\renewcommand{\summarypreamble}{Commands or environments defined in the \LaTeX\ kernel are always
available.}

\defgactivechar
 {backslashchar}
 {\backslashsym}
 {}
 {\LaTeX\ Kernel}
 {Escape character indicating a~command. (See \novices{command}.)}
 {}

\defgactivechar
 {dbbackslashchar}
 {\dbbackslashsym}
 {}
 {\LaTeX\ Kernel}
 {Starts a new row in \nxgls{env-tabbing} or tabular-style
environments. (See \novices{rowscols}.)}
 {}

\defgcs{tabularnewline}
 {}
 {\LaTeX\ Kernel}
 {Behaves like \nxglsi{tab.dbbackslashchar} in a
   \nxglsni{env-tabular}-like environment but helps to disambiguate a
    \nxglslink{newline.dbbackslashchar}{line break} in a paragraph
    cell from a \nxglslink{tab.dbbackslashchar}{row separator}.}
 {}

\defgidxactivechar
 {leftbracechar}
 {\leftbracesym}
 {}
 {\LaTeX\ Kernel}
 {Marks the beginning of a group. (See \novices{group}.)}
 {}

\defgidxactivechar
 {rightbracechar}
 {\rightbracesym}
 {}
 {\LaTeX\ Kernel}
 {Marks the end of a group. (See \novices{group}.)}
 {}

\defgidxactivecharcs
 {leftbrace}
 {\leftbracesym}
 {}
 {\LaTeX\ Kernel}
 {Left brace \{ character. In math mode may be used as a delimiter.}
 {}

\defgidxactivecharcs
 {rightbrace}
 {\rightbracesym}
 {}
 {\LaTeX\ Kernel}
 {Right brace \{ character. In math mode may be used as a delimiter.}
 {}

\defgidxactivechar
 [!Z]
 {visiblespace}
 {\textvisiblespace}
 {}
 {}
 {A visual indication of a space in the code. When you type up 
   the code, replace all instances of this symbol with a space via the space bar on your 
   keyboard.}
 {}

\defgcs{begin}
 {\marg{\meta{env-name}}\oarg{\meta{env-option}}\marg{\meta{env-arg-1}}\ldots\marg{\meta{env-arg-n}}}
 {\LaTeX\ Kernel}
 {\relax
   Starts an environment. (Must have a matching \nxglsi{end}.
   See \novices{environment}.)
 }
 {\relax
   \BeginArgList
    \csentryargitem{env-name} The name of the environment. (\emph{No
      backslash.})
    \csentryargitem{env-option} An optional argument that may be 
      passed to the environment. Not all environments have optional
      arguments.
    \csentryargitem{env-arg-1}\ldots\meta{env-arg-n} Any mandatory
      arguments required by the environment. Not all environments
      require arguments.
   \EndArgList
 }

\defgcs{end}
  {\marg{\meta{env-name}}}
  {\LaTeX\ Kernel}
  {Ends an environment. (Must have a matching \nxglsi{begin}. 
   See \novices{environment}.)}
  {\relax
    \BeginArgList
      \csentryargitem{env-name} The name of the environment.
    \EndArgList
  }

\defgcs{documentclass}
 {\oarg{\meta{option-list}}\marg{\meta{class-name}}}
 {\LaTeX\ Kernel}
 {\relax
   Loads the document class file, which sets up the type of document
   you wish to write. (See \novices[ch:simpledoc]{simpledoc}.)\relax
 }
 {\relax
   \BeginArgList
    \csentryargitem{option-list} A comma-separated list of options to
     pass to the class file or any packages that will later be
     loaded.
    \csentryargitem{class-name} The name of the document class. This
    corresponds to a file called \meta{class-name}\texttt{.cls},
    which must be installed. 
   \EndArgList
 }

\defgcs{usepackage}
 {\oarg{\meta{option-list}}\marg{\meta{package-list}}}
 {\LaTeX\ Kernel}
 {\relax
   Loads the listed package(s). (See \novices{packages}.)\relax
 }
 {\relax
   \BeginArgList
    \csentryargitem{option-list} A comma-separated list of options to
     pass to the package.
    \csentryargitem{package-list} A comma-separated list of package
     names (without the \texttt{.sty} extension).
   \EndArgList
 }

\defgactivechar
 [\underscoresym]
 {underscorechar}
 {\_}
 {\marg{\meta{maths}}}
 {\LaTeX\ Kernel (Math Mode)}
 {Displays its argument as a subscript. (See
\novices[sec:scripts]{subsupscripts}.)}
 {\relax
   \BeginArgList
     \csentryargitem{maths} The subscript.
   \EndArgList
 }

\defgactivechar
 {dollarchar}
 {\dollarsym}
 {}
 {\LaTeX\ Kernel}
 {Switches in and out of in-line math mode. (See
\novices[sec:inline]{inlinemaths}.)}
 {}

\defgactivechar
 {hashchar}
 {\hashsym}
 {\meta{digit}}
 {\LaTeX\ Kernel}
 {Replacement text for argument \meta{digit}. (See
\novices[ch:newcom]{newcom}.)}
 {}

\defgactivechar
 {percentchar}
 {\percentsym}
 {}
 {\LaTeX\ Kernel}
 {Comment character used to ignore everything up to and including
  the newline character in the source code. Sometimes comments are
used to provide information to applications that build your
document, such as \appname{arara}.}
 {}

\defgactivechar
 {percentchar.arara}
 {\percentsym{} arara:}
 {}
 {}
 {Instruction to \appname{arara} indicating how to build the
document. This is ignored if you are not using \appname{arara}.}
 {}

\defgactivechar
 {ampchar}
 {\ampsym}
 {}
 {\LaTeX\ Kernel}
 {Alignment tab.}
 {}

\defgactivechar
 [\circumsym]
 {circumchar}
 {\textasciicircum}
 {\marg{\meta{maths}}}
 {\LaTeX\ Kernel (Math Mode)}
 {Displays its argument as a superscript. (See
\novices[sec:scripts]{subsupscripts}.)}
 {\relax
   \BeginArgList
     \csentryargitem{maths} The superscript.
   \EndArgList
 }

\defgactivechar
 [\tildesym]
 {tildechar}
 {\textasciitilde}
 {}
 {\LaTeX\ Kernel}
 {Unbreakable space. (See \novices[sec:chars]{symbols}.)}
 {}

\defgcs{left}
 {\meta{delimiter}}
 {\LaTeX\ Kernel (Math Mode)}
 {Indicates a left stretchable delimiter. Must have a matching
  \nxglsi{right}.}
 {\relax
   \BeginArgList
    \csentryargitem{delimiter} A delimiter symbol, such as
    \nxglsni{openparen}, or a delimiter command, such as \nxglsni{langle}.
   \EndArgList
 }

\defgcs{right}
 {\meta{delimiter}}
 {\LaTeX\ Kernel (Math Mode)}
 {Indicates a right stretchable delimiter. Must have a matching
  \nxglsi{left}.}
 {\relax
   \BeginArgList
    \csentryargitem{delimiter} A delimiter symbol, such as
    \nxglsni{closeparen}, or a delimiter command, such as \nxglsni{rangle}.
   \EndArgList
 }

\defgchar
 {periodchar}
 {\periodsym}
 {}
 {\LaTeX\ Kernel}
 {\nopostdesc}
 {}

\defgchildchar
 {sentence.periodchar}
 {periodchar}
 {\periodsym}
 {period (full stop) or decimal point}

\defgchildchar
 {delimiter.periodchar}
 {periodchar}
 {\periodsym}
 {invisible delimiter}

\defgcs{author}
 {\marg{\meta{name}}}
 {Most classes that have the concept of a title page}
 {Specifies the document author (or authors). This command doesn't
  display any text so may be used in the preamble, but if it's not
  in the preamble it must be placed before \nxglsi{maketitle}.}
 {\relax
   \BeginArgList
    \csentryargitem{name} The name (or names) of the document author
     (or authors).
   \EndArgList
   Note that some classes, such as those supplied by journals or
   conference proceedings, may also define an optional argument
   that can be used to specify an abbreviated author list for the
   page headers.
 }

\defgcs{title}
 {\marg{\meta{text}}}
 {Most classes that have the concept of a title page}
 {Specifies the document title. This command doesn't
  display any text so may be used in the preamble, but if it's not
  in the preamble it must be placed before \nxglsi{maketitle}.}
 {\relax
   \BeginArgList
    \csentryargitem{text} The title of the document.
   \EndArgList
   Note that some classes, such as those supplied by journals or
   conference proceedings, may also define an optional argument
   that can be used to specify an abbreviated title for the page
   headers.
 }

\defgcs{date}
 {\marg{\meta{text}}}
 {Most classes that have the concept of a title page}
 {Specifies the document date. This command doesn't
  display any text so may be used in the preamble, but if it's not
  in the preamble it must be placed before \nxglsi{maketitle}. If omitted, most
  classes assume the current date.}
 {\relax
   \BeginArgList
    \csentryargitem{text} The document date.
   \EndArgList
 }

\defgcs{thanks}
 {\marg{\meta{text}}}
 {Most classes that have the concept of a title page}
 {Inserts a special type of footnote in one of the titling fields,
  such as \nxglsi{author} or \nxglsni{title}. Usually used for some form of
  acknowledgement or affiliation.}
 {}

\defgcs{today}
 {}
 {Most of the commonly-used classes}
 {Inserts into the output file the date when the \LaTeX\ 
  application created it from the source code.}
 {}

\defgcs{titlehead}
 {\marg{\meta{text}}}
 {\cls{scrartcl}, \cls{scrreprt}, \cls{scrbook} classes}
 {Specifies the title header (typeset at the top of the title page).}
 {\relax
   \BeginArgList
    \csentryargitem{text} The title header text.
   \EndArgList
 }

\defgcs{subtitle}
 {\marg{\meta{text}}}
 {\cls{scrartcl}, \cls{scrreprt}, \cls{scrbook} classes}
 {Specifies the subtitle (set just below the title).}
 {\relax
   \BeginArgList
    \csentryargitem{text} The subtitle text.
   \EndArgList
 }

\defgcs{subject}
 {\marg{\meta{text}}}
 {\cls{scrartcl}, \cls{scrreprt}, \cls{scrbook} classes}
 {Specifies the subject (set just above the title).}
 {\relax
   \BeginArgList
    \csentryargitem{text} The subject.
   \EndArgList
 }

\defgcs{publishers}
 {\marg{\meta{text}}}
 {\cls{scrartcl}, \cls{scrreprt}, \cls{scrbook} classes}
 {Specifies the publisher (set after all the other titling
   information).}
 {\relax
   \BeginArgList
    \csentryargitem{text} The publisher text.
   \EndArgList
 }

\defgcs{maketitle}
 {}
 {Most classes that have the concept of a title page}
 {Generates the title page (or title block). This command is usually
  placed at the beginning of the document environment.}
 {}

\defgenv{titlepage}
 {}
 {Most classes that have the concept of a title page}
 {The contents of this environment are displayed on a 
  single-column page with no header or footer and the page counter
  is set to 1.}
 {}

\defgcs{mathbf}
 {\marg{\meta{maths}}}
 {\LaTeX\ Kernel (Math Mode)}
 {Renders \meta{maths} in the predefined maths bold font. (Doesn't
  work with numbers and nonalphabetical symbols. See \novices{mathsfonts}.)}
 {\relax
   \BeginArgList
     \csentryargitem{maths} The text on which to apply the font change.
   \EndArgList
 }

\defgcs{textbf}
 {\marg{\meta{text}}}
 {\LaTeX\ Kernel}
 {Renders \meta{text} with a bold weight in the current font family,
if it exists. (See \novices{fontstyle}.)}
 {\relax
   \BeginArgList
     \csentryargitem{text} The text on which to apply the font change.
   \EndArgList
 }

\defgcs{texttt}
 {\marg{\meta{text}}}
 {\LaTeX\ Kernel}
 {Renders \meta{text} in the predefined monospaced font. 
 (See \novices{fontstyle}.)}
 {\relax
   \BeginArgList
     \csentryargitem{text} The text on which to apply the font
change.
   \EndArgList
 }

\defgcs{emph}
 {\marg{\meta{text}}}
 {\LaTeX\ Kernel}
 {Toggles the upright and italic\slash slanted rendering of
\meta{text}. (See \novices{fontstyle}.)}
 {\relax
   \BeginArgList
     \csentryargitem{text} The text on which to apply the font change.
   \EndArgList
 }

\defgcs{normalfont}
 {}
 {\LaTeX\ Kernel}
 {Switches to the default font style. (See \novices{fontstyle}.)}
 {}

\defgcs{large}
 {}
 {Most document classes}
 {Switches to large sized text. (See \novices{fontsize}.)}
 {\relax
 }

\defgcs{bfseries}
 {}
 {\LaTeX\ Kernel}
 {Switches to the bold weight in the current font
  family. (See \novices{fontstyle}.)}
 {}

\defgcs{scshape}
 {}
 {\LaTeX\ Kernel}
 {Switches to the small-caps form of the current font 
  family, if it exists. (See \novices{fontstyle}.)}
 {}

\defgcs{rmfamily}
 {}
 {\LaTeX\ Kernel}
 {Switches to the predefined serif font. (See \novices{fontstyle}.)}
 {}

\defgcs{ttfamily}
 {}
 {\LaTeX\ Kernel}
 {Switches to the predefined monospaced font. (See \novices{fontstyle}.)}
 {}

\defgcs{itshape}
 {}
 {\LaTeX\ Kernel}
 {Switches to the italic form of the current font family, if it
exists. (See \novices{fontstyle}.)}
 {}

\defgcs{chapter}
 {\oarg{\meta{short title}}\marg{\meta{title}}}
 {Book-style classes (such as \cls{scrbook} or \cls{scrreprt}) that have 
  the concept of chapters}
 {Inserts a chapter heading.}
 {\relax
   \BeginArgList
     \csentryargitem{short title} An abbreviated form of the title to
       go in the table of contents or the page header.
     \csentryargitem{title} The title.
   \EndArgList
   The starred form of this command doesn't have an optional
   argument and doesn't increment or display the chapter counter.
 }

\defgcs{section}
 {\oarg{\meta{short title}}\marg{\meta{title}}}
 {Most classes that have the concept of document structure}
 {Inserts a section header.}
 {\relax
   \BeginArgList
     \csentryargitem{short title} An abbreviated form of the title to
       go in the table of contents or the page header.
     \csentryargitem{title} The title.
   \EndArgList
   The starred form of this command doesn't have an optional
   argument and doesn't increment or display the section counter.
 }

\defgcs{subsection}
 {\oarg{\meta{short title}}\marg{\meta{title}}}
 {Most classes that have the concept of document structure}
 {Inserts a subsection header.}
 {\relax
   \BeginArgList
     \csentryargitem{short title} An abbreviated form of the title to
       go in the table of contents or the page header.
     \csentryargitem{title} The title.
   \EndArgList
   The starred form of this command doesn't have an optional
   argument and doesn't increment or display the subsection counter.
 }

\defgcs{subsubsection}
 {\oarg{\meta{short title}}\marg{\meta{title}}}
 {Most classes that have the concept of document structure}
 {Inserts a subsubsection header.}
 {\relax
   \BeginArgList
     \csentryargitem{short title} An abbreviated form of the title to
       go in the table of contents or the page header.
     \csentryargitem{title} The title.
   \EndArgList
   The starred form of this command doesn't have an optional
   argument and doesn't increment or display the subsubsection counter.
 }

\defgcs{paragraph}
 {\oarg{\meta{short title}}\marg{\meta{title}}}
 {Most classes that have the concept of document structure}
 {Inserts a subsubsubsection header. Most classes default to an 
  unnumbered running header for
  this sectional unit.}
 {\relax
   \BeginArgList
     \csentryargitem{short title} An abbreviated form of the title to
       go in the table of contents or the page header.
     \csentryargitem{title} The title.
   \EndArgList
   The starred form of this command doesn't have an optional
   argument and doesn't increment or display the associated counter.
 }

\defgcs{subparagraph}
 {\oarg{\meta{short title}}\marg{\meta{title}}}
 {Most classes that have the concept of document structure}
 {Inserts a subsubsubsubsection header. Most classes default 
  to an unnumbered running header for
  this sectional unit.}
 {\relax
   \BeginArgList
     \csentryargitem{short title} An abbreviated form of the title to
       go in the table of contents or the page header.
     \csentryargitem{title} The title.
   \EndArgList
   The starred form of this command doesn't have an optional
   argument and doesn't increment or display the associated counter.
 }

\defgcs{minisec}
 {\marg{\meta{heading}}}
 {\cls{scrartcl}, \cls{scrreprt} and
  \cls{scrbook} classes}
 {An unnumbered heading not associated with any structuring level.}
 {\relax
   \BeginArgList
    \csentryargitem{heading} The heading text.
   \EndArgList
 }

\defgcs{addtokomafont}
 {\marg{\meta{element name}}\marg{\meta{commands}}}
 {\cls{scrartcl}, \cls{scrreprt} and
  \cls{scrbook} classes}
 {Sets the font characteristics for the given \koma\ element. 
  (See \novices[sec:section]{sectionunits}.)}
 {\relax
   \BeginArgList
     \csentryargitem{element name} The element's name, for example
      \texttt{chapter}. See the \koma\ manual for a full list
      of defined elements.
     \csentryargitem{commands} The font changing commands to apply
      to that element.
   \EndArgList
 }

\defgcs{appendix}
 {}
 {Most classes that have the concept of document structure}
 {Indicates (but doesn't print anything) that the document is
  switching to the appendices. If chapters exist, the chapter
  numbering is reset and switched to a different format
  (usually upper case letters) otherwise the section numbering
  is reset and switched to a different format.}
 {}

\defgcs{tableofcontents}
 {}
 {Most classes that have the concept of document structure}
 {Inserts the table of contents. A second (possibly third) run
  is required to ensure the page numbering is correct.}
 {}

\defgcs{label}
 {\marg{\meta{string}}}
 {\LaTeX\ Kernel}
 {Assigns a unique textual label linked to the most recently
  incremented cross-referencing counter in the current scope. 
  (See \novices{crossref}.)}
 {\relax
   \BeginArgList
    \csentryargitem{string} A unique label that can be referenced
     elsewhere in the document with \nxglsi{ref}. (It's best to 
     just use alphanumerics and non-active punctuation characters in
     the label.)
   \EndArgList
 }

\defgcs{ref}
 {\marg{\meta{string}}}
 {\LaTeX\ Kernel}
 {References the value of the counter linked to the given label.
  A second (possibly third) run of \LaTeX\ is required to ensure the cross-references
  are up-to-date.
  (See \novices{crossref}.)
 }
 {\relax
   \BeginArgList
    \csentryargitem{string} The required label that was used in the
     corresponding \nxglsi{label}.
   \EndArgList
 }

\defgcs{pageref}
 {\marg{\meta{string}}}
 {\LaTeX\ Kernel}
 {Similar to \nxglsi{ref} but inserts the page number where the given
  label was defined.
  A second (possibly third) run of \LaTeX\ is required to ensure the cross-references
  are up-to-date.
 }
 {\relax
   \BeginArgList
    \csentryargitem{string} The required label that was used in the
     corresponding \nxglsi{label}.
   \EndArgList
 }

\defgcs{vref}
 {\marg{\meta{string}}}
 {\sty{varioref} package}
 {Like \nxglsi{ref} but also adds information about the location, such
  as \dq{on page~\meta{n}} or \dq{on the following page}.
 }
 {\relax
   \BeginArgList
    \csentryargitem{string} The required label that was used in the
     corresponding \nxglsi{label}.
   \EndArgList
 }

\defgcs{bibitem}
 {\oarg{\meta{tag}}\marg{\meta{key}}}
 {\LaTeX\ Kernel}
 {Indicates the start of a new reference in the bibliography. May
  only be used inside the contents of \nxglsi{env-thebibliography}
  environment. (See \novices[sec:bib]{biblio}.)}
 {\relax
   \BeginArgList
    \csentryargitem{tag} If present, overrides the marker at the
    start of the reference.
    \csentryargitem{key} A unique key that identifies this reference so
    it can be cited elsewhere in the document using \nxglsi{cite}.
   \EndArgList
 }

\defgcs{cite}
 {\oarg{\meta{text}}\marg{\meta{key list}}}
 {\LaTeX\ Kernel}
 {Inserts the citation markers of each reference identified in the
  key list. A second run is required to ensure the reference is
  correct. When used with \sty{biblatex}, this command has two
  optional arguments.}
 {\relax
    \BeginArgList
      \csentryargitem{text} Additional text to insert into the citation
       (such as the chapter number, or a particular page or
       page range within the citation).
      \csentryargitem{key list} A comma-separated list of keys used in
      the corresponding \nxglsi{bibitem}.
    \EndArgList
 }

\defgcs{Cite}
 {\oarg{\meta{prenote}}\oarg{\meta{postnote}}\marg{\meta{key}}}
 {\sty{biblatex} package}
 {Like \nxglsni{cite} but for use at the start of a sentence.}
 {\relax
    \BeginArgList
      \csentryargitem{prenote} a prenote, such as \dq{see}.
      \csentryargitem{postnote} a postnote, such as the chapter or
         section within the work.
      \csentryargitem{key list} A comma-separated list of keys
       identifying the entries.
    \EndArgList
 }

\defgcs{parencite}
 {\oarg{\meta{prenote}}\oarg{\meta{postnote}}\marg{\meta{key}}}
 {\sty{biblatex} package}
 {Like \nxgls{cite} but the citation is enclosed in parentheses.}
 {\relax
    \BeginArgList
      \csentryargitem{prenote} a prenote, such as \dq{see}.
      \csentryargitem{postnote} a postnote, such as the chapter or
         section within the work.
      \csentryargitem{key list} A comma-separated list of keys
       identifying the entries.
    \EndArgList
 }

\defgcs{Parencite}
 {\oarg{\meta{prenote}}\oarg{\meta{postnote}}\marg{\meta{key}}}
 {\sty{biblatex} package}
 {Like \nxglsni{parencite} but for use at the start of a sentence.}
 {\relax
    \BeginArgList
      \csentryargitem{prenote} a prenote, such as \dq{see}.
      \csentryargitem{postnote} a postnote, such as the chapter or
         section within the work.
      \csentryargitem{key list} A comma-separated list of keys
       identifying the entries.
    \EndArgList
 }

\defgcs{textcite}
 {\oarg{\meta{prenote}}\oarg{\meta{postnote}}\marg{\meta{key}}}
 {\sty{biblatex} package}
 {Like \nxgls{cite} but designed for use in the flow of text.}
 {\relax
    \BeginArgList
      \csentryargitem{prenote} a prenote, such as \dq{see}.
      \csentryargitem{postnote} a postnote, such as the chapter or
         section within the work.
      \csentryargitem{key list} A comma-separated list of keys
       identifying the entries.
    \EndArgList
 }

\defgcs{Textcite}
 {\oarg{\meta{prenote}}\oarg{\meta{postnote}}\marg{\meta{key}}}
 {\sty{biblatex} package}
 {Like \nxgls{textcite} but for use at the start of a sentence.}
 {\relax
    \BeginArgList
      \csentryargitem{prenote} a prenote, such as \dq{see}.
      \csentryargitem{postnote} a postnote, such as the chapter or
         section within the work.
      \csentryargitem{key list} A comma-separated list of keys
       identifying the entries.
    \EndArgList
 }

\defgcs{caption}
 {\oarg{\meta{short caption}}\marg{\meta{caption text}}}
 {\LaTeX\ Kernel}
 {Inserts the caption for a float such as a figure or table. 
 (See \novices[ch:floats]{floats}.)}
 {\relax
   \BeginArgList
    \csentryargitem{short caption} If provided, an abbreviated caption
      to go in the list of figures\slash tables etc.
    \csentryargitem{caption text} The caption contents.
   \EndArgList
 }

\defgcs{listoffigures}
 {}
 {Most classes that have the concept of document structure}
 {Inserts the list of figures. A second (possibly third) run
  is required to ensure the page numbering is correct.}
 {}

\defgcs{listoftables}
 {}
 {Most classes that have the concept of document structure}
 {Inserts the list of tables. A second (possibly third) run
  is required to ensure the page numbering is correct.}
 {}

\defgcs{par}
 {}
 {\LaTeX\ Kernel}
 {Insert a paragraph break.}
 {}

\defgcs{newcommand}
 {\marg{\meta{cmd}}\oarg{\meta{n-args}}\oarg{\meta{default}}\marg{\meta{text}}}
 {\LaTeX\ Kernel}
 {Defines a new command. (See \novices[ch:newcom]{newcom}.)}
 {\relax
   \BeginArgList
    \csentryargitem{cmd} The new command name (including initial backslash).
    \csentryargitem{n-args} The number of arguments this new command
     should have.
    \csentryargitem{default} If the first argument should be optional,
     the default value if omitted.
    \csentryargitem{text} What actions the command should perform.
   \EndArgList
 }

\defgcs{renewcommand}
 {\marg{\meta{cmd}}\oarg{\meta{n-args}}\oarg{\meta{default}}\marg{\meta{text}}}
 {\LaTeX\ Kernel}
 {Redefines an existing command. (See \novices{renewcom}.)}
 {\relax
   \BeginArgList
    \csentryargitem{cmd} The command name (including initial backslash).
    \csentryargitem{n-args} The number of arguments this command
     should have.
    \csentryargitem{default} If the first argument should be optional,
     the default value if omitted.
    \csentryargitem{text} What actions the command should perform.
   \EndArgList
 }

\defgcs{newenvironment}
 {\marg{\meta{env-name}}\oarg{\meta{n-args}}\oarg{\meta{default}}\marg{\meta{begin-code}}\marg{\meta{end-code}}}
 {\LaTeX\ Kernel}
 {Defines a new environment.}
 {\relax
   \BeginArgList
    \csentryargitem{env-name} The new environment name (\emph{no} backslash).
    \csentryargitem{n-args} The number of arguments this new
     environment should have.
    \csentryargitem{default} If the first argument should be optional,
     the default value if omitted.
    \csentryargitem{begin-code} Actions to perform at the start of the
      environment.
    \csentryargitem{end-code} Actions to perform at the end of the
      environment.
   \EndArgList
 }

\defgcs{renewenvironment}
 {\marg{\meta{env-name}}\oarg{\meta{n-args}}\oarg{\meta{default}}\marg{\meta{begin-code}}\marg{end-code}}
 {\LaTeX\ Kernel}
 {Redefines an existing environment.}
 {\relax
   \BeginArgList
    \csentryargitem{env-name} The environment name (\emph{no} backslash).
    \csentryargitem{n-args} The number of arguments this new
     environment should have.
    \csentryargitem{default} If the first argument should be optional,
     the default value if omitted.
    \csentryargitem{begin-code} Actions to perform at the start of the
      environment.
    \csentryargitem{end-code} Actions to perform at the end of the
      environment.
   \EndArgList
 }

\defgcs{makeindex}
 {}
 {\LaTeX\ Kernel (Preamble Only)}
 {Enables \nxglsi{index}.}
 {}

\defgcs{index}
 {\marg{\meta{text}}}
 {\LaTeX\ Kernel}
 {Adds indexing information to an external index file. The command
   \nxglsi{makeindex} must be used in the preamble to enable
   this command. The external index file must be post-processed with
   an indexing application, such as \nxiappname{makeindex}.}
 {\relax
   \BeginArgList
     \csentryargitem{text} The text to go in the index.
   \EndArgList
 }

\defgcs{printindex}
 {}
 {\sty{makeidx} package}
 {Prints the index. Must be used with \nxglsi{makeindex} and \nxglsi{index}.
 (The external index file must first be processed by an indexing application.)}
 {}

\defgcs{makeglossaries}
 {}
 {\sty{glossaries} package}
 {Activates \nxglsi{printglossaries} (and \nxglsi{printglossary}).}
 {}

\defgcs{printglossary}
 {\oarg{\meta{key-val option list}}}
 {\sty{glossaries} package}
 {Prints the glossary identified in the optional argument or the
default glossary if none identified.}
 {\relax
   \BeginArgList
     \csentryargitem{key-val option list} A comma-separated list of
\meta{key}=\meta{value} options
   \EndArgList
 }

\defgcs{printglossaries}
 {}
 {\sty{glossaries} package}
 {Prints all of the defined glossaries.}
 {}

\defgcs{newglossaryentry}
 {\marg{\meta{label}}\marg{\meta{key-\reportlinebreak val list}}}
 {\sty{glossaries} package}
 {Defines a new glossary entry or term.}
 {\relax
   \BeginArgList
     \csentryargitem{label} A unique label identifying this entry.
     \csentryargitem{key-val list} A comma-separated list that
define this entry.
   \EndArgList
 }

\defgcs{newacronym}
 {\oarg{\meta{key-val list}}\reportlinebreak\marg{\meta{label}}\booklinebreak\marg{\meta{abbrv}}\marg{\meta{long}}}
 {\sty{glossaries} package}
 {Shortcut that uses \nxglsni{newglossaryentry} to define an acronym.}
 {\relax
   \BeginArgList
     \csentryargitem{key-val list} A comma-separated list (same as
for \nxglsni{newglossaryentry}) that can be used to override
\cmdname{newacronym} defaults or add additional information.
     \csentryargitem{abbrv} The acronym.
     \csentryargitem{long} The long (expanded) form of the acronym.
   \EndArgList
 }

\defgcs{glsreset}
 {\marg{\meta{label}}}
 {\sty{glossaries} and \sty{datagidx} packages}
 {Resets a glossary term's first use flag.}
 {\relax
   \BeginArgList
     \csentryargitem{label} the label identifying the term that
needs resetting.
   \EndArgList
 }

\defgcs{glsunset}
 {\marg{\meta{label}}}
 {\sty{glossaries} and \sty{datagidx} packages}
 {Unsets a glossary term's first use flag.}
 {\relax
   \BeginArgList
     \csentryargitem{label} the label identifying the term that
needs unsetting.
   \EndArgList
 }

\defgcs{gls}
 {\oarg{\meta{options}}\marg{\meta{label}}\oarg{\meta{insert}}}
 {\sty{glossaries} package}
 {Displays a glossary term according to its first use flag.}
 {\relax
   \BeginArgList
    \csentryargitem{options} a \meta{key}=\meta{value} list of
options.
    \csentryargitem{label} the label uniquely identifying the term.
    \csentryargitem{insert} text to insert after the term (but
inside the hyperlink, if used with the \sty{hyperref} package).
   \EndArgList
 }

\defgcs{glspl}
 {\oarg{\meta{options}}\marg{\meta{label}}\oarg{\meta{insert}}}
 {\sty{glossaries} package}
 {Displays the plural form of a glossary term according to its first use flag.}
 {\relax
   \BeginArgList
    \csentryargitem{options} a \meta{key}=\meta{value} list of
options.
    \csentryargitem{label} the label uniquely identifying the term.
    \csentryargitem{insert} text to insert after the term (but
inside the hyperlink, if used with the \sty{hyperref} package).
   \EndArgList
 }

\defgcs{Gls}
 {\oarg{\meta{options}}\marg{\meta{label}}\oarg{\meta{insert}}}
 {\sty{glossaries} package}
 {Displays a glossary term according to its first use flag. The
first letter of the term is converted to uppercase.}
 {\relax
   \BeginArgList
    \csentryargitem{options} a \meta{key}=\meta{value} list of
options.
    \csentryargitem{label} the label uniquely identifying the term.
    \csentryargitem{insert} text to insert after the term (but
inside the hyperlink, if used with the \sty{hyperref} package).
   \EndArgList
 }

\defgcs{Glspl}
 {\oarg{\meta{options}}\marg{\meta{label}}\oarg{\meta{insert}}}
 {\sty{glossaries} package}
 {Displays the plural form of a glossary term according to its first
use flag. The first letter of the plural text is converted to
uppercase.}
 {\relax
   \BeginArgList
    \csentryargitem{options} a \meta{key}=\meta{value} list of
options.
    \csentryargitem{label} the label uniquely identifying the term.
    \csentryargitem{insert} text to insert after the term (but
inside the hyperlink, if used with the \sty{hyperref} package).
   \EndArgList
 }

\defgcs{glssymbol}
 {\oarg{\meta{options}}\marg{\meta{label}}\oarg{\meta{insert}}}
 {\sty{glossaries} package}
 {Displays the symbol element of a glossary entry.}
 {\relax
   \BeginArgList
    \csentryargitem{options} a \meta{key}=\meta{value} list of
options.
    \csentryargitem{label} the label uniquely identifying the term.
    \csentryargitem{insert} text to insert after the term (but
inside the hyperlink, if used with the \sty{hyperref} package).
   \EndArgList
 }

\defgcs{acrshort}
 {\oarg{\meta{options}}\marg{\meta{label}}\oarg{\meta{insert}}}
 {\sty{glossaries} package}
 {Displays the short form of the given acronym.}
 {\relax
   \BeginArgList
    \csentryargitem{options} a \meta{key}=\meta{value} list of
options.
    \csentryargitem{label} the label uniquely identifying the
acronym (as defined by \nxglsni{newacronym}).
    \csentryargitem{insert} text to insert after the acronym (but
inside the hyperlink, if used with the \sty{hyperref} package).
   \EndArgList
 }

\defgcs{Acrshort}
 {\oarg{\meta{options}}\marg{\meta{label}}\oarg{\meta{insert}}}
 {\sty{glossaries} package}
 {Displays the short form of the given acronym, the first letter
converted to uppercase.}
 {\relax
   \BeginArgList
    \csentryargitem{options} a \meta{key}=\meta{value} list of
options.
    \csentryargitem{label} the label uniquely identifying the
acronym (as defined by \nxglsni{newacronym}).
    \csentryargitem{insert} text to insert after the acronym (but
inside the hyperlink, if used with the \sty{hyperref} package).
   \EndArgList
 }

\defgcs{acrlong}
 {\oarg{\meta{options}}\marg{\meta{label}}\oarg{\meta{insert}}}
 {\sty{glossaries} package}
 {Displays the long form of the given acronym.}
 {\relax
   \BeginArgList
    \csentryargitem{options} a \meta{key}=\meta{value} list of
options.
    \csentryargitem{label} the label uniquely identifying the
acronym (as defined by \nxglsni{newacronym}).
    \csentryargitem{insert} text to insert after the term (but
inside the hyperlink, if used with the \sty{hyperref} package).
   \EndArgList
 }

\defgcs{Acrlong}
 {\oarg{\meta{options}}\marg{\meta{label}}\oarg{\meta{insert}}}
 {\sty{glossaries} package}
 {Displays the long form of the given acronym, the first letter
converted to uppercase.}
 {\relax
   \BeginArgList
    \csentryargitem{options} a \meta{key}=\meta{value} list of
options.
    \csentryargitem{label} the label uniquely identifying the
acronym (as defined by \nxglsni{newacronym}).
    \csentryargitem{insert} text to insert after the term (but
inside the hyperlink, if used with the \sty{hyperref} package).
   \EndArgList
 }

\defgcs{acrfull}
 {\oarg{\meta{options}}\marg{\meta{label}}\oarg{\meta{insert}}}
 {\sty{glossaries} package}
 {Displays the long and short form of the given acronym.}
 {\relax
   \BeginArgList
    \csentryargitem{options} a \meta{key}=\meta{value} list of
options.
    \csentryargitem{label} the label uniquely identifying the
acronym (as defined by \nxglsni{newacronym}).
    \csentryargitem{insert} text to insert after the term (but
inside the hyperlink, if used with the \sty{hyperref} package).
   \EndArgList
 }

\defgcs{Acrfull}
 {\oarg{\meta{options}}\marg{\meta{label}}\oarg{\meta{insert}}}
 {\sty{glossaries} package}
 {Displays the long and short form of the given acronym, the first letter
converted to uppercase.}
 {\relax
   \BeginArgList
    \csentryargitem{options} a \meta{key}=\meta{value} list of
options.
    \csentryargitem{label} the label uniquely identifying the
acronym (as defined by \nxglsni{newacronym}).
    \csentryargitem{insert} text to insert after the term (but
inside the hyperlink, if used with the \sty{hyperref} package).
   \EndArgList
 }

\defgcs{acs}
 {\oarg{\meta{options}}\marg{\meta{label}}\oarg{\meta{insert}}}
 {\sty{glossaries} package}
 {A synonym for \nxglsni{acrshort}. This command is only available if
the package option \optfmt{shortcuts} is used.}
 {\relax
   \BeginArgList
    \csentryargitem{options} a \meta{key}=\meta{value} list of
options.
    \csentryargitem{label} the label uniquely identifying the
acronym (as defined by \nxglsni{newacronym}).
    \csentryargitem{insert} text to insert after the term (but
inside the hyperlink, if used with the \sty{hyperref} package).
   \EndArgList
 }

\defgcs{acl}
 {\oarg{\meta{options}}\marg{\meta{label}}\oarg{\meta{insert}}}
 {\sty{glossaries} package}
 {A synonym for \nxglsni{acrlong}. This command is only available if
the package option \optfmt{shortcuts} is used.}
 {\relax
   \BeginArgList
    \csentryargitem{options} a \meta{key}=\meta{value} list of
options.
    \csentryargitem{label} the label uniquely identifying the
acronym (as defined by \nxglsni{newacronym}).
    \csentryargitem{insert} text to insert after the term (but
inside the hyperlink, if used with the \sty{hyperref} package).
   \EndArgList
 }

\defgcs{acf}
 {\oarg{\meta{options}}\marg{\meta{label}}\oarg{\meta{insert}}}
 {\sty{glossaries} package}
 {A synonym for \nxglsni{acrfull}. This command is only available if
the package option \optfmt{shortcuts} is used.}
 {\relax
   \BeginArgList
    \csentryargitem{options} a \meta{key}=\meta{value} list of
options.
    \csentryargitem{label} the label uniquely identifying the
acronym (as defined by \nxglsni{newacronym}).
    \csentryargitem{insert} text to insert after the term (but
inside the hyperlink, if used with the \sty{hyperref} package).
   \EndArgList
 }

\defgcs{ac}
 {\oarg{\meta{options}}\marg{\meta{label}}\oarg{\meta{insert}}}
 {\sty{glossaries} package}
 {A synonym for \nxgls{gls}. This command is only available if
the package option \optfmt{shortcuts} is used.}
 {\relax
   \BeginArgList
    \csentryargitem{options} a \meta{key}=\meta{value} list of
options.
    \csentryargitem{label} the label uniquely identifying the
acronym (as defined by \nxglsni{newacronym}).
    \csentryargitem{insert} text to insert after the term (but
inside the hyperlink, if used with the \sty{hyperref} package).
   \EndArgList
 }

\defgcs{Ac}
 {\oarg{\meta{options}}\marg{\meta{label}}\oarg{\meta{insert}}}
 {\sty{glossaries} package}
 {A synonym for \nxgls{Gls}. This command is only available if
the package option \optfmt{shortcuts} is used.}
 {\relax
   \BeginArgList
    \csentryargitem{options} a \meta{key}=\meta{value} list of
options.
    \csentryargitem{label} the label uniquely identifying the
acronym (as defined by \nxglsni{newacronym}).
    \csentryargitem{insert} text to insert after the term (but
inside the hyperlink, if used with the \sty{hyperref} package).
   \EndArgList
 }

\defgcs{glsentrytext}
 {\marg{\meta{label}}}
 {\sty{glossaries} package}
 {Displays the value of the \optfmt{text} key for a glossary entry.}
 {\relax
   \BeginArgList
     \csentryargitem{label} the label uniquely identifying the
entry.
   \EndArgList
 }

\defgcs{glsentryfirst}
 {\marg{\meta{label}}}
 {\sty{glossaries} package}
 {Displays the value of the \optfmt{first} key for a glossary entry.}
 {\relax
   \BeginArgList
     \csentryargitem{label} the label uniquely identifying the
entry.
   \EndArgList
 }

\defgcs{newglossary}
 {\oarg{\meta{log-ext}}\marg{\meta{name}}\marg{\meta{in-ext}}\marg{\meta{out-ext}}\marg{\meta{title}}\oarg{\meta{counter}}}
 {\sty{glossaries} package}
 {Defines a new glossary.}
 {\relax
   \BeginArgList
     \csentryargitem{log-ext} the extension of the associated log
file.
     \csentryargitem{name} a label that uniquely identifies this
new glossary
     \csentryargitem{in-ext} the associated glossary input file extension.
     \csentryargitem{out-ext} the associated glossary output file extension.
     \csentryargitem{title} the title used when the glossary is
displayed.
     \csentryargitem{counter} the default counter to use in this
glossary's location lists.
   \EndArgList
 }

\defgcs{glsadd}
 {\oarg{\meta{options}}\marg{\meta{label}}}
 {\sty{glossaries} package}
 {Adds the given entry to the glossary without displaying any text.}
 {\relax
   \BeginArgList
     \csentryargitem{options} a comma-separated list of
\meta{key}=\meta{value} options.
     \csentryargitem{label} the label uniquely identifying the
entry.
   \EndArgList
 }

\defgcs{glsaddall}
 {\oarg{\meta{options}}}
 {\sty{glossaries} package}
 {Adds all the defined entries without displaying any text.}
 {\relax
   \BeginArgList
     \csentryargitem{options} a comma-separated list of
\meta{key}=\meta{value} options.
   \EndArgList
 }

\defgsymcs[beginmath]{\openparensym}
 {}
 {\LaTeX\ Kernel}
 {Equivalent to \nxglsni{begin}\marg{math}.}
 {}

\defgsymcs[endmath]{\closeparensym}
 {}
 {\LaTeX\ Kernel}
 {Equivalent to \nxglsni{end}\marg{math}.}
 {}

\defgchar
 {openparen}
 {\openparensym}
 {}
 {\LaTeX\ Kernel}
 {\relax
   Opening parenthesis in text mode or left round bracket
   delimiter in math mode.\relax
 }
 {}

\defgchar
 {closeparen}
 {\closeparensym}
 {}
 {\LaTeX\ Kernel}
 {\relax
   Closing parenthesis in text mode or right round bracket
   delimiter in math mode.\relax
 }
 {}

\defgsymcs[begindispmath]{\opensqsym}
 {}
 {\LaTeX\ Kernel (inconsistency corrected in \sty{amsmath})}
 {Starts an unnumbered single-line of displayed maths.}
 {}

\defgsymcs[enddispmath]{\closesqsym}
 {}
 {\LaTeX\ Kernel (inconsistency corrected in \sty{amsmath})}
 {Ends an unnumbered single-line of displayed maths.}
 {}

\defgchar
 {opt.opensq}
 {\opensqsym}
 {}
 {\LaTeX\ Kernel}
 {Open delimiter of an optional argument. (See \novices{optional}.)}
 {}

\defgchar
 {opt.closesq}
 {\closesqsym}
 {}
 {\LaTeX\ Kernel}
 {Closing delimiter of an optional argument. (See \novices{optional}.)}
 {}

\defgenv{document}
 {}
 {\LaTeX\ Kernel}
 {The body of the document.}
 {}

\defgenv{thebibliography}
 {\marg{\meta{widest entry label}}}
 {Most classes that define sectioning commands}
 {Bibliographic list. (See \novices[sec:bib]{biblio}.)}
 {\relax
   \BeginArgList
    \csentryargitem{widest entry label} The widest label in the
     bibliography list.
   \EndArgList
 }

\defgcs{frontmatter}
 {}
 {Most book-style classes, such as \cls{scrbook}}
 {Switches to lower case Roman numeral page numbering. Also suppresses
   chapter and section numbering, but still adds unstarred sectional
   units to the table of contents. (See also \nxglsi{mainmatter}
   and \nxglsi{backmatter}.)}
 {}

\defgcs{mainmatter}
 {}
 {Most book-style classes, such as \cls{scrbook}}
 {Switches to Arabic page numbering and enables
   chapter and section numbering. (See also
   \nxglsi{frontmatter} and \nxglsi{backmatter}.)}
 {}

\defgcs{backmatter}
 {}
 {Most book-style classes, such as \cls{scrbook}}
 {Suppresses chapter and section numbering, but still adds unstarred
  sectional units to the table of contents. (See also \nxglsi{frontmatter}
  and \nxglsi{mainmatter}.)}
 {}

\defgenv{abstract}
 {}
 {Most article- or report-style classes, such as \cls{scrartcl} or
  \cls{scrreprt}. Not usually defined in book-style classes, such
  as \cls{scrbook}, but is defined in \cls{memoir}}
 {Displays its contents as an abstract.}
 {}

\defgcs{pagenumbering}
 {\marg{\meta{style}}}
 {\LaTeX\ Kernel}
 {Sets the style of the page numbers.}
 {\relax
   \BeginArgList
     \csentryargitem{style} The page numbering style (e.g.\
      \texttt{roman} for lower case Roman numerals).
   \EndArgList
 }

\defgcs{pagestyle}
 {\marg{\meta{style}}}
 {\LaTeX\ Kernel}
 {Sets the style of the headers and footers.}
 {\relax
   \BeginArgList
     \csentryargitem{style} The name of the page style. The \LaTeX\
     kernel defines only two styles: \nxipagestyle{empty} and
     \nxipagestyle{plain}. Most of the standard classes also provide
     the \nxipagestyle{headings} style.
   \EndArgList
 }

\defgcs{thispagestyle}
 {\marg{\meta{style}}}
 {\LaTeX\ Kernel}
 {Like \nxglsi{pagestyle} but only affects the current page.}
 {\relax
   \BeginArgList
     \csentryargitem{style} The name of the page style.
   \EndArgList
 }

\defgcs{ihead}
 {\oarg{\meta{scrplain}}\marg{\meta{scrheadings}}}
 {\sty{scrpage2} package}
 {Indicates what to put in the inner heading area for the
  \pagestylefmt{scrplain} and \pagestylefmt{scrheadings} page styles.}
 {\relax
   \BeginArgList
     \csentryargitem{scrplain} The text used by
       \pagestylefmt{scrplain} page style.
     \csentryargitem{scrheadings} The text used by
       \pagestylefmt{scrheadings} page style.
   \EndArgList
 }

\defgcs{chead}
 {\oarg{\meta{scrplain}}\marg{\meta{scrheadings}}}
 {\sty{scrpage2} package}
 {Indicates what to put in the centre heading area for the
  \pagestylefmt{scrplain} and \pagestylefmt{scrheadings} page styles.}
 {\relax
   \BeginArgList
     \csentryargitem{scrplain} The text used by
       \pagestylefmt{scrplain} page style.
     \csentryargitem{scrheadings} The text used by
       \pagestylefmt{scrheadings} page style.
   \EndArgList
 }

\defgcs{ohead}
 {\oarg{\meta{scrplain}}\marg{\meta{scrheadings}}}
 {\sty{scrpage2} package}
 {Indicates what to put in the outer heading area for the
  \pagestylefmt{scrplain} and \pagestylefmt{scrheadings} page styles.}
 {\relax
   \BeginArgList
     \csentryargitem{scrplain} The text used by
       \pagestylefmt{scrplain} page style.
     \csentryargitem{scrheadings} The text used by
       \pagestylefmt{scrheadings} page style.
   \EndArgList
 }

\defgcs{ifoot}
 {\oarg{\meta{scrplain}}\marg{\meta{scrheadings}}}
 {\sty{scrpage2} package}
 {Indicates what to put in the inner footer area for the
  \pagestylefmt{scrplain} and \pagestylefmt{scrheadings} page styles.}
 {\relax
   \BeginArgList
     \csentryargitem{scrplain} The text used by
       \pagestylefmt{scrplain} page style.
     \csentryargitem{scrheadings} The text used by
       \pagestylefmt{scrheadings} page style.
   \EndArgList
 }

\defgcs{cfoot}
 {\oarg{\meta{scrplain}}\marg{\meta{scrheadings}}}
 {\sty{scrpage2} package}
 {Indicates what to put in the centre footer area for the
  \pagestylefmt{scrplain} and \pagestylefmt{scrheadings} page styles.}
 {\relax
   \BeginArgList
     \csentryargitem{scrplain} The text used by
       \pagestylefmt{scrplain} page style.
     \csentryargitem{scrheadings} The text used by
       \pagestylefmt{scrheadings} page style.
   \EndArgList
 }

\defgcs{ofoot}
 {\oarg{\meta{scrplain}}\marg{\meta{scrheadings}}}
 {\sty{scrpage2} package}
 {Indicates what to put in the outer footer area for the
  \pagestylefmt{scrplain} and \pagestylefmt{scrheadings} page styles.}
 {\relax
   \BeginArgList
     \csentryargitem{scrplain} The text used by
       \pagestylefmt{scrplain} page style.
     \csentryargitem{scrheadings} The text used by
       \pagestylefmt{scrheadings} page style.
   \EndArgList
 }

\defgcs{headfont}
 {}
 {\sty{scrpage2} package}
 {Determines the font used by the header and footer with the
 \pagestylefmt{scrplain} and \pagestylefmt{scrheadings} page styles.}
 {}

\defgcs{pnumfont}
 {}
 {\sty{scrpage2} package}
 {Determines the font used by \nxglsni{pagemark} with the
 \pagestylefmt{scrplain} and \pagestylefmt{scrheadings} page styles.}
 {}

\defgcs{pagemark}
 {}
 {\sty{scrpage2} package}
 {Used in commands like \nxglsni{ihead} to insert the current page
  number.}
 {}

\defgcs{headmark}
 {}
 {\sty{scrpage2} package}
 {Used in commands like \nxglsni{ihead} to insert the current 
   running header.}
 {}

\defgcs{include}
 {\marg{\meta{file name}}}
 {\LaTeX\ Kernel}
 {Issues a \nxgls{clearpage}, creates an associated auxiliary file,
  inputs \meta{file name} and issues another \nxgls{clearpage}.
  (See also \nxglsni{input}.)}
 {\relax
   \BeginArgList
     \csentryargitem{file name} The name of the file to be included.
    (The \texttt{.tex} extension may be omitted.)
   \EndArgList
 }

\defgcs{input}
 {\marg{\meta{file name}}}
 {\LaTeX\ Kernel}
 {Reads in the contents of \meta{file name}. (See also
  \nxglsni{include}.)}
 {\relax
   \BeginArgList
     \csentryargitem{file name} The name of the file to be read in.
    (The \texttt{.tex} extension may be omitted.)
   \EndArgList
 }

\defgcs{includeonly}
 {\meta{\meta{file list}}}
 {\LaTeX\ Kernel (Preamble Only)}
 {Lists which of the files that are included using \nxglsni{include}
  should be read in. Any files not in the list won't be included.}
 {\relax
   \BeginArgList
     \csentryargitem{file list} Comma-separated list of file names.
   \EndArgList
 }


\defgcs{excludeonly}
 {\meta{\meta{file list}}}
 {\sty{excludeonly} Package}
 {Lists which of the files that are not to be included using \nxgls{include}. 
  Only those files not in the list will be included. (The opposite
  effect of \nxgls{includeonly}.)}
 {\relax
   \BeginArgList
     \csentryargitem{file list} Comma-separated list of file names.
   \EndArgList
 }

\defgcs{clearpage}
 {}
 {\LaTeX\ Kernel}
 {Inserts a page break and processes any unprocessed floats}
 {}

\defgcs{raggedsection}
 {}
 {\koma\ classes, such as \cls{scrbook} and \cls{scrreprt}}
 {Governs the justification of headings. Defaults to
\nxglsni{raggedright}}
 {}

\defgcs{raggedright}
 {}
 {\LaTeX\ Kernel}
 {Ragged-right paragraph justification. (See
\novices{declaration}.)}
 {}

\defgcs{centering}
 {}
 {\LaTeX\ Kernel}
 {Switches the paragraph alignment to centred. (See
\novices{declaration}.)}
 {}

\defgcs{singlespacing}
 {}
 {\sty{setspace} package}
 {Switches to single line-spacing.}
 {}

\defgcs{onehalfspacing}
 {}
 {\sty{setspace} package}
 {Switches to one-half line-spacing.}
 {}

\defgcs{doublespacing}
 {}
 {\sty{setspace} package}
 {Switches to double line-spacing.}
 {}

\defgcs{hspace}
 {\marg{\meta{length}}}
 {\LaTeX\ Kernel}
 {Inserts a horizontal gap of the given width.}
 {\relax
   \BeginArgList
    \csentryargitem{length} The width of the horizontal gap.
   \EndArgList
 }

\defgcs{vspace}
 {\marg{\meta{length}}}
 {\LaTeX\ Kernel}
 {Inserts a vertical gap of the given height.}
 {\relax
   \BeginArgList
    \csentryargitem{length} The height of the vertical gap.
   \EndArgList
 }

\defgcs{hfill}
 {}
 {\LaTeX\ Kernel}
 {Inserts a horizontal space that will expand to fit the available
width.}
 {}

\defgcs{vfill}
 {}
 {\LaTeX\ Kernel}
 {Inserts a vertical space that will expand to fit the available
  height.}
 {}

\defgenv{verbatim}
 {}
 {\LaTeX\ Kernel}
 {Typesets the contents of the environment as is. (Can't be used in
  the argument of a command.)}
 {}

\defgcs{lstset}
 {\marg{\meta{options}}}
 {\sty{listings} package}
 {Sets options used by the \sty{listings} package.}
 {\relax
   \BeginArgList
     \csentryargitem{options} comma-separated list of
      \meta{key}=\meta{value} options.
   \EndArgList
 }

\defgenv{lstlisting}
 {\oarg{\meta{options}}}
 {\sty{listings} package}
 {Typesets the contents of the environment as displayed code.}
 {\relax
   \BeginArgList
     \csentryargitem{options} comma-separated list of
      \meta{key}=\meta{value} options.
   \EndArgList
 }

\defgcs{lstinline}
 {\oarg{\meta{opts}}\meta{char}\meta{code}\meta{char}}
 {\sty{listings} package}
 {Typesets \meta{code} as an inline code snippet.}
 {\relax
   \BeginArgList
     \csentryargitem{opts} comma-separated list of
      \meta{key}=\meta{value} options.
     \csentryargitem{char} single character delimiter that doesn't
occur in \meta{code}
     \csentryargitem{code} the code snippet
   \EndArgList
 }

\defgcs{lstinputlisting}
 {\marg{\meta{options}}\marg{\meta{filename}}}
 {\sty{listings} package}
 {Reads in \meta{filename} and typesets the contents as displayed
  code.}
 {\relax
   \BeginArgList
     \csentryargitem{options} comma-separated list of
      \meta{key}=\meta{value} options.
     \csentryargitem{filename} the name of the file to input.
   \EndArgList
 }

\defgcs{lstlistoflistings}
 {}
 {\sty{listings} package}
 {Prints a list of listings for those listings with the caption set.}
 {}

\defgcs{sqrt}
 {\oarg{\meta{order}}\marg{\meta{operand}}}
 {\LaTeX\ Kernel (Math Mode)}
 {Displays a root. (See \novices{roots}.)}
 {\relax
    \BeginArgList
     \csentryargitem{order} The order of the root. (If omitted, a
       square root).
     \csentryargitem{operand} The operand.
    \EndArgList
 }

\defgidxactivecharcs
 {doublequote}
 {\doublequotesym}
 {\marg{\meta{c}}}
 {\LaTeX\ Kernel}
 {\relax
   Umlaut over \meta{c}. Example:
   \quotecs\marg{o} produces \oumlaut. (See
\novices[sec:chars]{symbols}.)\relax
 }
 {\relax
   \BeginArgList
    \csentryargitem{c} The character that requires an umlaut
     over it.
   \EndArgList
 }

\defgchar
 {quotedblright}
 {\quotedblrightcs}
 {}
 {\LaTeX\ Kernel}
 {Closing double quote \textquotedblright\ symbol in text mode
  or double prime \mathdoubleprime\ in math mode. 
(See \novices[sec:chars]{symbols}.)}
 {}

\defgchar
 {quotedblleft}
 {\quotedblleftcs}
 {}
 {\LaTeX\ Kernel}
 {Open double quote \textquotedblleft\ symbol. (See
\novices[sec:chars]{symbols}.)}
 {}

\defgchar
 {quoteright}
 {\quoterightcs}
 {}
 {\LaTeX\ Kernel}
 {Closing quote or apostrophe \textquoteright\ symbol in text mode 
  or prime symbol \mathsingleprime\ in math mode. 
(See \novices[sec:chars]{symbols}.)}
 {}

\defgsymcs[hyphen]{\dashcs}
 {}
 {\LaTeX\ Kernel}
 {\nopostdesc}
 {}

\defgchildsymcs
 {hyphen-discretionary}
 {hyphen}
 {\dashcs}
 {Outside \nxgls{env-tabbing} environment inserts a discretionary hyphen}

\defgchildsymcs
 {hyphen-tab}
 {hyphen}
 {\dashcs}
 {Inside \nxgls{env-tabbing} environment shifts the left border by one tab stop}

\defgsymcs[equals]{\equalsym}
 {}
 {\LaTeX\ Kernel}
 {\nopostdesc}
 {}

\defgchildsymcs
 {macron}
 {equals}
 {\equalsym}
 {Outside \nxgls{env-tabbing} environment puts a macron accent over the
  following character\relax
 }

\defgchildsymcs
 {tabstop}
 {equals}
 {\equalsym}
 {Inside \nxgls{env-tabbing} environment sets a tab-stop.\relax
 }

\defgsymcs[plus]{\pluscs}
 {}
 {\nxgls{env-tabbing} environment}
 {Shifts the left border by one tab stop to the right.}
 {}

\defgsymcs[lessthan]{\lesssym}
 {}
 {\nxgls{env-tabbing} environment}
 {Jumps to the next tab stop.}
 {}

\defgsymcs[greaterthan]{\greatersym}
 {}
 {\nxgls{env-tabbing} environment}
 {Jumps to the previous tab stop.}
 {}

\defgenv{tabbing}
 {}
 {\LaTeX\ Kernel}
 {Allows you to define tab stops from the left margin.}
 {}

\defgcs{kill}
 {}
 {\nxgls{env-tabbing} environment}
 {Sets the tab stops defined in the line but won't typeset the
  actual line.}
 {}

\defgcs{a}
 {\meta{accent symbol}\marg{\meta{character}}}
 {\LaTeX\ Kernel}
 {Used in the \nxgls{env-tabbing} environment to create accented characters.}
 {\relax
    \BeginArgList
      \csentryargitem{accent symbol} the symbol you would usually
        use in the normal accent command.
      \csentryargitem{character} the character that requires the
        accent
    \EndArgList
 }

\defgcs{newtheorem}
 {\marg{\meta{name}}\oarg{\meta{counter}}\reportlinebreak\booklinebreak\marg{\meta{title}}\oarg{\meta{outer counter}}}
 {\LaTeX\ Kernel}
 {Defines a new theorem-like environment. The optional arguments are
  mutually exclusive. Some packages, such as \sty{ntheorem}
  and \sty{amsthm}, redefine this command to have a starred variant 
  that defines unnumbered theorem-like environments.}
 {\relax
   \BeginArgList
     \csentryargitem{name} the name of the new environment
     \csentryargitem{counter} the counter to be used by the new environment
     \csentryargitem{title} the title for the new environment
     \csentryargitem{outer counter} the parent counter
   \EndArgList
 }

\defgcs{listtheorems}
 {\marg{\meta{list}}}
 {\sty{ntheorem} package}
 {Inserts a list of theorems for the theorem types given in \meta{list}.}
 {\relax
    \BeginArgList
      \csentryargitem{list} a comma-separated list of theorem types.
    \EndArgList
 }

\defgcs{theoremstyle}
 {\marg{\meta{style name}}}
 {\sty{ntheorem} and \sty{amsthm} packages}
 {Changes the current theorem style to \meta{style name}.}
 {\relax
    \BeginArgList
      \csentryargitem{style name} the name of the theorem style
    \EndArgList
 }

\defgcs{theoremheaderfont}
 {\marg{\meta{declarations}}}
 {\sty{ntheorem} package}
 {Changes the current theorem header fonts to \meta{declarations}.}
 {\relax
    \BeginArgList
      \csentryargitem{declarations} font declarations (such as
        \nxgls{normalfont})
    \EndArgList
 }

\defgcs{theorembodyfont}
 {\marg{\meta{declarations}}}
 {\sty{ntheorem} package}
 {Changes the current theorem body fonts to \meta{declarations}.}
 {\relax
    \BeginArgList
      \csentryargitem{declarations} font declarations (such as
        \nxgls{normalfont})
    \EndArgList
 }

\defgcs{theoremnumbering}
 {\marg{\meta{style}}}
 {\sty{ntheorem} package}
 {Changes the current theorem numbering style to \meta{style}.}
 {\relax
    \BeginArgList
      \csentryargitem{style} counter style, such as \texttt{arabic}
        or \texttt{roman}.
    \EndArgList
 }

\defgcs{sim}
 {}
 {\LaTeX\ Kernel (Math Mode)}
 {Relational \mathtilde\ symbol.}
 {}

\defgcs{vee}
 {}
 {\LaTeX\ Kernel (Math Mode)}
 {Operator \mathvee\ symbol. (See \novices{mathssym}.)}
 {}

\defgcs{wedge}
 {}
 {\LaTeX\ Kernel (Math Mode)}
 {Operator \mathwedge\ symbol. (See \novices{mathssym}.)}
 {}

\defgcs{equiv}
 {}
 {\LaTeX\ Kernel (Math Mode)}
 {Relational \mathequiv\ symbol. (See \novices{mathssym}.)}
 {}

\defgcs{sum}
 {}
 {\LaTeX\ Kernel (Math Mode)}
 {Summation \mathsum\ symbol. (See \novices{mathssym}.)}
 {}

\defgcs{vec}
 {\marg{\meta{c}}}
 {\LaTeX\ Kernel (Math Mode)}
 {Typesets its argument as a vector. (See \novices[sec:vec]{vectors}.)}
 {\relax
   \BeginArgList
    \csentryargitem{c} The character or symbol that represents a vector.
   \EndArgList
 }

\defgcs{lvert}
 {}
 {\sty{amsmath} (Math Mode)}
 {Left vertical bar \mathlvert\ delimiter. (See \novices{delimiters}.)}
 {}

\defgcs{rvert}
 {}
 {\sty{amsmath} (Math Mode)}
 {Right vertical bar \mathrvert\ delimiter. (See \novices{delimiters}.)}
 {}

\defgcs{frac}
 {\marg{\meta{numerator}}\marg{\meta{denominator}}}
 {\LaTeX\ Kernel (Math Mode)}
 {Displays a fraction. (See \novices{fractions}.)}
 {\relax
    \BeginArgList
     \csentryargitem{numerator} The numerator (above the line).
     \csentryargitem{denominator} The denominator (below the line).
    \EndArgList
 }

\defgcs{parindent}
 {}
 {\LaTeX\ Kernel}
 {A length register that stores the indentation at the start of
paragraphs. (See \novices{length}.)}
 {}

\defgcs{newline}
 {}
 {\LaTeX\ Kernel}
 {Forces a line break.}
 {}

\defgcs{epsilon}
 {}
 {\LaTeX\ Kernel (Math Mode)}
 {Greek lower case epsilon \mathepsilon. (See \novices{greek}.)}
 {}

\defgenv{algorithm}
 {\oarg{\meta{placement}}}
 {\sty{algorithm2e} package}
 {A floating environment for typesetting algorithms.}
 {\relax
    \BeginArgList
      \csentryargitem{placement} floating placement specifier
    \EndArgList
 }

\defgenv{algorithm2e}
 {\oarg{\meta{placement}}}
 {\sty{algorithm2e} package}
 {Replacement for \nxglsni{env-algorithm} when used with the 
   \optfmt{algo2e} package option.}
 {\relax
    \BeginArgList
      \csentryargitem{placement} floating placement specifier
    \EndArgList
 }

\defgenv{figure}
 {\oarg{\meta{placement}}}
 {Most classes that define sectioning commands}
 {\relax
   Floats the contents to the nearest location according to the
   preferred placement options, if possible. Within the environment,
   \nxglsi{caption} may be used one or more times, as required. 
   (See \novices{figures}.)
 }
 {\relax
   \BeginArgList
    \csentryargitem{placement} The preferred placement.
   \EndArgList
 }

\defgenv{table}
 {\oarg{\meta{placement}}}
 {Most classes that define sectioning commands}
 {\relax
   Floats the contents to the nearest location according to the
   preferred placement options, if possible. Within the environment,
   \nxglsi{caption} may be used one or more times, as required. 
   (See \novices{tables}.)}
 {\relax
   \BeginArgList
    \csentryargitem{placement} The preferred placement.
   \EndArgList
 }

\defgsymcs[space.semicolon]{\semicolonsym}
 {}
 {\LaTeX\ Kernel (Math Mode)}
 {Thick space.}
 {}

\defgsymcs[algo2e.semicolon]{\semicolonsym}
 {}
 {\sty{algorithm2e} package}
 {When used in the body of one of the environments defined by
   \sty{algorithm2e}, such as \nxglsni{env-algorithm}, marks the end of
   the line. Outside those environments, this is a math spacing
   command.}
 {}

\defgcs{DontPrintSemicolon}
 {}
 {\sty{algorithm2e} package}
 {Switches off the end of line semi-colon. (See also
  \nxgls{PrintSemicolon}.)}
 {}

\defgcs{PrintSemicolon}
 {}
 {\sty{algorithm2e} package}
 {Switches on the end of line semi-colon. (See also
  \nxgls{DontPrintSemicolon}.)}
 {}

\defgcs{For}
 {\marg{\meta{condition}}\marg{\meta{body}}}
 {\sty{algorithm2e} package}
 {For use in algorithm-like environments to indicate a for-loop}
 {\relax
   \BeginArgList
    \csentryargitem{condition} the for-loop condition.
    \csentryargitem{body} the for-loop body.
   \EndArgList
 }

\defgcs{While}
 {\marg{\meta{condition}}\marg{\meta{body}}}
 {\sty{algorithm2e} package}
 {For use in algorithm-like environments to indicate a while-loop}
 {\relax
   \BeginArgList
    \csentryargitem{condition} the while-loop condition.
    \csentryargitem{body} the while-loop body.
   \EndArgList
 }

\defgcs{If}
 {\marg{\meta{condition}}\marg{\meta{block}}}
 {\sty{algorithm2e} package}
 {For use in algorithm-like environments to indicate an if-statement}
 {\relax
   \BeginArgList
    \csentryargitem{condition} the if-statement condition.
    \csentryargitem{block} the true-part of the if-statement.
   \EndArgList
 }

\defgcs{uIf}
 {\marg{\meta{condition}}\marg{\meta{block}}}
 {\sty{algorithm2e} package}
 {Like \nxgls{If} but doesn't put \dq{end} after \meta{block}}
 {\relax
   \BeginArgList
    \csentryargitem{condition} the if-statement condition.
    \csentryargitem{block} the true-part of the if-statement.
   \EndArgList
 }

\defgcs{ElseIf}
 {\marg{\meta{block}}}
 {\sty{algorithm2e} package}
 {For use in algorithm-like environments to indicate an elseif-block}
 {\relax
   \BeginArgList
    \csentryargitem{block} the elseif-block.
   \EndArgList
 }

\defgcs{uElseIf}
 {\marg{\meta{condition}}\marg{\meta{block}}}
 {\sty{algorithm2e} package}
 {Like \nxgls{ElseIf} but doesn't put \dq{end} after \meta{block}}
 {\relax
   \BeginArgList
    \csentryargitem{block} the elseif-block.
   \EndArgList
 }

\defgcs{Else}
 {\marg{\meta{block}}}
 {\sty{algorithm2e} package}
 {For use in algorithm-like environments to indicate an else-block}
 {\relax
   \BeginArgList
    \csentryargitem{block} the else-block.
   \EndArgList
 }

\defgcs{KwIn}
 {\marg{\meta{text}}}
 {\sty{algorithm2e} package}
 {For use in algorithm-like environments to indicate the algorithm
input}
 {\relax
   \BeginArgList
    \csentryargitem{text} the algorithm input
   \EndArgList
 }

\defgcs{KwOut}
 {\marg{\meta{text}}}
 {\sty{algorithm2e} package}
 {For use in algorithm-like environments to indicate the algorithm
output}
 {\relax
   \BeginArgList
    \csentryargitem{text} the algorithm output
   \EndArgList
 }

\defgcs{KwData}
 {\marg{\meta{text}}}
 {\sty{algorithm2e} package}
 {For use in algorithm-like environments to indicate the algorithm
input data}
 {\relax
   \BeginArgList
    \csentryargitem{text} the algorithm input data.
   \EndArgList
 }

\defgcs{KwResult}
 {\marg{\meta{text}}}
 {\sty{algorithm2e} package}
 {For use in algorithm-like environments to indicate the algorithm
output}
 {\relax
   \BeginArgList
    \csentryargitem{text} the algorithm output.
   \EndArgList
 }

\defgcs{KwTo}
 {}
 {\sty{algorithm2e} package}
 {For use in algorithm-like environments to indicate ``to'' keyword}
 {\relax
 }

\defgcs{KwRet}
 {\marg{\meta{value}}}
 {\sty{algorithm2e} package}
 {For use in algorithm-like environments to indicate a value
returned}
 {\relax
   \BeginArgList
    \csentryargitem{value} the return value.
   \EndArgList
 }

\defgcs{Return}
 {\marg{\meta{value}}}
 {\sty{algorithm2e} package}
 {For use in algorithm-like environments to indicate a value
returned}
 {\relax
   \BeginArgList
    \csentryargitem{value} the return value.
   \EndArgList
 }

\defgcs{leftarrow}
 {}
 {\LaTeX\ Kernel (Math Mode)}
 {Left arrow \mathleftarrow. (See \novices{mathssym}.)}
 {}

\defgcs{num}
 {\marg{\meta{number}}}
 {\sty{siunitx} package}
 {Typesets \meta{number} with appropriate spacing.}
 {\relax
   \BeginArgList
     \csentryargitem{number} the number to typeset
   \EndArgList
 }

\defgcs{ang}
 {\marg{\meta{angle}}}
 {\sty{siunitx} package}
 {Typesets \meta{angle} where \meta{angle} is a single number or
  three semi-colon separated values.}
 {\relax
   \BeginArgList
     \csentryargitem{angle} the angle (degrees) to typeset or
      \meta{degrees}\texttt{;}\meta{minutes}\texttt{;}\meta{seconds}
   \EndArgList
 }

\defgcs{si}
 {\marg{\meta{unit}}}
 {\sty{siunitx} package}
 {Typesets the given unit.}
 {\relax
   \BeginArgList
     \csentryargitem{unit} the unit to typeset
   \EndArgList
 }

\defgcs{SI}
 {\marg{\meta{number}}\marg{\meta{unit}}}
 {\sty{siunitx} package}
 {Typesets a number and unit, combining the functionality of
  \nxgls{num} and \nxglsni{si}.}
 {\relax
   \BeginArgList
     \csentryargitem{unit} the unit to typeset
   \EndArgList
 }

\defgcs{metre}
 {}
 {\sty{siunitx} package}
 {Indicates the metre unit for use in commands like \nxglsni{si}.}
 {}

\defgcs{second}
 {}
 {\sty{siunitx} package}
 {Indicates the second unit for use in commands like \nxglsni{si}.}
 {}

\defgcs{per}
 {}
 {\sty{siunitx} package}
 {Indicates a divider in commands like \nxglsni{si}.}
 {}

\defgcs{square}
 {\meta{unit}}
 {\sty{siunitx} package}
 {Indicates a squared unit in commands like \nxglsni{si}.}
 {}

\defgcs{squared}
 {}
 {\sty{siunitx} package}
 {Indicates a squared term in commands like \nxglsni{si} after a unit command such as \nxgls{metre}.}
 {}

\defgcs{kilo}
 {}
 {\sty{siunitx} package}
 {Indicates a kilo multiplier in commands like \nxglsni{si}.}
 {}

\defgcs{gram}
 {}
 {\sty{siunitx} package}
 {Indicates a gram in commands like \nxglsni{si}.}
 {}

\defgcs{MakeUppercase}
 {\marg{\meta{text}}}
 {\LaTeX\ Kernel}
 {Converts its argument to upper case.}
 {\relax
    \BeginArgList
       \csentryargitem{text} the text to be converted.
    \EndArgList
 }

\defgidxactivechar
 {endash}
 {\enDashcs}
 {}
 {\LaTeX\ Kernel}
 {En-dash \textendash\ symbol.  (Normally used for number ranges. 
See \novices[sec:chars]{symbols}.)}
 {}

\defgcs{bibliographystyle}
 {\marg{\meta{style-name}}}
 {\LaTeX\ Kernel}
 {Specifies the bibliography style to be used by bibtex.}
 {\relax
    \BeginArgList
      \csentryargitem{style-name} the name of the bibliography style
(corresponds to a file called \meta{style-name}\texttt{.bst}).
    \EndArgList
 }

\defgcs{bibliography}
 {\marg{\meta{bib list}}}
 {\LaTeX\ Kernel}
 {Inputs the \texttt{.bbl} file (if it exists) and identifies the
  name(s) of the bibliography database files where the citations are
  defined.}
 {\relax
   \BeginArgList
     \csentryargitem{bib list} a comma-separated list of database names
     (without the \texttt{.bib} extension).
   \EndArgList
 }

\defgcs{citep}
 {\oarg{\meta{pre}}\oarg{\meta{post}}\marg{\meta{key}}}
 {\sty{natbib} package}
 {Parenthetical citation.}
 {\relax
    \BeginArgList
      \csentryargitem{pre} prefix text
      \csentryargitem{post} suffix text
      \csentryargitem{key} key identifying required citation or
       comma-separated list of keys.
    \EndArgList
 }

\defgcs{citet}
 {\oarg{\meta{pre}}\oarg{\meta{post}}\marg{\meta{key}}}
 {\sty{natbib} package}
 {Textual citation.}
 {\relax
    \BeginArgList
      \csentryargitem{pre} prefix text
      \csentryargitem{post} suffix text
      \csentryargitem{key} key identifying required citation or
       comma-separated list of keys.
    \EndArgList
 }

\defgcs{addbibresource}
 {\oarg{\meta{options}}\marg{\meta{resource}}}
 {\sty{biblatex} package}
 {Adds a resource, such as a \texttt{.bib} file}
 {\relax
    \BeginArgList
      \csentryargitem{options} comma-separated list of
        \meta{key}=\meta{value} options
      \csentryargitem{resource} the file name or URL of the resource
    \EndArgList
 }

\defgcs{printbibliography}
 {\oarg{\meta{options}}}
 {\isty{biblatex} package}
 {Prints the bibliography.}
 {\relax
   \BeginArgList
     \csentryargitem{options} comma-separated list of
       \meta{key}=\meta{value} options
   \EndArgList
 }

\defgidxactivecharcs
 {at}
 {\atsym}
 {}
 {\LaTeX\ Kernel}
 {Used when a sentence ends with a capital letter.
   This command should be placed after the letter and before the
   punctuation mark. (See \novices{intersentencespacing}.)}
 {}

\defgidxactivechar
 {atchar}
 {\atsym}
 {}
 {}
 {\relax
   Used in the argument of \nxglsi{index} to separate the sort key
   from the term being indexed.\relax
 }
 {}
  
\defgcs{AE}
 {}
 {\LaTeX\ Kernel}
 {\protect\AE\ ligature.}
 {}

\defgenv{proof}
 {\oarg{\meta{title}}}
 {\sty{amsthm} package}
 {Typesets its contents as a proof.}
 {\relax
   \BeginArgList
    \csentryargitem{title} replacement for the default title.
   \EndArgList
 }

\defgenv{Proof}
 {\oarg{\meta{title}}}
 {\sty{ntheorem} package with \optfmt{standard} package option}
 {Typesets its contents as a proof.}
 {\relax
   \BeginArgList
    \csentryargitem{title} if supplied this is appended in
parentheses to the proof title.
   \EndArgList
 }

\defgcs{qedhere}
 {}
 {\sty{amsthm} package}
 {Overrides default location of QED marker in \nxglsni{env-proof}
   environment.}
 {}

\defgcs{qedsymbol}
 {}
 {\sty{amsthm} package}
 {QED symbol used at the end of the \nxglsni{env-proof} environment.}
 {}

\defgcs{newtheoremstyle}
 {\marg{\meta{name}}\marg{\meta{space 
above}}\reportlinebreak\marg{\meta{space below}}\reportlinebreak\marg{\meta{body 
font}}\reportlinebreak\marg{\meta{indent}}\reportlinebreak\marg{\meta{head
font}}\reportlinebreak\marg{\meta{post head punctuation}}\reportlinebreak\marg{\meta{post head
space}}\reportlinebreak\marg{\meta{head spec}}}
 {\sty{amsthm} package}
 {Defines a new theorem style called \meta{name}.}
 {\relax
    \BeginArgList
      \csentryargitem{name} the name of the new theorem style
      \csentryargitem{space above} the amount of space above the
theorem-like environment
      \csentryargitem{space below} the amount of space below the
theorem-like environment
      \csentryargitem{body font} the font to use in the main theorem
body
      \csentryargitem{indent} the amount of indentation (empty means
no indent or use \nxgls{parindent} for normal paragraph indentation)
      \csentryargitem{head font} the font to use in the theorem
header
      \csentryargitem{post head punctuation} the punctuation to use
after the theorem head
      \csentryargitem{post head space} the space to put
after the theorem head (use \marg{ } for normal interword space or
\nxgls{newline} for a linebreak)
      \csentryargitem{head spec} the theorem head spec
    \EndArgList
 }

\defgidxactivechar
 {index.barchar}
 {\vbarsym}
 {}
 {}
 {When used in \nxglsi{index}, this symbol indicates that the rest
of the argument list is to be used as the
encapsulating command for the page number.}
 {}

\defgidxactivechar
 {makeindex.exclamchar}
 {\exclamsym}
 {}
 {}
 {\nxiappname{makeindex} sublevel special character}
 {}


\defgidxactivechar
 {makeindex.doublequote}
 {\doublequotesym}
 {}
 {}
 {\nxiappname{makeindex} escape special character}
 {}

\defgcs{mathcal}
 {\marg{\meta{maths}}}
 {\LaTeX\ Kernel (Math Mode)}
 {Typesets its argument in the maths calligraphic font.
  Example: 
  \nxglsni{dollarchar}\protect\cmdname{mathcal}\marg{S}\nxglsni{dollarchar} 
  produces
 \mathcalS. (See \novices{mathsfonts}.)}
 {\relax
   \BeginArgList
    \csentryargitem{maths} The maths to be displayed in a calligraphic
      font.
   \EndArgList
 }

\defgcs{ensuremath}
 {\marg{\meta{maths}}}
 {\LaTeX\ Kernel}
 {Ensures that its argument is displayed in maths mode. (If it's
already in maths mode, it just displays its argument, but if it's
not already in maths mode, it will typeset its argument in in-line
maths mode.) This command is usually only used in definitions, such as in
\nxgls{newglossaryentry}, where it may be used in either text or
math mode.}
 {\relax
   \BeginArgList
     \csentryargitem{maths} The maths to be displayed.
   \EndArgList
 }

\defgsymcs[acute]{\quoterightcs}
 {\marg{\meta{c}}}
 {\LaTeX\ Kernel}
 {Acute accent over \meta{c}. Example:
   \cmdname{'}\marg{o} produces \protect\'{o}. (See
\novices[sec:chars]{symbols}.)\relax
 }
 {\relax
   \BeginArgList
    \csentryargitem{c} The character that requires an acute
     accent over it.
   \EndArgList
 }

\defgcs{newgidx}
 {\marg{\meta{label}}\marg{\meta{title}}}
 {\sty{datagidx} package}
 {Defines a new index (or glossary) database.}
 {\relax
   \BeginArgList
     \csentryargitem{label} a label uniquely identifying this
database.
     \csentryargitem{title} the title to be used when this index (or
glossary) is displayed.
   \EndArgList
 }

\defgcs{newterm}
 {\oarg{\meta{options}}\marg{\meta{name}}}
 {\sty{datagidx} package}
 {Defines a new index or glossary term.}
 {\relax
   \BeginArgList
     \csentryargitem{options} a comma-separated list of
\meta{key}=\meta{value} options.
     \csentryargitem{name} the term.
   \EndArgList
 }

\defgcs{newacro}
 {\oarg{\meta{options}}\marg{\meta{short}}\marg{\meta{long}}}
 {\sty{datagidx} package}
 {Defines a new acronym.}
 {\relax
   \BeginArgList
     \csentryargitem{options} a comma-separated list of
\meta{key}=\meta{value} options (same as \nxgls{newterm}).
     \csentryargitem{short} the abbreviation.
     \csentryargitem{long} the long form.
   \EndArgList
 }

\defgcs{DTLgidxSetDefaultDB}
 {\marg{\meta{database label}}}
 {\sty{datagidx} package}
 {Sets the default indexing database.}
 {\relax
   \BeginArgList
     \csentryargitem{database label} the label uniquely identifying
the required database.
   \EndArgList
 }

\defgcs{acronymfont}
 {\marg{\meta{text}}}
 {\sty{glossaries} and \sty{datagidx} packages}
 {Font used to display acronyms.}
 {\relax
   \BeginArgList
    \csentryargitem{text} the acronym.
   \EndArgList
 }

\defgcs{DTLgidxAcrStyle}
 {\marg{\meta{long}}\marg{\meta{short}}}
 {\sty{datagidx} package}
 {Formats the long and short form of an acronym.}
 {\relax
    \BeginArgList
      \csentryargitem{long} the long form of the acronym.
      \csentryargitem{short} the abbreviation.
    \EndArgList
 }
\defgcs{MakeTextUppercase}
 {\marg{\meta{text}}}
 {\sty{textcase} package}
 {Converts \meta{text} to uppercase.}
 {\relax
    \BeginArgList
      \csentryargitem{text} the text that needs to be uppercased.
    \EndArgList
 }

\defgcs{capitalisewords}
 {\marg{\meta{text}}}
 {\sty{mfirstuc} package}
 {Converts the initial letter of each word in \meta{text} to
uppercase.}
 {\relax
   \BeginArgList
     \csentryargitem{text} the text that needs capitalising.
   \EndArgList
 }

\defgcs[][datagidx.gls]{gls}
 {\marg{\oarg{\meta{format}}\meta{label}}}
 {\sty{datagidx} package}
 {Displays a glossary or index term.}
 {\relax
   \BeginArgList
    \csentryargitem{format} the name of the formatting command
\emph{without} the initial backslash to be used for this location.
    \csentryargitem{label} the label uniquely identifying the term.
   \EndArgList
 }

\defgcs[][datagidx.glspl]{glspl}
 {\marg{\oarg{\meta{format}}\meta{label}}}
 {\sty{datagidx} package}
 {Displays the plural form of a~glossary or index term.}
 {\relax
   \BeginArgList
    \csentryargitem{format} the name of the formatting command
\emph{without} the initial backslash to be used for this location.
    \csentryargitem{label} the label uniquely identifying the term.
   \EndArgList
 }

\defgcs[][datagidx.Gls]{Gls}
 {\marg{\oarg{\meta{format}}\meta{label}}}
 {\sty{datagidx} package}
 {Displays a glossary or index term with the first letter converted
to uppercase.}
 {\relax
   \BeginArgList
    \csentryargitem{format} the name of the formatting command
\emph{without} the initial backslash to be used for this location.
    \csentryargitem{label} the label uniquely identifying the term.
   \EndArgList
 }

\defgcs[][datagidx.Glspl]{Glspl}
 {\marg{\oarg{\meta{format}}\meta{label}}}
 {\sty{datagidx} package}
 {Displays the plural form of a~glossary or index term with the first
letter converted to uppercase.}
 {\relax
   \BeginArgList
    \csentryargitem{format} the name of the formatting command
\emph{without} the initial backslash to be used for this location.
    \csentryargitem{label} the label uniquely identifying the term.
   \EndArgList
 }

\defgcs[][datagidx.acr]{acr}
 {\marg{\oarg{\meta{format}}\meta{label}}}
 {\sty{datagidx} package}
 {Displays an acronym. On first use the full form is displayed. On
subsequent use only the short form is displayed.}
 {\relax
   \BeginArgList
    \csentryargitem{format} the name of the formatting command
\emph{without} the initial backslash to be used for this location.
    \csentryargitem{label} the label uniquely identifying the term.
   \EndArgList
 }

\defgcs[][datagidx.Acr]{Acr}
 {\marg{\oarg{\meta{format}}\meta{label}}}
 {\sty{datagidx} package}
 {As \nxglsni{datagidx.acr} but the first letter is converted to
uppercase.}
 {\relax
   \BeginArgList
    \csentryargitem{format} the name of the formatting command
\emph{without} the initial backslash to be used for this location.
    \csentryargitem{label} the label uniquely identifying the term.
   \EndArgList
 }

\defgcs[][datagidx.acrpl]{acrpl}
 {\marg{\oarg{\meta{format}}\meta{label}}}
 {\sty{datagidx} package}
 {Displays the plural of an acronym. On first use the full form is displayed. On
subsequent use only the short form is displayed.}
 {\relax
   \BeginArgList
    \csentryargitem{format} the name of the formatting command
\emph{without} the initial backslash to be used for this location.
    \csentryargitem{label} the label uniquely identifying the term.
   \EndArgList
 }

\defgcs[][datagidx.Acrpl]{Acrpl}
 {\marg{\oarg{\meta{format}}\meta{label}}}
 {\sty{datagidx} package}
 {As \nxglsni{datagidx.acrpl} but the first letter is converted to
uppercase.}
 {\relax
   \BeginArgList
    \csentryargitem{format} the name of the formatting command
\emph{without} the initial backslash to be used for this location.
    \csentryargitem{label} the label uniquely identifying the term.
   \EndArgList
 }

\defgcs{printterms}
 {\oarg{\meta{options}}}
 {\sty{datagidx} package}
 {Displays the index or glossary or list of acronyms.}
 {\relax
   \BeginArgList
     \csentryargitem{options} a comma-separated list of options.
   \EndArgList
 }

\defgcs[][datagidx.glsadd]{glsadd}
 {\marg{\meta{label}}}
 {\sty{datagidx} package}
 {Adds the given entry to the glossary or index without displaying any text.}
 {\relax
   \BeginArgList
     \csentryargitem{label} the label uniquely identifying the
entry.
   \EndArgList
 }

\defgcs[][datagidx.glsaddall]{glsaddall}
 {\marg{\meta{database name}}}
 {\sty{datagidx} package}
 {Adds all the defined entries in the named database without displaying any text.}
 {\relax
   \BeginArgList
     \csentryargitem{database name} the label uniquely identifying
the required database.
   \EndArgList
 }

\defgcs{afterpage}
 {\marg{\meta{code}}}
 {\sty{afterpage} package}
 {Indicates code that should be implemented at the next page break.}
 {\relax
   \BeginArgList
    \csentryargitem{code} the relevant code.
   \EndArgList
 }

\documentclass[onedown-examples]{subfiles}

\begin{document}
\setdefaults{bidfirst=S}

Here is a diagram with four hands. We get rid of the past by calling
|\newgame|. Then we define our hands and set the first bid column to
\south and we show only one pair of bidders. Because the bidding table
does not fit on the same line, it appears below the compass.
(In fact it actually would fit on the line. But to illustrate the point we
used |\handskip| to widen the diagram a bit).
Note that the token |-| in the |bidding| environment suppresses the output
of it. Only when calling |\showAll+| (with a |+| token), the saved bidding
diagram is printed together with the card diagram.

\vfill
\begin{HBox}{not fitting on the line\hfill File: \FileName}
\newgame
\handskip{2em}
\northhand{KJ82}{KQ9}{J32}{JT8}
\southhand{AQ953}{A85}{Q84}{K6}
\easthand{T6}{764}{KT9}{Q5432}
\westhand{74}{JT32}{A765}{A97}
\setdefaults{bidfirst=S}
\begin{biddingpair}-
1S & 4S \\
p &  \\
\end{biddingpair}
\showAll+
\end{HBox}
\vfill
Here the hands and the bidding are the same as previous, but the complete diagram
(biddingtable inclusive) is centered and we use the short notation for the
players in the bidding table (by using the token |!|). 
Being shorter now, the bidding table does fit on the same line.
\vfill
\begin{HBox}{centered and fitting\hfill File: \FileName}
\setdefaults{bidfirst=S}
\begin{biddingpair}!-
1S & 4S \\
p &  \\
\end{biddingpair}
\showAll*+
\end{HBox}
\vfill
\needspace{8\baselineskip}
The same deal, but now we show all bidders in the bidding table with the long notation.
Again it is to wide to fit on the line.
\vfill
\begin{HBox}{left aligned\hfill File: \FileName}
\begin{bidding}-(\explainit{15-17\HLP})
p & p  & P & 1S \\
p & 2C & {\Pass*!} & 2N\markit \\
p & 3N & {\Allpass!} \\
\end{bidding}
\showAll+
\end{HBox}
\vfill
Finally the same, but the diagram is centered. We also set the extra
handskip to 0 again.
\vfill
\begin{HBox}{centered\hfill File: \FileName}
\handskip{0em}
\showAll*+
\end{HBox}
\vfill
\end{document}
\endinput


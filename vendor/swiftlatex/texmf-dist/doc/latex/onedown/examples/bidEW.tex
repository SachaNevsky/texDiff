\documentclass[onedown-examples]{subfiles}

\begin{document}
\vfill
In the previous example we changed the gamefont to scriptsize, which is
still in effect here, as you can see in this example. You wonder why the
bidding is shown for \north and \south, but the hands are those of \east and
\west? Well, if you say |\showEW| you get the \eastwest hands. And if you
say |\setdefaults{bidfirst=S}|, then the first column is of \south{}! (BTW:
if you want to get \south{}! as output, you must write |\south{}!|, then
|\south!| and even |\south !| will regard the |!| as a token to |\south|
and will produce \south !).

We reset the gamefont before we draw the 2nd diagram. Please note that we
have to redefine the bidding. The saved version from the 1st diagram will
not change its font!

\vfill
\begin{HBox}{bidEWpair\hfill File: \FileName}
\northhand{KJ82}{KQ9}{J}{AQJT8}
\southhand{AQT95}{A52}{864}{K6}
\easthand{63}{J764}{KQT9}{953}
\westhand{74}{T83}{A7532}{742}
\setdefaults{bidfirst=S}
\begin{biddingpair}!-(\explainit{splinter}
  \explainit{control})
1S        & 4D\markit \\
4H\markit & 6S \\
\end{biddingpair}
\showEW+
\gamefont{\sffamily\bfseries\normalsize}
\par\vspace{1\baselineskip}
\showEW+
\begin{biddingpair}!-(\explainit{splinter}
  \explainit{control})
1S        & 4D\markit \\
4H\markit & 6S \\
\end{biddingpair}
\par\vspace{1\baselineskip}
\showEW+
\end{HBox}
\vfill
\begin{HBox}{bidEWpair centered\hfill File: \FileName}
\showAll+

\showEW*+
\end{HBox}
\vfill
\end{document}
\endinput


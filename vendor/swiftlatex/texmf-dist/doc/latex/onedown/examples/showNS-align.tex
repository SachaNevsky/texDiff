\documentclass[onedown-examples]{subfiles}
\begin{document}
These examples show the alignment of diagrams.
The alignment is given as an optional argument to the |showXX| commands. It
bears the usual name for alignments: \textbf{t} for top, \textbf{b} for
bottom and \textbf{c} for centered. Note that the |\smaller\smaller| from
a previous example is still in effect: The command |\newgame| does not
reset any font. Please note the difference between diagram 1 and 3. The
font used for the suits is \textbf{not} the same.

\vfill
\begin{HBox}{showNS top algned\hfill File: \FileName}
\newgame
\northhand{KJ92}{KQ9}{J}{AQJT8}
\southhand{AQT85}{A52}{64}{K96}
A \showNS[t] B
\resetfonts
C \showNS[t] D
\gamefont{\sffamily\bfseries\smaller\smaller}
E \showNS[t] F
\end{HBox}
\vfill
We do need to call |\resetfonts| (or |\gamefont|) explicitly to reset the
font or size.
\vfill
\begin{HBox}{showNS center aligned\hfill File: \FileName}
\resetfonts
% \gamefont would do the job as well
%\gamefont{\bfseries\sffamily}
A \showNS[c] B
\end{HBox}
\vfill
\begin{HBox}{showNS bottom aligned\hfill File: \FileName}
A \showNS[b] B
\end{HBox}
\vfill
\end{document}
\endinput


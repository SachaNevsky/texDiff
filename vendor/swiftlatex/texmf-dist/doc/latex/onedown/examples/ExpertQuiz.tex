\documentclass[onedown-examples]{subfiles}
\begin{document}
\iffalse
\newgame\setdefaults{bidfirst=W}
\dealer[\east]\vulner[\eastwest]
\hand!{Q952}{K32}{765}{AQ7}

\begin{bidding}(\explainit{gameforcing}
 \explainit{natural, 6-card ab ab ab ab ab ab})%
  &    & 2D\markit & 2H\markit \\
p & 3H & 4D        & p \\
? \\
\end{bidding}

\makeatletter
   :\usebox\ODw@BidBox:
\makeatother



\begin{biddingpair}(\explainit{game forcing}
 \explainit{natural, 6-card})%
   2D\markit & 2H\markit \\
   3H        & 4D        \\
? \\
\end{biddingpair}

\makeatletter
   :\usebox\ODw@BidBox:
\makeatother
\fi
\vfill
A special feature of \OneDown is the so called |expertquiz|. It essentially
displays a hand with a (partial) bidding sequence. The reader has to guess
the best next bid with respect to the hand and the bidding so far. The
different answers are rewarded with points.

Observe the token '\textbf{-}' in |\hand-| and |\begin{bidding}-| that
suppresses the output.

With the optional argument we add some extra information to the quiz.
\vfill
\begin{HBox}{expertquiz \hfill File: \FileName}
\newgame\setdefaults{bidfirst=W}
\dealer[\east]\vulner[\eastwest]
\hand!-{Q952}{K32}{765}{AQ7} 
\begin{bidding}-(\explainit{gameforcing}
 \explainit{natural, 6-card})%
  &    & 2D\markit & 2H\markit \\
p & 3H & 4D        & p \\
? \\
\end{bidding}
\expertquiz[Team: \dealertext, \vulnertext]{%
  4\He= 10, 4\Sp= 7,
  6\Di= 4,
  4\NT/5\NT= 3, 5\Di=~1
}
\end{HBox}
\vfill
Here we show the same quiz, but the layout resembles the one used by the
DBV\footnote{German Bridge League} in their monthly, called \emph{Bridge
Magazin}. We get this special layout by using the token '\textbf{!}'. Of
course we set it in the German language.
\vfill
\begin{HBox}{expert quiz a la DBV\hfill File: \FileName}
\begin{otherlanguage}{german}
\expertquiz![Team: \dealertext, \vulnertext]{%
  4\He= 10, 4\Sp= 7,
  6\Di= 4,
  4\NT/5\NT= 3, 5\Di= 1
}
\end{otherlanguage}
\end{HBox}
\vfill
\end{document}
\endinput


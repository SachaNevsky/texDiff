% robustmanual.tex 2018/04/11
\documentclass{article}
\usepackage{makeidx}
\usepackage[multind]{robustindex}[2018/04/15]
\usepackage[hyperindex=false,colorlinks=true]{hyperref}

\title {Manual of robustindex package}

\author{Wilberd van der Kallen\index{Wilberd}}
\date{April 2018}

\makeindex    % obligatory with robustindex

\begin{document}

\maketitle

\section*{Introduction}

\index{discussion|(}

Amazingly often a third party changes the page numbers without rerunning
makeindex. We therefore want to have an index in which the page numbers automatically synchronise with the document.

The package \texttt{robustindex} achieves this
by invoking the \verb|\pageref| mechanism on automatically generated labels
of the form \verb|ind.2|, \verb|ind.4|, \dots.

Only after adding/deleting/modifying an \verb|\index{|\textit{entry}\verb|}|
command, or after changing the order of \verb|\index{|\textit{entry}\verb|}|
commands, does one have to rerun makeindex. (How would you change the order? By moving a block of text that contains 
an \verb!\index! command.)

Make sure to run makeindex at least once after the last change to the \verb!\index! commands.
Keep an eye on the \TeX\ log file. It often contains a warning that you  need to rerun \LaTeX\ or makeindex. 

\section*{Standard usage}
Put 
$$
\verb!\usepackage{makeidx}! 
$$
$$
\verb!\usepackage{robustindex}! 
$$
$$\verb!\makeindex!$$
in your preamble.
If you use the hyperref package, go against their advice and put 
$$\verb!\usepackage{robustindex}! $$
\emph{after} the hyperref declaration, \emph{or} 
turn off \verb!hyperindex! as in 
$$
\verb!\usepackage[plainpages=false,hyperindex=false]{hyperref}!.
$$

That should be all. (You may have to trash old auxiliary files, the \verb!ind! file included.)

You can now use \index{alpha}\index{gamma|textbf}
$$\verb!\index{alpha}!$$
$$\verb?\index{alpha!see beta}?$$
$$\verb!\index{gamma|textbf}!$$
$$\verb!\index{ampersand@\&|textbf}!$$
in the usual manner. 
\index{ampersand@\&|textbf}

See \url{https://en.wikibooks.org/wiki/LaTeX/Indexing}.\\



The symbol \verb!|!  has a special meaning inside an \verb!\index! command and cannot be escaped.
To get \verb!|! as output use the \LaTeX\ command \verb!\vert!.


We also have a command \verb!\gobblepageref!, a variant of \verb!\see!,
used as in 
$$\verb?\index{alpha!see also gamma\gobblepageref}?$$
\index{alpha!see also gamma\gobblepageref}

\subsection*{Page ranges}
Some features of makeindex had to be repurposed.
In particular, \emph{implicit} page ranges are no longer supported. 


If you want a page range in the index you have to use 
the \emph{explicit} page range mechanism of makeindex as in
$$\verb!\index{discussion|(}!$$% used above




on the first page of the range, followed by
$$\verb!\index{discussion|)}!$$\index{discussion|)}on the last page of the range.\\

\subsection*{Table of contents}
If there is a table of contents and you want the index listed in it, put the command
$$\verb!\indexincontents!$$
somewhere before the \verb!\printindex!.
\index{alpha!see beta}

Before discussing the \verb|multind|  option we now use $\verb!\printindex!$ to get an index (on a new page).

\printindex

\section*{The \texttt{multind} option}
The \verb!multind!  option provides
support for many indexes. 
Suppose the preamble contains  \verb!\usepackage[multind]{robustindex}[2018/04/15]!.

Let us say we want to use four indexes. First we choose tags to identify them.
Say we use the tags \verb|main|, \verb|bis|, \verb|a1|, \verb|b1|.
The tag \verb|main| is always known and
the index with tag \verb|main|  becomes active at the \verb|\begin{document}|.

With the \verb!\setindex! command we activate an index.
Thus \verb!\setindex{bis}! tells \LaTeX\ that the active index is now the index identified by the tag
\verb|bis|, until the next \verb!\setindex! command. The other indexes are inactivated.
If a tag has not been encountered 
before, then a new index with that tag is created. 
The \verb!\index! command and
the \verb|\printindex| commands write to/from the active index. All indexes are hidden in one
big index  file (with extension \verb!ind!) and \LaTeX\ knows how to find the active index in there.

If you wish you may use \verb!\sindex[bis]{!\emph{entry}\verb!}! as shorthand for\\

\hspace{5em} \verb!\setindex{bis}\index{!\emph{entry}\verb!}!\\
 
and \verb!\sindex{!\emph{entry}\verb!}! as shorthand for\\

 \hspace{5em} \verb!\setindex{main}\index{!\emph{entry}\verb!}!.\\



Option \verb!multind! works with the original compilation sequence. That is the main point of all our hacking.
For instance, the file \verb!multisample.tex! produces multiple indexes and may be processed in this 
old fashioned way:

latex multisample.tex

makeindex multisample

latex multisample.tex

latex multisample.tex\\

Older versions of  \verb!robustindex.sty! 
may give different results. Use version \verb!2018/04/15! or later.
We recommend to bundle \verb!robustindex.sty!  with your \verb!tex!  and \verb!ind! file when moving
files to another computer.


If you have an entry that should go before the letter a, then you may start the argument of \verb!\index! with \verb!<!,
as in \index{<@$<$ comes before alphabet}\verb!\index{<@$<$ comes before alphabet}!.

\subsection*{Embellishment}
One may embellish an index with letter headings, like this.

\verb!\setindex{main}!

\verb!\renewcommand{\indexname}{Embellished Index}!

\verb!\renewcommand{\indexcapstyle}[1]{\indexspace\textsc{#1}\par}!%

\verb!\printindex! \\

\noindent This gives 

\setindex{main}

\renewcommand{\indexname}{Embellished Index}

\renewcommand{\indexcapstyle}[1]{\indexspace\textsc{#1}\par}%

\printindex


\end{document}

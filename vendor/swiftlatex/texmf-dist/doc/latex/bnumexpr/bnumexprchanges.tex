%%
%% This is file `bnumexprchanges.tex',
%% generated with the docstrip utility.
%%
%% The original source files were:
%%
%% bnumexpr.dtx  (with options: `changes')
%% 
%% IMPORTANT NOTICE:
%% 
%% For the copyright see the source file.
%% 
%% Any modified versions of this file must be renamed
%% with new filenames distinct from bnumexprchanges.tex.
%% 
%% For distribution of the original source see the terms
%% for copying and modification in the file bnumexpr.dtx.
%% 
%% This generated file may be distributed as long as the
%% original source files, as listed above, are part of the
%% same distribution. (The sources need not necessarily be
%% in the same archive or directory.)
%% ---------------------------------------------------------------
%% The bnumexpr package: Expressions with big integers
%% Copyright (C) 2014-2019 by Jean-Francois Burnol
%%
\item[1.2e (2019/01/08)]
  Fixes a documentation glitch (extra braces when mentioning |\the\numexpr| or
  |\thebnumexpr|).

\item[1.2d (2019/01/07)]
  \begin{itemize}
  \item requires \xintcorename |1.3d| or later (if not using option |custom|).
  \item adds |\bnumeval|\marg{expression} user interface.
  \end{itemize}

\item[1.2c (2017/12/05)] \textbf{Breaking changes:}
  \begin{itemize}
  \item requires \xintcorename |1.2p| or later (if not using option |custom|).
  \item |divtrunc| key of |\bnumexprsetup| is renamed to |div|.
  \item the |//| and |/:| operators are now by default associated to the
    \emph{floored} division. This is to keep in sync with the change of
    \xintcorename at |1.2p|.
  \item for backwards compatibility, one
    may add to existing document:\newline
    \string\bnumexprsetup\{div=\string\xintiiDivTrunc,
    mod=\string\xintiiModTrunc\}
  \end{itemize}

\item[1.2b (2017/07/09)]
  \begin{itemize}
  \item the |_| may be used to separate visually blocks of digits in long
    numbers.
  \end{itemize}

\item[1.2a (2015/10/14)]
  \begin{itemize}
  \item requires \xintcorename |1.2| or later (if not using option |custom|).
  \item additions to the syntax: factorial |!|, truncated division
    |//|, its associated modulo |/:| and |**| as alternative to |^|.
  \item all options removed except |custom|.
  \item new command |\bnumexprsetup| which replaces the commands such as
    |\bnumexprusesbigintcalc|.
  \item the parser is no more limited to numbers with at most 5000 digits.
  \end{itemize}

\item[1.1b (2014/10/28)]
  \begin{itemize}
  \item README converted to |markdown/pandoc| syntax,
  \item the package now loads only |xintcore|, which belongs to |xint| bundle
    version |1.1| and extracts from the earlier |xint| package the core arithmetic
    operations as used by |bnumexpr|.
  \end{itemize}

\item[1.1a (2014/09/22)]
  \begin{itemize}
  \item added |l3bigint| option to use experimental \LaTeX3
    package of the same name,

  \item added Changes and Readme sections to the documentation,

  \item better |\BNE_protect| mechanism for use of
  |\bnumexpr...\relax| inside an |\edef| (without |\bnethe|). Previous one,
  inherited from |xintexpr.sty 1.09n|, assumed that the |\.=<digits>| dummy
  control sequence encapsulating the computation result had |\relax|
  meaning. But removing this assumption was only a matter of letting
  |\BNE_protect| protect two, not one, tokens. This will be backported to
  next version of \xintexprname, naturally (done with |xintexpr.sty 1.1|).
  \end{itemize}

\item[1.1 (2014/09/21)] First release. This is down-scaled from the
  (development version of) \xintexprname. Motivation came the previous day
  from a chat with \textsc{Joseph Wright} over big int status in \LaTeX3.
  The |\bnumexpr...\relax| parser can be used on top of big int macros of
  one's choice. Functionalities limited to the basic operations. I leave
  the power operator |^| as an option.
\endinput
%%
%% End of file `bnumexprchanges.tex'.

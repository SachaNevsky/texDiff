%%
%% This is file `morefloats-example.tex',
%% generated with the docstrip utility.
%%
%% The original source files were:
%%
%% morefloats.dtx  (with options: `example')
%% 
%% This is a generated file.
%% 
%% Project: morefloats
%% Version: 2015/07/22 v1.0h
%% 
%% Copyright (C) 2010 - 2015 by
%%     H.-Martin M"unch <Martin dot Muench at Uni-Bonn dot de>
%% Portions of code copyrighted by other people as marked.
%% 
%% The usual disclaimer applies:
%% If it doesn't work right that's your problem.
%% (Nevertheless, send an e-mail to the maintainer
%%  when you find an error in this package.)
%% 
%% This work may be distributed and/or modified under the
%% conditions of the LaTeX Project Public License, either
%% version 1.3c of this license or (at your option) any later
%% version. This version of this license is in
%%    http://www.latex-project.org/lppl/lppl-1-3c.txt
%% and the latest version of this license is in
%%    http://www.latex-project.org/lppl.txt
%% and version 1.3c or later is part of all distributions of
%% LaTeX version 2005/12/01 or later.
%% 
%% This work has the LPPL maintenance status "maintained".
%% 
%% The Current Maintainer of this work is H.-Martin Muench.
%% 
%% LaTeX 2015 provides the extrafloats command.
%% Don Hosek, Quixote, 1990/07/27 (Thanks!)
%% invented the main code for handling more floats
%% before extrafloats was available.
%% Maintenance has been taken over in September 2010
%% by H.-Martin Muench.
%% David Carlisle pointed the maintainer to the new
%% extrafloats command (Thanks!).
%% 
%% This work consists of the main source file morefloats.dtx,
%% the README, and the derived files
%%    morefloats.sty, morefloats.pdf,
%%    morefloats.ins, morefloats.drv,
%%    morefloats-example.tex, morefloats-example.pdf.
%% 
%% In memoriam
%%  Claudia Simone Barth + 1996/01/30
%%  Tommy Muench + 2014/01/02
%%  Hans-Klaus Muench + 2014/08/24
%% 
\documentclass[british]{article}[2014/09/29]%      v1.4h
%%%%%%%%%%%%%%%%%%%%%%%%%%%%%%%%%%%%%%%%%%%%%%%%%%%%%%%%%%%%%%%%%%%%%
\usepackage[maxfloats=25]{morefloats}[2015/07/22]% v1.0h
%% \maxdeadcycles is the maximum number of calls of \output
%% without a \shipout.
\gdef\unit#1{\mathord{\thinspace\mathrm{#1}}}%
\listfiles
\begin{document}

\makeatletter

\section*{Example for morefloats}
\markboth{Example for morefloats}{Example for morefloats}

This example demonstrates the use of package\newline
\textsf{morefloats}, v1.0h as of 2015/07/22 (HMM).\newline
The package takes options (here:
\verb|maxfloats=|\texttt{\morefloats@maxfloats} is used).\newline
For more details please see the documentation!\newline

To reproduce the\newline
\LaTeX{} \texttt{ Error: Too many unprocessed floats},\newline
comment out the \verb|\usepackage...| in the preamble
(line~3)\newline
(by placing a \% before it).\newline

\bigskip

Save per page about $200\unit{ml}$~water, $2\unit{g}$~CO$_{2}$
and $2\unit{g}$~wood:\newline
Therefore please print only if this is really necessary.\newline
I do NOT think, that it is necessary to print THIS file, really!

\bigskip

There follow \morefloats@maxfloats{} floating tables.

\pagebreak

\@tempcnta=18\relax% default floats
\advance\@tempcnta by \morefloats@morefloats%
\loop
  \ifnum\@tempcnta>0\relax%
  \begin{table}[t]\centering%
    \begin{tabular}{|l|}%
      \hline%
      A table, which will keep floating.\\%
      \hline
    \end{tabular}%
    \caption{A floating Table.}%
  \end{table}%
  \advance\@tempcnta by -1\relax%
\repeat

\makeatother

\end{document}
\endinput
%%
%% End of file `morefloats-example.tex'.

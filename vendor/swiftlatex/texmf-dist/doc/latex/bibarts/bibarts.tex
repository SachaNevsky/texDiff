  \documentclass[12pt,a4paper]{article}
   \usepackage{bibarts}   %% Reihenfolge ist gleichgueltig.
          \usepackage{ngerman} 
          %%\usepackage{german}
          %%\usepackage[germanb]{babel}
          %%\usepackage[ngermanb]{babel}
\setlength{\footnotesep}{2ex}   %% ... wie in bibarts.sty 2.0; siehe README.txt %%
       %\usepackage[cp850]{inputenc} %%\usepackage[utf8]{inputenc} %% entsprechende Zeichen sind nicht in bibarts.tex %%
       \usepackage[T1]{fontenc} %% Automatische Trennung von Worten mit Umlauten. %%



%%\documentstyle[12pt,german,a4,bibarts]{article}
 %\notnegcorrdefabk     %% LaTeX 2.09 %%



       \title{\sffamily Das \LaTeX-Paket \BibArts}
                        
       \author{\textsc{Timo Baumann}}
                        
       \date{\small Version 2.2, \copyright~2019.
         \hspace{.5em}\textbf{Inhalt S.\,\pageref{SectIn}}.}



\hyphenation{Stern-ar-gu-ment 
Vor-namens-argu-ment
Voll-ein-trag 
Satz-ende 
Nach-kom-ma-stel-len
De-zi-mal-zah-len-er-ken-nung
Auf-marsch-an-wei-sun-gen
Na-mens-ar-gu-men-te
Nach-na-mens-ar-gu-men-ten
Sprach-ein-stel-lung
Stern-ar-gu-men-te
Text-edi-tor
under-sco-res
Zei-len-um-bruch
Zei-len-um-br"u-che
}

       \newcommand{\pdfko}[1]{\kern #1pt
                              \rule[-0.4ex]{0.8pt}{0ex}\kern -0.8pt
                          \strut\ignorespaces}%
     %\newcommand{\pdfko}[1]{\strut\ignorespaces}% Falls NICHT mit pdflatex uebersetzt wird

       \newcommand{\pbs}{\string\ \unskip}
       \newcommand{\bs}{\protect\pbs}


\long\def\Doppelbox#1#2{\nopagebreak\vspace{2.25ex}%
       {\fbox{\parbox{.45\textwidth}{\raggedright\footnotesize\ttfamily
       \ignorespaces #1}\hfill\parbox{.45\textwidth}{\vspace{1ex}%
               \begin{minipage}{.45\textwidth}\sloppy\small
               \renewcommand{\thempfootnote}{\arabic{mpfootnote}}%
               \setcounter{mpfootnote}{\arabic{footnote}}%
               {\ignorespaces #2}%
               \setcounter{footnote}{\arabic{mpfootnote}}%
       \end{minipage}
               \vspace{1ex}}}\vspace{2.4ex}}}

 \begin{document}

 \maketitle 

 \noindent
 \BibArts\ soll \LaTeX\hy Anwender beim Schreiben geisteswissenschaftlicher
 Texte unterst"utzen (\kern-.05em\textit{arts faculty}). Der Vorspann eines deutschen 
 \LaTeX\hy Textdokuments, das das Stylefile \verb|bibarts.sty| einl"adt, 
 sieht typischerweise so aus:

  \vspace{-1.5ex}
        {\small\begin{verbatim}
   \documentclass[12pt,a4paper]{article} 
     \usepackage{bibarts}          \usepackage{ngerman}
     \usepackage[utf8]{inputenc}   \usepackage[T1]{fontenc}
  \end{verbatim}}
 
 \vspace{-4ex}\noindent
 Das separate Tippen von Anh"angen kann nun weitgehend entfallen. 
 Dazu werden Kopien von Literaturangaben, die sich im Haupttext oder 
 den Fu"snoten befinden, in einer Literaturliste und weiteren Listen 
 sortiert ausgedruckt.
 
 \vspace{1ex}\noindent
 Der wichtigste dieser Ausdruckbefehle, \verb|\printvli|, verh"alt sich 
 "ahnlich wie \verb|\tableofcontents| f"ur das Inhaltsverzeichnis: 
 Dazu muss im \LaTeX\hy Text ja auch
 \verb|\section{|\texttt{"Uberschriftentext}\verb|}| markiert werden. 
 Und f"ur \BibArts\ gilt:
 
 \Doppelbox
 {...\bs footnote\{Ein Beispiel f"ur Geschichtsliteratur ist
  \bs vli\{Hans-Ulrich\}\{Wehler\}\{Das Deutsche Kaiserreich, 
  \\ G"ottingen 1994\}.\}
 }
 {Einfaches Vollzitat.\footnote{Ein Beispiel f"ur Geschichtsliteratur ist
  \vli{Hans-Ulrich}{Wehler}{Das Deutsche Kaiserreich, 
  G"ottingen 1994}.}
 }

\noindent
In \verb|\vli| l"asst sich ein sp"ater verwendeter Kurztitel 
mit \verb|\ktit| so einf"uhren:

\Doppelbox
{...\bs footnote\{Soziologie:
 \bs vli\{Niklas\} \{Luhmann\} \{\bs ktit\{Soziale Systeme\}. 
 \\ Grundri"s einer allgemeinen Theorie, 1984: Frankfurt/M.\}.\}
}
{...\footnote{Soziologie:
 \vli{Niklas} {Luhmann} {\ktit{Soziale Systeme}. Grundri"s einer 
 allgemeinen Theorie, 1984: Frankfurt/M.}.\balabel{luhmann}}
}

\vfill\noindent 
Beide Arten der Eingabe (mit und ohne \verb|\ktit|) kommen in die 
Literaturliste, die das erw"ahnte \verb|\printvli| ausdruckt.
Dies hat nichts mit \textsc{Bib}\TeX\ zu tun:\pdfko{.25}

 \printvli
  
 \vspace{1ex}\noindent
 Wie von der Erzeugung des Inhaltsverzeichnisses her bekannt, sind
 "Anderungen erst nach zweimaligem Start von \LaTeX\
 im Preview oder Ausdruck zu sehen. F"ur \BibArts\ gilt au"serdem, dass 
 zwischen den beiden Bearbeitungen das Programm \verb|bibsort| gestartet
 werden muss, um die Literaturliste zu sortieren. Hei"st eine 
 \LaTeX\hy Textdatei \verb|meintext.tex|, ist typischerweise
 
 \vspace{-.5ex}
 \begin{verbatim} 
     bibsort -g1 -k -h  meintext
 \end{verbatim}

 \vspace{-3.25ex}\noindent
 in die Kommandozeile einzugeben. Dann liest \verb|bibsort| die mit
 \verb|\vli|-Li"-teraturangaben bef"ullte Datei \verb|meintext.aux| ein, 
 sortiert die Vollzitate "uber die Option
 \verb|-g1| nach deutschen Sortierregeln und legt das Ergebnis ab in einer
 Datei \verb|meintext.vli|, die im Text mit \verb|\printvli| bei der
 zweiten \LaTeX\hy "Ubersetzung ausgedruckt werden kann. Die zus"atzlich 
 gesetzte Option \verb|-k| \hspace{.3em}sorgt\pdfko{.25}\ 
 daf"ur, dass \textit{ein} Autor (Vor- und Nachname gleich), der mit mehreren 
 Werken zitiert wird, ab seiner zweiten Nennung auf der Literaturliste 
 als $\sim$ erscheint. \verb|-h| bewirkt, dass Bindestriche (Minuszeichen)
 als Leerzeichen sortiert werden.

 \vspace{1ex}
 Viele Texteditoren k"onnen den Start automatisieren. Falls sich die
 Datei \verb|bibsort.exe| im Verzeichnis \verb|C:\pfadangabe| befindet, dann
 gilt beispielsweise f"ur das \TeX nicCenter: Der automatisierte Start von
 \verb|bibsort| erfolgt durch Eingabe von \verb|C:\pfadangabe\bibsort.exe|
 in das Men"u \fbox{Ausgabe} $\Rightarrow$ \fbox{Ausgabeprofile definieren}
 $\Rightarrow$ \fbox{Vorbearbeitung} in die Zeile \underline{Anwendung} und
 \\[1ex]\verb|-i %tm -g1 -k -h|\hspace{.5em} 
 hinter \underline{Argumente} darunter.\hspace{.3em} (Evtl.\ versionsabh"angig.)

 \vspace{2.75ex}\noindent
 Nochmal zum \verb|\ktit|\hy Befehl: Dessen Verwendung im \verb|\vli|\hy Befehl 
 macht \textit{zus"atzlich} den Ausdruck eines Kurzzitate\hy Verzeichnisses mittels 
 \verb|\printnumvkc| m"oglich. \textsc{Wehler} fehlt nat"urlich; 
 aber vgl.\ \textsc{Luhmann} \baref[von]{luhmann}:

 \vspace{1.5ex}\label{vkc}{\footnotesize \batwocolitemdefs\printnumvkclist}

 \newpage
 \noindent
 Falls in Vollzitaten der Kurztitel mit \verb|\ktit| markiert 
 ist, erscheint das Kurzzitat (Nachname plus Kurztitel) also im Verzeichnis. 
 Um \verb|\printvli| zu  nutzen, m"ussen Sie \verb|\ktit| also nicht 
 verwenden $-$ aber dazu, \verb|\printnumvkc| zu bef"ullen. Dahinter folgten
 Orte mit `echten' Kurzzitaten; dazu gleich unten.

 Anwender, die das [L] $-$\,hei"st: \textit{Volltitel findet sich auf der Literaturliste}\,$-$ nicht 
 wollen, k"onnen im Vorspann \verb|\notprinthints| setzen; das unterdr"uckt den Ausdruck 
 von [L] und [Q] \,(\kern-.1em\textit{Volltitel im Verzeichnis gedruckter Quellen}).



\section{Vollzitate und Kurzzitate (\ko\textit{v}\hy\ und \textit{k}\fhy Befehle)}\label{Sect1}

 Nachdem ein Buch einmal vollzitiert wurde, kann es anschlie"send an
 weiteren Belegstellen kurzzitiert werden. Zur formatierten Eingabe von
 Literatur dienen f"ur Vollzitate die beiden Befehle\hspace{-.1em} \verb|\vli| f"ur
 Literatur und\hspace{-.1em} \verb|\vqu| f"ur gedruckte Quellen (Quelleneditionen); f"ur
 Kurzzitate dienen\hspace{-.1em} \verb|\kli| und\hspace{-.1em} \verb|\kqu|. Letztere haben jeweils ein
 Argument weniger als die Vollangaben, weil das Vornamensargument im
 Kurzzitat wegf"allt. Der Titel wird im Kurzzitat als Kurztitel angegeben.
 Falls dieser Kurztitel im Vollzitat bereits mit \verb|\ktit| markiert
 wurde, kann \BibArts\ mitkontrollierten, ob kurzzitierte Literatur weiter
 oben in Ihrem \LaTeX\hy Text bereits eingef"uhrt wurde. Dieser Aufgabe kommt
 das Sortierprogramm \verb|bibsort| nach, indem es Warnungen auf den
 Bildschirm ausgibt. Fehlt im Text das Vollzitat mit\hspace{-.1em} \verb|\ktit|, dann f"uhrt das
 Kurzzitieren der Quellenedition\hspace{.1em} 
 \verb|\kqu{Clausewitz}{Vom Kriege}|\hspace{.15em} zur \verb|bibsort|\hy Warnung:

  {\scriptsize\begin{verbatim}
   %%>   Info: Short-qu-title file 1 line 143 is NOT yet introduced.
   %%      (Clausewitz)  (Vom Kriege)  
   %%       ...  Change that short-title into missing full-title (\ktit)?
  \end{verbatim}}

 \vspace{-2ex}\noindent 
 Die Kontrolle macht \BibArts\ mittels der Daten, die auch in die\hspace{-.2em} \verb|.vkc|\hy Datei kommen: 
 (1)~Nachname und\hspace{.1em} Argument von\hspace{-.1em} \verb|\ktit|\hspace{.1em} aus \underline{v\ko}\hy 
 Belegen, (2)~Names- und Kurztitel\hy Argument aus \underline{k}\hy Belegen. 
 Die\hspace{-.2em} \verb|.vkc|\hy Datei, die `\ko\textit{\underline{c}ites}' 
 von Literatur \textit{und} gedruckten Quellen enth"alt, wurde oben mit 
 \verb|\printnumvkc| ausgedruckt.

 \vspace{1ex}\noindent
 Erfolgt irrt"umlich \textit{erst} das Kurzzitat und
 \textit{weiter unten} das Vollzitat ...

\Doppelbox
{...\bs footnote\{\bs kqu\{Clausewitz\}
 \\ \hspace{2em} \{Vom Kriege\}.\} 
 \ Aber ... \\
 ...\bs footnote \b{\b{\{}}Siehe dazu weiter 
 \\ \hspace{.5em} \bs vqu \{Carl von\}
 \{Clausewitz\} 
 \\ \hspace{2.5em} \b{\{}\bs ktit\{Vom Kriege\}.
 \\ \hspace{3em} Hinterlassenes Werk, 
 3.\bs,Auf\string"|l. Frankfurt/M 1991\b{\}}.\b{\b{\}}}
}
{...\footnote{\kqu{Clausewitz} {Vom Kriege}.}
 Aber ...
 ...\footnote{Siehe dazu weiter \vqu {Carl von}{Clausewitz} {\ktit{Vom Kriege}.
 Hinterlassenes Werk, 3.\,Auf"|l.\ Frankfurt/M 1991}.}
}

 \noindent
 ... dann warnt \verb|bibsort| danach auf dem
 Bildschirm etwa (siehe Folgeseite):

 \vspace{-.75ex}
  {\scriptsize\begin{verbatim} 
   %%>   Info: Short-qu-title file 1 line 193 is NOT yet introduced.
   %%      (Clausewitz)  (Vom Kriege)  
   %%       ...  Exchange it with the full-title in file 1 line 196.
  \end{verbatim}} %% Zeilenangaben nur ausserhalb \Doppelbox brauchbar %%

 \vspace{-2.75ex}\noindent 
 Die mehrfache Vergabe \textit{gleicher} Kurzzitate ... \footnote{\vqu{Carl
 von}{Clausewitz}{\ktit{Vom Kriege}. Hinterlassenes Werk, 3.\,Auf"|l.\
 Frankfurt/M 1991}.} ... ergibt solche Warnungen
 (der Vergleich ist f"ur \textit{aktive} \verb|"| sensitiv und \verb|!"| k"undigt deren Vorkommen an):

  \vspace{-.75ex}
  {\scriptsize\begin{verbatim} 
   %%>   Info: Introduction 2 of short-qu-title in file 1 line 241.
   %%      (Clausewitz)  (Vom Kriege)
   %%    First introduction has been in file 1 line 225 (\ktit).
  \end{verbatim}}

 \vspace{-2.75ex}\noindent  
 Falls Sie von \textit{einem} Autor in direkt aufeinanderfolgenden Fu"snoten 
 \textit{verschiedene} Werke zitieren, erscheint im 
 \LaTeX\kern.05em-\hspace{-.05em}\verb|.log|\kern.05em\hy File und auf dem Bildschirm:\balabel{DERS}%

  \vspace{-.75ex}
  {\footnotesize\begin{verbatim} 
  BibArts Warning: ...vqu-cmd repeats (first) author's lastname 
     on input line 180. `{Clausewitz}'. Change to `...vqu[m,f,p]'?
  \end{verbatim}}

 \vspace{-3.25ex}\noindent 
 Dieser Hinweis informiert, dass der Autorenname
 bei der zweiten Nennung in der direkt folgenden Fu"snote ersetzt 
 werden \textit{kann} durch den in diesem Fall\pdfko{.5}\ 
 "ublichen Hinweis, dass er derselbe 
 ist (\textsc{ders.}). Den entsprechenden Schalter\pdfko{.5}\ 
 m"ussen Sie
 selbst umlegen und dabei das Geschlecht des Autors einstellen. 
 Der Schalter l"asst sich zusammen mit allen v- und k-Befehlen 
 verwenden. Verf"ugbare Schalter sind\hspace{.1em}
 \verb|f| (weiblich),\hspace{.1em} \verb|m| (m"annlich) und\hspace{.1em} \verb|p{}|
 (plural: S.\,\pageref{p}):\pdfko{.25}

 \vspace{1.ex}{\small\noindent
\verb|  \footnote{\vqu [m] {Carl von}{Clausewitz}{\ktit{Strategie}.|\\
\verb|    Hrsg. von \vauthor{Eberhard}{Kessel}, Hamburg 1937}[58].}  => |}
{\renewcommand{\thefootnote}{{\bfseries\arabic{footnote}}}%
 \footnote{\vqu [m] {Carl von}{Clausewitz}{\ktit{Strategie}. 
 Hrsg.\ von \vauthor{Eberhard}{Kessel}, Hamburg 1937}[58].
 \hspace{.8em} $\Leftarrow$ \ \texttt{... \{Clausewitz\} \{\bs ktit\{Strategie\}. ...\}[58].}}%
}

\vspace{1.ex}\noindent 
Dabei wurde zudem eine Seitenangabe (\kern.05em...\kern.1em\texttt{\underline{\}[}58]}
\textit{ohne} \underline{Leerzeichen}) gemacht. 

Die Ank"undigung des
Kurztitels mit \verb|\ktit| erm"oglicht ein \textsc{ebd.}\hy Setzen:
Falls das folgende Kurzzitat auf 
dieselbe Seite der Quellenedition verweist, ergibt sich ... 
{\renewcommand{\thefootnote}{{\bfseries\itshape\arabic{footnote}}}%
 \footnote{\kqu{Clausewitz}{Strategie}[58].\label{ErsterFall} 
 \hspace{5.95em} $\Leftarrow$ \ \texttt{\bs kqu\{Clausewitz\}\{Strategie\}[58].}}
} ... w"ahrend eine andere Seite (\verb|[60]|) gedruckt wird 
als~...~\footnote{\kqu{Clausewitz}{Strategie}[60]. 
\hspace{3.05em} $\Leftarrow$ \ \texttt{\bs kqu\{Clausewitz\}\{Strategie\}[60].}}\kern.05em.

\BibArts\ druckt in Fu"snote\,\ref{ErsterFall} nur den Abk"urzungspunkt von \textsc{ebd.}, 
nicht aber den direkt folgenden Punkt am Satzende. Dies funktioniert nur, 
wenn zwischen \verb|[|\textit{Seitenzahl}\verb|]| und \verb|.|
\textit{keine Klammern oder Leerzeichen} stehen ... \footnote{\textit{\kqu{Clausewitz}{Strategie}[60]}.
\hspace{5.68em} $\Leftarrow$ \ \texttt{\bs textit\b{\{}\bs kqu\{Clausewitz\}\{Strategie\}[60]\b{\}}.
\texttt{\%Fehler}}}!

Das automatischen Ebenda-Setzen f"uhrt \BibArts\ in einer Fu"snote nicht durch, 
wenn in der vorausgehenden Fu"snote zwei verschiedene Werke angegeben sind 
(weil dies nicht eindeutig w"are): ...
\footnote{\kqu[m]{Clausewitz}{Strategie}[60] und \kqu[m]{Clausewitz}{Vom Kriege}.
\hspace{3em} \texttt{\%\% Ein Autor mit zwei Werken. \%\%}}
$\leftarrow$
\footnote{\kqu[m]{Clausewitz}{Strategie}[12]. 
\hspace{3.85em} $\Leftarrow$ \ \texttt{\bs kqu[m]\{Clausewitz\}\{Strategie\}[12].}}
... Mit \verb|\notibidemize| l"asst sich das automatische Ebenda-Setzen ausschalten
(nicht demonstriert).

In jedem Fall ist sinnvoll, im letzten Argument eines v\fhy Befehls 
einen Teil des Volltitels mittels \verb|\ktit| als Kurztitel zu
markieren: Dies bewirkt einerseits den Eintrag ins\hspace{-.15em} \verb|.vkc|\hy Verzeichnis
und erm"oglicht \texttt{bibsort} parallel die Kontrolle, ob beim Erstzitieren stets
vollzitiert wurde. Andererseits kann nur dann ein k\fhy Befehl in der 
folgenden Fu"snote als \textsc{ebd.} ausgedruckt werden. 

\BibArts\ erzwingt aber keine Verwendung von \verb|\ktit| in normalen
v\fhy Befehlen, denn sicherlich wollen manche Anwender die k\hy Befehle
\verb|\kli| und \verb|\kqu|\pdfko{.5}\ 
"uberhaupt nicht benutzen.\footnote{Das
unten eingef"uhrte Setzen von \textit{inneren v\fhy Befehlen} (im letzten Argument
eines \textit{"au"seren v\fhy Befehls}) ist dann allerdings nicht m"oglich; 
\textit{innere v\fhy Befehle} m"ussen immer ein \texttt{\bs ktit} haben.
Das liegt daran, dass auf den v\fhy Listen \textit{innere v\fhy Befehle} 
stets als Kurztitel ausgedruckt werden; \texttt{bibsort} druckt den Volltitel
automatisch in eigenen Listenpunkt! Falls Sie das nicht wollen, k"onnen
Sie \textit{innen} \texttt{\bs ntvauthor} verwenden (siehe unten S.\,\pageref{ntvauthor}).} 
In \BibArts~2.2 erzeugt \texttt{bibsort} "ubrigens Meldungen als Kommentare 
in den erzeugten Dateien\hspace{-.15em} \verb|.vli| und\hspace{-.15em} \verb|.vqu|\kern.05em, falls \verb|\ktit| 
gelegentlich verwendet und gelegentlich vergessen wurde.

Das Weglassen von \verb|\ktit| ist aber keine gute Methode, im Ausdruck von
Vollzitaten die\hspace{.2em} {\small im Folgenden\,...}\kern.05em\hy Ank"undigung\hspace{.1em}  (wie nachfolgend 
kurzzitiert werden wird) auszuschalten.\footnote{\texttt{bibsort}
w"urde am ersten Kurzzitat vorschlagen, \textit{dieses} in ein Vollzitat zu
verwandeln.} Dazu dient vielmehr \verb|\notannouncektit|, das\pdfko{.25}\ 
im Dokumentenvorspann \textit{global} gesetzt werden kann. Das Beispiel zeigt, 
wie es \textit{lokal} $-$\,also zusammen mit dem v\fhy Befehl eingeklammert\,$-$ zu setzen 
ist:\footnote{Ein lokales Setzen eines v\fhy Befehls unter 
\texttt{\bs notannouncektit} scheint zwar naheliegend, wenn ein Werk nur 
einmal pro Text angef"uhrt wird. Aber falls Sie sp"ater \textit{dasselbe} Werk 
doch noch kurzzitieren sollten, macht \BibArts\ keine Meldung, dass
\textit{der} Kurztitel \textit{im Text}\pdfko{.1}\ 
nicht vorangek"undigt wurde: \texttt{bibsort}
wei"s von Ihrem \texttt{\bs notannouncektit} nicht~(nie!).}

\Doppelbox
{... \bs vli\{Niklas\}\{Luhmann\} 
  \\ \{\bs ktit\{Soziale Systeme\}. 
  \\ \ Grundri"s einer allgemeinen 
        \\ \ Theorie, 1984: Frankfurt/M.\}
  \\[1.25ex] 
  Unannonciert:
  \b{\b{\{}}\bs notannouncektit 
    \bs vli\{Niklas\}\{Luhmann\} \b{\{}\bs ktit\{Soziale Systeme\}. 
   \\ \ Grundri"s einer allgemeinen 
   \\ \ Theorie, 1984: Frankfurt/M.\b{\}}\b{\b{\}}}
   \\[.75ex] \%\% Zu \}. siehe unten S.\pageref{v-Ausnahme}! \%\%
}
{\vspace{.375ex}
 Annonciert: \vli{Niklas}{Luhmann} {\ktit{Soziale Systeme}. 
      Grundri"s einer allgemeinen Theorie, 1984: Frankfurt/M.}
  \\[4.6ex] Unannonciert:
  {\notannouncektit
   \vli{Niklas}{Luhmann} {\ktit{Soziale Systeme}. 
        Grundri"s einer allgemeinen 
        Theorie, 1984: Frankfurt/M.}}
}

\vspace{.5ex}\noindent
Beide \verb|\vli|\hy Befehle ergeben $-$\,weil beide ein \verb|\ktit| haben\,$-$ 
auch Eintr"age in die\hspace{-.15em} \verb|.vkc|\hy Datei\hspace{.25em} (die\hspace{.1em} 
{\small\balistnumemph\thepage}\hspace{.1em} hinter 
{\small\printonlykli{Luhmann}{Soziale Systeme}} in der Liste S.\,\pageref{vkc}). 

\vspace{2ex}\noindent
Das automatische \textsc{ebd.}\hy Setzen f"uhrt \BibArts\ "ubrigens nur von 
Fu"snote zu Fu"snote durch, nicht im Haupttext. Nie in \textsc{ebd.} umgewandelt 
werden v\fhy Befehle.


\vspace{1ex}\noindent
Auch im speziellen Fall des Texttyps \textbf{Aufsatz} $-$\,gemeint ist:
falls \verb|\printvli| nicht verwendet wird\,$-$ ist die Verwendung von \verb|\ktit|
sinnvoll. In solchen Texten ist n"amlich w"unschenswert, beim Kurzzitat
Querverweise auf das Vollzitat zu drucken, um auf die vollst"andigen bibliographischen 
Angaben hinzuweisen. \BibArts\ kann deshalb optional aus jedem v\fhy Befehl
(Vollzitat) eine Marke aus Autornachname und Kurztitel erzeugen,
damit zugeh"orige k\fhy Befehle automatisiert einen Querverweis drucken k"onnen. 
Das Einschalten dieses Aufsatz\hy Modus erfolgt mit dem \BibArts-Befehl
\verb|\conferize|. Der sollte global gelten, also im Vorspann von
\LaTeX\hy Textdateien gesetzt werden.\footnote{Dies hat nichts
damit zu tun, ob f"ur den \LaTeX-Text der Dokumentenstil \texttt{\{article\}}\pdfko{.5}\ 
oder \texttt{\{book\}} gew"ahlt wird. Vielmehr sind die Auswahl des 
Dokumentenstils und das Setzen von \texttt{\bs conferize} zwei voneinander 
unabh"angige Entscheidungen. -- Studentische Hausarbeiten werden zwar oft
als Aufs"atze bezeichnet, sollen aber meist eine Literaturliste haben.} 
Ein Blick auf die Fu"snoten \ref{vz} und \ref{fz} 
im Kurzzitateverzeichnis (S.\,\pageref{vkc}) belegt, dass es sich auch im
\verb|\conferize|\hy Modus zu Kontrollzwecken ausdrucken l"asst.

\Doppelbox
{\bs conferize\ ...\bs footnote\b{\{}
   \\ \ \ Vollzitat: \bs vli\{Niklas\} 
   \\ \ \ \{Luhmann\} \{\bs ktit\{Soziale 
   \\ \ \ \ \ \ Systeme\}. Grundri"s einer 
   \\ \ \ \ \ allgemeinen Theorie, 
   \\ \ \ \ \ 1984: Frankfurt/M.\}.\b{\}} 
   \\[.5ex] ...\bs footnote\{\} \ \% kein Ebd.
   \\[.5ex] ...\bs footnote\H{\{}Kurzzitat: 
   \\ \ \ \bs kli\{Luhmann\} \ \ \{Soziale 
   \\ \ \ \ \ \ Systeme\}[23\bs f].\H{\}}
}
{\conferize ...\footnote{\label{vz}%
                  Vollzitat: \vli{Niklas}{Luhmann} {\ktit{Soziale Systeme}. 
     Grundri"s einer allgemeinen Theorie, 1984: Frankfurt/M.}.} 
  ...\footnote{}
  ...\footnote{\label{fz}Kurzzitat: \kli{Luhmann} {Soziale Systeme}[23\f].}
}


\noindent
Im Programmcode von \verb|bibarts.sty| wurde gro"ser Aufwand damit
betrieben, dass dies immer funktioniert, also auch dann, wenn sich
\LaTeX\hy Befehle in den Argumenten der 
v- und k\fhy Befehle befinden. Dazu durchsucht \BibArts\ sie und kopiert 
nur bestimmte Teile von \textit{Nachname} und \textit{Kurztitel} in das 
automatisch erzeugte \textit{Schl"usselwort} f"ur die Marke (vorz"uglich 
die Buchstaben).\footnote{\BibArts\ bildet zudem einige 
Sonderzeichen bzw.\ Akzente im Schl"usselwort ab, etwa:\hspace{.8em} 
\texttt{\bs \string`}\nobreak\hspace{.25em}$\Rightarrow$\nobreak\hspace{.2em}\texttt{[}\hspace{.9em}
\texttt{\bs \string'}\nobreak\hspace{.25em}$\Rightarrow$\nobreak\hspace{.2em}\texttt{]}\hspace{.9em} 
\texttt{\bs \string"}\nobreak\hspace{.25em}$\Rightarrow$\nobreak\hspace{.2em}\texttt{*}\hspace{.9em}
\texttt{\bs ss}\nobreak\hspace{.25em}$\Rightarrow$\nobreak\hspace{.2em}\texttt{(ss}\hspace{.9em}
\texttt{\bs l}\nobreak\hspace{.25em}$\Rightarrow$\nobreak\hspace{.2em}\texttt{(l}\hspace{.9em}
\texttt{\bs o}\nobreak\hspace{.25em}$\Rightarrow$\nobreak\hspace{.2em}\texttt{(o}\hspace{.9em}
\texttt{\bs \string^}\nobreak\hspace{.25em}$\Rightarrow$\nobreak\hspace{.2em}\texttt{)}\hspace{.9em}
\texttt{\bs \string~}\nobreak\hspace{.25em}$\Rightarrow$\nobreak\hspace{.2em}\texttt{-}\hspace{.9em} 
\texttt{\bs c}\nobreak\hspace{.25em}$\Rightarrow$\nobreak\hspace{.2em}\texttt{+}\pdfko{.3}\hspace{.9em}
(\kern.05em in Version~2.2\hspace{.1em} sind\hspace{.7em}\texttt{\bs \string=}\hspace{.5em} \texttt{\bs \string.}\hspace{.225em} 
\texttt{\bs b}\hspace{.4em} \texttt{\bs d}\hspace{.4em} sowie das\hspace{.1em} \textit{aktive} \texttt{\string"}\hspace{.15em} 
ausgeschieden).\hspace{.3em} In Fu"snoten z.\,B.\pdfko{.1}\ setzt\hspace{.3em} 
\texttt{\bs vli\hspace{.2em}\{Peter\}\hspace{.2em}\{M\bs\string"uller\}\hspace{.2em}%
\{Die \bs ktit\{Reise\}, Verlagsstadt 2002\}}\hspace{.3em} 
die Querverweis\hy Marke\hspace{.325em} 
\texttt{\bs newlabel\{baf.M*uller..Reise\}\{\{{\normalfont 
\textit{Fu"snote}}\}\{{\normalfont \textit{Seite}}\}\}}\hspace{.35em} ins\hspace{-.15em} \texttt{.aux}\hy File ab.}

Sorgfalt erfordert mehr das automatische \textsc{ebd.}\hy Setzen 
(\verb|\notibidemize| schaltet es aus). \BibArts\ erkennt zwei 
Argumente nur als gleich an, wenn sie zeichengleich sind. 
\verb|\only|...\hy Befehle sind zwar erlaubt (sollten jedoch
jeweils getestet werden f"ur den Querverweis \textit{und} die 
\textsc{ebd.}\hy Setzung). Falls Sie aber \verb|\underline| im v\fhy Befehl mit
\verb|\protect| sch"utzen und im zugeh"origen k\fhy Befehl nicht, 
gibt es zwei Eintr"age in der\hspace{-.15em} \verb|.vkc|-Liste und 
definitiv kein \textsc{ebd.}\hy Setzen.

Die ohnehin f"ur geisteswissenschaftliche Texte g"ultige Spielregel, dass 
jedes Kurzzitat aus Autoren\hy Nachname plus Kurztitel ein bestimmtes Werk 
eindeutig bezeichnen muss, schlie"st aus, dass eine Marke absichtlich 
zweimal vorkommt (Kurztitel d"urfen sogar mehrfach gleich
sein, wenn sich nur die Nachnamen unterscheiden). Trotzdem steht 
das Befehlspaar \verb|\balabel| und\pdfko{.5}\ 
\verb|\baref| bereit, um 'von Hand' Marken setzen zu k"onnen, 
wie k\fhy Befehle es im\pdfko{.3}\  
\verb|\conferize|\hy Stil tun: \verb|\baref| bemerkt ebenfalls selbst, ob 
es auf eine Fu"snote\pdfko{.3}\   
verweist. In den 'von Hand' zu tippenden Schl"usselworten in den
Argumenten von \verb|\balabel| und \verb|\baref| sind Sonderzeichen 
allerdings verboten.

\Doppelbox
{M"uller \bs balabel\{Mueller\} im 
 Text.\bs footnote\b{\{}Maier in 
 \\ \ Fu"snote.\bs balabel\{Maier\}\b{\}}
 \\ ... M"uller ist nochmal erw"ahnt \bs baref\{Mueller\} 
 und Maier ebenfalls \bs baref[vgl.]\{Maier\}.
}
{M"uller \balabel{Mueller} im 
 Text.\footnote{Maier in 
 Fu"snote.\balabel{Maier}}
 ... M"uller ist nochmal erw"ahnt \baref{Mueller} 
 und Maier ebenfalls \baref[vgl.]{Maier}.
}

\noindent
Das \texttt{[}\textit{OptionalArg}\texttt{]} "uberschreibt 'siehe'
(\verb|\grefverbname|; vgl.\ unten S.\,\pageref{grefverbname}). 


\vspace{2ex}\noindent
Zur"uck zu den v\fhy Befehlen. Bei der Auswahl eines Kurztitels aus
dem Volltitel mit \verb|\ktit| wird es gelegentlich so sein, dass 
der Kurztitel dort mit einem Kleinbuchstaben beginnt. \BibArts\
erkennt die Verbindung mit sp"ater in k\fhy Befehlen gro"sgeschriebenen 
Kurztiteln mittels \verb|\onlyvoll| und \verb|\onlykurz|:

\label{Ferguson}%
\Doppelbox
{...\bs footnote\{\bs vli\{Niall\}
  \\ \ \{Ferguson\} \b{\b{\{}}Der
  \\ \ \bs ktit\b{\{}\bs onlykurz\{F\}\%
  \\ \ \ \bs onlyvoll\{f\}alsche\bs onlykurz\{r\} 
  \\ \ \ Krieg\b{\}}, M"unchen 2001\b{\b{\}}}[22].\}
 ...\bs footnote\{\bs kli\{Ferguson\}
   \\ \ \{Falscher Krieg\}[23].\}
 ...\bs footnote\{\bs clearbamem 
 \\ \ \bs kli\{Ferguson\}\{Falscher Krieg\}.\} 
}
{...\footnote{\vli{Niall}
      {Ferguson} {Der
     \ktit{\onlykurz{F}%
                 \onlyvoll{f}alsche\onlykurz{r} Krieg},
            M"unchen 2001}[22].}
 ...\footnote{\kli{Ferguson}
      {Falscher Krieg}[23].}
 ...\footnote{\clearbamem \kli{Ferguson}{Falscher Krieg}.\label{clearbamem}} 
}

\vspace{.5ex}\noindent 
Die Fu"snote\,\ref{clearbamem} soll sich auf das ganze 
Werk beziehen; nur \textsc{ebd.} wieder mittels \verb|\kli{Ferguson}{Falscher Krieg}[23]|
zu erzeugen, w"are falsch. Stattdessen l"oschte \verb|\clearbamem| die 
Zwischenspeicher. Sonst h"atte der \verb|\kli|-Befehl ohne \verb|[|\textit{Seite}\verb|]|
bei der "Ubersetzung mit \LaTeX\ diese Fehlermeldung ausgel"ost:

\vspace{0.5ex}\label{before}
{\scriptsize\begin{verbatim}
  ! Same title, before with :{p}{23}:, has now no page/folio number.
    . . . . . . . . . . . .
  \errmessage@ba ...
   \space . . . . . . . . . . . }
                                                  }
  l.461 \footnote{\kli{Ferguson}{Falscher Krieg}.}
                                                  }
\end{verbatim}}

Neben Monografien gibt es noch B"ucher, die aus mehreren Aufs"atzen bestehen. 
Es ist genug, auch \textbf{Herausgeberwerke} nur einmal voll zu zitieren. Bei der 
Ersteinf"uhrung des \textit{zweiten} Aufsatzes darf das Buch (im letzten 
Argument des '"au"seren' v\fhy Befehls) kurzzitiert sein, denn es ist ja schon 
bekannt. Es steht ein 'inneres' \textsc{ebd.}\hy Setzen an, falls Sie beide 
Aufs"atze in aufeinander folgenden Fu"snoten einf"uhren. \BibArts\ 
hat daf"ur eine zweite Speicherebene.\footnote{Falls Sie das Herausgeberwerk 
sp"ater \textit{eigenst"andig} kurzzitieren wollen (wie unten S.\,\pageref{MEhlert}:
also nicht im letzten Argument eines v- oder k\fhy Befehls) 
\textit{und} dort \textsc{[Hrsg.]} nicht mehr setzen wollen, dann m"ussen 
Sie hier \textsc{[Hrsg.]} mit \texttt{\bs onlyvoll} im inneren v\fhy Befehl und mit 
\texttt{\bs vollout} im inneren k\fhy Befehl maskieren 
(Leerzeichen so: \texttt{\{Gro"s\}\bs onlyvoll\{ [Hrsg.]\}}).}

\vspace{.75ex}
\Doppelbox
{...\bs footnote\{Innen vollzitiert: 
 \\ \bs vqu \{\} \{\} 
 \\ \b{\b{\{}}\bs ktit\{Aufmarschanweisungen 
 \\ \ \ 1912\}, abgedruckt in: 
 \\ \ \bs xvqu\{Hans\} \{Ehlert\} 
 \\{} \ \ *\b{\{}\bs midvauthor\{Michael\}
 \\ \ \ \ \ \ \{Epkenhans\} 
 \\ \ \ \ \ \bs vauthor\{Gerhard P.\} 
 \\ \ \ \ \ \ \{Gro"s\} [Hrsg.]\b{\}} 
 \\ \ \ \{Der \bs ktit\{Schlieffenplan\},
 \\ \ \ \ Paderborn 
 \\ \ \ \ 2007\}[462-466]\b{\b{\}}}*[463].\}
 \\ \
 \\ ...\bs footnote\{Innen kurz: 
 \\ \bs vqu \{\} \{\} 
 \\ \H{\{}\bs ktit\{Aufmarsch 1913/14\},
 \\ \ abgedruckt in: 
 \\ \ \bs xkqu\{Ehlert\}
 \\{}\ \ *\{\bs midkauthor\{Epkenhans\} 
 \\ \ \ \ \ \bs kauthor\{Gro"s\} [Hrsg.]\}
 \\ \ \ \{Schlieffenplan\%
 \\ \ \ \}[467-477]\H{\}}*[469].\} 
}
{\vspace{1ex}
 \fbox{\parbox{.95\textwidth}{Siehe \texttt{.vkc}-Eintr"age oben S.\,\pageref{vkc}: 
 \\[1ex] \scshape[Anonym]\\{}[Anonym]\\{}[...]\\Ehlert\baslash Epkenhans\baslash Gro"s}}\\
 \\[1ex] ...\footnote{Innen vollzitiert: \vqu {} {} 
   {\ktit{Aufmarschanweisungen 1912}, 
   abgedruckt in: \xvqu{Hans} {Ehlert}
   *{\midvauthor{Michael} {Epkenhans}
     \vauthor{Gerhard P.} {Gro"s} [Hrsg.]} 
                {Der \ktit{Schlieffenplan},
     Paderborn 2007}[462-466]}*[463].}
                
 ...\footnote{Innen kurz: \vqu {} {} 
    {\ktit{\onlyhere{\onlykurz{\,}}Aufmarsch 1913/14},
    abgedruckt in: \xkqu{Ehlert}
    *{\midkauthor{Epkenhans} \kauthor{Gro"s} [Hrsg.]}
    {Schlieffenplan%
      }[467-477]}*[469].}
}


\vspace{.75ex}\noindent
\verb|*[463]| und \verb|*[469]| ergeben 'dort: S.' zur Bezeichnung der 
zitierten Einzelseite innerhalb des zuvor genannten Seitenbereichs des 
Teiltextes. Vor \verb|*[| darf\pdfko{1}\ kein Leerzeichen stehen; vor 
\verb|[462-466]| und \verb|[467-477]| auch nicht. Setzen\pdfko{.5}\ 
von runden statt eckigen Klammern w"urde Bl.\ statt S.\ ausdrucken.

Das Beispiel f"uhrte zudem das 'Sternargument' ein, das in allen v- 
und k\hy Befehlen nach dem Nachnamensargument \verb|*{|\textit{optional}\verb|}| 
stehen darf, um Koautoren aufzunehmen. In v\fhy Befehlen sind vauthor\hy 
Formatierer und in k\fhy Befehlen kauthor\fhy Formatierer zu verwenden. 
\BibArts\ setzt \textsc{ebd.} nur dann, wenn\pdfko{.5}\  
\textit{auch} gleiche Nachnamen in den vauthor\hy\ und kauthor\hy 
Formatierern stehen. 

Dabei benennen \verb|\vauthor| und \verb|\kauthor| stets den letzten von 
jeweils mehreren Autoren. Falls $-$\,wie oben\,$-$ mehr als zwei
Autoren genannt werden, sind alle davor im Sternargument mit 
\verb|\midvauthor| bzw.\ \verb|\midkauthor| zu kennzeichnen. Die setzen 
Schr"agstriche nach dem Nachnamen. Der Schr"agstrich nach dem Erstautor 
wird von x\fhy Befehlen erzeugt. Im letzten Beispiel waren das \verb|\xvqu| 
und \verb|\xkqu|, bei Literatur sind es \verb|\xvli| und 
\verb|\xkli|. 

Auch 'normale' v- und k\fhy Befehlen d"urfen Sternargumente haben.
Nach \verb|\vli| und \verb|\vqu| k"onnen sie Attribute wie 
\verb|*{\onlyvoll{[Hrsg.]}}| aufnehmen. Das Sternargument 
des v\fhy Befehls ist hier \textit{komplett} mit \verb|\onlyvoll| maskiert, sodass sp"atere 
k\fhy Befehle kein Sternargument brauchen (\textsc{ebd.}\hy Setzung):

\vspace{-.25ex}
\Doppelbox
{
\vspace{-.3ex}
...\bs footnote\{\bs vli\{Peter\}\{Maier\} 
\\[.4ex] \ \ \ \ \ \ *\b{\b{\{}}\bs onlyvoll\{[Hrsg.]\}\b{\b{\}}} 
\\[.3ex] \ \ \ \ \ \ \b{\{}Das \bs ktit\{Buch\}\b{\}}.\} 
\\[.7ex] Go!\bs footnote\{\bs kli\{Maier\}\{Buch\}.\} 
}
{Nicht in Listen "ubernommen.\footnote{\notktitaddtok
\printonlyvli{Peter}{Maier} *{\onlyvoll{[Hrsg.]}} {Das \ktit{Buch}}.}
Go!\footnote{\printonlykli{Maier}{Buch}.}
}


\vfill\noindent
Werden direkt hintereinander zwei verschiedene Aufs"atze aus einem 
Herausgeberwerk zitiert, erfolgt das innere und "au"sere \textsc{ebd.}\hy 
Setzen automatisch:

\vspace{.5ex}
\Doppelbox
{\scriptsize
 ...\bs footnote\{Innen vollzitiert: 
 \\ \bs vqu \{\} \{\} 
 \\ \b{\b{\{}}\bs ktit\{Aufmarschanweisungen 
 \\ \ \ 1912\}, abgedruckt in: 
 \\ \ \bs xvqu\{Hans\} \{Ehlert\} 
 \\{} \ \ *\b{\{}\bs midvauthor\{Michael\}
 \\ \ \ \ \ \ \{Epkenhans\} 
 \\ \ \ \ \ \bs vauthor\{Gerhard P.\} 
 \\ \ \ \ \ \ \{Gro"s\} [Hrsg.]\b{\}} 
 \\ \ \ \{Der \bs ktit\{Schlieffenplan\},
 \\ \ \ \ Paderborn 
 \\ \ \ \ 2007\}[462-466]\b{\b{\}}}*[463].\}
 \\[1ex] ...\bs footnote\{\bs kqu\{\}
 \\ \ \ \ \ \ \{Aufmarschanweisungen 
 \\ \ \ \ \ \ \ 1912\}[464].\}
 \\[1ex] ...\bs footnote\{Innen kurz: 
 \\ \bs vqu \{\} \{\} 
 \\ \H{\{}\bs ktit\{Aufmarsch 1913/14\},
 \\ \ abgedruckt in: 
 \\ \ \bs xkqu\{Ehlert\}
 \\{}\ \ *\{\bs midkauthor\{Epkenhans\} 
 \\ \ \ \ \ \bs kauthor\{Gro"s\} [Hrsg.]\}
 \\ \ \ \{Schlieffenplan\%
 \\ \ \ \}[467-477]\H{\}}*[469].\} 
}
{\vspace{9ex}\showbamem
 ...\footnote{Innen vollzitiert: \vqu {} {} 
   {\ktit{Aufmarschanweisungen 1912}, 
   abgedruckt in: \xvqu{Hans} {Ehlert}
   *{\midvauthor{Michael} {Epkenhans}
     \vauthor{Gerhard P.} {Gro"s} [Hrsg.]} 
                {Der \ktit{Schlieffenplan},
     Paderborn 2007}[462-466]}*[463].\label{InnVoll}}
                
   ...\footnote{\kqu{} {Aufmarschanweisungen 1912}[464].\label{zweite}}
   %%
   %\footnote{\kqu{}{Aufmarschanweisungen 1912!!!!!}[464].}    %% Aussen anders %%
   %%
   %\newbox\mybox
   %\setbox\mybox=\hbox{\footnotetext{\printonlyvqu{}{} {\xprintonlykqu{Ehlert} 
   %   *{\midkauthor{Epkenhans} \kauthor{Gro"s} [Hrsg.]}{Schlieffenplan}}}}

 ...\footnote{Innen kurz: \vqu {} {} 
    {\ktit{\onlyhere{\onlykurz{\,}}Aufmarsch 1913/14},
    abgedruckt in: \xkqu{Ehlert}
    *{\midkauthor{Epkenhans} \kauthor{Gro"s} [Hrsg.]}
    {Schlieffenplan%
      }[467-477]}*[469].\label{dritte}}
}


\vspace{.5ex}\noindent
Ein k\fhy Befehl, der den Eintrag der "au"seren Speicherebene wiederholt, 
l"asst die innere Ebene also unber"uhrt: In Fu"snote\,\ref{dritte} wurde 
ein inneres \textsc{ebd.} gesetzt. 

Die Zwischenspeicher lassen sich mit \verb|\showbamem| auch ansehen. 
Dies kann bei Problemen mit dem \textsc{ebd.}\hy Setzen
helfen.\footnote{\BibArts\ gibt der \LaTeX-\texttt{minipage}\hy Umgebung 
eigene Speicher; die \textsc{ebd.}\hy Setzung in \texttt{minipage}\hy
Fu"snoten erfolgt deshalb unabh"angig von Fu"snoten im "ubrigen Text.}
\BibArts\ druckt auf den Bildschirm aus (\verb|o-ref| bzw.\ \verb|i-ref| nennen 
dabei den f"ur \textsc{ebd.} gesuchten Inhalt):

{\scriptsize\vspace{3ex}\noindent
\verb|   FNT |\texttt{\ref{zweite}}
\vspace{-3ex}
\begin{verbatim}
   -- outer:  {qu}{}{}{Aufmarschanweisungen 1912} -- 
   ---------- inner:  {qu}{Ehlert}{\midkauthor {Epkenhans}
                        \kauthor {Gro"s} [Hrsg.]}{Schlieffenplan} -- 
   -- o-ref:  {qu}{}{}{Aufmarschanweisungen 1912} -- 
\end{verbatim}}


\vspace{-.5ex}\noindent
Nun h"atte in der mittleren Fu"snote alternativ auch ein dritter Aufsatz aus dem gleichen
Herausgeberband kurzzitiert sein k"onnen. Rein logisch d"urfte in der letzten 
Fu"snote dann weiterhin \textsc{ebd.} stehen. Steht aber etwas Anderes als 
\verb|\kqu{}{Aufmarschanweisungen 1912}| in der 
zweiten Fu"snote, unterbleibt ohne weitere Ma"snahmen das innere \textsc{ebd.}\hy Setzen in 
der dritten.\footnote{Um doch das innere \textsc{ebd.} zu kriegen:
\texttt{\bs newbox\bs mybox} im Vorspann und vor Fu"snote~\ref{dritte}:
\\[.5ex] \hspace*{.75em} \texttt{\bs setbox\bs mybox=\bs hbox\{\bs footnotetext\{\bs printonlyvqu\{\}\{\}
\\ \hspace*{1em} \{\bs xprintonlykqu\{Ehlert\} 
\\ \hspace*{1.5em} *\{\bs midkauthor\{Epkenhans\} \bs kauthor\{Gro"s\} [Hrsg.]\}
\\ \hspace*{1.5em} \{Schlieffenplan\}\}\}\}}}

\vspace{1.5ex}\noindent
Nun wird das Verzeichnis gedruckter Quellen mit \verb|\printnumvqu| gedruckt:


\balabel{printnumvqu}\printnumvqu

\noindent
Der Herausgeberband 
\textsc{Ehlert}, Hans\baslash""Michael \textsc{Epkenhans}\baslash""Gerhard P.\pdfko{.25}\  
\textsc{Gro"s} 
bekam auf der Liste einen \textit{eigenen} Volleintrag, den \BibArts\ automatisch\pdfko{1}\  
aus dem 'inneren' Vollzitat in Fu"snote~\ref{InnVoll} erzeugte 
(S.\,\pageref{InnVoll}). In den Listenpunkten "`Aufmarsch"' und 
"`Aufmarschanweisungen"' druckte \BibArts\ die 'inneren'\pdfko{1}\ 
Angaben dagegen als Kurzzitat. Damit \BibArts\ dort v\hy\ in k\fhy Angaben 
umwandeln kann, m"ussen Kurztitel in 'inneren' v\fhy Befehlen stets mit 
\verb|\ktit| markiert\pdfko{1.25}\  
sein (nur bei '"au"seren' v\fhy Befehlen ohne \verb|\ktit| gibt es keine Fehlermeldung). 

Beim Ausdruck von v\fhy Listen ergeben Zug"ange, die auf v\fhy Befehle mit leeren 
Namensargumenten  (\verb|\vqu{}{}{...}|) zur"uckgehen, stets \printonlyvqu{}{}{...}\,.\pdfko{.75}\  
Und trotz \verb|bibsort -k| wird der zweite anonyme Autor nicht als $\sim$ gedruckt.\pdfko{.5}


\vspace{2ex}\noindent
Gelegentlich sollen Teile der Literaturangaben nur in der Liste erscheinen, jedoch 
nicht in der Fu"snote. Die Reihenangaben hier sind nur in obiger Liste:

\vspace{.75ex}
\Doppelbox
{...\bs footnote\{\bs vqu \{Karl\}\{Marx\}
 \\ \b{\b{\{}}Das \bs ktit\{Kapital\%
 \\ \ \bs onlyhere\{\string~I\}\}\%
 \\ \ \bs onlyout \{. Kritik der 
 \\ \ \ politischen "Okonomie, 
 \\ \ \ erster Band; das ist 
 \\ \ \ Bd.\bs,23 (1962) von:\}%
 \\ \ \bs onlyhere\{, in:\}
 \\ \ \bs xvqu [m]\{Karl\}\{Marx\}
 \\ \ \ *\{\bs vauthor\{Friedrich\}\{Engels\}\}
 \\ \ \ \b{\{}\bs ktit\{Werke\}, 
 \\ \ \ \ \bs onlyout \{hrsg. vom Institut 
 \\ \ \ \ \ f"ur Marxismus-Leninismus 
 \\ \ \ \ \ beim ZK der SED, 40\string~Bde.
 \\ \ \ \ \ Berlin 1958--1971\}\%
 \\ \ \ \ \bs onlyhere\{Berlin 1962\}\b{\}}\b{\b{\}}}[49].\}
 \\ \
 \\ ...\bs footnote\{\bs kqu\{Marx\} 
 \\ \ \ \{Kapital\bs onlyhere\{\string~I\}\}[49].\}
}
{...\footnote{\vqu {Karl}{Marx}
 {\onlyhere{~}Das \ktit{Kapital%
   \onlyhere{~I}}%
   \onlyout {.
   Kritik der politischen "Okonomie, erster Band;
   das ist Bd.\,23 (1962) von:}%
          \onlyhere{, in:}
    \xvqu [m]{Karl}{Marx}
     *{\vauthor{Friedrich}{Engels}}
     {\ktit{Werke}, \onlyout {hrsg.\ vom Institut f"ur
              Marxismus-Leninismus beim ZK der SED, 40~Bde.\
      Berlin 1958--1971}%
                        \onlyhere{Berlin 1962}}}[49].\balabel{ders}}
                        
 ...\footnote{\kqu{Marx} {Kapital\onlyhere{~I}}[49].\label{I}}
}

\vspace{.75ex}\noindent 
Das Argument von \verb|\onlyhere| wird nur in
Haupttext oder Fu"snote, das Argument von \verb|\onlyout|
nur in den Listen ausgedruckt. Im Beispiel steht von der
"au"seren Angabe die Nummer~("`I"') in der Fu"snote und statt
dessen eine genauere Angabe zum Band auf der Liste. Von
der inneren Angabe wurde die Institution der Herausgeber 
nur auf der Liste (ganz unten) ausgedruckt.\kern-1pt\footnote{Falls 
Fu"snote~\ref{I} \texttt{\bs kqu\{Marx\}\{Kapital\string~I\}}
enthielte, w"urde dort auch \textsc{ebd.} gesetzt; Ziel war
aber, im Kurzzitateverzeichnis S.\,\pageref{vkc} nur 
\textit{einen} Eintrag \textsc{Marx}: Kapital [Q] f"ur alle
Teilb"ande des "`Kapital"' zu bekommen.}\pdfko{.5}

Ein Vergleich mit der Liste auf der Vorseite zeigt, dass das 
\verb|[m]| nach dem\pdfko{1}\ 
inneren \verb|\xvqu|\hy Befehl 
\textsc{ders.\baslash Engels} erzeugte \baref[Eintrag von]{ders}.

Neben der Markierung von Text in \BibArts\hy Befehlen mit 
\verb|\onlyhere| und\pdfko{1}\ 
\verb|\onlyout| gibt es eine "altere M"oglichkeit, 
unterschiedliche Eintr"age in Text und Liste zu erzeugen: \BibArts\hy 
Hauptbefehle (S.\,\pageref{Hauptbefehle}) lassen sich aufsplitten in 
eine printonly- und eine addto\hy Komponente, also in die 
Aufgabenteile 'Schreibe an Ort und Stelle' und 'Schreibe in die Liste'. 
\label{printonly} \verb|\vqu| beispielsweise l"asst\pdfko{1}\ 
sich durch \verb|\printonlyvqu| plus \verb|\addtovqu| ersetzen. Die 
Syntax ist identisch.

\Doppelbox
{...\bs footnote\{
 \\ \ \bs addtovqu\{Carl von\}\{Clausewitz\}
 \\ \ \ \b{\b{\{}}\bs ktit\{Strategie\}. Hrsg. von 
 \\ \ \ \ \bs vauthor\{Eberhard\}\{Kessel\}, 
 \\ \ \ \ Hamburg 1937\b{\b{\}}}\%
 \\[.2ex] \ \bs printonlyvqu\{Carl von\}
 \\ \ \ \ \ \{Clausewitz\}
 \\ \ \ \b{\{}\bs ktit\{Strategie\}, 
 \\ \ \ \ Hamburg 1937\b{\}}[58].\}
}
{Der Herausgeber Eberhard Kessel erscheint
 nur auf der Liste der gedruckten Quellen,
 aber nicht in der Fu"snote.\footnote{
 \addtovqu{Carl von}{Clausewitz}
  {\ktit{Strategie}. Hrsg.\ von 
         \vauthor{Eberhard}{Kessel}, 
         Hamburg 1937}%
 \printonlyvqu{Carl von}{Clausewitz}
  {\ktit{Strategie}, 
         Hamburg 1937}[58].}
}

\noindent
Da \textit{in} addto\hy Komponenten \textit{innere Befehle} nicht abgearbeitet 
werden, m"ussen solche Komponenten au"serhalb stehen. 
\verb|\onlyhere| und \verb|\onlyout| sind also zu bevorzugen. 
Weiter lassen sich innere v\hy\ und k\fhy Befehle so ganz vermeiden:

\Doppelbox
{\vspace{.3em}
 \bs vqu \{Karl\} \{Marx\} 
 \\ \b{\{}Das \bs ktit\{Kapital\}. Kritik der 
 \\ \ politischen "Okonomie, erster 
 \\ \ Band; das ist Bd.\bs,23 (1962) 
 \\ \ von: \bs midkauthor\{ders.\} 
 \\ \ \bs ntvauthor\{Friedrich\}\{Engels\}
 \\ \ Werke, 
 \\ \ \bs ersch\string|40\string|\{Berlin\}\{1958-{}-1971\}\b{\}}
 \vspace{.2em}
}
{\footnotesize\notktitaddtok
 \texttt{\%HIER nicht in Listen "ubernommen\%}
 \\[.5ex] \printonlyvqu {Karl} {Marx} {Das \ktit{Kapital}. Kritik der politischen 
 "Okonomie, erster Band; das ist Bd.\,23 (1962) von: 
 \midkauthor{ders.} \ntvauthor{Friedrich}{Engels} 
 Werke, 
 \ersch|40|{Berlin}{1958--1971}}
}\label{ntvauthor}%

\vspace{.5ex}\noindent 
Dies leitet "uber zur \textbf{Umdefinition vorgefertigter Textelemente}:
\BibArts\ l"asst gro"se Freiheiten bei der Wahl von Zitierkonventionen. 
Die Schr"agstiche definiert \verb|\nsep|, \label{nsep1} das seinerseits \verb|\baslash| 
('\baslash') ausf"uhrt. mid\hy Befehle und\pdfko{1}\ 
die Sternargumente von 
x\fhy Befehlen nutzen ihn. \verb*|\renewcommand{\nsep}{, }|\pdfko{.75}\  
w"urde Komma statt Schr"agstrich zwischen Namen drucken. Dies kann 
auch lokal geschehen: Die jeweils aktuelle Definition von \verb|\nsep| reist 
\label{Ausreise} mit jedem v- und k\fhy Zugang \textit{separat} in die Listen  
und wird dort reproduziert. Die Voreinstellung l"asst sich mit
\verb|\renewcommand{\nsep}{\baslash}| wiederherstellen.

Ein weiterer Separator, \verb|\ntsep|, \label{ntsepA} der zwischen Name und 
Titel '\printntsep' druckt,\pdfko{.5}\ 
sollte dagegen nur im Dokumentenvorspann ge"andert werden. 
Ausgef"uhrt wird \verb|\ntsep| von v- und k\fhy Befehlen 
sowie \verb|\ntvauthor| und \verb|\ntkauthor|. Gelegentlich 
ist ein lokal auf den \textit{Ausdruck ganzer Listen} 
beschr"anktes "Andern\pdfko{1.125}\ 
von \verb|\ntsep| sinnvoll und k"onnte etwa 
\verb*|\renewcommand{\ntsep}{, }| lauten. 

Im letzten Beispiel wurde auch \verb!\ersch|40|{Berlin}{1958--1971}! \label{ersch}
verwendet, was ausgedruckt ergibt:\hspace{.1em} {\small \ersch|40|{Berlin}{1958--1971}}\kern.1em.
Dabei ist \verb!|40|! optional. Ein normales Buch kann \textit{am Ende} des
letztes Arguments eines v\fhy Befehls stets etwas stehen haben wie\hspace{.1em}
\verb!\ersch{Berlin}{2003}!\hspace{.2em} $-$ das ergibt:\hspace{.1em} {\small \ersch{Berlin}{2003}} $-$\,; 
oder auch\hspace{.1em} \verb!\ersch[2]{Berlin}{2003}!\,, was\hspace{.1em} {\small \ersch[2]{Berlin}{2003}}\hspace{.1em}
ergibt. Und\hspace{.1em} \verb!\ersch{}{}!\hspace{.1em} druckt\hspace{.1em} 
{\small \ersch{}{}}\hspace{.1em} $-$ also: ohne Ort, ohne
Jahr. Nach\hspace{.1em} \verb|\exponenteditionnumber|\hspace{.1em} druckt
\verb!\ersch|5|[2]{Mainz}{2008}! aus:\hspace{.1em} {\small \exponenteditionnumber
\ersch|5|[2]{Mainz}{2008}}\hspace{.1em} (Auf"|lageexponent). Das sonst verwendete 
{\small `\gerscheditionname'}\,, definiert als\hspace{.2em} 
\verb|{\teskip Auf{\kern.03em}l.,}|\,, kann dudengerecht  ge"andert werden in
{\small \renewcommand{\gerscheditionname}{\teskip Aufl.,} `\gerscheditionname'}\hspace{.2em} 
mittels\hspace{.2em} \verb+\renewcommand{\gerscheditionname}{\teskip Aufl.,}+\,. 
%% ohne Ligatur sagt: Duden S.96 %%


\vspace{1.25ex}\label{female}\noindent
Falls in den \textit{v\fhy Listen bei Autorwiederholung} \textsc{dies.} 
oder \textsc{ders.} statt $\sim$ stehen soll, k"onnen Sie \verb|\female|
bzw.\ \verb|\male| in die v-Befehle zu Anfang der Vornamensargumente
tippen. Beispiel: \verb|\vqu{\male Karl}{Marx}{|...\verb|}| Das muss $-$\,einmal
etwa f"ur die vli-Liste angefangen\,$-$ dann aber in jedem vli-Befehl stehen
(ausgenommen anonyme Autoren \verb|\vli{}{}{|...\verb|}|): Nur so wird w\kern-.1em/m von 
\hspace{.2em}\verb|bibsort -k|\hspace{.2em} richtig zugeordnet (gleiche Namen 
gelten auch dann als gleich, wenn \verb|\female| oder \verb|\male| vergessen
wird; evtl.\ gilt dann das Geschlecht der vorausgehenden Person). 
Sind dann auch alle Koautoren gleich, wird automatisch \textsc{diesn.} 
f"ur 'Dieselben' gesetzt. Falls nur die ersten von mehreren Koautoren gleich sind, 
wird f"ur \textit{die} weiterhin {\small$\sim$} oder {\small$\sim$\baslash$\sim$} 
gesetzt.\pdfko{.5}  

\vspace{.675ex}\noindent
Um den \textit{Text} zu ver"andern (\textit{nicht die} \textsc{schrift}\ko\textit{!}), k"onnen
\verb|\geademname| (f"ur \textsc{die"-selbe}),
 \verb|\gidemname| (\textsc{der"-selbe}) und 
\verb|\giidemname| (\textsc{die"-selben})
mit \verb|\renewcommand| ver"andert werden. Diese Definitionen bestimmen auch, 
was von \verb|[f]|, \verb|[m]| und \verb|[p{}]| ausgedruckt wird (Schalter von  v- und k\fhy Befehlen).


\vspace{1.25ex}\noindent
Dagegen erfolgt ein \textit{Umstellen von \textsc{ebd.}} mit \label{setibidem}
\verb|\setibidem{g}{ebenda}{}| in\pdfko{1.25}\ 
\textsc{ebenda}. Die Voreinstellung ist
\verb|\setibidem{g}{ebd\kern -0.07em}{.}| in\pdfko{1.25}\  
\verb|bibarts.sty|. Das dritte
Argument kann nur entweder leer sein oder einen Punkt enthalten; es dient 
dazu, \BibArts\ mitzuteilen, wie beim automatischen\pdfko{.5}\  
\textsc{ebd.}\hy Setzen mit
einem nach dem k\fhy Befehl stehenden Punkt umgegangen werden soll (um 
\textsc{ebd.}.\ zu vermeiden). Nur hier ist \verb|\renewcommand| verboten!\pdfko{1.25}


\vspace{1.25ex}\noindent
Die Schrift, in der \textit{Autoren\hy Nachnamen} gesetzt sind, ist \verb|\authoremph|.
Mit\pdfko{1.25}\  
\verb|\renewcommand{\authoremph}{\upshape}| lie"se sich die voreingestellte
Hervorhebung von \textsc{Nachnamen} beim Ausdruck von v- und k\fhy Befehlen 
aufheben. Alternativ kann \verb|\stressing| \textit{ein} Schriftbefehl
ohne~\kern-.15em\verb|\| "ubergeben werden: \textit{Etwa} \verb|\stressing{underbar}| 
\textit{initiiert 
{\itshape\notprinthints\stressing{underbar}\printonlykli{Meyer}{}} auch
in kursivem Umfeld.}


\vfill\noindent{\sffamily 
Sprachabh"angig vorgefertigte Textelemente folgen in Kapitel~\ref{SprachSep} 
unten ab S.\,\pageref{SprachSep};}
\\[.25ex]{\sffamily 
einstellbare Texthervorhebungen liste ich in Kapitel~\ref{hervor} 
unten S.\,\pageref{hervor} auf;}
\\[.25ex]{\sffamily 
und der Literaturtyp \textit{Zeitschriften} kommt gleich in 
Kapitel~\ref{per} unten ab S.\,\pageref{per}.}%


\newpage 
Da das \textbf{\textsc{ders.}\hy Setzen} mit \verb|[f]|, \verb|[m]| 
oder \verb|[p{}]| fehleranf"allig ist, falls Textteile im 
Texteditor ausgeschnitten und verschoben werden, gibt es 
eine weitere \textbf{Kontrollm"oglichkeit}: "Uber den \LaTeX\hy 
Bildschirmausdruck hinaus \baref{DERS} k"onnen Sie sich testweise 
im Ausdruck selbst informieren lassen:

\vspace{1ex}
\Doppelbox
{\bs writeidemwarnings
 \\ \ \bs footnote\{\bs kqu[m]\{Clausewitz\} 
 \\ \ \ \ \{Strategie\}[61] und 
 \\ \ \ \bs kqu\{Clausewitz\}
 \\ \ \ \ \{Vom Kriege\}[62].\} 
 \\[1ex] \ \bs footnote\{\bs kqu[m]\{Clausewitz\} 
 \\ \ \ \{Strategie\}[63].\}
 \\[1ex] \ \bs footnote\{\bs kqu[m]\{Clausewitz\} 
 \\ \ \ \ \{Vom Kriege\}[64] und 
 \\ \ \ \bs kli[m]\{Luhmann\}\{Soziale 
 \\ \ \ \ Systeme\}[65].\}
 \\[1ex] \ \bs footnote\{\bs kqu[m]\{Clausewitz\} 
 \\ \ \ \{Strategie\}[66].\}
 \\[1ex] \ \bs footnote\{\bs vqu[m]\{Karl\}\{Marx\} 
 \\ \ \ \b{\b{\{}}Das 
 \\ \ \ \ \bs ktit\{Kapital\bs onlyhere\{\string~II\}\}\%
 \\ \ \ \ \bs onlyout\{. Kritik der 
 \\ \ \ \ \ politischen "Okonomie, 
 \\ \ \ \ \ zweiter Band; das ist 
 \\ \ \ \ \ Bd.\bs,24 (1962) 
 \\ \ \ \ \ von:\}\bs onlyhere\{, in:\} 
 \\ \ \ \ \bs xvqu \{Karl\}\{Marx\}
 \\ \ \ \ \ *\{\bs vauthor
 \\ \ \ \ \ \ \ \ \{Friedrich\}\{Engels\}\} 
 \\ \ \ \ \ \b{\{}\bs ktit\{Werke\}, 
 \\ \ \ \ \ \ \bs onlyout\{hrsg.\ vom Institut 
 \\ \ \ \ \ \ \ f"ur Marxismus-Leninismus 
 \\ \ \ \ \ \ \ beim ZK der SED, 40\string~Bde. 
 \\ \ \ \ \ \ \ Berlin 1958-{}-1971\}\%
 \\ \ \ \ \ \ \bs onlyhere\{Berlin 
 \\ \ \ \ \ \ \ 1962\}\b{\}}\b{\b{\}}}[67].\}
}
{\writeidemwarnings
 \footnote{\kqu[m]{Clausewitz} {Strategie}[61] und 
      \kqu{Clausewitz} {Vom Kriege}[62].} 
 \\[8.25ex]
 \footnote{\kqu[m]{Clausewitz} {Strategie}[63].}
 \\[3ex]
 \footnote{\kqu[m]{Clausewitz} {Vom Kriege}[64] und 
      \kli[m]{Luhmann}{Soziale Systeme}[65].}
 \\[8.25ex]
 \footnote{\kqu[m]{Clausewitz} {Strategie}[66].}
 \\[3ex]
 \footnote{\vqu[m]{Karl}{Marx} {Das 
  \ktit{Kapital\onlyhere{~II}}%
                     \onlyout{. Kritik der politischen 
  "Okonomie, zweiter Band; das ist Bd.\,24 (1962) 
  von:}\onlyhere{, in:} \xvqu {Karl}{Marx}
  *{\vauthor{Friedrich}{Engels}} {\ktit{Werke}, \onlyout{hrsg.\ 
  vom Institut f"ur Marxismus-Leninismus beim ZK der SED, 
  40~Bde.\ Berlin 1958--1971}%
        \onlyhere{Berlin 1962}}}[67].
        \\
        \texttt{\% Vgl.\ DERS. in der Liste \baref[]{printnumvqu}}
        }
}

\vspace{1ex}\noindent
Nach Setzen von \verb|\writeidemwarnings| druckten v- und k\fhy Befehle 
dabei in Klammern deren Nachnamensargument hinter folgenden Symbolen aus:

 \vspace{2ex}\noindent
 {\small\begin{tabular}{cl}%
 $\bullet$    & \textsc{\footnotesize DERS.} \sffamily fehlt m"oglicherweise (gleiche Nachnamen registriert).              \\
 $\heartsuit$ & \textsc{\footnotesize DERS.} \sffamily ist offenbar richtig gesetzt (gleiche Nachnamen registriert).       \\
 $\nabla$     & \textsc{\footnotesize DERS.} \sffamily wegen fehlender Autoren in vorausgehender Fu"snote unberechtigt.    \\
 $\spadesuit$ & \textsc{\footnotesize DERS.} \sffamily "uberschreibt einen Namen, der nicht der vorausgehende ist.          \\
 $\clubsuit$  & \textsc{\footnotesize DERS.} \sffamily steht irref"uhrenderweise nach einer Fu"snote mit mehreren Autoren. \\
 \end{tabular}}

\vspace{2ex}\noindent
\BibArts\ kontrolliert niemals Koautoren. Falls auch die in
aufeinanderfolgenden Fu"snoten gleich sind, lassen sie sich zwar durch 
Ersatzworte ersetzen, wozu \BibArts\ bei Anwenderfehlern aber nicht warnt. 
Hier ein Beispiel ohne Fehler:

\Doppelbox
{\vspace{1.25ex}%
 \bs footnote\{.... \bs xkli 
  \\ \ \{Maier\} *\H{\{}\bs midkauthor\{M"uller\} 
        \\ \ \ \ \bs kauthor\{Huber\}\H{\}} \{Geld\}[i].\}
 \\[1ex] \bs footnote\{\bs xkli[p\{\}] 
  \\ \ \{Maier\}  *\{\bs midkauthor\{M"uller\} 
        \\ \ \ \ \bs kauthor\{Huber\}\} \{Haus\}[ii].\}
 \\[1ex] \bs footnote\{\bs xkli 
 \\ \ [p\b{\b{\{}} ersten beiden und 
 \\ \ \ \ \ \ \bs kauthor\{Schmidt\} \bs editors\b{\b{\}}}]
 \\ \  \{Maier\} *\b{\{}\bs midkauthor\{M"uller\} 
 \\ \ \ \ \ \ \bs kauthor\{Schmidt\} \bs editors\b{\}} 
 \\ \ \{Vorsorge\}[iii].\}
 \vspace{1ex}
}
{\texttt{\% HIER nicht in den Listen \%}
 \\[3ex]
 \footnote{Lokal erst \xprintonlykli 
    {Maier} *{\midkauthor{M"uller} \kauthor{Huber}} {Geld}[i].}
                
 \footnote{\xprintonlykli[p{}] 
    {Maier} *{\midkauthor{M"uller} \kauthor{Huber}} {Haus}[ii].}
                
 \footnote{\xprintonlykli[p{ ersten beiden und \kauthor{Schmidt} \editors}]
    {Maier} *{\midkauthor{M"uller} \kauthor{Schmidt} \editors} {Vorsorge}[iii].\label{wieder}}
}


\noindent\label{p}%
Nur mit \verb|[p{}]| werden alle Namen mit \textsc{diesn.}\ko\ (\verb|\giidemname|)
"uberschrieben $-$ und nicht wie mit \verb|[p]| 
nur der erste Name! Falls mehrere, aber eben 
nicht alle Autoren \textit{dieselben} sind, k"onnen Sie die zuviel mit
\textsc{diesn.} "uberschriebenen wie oben in Anm.\,\ref{wieder} gezeigt in 
\verb|[p{|\textit{xx}\verb|}]| wieder nennen. Auch Attribute wie \verb|\editors|
w"aren in \textit{xx} zu nennen. (Bei der v\fhy Liste, die\hspace{.2em} 
\verb|bibsort -k|\hspace{.2em} erzeugt, werden\hspace{.2em} $\sim$\baslash$\sim$ \textit{Attribute}\hspace{.2em} 
"ubrigens \textit{aus den Hauptargumenten} "ubernommen.) 

Um \textit{richtig sortierte Listen} zu erzeugen, sollten in v- 
und k\fhy Befehlen die 'regul"aren' Namensargumente
\textit{in jedem Fall} vollst"andig bef"ullt sein (obwohl \textit{die} 
in der Fu"snote gar nicht gedruckt werden, sondern \textsc{ders.} etc.).

\vspace{1.25ex}\noindent
Dass bei \textit{inneren} v- und k\fhy Befehlen gesetzte Schalter \verb|[f]|, 
\verb|[m]| und \verb|[p{}]| in\pdfko{1}\ 
die Listen "ubernommen werden, wurde oben 
beim Ausdruck des Verzeichnisses der gedruckten Quellen anhand des ersten 
Bandes von Marx' Kapital demonstriert. Beim zweiten Band dagegen ist 
der 'innere' Marx nicht mit \verb|[m]| versehen; beim "Ubersetzen der 
\textit{Fu"snote}, aus der der Eintrag herstammt, erscheint die Warnung 
{\footnotesize\texttt{Inner ...vqu-cmd repeats author's lastname}}. 
Falls Sie\pdfko{.5}\ nur dort kein \textsc{ders.} haben wollten, k"onnen
Sie vor den inneren v- oder k\fhy Befehl \verb|\notwarnsamename| 
setzen, um die Warnung lokal auszuschalten. Beim 
"Ubersetzen der \textit{Listen} erfolgen solche Warnungen "ubrigens niemals.

\vfill\noindent
Falls Sie \textsc{dies.}, \textsc{ders.} und \textsc{diesn.} nicht
verwenden wollen, k"onnen Sie alle diesbez"uglichen Warnungen auch 
mit \verb|\notwarnsamename| im Vorspann Ihres \LaTeX\hy Textes
global ausschalten. Das unterbindet bei der \LaTeX\hy "Ubersetzung die 
Bildschirmwarnung {\small\texttt{cmd repeats (first) author's 
lastname}} samt allen weiteren Meldungen f"ur die eben 
aufgelisteten Fehlertypen. \textit{Zus"atzlich} wird 
\verb|\writeidemwarnings| unwirksam; \BibArts\ druckt also nicht 
mehr {\small$\bullet\heartsuit\nabla\spadesuit\clubsuit$}\,.


\newpage\noindent
Zum \textbf{Zitieren mehrb"andiger Werke} gibt es einen Speicher 
f"ur Bandnummern. Das optionale Argument \verb+|+\textit{Band}\verb+|+
zur \textsc{ebd}.\hy Setzung steht ohne Leerzeichen vor dem 
Seitenargument (\verb+\ersch|+\textit{Band}\verb+|+... bef"ullt den Speicher nicht):

\label{Reinhard}%
\Doppelbox
{\bs footnote\{... \bs vli\{Wolfgang\} 
 \\ \ \{Reinhard\} \b{\b{\{}}Geschichte der
 \\ \ \ \bs ktit\b{\{}\bs onlyvoll\{e\}\%
 \\ \ \ \ \ \ \ \ \ \bs onlykurz\{E\}urop"aische\%
 \\ \ \ \ \ \ \ \ \ \bs onlyvoll\{n\} Expansion\b{\}}, 
 \\ \ \ \bs ersch\string|4\string|\{Stuttgart\}
 \\ \ \ \ \ \ \ \ \ \ \ \{1983-{}-1990\}\b{\b{\}}}\string|2\string|[98].\}
 \\[1ex] \bs footnote\{\bs kli \{Reinhard\} \{Europ"aische Expansion\}\string|2\string|[98].\}
 \\[1ex] \bs footnote\{\bs kli \{Reinhard\} \{Europ"aische Expansion\}\string|3\string|[1].\}
}
{\vspace{1.5ex}%
 \footnote{Band aus Reihe: \notktitaddtok\printonlyvli{Wolfgang} {Reinhard} {Geschichte der
 \ktit{\onlyvoll{e}%
   \onlykurz{E}urop"aische%
         \onlyvoll{n} Expansion},
 \ersch|4|{Stuttgart} {1983--1990}}|2|[98].\label{Ranf}}
 
 \footnote{\printonlykli {Reinhard} {Europ"aische Expansion}|2|[98].}
 
 \footnote{\printonlykli {Reinhard} {Europ"aische Expansion}|3|[1].\label{Rend}}
}

\noindent
St"unden hier \verb|\vli| und \verb|\kli| statt der
tats"achlich verwendeten printonly-Befehle, ginge ins Kurzzitateverzeichnis  
\textsc{Reinhard}: Europ"aische Expansion [L] \hspace{.5em} 
\textsf{\pageref{Ranf}$^{\ref{Ranf}-\ref{Rend}}$} 
und ins Literaturverzeichnis: 
\textsc{Reinhard}, Wolfgang: Geschichte der europ"aischen Expansion, 
4 Bde., Stuttgart 1983--1990.

\vspace{.1ex}
W"urde die Bandangabe \verb+|3|+ \hspace{.1em}(oder Nummer \hspace{-.3em}\verb+_n_+)\hspace{.1em} in 
Fu"snote~\ref{Rend} fehlen, dann erschiene w"ahrend der \LaTeX\hy "Ubersetzung die 
Fehlermeldung:\footnote{Eine Warnung bei 'innerer' \textsc{ebd.}\hy Setzung erfolgt nur,
wenn \textit{innen} Band- bzw.\ Seitenangaben fehlen
\textit{und} zuvor (auch) entsprechende '"au"sere' Angaben standen. Sie erhalten\pdfko{.5}\
also \textsc{ebd.}, aber u.\,U. keine Warnung, wenn Sie innere 
Bandangaben zu tippen vergessen!}

\vspace{.5ex}\noindent 
\label{pervol}%
{\small\texttt{ !~Same title, before with :\{pervol\}\{2\}:, has now no no./vol number.}}

\vspace{.75ex}\noindent
Dagegen d"urfte die \verb|[1]| nach der \verb+|3|+ wegbleiben, weil es 
sich um einen anderen Band als in der vorausgehenden 
Fu"snote handelt. $-$ Im Falle von \textbf{mehrb"andigen Herausgeberwerken}
sind auch 'innere' Bandangaben erlaubt:

\vspace{-.2ex}%
\Doppelbox
{\vspace{1.75ex}\bs footnote\{\bs vli \{\}\{\}
 \\ \ \b{\{}\bs ktit\{Au"sen~1\}, in:
 \\ \ \ \bs vli \{\}\{\} \b{\b{\{}}\bs ktit\{Innen\}\b{\b{\}}}\%
 \\ \ \ \ \ \ \ \string|12\string|[100-199]\b{\}}*[111].\}
 \\[.5ex] \bs footnote\{\bs kli\{\}\{Au"sen~1\}[111].\}
 \\[.5ex] \bs footnote\{... \bs vli\{\}\{\}
 \\ \ \H{\{}\bs ktit\{Au"sen~2\}, in: \bs kli \{\} 
 \\ \ \ \ \{Innen\}\string|12\string|[200-299]\H{\}}*[222].\}
}
{\footnote{\notktitaddtok\printonlyvli {}{}%
  {\ktit{Au"sen~1}, in:
    \printonlyvli {}{}
        {\ktit{Innen}}%
          |12|[100-199]}*[111].}
 \footnote{\printonlykli{}{Au"sen~1}[111].}
 \footnote{Gleiche Reihe: \notktitaddtok\printonlyvli{}{}
        {\ktit {Au"sen~2}, in:
    \printonlykli {} {Innen}|12|[200-299]}*[222].}
}

\vspace{-.2ex}\noindent
\BibArts\ pr"uft nicht, ob \verb|*[|\textit{Seite}\verb|]| innerhalb des 
genannten Seitenintervalls liegt.

\newpage\noindent
Wie sind \textbf{Werke} in die Listen aufzunehmen, \textbf{die im Text nie 
verwendet wurden}? Solche Werke d"urfen auf den num\hy Listen nicht mit den 
Seiten\fhy/""Fu"snotennummern der Stelle gedruckt werden, an der die Angaben 
im \hspace{-.2em}\verb|.tex|\hy File stehen; normale addto\hy Befehle sind also 
ungeeignet. Stattdessen gibt es die \balabel{unused} \verb|{unused}|\hy 
Umgebung. Dort eingef"ugte \BibArts\hy Hauptbefehle bef"ullen nur\pdfko{.25}\ 
die Listen, sind aber im DVI\fhy\slash PDF\hy File unsichtbar. Ein Beispiel:

\vspace{-.25ex}
{\small\begin{verbatim}
  \begin{unused}
    \vli{James M.}{McPherson}{\ktit{Battle Cry of Freedom}. The 
       American Civil War, Oxford 1988}[vi] 
  \end{unused}
\end{verbatim}}

\vspace{-.25ex}
%\begin{unused}
%\vli{James M.}{McPherson}{\ktit{Battle Cry of Freedom}. The 
%American Civil War, Oxford 1988}[vi]
%\end{unused}
%%
\noindent Seitenzahlen in der Literaturangabe wie hier \verb|[vi]| 
werden ignoriert. Somit lassen sich Literaturangaben 
w"ahrend des Schreibens einfacher von einer Fu"snote in die 
\verb|{unused}|\hy Umgebung verschieben. In diesen Umgebungen d"urfen Sie 
alle \BibArts\hy Hauptbefehle wie etwa \verb|\vli| und \verb|\vqu| sowie 
die unten Seite~\pageref{per} und \pageref{archivquellen} eingef"uhrten 
Befehle \verb|\per| und \verb|\arq| nutzen; dazwischen d"urfen 
Leerzeichen und \textit{einfache} Zeilenumbr"uche stehen. 
Au"serdem sind addto\hy Befehle erlaubt; tats"achlich sieht \BibArts\ 
\verb|\vli|\hy Befehle nun als \verb|\addtovli|\hy Befehle unbestimmter Herkunft.
\verb|\printonlyvli| hat in \verb|{unused}|\hy Umgebungen nat"urlich nichts 
zu suchen. Ganz am Ende Ihrer\hspace{-.2em} \verb|.tex|\hy Datei 
$-$ insbesondere nach Ende einer \verb|twocolum|\fhy Umgebung $-$ 
sind \verb|{unused}|\hy Umgebungen schlecht platziert, da sie dort nicht 
mehr umgesetzt werden und deshalb in die Listen nichts geschrieben w"urde. 
Ein guter Platz zum Sammeln ist dagegen \textit{vor} dem zugeh"origen 
Listenausdruckbefehl.

\vspace{2ex}\noindent
In \verb|{unused}|\hy Umgebungen gilt "ahnliches wie f"ur addto\fhy 
Befehle au"serhalb: 

\vspace{1ex}\noindent
\parbox{1.8em}{(1)}Die dort in v\fhy Befehlen mit \verb|\ktit| markierten 
Kurztitel bewirken keinen Eintrag ins Kurzzitateverzeichnis. Falls 
Sie einen Text schreiben wollen, in welchem von der ersten Fu"snote an
kurzzitiert wird, m"ussten Sie ggf.\ gew"unschte Eintr"age f"ur das 
Kurzzitateverzeichnis in der Umgebung mit Separaten \verb|\kli|\hy\ und 
\verb|\kqu|\hy Eintr"agen selbst erzeugen. (Im normalen Text 
schreibt das \verb|\ktit| der \verb|\printonlyvli|\hy Komponente den 
Eintrag in die\hspace{-.2em} \verb|.vkc|\hy Datei; das \verb|\ktit| der 
\verb|\addtovli|\hy Komponente w"urde ebenfalls 'verschluckt'.)

\vspace{1ex}\noindent
\parbox{1.8em}{(2)}In der \verb|{unused}|\hy Umgebung erzeugen
\textit{innere v\fhy Befehle} keinen eigenen\pdfko{.25}\ 
Volleintrag auf den v\fhy Listen. 
\BibArts\ erzeugt nur einen Eintrag in den v\fhy Listen, wobei es
innere v\fhy Eintr"age als Kurzzitat ausdruckt. Innere v\fhy Befehle 
m"ussen deshalb kopiert und nochmals separat in die Umgebung 
eingef"ugt werden. 


\vfill\noindent
\textsf{Bevor ich Zeitschriftenbelege vorstelle, kommt nun erst das w"ortliche 
Zitieren.}


\newpage
\section{W"ortliche Zitate in verschiedenen Sprachen}\label{Sect2}

\BibArts\ stellt eine Umgebung bereit, um l"angere w"ortliche Zitate 
zur besseren Erkennbarkeit vom restlichen Text deutlich abgesetzt
ausdrucken zu k"onnen:\vspace{-.4ex}

\Doppelbox
{\ \ ... (hier deutsch):
 \\[.5ex] \bs begin\{originalquote\}
 \\[.35ex] \ \ \string"\string`Der Krieg entsteht  
 \\ \ \ nicht urpl"otzlich; seine 
 \\ \ \ Verbreitung ist nicht das 
 \\ \ \ Werk eines Augenblicks, 
 \\[.125ex] \ \ [...].\string"\string'\bs footnote \b{\{}
 \\[.125ex] \ \ \ \ \bs kqu\{Clausewitz\} 
 \\[.2ex] \ \ \ \ \ \ \ \ \{Vom Kriege\}[22].\b{\}}
 \\[.25ex] \bs end\{originalquote\}
}
{\renewcommand{\originalquotetype}{\footnotesize}%
 Dieses Zitat hat den Trennsatz des umgebenden Textes (hier deutsch):
 \begin{originalquote}
 "`Der Krieg entsteht nicht urpl"otzlich; seine Verbreitung 
 ist nicht das Werk eines Augen"-blicks, 
 [...]."'\footnote {
 \kqu{Clausewitz} {Vom Kriege}[22].}
 \end{originalquote}
}

\vspace{.25ex}\noindent
Gr"unde f"ur die neue Umgebung: Die \LaTeX\hy Umgebung \verb|{quote}| setzt den
vertikalen Abstand zum Fu"snotenbereich viel zu klein, wenn mitten in den
"ubersetzten Zitatblock ein Seitenumbruch f"allt. Damit die
\verb|{originalquote}|\hy Umgebung reagieren kann, versieht \verb|bibarts.sty| 
den bestehenden \LaTeX\hy Befehl\pdfko{1.25}\ 
\verb|\footnoterule| mit einem Zusatz.\footnote{
Au"serdem wird der \textit{Fu"snotenbereich} an den Fu"s der Seite geschoben 
durch Einf"ugen von zus"atzlichem vertikalem Zwischenraum. 
Deshalb sollten Sie in Texten mit \texttt{\{originalquote\}} den
\LaTeX\hy Befehl \texttt{\bs flushbottom} nicht verwenden. $-$ 
\BibArts\ setzte den \textit{Abstand zwischen Fu"snoten} in Version 2.0 
auf \texttt{2ex}. Falls Sie dies weiterhin wollen, m"ussen Sie nun 
selbst \texttt{\bs setlength\{\bs footnotesep\}\{2ex\}} im Vorspann definieren.} 
Wenn Sie \verb|\footnoterule| anschlie"send\pdfko{1}\ 
einfach umdefinieren, dann schalten Sie diese Eigenschaft
aus.\footnote{Um die Dicke des Strichs vor dem Fu"snotenbereich zu
"andern, m"ussen Sie unter \BibArts\pdfko{1.25}\ 
den Befehl \texttt{\bs fnrbasave} statt 
lehrbuchgem"a"s \texttt{\bs footnoterule} "andern, beispielsweise: \\[.25ex]
\hspace*{2em}{\ttfamily
 \bs renewcommand\{\bs fnrbasave\}\{\bs noindent\bs rule\{5cm\}\{0.5mm\}\bs vspace\{1ex\}\} }}

Ein zweiter Grund f"ur die \verb|{originalquote}|\hy Umgebung betrifft
den Zeilenumbruch. Geistes- und
SozialwissenschaftlerInnen zitieren oft in einer von der Basissprache
ihres Textes abweichenden Sprache. Dann muss zur richtigen Worttrennung 
aber \textit{nur} der Trennsatz umgestellt werden. 
\verb|\selectlanguage|\pdfko{.5}\ 
aus \texttt{ngerman.sty} bewirkt aber gleichzeitig, 
dass am Kapitelkopf nach Setzen von 
\verb|{french}| \textit{Chapitre} 
steht und eine Seite \textit{Page} hei"st. \BibArts\ separiert\pdfko{.5}\ 
beides (vgl.\ sprachabh"angige Textelemente unten S.\,\pageref{SprachSep}). 
Zur Einstellung des\pdfko{.25}\  
Trennsatzes mit \BibArts\hy Befehlen dienen 
dieselben Schl"usselbegriffe, die Sie\pdfko{.5}\  
auch als Argument f"ur \verb|\selectlanguage| verwenden. 
Folgendes Beispiel ist\pdfko{.5}\ 
englisch und druckt das Zitat zudem in Gr"o"se der umgebenden Schrift aus: 


{\notktitaddtok\renewcommand{\originalquotetype}{}
  \begin{originalquote}[UKenglish]
     "`Virginia brought crucial resources to the Confederacy. 
     Her population was the South's largest. Her industrial 
     capacity was nearly as great as that of the seven original 
     Confederate states combined."'\footnote{Auch englische 
     Trennung: \printonlyvli{James M.}{McPherson}{\ktit{Battle 
     Cry of Freedom}. The American Civil War, Oxford 1988}.}
  \end{originalquote}}

\vspace{-.5ex}\noindent
Dieses w"ortliche Zitat wurde mit folgendem \LaTeX\hy Code erzeugt:

\vspace{-.75ex}{\small\begin{verbatim}
  {\renewcommand{\originalquotetype}{}     %% Statt \small
   \begin{originalquote}[UKenglish]
      "`Virginia brought crucial resources to the Confederacy. 
      Her population was the South's largest. Her industrial 
      capacity was nearly as great as that of the seven original 
      Confederate states combined."'\footnote{Auch englische 
      Trennung: \vli{James M.}{McPherson}{\ktit{Battle Cry of 
      Freedom}. The American Civil War, Oxford 1988}.}
   \end{originalquote}}
\end{verbatim}}


\vspace{-.5ex}\noindent
Falls dort {\small\verb|\begin{originalquote}[eglihs]|} st"unde, w"are die 
Fehlermeldung bei der \LaTeX\hy "Ubersetzung:

\vspace{-.5ex}{\scriptsize\begin{verbatim}
     ** Arg(s) of BibArts' sethyphenation-command: Error around line 1371!
        You've called \begin{originalquote}[eglihs].
     <H><return>  for immediate help, 
     <return>     to continue.
   ! Language-name `eglihs' is undefined. (Old VALUE remains valid: 42).
     . . . . . . . . . . . .
   \errmessage@ba ...
    \space . . . . . . . . . . . }
                                                  }
   l.1145 \begin{originalquote}[eglihs]
\end{verbatim}}

\vspace{-.75ex}\noindent
Da oben tats"achlich ein \verb|\printonlyvli|\hy Befehl
steht, ist eine Besonderheit von \BibArts\ nur 
S.\,\pageref{Trennbeispiel} bei "`Zum Schluss ..."' zu sehen:  
\verb|bibsort| reproduziert den am Zugang g"ultigen Trennsatz 
\textit{beim Listenausdruck}. Bei der \LaTeX\hy "Ubersetzung 
der \textit{Datei mit der erzeugten Liste} kommen Bildschirm\hy 
Meldungen:\label{hyphenation}%

\vspace{-0.5ex}
{\scriptsize\begin{verbatim}
   [bibsort] Reproduce hyphenation 67 in line 33 of BibArts file. 
   [bibsort] Reproduce hyphenation 42 in line 43 of BibArts file.
\end{verbatim}}

\vspace{-0.75ex}\noindent 
Das ist die Trennsatz\hy Umschaltung \textit{vor} dem englischen Listenpunkt 
und das Zur"uckschalten ins Deutsche \textit{dahinter} (42 f"ur Deutsch ist
versionsabh"angig).

\vspace{1.25ex}\noindent
Um Trenns"atze $-$ und nur die $-$ auch au"serhalb von \verb|{originalquote}| 
einzustellen, bietet \BibArts\ den weiteren Befehl \verb|\sethyphenation| an. 
Ein deutschsprachiges Wort\footnote{In einem Zitat aus \kqu{Clausewitz}
{Vom Kriege}[75 (I.6)].} ist unten f"alschlicherweise franz"osisch getrennt. 
Falls Sie diesen Text mit \LaTeX\ "ubersetzen und nicht widers-prechend 
getrennt wird, verf"ugt Ihre \LaTeX\hy Version entweder "uber keinen 
franz"osischen Trennsatz oder reagiert auf Umschaltungen \textit{in} 
Abs"atzen nicht (sondern nur am Absatzkopf):

    \vspace{2ex}%
    \noindent\hspace{1.9em}\parbox{12.5cm}{\small\ttfamily
    \string"\string`Ein gro"ser Teil der Nachrichten, die man im Kriege \\
    bekommt, ist \{\bs sethyphenation\{french\} widersprechend\}, \\
    ein noch gr"o"serer ist falsch und bei weitem der \\
    gr"o"ste einer ziemlichen Ungewi"sheit unterworfen.\string"\string'
    }

    \vspace{2ex}%
    \noindent\hspace{1.9em}\parbox{12.5cm}{\small
    "`Ein gro"ser Teil der Nachrichten, die man im Kriege 
    bekommt, ist {\sethyphenation{french} widersprechend}, 
    ein noch gr"o"serer ist falsch und bei weitem der 
    gr"o"ste einer ziemlichen Ungewi"sheit unterworfen."'
    }

\vspace{2.5ex}\noindent
Die verschiedenen Befehle zur Trennsatz\hy Einstellung sind
kombinierbar. Falls in einer \verb|{originalquote}|\hy Umgebung 
der Titel des zitierten Werkes eine andere Sprache als das 
w"ortliche Zitat hat, darf \verb|\sethyphenation| am Kopf\pdfko{1.5}\ 
der Fu"snote stehen. Falls Sie \verb|\sethyphenation| oder 
\verb|\selectlanguage| zudem \textit{in} den 
\BibArts\hy Argumenten verwenden, ist dies (samt Argument) f"ur die
Sortierreihenfolge unerheblich. Speziell aber im Nachnamensargument
von \verb|\vli| und \verb|\kli| sollten Sie solche Befehle wegen der
\textsc{ebd.}\hy Setzung vermeiden. Wenn Sie in Autorennamen (also)
\textit{Trennhilfen} nutzen, sollten die bei allen v- und k\fhy Nennungen 
zumindest \textit{dieses einen Werkes} einheitlich gesetzt
sein.\footnote{Mehrere (ansonsten) zeichengleiche Listenzug"ange, 
bei denen \texttt{\bs sethyphenation} oder\pdfko{.5}\ 
\texttt{\bs selectlanguage} mal gesetzt und 
mal vergessen (oder mit verschiedenen Sprachen besetzt) 
wurde, ergeben mehrere Listeneintr"age; uneinheitliche 
Trennhilfen \texttt{\bs-} und \texttt{\dq-} auch.
$-$ Bei wechselnden Trennhilfen \texttt{\bs-} und \texttt{\dq-} setzt
\texttt{bibsort\hspace{.3em}-k} \hspace{.2em}das $\sim$ 
dagegen richtig.} 

\vspace{1.5ex}\noindent
Nebenbei: Die \verb|{originalquote}|\hy Umgebung und der
\verb|\sethyphenation|\hy Befehl "andern absichtlich auch das 
\textit{spacing} nicht, weil dies in einem Text durchgehend gleich sein sollte.
Vgl. unten Kap.\,\ref{nonfrenchspacing} ab S.\,\pageref{nonfrenchspacing} und 
Kap.\,\ref{Punkte} ab S.\,\pageref{Punkte}.

\vspace{1.5ex}\noindent
\BibArts\ reproduziert dar"uber hinaus den \verb|german.sty|\hy\ bzw.\ 
\verb|ngerman.sty|\hy Befehl \verb|\originalTeX| beim Listenausdruck, 
falls ein Eintrag aus einem Umfeld mit ver"andertem \textit{catcode} f"ur \verb|"| 
herstammt. Von den beiden \verb|"a| unten S.\,\pageref{originaltex}
ist nur eines als "a einsortiert. \verb|\originalTeX| schaltet zudem
den englischen Trennsatz ein. W"ahrend der "Ubersetzung einer 
\BibArts\hy Liste wird gemeldet:

\vspace{-.75ex}{\scriptsize\begin{verbatim}
     [bibsort] Set \baoriginalTeX in line 64 of BibArts file. 
     [bibsort] Reproduce hyphenation 255 in line 65 of BibArts file. 
     [bibsort] Set \bagermanTeX in line 75 of BibArts file.  (new)
     [bibsort] Reproduce hyphenation 42 in line 76 of BibArts file.
\end{verbatim}}

\vspace{-1ex}\noindent
\verb|\baoriginalTeX| f"uhrt \verb|\originalTeX| aus, \verb|\bagermanTeX| 
f"uhrt selbst"andig \verb|\germanTeX| oder \verb|\ngermanTeX|
(mit Meldungen \texttt{\footnotesize (old)} oder \texttt{\footnotesize (new)}) aus, 
je nachdem, ob Sie \verb|german.sty|
oder \verb|ngerman.sty| geladen haben.\footnote{\label{bagermanTeX}Die Zwischenstufe mit
\texttt{\bs baoriginalTeX} bzw.\ \texttt{\bs bagermanTeX} dient dazu, dass
Sie mit \texttt{\bs renewcommand} beide Definitionen ausschalten k"onnen,
falls es in Ihrem Text eine andere Bedeutung hat, wenn sich der 
\textit{catcode} des \texttt{\dq} "andert (vgl.\ unten S.\,\pageref{original} samt Anm.\,\ref{original}).}
Ganz allgemein kommt \verb|bibarts.sty| klar, falls \verb|~":;!?'`<>| 
\textit{aktive} Zeichen sind, doch "Anderungen \label{catcode}
des \textit{catcode} reproduziert \verb|bibsort| nur bez"uglich~\verb|"|.



\section{Formatierungs- und Editionshilfen}\label{Sect3}

Um Datumsangaben gutformatiert drucken zu k"onnen, verf"ugt 
\BibArts\ f"ur das\pdfko{1}\ 
Deutsche "uber den Befehl \verb|\te|, der einen Punkt und ein 
kurzes Leerzeichen (ohne Zeilenumbrucherlaubnis) druckt:
\hspace{.4em}\verb|Der 1\te April| \hspace{.4em}\verb|=>| \hspace{.4em}Der 1\te April.

\vspace{1ex}\noindent
F"ur englische Texte wird 
\hspace{.1em}\verb|\eordinal{|\textit{arabische Zahl}\verb|}| 
\hspace{.2em}bereitgestellt:

\vspace{.75ex}{\small\noindent
\verb|     \eordinal{1} Assistant  => | \eordinal{1} Assistant. \\
\verb|     \eordinal{2} Assistant  => | \eordinal{2} Assistant. \\
\verb|     \eordinal{3} Assistant  => | \eordinal{3} Assistant. \\
\verb|     \eordinal{4} Assistant  => | \eordinal{4} Assistant. \\
\verb|    \eordinal{11} Assistant  => | \eordinal{11} Assistant. \\
\verb|    \eordinal{21} Assistant  => | \eordinal{21} Assistant.} 

\vspace{1ex}\noindent
Im Franz"osischen ergibt sich bei \verb|{1}| ein geschlechtsspezifischer Unterschied:

\vspace{.75ex}{\small\noindent
\verb|    Le \fordinalm{1} homme   => | Le \fordinalm{1} homme. \\
\verb|    La \fordinalf{1} femme   => | La \fordinalf{1} femme. \\
\verb|    Le \fordinalm{2} homme   => | Le \fordinalm{2} homme. \\
\verb|    La \fordinalf{2} femme   => | La \fordinalf{2} femme.}  

\vspace{1ex}\noindent
Die ordinal\hy Befehle dienen auch als Hilfsbefehle f"ur den Befehl 
\verb|\ersch| (oben S.\,\pageref{ersch}). \verb|\ersch| nutzt verschiedene 
ordinal\hy Befehle, wenn  \verb|\bacaptionsgerman|, 
\verb|\bacaptionsenglish| oder \verb|\bacaptionsfrench| gilt 
(vgl.\ Kapitel~\ref{SprachSep} unten ab S.\,\pageref{SprachSep}). 
\textit{Auf"|l.} und \textit{edition} lassen sich direkt "andern 
(\verb|\gerscheditionname| und \verb|\eerscheditionname| unten 
S.\,\pageref{erscheditionname}). Weil aber \verb|\ferscheditionname|
das feminine Wort \textit{\'edi\-tion} druckt, setzt \BibArts\ 
\verb|\fordinalf| in \verb|\ersch| ein.\footnote{\texttt{\bs ersch} 
nutzt unter \texttt{\bs bacaptionsgerman} statt \texttt{\bs te} den 
Befehl \texttt{\bs gordinal}. Seit Version 2.2 wird\hspace{.2em} \texttt{1\bs te X}\hspace{.2em} wie\hspace{.2em}
\texttt{1.}\ \texttt{X}\hspace{.2em} \textit{einsortiert}, 
aber\hspace{.2em} \texttt{\bs gordinal\{1\} X}\hspace{.2em} 
wie\hspace{.2em} \texttt{1 X}\hspace{.2em} (relevant f"ur \texttt{-p}).} 
Bei Wechsel zu einem maskulinen Wort 
m"ussten Sie zudem \verb|\fordinal| anpassen:\label{fordinalf}%

\vspace{.75ex}{\small\noindent
\verb|   \bacaptionsfrench| \\
\verb|     \ersch[1]{Paris}{1976}   => | {\bacaptionsfrench\ersch[1]{Paris}{1976} \\
\verb|       \renewcommand{\ferscheditionname}{\fupskip classement}| \\[-.25ex]
\verb|       \renewcommand{\fordinal}{\fordinalm}| \\[.1ex]
\verb|     \ersch[1]{Paris}{1976}   => | \renewcommand{\fordinal}{\fordinalm}%
\renewcommand{\ferscheditionname}{\fupskip classement,}%
\ersch[1]{Paris}{1976}}}

\vspace{1.25ex}\noindent
Zum Hochstellen von freien Texteingaben dient 
\verb|\fup{|\textit{\kern-.05em Text}\verb|}| 
\hspace{.2em}(\kern-.025em\textit{F}\kern-.05em rench \textit{up}). In\pdfko{.25}\  
schr"aggestelltem Umfeld wird automatisch eine \textit{italics}\hy
Korrektur gesetzt. Die l"asst sich nach \hspace{-.1em}\verb|.| mit \verb|\bahasdot| 
unterbinden (vgl.\ Kapitel~\ref{Punkte} ab 
S.\,\pageref{Punkte}): 

\vspace{1ex}{\small\noindent
\verb|          S\fup{te} Claire             => |         \hbox to 6em{S\fup{te} Claire\hfill}\texttt{\% Laden von} \\
\verb|  \textit{S\fup{te} Claire}            => | \hbox to 6em{\textit{S\fup{te} Claire\hfill}}\texttt{\% babel-french} \\
\verb|  \textit{S.\fup{te} Claire}           => | \hbox to 6em{\textit{S.\fup{te} Claire\hfill}}\texttt{\% "uberschreibt} \\
\verb|  \textit{S.\bahasdot\fup{te} Claire}  => | \hbox to 6em{\textit{S.\bahasdot\fup{te} Claire\hfill}}\texttt{\% dieses \bs fup!}}


\newpage\noindent
F"ur \textit{Editionsarbeiten} (w"ortliches Zitieren) stellt \BibArts\ 
\verb|\abra{|\textit{Symbol}\verb|}| und\pdfko{1}\
\verb|\fabra{|\textit{Symbol}\verb|}| 
bereit. Als \textit{Symbol} lassen sich i.\,O. vergessene Satzzeichen 
nachtragen, die dann in eckigen Klammern (\kern-.1em\textit{a}ngular \textit{bra}ckets) 
ausgedruckt werden, um sie als \textit{editorische Zus"atze} zu kennzeichnen. 
Der Fixier\hy Befehl \verb|\fabra| verbietet einen Zeilenumbruch direkt
\textit{nach} dem \textit{Symbol}. 

Besonderheit der beiden Befehle ist, dass sie etliche kleine Symbole 
automatisch in \textit{h"ohenangepassten} Klammern ausdrucken. 'Unbekannte' 
Zeichen werden in ein normales eckiges Klammerpaar gesetzt. Bekannte Symbole sind:

\vspace{1.75ex}\noindent
   \verb|   \abra{,}      => | Rot\abra{,} blau und gr"un \\[-.5ex]
   \verb|   \abra{.}      => | kamen vor\abra{.} Und da \\[-.5ex]
   \verb|   \abra{...}    => | \abra{...} waren \\[-.5ex]
   \verb|   \abra{\dots}  => | \abra{\dots} noch \\[-.5ex]
   \verb|   \abra{$-$}    => | \abra{$-$} glaube ich \abra{$-$} \\[-.5ex]
   \verb|   \abra{-}      => | gr"un\abra{-} und gelb\abra{-}farbene \\[-.5ex]
   \verb|   \abra{--}     => | Punkte \abra{--}\\[-.5ex]
   \verb|   \abra{---}    => | Englischer\abra{---}Gedankenstrich.\\[-.5ex]
   \verb|  \fabra{`}      => | \hbox to 3em{\fabra{`}So,\hfill} \\[-.5ex]
   \verb|  \fabra{'}      => | \hbox to 3em{so\abra{'}.\hfill}  \\[-.5ex]
   \verb|  \fabra{\glq}   => | \hbox to 3em{\fabra{\glq}So,\hfill} \\[-.5ex]
   \verb|  \fabra{\grq}   => | \hbox to 3em{so\abra{\grq}.\hfill}  \\[-.5ex]
   \verb|  \fabra{\glqq}  => | \fabra{\glqq}Das kann \\[-.5ex]
   \verb|   \abra{\grqq}  => | nicht sein.\abra{\grqq} \\[-.5ex]
   \verb|  \fabra{"`}     => | \fabra{"`}Das kann \\[-.5ex]
   \verb|   \abra{"'}     => | nicht sein.\abra{"'} \\[-.5ex]
   \verb|   \abra{``}     => | ,,Gut!\abra{``} \\[-.5ex]
   \verb|   \abra{''}     => | ''Good?\abra{''} \\
   \verb|  \fabra{g}gf.   => | \hbox to 3em{\fabra{g}gf.\hfill} \verb| % unbekannt => 'grosse' Klammern|


\vspace{2ex}\noindent
Damit \BibArts\ die \textit{Symbole} erkennen kann, m"ussen sie genau "ubereinstimmen,
d"urfen also auch keine Leerzeichen enthalten. F"ur das \verb|"| in
\verb|"`| und \verb|"'| ist zudem Voraussetzung, dass es einen \textit{catcode} 
von 13 (\textit{aktiv}) hat, wie es nach Laden etwa von \verb|ngerman.sty| oder 
babel\hy ngerman der Fall ist. Sonst (Englisch\hy Original\hy\LaTeX) sind auch
\verb|\abra{"}| bzw.\ \verb|\fabra{"}| m"oglich:~{\originalTeX\fabra{"}}

\vspace{1ex}\vfill\noindent
In den 'kleinen' Klammern der abra-Befehle setzt \BibArts\ 
die \textit{Symbole} aufrecht,
weil die sonst in einigen schr"aggestellten Schriften 
schlecht zentriert in den Klammern erscheinen w"urden. 
\verb|\abra| und \verb|\fabra| machen eine 
\textit{italics}\hy Korrektur. Sie wird durch ein direkt davor getipptes 
\verb|\bahasdot| unterbunden:

\vspace{1.75ex}\noindent
\verb|    \fabra{"`}Haus\abra{"'}         => | {\fabra{"`}Haus\abra{"'}} \\
\verb|  \itshape | \\
\verb|    \fabra{"`}Haus\abra{"'}         => | {\itshape \fabra{"`}Haus\abra{"'}} \\
\verb|    \fabra{"`}H.\abra{"'}           => | {\itshape \fabra{"`}H.\abra{"'}} \\
\verb|    \fabra{"`}H.\bahasdot\abra{"'}  => | {\itshape \fabra{"`}H.\bahasdot\abra{"'}}

\vspace{2ex}\noindent
Weil normale \textit{Minuszeichen} in Worten die Silbentrennung ausschalten,
stellt \BibArts\ zudem \verb|\hy| und \verb|\fhy| bereit. 
\verb|\hy| erlaubt die Trennung direkt nach dem gedruckten Minuszeichen 
(\verb|Haber\hy Bosch\hy Verfahren =>| Haber\hy Bosch\hy Verfahren),
w"ahrend \verb|\fhy| ein Minuszeichen druckt, das fest am
Folgewort klebt: \verb|Truppenaufmarsch und \fhy abzug =>| 
Truppenaufmarsch und \fhy abzug. Gegebenenfalls w"urde auch ab-zug getrennt
(anders als nach~\verb|"~|).\label{hy}

\vspace{1ex}\noindent
\verb|\hy| machte im Beispiel oben auch ein \textit{kerning} zum V\ko,
das es nach direkt angetippten Minuszeichen nicht gibt:
\hspace{.3em}\verb|Haber-Bosch-Verfahren| 
\hspace{.3em}\verb|=>| \hspace{.2em}Haber-Bosch-Verfahren. 
Das \textit{kerning} erfolgt vor A, T\ko, v, V\ko, w, W\ko, x, X, y und Y\ko, sowie vor 
\verb|`|, \verb|'|, \verb|\glq|, \verb|)|, \verb|]| und \verb|\}| automatisch.
Es funktioniert auch dann, wenn der Buchstabe \textit{einen} Akzent hat
(\kern-.1em\textit{aktives} \hspace{-.1em}\verb|"|, \verb|\"|, 
\verb|\.|, \verb|\=|, \verb|\^|, \verb|\'|, \verb|\`|, 
\verb|\~|,\pdfko{.75}\ 
\verb*|\accent |\textit{num}\verb*| |,
\verb|\b|, \verb|\c|, \verb|\d|, \verb|\H|, \verb|\k|, 
\verb|\r|, \verb|\u| oder \verb|\v|; nur \verb|\t| 
funktioniert nicht).\pdfko{.5}

\vspace{1.25ex}\noindent
\verb|  -Yser        => | -Yser \\
\verb|  \hy Yser     => | \hy Yser \\
\verb|  \hy\"Yser    => | \hy \"Yser   \hspace{1em} \texttt{\% Kein utf8-\"Y, sondern \bs\dq Y (auch \dq A)} \\
\verb|  \hy\"{Y}ser  => | \hy \"{Y}ser

\vspace{1.5ex}\noindent
Dieses automatische \textit{kerning} l"asst sich durch \verb|\nothyko| 
ausschalten (Wiedereinschalten mit \verb|\hyko|). Setzen von 
\verb|\hy{}|\hspace{-.15em}\textit{Wort} bzw.\ 
\verb|\fhy{}|\hspace{-.15em}\textit{Wort} unterbindet es ebenfalls. 
Das folgende Wort kann dann immer noch getrennt werden. 
In einem \texttt{typewriter}\hy Umfeld sollten Sie weiterhin 
Minuszeichen '\verb|-|' tippen.

\vspace{1.5ex}\noindent
\hspace*{-.16em}\textit{Vor} \verb|\hy| oder \verb|\fhy| kann 
$-$ falls ein penibler Textsatz gew"unscht ist $-$ kein 
automatisches \textit{kerning} durchgef"uhrt werden. \BibArts\ 
stellt den Korrekturbefehl \verb|\ko| bereit. Die Kosmetik 
ist (\kern-.05em\textit{wenn "uberhaupt!}) n"otig nach 
Gro"sbuchstaben, die sehr weit vom nachfolgenden Minuszeichen 
entfernt sind: T\ko, V\ko, W und Y\ko. 

\vspace{1.25ex}\noindent
\verb|  T\hy Zacke     => | T\hy Zacke    \\
\verb|  T\ko\hy Zacke  => | T\ko\hy Zacke \\
\verb|  V\hy Form      => | V\hy Form     \\
\verb|  V\ko\hy Form   => | V\ko\hy Form

\vspace{1.5ex}\noindent
Die Definition von \verb|\ko| kann Ihnen als Beispiel f"ur "ahnliche Befehle dienen

\vspace{1.25ex}{\small\noindent
\verb|  \newcommand{\pko}{\ifhmode\nobreak\hskip -0.07em plus 0em\fi}| \\
\verb|  \newcommand{\ko}{\protect\pko}|\label{ProtectBeispiel}}

\vspace{1.5ex}\noindent
falls Sie die Korrektur zwischen V und Punkt oder Komma zu klein finden:

\vspace{1.25ex}\noindent
\verb|  V\te Armee     => | V\te Armee    \\
\verb|  V\ko\te Armee  => | V\ko\te Armee

\vspace{1.5ex}\noindent
Sicher w"are der Abstand von V und \hspace{-.1em}\verb|.| 
aber besser in den Ligaturtabellen definiert (worauf
\verb|\te| reagiert: \verb| P\te I und P{}\te I => | 
P\te I und P{}\te I).



\newpage
\section{Abk"urzungen}\label{Sect4}

\BibArts\ stellt Instrumente zur Verwaltung von Abk"urzungen zur Verf"ugung.
Dies ist ein zus"atzliches Feature, das sie (unabh"angig von der \BibArts\hy
Verwaltung von Belegstellen mit \verb|\vli| etc.) verwenden k"onnen, um
sich ein Abk"urzungsverzeichnis ausdrucken zu lassen. Sie werden von
\verb|bibsort| zudem mittels Bildschirm\hy Meldung gewarnt, falls Sie eine  
Abk"urzung verwenden, ohne sie f"ur Ihren Leser aufgel"ost zu haben. 
Spielregeln dabei sind: Falls Sie eine Abk"urzung bereits \textit{in einer Fu"snote} 
auf"|l"osten, darf die Abk"urzung \textit{in\pdfko{.75}\ 
weiteren Fu"snoten} ohne neuerliche 
Erkl"arung verwendet werden; erfolgte die Definition der Bedeutung dagegen im 
Haupttext, darf die Abk"urzung danach "uberall verwendet werden. Das 
Abk"urzungsverzeichnis wird jedoch in jedem Fall mit Abk"urzungen 
gef"uttert, sofern deren Auf"|l"osung vorliegt; \verb|bibsort| warnt, falls 
(stets m"ogliche) Mehrfach\hy Auf"|l"osungen \textit{voneinander abweichen}.

Abk"urzungen sind also zun"achst zu definieren. Dabei ist wahlfrei, ob erst
die Abk"urzung und dann ihre Auf"|l"osung gesetzt wird oder umgekehrt: 

\Doppelbox
{... eine
 \bs abkdef\{OHG\}\{Offene 
 \\ \ \ \ Handelsgesellschaft\}.
 \\ Oder: \bs defabk\{Offene 
 \\ \ Handelsgesellschaft\}\{OHG\}.
 Nun d"urfen Sie \bs abk\{OHG\} benutzen.
}
{Das Unternehmen ist eine 
 \abkdef{OHG}{Offene 
 Handelsgesellschaft}.
 Oder: \defabk{Offene 
 Handelsgesellschaft}{OHG}.
 Nun d"urfen Sie \abk{OHG} benutzen.
}\label{defabk}
 

\noindent
Falls Sie die weitere Abk"urzung \abk{GmbH} mit \verb|\abk{GmbH}| setzen,
aber \textit{nie} definieren, wird sie nicht ins Abk"urzungsverzeichnis 
"ubernommen; stattdessen druckt \verb|bibsort| eine Warnung folgenden 
Typs auf den Bildschirm: 

\vspace{-.5ex}
{\scriptsize\begin{verbatim}
  %%>   Warning: Abbreviation "GmbH" is NEVER defined!
  %%     The entry (file 1 line 1764) is rejected. Use \abkdef?
\end{verbatim}}

\vspace{-.5ex}\noindent
Falls Sie die Abk"urzung mit \verb|\abkdef| oder
\verb|\defabk| definieren, dies im Texteditor aber in einer 
Zeile \textit{nach} \verb|\abk{GmbH}| tun, kommt sie ins 
Abk"urzungsverzeichnis. \verb|bibsort| warnt in seinem 
Bildschirmausdruck allerdings etwa so:

%\abk{GmbH}
%\abkdef{GmbH}{Gemeinschaft mit beschr"ankter Haftung}

\vspace{-.5ex}
{\scriptsize\begin{verbatim}
  %%>   Warning: Abbreviation "GmbH" is used in
  %%     file 1 line 1782 and def in file 1 line 1783!
\end{verbatim}}

\vspace{-.5ex}\noindent
Verwenden Sie Abk"urzungen im Text, die \textit{nur} in Fu"snoten 
aufgel"ost sind, erscheint (nicht\hy abschaltbar) eine Bildschirm\hy 
Warnung. Falls Sie tippen ...

\Doppelbox
{...\bs footnote\b{\{}Ein \bs abkdef\{e.\bs,V.\} 
 \\ \ \ \ \ \ \ \{eingetragener Verein\} hat 
 \\ \ mehrere Mitglieder.\b{\}} Der Verein hat \bs abk\{e.\bs,V.\} als Form.
}
{...\footnote{Ein \abkdef{e.\,V.} 
 {eingetragener Verein} hat 
 mehrere Mitglieder.} Der Verein hat \abk{e.\,V.} als Form.
}

\noindent
... "ubernimmt \verb|bibsort| Abk"urzung und zugeh"orige Auf"|l"osung 
zwar ins Abk"urzungsverzeichnis, meldet allerdings stets etwas wie:

{\scriptsize\begin{verbatim}
  %%>   Warning: Abbreviation "e.\,V." is used in
  %%     file 1 line 1805 and def in A FNT file 1 line 1805!
\end{verbatim}}

\noindent
Durch eine Eigenart von \LaTeX2e\ nennt die Meldung
die Zeilennummer, in der die Fu"snote endet,\footnote{In \LaTeX~2.09 evtl.\ auch bez"uglich der Zeile,
in der sie anf"angt.} w"ahrend \verb|\abkdef| im Beispiel sich tats"achlich
in einer vorausgehenden Editorzeile befand. 

\vspace{2ex}\noindent
Das Abk"urzungsverzeichnis wird hier mit \verb|\printnumabklist| ausgedruckt. 
Die Befehle \verb|\printabk| bzw.\ das zentrale \verb|\printnumabk| ergeben 
einen doppelspaltigen Ausdruck in \verb|\footnotesize| auf einer neuen 
Seite unter der "Uberschrift \textbf{Abk"urzungen}, was hier nur aus 
Platzgr"unden unterbleibt:  

{\batwocolitemdefs\printnumabklist}\balabel{abklist}

\noindent
Die K"opfe der Listenpunkte wurden dabei in \verb|\abklistemph|
ausgedruckt, das\pdfko{.75}\ 
defaultm"a"sig \verb|\bfseries| ausf"uhrt 
(\textbf{fett}). Die Seiten, von denen Definitionen\pdfko{.5}\  
herstammen, sind in der Auf"|listung von Seitenzahlen nicht 
hervorgehoben. Eine Hervorhebung einzelner Seitenzahlen sieht 
auch \texttt{bibsort} 2.2 nie vor.

F"ur Abk"urzungen wie \printonlyabk{u.\,a.}, die Allgemeingut sind und deshalb
vielleicht nicht ins Abk"urzungsverzeichnis sollen, kann 
\verb|\printonlyabk{u.\,a.}| genutzt werden, um das 
Argument einheitlich in der Schrift aller Abk"urzungen 
ausgedruckt zu bekommen. Die Kontrolle durch \verb|bibsort| entf"allt 
dann. Radikaler k"onnen Sie insofern im Vorspann mittels 
\verb|\renewcommand{\abkemph}{}|
die Hervorhebung von Abk"urzungen ausschalten und dann \verb|u.\,a.| tippen.

Falls ein Eintrag ins Abk"urzungsverzeichnis soll, man sich die 
Auf"|l"osung im Text aber sparen will, hilft die bereits erw"ahnte 
\verb|{unused}|\hy Umgebung:

\vspace{-1.25ex}
{\small
\begin{verbatim}
    \begin{unused}
      \abkdef{u.\,a.}{unter anderem}     %vgl. Liste oben%
    \end{unused}
\end{verbatim}}
%
    \begin{unused}
      \abkdef{u.\,a.}{unter anderem}     %vgl. Liste oben%
    \end{unused}

\vspace{-1ex}\noindent
Solche Definitionen kommen ohne Seiten\fhy/""Fu"snotennummer in die 
num\hy Liste. Zudem kann \verb|\abk{u.\,a.}| dann "uberall im Text 
(also auch davor) verwendet werden, ohne dass \verb|bibsort| das 
Fehlen der Auf"|l"osung bem"akelt. 

%Wenn das oben nur in einer Fu"snote erkl"arte \verb|\abk{e.\,V.}| jetzt  
%nochmal im Haupttext -- anders als oben: \textit{auf einer anderen Seite 
%als die Definition} -- verwendet wird (hier: \abk{e.\,V.}), meldet \verb|bibsort| 
%diesen Fehler "ubrigens anders:
%
%\vspace{-1ex}
%{\scriptsize\begin{verbatim}
%  %%>   Warning: Abbreviation "e.\,V." is used
%  %%     in TEXT file 1 line 1384, whereas all foregoing defs were in FNTs!
%\end{verbatim}}

\noindent
Mehrfach verwendete Abk"urzungen sowie mehrfach verwendete Auf"|l"osungen
m"ussen zeichengleich sein, um von \verb|bibsort| als gleich erkannt zu
werden. Wird das bereits oben aufgel"oste \abk{OHG}
nochmals erkl"art (vielleicht wollen Sie die Bedeutung einiger bereits definierter
Abk"urzungen am Anfang eines neuen Gro"skapitels nochmal erkl"aren),
wird dies akzeptiert. Wenn Sie dann aber \verb|\abkdef{OHG}{Offene Handelsgschaft}| 
tippen, meldet \verb|bibsort|:\addtoabkdef{OHG}{Offene Handelsgschaft}

\vspace{-.5ex}
{\scriptsize\begin{verbatim}
  %%>   Warning: Different defs for abbreviation "OHG":
  %%     *Accept file 1 line 1761 "Offene Handelsgesellschaft";
  %%     *Reject file 1 line 1890 "Offene Handelsgschaft".
\end{verbatim}}

\vspace{-1ex}\noindent
... und im Abk"urzungsverzeichnis erscheint nur die akzeptierte Variante.


\vspace{1ex}\noindent
Falls die Auf"|l"osung einer Abk"urzung im Abk"urzungsverzeichnis anders sein
soll als im Text, lassen sich die Befehle \verb|\abkdef| (oder \verb|\defabk|)
aufsplitten\pdfko{.4}\ 
in ihre Teilkomponenten. Vergleichen Sie "`(Kochsalz)"' hier und in der Liste:

\Doppelbox
{\vspace{.25ex}
 Das ist \bs addtoabkdef\{NaCl\} 
 \\ \ \{Natriumchlorid (Kochsalz)\}
 \bs printonlyabkdef\{NaCl\} 
 \\ \ \{Natriumchlorid\}.
 \vspace{.25ex}
}
{Das ist \addtoabkdef{NaCl} 
 {Natriumchlorid (Kochsalz)}
 \printonlyabkdef{NaCl} 
 {Natriumchlorid}.
}

\noindent
Dasselbe l"asst sich erreichen durch \verb|\onlyout| in \verb|\abkdef| (oder \verb|\defabk|):

\Doppelbox
{\vspace{.25ex}
 Das ist \bs abkdef\{NaCl\} 
 \\ \ \b{\{}Natriumchlorid\%
 \\ \ \ \ \bs onlyout\{ (Kochsalz)\}\b{\}}.
}
{\vspace{1.5ex}
 Das ist \abkdef{NaCl} 
  {Natriumchlorid%
         \onlyout{ (Kochsalz)}}.
}


\noindent
Es gibt somit auch zwei Notationsarten, um eine in Text und Liste 
abweichende Gro"s\fhy/""Kleinschreibung der Abk"urzung auszudrucken:

\Doppelbox
{\vspace{.25ex}
 \bs printonlyabk\{E.\bs,V.\}\string's \bs addtoabk\{e.\bs,V.\} 
 sind beim Amtsgericht anzumelden. 
 \\[2ex]
 \bs abk\b{\{}\bs onlyhere\{E\}\%
  \\ \ \ \bs onlyout\{e\}.\bs,V.\b{\}} kann auch
 \\ alternativ so notiert sein.
 \vspace{.25ex}
}
{\printonlyabk{E.\,V.}'s \addtoabk{e.\,V.} 
 sind beim Amtsgericht anzumelden. 
 
 \vspace{1.1ex}
 \abk{\onlyhere{E}%
  \onlyout{e}.\,V.} kann auch
         alternativ so notiert sein.
}


\vspace{.25ex}\noindent
Im Text werden das Argument von \verb|\abk| sowie
die Abk"urzungen in \verb|\abkdef| und \verb|\defabk|  
in der Schrift \verb|\abkemph| gesetzt; der Befehl f"uhrt
defaultm"a"sig \verb|\sffamily| aus ({\sffamily{sans serif}\/}),
also anders als \verb|\abklistemph|. Im Text stellt
\verb|\renewcommand{\abkemph}{}| wie erw"ahnt
Umfeldschrift ein. Alternativ w"aren sogar \verb|\itshape| oder 
\verb|\slshape| erlaubt; nur auf Befehle der Art \verb|\textbf| oder 
\verb|\textit| sollten Sie wie immer verzichten. Wie sich \verb|\abk| 
zusammen mit schr"aggestellten Schriften verh"alt, wird unten in 
Kap.\,\ref{schraegkap} ab S.\,\pageref{schraegkap} erkl"art.


\newpage
\section{\texttt{\bs abk\{X.X.X.\}} unter \texttt{\bs nonfrenchspacing}}\label{Sect5}\label{nonfrenchspacing}

\parbox{1.8em}{(1)}Falls Sie \verb|\nonfrenchspacing| einschalten (originaler 
\LaTeX\hy Textsatz mit vergr"o"serten Leerzeichen am Satzende), gilt in 
\LaTeX\ \textit{normalerweise} eine\pdfko{.25}\ 
Vorschrift f"ur Abk"urzungen, \textit{die mit einem Kleinbuchstaben und einem
Punkt\pdfko{.4}\ 
enden}: \textbf{Falls der Satz danach weiter geht}, ist\hspace{.1em} 
\verb*|y.y.y.\ |\hspace{.2em} zu tippen. 

Im Argument von \verb|\abk| ist dagegen egal, ob der letzte Buchstabe klein oder gro"s ist. 
\BibArts\ pr"uft, ob \textit{nach} dem Argument ein Punkt steht; falls nein, geht es davon aus, 
dass ein Leerzeichen mit 'normaler' L"ange zu setzen ist:\footnote{Falls
\texttt{ \}? \}! \}: \}; \}, }folgen, stellen die die Leerzeichenl"ange stets eigenst"andig 
ein.}\pdfko{1}

\vspace{1ex}
{\sffamily\nonfrenchspacing\noindent
\hbox to 10em{\texttt{ \ \ \ \ \ Dr. Maier }\hfill}\verb| => |\hbox to 6em{Dr. Maier\hfill} {\footnotesize\verb| %% falsch in US-Voreinstellung|} \\
\hbox to 10em{\texttt{ \ \ \ \ \ Dr.\bs\ Maier}\hfill}\verb| => |\hbox to 6em{Dr.\ Maier\hfill} {\footnotesize\verb| %% ueblich in US-Voreinstellung|} \\
\hbox to 10em{\texttt{ \bs abk\{Dr.\} Maier}\hfill}\verb| => |\hbox to 6em{\printonlyabk{Dr.} Maier\hfill}  {\footnotesize\verb| %% ausreichend in BibArts|} 
}


\vspace{2ex}\noindent
\parbox{1.8em}{(2)}Falls die Abk"urzung dagegen \textbf{am Satzende} steht, ist im 
\LaTeX\hy Standard '\verb|\nonfrenchspacing|' nur dann etwas zu
unternehmen, falls die Abk"urzung mit einem Gro"sbuchstaben
endet\hspace{.1em} (danach ist \verb|\@.|\hspace{.05em} statt\hspace{-.05em} \verb|.| zu setzen). 

Nach \verb|\abk| m"ussen Sie dagegen \textit{am Satzende} etwas 
unternehmen, \textit{falls\pdfko{1}\ der Punkt zur Abk"urzung geh"ort}. 
Bei \verb|\abk{NASA.}| statt \verb|\abk{NASA}| etwa:

\vspace{1.25ex}
{\sffamily\nonfrenchspacing\noindent
\hbox to 10em{\texttt{ \ \ \ \ \ NASA\bs @.\ Next}\hfill}\verb| => |\hbox to 6em{NASA\@. Next\hfill} {\footnotesize\verb| %% US-Voreinstellung richtig|} \\
\hbox to 10em{\texttt{ \bs abk\{NASA\}.\ \ Next}\hfill}\verb| => |\hbox to 6em{\printonlyabk{NASA}. Next\hfill} {\footnotesize\verb| %% BibArts ohne Punkt richtig|} \\
\hbox to 10em{\texttt{ \bs abk\{NASA.\} \ Next}\hfill}\verb| => |\hbox to 6em{\printonlyabk{NASA.} Next\hfill} {\footnotesize\verb| %% BibArts mit Punkt falsch!|} \\
\hbox to 10em{\texttt{ \bs abk\{NASA.\}.\ Next}\hfill}\verb| => |\hbox to 6em{\printonlyabk{NASA.}. Next\hfill} {\footnotesize\verb| %% BibArts mit Punkt richtig|}
}

\vspace{1.5ex}\noindent
Nach \verb|\abk| \textit{d"urfen} Sie den 'Satzende-Punkt' \textit{immer} 
zus"atzlich setzen: Er wird\pdfko{.25}\ 
'verschluckt', falls die Abk"urzung selbst
schon mit einem Punkt endet; ein folgendes Leerzeichen wird nur im
\verb|\nonfrenchspacing| verl"angert.\footnote{\BibArts\ 
pr"uft erst, ob \texttt{.} einen \texttt{\bs sfcode} von 3000 hat (gilt unter 
\texttt{\bs nonfrenchspacing}); falls das nicht gilt, 'verl"angert' es 
keine Leerzeichen. Unter \texttt{\bs frenchspacing} hat der Punkt einen
\texttt{\bs sfcode} von 1000; falls Sie einen dritten Wert verwenden, 
k"onnen Sie in einer Kopie von \texttt{bibarts.sty} alle 
3000er\hy Stellen gegen Ihre Zahl austauschten und die Kopie nutzen.}

In jedem Fall sollten Sie direkt nach dem letzten Argument eines 
\BibArts\hy Hauptbefehls (siehe S.\,\pageref{Hauptbefehle}) nie \verb|\@| 
setzen: Eine Behandlung von \verb|\@.| ist dort \textit{nicht} vorgesehen.
Im sonstigen Text gelten \verb|\@.| und \verb*|.\ |\hspace{.2em} aber 
weiterhin.

\vspace{2ex}\noindent
Die hier genannten Spielregeln f"ur das \textit{spacing} gelten
auch f"ur andere \BibArts\hy Befehle (vgl.\ unten ab S.\,\pageref{Punkte}). 
So viel vorab: Unter \verb|\frenchspacing| (gilt 
nach Laden von \verb|german.sty| oder \verb|ngerman.sty|) ist 
beim Schreiben an nichts zu denken, weil im deutschen
Textsatz alle Leerzeichen gleich gro"s sind. Sie m"ussen am 
Satzende also nicht \verb*|.}. | setzen; und wenn Sie es doch 
t"aten, w"urde nur ein Punkt gedruckt (ohne Einfluss auf die 
folgende Leerzeichenl"ange).


\section{Zeitschriften und allgemein Bandangaben}\label{Sect6}\label{per}

\BibArts\ stellt zum Zitieren gedruckter Literatur als weitere Klasse
\textit{Zeitschriften}\pdfko{.1}\ bereit. Die kommen ins Argument von \verb|\per| 
(\textit{periodical}). Typischerweise steht\pdfko{.3}\ \verb|\per| im letzten 
Argument von \verb|\vli|, um Aufs"atze in Zeitschriften anzugeben. 
Nach dem Pflichtargument von \verb|\per|\hspace{.2em} kann zwischen 
\textit{underscores} optional\pdfko{.6}\ 
eine Angabe zur \verb|_|\textit{Heftnummer}\ko\ko\verb|_|\hspace{.2em} folgen. 
Vor \verb|_| darf kein Leerzeichen stehen.

\Doppelbox
{
...\bs footnote\{\bs vqu \{John 
\\ \ \ Frederick Charles\} \{Fuller\} 
\\ \ \H{\{}Gold Medal (Military) 
\\ \ \ \bs ktit\{Prize Essay\} for 1919, 
\\ \ \ in: \bs per\{Journal of the
\\ \ \ \ \ \ \ \ Royal United Service 
\\ \ \ \ \ \ \ \ Institution\}\string_458 
\\ \ \ \ \ \ \ \ \ (1920)\string_[239-274]\H{\}}*[240].\}
\\[1.5ex]
...\bs footnote\{\bs kqu \{Fuller\} 
\\ \ \ \ \ \ \ \ \ \ \ \ \{Prize Essay\}[241].\}
\\[1.5ex]
...\bs footnote\{\bs vqu\{R[ichard]\} 
\\ \ \{Chevenix Trench\} 
\\ \ \{Gold Medal (Military) 
\\ \ \ \bs ktit\{Prize Essay\} for 1922, 
\\ \ \ in: \bs per\{Journal of the 
\\ \ \ \ \ \ \ \ Royal United Service 
\\ \ \ \ \ \ \ \ Institution\}\string_470 
\\ \ \ \ \ \ \ \ \ (1923)\string_[199-227]\}*[200].\}
}
{\notktitaddtok
\fbox{Als Beispiel gedruckte Quellen:}
\\[1.5ex]
...\footnote{\printonlyvqu {John\, Frederick\, Charles\,} {Fuller} 
{Gold Medal (Military) \ktit{Prize Essay} for 1919, 
in: \per{Journal of the Royal United Service 
Institution}_458 (1920)_[239-274]}*[240].\label{noNr}\label{Fuller1}}

"Au"sere Ebenda\hy Setzung.\footnote{\printonlykqu {Fuller} 
{Prize Essay}[241].\label{Fuller2}}

Innere Ebenda\hy Setzung.\footnote{\printonlyvqu{R[ichard]\ \ } {Chevenix\ \ Trench} 
{Gold Medal (Military) \ktit{Prize Essay} for 1922, 
in: \per{Journal of the Royal United Service 
Institution}_470 (1923)_[199-227]}*[200].\balabel{Nr}\label{Trench}}
}

\noindent
Im Kurzzitateverzeichnis, das mit \verb|\printnumvkc| ausdruckt wird, erg"abe dies:

\vspace{-0.4ex}
{\small\begin{description}\parsep 0ex \itemsep -.5ex
\item \textsc{Chevenix Trench}: Prize Essay [Q] \ \textsf{\pageref{Trench}$^{\ref{Trench}}$}
\item \textsc{Fuller}: Prize Essay [Q] \ \textsf{\pageref{Fuller1}$^{\ref{Fuller1},\ \ref{Fuller2}}$}
\end{description}}

\vspace{-0.25ex}\noindent
In die Liste \verb|\printvqu| kommt (in \verb|bibarts.vqu| tats"achlich nicht umgesetzt):

\vspace{-0.4ex}
{\small
\begin{description}\parsep 0ex \itemsep -.5ex
\item \textsc{Chevenix Trench}, R[ichard]: Gold Medal (Military) Prize Essay for
1922, in:\pdfko{.2}\ 
\textsc{Journal of the Royal United Service Institution} 470 (1923), S.\,199-227.
\item \textsc{Fuller}, John Frederick Charles: Gold Medal (Military) Prize Essay for
1919, in: \textsc{Journal of the Royal United Service Institution} 458 (1920), S.\,239-274.
\end{description}}

\vspace{-0.25ex}\noindent
Au"serdem lassen sich die verwendeten Zeitschriften in einer separaten Liste ausdrucken. 
M"oglich ist, dabei an einzelne Eintr"age \textit{Zusatztext} anzuh"angen: 

\vspace{.4ex}%
{\footnotesize
\hspace{1em}\verb|\fillper{Journal of the Royal United Service Institution}| \\[-.5ex]
\hspace*{6.95em}\verb|{Zeitschrift gegr|"u\verb|ndet 1857}     %% ist umgesetzt %%|
}
\fillper{Journal of the Royal United Service Institution}
{Zeitschrift gegr"undet 1857}%

\vspace{.75ex}\vfill\noindent
\verb|\printnumper| druckt die Liste der Zeitschriften 
(das\hspace{-.1em} \verb|.per|\hy File) dann so:

\printnumper\label{printnumper}

\noindent
Der nur einmal gesetzte fill\hy Befehl diente dazu, einen Zusatz anzuh"angen, 
der zur Vereinfachung nicht bei jedem Zitat aus der Zeitschrift getippt
werden soll. Zu den Gedankenstrichen vor den fill\hy Eintr"agen 
siehe \verb|$-$| unten S.\,\pageref{perlistopen}. 

\vspace{1ex}\noindent
Wie nach allen \BibArts\hy Befehlen (vgl.\ S.\,\pageref{Hauptbefehle}) 
sind Sie frei, \verb+|+\ko\textit{Bandangaben}\ko\verb+|+ \textbf{oder}\pdfko{.5}\
\verb+_+\textit{Heftnummern}\verb+_+ zu setzen. Beide drucken jeweils 
eigene vorgefertigte Textelemente (\textit{captions}). Im Text hier 
wurden f"ur Zeitschriften die \textit{underscores}\pdfko{.1}\ gew"ahlt
und\hspace{.2em} \verb|_|\textit{Heftnummer} (\textit{Jahr})\verb|_|\hspace{.2em} 
eingetragen. In der letzten Fu"snote \baref{Nr} stand nach \textsc{ebd.} 
zus"atzlich {\footnotesize Nr.}, was zwei Fu"snoten zuvor\pdfko{.25}\  
unterblieb, wo keine \textsc{ebd.}\hy Setzung erfolgte (Anm.\,\ref{noNr}). 
Dies ist so definiert:

\vspace{-1ex}
{\small
\begin{verbatim}
  \gpername   =>  {\ifbaibidem{, Nr.\,}{\pernosep}}    % _X_
  \gperpname  =>  {\ifbaibidem{, Nr.\,}{\pernosep}}    % _X, Y_
\end{verbatim}}

\vspace{-.75ex}\noindent
Dabei f"uhrt \verb|\ifbaibidem| sein erstes Argument im Ebenda\hy Fall,
sonst sein zweites Argument aus (das ein Leerzeichen druckt). 
\verb|\gperpname| $-$\,Plural\,$-$ f"uhrt
\BibArts\ statt \verb|\gpername| dann aus, wenn im Argument zwischen
den \textit{underscores} sich ein Minuszeichen (auch \verb|\hy|), 
ein Komma, \verb|\f| oder \verb|\ff| findet, also eine Auf"|listung 
von mehreren Zeitschriftennummern enthalten ist. 

Dies gilt "aquivalent f"ur \verb+|+\ko\textit{Bandangaben}\ko\verb+|+, 
die besonders nach dem letzten\pdfko{.25}\ Pflichtargument von 
v- oder k\fhy Befehlen stehen d"urfen (vgl.\ oben S.\,\pageref{Reinhard}):

\vspace{-1ex}
{\small
\begin{verbatim}
  \gvolname   =>  {, Bd.\,}                            % |X|
  \gvolpname  =>  {, Bde.\,}                           % |X, Y|
\end{verbatim}}

\vspace{-.75ex}\noindent
wobei Singular und Plural erkennbar unterschiedliche Separatoren drucken:

\vfill
\Doppelbox
{\vspace{.75ex}%
  \bs footnote\{Wieder \bs xkqu \{Marx\} 
\\[-1pt] \ \ \ *\{\bs kauthor\{Engels\}\} 
\\[-1pt] \ \ \ \{Werke\}\string|11-13\string|.\}
\\[3pt] \bs footnote\{\bs xkqu \{Marx\} 
\\[-1pt] \ \ \ *\{\bs kauthor\{Engels\}\} 
\\[-1pt] \ \ \ \{Werke\}\string|14\string|.\}
\\[3pt] \bs footnote\{\bs xkqu \{Marx\} 
\\[-1pt] \ \ \ *\{\bs kauthor\{Engels\}\} 
\\[-1pt] \ \ \ \{Werke\}\string|15\bs f\string|.\}
\\[3pt] \bs footnote\{\bs xkqu \{Marx\} 
\\[-1pt] \ \ \ *\{\bs kauthor\{Engels\}\} 
\\[-1pt] \ \ \ \{Werke\}\string|17, 18\string|.\}
\vspace{.75ex}%
}
{
\footnote{Wieder \xkqu {Marx} *{\kauthor{Engels}} {Werke}|11-13|.}

\footnote{\xkqu {Marx} *{\kauthor{Engels}} {Werke}|14|.}

\footnote{\xkqu {Marx} *{\kauthor{Engels}} {Werke}|15\f|.}

\footnote{\xkqu {Marx} *{\kauthor{Engels}} {Werke}|17, 18|.}
}

\noindent 
\verb|\gpername|, \verb|\gperpname| sowie \verb|\gvolname| 
und \verb|\gvolpname| lassen sich etwa\pdfko{.5}\ 
mittels \verb|\renewcommand{\gpername}{, Heft }| ver"andern
(Beispiel ohne \textit{if}):

\vspace{.7ex}
\verb+\per{ZfG.}_5_.+\hspace{1em} \verb|=>|\hspace{.5em} 
{\renewcommand{\gpername}{, Heft }\per{ZfG.}_5_.}

\vspace{.9ex}\noindent
Falls \BibArts\ nach v\fhy, k- oder per\hy Befehlen im Eintrag zwischen den 
\textit{senkrechten\pdfko{.5}\ Strichen} bzw.\ zwischen den \textit{underscores} 
Singular und Plural nicht richtig erkennt, l"asst sich mit 
\verb|\basingular| bzw.\ \verb|\baplural| \textit{am Ende} 
nachjustieren:\label{baplural}

\vspace{-.25ex}
\Doppelbox
{
\bs footnote\{\bs per \{ZfG.\}\string|11 u. 
\\ \ \ 13\string|.\}
\\[1ex]
\bs footnote\{\bs per \{ZfG.\}\string|11 u. 
\\ \ \ 13\bs baplural\string| (erzwungen).\}
\\[2ex]
\bs footnote\{\bs per \{ZfG.\}\string|17, 
\\ \ \ 18 oder 19\string|.\}
\\[1ex]
\bs footnote\{\bs per \{ZfG.\}\string|17, 
\\ \ \ 18 oder 19\bs basingular\string| (dito).\}
\\[1ex]
\bs footnote\{\bs per \{ZfG.\}\string_17, 
\\ \ \ 18 oder 19\bs basingular\string_.\}
\\[1ex]
\bs footnote\{\bs per \{ZfG.\}\string_17, 
\\ \ \ 18 oder 19\string_.\}
}
{
\footnote{\per {ZfG.}|11 u. 13|.}

\footnote{\per {ZfG.}|11 u. 13\baplural| (erzwungen).}

\footnote{\per {ZfG.}|17, 18 oder 19|.}

\footnote{\per {ZfG.}|17, 18 oder 19\basingular| (dito).}

\footnote{\per {ZfG.}_17, 18 oder 19\basingular_. 
   \texttt{ \ \% Wechsel auf \string_...\string_ \%}\label{Wechsel}}

\footnote{\per {ZfG.}_17, 18 oder 19_.}
}

\vspace{-.25ex}\noindent
Falls Sie zwischen \verb+|+...\verb+|+ und \verb+_+...\verb+_+ 
unbeabsichtigt wechseln (vgl.\ Anm.\,\ref{Wechsel}), 
erhalten Sie keine Warnung (die Zahlen werden im selben Speicher
hinterlegt). \BibArts\ macht auch hier nur die oben 
S.\,\pageref{before} und \pageref{pervol} beschrieben Fehlermeldungen.

\vspace{1ex}\noindent
Wenn Zeitschriften abgek"urzt werden \textit{und} die Abk"urzung 
zus"atzlich im Abk"urzungsverzeichnis erscheinen soll, vereinfacht
dies \verb|\abkper|: Das f"uhrt \verb|\per|\pdfko{1}\ aus 
(Liste S.\,\pageref{printnumper}) und zus"atzlich
\verb|\addtoabk| f"ur das Abk"urzungsverzeichnis\pdfko{.5}\  
\baref[siehe {\abklistemph{ZfG.}}]{abklist}. Die Abk"urzung 
erscheint dort nur, wenn sie definiert ist:

\vspace{-.75ex}
\Doppelbox
{\vspace{.5ex}
 Die \bs abkper \{ZfG.\} \bs addtoabkdef\{ZfG.\}\{Zeitschrift 
 \\ \ \ \ \ f"ur Geschichtswissenschaft\} ist ... Satzende:
 \bs abkper\{ZfG.\}.
 \vspace{.25ex}
}
{\vspace{.75ex}%
 Die \abkper {ZfG.} \addtoabkdef{ZfG.}{Zeitschrift f"ur 
 Geschichtswissenschaft} ist eine wissenschaftliche Zeitschrift.
 Am Satzende: \abkper{ZfG.}.
}

\vspace{-.5ex}\noindent
Hinter dem Hauptargument von \verb|\abkper| d"urfen Angaben zu
Heftnummern und Seitenzahlen stehen wie nach jedem \BibArts\hy 
Hauptbefehl (siehe S.\,\pageref{Hauptbefehle}).
%<<<

\vspace{1.25ex}\noindent
\verb|\per{|\textsc{Argument}\verb|}| und \verb|\abkper{|\textsc{Argument}\verb|}| 
werden in \verb|\peremph| ausgedruckt. 
Dessen Definition darf nicht leer sein; zumindest \verb|\upshape| 
sollte\pdfko{.1}\ darin stehen $-$ denn mit \verb|\renewcommand{\peremph}{}| 
allein w"urden alle per\hy Befehle, die in schr"aggestelltem 
Schriftumfeld stehen, etwas melden wie:

\vspace{.5ex}{\scriptsize\begin{verbatim}
   BibArts Warning: Add \upshape to \peremph on input line 1696.
\end{verbatim}}


%>>>
\noindent
Die Aufgaben von \verb|\per| lassen sich in
\verb|\addtoper| und \verb|\printonlyper| teilen: 

\Doppelbox
{
Die \bs printonlyper\{Zeitschrift
\\ \ \ \ f"ur Geschichtswissenschaft\} 
\bs addtoper\{ZfG.\} 
\\ soll als Abk"urzung ins Zeitschriftenverzeichnis. 
\\[.25ex] Alternativ gibt auch 
\bs per\b{\{}Z\bs onlyhere\{eitschrift \}\%
\\ \ \ \ \ \ f\bs onlyhere\{"ur \}\%
\\ \ \ \ \ \ G\bs onlyhere
\\ \ \ \ \ \ \ \{eschichtswissenschaft\}\%
\\ \ \bs onlyout\{.\}\b{\}} nur einen Eintrag.
}
{
Die \printonlyper{Zeitschrift f"ur Geschichtswissenschaft} 
\addtoper{ZfG.} 
soll als Abk"urzung ins Zeitschriftenverzeichnis.
Alternativ gibt auch 
\per{Z\onlyhere{eitschrift }%
f\onlyhere{"ur }%
G\onlyhere{eschichtswissenschaft}%
\onlyout{.}} nur einen Eintrag.
\\[.75ex]
\fbox{\parbox{.95\textwidth}{Vergleichen Sie dazu die Angabe 
 der Seite \label{hier}{\balistnumemph{\pageref{hier}}} hier
 nach {\perlistemph{ZfG.}}\ im Zeitschriftenverzeichnis 
 oben S.\,{\pageref{printnumper}}.}}\vspace{-.75ex}
}

\vfill\noindent
In Voreinstellung \verb|\printlongpervol| wird vor der 
\verb+|+\ko\textit{Bandnummer}\ko\verb+|+ der Separator\hspace{.2em} 
{\footnotesize Bd.}\hspace{.2em} 
ausdruckt. Mit \verb|\notprintlongpervol| wird stattdessen nur 
ein Leerzeichen gedruckt; davor bleibt nach {\footnotesize [L]} bzw.\ 
{\footnotesize [Q]} das Komma erhalten:

\Doppelbox
{
Voreinstellung.\bs footnote\{ 
\\ \ \ \bs kqu\{Marx\}\{Kapital\}\string|1\string|[2].\}
\\[2ex] \bs footnote\{Kein Ebenda.\}
\\[4ex] \bs notprintlongpervol
\\[1ex] ...\bs footnote\{ 
\\ \ \ \bs kqu\{Marx\}\{Kapital\}\string|1\string|[3].\}
\\[1ex] ...\bs footnote\{ 
\\ \ \ \bs kqu\{Marx\}\{Kapital\}\string|1\string|[4].\}
}
{
Voreinstellung.\footnote{
\kqu{Marx}{Kapital}|1|[2].}

\footnote{Kein Ebenda.}

\

\notprintlongpervol
...\footnote{
\kqu{Marx}{Kapital}|1|[3].}

...\footnote{
\kqu{Marx}{Kapital}|1|[4].}
}\label{notprintlongpagefolio}

\noindent
\verb|\notprintlongpagefolio| stellt ein, dass vor der Seitenzahl 
ein Doppelpunkt statt\hspace{.1em} {\footnotesize S.}\hspace{.1em} 
steht. Mit dem erw"ahnten \verb|\notprintlongpervol| wird bei 
\verb+_+\textit{Heftnummern}\verb+_+ im Falle von \textsc{ebd.}\hy 
Setzung zus"atzlich\hspace{.1em} {\footnotesize Nr.}\hspace{.1em} nicht ausgedruckt:

\Doppelbox
{
 \bs notprintlongpagefolio
 \\[.5ex] \bs footnote\{ 
 \\ \ \ \bs per\{ShortMagazine\}\string_25\string_[4].\}
 \\[.2ex] \bs footnote\{ 
 \\ \ \ \bs per\{ShortMagazine\}\string_25\string_[5].\}
 \\[.2ex] \bs footnote\{ 
 \\ \ \ \bs per\{ShortMagazine\}\string_26\string_[6].\}
 \\[3ex] \bs notprintlongpervol
 \\[.5ex] \bs footnote\{
 \\ \ \ \bs per\{ShortMagazine\}\string_27\string_[7].\}
}
{
 \notprintlongpagefolio
 \footnote{\per {ShortMagazine}_25_[4].}

 \footnote{\per {ShortMagazine}_25_[5].}

 \footnote{\per{ShortMagazine}_26_[6].}

 \notprintlongpervol
 \footnote{\per{ShortMagazine}_27_[7].}
}



\noindent
Das folgende Beispiel zeigt (unten), was Setzen von
\verb|\notprintlongpervol| und 
\verb|\notprintlongpagefolio|\footnote{\texttt{\bs notprintlongpagefolio}\hspace{.1em} 
bewirkt zudem, dass vor\hspace{.05em} \texttt{(}\textit{n}\texttt{)}\hspace{.05em} nicht "`Bl."' 
ausgedruckt wird; vgl.\hspace{-.1em} \texttt{\bs arq}\hspace{.1em} im 
folgenden Kap.\,\ref{archivquellen} \baref{BdSarq}.} 
zusammen mit \verb|\notannouncektit| ergibt:

\vspace{.5ex}
\Doppelbox
{
\vspace{1.5ex}
\bs notprintlongpagefolio
\\[1.5ex]
Voll.\bs footnote\{\bs vqu \{John 
\\ \ \ Frederick Charles\} \{Fuller\} 
\\ \ \H{\{}Gold Medal (Military) 
\\ \ \ \bs ktit\{Prize Essay\} for 1919, 
\\ \ \ in: \bs per\{Journal of the
\\ \ \ \ \ \ \ \ Royal United Service 
\\ \ \ \ \ \ \ \ Institution\}\string_458 
\\ \ \ \ \ \ \ \ \ (1920)\string_[239-274]\H{\}}*[240].\}
\\[1.5ex]
...\bs footnote\{\bs kqu \{Fuller\} 
\\ \ \ \ \ \ \ \ \ \ \ \ \{Prize Essay\}[241].\}
\\[3.5ex]
   \bs notannouncektit
\\[.25ex]
   \bs notprintlongpervol
\\[1.5ex]
...\bs footnote\{\bs vqu\{R[ichard]\} 
\\ \ \{Chevenix Trench\} 
\\ \ \{Gold Medal (Military) 
\\ \ \ \bs ktit\{Prize Essay\} for 1922, 
\\ \ \ in: \bs per\{Journal of the 
\\ \ \ \ \ \ \ \ Royal United Service 
\\ \ \ \ \ \ \ \ Institution\}\string_470 
\\ \ \ \ \ \ \ \ \ (1923)\string_[199-227]\}*[200].\}
\\[-1ex] \strut
}
{\notktitaddtok
\notprintlongpagefolio
\texttt{\% Solche Befehle sollten nur}\\
\texttt{\% global gesetzt werden; hier}\\
\texttt{\% geht es darum, die Konse-}\\
\texttt{\% quenzen zu demonstrieren.}
\\[7.5ex]
Voll.\footnote{\printonlyvqu {John\, Frederick\, Charles\,} {Fuller} 
{Gold Medal (Military) \ktit{Prize Essay} for 1919, 
in: \per{Journal of the Royal United Service 
Institution}_458 (1920)_[239-274]}*[240].\label{erste}}

\vspace{3.5ex}
"Au"seres Ebenda.\footnote{\printonlykqu {Fuller} 
{Prize Essay}[241].}

\notannouncektit
\notprintlongpervol
\vspace{1.7ex}
Voll und inneres Ebenda.\footnote{\printonlyvqu{R[ichard]\ \ } {Chevenix\ \ Trench} 
{Gold Medal (Military) \ktit{Prize Essay} for 1922, 
in: \per{Journal of the Royal United Service 
Institution}_470 (1923)_[199-227]}*[200].}
}

\vspace{.5ex}\noindent
\verb|\notprintlongpervol| ordnete an, dass in der letzten Fu"snote
nach dem inneren \textsc{ebd.} (vor {\footnotesize 470}) kein {\footnotesize Nr.}\ 
ausgedruckt wurde. In Fu"snote~\ref{erste} fehlte {\footnotesize Nr.}\pdfko{.5}\
bereits, weil dort kein \textsc{ebd.} gesetzt ist und
in \verb|\gpername| das zweite Argument von \verb|\ifbaibidem| dann 
\verb|\pernosep| ausdruckt, ein gesch"utztes Leerzeichen.\footnote{
\texttt{\bs renewcommand\{\bs pernosep\}\{\bs bastrut\bs\ \bs bacorr\}}
\\ w"urde stattdessen einen Zeilenumbruch am Leerzeichen erlauben \balabel{pernosep}
("Anderung in \BibArts\ 2.2).}\pdfko{.5}

\vspace{3ex}\noindent
Die Befehle sind auch auf Listen anwendbar. Ausgedruckt
werden w"urde beispielsweise ein Eintrag unter\hspace{.1em} 
\verb|{\notprintlongpagefolio|\,\verb|\printvqu|\,\verb|}|\hspace{.1em} so:

\vspace{.1ex}
{\footnotesize
\begin{description}\parsep 0ex \itemsep -.5ex
\item \textsc{[Anonym]}: Aufmarschanweisungen 1912, 
abgedruckt in: \textsc{Ehlert}\baslash \textsc{Epkenhans}\baslash \textsc{Gro"s}\pdfko{1}\
[Hrsg.]: Schlieffenplan [Q]: 462-466.
\end{description}}



\newpage
\section{Archivquellen}\label{Sect7}\label{archivquellen}

Historische Forschungsliteratur weist h"aufig ein separates Verzeichnis 
f"ur \textit{ungedruckte Quellen} auf, die \BibArts\ wiederum aus 
Haupttext oder Fu"snoten gewinnen kann. Zudem ist eine korrekte 
\textsc{ebd.}\hy Setzung in Fu"snoten n"otig. Beides bew"altigt der Befehl\hspace{-.1em} 
\verb|\arq| mittels zwei Pflicht- und zwei optionalen Argumenten. 
Das erste Pflichtargument nennt ein Schriftst"uck und das zweite eine 
Archivsignatur (eventuell samt Eigennamen des Quellenbestandes). Die 
\textsc{ebd.}\hy Setzung kann mal Schriftst"uck \textit{und} Signatur betreffen, 
mal nur die\pdfko{.25}\ 
Signatur (wenn Sie ein anderes Schriftst"uck aus derselben 
Akte zitieren). Nur das zweite Pflichtargument kommt ins Verzeichnis 
ungedruckter Quellen.

Falls Sie in Ihren Fu"snoten h"aufig verschiedene Mappen \textit{einer} 
Akte (gleiche Signatur) verwenden, k"onnen Sie zudem etwas wie\hspace{.1em}  
{\footnotesize\textsc{Ebd.}, Bd.\,2}\hspace{.15em} drucken lassen: Die Mappen\hy 
Nummern w"aren \textit{dann} stets in senkrechten Strichen nach dem zweiten 
Pflichtargument zu nennen. Falls die Schriftst"ucke in der Mappe paginiert 
sind, kann die Blattnummer \textit{in jedem Fall} zuletzt in 
runden Klammern stehen. Vor {\small\verb+|+\ko\textit{Band}\verb+|+} sowie 
vor {\small\verb+(+\textit{Blatt}\verb+)+} darf kein Leerzeichen sein:

\vspace{-.125ex}
\Doppelbox
{\strut\\[-2ex]
...\bs footnote\{\bs arq\{Haber 
\\ \ am 17.12.1914 an 
Kultusminister\} \{GStAPK, HA\bs ,1, Rep\string~76\string~Vc, 
\\ \ Sekt\string~1, 
Tit\string~23, Litt\string~A, 
\\ \ Nr.\bs ,108\}\string|2\string|(223\bs f).\}
\\[1ex]
...\bs footnote\{\bs arq\{Setsuro Tamaru 
\\ \ \ \ am 24.12.1914 an 
Clara Haber\} \{GStAPK, HA\bs ,1, Rep\string~76\string~Vc, 
\\ \ Sekt\string~1, 
Tit\string~23, Litt\string~A, 
\\ \ Nr.\bs ,108\}\string|2\string|(226-231).\}
\\[1ex]
...\bs footnote\{\bs arq\{Setsuro Tamaru 
\\ \ \ \ am 24.12.1914 an 
Clara Haber\} \{GStAPK, HA\bs ,1, Rep\string~76\string~Vc, 
\\ \ Sekt\string~1, 
Tit\string~23, Litt\string~A, 
\\ \ Nr.\bs ,108\}\string|2\string|(226-231).\}
\\[1ex]
...\bs footnote\{\bs arq\{Valentini am 
\\ \ \ \ 13.3.1911 an Schmidt\} \{GStAPK, 
HA\bs ,1, Rep\string~76\string~Vc, Sekt\string~1, Tit\string~23, Litt\string~A, 
Nr.\bs ,108\}\string|1\string|(47).\}
\\[-2.25ex]\strut
}
{
Solche komplexen Signaturen m"ussen nat"urlich
mit der Kopier\hy Funktion des Texteditors eingef"ugt
werden; nur bei gleichen Eintr"agen f"uhrt \BibArts\ eine Ebendasetzung durch:

\vspace{.5ex}
Neue Akte, erste Mappe.\footnote{\arq{Haber am 17.12.1914 an 
Kultusminister} {GStAPK, HA\,1, Rep~76~Vc, Sekt~1, 
Tit~23, Litt~A, Nr.\,108}|2|(223\f).}

Selbe Mappe, anderes Schriftst"uck.\footnote{\arq{Setsuro Tamaru am 24.12.1914 an 
Clara Haber} {GStAPK, HA\,1, Rep~76~Vc, Sekt~1, 
Tit~23, Litt~A, Nr.\,108}|2|(226-231).}

Selbes Schriftst"uck.\footnote{\arq{Setsuro Tamaru am 24.12.1914 an 
Clara Haber} {GStAPK, HA\,1, Rep~76~Vc, Sekt~1, 
Tit~23, Litt~A, Nr.\,108}|2|(226-231).}

Selbe Akte, zweite Mappe.\footnote{\arq{Valentini am 13.3.1911 an Schmidt} 
{GStAPK, HA\,1, Rep~76~Vc, Sekt~1, Tit~23, Litt~A, 
Nr.\,108}|1|(47).} 
}

\vspace{-.25ex}\noindent
Blattnummern werden (deutsch) im Singular und Plural gleich angek"undigt:

\vspace{-.625ex}
{\small
\begin{verbatim}
  \gisonfolioname  => {, Bl.\,}
  \gisonfoliopname => {, Bl.\,}
\end{verbatim}}

\vfill\noindent
Ohne dies mit \verb|\renewcommand| zu ver"andern, wirkt sich alternativ der 
erw"ahnte Schalter \verb|\notprintlongpagefolio| so auf \verb|\arq| aus:


\newpage
\Doppelbox
{\strut\\[-2ex]
\bs notprintlongpervol
\bs notprintlongpagefolio
\\[1ex]
...\bs footnote\{ \bs arq\{Haber am 
\\ \ \ \ 17.12.1914 an Kultusminister\} 
\\ \ \ \b{\{}GStAPK, HA\bs ,1, Rep\string~76\string~Vc, 
\\ \ \ \ \ \ Sekt\string~1, Tit\string~23, Litt\string~A, 
\\ \ \ \ \ \ \ \ Nr.\bs ,108\b{\}}\string|2\string|(223 \bs f).\}
\\[-2.25ex]\strut
}
{\notprintlongpervol
 \notprintlongpagefolio
Die \texttt{|}\ko\textit{Bandnummer}\ko\texttt{|} 
wird hier ohne `Bd.'\slash`Bde.' ausgedruckt und 
\texttt{(}\textit{Blatt}\texttt{)} steht hinter `:' 
statt `Bl.'.\footnote{ \arq{Haber am 17.12.1914 an 
Kultusminister} {GStAPK, HA\,1, Rep~76~Vc, Sekt~1, 
Tit~23, Litt~A, Nr.\,108}|2|(223 \f).\balabel{BdSarq}}
}\balabel{banotlong}%


\vspace{1.75ex}\noindent
In den\hspace{.1em} \textbf{Ausdruck des Archivquellenverzeichnisses}\hspace{.1em} lassen sich 
optional "Uberschriften einf"ugen. Zur korrekten Sortierung 
muss das erste Argument mit \textit{den ersten Buchstaben
der jeweils "uberschriebenen Signatur} beginnen:

\vspace{1.25ex}\noindent
{\small
\verb|  \arqsection{GStAPK}{Geheimes Staatsarchiv|\\
\verb|                         Preu|\texttt{"s}\verb|ischer Kulturbesitz}|\\
\verb|  \arqsection{BA} {Bundesarchiv}|
}
  \arqsection{GStAPK}{Geheimes Staatsarchiv 
                      Preu"sischer Kulturbesitz}      
  \arqsection{BA} {Bundesarchiv}

\vspace{2ex}\noindent
\verb|\arqsubsection| \label{arqsection} erzeugt eine Unter\fhy,
\verb|\arqsubsubsection| eine Unter\hy Unter\hy "Uberschrift;
sie m"ussen je in \underline{mehr} Zeichen mit den "uberschriebenen 
Signaturen "ubereinstimmen. 
\verb|\arqsubsection {GStAPK|\underline{\texttt{, HA}}\verb|} {Hauptabteilung}|
\arqsubsection {GStAPK, HA} {Hauptabteilung} wurde hier verwendet. 
Die Zahl der Mappen in Nr.\,108 kann au"serdem mit einmaligem 
fill\hy Befehl an den \verb|\arq|\hy Zugang im Verzeichnis angeh"angt werden:

\vspace{-.25ex}\noindent
{\small
\begin{verbatim}
  \fillarq{GStAPK, HA\,1, Rep~76~Vc, Sekt~1, 
                     Tit~23, Litt~A, Nr.\,108} {2\,Bde.}
\end{verbatim}}
  \fillarq{GStAPK, HA\,1, Rep~76~Vc, Sekt~1, 
                     Tit~23, Litt~A, Nr.\,108} {2\,Bde.}


\noindent
Um im Verzeichnis "Uberschriften auf neue Seiten zu setzen,
w"aren alternativ zudem Angaben wie \verb|\arqsection[\newpage]{BA}{Bundesarchiv}| m"oglich.

\vspace{2ex}\noindent
\verb|\printarq| druckt die von \verb|bibsort| erzeugte\hspace{-.15em} 
\verb|.arq|\hy Datei so aus (vgl.\ S.\,\pageref{newpage}):

\vfill
\printarq


\newpage\noindent
Falls Sie\hspace{-.1em} \verb|\arq| in\hspace{-.1em} \verb|\printonlyarq| 
und\hspace{-.1em} \verb|\addtoarq| separieren, m"ussen beide die genannten 
zwei Pflichtargumente haben. Das erste Argument von\hspace{-.1em} 
\verb|\addtoarq| sollten Sie dabei nicht leer lassen, sondern 
wie\hspace{-.1em} \verb|\printonlyarq| bef"ullen ... 

\Doppelbox
{
...\bs footnote\b{\{} \bs printonlyarq\{\% 
\\ \ \ Gesellschaftsvertrag der KCAG\} 
\\ \ \{BA R\bs ,8729\string~4\}(94). 
\\[.25ex] \ \bs addtoarq\{Gesellschaftsvertrag 
\\ \ \ der KCAG\}\{BA Zwischenarchiv 
\\ \ \ Dahlwitz\bs hy Hoppegarten 
\\ \ \ R\bs ,8729\string~4\}(94)
Die ...\b{\}}
}
{
...\footnote{ \printonlyarq{% 
Gesellschaftsvertrag der KCAG} 
{BA R\,8729~4}(94). 
\addtoarq{Gesellschaftsvertrag 
der KCAG}{BA Zwischenarchiv 
Dahlwitz\hy Hoppegarten 
R\,8729~4}(94)
\\ Die \texttt{(94)} hinter dem addto\hy Befehl wird einfach 'verschluckt' (nicht gedruckt). 
\\ Grundlage der \textsc{ebd.}\hy Setzung ist die printonly\hy Komponente.}
}

\noindent
... denn es wird (wie sonst auch auch die jeweils
ersten Argumente von\hspace{-.1em} \verb|\arq|)
zur Nachvollziehbarkeit als \hspace{.1em} 
{\small\verb|%%|~\textit{Kommentar}\ko~\verb|%%|} 
\hspace{.1em} im\hspace{-.1em} \verb|.aux|\hy File festgehalten:

\vspace{-.75ex}
{\footnotesize\begin{verbatim}
   %\archqentry{BA Zwischenarchiv Dahlwitz\hy Hoppegarten 
      R\,8729~4}{}{{}{}{}{}}{{35}{97}{@}}[13][42](line 2704)
          %% (mpf)  Gesellschaftsvertrag der KCAG %%
\end{verbatim}}

\vspace{-.75ex}\noindent 
Die Eintr"age {\small\verb|(f)|} bzw.\ {\small\verb|(mpf)|} 
dokumentieren, ob Zug"ange aus einer normalen oder 
\verb|minipage|\hy Fu"snote herstammen (oder leer: nicht aus 
einer Fu"snote). 

\vspace{1.25ex}\noindent
Im\hspace{-.1em} \verb|.arq|\hy File stehen alle 
\textsf{R\,8729~4} zusammengefasst unter der BA\hy "Uberschrift.
Im {\small\verb|%%|~\textit{Kommentar}\ko~\verb|%%|} 
steht der erste Eintrag.
Die Band\fhy\ko\slash Mappen\hy Nummer\,\verb|4|\pdfko{1}\ stand stets
\textit{im} zweiten\hspace{-.1em} \verb|\arq|\hy Argument und wird im Verzeichnis ausgedruckt:

\vspace{-.5ex}
{\scriptsize\begin{verbatim}
   \archqentry{BA}{Bundesarchiv}{{\bastrut \ \bacorr
     $-$ }{\bahasdot }{1}{}}{{}{}{-}}[13][42](line 2641)
       %%  <- List-internal heading (class 1).

   \archqentry{BA Zwischenarchiv Dahlwitz\hy Hoppegarten
     R\,8729~4}{}{{}{}{}{}}{{35}{}{}}[13][42](line 2760)  %%  %%
       \first@baidx{35, 35$^{97}$, 41$^{101}$, 48}
\end{verbatim}}

\vspace{-.5ex}\noindent
Falls Sie Ihren Leser aber einfach nur darauf hinweisen wollen, dass 
es einen Bestand oder eine Akte gibt (also kein bestimmtes 
Schriftst"uck daraus zitieren), kann das erste Argument von arq\hy 
Befehlen freilich auch leer bleiben (das ist "ubrigens der erste
\textsf{R\,8729~4}\hy Eintrag S.\,\pageref{toparq}, der obiges\hspace{.3em} 
{\scriptsize\verb|%%  %%|}\hspace{.3em} erzeugte):

\Doppelbox
{
In \bs arq\{\}\{BA Zwischenarchiv 
\\ \ \ \ \ \ \ Dahlwitz\bs hy Hoppegarten 
\\ \ \ \ \ \ \ R\bs ,8729\string~4\} findet sich ...
}
{
In \arq{}{BA Zwischenarchiv 
Dahlwitz\hy Hoppegarten 
R\,8729~4} findet sich ...
\label{toparq}
}

\vfill\noindent
{\sffamily Zur Schrifteinstellung mit }\verb|\arqemph|{\sffamily\ 
und }\verb|\arqlistemph|{\sffamily\ unten S.\,\pageref{arqemph1} und \pageref{arqemph2}. \\
Und S.\,\pageref{Gedankenstriche} wird erkl"art, wie die 
fill\hy\ und section\hy Separatoren einzustellen sind.}


\newpage
\section{Orts\fhy, Sach\hy\ und Personenregister}\label{Sect8}

\BibArts\ stellt drei Register zur Verf"ugung. (\textsc{MakeIndex} kann
unabh"angig\pdfko{.5}\  
davon parallel verwendet werden.) \verb|bibsort| nutzt seine F"ahigkeit,
auch Fu"snotennummern zu verarbeiten. Bef"ullt werden die Register mit\hspace{-.1em} 
\verb|\addtogrr|\pdfko{.75}\ 
(Ortsregister), \verb|\addtosrr| (Sachregister) und \verb|\addtoprr|
(Personenregister). Ein vielfach verwendetes Stichwort \textit{kann} zudem mittels fill\hy 
Befehl einen ausf"uhr"-lich{\small(}\hspace{-.05em}er\hspace{-.05em}{\small)}en 
Zusatz erhalten, der nur einmal getippt zu werden braucht. 
fill\hy Befehle haben ein benutztes Stichwort als 
erstes und eine Erg"anzung dazu als zweites Argument.
Sie sind wie die addto\hy Befehle im Text unsichtbar:

\Doppelbox
{
\vspace{.75ex}
\underline{\bs fillgrr\{Rom\}\{Stadt in Italien\}}
\\[1.2ex] Hier geht es ums Ortsregister (Geografie).\bs footnote\b{\{}Rom 
\\[0.2ex] \ \ \underline{\bs addtogrr\{Rom\}} ist ein Ort.\b{\}} 
\\[1.1ex] Nero lebte in Rom.\underline{\bs addtogrr\{Rom\}} 
\\[1.1ex] \bs printnumgrrlist
\vspace{.25ex}
}
{
\fillgrr{Rom}{Stadt in Italien}
Hier geht es ums Ortsregister (Geografie).\footnote{Rom 
\addtogrr{Rom} ist ein Ort.} 
Nero lebte in Rom.\addtogrr{Rom} 
 %{\batwocolitemdefs
\printnumgrrlist
 %}
\vspace{-.25ex}
}


\vspace{-.25ex}\noindent
\verb|\printnumgrr| erg"abe (mit "Uberschrift \verb|\ggrrtitlename| im Deutschen: \ggrrtitlename) 
einen zweispaltigen Ausdruck, der hier vermieden wurde. Die beiden
anderen Register werden mit \verb|\printnumprr| und \verb|\printnumsrr| 
ausgedruckt.\footnote{\balabel{fromnopagexrrsep}Falls Eintr"age in den zahlen\hy losen Ausgaben  
\texttt{\bs printgrr}, \texttt{\bs printsrr} und \texttt{\bs printprr}\pdfko{1.5}\ 
mit einem Punkt enden sollen:\hspace{.25em}
\texttt{\bs renewcommand\{\bs fromnopagexrrsep\}\{\bs bapoint\}}} 
Stichworte wie "`Rom"' werden in \verb|\xrrlistemph| gesetzt, das \label{xrr}
sich mit \verb|\itshape| etc.\ belegen l"asst; verwendete \textit{fills} 
behalten die Umfeldschrift.

\vspace{2ex}\noindent
\verb|bibsort| kann keine Unterstichworte erzeugen, sondern nur 
Haupteintr"age. Argumente (Stichworte) werden wie immer so getippt, dass 
\LaTeX\ sie auch direkt drucken w"urde (anders als bei \verb|\index|
gibt es keine Sonderzeichen); nur zerbrechliche 
Befehle sollten Sie mit \verb|\protect| sch"utzen.

\vspace{2ex}\noindent
Neben \verb|\fillgrr| existieren \verb|\fillsrr| und \verb|\fillprr|. 
Falls Sie ein Stichwort "uberfl"ussigerweise zweimal und dann irrt"umlich
auch noch mit unterschiedlichen 
Zus"atzen bef"ullen, w"urde \verb|bibsort| z.\,B. folgende Warnung ausgeben:

\vspace{-.75ex}
{\scriptsize\begin{verbatim}
    %%>   Warning: Different fills for head "Rom":
    %%     *Accept file 1 line 2669 "Stadt in Italien";
    %%     *Reject file 1 line 2678 "Stadt in Mittel-Italien".
\end{verbatim}}

\vspace{-1ex}\noindent
Den Registern lassen sich Querverweise der Art "`Roma $\rightarrow$ Rom"' 
hinzuf"ugen:

\vspace{-1.25ex}{\small
\begin{verbatim}
  {\renewcommand{\xrrlistopen}{\bastrut\ \bacorr$\rightarrow$ }
   \renewcommand{\xrrlistclose}{}
   \fillgrr{Roma}{Rom}}            %\addtogrr{Roma} nicht verwenden!
\end{verbatim}}\label{xrrlistclose}%

In den Listen bestimmen allein die Stichworte die Sortierreihenfolge;
die F"ullungen haben kein Gewicht. Falls Sie die Zusatzf"ullungen nicht in
runden Klammern gedruckt haben wollen, k"onnen Sie \verb|\xrrlistopen| 
etwa in \verb|{, }|\pdfko{.5}\  
und \verb|\xrrlistclose| in \verb|{}|
"andern.\footnote{Im Beispiel sind wegen des niederen Zeichens~`\texttt{,}' 
am Kopf von \texttt{\bs xrrlistopen} weder \texttt{\bs bastrut} noch die 
\textit{italics}\hy Korrektur \texttt{\bs bacorr} n"otig; vgl. 
unten das Kap.\,\ref{schraegkap} ab S.\,\pageref{schraegkap}.} 
Die Serie von fill\hy Befehlen ist im folgenden sorgf"altigen Beispiel 
in eine \verb|{unused}|\hy Umgebung gesetzt \baref[vgl.]{unused}, 
damit an ihren Positionen im Text kein horizontaler Leerraum erzeugt wird:

\vspace{1ex}
{\small\begin{verbatim}
 %% Verschiedene Stellen im Text mit Namen:
       ... Winston Churchill \addtoprr{Churchill} ...
             ... Hans Maier \addtoprr{Maier, Hans} ...
                 ... Peter Maier \addtoprr{Maier, Peter} ...
  ... Theobald von Bethmann-Hollweg \addtoprr{Bethmann-Hollweg} ...
 
                    %% Eine gute Stelle zum Sammeln der optionalen 
                    %% Zusatzfuellungen ist VOR dem Ausdruckbefehl:             
    \begin{unused}
      {\renewcommand{\xrrlistopen}{, }%
       \renewcommand{\xrrlistclose}{}%
       \fillprr{Churchill}{Winston (1874-1965)}
       \fillprr{Bethmann-Hollweg}{Theobald von (1856-1921)}
      }%
                    %% Wieder in Default-Klammerung (...):
      \fillsrr{Maier, Peter}{1887-\protect\framebox{????}}
    \end{unused}

  \printnumprr
\end{verbatim}}\label{pgofexample}


\vspace{1.5ex}\noindent
Im obigen Beispiel sollen Personen, deren Nachnamen nur je \textit{eine}
Person im Text hat, in den \verb|\addtoprr|\hy Argumenten eben auch nur mit
diesem Nachnamen angetippt werden. Diese Nutzung ist wohl nur
f"ur Fortgeschrittene geeignet. Verdeutlicht werden sollte 
jedoch: Die lokalen Umdefinitionen von \verb|\xrrlistopen| und 
\verb|\xrrlistclose| reisen separat mit dem \verb|Churchill|\hy\ und\pdfko{.5}\ 
\verb|Bethmann-Hollweg|\hy \textit{fill} ins \hspace{-.2em}\verb|.prr|\hy File. 
Dann druckt \verb|\printnumprr| in etwa:


\vspace{-0.25ex}
\begin{description}\itemsep 0pt \parskip 0pt \lineskip 0pt \small
\item Bethmann-Hollweg, Theobald von \\ (1856-1921)\hspace{1.2em}\textsf{\pageref{pgofexample}}
\item Churchill, Winston (1874-1965)\hspace{1.2em}\textsf{\pageref{pgofexample}}
\item Maier, Hans\hspace{1.2em}\textsf{\pageref{pgofexample}}
\item Maier, Peter (1887-\framebox{????})\hspace{1.2em}\textsf{\pageref{pgofexample}}
\end{description}


\newpage
\section{\texttt{\bs protect} und zerbrechliche Befehle}\label{Sect9}\label{zerbrechen}

Ein \LaTeX\hy Befehl $-$\,etwa mit 
\verb|\newcommand{|\textit{Befehlsname}\verb|}{|\textit{Deklaration}\verb|}| 
definiert\,$-$ arbeitet bei der Ausf"uhrung seine Deklaration ab. Die
besteht oft aus mehreren schon vorhandenen \LaTeX\hy Befehlen. Falls
Sie den neuen Befehl\pdfko{.75}\ 
\textit{in das Argument} eines \BibArts\hy Befehls wie etwa
\verb|\vli| tippen, wird eine Kopie dieses Eintrags an Ort und Stelle
ausgedruckt und eine zweite Kopie in das\hspace{-.2em} \verb|.aux|\hy 
File geschrieben. Ist der neue Befehl nicht gesch"utzt, wird er beim 
Schreiben in das\hspace{-.2em} \verb|.aux|\hy File von \LaTeX\ 
allerdings teilweise ausgef"uhrt: Enth"alt die Deklaration gesch"utzte 
Befehle, werden \textit{sie} ins\hspace{-.2em} \verb|.aux|\hy File 
geschrieben; sind sie aber ungesch"utzt, wiederum deren Deklarationen 
$-$ u.\,s.\,w.

Wie weit ein Befehl in diesem Sinne `zerbricht', ist also unklar. Ist
ein neuer Befehl ungesch"utzt, droht zumindest, dass \verb|bibsort| Ihre
Eintr"age nicht richtig sortiert. Schlimmstenfalls wird beim Schreiben 
ins\hspace{-.2em} \verb|.aux|\hy File oder beim Drucken der daraus erzeugten Liste die 
\TeX\hy Kapazit"at "uberschritten und die \LaTeX\hy "Ubersetzung Ihres 
Textes abgebrochen.

Seit \LaTeX2e\ ist letzteres kaum noch ein Problem, da fast alle wichtigen
Befehle gesch"utzt definiert sind. Allerdings bleibt das Risiko, dass
Titel mit Ihren eigenen Neudefinitionen falsch einsortiert werden. Wenn Sie etwa
\verb|\newcommand{\meinspace}{{\hskip 3cm}}| definieren und
\verb|\meinspace| in das Argument eines \verb|\vli|\hy Befehls tippen, 
wird dies bei der \LaTeX\hy "Ubersetzung den Eintrag 
\verb|{\hskip 3cm}| im\hspace{-.2em} \verb|.aux|\hy File ergeben und Ihr
Literaturtitel von \verb|bibsort| im\hspace{-.2em} \verb|.vli|\hy File 
entsprechend der Zeichenfolge \verb|3cm| einsortiert.

\underbar{Gegenma"snahme:} Durch Tippen von \verb|\protect\meinspace|
in solche Argumente ist der Befehl gesch"utzt; es wird \verb|\meinspace| 
ins\hspace{-.2em} \verb|.aux|\hy File kopiert.

Dabei muss \verb|\protect| also nicht von \verb|\onlyout| maskiert werden!
Vielmehr arbeitet \verb|\protect| in der addto- und der printonly\hy Komponente
des \verb|\vli|\hy Befehls unterschiedliche Deklarationen ab; beim Ausdrucken 
tut es nichts.

Falls Sie eine Eigendefinition sehr oft benutzen, k"onnen Sie $-$\,wie
oben S.\,\pageref{ProtectBeispiel} f"ur \verb|\ko| vorgemacht\,$-$ den 
Schutz in eine Doppel\hy Definition einf"ugen.

Da die \BibArts\hy Befehle meist nur Text aufnehmen sollen, stellt sich das
Problem selten. In \verb|bibarts.sty| habe ich ggf.\ zerbrechliche Befehle  
gesichert, die zum Drucken von \textit{Buchstaben} dienen. Dieser Schutz erstreckt
sich auf die Argumente aller \BibArts\hy Befehle, jedoch nicht auf die 
Argumente von \LaTeX\hy Befehlen wie \verb|\section|, die ebenfalls in Dateien 
schreiben! Nirgendwo gesichert ist \verb|\underline{X}|, von dem 
(ohne \verb|\protect| davor) im\hspace{-.2em} \verb|.aux|\hy File\pdfko{.25}\
etwas wie\hspace{.1em} 
\verb|\relax| \verb|$\@@underline| \verb|{\hbox {X}}\mathsurround| \verb|\z@| \verb|$\relax|\pdfko{.25}\hspace{.1em} 
ankommt. Das wird von \verb|bibsort| zwar zwischen \verb|W|
und \verb|Y| einsortiert; trotzdem sollten Sie ein neues\hspace{-.2em} \verb|.aux|\hy File
bzw.\ die von \verb|bibsort| daraus erzeugten Dateien immer durchsehen,
nachdem Sie einen Befehl in ein \BibArts\hy Argument setzten,
"uber dessen Zerbrechlichkeit\slash Unzerbrechlichkeit Sie nichts wissen.

Ob ein Befehl mit \verb|\protect| gesch"utzt werden sollte, ist nicht immer nur am
Umstand erkennbar, ob das, was Sie im\hspace{-.15em} \verb|.tex|\hy File eingegeben haben,
auch ebenso im\hspace{-.15em} \verb|.aux|\hy File erscheint. Der \BibArts\hy Befehl \verb|\hyf|
ist ein Beispiel: Falls Sie \verb|\sethyphenation{german}| festgelegt haben,
wird ins\hspace{-.15em} \verb|.aux|\hy File \verb|\oldhyf| eingetragen, sonst \verb|ff|. Das
sollte eben nicht mit \verb|\protect| unterbunden werden (\verb|bibsort| kann
sogar nur \verb|\oldhyf| richtig als \verb|f| sortieren, nicht jedoch \verb|\hyf|).
Der Befehl macht aus \verb|Sto\hyf figur| unter Voreinstellung \verb|\sethyphenation{german}|, 
also alter deutscher Rechtschreibung: {\sethyphenation{german} `Sto\hyf figur'}; 
unter Voreinstellung \verb|\sethyphenation{ngerman}| wird dagegen {\sethyphenation{ngerman} `Sto\hyf figur'} 
ausgedruckt. Und \verb|bibsort| sortiert jeweils das, was gedruckt wird. Es war
n"otig, diesen Befehl in \BibArts\ einzuf"uhren, weil \verb|"ff| in Versionen von
\verb|ngerman.sty| `fff' ausdruckt, \label{ff} was \verb|bibsort| anhand von \verb|"ff| aber
nicht erkennen kann und jedenfalls stets als \verb|ff| sortiert. \verb|\hyf|~\verb|f| druckt
nach Einstellung von \verb|german| ein {\sethyphenation{german} `\hyf f'}, das am Zeilenende ff-f getrennt wird. 
Parallel zu \verb|german.sty| gibt es 
\verb|Scha\hyl leistung|, 
\verb|Sta\hym mutter|, 
\verb|Ke\hyn nummer|,
\verb|Ste\hyp pullover|, 
\verb|Sta\hyr rahmen| und 
\verb|Schri\hyt tempo|. 
Alle sollen ohne davorstehendes \verb|\protect|
verwendet werden. Aus Symmetriegr"unden existiert noch\hspace{.1em} \verb|Dru\hyc| \verb|ker| 
$\Rightarrow$ Dru\hyc ker, das sich bei Trennung in k- wandelt.

Zuletzt f"uhrt \BibArts~2.2 
den Befehl \verb|\newhyss| ein, das ein `Scharf\fhy S zwischen Vokalen' in neuer 
deutscher Rechtschreibung richtig trennt: In \textsc{small caps} trennt es s-s, 
und sonst, wie es der gew"ahlte Trennsatz f"ur Scharf\fhy S vorsieht (meist -\ss). 
\verb|\oldhyss| f"ur die alte deutsche Rechtschreibung trennt dagegen immer s-s. 
Verwendet werden soll \verb|\hyss|, \label{hyss}
das mit \verb|german|\hy Trennung \verb|\oldhyss| ausf"uhrt, sonst \verb|\newhyss|.
Der Befehl \verb|\hyss| ist wichtig, weil \BibArts\ Nachnamen in v- und k\fhy Befehlen
defaultm"a"sig in \textsc{small caps} setzt, in dem `\ss' nicht existiert, sondern \textsc{\ss}
gedruckt wird. Eine Trennung \verb|au"ser| $\Rightarrow$ \textsc{au"ser} f"ur ein `Scharf\fhy S zwischen Vokalen
in einer Schrift, in der das Scharf\fhy S als ss dargestellt wird', ist nicht dudengerecht, 
\verb|au\hyss er| trennt aber \textsc{au\hyss er} in neuer Rechtschreibung richtig;
in anderen Schrifts"atzen trennt es au\hyss er. An anderen Positionen sollten Sie Ihr Scharf\fhy S verwenden,
etwa \verb|da"s|. 

Der ganze Aufwand wird also nicht nur betrieben, um w"ortliche Zitate
aus der Zeit der alten deutschen Rechtschreibung original 
wiedergeben zu k"onnen: \verb|\hyss| wird auch f"ur die \texttt{ngerman}\hy Trennung ben"otigt.

Die auf dieser Seite eingef"uhrten \BibArts\hy
Befehle funktionieren auch im Argument von \verb|\MakeUppercase|. Die Worttennung
zwischen Silben an Positionen vor und nach den erw"ahnten Zeichen erfolgt ebenso,
wie sie auch mit \verb|"|\textit{x}\pdfko{.05}\ erfolgen w"urde. Es bleibt nur ein kleiner Sch"onheitsfehler:
Mit \textsc{small caps} speziell im \verb|ngerman|\hy Trennsatz zeigt \verb|\hyss| 
im Argument von \verb|\showhyphens| f"alschlicherweise als Bildschirmausdruck an, es 
werde -\ss\ getrennt (Anzeige z.\,B.: \texttt{au-\"yer}); tats"achlich wird im Preview und im Papierausdruck aber wie
gew"unscht vom dann ausgef"urten \verb|\newhyss| am Zeilenende \textsc{s-s} getrennt. 


\newpage
\section{Punkte, \,\texttt{\bs bahasdot} \,und \,\texttt{\bs banotdot}}\label{Sect10}\label{Punkte}

Die \BibArts\hy Befehle, die statt ihrer Argumente auch \textsc{ebd.} 
ausdrucken k"onnen,\pdfko{.5}\ 
d"urfen den Ausdruck eines unmittelbar nach ihnen 
getippten Punktes eigenst"andig unterbinden. Sonst w"urden 
am Satzende oft zwei Punkte gedruckt (\kern-.5pt\textsc{ebd.}.). 
Diese Befehle sind \verb|\kli|, \verb|\kqu|, \verb|\per| 
und \verb|\abkper|, sowie \verb|\arq|. 

\vspace{1ex}\noindent
In englischen Texten (unter \verb|\nonfrenchspacing|)\hspace{.1em} stellen
Punkte nach \BibArts\hy Befehlen zudem das richtige
\textit{spacing} ein: \hspace{.35em}\verb*|}. |\hspace{.25em} \textit{hinter} einem der oben aufgez"ahlten
Befehle markiert ein Satzende und verl"angert das 
\verb*| | entsprechend.\pdfko{.75}\
Auch\hspace{.1em} \verb*|\abk{X.X.X.}. |\hspace{.2em} bezeichnet ein Satzende und druckt 
\textit{immer} \hspace{.1em}\abk{X.X.X.}.

Auch im \verb|\frenchspacing| (dt.\ oder frz.\ Texte mit stets gleicher Leerzeichenl"ange) 
sollten Sie hinter \verb|\kli| und sogar unter \verb|\notprinthints| tippen:

\vspace{1ex}{\notprinthints\noindent\small
\verb| [deutsch/franz.:] \kli{Maier}{D.\,D.\,R.}. N =>| \printonlykli{Maier}{D.\,D.\,R.}. N \\
\verb| \nonfrenchspacing \kli{Maier}{D.\,D.\,R.}. N =>| {\nonfrenchspacing\printonlykli{Maier}{D.\,D.\,R.}. N}}

\vspace{1.25ex}\noindent
... denn nur dann k"onnen Sie sp"ater wieder auf \verb|\printhints| zur"uckschalten:\pdfko{1}\

\vspace{1ex}{\printhints\noindent\small
\verb| [deutsch/franz.:] \kli{Maier}{D.\,D.\,R.}. N =>| \printonlykli{Maier}{D.\,D.\,R.}. N \\
\verb| \nonfrenchspacing \kli{Maier}{D.\,D.\,R.}. N =>| {\nonfrenchspacing\printonlykli{Maier}{D.\,D.\,R.}. N}}

\vspace{1.75ex}\noindent
\BibArts\ durchsucht viele Argumente nach Punkten und verhindert ..\ 
eigenst"andig. Falls es doch ..\ druckt, `sieht' es den Punkt 
\textit{am Ende des Arguments}\pdfko{.5}\ 
nicht. Mit  \hspace{.2em}\verb|{|...\verb|\bahasdot}.| \hspace{-.1em}k"onnen
Sie befehlen, den Punkt zu `verschlucken'.

\vspace{1ex}\noindent
Diese Halbautomatisierung ist seit Version~2.2 auf Band- und Seitenangaben ausgedehnt. 
Weiterhin drucken\hspace{.1em} \verb|\f].| und\hspace{.15em} \verb|\sq].|\hspace{.1em} 
(f"ur \textit{folgende} bzw.\ \textit{sequentes}\ko) ein `f.'
unter {\small\verb|\bacaptionsgerman|} und (Beispiel) {\small\verb|\bacaptionsenglish|}:

\vspace{1.25ex}\noindent
{\small\bacaptionsenglish
\verb+  \per{ZfG.}[2~f.].                       => + \per{ZfG.}[2~f.]. \\
\verb+  \per{ZfG.}[2~f.\baplural].              => + \per{ZfG.}[2~f.\baplural]. \\
\verb+  \per{ZfG.}[2~f.\baplural\bahasdot].     => + \per{ZfG.}[2~f.\baplural\bahasdot]. \\
\verb+  \per{ZfG.}[2 \f].                       => + \per{ZfG.}[2 \f].
}

\vspace{1.75ex}\noindent
Ein Sonderfall: \verb|\bahasdot| darf \textit{nicht} 
nach \verb|!| oder \verb|?| ans Argumenten\hy Ende\pdfko{1}\ 
gesetzt werden, weil dann ggf.\ notwendige \textit{italics}\hy Korrekturen unterbleiben:

\vspace{1ex}\noindent
{\small
{\renewcommand{\kxxemph}{\itshape}\notprinthints \showbacorr 
 \verb|   \renewcommand{\kxxemph}{\itshape} \notprinthints \showbacorr| \\[.5ex]
 \verb+ (\kli{Kingsley}{Westward Ho!\bahasdot})  =>+ (\printonlykli{Kingsley}{Westward Ho!\bahasdot}) \\
 \verb+ (\kqu{Sienkiewicz}{Quo vadis?\bahasdot}) =>+ (\printonlykqu{Sienkiewicz}{Quo vadis?\bahasdot})
}}

\vspace{1.75ex}\noindent
Vielmehr ist es n"otig, nur das Setzen des nachfolgenden Punktes zu unterbinden,
die \textit{italics}\hy Korrektur aber zu belassen. \verb|\banotdot| ist zu verwenden:\pdfko{1}

\vspace{1.25ex}\noindent
{\small
{\renewcommand{\kxxemph}{\itshape}\showbacorr\notprinthints 
\verb+ (\kli{Kingsley}{Westward Ho!\banotdot})  =>+ (\printonlykli{Kingsley}{Westward Ho!\banotdot}) \\
\verb+ (\kqu{Sienkiewicz}{Quo vadis?\banotdot}) =>+ (\printonlykqu{Sienkiewicz}{Quo vadis?\banotdot})
}}

\vspace{1.75ex}\noindent
Damit \verb|bibsort| stets zeichengleiche Eintr"age bekommt, 
muss ein einmal begonnenes Setzen von \verb|\banotdot| beim 
jeweiligen Titel immer erfolgen:

\vspace{1ex}\noindent
{\footnotesize
{\renewcommand{\kxxemph}{\itshape}\notprinthints \showbacorr 
 \verb|  \renewcommand{\kxxemph}{\itshape} \notprinthints \showbacorr| \\[.25ex]
 \verb+ (\kli{Kingsley}{Westward Ho!\banotdot}.)  =>+ (\printonlykli{Kingsley}{Westward Ho!\banotdot}.) \\
 \verb+ \kli{Kingsley}{Westward Ho!\banotdot}[3]. =>+ \printonlykli{Kingsley}{Westward Ho!\banotdot}[3]. \\
 \verb+ \kqu{Sienkiewicz}{Quo vadis?\banotdot} in =>+ \printonlykqu{Sienkiewicz}{Quo vadis?\banotdot} in \\
 \verb+ \kqu{Sienkiewicz}{Quo vadis?\banotdot}|2| =>+ \printonlykqu{Sienkiewicz}{Quo vadis?\banotdot}|2|%
}}\label{kxxB}

\vspace{2.5ex}\noindent
Nicht automatisch bew"altigt wird \,\verb|!\banotdot}| \,vor \textit{mehreren} Punkten.
Zur\pdfko{1}\ 
L"osung dieses sehr seltenen Problems kann \verb|\strut| nach \verb|}| gesetzt
werden: \verb|\kli{N.}{XX!\banotdot}...| kann zur falschen \textit{italics}\hy Korrektur
{\renewcommand{\kxxemph}{\itshape}\notprinthints \showbacorr 
 \printonlykli{N.}{XX!\banotdot}...\pdfko{.5}\ 
f"uhren, w"ahrend 
\,\verb|\kli{N.}{XX!\banotdot}\strut...| \,zu\, 
\printonlykli{N.}{XX!\banotdot}\strut...} \,f"uhrt.\pdfko{.5}\ 
 
\vspace{2ex}\noindent
\fbox{\parbox{.98\textwidth}{\hfil\sffamily 
Einfacher ist es sicher, wenn Sie sich Kurztitel ohne Satzzeichen aussuchen.}}


\vspace{4ex}\noindent
Nach dem letzten Argument von \verb|\vli| oder \verb|\vqu| l"oscht 
\BibArts\ einen Punkt\pdfko{1}\  
\textit{im Text} \underline{nie} automatisch, 
weil es \textit{dort} das letzte Argument nicht durchsucht. 
Falls das letzte Argument eines v\fhy Befehls mit einem Punkt
enden sollte, ist die Verwendung von \verb|\bahasdot| in jeder
Sprache sinnvoll, denn nur dann ist ein sp"aterer Wechsel zwischen
\verb|\announcektit| und \verb|\notannouncektit|\pdfko{1.5}\  
m"oglich (die Ank"undigung der sp"ateren Kurzzitierweise). 

\Doppelbox
{
\bs notannouncektit
\\
Vers.\string~1: \bs vli\{Niklas\}\{Luhmann\} \{\bs ktit\{Soziale Systeme\}. Grundri"s
einer allgemeinen Theorie, 1984: Frankfurt/M.\}. \ \ \%\%~FALSCH
\\[1ex]
\% Nicht in die Listen umgesetzt:
\\[.25ex]
Vers.\string~2: \bs vli\{Niklas\}\{Luhmann\} \{\bs ktit\{Soziale Systeme\}. Grundri"s
einer allgemeinen Theorie, 1984: Frankfurt/M.\bs bahasdot\}. Das ...
}
{\vspace{1.1ex}
\notannouncektit
Vers.~1: \vli{Niklas}{Luhmann} {\ktit{Soziale Systeme}. Grundri"s
einer allgemeinen Theorie, 1984: Frankfurt/M.}. 
\\[2.375ex]
Vers.~2: \printonlyvli{Niklas}{Luhmann} {\ktit{Soziale Systeme}. Grundri"s
einer allgemeinen Theorie, 1984: Frankfurt/M.\bahasdot}. 
Das ist auch unter \texttt{\bs frenchspacing} besser!
}%
\label{v-Ausnahme}%

\vspace{.25ex}\noindent
Beim Drucken der \textit{Listen} wird im letzten Argument von v\fhy Befehlen 
aber nach 'Punkt' gesucht; Frankfurt/M.. ist so in \verb|\printvli|
und \verb|\printvqu|\pdfko{.5}\  
ausgeschlossen (solange nicht etwas wie \verb|.{}}| 
am Ende steht). Vers.\,2 ist in die Listen nicht umgesetzt,
um dort zwei Luhmann\hy Eintr"age zu vermeiden.

\vfill\noindent
Zusammengefasst gibt es eine Ausnahme bei v\fhy Befehlen,
nachdem der Schalter \verb|\notannouncektit| gesetzt wurde.
Falls Sie in deutschen Texten darauf (\texttt{ngerman.sty} setzt 
\verb|\frenchspacing|) und auf \verb|\notprinthints|
verzichten, brauchen Sie \verb|\banotdot| und \verb|\bahasdot| 
nicht unbedingt zu kennen. 


\newpage\noindent
F"ur die \textbf{Definition der Textelemente}, die f"ur den Ausdruck zwischen den
Argumenten von \BibArts\hy Befehlen vorgefertigt sind (`Separatoren'), 
dient der Befehl \verb|\bapoint| zum Drucken eines Punktes am Separatoren"-\textit{kopf}. 
\hspace{.1em}\verb|\bapoint| reagiert auf die Suche nach einem Punkt \textit{am Endes des 
Arguments davor}\pdfko{.25}\  
(bzw.\ auf Ihr \verb|\bahasdot| oder \verb|\banotdot|) und druckt dann 
\textit{keinen} Punkt. 

Falls Sie im Text in \verb|\arq| zwischen Schriftst"uck und Signatur 
einen Punkt statt ein Komma haben wollen, m"ussen Sie\hspace{-.1em} \verb|\arqsep| 
umdefinieren. Sie sollten nicht \verb*|{. }|\kern.1em\ zuweisen:\kern.2em\
\verb|\renewcommand{\arqsep}{\bapoint\newsentence}| reagiert automatisch
und druckt keinen Punkt, wenn die sp"atere Eingabe des Schriftst"ucks 
bereits selbst mit einem Punkt endet.\footnote{\balabel{arqsep}%
\texttt{\,\bs renewcommand\{\bs arqsep\}\{\bs bapoint\bs newsentence\} 
\ \% fuer beide spacings!}
\\[.75ex]
%\nonfrenchspacing  %%Testen Sie!%%
\renewcommand{\arqsep}{\bapoint\newsentence}
\strut\texttt{ \ \ \ \bs arq\{Gesellschaftsvertrag der KCAG\}\{BA} ... \texttt{ =>} \\[.5ex]
\arq{Gesellschaftsvertrag der KCAG} {BA Zwischenarchiv Dahlwitz\hy Hoppegarten R\,8729~4}. 
\\[1ex]
\strut\texttt{ \ \ \ \bs arq\{Test!\bs banotdot\}\{BA} ... \texttt{ =>} \\[.5ex]
\clearbamem\arq{Test!\banotdot} {BA Zwischenarchiv Dahlwitz\hy Hoppegarten R\,8729~4}. 
\\[1ex]
\strut\texttt{ \ \ \ \bs arq\{Abk.\}\{BA} ... \texttt{ =>} \\[.5ex] 
\clearbamem\arq{Abk.} {BA Zwischenarchiv Dahlwitz\hy Hoppegarten R\,8729~4}.} 

\balabel{nonum} Beim \textit{Ausdruck der Listen} wird \verb|\bapoint| am Ende jedes Listenpunkts
ausgef"uhrt von \verb|\printvli| und \verb|\printvqu| 
(durch \verb|\fromnopagevxxsep|) sowie von \verb|\printarq| 
(durch \verb|\fromnopagearqsep|) und von
\verb|\printper| (durch \verb|\fromnopagepersep|). Um die
einzelnen Listenpunkte in \verb|\printvkc| zu hinterpunkten, ist  
\verb|\renewcommand{\fromnopagevkcsep}{\bapoint}| zu befehlen;
bei \verb|\printabk| ist \verb|\fromnopageabksep| 
entsprechend umzudefinieren.


\vspace{1.75ex}\noindent
Obwohl \BibArts\ \textit{im Text} das letzte Argument des v\fhy Befehls nicht 
nach Punkten durchsucht, druckt ein dort ans Ende gesetztes 
\verb|\ersch{|\textit{Ort}\kern.1em\verb|}{}| mit leerer\pdfko{1}\ 
Jahresangabe "`\ko\ersch{\textit{Ort}}{}"' mit \textit{einem} Punkt aus. Sie k"onnen am 
Satzende also\pdfko{.4}\  
intuitiv vorgehen und den \underline{Punkt\kern-1.5pt} einfach 
hinter die Literaturangabe setzen:

\vspace{.75ex}\noindent
{\small
 \verb|   \vli{}{}{Titel, \ersch{Bonn}{}}|\underline{\texttt{.}}\verb|  => | 
 \printonlyvli{}{}{Titel, \ersch{Bonn}{}}.}

\vspace{.75ex}\noindent
Nicht gedruckt wird \oJ. deshalb, weil das leere \verb|\ersch|\hy Argument 
\verb|\oJ| ausf"uhrt, das seinerseits ganz am 
Ende \verb|\bahasdot| setzt. Da \verb|\oD|, \verb|\oO| und \verb|\oJ|\pdfko{1.25}\
zun"achst \verb|\protect|\hy gesch"utzt \verb|\poD|, \verb|\poO| und \verb|\poJ|
ausf"uhren, sollte \textit{an diesen} eine \label{poJ}
Umdefinition von \oD, \oO\ und \oJ\ ansetzen
(ggf.\ mit \verb|\bahasdot|\pdfko{.75}\ 
am Ende). \verb|\ersch| verwendet \verb|\oO| 
und \verb|\oJ| nur in deutschen Texten; deren\pdfko{1}\  
Umdefinition "andert \verb|\ersch| nur unter 
\verb|\bacaptionsgerman| (vgl.\ S.\,\pageref{SprachSep}, \pageref{gerschnoyearname}).
\verb|\ersch| dient zur Verwendung ganz am Ende des letzten v\fhy Arguments. 

Die bereits erw"ahnten Befehle \verb|\f| und \verb|\sq| setzen
\verb|\bahasdot| ebenfalls.\pdfko{.75}\ 
\textit{Beide} f"uhren von
der Spracheinstellung abh"angig entweder \verb|\gfolpagename|\pdfko{1.25}\ 
oder \verb|\efolpagename| oder \verb|\ffolpagename| aus und 
drucken f.\ im Deutschen\pdfko{1}\ 
und Englischen, aber sq.\ im Franz"osischen.
Es gibt auch \verb|\ff| (und \verb|\sqq|). 


\newpage\noindent
Falls Sie etwa eine\hspace{-.1em} \verb|\onlyout|\hy Konstruktion im 
Argument eines\hspace{-.1em} \verb|\kli|\hy Befehls \textit{ganz ans 
Ende} setzen, findet \BibArts\ u.\,U. irrt"umlich den `falschen' 
Punkt. Im folgenden Beispiel wird \verb|{}| in einem solchen Fall 
genutzt, um den am Ende von \verb|o.O.| stehenden Punkt `hier' 
auszublenden.\footnote{Am Argumenten\hy Ende ist `\texttt{\}}' 
erlaubt:\hspace{.2em} \texttt{\bs abk\{\bs protect\bs underline\{Abk.\}\}. =>}\hspace{.3em} 
\printonlyabk{\protect\underline{Abk.}}.} Damit der Punkt 
trotz \verb|{}| aber in der Liste `gesehen' wird, steht 
\verb|\bahasdot| am Ende von \verb|\onlyout|:


\vspace{1.25ex}\noindent
{\small
\verb|  \notprinthints| \notprinthints \\[.25ex]
\verb|  \kli{}{Buch\onlyout{ o.O.\bahasdot}{}}.      | \verb| =>|
\printonlykli{}{Buch\onlyout{ o.O.\bahasdot}{}}.
\\[.25ex]
\verb|  \kli{}{A.{\onlyhere{\bahasdot}}\onlyout{ N}}.| \verb| =>|
\printonlykli{}{A.{\onlyhere{\bahasdot}}\onlyout{ N}}.
}

\vspace{1.75ex}\noindent
Unter {\small\verb|\notprinthints| \verb|\renewcommand{\fromnopagevkcsep}{\bapoint}|} \balabel{fromnopagevkcsep2}
w"urde dann auch {\small\verb|\printvkc|} korrekt mit \textit{einem} Punkt je Eintrag ausgedruckt.


\vspace{2.5ex}\noindent
Eine Nachbesserung in Version~2.2 von \BibArts\ ist, dass nun auch die Namensargumente
von v- und k\fhy Befehlen daraufhin gepr"uft werden, ob Punkte am Ende stehen.
Auch die optionalen Sternargumente werden jetzt durchsucht.
Damit ist m"oglich, \verb|\bapoint| in die Definition des Separators \verb|\ntsep|
aufzunehmen, ohne z.\,B. nach abgek"urzten Namen \verb|\bahasdot| setzen zu m"ussen:

\Doppelbox
{
\vspace{.5ex}
\bs renewcommand\{\bs ntsep\}\{\bs bapoint\bs\ \} \\[1.75ex]
\bs vli\{N\}\{N\} \\ 
\ \ \ \{Das \bs ktit\{Buch\}, London\} \\[1.75ex]
\bs vli\{N.\}\{N.\} \\
\ \ \ \{Das \bs ktit\{Buch\}, London\} \\[1.75ex]
\bs vli\{N\}\{N\}*\{Hg\} \\
\ \ \ \{Das \bs ktit\{Buch\}, London\} \\[1.75ex]
\bs xvli\{N.\}\{N.\} \\
\ \ \ *\{\bs vauthor\{M.\}\{M.\}\} \\
\ \ \ \{Das \bs ktit\{Buch\}, London\} 
\vspace{.5ex}
}
{
\vspace{2.5ex}
\renewcommand{\ntsep}{\bapoint\ }\notktitaddtok
\printonlyvli{N}{N}{Das \ktit{Buch}, London} \\[.9ex]
\printonlyvli{N.}{N.}{Das \ktit{Buch}, London} \\[.9ex]
\printonlyvli{N}{N}*{Hg}{Das \ktit{Buch}, London} \\[.9ex]
\xprintonlyvli{N.}{N.}*{\vauthor{M.}{M.}}{Das \ktit{Buch}, London} 
}

\vspace{.5ex}\noindent
F"ur die hier in die Listen umgesetzten Titel ergibt dann
die Anwendung von\hspace{.2em}
{\footnotesize\verb|{\renewcommand{\ntsep}{\bapoint\newsentence} \nonfrenchspacing \printvli}|}\hspace{.2em}:

\vspace{-1ex}
{\renewcommand{\ntsep}{\bapoint\newsentence} \nonfrenchspacing \printvli}


%%%>>>>>

\newpage
\section{\textit{Italics}\hy Korrekturen und Separatoren}%
\label{schraegkap}\label{Sect11}

\textit{Italics}\hy Korrekturen sind bei den \BibArts\hy Hauptbefehlen 
gelegentlich n"otig an $\bullet$~ihrem Kopf, \hspace{.25em} $\bullet$~ihrem Ende,
\hspace{.25em} $\bullet$~den Separatoren zwischen den Argumenten. 
Die Stellen dieser\hspace{.1em} \textit{automatischen Korrekturen}\hspace{.1em}
macht \verb|\showbacorr| sichtbar.

\vspace{1.5ex}\noindent
Da v-, k-, per- und arq\hy Befehle am Kopf stets
in aufrechte Schrift umschalten, ist \textit{in schr"aggestelltem
Umfeld} ggf.\ eine Korrektur n"otig. Nach \verb*| |\hspace{.05em} 
oder am Absatzanfang wird diese \textbf{Kopf"|korrektur} 
nicht durchgef"uhrt.\footnote{Die Kopf"|korrektur ist \texttt{\bs/}\hspace{.2em}
(f"ur andere \textit{italics}\hy Korrekturen setzt \BibArts\ \texttt{\bs kern 0.1em}).} 


\vspace{1.5ex}\noindent\hspace{.5em}
{\small\itshape\showbacorr
\begin{tabular}{ll}
& \verb|   \itshape \showbacorr| \\[.25ex]
(\printonlyvli{}{}{Rest}) & \verb|(\vli{}{}{Rest})| \\
(\printonlyvli{}{Nachname}{Rest}) & \verb|(\vli{}{Nachname}{Rest})| \\
(\printonlyvli{Vorname}{Nachname}{Rest}) & \verb|(\vli{Vorname}{Nachname}{Rest})| \\
(\printonlykli{}{Kurztitel}) & \verb|(\kli{}{Kurztitel})| \\
(\printonlykli{Nachname}{Kurztitel}) & \verb|(\kli{Nachname}{Kurztitel})| \\
(\printonlyper{Zeitschrift}) & \verb|(\per{Zeitschrift})| \\
(\printonlyarq{}{Signatur}) & \verb|(\arq{}{Signatur})| \\
(\printonlyarq{Dokument}{Signatur}) & \verb|(\arq{Dokument}{Signatur})| \\
(\printonlyabkdef{Initialen}{Erkl"arung}) & \verb|(\abkdef{Initialen}{Erkl|{\upshape\texttt{"a}}\verb|rung})| \\
(\printonlyabk{Initialen}) & \verb|(\abk{Initialen})| \\[.25ex]
& \verb|   \renewcommand{\abkemph}{\upshape}| \\[.125ex]
 \renewcommand{\abkemph}{\upshape}%
(\printonlyabkdef{Initialen}{Erkl"arung}) & \verb|(\abkdef{Initialen}{Erkl|{\upshape\texttt{"a}}\verb|rung})| \\
 \renewcommand{\abkemph}{\upshape}%
(\printonlyabk{Initialen}) & \verb|(\abk{Initialen})| \\
\end{tabular}}\label{abkA}

\vspace{1.75ex}\noindent
\verb|\vauthor|, \verb|\midvauthor|, \verb|\kauthor| und 
\verb|\mitkauthor| korrigieren ebenfalls bei Bedarf `am Kopf' automatisch. 

Die Kopfkorrektur wird praktisch immer ben"otigt. Trotzdem existiert der
Befehl \verb|\notbafrontcorr|, um sie auszuschalten:

\vspace{-.5ex}
\Doppelbox
{
\bs itshape \bs showbacorr
\\[.1ex] \ \ (\bs arq\{Aktenst"uck\}\{Signatur\})
\\[.5ex] \{\bs notbafrontcorr 
\\ \ \ (\bs arq\{Aktenst"uck\}\{Signatur\})\}
}
{\vspace{2.6ex}
\itshape \showbacorr
   (\printonlyarq{Aktenst"uck}{Signatur})
\\[2.7ex] {\notbafrontcorr (\printonlyarq{Aktenst"uck}{Signatur})}
}


\vspace{2.25ex}\noindent
\textbf{Endkorrekturen} nach \verb|\abk|, \verb|\kli| und \verb|\kqu|
lassen sich nicht abschalten. Unter \verb|\renewcommand{\abkemph}{\itshape}|
korrigiert \verb|\abk| in aufrechtem Umfeld, falls das Argument 
\textit{nicht} mit `\texttt{.}' endet, oder kein `\texttt{.}' oder  
`\texttt{,}' folgt:

\vspace{1ex}\noindent
{\small\verb|          {\showbacorr \abk{GmbH},   \abk{GmbH}!}   =>|} 
{\renewcommand{\abkemph}{\itshape}\showbacorr \abk{GmbH}, \abk{GmbH}!}
\\
{\small\verb|          {\showbacorr \abk{e.\,V.}, \abk{e.\,V.}!} =>|} 
{\renewcommand{\abkemph}{\itshape}\showbacorr \abk{e.\,V.}, \abk{e.\,V.}!}



\vspace{1.25ex}\noindent
Dasselbe gilt f"ur \verb|\renewcommand{\kxxemph}{\itshape}|, \label{kxxA}
mit dem der Kurztitel in \verb|\kli| und \verb|\kqu| \textit{kursiv} 
gesetzt wird.\footnote{\texttt{\bs kxxemph} wirkt sich au"serdem noch auf die 
Vorank"undigung der Kurzzitate in den v\fhy Befehlen aus;
\renewcommand{\kxxemph}{\bfseries\itshape}\showbacorr
\texttt{\bs renewcommand\{\bs kxxemph\}\{\bs bfseries\bs itshape\}} \texttt{\bs showbacorr}
bewirkt:\pdfko{.5}\ \vli{Niklas} {Luhmann}{\ktit{Soziale Systeme}. Grundri"s 
einer allgemeinen Theorie, 1984: Frankfurt/M.}[123].} Im Fall von 
\verb|\notprinthints|, das den Ausdruck von {\small [L]} und {\small [Q]} 
unterbindet, wird automatisch korrigiert:

\vspace{.75ex}\noindent
{\small\verb|          {\showbacorr \kli{N}{K},   \kli{N}{K}!}   =>|}
{\renewcommand{\kxxemph}{\itshape}\notprinthints\showbacorr\printonlykli{N}{K}, \printonlykli{N}{K}!}
\\
{\small\verb|          {\showbacorr \kli{N}{K.},  \kli{N}{K.}!}  =>|}
{\renewcommand{\kxxemph}{\itshape}\notprinthints\showbacorr\printonlykli{N}{K.}, \printonlykli{N}{K.}!}

\vspace{1.25ex}\noindent
Auch im schr"aggestellten Umfeld verhalten sich beide Befehle weiter richtig:

\vspace{.85ex}\noindent
{\small\verb|  {\itshape\showbacorr \abk{GmbH},   \abk{GmbH}!}   =>|} 
{\itshape\renewcommand{\abkemph}{\itshape}\showbacorr\abk{GmbH}, \abk{GmbH}!}\\
{\small\verb|  {\itshape\showbacorr \kli{N}{K},   \kli{N}{K}!}   =>|}
{\itshape\renewcommand{\kxxemph}{\itshape}\notprinthints\showbacorr\printonlykli{N}{K}, \printonlykli{N}{K}!}


\vspace{4ex}\noindent
In den vorgefertigten Textelementen (\textbf{Separatoren}), 
die \BibArts\ zwischen den Argumenten ausdruckt, sind ebenfalls oft 
\textit{italics}\hy Korrekturen n"otig.\footnote{Separatoren sollen 
\textit{nur} vorgefertigten Text und \BibArts\hy\textit{italics}\hy Korrekturen enthalten;
\textsc{Schriftumschaltbefehle} werden in die \BibArts\hy \texttt{\bs}...\texttt{emph}\hy Befehle gesetzt
\baref{hervor}!}
Auch diese Korrekturen arbeiten in Voreinstellung automatisch. 

Zur \textit{Ver"anderung von Separatoren} l"asst sich \verb|\renewcommand| 
verwenden.\pdfko{.75}\ 
Danach sollten die Korrekturen weiterhin automatisch arbeiten. 
\textit{Italics}\hy Korrekturen sind n"otig,
wenn schr"age auf aufrechte Schriften treffen k"onnten und der 
Separator (\underline{\underline{unteres Beispiel}}) nicht nur aus `niederen' Zeichen besteht:

\vspace{-.5ex}
\Doppelbox
{
\bs itshape\bs showbacorr
\\[1ex]
\bs renewcommand\{\bs nsep\}\{\underline{,\vphantom{p} }\}   
\\[1.25ex] ... ... \bs xkqu\{Ehlert\} 
\\ \ \string*\b{\{}\bs midkauthor\{Epkenhans\} 
\\ \ \ \ \bs kauthor\{Gro"s\} [Hrsg.]\b{\}}
\\ \ \{Schlieffenplan\}[468].
\\[1.75ex] 
\bs renewcommand\{\bs nsep\}\{\underline{\underline{/\bs baupcorr}}\}
\\[.75ex] ... ... \bs xkqu\{Ehlert\} 
\\ \ \string*\{\bs midkauthor\{Epkenhans\} 
\\ \ \ \ \bs kauthor\{Gro"s\} [Hrsg.]\}
\\ \ \{Schlieffenplan\}[469].
}
{\vspace{2.5ex}
\itshape\showbacorr\noindent
\renewcommand{\nsep}{\protect\underline{, }}    %% nach Komma keine Korrektur %%
... ... \xkqu{Ehlert} *{\midkauthor{Epkenhans} \kauthor{Gro"s} [Hrsg.]}
   {Schlieffenplan}[468]. 

\vspace{.5cm}
\textup{\texttt{\bs nsep}} steht zwischen Namen in 
ggf.\ kursiver Umfeldschrift, w"ahrend die 
Namen aufrecht sind.
Enth"alt \textup{\texttt{\bs nsep}} `hohe' Zeichen,
sollte eine automatische italics\hy Korrektur also \emph{ans Ende}:

\vspace{.25cm}
\renewcommand{\nsep}{\protect\underline{\protect\underline{/\baupcorr}}}    %% nach + schon %%
... ... \xkqu{Ehlert} *{\midkauthor{Epkenhans} \kauthor{Gro"s} [Hrsg.]}
   {Schlieffenplan}[469].
}%
\label{MEhlert}\label{nsep2}%


\vspace{.25ex}\noindent
\verb|\baupcorr| korrigiert \textit{immer} dann, wenn es in
schr"aggestellten Umfeld steht. F"ur den \textit{Kopf} einer
Separator\hy Definition ist es damit ungeeignet, denn es hat keine Information
dar"uber, ob Sie in das \textit{davorstehende Argument des\pdfko{1}\  
\BibArts\hy Befehls} bei der Anwendung Text tippten, der mit einem Punkt endet. 


\newpage \BibArts\ durchsucht etliche Argumente deshalb
und stellt f"ur den folgenden Separator eine \textit{italics}\hy Korrektur bereit.
Dieser Korrektur\hy Befehl hei"st seit Version 2.2 einheitlich 
\hspace{-.1em}\verb|\bacorr|.\hspace{.1em} Er tut nichts, falls das vorausgehende Argument
mit einem Punkt endet. F"ur num\hy Listenausdrucke etwa steht 
\verb|\bacorr| am Kopf des Separators \verb|\frompagesep|, \label{frompagesep} 
um ggf.\ vor den stets aufrechten Indexzahlen zu korrigieren. 
(Das alte \verb|\balistcorr| speziell f"ur diesen Separator ist weiterhin Alternative.)
Das Argument von \verb|\frompagesep| und die Indexzahlen druckt \BibArts\ in 
\verb|\balistnumemph| aus (voreingestellt
\verb|\sffamily|):\footnote{\label{listnum}In \texttt{\bs balistnumemph} eingesetzte Schr"agschrift\hy Befehle 
ignoriert \BibArts\ einfach, denn vor den `Fu"snoten'\hy Exponenten in
num\hy Listen soll auf \textit{italics}\hy Korrekturen verzichtet werden.}


\vspace{.25ex}
{\small
\begin{verbatim}
  {\renewcommand{\frompagesep}{\bacorr ; }   % ; HOCH mit Korrektur
   \itshape \showbacorr \printnumvlilist }
\end{verbatim}}

\vspace{-2.5ex}
  {\renewcommand{\frompagesep}{\bacorr ; }   % ; HOCH mit Korrektur
    \itshape \showbacorr \baonecolitemdefs\printnumvlilist }\label{frompagesep1}%


\vspace{.5ex}\noindent
\verb|\bacorr| steht fast immer am Anfang des Arguments eines Separators. 
Ein \textbf{Sonderfall} ist \verb|\ntsep|, den \BibArts\ in Umfeldschrift 
zwischen Name und Titel druckt: Dort muss \verb|\bacorr| (falls verwendet) 
\textit{am Ende des Arguments} stehen: Eine Korrektur ist in \verb|\ntsep| 
\textit{nach} einem hohen Zeichen wie `:' dann n"otig, falls Sie eine
schr"aggestellte Umfeldschrift und f"ur den \textit{k\hy Titel} eine aufrechte 
Schrift einstellen. (\verb|\bacorr| "andert sich also nicht in Abh"angigkeit 
davon, ob das \textit{vorausgehende} Argument mit einem Punkt endet.) \verb|\ntsep| 
f"uhrt defaultm"a"sig\hspace{.2em} \verb*|{: \bacorr}|\hspace{.3em} \label{ntsepB} 
aus, wobei dieses \verb|\bacorr| nur in \textit{k\fhy Befehlen} etwas tut: 

\vspace{-.5ex}
\Doppelbox
{
   \bs renewcommand\{\bs kxxemph\}\{\bs upshape\}
\\ \bs showbacorr \bs itshape
\\[.4ex] \bs kli\{Ferguson\}\{Falscher Krieg\}
}
{
\vspace{5.5ex}
\renewcommand{\kxxemph}{\upshape}
\showbacorr \itshape
\kli{Ferguson}{Falscher Krieg}
}


\vfill\noindent
In \verb|\ntsep| k"onnen Sie den alten Befehl \verb|\bakntsepcorr| statt \verb|\bacorr| weiterhin verwenden.
Seit \BibArts~2.2 tun beide in \verb|\ntsep| nichts, falls sie in einem v\fhy Befehl stehen.
Damit ist diesbez"uglich nun "uberfl"ussig, die fr"uher n"otige Fallunterscheidung mittels \verb|\ifbashortcite|  zu 
definieren.\footnote{In v\fhy Befehlen werden \texttt{\bs ntsep} \textit{und} das folgende Argument in Umfeldschrift
gedruckt. Korrekt ist 
\texttt{\bs renewcommand\{\bs ntsep\}\{: \bs ifbashortcite\{\bs bakntsepcorr\}\{\}\}} weiterhin.} 
Doch k"onnen Sie mit
\hspace{-.05em} \verb|\ifbashortcite|\hspace{.4em}\verb|{|\textit{Text k\fhy Befehl}\hspace{.2em}\verb|}|\hspace{.4em}%
\verb|{|\textit{Text v\fhy Befehl}\hspace{.2em}\verb|}|\hspace{.4em} 
im Argument von \verb|\ntsep| weiterhin Textunterschiede f"ur k- und v\fhy Befehle festlegen.\footnote{\label{ntsepC}\texttt{%
\bs renewcommand}\hskip 1pt plus 2pt\texttt{\{\bs ntsep\}}\hskip 1pt plus 2pt\texttt{\{\bs ifbashortcite\{:\bs hskip0pt plus 0pt\bs bacorr\}\{: \}\}}
w"urde im v\hy Befehl \textit{ein Leerzeichen} nach dem `:' drucken, im k\fhy Befehl nicht.}


\newpage
\noindent
Solche \textbf{if\hy Befehle} haben zwei Argumente; \BibArts\ setzt das erste
bei Ja und das zweite bei Nein um (das jeweils unzutreffende Argument wird ignoriert):

\vspace{.875ex}{\small\noindent
\verb|  \ifbashortcite  | \verb|{|\textit{falls k-Befehl}\verb|}| \verb|{|\textit{sonst (falls v-, per- oder arq-Befehl)}\verb|}| \\
\verb|  \ifbaperiodical | \verb|{|\textit{falls per-Befehl}\verb|}| \verb|{|\textit{sonst (falls v-, k- oder arq-Befehl)}\verb|}| \\
\verb|  \ifbaprinthints | \verb|{|\textit{unter Voreinstellung}\verb|}| \verb|{|\textit{falls }\verb|\notprinthints|\textit{ gilt}\verb|}| \\
\verb|  \ifbaibidem     | \verb|{|\textit{falls }\textsc{ebd.}\textit{ (nach  Ausdruck von }\textsc{ebd.}\textit{ bereit)}\verb|}| \verb|{|\textit{sonst}\verb|}| \\
\verb|  \ifbahaspervol  | \verb|{|\textit{hat }\verb+|n|+\textit{ oder }\verb+_n_+\textit{ (ab }\verb|\gisonfolioname|\textit{ bereit)}\verb|}| \verb|{|\textit{sonst}\verb|}| \\
\verb|  \ifbahasdot     | \verb|{|\textit{am Separatorenkopf: falls Arg davor mit} 
 \hspace{-.15em}\verb|.| \hspace{-.1em}\textit{endet}\verb|}| \verb|{|\textit{sonst}\verb|}|}


\vspace{1.4ex}\noindent
Die wichtigsten Separatoren, f"ur die \BibArts\ \verb|\bacorr| bereitstellt, sind: 

\vspace{.75ex}\noindent
{\small 
\strut\kern 4mm$\bullet$\verb| \frompagesep      |\kern5mm(vgl.\ oben S.\,\pageref{frompagesep}; alternativ das alte \verb|\balistcorr|) \\
\strut\kern 4mm$\bullet$\verb| \ntsep            |\kern5mm(vgl.\ oben S.\,\pageref{ntsepA}, \pageref{ntsepB}; 
  alternativ \verb|\bakntsepcorr|) \\
\strut\kern 4mm$\bullet$\verb| \pagefolioshortsep| \hfill (alternativ \verb|\bakxxcorr|)\kern2mm\strut \\
\strut\kern 4mm$\bullet$\verb| \abkdefopen|, \verb|\defabkopen|, \verb|\defabkclose| \hfill (alternativ \verb|\baabkcorr|)\kern2mm\strut} 


\vspace{1.5ex}\noindent 
Die Definition von \verb|\pagefolioshortsep| lautet nur noch 
\verb|{\bacorr : }| in Version~2.2: \verb|\bacorr| wird 
nun ohne \verb|\ifba|...\hy Befehle eigenst"andig nach 
\verb|\kli| und nach \verb|\kqu| ausgef"uhrt 
(unter \verb|\notprintlongpagefolio|;\footnote{\texttt{\scriptsize\bs notprintlongpagefolio 
\bs renewcommand\{\bs kxxemph\}\{\bs itshape\} \bs notprinthints \bs showbacorr} 
\\ \hspace*{1cm}\texttt{\scriptsize
\bs kli\{Luhmann\}\{Soziale Systeme\}[23].  => } 
\clearbamem\notprintlongpagefolio 
\renewcommand{\kxxemph}{\itshape}\notprinthints \showbacorr 
\kli{Luhmann}{Soziale Systeme}[23].} 
vgl.\ S.\,\pageref{notprintlongpagefolio}).


\vspace{1.5ex}\noindent
Die \textbf{Definitionen von Abk"urzungen} werden in Klammern ausgedruckt. 
Die Separatoren \verb|\abkdefopen|, \verb|\defabkopen| und 
\verb|\defabkclose| enthalten je ein Klammer\hy Zeichen \textit{und} \verb|\bacorr|. 
In \verb|\abkdefopen| und \verb|\defabkclose| steht \verb|\bacorr| vor der Klammer, in 
\verb|\defabkopen| dahinter. Stehen \verb|\abkdef| oder \verb|\defabk| 
(oben S.\,\pageref{defabk}) in aufrechtem Umfeld, f"uhrt \verb|\bacorr| n"otige 
\textit{italics}\hy Korrekturen aus, falls 
\verb|\abkemph| eine schr"aggestellte Schrift einstellt:\footnote{Oder 
falls \texttt{\bs abkemph} eine aufrechte Schrift in schr"aggestelltem Umfeld einstellt.}

\vspace{-.5ex}
\Doppelbox
{\vspace{.75ex}
\bs renewcommand\{\bs abkemph\}\{\bs itshape\}
\\[.25ex] \ \ \bs showbacorr
\\[.5ex] \b{\{}\bs renewcommand\{\bs abkdefopen\}
\\[.1ex] \ \ \ \ \ \ \{\bs bacorr\bs\ [\kern.1em\} 
\\[.5ex] \ \bs renewcommand\{\bs abkdefclose\}\{\kern.1em]\}
\\[.5ex] \ \underline{\vphantom{g}\bs abkdef\{OHG\}\{Offene} 
\\ \ \ \ \ \ \underline{Handelsgesellschaft\}}\b{\}} \ u.\bs
\\[1ex]
 \b{\b{\{}}\bs renewcommand\{\bs defabkopen\}
\\ \ \ \ \ \ \ \{\bs bastrut\bs\ \string"\string<\bs bacorr\}\%
\\[.5ex] \ \bs renewcommand\{\bs defabkclose\}
\\ \ \ \ \ \ \ \{\bs bacorr \string"\string>\}\%
\\[.5ex] \ \underline{\vphantom{g}\bs defabk\{Offene} 
\\ \ \ \ \ \ \underline{Handelsgesellschaft\}\{OHG\}}.\b{\b{\}}}
}
{\vspace{1ex}
 \fbox{\parbox{.95\textwidth}{\sffamily Der alte Befehl \texttt{\bs baabkcorr} 
          funktioniert weiter (alternativ zu \texttt{\bs bacorr}).}}
 \\[5.5ex]
   \small\renewcommand{\abkemph}{\itshape} \showbacorr 
 {\renewcommand{\abkdefopen} {\bacorr\ [}%
  \renewcommand{\abkdefclose}{]}%
   \abkdef{OHG}{Offene Handelsgesellschaft}}  u.\
        \\[.25ex]
 {\renewcommand{\defabkopen}{\bastrut\ "<\bacorr}%
  \renewcommand{\defabkclose}{\bacorr ">}%
   \defabk{Offene Handelsgesellschaft}{OHG}.}
 \\[3.5ex]
 \fbox{\parbox{.95\textwidth}{\sffamily \texttt{\bs bastrut\bs\ }\,stellt sicher, 
         dass Zeilenumbr"uche am \texttt{\bs\ }\,stattfinden k"onnen.}}
}\balabel{abkdefopen}\balabel{defabkopen}%


\noindent
Wird statt \verb|{OHG}| alternativ \verb|{e.\,V.}| eingesetzt, unterbleibt die 
Korrektur:

\vspace{.5ex}
{\small\renewcommand{\abkemph}{\itshape}\showbacorr 
{\renewcommand{\abkdefopen} {\bacorr\ [}%
 \renewcommand{\abkdefclose}{]}%
  \abkdef{e.\,V.}{eingetragener Verein}} und
{\renewcommand{\defabkopen}{\bastrut\ "<\bacorr}%
 \renewcommand{\defabkclose}{\bacorr ">}%
  \defabk{eingetragener Verein}{e.\,V.}.}
}



\vspace{1ex}\noindent
In Voreinstellung macht \BibArts\ nach \verb|\defabkopen| (hier nach\hspace{-.1em} "<\kern.15em) 
einen in \LaTeX2e von \verb|\itshape| an dieser Stelle ausgedruckten horizontalen 
Abstand r"uckg"angig. Diese Gegenkorrektur l"asst sich im Dokumentenvorspann mit 
\verb|\notnegcorrdefabk| ausschalten (\LaTeX\,2.09 macht die Korrektur nicht).  


\vspace{1ex}\noindent
Das zweite Argument von \verb|\defabk| ist die
Abk"urzung, die in einstellbarer\pdfko{.5}\ 
Schrift gedruckt wird. 
\BibArts\ reagiert auf das \textit{folgende} Zeichen, z.\,B. ein~`!':


\Doppelbox
{\vspace{.25em}
    \bs renewcommand\{\bs abkemph\}\{\bs em\} 
 \\[.5ex] \bs showbacorr
 \\[.5ex] \bs renewcommand\{\bs defabkopen\}
 \\ \ \ \ \ \ \ \underline{\{\bs}ifbahasdot\{\bs bastrut\bs\ \}
 \\ \ \ \ \ \ \ \ \ \ \ \ \ \ \ \ \ \ \ \{ \bs bacorr\underline{\}\}}
 \\[.5ex] \bs renewcommand\{\bs defabkclose\}\{\}
 \\ \ \ \ \%\%=\{\bs defabkclose\}\{\bs bacorr\}
\\[2.5ex]
 Ein \bs defabk\{eingetragener 
\\ \ \ Verein\} \{e.\bs,V.\}!
\\[1.5ex]
 \bs defabk \{Offene 
\\ \ \ Handelsgesellschaft\}\{OHG\}!
\\[1.5ex]
   \bs itshape 
\\[.5ex] Noch ein \bs defabk\{eingetragener 
\\ \ \ Verein\}\{e.\bs,V.\}!
\\[1.5ex]
 \bs defabk\{Offene Handelsges.\}
\\ \ \ \{OHG\}!
\vspace{.25em}
}
{
 \vspace{19.4ex}
 \renewcommand{\abkemph}{\em} 
 \showbacorr
 \renewcommand{\defabkopen}
  {\ifbahasdot{\bastrut\ }
           { \bacorr}}
 \renewcommand{\defabkclose}{}   %=%{\defabkclose}{\bacorr}

 Ein \defabk {eingetragener Verein} {e.\,V.}!
 \\[3.5ex]
 \defabk {Offene Handelsgesellschaft}{OHG}!
 \\[6.6ex]
   \itshape 
 Noch ein \defabk{eingetragener Verein}{e.\,V.}!
 \\[3.6ex]
 \printonlydefabk{Offene Handelsges.} {OHG}!
}\label{abkB}


\vspace{-.25ex}\noindent
\verb|\ifbahasdot| ist dort wegen korrekter Zeilenumbr"uche n"otig (siehe S.\,\pageref{Systematik}).

\vspace{3ex}\noindent
In den Beispielen oben wurden die Klammersymbole lokal angepasst.
Dies wirkt sich \textit{nicht} auf das \textbf{Abk"urzungsverzeichnis} 
aus, denn f"ur dessen Ausdruck gelten eigene 
Separatoren: \verb|\abklistopen| und \verb|\abklistclose| 
legen fest, was dort vor und nach der \textit{Erkl"arung}
stehen soll. Im Abk"urzungsverzeichnis 
steht die Abk"urzung immer links und die Erkl"arung immer rechts.

\verb|bibarts.sty| 
legt f"ur \verb|\abklistopen| zun"achst \verb|{\protect\pabklo}| fest; 
das f"uhrt \verb|{\bastrut\hskip 1.2em minus 0.3em\bacorr}| aus. 
"Aquivalent h"angen \verb|\abklistclose|
und \verb|\pabklc| zusammen (das nichts tut: \verb|{}|). 

Im Beispiel werden die Listenseparatoren \verb|\pabklo| oder
\verb|\pabklc| direkt vor dem Befehl zum Ausdruck des Abk"urzungsverzeichnisses
lokal umdefiniert. Dies wirkt sich aus auf alle Eintr"age,
an deren Stellen im Text die Voreinstellungen f"ur
\verb|\abklistopen| und \verb|\abklistclose| nicht ver"andert wurden. 
Im Text ver"andert wurden die Listenseparatoren f"ur die Abk"urzung `S';
dazu gleich. Zuerst der Ausdruck des gesamten Abk"urzungsverzeichnisses:

\vspace{.5ex}{\small
\begin{verbatim}
  {\renewcommand{\pabklo}{\bastrut\ \bacorr =\ }   %% \bastrut\  er-
   \renewcommand{\pabklc}{!\banotdot}              %% laubt Umbruch
   \renewcommand{\abklistemph}{\itshape\bfseries}
   \showbacorr \small \printnumabklist}
\end{verbatim}}

\vspace{-.5ex}
  {\renewcommand{\pabklo}{\bastrut\ \bacorr =\ }
   \renewcommand{\pabklc}{!\banotdot}
   \renewcommand{\abklistemph}{\itshape\bfseries}
   \showbacorr \small \batwocolitemdefs\printnumabklist}


\vspace{1ex}\noindent
\verb|\printnumabk| h"atte einen \verb|twocolumn|\hy Ausdruck in 
\verb|\footnotesize| ergeben, was ich hier wieder nur aus Platzgr"unden unterlie"s.


\vspace{2ex}\noindent
Falls eine Abk"urzung
im Verzeichnis in \textit{abweichender} `Klammerung' ausgedruckt 
werden soll, sind \verb|\abklistopen| und \verb|\abklistclose| im Text 
am Ort der Abk"urzung lokal
anzupassen (zerbrechliche Befehle mit \verb|\protect| sch"utzen!). 
Der im Abk"urzungsverzeichnis abweichende Zugang kam so zustande:

\Doppelbox
{\vspace{1ex}
 Der 
 \\[.2ex] \b{\{}\bs renewcommand\{\bs abklistopen\}\{, \}\%
 \\ \ \bs renewcommand\{\bs abklistclose\}
 \\ \ \ \ \ \{ [Erkl"arung am Zugangsort]\}\%
 \\[.4ex] \ \bs abkdef\{S\}\{Sonderfall\}\b{\}} 
 \\[.2ex] in der Liste.
 \vspace{1.2ex}
}
{\vspace{8.125ex}
 \ \ Der {\renewcommand{\abklistopen}{, }%
  \renewcommand{\abklistclose}{ [Erkl"arung am Zugangsort]}%
 \abkdef{S}{Sonderfall}} in der Liste.
}\balabel{abklistopen}%


\vspace{-.2ex}\noindent
... wobei die Definition von \verb|\abklistopen| der "Ubersichtlichkeit 
halber mit einem `niederen' Zeichen beginnt, 
vor dem keine Korrektur n"otig ist; und \verb|\abklistclose| 
braucht keine. 
Eingetragen wird ins\hspace{-.15em} \verb|.aux|\hy File f"ur die
Einf"uhrung der Abk"urzungen \abk{OHG} und \abk{S} mit \verb|\abkdef| oder \verb|\defabk| etwas wie:

\vspace{0.325ex}
{\scriptsize\begin{verbatim}
   %\abkrzentry{OHG}{Offene Handelsgesellschaft}{{\pabklo }{\pabklc }{}{}}...(line 1762)
   ...
   %\abkrzentry{S}{Sonderfall}{{, }{ [Erkl\IeC {\"a}rung am Zugangsort]}{}{}}...(line 3756)
\end{verbatim}}

\vspace{.725ex}\noindent
Die Definitionen von \verb|\abklistopen| und 
\verb|\abklistclose| sind dabei jeweils ("ahnlich \verb|\nsep|\hspace{.075em}
oben S.\,\pageref{Ausreise}) zusammen mit den beiden Hauptargumenten von 
\verb|\abkdef| oder \verb|\defabk| ins 
\hspace{-.2em}\verb|.aux|\hy File gereist. Gelten f"ur Separatoren
im Falle mehrerer Erkl"arungen \textit{einer} Abk"urzung 
unterschiedliche Definitionen, 
warnt \texttt{bibsort} mittels Bildschirmmeldung; es verwendet auch bei den
Separatoren beim Ausdruck der Liste jeweils die erste Definition.



\newpage\noindent
Zum \textbf{Drucken von Archivquellenangaben} in Text oder Fu"snoten: Im 
Argument von \verb|\arqsep|, das zwischen Dokument und Archivsignatur gedruckt wird, ist 
kein corr\hy Befehl n"otig, da \BibArts\ f"ur beide Argumente
eine aufrechte Schrift \verb|\arqemph| erzwingt. Z.\,B. reicht \verb|\renewcommand{\arqsep}{: }| 
aus. Hier steht ein solcher \verb|\arq|\hy Befehl
{\showbacorr\itshape in schr"aggestelltem Umfeld: 
\renewcommand{\arqsep}{: }\arq{Gesellschaftsvertrag der 
KCAG}{BA Zwischenarchiv Dahlwitz\hy Hoppegarten R\,8729~4}(94)}. 

\vspace{1.25ex}\noindent
Beim \textbf{Drucken des Archivquellenverzeichnisses} darf 
f"ur \verb|\arqlistemph| (oder \verb|\arqemph|, siehe S.\,\pageref{arqemph2}) 
dagegen \textit{kursiv} eingestellt werden. Im \BibArts\hy Archivalienverzeichnis werden 
die einzelnen Dokumente nicht aufgelistet:

\vspace{.25ex}
{\footnotesize\begin{verbatim}
    {\renewcommand{\arqlistemph}{\itshape}\showbacorr \printnumarq }
\end{verbatim}}\label{arqemph1}

\vspace{-3.25ex}
    {\renewcommand{\arqlistemph}{\itshape}\showbacorr \printnumarq }


\vspace{2.5ex}\noindent
Auch bei Zeitschriften ist der $-$ im Text verbotene $-$
Kursivdruck von Titeln im \textbf{Zeitschriftenverzeichnis} erlaubt. Falls Sie 
\verb|\frompagesep| f"ur den num\hy Listenausdruck modifizieren und 
ein `hohes' Zeichen (hier im Beispiel $\rightarrow$\kern .2em) setzen,
dann sollten Sie \verb|\bacorr| vor das Zeichen setzen ({\small\verb|\bacorr$\rightarrow$|}):

{\footnotesize\begin{verbatim}
   {\renewcommand{\perlistemph}{\itshape}%
    \renewcommand{\frompagesep}{\bastrut\hskip0pt\bacorr$\rightarrow$}%
    \showbacorr \printnumper} % ^^^^^^^^^^^^^^^^^  <= Trennung erlauben
\end{verbatim}}\label{frompagesep2}%

\vspace{-3ex}
   {\renewcommand{\perlistemph}{\itshape}%
    \renewcommand{\frompagesep}{\bastrut\hskip0pt\bacorr$\rightarrow$}%
    \showbacorr \printnumper} % ^^^^^^^^^^^^^^^^^  <= Trennung erlauben

\vspace{1ex}\noindent
Die "offnenden und schlie"senden Separatoren f"ur die per\hy Liste
hei"sen (symmetrisch zu den abk\hy Befehlen) \label{perlistopen}
\verb|\perlistopen| und \verb|\perlistclose|. Sie f"uhren
\verb|\protect\pperlo| und \verb|\protect\pperlc| aus, um
beim Schreiben ins \hspace{-.15em}\verb|.aux|\hy File nicht zu zerbrechen.
Beim Ausdrucken der per\hy Liste f"uhren \verb|\pperlo| und 
\verb|\pperlc| gem"a"s Voreinstellung 
\verb|{\bastrut\ \bacorr $-$ }| und \verb|{}| aus. Daran k"onnen
Sie wiederum im Umfeld des Listenausdruckbefehls ansetzen:

\vspace{-.75ex}
{\small\begin{verbatim}
  {\renewcommand{\pperlo}{\bastrut\ \bacorr ((}
   \renewcommand{\pperlc}{))}
   \renewcommand{\perlistemph}{\itshape} \showbacorr \printper}
\end{verbatim}}

\vspace{-3.75ex}
  {\renewcommand{\pperlo}{\bastrut\ \bacorr ((}
   \renewcommand{\pperlc}{))}
   \renewcommand{\perlistemph}{\itshape} \showbacorr \printper}


\vspace{.5ex}\noindent
Im \textbf{Archivquellenverzeichnis} dienen 
\verb|\arqlistopen| und \verb|\arqlistclose| als Separatoren.
\label{Gedankenstriche} Sie f"uhren \verb|{\protect\parqlo}| 
und \verb|{\protect\parqlc}| aus und expandieren
zu \verb|{\bastrut\hskip 1em minus 0.3em\bacorr}| und \verb|{}|.
Die "Uberschriften, die im Verzeichnis ungedruckter Quellen
existieren k"onnen, enthalten die oben gezeigten Gedankenstriche,
die von den Befehlen \verb|\arqsectionopen|, \verb|\arqsubsectionopen|
und \verb|\arqsubsubsectionopen|\pdfko{1}\ 
initiiert werden. Alle f"uhren direkt \verb|{\bastrut\ \bacorr $-$ }| aus.

\vspace{1ex}\noindent
Weiter existieren \verb|\xrrlistopen| und \verb|\xrrlistclose|.
Sie schreiben f"ur die drei \BibArts\hy Register \verb|\protect|\hy gesch"utzt 
\verb|\pxrrlo| und \verb|\pxrrlc| ins \verb|.aux|\hy File und expandieren 
zu \verb|{\bastrut\ \bacorr(}| bzw.\ \verb|{)}| \,(vgl.\ oben S.\,\pageref{xrrlistclose}).




\newpage\noindent
\textbf{Wiederholung: }\texttt{\bs bacorr}\textbf{, }\texttt{\bs bapoint}\textbf{ und }\texttt{\bs bastrut}

\begin{itemize}
\item Ganz am Anfang der Definition eines Separators kann \verb|\bapoint|
      \textit{oder}\pdfko{.5}\ 
      \verb|\bastrut| stehen. Sie schlie"sen sich gegenseitig aus; es
      d"urfen nicht beide hintereinanderstehen. Falls Sie ganz an den
      Anfang eines Separators ein `echtes' Zeichen setzen (kein
      Leerzeichen), sind beide "uberfl"ussig.
\item \verb|\bapoint| tut nichts, wenn das \textit{im \BibArts\hy Befehl
      direkt zuvor gesetzte\pdfko{1.25}\ 
      Argument} mit einem Punkt oder \verb|\banotdot|
      oder \verb|\bahasdot| endet. Sonst druckt \verb|\bapoint| einen Punkt.
\item \verb|\bastrut| steht vor Leerzeichen (\,\verb*|\ |\,)\hspace{.1em} oder\hspace{.1em}
      \verb|\hskip|\hy\ oder \verb|\hspace|\hy Befehlen, um dort einen
      Zeilenumbruch \textit{sicher} zu erlauben. \hspace{.2em}\verb|\bastrut| ist
      freilich vor \verb|~| und anderen gesch"utzten Leerzeichen nicht n"otig.
\item Ist das erste `echte' Zeichen eines Separators ein hohes aufrechtes
      Zeichen, sollten Sie dann \verb|\bacorr| davorsetzen, wenn das
      vorausgehend gesetzte Argument schr"aggestellt sein k"onnte (vgl.\
      unten S.\,\pageref{hervor}). \hspace{.1em}\verb|\bacorr|\pdfko{1}\ 
      steht \textit{nie} vor Punkt, vor Komma oder vor \verb|\bapoint|.
\item F"ur Zeilenumbr"uche: Von zwei Ausnahmen abgesehen gilt am
      Kopf eines Separators immer die
      Reihenfolge \verb|\bastrut| $-$ Leerzeichen $-$ \verb|\bacorr|.\footnote{
      Bei einem Zeilenumbruch am Leerzeichen steht \texttt{\bs bacorr} direkt
      \textit{vor} dem linken Rand; irrt"umlich am rechten Rand stehend
      w"urde es dagegen eine (falsche!) Einr"uckung bewirken.} 
      Die beiden Ausnahmen sind: Falls \verb|\bacorr| in den Separatoren \verb|\abkdefopen| oder
      \verb|\defabkclose| steht, muss es \textit{vor} etwaigen Leerzeichen
      stehen! \label{Systematik} Nach einem solchen \verb|\bacorr| ist
      \verb|\bastrut| sinnlos.
\item \verb|\baupcorr| existiert aus Symmetriegr"unden, um \textit{italics}\hy
      Korrekturen in der Art anderer \BibArts\hy Korrekturen (\kern.01em\verb|0.1em|\kern.04em) zu erhalten.
      \verb|\baupcorr| tut \textit{immer} etwas in schr"aggestelltem Umfeld.
      \verb|\baupcorr| steht allenfalls am Ende von Separatoren. Es ist n"otig, 
      falls `hohe' Zeichen im Separator stehen \textit{und} schr"ag gedruckt werden 
      k"onnten \textit{und} das folgende Argument \textit{sicher} aufrecht ist. \verb|\baupcorr|
      k"onnte am Ende von \verb|\nsep| zum Einsatz kommen; es wird verwendet
      am Ende der sprachabh"angigen Separatoren (unten ab S.\,\pageref{SprachSep}) 
      \verb|\gannouncektitname| (\textit{im Folgenden}), \verb|\fannouncektitname| und 
      \verb|\eannouncektitname|.
\item \verb|\bacorr| steht in zwei F"allen am Ende des Separators. Diese Ausnahmen 
      sind \verb|\ntsep| und \verb|\defabkopen|: Die stehen beide \textit{vor} 
      Argumenten mit einstellbarer Schrift (abk und k\fhy Titel). \verb|\baupcorr|
      ist dort ungeeignet; aber \verb|\bacorr| reagiert, falls Sie f"ur das Argument
      dahinter eine schr"aggestellte Schrift einstellen (mit \verb|\abkemph| 
      bzw.\ mit \verb|\kxxemph|).
\end{itemize}



\newpage\noindent
\textbf{Sprachunabh"angige Separatoren allgemein}

\begin{small}
\vspace{1.5ex}\noindent
\verb|\arqsep  => {, } | \hfill \baref{arqsep}
\\ \hspace*{1em} Zwischen Dokument und Signatur in arq\hy Befehlen 
\\ \hspace*{1em} Alternativen: \verb|{: }| oder \verb|{\bapoint\newsentence}|
\\[1ex]
\verb|\nsep  => {\baslash} | \hfill (siehe S.\,\pageref{nsep1} und S.\,\pageref{nsep2})
\\ \hspace*{1em} Hinter Namen in x\fhy Befehlen sowie hinter \verb|\midvauthor| und \verb|\midkauthor|
\\ \hspace*{1em} Alternativen: \verb|{, }| oder \verb|{ / \baupcorr}| oder ...\,.\hspace{.25em} Ggf.\hspace{-.1em} \verb|\protect| setzen: 
\\ \hspace*{1em} Die lokale Definition von \verb|\nsep| reist \textit{auch} mit dem Eintrag in die Liste!
\\[1ex]
\verb|\ntsep  => {: \bacorr} | \hfill (siehe S.\,\pageref{ntsepA} und S.\,\pageref{ntsepB} samt Anm.\,\ref{ntsepC})
\\ \hspace*{1em} Vor dem letzten Hauptargument (`Rest' oder Kurztitel) von v- und k\fhy Befehlen
\\ \hspace*{1em} Alternativen: \verb|{, }| oder \verb|{\bapoint\ }|
\\[1ex]
\verb|\pagefolioshortsep  => {\bacorr : } | 
\\ \hspace*{1em} Unter \verb|\notprintlongpagefolio| \baref{banotlong} 
     vor \verb|[|\textit{Seite}\verb|]| oder \verb|(|\textit{Blatt}\verb|)|
\\ \hspace*{1em} "Anderung nicht empfohlen
\\[1ex]
\verb|\pernosep  => {\bacorr\nobreak\ } | \hfill \baref{pernosep}
\\ \hspace*{1em} Hilfsbefehl f"ur \verb|\pervolshortsep| und mehrere sprachabh"angige Separatoren
\\ \hspace*{1em} Alternative: \verb|{\bastrut\ \bacorr}|
\\[1ex]
\verb|\pervolshortsep |\hfill\verb|=> {\ifbaperiodical{\ifbaibidem{, }{\pernosep}}{, }}|
\\ \hspace*{1em} Unter \verb|\notprintlongpervol| \baref{banotlong} 
     vor \verb+|+\ko\textit{Band}\verb+|+ oder \verb|_|\textit{Nummer}\ko\verb|_|
\\ \hspace*{1em} "Anderung nicht empfohlen
\end{small}


\vspace{2.95ex}\noindent
\textbf{Sprachunabh"angige Separatoren speziell f"ur die Listen}

\begin{small}
\vspace{1.75ex}\noindent
\verb|\listlnfnsep  => {, }|
\\ \hspace*{1em} Komma zwischen Nachname und Vorname am Kopf von v\hy Eintr"agen
\\ \hspace*{1em} "Anderung etwa in \verb|{\bapoint\ }| prinzipiell m"oglich
\\[2.35ex]
\verb|\frompagesep  => |\verb|{\bastrut\hskip 1.25em minus 0.25em\bacorr}|
\\[-.55ex] \hspace*{1em} \hfill (S.\,\pageref{frompagesep1}, \pageref{frompagesep2}) 
\\ \hspace*{1em} Nach dem eigentlichen Eintrag und vor \textit{Seite}$^{\mbox{\scriptsize\itshape Fu"snote}}$ auf \textbf{allen num\hy Listen}
\\ \hspace*{1em} Alternativen: \verb|{, }| oder \verb|{\bastrut\hskip0pt\bacorr$\rightarrow$}|
\\[2.8ex]
Am Ende jedes Eintrags auf den verschiedenen nopage\hy Listen \baref[vgl.\ besonders]{nonum}
kann stets sowohl `nichts' als auch ein Punkt stehen; hier die Voreinstellungen:
\\[.9ex]
\verb|\fromnopageabksep  => {}| \hfill (\verb|\printabk|)
\\[.25ex]
\verb|\fromnopagearqsep  => {\bapoint}| \hfill (\verb|\printarq|)
\\[.25ex]
\verb|\fromnopagepersep  => {\bapoint}| \hfill (\verb|\printper|)
\\[.25ex]
\verb|\fromnopagevkcsep  => {}| \hfill \baref[\texttt{\bs printvkc}; auch]{fromnopagevkcsep2}
\\[.25ex]
\verb|\fromnopagevxxsep  => {\bapoint}| \hfill (\verb|\printvli| und \verb|\printvqu|)
\\[.25ex]
\verb|\fromnopagexrrsep  => {}| \hfill \baref[\texttt{\bs printgrr}, \texttt{\bs printsrr}, \texttt{\bs printprr}]{fromnopagexrrsep}

\end{small}



\newpage\noindent
\textbf{Sprachunabh"angige Klammerungen (open\fhy/close\hy Textelemente)}

\begin{small}
\vspace{1.7ex}\noindent
\verb|\abkdefopen  => {\bacorr\ (}| \hfill \baref{abkdefopen}
\\
\verb|\abkdefclose  => {)}|
\\ \hspace*{1em} Klammern, in die das zweite Argument von \verb|\abkdef| im Text gesetzt wird
\\[1.6ex]
\verb|\defabkopen  => {\bastrut\ (\bacorr}| \hfill \baref{defabkopen}
\\
\verb|\defabkclose  => {\bacorr)}| 
\\ \hspace*{1em} Klammern, in die das zweite Argument von \verb|\defabk| im Text gesetzt wird
\\[1.6ex]
\verb|\abklistopen  => {\protect\pabklo}| \hfill \baref{abklistopen}
\\[-.1ex] \hspace*{7em} \verb|\pabklo  => {\bastrut\hskip 1.2em minus 0.3em\bacorr}|
\\
\verb|\abklistclose  => {\protect\pabklc}|
\\[-.1ex] \hspace*{7em} \verb|\pabklc  => {}|
\\ \hspace*{1em} Klammern, in die eine \textit{Erkl"arung} auf der abk\hy Liste gesetzt wird; die loka- 
\\ \hspace*{1em} len Definitionen von \verb|\abklistopen| und \verb|\abklistclose| reisen in die Liste  
\\[1.6ex]
\verb|\arqlistopen  => {\protect\parqlo}|
\\[-.1ex] \hspace*{7em} \verb|\parqlo  => {\bastrut\hskip 1em minus 0.3em\bacorr}|
\\
\verb|\arqlistclose  => {\protect\parqlc}|
\\[-.1ex] \hspace*{7em} \verb|\parqlc  => {}|
\\ \hspace*{1em} Klammern um das zweite Argument von \verb|\fillarq| auf der arq\hy Liste; die lo-
\\ \hspace*{1em} kalen Definitionen von \verb|\arqlistopen| und \verb|\arqlistclose| reisen in die Liste  
\\[1.6ex]
\verb|\arqsectionopen  => {\bastrut\ \bacorr $-$ }|
\\[-.05ex]
\verb|\arqsectionclose  => {\bahasdot}            %% nie hinterpunkten|
\\
\verb|\arqsubsectionopen  => {\bastrut\ \bacorr $-$ }|
\\[-.05ex]
\verb|\arqsubsectionclose  => {\bahasdot}|
\\
\verb|\arqsubsubsectionopen  => {\bastrut\ \bacorr $-$ }|
\\[-.05ex]
\verb|\arqsubsubsectionclose  => {\bahasdot}|
\\[.05ex] \hspace*{1em} Klammerpaare um das zweite Argument von \verb|\arqsection| (Archiv
   S.\,\pageref{arqsection}, \pageref{Gedankenstriche}),
\\ \hspace*{1em} \verb|\arqsubsection| bzw.\hspace{-.1em} \verb|\arqsubsubsection|; die lokalen Definitionen dazu
\\ \hspace*{1em} reisen in die arq\hy Liste; vor zerbrechlichen Befehlen sollte \verb|\protect| stehen
\\[1.6ex]
\verb|\perlistopen  => {\protect\pperlo}| \hfill (siehe S.\,\pageref{perlistopen})
\\[-.1ex] \hspace*{7em} \verb|\pperlo  => {\bastrut\ \bacorr $-$ }|
\\
\verb|\perlistclose  => {\protect\pperlc}|
\\[-.1ex] \hspace*{7em} \verb|\pperlc  => {}|
\\ \hspace*{1em} Klammern um das zweite Argument von \verb|\fillper| auf der per\hy Liste; die lo-
\\ \hspace*{1em} kalen Definitionen von \verb|\perlistopen| und \verb|\perlistclose| reisen in die Liste  
\\[1.6ex]
\verb|\xrrlistopen  => {\protect\pxrrlo}| \hfill (siehe S.\,\pageref{xrrlistclose})
\\[-.1ex] \hspace*{7em} \verb|\pxrrlo  => {\bastrut\ \bacorr(}|
\\
\verb|\xrrlistclose  => {\protect\pxrrlc}|
\\[-.1ex] \hspace*{7em} \verb|\pxrrlc  => {)}|
\\ \hspace*{1em} Klammern um das zweite Argument von \verb|\fillgrr|, \verb|\fillprr| bzw.\hspace{-.1em} \verb|\fillsrr|; 
\\ \hspace*{1em} lokale Definitionen von \verb|\xrrlistopen| und \verb|\xrrlistclose| reisen in die Listen  

\end{small}


\newpage
\section{Sprachabh"angige Separatoren (\kern-.02em\textit{captions})}\label{Sect12}\label{SprachSep}

Wenn Sie mit dem \BibArts-Befehl \verb|\sethyphenation{|\textit{Sprache}\verb|}| 
oder alternativ f"ur Zitatbl"ocke mit
\verb|\begin{originalquote}[|\textit{Sprache}\verb|]| eine bestimmte\pdfko{1.125}\ 
Sprache einstellen, wird nur bestimmt, wie Worte getrennt werden (Trennsatz).
Andere Schalter stellen die Basissprache des Textes ein, bestimmen also,
in welcher Sprache Text\hy Separatoren (\kern-.1em\textit{captions}) zu drucken 
sind $-$\pdfko{1.25}\  
ob beispielsweise die Abk"urzung f"ur Seite \textit{S.} oder \textit{p.} lautet. 
W"ahrend der\pdfko{1.25}\ 
Trennsatz oft mehrfach in einem Text f"ur fremdsprachige w"ortliche Zitate 
jeweils angepasst wird, bleibt die Sprache der Text\hy Separatoren in einem 
Text meist durchgehend gleich. \BibArts\hy \textit{captions} werden also meist 
im Vorspann des\pdfko{1.25}\  
\LaTeX\hy Textes festgelegt; \verb|\bacaptionsgerman| 
ist voreingestellt. \BibArts\ stellt\pdfko{1.5}\  
gegenw"artig zudem \verb|\bacaptionsenglish| 
und \verb|\bacaptionsfrench| bereit.\pdfko{1}\  
Falls Sie \verb|ngerman.sty| nutzen, 
m"ussen Sie ggf.\ die dort bereitgestellten Befehle \verb|\captionsenglish| 
oder \verb|\captionsfrench| zus"atzlich setzen, denn die\pdfko{1.25}\  
\verb|\bacaptions|...\hy Befehle "andern die Voreinstellungen f"ur 
\LaTeX\hy \textit{captions} wie\pdfko{1.75}\  
etwa \textit{\chaptername} (\verb|\chaptername|) nicht. 
\verb|\bacaptions|...\hy Befehle "andern nur die\pdfko{1.5}\  
Voreinstellungen f"ur \BibArts\hy Befehle. Hier zwei wichtige Beispiele:

\vspace{.75ex}{\small\noindent
\verb|  \bacaptionsenglish \vli{}{}{Text}[20]  => |{\bacaptionsenglish \printonlyvli{}{}{Text}[20]} \\
\verb|  \bacaptionsgerman  \vli{}{}{Text}[20]  => |{\bacaptionsgerman  \printonlyvli{}{}{Text}[20]}}

\vspace{1.325ex}\noindent
Bei der \textit{Einstellung} sprachabh"angiger Separatoren gibt es einen
Sonderfall: \textsc{ebd.}\ wird mit\hspace{.15em} \verb|\setibidem{g}{ebenda}{}|\hspace{.25em} 
in\hspace{.15em} \textsc{ebenda}\hspace{.1em} umgestellt (nicht mit \verb|\renewcommand|\kern.05em). Das erste Argument 
\verb|{g}| weist die "Anderung in \textsc{ebenda} dabei \verb|\bacaptionsgerman| zu. 
Im Englischen lautet die Voreinstellung \textsc{ibidem} (\verb|\bacaptionsenglish|)
und\hspace{.1em} \textsc{ibid\kern -0.07em.}\hspace{.075em} im Franz"osischen (\verb|\bacaptionsfrench|). Auch diese 
Voreinstellungen lassen sich "andern, etwa vertauschen.\footnote{
Wie oben S.\,\pageref{setibidem} schon erw"ahnt,
muss ein am Ende des vorgefertigten Textelements ggf.\ vorhandener Abk"urzungspunkt in das letzte Argument;
sonst muss es leer bleiben: \\[1ex]
\hspace*{5.5mm} \texttt{\bs setibidem\{e\}\{ibid\bs kern -0.07em\}\{.\}} \setibidem{e}{ibid\kern -0.07em}{.} \\
\hspace*{5.5mm} \texttt{\bs setibidem\{f\}\{ibidem\}\{\}} \setibidem{f}{ibidem}{} \\[.25ex]
\hspace*{5.5mm} \texttt{\ \ \ \ \ \ \ \ \ \ \ \ \ \ \ \ \ \ \ \ \ \ \ \ \bs kli\{\}\{Text\}[20]. \ => } \bacaptionsgerman \printonlykli{}{Text}[20]. \\
\hspace*{5.5mm} \texttt{\bs bacaptionsenglish \ \ \ \ \ \bs kli\{\}\{Text\}[20]. \ => } \bacaptionsenglish \printonlykli{}{Text}[20]. \\
\hspace*{5.5mm} \texttt{\bs bacaptionsfrench \ \ \ \ \ \ \bs kli\{\}\{Text\}[20]. \ => } \bacaptionsfrench \printonlykli{}{Text}[20].}

\vspace{1ex}\noindent
Alle anderen \textit{captions} d"urfen Sie mit \verb|\renewcommand| "andern, 
wenn Ihnen\pdfko{1}\ 
die Voreinstellungen von \BibArts\ nicht gefallen. Etwa
\verb|\vli{}{}{Text}| w"urde nach \verb|\renewcommand{\ganonymousname}{[?]}| 
ausdrucken: {\renewcommand{\ganonymousname}{[?]} \printonlyvli{}{}{Text}}.\pdfko{.5}\
Die deutschen \textit{captions} beginnen mit\hspace{-.1em} \verb|\g|...\,, die englischen mit\hspace{-.1em} 
\verb|\e|...\ und die\pdfko{1}\ 
franz"osischen mit\hspace{-.1em} \verb|\f|...\,. Ich liste nun die 
in \verb|bibarts.sty| definierten Voreinstellungen auf. In Worten mit 
Sonderzeichen (wie "s oder \'e) ist jeweils die Trennhilfe\hspace{-.1em} \verb|\-| 
eingesetzt, damit auch in \texttt{OT1}\hy Kodierung richtig getrennt wird. 

\vspace{1ex}\noindent{\small
\verb|\ganonymousname => {[Anonym]}| \\[-.8ex]
\verb|\eanonymousname => {[Anonymous]}| \\[-.8ex]
\verb|\fanonymousname => {[Anonyme]}| \\[-.2ex]
Kein Autor in v- oder k-Befehlen eingetippt: \verb|\kli{}{T} =>| \printonlykli{}{T}.
\\[.8ex]%
\verb|\geditorname => {[\kern 0.04em Hrsg.]\kern 0.02em}| \\[-.8ex]
\verb|\eeditorname => {(\kern -0.03em ed.\kern -0.06em)\kern 0.02em}| \\[-.8ex]
\verb|\feditorname => {(\kern -0.03em \'ed.\kern -0.06em)\kern 0.02em}| \\[-.2ex]
Text f"ur Befehl \verb|\editor =>| \editor\ \ (ohne Sortiergewicht).
\\[.8ex]%
\verb|\geditorpname => {[\kern 0.04em Hrsg.]\kern 0.02em}| \\[-.8ex]
\verb|\eeditorpname => {(\kern -0.03em eds.\kern -0.08em)\kern 0.02em}| \\[-.8ex]
\verb|\feditorpname => {(\kern -0.03em \'ed.\kern -0.06em)\kern 0.02em}| \\[-.2ex]
Text f"ur Befehl \verb|\editors| (mehrere Herausgeber); mit \verb|\bacaptionsenglish|: {\bacaptionsenglish\editors}.
\\[.8ex]%
\verb|\gidemname => {ders\kern -0.04em.\bahasdot}     % 2.2: \bahasdot neu| \\[-.8ex]
\verb|\eidemname => {idem}| \\[-.8ex]
\verb|\fidemname => {le m\^e\-me}| \\[-.2ex]
Derselbe Autor wird direkt hintereinander mit verschiedenen Werken zitiert: \\[-.2ex]
\verb|[m]| direkt nach v- und k-Befehlen: \verb|\kli[m]{N}{T} =>| \printonlykli[m]{N}{T}.
\\[.8ex]%
\verb|\geademname => {dies\kern -0.04em.\bahasdot}    % 2.2: \bahasdot neu| \\[-.8ex]
\verb|\eeademname => {eadem}| \\[-.8ex]
\verb|\feademname => {la m\^e\-me}| \\[-.2ex]
\verb|[f]| direkt nach v- und k-Befehlen: \verb|\kli[f]{N}{T} =>| \printonlykli[f]{N}{T}.
\\[.8ex]%
\verb|\giidemname => {diesn\kern -0.07em.\bahasdot}   % 2.2: \bahasdot neu| \\[-.8ex]
\verb|\eiidemname => {iidem}| \\[-.8ex]
\verb|\fiidemname => {les m\^e\-mes}| \\[-.2ex]
\verb|[p{}]| direkt nach v- und k-Befehlen: \verb|\kli[p{}]{N1}*{N2}{T} =>| \printonlykli[p{}]{N1}*{N2}{T}.
\\[.8ex]%
\verb|\gvolname => {, Bd.\,}| \\[-.8ex]
\verb|\evolname => {, vol.\,}| \\[-.8ex]
\verb|\fvolname => {, vol.\,}| \\[-.2ex]
Bandangabe nach v-, k-, arq- und per-Befehlen: \verb+\per{ZfG.}|2| =>+ \per{ZfG.}|2|.
\\[.8ex]%
\verb|\gvolpname => {, Bde.\,}| \\[-.8ex]
\verb|\evolpname => {, vols.\,}| \\[-.8ex]
\verb|\fvolpname => {, vol.\,}| \\[-.2ex]
Nach v-, k-, arq- und per-Befehlen: 
\verb+\per{ZfG.}|2-3| =>+ \per{ZfG.}|2-3|.\footnote{\BibArts\ ermittelt
 einen vorliegenden Plural selbst"andig, indem es das Argument nach 
 \texttt{-}, [Komma], \texttt{\bs hy}, \texttt{\bs fhy}, \texttt{\bs f}, 
 \texttt{\bs ff}, \texttt{\bs sq}, und \texttt{\bs sqq} durchsucht,
 oder setzt die Plural\hy \textit{caption}\pdfko{.75}\ 
 ein, wenn Sie \texttt{\bs baplural} setzen; vgl.\ oben S.\,\pageref{baplural}.}
\\[.8ex]%
\verb|\gpername => {\ifbaibidem{, Nr.\,}{\pernosep}}| \\[-.8ex]
\verb|\epername => {\ifbaibidem{, no.\,}{\pernosep}}| \\[-.8ex]
\verb|\fpername => {\ifbaibidem{, n\fup{o}\,}{\pernosep}}| \\[-.2ex]
Heftangaben im Singular (mit/ohne \textsc{ebd.}) in v-, k-, arq- und per-Befehlen.\footnote{
 \texttt{\bs per\{ZfG.\}\_5\_ und \bs per\{ZfG.\}\_6\_ \ \ \ \ => } 
 \per{ZfG.}_5_ und \per{ZfG.}_6_.}
\\[.8ex]%
\verb|\gperpname => {\ifbaibidem{, Nr.\,}{\pernosep}}| \\[-.8ex]
\verb|\eperpname => {\ifbaibidem{, no.\,}{\pernosep}}| \\[-.8ex]
\verb|\fperpname => {\ifbaibidem{, n\fup{os}\,}{\pernosep}}| \\[-.2ex]
Heftangaben im Plural (mit/ohne \textsc{ebd.}) in v-, k-, arq- und per-Befehlen. \\[-.2ex]
Beispiel unter \verb|\bacaptionsfrench| in der Fu"snote.\footnote{\bacaptionsfrench
 \texttt{\bs per\{Jour\}\_4-5\_ et \bs per\{Jour\}\_6-7\_ \ => }
 \printonlyper{Jour}_4-5_ et \printonlyper{Jour}_6-7_.}
\newpage\noindent%\\[.8ex]%
\verb|\gisonfolioname => {, Bl.\,}| \\[-.8ex]
\verb|\eisonfolioname => {, folio\nobreak \ }| \\[-.8ex]
\verb|\fisonfolioname => {, folio\nobreak \ }| \\[-.2ex]
Blattangabe nach v-, k-, arq- und per-Befehlen: \verb|\arq{}{PRO}(2) =>| \printonlyarq{}{PRO}(2).
\\[.8ex]%
\verb|\gisonfoliopname => {, Bl.\,}| \\[-.8ex]
\verb|\eisonfoliopname => {, folii\nobreak \ }| \\[-.8ex]
\verb|\fisonfoliopname => {, folii\nobreak \ }| \\[-.2ex]
Sichtbar in {\bacaptionsenglish\verb|\bacaptionsenglish| \verb|\arq{}{PRO}(2-3) =>| \printonlyarq{}{PRO}(2-3).}
\\[.8ex]%
\verb|\gisonxfolioname => {, dort: Bl.\,}| \\[-.8ex]
\verb|\eisonxfolioname => {, there: Folio\nobreak \ }| \\[-.8ex]
\verb|\fisonxfolioname => {, l\`a: Folio\nobreak \ }| \\[-.2ex]
\verb|*|-Blatt nach v-, k-, arq- und per-Befehlen: \verb|\arq{}{PRO}*(2) =>| \printonlyarq{}{PRO}*(2).
\\[.8ex]%
\verb|\gisonxfoliopname => {, dort: Bl.\,}| \\[-.8ex]
\verb|\eisonxfoliopname => {, there: Folii\nobreak \ }| \\[-.8ex]
\verb|\fisonxfoliopname => {, l\`a: Folii\nobreak \ }| \\[-.2ex]
Sichtbar in {\bacaptionsenglish\verb|\bacaptionsenglish| \verb|\arq{}{PRO}*(2-3) =>| \printonlyarq{}{PRO}*(2-3).}
\\[.8ex]%
\verb|\gisonpagename => {, S.\,}| \\[-.8ex]
\verb|\eisonpagename => {, p.\,}| \\[-.8ex]
\verb|\fisonpagename => {, p.\,}| \\[-.2ex]
Seitenangabe nach v-, k-, arq- und per-Befehlen: \verb|\kli{N}{T}[2] =>| \printonlykli{N}{T}[2].
\\[.8ex]%
\verb|\gisonpagepname => {, S.\,}| \\[-.8ex]
\verb|\eisonpagepname => {, pp.\,}| \\[-.8ex]
\verb|\fisonpagepname => {, p.\,}| \\[-.2ex]
Sichtbar in {\bacaptionsenglish\verb|\bacaptionsenglish| \verb|\kli{N}{T}[2-3] =>| \printonlykli{N}{T}[2-3].}
\\[.8ex]%
\verb|\gisonxpagename => {, dort: S.\,}| \\[-.8ex]
\verb|\eisonxpagename => {, there: p.\,}| \\[-.8ex]
\verb|\fisonxpagename => {, l\`a: p.\,}| \\[-.2ex]
\verb|*|-Seite nach v-, k-, arq- und per-Befehlen: \verb|\kli{N}{T}*[2] =>| \printonlykli{N}{T}*[2].
\\[.8ex]%
\verb|\gisonxpagepname => {, dort: S.\,}| \\[-.8ex]
\verb|\eisonxpagepname => {, there: pp.\,}| \\[-.8ex]
\verb|\fisonxpagepname => {, l\`a: p.\,}| \\[-.2ex]
Sichtbar in {\bacaptionsenglish\verb|\bacaptionsenglish| \verb|\kli{N}{T}*[2-3] =>| \printonlykli{N}{T}*[2-3].}
\\[.8ex]%
\verb|\gbibtitlename => {Quellen und Literatur}| \\[-.8ex]
\verb|\ebibtitlename => {Bibliography}| \\[-.8ex]
\verb|\fbibtitlename => {Bibliographie}| \\[-.2ex]
Titel gesamter Belegapparat ("Uberschrift \BibArts-Anhang) \verb|\printbibtitle|.
\\[.8ex]%
\verb|\gabktitlename => {Ab\-k\"ur\-zungen}| \\[-.8ex]
\verb|\eabktitlename => {Abbreviations}| \\[-.8ex]
\verb|\fabktitlename => {Ab\-r\'e\-viations}| \\[-.2ex]
Titel Abk"urzungsverzeichnis \verb|\printabk| und \verb|\printnumabk| bzw.\ \verb|\printabktitle|.
\\[.8ex]%
\verb|\gvlititlename => {Literatur}| \\[-.8ex]
\verb|\evlititlename => {Literature}| \\[-.8ex]
\verb|\fvlititlename => {Travaux}| \\[-.2ex]
Titel Literaturliste \verb|\printvli| und \verb|\printnumvli| bzw.\ \verb|\printvlititle|.
\\[.8ex]%
\verb|\ghinttovliname => {[L]}          %\| \\[-.8ex]
\verb|\ehinttovliname => {[L]}          % )  Alle ohne italics-Korrektur!| \\[-.8ex]
\verb|\fhinttovliname => {[T]}          %/| \\[-.2ex]
Hinweis auf Liste mit vollen Literaturangaben: \verb|\kli{N}{T} =>| \printonlykli{N}{T}.
\\[.8ex]%
\verb|\gvqutitlename => {Gedruckte Quellen}| \\[-.8ex]
\verb|\evqutitlename => {Published Documents}| \\[-.8ex]
\verb|\fvqutitlename => {Sources im\-pri\-m\'ees}| \\[-.2ex]
Titel Verzeichnis gedruckter Quellen \verb|\printvqu| (plus -\verb|num|-) bzw.\ \verb|\printvqutitle|.
\\[.8ex]%
\verb|\ghinttovquname => {[Q]}          %\| \\[-.8ex]
\verb|\ehinttovquname => {[D]}          % )  Alle ohne italics-Korrektur!| \\[-.8ex]
\verb|\fhinttovquname => {[S]}          %/| \\[-.2ex]
Hinweis auf Verzeichnis mit vollen Quellenangaben: \verb|\kqu{N}{T} =>| \printonlykqu{N}{T}.
\\[.8ex]%
\verb|\gpertitlename => {Zeitschriften}| \\[-.8ex]
\verb|\epertitlename => {Periodicals}| \\[-.8ex]
\verb|\fpertitlename => {P\'e\-riodiques}| \\[-.2ex]
Titel Zeitschriftenverzeichnis \verb|\printper| und \verb|\printnumper| bzw.\ \verb|\printpertitle|.
\\[.8ex]%
\verb|\garqtitlename => {Ungedruckte Quellen}| \\[-.8ex]
\verb|\earqtitlename => {Unpublished Documents}| \\[-.8ex]
\verb|\farqtitlename => {Sources in\-\'edi\-tes}| \\[-.2ex]
Titel Archivquellenverzeichnis \verb|\printarq|,  \verb|\printnumarq| bzw.\ \verb|\printarqtitle|.
\\[.8ex]%
\verb|\gvkctitlename => {Verwendete Kurztitel}| \\[-.8ex]
\verb|\evkctitlename => {Shortened References}| \\[-.8ex]
\verb|\fvkctitlename => {Titres ab\-r\'e\-g\'ees}| \\[-.2ex]
Titel Kurzzitateverzeichnis \verb|\printnumvkc| und \verb|\printvkc| bzw.\ \verb|\printvkctitle|.
\\[.8ex]%
\verb|\ggrrtitlename => {Ortsregister}| \\[-.8ex]
\verb|\egrrtitlename => {Geographical index}| \\[-.8ex]
\verb|\fgrrtitlename => {Registre g\'eo\-graphique}| \\[-.2ex]
Titel Ortsregister f"ur Liste \verb|\printnumgrr| und \verb|\printgrr| bzw.\ \verb|\printgrrtitle|.
\\[.8ex]%
\verb|\gprrtitlename => {Personenregister}| \\[-.8ex]
\verb|\eprrtitlename => {Person index}| \\[-.8ex]
\verb|\fprrtitlename => {Registre des personnes}| \\[-.2ex]
Titel Personenreg.\ f"ur Liste \verb|\printnumprr| und \verb|\printprr| bzw.\ \verb|\printprrtitle|.
\\[.8ex]%
\verb|\gsrrtitlename => {Sachregister}| \\[-.8ex]
\verb|\esrrtitlename => {Subject index}| \\[-.8ex]
\verb|\fsrrtitlename => {Registre des sujets}| \\[-.2ex]
Titel Sachregister f"ur Liste \verb|\printnumsrr| und \verb|\printsrr| bzw.\ \verb|\printsrrtitle|.
\\[.8ex]%
\verb|\gfolpagename => {\badelspacebefore\,f\kern -0.1pt.\bahasdot}| \\[-.8ex]
\verb|\efolpagename => {\badelspacebefore\,f\kern -0.1pt.\bahasdot}| \\[-.8ex]
\verb|\ffolpagename => {\badelspacebefore\ sq.\bahasdot}| \\[-.2ex]
Abk"urzung 'folgende' (\verb|\f|\,=\,\verb|\sq|): \verb|\per{ZfG.}_2\sq_[3 \f] =>| \per{ZfG.}_2\sq_[3 \f].
\\[.8ex]%
\verb|\gxfolpagename => {\badelspacebefore\,ff\kern -0.1pt.\bahasdot}| \\[-.8ex]
\verb|\exfolpagename => {\badelspacebefore\,ff\kern -0.1pt.\bahasdot}| \\[-.8ex]
\verb|\fxfolpagename => {\badelspacebefore\ sqq.\bahasdot}| \\[-.2ex]
Abk"urzung 'mehrere folgende': \verb|\per{ZfG.}_2\ff_[3\sqq] =>| \per{ZfG.}_2\ff_[3\sqq].
\\[2.5ex]%
\verb|\gannouncektitname => |{\footnotesize\verb|{\bastrut\ (\kern 0.015em im Folgenden \baupcorr}|} \\[-.4ex]
\hspace*{5cm} \verb|\gannouncekendname => {)}| \\[-.4ex]
\verb|\eannouncektitname => |{\footnotesize\verb|{\bastrut\ (\kern -0.02em cited as \baupcorr}|} \\[-.4ex]
\hspace*{5cm} \verb|\eannouncekendname => {)}| \\[-.4ex]
\verb|\fannouncektitname => |{\footnotesize\verb|{\bastrut\ (\kern 0.02em par la suite \baupcorr}|} \\[-.4ex]
\hspace*{5cm} \verb|\fannouncekendname => {)}| \\[.2ex]
\verb|\ktit|""-Ank"undigung: \verb|\vli{V}{N}{\ktit{T}} =>| {\notktitaddtok\printonlyvli{V}{N}{\ktit{T}}}.
\newpage\noindent%\\[.8ex]%
\verb|\grefvbegname => {(}| \\[-.8ex]
\hspace*{1cm} \verb|\grefvendname => {\barefcorr)}| \\[-.8ex]
\verb|\erefvbegname => {[\nobreak \hskip 1pt plus 0pt}| \\[-.8ex]
\hspace*{1cm} \verb|\erefvendname => {\nobreak \hskip 1pt plus 0pt\barefcorr]}| \\[-.8ex]
\verb|\frefvbegname => {(}| \\[-.8ex]
\hspace*{1cm} \verb|\frefvendname => {\barefcorr)}| \\[-.2ex]
Klammern in \verb|\conferize| f"ur k-Befehle: \verb|\kli{Luhmann}{Soziale Systeme} =>| \\[-.5ex]
{\conferize \kli{Luhmann}{Soziale Systeme}}. \\[-.5ex]
Die \textit{italics}\hy Korrektur \verb|\barefcorr| ist f"ur Verweise bes.\ auf \verb|{minipage}|-Fu"snoten.
\\[.8ex]%
\verb|\gconfername => {\kern -0.03em wie}| \\[-.8ex]
\verb|\econfername => {\kern -0.05em cf.\bahasdot}| \\[-.8ex]
\verb|\fconfername => {\kern -0.03em op.\ cit.\bahasdot}| \\[-.2ex]
Bezugsworte im Querverweis des eben genannten \verb|\conferize|-Stils.
\\[.8ex]%
\verb|\grefvpagname => {S.\,}| \\[-.8ex]
\verb|\erefvpagname => {p.\,}| \\[-.8ex]
\verb|\frefvpagname => {p.\,}| \\[-.2ex]
Seitenabk"urzung im Querverweis des eben genannten \verb|\conferize|-Stils; und \\[-.5ex]
ebenso f"ur \BibArts-Querverweise: \verb|\baref{Mueller} =>| \baref{Mueller}, wozu auch \\[-.5ex]
die oben unter \verb|\grefvbegname| ... genannten Klammersymbole benutzt werden.
\\[.8ex]%
\verb|\grefverbname => {siehe}| \label{grefverbname} \\[-.8ex]
\verb|\erefverbname => {see}| \\[-.8ex]
\verb|\frefverbname => {voir}| \\[-.2ex]
Bezugswort im eben genannten \verb|\baref|-Querverweis.
\\[.8ex]%
\verb|\grefvfntname => {, Anm.\,}| \\[-.8ex]
\verb|\erefvfntname => {, n.\,}| \\[-.8ex]
\verb|\frefvfntname => {, n.\,}| \\[-.2ex]
Abk"urzung f"ur '\kern-.05em Anmerkung' oder 'Fu"snote' in den \verb|\conferize|-k-Befehle oben \\[-.5ex]
und f"ur \BibArts\hy Querverweise: \verb|\baref{XX} =>| \baref{XX}.\footnote{
\balabel{XX}\texttt{\bs balabel\{XX\}}.
\BibArts\ bemerkt automatisch, ob dies in einer Fu"snote steht.}
\\[.8ex]%
\label{erscheditionname}%
\verb|\gerscheditionname => {\teskip Auf{\kern.03em}l.,}| \\[-.8ex]
\verb|\eerscheditionname => {\fupskip edition,}| \\[-.8ex]
\verb|\ferscheditionname => {\fupskip \'edi\-tion,}   %%| 
       \texttt{Vgl. oben S. \pageref{fordinalf}.} \\[-.2ex]
Auf"|lage-Abk"urzung in \verb|\ersch[4]{Stuttgart}{1899} =>| \ersch[4]{Stuttgart}{1899}.
\\[.8ex]%
\verb|\gerschvolumename => {Bd.,}    \gerschvolumepname => {Bde.,}| \\[-.8ex]
\verb|\eerschvolumename => {vol.,}   \eerschvolumepname => {vols.,}| \\[-.8ex]
\verb|\ferschvolumename => {vol.,}   \ferschvolumepname => {vol.,}| \\[-.2ex]
Band-Abk"urzung in \verb+\ersch|3|{Stuttgart}{1899} =>+ \ersch|3|{Stuttgart}{1899}.
\\[.8ex]%
\verb|\gerschnohousename => {\oO,}| {\footnotesize\verb|=> {o.\kern 0.1em O\kern -0.08em.\bahasdot}|}\\[-.8ex]
\verb|\eerschnohousename => {n.\kern 0.15em p.,}            | ('no place') \\[-.8ex]
\verb|\ferschnohousename => {s.\kern 0.15em l\kern 0.02em.,}| \ ('sans lieu') \\[-.2ex]
Kein Erscheinungsort getippt in \verb|\ersch{}{1899} =>| \ersch{}{1899} \ ('ohne Ort').
\\[.8ex]%
\verb|\gerschnoyearname => {\oJ}  | {\footnotesize\verb|=> {o.\kern 0.1em J\kern -0.09em.\bahasdot}|} \\[-.8ex] \label{gerschnoyearname}
\verb|\eerschnoyearname => {n.\kern 0.13em d.\bahasdot}| \ ('no date') \\[-.8ex]
\verb|\ferschnoyearname => {s.\kern 0.13em d.\bahasdot}| \ ('sans date') \\[-.2ex]
Kein Jahr in \verb|\vli{}{Sam}{Titel, \ersch{Paris}{}} =>| \ersch{Paris}{} \hfill (weiter!\,$\rightarrow$)
}

\noindent
\textbf{Separater Ausdruck von vorgefertigten Textelementen}\\[1ex]
Wenn Befehle, die \verb|\bahasdot| oder \verb|\banotdot| nutzen, nicht am 
Ende des Arguments eines \BibArts\fhy Hauptbefehls stehen, kann dies 
einen Zeilenumbruch verhindern. An \verb*|\oJ, | wird umgebrochen, aber ein 
\verb|\oJ| \textit{direkt} folgendes Leerzeichen ist oft gesch"utzt. Dann k"onnen 
Sie \verb|\strut| einf"ugen: Im freien Text ist also \verb*|\oJ\strut\ | 
statt \verb*|\oJ\ | zu tippen. Das gilt auch, wenn ein Befehl \verb|\oJ| ausf"uhrt: 
\verb|\ersch{Stuttgart}{}\strut\ next =>| \ersch{Stuttgart}{}\strut\ next.
Am Ende von Hauptbefehlen macht \BibArts\ dies f"ur Sie eigenst"andig: 
\verb|\vli{}{N.}{...., \ersch{Stuttgart}{}} next =>| 
\printonlyvli{}{N.}{...., \ersch{Stuttgart}{}} next. Auch ist es der
Hauptbefehl, der Ihren Satzende\hy Punkt nach einer
automatisch eingef"ugten Abk"urzung `verschluckt'; Beispiel
ist der Hauptbefehl \verb|\vli{}{N.}{...., \ersch{Stuttgart}{}}. =>|
\printonlyvli{}{N.}{...., \ersch{Stuttgart}{}}. 
Oder \verb|\vli{}{N.}{...., \oJ}. =>| \printonlyvli{}{N.}{...., \oJ}.
Dagegen ergibt im freien Text \verb|\oJ. =>| \oJ. Und
\verb|\ersch{Stuttgart}{}. =>| \ersch{Stuttgart}{}. Auch stellen
nur \BibArts\hy Hauptbefehle das \textit{spacing} wie
oben S.\,\pageref{nonfrenchspacing} geschildert ein.
Sonst ist \verb|\ersch{Stuttgart}{}\bapoint\newsentence Next| stets m"oglich
(\verb|\bapoint| tut, was \verb|\strut| tut,
druckt jedoch nur \textit{einen} Punkt: etwa nach \verb|\renewcommand{\poJ}{ohne Jahr}|; siehe S.\,\pageref{poJ}).
Bei `englischem' \textit{spacing} ist \verb|\newsentence| sogar n"otig:

\vspace{1ex}
{\small
 %\renewcommand{\poJ}{ohne Jahr}%
{\frenchspacing    \verb|\frenchspacing     | \ersch{Stuttgart}{}\bapoint\newsentence Next
}

{\nonfrenchspacing \verb|\nonfrenchspacing  | \ersch{Stuttgart}{}\bapoint\newsentence Next
}
}

\vspace{1.5ex}\noindent
Vorgefertigte Textelemente, die speziell \verb|\bacorr| enthalten, 
sollten Sie nicht separat ausdrucken. Denn \verb|\bacorr|
ist nur innerhalb der \BibArts\hy Hauptbefehle mit einem sinnvollen
Inhalt belegt. Zum separaten Ausdruck speziell von \verb|\ntsep| gibt 
es deshalb den Befehl \verb|\printntsep|, der `\printntsep' mit lokal
ausgeschalteter \textit{italics}\hy Korrektur druckt. 
Sie sollten aber besser \verb|\ntvauthor| und \verb|\ntkauthor| verwenden 
(oben S.\,\pageref{ntvauthor}):\hspace{.5em} \verb|\ntkauthor{Meier} XX. => |\ntkauthor{Meier} XX.


\vspace{4ex}\noindent
\fbox{\parbox{.98\textwidth}{\sffamily 
Einfacher ist es sicher, vorgefertigte Textelemente nie separat auszudrucken
und \texttt{\bs ersch} nur ganz ans Ende des letzten Arguments von v\fhy Befehlen zu setzen.}}

\vfill\noindent
{\sffamily Nun folgen Kap.\,\ref{Sect13}, \ref{Sect14} und \ref{Sect15} mit Zusammenstellungen von \BibArts\hy Befehlen.}



\newpage
\section{Die \BibArts\kern .1em\hy Hauptbefehle}\label{Sect13}\label{Hauptbefehle}

\vspace{1ex}
 \begin{tabular}{rrrc}%
  \bf Basis      & \verb|= |\bf addto-Teil & \verb|+ |\bf printonly-Teil & \ \ \ \bf Zusatzf"ullung \\ \hline
                 &                       &                           &                 \\
  \verb|\vli|    & \verb|= \addtovli|    & \verb|+ \printonlyvli|    &                 \\
  \verb|\vqu|    & \verb|= \addtovqu|    & \verb|+ \printonlyvqu|    &                 \\
  \verb|\kli|    & \verb|= \addtokli|    & \verb|+ \printonlykli|    &                 \\
  \verb|\kqu|    & \verb|= \addtokqu|    & \verb|+ \printonlykqu|    &                 \\
                 &                       &                           &                 \\
  \verb|\xvli|   & \verb|= \xaddtovli|   & \verb|+ \xprintonlyvli|   &                 \\
  \verb|\xvqu|   & \verb|= \xaddtovqu|   & \verb|+ \xprintonlyvqu|   &                 \\
  \verb|\xkli|   & \verb|= \xaddtokli|   & \verb|+ \xprintonlykli|   &                 \\
  \verb|\xkqu|   & \verb|= \xaddtokqu|   & \verb|+ \xprintonlykqu|   &                 \\
                 &                       &                           &                 \\
  \verb|\per|    & \verb|= \addtoper|    & \verb|+ \printonlyper|    & \verb|\fillper| \\
  \verb|\arq|    & \verb|= \addtoarq|    & \verb|+ \printonlyarq|    & \verb|\fillarq| \\
                 &                       &                           &                 \\
  \verb|\abkdef| & \verb|= \addtoabkdef| & \verb|+ \printonlyabkdef| &                 \\
  \verb|\defabk| & \verb|= \addtodefabk| & \verb|+ \printonlydefabk| &                 \\
  \verb|\abk|    & \verb|= \addtoabk|    & \verb|+ \printonlyabk|    &                 \\
                 &                       &                           &                 \\
                 & \verb|\addtogrr|      &                           & \verb|\fillgrr| \\
                 & \verb|\addtoprr|      &                           & \verb|\fillprr| \\
                 & \verb|\addtosrr|      &                           & \verb|\fillsrr| \\
 \end{tabular}

\vspace{4.5ex}\noindent 
\textbf{Spielregeln:}

\begin{itemize}\itemsep .1ex
\item k\hy Befehle wie \verb|\kli| und \verb|\kqu| haben zwei Pflichtargumente 
      (Nachname, Kurztitel) und v\fhy Befehle drei (Vorname, Nachname, Rest). 
                        k\hy Belege d"urfen nach Einf"uhrung eines Werks mittels v\hy Beleg verwendet werden;
                        in diesem Fall muss der Kurztitel in `Rest' mittels \verb|\ktit| markiert sein.
\item abk wird erst nach Einf"uhrung mittels abkdef oder defabk verwendet
\item per hat ein Argument und arq zwei (Schriftst"uck und Signatur)
\item fill-Befehle \textit{k"onnen} im zweiten Argument einmal an zentraler Stelle
      Zusatztext f"ur die Liste aufnehmen, um das Stichwort im ersten Argument
      zu erkl"aren. Das erste Argument von \verb|\fillarq| muss die Zeichenfolge\pdfko{.25}\ 
                        des zweiten Arguments eines arq\hy Befehls enthalten (eine Signatur),
                        bei per\fhy, grr\fhy, prr- und srr\hy Befehlen den Text \textit{des} Arguments.
\end{itemize}

\vspace{1ex}\noindent 
Bei abgek"urzten Zeitschriften f"ullt \verb|\abkper| die \verb|\abk|- \textit{und} die \verb|\per|\hy Liste.



\newpage
\section{Schrifteinstellung in \BibArts\kern .1em\hy Argumenten}\label{Sect14}\label{hervor}

\vspace{1ex}\noindent
\begin{tabular}{lll}
\textbf{Befehl}       & \textbf{Voreinstellung}              & \textbf{Alternative}     \\ \hline
                      &                                      &                                            \\
\verb|\authoremph|    & \verb|{\normalfont\scshape}|         & \verb|{\upshape\|\abra{...}\verb|}|        \\[.25ex]
\verb|\kxxemph|       & \verb|{}| \ (\verb|\kli|- und \verb|\kqu|\hy Titel)& alles (S.\,\pageref{kxxB}, \pageref{kxxA}, \pageref{ntsepB}) \\[.25ex]
\verb|\edibidemph|    & \verb|{\scshape}| \ (\textsc{ebd.}, \textsc{ders.}\ko)& \textbf{KEINE!} \\[.25ex]
\verb|\abkemph|       & \verb|{\sffamily}|                   & alles (S.\,\pageref{abkA}, \pageref{abkB}) \\[.25ex]
\verb|\abklistemph|   & \verb|{\bfseries}|                   & alles; \verb|{\abkemph}| \\[.25ex]
\verb|\arqemph|       & \verb|{\normalfont\sffamily}|        & \verb|{\upshape\|\abra{...}\verb|}|        \\[.25ex]
\verb|\arqlistemph|   & \verb|{\arqemph\relax\normalsize}|   & alles (vgl.\ unten)      \\[.25ex]
\verb|\peremph|       & \verb|{\normalfont\scshape}|         & \verb|{\upshape\|\abra{...}\verb|}|        \\[.25ex]
\verb|\perlistemph|   & \verb|{\peremph}|                    & alles (siehe\ unten)     \\[.25ex]
\verb|\xrrlistemph|   & \verb|{}| \ (Register-Stichworte)    & alles (S.\,\pageref{xrr})\\[.25ex]
\verb|\balistnumemph| & \verb|{\sffamily}| \ (Index-Zahlen)  & \verb|{}| (auto-up: S.\,\pageref{listnum}) \\
\end{tabular}\label{arqemph2}%

\vspace{3ex}\noindent
"Anderungen an diesen Befehlen lassen sich mit \verb|\renewcommand|
durchf"uhren.\pdfko{1}\ 
\verb|\authoremph|, \verb|\edibidemph|, \verb|\arqemph| 
und \verb|\peremph| l"asst sich auch etwa\pdfko{1.5}\  
\verb|\bfseries| zuweisen, 
aber nur \textit{nach} \verb|\upshape| oder besser \verb|\normalfont|
(in\pdfko{1.5}\ 
schr"aggestelltem "au"seren Umfeld w"urde \BibArts\ sonst jedes Mal 
warnen).
 
\vspace{.25ex}
Zudem sollten f"ur \verb|\edibidemph| nur \textsc{kleine kapitelle} 
als Basis verwendet
werden, denn nur das umgeht das Problem der 
Klein-/""Gro"sschreibung (\textsc{ebd.} und \textsc{ders.} m"ussen ja 
nicht immer am Anfang eines Satzes stehen)!

\vspace{1.5ex}\noindent
\verb|\arqlistemph| und \verb|\perlistemph| sind so voreingestellt, dass 
sie (im Wesentlichen) die Einstellungen von \verb|\arqemph| und 
\verb|\peremph| f"ur den Listenausdruck "ubernehmen. 
F"ur listemph\hy Befehle gilt "ubertragbar:

\vspace{.5ex}
{\footnotesize
\begin{verbatim}
   {\renewcommand{\perlistemph}{\slshape}%
    Auf den Ausdruck von \per{ShortMagazine} wirkt sich dies nicht aus!
    \renewcommand{\balistnumemph}{}    %% Zahlen nicht in sans serif %%
    \printnumper}
\end{verbatim}}

\vspace{.5ex}
{\renewcommand{\perlistemph}{\slshape}%
 \hspace{.em}Auf den Ausdruck von \per{ShortMagazine} wirkt sich dies nicht aus!
 \renewcommand{\balistnumemph}{}    %% Zahlen nicht in sans serif %%
\vspace{-1.25ex}
 \printnumper}


\newcommand{\mynwarrow}{\hspace{1cm}\raisebox{1.1ex}{{\footnotesize$\nwarrow$}}\hspace{.3em}}%
\newpage
\section{\BibArts\kern .1em\hy Ein\fhy\ko/\ko\ko Ausschalter}\label{Sect15}

\vspace{1ex}\

\hspace*{7.5mm}
\begin{tabular}{ll}
 \bf Voreinstellung "andern\hspace{1cm}  & \bf $\sim$ wiederherstellen\hspace{1.1em}(1/2) \\ \hline
   \multicolumn{2}{l}{} \\[1ex]
         %%
 \verb|   \affixhints|            & \verb|\notaffixhints|            \\
   \multicolumn{2}{l}{\mynwarrow {\footnotesize\sffamily Vor [L] und [Q] keinen Zeilenumbruch erlauben (\ko neu in 2.2)}} \\[3ex]
         %%
 \verb|\notannouncektit|          & \verb|   \announcektit|          \\
   \multicolumn{2}{l}{\mynwarrow {\footnotesize\sffamily v\hy Befehl druckt den sp"ater verwendeten Kurztitel nicht aus}} \\[3ex]
         %%
 \verb|\notbafrontcorr|           & \verb|   \bafrontcorr|           \\
   \multicolumn{2}{l}{\mynwarrow {\footnotesize\sffamily {\normalfont\footnotesize\textit{Italics}\hy}Korrektur am Kopf von \BibArts\hy Befehlen unterlassen}} \\[3ex]
         %%
 \verb|   \baonecolitemdefs|      & \verb|\notbaitemdefs|            \\
   \multicolumn{2}{l}{\mynwarrow {\footnotesize\sffamily \texttt{\bs print...list}\hy Befehle bekommen item\hy Abst"ande wie \texttt{\bs printvli}}} \\[3ex]
         %%
 \verb|   \batwocolitemdefs|      & \verb|\notbaitemdefs|            \\
   \multicolumn{2}{l}{\mynwarrow {\footnotesize\sffamily \texttt{\bs print...list}\hy Befehle bekommen item\hy Abst"ande wie \texttt{\bs printvkc}}} \\[3ex]
         %%
 \verb|   \conferize|             & \verb|\notconferize|             \\    
   \multicolumn{2}{l}{\mynwarrow {\footnotesize\sffamily Verweis vom k\hy Befehl auf Stelle des zugeh"origen v\hy Befehls}} \\[3ex]
         %%
 \verb|   \exponenteditionnumber| & \verb|\notexponenteditionnumber| \\
   \multicolumn{2}{l}{\mynwarrow {\footnotesize\sffamily{\normalfont\footnotesize\texttt{\bs ersch}}\hy Befehl druckt Nummer der {\normalfont\footnotesize\texttt{[}}{\normalfont\footnotesize\textit{Auf"|lage}}{\normalfont\footnotesize\texttt{]}} als Exponent}} \\[3ex]
         %%
 \verb|\nothyko|                  & \verb|   \hyko|             \\
   \multicolumn{2}{l}{\mynwarrow {\footnotesize\sffamily Automatisches 
           {\normalfont\footnotesize\textit{kerning}} nach {\normalfont\footnotesize\texttt{\bs hy}} 
                 und {\normalfont\footnotesize\texttt{\bs fhy}} ausschalten: \hyko\fhy Y\ \nothyko\fhy Y}} \\[3ex]
         %%
 \verb|\notibidemize|             & \verb|   \ibidemize|             \\
   \multicolumn{2}{l}{\mynwarrow {\footnotesize\sffamily Automatisches {\normalfont\footnotesize\textsc{ebd.}}\hy Setzen ausschalten}} \\[3ex]
         %%
 \verb|\notktitaddtok|            & \verb|   \ktitaddtok|            \\
   \multicolumn{2}{l}{\mynwarrow {\footnotesize\sffamily{\normalfont\footnotesize\texttt{\bs ktit}} im v\fhy Befehl erzeugt keinen \kern-.2em{\normalfont\footnotesize\texttt{.vkc}}\hy Eintrag wie ein k\fhy Befehl}} \\[3ex]
         %%
 \verb|\notkurzaddtoarq|          & \verb|   \kurzaddtoarq|            \\
   \multicolumn{2}{l}{\mynwarrow {\footnotesize\sffamily{\normalfont\footnotesize\texttt{\bs kurz}} 
            (\kern-.075em {\normalfont\footnotesize\textit{Vorl"aufer}} von 
                         {\normalfont\footnotesize\texttt{\bs ktit}}) erzeugt keinen 
                         \hspace{-.2em}{\normalfont\footnotesize\texttt{.arq}}\hy Eintrag}} \\%[3ex]
\end{tabular}


\newpage
\section*{\hspace{2em}\BibArts\kern .1em\hy Ein\fhy\ko/\ko\ko Ausschalter}

\vspace{1ex}\

\hspace*{7.5mm}
\begin{tabular}{ll}
 \bf Voreinstellung "andern\hspace{1cm}  & \bf $\sim$ wiederherstellen\hspace{1.1em}\hfill (2/2) \\ \hline
   \multicolumn{2}{l}{} \\[1ex]
         %%
 \verb|\notnegcorrdefabk|         & \verb|   \negcorrdefabk|         \\
   \multicolumn{2}{l}{\mynwarrow {\footnotesize\sffamily Kein negativer Abstand nach Klammer-Auf in Abk"urzungen}} \\[3ex]
         %%
 \verb|\notprinthints|            & \verb|   \printhints|            \\
   \multicolumn{2}{l}{\mynwarrow {\footnotesize\sffamily k\hy Befehle sollen [L]- bzw\ko.\ [Q]\hy Hinweise nicht drucken}} \\[3ex]
         %%
 \verb|\notprintlongpagefolio|    & \verb|   \printlongpagefolio|    \\
   \multicolumn{2}{l}{\mynwarrow {\footnotesize\sffamily Statt `S.' bei {\normalfont\footnotesize\texttt{[num]}} bzw\ko.\ `Bl.' bei {\normalfont\footnotesize\texttt{(num)}} einen Doppelpunkt drucken}} \\[3ex]
         %%
 \verb|\notprintlongpervol|       & \verb|   \printlongpervol|       \\
   \multicolumn{2}{l}{\mynwarrow {\footnotesize\sffamily `Bd.' bei {\normalfont\footnotesize\texttt{\string|n\string|}} bzw\ko.\ `Nr.' bei {\normalfont\footnotesize\texttt{\string_n\string_}} nicht drucken}} \\[3ex]
         %%
 \verb|   \bibsortheads|          & \verb|\notbibsortheads|        \\
   \multicolumn{2}{l}{\mynwarrow {\footnotesize\sffamily Listen: Initialen vor Eintr"agen mit neuem Anfangsbuchstaben}} \\[3ex]
         %%
 \verb|   \bibsortspaces|         & \verb|\notbibsortspaces|       \\
   \multicolumn{2}{l}{\mynwarrow {\footnotesize\sffamily Listen: Abstand zw\ko.\ Eintr"agen mit versch.\ Anfangsbuchstaben}} \\[3ex]
         %%
 \verb|   \showbacorr|            & \verb|\notshowbacorr|            \\
   \multicolumn{2}{l}{\mynwarrow {\footnotesize\sffamily Stelle mit \BibArts{\normalfont\footnotesize\hy\ko\textit{italics}\hy}Korrektur im Ausdruck markieren}} \\[3ex]
         %%
 \verb|   \showbamem|             & \verb|\notshowbamem|             \\
   \multicolumn{2}{l}{\mynwarrow {\footnotesize\sffamily \BibArts\hy Zwischenspeicher auf Bildschirm drucken (\ko{\normalfont\footnotesize\textsc{ebd.}}\hy Setzung)}} \\[3ex]
         %%
 \verb|\notwarnsamename|          & \verb|   \warnsamename|          \\
   \multicolumn{2}{l}{\mynwarrow {\footnotesize\sffamily Bildschirmwarnung bei Wiederholung von Autornachnamen aus}} \\[3ex]
         %%
 \verb|   \writeidemwarnings|     & \verb|\notwriteidemwarnings|     \\
   \multicolumn{2}{l}{\mynwarrow {\footnotesize\sffamily{\normalfont\footnotesize\textsc{ders.}}\hy Setzung im Ausdruck testhalber mit {\small$\bullet\heartsuit\nabla\spadesuit\clubsuit$} markieren}} \\
         %%
\end{tabular}


\newpage
\section{\BibArts\kern .1em\hy\hspace{-.025em}1.3\hspace{.075em}\hy Texte unter \BibArts~2.x}\label{Sect16}\label{compabil}

\BibArts~2.x hat so viele Neuerungen, dass ein Text in Version~1.3
vor der \LaTeX\hy Bearbeitung "uberarbeitet werden m"usste. An
den Befehlen \verb|\schrift| (f"ur ganze v\fhy Befehle), 
\verb|\barschrift| und \verb|\indschrift| mit 
\verb|\renewcommand| ansetzende "Anderungen sind heute
\textit{wirkungslos}.\footnote{\texttt{\bs frompagesep} 
(oben S.\,\pageref{frompagesep}) ersetzt zudem \texttt{\bs verw}; \kern 1pt und 
\texttt{\bs ntsep} (S.\,\pageref{ntsepA}) \texttt{\bs punctuation}\hspace{.05em}.\kern1pt} 
Lesen Sie \verb|README.txt|.

\vspace{.05ex}%
\textbf{Behalten Sie zur "Ubersetzung alter \BibArts\hy Texte
die Programmdateien Ihrer \BibArts\kern .1em\hy 1.x\hy Version zur"uck!} 
... Falls Sie dies vers"aumten:

\vspace{.075ex}%
\BibArts~1.3 hatte keine automatische \textsc{ebd.}\hy Setzung. 
Dort konnte \verb|\kurz|\pdfko{1}\ 
ganz am Ende des letzten Arguments 
eines v\fhy Befehls stehen; es druckte sein Argument nach 
\textsf{im folgenden} (und in v\fhy Listen in eckigen Klammern)
einfach\pdfko{.25}\  
aus. \BibArts~2.2 erkennt alte 
\hspace*{-.15em}\texttt{.tex}\hy Dateien und startet eine 
Emulation.\footnote{Wird vom alten Vorspannbefehl 
\texttt{\bs makebar} eingeschaltet (stehen lassen!) und redefiniert auch 
\hspace{-.1em}\texttt{\bs printvli}, das in 1.3 keine "Uberschrift
druckte. Sonst wird \texttt{\bs makebar} nicht\pdfko{1}\
mehr ben"otigt. Es gibt kein \hspace*{-.15em}\texttt{.bar}\hy File mehr: 
\BibArts\ nutzt nun \hspace{-.25em}\texttt{.aux}\hy Files 
(\textbf{dazu Kap.\,\ref{bibsort}}).}
Kopien der Argumente von \verb|\kurz| sowie der alten 
\verb|\bib|\hy Befehle gehen heute\pdfko{1.125}\  
ins \hspace*{-.15em}\texttt{.arq}\hy 
Verzeichnis, das es in \BibArts~1.3 nicht gab; eine 
\verb|\printind|\hy Emulation druckt alles aus $-$ nach Bearbeitung 
mit \texttt{bibsort}. F"ur \textsc{MakeIndex}\pdfko{1.25}\  
gedachte Steuerzeichen werden jetzt also ausgedruckt 
(\kern -.05em vgl.\ unten S.\,\pageref{subitem})!

\vspace{-.625ex}
\Doppelbox
{      \bs notkurzaddtoarq \% (jetzt nicht)
    \\ Fast wie 1.3: \bs vli\{Norbert\} 
    \\ \ \{Schwarz\} \b{\{}Einf"uhrung in 
    \\ \ \ \ \bs protect\bs TeX, Bonn
    \\ \ \ \ 1988 \bs kurz\{Schwarz\}\b{\}}
}
{\notkurzaddtoarq
  \texttt{ \footnotesize \%\% \bs kurz druckt in \bs arqemph \%\%} \\[.2ex]
    Fast wie 1.3: \printonlyvli{Norbert} 
                {Schwarz} {Einf"uhrung in 
                \protect\TeX, Bonn
    1988 \kurz{Schwarz}}
}

\vspace{-.125ex}\noindent
Einige Befehle sind auch in 2.2\hy Texten
brauchbar. \verb|\stressing{underline}| stellt wie in 1.3 die 
Autorenhervorhebung ein und ist heute Alternative f"ur 
\verb|\renewcommand{\authoremph}{\upshape\underline}|. Auch 
das Paar\pdfko{1.125}\  
\verb|\bibmark| und \verb|\bibref| existiert weiter 
(die x\fhy Befehle sind nun unn"otig):

\vspace{-.875ex}
\Doppelbox
{
 Text.\bs footnote\b{\{}Albert Lecl\bs\string`erc: 
        \\[.25ex] \ \ Der Sommerregen, Paris 1985 
        \\[.29ex] \ \ (\bs bibmark\{Lecl\bs\string`erc\}).\b{\}}
        \\[.47ex] Schon Lecl\bs\string`erc wollte 
        \\[.25ex] freie Eingaben.\bs footnote\b{\b{\{}}
        \\[.25ex] \ \ \bs bibref\{\bs scshape Lecl\bs\string`erc\}.\b{\b{\}}}
}
{
 Text.\footnote{Albert Lecl\`erc: Der Sommerregen, Paris 1985 (\bibmark{Lecl\`erc}).}
        Schon Lecl\`erc wollte freie Eingaben.\footnote{
        \bibref{\scshape Lecl\`erc}.}
}

\vspace{-.325ex}\noindent
\verb|\bibref| passt sich an, wenn \verb|\bibmark| in 
keiner Fu"snote war. Neu sind dazu \textit{captions} 
\verb|\gbibmarkname| (`{im Folgenden: }'), 
\verb|\fbibmarkname| (`{par la suite: }')
und \verb|\ebibmarkname| (`{cited as: }'),
deren Definitionen mit Leerzeichen enden.



\newpage
\section{Listenausdruck (\BibArts\kern .1em\hy Belegapparat)}\label{Sect17}

Wie die von \verb|bibsort| erzeugten Dateien (\kern -.05em vgl.\ unten 
ab S.\,\pageref{bibsort}) auszudrucken
sind, wurde in den jeweiligen Kapiteln bereits fallweise abgehandelt:
\verb|bibarts.sty| stellt dazu print- und printnum\hy Befehle bereit $-$ wobei
die\pdfko{1.875}\ 
print\hy Befehle die Zug"ange als Liste und die printnum\hy Befehle
zus"atzlich hinter jeden Listenpunkt die Zugangsstellen indexartig drucken.
Bei beiden Befehlsklassen enth"alt das Befehlswort zum Ausdruck der
jeweiligen Liste dieselben drei Buchstaben, die auch der Befehl zum
F"ullen der Liste aufweist. Auch das Dateinamen\hy Suffix der von
\verb|bibsort| erzeugten Liste hat diese Zeichen: Eintr"age des 
\BibArts\hy Befehls \verb|\vli| kommen in eine Datei \verb|.vli|, 
die Sie\pdfko{1}\  
mit \verb|\printvli| oder \verb|\printnumvli| im Anhang 
Ihres Textes ausdrucken k"onnen. Entsprechendes gilt f"ur \verb|\vqu|, 
\verb|\arq| und \verb|\per|. Ausnahme ist das Kurzzitateverzeichnis 
\verb|.vkc|, das die Zug"ange der \verb|\kli|- \textit{und} 
\verb|\kqu|\hy Eintr"age erh"alt (sowie der Zug"ange, die \BibArts\ 
aus den Argumenten von \verb|\ktit| und den Nachnamensargumenten 
der v\fhy Befehle \textit{automatisch} erzeugt); das Kurzzitateverzeichnis
wird mit \verb|\printvkc| oder \verb|\printnumvkc| ausgedruckt. Und 
f"ur das\pdfko{.75}\ 
Abk"urzungsverzeichnis, das mit \verb|\printabk| oder 
\verb|\printnumabk| ausgedruckt\pdfko{.5}\ 
wird, bef"ullen die \BibArts\hy Befehle 
\verb|\abkdef| oder \verb|\defabk| eine von \verb|bibsort| erzeugte 
Datei \verb|.abk|; f"ur so eingef"uhrte Abk"urzungen liefern 
\verb|\abk|\hy Befehle weitere Seitenzahlen und ggf.\ Fu"snotennummern, 
die \verb|\printnumabk| ausdruckt. 

F"ur alle diese Listen liest \verb|bibsort| das\slash die \verb|.aux|\hy 
File{\small(}s{\small)} Ihres \LaTeX\hy Textes ein und erzeugt daraus 
die genannten Dateien. Das Namens\hy Pr"afix ist dasjenige des \LaTeX\hy 
Haupttextes (die Literaturliste \textit{hier} ist \verb|bibarts.vli|). 

Das Orts\fhy, Personen und Sachregister wird jeweils nur mit addto\hy 
Befehlen bef"ullt, etwa \verb|\addtogrr|. Das sind Befehle, die nichts 
an Ort und Stelle\pdfko{.75}\ 
drucken. Verwechseln Sie die print- und printnum\hy 
Befehle nicht mit Befehlen wie \verb|\printonlyvli|, die \textit{nur} 
an Ort und Stelle drucken (vgl.\ oben S.\,\pageref{printonly}). 

Listen werden defaultm"a"sig unter den in Kapitel~\ref{SprachSep}
genannten "Uberschriften ausgedruckt: \verb|\printvli|
druckt unter \verb|\bacaptionsgerman| den Text \verb|\gvlititlename|.
$-$ \verb|\print|...\verb|list|\hy Befehle drucken Listen ohne "Uberschrift.

Die drei Register sowie das Abk"urzungs\hy\ und das Kurzzitateverzeichnis 
werden defaultm"a"sig in fixer Schriftgr"o"se \textit{und} zweispaltig
gedruckt. Beides ist nicht der Fall bei 
\verb|\printvli|, \verb|\printvqu|, \verb|\printper| und 
\verb|\printarq| samt ihren \verb|num|\hy Varianten sowie den 
\verb|\print|...\verb|list|\hy Befehlen (wie \verb|\printabklist|).

Der Ausdruck von "Uberschrift und Liste 
l"asst sich immer trennen. Die vli\hy "Uberschrift etwa k"onnen Sie mit 
\verb|\printvlititle| drucken. Wie auch bei\pdfko{.75}\ 
\verb|\printvli| oder \verb|\printnumvli| kommt der 
Titel ohne Kapitelnummer ins Inhaltsverzeichnis. Alternativ 
k"onnen Sie etwa \verb|\subsection{|\kern -.1em\textit{"Uberschrift}\verb|}| 
tippen, falls Sie dort Kapitelnummern haben wollen.
Die Liste l"asst sich darunter in beiden F"allen mit \verb|\printvlilist| 
oder \verb|\printnumvlilist| ausdrucken. 


\noindent
\verb|\printbibtitle| ist "Uberschrift f"ur den gesamten Belegapparat, 
defaultm"a"sig in section\hy Gr"o"se. Die anderen title\hy Befehle 
verwenden eine Gr"o"se kleiner:

\vspace{1.25ex}{\small\noindent
\verb|    |\hbox to 14em{\textit{Beide drucken "Uberschrift}\hfill}%
\verb|   |\hbox to 8.6em{\textit{mit Text im dt.}\hfill}%
\verb|   |\hbox to 6em{\textit{Default}\hfill} \\[-.25ex]
\verb|             \printbibtitle  =>  \gbibtitlename  =>  section| \\[-.5ex]
\verb|   \printvli \printvlititle  =>  \gvlititlename  =>  subsection| \\[-.5ex]
\verb|   \printvqu \printvqutitle  =>  \gvqutitlename  =>  subsection| \\[-.5ex]
\verb|   \printabk \printabktitle  =>  \gabktitlename  =>  subsection| \\[-.5ex]
\verb|   \printper \printpertitle  =>  \gpertitlename  =>  subsection| \\[-.5ex]
\verb|   \printarq \printarqtitle  =>  \garqtitlename  =>  subsection| \\[-.5ex]
\verb|   \printvkc \printvkctitle  =>  \gvkctitlename  =>  subsection| \\[-.5ex]
\verb|   \printgrr \printgrrtitle  =>  \ggrrtitlename  =>  subsection| \\[-.5ex]
\verb|   \printprr \printprrtitle  =>  \gprrtitlename  =>  subsection| \\[-.5ex]
\verb|   \printsrr \printsrrtitle  =>  \gsrrtitlename  =>  subsection|}

\vspace{1.25ex}\noindent
Hinter print\fhy, printnum\hy\ und title\hy Befehlen kann ein 
optionales Argument die "Uberschriftengr"o"se "andern, 
\verb|\printvli[section]| etwa (einzusetzen ist ein
"Uberschriftenbefehl ohne \textit{backslash}). Bei Befehlen, die
Listen zweispaltig drucken, ist \verb|[chapter]| 
verboten. Die "Uberschrift kommt stets ins Inhaltsverzeichnis 
(\verb|\tableofcontents|) und unter \verb|\pagestyle{headings}| 
zudem in die Kopfzeile; Befehle, die zwei Spalten anordnen, setzen 
die Anfangsseite \texttt{plain}. \hspace{.1em}Die list\hy 
Befehle setzen nichts in Inhaltsverzeichnis oder Kopfzeile.

\vspace{1ex}\noindent
Die normalen print\hy Befehle drucken Listen unter einer passenden "Uberschrift; die
printnum\hy Befehle arbeiten "aquivalent (vgl.\ die Auf"|listung unten S.\,\pageref{Auflistung}):

\vspace{.75ex}{\small\noindent
\verb|   \printvli  =  \printvlititle + \printvlilist |in \textit{Umfeldschrift} \\[-.5ex]
\verb|   \printvqu  =  \printvqutitle + \printvqulist |in \textit{Umfeldschrift} \\[-.5ex]
\verb|   \printabk  =  \printabktitle + \printabklist |in \verb|\twocolumn| und \\[-.7ex]
\verb|                                                     \footnotesize| \\[-.2ex]
\verb|   \printper  =  \printpertitle + \printperlist |in \textit{Umfeldschrift} \\[-.5ex]
\verb|   \printarq  =  \printarqtitle + \printarqlist |in \textit{Umfeldschrift} \\[-.5ex]
\verb|   \printvkc  =  \printvkctitle + \printvkclist |in \verb|\twocolumn\small| \\[-.5ex]
\verb|   \printgrr  =  \printgrrtitle + \printgrrlist |in \verb|\twocolumn\small| \\[-.5ex]
\verb|   \printprr  =  \printprrtitle + \printprrlist |in \verb|\twocolumn\small| \\[-.5ex]
\verb|   \printsrr  =  \printsrrtitle + \printsrrlist |in \verb|\twocolumn\small|}


\vspace{2ex}\noindent
Auf diese Weise w"urde der \BibArts\hy Anhang unter der nummerierten
"Uberschrift \textbf{A~~Belegapparat und Register} gedruckt:\footnote{
...\,\texttt{\bs pagestyle\{headings\}}\,\textit{"Uberschrift} 
\texttt{\bs pagestyle\{myheadings\}}\,...\,\texttt{\bs end\{appendix\}} \\
druckt die Kopfzeile von \textit{"Uberschrift} im \textit{ganzen} Appendix
(dann kein \texttt{\bs markboth} setzen!).\hspace*{.2em}}

\vspace{-1.25ex}
{\footnotesize
\begin{verbatim}
   \clearpage \begin{appendix} \pagestyle{headings}
   \section{Belegapparat und Register}\thispagestyle{plain}\vspace{7mm}
   {\small \printarq \newpage \printvqu \printvli \newpage}\printnumgrr
   \end{appendix} %%Einspaltige Bereiche enden mit \newpage (Kopfzeile!)
\end{verbatim}}

\vspace{-1ex}\vfill\noindent
Im folgenden Beispiel wird das Kurzzitateverzeichnis einspaltig ausgedruckt:

\newpage
\noindent{\small
\verb| \clearpage {\pagestyle{headings}\small \printbibtitle \printvqu| \\
\verb| \printvli \printvkctitle\baonecolitemdefs\printnumvkclist \newpage}|}

\vspace*{1.5ex}
                 {\pagestyle{headings}\small \printbibtitle \printvqu
      \printvli \printvkctitle\baonecolitemdefs\printnumvkclist \newpage}



\twocolumn[\subsubsection*{Auf"|listung der print-, printnum-, title-, list- und num...list-Befehle}
Hier \texttt{[}\kern-.05em\textit{OptArg}\texttt{]}'s f"ur "Uberschriften, die eine Stufe gr"o"ser als der Default sind. Die jeweils ersten zwei Befehle sind in den title- und einen list-Befehl teilbar. \\ 
\strut \\
\texttt{\bs printbibtitle\lbrack chapter\rbrack} \hfill
{\footnotesize Dokumentenklasse \texttt{\{report\}}\ }\vspace{1.75ex}]\label{Auflistung}%
\noindent
%%
\textbf{\gvlititlename} \\[0.25ex]
\verb|\printvli[section]| \\
\verb|\printnumvli[section]| \\[0.5ex]
\verb|\printvlititle[section]| \\
\verb|\printvlilist| \\
\verb|\printnumvlilist| \\[1.75ex]
%%
\textbf{\gvqutitlename} \\[0.25ex]
\verb|\printvqu[section]| \\
\verb|\printnumvqu[section]| \\[0.5ex]
\verb|\printvqutitle[section]| \\
\verb|\printvqulist| \\
\verb|\printnumvqulist| \\[1.75ex]
%%
\textbf{\gvkctitlename} \\[0.25ex]
\verb|\printvkc[section]|~$^{(t, s)}$ \\
\verb|\printnumvkc[section]|~$^{(t, s)}$ \\[0.5ex]
\verb|\printvkctitle[section]| \\
\verb|\printvkclist| \\
\verb|\printnumvkclist| \\[1.75ex]
%%
\textbf{\gpertitlename} \\[0.25ex]
\verb|\printper[section]| \\
\verb|\printnumper[section]| \\[0.5ex]
\verb|\printpertitle[section]| \\
\verb|\printperlist| \\
\verb|\printnumperlist| \\[1.75ex]
%%
\textbf{\garqtitlename} \\[0.25ex]
\verb|\printarq[section]| \\
\verb|\printnumarq[section]| \\[0.5ex]
\verb|\printarqtitle[section]| \\
\verb|\printarqlist| \\
\verb|\printnumarqlist| \\[1.75ex]
%%
\textbf{\gabktitlename} \\[0.25ex]
\verb|\printabk[section]|~$^{(t, f)}$ \\
\verb|\printnumabk[section]|~$^{(t, f)}$ \\[0.5ex]
\verb|\printabktitle[section]| \\
\verb|\printabklist| \\
\verb|\printnumabklist| \\[1.75ex]
%%
\textbf{\ggrrtitlename} \\[0.25ex]
\verb|\printgrr[section]|~$^{(t, s)}$ \\
\verb|\printnumgrr[section]|~$^{(t, s)}$ \\[0.5ex]
\verb|\printgrrtitle[section]| \\
\verb|\printgrrlist| \\
\verb|\printnumgrrlist| \\[1.75ex]
%%
\textbf{\gprrtitlename} \\[0.25ex]
\verb|\printprr[section]|~$^{(t, s)}$ \\
\verb|\printnumprr[section]|~$^{(t, s)}$ \\[0.5ex]
\verb|\printprrtitle[section]| \\
\verb|\printprrlist| \\
\verb|\printnumprrlist| \\[1.75ex]
%%
\textbf{\gsrrtitlename} \\[0.25ex]
\verb|\printsrr[section]|~$^{(t, s)}$ \\
\verb|\printnumsrr[section]|~$^{(t, s)}$ \\[0.5ex]
\verb|\printsrrtitle[section]| \\
\verb|\printsrrlist| \\
\verb|\printnumsrrlist|\\[1.75ex]
%%
\textbf{Legende} \\[0.25ex]
{\footnotesize
\hbox to 2em{$^{(t, f)}$\hfill} \verb|\twocolumn \footnotesize| \\[-.5ex]
\hbox to 2em{$^{(t, s)}$\hfill} \verb|\twocolumn \small| \\[-.25ex]
Befehle, die \verb|[|\textit{Arg}\verb|]| annehmen, erzeugen \\[-.5ex]
$-$ "Uberschrift in \textit{Default}\fhy/\kern-.1em\textit{Arg}\fhy Gr"o"se \\[-.65ex]
$-$ Inhaltsverzeichnis-Eintrag (\kern-.05em\textit{dito}) \\[-.65ex]
$-$ Kopfzeilen-Eintrag unter \texttt{headings}}
\onecolumn


\newpage\noindent
\verb|bibsort| bereitet f"ur den Ausdruck der Listen vor, 
den Wechsel von Eintr"agen mit unterschiedlichen Anfangsbuchstaben zu betonen.
Es gibt vergr"o"serte Abst"ande und Buchstaben: 
\verb|{\bibsortspaces\printnumvkc}| und\pdfko{1.5}\ 
\verb|{\bibsortheads\printnumvkc}| erg"aben tats"achlich jeweils eigenen Seiten:

\vspace{-1ex}\noindent\hspace{1em}%
\parbox[t]{.45\textwidth}
{\subsection*{\gvkctitlename}\vspace{\batwocoltopskip}\bibsortspaces\batwocolitemdefs\small\printnumvkclist}
\hspace{.5em}%
{\makeatletter\def\@baitemdefs{\parsep 0pt \itemsep 0pt \parskip 0pt \lineskip 0pt \rightskip 1cm minus 1cm}\makeatother
\parbox[t]{.45\textwidth}
{\subsection*{\gvkctitlename}\vspace{\batwocoltopskip}\bibsortheads\small\printnumvkclist}}%
\label{head}%

\vfill\noindent
print\hy Befehle, die \textit{selbst} zweispaltig drucken,
setzen strikte Vorgaben f"ur Abst"ande um, etwa 
\verb|\itemsep| \verb|0pt|. F"ur andere list\hy Befehle
gelten nur die Vorgaben der \verb|{description}|\hy Liste. 
Um dann Abst"ande, die f"ur zweispaltigen Ausdruck gelten, einzustellen,
kann \verb|\batwocolitemdefs| gesetzt werden.

\vspace{1ex}\noindent
F"ur list\hy Befehle in einspaltigem Umfeld dient 
\verb|\baonecolitemdefs|, das viel weniger Vorgaben
macht und Spielr"aume l"asst. Eigene Definitionen
legt etwa 
\verb|{\bamyitemdefs{\rightskip| \verb|1cm minus| \verb|1cm}\printvkclist}|
fest. Alle f"ur \verb|\print|...\verb|list| oder 
\verb|\printnum|...\verb|list| eventuell gemachten itemdef\hy Vorgaben 
schaltet \verb|\notbaitemdefs| aus (es stellt die
Voreinstellung wieder her).

\vspace{1.5ex}\noindent
\verb|\printvkclist| l"asst sich \textit{mit Zus"atzen} 
genauso wie \verb|\printvkc| ausdrucken
(weil die {\small\texttt{[}...\texttt{]}} abschirmen, w"are 
eine Kopfzeile danach nochmal zu definieren):

\vspace{1.ex}{\small\noindent
\verb|   \twocolumn[\printvkctitle\vspace{\batwocoltopskip}] %\markboth|...\\
\verb|   {\small\bibsortheads \batwocolitemdefs\printvkclist}\onecolumn|}

\vspace{1.25ex}\noindent
Unter \verb|\bibsortspaces| bzw.\ \verb|\bibsortheads| wird
\verb|\batwocoltopskip| eigenst"andig von print\hy Befehlen, die 
zweispaltigen Druck anordnen, ausgef"uhrt. 
Falls \verb|\twocolumn| und \verb|\bibsortheads| gleichzeitig gelten, 
setzen list\hy Befehle \verb|\batwocolitemdefs| selbst"andig (dann kann 
lassen sich die Zeilenabst"ande nur noch z.\,B. durch \verb|\renewcommand{\baselinestretch}{1.1}| 
"andern).

\vspace{1.5ex}\noindent
\verb|\bibsortspaces| und \verb|\bibsortheads| schalten sich 
gegenseitig ab: Automatisch gilt also immer nur eines von beiden. 
Zudem l"ost \verb|\notbibsortheads| auch \verb|\notbibsortspaces| 
aus $-$ und umgekehrt.

\vspace{1.5ex}\noindent
Beim Archivquellenverzeichnis kann ein Konflikt auftreten:
Sie sollten sich entscheiden, ob Sie \verb|\bibsortspaces| 
bzw.\ \hspace{-.05em}\verb|\bibsortheads| aktivieren
m"ochten, \textit{oder} \verb|\arqsection|,
\verb|\arqsubsection| und \verb|\arqsubsubsection| nutzen.
Und nur im Archivquellenverzeichnis k"onnen Sie \textit{innerhalb 
einer Liste} in den Seitenumbruch eingreifen. Umbruchbefehle
wie \verb|\newpage| k"onnen direkt nach \verb|\arqsection|, 
\verb|\arqsubsection| oder \verb|\arqsubsubsection| optional 
"ubergeben werden:\label{newpage} 
\verb|\arqsection[\newpage]{BA}{Bundesarchiv}| \hspace{.1em}ist ein Beispiel.
Zerbrechliche Befehle sollten mit \verb|\protect| gesch"utzt werden.
Befehle mit Argumenten in eckigen Klammern wie \verb|\rule[2ex]{1cm}{1cm}| lassen
sich so einsetzen: 
...\verb|[\protect\rule\lbrack 2ex\rbrack{1cm}{1cm}]|...\footnote{Ist 
eine arq\hy "Uberschrift der allererste Listeneintrag, dann sind \textit{nur}
Abstandsbefehle wie etwa \texttt{\bs vspace} im optionalen Argument 
erlaubt, denn \texttt{bibsort} setzt die optionalen Argumente innerhalb 
des \texttt{.arq}\hy Files in eine Zeile \textit{vor} die arq\hy section; 
in einer \LaTeX\hy Liste darf jedoch zu druckender Text oder 
\texttt{\bs rule} erst \textit{nach} dem ersten \texttt{\bs item}\hy Befehl stehen.} 
Einfach l"asst sich \textit{zus"atzlicher Abstand zum vorausgehenden 
Listenpunkt} etwa mit \verb|\arqsection| \verb|[\vspace{2ex}]| \verb|{BA}{Bundesarchiv}| 
einstellen.\footnote{\texttt{\bs arqsectionbegin}, 
\texttt{\bs arqsubsectionbegin} und \texttt{\bs arqsubsubsectionbegin} 
legen den Basisabstand fest; sie werden von \texttt{[}...\texttt{]} 
nicht "uberschrieben, sondern \textit{danach} ausgef"uhrt. Diese Befehle sind 
untereinander austariert; Anf"anger sollten sie unver"andert lassen.}  



\newpage
\section{\texttt{bibsort} samt Neuerungen seit Version~2.0}\label{Sect18}\label{bibsort}

\verb|bibsort| ist das Sortierprogramm von \BibArts\kern.1em. Eine
Datei \verb|bibsort.exe| liegt dem Paket neben \verb|bibarts.sty| bei. 
Beide zusammen sollen den Anhang Ihres \LaTeX\hy Textes erzeugen. 
Anwender, bei denen \verb|bibsort.exe| nicht startet, sollten  
\verb|bibsort.c| mit einem f"ur ihr Betriebssystem
geeigneten C\fhy Compiler selbst in eine Bin"ardatei "ubersetzen und 
dann diese einsetzen. Der Quellcode von \verb|bibsort.c| setzt kein 
bestimmtes Betriebssystem voraus (ANSI~C).\footnote{F"ur GNU\hy Software habe ich
das getestet: Auch\hspace{.3em} \texttt{gcc -c bibsort.c -o bibsort.o}\hspace{.3em}
und\hspace{.3em} \texttt{gcc bibsort.o -o bibsort.exe}\hspace{.3em} ergaben bei mir eine 
brauchbare Bin"ardatei. Bei mir machte der \texttt{Dev-Cpp\_5.4.0}\kern .1em\hy Editor
Schwierigkeiten, der \texttt{gcc} pers"onlich jedoch nie.}

\BibArts\ kommt heute (Version 2.x) ohne \textsc{MakeIndex} aus.  
\verb|bibsort| verarbeitet keine Steuerzeichen, hat kein 
Maskierungszeichen f"ur Steuerzeichen und kein
Steuerfile.\footnote{\BibArts\,1.3 brauchte \textsc{MakeIndex}, um einen
Belegstellenindex zu erzeugen (siehe S.\,\pageref{compabil}).}
Anders als \textsc{MakeIndex} erzeugt es keine 
\verb|\subitem|\kern.05em s.\label{subitem}% 

Ganz oben wurde bereits erkl"art, wie \verb|bibsort.exe| zusammen mit
speziellen \LaTeX\hy Editoren benutzt wird. Ansonsten kann es durch Antippen
von \verb|bibsort| plus Dateinamens\hy Pr"afix und Optionen in der
Eingabeauf"|forderung des Betriebssystems gestartet werden. Bei 
\LaTeX\hy Texten, die aus mehreren Dateien bestehen, ist das Namenspr"afix der Hauptdatei 
anzutippen. \verb|bibsort| liest die zugeh"orige \hspace{-.15em}\texttt{.aux}\hy Datei ein; 
die ggf.\ enthaltene \verb|\@include|\hy Liste wird abgearbeitet, sodass auch 
bei sequenzieller "Ubersetzung eines \LaTeX\hy Textes (wenn \verb|\includeonly|
nicht alle Dateien nennt) f"ur den \BibArts\hy Anhang immer vollst"andige Listen 
erzeugt werden. F"ur \verb|bibarts.tex| hier ist 
\verb|bibarts| das Namenspr"afix. Die aus
\verb|bibarts.aux| erzeugte Datei \verb|bibarts.vli| enth"alt die
Literaturliste, \verb|bibarts.abk| das Abk"urzungsverzeichnis, etc.

\verb|bibsort| liest aus einer \hspace{-.15em}\texttt{.aux}\hy Datei nur die Zeilen ein, die
mit \BibArts\ erzeugt wurden. In einem ersten Schritt sortiert es 
diese Zeilen klein\fhy\slash gro"s\hy schreibungs\hy invariant. Das Programm sortiert 
\textit{in Grundeinstellung} "a, "o und~"u als a, o und u; weiter sind \verb|\ss| und 
\verb|\3| (sowie \verb|"s|, falls \verb|"| \textit{aktiv} ist\footnote{\BibArts\
"ubergibt jedem Eintrag ins \hspace{-.15em}\texttt{.aux}\hy File den an der
entsprechenden Stelle g"ultigen \textit{catcode} von \string" und die
zur Worttrennung eingestellte Sprache. Beides wird beim Ausdruck
des entsprechenden Listenpunkts reproduziert (vgl.\ oben S.\,\pageref{hyphenation}\,f. und 
unten S.\,\pageref{dqakt}, Anm.\,\ref{dqakt}).
Dies bestimmt f"ur die Listen Ausdruck und 
Sortierreihenfolge z.\,B. von~\texttt{\string"a}\kern.1em.}) gleich~\verb|s|. 
Wird \texttt{bibsort} die Option \ko\verb|-g2|~(Wortliste) "ubergeben,
sortiert es "s als \verb|ss|; mit \ko\verb|-g1|~(Namensliste) 
gelten \textit{zudem} die Umlaute als \verb|ae|, \verb|oe| und \verb|ue| 
(letzteres\pdfko{.25}\ entspricht DIN\,5007\fhy 2). Zahlen werden \textit{in Grundeinstellung} 
vor Buchstaben sortiert; mit \ko\verb|-g1| oder \ko\verb|-g2| 
ist es umgekehrt. Nur, wenn Zeilen anhand der enthaltenen Zahlen sowie 
gro"s\fhy\slash klein\hy invarianten Buchstaben keinen Unterschied gegen"uber
anderen Zeilen aufweisen, wird die Gro"s\fhy\slash Klein\hy Schreibung beachtet,
danach etwaige Akzente auf den Buchstaben, dann weitere Zeichen.

Etliche Einstellungen sind nicht ver"anderbar. Etwa wird \verb|\o|~(\o) im 
ersten Schritt immer als \verb|o| einsortiert, dann vor allen \verb|o|'s mit 
`aufgesetzten' Akzenten. Ebenso nicht\hy einstellbar ist, dass 
\verb|$\alpha$| als \verb|a| einsortiert wird. Um im Einzelfall abweichende 
Sortierreihenfolgen zu erzwingen, k"onnen Sie den \BibArts\hy Befehl 
\verb|\sort{|\textit{Zeichenfolge}\verb|}| \balabel{sort} verwenden (dessen Argument
sortiert, aber nicht ausgedruckt wird). Die meisten weiteren \LaTeX\hy Befehle 
ignoriert\pdfko{1}\ 
\verb|bibsort| einfach. Bei anderen wie etwa \verb|\parbox| werden 
$-$\,in der Routine\pdfko{1}\ 
\verb|transformtable| in \verb|bibsort.c| definiert\,$-$ 
die L"angen\hy\ und Positionierungs\hy Angaben ignoriert. 
\verb|\diskretionary{A}{B}{C}| wird wie \verb|C| sortiert und $-$ neu in Version 2.2 $-$ \verb|"ff| 
mit \verb|-g1| oder \verb|-g2| wie \verb|ff| statt wie \texttt{\"ff} (vgl.\ S.\,\pageref{ff}). 
In \verb|\protect\pageref{|\textit{xx}\verb|}| wird 
\textit{xx} nicht gewichtet (suche \verb|\pageref{X1}|).

\vspace{.75ex}\noindent
Nur \verb|bibsort| in der Eingabeauf"|forderung getippt druckt auf den Bildschirm: 

{\scriptsize
\catcode`\|=\active
\def|{\hspace{-.25em}\hbox to 1em{$\sim$\hfill}\hspace{-.25em}}%
\begin{verbatim}
   %%>  This is bibsort 2.2  (for help:  bibsort -?)
   %%      bibsort 2.2 is part of BibArts 2.2    (C) Timo Baumann  2019.
   %%   I read a LaTeX .aux file (follow \@input), and I write my output in files
   %%     .vli  Literature     .vqu  Published sources     .grr  Geographic index
   %%     .vkc  Short titles   .arq  Unpublished sources   .prr  Person index
   %%     .per  Periodicals    .abk  Abbreviations         .srr  Subject index
   %%
   %%  bibsort <LaTeXFile> [-o <OutFile>] [-g1 [OR] -g2] [<OtherOptions>]
   %%
   %%         DefaultSort:  0, .., 9, (A a), (B b), .., (s \ss S), .., (Z z)  and
   %%         (a [\.\'\`\^\"\=\~]a \aa=\r{a} [\b\c\k\d\H\t\u\v]{a} \ae),  b, ..
   %%   -g<n> GermanSort:   (a A), .., (z Z), 0, .., 9;   and (\" or active "):
   %%         -g1  GermanTelefonebookStyle:  "a = \"a = ae, ..., "s = \ss = ss;
   %%         -g2  ModernGermanDictionary:   "a = \"a = a,  ..., "s = \ss = ss.
   %%     -x  DoNotExpect german.sty: Active "-characters do NOT produce umlauts.
   %%     -k  Idemize | multiple used authors in the .vli and .vqu lists (kill).
   %%     -d  \include{ / is \ } (dos).   -h  Sort a hyphen as a space.
   %%   -i=j  Sort 'i' as 'j'.            -p  Sort "P.S." before "Peter" (point).
   %%     -l  Ignore spaces (leer).       -c  Don't set bad page/fnt nums in {}.
   %%     -r  Typeout license (read).    -??  Further options.
   %%
   %%>  I give up my job, because I get no <FileNamePrefix> of an auxiliary file.
\end{verbatim}}


\noindent
Dabei ist die ganze Serie der neun Hilfsdateien aufgelistet, die \verb|bibsort| 
erzeugen kann $-$ \textit{und l"oschen darf}, falls keine Eintr"age da 
sind!\footnote{Wenn Sie z.\,B. 
\ko\texttt{\bs vli} in Ihrem Text verwenden, erzeugt \texttt{bibsort} eine Datei 
\ko\texttt{.vli}; falls
Sie \texttt{\bs vli} sp"ater wieder l"oschen und Ihr mit 
\LaTeX\ "ubersetzter Text derartige Literatur\-\hbox{anga}\-ben nicht mehr enth"alt, 
l"oscht ein weiterer Start von \texttt{bibsort} die \ko\texttt{.vli}\hy Datei.
$-$ Unter \texttt{\bs nofiles} bleiben die \ko\texttt{.aux}\hy Files unver"andert
stehen und \texttt{bibsort} erzeugt die Datei
\texttt{.vli} immer wieder gleich, d.\,h.: ohne neue/""ver"anderte 
\texttt{\bs vli}\hy Befehle im Text aufzunehmen.}

\vspace{.75ex}
\verb|-d|\hspace{.1em} dient dazu, dass \verb|bibsort| Dateien findet, wenn im Text in 
\verb|\include|\hy Argumenten Pfadangaben stehen. Dort m"ussen Sie \verb|/| 
antippen. Dies akzeptieren einige Betriebssysteme nicht. 
\hspace{.1em} \verb|bibsort -d| \hspace{.1em} ruft Dateien mit \verb|\| auf.


\vspace{.75ex}
Die Option \ko\verb|-k| zum Drucken von $\sim$ f"ur mehrfach genannte Autoren 
in den vli- und vqu\hy Listen wurde schon erkl"art. Vgl.\ 
\verb|\female| und \verb|\male| oben S.\,\pageref{female}.

\vspace{.75ex}
\verb|-x| ist f"ur Anwender, bei denen ein aktives Anf"uhrungszeichen etwas Anderes bedeutet
als in \verb|german|, \verb|ngerman|, \verb|germanb| oder \verb|ngermanb| 
(\verb|"a|\,$\Rightarrow$\,"a). Insbesondere wird \verb|"a| zusammen mit \verb|-g1| nicht mehr als \verb|ae|
einsortiert.\footnote{\label{original}Mit der Option \texttt{-x} setzt \texttt{bibsort} zudem
nicht (wie sonst) \texttt{\bs bagermanTeX} oder \texttt{\bs baoriginalTeX} in einer \textit{if}\hy Konstruktion
in die erzeugten Dateien, um eine "Anderung des \textit{catcodes} ggf.\
zu reproduzieren. $-$ \label{dqakt} Im babel\hy Paket "andert sich der \textit{catcode} von Zeichen offenbar nie.
\texttt{bibsort} kann Ihre Nutzung von \texttt{\bs originalTeX} auf den Listen nur reproduzieren,
falls \texttt{german.sty} oder \texttt{ngerman.sty} verwendet wird (dort "andert \texttt{\bs originalTeX}
den \textit{catcode} des Anf"uhrungszeichens). Vgl.\ oben S.\,\pageref{bagermanTeX} samt Anm.\,\ref{bagermanTeX}. 
$-$ Trotzdem muss heute unter babel \textbf{weder} \texttt{bibsort} mit der Option \texttt{-x} gestartet 
\textbf{noch} \texttt{\bs baoriginalTeX} mit
\texttt{\bs renewcommand} ausgeschaltet werden: Falls babel geladen ist, f"uhrt das \texttt{\bs baoriginalTeX} 
aus \BibArts~2.2 den Befehl \texttt{\bs originalTeX} nicht aus
(babel\hy french meldete unter \BibArts~2.1 ohne den \texttt{\bs renewcommand}\hy Befehl 
irref"uhrend, es sei ein\hspace{.2em} \texttt{Incomplete} \texttt{\bs ifx}~...\hspace{.2em} vorhanden).}

\vspace{.75ex}
Weitere Optionen wie \verb|-l|\,, \verb|-i=j| und \verb|-p| werden unten ab Seite~\pageref{plit} erkl"art.

\vspace{1.25ex}\noindent
In Version 2.2 gibt es \textbf{drei neue Optionen} f"ur \verb|bibsort|:

\vspace{.75ex}
\verb|-h| sortiert Minuszeichen wie Leerzeichen. Das bezieht sich neben `\verb|-|' auf
\verb|\hy| und \verb|\fhy| sowie\footnote{Falls das Doppelanf"uhrungszeichen \textit{aktiv} ist und \texttt{-x} nicht gesetzt wurde.} auf \verb|"=| und \verb|"~|. Dabei sind \verb|\hy| und \verb|\fhy| in \BibArts\ definiert
(siehe oben S.\,\pageref{hy}), w"ahrend \verb|"=| und \verb|"~| von \verb|german|, \verb|ngerman|
oder den entsprechenden Versionen von babel bereitgestellt werden m"ussen.

\vspace{.75ex}
\verb|-b| beeinflusst die Sortierreihenfolge, falls \verb|\bago| im Text verwendet wird.
\verb|\vli{J.}{Smith}{The| \verb|\bago| \verb|\ktit{Book},| \verb|London| \verb|2005}| wird sortiert als
\verb|Smith J Book London 2005|\hspace{.2em} (das \verb*|The | wurde ausgeklammert, weil im entsprechenden 
Argument nur Zeichen hinter \verb|\bago| ein Sortiergewicht erhalten). Vergleichbare Regeln
sind bis heute in mehreren Sprachen "ublich. Fr"uher schrieben die Preu\hyss ischen 
Instruktionen vor, dass das erste Substantiv, das im Nominativ steht,
sortierrelevant ist. Ohne \verb|-b|\hspace{.1em} hat \verb|\bago| keinen Einfluss.

\vspace{.75ex}
\verb|-n1| dient dazu, dass in den num\hy Listen keine Zusammenfassungen von Fu"snoten
"uber mehrere Seiten hinweg erfolgen. Dies dient der Vermeidung seltener Fehler in Texten,
in denen in jedem Kapitel die Fu"snotenz"ahlung neu mit 1 beginnt. Dann k"onnte ein Kapitel 
auf seiner letzten Seite beispielsweise in den Fu"snoten 2 bis 4 dieselbe
Belegstelle nennen $-$ und zudem k"onnte auf der n"achsten Seite in Fu"snote 5 
nochmals dieselbe Belegstelle folgen. Ohne \verb|-n1| w"urde dies zu $6^2$$-$$7^5$
zusammengefasst\hspace{.1em} (obwohl der Beleg auf Seite\hspace{-.1em} 7 in den Fu"snoten 1 bis 4 nicht vorkommt). 
Mit \verb|-n1| wird dagegen richtig $6^{2-4}$$,\ $$7^5$ ausgedruckt. Gerechtfertigte
Zusammenfassungen unterbleiben aber ebenso. 

\vspace{1.25ex}\vfill\noindent
Nun ein \textbf{Beispiel f"ur die Sortierreihenfolge}. Achten Sie etwa darauf, dass bei Buchstaben mit
mehreren Akzenten (\c{\"a}) die Akzente von hinten nach vorne gewichtet werden, d.\,h. der zuletzt
getippte Akzent die Einsortierung am st"arksten bestimmt (im Falle von
Eintr"agen mit gleichen Buchstabenkernen): 



 \newcommand{\demotext}[1]{\nosort{\protect\printdemotext{#1}}}%
{\newcommand{\printdemotext}[1]{{\hskip 0.1cm plus 0.5cm\footnotesize\texttt{#1}}}%
 \renewcommand{\gprrtitlename}{Das Personenregister als Beispiel f"ur das Sortieren}%
 \printprr}



%% \oldhy...-Befehle (und - falls undefiniert - auch \ck) sind in bibarts.sty festgelegt, 
%% (1) um eine Paralle zu "' und \grqq zu haben;
%% (2) um Zitate in alter deutscher Rechtschreibung auch ohne german.sty machen zu koennen.
%% In alter deutscher Rechtschreibung werden manche ck als k-k getrennt. german.sty und
%% ngerman.sty bieten Loesungen, die in ngerman.sty aber zudem eine Warnmeldung ausloesen: 
  % \addtoprr{zz\ck a\demotext{ zz\bs ck a}}
  % \addtoprr{zz"cka\demotext{ zz\protect\string"cka \ \ \ \ \ \ \ \% neu in 2.2}}
  %  \addtoprr{zz\oldhyc ka\demotext{ zz\bs oldhyc ka \ \%\bs old...:~2.2}}
  % \addtoprr{zz"CKa\demotext{ zz\protect\string"CKa}}
  %  \addtoprr{zz\OLDHYC Ka\demotext{ zz\bs OLDHYC Ka}}
%% "FF dient in alter deutscher Rechtschreibung fuer ein FF, das FF-F getrennt wird (etc.).
%% "FF druckt (Fall: ohne Zeilenumbruch) unter ngerman.sty FFF statt FF aus, wird aber von
%% bibsort stets unter FF einsortiert: "FF nur mit german.sty verwenden, das FF ausdruckt.
  % \addtoprr{zz"lla\demotext{ zz\protect\string"lla}}
  %  \addtoprr{zz\oldhyl la\demotext{ zz\bs oldhyl la}}
  % \addtoprr{zz"LLa\demotext{ zz\protect\string"LLa}}
  %  \addtoprr{zz\OLDHYL La\demotext{ zz\bs OLDHYL La}}
  % \addtoprr{zz"mma\demotext{ zz\protect\string"mma}}
  %  \addtoprr{zz\oldhym ma\demotext{ zz\bs oldhym ma}}
  % \addtoprr{zz"MMa\demotext{ zz\protect\string"MMa}}
  %  \addtoprr{zz\OLDHYM Ma\demotext{ zz\bs OLDHYM Ma}}
  % \addtoprr{zz"nna\demotext{ zz\protect\string"nna}}
  %  \addtoprr{zz\oldhyn na\demotext{ zz\bs oldhyn na}}
  % \addtoprr{zz"NNa\demotext{ zz\protect\string"NNa}}
  %  \addtoprr{zz\OLDHYN Na\demotext{ zz\bs OLDHYN Na}}
  % \addtoprr{zz"ppa\demotext{ zz\protect\string"ppa}}
  %  \addtoprr{zz\oldhyp pa\demotext{ zz\bs oldhyp pa}}
  % \addtoprr{zz"PPa\demotext{ zz\protect\string"PPa}}
  %  \addtoprr{zz\OLDHYP Pa\demotext{ zz\bs OLDHYP Pa}}
  % \addtoprr{zz"rra\demotext{ zz\protect\string"rra}}
  %  \addtoprr{zz\oldhyr ra\demotext{ zz\bs oldhyr ra}}
  % \addtoprr{zz"RRa\demotext{ zz\protect\string"RRa}}
  %  \addtoprr{zz\OLDHYR Ra\demotext{ zz\bs OLDHYR Ra}}
  % \addtoprr{zz"tta\demotext{ zz\protect\string"tta}}
  %  \addtoprr{zz\oldhyt ta\demotext{ zz\bs oldhyt ta}}
  % \addtoprr{zz"TTa\demotext{ zz\protect\string"TTa}}
  %  \addtoprr{zz\OLDHYT Ta\demotext{ zz\bs OLDHYT Ta}}
  % \addtoprr{zz"ffa\demotext{ zz\protect\string"ffa}}
  %  \addtoprr{zz\oldhyf fa\demotext{ zz\bs oldhyf fa}}
  % \addtoprr{zz"FFa\demotext{ zz\protect\string"FFa}}
  %  \addtoprr{zz\OLDHYF Fa\demotext{ zz\bs OLDHYF Fa}}
%%
  % \addtoprr{\ck\demotext{ \bs ck \ \%neu in 2.2}}
%%
\addtoprr{\"y\demotext{ \bs\protect\string"y}}
\addtoprr{ya}
\addtoprr{DiFabio}
\addtoprr{Di Niro}
\addtoprr{ba}
\addtoprr{ a}
\addtoprr{5.000}
\addtoprr{0,5}
\addtoprr{0.5}
\addtoprr{0.a6}
\addtoprr{0.26}
\addtoprr{0.25}
\addtoprr{0,251}
\addtoprr{0.251}
\addtoprr{1.500}
\addtoprr{1,500}
\addtoprr{1501}
\addtoprr{1,5}
\addtoprr{1,1}
\addtoprr{1,125}
\addtoprr{1,45}
\addtoprr{1.750}
\addtoprr{1,75}
\addtoprr{$\frac{8}{2}$\demotext{ \$\bs frac\{8\}\{2\}\$}}
\addtoprr{$\frac{8} {2}$\demotext{ \$\bs frac\{8\} \{2\}\$}}
\addtoprr{$\frac{10}{1}$}
\addtoprr{$\frac{13}{2}$}
\addtoprr{$\frac{1000}{100}$}
\addtoprr{$\frac{7}{2}$}
\addtoprr{$\frac{9}{2}$}
\addtoprr{\sort{9,5}$\frac{19}{2}$\demotext{ \bs sort\{9,5\}\$\bs frac\{19\}\{2\}\$}}
\addtoprr{$\frac{b}{a}$\demotext{ \$\bs frac\{b\}\{a\}\$}}
\addtoprr{1}
\addtoprr{14\te Auf"|l.\demotext{ 14\bs te Auf\protect\string"\protect\string|l.}}
\addtoprr{15}
\addtoprr{22}
\addtoprr{2}
%\addtoprr{\protect\underline{2}\demotext{ \bs protect\bs underline\{2\}}}
\addtoprr{b \sort{b}c\demotext{ \ \ \ b \bs sort\{b\}c}}
\addtoprr{b\sort{b} c\demotext{ b\bs sort\{b\} c}}
\addtoprr{b\sort{b}b\demotext{ b\bs sort\{b\}b}}
\addtoprr{b\sort{b}a\demotext{ b\bs sort\{b\}a}}
\addtoprr{b\sort{b}\demotext{ \ b\bs sort\{b\}}}
%\addtoprr{*b\demotext{ \protect\string*b}}
\addtoprr{bb}
\addtoprr{bc}
\addtoprr{$b$\demotext{ \ \$b\$}}
\addtoprr{b$^{2}$\demotext{ b\$\protect\string^\{2\}\$}}
\addtoprr{b$_{2}$\demotext{ b\$\protect\string_\{2\}\$}}
\addtoprr{b\fup{2}\demotext{ \ b\bs fup\{2\}}}
\addtoprr{b\fup{b}\demotext{ \ \ b\bs fup\{b\}}}
\addtoprr{b1}
\addtoprr{b2}
\addtoprr{b 2}
\addtoprr{b3}
\addtoprr{b20}
\addtoprr{b98}
\addtoprr{b100}
\addtoprr{b\fup{\baRomannum{99}}\demotext{ b\bs fup\{\bs baRomannum\{99\}\}}}
\addtoprr{{c}\demotext{ \{c\}}}
\addtoprr{{}c\demotext{ \{\}c}}
\addtoprr{c{}\demotext{ c\{\}}}
\addtoprr{"a\demotext{ \protect\string"a}}
\addtoprr{\"a\demotext{ \bs\protect\string"a}}
\addtoprr{"A\demotext{ \protect\string"A}}
\addtoprr{\"A\demotext{ \bs\protect\string"A}}
\addtoprr{ae}
\addtoprr{\c{\"a}\demotext{ \bs c\{\bs\protect\string"a\}}}
\addtoprr{\c{"a}\demotext{ \bs c\{\protect\string"a\}}}
\addtoprr{af}
\addtoprr{sr}
\addtoprr{ss}
\addtoprr{\3\demotext{ \ \bs 3}}
\addtoprr{\hyss\demotext{ \ \bs hyss \ \%siehe~S.~\pageref{hyss}}}
\addtoprr{"s\demotext{ \ \protect\string"s}}
\addtoprr{\sz\demotext{ \bs sz \ \%neu in 2.1}}
\addtoprr{{\scshape \ss}\demotext{ \{\bs scshape \bs ss\}}}
\addtoprr{"z\demotext{ \ \protect\string"z}}
\addtoprr{Stra"se\demotext{ \ \ Stra\protect\string"se}}
\addtoprr{Stra{\ss}e\demotext{ \ \ Stra\{\bs ss\}e}}
\addtoprr{Stra"sburg\demotext{ Stra\protect\string"sburg}}
\addtoprr{Stra{\ss}burg\demotext{ Stra\{\bs ss\}burg}}
\addtoprr{Sto"ffigur\demotext{ Sto\protect\string"ffigur \ \%!~S.~\pageref{ff}}}

 \addtoprr{Zum Schluss die Worttrennung american im \label{Trennbeispiel} deutschen Trennsatz}
{\sethyphenation{UKenglish}
 \addtoprr{Zum Schluss die Worttrennung american im englischen Trennsatz}}

\addtoprr{A}
\addtoprr{B}
\addtoprr{C}
\addtoprr{D}
\addtoprr{E}
\addtoprr{F}
\addtoprr{G}
\addtoprr{H}
\addtoprr{I}
\addtoprr{J}
\addtoprr{K}
\addtoprr{L}
\addtoprr{M}
\addtoprr{N}
\addtoprr{O}
\addtoprr{P}
\addtoprr{Q}
\addtoprr{R}
\addtoprr{S}
\addtoprr{T}

\addtoprr{W}
\addtoprr{X}
\addtoprr{Y}
\addtoprr{Z}
\addtoprr{Ba}
\addtoprr{Ca}
\addtoprr{Ci}
\addtoprr{Da}
%\addtoprr{Ea}
%\addtoprr{Fa}
\addtoprr{Ga}
%\addtoprr{Ha}
\addtoprr{Ia}
%\addtoprr{Ja}
%\addtoprr{Ka}
{\sethyphenation{ngerman}\addtoprr{Sto\hyf figur\demotext{ Sto\bs hyf figur~\%ngerman}}}
{\sethyphenation{german}\addtoprr{Sto\hyf figur\demotext{ Sto\bs hyf figur~~\%german}}}
\addtoprr{stoffig\demotext{ \ \ stoffig}}
%\addtoprr{La}
\addtoprr{Ma}
\addtoprr{Na}
\addtoprr{Oa}
\addtoprr{"o\demotext{ \protect\string"o}}
\addtoprr{Ra}
\addtoprr{Sa}
\addtoprr{Ta}
\addtoprr{Ti}
\addtoprr{th}
\addtoprr{Th}
\addtoprr{Xa}
\addtoprr{Ya}
\addtoprr{Za}

\addtoprr{a}
\addtoprr{b}
\addtoprr{c}
\addtoprr{d}
\addtoprr{e}
\addtoprr{f}
\addtoprr{g}
\addtoprr{h}
\addtoprr{i}
\addtoprr{j}
\addtoprr{k}
\addtoprr{l}
\addtoprr{m}
\addtoprr{n}
\addtoprr{o}
\addtoprr{p}
\addtoprr{q}
\addtoprr{r}
\addtoprr{s}
\addtoprr{t}
\addtoprr{u}
\addtoprr{v}
\addtoprr{w}
\addtoprr{x}
\addtoprr{y}
\addtoprr{z}
\addtoprr{aa}
\addtoprr{ba}
\addtoprr{ca}
\addtoprr{ci}
\addtoprr{da}
%\addtoprr{ea}
%\addtoprr{fa}
\addtoprr{ga}
%\addtoprr{ha}
\addtoprr{ia}
\addtoprr{ja}
\addtoprr{ka}
\addtoprr{la}
\addtoprr{na}
\addtoprr{oa}
\addtoprr{pa}
\addtoprr{ra}
\addtoprr{sa}
\addtoprr{ta}
\addtoprr{ti}

\addtoprr{xa}
\addtoprr{ya}
\addtoprr{za}

\addtoprr{ph}
\addtoprr{pi}
\addtoprr{ps}
\addtoprr{pt}
\addtoprr{Ph}
\addtoprr{Pi}
\addtoprr{Ps}
\addtoprr{Pt}

\addtoprr{b a}
\addtoprr{B a}

\addtoprr{0000000000}
\addtoprr{100000000}
\addtoprr{2~100~000,7\demotext{ 2\protect\string~100\protect\string~000,7}}
\addtoprr{2\,100\,000,65\demotext{ 2\bs,100\bs,000,65}}
\addtoprr{2.099.999}
\addtoprr{2.100.001}
\addtoprr{\sort{2.100.000}2,1\,Mio.\demotext{ \ \bs sort\{2.100.000\}2,1\bs,Mio.}}
\addtoprr{\baromannum{14}\te Bd.\demotext{ \bs baromannum\{14\}\bs te Bd.}}
\addtoprr{\baRomannum{15}\te Bd.\demotext{ \bs baRomannum\{15\}\bs te Bd.}}
\addtoprr{400000}
\addtoprr{6000}
\addtoprr{700}
\addtoprr{9}
%\addtoprr{.}
%\addtoprr{,}
%\addtoprr{;}
%\addtoprr{:}
%\addtoprr{?}
%\addtoprr{!}
%\addtoprr{?`\demotext{ \protect\string?{}\protect\string`}}
%\addtoprr{!`\demotext{ \protect\string!{}\protect\string`}}
\addtoprr{*\demotext{ \protect\string*}}
\addtoprr{\$\demotext{ \bs\protect\$}}
%\addtoprr{\#\demotext{ \bs\protect\#}}
\addtoprr{"`a"'\demotext{ \protect\string"\protect\string`a\protect\string"\protect\string' \ \ \% ge"andert in 2.2}}
\addtoprr{"<a">\demotext{ \protect\string"\protect\string<a\protect\string"\protect\string> \ \% ge"andert in 2.2}}
\addtoprr{\nosort{'}a'\demotext{ \bs nosort\{{\protect\string'}\}a'}}
\addtoprr{'a'\demotext{ {\protect\string'}a{\protect\string'}}}
\addtoprr{a'a\demotext{ a\protect\string'a \ \ \%\% vgl. \bs nosort\{\protect\string'\}a}}
{\originalTeX \addtoprr{"a\demotext{ \{\bs originalTeX \bs addtoprr\{\protect\string"a\}\}\protect\label{originaltex}}}}
\addtoprr{\glqq a\grqq\demotext{ \bs glqq a\bs grqq}}
\addtoprr{\flqq a\frqq\demotext{ \bs flqq a\bs frqq}}
\addtoprr{b\$2\demotext{ b\bs\$2}}
\addtoprr{b\&2\demotext{ b\bs\&2}}
\addtoprr{b/2}
\addtoprr{$b\;2$\demotext{ \$b\bs;2\$}}
\addtoprr{$b\:2$\demotext{ \$b\bs:2\$}}
\addtoprr{(b)}
\addtoprr{[b]}
\addtoprr{\{b\demotext{ \bs\{b}}
\addtoprr{\%\demotext{ \bs\%}}
\addtoprr{b\_2\demotext{ b\bs\protect\string_2}}
\addtoprr{b=2}
\addtoprr{b<2}
\addtoprr{b>2}
\addtoprr{b@2}
\addtoprr{b|2\demotext{ b\protect\string|2}}
\addtoprr{b+2}
\addtoprr{b-2}
\addtoprr{b--2\demotext{ \ b{\protect\string-}{\protect\string-}2}}
\addtoprr{b---2\demotext{ b{\protect\string-}{\protect\string-}{\protect\string-}2}}
\addtoprr{Ae}
\addtoprr{Ac}
\addtoprr{Af}
\addtoprr{\.{b}\demotext{ \bs.\{b\}}}
\addtoprr{$\dot{b}$\demotext{ \$\bs dot\{b\}\$}}
\addtoprr{\'{b}\demotext{ \bs\protect\string'\{b\}}}
\addtoprr{$\acute{b}$\demotext{ \$\bs acute\{b\}\$}}
\addtoprr{\`{b}\demotext{ \bs\protect\string`\{b\}}}
\addtoprr{$\grave{b}$\demotext{ \$\bs grave\{b\}\$}}
\addtoprr{\^{b}\demotext{ \bs\protect\string^\{b\}}}
\addtoprr{$\hat{b}$\demotext{ \$\bs hat\{b\}\$}}
\addtoprr{$\ddot{b}$\demotext{ \$\bs ddot\{b\}\$}}
\addtoprr{\={b}\demotext{ \bs=\{b\}}}
\addtoprr{$\bar{b}$\demotext{ \$\bs bar\{b\}\$}}
\addtoprr{$\vec{b}$\demotext{ \$\bs vec\{b\}\$}}
\addtoprr{\~{b}\demotext{ \bs\protect\string~\{b\}}}
\addtoprr{$\tilde{b}$\demotext{ \$\bs tilde\{b\}\$}}
\addtoprr{\r{a}\demotext{ \bs r\{a\}}}
\addtoprr{\r{A}\demotext{ \bs r\{A\}}}
\addtoprr{\aa\demotext{ \bs aa}}
\addtoprr{\AA\demotext{ \bs AA}}
\addtoprr{\r{b}\demotext{ \bs r\{b\}}}
\addtoprr{$\mathring{b}$\demotext{ \$\bs mathring\{b\}\$ \ \%neu in 2.1}}
\addtoprr{\b{b}\demotext{ \bs b\{b\}}}
\addtoprr{\c{b}\demotext{ \bs c\{b\}}}
\addtoprr{\k{b}\demotext{ \bs k\{b\}}}
\addtoprr{\d{b}\demotext{ \bs d\{b\}}}
\addtoprr{\H{b}\demotext{ \bs H\{b\}}}
\addtoprr{\t{b}\demotext{ \bs t\{b\}}}
\addtoprr{\u{b}\demotext{ \bs u\{b\}}}
\addtoprr{$\breve{b}$\demotext{ \$\bs breve\{b\}\$}}
\addtoprr{\v{b}\demotext{ \bs v\{b\}}}
\addtoprr{$\check{b}$\demotext{ \$\bs check\{b\}\$}}
\addtoprr{\protect\underbar{\"b}\demotext{ \bs protect\bs underbar\{\bs\protect\string"b\}}}
\addtoprr{\protect\underline{\c{b}}\demotext{ \bs protect\bs underline\{\bs c\{b\}\}}}
\addtoprr{\protect\underline{\"b}\demotext{ \bs protect\bs underline\{\bs\protect\string"b\}}}
\addtoprr{ab}
\addtoprr{a b}
\addtoprr{a\,b\demotext{ a\bs,b}}
\addtoprr{a\protect\space b\demotext{ a\bs protect\bs space b}}
%\addtoprr{a\relax b\demotext{ a\bs relax b}}
\addtoprr{a!b\demotext{ a\protect\string!b}}
\addtoprr{a\nosort{!}b\demotext{ a\bs nosort\{\protect\string!\}b}}
\addtoprr{a~b\demotext{ a\protect\string~b}}
\addtoprr{aa}
\addtoprr{ab}
\addtoprr{aba}
\addtoprr{a-b\demotext{ a-b}}
\addtoprr{a"|b\demotext{ a\protect\string"{\protect\string|}b \ \ \% ge"andert in 2.2}}
\addtoprr{a"~b\demotext{ a\protect\string"{\protect\string~}b \ \ \% ge"andert in 2.2}}
\addtoprr{a"=b\demotext{ a\protect\string"{\protect\string=}b \ \ \% ge"andert in 2.2}}
\addtoprr{a"-b\demotext{ a\protect\string"{\protect\string-}b \ \ \% ge"andert in 2.2}}
\addtoprr{a\-b\demotext{ a\bs{\protect\string-}b}}
\addtoprr{a\hy b\demotext{ a\bs hy b}}
\addtoprr{a\fhy b\demotext{ a\bs fhy b}}
%\addtoprr{ac}
\addtoprr{\i\demotext{ \bs i}}
\addtoprr{\'\i\demotext{ \bs\protect\string'\bs i}}
\addtoprr{\^\i\demotext{ \bs\protect\string^\bs i}}
\addtoprr{\`\i\demotext{ \bs\protect\string`\bs i}}
\addtoprr{$\imath$\demotext{ \$\bs imath\$}}
\addtoprr{\j\demotext{ \bs j}}
\addtoprr{$\jmath$\demotext{ \$\bs jmath\$}}
\addtoprr{\ae\demotext{ \bs ae}}
\addtoprr{\AE\demotext{ \bs AE}}
\addtoprr{\copyright\demotext{ \bs copyright}}
\addtoprr{\dh\demotext{ \bs dh}}
\addtoprr{\dj\demotext{ \bs dj}}
\addtoprr{\DH\demotext{ \bs DH}}
\addtoprr{\DJ\demotext{ \bs DJ}}
\addtoprr{\l\demotext{ \bs l}}
\addtoprr{$\ell$\demotext{ \$\bs ell\$}}
\addtoprr{\ij\demotext{ \bs ij niederl"and.\ als i sortiert}}
\addtoprr{\IJ\demotext{ \bs IJ "aquivalent zu \bs ij}}
\addtoprr{$\Im$\demotext{ \$\bs Im\$}}
\addtoprr{\L\demotext{ \bs L}}
\addtoprr{\LaTeX\demotext{ \bs LaTeX}}
\addtoprr{LATEX}
\addtoprr{LaTeX}
\addtoprr{\ng\demotext{ \bs ng}}
\addtoprr{\NG\demotext{ \bs NG}}
\addtoprr{\o\demotext{ \bs o}}
\addtoprr{\O\demotext{ \bs O}}
\addtoprr{\oe\demotext{ \bs oe}}
\addtoprr{\OE\demotext{ \bs OE}}
\addtoprr{\textsterling\demotext{ \bs textsterling}}
\addtoprr{\pounds\demotext{ \bs pounds}}
\addtoprr{\textregistered\demotext{ \bs textregistered}}
\addtoprr{$\Re$\demotext{ \$\bs Re\$}}
\addtoprr{\th\demotext{ \bs th}}
\addtoprr{\TH\demotext{ \bs TH}}
\addtoprr{\protect\TeX\demotext{ \bs protect\bs TeX}}
\addtoprr{TEX\demotext{ TEX}}
\addtoprr{TeX\demotext{ TeX}}
\addtoprr{$\alpha$\demotext{ \$\bs alpha\$}}
\addtoprr{$\beta$\demotext{ \$\bs beta\$}}
\addtoprr{$\chi$\demotext{ \$\bs chi\$}}
\addtoprr{$\Chi$\demotext{ \$\bs Chi\$ \ \%neu in 2.1}}
\addtoprr{$\delta$\demotext{ \$\bs delta\$}}
\addtoprr{$\Delta$\demotext{ \$\bs Delta\$}}
\addtoprr{$\epsilon$\demotext{ \$\bs epsilon\$}}
\addtoprr{$\varepsilon$\demotext{ \$\bs varepsilon\$}}
\addtoprr{$\eta$\demotext{ \$\bs eta\$}}
\addtoprr{$\gamma$\demotext{ \$\bs gamma\$}}
\addtoprr{$\Gamma$\demotext{ \$\bs Gamma\$}}
\addtoprr{$\iota$\demotext{ \$\bs iota\$}}
\addtoprr{$\kappa$\demotext{ \$\bs kappa\$}}
\addtoprr{$\lambda$\demotext{ \$\bs lambda\$}}
\addtoprr{$\Lambda$\demotext{ \$\bs Lambda\$}}
\addtoprr{$\mu$\demotext{ \$\bs mu\$}}
\addtoprr{$\nu$\demotext{ \$\bs nu\$}}
\addtoprr{\d{$\tilde\nu$}\demotext{ \bs d\{\$\bs tilde\bs nu\$\}}}
\addtoprr{$\tilde\nu$\demotext{ \$\bs tilde\bs nu\$}}
\addtoprr{$\breve\nu$\demotext{ \$\bs breve\bs nu\$}}
\addtoprr{$\omicron$\demotext{ \$\bs omicron\$ (bibarts.sty)}}   %% existiert in LaTeX nicht
\addtoprr{$\omega$\demotext{ \$\bs omega\$}}
\addtoprr{$\Omega$\demotext{ \$\bs Omega\$}}
\addtoprr{$\pi$\demotext{ \$\bs pi\$}}
\addtoprr{$\Pi$\demotext{ \$\bs Pi\$}}
\addtoprr{$\varpi$\demotext{ \$\bs varpi\$}}
\addtoprr{$\phi$\demotext{ \$\bs phi\$}}
\addtoprr{$\Phi$\demotext{ \$\bs Phi\$}}
\addtoprr{$\varphi$\demotext{ \$\bs varphi\$}}
\addtoprr{$\psi$\demotext{ \$\bs psi\$}}
\addtoprr{$\Psi$\demotext{ \$\bs Psi\$}}
\addtoprr{$\rho$\demotext{ \$\bs rho\$}}
\addtoprr{$\varrho$\demotext{ \$\bs varrho\$}}
\addtoprr{$\sigma$\demotext{ \$\bs sigma\$}}
\addtoprr{$\Sigma$\demotext{ \$\bs Sigma\$}}
\addtoprr{$\varsigma$\demotext{ \$\bs varsigma\$}}
\addtoprr{$\tau$\demotext{ \$\bs tau\$}}
\addtoprr{$\theta$\demotext{ \ \$\bs theta\$}}
\addtoprr{$\Theta$\demotext{ \$\bs Theta\$}}
\addtoprr{$\vartheta$\demotext{ \$\bs vartheta\$}}
\addtoprr{$\xi$\demotext{ \$\bs xi\$}}
\addtoprr{$\Xi$\demotext{ \$\bs Xi\$}}
\addtoprr{$\upsilon$\demotext{ \$\bs upsilon\$}}
\addtoprr{$\Upsilon$\demotext{ \$\bs Upsilon\$}}
\addtoprr{$\zeta$\demotext{ \$\bs zeta\$}}
%\addtoprr{$\Zeta$\demotext{ \$\bs Zeta\$}}
\addtoprr{$\Rho \acute\omicron \delta \omicron \varsigma$ 
   \demotext{ \$\bs Rho \bs acute\bs omicron \bs delta \bs omicron \bs varsigma\$}}
\addtoprr{$\Eta \rho \alpha$ (Hera) 
   \demotext{ \$\bs Eta \bs rho \bs alpha\$}}

\addtoprr{b\\[-2mm]b\demotext{ b\bs\bs[-2mm]b}}
\addtoprr{b\vspace{2mm}b\demotext{ b\bs vspace\{2mm\}b}}
\addtoprr{b\hspace{2mm}b\demotext{ \ b\bs hspace\{2mm\}b}}
\addtoprr{b\index{X}b\demotext{ \ \ b\bs index\{X\}b}}
\addtoprr{b \index{X}b\demotext{ \ \ b \bs index\{X\}b}}
\addtoprr{b \index{X} b\demotext{ \ \ b \bs index\{X\} b}}
\addtoprr{b\glossary{X}b\demotext{ \ b\bs glossary\{X\}b}}
\addtoprr{b \label{X1} b\demotext{ \ b \bs label\{X1\} b}}
\addtoprr{b \pageref{X1} b\demotext{ b \bs pageref\{X1\} b}}
\addtoprr{b \baref{X2} b\demotext{ b \bs baref\{X2\} b}}
\addtoprr{b \balabel{X2} b\demotext{ \ b \bs balabel\{X2\} b}}
\addtoprr{b \protect\pageref{X1} b\demotext{ b \bs protect\bs pageref\{X1\} b}}
\addtoprr{b\protect\linebreak[1]b\demotext{ \ \ \ b\bs protect\bs linebreak[1]b}}
\addtoprr{b\protect\nolinebreak[1]b\demotext{ b\bs protect\bs nolinebreak[1]b}}
\addtoprr{b\protect\pagebreak[2]b\demotext{ b\bs protect\bs pagebreak[2]b}}
\addtoprr{b\protect\nopagebreak[2]b\demotext{ b\bs protect\bs nopagebreak[2]b}}
\addtoprr{b\protect\footnote[144]{X} b\demotext{ b\bs protect\bs footnote[144]\{X\} \protect\strut\hspace{2em} b}}
\addtoprr{b\protect\footnotetext[145]{Y} b\demotext{ b\bs protect\bs footnotetext[145]\{Y\} \protect\strut\hspace{2em} b}}
\addtoprr{b\protect\footnotemark[146] b\demotext{ b\bs protect\bs footnotemark[146] \protect\strut\hspace{2em} b}}
\addtoprr{b\protect\framebox [1cm][l]{b}\demotext{ b\bs protect\bs framebox [1cm][l]\{b\}}}
\addtoprr{b\protect\makebox [1cm][r]{b}\demotext{ b\bs protect\bs makebox [1cm][r]\{b\}}}
\addtoprr{b\protect\parbox[t]{5mm}{b}\demotext{ b\bs protect\bs parbox[t]\{5mm\}\{b\}}}
\addtoprr{b\protect\raisebox{0.5ex} [3mm] [3mm] {b}\demotext{ b\bs protect\bs raisebox\{0.5ex\} [3mm] [3mm] \{b\}}}
\addtoprr{b\protect\rule[1mm] {1mm} {2mm}b\demotext{ b\bs protect\bs rule[1mm] \{1mm\} \{2mm\}b}}
%
\addtoprr{b Nach Titel\demotext{ \ \ b Nach Titel}}
\addtoprr{b \vli {Vor} {Nach} {Titel}\demotext{ \ \ b \bs vli \{Vor\} \protect\strut\hspace{6em} \{{\bf Nach}\} \{Titel\}}}
\addtoprr{b \vli {Vor} {Nach}{Titel}\demotext{ \ b \bs vli \{Vor\} \protect\strut\hspace{6em} \{{\bf Nach}\}\{Titel\}}}
\addtoprr{b \kli {Nach} {Titel}\demotext{ \ b \bs kli \{{\bf Nach}\} \protect\strut\hspace{8em} \{Titel\}}}
\addtoprr{b \kli [f]{Nach} {Titel}\demotext{ \ b \bs kli [f]\{{\bf Nach}\} \protect\strut\hspace{8em} \{Titel\}}}
\addtoprr{b \vli[m] {Vor} {Nach} {Titel}\demotext{ \ \ \ b \bs vli[m] \{Vor\} \protect\strut\hspace{6em} \{{\bf Nach}\} \{Titel\}}}
\addtoprr{b Nach\demotext{ \ b Nach}}
\addtoprr{b Nach Vor\demotext{ \ \ \ \ b Nach Vor}}
\addtoprr{b Nach Vor U\demotext{ \ \ \ \ b Nach Vor U}}
\addtoprr{b Nach Titel-a\demotext{ \ \ \ b Nach Titel-a}}
\addtoprr{b \vauthor{Vor}{Nach}\demotext{ b \bs vauthor\{Vor\}\{{\bf Nach}\}}}
\addtoprr{b \kauthor{Nach}\demotext{ \ \ b \bs kauthor\{{\bf Nach}\}}}
%
\addtoprr{\protect\begin{large}b\protect \end{large}b\demotext{ \bs protect\bs begin\{large\}b\bs protect \bs end\{large\}b}}
\addtoprr{b\protect\typeout{9}b\demotext{ b\bs protect\bs typeout\{9\}b}}
\addtoprr{b\message{9}b\demotext{ b\bs message\{9\}b}}
\addtoprr{b\mathhexbox{1}{2}{3}b\demotext{ b\bs mathhexbox\{1\}\{2\}\{3\}b}}
\addtoprr{b\protect\phantom{X}b\demotext{ \ b\bs protect\bs phantom\{X\}b}}
\addtoprr{b\protect\vphantom{X}b\demotext{ \ b\bs protect\bs vphantom\{X\}b}}
\addtoprr{b\protect\hphantom{X}b\demotext{ \ b\bs protect\bs hphantom\{X\}b}}
\addtoprr{b\sethyphenation{french}b\demotext{ b\bs sethyphenation\{french\}b}}
\addtoprr{b\selectlanguage{french}b\demotext{ b\bs selectlanguage\{french\}b}}
\addtoprr{b\discretionary{a-}{c} {b}\demotext{ b\bs discretionary\{a-\}\{c\} \{b\}}}

\noindent
Zur Erzeugung der vorausgehenden Liste bekam \verb|bibsort| als
\textit{Sortier}\hy Optio"-n(\ko en\ko) \texttt{\bibsortargs} "ubergeben. 
\verb|\bibsortargs| steht hier vor "`"ubergeben"', um die 
Optionen auszudrucken; sie werden in diesem Befehl
beim Ausdruck einer \verb|bibsort|\hy Datei $-$ hier \verb|\printprr| $-$
hinterlegt. Die Optionen 
\texttt{-d} und \texttt{-m} werden nie hinterlegt; 
sie beeinflussen die Reihenfolge nicht.

Die vorausgehenden Seiten zeigten auch viele \LaTeX\hy Befehle,
die \verb|bibsort| verarbeitet. In \verb|bibsort.c| k"onnen Sie sehen, 
welche Befehle einen Sortierwert erhalten.\footnote{ 
\LaTeX\hy Befehlsnamen \textit{aus Buchstaben} m"ussen in der 
\texttt{weighttable}\hy Liste in \texttt{bibsort.c} mit \texttt{\bs t} 
enden, \textit{aus einzelnen Zeichen} bestehende Befehle wie \texttt{\bs\dq} d"urfen das nicht.} 
Befehle, die \verb|bibsort| nicht kennt, sortiert es "ahnlich wie Satzzeichen,
gewichtet sie also nur, wenn Zeilen \textit{sonst} nur 
\textit{gleiche Buchstaben und Zahlen} enthalten. Zum Verst"andnis des
schichtweisen Sortierens beachten Sie bitte \verb|Stra"sburg| und \verb|Stra{\ss}burg|. 
In Ihrem Text d"urften Sie Scharf\fhy S auch als \verb|utf8|\hy Zeichen tippen; 
es w"urde ebenfalls vor \verb|Stra"se| und \verb|Stra{\ss}e| sortiert
(ggf.\ \verb|\usepackage[utf8]{inputenc}| im Vorspann).

Fu"snotenexponenten (\verb|\footnotemark[146]|) werden absichtlich nicht gewichtet, 
mathematische Exponenten(\verb|$^{2}$|) schon. 
\verb|bibsort| macht auch Unterschiede zwischen Ausdruckreihenfolge
und Sortierreihenfolge:
Die Argumente von \verb|\vauthor| und \verb|\midvauthor| sowie  \verb|\ntvauthor| 
sortiert \verb|bibsort| zuerst nach Nachnamen, und nur bei gleichen Nachnamen 
\textit{anschlie"send} nach Vornamen. 
Und Argumente nach vielen Befehlen\footnote{In \texttt{transformtable} in \texttt{bibsort.c} aufgelistet.
Von "Anderungen wird abgeraten.} 
werden `ausgerichtet': Z.\,B. werden `innere'\hspace{.1em} \verb|\kli{A}{B}|\hspace{.1em} und\hspace{.1em}
\verb*|\kli{A} {B}|\hspace{.1em} stets als\hspace{.15em} \verb*|A B|\hspace{.15em} sortiert. 

\vspace{2ex}\noindent
Die Spracheinstellung bestimmt, worin \verb|bibsort| eine 
\textbf{Dezimalzahl} sieht: Mit \verb|-g1| oder \verb|-g2| kommt 
\verb|0,251| \textit{vor} \verb|0,5| (deutsche `Nachkommastellen'), 
\textit{sonst}\pdfko{0}\ zwischen 22 und 700 (englisch gelesen 
nulltausendzweihunderteinundf"unfzig). 

\textit{Strukturierungszeichen zum besseren 
Lesen gro"ser Zahlen} weichen im Deutschen vom \textit{default} ab:  
\verb|1.000| wird mit \ko\verb|-g1| oder \ko\verb|-g2| als tausend verstanden, 
sonst (\ko\textit{default}$\,=\,$englisch) stellt \verb|1,000| die Zahl Tausend dar. 
Mit \ko\verb|-g1| oder \ko\verb|-g2| wird \verb|0.251| als 251 verstanden; 
nur bei \textit{drei} `Nachpunktstellen' ist `\verb|.|' Strukturierungszeichen:
\verb|Bd.\,3.1| gilt als `drei-Punkt-eins' und wird vor \verb|Bd.\,10.2| 
einsortiert (anders als \verb|Bd.\,3.100|).

Bei Punkt und Komma l"asst sich die f"ur einen Text einmal gew"ahlte 
Sprachkonvention sp"ater also nur noch schwer "andern. Setzen der Punkte oder
Kommata in geschweifte Klammern schaltet jedoch die Dezimalzahlenerkennung aus;
\texttt{100\{.\}200} gelten als zwei Zahlen \verb|100| und \verb|200| hintereinander.

Sprachunabh"angig gelten einzelne Leerzeichen, \verb|\,| oder 
\verb|~| \textit{vor Dreierkolonnen von Zahlzeichen} nicht als 
Unterbrechung einer Zahl. Dementsprechend steht \verb|1\,000\,000| 
immer f"ur eine Million. Und \verb|1000| ist
immer tausend.\footnote{Die Zahlenerkennung funktioniert bis\hspace{.1em}
\texttt{999.999.999.999.999} vor dem `Komma' (deutsch) und bis zu zus"atzlich
16 Stellen nach dem `Komma' (wobei in den Nachkommastellen keine Strukturierungszeichen enthalten
sein d"urfen: \texttt{0,0000000000000001} ist die kleinste korrekt sortierbare Zahl).
Ziffern nach der 15ten bzw.\ 16ten Stelle werden ignoriert von den `h"oheren' 
Sortierschichten (die zuerst die Reihenfolge bestimmen) 
und als neue Zahlen begriffen.}

Negative Zahlen werden f"alschlicherweise nach ihrem Betrag sortiert. 
F"ur ein Buch, das 700 v.\,Chr.\ erschien, k"onnen Sie 1/700 im  
Taschenrechner bestimmen und unter \verb|-g1| oder \verb|-g2| dann 
\verb|\sort{0,001429}700 v.\,Chr.| im\pdfko{1}\ 
Text setzen;
entsprechend lassen sich alle `negativen' Jahre vor das 
Jahr \verb|+1|\pdfko{1}\ 
einsortieren (das Jahr \verb|0| kommt aber 
weiterhin vor \textit{allen} anderen Zahlen).

\vspace{2ex}\noindent
Es gibt weitere Optionen f"ur \verb|bibsort| zum
\textbf{Sortieren der Argumente}:

\vspace{1ex}
Etwas Anderes als Punkte zwischen Zahlen sind Punkte nach
Buchstaben, n"amlich Abk"urzungen. Die Option\hspace{-.2em} \verb|-p| stellt ein,\label{plit}
dass ein `\verb|.|' im Unterschied zu anderen 
Satzzeichen wie ein Buchstabe z"ahlt
und dementsprechend in der obersten Sortierschicht Gewicht bekommt:
\verb|P.S.| kommt dann vor \verb|Peter|.

\vspace{1ex}
\verb|-l| bringt \verb|bibsort| dazu, Leerzeichen nicht zu gewichten.
Dann wird \verb|DiFabio| vor \verb|Di Niro| einsortiert (entgegen der
Grundeinstellung). \verb|-l|~wirkt sich allerdings nur auf Leerzeichen 
\textit{in} Argumenten der \BibArts\hy Befehle aus; falls Sie 
\verb|\vli{Di}{Niro}{|...\verb|}| tippen, hat es keine Auswirkung.
Bitte beachten Sie, dass \verb*|P. S.| und \verb|P.\,S.|
anders als \verb|P.S.| von \verb|-l| beeinflusst werden.

\vspace{1ex}
\verb|-i=j| sortiert \verb|i| unter \verb|j|; Zug"ange mit  
\textit{beiden} Anfangsbuchstaben bilden in den Listen also 
\textit{einen} Block (Zettelkataloge nach Preu"sischen Instruktionen).

\vspace{1ex}
\verb|-t1| stellt einen Versuch dar, Zeichen aus der zweiten
H"alfte der ASCII\hy Codetabelle zu sortieren.\footnote{In 
\texttt{bibsort.c} in \texttt{teinzerw} fix definiert.} 
Solche Zeichen erschienen in fr"uheren \LaTeX\hy Versionen im\hspace{-.15em} 
\verb|.aux|\hy File, falls kein \verb|\usepackage[cp850]{inputenc}| o.\,"A. 
im Dokumenten\hy Vorspann gesetzt und im Text trotzdem Zeichen des erweiterten Teils getippt wurden.
Mit neuesten \LaTeX\hy Versionen ist \verb|-t1| offenbar "uberfl"ussig, weil 
solche Zeichen im\hspace{-.15em} \verb|.aux|\hy File nicht mehr ankommen.

\vspace{2.5ex}\noindent
Neben \verb|\sort| \baref{sort} l"asst sich die Reihenfolge
mit dem Befehl \verb|\nosort| steuern. Sein Argument wird gedruckt, 
aber beim Sortieren weitgehend ausgeblendet. 
\verb|'a'| wird vor \verb|a| einsortiert, \verb|\nosort{'}a'| dahinter
(falls beide Eintr"age sonst nur \textit{gleiche} Buchstaben enthalten).
Dies liegt daran, dass in dieser `unteren' Sortierschicht 
ein Vergleich zwischen dem \textit{backslash} von \verb|\nosort| 
und dem Apostroph am Anfang von \verb|'a'| stattfindet; \texttt{bibsort}
sortiert (bei ihm unbekannten Befehlen) den \textit{backslash} 
hinter alle anderen Zeichen. Und beim Konflikt von
\verb|\nosort{'}a'| und \verb|\glqq|~\verb|a| kommen \verb|n| und \verb|g| zum Zuge.

\newpage
\noindent
\verb|bibsort| schreibt keine Protokolldatei, sondern setzt seine
Fehlermeldungen als Kommentarzeilen in die erzeugten Dateien. 
Wenn anders sortiert wird als erwartet, k"onnen Sie \ko\verb|-m| 
setzen; dann f"ugt \verb|bibsort| zu jedem Eintrag als\pdfko{.5}\   
\LaTeX\hy 
Kommentar seine zum Sortieren genutzten Meta\hy Zeilen hinzu.
(Die\pdfko{.25}\    
Sonderzeichen zur Nachbewertung bilden manche Editoren 
nur teilweise ab!)

\vspace{1ex}\noindent
Wie oben S.\,\pageref{head} vorgef"uhrt, bewirkt \verb|\bibsortheads|,
dass Bl"ocke mit gleichen Anfangsbuchstaben in den Listen mit 
"Uberschriftenbuchstaben versehen werden; \verb|\bibsortspaces|
setzt an diesen Stellen alternativ vergr"o"serte vertikale Abst"ande.
Die Buchstaben stellt \verb|bibsort| in allen neun
Dateien immer bereit;
\verb|\bibsortheads| ordnet nur an, dies nicht mehr auszublenden. 
Ein Umstellen der Schrift zum Drucken der "Uberschriftenbuchstaben 
ist nicht vorgesehen.

%%%%%%%%%%

\vspace{1.5ex}\noindent
Nun zu \textbf{Seiten- und Fu"snotennummern in num\hy Ausdruckbefehlen} (wie \verb|\printnumvkc|):
Die drucken hinter den Text des Listeneintrags die Seitenzahlen und eventuell Fu"snotennummern aus,
von denen mehrere textgleiche Zug"ange herstammen. Die Reihenfolge,
in der Zahlentypen ausgedruckt werden, hat \textbf{defaultm"a"sig} (ohne Setzen von Optionen) diese 
\textbf{Reihenfolge}:

\vspace{1.75ex}{\small\sffamily
\hspace{.25cm}\begin{tabular}{rl}
 T4   & \verb|\fnsymbol|,\footnotemark\ also \hspace{.5em} {\small 
 $* \hspace{1em} \dagger \hspace{1em} \ddagger \hspace{1em} \mathchar "278 \hspace{1em} \mathchar "27B \hspace{1em} \delimiter "026B30D \hspace{1em} ** \hspace{1em} \dagger \dagger \hspace{1em} \ddagger \ddagger$} \\[.2ex]
 T5,6 & r"omische Zahlen aus \,\verb|i v x l c d m|\,, dann aus \,\verb|I V X L C D M| \\[.2ex]
 T7   & arabische Zahlen aus \,\verb|0| \,bis \,\verb|9| \\[.2ex]
 T8,9 & Buchstaben\hy Z"ahler aus \,\verb|a| \,bis \,\verb|z|\,, dann aus \,\verb|A| \,bis \,\verb|Z| \\[.2ex]
 T10  & Zeichenfolgen, die nicht als Zahl (\ko an\ko)erkannt werden \\
\end{tabular}}\footnotetext{\texttt{\bs mathchar \string"278}
\,sowie \texttt{\bs ensuremath \{\bs mathsection \}} 
werden als $\mathchar "278$~(Symbolz"ahlerstand~4) akzeptiert,
\texttt{\bs mathchar \string"27B} 
\,sowie \texttt{\bs ensuremath \{\bs mathparagraph \}} 
als $\mathchar "27B$~(5). $-$ \texttt{bibsort} akzeptiert seit 2.1 zudem 
\texttt{\bs TextOrMath\{}\textit{Textmodus}\texttt{\}\{}\textit{Mathematikmodus}\texttt{\}} 
und zieht zur Bewertung das zweite Argument heran.} 

\vspace{2ex}\noindent
Ein Unterschied zwischen der Seiten- und Fu"snoten\hy Nummerierung ergibt sich
trotz der fixen Reihenfolge, in der \verb|bibsort| die Zahlentypen 
defaultm"a"sig ausdruckt, "uber die Reihenfolge, in der es seine Instrumente 
anwendet: Bei den \textbf{Seitenzahlen} \textit{pr"uft} es erst auf kleine r"omische Zahlen; 
und nur dann, falls es andere Zeichen als\hspace{.2em} \verb|i v x l c d m|\hspace{.2em} findet, auf kleine 
Buchstaben. Damit \textit{gilt} ein \verb|c| defaultm"a"sig als 100, auch wenn es Drei bedeuten \textit{soll}. 
Um dies zu "andern, bietet \verb|bibsort| zwei Typen von Schaltern an. Der erste
Typ tauscht einfach die Bewertungs\hy Instrumente:
Falls Sie in Ihrem Appendix \verb|\pagenumbering{alph}| wollen, k"onnen Sie \verb|bibsort|
mit \verb|-s1| starten. Dann \textit{d"urfen} Sie in Ihrer Einleitung zudem 
\verb|\pagenumbering{Roman}| nutzen (statt der dort defaultm"a"sig \textit{erlaubten} 
kleinen r"omischen Seitenzahlen). 

\vspace{1.5ex}\noindent
Beim Auslesen der \textbf{Fu"snotennummern} pr"uft \verb|bibsort| eine
eingelesene Zahl dagegen defaultm"a"sig darauf ab, ob es sich um eine
r"omische Zahl in Gro"sbuchstaben handelt; weitert erwartet
\verb|bibsort| die auch von \LaTeX\ defaultm"a"sig in \texttt{minipages}
verwendeten Kleinbuchstaben\hy Fu"snotenexponenten. Bei den Fu"snoten dreht
\ko\verb|-f1| die Basis\hy Bewertungsreihenfolge um: Dann sind kleine r"omische Zahlen
und Gro"sbuchstaben als Fu"snotennummern m"oglich (freilich neben dem
stets m"oglichen {\small\verb|\fnsymbol|} und den arabischen Zahlen).\footnote
{Falls Sie \texttt{\bs renewcommand\{\bs thefootnote\}\{\bs Alph\{footnote\}\}}
ohne \texttt{-f1} verwenden, wird \texttt{bibsort} 
textgleiche Zug"ange aus den Fu"snoten 
{\renewcommand{\thefootnote}{\Alph{footnote}}\footnotemark[1] \footnotemark[2] \footnotemark[3]}
nicht zu $^{A-C}$ zusammenfassen, sondern 
$^{C}$$^{,}$ $^{A}$$^{,}$ $^{B}$ ausdrucken,
denn \texttt{C} w"urde als r"omische Zahl (\textsf{T6}) gesehen und vor 
den als Gro"sbuchstaben (\textsf{T9}) bewerteten \texttt{A} und \texttt{B} 
einsortiert.
(Falls Sie alternativ \texttt{-f2} \textit{xxxx}\pdfko{.1}\ nutzen, aber \texttt{A} in
\textit{xxxx} nicht vorkommt, gelten die Fu"snoten \texttt{A} und \texttt{B} als \textsf{T10}).}

\vspace{1.75ex}\noindent
Mit einem zweiten Typ von Schalter k"onnen Sie die \textbf{Reihenfolge} 
einstellen, in der \verb|bibsort| \textsf{T4} bis \textsf{T9} 
bewertet \textit{und ausdruckt}: mit \verb|bibsort|~\verb|-s2|~\textit{xxxx}\pdfko{.5}\ 
die Seitenzahlen und separat mit \verb|-f2|~\textit{xxxx}
die Fu"snotennummern. \textit{xxxx} muss\pdfko{.75}\ 
vier der sechs Buchstaben \verb|a|, \verb|A|, \verb|r|, 
\verb|R|, \verb|n| und \verb|s| enthalten, wobei 
\verb|a|~alph, \verb|A|~Alph, \verb|n|~arabic, 
\verb|r|~roman, \verb|R|~Roman und \verb|s|~{\small\verb|\fnsymbol|} bedeutet;
in dieser Reihenfolge wird dann der Zahlenindex in den
num\hy Listen gedruckt. Es stehen vier statt alle sechs
Attribute zur Auswahl, weil etwa die Seitennummern \verb|c| oder \verb|C|
weiterhin sowohl einer Buchstabenz"ahlung wie auch einer r"omischen 
Z"ahlung entstammen k"onnten: Was gemeint ist, sagen Sie 
\texttt{bibsort} nun explizit. \verb|bibsort| stoppt mit einer
Fehlermeldung, falls Sie in \textit{xxxx} \verb|A| \textbf{und} \verb|R| setzen, 
oder falls Sie \verb|a| \textbf{und} \verb|r| setzen.
\verb|A| und \verb|a| k"onnen Sie aber in beliebiger Reihenfolge setzen, 
alternativ auch \verb|A| und \verb|r|, \verb|R| und \verb|r|, oder
\verb|R| und \verb|a|. Das bedeutet gleichzeitig, dass Sie 
z.\,B. {\small\verb|Roman|} in Ihrem Text dann als Seitenz"ahler nicht verwenden 
d"urfen, falls Sie dazu bereits {\small\verb|Alph|} nutzen. Auf in \textit{xxxx} ungenannte Z"ahler wird 
in diesem Modus nicht mehr gepr"uft; sie gelten (wie korrupte Nummern) 
als \textsf{T10}\,!\hspace{.2em} Stets m"ussen \verb|n| und \verb|s| in \textit{xxxx} 
getippt werden (auch wenn Sie beispielsweise {\small\verb|\fnsymbol|} im Text gar nicht 
nutzen), um auf die stets geforderten vier Buchstaben zu kommen. 
Durch mehrere Starts von \texttt{bibsort} erzeugte ich 
aus einer dazwischen unver"anderten \LaTeX\hy Textdatei "uber unterschiedlich
gesetzte Optionen diese verschiedenen Zahlenkolonnen:


\vspace{1.25ex}\noindent
\verb| bibsort -s2 sArn -f2 snRa |\abra{...}\verb|    |\textsf{A$^{I}$--C$^{III}$$^{,}$\ $^{d-f}$, i$^{*}$--v$^{\mathparagraph }$, 1$^{1}$--5$^{5}$}

\vspace{.5ex}\noindent
\verb| bibsort -s2 sArn -f2 asnR |\abra{...}\verb|    |\textsf{A$^{I}$,\ B$^{II}$, C$^{d-f}$$^{,}$\ $^{III}$, i$^{*}$--v$^{\mathparagraph }$, 1$^{1}$--5$^{5}$}

\vspace{.5ex}\noindent
\verb| bibsort -s2 nsrR          |\abra{...}\verb|    |\textsf{1$^{1}$--5$^{5}$, i$^{*}$--v$^{\mathparagraph }$, C$^{III}$$^{,}$\ $^{d-f}$, \{{A\/}\}$^{I}$, \{{B\/}\}$^{II}$}

\vspace{1.75ex}\noindent
Als unbewertbar bewertete `Zahlen' (\textsf{T10}) druckt \verb|bibsort| 
alphabetisch sortiert in geschweiften Klammern
aus; es gibt keine Zusammenfassung etwa zu \textsf{A--C}.
Sie k"onnen mit \verb|-c| den Ausdruck der geschweiften Klammern 
unterdr"ucken. Leere Z"ahlerstandsausdrucke erscheinen als\hspace{.1em} \textsf{[]}\hspace{.1em}  
und die \LaTeX\hy Fehlermeldung\hspace{.15em} \verb|Counter too large|\hspace{.15em}
als\hspace{.2em} \textsf{()}\,;\footnote{Ausdruck eines Z"ahlerstandes von \texttt{0} als r"omische Zahl bzw.\
gr"o"ser \texttt{26} als Buchstabe.} beides l"asst sich nicht ausschalten.

\vspace{2ex}\noindent
\verb|bibsort| akzeptiert in den Zahlenargumenten die 
"ublichen Befehle zur Einstellung des Schriftgrades. 
Beispielsweise akzeptiert \BibArts\ Ihre Eingabe: 

\vspace{1ex}\noindent{\small
\verb|  \renewcommand{\thempfootnote}{{\itshape\Alph{mpfootnote}}}|}

\vspace{1.25ex}\noindent
Entsprechendes gilt f"ur \verb|\thepage| und \verb|\thefootnote|.
Schriftgr"o"sen\hy Befehle wie \verb|\large| weist \verb|bibsort| dagegen 
zur Index\hy Zahlenverarbeitung zur"uck und wertet solche Nummern als 
\textsf{T10} (TEXT). Es gibt aber \verb|\bapageframe| und\pdfko{.75}\ 
\verb|\bafootnoteframe|, um Befehlscode oder Text vor \verb|bibsort| zu
verbergen:

\vspace{1.25ex}\noindent{\small
\verb|  \renewcommand{\thepage}{{\bapageframe{\roman{page}}}}| \\[.25ex]
\verb|  \renewcommand{\thefootnote}{{\bafootnoteframe{\arabic{footnote}}}}|}

\vspace{1.5ex}\noindent
Die drucken \textit{in Voreinstellung} die Seitenzahl und die Fu"snotenexponenten 
\textit{auf der Seite} in Schr"agstrichen aus. \verb|bibsort| 
druckt die Schr"agstriche nicht aus, erkennt aber den Wert der Z"ahler. 
(Andere Programme wie \textsc{MakeIndex} akzeptieren derart ver"anderte 
Z"ahler jedoch nicht mehr!) 

\vspace{1ex}\noindent
Falls Sie andere Symbole ausgedruckt haben wollen, hier ein Beispiel,
um die Seitenzahl in geschweiften Klammern und die 
Fu"snotennummer fett in fetten runden Klammern auszudrucken 
(sowie auch \LaTeX- und \BibArts\kern.1em\hy Querverweise
\verb|\ref{|\textit{xyz}\verb|}| und \verb|\baref{|\textit{xyz}\verb|}|, die auf 
solche Seiten oder Fu"snoten zeigen):

\vspace{1.25ex}\noindent{\small
\verb|  \renewcommand{\pbapageframe}[1]{\{#1\}}| \\[.4ex]
\verb|  \renewcommand{\pbafootnoteframe}[1]{(#1)}| \\[.4ex]
\verb|  \renewcommand{\thepage}{{\bapageframe{\roman{page}}}}| \\[.6ex]
\verb|  \renewcommand{\thefootnote}| \\[.3ex]
\verb|      {{\bfseries\bafootnoteframe{\arabic{footnote}}}}|}


\vspace{2.5ex}\noindent
Falls Sie\hspace{.2em} \verb|bibsort|~\textit{file}\hspace{.1em} befehlen, 
liest \verb|bibsort| das von \LaTeX\ zuvor erzeugte\hspace{.15em} \textit{file}\ko\verb|.aux|\hspace{.15em} 
ein und erzeugt die bis zu neun \BibArts\kern.1em\hy
Dateien (etwa\hspace{.15em} \textit{file}\ko\verb|.vli|\kern.1em, falls Sie vli\fhy Befehle verwenden).  
Alle bei \textit{einem} \texttt{bibsort}\hy Start erzeugte Dateien haben dasselbe Pr"afix, \textit{defaultm"a"sig}
dasjenige der Input\hy Datei, also \textit{file}\kern.01em.\hspace{.1ex} 
Seit \BibArts~2.1 muss ein abweichendes Pr"afix\hspace{.1em} \textit{outfile}\hspace{.15em} 
mit\hspace{.1em} \verb|-o|\hspace{.2em} angek"undigt werden:

\vspace{1.5ex}
\verb|bibsort |\textit{file}\verb| -o |\textit{outfile}

\vspace{1.75ex}\noindent
Weiter \textit{kann} das Pr"afix der Input\hy Datei beim Aufruf von\hspace{.15em} \verb|bibsort|\hspace{.15em} 
seit 2.1 mit der Option\hspace{.1em} 
\verb|-i|~\textit{file}\hspace{.15em} nun explizit gekennzeichnet 
werden. So lassen sich auch Dateien\hspace{.15em} \textit{file}\ko\verb|.aux|\hspace{.15em} bearbeiten, deren 
Name\hspace{.15em} mit einem Minuszeichen beginnt.


%%

\vspace{1ex}\vfill\noindent
\verb|bibsort| sortiert {\small\verb|\fnsymbol|} in Grundeinstellung
deswegen zuerst ein, weil \LaTeX\ diese Marken f"ur Fu"snoten in 
seiner Titelkonstruktion vorsieht. Innerhalb des Arguments
von {\small\verb|\title|} separiere ich im folgendem Beispiel {\small\verb|\footnote|}\pdfko{.2}\ in 
{\small\verb|\footnotemark|} und {\small\verb|\footnotetext|}, sonst droht ein Speicher"uberlauf:


\newpage

\vspace{.5cm}
\begin{tt}\small\noindent
\%\% Ein Text mit einer Fussnote im Argument des \bs title-Befehls \%\%\\[2ex]
\bs documentclass[12pt,a4paper]\{article\} \\[.25ex]
\hspace*{.5cm}   \bs usepackage[T1]\{fontenc\} \ \bs usepackage[utf8]\{inputenc\} \\
\hspace*{.5cm}   \bs usepackage\{ngerman\} \ \ \ \ \ \bs usepackage\{bibarts\} \\[2ex]
\hspace*{.5cm}   \bs author\{Peter Maier\} \\
\hspace*{.5cm}   \bs title\{Aufsatz\bs footnotemark[1]\} \\[4ex]
\bs begin\{document\} \\[2ex]
\hspace*{.25cm}  \%\% Aus \bs title ausgelagerte Eingabe des Fussnotentextes \\
\hspace*{.25cm}  \%\% und lokales Anpassen des Exponenten * im Fussnotenbereich: \\[.75ex]
\hspace*{.25cm}  \{\bs renewcommand\{\bs thefootnote\}\{\bs fnsymbol\{footnote\}\} \\[.75ex]
\hspace*{.5cm} \bs footnotetext[1]\{Vgl.\bs\ dazu \bs vli\{Niall\}\{Ferguson\}\{Der \\[.25ex]
\hspace*{1cm}    \bs ktit\{\bs onlykurz\{F\}\bs onlyvoll\{f\}alsche\bs onlykurz\{r\} Krieg\}, \\[.4ex]
\hspace*{1cm}    M"unchen 2001\}[22].\} \\
\hspace*{.25cm} \} \\[2ex]
\bs maketitle \\[2ex]
\bs noindent \\
Der erste Satz.\bs footnote\{\bs kli\{Ferguson\}\{Falscher Krieg\}[23].\} \\[2ex]
\bs end\{document\}
\end{tt}

\vfill\noindent
\begin{minipage}{.98\textwidth}
\vspace{.75cm}\noindent
{\renewcommand{\thempfootnote}{\fnsymbol{mpfootnote}}%
\begin{center}
{\LARGE \hspace*{.4em}Aufsatz$^{*}$}\\[4.05ex]
{\large Peter Maier}\\[2.65ex]
{\large \today}\\
\end{center}
 \footnotetext[1]{Vgl.\ dazu \vli{Niall}{Ferguson}{Der
 \ktit{\onlykurz{F}\onlyvoll{f}alsche\onlykurz{r} Krieg}, 
       M"unchen 2001}[22].}%
}%
{\renewcommand{\thempfootnote}{\arabic{mpfootnote}}%
\vspace{2ex}\noindent
Der erste Satz.\footnote{\kli{Ferguson}{Falscher Krieg}[23].\vspace{1ex}}}
\vspace{.75cm}
\end{minipage}



\newpage
\thispagestyle{empty}
\section*{Inhaltsbeschreibung}\label{SectIn}
\newcommand\tocline[2]{\hbox to .5cm{\hfill\bfseries{\ref{#1}}} \ {#2} \dotfill\hbox to .5cm {\hfill\ttfamily\pageref{#1}}}

\strut \\[-.5ex]
\hspace*{2cm}\textsf{Zun"achst werden die zentralen \BibArts\hy Befehle erkl"art:} \\[.875ex]
\tocline{Sect1} {Vollzitate und Kurzzitate (\ko\textit{v}\hy\ und \textit{k}\fhy Befehle)} \\[.25ex]
\tocline{Sect2} {W"ortliche Zitate in verschiedenen Sprachen} \\[.25ex]
\tocline{Sect3} {Formatierungs- und Editionshilfen} \\[.25ex]
\tocline{Sect4} {Abk"urzungen} \\[.25ex]
\tocline{Sect5} {\texttt{\bs abk\{X.X.X.\}} unter \texttt{\bs nonfrenchspacing}} \\[.25ex]
\tocline{Sect6} {Zeitschriften und allgemein Bandangaben} \\[.25ex]
\tocline{Sect7} {Archivquellen} \\[.25ex]
\tocline{Sect8} {Orts\fhy, Sach\hy\ und Personenregister} \\[1.875ex]
\hspace*{2cm}\textsf{Dann beschreibe ich Sonderf"alle und Hintergrundbefehle:} \\[.875ex]
\tocline{Sect9} {\texttt{\bs protect} und zerbrechliche Befehle} \\[.25ex]
\tocline{Sect10}{Punkte, \,\texttt{\bs bahasdot} \,und \,\texttt{\bs banotdot}} \\[.25ex]
\tocline{Sect11}{\textit{Italics}\hy Korrekturen und Separatoren} \\[.25ex]
\tocline{Sect12}{Sprachabh"angige Separatoren (\kern-.02em\textit{captions})} \\[1.875ex]
\hspace*{2cm}\textsf{Hier kommen Zusammenstellungen nach Aufgabentyp:} \\[.875ex]
\tocline{Sect13}{Die \BibArts\kern .1em\hy Hauptbefehle} \\[.25ex]
\tocline{Sect14}{Schrifteinstellung in \BibArts\kern .1em\hy Argumenten} \\[.25ex]
\tocline{Sect15}{\BibArts\kern .1em\hy Ein\fhy\ko/\ko\ko Ausschalter} \\[.25ex]
\tocline{Sect16}{\BibArts\kern .1em\hy\hspace{-.025em}1.3\hspace{.075em}\hy Texte unter \BibArts~2.x} \\[.25ex]
\tocline{Sect17}{Listenausdruck (\BibArts\kern .1em\hy Belegapparat)} \\[1.875ex]
\hspace*{2cm}\textsf{Und zuletzt folgen Sortierprogramm und Sortierreihenfolge:} \\[.875ex]
\tocline{Sect18}{\texttt{bibsort} samt Neuerungen seit Version~2.0} \\ \strut



\vfill
\hbox{\parbox{7.7cm}{\footnotesize\noindent
\textbf{\BibArts~2.2 (9 Dateien, 8 vom 03.03.2019):} \\[.85ex]
 \begin{tabular}{ll}%
 \texttt{README.txt}   & Versionsgeschichte seit 1.3        \\[-1.75pt]
 \texttt{bibarts.sty}  & Das \LaTeX-Style-File              \\[-1.75pt]
 \texttt{bibarts.pdf}  & Diese Dokumentation hier           \\[-1.75pt]
 \texttt{bibarts.tex}  & Quellcode von \texttt{bibarts.pdf} \\[-1.75pt]
 \texttt{ba-short.pdf} & Englische Kurzdokumentation        \\[-1.75pt]
 \texttt{ba-short.tex} & Quellcode von \texttt{ba-short.pdf}\\[-1.75pt]
 \texttt{bibsort.exe}  & Bin"ardatei f"ur die Listen        \\[-1.75pt]
 \texttt{bibsort.c}    & Quellcode von \texttt{bibsort.exe} \\[-1.75pt]
 \texttt{COPYING}      & Lizenz (vom 28.11.1993)                \\
 \end{tabular}} 
\\
\hspace*{2.4mm}
\parbox{5.6cm}{\vspace{1.6ex}\sethyphenation{UKenglish}\tiny\sffamily
    This program is free software; you can redistribute it and/or modify
    it under the terms of the GNU General Public License as published by
    the Free Software Foundation; either version 2 of the License, or
    (at your option) any later version.

    This program is distributed in the hope that it will be useful,
    but WITHOUT ANY WARRANTY; without even the implied warranty of
    MERCHANTABILITY or FITNESS FOR A PARTICULAR PURPOSE.  See the
    GNU General Public License for more details.

    You should have received a copy of the GNU General Public License
    along with this program; if not, write to the Free Software
    Foundation, Inc., 675 Mass Ave, Cambridge, MA 02139, USA.
}}

{\footnotesize\vspace{1.5ex}\noindent \BibArts\ ist kostenlos. Bitte
dokumentieren Sie "Anderungen vor der Weitergabe. \\[-.5ex]
Zur Diskussion k"onnen Sie mir an\hspace{.1em}
\texttt{bibarts}\kern.1em(\kern-.05em at\kern-.075em)\kern.1em\texttt{gmx.de}\hspace{.1em} 
schreiben.}


%%%%%%%%%%%%%%%%%%%%%%%%%%%%%%%%%%%%


%\tableofcontents

\end{document}

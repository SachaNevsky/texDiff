\documentclass{article}
\usepackage{parskip}
\usepackage{booktabs}

\usepackage{lmodern}
\usepackage{textcomp}
\usepackage[LGR,T1]{fontenc}

% UTF8 input encoding for pdfTeX or LuaTeX in 8-bit compatibility mode:
% (XeTeX in compatibility mode would require xetex-inputenc.sty, which is
% not at CTAN but on https://github.com/wspr/xetex-inputenc)
\usepackage[utf8]{luainputenc}

\usepackage[pdfencoding=auto,colorlinks=true,linkcolor=blue]{hyperref}
\usepackage{bookmark}
% hyperrefs PU encoding supports Greek LICR macros
\DeclareTextCommand{\ensuregreek}{PU}[1]{#1}

\usepackage[normalize-symbols, % comment option out to test error reporting
            keep-semicolon%
           ]{textalpha}

\begin{document}

\title{The \emph{textalpha} package}
\author{Günter Milde}
\maketitle

\abstract{\noindent
The \emph{textalpha} package enables the use of Greek characters
in text independent of font encoding or TeX engine. Input is possible via
text commands (\verb|\textalpha| \ldots
\verb|\textOmega|) or Unicode literals\footnote{%
Requires \emph{\href{http://www.ctan.org/pkg/greek-inputenc}{greek-inputenc}}
or XeTeX/LuaTeX.}.
}
\tableofcontents

\section{Usage}

Load this package in the preamble of your document with
\begin{verbatim}
  \usepackage{textalpha}
\end{verbatim}
eventually with options \texttt{normalize-symbols} and/or
\texttt{keep-semicolon}.

If \emph{textalpha} is loaded after the setup of Unicode fonts with the
\href{http://www.ctan.org/pkg/fontspec}{\emph{fontspec}} package under
LuaTeX and XeTeX, it provides a compatible interface for Greek in text mode.


See the source of this document \url{textalpha-doc.tex} for a setup and
usage example, the literate source of the package
\href{textalpha.sty.html}{textalpha.sty} for the implementation, and
\href{tuenc-greek-doc.pdf}{tuenc-greek-doc.pdf} for
\emph{Font setup for Greek with XeTeX/LuaTeX}.


\subsection{option \texttt{normalize-symbols}}

Mathematical notation uses variant shapes of some Greek letters as
additional symbols.  There are separate code points for the symbol variants
in Unicode. TeX supports some of the variant shape symbols in mathematical
mode ($\theta|\vartheta, \phi|\varphi, \pi|\varpi, \rho|\varrho,
\epsilon|\varepsilon$) but not in the LGR font encoding used for Greek in
8-bit TeX.

The variations have no syntactic meaning in Greek text and text fonts may
use the variant shapes in place of the “regular” ones as a stylistic choice.
However, some Greek texts use these GREEK ... SYMBOL characters in place of
the corresponding GREEK LETTER ... characters.

The \texttt{normalize-symbols} option merges letters and symbols to Greek
letters.
This way, text copied from external sources can be compiled without
errors even if it contains GREEK SYMBOL characters in place of GREEK LETTERS:
\begin{quote}
  This text uses both variants for beta (β|ϐ), theta (θ|ϑ), phi (φ|ϕ), pi
  (π|ϖ), kappa (κ|ϰ), rho (ρ|ϱ), Theta (Θ|ϴ), and epsilon (ε|ϵ) in the LaTeX
  source.

  \greekscript
  This text uses both variants for beta (β|ϐ), theta (θ|ϑ), phi (φ|ϕ), pi
  (π|ϖ), kappa (κ|ϰ), rho (ρ|ϱ), Theta (Θ|ϴ), and epsilon (ε|ϵ) in the LaTeX
  source.
\end{quote}

\textbf{Attention}: Do not use this option in cases where the distinction
between the symbol variants may be important (e.g. in a mathematical or
scientific context). Try the \emph{alphabeta} package with the
respective characters in math mode or use XeTeX/LuaTeX with Unicode fonts in
these cases.

This option is ignored with Unicode fonts.

\subsection{\texttt{keep-semicolon}}

LGR is no ``standard font encoding''. Latin characters and some other ASCII
symbols are mapped to Greek ``equivalents'' if LGR is the active font
encoding. (See
\href{http://mirrors.ctan.org/language/babel/contrib/greek/usage.pdf}{usage.pdf}
for a description of this Latin-Greek transliteration.)

Special care is required with the question mark characters: The LGR font
encoding uses the Latin question mark as input for the \emph{erotimatiko}
and maps the semicolon to a middle dot (\emph{ano teleia}).
As a result, Unicode-encoded texts that use the semicolon as
\emph{erotimatiko} end up with an \emph{ano teleia} in its place! 
Without special care, only the deprecated character 037E GREEK QUESTION MARK%
\footnote{The Unicode standard provides the code point 037E GREEK QUESTION MARK
        but says character 003B SEMICOLON and not 037E is the preferred
        character for a `Greek question mark' (erotimatiko).}
works with both, Xe/LuaTeX and 8-bit TeX.

The \verb|\textsemicolon| command inserts an \emph{erotimatiko} in LGR and a
semicolon else (i.e. always a character that looks like a semicolon):
\begin{quote}
  Latin (T1) a\textsemicolon{} b,
  Greek (LGR) \ensuregreek{a\textsemicolon{} b}
\end{quote}

With the \texttt{keep-semicolon} option, character 003B SEMICOLON can be used
for the \emph{erotimatiko} also with LGR encoded fonts:

\begin{center}
\begin{tabular}{lll}
  Latin (T1) & Greek (LGR) & question mark character \\
  \midrule
  Τί φήις; & \ensuregreek{Τί φήις;} & 037E GREEK QUESTION MARK \\
  Τί φήις; & \ensuregreek{Τί φήις;} & 003B SEMICOLON \\
  Τί φήις? & \ensuregreek{Τί φήις?} & 003F QUESTION MARK \\
\end{tabular}
\end{center}

This option is ignored with Unicode fonts.


\section{Limitations \label{sec:limitations}}

Because the internal font encoding switch interferes with other work behind
the scenes, kerning, diacritics and up/down-casing show problems if Greek
letters are used without explicit change of the font encoding.

These problems can be avoided by use of \emph{babel} and the correct
language setting (greek), an explicit font encoding switch (e.g.
wrapping the Greek text in \verb|\ensuregreek|%
\footnote{The \texttt{\textbackslash ensuregreek} macro ensures the argument
  is set in a font encoding supporting Greek.
  % This can be used to fix these
  % problems without adverse side-effects if the active font encoding is
  % already LGR or TU.
  },
or XeTeX/LuaTeX with Unicode fonts.


\subsection{Diacritics}

Composition of diacritics (like \verb|\accdasia\acctonos| or \verb|\<\'|)
fails in other font encodings. Long names (like \verb|\accdasiaoxia|) work.

With LGR, pre-composed glyphs are chosen if available. In other font
encodings, accent macros do not select pre-composed characters. (The
difference is a sub-optimal placement of the accent and becomes obvious if
you drag-and-drop text from the PDF version of this document.):

\begin{quote}
  \ensuregreek{\<'a \accdasia\acctonos a \accdasiaoxia a
  \accdasiaoxia\textalpha} (LGR) vs. \accdasiaoxia\textalpha{} (T1).
\end{quote}

According to Greek typographical tradition, diacritics (except the
dialytika) are placed before capital letters in Titlecase and dropped in
UPPERCASE:
%
\begin{quote}
  \ensuregreek{%
    \<{\textalpha} \>{\textepsilon} \"'{\textiota} \`>\texteta{}
    \'<{\textomicron} \~<{\textupsilon} \~>{\textomega}
    \quad
    \<{\textAlpha} \>{\textEpsilon} \"{\textIota} \`>\textEta{}
    \'<{\textOmicron} \~<{\textUpsilon} \~>{\textOmega}
    \quad
    \MakeUppercase{%
      \<{\textalpha} \>{\textepsilon} \"'{\textiota} \`\>\texteta{}
      \'<{\textomicron} \~<{\textupsilon} \~>{\textomega}
    }
  }
\end{quote}
%
This fails for accent macros if the active font encoding does not support
Greek. Pre-composed literal Unicode characters are handled correctly:
\begin{quote}
    \begin{tabular}{ccc}
        & LICR                       & Unicode         \\ \hline 
    LGR & \ensuregreek{\'\textAlpha} & \ensuregreek{Ά} \\
    T1  & \'\textAlpha               & Ά               \\
    \end{tabular}
\end{quote}

The dialytika marks a \emph{hiatus} (break-up of a diphthong). It must be
present in UPPERCASE even where it is redundant in lowercase (the hiatus can
also be marked by an accent on the first character of a diphthong). The
auto-hiatus feature works in LGR and TU font encodings only:
\begin{quote}
  \acctonos\textalpha\textupsilon{}, \acctonos\textepsilon\textiota{} $\mapsto$
  \MakeUppercase{\ensuregreek{
    \acctonos\textalpha\textupsilon{}, \acctonos\textepsilon\textiota{}
  }} (LGR) vs.
  \MakeUppercase{
    \acctonos\textalpha\textupsilon{}, \acctonos\textepsilon\textiota{}%
  } (T1)
\end{quote}

Currently, the second vowel of the diphthong must be given as macro, not
Unicode literal if the auto-hiatus feature should work:
\begin{quote}
\ensuregreek{ἀ\textupsilon{}πνία} $\mapsto$
\ensuregreek{\MakeUppercase{\ensuregreek{ἀ\textupsilon{}πνία}}} (LGR) vs.
\ensuregreek{\MakeUppercase{\ensuregreek{ἀυπνία}}} (T1).
\end{quote}

\subsection{Kerning}

No kerning occurs between Greek characters in non-Greek text due to the
internal font encoding switch: compare
\ensuregreek{\textAlpha\textUpsilon\textAlpha} (LGR) to
\textAlpha\textUpsilon\textAlpha (T1).

Compiling with LuaTeX provides kerning also on font encoding boundaries.

\section{Test and Examples}

\subsection{Greek alphabet}

Greek letters via Latin transcription in LGR font encoding:

\begin{quote}
  \ensuregreek{A B G D E Z H J I K L M N X O P R S T U F Q Y W}\\
  \ensuregreek{a b g d e z h j i k l m n x o p r sv c t u f q y w}
\end{quote}

\smallskip\noindent
Greek letters via default macros in other font encoding (here T1):

\begin{quote}
  \textAlpha{} \textBeta{} \textGamma{} \textDelta{} \textEpsilon{}
  \textZeta{} \textEta{} \textTheta{} \textIota{} \textKappa{}
  \textLambda{} \textMu{} \textNu{} \textXi{} \textOmicron{} \textPi{}
  \textRho{} \textSigma{} \textTau{} \textUpsilon{} \textPhi{}
  \textChi{} \textPsi{} \textOmega{}
  \\
  \textalpha{} \textbeta{} \textgamma{} \textdelta{} \textepsilon{}
  \textzeta{} \texteta{} \texttheta{} \textiota{} \textkappa{}
  \textlambda{} \textmu{} \textnu{} \textxi{} \textomicron{} \textpi{}
  \textrho{} \textsigma{} \textvarsigma{} \texttau{} \textupsilon{}
  \textphi{} \textchi{} \textpsi{} \textomega{}
\end{quote}

\smallskip\noindent
Archaic Greek letters and Greek punctuation

\begin{quote}
  \textDigamma
  \textanoteleia
  \textQoppa
  \textSampi
  \textStigma
  \textnumeralsigngreek
  \texterotimatiko
  \\
  \textdigamma
  \textkoppa
  \textqoppa
  \textsampi
  \textstigma
  \textnumeralsignlowergreek
  \textvarstigma
\end{quote}

\smallskip\noindent
Diacritics

\begin{quote}
  Symbol macros:%
  \footnote{Composite diacritics require wrapping in
  \texttt{\textbackslash ensuregreek}.}
  \"{} \'{} \`{} \~{} \<{} \>{} \u{} \={}
  \ensuregreek{\"~{} \"'{} \"`{} \<~{} \<`{} \<'{} \>~{} \>'{} \>`{}}

  Named macros:
  \accdialytika{}
  \acctonos{}
  \accvaria{}
  \accperispomeni{}
  \accdasia{}
  \accpsili{}
  \ypogegrammeni{}
  \prosgegrammeni{}
  %
  \accdialytikaperispomeni{}
  \accdialytikatonos{}
  \accdialytikavaria{}
  \accdasiaperispomeni{}
  \accdasiavaria{}
  \accdasiaoxia{}
  \accpsiliperispomeni{}
  \accpsilioxia{}
  \accpsilivaria{}
  %
  \accinvertedbrevebelow{} % == \textsubarch{}
  \accbrevebelow{}


\end{quote}

\medskip\noindent
Accent macros can start with \verb|\a| instead of \verb|\| when the
short form is redefined, e.\,g. inside a \emph{tabbing} environment.
This also works for the new-defined Dasia and Psili shortcuts:
\begin{quote}
  \begin{tabbing}
       COL1\quad \= COL2\quad \= COL3\quad \= COL4\quad \\
       COL1 \>          \> COL3 \\
       Viele    \> Gr\a"u\ss e
       \> \greekscript \a<\textalpha{}
       \> \greekscript \a>\textomega
  \end{tabbing}
\end{quote}


\subsection{Greek Unicode characters in non-Greek text}

With the \emph{textalpha} package,
\href{http://www.ctan.org/pkg/greek-inputenc}{greek-inputenc} and input
encoding \texttt{utf8}, Greek Unicode characters can be used in text with
any font encoding. See Tables \ref{tab:greek-and-coptic} and
\ref{tab:greek-extended}.

Kerning is preserved if the font encoding is LGR: \ensuregreek{AΫA}

\begin{table}[tbp]
\centerline{
\begin{tabular}{rrrrrrrrrrrrrrrrr}
\toprule
    & 0 & 1 & 2 & 3 & 4 & 5 & 6 & 7 & 8 & 9 & A & B & C & D & E & F\\
\midrule
370 & ◦ & ◦ & ◦ & ◦ & ʹ & ͵ & ◦ & ◦ &   &   & ͺ & ◦ & ◦ & ◦ & ; &  \\
380 &   &   &   &   & ΄ & ΅ & Ά & · & Έ & Ή & Ί &   & Ό &   & Ύ & Ώ\\
390 & ΐ & Α & Β & Γ & Δ & Ε & Ζ & Η & Θ & Ι & Κ & Λ & Μ & Ν & Ξ & Ο\\
3A0 & Π & Ρ &   & Σ & Τ & Υ & Φ & Χ & Ψ & Ω & Ϊ & Ϋ & ά & έ & ή & ί\\
3B0 & ΰ & α & β & γ & δ & ε & ζ & η & θ & ι & κ & λ & μ & ν & ξ & ο\\
3C0 & π & ρ & ς & σ & τ & υ & φ & χ & ψ & ω & ϊ & ϋ & ό & ύ & ώ &  \\
3D0 & ◦ & ◦ & ◦ & ◦ & ◦ & ◦ & ◦ & ◦ & Ϙ & ϙ & Ϛ & ϛ & Ϝ & ϝ & ◦ & ϟ\\
3E0 & Ϡ & ϡ & ◦ & ◦ & ◦ & ◦ & ◦ & ◦ & ◦ & ◦ & ◦ & ◦ & ◦ & ◦ & ◦ & ◦\\
3F0 & ◦ & ◦ & ◦ & ◦ & ◦ & ◦ & ◦ & ◦ & ◦ & ◦ & ◦ & ◦ & ◦ & ◦ & ◦ & ◦\\
\bottomrule
\end{tabular}
} % end centerline
\caption{Greek and Coptic Unicode Block, input as literal Unicode
   characters in T1 font encoding (legend: ◦ glyph missing in LGR).}
\label{tab:greek-and-coptic}
\end{table}


\begin{table}[tbp]
\centerline{
\begin{tabular}{rrrrrrrrrrrrrrrrr}
\toprule
     & 0 & 1 & 2 & 3 & 4 & 5 & 6 & 7 & 8 & 9 & A & B & C & D & E & F\\
\midrule
1F00 & ἀ & ἁ & ἂ & ἃ & ἄ & ἅ & ἆ & ἇ & Ἀ & Ἁ & Ἂ & Ἃ & Ἄ & Ἅ & Ἆ & Ἇ\\
1F10 & ἐ & ἑ & ἒ & ἓ & ἔ & ἕ &   &   & Ἐ & Ἑ & Ἒ & Ἓ & Ἔ & Ἕ &   &  \\
1F20 & ἠ & ἡ & ἢ & ἣ & ἤ & ἥ & ἦ & ἧ & Ἠ & Ἡ & Ἢ & Ἣ & Ἤ & Ἥ & Ἦ & Ἧ\\
1F30 & ἰ & ἱ & ἲ & ἳ & ἴ & ἵ & ἶ & ἷ & Ἰ & Ἱ & Ἲ & Ἳ & Ἴ & Ἵ & Ἶ & Ἷ\\
1F40 & ὀ & ὁ & ὂ & ὃ & ὄ & ὅ &   &   & Ὀ & Ὁ & Ὂ & Ὃ & Ὄ & Ὅ &   &  \\
1F50 & ὐ & ὑ & ὒ & ὓ & ὔ & ὕ & ὖ & ὗ &   & Ὑ &   & Ὓ &   & Ὕ &   & Ὗ\\
1F60 & ὠ & ὡ & ὢ & ὣ & ὤ & ὥ & ὦ & ὧ & Ὠ & Ὡ & Ὢ & Ὣ & Ὤ & Ὥ & Ὦ & Ὧ\\
1F70 & ὰ & ά & ὲ & έ & ὴ & ή & ὶ & ί & ὸ & ό & ὺ & ύ & ὼ & ώ &   &  \\
1F80 & ᾀ & ᾁ & ᾂ & ᾃ & ᾄ & ᾅ & ᾆ & ᾇ & ᾈ & ᾉ & ᾊ & ᾋ & ᾌ & ᾍ & ᾎ & ᾏ\\
1F90 & ᾐ & ᾑ & ᾒ & ᾓ & ᾔ & ᾕ & ᾖ & ᾗ & ᾘ & ᾙ & ᾚ & ᾛ & ᾜ & ᾝ & ᾞ & ᾟ\\
1FA0 & ᾠ & ᾡ & ᾢ & ᾣ & ᾤ & ᾥ & ᾦ & ᾧ & ᾨ & ᾩ & ᾪ & ᾫ & ᾬ & ᾭ & ᾮ & ᾯ\\
1FB0 & ᾰ & ᾱ & ᾲ & ᾳ & ᾴ &   & ᾶ & ᾷ & Ᾰ & Ᾱ & Ὰ & Ά & ᾼ & ᾽ & ι & ᾿\\
1FC0 & ῀ & ῁ & ῂ & ῃ & ῄ &   & ῆ & ῇ & Ὲ & Έ & Ὴ & Ή & ῌ & ῍ & ῎ & ῏\\
1FD0 & ῐ & ῑ & ῒ & ΐ &   &   & ῖ & ῗ & Ῐ & Ῑ & Ὶ & Ί &   & ῝ & ῞ & ῟\\
1FE0 & ῠ & ῡ & ῢ & ΰ & ῤ & ῥ & ῦ & ῧ & Ῠ & Ῡ & Ὺ & Ύ & Ῥ & ῭ & ΅ & `\\
1FF0 &   &   & ῲ & ῳ & ῴ &   & ῶ & ῷ & Ὸ & Ό & Ὼ & Ώ & ῼ & ´ & ῾ &  \\
\bottomrule
\end{tabular}
} % end centerline
\caption{Greek Extended Unicode Block, input as literal Unicode
characters in T1 font encoding.}
\label{tab:greek-extended}
\end{table}

Combined Diacritics work ᾅ, diacritics (except diaeresis) are dropped with
MakeUppercase (μαΐστρος $\mapsto$ \MakeUppercase{μαΐστρος}).

\subsection{PDF strings}

With \emph{textalpha} and
\emph{\href{http://www.ctan.org/pkg/greek-inputenc}{greek-inputenc}}, there
are two options to get Greek letters in PDF strings: LICR macros and literal
Unicode input.

\subsubsection{\textlambda\textomicron\textgamma\textomicron\textvarsigma{},
         λογος and \ensuregreek{logos}}

The subsection title above uses: LICR macros, Unicode input and the LGR
transcription for the Greek word \ensuregreek{logos}. Check the table of
contents in the PDF viewer: LICR macros and Unicode literals work fine, the
Latin transcription remains Latin in the PDF metadata.

\end{document}

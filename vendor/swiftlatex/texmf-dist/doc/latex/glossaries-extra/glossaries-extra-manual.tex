\documentclass[report,inlinetitle,noprint]{nlctdoc}

\makeatletter
\@ifpackageloaded{tex4ht}
{
  \DeclareUnicodeCharacter{00FE}{\HCode{&\#x00FE;}}
  \DeclareUnicodeCharacter{00D0}{\HCode{&\#x00D0;}}
  \DeclareUnicodeCharacter{01F7}{\HCode{&\#x01F7;}}
  \DeclareUnicodeCharacter{A77D}{\HCode{&\#xA77D;}}
  \DeclareUnicodeCharacter{017F}{\HCode{&\#x017F;}}
  \DeclareUnicodeCharacter{018F}{\HCode{&\#x018F;}}
}
{
  \usepackage{fontspec}
  \setmainfont{FreeSerif}
}
\makeatother

\usepackage{alltt}
\usepackage{hologo}
\usepackage[colorlinks,
            hyperindex=false,
            pdfauthor={Nicola L.C. Talbot},
            pdftitle={glossaries-extra.sty: an extension to the glossaries
package},
            pdfkeywords={LaTeX,package,glossary,abbreviations}]{hyperref}
\usepackage[nogroupskip,nonumberlist]{glossaries-extra}

\makeglossaries

\GlsXtrEnableEntryUnitCounting{general}{0}{page}

\renewcommand*{\glsxtrifcounttrigger}[3]{#3}

\renewcommand*{\glslinkcheckfirsthyperhook}{%
  \ifnum\glsentrycurrcount\glslabel>0
   \setkeys{glslink}{hyper=false}%
  \fi
}

\newcommand*{\XeLaTeX}{\hologo{XeLaTeX}}
\newcommand*{\LuaLaTeX}{\hologo{LuaLaTeX}}
\newcommand*{\pdfLaTeX}{\hologo{pdfLaTeX}}

\IndexPrologue{\chapter*{\indexname}
 \addcontentsline{toc}{chapter}{\indexname}%
 \markboth{\indexname}{\indexname}}

\renewcommand*{\main}[1]{\hyperpage{#1}}
\newcommand*{\htextbf}[1]{\textbf{\hyperpage{#1}}}

\glssetcategoryattribute{general}{indexname}{htextbf}
\glssetcategoryattribute{general}{dualindex}{main}

\newglossaryentry{xindy}{
  name={\appfmt{xindy}},
  sort={xindy},
  description={An flexible indexing application with multilingual
  support written in Perl.}
}

\newglossaryentry{makeindex}{%
  name={\appfmt{makeindex}},%
  sort={makeindex},%
  description={An indexing application.},
}

\newglossaryentry{makeglossaries}{%
name={\appfmt{makeglossaries}},%
sort={makeglossaries},%
description={A custom designed Perl script interface
provided with the \styfmt{glossaries} package
to run \gls{xindy} or \gls{makeindex} according to 
the document settings.}}

\newglossaryentry{makeglossaries-lite.lua}{%
name={\appfmt{makeglossaries-lite}\index{makeglossaries-lite=\appfmt{makeglossaries-lite}|htextbf}},%
sort={makeglossaries-lite},%
text={\appfmt{makeglossaries-lite}\iapp{makeglossaries-lite}},%
description={A custom designed Lua script interface
to \gls{xindy} and \gls{makeindex} provided with the
\styfmt{glossaries}
package. This is a cut-down alternative to the Perl
\appfmt{makeglossaries} script. If you have Perl installed, use the
Perl script instead. This script is distributed in the source code
on CTAN with the file name
\texttt{makeglossaries-lite.lua} but TeX Live on Unix-like systems 
creates a symbolic link
called \texttt{makeglossaries-lite} (without the \texttt{.lua}
extension) to the actual \texttt{makeglossaries-lite.lua} script, and 
\TeX\ distributions on Windows convert the script to an executable
\texttt{makeglossaries-lite.exe}.}
}

\newglossaryentry{bib2gls}{%
name={\appfmt{bib2gls}},%
sort={bib2gls},%
description={A command line Java application that selects
entries from a .bib file and converts them to glossary
definitions (like \appfmt{bibtex} but also performs hierarchical
sorting and collation, thus omitting the need for \gls{xindy} or
\gls{makeindex}). Further details at:
\url{http://www.dickimaw-books.com/software/bib2gls/}.}%
}

\newglossaryentry{numberlist}{%
name={number list},%
description={A list of \glspl{entrylocation} (also 
called a location list). The number list can be suppressed using the 
\pkgopt{nonumberlist} package option.}
}

\newglossaryentry{entrylocation}{%
name={entry location},%
description={The location of the entry in the document. This
defaults to the page number on which the entry appears. An entry may
have multiple locations.}
}

\newglossaryentry{locationlist}{%
name={location list},%
description={A list of \glspl{entrylocation}.
See \gls{numberlist}.}%
}

\newglossaryentry{linktext}{%
name={link-text},
description={The text produced by commands such as \ics{gls}. It may
or may not have a hyperlink to the glossary.}
}

\let\glsd\glsuseri
\let\glsation\glsuserii

\newglossaryentry{firstuseflag}{%
name={first use flag},
description={A conditional that determines whether or not the entry
has been used according to the rules of \gls{firstuse}.}%
}

\newglossaryentry{firstusetext}{%
name={first use text},
description={The text that is displayed on \gls{firstuse}, which is
governed by the \gloskey{first} and \gloskey{firstplural} keys of
\ics{newglossaryentry}. (May be overridden by
\ics{glsdisp}.)}%
}

\newglossaryentry{firstuse}{%
  name={first use},
  user1={first used},
  see={firstuseflag,firstusetext},
  description={The first time a glossary entry is used (from the start
  of the document or after a reset) with one of the
  following commands: \ics{gls}, \ics{Gls}, \ics{GLS}, \ics{glspl},
  \ics{Glspl}, \ics{GLSpl} or \ics{glsdisp}.}%
}

\setcounter{IndexColumns}{2}

\newrobustcmd*{\eq}{\protect=}% hide from makeindex

\newcommand*{\igloskey}[2][newglossaryentry]{\icsopt{#1}{#2}}
\newcommand*{\gloskey}[2][newglossaryentry]{\csopt{#1}{#2}}
\newcommand*{\optfmt}[1]{\textsf{#1}}

\newcommand*{\glostyle}[1]{\textsf{#1}\index{glossary styles:>#1={\protect\sffamily#1}|main}}

\newcommand*{\acrstyle}[1]{\textsf{#1}\index{acronym
styles (glossaries):>#1={\protect\ttfamily#1}|main}}

\newcommand*{\abbrstylenolink}[1]{%
 \textsf{#1}\index{abbreviation styles:>#1={\protect\sffamily#1}|main}}

\newcommand*{\abbrstyle}[1]{%
 \hyperlink{abbrstyle.#1}{\textsf{#1}}%
 \index{abbreviation styles:>#1={\protect\sffamily#1}|main}%
}

\makeatletter
\newcommand{\htarget}[2]{\Hy@raisedlink{\hypertarget{#1}{}}#2}
\makeatother

\newcommand*{\abbrstyledef}[1]{%
 \htarget{abbrstyle.#1}{\textsf{#1}}%
 \index{abbreviation styles:>#1={\protect\sffamily#1}|main}%
}

\newcommand*{\depabbrstyle}[1]{\textsf{#1}\index{abbreviation styles (deprecated):>#1={\protect\sffamily#1}|main}}

\newcommand*{\category}[1]{\textsf{#1}\index{categories:>#1={\protect\sffamily#1}|main}}

\newcommand*{\catattrnolink}[1]{%
 \textsf{#1}\index{category attributes:>#1={\protect\sffamily#1}|main}}

\newcommand*{\catattr}[1]{%
 \hyperlink{catattr.#1}{\textsf{#1}}%
 \index{category attributes:>#1={\protect\sffamily#1}|main}%
}

\newcommand*{\catattrdef}[1]{%
 \htarget{catattr.#1}{\textsf{#1}}%
 \index{category attributes:>#1={\protect\sffamily#1}|main}%
}

\newcommand*{\postlinkcat}{%
 \cs{glsxtrpostlink}\meta{category}%
 \index{glsxtrpostlinkcategory=\cs{glsxtrpostlink}\meta{category}|hyperpage}%
}

\renewcommand*{\pkgopt}[2][]{%
 \hyperlink{styopt.#2}{\pkgoptfmt{#2}}%
 \ifstrempty{#1}{}{\pkgoptfmt{=#1}}%
 \ipkgopt[#1]{#2}%
}

\newcommand*{\pkgoptdef}[1]{%
 \htarget{styopt.#1}{\pkgoptfmt{#1}}%
 \ipkgopt{#1}%
}

\setlength\marginparwidth{70pt}

\begin{document}
\DeleteShortVerb{\|}
\MakeShortVerb{"}

 \title{glossaries-extra.sty v1.44:
 an extension to the glossaries package}
 \author{Nicola L.C. Talbot\\[10pt]
Dickimaw Books\\
\url{http://www.dickimaw-books.com/}}

 \date{2020-03-23}
 \maketitle

\begin{abstract}
The \styfmt{glossaries-extra} package is an extension to the
\styfmt{glossaries} package, providing additional features.
Some of the features provided by this package are only available
with \styfmt{glossaries} version 4.19 (or above).
This document assumes familiarity with the \styfmt{glossaries} package.

The file \nolinkurl{example-glossaries-xr.tex} contains dummy
entries with cross-references that may be used for creating
minimal working examples for testing the \styfmt{glossaries-extra}
package. (The base \styfmt{glossaries} package provides additional files,
but this one needs \styfmt{glossaries-extra}.) There are equivalent
\texttt{.bib} files for use with \gls{bib2gls}.

\end{abstract}

\begin{important}
Since \styfmt{glossaries-extra} internally loads the \styfmt{glossaries}
package, you also need to have \styfmt{glossaries} installed and
all the packages that \styfmt{glossaries} depends on (including, but
not limited to, \sty{tracklang}, \sty{mfirstuc}, \sty{etoolbox}, 
\sty{xkeyval} (at least version dated 2006/11/18), \sty{textcase},
\sty{xfor}, \sty{datatool-base} and \sty{amsgen}. These
packages are all available in the current \TeX\ Live and Mik\TeX\
distributions. If any of them are missing, please update your
\TeX\ distribution using your update manager.
(For help on this see, for example,
\href{http://tex.stackexchange.com/questions/55437/how-do-i-update-my-tex-distribution}{How do I update my \TeX\ distribution?}
or
\href{http://tex.stackexchange.com/questions/14925/updating-tex-on-linux}{Updating
\TeX\ on Linux}.)
\end{important}

Additional resources:
\begin{itemize}
\item The \styfmt{glossaries-extra} documented code
\href{http://mirrors.ctan.org/macros/latex/contrib/glossaries-extra/glossaries-extra-code.pdf}{glossaries-extra-code.pdf}.

\item The
\href{http://www.dickimaw-books.com/gallery/#glossaries-extra}{\styfmt{glossaries-extra} gallery}.

\item \styfmt{glossaries-extra} and \gls{bib2gls}: An Introductory
Guide. (\href{http://mirrors.ctan.org/support/bib2gls/bib2gls-begin.pdf}{\texttt{bib2gls-begin.pdf}}).

\item
\href{https://www.dickimaw-books.com/latex/buildglossaries/}{Incorporating
makeglossaries or makeglossaries-lite or bib2gls into the document
build}.

\item The
\href{http://www.dickimaw-books.com/software/bib2gls}{\appfmt{bib2gls}}
application.

\item The base \href{https://ctan.org/pkg/glossaries}{\styfmt{glossaries}} package.
\end{itemize}

\tableofcontents

\chapter{Introduction}
\label{sec:intro}

The \styfmt{glossaries} package is a flexible package, but it's also a
heavy-weight package that uses a lot of resources. As package
developer, I'm caught between those users who complain about the
drawbacks of a heavy-weight package with a large user manual and
those users who want more features (which necessarily adds to the
package weight and manual size).

The \styfmt{glossaries-extra} package is an attempt to provide
a~compromise for this
conflict. Version 4.22 of the \styfmt{glossaries} package is
the last version to incorporate new features.\footnote{4.21 was
originally intended as the last release of \styfmt{glossaries} to
incorporate new features, but a few new minor features slipped in with
some bug fixes in v4.21.} Future versions of
\styfmt{glossaries} will just be bug fixes. New features will
instead be added to \styfmt{glossaries-extra}.
This means that the base \styfmt{glossaries} package won't increase
in terms of package loading time and allocation of resources, but
those users who do want extra features available will have more of a
chance of getting their feature requests accepted.

\section{Package Defaults}
\label{sec:defaults}

I'm not happy with some of the default settings assumed by the
\styfmt{glossaries} package, and, judging from code I've seen, other
users also seem unhappy with them, as certain package options are
often used in questions posted on various sites. I can't change the default
behaviour of \styfmt{glossaries} as it would break backward
compatibility, but since \styfmt{glossaries-extra} is a separate
package, I have decided to implement some of these commonly-used 
options by default. You can switch them back if they're not 
appropriate.

The new defaults are:
\begin{itemize}
\item \pkgopt[true]{toc} (add the glossaries to the table of
contents). Use \pkgopt[false]{toc} to switch this back off.

\item \pkgopt[true]{nopostdot} (suppress the terminating full stop
after the description in the glossary). Use
\pkgopt[false]{nopostdot} or just \pkgopt{postdot} to restore the
terminating full stop (period).

\item \pkgopt[true]{noredefwarn} (suppress the warnings that occur when
the \env{theglossary} environment and \cs{printglossary} are redefined while
\styfmt{glossaries} is loading). To restore the warnings, use
\pkgopt[false]{noredefwarn}. Note that this won't have any effect if
the \styfmt{glossaries} package has already been loaded before you use
the \styfmt{glossaries-extra} package.

\item If \sty{babel} has been loaded, the \pkgopt[babel]{translate}
option is switched on. To revert to using the \sty{translator}
interface, use \pkgopt[true]{translate}. There is no change to the
default if \sty{babel} hasn't been loaded.

\end{itemize}

The examples below illustrate the difference in explicit
package options between \styfmt{glossaries} and
\styfmt{glossaries-extra}. There may be other differences
resulting from modifications to commands provided by
\styfmt{glossaries} (see \sectionref{sec:modifications}).

\begin{enumerate}
\item 
\begin{verbatim}
\documentclass{article}
\usepackage{glossaries-extra}
\end{verbatim}
This is like:
\begin{verbatim}
\documentclass{article}
\usepackage[toc,nopostdot]{glossaries}
\usepackage{glossaries-extra}
\end{verbatim}

\item
\begin{verbatim}
\documentclass[british]{article}
\usepackage{babel}
\usepackage{glossaries-extra}
\end{verbatim}
This is like:
\begin{verbatim}
\documentclass[british]{article}
\usepackage{babel}
\usepackage[toc,nopostdot,translate=babel]{glossaries}
\usepackage{glossaries-extra}
\end{verbatim}

\item
\begin{verbatim}
\documentclass{memoir}
\usepackage{glossaries-extra}
\end{verbatim}
This is like:
\begin{verbatim}
\documentclass{memoir}
\usepackage[toc,nopostdot,noredefwarn]{glossaries}
\usepackage{glossaries-extra}
\end{verbatim}
\emph{However}
\begin{verbatim}
\documentclass{memoir}
\usepackage{glossaries}
\usepackage{glossaries-extra}
\end{verbatim}
This is like:
\begin{verbatim}
\documentclass{memoir}
\usepackage[toc,nopostdot]{glossaries}
\usepackage{glossaries-extra}
\end{verbatim}
Since by the time \styfmt{glossaries-extra} has been loaded,
\styfmt{glossaries} has already redefined \cls{memoir}'s
glossary-related commands.
\end{enumerate}

Another noticeable change is that by default \cs{printglossary}
will now display information text in the document if the external
glossary file doesn't exist. This is explanatory text to help new
users who can't work out what to do next to complete the document
build. Once the document is set up correctly and the external files
have been generated, this text will disappear.

This change is mostly likely to be noticed by users
with one or more redundant empty glossaries who ignore
transcript messages, explicitly use \gls{makeindex}\slash\gls{xindy}
on just the non-empty glossary (or glossaries) and use
the iterative 
\ics{printglossaries} command instead of \ics{printglossary}. For example,
consider the following:
\begin{verbatim}
\documentclass{article}

\usepackage[acronym]{glossaries}

\makeglossaries

\newacronym{laser}{laser}{light amplification by stimulated
emission of radiation}

\begin{document}

\gls{laser}

\printglossaries

\end{document}
\end{verbatim}
The above document will only display the list of
acronyms at the place where \ics{printglossaries} occurs. However it
will also attempt to input the \texttt{.gls} file associated with
the \texttt{main} glossary.

If you use \gls{makeglossaries}, you'll get the warning message:
\begin{verbatim}
Warning: File 'test.glo' is empty.
Have you used any entries defined in glossary 'main'?
Remember to use package option 'nomain' if you
don't want to use the main glossary.
\end{verbatim}
(where the original file is called \texttt{test.tex})
but if you simply call \gls{makeindex} directly to generate the
\texttt{.acr} file (without attempting to create the \texttt{.gls}
file) then the transcript file will always contain the message:
\begin{verbatim}
No file test.gls.
\end{verbatim}
This doesn't occur with \gls{makeglossaries} as it will create
the \texttt{.gls} file containing the single command \cs{null}.

If you simply change from \styfmt{glossaries} to
\styfmt{glossaries-extra} in this document, you'll find a change in the
resulting PDF if you don't use \gls{makeglossaries} and you only
generate the \texttt{.acr} file with \gls{makeindex}.

The transcript file will still contain the message
about the missing \texttt{.gls}, but now you'll also see
information in the actual PDF document. The simplest remedy is to
follow the advice inserted into the document at that point, which is
to add the \pkgopt{nomain} package option:
\begin{verbatim}
\documentclass{article}

\usepackage[nomain,acronym,postdot]{glossaries-extra}

\makeglossaries

\setabbreviationstyle[acronym]{long-short}

\newacronym{laser}{laser}{light amplification by stimulated
emission of radiation}

\begin{document}

\gls{laser}

\printglossaries

\end{document}
\end{verbatim}
(Note the need to set the acronym style using
\cs{setabbreviationstyle} before \cs{newacronym}.
See \sectionref{sec:abbreviations} for further details.)

\section{New or Modified Package Options}
\label{sec:pkgopts}

If you haven't already loaded \styfmt{glossaries}, you can use any of
the package options provided by \styfmt{glossaries} when you load
\styfmt{glossaries-extra} and they will automatically be passed to
\styfmt{glossaries} (which \styfmt{glossaries-extra} will load). If
\styfmt{glossaries} has already been loaded, then those options will be
passed to \ics{setupglossaries}, but remember that not all of the
\styfmt{glossaries} package options may be used in that command.

\begin{important}
This section only lists options that are either unrecognised by
the \styfmt{glossaries} package or are a modified version of options 
of the same name provided by \styfmt{glossaries}. See the
\styfmt{glossaries} user manual for details about the unmodified
options.
\end{important}

The new and modified options provided by \styfmt{glossaries-extra}
are described below:

\begin{description}

\item[{\pkgoptdef{debug}}] The \styfmt{glossaries} package
has a \pkgoptfmt{debug} option that allows the values
\pkgoptfmt{false}, \pkgoptfmt{true} and \pkgoptfmt{showtargets}.
The \pkgopt[showtargets]{debug} option was introduced to
\styfmt{glossaries} v4.32, so if you want to use this option
with \styfmt{glossaries-extra} you must make sure that
your version of \styfmt{glossaries} supports it.

The \styfmt{glossaries-extra} package extends this option
to provide the additional values \pkgopt[showwrgloss]{debug}
and \pkgopt[all]{debug}.

The \pkgopt[showwrgloss]{debug} option implements
\pkgopt[true]{debug} and uses
\begin{definition}[\DescribeMacro\glsxtrwrglossmark]
\cs{glsxtrwrglossmark}
\end{definition}
to show a mark \ensuremath{\cdot} just before the write operation
performed by the indexing commands. If you use
\pkgopt[alsoindex]{record} there will be a mark for the
write operation to the \texttt{.aux} file for \gls{bib2gls} 
and a mark for the write operation to the associated glossary
file for \gls{makeindex} or \gls{xindy}.

The \pkgopt[all]{debug} option implements both \pkgopt[showtargets]{debug}
and \pkgopt[showwrgloss]{debug}.

\item[{\pkgoptdef{postdot}}] (New to version 1.12.)
This option is just a shortcut for \pkgopt[false]{nopostdot}.

\item[{\pkgoptdef{postpunc}}] (New to version 1.21.)
This option sets the post-description punctuation to the given
value. For example: \pkgoptfmt{postpunc={;}} does
\begin{verbatim}
\renewcommand{\glspostdescription}{;}
\end{verbatim}
The value may also be one of the following keywords: 
\begin{description}
\item[\texttt{comma}:] \pkgopt[comma]{postpunc} is
equivalent to
\begin{verbatim}
\renewcommand{\glspostdescription}{,}
\end{verbatim}

\item[\texttt{dot}:] \pkgopt[dot]{postpunc} is
equivalent to
\begin{verbatim}
\renewcommand{\glspostdescription}{.\spacefactor\sfcode`\. }
\end{verbatim}

\item[\texttt{none}:] \pkgopt[none]{postpunc} is
equivalent to
\begin{verbatim}
\renewcommand{\glspostdescription}{}
\end{verbatim}
\end{description}
The default definition is
\begin{verbatim}
\newcommand*{\glspostdescription}{%
  \ifglsnopostdot\else.\spacefactor\sfcode`\. \fi
}
\end{verbatim}
where the conditional is determined by the \pkgopt{nopostdot}
package option. The \pkgopt{postpunc} option removes the conditional
from the definition of \cs{glspostdescription}. The package options
\pkgopt{nopostdot} and \pkgopt{postdot} will restore the original definition of
\cs{glspostdescription}.

The \sty{glossaries-extra-stylemods} package adjusts 
the predefined styles that had a hard-coded \cs{space}
before the \gls{numberlist} so that they use \cs{glsxtrprelocation}
instead (which is defined to \cs{space}). You can therefore
redefine this command in combination with \pkgopt{postpunc}
to alter the separator before the \gls{numberlist}.
For example, to have a comma followed by \cs{hfil}:
\begin{verbatim}
\usepackage[postpunc=comma,stylemods]{glossaries-extra}
\renewcommand{\glsxtrprelocation}{\hfil}
\end{verbatim}
Be careful with doing this as it will look odd if the \gls{numberlist} is
missing. (With \gls{bib2gls} you can instead redefine
\cs{glsxtrprelocation} to do nothing and set the
location prefixes with \texttt{loc-prefix} which will only
apply if the entry has a \gls{numberlist}.)

\item[{\pkgoptdef{prefix}}] Load the \sty{glossaries-prefix}
package (if not already loaded).

\item[{\pkgoptdef{accsupp}}] Load the \sty{glossaries-accsupp}
package (if not already loaded). 

\begin{important}
The \sty{glossaries-accsupp} package is still experimental and so
accessibility features are liable to change.
\end{important}

If you want to define styles that can interface with the
accessibility support provided by \sty{glossaries-accsupp} use
the \cs{glsaccess\meta{xxx}} type of commands instead of
\cs{glsentry\meta{xxx}} (for example, \ics{glsaccesstext} instead of
\ics{glsentrytext}). If \sty{glossaries-accsupp} hasn't been loaded
those commands are equivalent (for example, \cs{glsaccesstext}
just does \cs{glsentrytext}) but if it has been loaded, then the
\cs{glsaccess\meta{xxx}} commands will add the accessibility
information. (See \sectionref{sec:accsupp} for further details.)

Note that the \pkgopt{accsupp} option can only be used as 
a~package option (and can't be set with \ics{glossariesextrasetup}) since the \sty{glossaries-accsupp}
package must be loaded before \styfmt{glossaries-extra} if it's
required.

\item[{\pkgoptdef{stylemods}}] This is a
\meta{key}=\meta{value} option used to load the
\sty{glossaries-extra-stylemods} package. The value may be a
comma-separated list of options to pass to that package. (Remember
to group \meta{value} if it contains any commas.) The value may be
omitted if no options need to be passed. See \sectionref{sec:glosstylemods}
for further details. There are two special keyword values:
\pkgopt[default]{stylemods} (equivalent to omitting the value)
and \pkgopt[all]{stylemods}, which loads all the
predefined styles.

\item[{\pkgoptdef{undefaction}}] This is a \meta{key}=\meta{value}
option, which has two allowed values: \pkgoptfmt{warn} and
\pkgoptfmt{error}. This indicates what to do if an undefined
glossary entry is referenced. The default behaviour is
\pkgopt[error]{undefaction}, which produces an error message (the
default for \styfmt{glossaries}). You can switch this to a warning
message (and ?? appearing in the text) with
\pkgopt[warn]{undefaction}.

\begin{important}
Undefined entries can't be picked up by any commands that iterate
over a glossary list. This includes \ics{forglsentries} and
\ics{glsaddall}.
\end{important}

Note that \ics{ifglsused} will display ?? in the document text with
\pkgopt[warn]{undefaction} if the entry hasn't been defined, as the
underlying boolean variable doesn't exist and so is neither true nor
false. (There will also be a warning in the transcript.) See
\sectionref{sec:glsunset} for further details.

\item[{\pkgoptdef{indexcrossrefs}}] This is a boolean option.
If \pkgoptfmt{true}, this will automatically index any
cross-referenced entries that haven't been marked as used at
the end of the document. Note that this necessarily adds to
the overall document build time, especially if you have defined
a large number of entries, so this defaults to \pkgoptfmt{false},
but it will be automatically switched on if you use the \gloskey{see}
or \gloskey{seealso} keys in any entries (unless
\pkgopt[false]{autoseeindex}). To force it off, even if you use the
\gloskey{see} or \gloskey{seealso} key, you
need to explicitly set \pkgopt{indexcrossrefs} to \pkgoptfmt{false}.

Note that \gls{bib2gls} can automatically find dependent
entries when it parses the \texttt{.bib} source file.
The \pkgopt{record} option automatically implements
\pkgopt[false]{indexcrossrefs}.

\item[{\pkgoptdef{autoseeindex}}]\label{opt:autoseeindex} (New to
v1.16.) This is a boolean option. If true (default), this makes 
the \gloskey{see} and \gloskey{seealso} keys automatically index the
cross-reference when an entry is defined. If false, the value
of those keys will still be stored in their corresponding fields
(and can be accessed using commands like \cs{glsxtrusesee} and
\cs{glsxtruseseealso}) but cross-reference won't be automatically
indexed.

Note that the \pkgopt[only]{record} option automatically implements
\pkgopt[false]{autoseeindex}.

For example, if an entry is defined as
\begin{verbatim}
\newglossaryentry{foo}{name={foo},description={},see={bar,baz}}
\end{verbatim}
then with \texttt{autoseeindex=true}, this is equivalent to
\begin{verbatim}
\newglossaryentry{foo}{name={foo},description={}}
\glssee{foo}{bar,baz}
\glossariesextrasetup{indexcrossrefs=true}
\GlsXtrSetField{foo}{see}{bar,baz}
\end{verbatim}
but with \texttt{autoseeindex=false}, this is equivalent to
\begin{verbatim}
\newglossaryentry{foo}{name={foo},description={}}
\GlsXtrSetField{foo}{see}{bar,baz}
\end{verbatim}
Note that \pkgopt{indexcrossrefs} isn't automatically implemented
by the presence of the \gloskey{see} key when \pkgopt{autoseeindex}
is false.

It's therefore possible to remove the cross-references from the
location lists and set their position within the glossary style.

Another method of preventing the automatic indexing is to 
define the entries before the external indexing files have been
opened with \cs{makeglossaries}. Since the appropriate file isn't
open, the information can't be written to it. This will need
the package option \pkgoptfmt{seenoindex=ignore} (provided by 
\styfmt{glossaries}) to prevent an error occurring.

\item[{\pkgoptdef{record}}]\label{opt:record}(New to v1.08.) This is
a \meta{key}=\meta{value} option provided for the benefit of
\gls{bib2gls} (see \sectionref{sec:bib2gls}).

The option may only be set in the preamble and can't be used after
\cs{GlsXtrLoadResources}. If the value is missing
\pkgopt[only]{record} is assumed. Permitted values:
\begin{description}
\item[\pkgoptfmt{off}] This is the default setting. The indexing
is performed as normal using either \cs{makeglossaries}
or \cs{makenoidxglossaries}. This setting implements \pkgopt[error]{undefaction}.

\item[\pkgoptfmt{only}] The indexing is performed by
\gls{bib2gls} (see \sectionref{sec:bib2gls}). Neither
\cs{makeglossaries} nor \cs{makenoidxglossaries} is permitted.
This setting implements \pkgopt[warn]{undefaction} and automatically
loads the supplementary \sty{glossaries-extra-bib2gls} package.
(There should be no need to explicitly load \sty{glossaries-extra-bib2gls}.)

The glossaries should be displayed using \cs{printunsrtglossary}
(or \cs{printunsrtglossaries}).

The document build process is (assuming the file is called
\texttt{myDoc.tex}):
\begin{verbatim}
pdflatex myDoc
bib2gls myDoc
pdflatex myDoc
\end{verbatim}

Note that \pkgopt[only]{record} will prevent the \gloskey{see} from
automatically implementing \cs{glssee}. (\gls{bib2gls} deals with the 
\gloskey{see} field.) You may explicitly use \cs{glssee} in the
document, but \gls{bib2gls} will ignore the cross-reference if the
\gloskey{see} field was already set for that entry.

The \pkgopt[only]{record} option will automatically set
the \styfmt{glossaries} package's \pkgopt[none]{sort} option
if available. (That option value was only introduced to 
\styfmt{glossaries} v4.30.)

\item[\pkgoptfmt{nameref}] (New to v1.37 and requires \gls{bib2gls}
v1.8+.) This option is like \pkgopt[only]{record} but additionally
records the current label information given by \cs{@currentlabel}
and \cs{@currentHref}, and provides a more reliable way of saving
the \cs{theH}\meta{counter} for the given location. This option
requires \sty{hyperref} otherwise it will fall back on the usual
location records. Remember that \cs{@currentHref} is always globally
updated whenever \cs{refstepcounter} is used, but \cs{@currentlabel}
isn't. This can cause some undesired side-effects with some settings. Remember also that the \pkgopt{indexcounter} option
increments the associated counter every time an entry is indexed,
which affects this option.  See \sectionref{sec:recordnameref} for
further details.

This option is best used with \pkgopt[chapter]{counter} or
\pkgopt[section]{counter} if you want the title included in the
location list. If the indexing counter is the default
\ctr{page}, only the location number is shown. Similarly for
\pkgopt[equation]{counter} (or \pkgopt[true]{equations}).

\item[\pkgoptfmt{alsoindex}] This is a hybrid setting
that uses \gls{bib2gls} to fetch entry information from
\texttt{.bib} files, but uses \gls{makeindex} or \gls{xindy}
to create the glossary files. This option should be used
with \cs{makeglossaries} but not with its optional argument.
This option should not be used with \cs{makenoidxglossaries}.
You may need to change the transcript file used by \gls{bib2gls}
to avoid a clash with the transcript file used by
\gls{makeindex} or \gls{xindy}. (This can be done with
\gls{bib2gls}['s] \verb|--log-file| or \verb|-t| option.)

The glossaries should be displayed using \cs{printglossary}
(or \cs{printglossaries}). This option is expected to be used with
\gls{bib2gls}'s \texttt{sort=none} setting and so
\sty{glossaries-extra-bib2gls} is not automatically loaded.

The document build process is (assuming the file is called
\texttt{myDoc.tex}):
\begin{verbatim}
pdflatex myDoc
bib2gls myDoc
pdflatex myDoc
makeglossaries myDoc
pdflatex myDoc
\end{verbatim}
\end{description}

With the recording on (\pkgopt[only]{record} or
\pkgopt[alsoindex]{record}), any of the commands that would typically
index the entry (such as \cs{gls}, \cs{glstext} or \cs{glsadd})
will add a \cs{glsxtr@record} entry to the \texttt{.aux} file.
\gls{bib2gls} can then read these lines to find
out which entries have been used. (Remember that commands like
\cs{glsentryname} don't index, so any use of these commands won't
add a corresponding \cs{glsxtr@record} entry to the \texttt{.aux}
file.) See \sectionref{sec:bib2gls} for further details.

\item[{\pkgoptdef{equations}}] (New to v1.37.) This option will
cause the default location counter to automatically switch
to \ctr{equation} when inside a numbered equation environment, such
as \env{equation} or \env{align}. The counter can be explicitly
overridden with \gloskey[glslink]{counter} in the optional
arguments of commands like \cs{glslink} or \cs{gls} as usual.

\item[{\pkgoptdef{floats}}] (New to v1.37.) This option will
cause the default location counter to automatically switch
to the corresponding counter when inside a floating environment,
such as \env{figure} or \env{table}. The counter can be explicitly
overridden with \gloskey[glslink]{counter} in the optional
arguments of commands like \cs{glslink} or \cs{gls} as usual.
Remember that within floats it's the \cs{caption} command that actually uses
\cs{refstepcounter}, so indexing before the caption will result in
the wrong reference. The commands for use in captions and sections,
such as \cs{glsfmttext} and \cs{glsfmtshort}, don't index. (See
\sectionref{sec:headtitle}). You may
want to consider using \cs{glsadd} after the caption (not before). For example:
\begin{verbatim}
\begin{figure}[htbp]
  \centering
  \includegraphics{example-image}
  \caption{Sample \glsfmttext{foobar} figure}
  \glsadd{foobar}
\end{figure}
\end{verbatim}

\item[{\pkgoptdef{indexcounter}}] (New to v1.29.) This option (which
doesn't take a value) is primarily intended for use with
\gls{bib2gls} (v1.4+) and \sty{hyperref}.  It can be used with
\gls{makeindex} or \gls{xindy} but it will interfere with the
\gls{numberlist} collation, so you won't have ranges and you'll have
duplicate page numbers present (but each page number will link to
the relevant part of the page where the indexing occurred). This
option automatically implements \pkgopt[wrglossary]{counter}.

\begin{important}
This option works by incrementing \ctr{wrglossary} and
adding \ics{label}. This can cause a problem if the indexing
occurs in an \env{equation} environment as \sty{amsmath}
forbids multiple occurrences of \cs{label} (resulting in the
\qt{Multiple \cs{label}'s} error). It's best to change the counter
to \ctr{page} or \ctr{equation} when in maths mode with this option.
For example:
\begin{verbatim}
\renewcommand{\glslinkpresetkeys}{%
 \ifmmode \setkeys{glslink}{counter=equation}\fi}
\renewcommand{\glsaddpresetkeys}{%
 \ifmmode \setkeys{glossadd}{counter=equation}\fi}
\end{verbatim}
\end{important}

By default (with \sty{hyperref}), the page numbers in
\glspl{numberlist} link back to the top of the relevant page
(provided the format uses \cs{glshypernumber}). The
\pkgopt{indexcounter} option is designed to link back to the place
within the page where the indexing occurred. It does this by
creating a new counter (called \ctr{wrglossary}) that's incremented
with \cs{refstepcounter} every time an entry is indexed (but not via
cross-referencing commands, such as \cs{glssee}).  A \cs{label} is
placed immediately after the increment command allowing the
back-referenced to be obtained with \cs{pageref}.  The location, as
seen by the indexing application, is the value of the
\ctr{wrglossary} counter but this value is substituted with the page
reference when \gls{numberlist} is typeset. Since the counter is
used by all entries and is incremented every time any indexing
occurs, neither \gls{makeindex} nor \gls{xindy} can correctly
collate the lists. For example, if the first term to be referenced
is indexed three times on page~5 without any intervening terms then
the actual locations obtained from \ctr{wrglossary} will be 1, 2 and
3, which \gls{xindy} and \gls{makeindex} will try to form into the
range 1--3, but they should actually all simply appear as page~5,
whereas it can actually end up with 5--5. Conversely, a range may
not be formed where it would naturally occur if just the \ctr{page}
counter was used.

Since \gls{bib2gls} is designed specifically to work with 
\styfmt{glossaries-extra}, \gls{bib2gls} (v1.4+) will check for
\ctr{wrglossary} locations. If the default
\texttt{--merge-wrglossary-records} is on, then any records for the
same page for a given entry will be merged. In the above example
with three references on page~5, only a single record for page~5 for
that entry will be added to the \gls{numberlist} and it will link
back to the first instance on that page. Whereas if you use the
\texttt{--no-merge-wrglossary-records} switch, the \gls{numberlist}
will contain three instance of page~5, with each linking to the
corresponding place on that page. In both cases, consecutive pages
can form ranges, but it may look strange in the second case.

See the \gls{bib2gls} documentation for the
\texttt{save-index-counter} resource option for more details.

\item[{\pkgoptdef{docdef}}] This option governs the use of
\cs{newglossaryentry}. It was originally a boolean option,
but as from version 1.06, it can now take one of the following values (if
the value is omitted, \pkgoptfmt{true} is assumed):
\begin{itemize}
\item[{\pkgopt[false]{docdef}}] \ics{newglossaryentry} is not 
permitted in the \env{document} environment (default).

\item[{\pkgopt[true]{docdef}}] \cs{newglossaryentry} behaves as it
does in the base \styfmt{glossaries} package. That is, where
its use is permitted in the \env{document} environment, it
uses the \texttt{.glsdefs} temporary file to store the entry
definitions so that on the next \LaTeX\ run the entries are
defined at the beginning of the \env{document} environment.
This allows the entry information to be referenced in the glossary,
even if the glossary occurs before \cs{newglossaryentry}.
(For example, when the glossary is displayed in the front matter.)
This method of saving the definitions for the next \LaTeX\ run
has drawbacks that are detailed in the \styfmt{glossaries} user
manual.

Remember that if \cs{newglossaryentry} wouldn't be allowed
in the \env{document} environment with the
base \styfmt{glossaries} package, then it still won't
be allowed with \pkgopt[true]{docdefs}.
If your glossaries occur at the end of the document, consider using
\pkgopt[restricted]{docdef} instead.

\item[{\pkgopt[restricted]{docdef}}] (new to version 1.06)
\ics{newglossaryentry} is permitted in the \env{document}
environment without using the \texttt{.glsdefs} file. This means
that all entries must be defined before the glossary is displayed,
but it avoids the complications associated with saving the
entry details in a temporary file. You will still need to take
care about any changes made to characters that are required
by the \meta{key}=\meta{value} mechanism (that is, the comma and
equal sign) and any \gls{makeindex} or \gls{xindy} character that
occurs in the \gloskey{sort} key or label. If any of those
characters are made active in the document, then it can cause
problems with the entry definition. This option will allow
\cs{newglossaryentry} to be used in the document with
\cs{makenoidxglossaries}, but note that \cs{longnewglossaryentry}
remains a preamble-only command.

With this option, if an entry appears in the glossary before
it has been defined, an error will occur (or a warning if
the \pkgopt[warn]{undefaction} option is used.) If you edit your
document and either remove an entry or change its label, you may
need to delete the document's temporary files (such as the
\texttt{.aux} and \texttt{.gls} files).

\item[{\pkgopt[atom]{docdef}}] (new to version 1.34)
This option behaves like \pkgopt[restricted]{docdef} but creates the
\texttt{.glsdefs} file for
\href{https://atom.io/packages/autocomplete-glossaries}{atom's
autocomplete support}. This file isn't input by
\styfmt{glossaries-extra} and so associated problems with the use of
this file are avoided, but it allows the autocomplete support to
find the labels in the file. As with \pkgopt[restricted]{docdef},
entries may be defined in the preamble or anywhere in the document,
but they may only be referenced after they have been defined.
Entries must be defined before the associated glossary is displayed.

\end{itemize}

The \styfmt{glossaries} package allows
\cs{newglossaryentry} within the \env{document} environment (when
used with \gls{makeindex} or \gls{xindy}) but the user manual warns 
against this usage. By default the \styfmt{glossaries-extra} package
\emph{prohibits} this, only allowing definitions within the
preamble. If you are really determined to define entries in the
\env{document} environment, despite all the associated drawbacks,
you can restore this with \pkgopt[true]{docdef}. Note that this
doesn't change the prohibitions that the \styfmt{glossaries}
package has in certain circumstances (for example, when using
\qt{option~1}). See the \styfmt{glossaries} user manual for further
details. A better option if document definitions are required
is \pkgopt[restricted]{docdef}. Only use \pkgopt[true]{docdef}
if document definitions are necessary and one or more of the
glossaries occurs in the front matter.

\begin{sloppypar}
This option affects commands that internally use
\cs{newglossaryentry}, such as \cs{newabbreviation}, but not
the \qt{on-the-fly} commands described in \sectionref{sec:onthefly}.
\end{sloppypar}

\item[{\pkgoptdef{nomissingglstext}}] This is a boolean option.
If true, this will suppress the warning written to the transcript
and the warning text that will appear in the
document if the external glossary files haven't been generated due
to an incomplete document build. However, it's probably simpler 
just to fix whatever has caused the failure to build the external
file or files.

\item[{\pkgoptdef{abbreviations}}] This option has no value and can't
be cancelled. If used,
it will automatically create a new glossary with the label
\texttt{abbreviations} and redefines \cs{glsxtrabbrvtype} to this
label. (The file extensions are \filetype{glg-abr},
\filetype{gls-abr} and \filetype{glo-abr}.) 
In addition, this option defines a shortcut command
\begin{definition}[\DescribeMacro\printabbreviations]
\cs{printabbreviations}\oarg{options}
\end{definition}
which is equivalent to
\begin{alltt}
\cs{printglossary}[type=\cs{glsxtrabbrvtype},\meta{options}]
\end{alltt}
If \sty{glossaries-extra-bib2gls} is also loaded then this option
will additionally provide:
\begin{definition}[\DescribeMacro\printunsrtabbreviations]
\cs{printunsrtabbreviations}\oarg{options}
\end{definition}
which uses \ics{printunsrtglossary}.

The title of the new glossary is given by
\begin{definition}[\DescribeMacro\abbreviationsname]
\cs{abbreviationsname}
\end{definition}
If this command is already defined, it's left unchanged. Otherwise
it's defined to \qt{Abbreviations} if \sty{babel} hasn't been loaded
or \cs{acronymname} if \sty{babel} has been loaded. However, if
you're using \sty{babel} it's likely you will need to change this.
(See \sectionref{sec:lang} for further details.)

\begin{important}
If you don't use the \pkgopt{abbreviations} package option,
the \cs{abbreviationsname} command won't be defined (unless it's
defined by an included language file).
\end{important}

If the \pkgopt{abbreviations} option is used and the 
\pkgopt{acronym} option provided by the \styfmt{glossaries}
package hasn't been used, then \ics{acronymtype}
will be set to \ics{glsxtrabbrvtype} so that acronyms defined with
\ics{newacronym} can be added to the list of abbreviations. If you
want acronyms in the \texttt{main} glossary and other abbreviations in the
\texttt{abbreviations} glossary then you will need to redefine
\cs{acronymtype} to \texttt{main}:
\begin{verbatim}
\renewcommand*{\acronymtype}{main}
\end{verbatim}

Note that there are no analogous options to the \styfmt{glossaries}
package's \pkgopt{acronymlists} option (or associated commands)
as the abbreviation mechanism is handled differently with
\styfmt{glossaries-extra}.

\item[{\pkgoptdef{symbols}}] This is passed to \styfmt{glossaries} but
will additionally define
\begin{definition}[\DescribeMacro\glsxtrnewsymbol]
\cs{glsxtrnewsymbol}\oarg{options}\marg{label}\marg{symbol}
\end{definition}
which is equivalent to
\begin{alltt}
\cs{newglossaryentry}\marg{label}\{name=\marg{symbol},
 sort=\marg{label},type=symbols,category=symbol,\meta{options}\}
\end{alltt}
Note that the \gloskey{sort} key is set to the \meta{label}
not the \meta{symbol} as the symbol will likely contain commands.

If \sty{glossaries-extra-bib2gls} is also loaded then this option
will additionally provide:
\begin{definition}[\DescribeMacro\printunsrtsymbols]
\cs{printunsrtsymbols}\oarg{options}
\end{definition}
which uses \ics{printunsrtglossary}.

\item[{\pkgoptdef{numbers}}] This is passed to \styfmt{glossaries} but
will additionally define
\begin{definition}[\DescribeMacro\glsxtrnewnumber]
\cs{glsxtrnewnumber}\oarg{options}\marg{number}
\end{definition}
which is equivalent to
\begin{alltt}
\cs{newglossaryentry}\marg{label}\{name=\marg{number},
 sort=\marg{label},type=numbers,category=number,\meta{options}\}
\end{alltt}

If \sty{glossaries-extra-bib2gls} is also loaded then this option
will additionally provide:
\begin{definition}[\DescribeMacro\printunsrtnumbers]
\cs{printunsrtnumbers}\oarg{options}
\end{definition}
which uses \ics{printunsrtglossary}.

\item[{\pkgoptdef{acronyms} (or \pkgoptdef{acronym})}] This is
passed to \styfmt{glossaries} but if 
\sty{glossaries-extra-bib2gls} is also loaded then this option
will additionally provide:
\begin{definition}[\DescribeMacro\printunsrtacronyms]
\cs{printunsrtacronyms}\oarg{options}
\end{definition}
which uses \ics{printunsrtglossary}.

\begin{important}
This option defines a new glossary with the label \texttt{acronym}
not \texttt{acronyms}. You may find it easier to reference it
with the command \ics{acronymtype}.
\end{important}

\item[{\pkgoptdef{index}}] This is
passed to \styfmt{glossaries} but if 
\sty{glossaries-extra-bib2gls} is also loaded then this option
will additionally provide:
\begin{definition}[\DescribeMacro\printunsrtindex]
\cs{printunsrtindex}\oarg{options}
\end{definition}
which uses \ics{printunsrtglossary}.

\item[{\pkgoptdef{shortcuts}}] Unlike the \styfmt{glossaries} package
option of the same name, this option isn't boolean but has multiple
values:
\begin{itemize}
\item \pkgopt[acronyms]{shortcuts} (or \pkgopt[acro]{shortcuts}):
set the shortcuts provided by the \styfmt{glossaries} package for
acronyms (such as \cs{ac}). Note that the short and long forms don't
use \cs{glsxtrshort} and \cs{glsxtrlong} but use the original
\cs{acrshort} and \cs{acrlong}, which don't recognise multiple
abbreviation styles. The better option with \styfmt{glossaries-extra}
is \pkgopt[ac]{shortcuts}.

\item \pkgopt[ac]{shortcuts}:
set the shortcuts provided by the \styfmt{glossaries} package for
acronyms (such as \cs{ac}) but uses the \styfmt{glossaries-extra}
interface (such as \cs{glsxtrshort} rather than \cs{acrshort}).
In this case \cs{ac} is defined as \cs{cgls} rather than
\cs{gls}.

\item \pkgopt[abbreviations]{shortcuts} (or
\pkgopt[abbr]{shortcuts}):
set the abbreviation shortcuts provided by \styfmt{glossaries-extra}. (See
\sectionref{sec:abbrshortcuts}.) These settings don't switch on the
acronym shortcuts provided by the \styfmt{glossaries} package.

\item \pkgopt[other]{shortcuts}: set the \qt{other}
shortcut commands, but not the shortcut commands for abbreviations
or the acronym shortcuts provided by \styfmt{glossaries}.
The \qt{other} shortcuts are:
\begin{itemize}
\item \ics{newentry} equivalent to \ics{newglossaryentry}
\item \ics{newsym} equivalent to \ics{glsxtrnewsymbol} (see the
\pkgopt{symbols} option).
\item \ics{newnum} equivalent to \ics{glsxtrnewnumber} (see the
\pkgopt{numbers} option).
\end{itemize}

\item \pkgopt[all]{shortcuts} (or \pkgopt[true]{shortcuts}): 
implements \pkgopt[ac]{shortcuts}, \pkgopt[abbreviations]{shortcuts} 
and \pkgopt[other]{shortcuts}.

\item \pkgopt[none]{shortcuts} (or \pkgopt[false]{shortcuts}):
don't define any of the shortcut commands (default).
\end{itemize}

Note that multiple invocations of the \pkgopt{shortcuts} option
\emph{within the same option list} will override each other.
\end{description}

After the \styfmt{glossaries-extra} package has been loaded, you can
set available options using
\begin{definition}[\DescribeMacro\glossariesextrasetup]
\cs{glossariesextrasetup}\marg{options}
\end{definition}
The \pkgopt{abbreviations} and \pkgopt{docdef} options may only be
used in the preamble. Additionally, \pkgopt{docdef} can't be used
after \ics{makenoidxglossaries}.

\chapter{Modifications to Existing Commands and Styles}
\label{sec:modifications}

\section{Defining Entries}

The \styfmt{glossaries} package provides \ics{nopostdesc}
which may be used in the \gloskey{description} to suppress
the post-description hook. The \styfmt{glossaries-extra} package
provides another command
\begin{definition}[\DescribeMacro\glsxtrnopostpunc]
\cs{glsxtrnopostpunc}
\end{definition}
which has a similar function but only suppresses the
post-description punctuation. It doesn't suppress the use of
\cs{glsxtrpostdescription} which allows the use of
category-dependent post-description hooks. (Note that the
punctuation, which is in the original base hook \cs{glspostdescription}, 
comes after the extended post-description hook
\cs{glsxtrpostdescription} not before.) The post-description hook
can counter-act the effect of \cs{glsxtrnopostpunc} using
\begin{definition}[\DescribeMacro\glsxtrrestorepostpunc]
\cs{glsxtrrestorepostpunc}
\end{definition}
These commands have no effect outside of the glossary (except with
standalone entries that use \cs{glsxtractivatenopost} and
\cs{glspostdescription}, see \sectionref{sec:glossentry}).

The commands used by \styfmt{glossaries} to automatically produce an
error if an entry is undefined (such as \ics{glsdoifexists}) are
changed to take the \pkgopt{undefaction} option into account.

The \cs{newglossaryentry} command has three new keys:
\begin{itemize}
\item \gloskey{category}, which sets the category label for the given
entry. By default this is \texttt{general}. See
\sectionref{sec:categories} for further information about
categories.
\item \gloskey{alias}, which allows an entry to be alias to another
entry. See \sectionref{sec:alias} for further details.
\item \gloskey{seealso}, which performs much like \gloskey{see}, but
allows for separate \qt{see} and \qt{see also} treatment.
See \sectionref{sec:see} for further details.
\end{itemize}
This apply to all entry defining commands (such as
\cs{newabbreviation}).

The test file \nolinkurl{example-glossaries-xr.tex} contains
dummy entries with a mixture of \gloskey{see}, \gloskey{alias}
and \gloskey{seealso} keys for use with minimal working examples.
There are also \nolinkurl{example-glossaries-*.bib} files that
correspond to each \nolinkurl{example-glossaries-*.tex} file
for testing \gls{bib2gls}.

The \cs{longnewglossaryentry} command now has a starred version
(as from v1.12) that doesn't automatically insert
\begin{verbatim}
\leavevmode\unskip\nopostdesc
\end{verbatim}
at the end of the description field.
\begin{definition}[\DescribeMacro\longnewglossaryentry]
\cs{longnewglossaryentry}*\marg{label}\marg{options}\marg{description}
\end{definition}
The \gloskey{descriptionplural} key is left unset unless explicitly
set in \meta{options}.

The unstarred version no longer hard-codes the above code (which
removes trailing space and suppresses the post-description hook) but instead
uses:
\begin{definition}[\DescribeMacro\glsxtrpostlongdescription]
\cs{glsxtrpostlongdescription}
\end{definition}
This can be redefined to allow the post-description hook to work
but retain the \cs{unskip} part if required.
For example:
\begin{verbatim}
\renewcommand*{\glsxtrpostlongdescription}{\leavevmode\unskip}
\end{verbatim}
This will discarded unwanted trailing space at the end of the description
but won't suppress the post-description hook.

The unstarred version also alters the base \sty{glossaries} package's
treatment of the \gloskey{descriptionplural} key. Since a
plural description doesn't make much sense for multi-paragraph
descriptions, the default behaviour with 
\sty{glossaries-extra}'s
\cs{longnewglossaryentry} is to simply leave the plural description
unset unless explicitly set using the \gloskey{descriptionplural}
key. The \styfmt{glossaries.sty} version of this command sets the description's
plural form to the same as the singular.\footnote{The
\gloskey{descriptionplural} key is a
throwback to the base package's original acronym mechanism before the introduction of
the \gloskey{long} and \gloskey{short} keys, where the long form was
stored in the \gloskey{description} field and the short form was stored in the
\gloskey{symbol} field.}

Note that this modified unstarred version doesn't append
\cs{glsxtrpostlongdescription} to the description's plural form.

The \ics{newterm} command (defined through the \pkgopt{index} package
option) is modified so that the category defaults to \category{index}.
The \ics{newacronym} command is modified to use the new abbreviation
interface provided by \styfmt{glossaries-extra}.
(See \sectionref{sec:abbreviations}.)

The \cs{makeglossaries} command now has an optional argument.
\begin{definition}[\DescribeMacro\makeglossaries]
\cs{makeglossaries}\oarg{list}
\end{definition}
If \meta{list} is empty, \cs{makeglossaries} behaves as per
its original definition in the \styfmt{glossaries} package,
otherwise \meta{list} can be a comma-separated list of glossaries
that need processing with an external indexing application.

This command is not permitted with the \pkgopt[only]{record}
package option. Without the optional argument, it's permitted
with \pkgopt[alsoindex]{record}. With the optional argument,
it's only permitted with the default \pkgopt[off]{record}.

It should then be possible to use \cs{printglossary} for those
glossaries listed in \meta{list} and \cs{printnoidxglossary}
for the other glossaries. (See the accompanying file
\texttt{sample-mixedsort.tex} for an example.)

\begin{important}
If you use the optional argument \meta{list}, you can't define
entries in the document (even with the \pkgopt{docdef} option).
\end{important}

You will need at least version 2.20 of \gls{makeglossaries} or at
least version 1.3 of the Lua alternative \gls{makeglossaries-lite.lua} (both distributed
with \styfmt{glossaries} v4.27) to allow for this use of
\cs{makeglossaries}\oarg{list}. Alternatively, use the
\pkgopt{automake} option.

\section{Entry Indexing}
\label{sec:wrglossary}

As from version 1.31, there is a new command like \cs{glsadd}
where the mandatory argument is a comma-separated list
of labels:
\begin{definition}[\DescribeMacro\glsaddeach]
\cs{glsaddeach}\oarg{options}\marg{list}
\end{definition}
This simply iterates over \meta{list} and does
\cs{glsadd}\oarg{options}\marg{label} for each entry in the list.

As from version 1.37, you can make commands like \cs{gls} or
\cs{glslink} automatically use \cs{glsadd} with specific options
when a given format is used (identified with
\texttt{format=\marg{format}} in the optional argument of the
corresponding \cs{gls}, \cs{glslink} etc).
\begin{definition}[\DescribeMacro\GlsXtrAutoAddOnFormat]
\cs{GlsXtrAutoAddOnFormat}\oarg{label}\marg{format list}\marg{glsadd options}
\end{definition}
The optional argument \meta{label} defaults to \cs{glslabel} and
indicates the label to use in \cs{glsadd} and so needs to be
expandable. The \meta{format list} is a comma-separated list
of format values that will trigger the automated adding. The
\meta{glsadd options} are the options to pass to \cs{glsadd} with
\texttt{format=\marg{format}} prepended to the list.

For example, with:
\begin{verbatim}
\GlsXtrAutoAddOnFormat{hyperbf}{counter=chapter}
\end{verbatim}
then \verb|\gls[format=hyperbf]{sample}| will be equivalent to
\begin{verbatim}
\glsadd[format=hyperbf,counter=chapter]{sample}\gls[format=hyperbf]{sample}
\end{verbatim}
Note that the explicit range markers will prevent a match
unless you include them in \meta{format list} (in which case, be
sure to add both the start and end formats).

Here's another example:
\begin{verbatim}
\GlsXtrAutoAddOnFormat[dual.\glslabel]{hyperbf}{}
\end{verbatim}
In this case \verb|\gls[format=hyperbf]{sample}| will now be
equivalent to:
\begin{verbatim}
\glsadd[format=hyperbf]{dual.sample}\gls[format=hyperbf]{sample}
\end{verbatim} 

\begin{important}
\cs{GlsXtrAutoAddOnFormat} is not applied to \cs{glsadd}
as it could cause an infinite loop.
\end{important}

The \styfmt{glossaries-extra} package provides extra keys for
commands like \cs{gls} and \cs{glstext}:
\begin{description}
\item[{\gloskey[glslink]{noindex}}] This is a boolean key. If true,
this suppresses the indexing. (That is, it prevents \cs{gls} or
whatever from adding a line to the external glossary file.)
This is more useful than the \pkgopt{indexonlyfirst} package option
provided by \styfmt{glossaries}, as the \gls{firstuse} might not be the most
pertinent use. (If you want to apply \pkgopt{indexonlyfirst}
to selected entries, rather than all of them, you can use
the \catattr{indexonlyfirst} attribute, see
\sectionref{sec:categories} for further details.)
Note that the \gloskey[glslink]{noindex} key isn't available
for \ics{glsadd} (and \ics{glsaddall}) since the whole purpose
of that command is to index an entry.

\item[{\gloskey[glslink]{wrgloss}}] (New to v1.14.) 
This is may only take the
values \optfmt{before} or \optfmt{after}. By default, commands
that both index and display link text (such as \cs{gls},
\cs{glstext} and \cs{glslink}), perform the indexing before
the link text as the indexing creates a whatsit that can cause
problems if it occurs after the link text. However, it may
be that in some cases (such as long phrases) you may actually
want the indexing performed after the link text. In this
case you can use \texttt{wrgloss=after} for specific 
instances. Note that this option doesn't 
have an effect if the indexing has been suppressed through
other settings (such as \gloskey[glslink]{noindex}).

The default value is set up using
\begin{definition}[\DescribeMacro\glsxtrinitwrgloss]
\cs{glsxtrinitwrgloss}
\end{definition}
which is defined as:
\begin{verbatim}
\newcommand*{\glsxtrinitwrgloss}{%
 \glsifattribute{\glslabel}{wrgloss}{after}%
 {%
   \glsxtrinitwrglossbeforefalse
 }%
 {%
   \glsxtrinitwrglossbeforetrue
 }%
}
\end{verbatim}
This sets the conditional
\begin{definition}
\ics{ifglsxtrinitwrgloss}
\end{definition}
which is used to determine where to perform the indexing.

This means you can set the \catattr{wrgloss} 
attribute to \optfmt{after} to automatically use this as
the default for entries with that category attribute. (Note that
adding \gloskey[glslink]{wrgloss} to the default options
in \ics{GlsXtrSetDefaultGlsOpts} will override 
\cs{glsxtrinitwrgloss}.)

\item[{\gloskey[glslink]{hyperoutside}}] (New to v1.21.) This is a
boolean key. The default is \texttt{hyperoutside=true}, which puts
the hyperlink outside \cs{glstextformat}, so that commands like
\cs{gls} will effectively do
\begin{alltt}
\cs{hyperlink}\marg{target}\{\cs{glstextformat}\marg{link text}\}
\end{alltt}
This is the same behaviour as with the base \styfmt{glossaries}
package. With \texttt{hyperoutside=false}, \cs{hyperlink} is placed
inside the argument of \cs{glstextformat}:
\begin{alltt}
\cs{glstextformat}\{\cs{hyperlink}\marg{target}\marg{link text}\}
\end{alltt}
You can use the \catattr{hyperoutside} category attribute to set
the default for a given category. This can be combined
with the use of the \catattr{textformat} attribute to counteract
any interference caused by \cs{hyperlink}.
For example:
\begin{verbatim}
\glssetcategoryattribute{mathrelation}{hyperoutside}{false}
\end{verbatim}
will set \texttt{hyperoutside=false} for all entries that are
assigned to the category \texttt{mathrelation} and
\begin{verbatim}
\glssetcategoryattribute{mathrelation}{textformat}{mathrel}
\end{verbatim}
will use \cs{mathrel} instead of \cs{glstextformat} resulting in:
\begin{alltt}
\cs{mathrel}\{\cs{hyperlink}\marg{target}\marg{link text}\}
\end{alltt}
for entries with the \gloskey{category} key set to
\texttt{mathrelation}.

\item[{\gloskey[glslink]{textformat}}] This key must have a 
control sequence name as its value. The command formed from 
this name must exist and must take one argument. (Use
\texttt{relax} for default behaviour.) If set, this overrides
the \catattr{textformat} attribute and \cs{glstextformat}.
See the \sty{soul} example in \sectionref{sec:glsunset}.

\item[{\gloskey[glslink]{prefix}}] Locally redefines
\ics{glolinkprefix} to the given value. It should match the prefix
for the desired glossary.

\item[{\gloskey[glslink]{thevalue}}] Explicitly set the location to
this value (see below).

\item[{\gloskey[glslink]{theHvalue}}] Set the corresponding
hyperlink location (see below).

\end{description}

You can set the default options used by \ics{glslink}, \ics{gls}
etc with:
\begin{definition}[\DescribeMacro\GlsXtrSetDefaultGlsOpts]
\cs{GlsXtrSetDefaultGlsOpts}\marg{options}
\end{definition}
For example, if you mostly don't want to index entries then
you can do:
\begin{verbatim}
\GlsXtrSetDefaultGlsOpts{noindex}
\end{verbatim}
and then use, for example, \verb|\gls[noindex=false]{sample}|
when you actually want the location added to the \gls{numberlist}.
These defaults may be overridden by other settings (such as
category attributes) in addition to any settings passed in the 
option argument of commands like \cs{glslink} and \cs{gls}.

Note that if you don't want \emph{any} indexing, just omit
\cs{makeglossaries} and \cs{printglossaries} (or analogous
commands). If you want to adjust the default for
\gloskey[glslink]{wrgloss}, it's better to do this by redefining
\ics{glsxtrinitwrgloss} instead.

\begin{important}
\cs{GlsXtrSetDefaultGlsOpts} doesn't affect \cs{glsadd}.
\end{important}

If you want to change the default value of \gloskey[glsadd]{format}, 
you can instead use:
\begin{definition}[\GlsXtrSetDefaultNumberFormat]
\cs{GlsXtrSetDefaultNumberFormat}\marg{format}
\end{definition}
This has the advantage of also working for \cs{glsadd}.
For example, if you want all locations in the back matter to appear
in italic (unless explicitly overridden):
\begin{verbatim}
\backmatter
\GlsXtrSetDefaultNumberFormat{hyperit}
\end{verbatim}

Commands like \ics{gls} have star (\texttt{*}) and plus 
(\texttt{+}) modifiers as a short cut for \texttt{hyper=false}
and \texttt{hyper=true}. The \styfmt{glossaries-extra} package
provides a way to add a third modifier, if required, using
\begin{definition}[\DescribeMacro\GlsXtrSetAltModifier]
\cs{GlsXtrSetAltModifier}\marg{char}\marg{options}
\end{definition}
where \meta{char} is the character used as the modifier and
\meta{options} is the default set of options (which may be
overridden). Note that \meta{char} must be a single character
(not a UTF-8 character, unless you are using \XeLaTeX\ or
\LuaLaTeX).

\begin{important}
When choosing the character \meta{char} take care of any
changes in category code.
\end{important}

Example:
\begin{verbatim}
\GlsXtrSetAltModifier{!}{noindex}
\end{verbatim}
This means that \verb|\gls!{sample}| will be equivalent to
\verb|\gls[noindex]{sample}|. It's not possible to mix modifiers.
For example, if you want to do 
\begin{verbatim}
\gls[noindex,hyper=false]{sample}
\end{verbatim}
you can use \verb|\gls*[noindex]{sample}| or
\verb|\gls![hyper=false]{sample}| but you can't combine the
\texttt{*} and \texttt{!} modifiers.

There is a new hook that's used each time indexing information is
written to the external glossary files:
\begin{definition}[\DescribeMacro\glsxtrdowrglossaryhook]
\cs{glsxtrdowrglossaryhook}\marg{label}
\end{definition}
where \meta{label} is the entry's label. This does nothing by
default but may be redefined. (See, for example, the 
accompanying sample file \texttt{sample-indexhook.tex}, 
which uses this hook to determine which entries haven't been
indexed.)

There's also a new hook (from v1.26) that's used immediately
before the options are set by the \cs{gls}-like and
\cs{glstext}-like commands:
\begin{definition}[\DescribeMacro\glslinkpresetkeys]
\cs{glslinkpresetkeys}
\end{definition}
(The base package provides \ics{glslinkpostsetkeys} that's
used immediately after the options are set.)

As from version 1.30 there are also similar hooks for \cs{glsadd}:
\begin{definition}[\DescribeMacro\glsaddpresetkeys]
\cs{glsaddpresetkeys}
\end{definition}
and
\begin{definition}[\DescribeMacro\glsaddpostsetkeys]
\cs{glsaddpostsetkeys}
\end{definition}
For example, to default to using the \ctr{equation} counter in maths
mode:
\begin{verbatim}
\renewcommand{\glslinkpresetkeys}{%
 \ifmmode \setkeys{glslink}{counter=equation}\fi}
\renewcommand{\glsaddpresetkeys}{%
 \ifmmode \setkeys{glossadd}{counter=equation}\fi}
\end{verbatim}
In this case, the counter can be overridden with an explicit use of
\gloskey{counter} in the optional argument of \cs{gls} or
\cs{glsadd}.  (As from version 1.37, a simpler method is to just use
the \pkgopt{equations} package option.)

Alternatively, to enforce this (overriding the option
argument):
\begin{verbatim}
\renewcommand{\glslinkpostsetkeys}{%
 \ifmmode \setkeys{glslink}{counter=equation}\fi}
\renewcommand{\glsaddpostsetkeys}{%
 \ifmmode \setkeys{glossadd}{counter=equation}\fi}
\end{verbatim}

As from version 1.14, there are two new keys for \cs{glsadd}: 
\gloskey[glsadd]{thevalue} and \gloskey[glsadd]{theHvalue}. 
These keys are designed for manually adding explicit locations
rather than obtaining the value from the associated counter.
As from version 1.19, these two keys are also available for commands
like \cs{gls} and \cs{glslink}. The \gloskey[glsadd]{thevalue} keys is
intended primarily for adding locations in supplementary material
that can't be obtained from a counter.

The principle key \gloskey[glslink]{thevalue} is for the location value. The
other key \gloskey[glslink]{theHvalue} can be used to
extract a prefix value for the first argument of commands like
\ics{glsnoidxdisplayloc}. Its value must be in the format
\meta{prefix}\meta{location}. In general, there's little need for
this key as the prefix is typically associated with a counter that can be
used to form hypertargets.

\begin{important}
If you use \gloskey[glsadd]{thevalue}, you must make sure that you
use an indexing application that will accept the given value.
\end{important}

For example, \app{makeindex} will only accept locations in the form
[\meta{num}\meta{sep}]*\meta{num} where \meta{num} is an arabic
number (0, 1, \ldots), roman numeral (i, ii, \ldots\ or I, II,
\ldots) or a character from a, \ldots, z or A, \ldots, Z, and
[\meta{num}\meta{sep}]* indicates zero or more instances of a number
followed by the recognised separator character (set with
\cs{glsSetCompositor}). This means that \app{makeindex} won't
accept, for example,
\begin{verbatim}
\glsadd[thevalue={Supplementary Material}]{sample}
\end{verbatim}
This location value will be accepted by \gls{bib2gls}, since it will
allow any location and will only try forming ranges if the location
matches any of its numerical patterns.
In the case of \app{xindy}, you'll need to add a rule that can match
the value. If you're using \sty{hyperref}, you may need to use the
\gloskey[glslink]{format} key to prevent a hyperlink if one can't
naturally be formed from the prefix, counter and location value.

For example, suppose the file \texttt{suppl.tex} contains:
\begin{verbatim}
\documentclass{article}

\usepackage{glossaries-extra}

\newglossaryentry{sample}{name={sample},description={an example}}

\renewcommand{\thepage}{S.\arabic{page}}

\begin{document}
First page.
\newpage
\gls{sample}.
\end{document}
\end{verbatim}
This has an entry on page S.2. Suppose another document wants to
include this location in the glossary. Then this can be done by
setting \gloskey[glsadd]{thevalue} to \texttt{S.2}. For example:
\begin{verbatim}
\documentclass{article}

\usepackage{glossaries-extra}

\makeglossaries

\newglossaryentry{sample}{name={sample},description={an example}}

\begin{document}
Some \gls{sample} text.

\printglossaries
\glsadd[thevalue={S.2}]{sample}
\end{document}
\end{verbatim}
This location value will be accepted by \app{makeindex} as it's in
the form \meta{num}\meta{sep}\meta{num}.

If you want hyperlinks, things are more complicated. First
you need to set the \catattr{externallocation} attribute to the location of
the PDF file. For example:
\begin{verbatim}
\glssetcategoryattribute{supplemental}{externallocation}{suppl.pdf}

\newglossaryentry{sample}{category=supplemental,
 name={sample},description={an example}}
\end{verbatim}
Next you need to add \texttt{glsxtrsupphypernumber} as the format:
\begin{verbatim}
\glsadd[thevalue={S.2},format=glsxtrsupphypernumber]{sample}
\end{verbatim}
Both documents will need to use the \sty{hyperref} package.
Remember that the counter used for the location also needs to match.
If \cs{theH}\meta{counter} is defined in the other document
and doesn't match in the referencing document, then you need
to use \gloskey[glsadd]{theHvalue} to set the appropriate value.
See the accompanying sample files \texttt{sample-suppl-hyp.tex}
and \texttt{sample-suppl-main-hyp.tex} for an example that uses
\sty{hyperref}.

\begin{important}
The hyperlink for the supplementary location may or \emph{may not}
take you to the relevant place in the external PDF file
\emph{depending on your PDF viewer}. Some may not support external
links, and some may take you to the first page or last visited page.
\end{important}

For example, if both \texttt{sample-suppl-hyp.pdf}
and \texttt{sample-suppl-main-hyp.pdf} are in the same directory,
then viewing \texttt{sample-suppl-main-hyp.pdf} in Evince will
take you to the correct location in the linked document (when you
click on the S.2 external link), but Okular will take you to the top
of the first page of the linked document.

This method can only be used where there is one external source
for the designated category (identified by the
\catattr{externallocation} attribute). For multiple sources, you need to use
\gls{bib2gls} version 1.7+, which is the better method in general as it can
automatically fetch the relevant locations from the \texttt{.aux}
files of the designated external documents without the need to
explicitly use \cs{glsadd}.

\section{Cross-References (\qt{see} and \qt{see also})}
\label{sec:see}

The value of the \gloskey{see} key is now saved
as a field. This isn't the case with \styfmt{glossaries}, where
the \gloskey{see} value is simply used to directly
write a line to the corresponding glossary file and is then 
discarded. This is why the \gloskey{see} key can't be
used before \ics{makeglossaries} (since the file hasn't been opened
yet). It's also the reason why the \gloskey{see} key doesn't have
any effect when used in entries that are defined in the
\env{document} environment. Since the value isn't saved,
it's not available when the \texttt{.glsdefs} file is created at the
end of the document and so isn't available at the start of the
\env{document} environment on the next run.

This modification allows \styfmt{glossaries-extra} to provide
\begin{definition}[\DescribeMacro\glsxtraddallcrossrefs]
\cs{glsxtraddallcrossrefs}
\end{definition}
which is used at the end of the document to automatically add
any unused cross-references unless the package option
\pkgopt{indexcrossrefs} was set to false.

As a by-product of this enhancement, the \gloskey{see} key will now
work for entries defined in the \env{document} environment, but it's still
best to define entries in the preamble, and the \gloskey{see} key
still can't perform any indexing before the file has been opened by
\cs{makeglossaries}. Note that \styfmt{glossaries} v4.24 introduced
the \pkgopt{seenoindex} package option, which can be used to
suppress the error when the \gloskey{see} key is used before
\cs{makeglossaries}, so \pkgopt[ignore]{seenoindex} will allow the 
\gloskey{see} value to be stored even though it may not be possible
to index it at that point.

As from version 1.06, you can display the cross-referenced
information for a given entry using
\begin{definition}[\DescribeMacro\glsxtrusesee]
\cs{glsxtrusesee}\marg{label}
\end{definition}
This internally uses
\begin{definition}[\DescribeMacro\glsxtruseseeformat]
\cs{glsxtruseseeformat}\marg{tag}\marg{xr list}
\end{definition}
where \meta{tag} and \meta{xr list} are obtained from the value of
the entry's \gloskey{see} field (if non-empty).
By default, this just does \cs{glsseeformat}\oarg{tag}\marg{xr
list}\verb|{}|, which is how the cross-reference is displayed in the
\gls{numberlist}. Note that \cs{glsxtrusesee} does nothing if the
\gloskey{see} field hasn't been set for the entry given by
\meta{label}.

As with the base \sty{glossaries} package, \cs{glsseeformat}
is defined to do \texttt{\cs{emph}\marg{tag} \cs{glsseelist}\marg{xr
list}}. The third argument is always ignored and is present for
\app{makeindex}, which always requires a final argument to
encapsulate the associated location. The command
\begin{definition}
\ics{glsseelist}\marg{xr list}
\end{definition}
used to iterate over the list of cross-reference labels is also
unchanged from the base \sty{glossaries} package, with each item in
the list formatted according to:
\begin{definition}
\ics{glsseeitem}\marg{label}
\end{definition}
This is defined by the \sty{glossaries} package to:
\begin{alltt}
\cs{glshyperlink}[\cs{glsseeitemformat}\marg{label}]\marg{label}
\end{alltt}
So the actual formatting for each cross-referenced entry is
performed by \ics{glsseeitemformat}, which is redefined by
\sty{glossaries-extra}, as described in
\sectionref{sec:modifications}. This now displays the value
of the \gloskey{text} field for abbreviations and the value
of the \gloskey{name} field otherwise. There's no indication
of the entry hierarchy, which could confuse the reader. Therefore,
as from \sty{glossaries-extra} v1.37, there are some new commands
that include the hierarchical information. You may prefer to
redefine \cs{glsseeitemformat} to use one of this if you have
sub-entries. For example:
\begin{verbatim}
\renewcommand*{\glsseeitemformat}[1]{\glsxtrhiername{#1}}
\end{verbatim}

The \styfmt{glossaries} package provides
\begin{definition}[\DescribeMacro\glsseeitemformat]
\cs{glsseeitemformat}\marg{label}
\end{definition}
to format items in a cross-reference list (identified with the
\gloskey{see} key or \cs{glssee}). This was originally defined to
use \cs{glsentryname}\marg{label} since it makes more sense for the
cross-reference to match the way the term appears in the glossary.
Unfortunately this caused a problem when the \gloskey{name} field
was sanitized, which used to be the default setting, so
\styfmt{glossaries} v3.0 changed the default definition of this
command to use \cs{glsentrytext} instead. Since the \gloskey{name}
and \gloskey{text} field are quite often the same, this change
usually doesn't have a noticeable effect. However, now that the
\gloskey{name} field is no longer sanitized (following the redesign
of \styfmt{glossaries} v4.0) it makes more sense to restore this
command to its original behaviour, but to take account of
abbreviations \styfmt{glossaries-extra} redefines this as:
\begin{verbatim}
\renewcommand*{\glsseeitemformat}[1]{%
 \ifglshasshort{\glslabel}{\glsfmttext{#1}}{\glsfmtname{#1}}%
}
\end{verbatim}
(Note that as from \styfmt{glossaries-extra} version 1.42, this now
uses \cs{glsfmttext} and \cs{glsfmtname} instead of just referencing
the \gloskey{text} and \gloskey{name} fields. This helps to ensure
that any formatting is correctly applied.)

If you want to restore the \styfmt{glossaries} v3.0+ definition just do:
\begin{verbatim}
\renewcommand*{\glsseeitemformat}[1]{\glsentrytext{#1}}
\end{verbatim}

The \sty{glossaries-extra} package provides \ics{glsxtrhiername}
and its case-changing variants that may be used within
the definition of \cs{glsseeitemformat} if required. These
display the hierarchy for sub-entries rather than just the name,
which may be more helpful in cross-references.
\begin{definition}[\DescribeMacro\glsxtrhiername]
\cs{glsxtrhiername}\marg{label}
\end{definition}
performs a recursive action:
\begin{enumerate}
\item If the entry given by \meta{label} has a parent, then
\cs{glsxtrhiername}\marg{parent label} is done followed by
\cs{glsxtrhiernamesep} then:
\item If the entry given by \meta{label} is an abbreviation
(that is, it has the \gloskey{short} field set) then
the short form is displayed (using \cs{glsfmtshort})
otherwise the name is displayed (using \cs{glsfmtname}).
\end{enumerate}
The first step above is skipped if the entry doesn't have a parent.
Each level is separated by:
\begin{definition}[\DescribeMacro\glsxtrhiernamesep]
\cs{glsxtrhiernamesep}
\end{definition}
which defaults to \qt{\glsxtrhiernamesep}. This can be redefined as
appropriate.

There are some case-changing variants:
\begin{definition}[\DescribeMacro\Glsxtrhiername]
\cs{Glsxtrhiername}\marg{label}
\end{definition}
The top-level has the first letter changed to upper case 
(either \cs{Glsfmtshort} or \cs{Glsfmtname}). There's
no case-change for sub-entries.

\begin{definition}[\DescribeMacro\GlsXtrhiername]
\cs{GlsXtrhiername}\marg{label}
\end{definition}
All levels have the first letter changed to upper case 
(either \cs{Glsfmtshort} or \cs{Glsfmtname}).

\begin{definition}[\DescribeMacro\GLSxtrhiername]
\cs{GLSxtrhiername}\marg{label}
\end{definition}
The top-level is converted to upper case 
(either \cs{GLSfmtshort} or \cs{GLSfmtname}). There's
no case-change for sub-entries.

\begin{definition}[\DescribeMacro\GLSXTRhiername]
\cs{GLSXTRhiername}\marg{label}
\end{definition}
All levels are converted to upper case 
(either \cs{GLSfmtshort} or \cs{GLSfmtname}).

Suppose you want to suppress the \gls*{numberlist}
using \pkgopt{nonumberlist}. This will automatically prevent the
cross-references from being displayed. The
\pkgopt{seeautonumberlist} package option will automatically
enable the \gls*{numberlist} for entries that have the \gloskey{see}
key set, but this will also show the rest of the \gls*{numberlist}.

Another approach in this situation is to use the post description
hook with \cs{glsxtrusesee} to append the cross-reference after
the description. For example:
\begin{verbatim}
\renewcommand*{\glsxtrpostdescgeneral}{%
 \ifglshasfield{see}{\glscurrententrylabel}
 {, \glsxtrusesee{\glscurrententrylabel}}%
 {}%
}
\end{verbatim}
Now the cross-references can appear even though the \gls{numberlist}
has been suppressed.

As from v1.16, there's a separate \gloskey{seealso} key. Unlike
\gloskey{see}, this doesn't have an optional part for the textual
tag. The syntax \texttt{seealso=\marg{xr-labels}} works in much the same way as using 
\texttt{see=[\cs{seealsoname}]\marg{xr-labels}} but the information
is stored in a separate field. If you need a different tag, use the
\gloskey{see} key instead (or redefine \cs{seealsoname} or
\cs{glsxtruseseealsoformat}, described below).

You can display the formatted list of
cross-references stored in the \gloskey{seealso} key using:
\begin{definition}[\DescribeMacro\glsxtruseseealso]
\cs{glsxtruseseealso}\marg{label}
\end{definition}
This works in much the same way as \cs{glsxtrusesee}
but it internally uses
\begin{definition}[\DescribeMacro\glsxtruseseeformat]
\cs{glsxtruseseealsoformat}\marg{xr list}
\end{definition}
For example:
\begin{verbatim}
\renewcommand*{\glsxtrpostdescgeneral}{%
 \ifglshasfield{see}{\glscurrententrylabel}
 {, \glsxtrusesee{\glscurrententrylabel}}%
 {}%
 \ifglshasfield{seealso}{\glscurrententrylabel}
 { (\glsxtruseseealso{\glscurrententrylabel})}%
 {}%
}
\end{verbatim}

The \gloskey{alias} key only contains a single label not a list, but
if you want to ensure consistent formatting with \cs{glsxtrusesee}
and \cs{glsxtruseseealso} you can use (from v1.42):
\begin{definition}[\DescribeMacro\glsxtrusealias]
\cs{glsxtrusealias}\marg{label}
\end{definition}

The actual unformatted comma-separated list \meta{xr-list}
stored in the \gloskey{seealso} field can be accessed with:
\begin{definition}[\DescribeMacro\glsxtrseealsolabels]
\cs{glsxtrseealsolabels}\marg{label}
\end{definition}
This will just expand to the \meta{xr-labels} provided in the value
of the \gloskey{seealso} key. There's no corresponding command to
access the \gloskey{see} field. If you really need to access it, you
can use commands like \cs{glsxtrfielduse}, but remember that it
may start with \oarg{tag}, so it can't be automatically treated
as a simple comma-separated list.

As mentioned above, the base \styfmt{glossaries} package provides \ics{glsseelist},
which requires a comma-separated list of labels as the argument.
The argument isn't fully expanded, so it's not suitable to use, for
example, \cs{glsxtrseealsolabels}\marg{label} as the argument.
For convenience, \styfmt{glossaries-extra} provides
\begin{definition}[\DescribeMacro\glsxtrseelist]
\cs{glsxtrseelist}\marg{xr list}
\end{definition}
which fully expands its argument and passes it to \cs{glsseelist}.

The \gloskey{seealso} key implements the automatic indexing using
\begin{definition}[\DescribeMacro\glsxtrindexseealso]
\cs{glsxtrindexseealso}\marg{label}\marg{xr list}
\end{definition}
which just does
\begin{alltt}
\cs{glssee}[\cs{seealsoname}]\marg{label}\marg{xr list}
\end{alltt}
unless the \pkgopt{xindy} option is used with \sty{glossaries}
v4.30+, in which case a distinct \texttt{seealso} cross-reference
class is used instead.

The command that produces this \qt{see also} text is
\begin{definition}[\DescribeMacro\seealsoname]
\cs{seealsoname}
\end{definition}
If \ics{alsoname} (provided by language packages) is defined then
\cs{seealsoname} will simply be defined to \cs{alsoname} otherwise
it will be defined to \texttt{see also}.

The language-sensitive \cs{alsoname} is used by general indexing
packages, such as \sty{makeidx}, so if you redefine \cs{alsoname}
the change will apply to normal indexes as well as glossaries. If
you only want to change the text produced by the \gloskey{seealso}
key without affecting general indexing (with \cs{index}) then
redefine \cs{seealsoname} instead.

\section{Entry Display Style Modifications}
\label{sec:entryfmtmods}

Recall from the \styfmt{glossaries} package that commands such as \cs{gls}
display text at that point in the document (optionally with a
hyperlink to the relevant line in the glossary). This text is
referred to as the \qt{\gls{linktext}} regardless of whether or not it
actually has a hyperlink. The actual text and the way it's displayed
depends on the command used (such as \cs{gls}) and the entry format.

The default entry format (\ics{glsentryfmt}) used in the
\gls{linktext} by commands like \ics{gls}, \ics{glsxtrfull},
\ics{glsxtrshort} and \ics{glsxtrlong} (but not commands like
\ics{glslink}, \ics{glsfirst} and \cs{glstext}) is changed by
\styfmt{glossaries-extra} to test for regular entries, which are
determined as follows:

\begin{itemize}
\item If an entry is assigned to a category that has the
\catattr{regular} attribute set (see \sectionref{sec:categories}), the entry is considered a~regular
entry, even if it has a~value for the \gloskey{short} key.
In this case \cs{glsentryfmt} uses \ics{glsgenentryfmt}
(provided by \styfmt{glossaries}), which uses the \gloskey{first}
(or \gloskey{firstplural}) value on \gls{firstuse} and the
\gloskey{text} (or \gloskey{plural}) value on subsequent use.

\item An entry that doesn't have a~value for the \gloskey{short} 
key is assumed to be a~regular entry, even if the
\catattr{regular} attribute isn't set to \qt{true} (since it can't
be an abbreviation without the short form).
In this case \cs{glsentryfmt} uses \ics{glsgenentryfmt}.

\item If an entry does has a~value for the \gloskey{short} key
and hasn't been marked as a~regular entry through the 
\catattr{regular} attribute, it's not considered a regular entry. 
In this case \cs{glsentryfmt} uses \ics{glsxtrgenabbrvfmt}
(defined by \styfmt{glossaries-extra}) which is governed
by the abbreviation style (see \sectionref{sec:abbrstyle}).

\end{itemize}

This means that entries with a short
form can be treated as regular entries rather than
abbreviations if it's more appropriate for the desired style.

As from version 1.04, \ics{glsentryfmt} now puts \ics{glsgenentry}
in the argument of the new command
\begin{definition}[\DescribeMacro\glsxtrregularfont]
\cs{glsxtrregularfont}\marg{text}
\end{definition}
This just does its argument \meta{text} by default. This means that
if you want regular entries in a different font but don't want
that font to apply to abbreviations, then you can redefine
\cs{glsxtrregularfont}. This is more precise than changing
\ics{glstextformat} which is applied to all linking commands
for all entries, unless overridden by the \catattr{textformat}
attribute.

For example:
\begin{verbatim}
\renewcommand*{\glsxtrregularfont}[1]{\textsf{#1}}
\end{verbatim}
You can access the label through \cs{glslabel}. For example,
you can query the category:
\begin{verbatim}
\renewcommand*{\glsxtrregularfont}[1]{%
 \glsifcategory{\glslabel}{general}{\textsf{#1}}{#1}}
\end{verbatim}
or query the category attribute, for example, provide a custom
attribute called \texttt{font}:
\begin{verbatim}
\glssetcategoryattribute{general}{font}{sf}

\renewcommand*{\glsxtrregularfont}[1]{%
 \glsifattribute{\glslabel}{font}{sf}{\textsf{#1}}{#1}}
\end{verbatim}
As from version 1.21, it's simpler to just do, for example:
\begin{verbatim}
\glssetcategoryattribute{general}{textformat}{textsf}
\end{verbatim}
without redefining \cs{glsxtrregularfont}.

As from version 1.30, there is also a command for abbreviations
that encapsulates \cs{glsxtrgenabbrvfmt}:
\begin{definition}[\DescribeMacro\glsxtrabbreviationfont]
\cs{glsxtrabbreviationfont}\marg{text}
\end{definition}
This also just does its argument by default. Font changes
made by abbreviation styles are within \meta{text}.

The \cs{glspostlinkhook} provided by the \styfmt{glossaries} package to
insert information after the \gls{linktext} produced by commands like \cs{gls} and \cs{glstext} is
redefined to
\begin{definition}[\DescribeMacro\glsxtrpostlinkhook]
\cs{glsxtrpostlinkhook}
\end{definition}
This command will discard a following full stop (period) if the
\catattr{discardperiod} attribute is set to \qt{true} for the
current entry's category. It will also do
\begin{definition}[\DescribeMacro\glsxtrpostlink]
\cs{glsxtrpostlink}
\end{definition}
if a full stop hasn't be discarded and
\begin{definition}[\DescribeMacro\glsxtrpostlinkendsentence]
\cs{glsxtrpostlinkendsentence}
\end{definition}
if a full stop has been discarded.

It may be that you want to check some other setting (rather
than a category attribute) to determine whether or not to discard
a following full stop. In which case you can redefine:
\begin{definition}[\DescribeMacro\glsxtrifcustomdiscardperiod]
\cs{glsxtrifcustomdiscardperiod}\marg{true}\marg{false}
\end{definition}
You can access the field's label using \cs{glslabel}.
This command should do \meta{true} if the post-link hook
should check if a period follows and \meta{false} otherwise.
The default definition is simply:
\begin{verbatim}
\newcommand*{\glsxtrifcustomdiscardperiod}[2]{#2}
\end{verbatim}
which means that no additional checks are performed.
(Only the recognised category attributes will be checked.)

\begin{important}
Avoid the use of \cs{gls}-like and \cs{glstext}-like commands
within the post-link hook as they will cause interference.
Consider using commands like \cs{glsentrytext}, \cs{glsaccesstext}
or \cs{glsxtrp} (\sectionref{sec:nested}) instead.
\end{important}

By default \cs{glsxtrpostlink} just does \postlinkcat\ if it exists, where
\meta{category} is the category label for the current entry.
(For example,  for the \category{general} category, 
\cs{glsxtrpostlinkgeneral} if it has been defined.)

You can define the post-link hook command using \cs{newcommand}, for
example:
\begin{verbatim}
\newcommand*{\glsxtrpostlinkgeneral}{%
 \glsxtrpostlinkAddDescOnFirstUse
}
\end{verbatim}
or, as from v1.31, you can use
\begin{definition}[\DescribeMacro\glsdefpostlink]
\cs{glsdefpostlink}\marg{category}\marg{definition}
\end{definition}
This is simply a shortcut for:
\begin{alltt}
\cs{csdef}\{glsxtrpostlink\meta{category}\}\marg{definition}
\end{alltt}
Note that it doesn't check if the command has already been defined.

\begin{sloppypar}
The sentence-ending hook is slightly more complicated.
If the command \postlinkcat\ is defined the hook will do that
and then insert a full stop with the space factor adjusted to match
the end of sentence. If \postlinkcat\ hasn't
been defined, the space factor is adjusted to match the end of
sentence. This means that if you have, for example, an entry that
ends with a full stop, a redundant following full stop will be
discarded and the space factor adjusted (in case the entry is in
uppercase) unless the entry is followed by additional material, in
which case the following full stop is no longer redundant and needs
to be reinserted.
\end{sloppypar}

There are some convenient commands you might want to use
when customizing the post-\gls{linktext} category hooks:
\begin{definition}[\DescribeMacro\glsxtrpostlinkAddDescOnFirstUse]
\cs{glsxtrpostlinkAddDescOnFirstUse}
\end{definition}
This will add the description in parentheses on \gls{firstuse}.

For example, suppose you want to append the description in
parentheses on \gls{firstuse} for entries in the \category{symbol}
category:
\begin{verbatim}
\newcommand*{\glsxtrpostlinksymbol}{%
  \glsxtrpostlinkAddDescOnFirstUse
}
\end{verbatim}

\begin{definition}[\DescribeMacro\glsxtrpostlinkAddSymbolOnFirstUse]
\cs{glsxtrpostlinkAddSymbolOnFirstUse}
\end{definition}
This will append the symbol (if defined) in parentheses on
\gls{firstuse}. (Does nothing if the symbol hasn't been set.)

\begin{definition}[\DescribeMacro\glsxtrpostlinkAddSymbolDescOnFirstUse]
\cs{glsxtrpostlinkAddSymbolDescOnFirstUse}
\end{definition}
(New to v1.31.) On \gls{firstuse},
this will append \cs{space}(\meta{symbol}, \meta{description}) if
the symbol is defined otherwise it just appends
\cs{space}(\meta{description}).

If you want to provide your own custom format be aware that you
can't use \ics{ifglsused} within the post-\gls{linktext} hook as by this point
the \gls{firstuseflag} will have been unset. Instead you can use
\begin{definition}[\DescribeMacro\glsxtrifwasfirstuse]
\cs{glsxtrifwasfirstuse}\marg{true}\marg{false}
\end{definition}
This will do \meta{true} if the last used entry was the
\gls{firstuse}
for that entry, otherwise it will do \meta{false}. (Requires at
least \styfmt{glossaries} v4.19 to work properly.) This command is
locally set by commands like \cs{gls}, so don't rely on it outside
of the post-\gls{linktext} hook.

\begin{important}
Note that commands like \ics{glsfirst} and \ics{glsxtrfull} fake
\gls{firstuse} for the benefit of the post-\gls{linktext} hooks by setting 
\cs{glsxtrifwasfirstuse} to \cs{@firstoftwo}.
(Although, depending on the styles in use, they may not exactly match
the text produced by \ics{gls}-like commands on \gls{firstuse}.)
However, the \abbrstyle{short-postfootnote} style alters \cs{glsxtrfull}
so that it fakes non-\gls{firstuse} otherwise the inline full format
would include the footnote, which is inappropriate.
\end{important}

For example, if you want to place the description in a footnote
after the \gls{linktext} on \gls{firstuse} for the \category{general} category:
\begin{verbatim}
\newcommand*{\glsxtrpostlinkgeneral}{%
  \glsxtrifwasfirstuse{\footnote{\glsentrydesc{\glslabel}}}{}%
}
\end{verbatim}

The \abbrstyle{short-postfootnote} abbreviation style uses the 
post-\gls{linktext} hook to
place the footnote after trailing punctuation characters.


\section{Entry Counting Modifications}
\label{sec:entrycountmods}

If you are using \gls{bib2gls} you may find it more convenient
to use the record count commands described in
\sectionref{sec:bib2gls} instead.

The \ics{glsenableentrycount} command is modified to allow
for the \catattr{entrycount} attribute. This means that
you not only need to enable entry counting with
\cs{glsenableentrycount}, but you also need
to set the appropriate attribute (see \sectionref{sec:categories}).

For example, instead of just doing:
\begin{verbatim}
\glsenableentrycount
\end{verbatim}
you now need to do:
\begin{verbatim}
\glsenableentrycount
\glssetcategoryattribute{abbreviation}{entrycount}{1}
\end{verbatim}
This will enable the entry counting for entries in the
\category{abbreviation} category, but any entries assigned to
other categories will be unchanged.

Further information about entry counting, including the
new per-unit feature, is described in \sectionref{sec:entrycount}.

\section{First Use Flag}
\label{sec:glsunset}

The \styfmt{glossaries} package provides
\begin{definition}
\ics{ifglsused}\marg{label}\marg{true}\marg{false}
\end{definition}
to determine whether or not an entry has been used. This requires
the entry to have been defined. If the entry is undefined, then the
underlying boolean variable doesn't exist and so is neither true
nor false. This command will produce an error with the default
\pkgopt[error]{undefaction} mode and a warning with
\pkgopt[warn]{undefaction} and, where appropriate, displays ?? in
the document text to denote an undefined reference. With both modes,
neither \meta{true} nor \meta{false} will be performed if
\meta{label} doesn't exist.

The \pkgopt{record} option automatically implements
\pkgopt[warn]{undefaction} since no entries will be defined until
they have been selected by \gls{bib2gls}. In this case, you
may prefer to use:
\begin{definition}[\DescribeMacro\GlsXtrIfUnusedOrUndefined]
\cs{GlsXtrIfUnusedOrUndefined}\marg{label}\marg{true}\marg{false}
\end{definition}
(new to v1.34) which does \meta{true} if the entry given by
\meta{label} is undefined or if the entry is marked as unused.
Remember that neither \cs{ifglsused} nor
\cs{GlsXtrIfUnusedOrUndefined} should be used in the post-link
hook as the \gls{firstuseflag} will have already been unset
before the hook is performed. (Instead, \ics{glsxtrifwasfirstuse}
should be used, see \sectionref{sec:entryfmtmods}.)

There are two new commands provided with version 1.31+:
\begin{definition}[\DescribeMacro\glslocalreseteach]
\cs{glslocalreseteach}\marg{list}
\end{definition}
and 
\begin{definition}[\DescribeMacro\glslocalunseteach]
\cs{glslocalunseteach}\marg{list}
\end{definition}
These behave like \cs{glslocalreset} and \cs{glslocalunset}
but the argument is a comma-separated list of labels.

The internal command used by \cs{glsunset} is modified first to
allow for the changing in entry counting, described above, but
also to allow for buffering pending unsets when commands like
\cs{gls} are used in a context where changing a boolean variable 
can cause things to go wrong. One example of this is using
\cs{gls} in one of the commands provided with the \isty{soul}
package. For example:
\begin{verbatim}
\ul{Some text about \gls{html}.}
\end{verbatim}
This causes the confusing error:
\begin{verbatim}
Glossary entry `{html}' has not been defined.
\end{verbatim}
The simplest workaround is to put \verb|\gls{html}| inside the
argument of \cs{mbox}. For example:
\begin{verbatim}
\ul{Some text about \mbox{\gls{html}}.}
\end{verbatim}
This can work provided it's not the first use of this entry.
It if is, then unsetting the first use flag causes a problem
and results in the error:
\begin{verbatim}
! Package soul Error: Reconstruction failed.
\end{verbatim}
The \styfmt{glossaries-extra} package provides a way of temporarily
switching off \cs{glsunset} so that it just makes a note of the
entry's label but doesn't actually perform the change:
\begin{definition}[\DescribeMacro\GlsXtrStartUnsetBuffering]
\cs{GlsXtrStartUnsetBuffering}
\end{definition}
The unstarred version doesn't check for duplicates, so the internal
list may end up with multiple occurrences of the same label. The
starred version only adds a label to the internal list if it's not
already in it. Note that this buffering only applies to the
global \cs{glsunset} and does not affect the local
\cs{glslocalunset}.

Later you can restore \cs{glsunset} and unset all buffered
labels using:
\begin{definition}[\DescribeMacro\GlsXtrStopUnsetBuffering]
\cs{GlsXtrStopUnsetBuffering}
\end{definition}
The starred form \cs{GlsXtrStopUnsetBuffering*} uses 
\cs{glslocalunset} instead. For example:
\begin{verbatim}
\documentclass{article}

\usepackage[T1]{fontenc}
\usepackage{soul}
\usepackage{glossaries-extra}

\newabbreviation{html}{HTML}{hypertext markup language}

\begin{document}
\GlsXtrStartUnsetBuffering
\ul{Some text about \mbox{\gls{html}}.}
\GlsXtrStopUnsetBuffering

Next use: \gls{html}.
\end{document}
\end{verbatim} 
Before you stop the unset buffering, you can iterate over the
current buffer using
\begin{definition}[\DescribeMacro\GlsXtrForUnsetBufferedList]
\cs{GlsXtrForUnsetBufferedList}\marg{cs}
\end{definition}
where \meta{cs} is a control sequence that takes a single argument
(which is the entry label). This is best used with the starred
version \cs{GlsXtrStartUnsetBuffering*} to avoid duplicates.

You can discard the buffer and restore \cs{glsunset} to its normal
behaviour with (new to v1.42):
\begin{definition}[\DescribeMacro\GlsXtrDiscardUnsetBuffering]
\cs{GlsXtrDiscardUnsetBuffering}
\end{definition}

\begin{important}
Note that since the change in the first use flag now doesn't
occur until \cs{GlsXtrStopUnsetBuffering}, multiple references
of the same term within the buffering zone will always be
treated as first use (if the term wasn't used before the
buffering started).
\end{important}

There can still be a problem here as content within \cs{mbox}
can't break across a line so you may end up with an overfull
line or excessive white space within the paragraph.

An alternative is to use \cs{protect}:
\begin{verbatim}
\GlsXtrStartUnsetBuffering
\ul{Some text about \protect\gls{html}.}
\GlsXtrStopUnsetBuffering
\end{verbatim}
but the formatting (underlining in this example) won't be applied.
Another possibility is:
\begin{verbatim}
\usepackage[T1]{fontenc}
\usepackage{soul}
\usepackage{glossaries-extra}

\newabbreviation{html}{HTML}{hypertext markup language}

\newrobustcmd{\gul}[1]{%
  {%
   \def\glsxtrabbreviationfont##1{\GlsXtrExpandedFmt{\ul}{##1}}%
   \def\glsxtrregularfont##1{\GlsXtrExpandedFmt{\ul}{##1}}%
   #1%
  }%
}

\begin{document}
\ul{Some text about }\gls[textformat=gul]{html}.

Next use: \gls{html}.
\end{document}
\end{verbatim}
This moves \cs{gls} outside of \cs{ul} and uses
\gloskey[glslink]{textformat} to locally change the formatting
command used by \cs{gls} (which is normally given by
\cs{glstextformat} or the \catattr{textformat} attribute)
to a custom command \cs{gul}, which locally changes the regular font
and the default abbreviation font to use \cs{ul}. It then uses
\begin{definition}[\DescribeMacro\GlsXtrExpandedFmt]
\cs{GlsXtrExpandedFmt}\marg{cs}\marg{text}
\end{definition}
which (protected) fully expands \meta{text} before applying
\meta{cs}, which must be a control sequence that takes a single
argument. This allows \cs{ul} to move much further inside
and increases its chances of working. It can still break if
\meta{text} expands to something that \cs{ul} can't deal with.
For example, if an abbreviation style is used that contains
complex formatting or if the field value contains problematic
content.

\section{Plurals}

Some languages, such as English, have a general rule that plurals
are formed from the singular with a suffix appended. This isn't 
an absolute rule. There are plenty of exceptions (for example,
geese, children, churches, elves, fairies, sheep). The
\sty{glossaries} package allows the \gloskey{plural} key to be
optional when defining entries. In some cases a plural may not make
any sense (for example, the term is a symbol) and in some cases
the plural may be identical to the singular.

To make life easier for languages where the majority of plurals can
simply be formed by appending a suffix to the singular, the
\sty{glossaries} package lets the \gloskey{plural} field default
to the value of the \gloskey{text} field with \ics{glspluralsuffix}
appended. This command is defined to be just the letter \qt{s}.
This means that the majority of terms don't need to have the
\gloskey{plural} supplied as well, and you only need to use it for the
exceptions.

For languages that don't have this general rule, the \gloskey{plural}
field will always need to be supplied, where needed.

There are other plural fields, such as \gloskey{firstplural},
\gloskey{longplural} and \gloskey{shortplural}. Again, if you are using
a language that doesn't have a simple suffix rule, you'll have to
supply the plural forms if you need them (and if a plural makes
sense in the context).

If these fields are omitted, the \sty{glossaries} package follows
these rules:
\begin{itemize}
\item If \gloskey{firstplural} is missing, then \cs{glspluralsuffix}
is appended to the \gloskey{first} field, if that field has been
supplied. If the \gloskey{first} field hasn't been supplied but the
\gloskey{plural} field has been supplied, then the \gloskey{firstplural}
field defaults to the \gloskey{plural} field. If the \gloskey{plural}
field hasn't been supplied, then both the \gloskey{plural} and
\gloskey{firstplural} fields default to the \gloskey{text} field (or
\gloskey{name}, if no \gloskey{text} field) with \cs{glspluralsuffix}
appended.

\item If the \gloskey{longplural} field is missing, then 
\cs{glspluralsuffix} is appended to the \gloskey{long} field, if the
\gloskey{long} field has been supplied.

\item If the \gloskey{shortplural} field is missing then, \emph{with
the base \sty{glossaries} acronym mechanism}, \ics{acrpluralsuffix}
is appended to the \gloskey{short} field.

\end{itemize}

This \emph{last case is changed} with \styfmt{glossaries-extra}.
With this extension package, the \gloskey{shortplural} field
defaults to the \gloskey{short} field with \ics{abbrvpluralsuffix}
appended unless overridden by category attributes. This
suffix command is set by the abbreviation styles. This means that
every time an abbreviation style is implemented,
\cs{abbrvpluralsuffix} is redefined. In most cases its redefined to
use
\begin{definition}[\DescribeMacro\glsxtrabbrvpluralsuffix]
\cs{glsxtrabbrvpluralsuffix}
\end{definition}
which defaults to just \cs{glspluralsuffix}. Some of the
abbreviation styles have their own command for the plural suffix,
such as \cs{glsxtrscsuffix},
so if you want to completely strip all the plural suffixes used for
abbreviations then you need to redefine \cs{glsxtrabbrvpluralsuffix}
\emph{not} \cs{abbrvpluralsuffix}, which changes with the style. Redefining
\cs{acrpluralsuffix} will have no affect, since it's not used by the
new abbreviation mechanism.

If you require a mixture (for example, in a multilingual document),
there are two attributes that affect the short plural suffix
formation. The first is \catattr{aposplural} which uses the suffix
\begin{verbatim}
'\abbrvpluralsuffix
\end{verbatim}
That is, an apostrophe followed by \cs{abbrvpluralsuffix} is
appended. The second attribute is \catattr{noshortplural} which
suppresses the suffix and simply sets \gloskey{shortplural} to the
same as \gloskey{short}.

\section{Nested Links}
\label{sec:nested}

Complications arise when you use \ics{gls} in the 
value of the \gloskey{name} field (or \gloskey{text} 
or \gloskey{first} fields, if set). This tends to occur with
abbreviations that extend other abbreviations. For example,
SHTML is an abbreviation for SSI enabled HTML, where SSI
is an abbreviation for Server Side Includes and HTML
is an abbreviation for Hypertext Markup Language.

Things can go wrong if we try the following with the
\styfmt{glossaries} package:
\begin{verbatim}
\newacronym{ssi}{SSI}{Server Side Includes}
\newacronym{html}{HTML}{Hypertext Markup Language}
\newacronym{shtml}{S\gls{html}}{\gls{ssi} enabled \gls{html}}
\end{verbatim}

The main problems are:
\begin{enumerate}
\item\label{itm:nestedfirstucprob} The first letter upper casing commands, such as \ics{Gls},
won't work for the \texttt{shtml} entry on \gls{firstuse} if the
long form is displayed before the short form (which is the
default abbreviation style). This will attempt to do
\begin{verbatim}
\gls{\uppercase ssi} enabled \gls{html}
\end{verbatim}
which just doesn't work. Grouping the \verb|\gls{ssi}| doesn't
work either as this will effectively try to do
\begin{verbatim}
\uppercase{\gls{ssi}} enabled \gls{html}
\end{verbatim}
This will upper case the label \texttt{ssi} so the entry won't be
recognised. This problem will also occur if you use the all capitals
version, such as \ics{GLS}.

\item\label{itm:nonexpandprob} The long and abbreviated forms accessed through
\ics{glsentrylong} and \ics{glsentryshort} are no longer expandable
and so can't be used be used in contexts that require this,
such as PDF bookmarks.

\item\label{itm:nestedsortprob} The nested commands may end up in the \gloskey{sort} key,
which will confuse the indexing.

\item\label{itm:inconsistentfirstuseprob} The \texttt{shtml} entry produces inconsistent results
depending on whether the \texttt{ssi} or \texttt{html} entries have
been used. Suppose both \texttt{ssi} and \texttt{html} are used
before \texttt{shtml}. For example:
\begin{verbatim}
This section discusses \gls{ssi}, \gls{html} and \gls{shtml}.
\end{verbatim}
This produces:
\begin{quote}
This section discusses server side includes (SSI), hypertext markup
language (HTML) and SSI enabled HTML (SHTML).
\end{quote}
So the \gls{firstuse} of the \texttt{shtml} entry produces
\qt{SSI enabled HTML (SHTML)}.

Now let's suppose the \texttt{html} entry is used before the
\texttt{shtml} but the \texttt{ssi} entry is used after the
\texttt{shtml} entry, for example:
\begin{verbatim}
The sample files are either \gls{html} or \gls{shtml}, but let's
first discuss \gls{ssi}.
\end{verbatim}
This produces:
\begin{quote}
The sample files are either hypertext markup language (HTML) or
server
side includes (SSI) enabled HTML (SHTML), but let's first discuss
SSI.
\end{quote}
So the \gls{firstuse} of the \texttt{shtml} entry now produces
\qt{server side includes (SSI) enabled HTML (SHTML)}, which looks
a bit strange.

Now let's suppose the \texttt{shtml} entry is used before (or
without) the other two entries:
\begin{verbatim}
This article is an introduction to \gls{shtml}.
\end{verbatim}
This produces:
\begin{quote}
This article is an introduction to server side includes (SSI)
enabled hypertext markup language (HTML) (SHTML).
\end{quote}
So the \gls{firstuse} of the \texttt{shtml} entry now produces
\qt{server side includes (SSI) enabled hypertext markup language (HTML) 
(SHTML)}, which is even more strange.

This is all aggravated by setting the style using the
\styfmt{glossaries} package's
\cs{setacronymstyle}. For example:
\begin{verbatim}
\setacronymstyle{long-short}
\end{verbatim}
as this references the label through the use of \cs{glslabel}
when displaying the long and short forms, but this value
changes with each use of \cs{gls}, so instead of displaying
\qt{(SHTML)} at the end of the \gls{firstuse}, it now displays
\qt{(HTML)}, since \cs{glslabel} has been changed to \texttt{html}
by \verb|\gls{html}|.

Another oddity occurs if you reset the \texttt{html} entry between
uses of the \texttt{shtml} entry. For example:
\begin{verbatim}
\gls{shtml} ... \glsreset{html}\gls{shtml}
\end{verbatim}
The next use of \texttt{shtml} produces \qt{Shypertext markup
language (HTML)}, which is downright weird.

Even without this, the short form has nested formatting commands,
which amount to \verb|\acronymfont{S\acronymfont{HTML}}|. This
may not be a problem for some styles, but if you use one of the 
\qt{sm} styles (that use \ics{textsmaller}), this will produce
an odd result.

\item\label{itm:indexingprob} Each time the \texttt{shtml} entry is used, the 
\texttt{html} entry will also be indexed and marked as used,
and on \gls{firstuse} this will happen to both the \texttt{ssi}
and \texttt{html} entries.  This kind of duplication in the location
list isn't usually particularly helpful to the reader.

\item\label{itm:nestedhyplinkprob} If \sty{hyperref} is in use, you'll get nested hyperlinks
and there's no consistent way of dealing with this across the 
available PDF viewers. If on the \gls{firstuse} case, the user
clicks on the \qt{HTML} part of the \qt{SSI enabled HTML (SHTML)}
link, they may be directed to the HTML entry in the glossary or 
they may be directed to the SHTML entry in the glossary.

\end{enumerate}

For these reasons it's better to use the simple expandable commands like
\ics{glsentrytext} or \ics{glsentryshort} in the definition
of other entries (although that doesn't fix the first problem).
Alternatively use something like:
\begin{verbatim}
\newacronym
 [description={\acrshort{ssi} enabled \acrshort{html}}]
 {shtml}{SHTML}{SSI enabled HTML}
\end{verbatim}
with \styfmt{glossaries} or:
\begin{verbatim}
\newabbreviation
 [description={\glsxtrshort{ssi} enabled \glsxtrshort{html}}]
 {shtml}{SHTML}{SSI enabled HTML}
\end{verbatim}
with \styfmt{glossaries-extra}. This fixes all the above
listed problems (as long as you don't use \ics{glsdesc}).
Note that replacing \cs{gls} with \cs{acrshort} in the original
example may fix the \gls{firstuse} issue, but it doesn't
fix any of the other problems listed above.

If it's simply that you want to use the abbreviation font, you can
use \cs{glsabbrvfont}:
\begin{verbatim}
\setabbreviationstyle{long-short-sc}

\newabbreviation{ssi}{ssi}{server-side includes}
\newabbreviation{html}{html}{hypertext markup language}
\newabbreviation{shtml}{shtml}{\glsabbrvfont{ssi} enabled
\glsabbrvfont{html}}
\end{verbatim}
This will pick up the font style setting of the outer entry (shtml,
in the above case). This isn't a problem in the above example as all
the abbreviations use the same style.

However if you're really determined to use
\ics{gls} in a field that may be included within 
some \gls{linktext}, \styfmt{glossaries-extra} patches internals
used by the linking commands so that if \cs{gls} (or plural or
case changing variants) occurs in the \gls{linktext} it will
behave as though you used 
\texttt{\ics{glstext}[hyper=false,noindex]}
instead. Grouping is also added so that, for example, when
\verb|\gls{shtml}| is used for the first time the long form
\begin{verbatim}
\gls{ssi} enabled \gls{html}
\end{verbatim}
is treated as
\begin{verbatim}
{\glstext[hyper=false,noindex]{ssi}} enabled 
{\glstext[hyper=false,noindex]{html}}
\end{verbatim}
This overcomes problems~\ref{itm:inconsistentfirstuseprob}, 
\ref{itm:indexingprob} and \ref{itm:nestedhyplinkprob} listed 
above, but still doesn't fix problems~\ref{itm:nestedfirstucprob}
and \ref{itm:nonexpandprob}.
Problem~\ref{itm:nestedsortprob} usually won't be an issue as most abbreviation
styles set the \gloskey{sort} key to the short form, so using these
commands in the long form but not the short form will only affect
entries with a style that sorts according to the long form (such as
\abbrstyle{long-noshort-desc}).

Additionally, any instance of the long form commands, such
as \ics{glsxtrlong} or \ics{acrlong} will be temporarily
redefined to just use
\begin{alltt}
\{\ics{glsentrylong}\marg{label}\meta{insert}\}
\end{alltt}
(or case-changing versions). Similarly the short form commands,
such as \ics{glsxtrshort} or \ics{acrshort} will use
\ics{glsentryshort} in the argument of either \cs{glsabbrvfont}
(for \cs{glsxtrshort}) or \cs{acronymfont} (for \cs{acrshort}).
So if the \texttt{shtml} entry had instead been defined as:
\begin{verbatim}
\newacronym{shtml}{SHTML}{\acrshort{ssi} enabled \acrshort{html}}
\end{verbatim}
then (using the \abbrstyle{long-short} style) the \gls{firstuse} 
will be like
\begin{verbatim}
{\acronymfont{\glsentryshort{ssi}}} enabled 
{\acronymfont{\glsentryshort{html}}} (SHTML)
\end{verbatim}
whereas if the entry is defined as:
\begin{verbatim}
\newabbreviation{shtml}{SHTML}{\glsxtrshort{ssi} enabled
\glsxtrshort{html}}
\end{verbatim}
then the \gls{firstuse} will be like:
\begin{verbatim}
{\glsabbrvfont{\glsentryshort{ssi}}} enabled 
{\glsabbrvfont{\glsentryshort{html}}} (SHTML)
\end{verbatim}
Note that the first optional argument of
\cs{acrshort} or \cs{glsxtrshort} is ignored in this context. 
(The final optional argument will be inserted, if present.)
The abbreviation style that governs \ics{glsabbrvfont} will be 
set for \ics{glsxtrshort}. Note that \cs{acrshort} doesn't
set the abbreviation style.

Alternatively you can use:
\begin{definition}[\DescribeMacro\glsxtrp]
\cs{glsxtrp}\marg{field}\marg{label}
\end{definition}
where \meta{field} is the field label and corresponds to a 
command in the form \cs{gls}\meta{field} (e.g.\ \cs{glstext}) or
in the form \cs{glsxtr}\meta{field} (e.g.\ \cs{glsxtrshort}).

There's a shortcut command for the most common fields:
\begin{definition}[\DescribeMacro\glsps]
\cs{glsps}\marg{label}
\end{definition}
which is equivalent to \verb|\glsxtrp{short}|\marg{label}, and
\begin{definition}[\DescribeMacro\glspt]
\cs{glspt}\marg{label}
\end{definition}
which is equivalent to \verb|\glsxtrp{text}|\marg{label}.

The \cs{glsxtrp} command behaves much like the
\cs{glsfmt}\meta{field} commands described in
\sectionref{sec:headtitle} but the post-link hook is also
suppressed and extra grouping is added. It automatically sets
\gloskey[glslink]{hyper} to \texttt{false} and
\gloskey[glslink]{noindex} to \texttt{true}.  If you want to change
this, you can use
\begin{definition}[\DescribeMacro\glsxtrsetpopts]
\cs{glsxtrsetpopts}\marg{options}
\end{definition}
For example:
\begin{verbatim}
\glsxtrsetpopts{hyper=false}
\end{verbatim}
will just switch off the hyperlinks but not the indexing.
Be careful using this command or you can end up back to the
original problem of nested links.

The hyper link is re-enabled within glossaries. This is 
done through the command:
\begin{definition}[\DescribeMacro\glossxtrsetpopts]
\cs{glossxtrsetpopts}
\end{definition}
which by default just does
\begin{verbatim}
\glsxtrsetpopts{noindex}
\end{verbatim}
You can redefine this if you want to adjust the setting when
\cs{glsxtrp} is used in the glossary. For example:
\begin{verbatim}
\renewcommand{\glossxtrsetpopts}{\glsxtrsetpopts{noindex=false}}
\end{verbatim}

For example,
\begin{verbatim}
\glsxtrp{short}{ssi}
\end{verbatim}
is equivalent to
\begin{verbatim}
{\let\glspostlinkhook\relax
 \glsxtrshort[hyper=false,noindex]{ssi}[]%
}
\end{verbatim}
in the main body of the document or
\begin{verbatim}
{\let\glspostlinkhook\relax
 \glsxtrshort[noindex]{ssi}[]%
}
\end{verbatim}
inside the glossary. (Note the post-link hook is locally disabled.)

If \verb|\glsxtrp{short}{ssi}| occurs in a sectioning mark,
it's equivalent to
\begin{verbatim}
{\glsxtrheadshort{ssi}}
\end{verbatim}
(which recognises the \catattr{headuc} attribute.)

If \sty{hyperref} has been loaded, then the bookmark will 
use \cs{glsentry}\meta{field}
(\verb|\glsentryshort{ssi}| in the above example).

There are similar commands
\begin{definition}[\DescribeMacro\Glsxtrp]
\cs{Glsxtrp}\marg{field}\marg{label}
\end{definition}
for first letter upper case and
\begin{definition}[\DescribeMacro\GLSxtrp]
\cs{GLSxtrp}\marg{field}\marg{label}
\end{definition}
for all upper case.

\begin{important}
If you use any of the case-changing commands, such as \ics{Gls}
or \ics{Glstext}, (either all caps or first letter upper casing)
don't use any of the linking commands, such as \ics{gls} or
\ics{glstext}, in the definition of entries for any of the
fields that may be used by those case-changing commands.
\end{important}

You can, with care, protect against issue~\ref{itm:nestedfirstucprob} by
inserting an empty group at the start if the long form starts with a
command that breaks the first letter uppercasing commands like
\cs{Gls}, but you still won't be able to use the all caps commands,
such as \cs{GLS}.

If you \emph{really need} nested commands, the safest method is
\begin{verbatim}
\newabbreviation{shtml}{shtml}{{}\glsxtrp{short}{ssi} enabled
\glsxtrp{short}{html}}
\end{verbatim}
but be aware that it may have some unexpected results occasionally.

Example document:
\begin{verbatim}
\documentclass{report}

\usepackage[utf8]{inputenc}
\usepackage[T1]{fontenc}
\usepackage{slantsc}
\usepackage[colorlinks]{hyperref}
\usepackage[nopostdot=false]{glossaries-extra}

\makeglossaries

\setabbreviationstyle{long-short-sc}

\newabbreviation{ssi}{ssi}{server-side includes}
\newabbreviation{html}{html}{hypertext markup language}
\newabbreviation{shtml}{shtml}{{}\glsps{ssi} enabled {}\glsps{html}}

\pagestyle{headings}

\glssetcategoryattribute{abbreviation}{headuc}{true}
\glssetcategoryattribute{abbreviation}{glossdesc}{title}

\begin{document}

\tableofcontents

\chapter{\glsfmtfull{shtml}}

First use: \gls{shtml}, \gls{ssi} and \gls{html}.

Next use: \gls{shtml}, \gls{ssi} and \gls{html}.

\newpage
Next page.

\printglossaries
\end{document}
\end{verbatim}

\section{Acronym Style Modifications}
\label{sec:acronymmods}

The \styfmt{glossaries-extra} package provides a new way of dealing
with abbreviations and redefines \ics{newacronym} to
use \cs{newabbreviation} (see \sectionref{sec:abbreviations}).
The simplest way to update a document that uses
\cs{newacronym} from \styfmt{glossaries} to
\styfmt{glossaries-extra} is do just add
\begin{verbatim}
\setabbreviationstyle[acronym]{long-short}
\end{verbatim}
before you define any entries. If you have used commands like
\ics{acrshort}, \ics{acrlong} and \ics{acrfull}, you need to
substitute them with \ics{glsxtrshort}, \ics{glsxtrlong} and
\ics{glsxtrfull}. (This includes plural and case-changing forms.)
Your text editor's search and replace function should help with
this. If you have used the \pkgopt{shortcuts} package option then
use \pkgopt[ac]{shortcuts} in order to continue using the shortcut
commands like \ics{ac}. With this setting, \ics{acs} will use
\ics{glsxtrshort} instead of \ics{acrshort} etc.

For example, the following document using just \styfmt{glossaries}
\begin{verbatim}
\documentclass{article}
\usepackage[acronym,nopostdot,toc]{glossaries}
\makeglossaries
\setacronymstyle{long-short}
\newacronym{html}{HTML}{hypertext markup language}
\begin{document}
\gls{html}
\printglossaries
\end{document}
\end{verbatim}
can be easily adapted to use \styfmt{glossaries-extra}:
\begin{verbatim}
\documentclass{article}
\usepackage[acronym]{glossaries-extra}
\makeglossaries
\setabbreviationstyle[acronym]{long-short}
\newacronym{html}{HTML}{hypertext markup language}
\begin{document}
\gls{html}
\printglossaries
\end{document}
\end{verbatim}
Table~\ref{tab:acrabbrvstyles} lists the nearest equivalent
\styfmt{glossaries-extra} abbreviation styles for 
the predefined acronym styles provided by \styfmt{glossaries}, but
note that the new styles use different formatting commands. See 
\sectionref{sec:predefabbrvstyles} for further details.

\begin{table}[htbp]
\caption[Old Acronym Styles Verses New Abbreviation Styles]{Old Acronym Styles 
\cs{setacronymstyle}\marg{old-style-name} Verses New Abbreviation 
Styles \cs{setabbreviationstyle}\oarg{category}\marg{new-style-name}}
\label{tab:acrabbrvstyles}
\centering
\begin{tabular}{@{}p{\dimexpr0.3\textwidth-2\tabcolsep}p{0.7\textwidth}@{}}
\bfseries Old Style Name & 
\bfseries New Style Name\\
\acrstyle{long-sc-short} & \abbrstyle{long-short-sc}\\
\acrstyle{long-sm-short} & \abbrstyle{long-short-sm}\\
\acrstyle{long-sp-short} & \abbrstyle{long-short} \newline with
\ttfamily\cs{renewcommand}\{\cs{glsxtrfullsep}\}[1]\{\cs{glsacspace}\{\#1\}\}\\
\acrstyle{short-long} & \abbrstyle{short-long}\\
\acrstyle{sc-short-long} & \abbrstyle{short-sc-long}\\
\acrstyle{sm-short-long} & \abbrstyle{short-sm-long}\\
\acrstyle{long-short-desc} & \abbrstyle{long-short-desc}\\
\acrstyle{long-sc-short-desc} & \abbrstyle{long-short-sc-desc}\\
\acrstyle{long-sm-short-desc} & \abbrstyle{long-short-sm-desc}\\
\acrstyle{long-sp-short-desc} & \abbrstyle{long-short-desc} \newline with
\ttfamily\cs{renewcommand}\{\cs{glsxtrfullsep}\}[1]\{\cs{glsacspace}\{\#1\}\}\\
\acrstyle{short-long-desc} & \abbrstyle{short-long-desc}\\
\acrstyle{sc-short-long-desc} & \abbrstyle{short-sc-long-desc}\\
\acrstyle{sm-short-long-desc} & \abbrstyle{short-sm-long-desc}\\
\acrstyle{dua} & \abbrstyle{long-noshort}\\
\acrstyle{dua-desc} & \abbrstyle{long-noshort-desc}\\
\acrstyle{footnote} & \abbrstyle{short-footnote}\\
\acrstyle{footnote-sc} & \abbrstyle{short-sc-footnote}\\
\acrstyle{footnote-sm} & \abbrstyle{short-sm-footnote}\\
\acrstyle{footnote-desc} & \abbrstyle{short-footnote-desc}\\
\acrstyle{footnote-sc-desc} & \abbrstyle{short-sc-footnote-desc}\\
\acrstyle{footnote-sm-desc} & \abbrstyle{short-sm-footnote-desc}
\end{tabular}
\end{table}

The reason for introducing the new style of abbreviation commands
provided by \styfmt{glossaries-extra} is because the original
acronym commands provided by \styfmt{glossaries} are too restrictive
to work with the internal modifications made by
\styfmt{glossaries-extra}. However, if you really want to restore
the generic acronym function provided
by \styfmt{glossaries} you can use
\begin{definition}[\DescribeMacro\RestoreAcronyms]
\cs{RestoreAcronyms}
\end{definition}
(before any use of \cs{newacronym}).

\cs{RestoreAcronyms} should not be used in combination with the newer
\styfmt{glossaries-extra} abbreviations. Don't combine old and 
new style entries with the same \gloskey{type}. The
original \styfmt{glossaries} acronym mechanism doesn't work well
with the newer \styfmt{glossaries-extra} commands.

\begin{important}
If you use \cs{RestoreAcronyms}, don't use any of the commands
provided by \styfmt{glossaries-extra} intended for abbreviations
(such as \ics{glsxtrshort} or \ics{glsfmtshort}) with entries
defined via \ics{newacronym} as it will cause unexpected 
results.
\end{important}

In general, there's rarely any need for \cs{RestoreAcronyms}. If you
have a document that uses \ics{newacronymstyle}, then it's best to
either stick with just \styfmt{glossaries} for that document or
define an equivalent abbreviation style with
\ics{newabbreviationstyle}. (See \sectionref{sec:newabbrvstyle} for
further details.)

\begin{definition}[\DescribeMacro\glsacspace]
\cs{glsacspace}\marg{label}
\end{definition}
The space command \cs{glsacspace} used by the
\acrstyle{long-sp-short} acronym style provided by \styfmt{glossaries}
is modified so that it uses
\begin{definition}[\DescribeMacro\glsacspacemax]
\cs{glsacspacemax}
\end{definition}
instead of the hard-coded 3em. This is a command not a length
and so can be changed using \cs{renewcommand}.

Any of the new abbreviation styles that use \ics{glsxtrfullsep}
(such as \abbrstyle{long-short}) can easily be changed to use
\cs{glsacspace} with
\begin{verbatim}
\renewcommand*{\glsxtrfullsep}[1]{\glsacspace{#1}}
\end{verbatim}

The \gls{firstuse} acronym font command
\begin{definition}[\firstacronymfont]
\cs{firstacronymfont}\marg{text}
\end{definition}
is redefined to use the \gls{firstuse} abbreviation font command
\ics{glsfirstabbrvfont}. This will be reset if you use
\cs{RestoreAcronyms}.

The subsequent use acronym font command
\begin{definition}[\acronymfont]
\cs{acronymfont}\marg{text}
\end{definition}
is redefined to use the subsequent use abbreviation font command
\ics{glsabbrvfont}. This will be reset if you use
\cs{RestoreAcronyms}.

\section{Glossaries}

The \ics{newignoredglossary}\marg{type} command now (as from v1.11) has a starred
version that doesn't automatically switch off the hyperlinks.
This starred version may be used with the \catattr{targeturl}
attribute to create a link to an external URL. (See
\sectionref{sec:categories} for further details.)
As from v1.12 both the starred and unstarred version check
that the glossary doesn't already exist. (The \styfmt{glossaries}
package omits this check.)

You can now provide an ignored glossary with:
\begin{definition}[\DescribeMacro\provideignoredglossary]
\cs{provideignoredglossary}\marg{type}
\end{definition}
which will only define the glossary if it doesn't already exist.
This also has a starred version that doesn't automatically switch
off hyperlinks.

The individual glossary displaying commands \ics{printglossary},
\ics{printnoidxglossary} and \cs{printunsrtglossary} have extra
keys:

\begin{itemize}
\item \gloskey[printglossary]{target}. This is a boolean key which can
be used to switch off the automatic hypertarget for each entry.
Unlike \cs{glsdisablehyper} this doesn't switch off hyperlinks, so
any cross-references within the glossary won't be affected. This is
a way of avoiding duplicate target warnings if a glossary needs to
be displayed multiple times.
\item
\gloskey[printglossary]{targetnameprefix}\texttt{=}\marg{prefix}. Another way
of avoiding duplicate target names is to set a prefix
used for the names. Unlike changing \cs{glolinkprefix} this doesn't affect
any hyperlinks (such as those created with \cs{gls}).
\item
\gloskey[printglossary]{prefix}\texttt{=}\marg{prefix}. If you do
actually want to locally change \cs{glolinkprefix}, you can use
the \gloskey{prefix} key instead. You need to use the matching
\gloskey[glslink]{prefix} key in commands like \cs{gls}.

\item
\gloskey[printglossary]{label}\texttt{=}\marg{label}. Labels
the glossary with \texttt{\cs{label}\marg{label}} and also
defines \cs{@currentlabelname} to the glossary's TOC title
(to provide text for \cs{nameref}). You can also set the
label before the glossary using:
\begin{definition}[\DescribeMacro\glsxtrsetglossarylabel]
\cs{glsxtrsetglossarylabel}\marg{label}
\end{definition}
but take care of duplicate labels if it's not scoped.

\item
\gloskey[printunsrtglossary]{groups}. This is a boolean key (new to
v1.44) designed for use with \cs{printunsrtglossary} (and other
\cs{printunsrt\ldots} variations). It has no
effect with \cs{printglossary} and \cs{printnoidxglossary}.

If true (default), group formation (group header and group skip)
will be attempted if the \gloskey{group} key has been defined and set. Note
that some styles ignore the header and group skip commands so there
may not be a noticeable difference in those cases. If this key is
set to false, no group formation will occur so there won't be any
group markup to separate letter groups so there will be no visual
separation regardless of style or the group skip setting.

\item
\gloskey[printunsrtglossary]{leveloffset}. This key (new to
v1.44) is designed for use with \cs{printunsrtglossary} (and other
\cs{printunsrt\ldots} variations). It has no
effect with \cs{printglossary} and \cs{printnoidxglossary}.

The value may either be a simple integer \meta{n} to indicate assignment or in the form
\texttt{++\meta{n}} to indicate an increment. This will locally
assign or increment the level offset. The default value is~0. This
will cause entries to be displayed as though their hierarchical
level is $N$ more than it would normally be (where $N$ is the level
offset). For example, a top-level entry (that is, an entry without a
parent) would normally have a hierarchical level of~0. With a level
offset of~1, it would be treated by the glossary style as though it
was actually a level~1 entry. (Remember that some styles don't
support hierarchical glossaries.)

Note that the group formation (if supported) will still occur
between entries that don't have a parent, regardless of the level
offset. This can cause odd results.
\end{itemize}

\subsection{Glossary Style Modifications}
\label{sec:glosstylemods}

The \sty{glossaries-extra-stylemods} package (more conveniently
loaded through the \styfmt{glossaries-extra} 
\pkgopt{stylemods} option) modifies some of the predefined
styles that are provided with the \styfmt{glossaries} package.
These modifications are described in more detail in 
\sectionref{sec:stylemods}.

The \sty{glossaries} package tries to determine the group
title from its label by first checking if 
\cs{}\meta{label}\texttt{groupname} exists. If it doesn't exist,
then the title is assumed to be the same as the label.
For example, when typesetting the \qt{A} letter group,
\sty{glossaries} first checks if \cs{Agroupname} exists.
This could potentially cause conflict with another package
that may have some other meaning for \cs{Agroupname}, so
\sty{glossaries-extra} first checks for the existence
of the internal command \cs{glsxtr@grouptitle@}\meta{label}
which shouldn't clash with another package. You can set the
group title using
\begin{definition}[\DescribeMacro\glsxtrsetgrouptitle]
\cs{glsxtrsetgrouptitle}\marg{label}\marg{title}
\end{definition}
For example:
\begin{verbatim}
\glsxtrsetgrouptitle{A}{A (a)}
\end{verbatim}
This uses a global assignment. If you need to scope the change
you can use
\begin{definition}[\DescribeMacro\glsxtrlocalsetgrouptitle]
\cs{glsxtrlocalsetgrouptitle}\marg{label}\marg{title}
\end{definition}

\subsubsection{Style Hooks}
\label{sec:stylehooks}

The commands \ics{glossentryname} and \ics{glossentrydesc} are
modified to take into account the \catattr{glossname},
\category{glossnamefont}, \catattr{glossdesc} and \catattr{glossdescfont}
attributes (see \sectionref{sec:categories}).  This means you can
make simple font or case-changing modifications to the name and description
without defining a new glossary style.

(New to version 1.42.)
The command \ics{glossentrysymbol} is modified to take into account
the \catattr{glosssymbolfont}. Note that, unlike the above, there's
no corresponding attribute to change the case as it's usually not
appropriate to change the case of a symbol (and for some symbols,
such as pictographs, there's no concept of case). If
\ics{texorpdfstring} has been defined \ics{glossentrysymbol} will
be defined to do:
\begin{alltt}
\ics{texorpdfstring}\marg{\TeX\ code}\marg{PDF}
\end{alltt}
The \meta{\TeX\ code} part is robust and deals with the actual
typesetting of the symbol. The \meta{PDF} part is simply:
\begin{definition}[\DescribeMacro\glsentrypdfsymbol]
\cs{glsentrypdfsymbol}\marg{label}
\end{definition}
which is defined to just do \cs{glsentrysymbol}\marg{label}. The
chances are that the code in the \gloskey{symbol} key won't be valid
in the PDF bookmarks, so you can redefine \cs{glsentrypdfsymbol} to
use a more appropriate field. (If you do redefine this command,
remember that it needs to fully expand.)

For example, if you are using \sty{glossaries-accsupp}, you could
use the \gloskey{symbolaccess} field:
\begin{verbatim}
\renewcommand{\glsentrypdfsymbol}[1]{\glsentrysymbolaccess{#1}}
\end{verbatim}
Alternatively, if you are using \gls{bib2gls} you can use the \TeX\
parser library to interpret a copy of the \gloskey{symbol} field and
use that. For example, with the resource options:
\begin{verbatim}
  replicate-fields={symbol=user1},
  interpret-fields={user1}
\end{verbatim}
This stores the interpreted value of the symbol in the
\gloskey{user1} field, so you can then do:
\begin{verbatim}
\renewcommand{\glsentrypdfsymbol}[1]{\glsentryuseri{#1}}
\end{verbatim}
(You may need \XeLaTeX\ or \LuaLaTeX\ with this method.) This allows
\cs{glossentrysymbol} to be used in a section heading with
standalone definitions.

If you want to adapt a style to use another field instead
of \gloskey{name}, you can use
\begin{definition}[\DescribeMacro\glossentrynameother]
\cs{glossentrynameother}\marg{label}\marg{field}
\end{definition}
This behaves just like \cs{glossentryname} (that is, it obeys
\catattr{glossname}, \catattr{glossnamefont} or \cs{glsnamefont} and uses the post-name
hook) but it uses the given \meta{field} instead of \gloskey{name}.
The \meta{field} argument must be the internal field name
(for example \texttt{desc} rather than \texttt{description}).
See the key to field mappings table in the \styfmt{glossaries}
user manual.

There is a hook after \ics{glossentryname} and 
\ics{Glossentryname}:
\begin{definition}[\DescribeMacro\glsxtrpostnamehook]
\cs{glsxtrpostnamehook}\marg{label}
\end{definition}
By default this checks the \catattr{indexname} attribute.
If the attribute exists for the category to which the label belongs,
then the name is automatically indexed using
\begin{definition}
\ics{glsxtrdoautoindexname}\marg{label}\texttt{\{indexname\}}
\end{definition}
See \sectionref{sec:autoindex} for further details.

\begin{sloppypar}
As from version 1.04, the post-name hook \cs{glsxtrpostnamehook}
will also use \cs{glsxtrpostname}\meta{category} if it exists.
You can use \ics{glscurrententrylabel} to obtain the entry label
with the definition of this command. For example, suppose you are
using a glossary style the doesn't display the symbol, you can
insert the symbol after the name for a particular category, say,
the \qt{symbol} category:
\end{sloppypar}
\begin{verbatim}
\newcommand*{\glsxtrpostnamesymbol}{\space
 (\glsentrysymbol{\glscurrententrylabel})}
\end{verbatim}
For convenience, as from v1.31, you can use
\begin{definition}[\DescribeMacro\glsdefpostname]
\cs{glsdefpostname}\marg{category}\marg{definition}
\end{definition}
This is simply a shortcut for:
\begin{alltt}
\cs{csdef}\{glsxtrpostname\meta{category}\}\marg{definition}
\end{alltt}
Note that it doesn't check if the command has already been defined.

As from version 1.25, the post-name hook also does
\begin{definition}[\DescribeMacro\glsextrapostnamehook]
\cs{glsextrapostnamehook}\marg{label}
\end{definition}
(before \cs{glsxtrpostname}\meta{category})
to allow for additional non-category related code. This does nothing
by default.

The post-description code used within the glossary is modified so
that it also does
\begin{definition}[\DescribeMacro\glsxtrpostdescription]
\cs{glsxtrpostdescription}
\end{definition}
This occurs before the original \cs{glspostdescription}, so if the
\pkgopt[false]{nopostdot} option is used, it will be inserted before
the terminating full stop.

This new command will do \cs{glsxtrpostdesc\meta{category}}
if it exists, where \meta{category} is the category label associated
with the current entry. For example \cs{glsxtrpostdescgeneral}
for entries with the category set to \category{general}
or \cs{glsxtrpostdescacronym} for entries with the category set to
\category{acronym}.
For convenience, as from v1.31, you can use
\begin{definition}[\DescribeMacro\glsdefpostdesc]
\cs{glsdefpostdesc}\marg{category}\marg{definition}
\end{definition}
This is simply a shortcut for:
\begin{alltt}
\cs{csdef}\{glsxtrpostdesc\meta{category}\}\marg{definition}
\end{alltt}
Note that it doesn't check if the command has already been defined.

Since both \cs{glossentry} and \cs{subglossentry} set
\begin{definition}[\DescribeMacro\glscurrententrylabel]
\cs{glscurrententrylabel}
\end{definition}
to the label for the current entry, you can use this within the
definition of these post-description hooks if you need to reference
the label.

For example, suppose you want to insert the plural form in brackets
after the description in the glossary, but only for entries in the
\category{general} category, then you could do:
\begin{verbatim}
\renewcommand{\glsxtrpostdescgeneral}{\space
 (plural: \glsentryplural{\glscurrententrylabel})}
\end{verbatim}
This means you don't have to define a custom glossary style, which
you may find more complicated. (It also allows more flexibility if
you decide to change the underlying glossary style.)

\begin{important}
This feature can't be used for glossary styles that ignore
\cs{glspostdescription} or if you redefine
\cs{glspostdescription} without including \cs{glsxtrpostdescription}.
(For example, if you redefine \cs{glspostdescription} to do
nothing instead of using the \pkgopt{nopostdot} option to suppress
the terminating full stop.) See \sectionref{sec:stylemods} to patch
the predefined styles provided by \styfmt{glossaries} that are missing 
\cs{glspostdescription}.
\end{important}

\subsubsection{Number List}
\label{sec:glosstylenumlist}

The \gls{numberlist} is now placed inside the argument of
\begin{definition}[\DescribeMacro\GlsXtrFormatLocationList]
\cs{GlsXtrFormatLocationList}\marg{number list}
\end{definition}
This is internally used by \cs{glossaryentrynumbers}. The
\pkgopt{nonumberlist} option redefines \cs{glossaryentrynumbers} so that it
doesn't display the \gls{numberlist}, but it still saves the
\gls{numberlist} in case it's required.

\begin{important}
If you want to suppress the \gls{numberlist} always use the
\pkgopt{nonumberlist} option instead of redefining
\cs{glossaryentrynumbers} to do nothing.
\end{important}

For example, to change the font for the entire
\gls{numberlist} redefine \cs{GlsXtrFormatLocationList} as
appropriate. Don't modify \cs{glossaryentrynumbers}.

Sometimes users like to insert \qt{page} or \qt{pages} in front of
the \gls{numberlist}. This is quite fiddly to do with the base
\styfmt{glossaries} package, but \styfmt{glossaries-extra}
provides a way of doing this. First you need to enable this
option and specify the text to display using:
\begin{definition}[\DescribeMacro\GlsXtrEnablePreLocationTag]
\cs{GlsXtrEnablePreLocationTag}\marg{page}\marg{pages}
\end{definition}
where \meta{page} is the text to display if the \gls{numberlist} only
contains a single location and \meta{pages} is the text to display
otherwise. For example:
\begin{verbatim}
\GlsXtrEnablePreLocationTag{Page: }{Pages: }
\end{verbatim}
An extra run is required when using this command.

\begin{important}
Use \texttt{glsignore} not \texttt{@gobble} as the format if you
want to suppress the page number (and only index the entry once).
\end{important}

See the accompanying sample file \texttt{sample-pages.tex}.

Note that \gls{bib2gls} can be instructed to insert
a prefix at the start of non-empty location lists, which
can be used as an alternative to \cs{GlsXtrEnablePreLocationTag}.

\Glspl{locationlist} displayed with \cs{printnoidxglossary}
internally use
\begin{definition}[\DescribeMacro\glsnoidxdisplayloc]
\cs{glsnoidxdisplayloc}\marg{prefix}\marg{counter}\marg{format}\marg{location}
\end{definition}
This command is provided by \styfmt{glossaries}, but is modified by
\styfmt{glossaries-extra} to check for the start and end range
formation identifiers \verb|(| and \verb|)| which are discarded to
obtain the actual control sequence name that forms the location
formatting command.

If the range identifiers aren't present, this just uses
\begin{definition}[\DescribeMacro\glsxtrdisplaysingleloc]
\cs{glsxtrdisplaysingleloc}\marg{format}\marg{location}
\end{definition}
otherwise it uses
\begin{definition}[\DescribeMacro\glsxtrdisplaystartloc]
\cs{glsxtrdisplaystartloc}\marg{format}\marg{location}
\end{definition}
for the start of a range (where the identifier has been stripped
from \meta{format}) or
\begin{definition}[\DescribeMacro\glsxtrdisplayendloc]
\cs{glsxtrdisplayendloc}\marg{format}\marg{location}
\end{definition}
for the end of a range (where the identifier has been stripped
from \meta{format}).

By default the start range command saves the format in
\begin{definition}[\DescribeMacro\glsxtrlocrangefmt]
\cs{glsxtrlocrangefmt}
\end{definition}
and does
\begin{display}
\cs{glsxtrdisplaysingleloc}\marg{format}\marg{location}
\end{display}
(If the format is empty, it will be replaced with 
\texttt{glsnumberformat}.)

The end command checks that the format matches the start of the
range, does
\begin{definition}[\DescribeMacro\glsxtrdisplayendlochook]
\cs{glsxtrdisplayendlochook}\marg{format}\marg{location}
\end{definition}
(which does nothing by default), followed by
\begin{display}
\cs{glsxtrdisplaysingleloc}\marg{format}\marg{location}
\end{display}
and then sets \cs{glsxtrlocrangefmt} to empty.

This means that the list
\begin{verbatim}
\glsnoidxdisplayloc{}{page}{(textbf}{1},
\glsnoidxdisplayloc{}{page}{textbf}{1},
\glsnoidxdisplayloc{}{page}{)textbf}{1}.
\end{verbatim}
doesn't display any differently from 
\begin{verbatim}
\glsnoidxdisplayloc{}{page}{textbf}{1},
\glsnoidxdisplayloc{}{page}{textbf}{1},
\glsnoidxdisplayloc{}{page}{textbf}{1}.
\end{verbatim}
but it does make it easier to define your own custom list handler
that can accommodate the ranges.

\subsubsection{The \styfmt{glossaries-extra-stylemods} Package}
\label{sec:stylemods}

As from v1.02, \styfmt{glossaries-extra} now includes the package
\sty{glossaries-extra-stylemods} that will redefine the predefined
styles to include the post-description hook (for those that are
missing it).  You will need to make sure the styles have already
been defined before loading \styfmt{glossaries-extra}. For example:
\begin{verbatim}
\usepackage{glossaries-extra}
\usepackage{glossary-longragged}
\usepackage{glossaries-extra-stylemods}
\end{verbatim}
Alternatively you can load
\texttt{glossary-}\meta{name}\texttt{.sty} at the same time by
passing \meta{name} as a package option to
\sty{glossaries-extra-stylemods}. For example:
\begin{verbatim}
\usepackage{glossaries-extra}
\usepackage[longragged]{glossaries-extra-stylemods}
\end{verbatim}
Another option is to use the \pkgopt{stylemods} key when you
load \styfmt{glossaries-extra}. You can omit a value if you only want to use
the predefined styles that are automatically loaded by
\styfmt{glossaries} (for example, the \glostyle{long3col} style):
\begin{verbatim}
\usepackage[style=long3col,stylemods]{glossaries-extra}
\end{verbatim}
Or the value of \pkgopt{stylemods} may be a comma-separated list
of the style package identifiers. For example:
\begin{verbatim}
\usepackage[style=mcoltree,stylemods=mcols]{glossaries-extra}
\end{verbatim}
Remember to group the value if it contains any commas:
\begin{verbatim}
\usepackage[stylemods={mcols,longbooktabs}]{glossaries-extra}
\end{verbatim}

Note that the \glostyle{inline} style is dealt with slightly
differently. The original definition provided by the
\sty{glossary-inline} package uses \cs{glspostdescription} at the
end of the glossary (not after each entry description) within the
definition of \cs{glspostinline}. The style modification changes
this so that \cs{glspostinline} just does a full stop followed by
space factor adjustment, and the description
\cs{glsinlinedescformat} and sub-entry description formats
\cs{glsinlinesubdescformat} are redefined to include
\cs{glsxtrpostdescription} (not \cs{glspostdescription}). This means
that the modified \glostyle{inline} style isn't affected by the
\pkgopt{nopostdot} option, but the post-description category hook
can still be used.

The \env{tabular}-like styles, such as \glostyle{long} are
adjusted so that the \ics{ifglsnogroupskip} conditional (set with
\pkgopt{nogroupskip}) is moved outside of the definition of
\cs{glsgroupskip} to avoid problems that cause an \qt{Incomplete
\cs{iftrue}} error with \cs{printunsrtglossary} and
\cs{printnoidxglossary}.  This means that if you want to change this
conditional using \cs{setupglossaries} or using the
\gloskey[printglossary]{nogroupskip} option in \cs{printglossary},
\cs{printnoidxglossary} or \cs{printunsrtglossary}, you must also
reset the glossary style.

As from version 1.21, the hard-coded \cs{space} before the
\gls{numberlist} in many of the predefined styles is replaced
with
\begin{definition}[\DescribeMacro\glsxtrprelocation]
\cs{glsxtrprelocation}
\end{definition}
This just defaults to \cs{space} but may be redefined as required.
For example:
\begin{verbatim}
\renewcommand{\glsxtrprelocation}{\dotfill}
\end{verbatim}

The \glostyle{list} styles use
\begin{definition}[\DescribeMacro\glslistprelocation]
\cs{glslistprelocation}
\end{definition}
(which defaults to \cs{glsxtrprelocation}) for top-level items and
\begin{definition}[\DescribeMacro\glslistchildprelocation]
\cs{glslistchildprelocation}
\end{definition}
(which defaults to \cs{glslistprelocation}) for child items.

As from v1.31, the description (including the post-description hook)
is governed by:
\begin{definition}[\DescribeMacro\glslistdesc]
\cs{glslistdesc}\marg{label}
\end{definition}
for the \glostyle{list} and \glostyle{altlist} styles (but not the
\glostyle{listdotted} variations).

For just the \glostyle{list} style and its letter group variations
(not the \glostyle{altlist} or \glostyle{listdotted} variations) the
\gls{numberlist} for child entries is followed by
\begin{definition}[\DescribeMacro\glslistchildpostlocation]
\cs{glslistchildpostlocation}
\end{definition}
which defaults to a full stop.

The default value of \ics{glslistdottedwidth} is changed so that
it's set at the start of the document (if it hasn't been changed in
the preamble). This should take into account situations where
\cs{hsize} isn't set until the start of the document.

The separator between groups (if not \pkgopt{nogroupskip}) is now
given by:
\begin{definition}[\DescribeMacro\glslistgroupskip]
\cs{glslistgroupskip}
\end{definition}
This defaults to \cs{indexspace} with penalties to deter page
breaks.
This command isn't used if \pkgopt{nogroupskip} is set.

The \sty{glossary-tree} package introduced new commands in
v4.22, \ics{glstreegroupheaderfmt}
and \ics{glstreenavigationfmt}, which are used to format
the letter group headings and the navigation elements for the
appropriate styles. These two new commands are defined in terms of
\ics{glstreenamefmt} since that was the command originally used
for the group headings and navigation. This now allows these
different elements to be defined independently, but the most common
redefinition is for \cs{glstreenamefmt} to remove the bold in the
name. If the bold is still required for the group heading and navigation
elements, then both other commands also need redefining. To simplify
matters, if \cs{glstreenamefmt} has been defined, as from v1.31
\sty{glossaries-extra-stylemods} defines:
\begin{definition}[\DescribeMacro\glstreedefaultnamefmt]
\cs{glstreedefaultnamefmt}\marg{text}
\end{definition}
which simply does \cs{textbf}\marg{text} and redefines
\cs{glstreenamefmt}, \cs{glstreegroupheaderfmt} and
\cs{glstreenavigationfmt} all in terms of
\cs{glstreedefaultnamefmt}. 

This means that if you want to change
all three to use a particular style you only need to redefine
\cs{glstreedefaultnamefmt}, but if you only want to redefine
\cs{glstreenamefmt} without affecting the other two commands, then
you now can.

The separator between groups without headers is given by:
\begin{definition}[\DescribeMacro\glstreegroupskip]
\cs{glstreegroupskip}
\end{definition}
This defaults to just \ics{indexspace} without penalties. This
command isn't used if \pkgopt{nogroupskip} is set. (The penalties
introduced in v1.41 were moved to \cs{glstreeheadergroupskip} in
v1.42 as they are inappropriate when there's no header.)

The separator between groups with headers is now given by (as from v1.42):
\begin{definition}[\DescribeMacro\glstreeheadergroupskip]
\cs{glstreegroupheaderskip}
\end{definition}
This defaults to \cs{glstreegroupskip} with penalties to deter page
breaks after the group heading.

The styles that display the group titles now use:
\begin{definition}[\DescribeMacro\glstreePreHeader]
\cs{glstreePreHeader}\marg{label}\marg{title}
\end{definition}
where \meta{label} is the group label and \meta{title} is the group
title. This does nothing by default
and is inserted before the group title. You can redefine it to add
the group title to the PDF bookmarks. For example, if the
glossary title uses \cs{chapter} then:
\begin{verbatim}
\renewcommand{\glstreePreHeader}[2]{%
 \pdfbookmark[1]{#2}{\currentglossary.#1}%
}
\end{verbatim}
will insert section-level bookmarks. The use of \cs{currentglossary}
helps to provide unique bookmark labels in the event of multiple
glossaries.

The \glostyle{index}-like and \glostyle{tree}-like styles
insert the pre-\gls{numberlist} space with
\begin{definition}[\DescribeMacro\glstreeprelocation]
\cs{glstreeprelocation}
\end{definition}
(which defaults to \cs{glsxtrprelocation}) for top-level items and
\begin{definition}[\DescribeMacro\glstreechildprelocation]
\cs{glstreechildprelocation}
\end{definition}
(which defaults to \cs{glstreeprelocation}) for child items.

As from version 1.31, the \styfmt{glossaries-extra-stylemods}
package also provides:
\begin{definition}[\DescribeMacro\glstreenonamedesc]
\cs{glstreenonamedesc}\marg{label}
\end{definition}
which is used by the \glostyle{treenoname} styles to display the
pre-description separator, the description and the post-description
hook. Similarly for the symbol:
\begin{definition}[\DescribeMacro\glstreenonamesymbol]
\cs{glstreenonamesymbol}\marg{label}
\end{definition}
The above are just used for top-level entries. Child entries don't
have the name or symbol displayed for the \glostyle{treenoname}
styles, so there's only a command for the child description:
\begin{definition}[\DescribeMacro\glstreenonamechilddesc]
\cs{glstreenonamechilddesc}\marg{label}
\end{definition}

For the \glostyle{tree} styles (but not the \glostyle{treenoname} or
\glostyle{alttree} styles), the description is displayed using:
\begin{definition}[\DescribeMacro\glstreedesc]
\cs{glstreedesc}\marg{label}
\end{definition}
and the symbol with:
\begin{definition}[\DescribeMacro\glstreesymbol]
\cs{glstreesymbol}\marg{label}
\end{definition}
Again the above two commands are just for top-level entries. The
child entries use:
\begin{definition}[\DescribeMacro\glstreechilddesc]
\cs{glstreechilddesc}\marg{label}
\end{definition}
for the description and
\begin{definition}[\DescribeMacro\glstreechildsymbol]
\cs{glstreechildsymbol}\marg{label}
\end{definition}
for the symbol.
As from version 1.41, there are now wrapper commands for
\cs{glstreedesc} and \cs{glstreechilddesc} that check for
the description and symbol to determine what separator to use
before the page list. These are:
\begin{definition}[\DescribeMacro\glstreeDescLoc]
\cs{glstreeDescLoc}\marg{label}\marg{location}
\end{definition}
for top-level entries and
\begin{definition}[\DescribeMacro\glstreeChildDescLoc]
\cs{glstreeChildDescLoc}\marg{label}\marg{location}
\end{definition}
for sub-entries. 

If either the symbol or description is present these will use 
\cs{glstreeprelocation} or \cs{glstreechildprelocation},
respectively. Otherwise, both will use (from v1.42):
\begin{definition}[\DescribeMacro\glstreeNoDescSymbolPreLocation]
\cs{glstreeNoDescSymbolPreLocation}
\end{definition}
The default is a space. This means that you could have, say, a
comma followed by a space for terms that are simply an alias, but
just have a space for terms that have a description that ends with a
full stop (or that just have a symbol without a description) where
the comma would be inappropriate.

Note that version 1.42 has corrected an error that was introduced to
v1.41 that caused the name to run into the location list if there
was no symbol and no description.

As from version 1.05, the \styfmt{glossaries-extra-stylemods}
package provides some additional commands for use with the
\glostyle{alttree} style to make it easier to modify.
These commands are only defined if the
\sty{glossary-tree} package has already been loaded, which is
typically the case unless the \pkgopt{notree} option has been used
when loading \styfmt{glossaries}.

\begin{definition}[\DescribeMacro\gglssetwidest]
\cs{gglssetwidest}\oarg{level}\marg{name}
\end{definition}
(New to version 1.21.)
This is like \cs{glssetwidest} (provided by \sty{glossary-tree})
but performs a global assignment.

\begin{definition}[\DescribeMacro\eglssetwidest]
\cs{eglssetwidest}\oarg{level}\marg{name}
\end{definition}
This is like \cs{glssetwidest} 
but performs a protected expansion on \meta{name}. This has
a localised effect. For a global setting, use
\begin{definition}[\DescribeMacro\xglssetwidest]
\cs{xglssetwidest}\oarg{level}\marg{name}
\end{definition}
The following only set the value if \meta{name} is wider than the
current value (new to version 1.23). Local update:
\begin{definition}[\DescribeMacro\glsupdatewidest]
\cs{glsupdatewidest}\oarg{level}\marg{name}
\end{definition}
Global update:
\begin{definition}[\DescribeMacro\gglsupdatewidest]
\cs{gglsupdatewidest}\oarg{level}\marg{name}
\end{definition}
Locale update (expands \meta{name}):
\begin{definition}[\DescribeMacro\eglsupdatewidest]
\cs{eglsupdatewidest}\oarg{level}\marg{name}
\end{definition}
Global update (expands \meta{name}):
\begin{definition}[\DescribeMacro\xglsupdatewidest]
\cs{xglsupdatewidest}\oarg{level}\marg{name}
\end{definition}


The widest entry value can later be retrieved using
\begin{definition}[\DescribeMacro\glsgetwidestname]
\cs{glsgetwidestname}
\end{definition}
for the top-level entries and
\begin{definition}[\DescribeMacro\glsgetwidestsubname]
\cs{glsgetwidestsubname}\marg{level}
\end{definition}
for sub-entries, where \meta{level} is the level number.

Note that if you are using \gls{bib2gls}, you can use the
resource option \texttt{set-widest} which will try to determine the
widest name of all the selected entries. This isn't guaranteed
to work as it may depend on fonts or commands that \gls{bib2gls}
can't replicate, but it should be suitable for names that just
consist of text, and can be more efficient than iterating over all
the defined entries using \TeX. 

The command \cs{glsfindwidesttoplevelname} provided by
\sty{glossary-tree} has a CamelCase synonym:
\begin{definition}[\DescribeMacro\glsFindWidestTopLevelName]
\cs{glsFindWidestTopLevelName}\oarg{glossary list}
\end{definition}
Similar commands are also provided:
\begin{definition}[\DescribeMacro\glsFindWidestUsedTopLevelName]
\cs{glsFindWidestUsedTopLevelName}\oarg{glossary list}
\end{definition}
This has an additional check that the entry has been used.
Naturally this is only useful if the glossaries that use
the \glostyle{alttree} style occur at the end of the document.
This command should be placed just before the start of the glossary.
(Alternatively, place it at the end of the document and save
the value in the auxiliary file for the next run.)

\begin{definition}[\DescribeMacro\glsFindWidestUsedAnyName]
\cs{glsFindWidestUsedAnyName}\oarg{glossary list}
\end{definition}
This is like the previous command but if doesn't check the
\gloskey{parent} key. This is useful if all levels should have the
same width for the name.

\begin{definition}[\DescribeMacro\glsFindWidestAnyName]
\cs{glsFindWidestAnyName}\oarg{glossary list}
\end{definition}
This is like the previous command but doesn't check if the entry
has been used.

\begin{definition}[\DescribeMacro\glsFindWidestUsedLevelTwo]
\cs{glsFindWidestUsedLevelTwo}\oarg{glossary list}
\end{definition}
This is like \cs{glsFindWidestUsedTopLevelName} but also sets
the first two sub-levels as well. Any entry that has a
great-grandparent is ignored.

\begin{definition}[\DescribeMacro\glsFindWidestLevelTwo]
\cs{glsFindWidestLevelTwo}\oarg{glossary list}
\end{definition}
This is like the previous command but doesn't check if the entry has
been used.

\begin{definition}[\DescribeMacro\glsFindWidestUsedAnyNameSymbol]
\cs{glsFindWidestUsedAnyNameSymbol}\oarg{glossary
list}\marg{register}
\end{definition}
This is like \cs{glsFindWidestUsedAnyName} but also measures the
symbol. The length of the widest symbol is stored in
\meta{register}.

\begin{definition}[\DescribeMacro\glsFindWidestAnyNameSymbol]
\cs{glsFindWidestAnyNameSymbol}\oarg{glossary list}\marg{register}
\end{definition}
This is like the previous command but it doesn't check if the entry
has been used.

\begin{definition}[\DescribeMacro\glsFindWidestUsedAnyNameSymbolLocation]
\cs{glsFindWidestUsedAnyNameSymbolLocation}\oarg{glossary
list}\marg{symbol register}\marg{location register}
\end{definition}
This is like \cs{glsFindWidestUsedAnyNameSymbol} but also
measures the \gls{numberlist}. This requires
\ics{glsentrynumberlist} (see the \styfmt{glossaries} user manual).
The length of the widest symbol is stored in \meta{symbol register}
and the length of the widest \gls*{numberlist} is stored in
\meta{location register}.

\begin{definition}[\DescribeMacro\glsFindWidestAnyNameSymbolLocation]
\cs{glsFindWidestAnyNameSymbolLocation}\oarg{glossary
list}\marg{symbol register}\marg{location register}
\end{definition}
This is like the previous command but it doesn't check if the entry
has been used.

\begin{definition}[\DescribeMacro\glsFindWidestUsedAnyNameLocation]
\cs{glsFindWidestUsedAnyNameLocation}\oarg{glossary
list}\marg{register}
\end{definition}
This is like \cs{glsFindWidestUsedAnyNameSymbolLocation} but doesn't
measure the symbol. The length of the widest \gls{numberlist}
is stored in \meta{register}.

\begin{definition}[\DescribeMacro\glsFindWidestAnyNameLocation]
\cs{glsFindWidestAnyNameLocation}\oarg{glossary
list}\marg{register}
\end{definition}
This is like the previous command but doesn't check if the entry has
been used.

The layout of the symbol, description and \gls{numberlist}
is governed by
\begin{definition}[\DescribeMacro\glsxtralttreeSymbolDescLocation]
\cs{glsxtralttreeSymbolDescLocation}\marg{label}\marg{number list}
\end{definition}
for top-level entries and
\begin{definition}[\DescribeMacro\glsxtralttreeSubSymbolDescLocation]
\cs{glsxtralttreeSubSymbolDescLocation}\marg{label}\marg{number list}
\end{definition}
for sub-entries.

There is now a user level command that performs the initialisation
for the \glostyle{alttree} style:
\begin{definition}[\DescribeMacro\glsxtralttreeInit]
\cs{glsxtralttreeInit}
\end{definition}

The paragraph indent for subsequent paragraphs in multi-paragraph
descriptions is provided by the length
\begin{definition}[\DescribeMacro\glsxtrAltTreeIndent]
\cs{glsxtrAltTreeIndent}
\end{definition}

For additional commands that are available with the 
\glostyle{alttree} style, see the documented code
(\nolinkurl{glossaries-extra-code.pdf}). See also
the accompanying sample files \nolinkurl{sample-alttree.tex},
\nolinkurl{sample-alttree-sym.tex} and
\nolinkurl{sample-alttree-marginpar.tex}.

\chapter{New Glossary Styles}

The \styfmt{glossaries-extra} package comes with some new styles.
The associated style package needs to be loaded. This can be done
with \cs{usepackage} but it's simpler to use the \pkgopt{stylemods}
option.

\section{\styfmt{glossary-bookindex} package}
\label{sec:bookindex}

As from v1.21, \styfmt{glossaries-extra} has a new supplementary
package \sty{glossary-bookindex} which provides the glossary style
\glostyle{bookindex}. This is very similar to the
\glostyle{mcolindexgroup} style but is designed for indexes, so by
default only the name and location list are displayed.  You can
either load this package explicitly and then set the style:
\begin{verbatim}
\usepackage{glossaries-extra}
\usepackage{glossary-bookindex}
\setglossarystyle{bookindex}
\end{verbatim}
or use both the \pkgopt{stylemods} and \sty{style} options:
\begin{verbatim}
\usepackage[stylemods=bookindex,style=bookindex]{glossaries-extra}
\end{verbatim}
The \glostyle{bookindex} style only supports a maximum
hierarchical level of 2 (top-level, level~1 and level~2). It's
primarily designed for use with \gls{bib2gls}. It may be used with
other indexing options, but some features may not be present and
UTF-8 characters may cause a problem with non-Unicode engines in
letter group headings or PDF bookmarks. (\gls{bib2gls} uses numeric
identifies by default to avoid these problems. If numbers show up in
the group headings instead of the appropriate characters, check that
you have used the \pkgopt{record} package option.)

The number of columns is given by
\begin{definition}[\DescribeMacro\glsxtrbookindexcols]
\cs{glsxtrbookindexcols}
\end{definition}
which defaults to 2.

This style uses the \env{multicols} environment. If
the command
\begin{definition}[\DescribeMacro\glsxtrbookindexcolspread]
\cs{glsxtrbookindexcolspread}
\end{definition}
isn't empty then it's supplied as the optional argument
following \verb|\begin{multicols}|\marg{n}. You can switch from
\env{multicols} to \env{multicols*} by redefining
\begin{definition}[\DescribeMacro\glsxtrbookindexmulticolsenv]
\cs{glsxtrbookindexmulticolsenv}
\end{definition}
For example
\begin{verbatim}
\renewcommand{\glsxtrbookindexmulticolsenv}{multicols*}
\end{verbatim}

Each top-level entry is displayed using
\begin{definition}[\DescribeMacro\glsxtrbookindexname]
\cs{glsxtrbookindexname}\marg{label}
\end{definition}
where the entry is identified by \meta{label}. This just
does \cs{glossentryname}\marg{label} by default.
For example, if you want the symbol to be included:
\begin{verbatim}
\renewcommand*{\glsxtrbookindexname}[1]{%
 \glossentryname{#1}%
 \ifglshassymbol{#1}{\space (\glossentrysymbol{#1})}{}%
}
\end{verbatim}
or if you want the description (if set):
\begin{verbatim}
\renewcommand*{\glsxtrbookindexname}[1]{%
 \glossentryname{#1}%
 \ifglshasdesc{#1}{\space \glossentrydesc{#1}\glspostdescription}{}%
}
\end{verbatim}
(which picks up the post-description hook).

Alternatively you can use the \cs{glsxtrpostname\meta{category}}
hook to append information after the name according to the entry's
category.

Sub-entries are displayed using
\begin{definition}[\DescribeMacro\glsxtrbookindexsubname]
\cs{glsxtrbookindexsubname}\marg{label}
\end{definition}
which just defaults to \cs{glsxtrbookindexname}\marg{label}.

The separator used before the location list for top-level
entries is given by
\begin{definition}[\DescribeMacro\glsxtrbookindexprelocation]
\cs{glsxtrbookindexprelocation}\marg{label}
\end{definition}
where \meta{label} is the entry's label. This checks if
the \texttt{location} field has been set. If it has, it
does
\begin{verbatim}
,\glsxtrprelocation
\end{verbatim}
otherwise it just does \cs{glsxtrprelocation} (which defaults
to \cs{space}).  If you're not using 
\gls{bib2gls}, the \gloskey{location} field won't be set.

The separator used before the
location list for sub-entries is given by
\begin{definition}[\DescribeMacro\glsxtrbookindexsubprelocation]
\cs{glsxtrbookindexsubprelocation}\marg{label}
\end{definition}
which defaults to \cs{glsxtrbookindexprelocation}\marg{label}.

The actual location list is encapsulated with:
\begin{definition}[\DescribeMacro\glsxtrbookindexlocation]
\cs{glsxtrbookindexlocation}\marg{label}\marg{location list}
\end{definition}
for top-level entries and
\begin{definition}[\DescribeMacro\glsxtrbookindexsublocation]
\cs{glsxtrbookindexsublocation}\marg{label}\marg{location list}
\end{definition}
for sub-entries. These both just do \meta{location list} by default.

The separator used between a top-level parent and child entry is
given by
\begin{definition}[\DescribeMacro\glsxtrbookindexparentchildsep]
\cs{glsxtrbookindexparentchildsep}
\end{definition}
This defaults to \cs{nopagebreak}.

The separator used between a sub-level parent and child entry is
given by
\begin{definition}[\DescribeMacro\glsxtrbookindexparentsubchildsep]
\cs{glsxtrbookindexparentsubchildsep}
\end{definition}
This defaults to \cs{glsxtrbookindexparentchildsep}.

The separator between top-level entries is given by
\begin{definition}[\DescribeMacro\glsxtrbookindexbetween]
\cs{glsxtrbookindexbetween}\marg{label1}\marg{label2}
\end{definition}
This comes after the entry given by \meta{label1}, if the entry
has no children, or after the last descendent otherwise,
so it always comes immediately before the entry given
by \meta{label2} unless the entry occurs at the start of
a group. This does nothing by default.

The separator between two level~1 entries is given by
\begin{definition}[\DescribeMacro\glsxtrbookindexsubbetween]
\cs{glsxtrbookindexsubbetween}\marg{label1}\marg{label2}
\end{definition}

The separator between two level~2 entries is given by
\begin{definition}[\DescribeMacro\glsxtrbookindexsubsubbetween]
\cs{glsxtrbookindexsubsubbetween}\marg{label1}\marg{label2}
\end{definition}
At the end of each letter group, the following hooks
are done in order:
\begin{definition}[\DescribeMacro\glsxtrbookindexsubsubatendgroup]
\cs{glsxtrbookindexsubsubatendgroup}\marg{sub-sub-label}
\end{definition}
\begin{definition}[\DescribeMacro\glsxtrbookindexsubatendgroup]
\cs{glsxtrbookindexsubatendgroup}\marg{sub-label}
\end{definition}
\begin{definition}[\DescribeMacro\glsxtrbookindexatendgroup]
\cs{glsxtrbookindexatendgroup}\marg{label}
\end{definition}
where \meta{sub-sub-label} is the label of the last
level~2 entry, \meta{sub-label} is the label of the last
level~1 entry and \meta{label} is the label of the last
level~0 entry.

For example, the resource option \texttt{seealso=omit}
instructs \gls{bib2gls} to omit the \texttt{seealso} cross-reference
from the location list. (The \texttt{see} cross-reference
will still be added unless you also have \texttt{see=omit}.)
The \texttt{seealso} cross-reference can instead be appended after
the child entries using:
\begin{verbatim}
\renewcommand{\glsxtrbookindexatendgroup}[1]{%
  \glsxtrifhasfield{seealso}{#1}{\glstreesubitem\glsxtruseseealso{#1}}{}%
}
\renewcommand{\glsxtrbookindexbetween}[2]{%
  \glsxtrbookindexatendgroup{#1}%
}%

\renewcommand{\glsxtrbookindexsubatendgroup}[1]{%
  \glsxtrifhasfield{seealso}{#1}{\glstreesubsubitem\glsxtruseseealso{#1}}{}%
}

\renewcommand{\glsxtrbookindexsubbetween}[2]{%
  \glsxtrbookindexsubatendgroup{#1}%
}

\renewcommand{\glsxtrbookindexsubsubatendgroup}[1]{%
  \glsxtrifhasfield{seealso}{#1}%
  {\glstreeitem\hspace*{40pt}\glsxtruseseealso{#1}}{}%
}

\renewcommand{\glsxtrbookindexsubsubbetween}[2]{%
  \glsxtrbookindexsubsubatendgroup{#1}%
}
\end{verbatim}
This uses \cs{glstreesubitem} and \cs{glstreesubsubitem} 
to indent the cross-reference according to the next level down,
so the cross-reference for a top-level entry is aligned with
the sub-entries, and a level~1 entry has its cross-reference
aligned with sub-sub-entries. In the event that a level~2
entry has a cross-reference, this is indented a bit further
(but it won't be aligned with any deeper level as the
\glostyle{bookindex} style only supports a maximum of two
sub-levels).

The \glostyle{bookindex} style uses group headings. (If you
use \gls{bib2gls} remember to invoke it with the \verb|--group|
or \verb|-g| switch.) The heading will use
\begin{definition}[\DescribeMacro\glsxtrbookindexbookmark]
\cs{glsxtrbookindexbookmark}\marg{group title}\marg{name}
\end{definition}
If \cs{pdfbookmark} has been defined, this will 
use that command to bookmark the group title. If
\pkgopt[chapter]{section} is set (default if chapters are defined)
then this uses level~1 otherwise it uses level~2. You can
redefine this command if this isn't appropriate.
If \cs{pdfbookmark} hasn't been defined, this command does nothin.

The group heading is formatted according to
\begin{definition}[\DescribeMacro\glsxtrbookindexformatheader]
\cs{glsxtrbookindexformatheader}\marg{group title}
\end{definition}
which is defined as
\begin{verbatim}
\newcommand*{\glsxtrbookindexformatheader}[1]{%
  \par{\centering\glstreegroupheaderfmt{#1}\par}%
}
\end{verbatim}
where \cs{glstreegroupheaderfmt} is provided by the
\sty{glossary-tree} package, which is automatically loaded.
Note that the entry names aren't encapsulated with
\cs{glstreenamefmt}.

The \sty{glossary-bookindex} package provides some supplementary
commands that aren't used by default, but may be used when
adjusting the style. These commands should only be
used within one of the \cs{print\ldots glossary} commands.
(That is, they should only be used in glossary styles.)
\begin{definition}[\DescribeMacro\glsxtrbookindexmarkentry]
\cs{glsxtrbookindexmarkentry}\marg{label}
\end{definition}
This writes information to the \texttt{.aux} file that
can be read on the next run to obtain the first and last entry on
each page of the glossary.

You can display the first entry
associated with the current page using:
\begin{definition}[\DescribeMacro\glsxtrbookindexfirstmark]
\cs{glsxtrbookindexfirstmark}
\end{definition}
and the last entry associated with the current page
using:
\begin{definition}[\DescribeMacro\glsxtrbookindexlastmark]
\cs{glsxtrbookindexlastmark}
\end{definition}
These do nothing if there are no entries marked on the current page
(or if the document build isn't up to date).

The entry is formatted using:
\begin{definition}[\DescribeMacro\glsxtrbookindexfirstmarkfmt]
\cs{glsxtrbookindexfirstmarkfmt}\marg{label}
\end{definition}
for the first instance and
\begin{definition}[\DescribeMacro\glsxtrbookindexlastmarkfmt]
\cs{glsxtrbookindexlastmarkfmt}\marg{label}
\end{definition}
for the last.

These commands are designed for use in page headers or footers
where the page number is stable. For example, \cs{glsxtrbookindexname}
can be redefined to mark the current entry:
\begin{verbatim}
\renewcommand{\glsxtrbookindexname}[1]{%
  \glsxtrbookindexmarkentry{#1}%
  \glossentryname{#1}%
}
\end{verbatim}
If you only want to mark the top-level entries, remember
to redefine \cs{glsxtrbookindexsubname} as it defaults
to \cs{glsxtrbookindexname}:
\begin{verbatim}
\renewcommand{\glsxtrbookindexsubname}[1]{%
  \glossentryname{#1}%
}
\end{verbatim}

Then if you're using \sty{fancyhdr} you can set the page style
to show the first and last entry for the current page with:
\begin{verbatim}
  \pagestyle{fancy}%
  \lhead{\thepage}%
  \lfoot{\glsxtrbookindexfirstmark}%
  \cfoot{}%
  \rfoot{\glsxtrbookindexlastmark}%
\end{verbatim}

\section{\styfmt{glossary-longextra} package}
\label{sec:longextra}

As from version 1.37, the \styfmt{glossaries-extra} package comes
with the supplementary package \sty{glossary-longextra} that
provides additional styles, listed below, that use the
\env{longtable} environment. If you know that your glossary won't
span more than a page and you need to use it in a context that's
incompatible with \env{longtable}, you can instead setup these styles
to use \env{tabular} instead. In order to do this you must use
\begin{definition}[\DescribeMacro\GlsLongExtraUseTabulartrue]
\cs{GlsLongExtraUseTabulartrue}
\end{definition}
\emph{before the style is set}. For example:
\begin{verbatim}
\GlsLongExtraUseTabulartrue
\setglossarystyle{long-name-desc}
\end{verbatim}
or
\begin{verbatim}
\GlsLongExtraUseTabulartrue
\printunsrtglossary[style={long-name-desc}]
\end{verbatim}
If you use this setting, you can change the default vertical
alignment with:
\begin{definition}[\DescribeMacro\glslongextraTabularVAlign]
\cs{glslongextraTabularVAlign}
\end{definition}
The default definition is \texttt{c}.

The column titles are formatted according to:
\begin{definition}[\DescribeMacro\glslongextraHeaderFmt]
\cs{glslongextraHeaderFmt}\marg{text}
\end{definition}
which simply does \cs{textbf}\marg{text} by default.

The name column has the title given by \ics{entryname} and
the column alignment is given by:
\begin{definition}[\DescribeMacro\glslongextraNameAlign]
\cs{glslongextraNameAlign}
\end{definition}
which expands to \texttt{l} (left) by default.

The symbol column (where applicable) has the title given by
\ics{symbolname} and the column alignment is given by:
\begin{definition}[\DescribeMacro\glslongextraSymbolAlign]
\cs{glslongextraSymbolAlign}
\end{definition}
which expands to \texttt{c} (centred) by default.

The location list column (where applicable) has the title given by
\ics{pagelistname} and the column alignment is given by:
\begin{definition}[\DescribeMacro\glslongextraLocationAlign]
\cs{glslongextraLocationAlign}
\end{definition}
which expands to
\begin{verbatim}
>{\raggedright}p{\glspagelistwidth}
\end{verbatim}
by default. (Ragged-right paragraph, \ics{glspagelistwidth} is
defined in \sty{glossary-long}, which is automatically loaded.)

The description column has the title given by \ics{descriptionname}
and the column alignment is given by:
\begin{definition}[\DescribeMacro\glslongextraDescAlign]
\cs{glslongextraDescAlign}
\end{definition}
which expands to
\begin{verbatim}
>{\raggedright}p{\glsdescwidth}
\end{verbatim}
by default. (Ragged-right paragraph, \ics{glsdescwidth} is
defined in \sty{glossary-long}, which is automatically loaded.)

Unlike the long styles provided by the base \styfmt{glossaries}
package, these new styles try to determine the value of
\cs{glsdescwidth} at the start of the glossary according to the 
number of columns provided by the style. The calculations are 
performed by the following commands:
\begin{definition}[\DescribeMacro\glslongextraSetDescWidth]
\cs{glslongextraSetDescWidth}
\end{definition}
This is used by the styles that have a name and description
column. The value of \cs{glsdescwidth} is computed as:
\[
  \cs{glsdescwidth} = \cs{linewidth} - 4\cs{tabcolsep} - W
\]
where $W$ is a guess at the width of the name column. This is first
set to the width of the name column header:
\begin{alltt}
\cs{settowidth}\marg{width}\{\cs{glslongextraHeaderFmt}\cs{entryname}\}
\end{alltt}
If any names in that column are larger than this, then you need
to specify the widest name using:
\begin{definition}[\DescribeMacro\glslongextraSetWidest]
\cs{glslongextraSetWidest}\marg{widest name}
\end{definition}
or
\begin{definition}[\DescribeMacro\glslongextraUpdateWidest]
\cs{glslongextraUpdateWidest}\marg{text}
\end{definition}
These work like the analogous commands \cs{glssetwidest} and
\cs{glsupdatewidest} provided with the \glostyle{alttree} style, but
in this case there's no hierarchy. The default widest name
is obtained from the \glostyle{alttree} top-level name if set,
otherwise it's empty, so you can use \gls{bib2gls}['s]
\texttt{set-widest} option. If you have the entry counter enabled,
you will need to include this with the name for the extra material to
be taken into account.

The name isn't shown for child entries by default, but if
you change this and you want to use \gls{bib2gls}['s]
\texttt{set-widest} option (for \gls{bib2gls} v1.8+) then you need to
redefine:
\begin{definition}[\DescribeMacro\glslongextraUpdateWidestChild]
\cs{glslongextraUpdateWidestChild}\marg{level}\marg{text}
\end{definition}
This does nothing by default, but if you are including the child
names then you need to redefine this command:
\begin{verbatim}
\renewcommand{\glslongextraUpdateWidestChild}[2]{%
 \glslongextraUpdateWidest{#2}%
}
\end{verbatim}

If you prefer to set an explicit width for the description
column then you need to redefine \cs{glslongextraSetDescWidth}. For
example:
\begin{verbatim}
\renewcommand{\glslongextraSetDescWidth}{%
  \setlength{\glsdescwidth}{0.6\linewidth}%
}
\end{verbatim}

The styles that have a name, symbol and description,
\cs{glsdescwidth} is set with:
\begin{definition}[\DescribeMacro\glslongextraSymSetDescWidth]
\cs{glslongextraSymSetDescWidth}
\end{definition}
This first uses \cs{glslongextraSetDescWidth} and then subtracts
2\cs{tabcolsep} and the width of the symbol column header from
\cs{glsdescwidth}. This assumes that the symbol column header
is larger than any of the symbols. If this isn't appropriate
then you can redefine this command. For example:
\begin{verbatim}
\renewcommand{\glslongextraSymSetDescWidth}{%
  \glslongextraSetDescWidth
  \addtolength{\glsdescwidth}{-3cm}%
}
\end{verbatim}
or
\begin{verbatim}
\renewcommand{\glslongextraSymSetDescWidth}{%
  \setlength{\glsdescwidth}{.5\linewidth}%
}
\end{verbatim}

For the styles that have a name, description and location column,
\cs{glsdescwidth} is set using:
\begin{definition}[\DescribeMacro\glslongextraLocSetDescWidth]
\cs{glslongextraLocSetDescWidth}
\end{definition}
This uses \cs{glslongextraSetDescWidth} and then subtracts
2\cs{tabcolsep} and \cs{glspagelistwidth} from \cs{glsdescwidth}.
You can redefine this command to set both \cs{glsdescwidth} and
\cs{glspagelistwidth} if appropriate.

For the styles that have a name, description, symbol and location
column, \cs{glsdescwidth} is set using:
\begin{definition}[\DescribeMacro\glslongextraSymLocSetDescWidth]
\cs{glslongextraSymLocSetDescWidth}
\end{definition}
This uses \cs{glslongextraSymSetDescWidth} and then subtracts
2\cs{tabcolsep} and \cs{glspagelistwidth} from
\cs{glsdescwidth}. Again, you can redefine this command
to explicitly set both lengths.

In all cases, the top-level name is formatted according to:
\begin{definition}[\DescribeMacro\glslongextraNameFmt]
\cs{glslongextraNameFmt}\marg{label}
\end{definition}
This does 
\begin{alltt}
\cs{glsentryitem}\marg{label}\cs{glstarget}\marg{label}\{\cs{glossentryname}\marg{label}\}
\end{alltt}
which includes the entry counter (if enabled), the target and the post-name link.

The top-level description is formatted according to:
\begin{definition}[\DescribeMacro\glslongextraDescFmt]
\cs{glslongextraDescFmt}\marg{label}
\end{definition}
This does \cs{glossentrydesc}\marg{label} followed by the
post-description hook.

The styles that have a symbol column format the symbol using:
\begin{definition}[\DescribeMacro\glslongextraSymbolFmt]
\cs{glslongextraSymbolFmt}\marg{label}
\end{definition}
This just does \cs{glossentrysymbol}\marg{label}.

The styles that have a location list column format the list using:
\begin{definition}[\DescribeMacro\glslongextraLocationFmt]
\cs{glslongextraLocationFmt}\marg{label}\marg{locations}
\end{definition}
This just does \meta{locations} and ignores the label.

The child entries have their name formatted according to:
\begin{definition}[\DescribeMacro\glslongextraSubNameFmt]
\cs{glslongextraSubNameFmt}\marg{level}\marg{label}
\end{definition}
where \meta{level} is the hierarchical level. This defaults to:
\begin{alltt}
\cs{glssubentryitem}\marg{label}\cs{glstarget}\marg{label}\{\cs{strut}\}
\end{alltt}
This includes the sub-entry counter (if enabled) and the target
but doesn't show the name. The child description is formatted according to:
\begin{definition}[\DescribeMacro\glslongextraSubDescFmt]
\cs{glslongextraSubDescFmt}\marg{level}\marg{label}
\end{definition}
which defaults to just \cs{glslongextraDescFmt}\marg{label}

The child symbol is formatted (where appropriate) according to:
\begin{definition}[\DescribeMacro\glslongextraSubSymbolFmt]
\cs{glslongextraSubSymbolFmt}\marg{level}\marg{label}
\end{definition}
This just does \cs{glslongextraSymbolFmt}\marg{label} by default.

The styles that have a location list column format the list for
child entries using:
\begin{definition}[\DescribeMacro\glslongextraLocationFmt]
\cs{glslongextraSubLocationFmt}\meta{level}\marg{label}\marg{locations}
\end{definition}
This just does \meta{locations} and ignores the level and label.

The letter group headings are formatted according to:
\begin{definition}[\DescribeMacro\glslongextraGroupHeading]
\cs{glslongextraGroupHeading}\marg{n}\marg{label}
\end{definition}
which does nothing by default. The first argument \meta{n} is the
number of columns in the table. The second argument \meta{label}
is the group label (not the title, although they may happen
to be the same).

This can be redefined to show the group heading. For example:
\begin{verbatim}
\renewcommand*{\glslongextraGroupHeading}[2]{%
  \glsxtrgetgrouptitle{#2}{\thisgrptitle}%
  \glslongextraHeaderFmt{\thisgrptitle}%
  \tabularnewline
  \noalign{\vskip\normalbaselineskip}%
}
\end{verbatim}
This ignores the first argument and just puts the group
title in the first column formatted according to
\cs{glslongextraHeaderFmt} (to match the column header).

Remember that you can also adjust the styles through category
attributes. The name column's title is given by
\cs{entryname}, the description column's title is given by
\cs{descriptionname} and (where present) the symbol column's 
title is given by \cs{symbolname}, as for the other long styles
that have headers.

\begin{description}
\item{\glostyle{long-name-desc}} This is like the
\glostyle{longragged-booktabs} style but doesn't show the location
list (regardless of the \pkgopt{nonumberlist} option). The name is
shown in the first column and the description in the second.

The symbol is not displayed. The header
row is produced with:
\begin{definition}[\DescribeMacro\glslongextraNameDescHeader]
\cs{glslongextraNameDescHeader}
\end{definition}
This essentially uses the same code as for \glostyle{longragged-booktabs}
but makes it easier to adjust the header without having to define a
new style. This is defined as:
\begin{verbatim}
\glslongextraNameDescTabularHeader\endhead
\glslongextraNameDescTabularFooter\endfoot
\end{verbatim}
where:
\begin{definition}[\DescribeMacro\glslongextraNameDescTabularHeader]
\cs{glslongextraNameDescTabularHeader}
\end{definition}
sets up the header and
\begin{definition}[\DescribeMacro\glslongextraNameDescTabularFooter]
\cs{glslongextraNameDescTabularFooter}
\end{definition}
sets up the footer. If you have setup the \env{tabular} version 
of this style then the above two commands are used at the
start and end of the \env{tabular} environment (and
\cs{glslongextraNameDescHeader} isn't used).

For example, to simply remove the header and footer (for the default
\env{longtable} version of the style):
\begin{verbatim}
\renewcommand{\glslongextraNameDescHeader}{}
\end{verbatim}
Or to change the name alignment to centred:
\begin{verbatim}
\renewcommand{\glslongextraNameAlign}{c}
\end{verbatim}

\item{\glostyle{long-name-desc-loc}} This is like the
\glostyle{long-name-desc} style but has a third column for the
location list. The \env{longtable} header is given by:
\begin{definition}[\DescribeMacro\glslongextraNameDescLocationHeader]
\cs{glslongextraNameDescLocationHeader}
\end{definition}
which similarly defined in terms of the commands used for the
\env{tabular} version:
\begin{definition}[\DescribeMacro\glslongextraNameDescLocationTabularHeader]
\cs{glslongextraNameDescLocationTabularHeader}
\end{definition}
and
\begin{definition}[\DescribeMacro\glslongextraNameDescLocationTabularFooter]
\cs{glslongextraNameDescLocationTabularFooter}
\end{definition}

\item{\glostyle{long-desc-name}} This is like the
\glostyle{long-name-desc} style but swaps the columns. Note that
if the entry counter is displayed it will appear at the start 
of the second column by default. The \env{longtable} header is formatted according
to:
\begin{definition}[\DescribeMacro\glslongextraDescNameHeader]
\cs{glslongextraDescNameHeader}
\end{definition}
which similarly defined in terms of the commands used for the
\env{tabular} version:
\begin{definition}[\DescribeMacro\glslongextraDescNameTabularHeader]
\cs{glslongextraDescNameTabularHeader}
\end{definition}
and
\begin{definition}[\DescribeMacro\glslongextraDescNameTabularFooter]
\cs{glslongextraDescNameTabularFooter}
\end{definition}

\item{\glostyle{long-loc-desc-name}} This has three columns
containing the location list, description and name. The
\env{longtable} header 
is formatted according to:
\begin{definition}[\DescribeMacro\glslongextraLocationDescNameHeader]
\cs{glslongextraLocationDescNameHeader}
\end{definition}
which similarly defined in terms of the commands used for the
\env{tabular} version:
\begin{definition}[\DescribeMacro\glslongextraLocationDescNameTabularHeader]
\cs{glslongextraLocationDescNameTabularHeader}
\end{definition}
and
\begin{definition}[\DescribeMacro\glslongextraLocationDescNameTabularFooter]
\cs{glslongextraLocationDescNameTabularFooter}
\end{definition}

\item{\glostyle{long-name-desc-sym}} This is has three columns, with
the name in the first, the description in the second and the symbol
in the third.

The \env{longtable} header row is produced with:
\begin{definition}[\DescribeMacro\glslongextraNameDescSymHeader]
\cs{glslongextraNameDescSymHeader}
\end{definition}
which similarly defined in terms of the commands used for the
\env{tabular} version:
\begin{definition}[\DescribeMacro\glslongextraNameDescSymTabularHeader]
\cs{glslongextraNameDescSymTabularHeader}
\end{definition}
and
\begin{definition}[\DescribeMacro\glslongextraNameDescSymTabularFooter]
\cs{glslongextraNameDescSymTabularFooter}
\end{definition}

\item{\glostyle{long-name-desc-sym-loc}} This is has four columns, with
the name in the first, the description in the second, the symbol
in the third and the location list in the fourth.

The \env{longtable} header row is produced with:
\begin{definition}[\DescribeMacro\glslongextraNameDescSymLocationHeader]
\cs{glslongextraNameDescSymLocationHeader}
\end{definition}
which similarly defined in terms of the commands used for the
\env{tabular} version:
\begin{definition}[\DescribeMacro\glslongextraNameDescSymLocationTabularHeader]
\cs{glslongextraNameDescSymLocationTabularHeader}
\end{definition}
and
\begin{definition}[\DescribeMacro\glslongextraNameDescSymLocationTabularFooter]
\cs{glslongextraNameDescSymLocationTabularFooter}
\end{definition}

\item{\glostyle{long-name-sym-desc}} This is like the
\glostyle{long-name-desc-sym} but the second and third column are
swapped. The \env{longtable} header row is given by:
\begin{definition}[\DescribeMacro\glslongextraNameSymDescHeader]
\cs{glslongextraNameSymDescHeader}
\end{definition}
which similarly defined in terms of the commands used for the
\env{tabular} version:
\begin{definition}[\DescribeMacro\glslongextraNameSymDescTabularHeader]
\cs{glslongextraNameSymDescTabularHeader}
\end{definition}
and
\begin{definition}[\DescribeMacro\glslongextraNameSymDescTabularFooter]
\cs{glslongextraNameSymDescTabularFooter}
\end{definition}

\item{\glostyle{long-name-sym-desc-loc}} This is like the
\glostyle{long-name-desc-sym-loc} but the second and third column are
swapped. The \env{longtable} header row is given by:
\begin{definition}[\DescribeMacro\glslongextraNameSymDescLocationHeader]
\cs{glslongextraNameSymDescLocationHeader}
\end{definition}
which similarly defined in terms of the commands used for the
\env{tabular} version:
\begin{definition}[\DescribeMacro\glslongextraNameSymDescLocationTabularHeader]
\cs{glslongextraNameSymDescLocationTabularHeader}
\end{definition}
and
\begin{definition}[\DescribeMacro\glslongextraNameSymDescLocationTabularFooter]
\cs{glslongextraNameSymDescLocationTabularFooter}
\end{definition}

\item{\glostyle{long-sym-desc-name}} This has the symbol in the
first column, the description in the second and the name in the
third. The \env{longtable} header row is given by:
\begin{definition}[\DescribeMacro\glslongextraSymDescNameHeader]
\cs{glslongextraSymDescNameHeader}
\end{definition}
which similarly defined in terms of the commands used for the
\env{tabular} version:
\begin{definition}[\DescribeMacro\glslongextraSymDescNameTabularHeader]
\cs{glslongextraSymDescNameTabularHeader}
\end{definition}
and
\begin{definition}[\DescribeMacro\glslongextraSymDescNameTabularFooter]
\cs{glslongextraSymDescNameTabularFooter}
\end{definition}

\item{\glostyle{long-loc-sym-desc-name}} This has the location list
in the first column, symbol in the
second column, the description in the third and the name in the
fourth. The \env{longtable} header row is given by:
\begin{definition}[\DescribeMacro\glslongextraLocationSymDescNameHeader]
\cs{glslongextraLocationSymDescNameHeader}
\end{definition}
which similarly defined in terms of the commands used for the
\env{tabular} version:
\begin{definition}[\DescribeMacro\glslongextraLocationSymDescNameTabularHeader]
\cs{glslongextraLocationSymDescNameTabularHeader}
\end{definition}
and
\begin{definition}[\DescribeMacro\glslongextraLocationSymDescNameTabularFooter]
\cs{glslongextraLocationSymDescNameTabularFooter}
\end{definition}

\item{\glostyle{long-desc-sym-name}} This has the description in the
first column, the symbol in the second and the name in the
third. The \env{longtable} header row is given by:
\begin{definition}[\DescribeMacro\glslongextraDescSymNameHeader]
\cs{glslongextraDescSymNameHeader}
\end{definition}
which similarly defined in terms of the commands used for the
\env{tabular} version:
\begin{definition}[\DescribeMacro\glslongextraDescSymNameTabularHeader]
\cs{glslongextraDescSymNameTabularHeader}
\end{definition}
and
\begin{definition}[\DescribeMacro\glslongextraDescSymNameTabularFooter]
\cs{glslongextraDescSymNameTabularFooter}
\end{definition}

\item{\glostyle{long-loc-desc-sym-name}} This has the location list
in the first column, the description in the
second column, the symbol in the third and the name in the
fourth. The \env{longtable} header row is given by:
\begin{definition}[\DescribeMacro\glslongextraLocationDescSymNameHeader]
\cs{glslongextraLocationDescSymNameHeader}
\end{definition}
which similarly defined in terms of the commands used for the
\env{tabular} version:
\begin{definition}[\DescribeMacro\glslongextraLocationDescSymNameTabularHeader]
\cs{glslongextraLocationDescSymNameTabularHeader}
\end{definition}
and
\begin{definition}[\DescribeMacro\glslongextraLocationDescSymNameTabularFooter]
\cs{glslongextraLocationDescSymNameTabularFooter}
\end{definition}

\end{description}

\section{\styfmt{glossary-topic} package}
\label{sec:topic}

As from version 1.40, the \styfmt{glossaries-extra} package comes
with the supplementary package \sty{glossary-topic} that
provides glossary styles designed for hierarchical glossaries
where the top-level entries are topic titles. This package
automatically loads the \sty{multicols} package. If the
\sty{glossary-tree} package is also loaded then commands like
\cs{glssetwidest} can be used on these styles in much the same way
as for the \glostyle{alttree} style. If a widest value isn't set
then these styles behave more like the \glostyle{tree} style.
\begin{description}
\item[\glostyle{topic}] This style is similar to the \glostyle{tree}
style but the indentation doesn't start until the second
sub-item level. The top-level entries have the name displayed
in a larger font with the description following in a new paragraph.
This style doesn't support the \pkgopt{nogroupskip} setting.

\item[\glostyle{topicmcols}] This style is like the \glostyle{topic}
style but the sub-entries are placed inside a \env{multicols}
environment. The environment name is supplied in the value of
the command:
\begin{definition}[\DescribeMacro\glstopicColsEnv]
\cs{glstopicColsEnv}
\end{definition}
You can change this to the starred form. For example:
\begin{verbatim}
\renewcommand{\glstopicColsEnv}{multicols*}
\end{verbatim}
The number of columns is given by the command:
\begin{definition}[\DescribeMacro\glstopicCols]
\cs{glstopicCols}
\end{definition}
The default value is 2.
\end{description}
Both styles use the following commands.
\begin{definition}[\DescribeMacro\glstopicParIndent]
\cs{glstopicParIndent}
\end{definition}
This command is a length that's used for the paragraph indentation
in any multi-paragraph description for top-level entries, but not
for the first paragraph (at the start of the description) which
isn't indented.

\begin{definition}[\DescribeMacro\glstopicSubIndent]
\cs{glstopicSubIndent}
\end{definition}
This command is a length that's used to calculate the hanging
indentation for sub-entries. The level~1 sub-entries don't indent
the name. Level~$n$ sub-entries have the name indented by
$(n-1)\times$\cs{glstopicSubIndent}. The hanging indent depends
on whether or not a widest name has been set for the level.

\begin{definition}[\DescribeMacro\glstopicInit]
\cs{glstopicInit}
\end{definition}
Hook used at the start of the glossary. Does nothing by default.

Although the styles don't support letter groups by default, if you
have many topics (top-level entries) and you feel that it would help
the reader to divide them up into headed letter groups, you can 
redefine:
\begin{definition}[\DescribeMacro\glstopicGroupHeading]
\cs{glstopicGroupHeading}\marg{group label}
\end{definition}
This does nothing by default. If you want to redefine it, you can
fetch the title corresponding to the group label with
\cs{glsxtrgetgrouptitle}. For example:
\begin{verbatim}
\renewcommand*{\glstopicGroupHeading}[1]{%
  \glsxtrgetgrouptitle{#1}{\thisgrptitle}%
  \section*{\thisgrptitle}%
}
\end{verbatim}
Remember that if you use \gls{bib2gls}, you will need the
\verb|--group| (or \verb|-g|) switch to support this.

\begin{definition}[\DescribeMacro\glstopicItem]
\cs{glstopicItem}\marg{label}\marg{location list}
\end{definition}
Used to format the name, symbol, description and location list
for the top-level entries. This starts with a paragraph break
followed by:
\begin{definition}[\DescribeMacro\glstopicPreSkip]
\cs{glstopicPreSkip}
\end{definition}
which defaults to \cs{medskip}. There is then a hook:
\begin{definition}[\DescribeMacro\glstopicMarker]
\cs{glstopicMarker}\marg{label}
\end{definition}
which does nothing by default, but may be redefined. For example,
to add a line to the table of contents.
The name and symbol are set in the
form of a title using:
\begin{definition}[\DescribeMacro\glstopicTitle]
\cs{glstopicTitle}\marg{label}
\end{definition}
This uses \cs{Glossentryname} which converts the first letter to
upper case. If there's a symbol, this is added in parentheses.
Both name and symbol (if present) are encapsulated by
\begin{definition}[\DescribeMacro\glstopicTitleFont]
\cs{glstopicTitleFont}\marg{text}
\end{definition}
This uses a bold, large font by default.

If the entry has the description key set (tested with
\cs{ifglshasdesc}) then a paragraph break is inserted followed by:
\begin{definition}[\DescribeMacro\glstopicMidSkip]
\cs{glstopicMidSkip}
\end{definition}
which defaults to \cs{smallskip}. This is followed by the
description which is formatted according to:
\begin{definition}[\DescribeMacro\glstopicDesc]
\cs{glstopicDesc}\marg{label}
\end{definition}
This just does \cs{Glossentrydesc}\marg{label} followed
by the post-description hook.

A paragraph break followed by:
\begin{definition}[\DescribeMacro\glstopicPostSkip]
\cs{glstopicPostSkip}
\end{definition}
comes next regardless of whether or not the description was
displayed. This defaults to \cs{smallskip}. This is then followed
by:
\begin{definition}[\DescribeMacro\glstopicLoc]
\cs{glstopicLoc}\marg{label}\marg{location list}
\end{definition}
which may be used to display the location list, but does nothing by
default.

The sub-entries first set up the paragraph and hanging indentations
using:
\begin{definition}[\DescribeMacro\glstopicAssignSubIndent]
\cs{glstopicAssignSubIndent}\marg{level}
\end{definition}
This uses:
\begin{definition}[\DescribeMacro\glstopicAssignWidest]
\cs{glstopicAssignWidest}\marg{level}
\end{definition}
to determine if a widest name has been set for the given level.

The sub-entry has its information displayed using:
\begin{definition}[\DescribeMacro\glstopicSubItem]
\cs{glstopicSubItem}\marg{level}\marg{label}\marg{location}
\end{definition}
This encapsulates the name with:
\begin{definition}[\DescribeMacro\glstopicSubNameFont]
\cs{glstopicSubNameFont}\marg{text}
\end{definition}
By default this just uses \cs{textbf}. This is followed by:
\begin{definition}[\DescribeMacro\glstopicSubItemSep]
\cs{glstopicSubItemSep}
\end{definition}
which defaults to \cs{quad}. The name and separator are 
passed in the \meta{text} argument of:
\begin{definition}[\DescribeMacro\glstopicSubItemBox]
\cs{glstopicSubItemBox}\marg{level}\marg{text}
\end{definition}
If a widest name was set for the given level, this will
put \meta{text} inside a box of that width otherwise it just does
\meta{text}.

This is followed by the symbol in parentheses if set. Then, if the
description is set, the description and post-description hook are
displayed followed by:
\begin{definition}[\DescribeMacro\glstopicSubPreLocSep]
\cs{glstopicSubPreLocSep}
\end{definition}
(This command isn't used if the description isn't set.)

Finally the location list is displayed using:
\begin{definition}[\DescribeMacro\glstopicSubLoc]
\cs{glstopicSubLoc}\marg{label}\marg{location}
\end{definition}
which just does \meta{location} by default.

\chapter{Abbreviations}
\label{sec:abbreviations}

The new abbreviation system provided by \styfmt{glossaries-extra}
is more flexible than the acronym handling provided by the
base \styfmt{glossaries} package. The \styfmt{glossaries-extra}
package modifies the underlying formatting used by 
\cs{gls} (and its variants) so that terms that are recognised as
abbreviations can have their formatting dealt with by the style. You
therefore need to select an abbreviation style that ensures that
\cs{gls} (and its variants) displays the desired output.
See the file
\href{http://mirrors.ctan.org/macros/latex/contrib/glossaries-extra/samples/sample-abbr-styles.pdf}{sample-abbr-styles.pdf}
for samples of all provided abbreviation styles.

\begin{important}
Don't use \ics{glsfirst}, \ics{Glsfirst} or \ics{GLSfirst} with
abbreviations as they aren't flexible enough to handle some
abbreviation styles and unexpected results may occur. (To a lesser
degree, this also applies to \ics{glstext} etc.) Use commands
like \cs{glsfmttext}, \cs{glsfmtshort} or \cs{glsfmtlong} in section headings or
captions instead of \cs{gls}. If you don't want the full form to
show on first use of \cs{gls} use one of the \qt{nolong} or
\qt{noshort} styles.
\end{important}

This lack of flexibility in \cs{glsfirst} can be demonstrated with
the following document:
\begin{verbatim}
\documentclass{article}

\usepackage{glossaries-extra}
\setabbreviationstyle{footnote}
\newabbreviation{ex}{EX}{Example}

\begin{document}
Compare \gls{ex}['s] with \glsfirst{ex}['s].
\end{document}
\end{verbatim}
The \gls{firstuse} \cs{gls} has the footnote marker after the
inserted material \qt{EX's\textsuperscript{1}} but \cs{glsfirst} has
it before the inserted material \qt{EX\textsuperscript{1}'s} which
is inappropriate. Note that if the style is changed to
\abbrstyle{postfootnote}, the footnote marker appears after the
inserted material as \ics{footnote} is in the post-link hook.

There are some instances where \cs{glstext} can be useful. It's used
internally by \ics{glsfmttext}, which in turn is used by
\ics{glsseelistformat} if the entry has a short form. If
\cs{glsfmtshort} is used, this enforces the short form, but if
\cs{glsfmttext} is used, then the long form will be used for the
\qt{noshort} styles, which is more appropriate. In this particular
situation, there's no need to worry about inserted material as the
final optional argument isn't supported by \cs{glsfmttext}.

Abbreviations include acronyms (words formed from initial letters,
such as \qt{laser}),
initialisms (initial letters of a phrase, such as \qt{html}, that
aren't pronounced as words) and contractions (where
parts of words are omitted, often replaced by an apostrophe, such as
\qt{don't}).
The \qt{acronym} code provided by the \styfmt{glossaries} package is
misnamed as it's more often than not used for initialisms instead.
Acronyms tend not to be \emph{expanded} on \gls{firstuse} (although they may
need to be \emph{described} for readers unfamiliar with the term). They are therefore more like a regular term,
which may or may not require a description in the glossary.

The \styfmt{glossaries-extra} package corrects this misnomer, and
provides better abbreviation handling, with
\begin{definition}[\DescribeMacro\newabbreviation]
\cs{newabbreviation}\oarg{options}\marg{label}\marg{short}\marg{long}
\end{definition}

This sets the \gloskey{category} key to \texttt{abbreviation} by
default, but that value may be overridden in \meta{options}.
The category may have attributes that modify the way abbreviations
are defined. For example, the \catattr{insertdots} attribute will
automatically insert full stops (periods) into \meta{short} or the
\catattr{noshortplural} attribute will set the default value
of the \gloskey{shortplural} key to just \meta{short} (without
appending the plural suffix). See \sectionref{sec:categories} for
further details.

See \sectionref{sec:nested} regarding the pitfalls of using
commands like \ics{gls} or \ics{glsxtrshort} within
\meta{short} or \meta{long}.

\begin{important}
Make sure that you set the category attributes before defining new
abbreviations or they may not be correctly applied.
\end{important}

The \ics{newacronym} command provided by the \styfmt{glossaries}
package is redefined by \styfmt{glossaries-extra} to use
\cs{newabbreviation} with the \gloskey{category} set to
\texttt{acronym} (see also \sectionref{sec:acronymmods}) so
\begin{definition}[\DescribeMacro\newacronym]
\cs{newacronym}\oarg{options}\marg{label}\marg{short}\marg{long}
\end{definition}
is now equivalent to
\begin{display}\raggedright\ttfamily
\cs{newabbreviation}[type=\cs{acronymtype},category=acronym,\meta{options}]\marg{label}\marg{short}\marg{long}
\end{display}

\begin{sloppypar}
The \cs{newabbreviation} command is superficially similar to the
\styfmt{glossaries} package's \cs{newacronym} but you can apply
different styles to different categories. The default style is
\abbrstyle{short-nolong} for entries in the \category{acronym} category and 
\abbrstyle{short-long} for entries in the \category{abbreviation}
category. (These aren't the same as the acronym styles provided by
the \styfmt{glossaries} package, although they may produce similar
results.)
\end{sloppypar}

The way the abbreviations are displayed by commands like \cs{gls}
varies according to the abbreviation style. The styles are set
according to the entry's category so, unlike the base
\styfmt{glossaries} package, you can have different abbreviation
styles within the same glossary.

There are two types of full forms. The display full form, which is
used on \gls{firstuse} by commands like \ics{gls} and the inline full
form, which is used by commands like \ics{glsxtrfull}.
For some of the abbreviation styles, such as \abbrstyle{long-short}, the display and inline forms
are the same. In the case of styles such as \abbrstyle{short-nolong} or
\abbrstyle{short-footnote}, the display and inline full forms are different.

These formatting commands aren't stored in the \gloskey{short},
\gloskey{shortplural}, \gloskey{long} or \gloskey{longplural}
fields, which means they won't be used within commands like 
\cs{glsentryshort} (but they are used within commands like
\cs{glsxtrshort} and \cs{glsfmtshort}).
Note that \ics{glsxtrlong} and the case-changing variants don't use
\cs{glsfirstlongfont}.

You can apply the formatting command used for the short form
to some arbitrary text using
\begin{definition}[\DescribeMacro\glsuseabbrvfont]
\cs{glsuseabbrvfont}\marg{text}\marg{category}
\end{definition}
where \meta{category} is the category label that identifies
the abbreviation style. Similarly for the formatting command
use by the long form:
\begin{definition}[\DescribeMacro\glsuselongfont]
\cs{glsuselongfont}\marg{text}\marg{category}
\end{definition}

You can't use the acronym commands provided by the base package with
the new abbreviations provided by the \styfmt{glossaries-extra}
package. The style commands that replace \cs{setacronymstyle},
\cs{acrshort} etc are described in \sectionref{sec:abbrstyle}.
The \pkgopt{acronymlists} package option and associated commands
aren't supported. The \ics{forallacronyms} command, which iterates
over all acronym lists, should be replaced with:
\begin{definition}[\DescribeMacro\forallabbreviationlists]
\cs{forallabbreviationlists}\marg{cs}\marg{body}
\end{definition}

\section{Tagging Initials}
\label{sec:tagging}

If you would like to tag the initial letters in the long form
such that those letters are underlined in the glossary but
not in the main part of the document, you can use
\begin{definition}[\DescribeMacro\GlsXtrEnableInitialTagging]
\cs{GlsXtrEnableInitialTagging}\marg{categories}\marg{cs}
\end{definition}
before you define your abbreviations.

This command (robustly) defines \meta{cs} (a control sequence) 
to accept a single argument, which is the letter (or letters)
that needs to be tagged. The normal behaviour of this command
within the document is to simply do its argument, but in the
glossary it's activated for those categories that have
the \catattr{tagging} attribute set to \qt{true}. For those
cases it will use
\begin{definition}[\DescribeMacro\glsxtrtagfont]
\cs{glsxtrtagfont}\marg{text}
\end{definition}
This command defaults to \ics{underline}\marg{text}
but may be redefined as required.

The control sequence \meta{cs} can't already be defined when
used with the unstarred version of 
\cs{GlsXtrEnableInitialTagging} for safety reasons.
The starred version will overwrite any previous definition
of \meta{cs}. As with redefining any commands, ensure that
you don't redefine something important. In fact, just forget
the existence of the starred version and let's pretend I didn't
mention it.

The first argument of \cs{GlsXtrEnableInitialTagging} is a
comma-separated list of category names. The \catattr{tagging}
attribute will automatically be set for those categories.
You can later set this attribute for other categories (see
\sectionref{sec:categories}) but this must be done before the
glossary is displayed.

The accompanying sample file \nolinkurl{sample-mixtures.tex}
uses initial tagging for both the \category{acronym} and
\category{abbreviation} categories:
\begin{verbatim}
\GlsXtrEnableInitialTagging{acronym,abbreviation}{\itag}
\end{verbatim}
This defines the command \cs{itag} which can be used in the
definitions. For example:
\begin{verbatim}
\newacronym
 [description={a system for detecting the location and
 speed of ships, aircraft, etc, through the use of radio
 waves}% description of this term
 ]
 {radar}% identifying label
 {radar}% short form (i.e. the word)
 {\itag{ra}dio \itag{d}etection \itag{a}nd \itag{r}anging}

\newabbreviation{xml}{XML}
 {e\itag{x}tensible \itag{m}arkup \itag{l}anguage}
\end{verbatim}
The underlining of the tagged letters only occurs in the
glossary and then only for entries with the \catattr{tagging}
attribute set.

\section{Abbreviation Styles}
\label{sec:abbrstyle}

The abbreviation style must be set before abbreviations are defined
using:
\begin{definition}[\DescribeMacro\setabbreviationstyle]
\cs{setabbreviationstyle}\oarg{category}\marg{style-name}
\end{definition}
where \meta{style-name} is the name of the style and \meta{category}
is the category label (\texttt{abbreviation} by default). New
abbreviations will pick up the current style according to their
given category. If there is no style set for the category, the
fallback is the style for the \texttt{abbreviation} category.
Some styles may automatically modify one or more of the attributes
associated with the given category. For example, the
\abbrstyle{long-noshort} and \abbrstyle{short-nolong} styles set the
\catattr{regular} attribute to \texttt{true}.

\begin{important}
If you want to apply different styles to groups of abbreviations,
assign a different category to each group and set the style for the
given category.
\end{important}

Note that \ics{setacronymstyle} is disabled by
\styfmt{glossaries-extra}. Use
\begin{alltt}
\cs{setabbreviationstyle}[acronym]\marg{style-name}
\end{alltt}
instead. The original acronym interface can be restored with
\ics{RestoreAcronyms} (see \sectionref{sec:acronymmods}). However the
original acronym interface is incompatible with all the commands
described here.

Abbreviations can be used with the standard \styfmt{glossaries}
commands, such as \ics{gls}, but don't use the acronym commands
like \ics{acrshort} (which use \ics{acronymfont}). The short form can be
produced with:
\begin{definition}[\DescribeMacro\glsxtrshort]
\cs{glsxtrshort}\oarg{options}\marg{label}\oarg{insert}
\end{definition}
(Use this instead of \ics{acrshort}.)

The long form can be produced with
\begin{definition}[\DescribeMacro\glsxtrlong]
\cs{glsxtrlong}\oarg{options}\marg{label}\oarg{insert}
\end{definition}
(Use this instead of \ics{acrlong}.)

The \emph{inline} full form can be produced with
\begin{definition}[\DescribeMacro\glsxtrfull]
\cs{glsxtrfull}\oarg{options}\marg{label}\oarg{insert}
\end{definition}
(This this instead of \ics{acrfull}.)

\begin{important}
In general, it's best not to use commands like \ics{glsfirst}
for abbreviations, especially if you use the \meta{insert}
optional argument. Use either \cs{gls} (possibly with a reset)
or \cs{glsxtrfull}.
\end{important}

As mentioned earlier, the inline full form may not necessarily match the format used on
\gls{firstuse} with \cs{gls}. For example, the
\abbrstyle{short-nolong} style
only displays the short form on \gls{firstuse}, but the full
form will display the long form followed by the short form in
parentheses.

\begin{important}
If you want to use an abbreviation in a chapter or section
title, use the commands described in \sectionref{sec:headtitle}
instead.
\end{important}

The arguments \meta{options}, \meta{label} and \meta{insert} are the
same as for commands such as \cs{glstext}. There are also analogous
case-changing commands:

First letter upper case short form:
\begin{definition}[\DescribeMacro\Glsxtrshort]
\cs{Glsxtrshort}\oarg{options}\marg{label}\oarg{insert}
\end{definition}

First letter upper case long form:
\begin{definition}[\DescribeMacro\Glsxtrlong]
\cs{Glsxtrlong}\oarg{options}\marg{label}\oarg{insert}
\end{definition}

First letter upper case inline full form:
\begin{definition}[\DescribeMacro\Glsxtrfull]
\cs{Glsxtrfull}\oarg{options}\marg{label}\oarg{insert}
\end{definition}

All upper case short form:
\begin{definition}[\DescribeMacro\Glsxtrshort]
\cs{GLSxtrshort}\oarg{options}\marg{label}\oarg{insert}
\end{definition}

All upper case long form:
\begin{definition}[\DescribeMacro\Glsxtrlong]
\cs{GLSxtrlong}\oarg{options}\marg{label}\oarg{insert}
\end{definition}

All upper case inline full form:
\begin{definition}[\DescribeMacro\GLSxtrfull]
\cs{GLSxtrfull}\oarg{options}\marg{label}\oarg{insert}
\end{definition}

Plural forms are also available.

Short form plurals:
\begin{definition}[\DescribeMacro\glsxtrshortpl]
\cs{glsxtrshortpl}\oarg{options}\marg{label}\oarg{insert}
\end{definition}
\begin{definition}[\DescribeMacro\Glsxtrshortpl]
\cs{Glsxtrshortpl}\oarg{options}\marg{label}\oarg{insert}
\end{definition}
\begin{definition}[\DescribeMacro\GLSxtrshortpl]
\cs{GLSxtrshortpl}\oarg{options}\marg{label}\oarg{insert}
\end{definition}

Long form plurals:
\begin{definition}[\DescribeMacro\glsxtrlongpl]
\cs{glsxtrlongpl}\oarg{options}\marg{label}\oarg{insert}
\end{definition}
\begin{definition}[\DescribeMacro\Glsxtrlongpl]
\cs{Glsxtrlongpl}\oarg{options}\marg{label}\oarg{insert}
\end{definition}
\begin{definition}[\DescribeMacro\GLSxtrlongpl]
\cs{GLSxtrlongpl}\oarg{options}\marg{label}\oarg{insert}
\end{definition}

Full form plurals:
\begin{definition}[\DescribeMacro\glsxtrfullpl]
\cs{glsxtrfullpl}\oarg{options}\marg{label}\oarg{insert}
\end{definition}
\begin{definition}[\DescribeMacro\Glsxtrfullpl]
\cs{Glsxtrfullpl}\oarg{options}\marg{label}\oarg{insert}
\end{definition}
\begin{definition}[\DescribeMacro\GLSxtrfullpl]
\cs{GLSxtrfullpl}\oarg{options}\marg{label}\oarg{insert}
\end{definition}

\begin{important}
Be careful about using \cs{glsentryfull}, \cs{Glsentryfull},
\cs{glsentryfullpl} and \cs{Glsentryfullpl}. These commands will use
the currently applied style rather than the style in use when the
entry was defined. If you have mixed styles, you'll need to use
\ics{glsxtrfull} instead. Similarly for \cs{glsentryshort} etc.
\end{important}

\section{Shortcut Commands}
\label{sec:abbrshortcuts}

The abbreviation shortcut commands can be enabled using
the \pkgopt[abbreviations]{shortcuts} package option 
(or \pkgopt[abbr]{shortcuts}) or \pkgopt[ac]{shortcuts}.
(You can use both settings at the same time.)
The provided shortcut commands listed in
\tableref{tab:abbrshortcuts}.

\begin{table}[htbp]
\caption{Abbreviation Shortcut Commands}
\label{tab:abbrshortcuts}
\centering
\begin{tabular}{lll}
\bfseries Shortcut & 
\bfseries Shortcut & 
\bfseries Equivalent Command\\
\bfseries (\pkgopt[abbreviations]{shortcuts}) & 
\bfseries (\pkgopt[ac]{shortcuts})\\
\ics{ab} & \ics{ac} & \ics{cgls}\\
\ics{abp} & \ics{acp} & \ics{cglspl}\\
\ics{as} & \ics{acs} & \ics{glsxtrshort}\\
\ics{asp} & \ics{acsp} & \ics{glsxtrshortpl}\\
\ics{al} & \ics{acl} & \ics{glsxtrlong}\\
\ics{alp} & \ics{aclp} & \ics{glsxtrlongpl}\\
\ics{af} & \ics{acf} & \ics{glsxtrfull}\\
\ics{afp} & \ics{acfp} & \ics{glsxtrfullpl}\\
\ics{Ab} & \ics{Ac} & \ics{cgls}\\
\ics{Abp} & \ics{Acp} & \ics{cglspl}\\
\ics{As} & \ics{Acs} & \ics{Glsxtrshort}\\
\ics{Asp} & \ics{Acsp} & \ics{Glsxtrshortpl}\\
\ics{Al} & \ics{Acl} & \ics{Glsxtrlong}\\
\ics{Alp} & \ics{Aclp} & \ics{Glsxtrlongpl}\\
\ics{Af} & \ics{Acf} & \ics{Glsxtrfull}\\
\ics{Afp} & \ics{Acfp} & \ics{Glsxtrfullpl}\\
\ics{AB} & \ics{AC} & \ics{cGLS}\\
\ics{ABP} & \ics{ACP} & \ics{cGLSpl}\\
\ics{AS} & \ics{ACS} & \ics{GLSxtrshort}\\
\ics{ASP} & \ics{ACSP} & \ics{GLSxtrshortpl}\\
\ics{AL} & \ics{ACL} & \ics{GLSxtrlong}\\
\ics{ALP} & \ics{ACLP} & \ics{GLSxtrlongpl}\\
\ics{AF} & \ics{ACF} & \ics{GLSxtrfull}\\
\ics{AFP} & \ics{ACFP} & \ics{GLSxtrfullpl}\\
\ics{newabbr} & \ics{newabbr} & \ics{newabbreviation}
\end{tabular}
\end{table}

\section{Predefined Abbreviation Styles}
\label{sec:predefabbrvstyles}

There are two types of abbreviation styles: those that treat the
abbreviation as a regular entry (so that \ics{gls} uses
\ics{glsgenentryfmt}) and those that don't treat the abbreviation as
a regular entry (so that \ics{gls} uses \ics{glsxtrgenabbrvfmt}).

The regular entry abbreviation styles set the \catattr{regular}
attribute to \qt{true} for the category assigned to each 
abbreviation with that style. This means that on \gls{firstuse},
\ics{gls} uses the value of the \gloskey{first} field and on
subsequent use \ics{gls} uses the value of the \gloskey{text} field
(and analogously for the plural and case-changing versions). The
\gloskey{short} and \gloskey{long} fields are set as appropriate
and may be accessed through commands like \ics{glsxtrshort}. 

The other abbreviation styles don't modify the \catattr{regular}
attribute. The \gloskey{first} and \gloskey{text} fields (and their
plural forms) are set and can be accessed through commands like 
\cs{glsfirst}, but they aren't used by commands like \ics{gls},
which instead use the short form (stored in the \gloskey{short} key)
and the display full format (through commands like
\ics{glsxtrfullformat} that are defined by the style).

In both cases, the \gls{firstuse} of \ics{gls} may not match the
text produced by \ics{glsfirst} (and likewise for the plural
and case-changing versions).

The sample file \nolinkurl{sample-abbr-styles.tex} demonstrates all
predefined styles described here.

\begin{important}
For the \qt{sc} styles that use \ics{textsc}, be careful about your choice
of fonts as some only have limited support. For
example, you may not be able to combine bold and small-caps. I
recommend that you at least use the \sty{fontenc} package with the
\pkgoptfmt{T1} option or something similar.
\end{important}

Many of the styles have helper commands in the form
\cs{glsxtr...name} and \cs{glsxtr...sort} that are expanded within the
\gloskey{name} and \gloskey{sort} fields when the abbreviation is defined.
These commands may use the helper token registers available within
\cs{newabbreviation}, such as \ics{glsshorttok}. The contents of a
register can be accessed with \ics{the}\meta{register}. It's essential that
these expand when the abbreviation is defined so don't hide them behind no-expandable content if you
redefine these helper commands.

The parenthetical styles, such as \abbrstyle{long-short}, 
use
\begin{definition}[\DescribeMacro\glsxtrparen]
\cs{glsxtrparen}\marg{text}
\end{definition}
to set the parenthetical material. This just puts parentheses around
the text: (\meta{text}).

The basic abbreviation styles, such as \abbrstyle{long-short}
and \abbrstyle{short-long} use
\begin{definition}[\DescribeMacro\glsabbrvdefaultfont]
\cs{glsabbrvdefaultfont}\marg{text}
\end{definition}
for the short form. This just does \meta{text} by default.
(That is, no font change is applied.) On first use, 
\begin{definition}[\DescribeMacro\glsfirstabbrvdefaultfont]
\cs{glsfirstabbrvdefaultfont}\marg{text}
\end{definition}
is used instead. By default, this just does 
\cs{glsabbrvdefaultfont}. The long form is formatted according to 
\begin{definition}[\DescribeMacro\glslongdefaultfont]
\cs{glslongdefaultfont}\marg{text}
\end{definition}
which again just does \meta{text} (no font change). On first use,
\begin{definition}[\DescribeMacro\glsfirstlongdefaultfont]
\cs{glsfirstlongdefaultfont}\marg{text}
\end{definition}
is used instead. This just does \cs{glslongdefaultfont}.
The plural suffix used for the short form is given by
\begin{definition}[\DescribeMacro\glsxtrabbrvpluralsuffix]
\cs{glsxtrabbrvpluralsuffix}
\end{definition}
which defaults to \ics{glspluralsuffix}.

The small-cap styles, such as \abbrstyle{long-short-sc}
and \abbrstyle{short-sc-long}, use
\begin{definition}[\DescribeMacro\glsabbrvscfont]
\cs{glsabbrvscfont}\marg{text}
\end{definition}
which uses \cs{textsc}.\footnote{For compatibility with earlier
versions, \cs{glsabbrvscfont} is defined to \cs{glsxtrscfont},
which is defined to use \cs{textsc}. Direct use of
\cs{glsxtrscfont} is now deprecated. Likewise for similar commands.}
On first use
\begin{definition}[\DescribeMacro\glsabbrvdefaultfont]
\cs{glsfirstabbrvscfont}\marg{text}
\end{definition}
is used instead. This uses \cs{glsabbrvscfont} by default.
So redefine, \cs{glsabbrvscfont} to change first and subsequent uses or
\cs{glsfirstabbrvscfont} to change just the first use.

The long form for the small-cap styles uses \cs{glslongdefaultfont}
or \cs{glsfirstlongdefaultfont}, as with the basic style.
The suffix is given by
\begin{definition}[\DescribeMacro\glsxtrscsuffix]
\cs{glsxtrscsuffix}
\end{definition}
This is defined as
\begin{verbatim}
\newcommand*{\glsxtrscsuffix}{\glstextup{\glsxtrabbrvpluralsuffix}}
\end{verbatim}
The \ics{glstextup} command is provided by \styfmt{glossaries}
and is used to switch off the small caps font for the suffix.
If you override the default short plural using the
\gloskey{shortplural} key when you define the abbreviation
you will need to make the appropriate adjustment if necessary.
(Remember that the default plural suffix behaviour can be modified
through the use of the \catattr{aposplural} and
\catattr{noshortplural} attributes. See 
\sectionref{sec:categories} for further details.)

The small styles, such as \abbrstyle{long-short-sm}
and \abbrstyle{short-sm-long}, use
\begin{definition}[\DescribeMacro\glsabbrvsmfont]
\cs{glsabbrvsmfont}\marg{text}
\end{definition}
which uses \cs{textsmaller}. (This requires the \sty{relsizes}
package, which isn't loaded by \styfmt{glossaries-extra}, so must be
loaded explicitly.)
On first use
\begin{definition}[\DescribeMacro\glsfirstabbrvsmfont]
\cs{glsfirstabbrvsmfont}\marg{text}
\end{definition}
is used instead. This uses \cs{glsabbrvsmfont} by default.

The long form for the smaller styles uses \cs{glslongdefaultfont}
or \cs{glsfirstlongdefaultfont}, as with the basic style.
The suffix is given by
\begin{definition}[\DescribeMacro\glsxtrsmsuffix]
\cs{glsxtrsmsuffix}
\end{definition}
which defaults to just \cs{glsxtrabbrvpluralsuffix}.

The \qt{short-em} (emphasize short) styles, such as 
\abbrstyle{long-short-em} or \abbrstyle{short-em-long}, use
\begin{definition}[\DescribeMacro\glsabbrvemfont]
\cs{glsabbrvemfont}\marg{text}
\end{definition}
On first use
\begin{definition}[\DescribeMacro\glsfirstabbrvemfont]
\cs{glsfirstabbrvemfont}\marg{text}
\end{definition}
is used instead. This uses \cs{glsabbrvemfont} by default.
The suffix is given by
\begin{definition}[\DescribeMacro\glsxtremsuffix]
\cs{glsxtremsuffix}
\end{definition}
which defaults to just \cs{glsxtrabbrvpluralsuffix}.
The long form is as for the basic style unless the style
is a \qt{long-em} style.

The \qt{long-em} (emphasize long) styles, such as
\abbrstyle{long-em-short-em} or \abbrstyle{short-em-long-em}, use
\begin{definition}[\DescribeMacro\glsfirstlongemfont]
\cs{glsfirstlongemfont}\marg{long-form}
\end{definition}
instead of \cs{glsfirstlongdefaultfont}\marg{long-form} and
\begin{definition}[\DescribeMacro\glslongemfont]
\cs{glslongemfont}\marg{long-form}
\end{definition}
instead of \cs{glslongdefaultfont}\marg{long-form}. The first form
\cs{glsfirstlongemfont} is initialised to use \cs{glslongemfont}.

The user styles have similar commands:
\begin{definition}[\DescribeMacro\glsabbrvuserfont]
\cs{glsabbrvuserfont}\marg{text}
\end{definition}
for the short form,
\begin{definition}[\DescribeMacro\glsfirstabbrvuserfont]
\cs{glsfirstabbrvuserfont}\marg{text}
\end{definition}
for the first use short form,
\begin{definition}[\DescribeMacro\glslonguserfont]
\cs{glslonguserfont}\marg{text}
\end{definition}
for the long form,
\begin{definition}[\DescribeMacro\glsfirstlonguserfont]
\cs{glsfirstlonguserfont}\marg{text}
\end{definition}
for the first use long form, and
\begin{definition}[\DescribeMacro\glsxtrusersuffix]
\cs{glsxtrusersuffix}
\end{definition}
for the short plural suffix.

Similarly for the hyphen styles:
\begin{definition}[\DescribeMacro\glsabbrvhyphenfont]
\cs{glsabbrvhyphenfont}\marg{text}
\end{definition}
for the short form,
\begin{definition}[\DescribeMacro\glsfirstabbrvhyphenfont]
\cs{glsfirstabbrvhyphenfont}\marg{text}
\end{definition}
for the first use short form,
\begin{definition}[\DescribeMacro\glslonghyphenfont]
\cs{glslonghyphenfont}\marg{text}
\end{definition}
for the long form,
\begin{definition}[\DescribeMacro\glsfirstlonghyphenfont]
\cs{glsfirstlonghyphenfont}\marg{text}
\end{definition}
for the first use long form, and
\begin{definition}[\DescribeMacro\glsxtrhyphensuffix]
\cs{glsxtrhyphensuffix}
\end{definition}
for the short plural suffix.

Similarly for the \qt{only} styles, such as
\abbrstyle{long-only-short-only}:
\begin{definition}[\DescribeMacro\glsabbrvonlyfont]
\cs{glsabbrvonlyfont}\marg{text}
\end{definition}
for the short form,
\begin{definition}[\DescribeMacro\glsfirstabbrvonlyfont]
\cs{glsfirstabbrvonlyfont}\marg{text}
\end{definition}
for the first use short form,
\begin{definition}[\DescribeMacro\glslongonlyfont]
\cs{glslongonlyfont}\marg{text}
\end{definition}
for the long form,
\begin{definition}[\DescribeMacro\glsfirstlongonlyfont]
\cs{glsfirstlongonlyfont}\marg{text}
\end{definition}
for the first use long form, and
\begin{definition}[\DescribeMacro\glsxtronlysuffix]
\cs{glsxtronlysuffix}
\end{definition}
for the short plural suffix.

Note that by default inserted material (provided in the final
optional argument of commands like \cs{gls}), is placed outside the
font command in the predefined styles. To move it inside, use:
\begin{definition}[\DescribeMacro\glsxtrinsertinsidetrue]
\cs{glsxtrinsertinsidetrue}
\end{definition}
This applies to all the predefined styles. For example:
\begin{verbatim}
\setabbreviationstyle{long-short}
\renewcommand*{\glsfirstlongdefaultfont}[1]{\emph{#1}}
\glsxtrinsertinsidetrue
\end{verbatim}
This will make the long form and the inserted text emphasized,
whereas the default (without \cs{glsxtrinsertinsidetrue}) would 
place the inserted text outside of the emphasized font.

Note that for some styles, such as the \abbrstyle{short-long}, the
inserted text would be placed inside the font command for the short
form (rather than the long form in the above example).

Remember that \cs{textsc} renders \emph{lowercase} letters as small
capitals. Uppercase letters are rendered as normal uppercase
letters, so if you specify the short form in uppercase, you won't
get small capitals unless you redefine \cs{glsabbrvscfont} to
convert its argument to lowercase. For example:
\begin{verbatim}
\renewcommand*{\glsabbrvscfont}[1]{\textsc{\MakeLowercase{#1}}}
\end{verbatim}

If you want to
easily switch between the \qt{sc} and \qt{sm} styles, you may find
it easier to redefine this command to convert case:
\begin{verbatim}
\renewcommand*{\glsabbrvscfont}[1]{\textsc{\MakeTextLowercase{#1}}}
\renewcommand*{\glsabbrvsmfont}[1]{\textsmaller{\MakeTextUppercase{#1}}}
\end{verbatim}

Some of the styles use
\begin{definition}[\DescribeMacro\glsxtrfullsep]
\cs{glsxtrfullsep}\marg{label}
\end{definition}
as a separator between the long and short forms. This is defined as
a space by default, but may be changed as required. For example:
\begin{verbatim}
\renewcommand*{\glsxtrfullsep}[1]{~}
\end{verbatim}
or
\begin{verbatim}
\renewcommand*{\glsxtrfullsep}[1]{\glsacspace{#1}}
\end{verbatim}

The new naming scheme for abbreviation styles is as follows:
\begin{itemize}
\item
\meta{field1}[\texttt{-}\meta{modifier1}]\texttt{-}[post]\meta{field2}[\texttt{-}\meta{modifier2}][\texttt{-user}]

This is for the parenthetical styles. The \texttt{-}\meta{modifier} parts may
be omitted. These styles display \meta{field1} followed by
\meta{field2} in parentheses. If \meta{field1} or \meta{field2} starts 
with \qt{no} then that element is omitted from the display style (no
parenthetical part) but is included in the inline style.

If the \texttt{-}\meta{modifier} part is present, then the field has
a font changing command applied to it. The special modifier
\texttt{-only} indicates that field is only present according to
whether or not the entry has been used.

If \texttt{post} is present then \meta{field2} is
placed after the \gls{linktext} using the post-link hook.

If the \texttt{-user} part is present, then the \gloskey{user1}
value, if provided, is inserted into the parenthetical material .
(The field used for the inserted material may be changed.)

Examples:
 \begin{itemize}
  \item\abbrstyle{long-noshort-sc}: \meta{field1} is the long
form, the short form is set in smallcaps but omitted in the display
style.
  \item\abbrstyle{long-em-short-em}: both the long form and the
short form are emphasized. The short form is in parentheses.
  \item\abbrstyle{long-short-em}: the
short form is emphasized but not the long form. The short form is in parentheses.
  \item\abbrstyle{long-short-user}: if the \gloskey{user1} key has
been set, this produces the style \meta{long} (\meta{short},
\meta{user1}) otherwise it just produces \meta{long} (\meta{short}).

  \item\abbrstyle{long-hyphen-postshort-hyphen}: the short
form and the inserted material (provided by the final optional
argument of commands like \cs{gls}) is moved to the post-link hook.
The long form is formatted according to \cs{glslonghyphenfont}
(or \cs{glsfirstlonghyphenfont} on first use). 
The short form is formatted according to \cs{glsabbrvhyphenfont}
(or \cs{glsfirstabbrvhyphenfont} on first use). 
 \end{itemize}

\item \meta{style}\texttt{-noreg}

Some styles set the \catattr{regular} attribute. In some cases,
there's a version of the style that doesn't set this attribute.
For example, \abbrstyle{long-em-noshort-em} sets the
\catattr{regular} attribute. The
\abbrstyle{long-em-noshort-em-noreg} style is a minor variation
that style that doesn't set the attribute.

There are a few \qt{noshort} styles, such as
\abbrstyle{long-hyphen-noshort-noreg}, that have
\qt{-noreg} version without a regular version. This is because the
style won't work properly with the \catattr{regular} set, but the
naming scheme is maintained for consistency with the other
\qt{noshort} styles.

\item
\meta{field1}[\texttt{-}\meta{modifier1}]\texttt{-}[\texttt{post}]\texttt{footnote}

The display style uses \meta{field1} followed by a footnote with the
other field in it. If \texttt{post} is present then the footnote is
placed after the \gls{linktext} using the post-link hook.
The inline style does \meta{field1} followed by the other field in
parentheses.

If \texttt{-}\meta{modifier1} is present, \meta{field1} has a
font-changing command applied to it.

Examples:
\begin{itemize}
\item \abbrstyle{short-footnote}: short form in the text with the
long form in the footnote.
\item \abbrstyle{short-sc-postfootnote}: short form in smallcaps
with the long form in the footnote outside of the \gls{linktext}.
\end{itemize}

\begin{important}
Take care with the footnote styles. Remember that there are some
situations where \ics{footnote} doesn't work.
\end{important}

\item \meta{style}\texttt{-desc}

Like \meta{style} but the \gloskey{description} key must be provided
when defining abbreviations with this style.

Examples:
\begin{itemize}

\item \abbrstyle{short-long-desc}: like \abbrstyle{short-long} but
requires a description.
\item \abbrstyle{short-em-footnote-desc}: like
\abbrstyle{short-em-footnote} but requires a description.
\end{itemize}
\end{itemize}

Not all combinations that fit the above syntax are provided.
Pre-version 1.04 styles that didn't fit this naming scheme are either
provided with a synonym (where the former name wasn't ambiguous) or
provided with a deprecated synonym (where the former name was
confusing).
The deprecated style names generate a warning using:
\begin{definition}[\DescribeMacro\GlsXtrWarnDeprecatedAbbrStyle]
\cs{GlsXtrWarnDeprecatedAbbrStyle}\marg{old-name}\marg{new-name}
\end{definition}
where \meta{old-name} is the deprecated name and \meta{new-name} is
the preferred name. You can suppress these warnings by redefining
this command to do nothing.

\subsection{Predefined Abbreviation Styles that Set the Regular
Attribute}
\label{sec:predefregabbrvstyles}

The following abbreviation styles set the \catattr{regular}
attribute to \qt{true} for all categories that have abbreviations
defined with any of these styles.

\begin{description}
\item[\abbrstyledef{short-nolong}]
This only displays the short form on \gls{firstuse}. The \gloskey{name}
is set to the short form through the command
\begin{definition}[\DescribeMacro\glsxtrshortnolongname]
\cs{glsxtrshortnolongname}
\end{definition}
(Similarly for the other
\texttt{short}\meta{modifier}\texttt{-nolong}\meta{modifier} styles,
unless indicated otherwise.)
This command is expanded as the entry is defined, so any
redefinition must be done before \cs{newabbreviation} (or
\cs{newacronym}) for it to take effect. Make sure to \cs{protect}
any formatting commands (or anything else that shouldn't be expanded).

The \gloskey{description} is set to the
long form. The inline full form displays
\meta{short} (\meta{long}). The long form on its own can be
displayed through commands like \ics{glsxtrlong}.

\item[\abbrstyledef{short}] A synonym for \abbrstyle{short-nolong}.

\item[\abbrstyledef{nolong-short}] Like \abbrstyle{short-nolong}
but the inline full form displays
\meta{long} (\meta{short}).

\item[\abbrstyledef{short-sc-nolong}]
Like \abbrstyle{short-nolong} but redefines \cs{glsabbrvfont} to
use \ics{glsabbrvscfont} (which defaults to \cs{textsc}).

\item[\abbrstyledef{short-sc}] A synonym for \abbrstyle{short-sc-nolong}

\item[\abbrstyledef{nolong-short-sc}] Like \abbrstyle{short-sc-nolong}
but the inline full form displays
\meta{long} (\meta{short}). The name is still obtained from
\cs{glsxtrshortnolongname} (similarly for the other styles in the
form \texttt{nolong}\meta{modifier}\texttt{-short}\meta{modifier}
unless indicated otherwise).

\item[\abbrstyledef{short-sm-nolong}]
Like \abbrstyle{short-nolong} but redefines \cs{glsabbrvfont} to
use \ics{glsabbrvsmfont} (which defaults to \cs{textsmaller}).

\item[\abbrstyledef{short-sm}] A synonym for \abbrstyle{short-sm-nolong}.

\item[\abbrstyledef{nolong-short-sm}] Like \abbrstyle{short-sm-nolong}
but the inline full form displays
\meta{long} (\meta{short}).

\item[\abbrstyledef{short-em-nolong}]
Like \abbrstyle{short-nolong} but redefines \cs{glsabbrvfont} to
use \ics{glsabbrvemfont}.

\item[\abbrstyledef{short-em}] A synonym for \abbrstyle{short-em-nolong}

\item[\abbrstyledef{nolong-short-em}] Like \abbrstyle{short-em-nolong}
but the inline full form displays
\meta{long} (\meta{short}).

\item[\abbrstyledef{short-nolong-desc}]
Like the \abbrstyle{short-nolong} style, but the \gloskey{name} is set to
the full form obtained by expanding
\begin{definition}[\DescribeMacro\glsxtrshortdescname]
\cs{glsxtrshortdescname}
\end{definition}
(Similarly for the other
\texttt{short}\meta{modifier}\texttt{-nolong}\meta{modifier}\texttt{-desc}
styles, unless indicated otherwise.) This command is expanded when the entry is defined, so
\cs{protect} fragile and formatting commands and only redefine this
command before \cs{newabbreviation} (or \cs{newacronym}).

The \gloskey{description} must be supplied by the
user. You may prefer to use the \abbrstyle{short-nolong} style with the
post-description hook set to display the long form and override
the \gloskey{description} key. (See the sample file
\texttt{sample-acronym-desc.tex}.)

\item[\abbrstyledef{short-desc}] A synonym for
\abbrstyle{short-nolong-desc}.

\item[\abbrstyledef{short-sc-nolong-desc}]
Like \abbrstyle{short-nolong} but redefines \cs{glsabbrvfont} to
use \ics{glsabbrvscfont} (which defaults to \cs{textsc}).

\item[\abbrstyledef{short-sc-desc}] A synonym for
\abbrstyle{short-sc-nolong-desc}.

\item[\abbrstyledef{short-sm-nolong-desc}]
Like \abbrstyle{short-nolong-desc} but redefines \cs{glsabbrvfont} to
use \ics{glsabbrvsmfont} (which defaults to \cs{textsmaller}).

\item[\abbrstyledef{short-sm-desc}] A synonym for
\abbrstyle{short-sm-nolong-desc}.

\item[\abbrstyledef{short-em-nolong-desc}]
Like \abbrstyle{short-nolong-desc} but redefines \cs{glsabbrvfont} to
use \ics{glsabbrvemfont}.

\item[\abbrstyledef{short-em-desc}] A synonym for
\abbrstyle{short-em-nolong-desc}.

\item[\abbrstyledef{long-noshort-desc}]
This style only displays the long form, regardless of first or
subsequent use of commands \ics{gls}. The short form may be 
accessed through commands like \ics{glsxtrshort}. The inline full
form displays \meta{long} (\meta{short}).

The \gloskey{sort} key are set to the long form.
The \gloskey{name} key is also set to the long form, but this is
done by expanding
\begin{definition}[\DescribeMacro\glsxtrlongnoshortdescname]
\cs{glsxtrlongnoshortdescname}
\end{definition} 
(Similarly for the other
\texttt{long}\meta{modifier}\texttt{-noshort}\meta{modifier}\texttt{-desc}
styles, unless indicated otherwise.) This command should only be redefined before abbreviations
are defined, and any fragile or formatting commands within it need protecting.

The \gloskey{description} must be provided by the user. The predefined
glossary styles won't display the short form. You can use the
post-description hook to automatically append the short form to the
description. The inline full form will display \meta{long}
(\meta{short}).

\item[\abbrstyledef{long-desc}] A synonym for
\abbrstyle{long-noshort-desc}.

\item[\abbrstyledef{long-noshort-sc-desc}]
Like the \abbrstyle{long-noshort-desc} style but the short form (accessed
through commands like \ics{glsxtrshort}) use \ics{glsabbrvscfont}.
(This style was originally called \depabbrstyle{long-desc-sc}. Renamed in version
1.04, but original name retained as a deprecated synonym for
backward-compatibility.)

\item[\abbrstyledef{long-noshort-sm-desc}]
Like \abbrstyle{long-noshort-desc} but redefines \cs{glsabbrvfont} to
use \ics{glsabbrvsmfont}.
(This style was originally called \depabbrstyle{long-desc-sm}. Renamed in version
1.04, but original name retained as a deprecated synonym for
backward-compatibility.)

\item[\abbrstyledef{long-noshort-em-desc}]
Like \abbrstyle{long-noshort-desc} but redefines \cs{glsabbrvfont} to
use \ics{glsabbrvemfont}. The long form isn't emphasized.
(This style was originally called \depabbrstyle{long-desc-em}. Renamed in version
1.04, but original name retained as a deprecated synonym for
backward-compatibility.)

\item[\abbrstyledef{long-em-noshort-em-desc}]
New to version 1.04, like \abbrstyle{long-noshort-desc} but redefines
\cs{glsabbrvfont} to use \ics{glsabbrvemfont}. The long form uses
\ics{glsfirstlongemfont} and \ics{glslongemfont}.

\item[\abbrstyledef{long-noshort}]
This style doesn't really make sense if you don't use the short
form anywhere in the document, but is provided for completeness.
This is like the \abbrstyle{long-noshort-desc} style, but 
the \gloskey{sort} key is set to the short form. The \gloskey{name}
key is also set to the short form, but this is done by expanding
\begin{definition}[\DescribeMacro\glsxtrlongnoshortname]
\cs{glsxtrlongnoshortname}
\end{definition}
(Similarly for other
\texttt{long}\meta{modifier}\texttt{-noshort}\meta{modifier} styles,
unless indicated otherwise.)
This command should only be redefined before abbreviations are
defined, and fragile or formatting commands should be protected.

The \gloskey{description} is set to the long form.

\item[\abbrstyledef{long}] A synonym for \abbrstyle{long-noshort}

\item[\abbrstyledef{long-noshort-sc}]
Like the \abbrstyle{long-noshort} style but the short form (accessed
through commands like \ics{glsxtrshort}) use \ics{glsabbrvscfont}.
(This style was originally called \depabbrstyle{long-sc}. Renamed in version
1.04, but original name retained as a deprecated synonym for
backward-compatibility.)

\item[\abbrstyledef{long-noshort-sm}]
Like \abbrstyle{long-noshort} but redefines \cs{glsabbrvfont} to
use \ics{glsabbrvsmfont}.
(This style was originally called \depabbrstyle{long-sm}. Renamed in version
1.04, but original name retained as a deprecated synonym for
backward-compatibility.)

\item[\abbrstyledef{long-noshort-em}]
This style is like \abbrstyle{long-noshort} but redefines \cs{glsabbrvfont} to
use \ics{glsabbrvemfont}. The long form isn't emphasized.
(This style was originally called \depabbrstyle{long-em}. Renamed in version
1.04, but original name retained as a deprecated synonym for
backward-compatibility.)

\item[\abbrstyledef{long-em-noshort-em}]
New to version 1.04, this style is like \abbrstyle{long-noshort} but redefines \cs{glsabbrvfont} to
use \ics{glsabbrvemfont}, \cs{glsfirstlongfont} to use
\cs{glsfirstlongemfont} and \cs{glslongfont} to use
\cs{glslongemfont}. The short form isn't used by commands like
\ics{gls}, but can be obtained using \ics{glsxtrshort}.
The related style \abbrstyledef{long-em-noshort-em-noreg}
doesn't set the \catattr{regular} attribute.

\end{description}

\subsection{Predefined Abbreviation Styles that Don't Set the Regular
Attribute}
\label{sec:predefnonregabbrvstyles}

The following abbreviation styles will set the 
\catattr{regular} attribute to \qt{false} if it has previously
been set. If it hasn't already been set, it's left unset.
Other attributes may also be set, depending on the style.

\begin{description}
\item[\abbrstyledef{long-short}]
On \gls{firstuse}, this style uses the format \meta{long} (\meta{short}).
The inline and display full forms are the same. The 
 \gloskey{sort} key is set to the short form. The \gloskey{name} is
also set to the short form through
\begin{definition}[\DescribeMacro\glsxtrlongshortname]
\cs{glsxtrlongshortname}
\end{definition}
(Similarly for other
\texttt{long}\meta{modifier}\texttt{-short}\meta{modifier} styles,
unless indicated otherwise.)
Any redefinition of this command must come before the abbreviations
are defined as it expands on definition. Make sure you protect
any commands that shouldn't be expanded. The long form can be
referenced with \cs{the}\cs{glslongtok} and the short form
can be referenced with \cs{the}\cs{glsshorttok}.

The \gloskey{description} is set to the long form. The long and short
forms are separated by \cs{glsxtrfullsep}. If you want to insert
material within the parentheses (such as a~translation), try the
\abbrstyle{long-short-user} style.

\item[\abbrstyledef{long-short-sc}]
Like \abbrstyle{long-short} but redefines \cs{glsabbrvfont} to
use \ics{glsabbrvscfont}.

\item[\abbrstyledef{long-short-sm}]
Like \abbrstyle{long-short} but redefines \cs{glsabbrvfont} to
use \ics{glsabbrvsmfont}.

\item[\abbrstyledef{long-short-em}]
Like \abbrstyle{long-short} but redefines \cs{glsabbrvfont} to
use \ics{glsabbrvemfont}.

\item[\abbrstyledef{long-em-short-em}]
New to version 1.04, this style is like \abbrstyle{long-short-em} but redefines \cs{glsfirstlongfont} to
use \ics{glsfirstlongemfont}.

\item[\abbrstyledef{long-only-short-only}]
New to version 1.17, this style only shows the long form
on first use and only shows the short form on subsequent use.
The inline full form \cs{glsxtrfull} shows the long form
followed by the short form in parentheses. The \gloskey{name}
field is obtained from
\begin{definition}[\DescribeMacro\glsxtronlyname]
\cs{glsxtronlyname}
\end{definition}
Any redefinition of this command must come before the abbreviations
are defined as it expands on definition. Make sure you protect
any commands that shouldn't be expanded. The long form can be
referenced with \cs{the}\cs{glslongtok} and the short form
can be referenced with \cs{the}\cs{glsshorttok}.

\item[\abbrstyledef{long-only-short-only-desc}]
New to version 1.17, this style is like
\abbrstyle{long-only-short-only} but the user must supply the
description. The \gloskey{name}
field is obtained from
\begin{definition}[\DescribeMacro\glsxtronlydescname]
\cs{glsxtronlydescname}
\end{definition}
Any redefinition of this command must come before the abbreviations
are defined as it expands on definition. Make sure you protect
any commands that shouldn't be expanded. The long form can be
referenced with \cs{the}\cs{glslongtok} and the short form
can be referenced with \cs{the}\cs{glsshorttok}.

\item[\abbrstyledef{long-em-noshort-em-noreg}]
New to version 1.17, this style is like 
\abbrstyle{long-em-noshort-em} but 
doesn't set the \catattr{regular} attribute.

\item[\abbrstyledef{long-short-user}]
This style was introduced in version 1.04. It's like the
\abbrstyle{long-short} style but additional information can be
inserted into the parenthetical material. This checks the value
of the field given by
\begin{definition}[\DescribeMacro\glsxtruserfield]
\cs{glsxtruserfield}
\end{definition}
(which defaults to \texttt{useri}) using \ics{ifglshasfield}
(provided by \styfmt{glossaries}).  If the field hasn't been set,
the style behaves like the \abbrstyle{long-short} style and
produces \meta{long} (\meta{short}) but if the field has been set,
the contents of that field are inserted within the parentheses in
the form \meta{long} (\meta{short}, \meta{field-value}).
The format is governed by
\begin{definition}[\DescribeMacro\glsxtruserparen]
\cs{glsxtruserparen}\marg{text}\marg{label}
\end{definition}
where \meta{text} is the short form (for the
\abbrstyle{long-short-user} style) or the long form (for the
\abbrstyle{short-long-user} style). This command first inserts
a space using \cs{glsxtrfullsep} and then the parenthetical content
(using \ics{glsxtrparen}).
The \gloskey{description} is set to
\begin{definition}[\DescribeMacro\glsuserdescription]
\cs{glsuserdescription}\marg{long}\marg{label}
\end{definition}
The default definition ignores the \meta{label} and encapsulates
\meta{long} with \cs{glslonguserfont}.

The \gloskey{name} is obtained by expanding
\cs{glsxtrlongshortname} (see above).
The \meta{text} argument includes the font formatting command,
\cs{glsfirstabbrvfont}\marg{short} in the case of the
\abbrstyle{long-short-user} style and
\cs{glsfirstlongfont}\marg{long} in the
case of the \abbrstyle{short-long-user} style.

For example:
\begin{verbatim}
\setabbreviationstyle[acronym]{long-short-user}

\newacronym{tug}{TUG}{\TeX\ User Group}

\newacronym
 [user1={German Speaking \TeX\ User Group}]
 {dante}{DANTE}{Deutschsprachige Anwendervereinigung \TeX\ e.V}

\end{verbatim}
On first use, \verb|\gls{tug}| will appear as:
\begin{quote}
\TeX\ User Group (TUG)
\end{quote}
whereas \verb|\gls{dante}| will appear as:
\begin{quote}
Deutschsprachige Anwendervereinigung \TeX\ e.V (DANTE, German Speaking \TeX\ User Group)
\end{quote}

The short form is formatted according to
\begin{definition}[\DescribeMacro\glsabbrvuserfont]
\cs{glsabbrvuserfont}\marg{text}
\end{definition}
and the plural suffix is given by
\begin{definition}[\DescribeMacro\glsxtrusersuffix]
\cs{glsxtrusersuffix}
\end{definition}

These may be redefined as appropriate. For example,
if you want a~smallcaps style, you can just set these commands
to those used by the \abbrstyle{long-short-sc} style:
\begin{verbatim}
\renewcommand{\glsabbruserfont}[1]{\glsabbrvscfont{#1}}
\renewcommand{\glsxtrusersuffix}{\glsxtrscsuffix}
\end{verbatim}

\item[\abbrstyledef{long-noshort-noreg}]
This style is like \abbrstyle{long-noshort} but it doesn't
set the \catattr{regular} attribute.

\item[\abbrstyledef{long-noshort-desc-noreg}]
This style is like \abbrstyle{long-noshort-desc} but it doesn't
set the \catattr{regular} attribute.

\item[\abbrstyledef{long-short-desc}]
On \gls{firstuse}, this style uses the format \meta{long} (\meta{short}).
The inline and display full forms are the same. The \gloskey{name}
is set to the full form. The \gloskey{sort} key is set to
\meta{long} (\meta{short}). Before version 1.04, this was
incorrectly set to the short form. If you want to revert back to
this you can redefine
\begin{definition}[\DescribeMacro\glsxtrlongshortdescsort]
\cs{glsxtrlongshortdescsort}
\end{definition}
For example:
\begin{verbatim}
\renewcommand*{\glsxtrlongshortdescsort}{\the\glsshorttok}
\end{verbatim}
The \gloskey{description} must be supplied by the user.
The long and short forms are separated by \cs{glsxtrfullsep}. The \gloskey{name}
field is obtained from
\begin{definition}[\DescribeMacro\glsxtrlongshortdescname]
\cs{glsxtrlongshortdescname}
\end{definition}
(Similarly for other
\texttt{long}\meta{modifier}\texttt{-short}\meta{modifier}\texttt{-desc}
styles, unless indicated otherwise.)
Any redefinition of this command must come before the abbreviations
are defined as it expands on definition. Make sure you protect
any commands that shouldn't be expanded. The long form can be
referenced with \cs{the}\cs{glslongtok} and the short form
can be referenced with \cs{the}\cs{glsshorttok}.

\item[\abbrstyledef{long-short-sc-desc}]
Like \abbrstyle{long-short-desc} but redefines \cs{glsabbrvfont} to
use \ics{glsabbrvscfont}.

\item[\abbrstyledef{long-short-sm-desc}]
Like \abbrstyle{long-short-desc} but redefines \cs{glsabbrvfont} to
use \ics{glsabbrvsmfont}.

\item[\abbrstyledef{long-short-em-desc}]
Like \abbrstyle{long-short-desc} but redefines \cs{glsabbrvfont} to
use \ics{glsabbrvemfont}.

\item[\abbrstyledef{long-em-short-em-desc}]
New to version 1.04, this style is like \abbrstyle{long-short-em-desc} but redefines \cs{glsfirstlongfont} to
use \ics{glsfirstlongemfont}.

\item[\abbrstyledef{long-em-noshort-em-desc-noreg}]
New to version 1.17, this style is like 
\abbrstyle{long-em-noshort-em-desc} but 
doesn't set the \catattr{regular} attribute.


\item[\abbrstyledef{long-short-user-desc}]
New to version 1.04, this style is like a~cross between the
\abbrstyle{long-short-desc} style and the
\abbrstyle{long-short-user} style. The display and inline forms are
as for \abbrstyle{long-short-user} and the \gloskey{name} key is
obtained from
\begin{definition}[\DescribeMacro\glsxtrlongshortuserdescname]
\cs{glsxtrlongshortuserdescname}
\end{definition}
Again, this should only be redefined before \cs{newabbreviation} (or
\cs{newacronym}), and fragile and formatting commands need protecting.

The \gloskey{description} key must be
supplied in the optional argument of \cs{newabbreviation}
(or \cs{newacronym}). The \gloskey{sort} key is set to \meta{long}
(\meta{short}) as per the \abbrstyle{long-short-desc} style.

\item[\abbrstyledef{short-nolong-noreg}]
This is like \abbrstyle{short-nolong} but doesn't set the
\catattr{regular} attribute.

\item[\abbrstyledef{nolong-short-noreg}]
This is like \abbrstyle{nolong-short} but doesn't set the
\catattr{regular} attribute.

\item[\abbrstyledef{short-long}]
On \gls{firstuse}, this style uses the format \meta{short} (\meta{long}).
The inline and display full forms are the same. The \gloskey{name}
and \gloskey{sort} keys are set to the short form. The 
\gloskey{description} is set to the long form.
The short and long forms are separated by \cs{glsxtrfullsep}.
If you want to insert
material within the parentheses (such as a~translation), try the
\abbrstyle{short-long-user} style.

The \gloskey{name} field is obtained from
\begin{definition}[\DescribeMacro\glsxtrshortlongname]
\cs{glsxtrshortlongname}
\end{definition}
(Similarly for other
\texttt{short}\meta{modifier}\texttt{-long}\meta{modifier} styles,
unless indicated otherwise.)
Any redefinition of this command must come before the abbreviations
are defined as it expands on definition. Make sure you protect
any commands that shouldn't be expanded. The long form can be
referenced with \cs{the}\cs{glslongtok} and the short form
can be referenced with \cs{the}\cs{glsshorttok}.


\item[\abbrstyledef{short-sc-long}]
Like \abbrstyle{short-long} but redefines \cs{glsabbrvfont} to
use \ics{glsabbrvscfont}.

\item[\abbrstyledef{short-sm-long}]
Like \abbrstyle{short-long} but redefines \cs{glsabbrvfont} to
use \ics{glsabbrvsmfont}.

\item[\abbrstyledef{short-em-long}]
Like \abbrstyle{short-long} but redefines \cs{glsabbrvfont} to
use \ics{glsabbrvemfont}.

\item[\abbrstyledef{short-em-long-em}]
New to version 1.04, this style is like \abbrstyle{short-em-long} but redefines \cs{glsfirstlongfont} to
use \ics{glsfirstlongemfont}.

\item[\abbrstyledef{short-long-user}]
New to version 1.04. This style is like the
\abbrstyle{long-short-user} style but with the long and short forms
switched. The parenthetical material is governed by the same command
\ics{glsxtruserparen}, but the first argument supplied to it is 
the long form instead of the short form. The \gloskey{name} field is
obtained by expanding
\begin{definition}[\DescribeMacro\glsxtrshortlongname]
\cs{glsxtrshortlongname}
\end{definition}
Again, this should only be redefined before \cs{newabbreviation} (or
\cs{newacronym}) and commands that should be expanded need to be
protected. The \gloskey{description} is set to
\cs{glsuserdescription}\marg{long}\marg{label}.

\item[\abbrstyledef{short-nolong-desc-noreg}]
This style is like \abbrstyle{short-nolong-desc} but it doesn't
set the \catattr{regular} attribute.

\item[\abbrstyledef{short-long-desc}]
On \gls{firstuse}, this style uses the format \meta{short} (\meta{long}).
The inline and display full forms are the same. The \gloskey{name}
is set to the full form. The 
\gloskey{description} must be supplied by the user.
The short and long forms are separated by \cs{glsxtrfullsep}. The \gloskey{name}
field is obtained from
\begin{definition}[\DescribeMacro\glsxtrshortlongdescname]
\cs{glsxtrshortlongdescname}
\end{definition}
(Similarly for other
\texttt{short}\meta{modifier}\texttt{-long}\meta{modifier}\texttt{-desc}
styles, unless indicated otherwise.)
Any redefinition of this command must come before the abbreviations
are defined as it expands on definition. Make sure you protect
any commands that shouldn't be expanded. The long form can be
referenced with \cs{the}\cs{glslongtok} and the short form
can be referenced with \cs{the}\cs{glsshorttok}.

\item[\abbrstyledef{short-sc-long-desc}]
Like \abbrstyle{short-long-desc} but redefines \cs{glsabbrvfont} to
use \ics{glsabbrvscfont}.

\item[\abbrstyledef{short-sm-long-desc}]
Like \abbrstyle{short-long-desc} but redefines \cs{glsabbrvfont} to
use \ics{glsabbrvsmfont}.

\item[\abbrstyledef{short-em-long-desc}]
Like \abbrstyle{short-long-desc} but redefines \cs{glsabbrvfont} to
use \ics{glsabbrvemfont}.

\item[\abbrstyledef{short-em-long-em-desc}]
New to version 1.04, this style is like \abbrstyle{short-em-long-desc} but redefines \cs{glsfirstlongfont} to
use \ics{glsfirstlongemfont}.

\item[\abbrstyledef{short-long-user-desc}]
New to version 1.04, this style is like a~cross between the
\abbrstyle{short-long-desc} style and the
\abbrstyle{short-long-user} style. The display and inline forms are
as for \abbrstyle{short-long-user}, but the \gloskey{name} key is
obtained from 
\begin{definition}[\DescribeMacro\glsxtrshortlonguserdescname]
\cs{glsxtrshortlonguserdescname}
\end{definition}
Again, this should only be redefined before \cs{newabbreviation} (or
\cs{newacronym}), and fragile and formatting commands need protecting.

The \gloskey{description} key must be
supplied in the optional argument of \cs{newabbreviation}
(or \cs{newacronym}).

\item[\abbrstyledef{short-footnote}]
On \gls{firstuse}, this style displays the short form with the long form
as a footnote. This style automatically sets the
\catattr{nohyperfirst} attribute to \qt{true} for the supplied
category, so the \gls{firstuse} won't be hyperlinked (but the footnote
marker may be, if the \sty{hyperref} package is used).

The inline full form uses the \meta{short}
(\meta{long}) style. The \gloskey{name} is set to the short form.
The \gloskey{description} is set to the long form. The
\gloskey{name} key is obtained by expanding:
\begin{definition}[\DescribeMacro\glsxtrfootnotename]
\cs{glsxtrfootnotename}
\end{definition}
(Similarly for other
\texttt{short\meta{modifier}-\meta{modifier}footnote} styles, unless
indicated otherwise.) Again, this command should only be redefined
before \cs{newabbreviation} (or \cs{newacronym}), and fragile or
formatting commands should be protected from expansion.

As from version 1.05, all the footnote styles use:
\begin{definition}[\DescribeMacro\glsfirstlongfootnotefont]
\cs{glsfirstlongfootnotefont}\marg{text}
\end{definition}
to format the long form on \gls{firstuse} or for the full form and
\begin{definition}[\DescribeMacro\glslongfootnotefont]
\cs{glslongfootnotefont}\marg{text}
\end{definition}
to format the long form elsewhere (for example, when used with
\cs{glsxtrlong}).

As from version 1.07, all the footnote styles use:
\begin{definition}[\DescribeMacro\glsxtrabbrvfootnote]
\cs{glsxtrabbrvfootnote}\marg{label}\marg{long}
\end{definition}
By default, this just does \ics{footnote}\marg{long} (the first
argument is ignored). For example, to make the footnote text
link to the relevant place in the glossary:
\begin{verbatim}
\renewcommand{\glsxtrabbrvfootnote}[2]{%
  \footnote{\glshyperlink[#2]{#1}}%
}
\end{verbatim}
or to include the short form with a hyperlink:
\begin{verbatim}
\renewcommand{\glsxtrabbrvfootnote}[2]{%
  \footnote{\glshyperlink[\glsfmtshort{#1}]{#1}: #2}%
}
\end{verbatim}
Note that I haven't used commands like \cs{glsxtrshort} to 
avoid interference (see \sectionref{sec:entryfmtmods}
and \sectionref{sec:nested}).

\item[\abbrstyledef{footnote}] A synonym for
\abbrstyle{short-footnote}.

\item[\abbrstyledef{short-sc-footnote}]
Like \abbrstyle{short-footnote} but redefines \cs{glsabbrvfont} to
use \ics{glsabbrvscfont}.
(This style was originally called \depabbrstyle{footnote-sc}. Renamed in version
1.04, but original name retained as a deprecated synonym for
backward-compatibility.)

\item[\abbrstyledef{short-sc-footnote}]
Like \abbrstyle{short-footnote} but redefines \cs{glsabbrvfont} to
use \ics{glsabbrvsmfont}.
(This style was originally called \depabbrstyle{footnote-sm}. Renamed in version
1.04, but original name retained as a deprecated synonym for
backward-compatibility.)

\item[\abbrstyledef{short-em-footnote}]
Like \abbrstyle{short-footnote} but redefines \cs{glsabbrvfont} to
use \ics{glsabbrvemfont}.
(This style was originally called \depabbrstyle{footnote-em}. Renamed in version
1.04, but original name retained as a deprecated synonym for
backward-compatibility.)

\item[\abbrstyledef{short-footnote-desc}]
(New to v1.42.) 
This is similar to \abbrstyle{short-footnote}, but the 
\gloskey{description} must be supplied by the user.
The \gloskey{name} field is obtained from:
\begin{definition}[\DescribeMacro\glsxtrfootnotedescname]
\cs{glsxtrfootnotedescname}
\end{definition}
which defaults to \meta{short} followed by \meta{long} in
parentheses, and the \gloskey{sort} field is obtained from:
\begin{definition}[\DescribeMacro\glsxtrfootnotedescsort]
\cs{glsxtrfootnotedescsort}
\end{definition}
which defaults to just the short form.  (Similarly for other
\texttt{short}\meta{modifier}\texttt{-[post]footnote}\texttt{-desc}
styles, unless indicated otherwise.)

Any redefinition of these commands must come before the abbreviations
are defined as it expands on definition. Make sure you protect
any commands that shouldn't be expanded. The long form can be
referenced with \cs{the}\cs{glslongtok} and the short form
can be referenced with \cs{the}\cs{glsshorttok}.

\item[\abbrstyledef{footnote-desc}]
A synonym for \abbrstyle{short-footnote-desc}.

\item[\abbrstyledef{short-sc-footnote-desc}]
Like \abbrstyle{short-footnote-desc} but redefines \cs{glsabbrvfont} to
use \ics{glsabbrvscfont}.

\item[\abbrstyledef{short-sm-footnote-desc}]
Like \abbrstyle{short-footnote-desc} but redefines \cs{glsabbrvfont} to
use \ics{glsabbrvsmfont}.

\item[\abbrstyledef{short-em-footnote-desc}]
Like \abbrstyle{short-footnote-desc} but redefines \cs{glsabbrvfont} to
use \ics{glsabbrvemfont}.

\item[\abbrstyledef{short-postfootnote}]
This is similar to the \abbrstyle{short-footnote} style but doesn't modify
the category attribute. Instead it changes \postlinkcat\ to 
insert the footnote after the
\gls{linktext} on \gls{firstuse}. This will also defer the footnote until after any
following punctuation character that's recognised by
\ics{glsxtrifnextpunc}.

The inline full form uses the \meta{short}
(\meta{long}) style. The \gloskey{name} is set to the short form.
The \gloskey{description} is set to the long form.
Note that this style will change \cs{glsxtrfull} (and its variants)
so that it fakes non-\gls{firstuse}. (Otherwise the footnote would
appear after the inline form.)

\item[\abbrstyledef{postfootnote}] A synonym for
\abbrstyle{short-postfootnote}.

\item[\abbrstyledef{short-sc-postfootnote}]
Like \abbrstyle{short-postfootnote} but redefines \cs{glsabbrvfont} to
use \ics{glsabbrvscfont}.
(This style was originally called \depabbrstyle{postfootnote-sc}. Renamed in version
1.04, but original name retained as a deprecated synonym for
backward-compatibility.)

\item[\abbrstyledef{short-sm-postfootnote}]
Like \abbrstyle{short-postfootnote} but redefines \cs{glsabbrvfont} to
use \ics{glsabbrvsmfont}.
(This style was originally called \depabbrstyle{postfootnote-sm}. Renamed in version
1.04, but original name retained as a deprecated synonym for
backward-compatibility.)

\item[\abbrstyledef{short-em-postfootnote}]
Like \abbrstyle{short-postfootnote} but redefines \cs{glsabbrvfont} to
use \ics{glsabbrvemfont}.
(This style was originally called \depabbrstyle{postfootnote-em}. Renamed in version
1.04, but original name retained as a deprecated synonym for
backward-compatibility.)

\item[\abbrstyledef{short-postfootnote-desc}]
(New to v1.42.) 
This is similar to \abbrstyle{short-postfootnote}, but the 
\gloskey{description} must be supplied by the user. The
\gloskey{name} and \gloskey{sort} are set as for
\abbrstyle{short-footnote-desc}.

\item[\abbrstyledef{postfootnote-desc}]
A synonym for \abbrstyle{short-postfootnote-desc}.

\item[\abbrstyledef{short-sc-postfootnote-desc}]
Like \abbrstyle{short-postfootnote-desc} but redefines \cs{glsabbrvfont} to
use \ics{glsabbrvscfont}.

\item[\abbrstyledef{short-sm-postfootnote-desc}]
Like \abbrstyle{short-postfootnote-desc} but redefines \cs{glsabbrvfont} to
use \ics{glsabbrvsmfont}.

\item[\abbrstyledef{short-em-postfootnote-desc}]
Like \abbrstyle{short-postfootnote-desc} but redefines \cs{glsabbrvfont} to
use \ics{glsabbrvemfont}.

\item[\abbrstyledef{short-postlong-user}]
This style was introduced in version 1.12. It's like the
\abbrstyle{short-long-user} style but defers the parenthetical
material to after the link-text. This means that you don't have such
a long hyperlink (which can cause problems for the DVI \LaTeX\ format)
and it also means that the user supplied material can include a
hyperlink to another location. The \gloskey{name} key is obtained
from \cs{glsxtrshortlongname}.

\item[\abbrstyledef{short-postlong-user-desc}]
This style was introduced in version 1.12. It's like the above
\abbrstyle{short-postlong-user} style but the \gloskey{description}
must be specified. The \gloskey{name} is obtained from
\cs{glsxtrshortlonguserdescname}.

\item[\abbrstyledef{long-postshort-user}]
This style was introduced in version 1.12. It's like the above
\abbrstyle{short-postlong-user} style but the long form is shown
first and the short form is in the parenthetical material (as for
\abbrstyle{long-short-user}) style.

\item[\abbrstyledef{long-postshort-user-desc}]
This style was introduced in version 1.12. It's like the above
\abbrstyle{long-postshort-user} style but the \gloskey{description}
must be specified. The \gloskey{name} is obtained from
\cs{glsxtrlongshortuserdescname}.

\item[\abbrstyledef{long-hyphen-short-hyphen}]
This style (new to v1.17) is designed to work with the \catattr{markwords}
category attribute. The full form is formatted using
\begin{definition}[\DescribeMacro\glsxtrlonghyphenshort]
\cs{glsxtrlonghyphenshort}\marg{label}\marg{long}\marg{short}\marg{insert}
\end{definition}
where \meta{insert} is the inserted material
provided in the final optional argument of commands like
\cs{insert}. If \meta{insert} start with a hyphen, then this 
locally redefines \ics{glsxtrwordsep} to a hyphen, which means that
if the \catattr{markwords} attribute is set then the long form
will become hyphenated. (If this attribute isn't set, there's
no alteration to the way the long form is displayed.)
The \gloskey{name} key is obtained from \cs{glsxtrlongshortname}.

Unlike the other \meta{long} (\meta{short}) type of styles,
such as \abbrstyle{long-short}, this style
also repeats the insertion in the parenthetical part, so that the
first use form is:
\begin{quote}
\cs{glsfirstlonghyphenfont}\marg{long}\meta{insert} 
(\cs{glsfirstabbrvhyphenfont}\marg{short}\meta{insert})
\end{quote}
The space before the parenthetical material is actually given by
\cs{glsxtrfullsep}\marg{label} which defaults to a space.
The \meta{insert} may be moved into the formatting commands 
according to the conditional \cs{ifglsxtrinsertinside}.

For example, if \texttt{ip} is defined using:
\begin{verbatim}
\glssetcategoryattribute{english}{markwords}{true}
\setabbreviationstyle[english]{long-hyphen-short-hyphen}
\newabbreviation[category=english]{ip}{IP}{Internet Protocol}
\end{verbatim}
then
\begin{verbatim}
\gls{ip}[-Adressen]
\end{verbatim}
will do 
\begin{quote}
Internet-Protocol-Adressen (IP-Adressen)
\end{quote}
on first use, whereas
\begin{verbatim}
\gls{ip}[ Address]
\end{verbatim}
will do
\begin{quote}
Internet Protocol Address (IP Address)
\end{quote}
on first use.

\begin{important}
Note that the hyphenation isn't applied when using commands like
\cs{glsxtrlong}. This means that
\begin{verbatim}
\glsxtrlong{ip}[-Adressen]
\end{verbatim}
will do
\begin{quote}
Internet Protocol-Adressen
\end{quote}
\end{important}

If the \catattr{markwords} attribute hadn't been set, then the
first use of 
\begin{verbatim}
\gls{ip}[-Adressen]
\end{verbatim}
would do 
\begin{quote}
Internet Protocol-Adressen (IP-Adressen)
\end{quote}
instead.

\begin{important}
If the inserted material \meta{insert} is likely to contain commands like
\cs{gls}, then use \abbrstyle{long-hyphen-postshort-hyphen}
instead to avoid nested links.
\end{important}

If you want the short version in small-caps, you can
just redefine \cs{glsabbrvhyphenfont} and \cs{glsxtrhyphensuffix}
to use the small-caps versions. For example:
\begin{verbatim}
\renewcommand{\glsabbrvhyphenfont}{\glsabbrvscfont}
\renewcommand{\glsxtrhyphensuffix}{\glsxtrscsuffix}
\end{verbatim}
Similarly for other font-changing variations.

\item[\abbrstyledef{long-hyphen-noshort-desc-noreg}]
New to version 1.17, this style is like
\abbrstyle{long-hyphen-short-hyphen-desc} except that the
parenthetical part is omitted and the long form is displayed
on subsequent use. The short form can be accessed
with \cs{glsxtrshort} but just uses the default abbreviation font.
There's no regular version of this style as the regular form
doesn't have the flexibility to deal with the hyphen switch.
The \gloskey{name} is obtained from \cs{glsxtrlongnoshortdescname}.

\item[\abbrstyledef{long-hyphen-noshort-noreg}]
New to version 1.17, this style is like
\abbrstyle{long-hyphen-noshort-desc-noreg} but the
name is set to the short form and the description is set to the long
form.

\item[\abbrstyledef{long-hyphen-short-hyphen-desc}]

New to version 1.17.
This is similar to \abbrstyle{long-hyphen-short-hyphen} but the user
supplies the description. The \gloskey{name} is obtained from
\cs{glsxtrlongshortdescname}.

\item[\abbrstyledef{long-hyphen-postshort-hyphen}]

New to version 1.17.
This is similar to \abbrstyle{long-hyphen-short-hyphen} but the
inserted and parenthetical material are moved to the post-link hook.
On first use, \cs{gls}\marg{label}\oarg{insert} will do
\begin{display}
\cs{glsxtrlonghyphen}\marg{long}\marg{label}\marg{insert}\cs{glsxtrposthyphenshort}\marg{label}\meta{insert}
\end{display}
where
\begin{definition}[\DescribeMacro\glsxtrposthyphenshort]
\cs{glsxtrposthyphenshort}\marg{label}\meta{insert}
\end{definition}
is in the post-link hook. This uses the format:
\begin{display}
\meta{insert} (\cs{glsfirstabbrvhyphenfont}\marg{short}\meta{isnert})
\end{display}

\begin{important}
The singular short form is always used here, even with \cs{glspl}.
(Unlike \abbrstyle{long-hyphen-long-hyphen}.)
\end{important}

The part in the link-text on first use:
\begin{definition}[\DescribeMacro\glsxtrlonghyphen]
\cs{glsxtrlonghyphen}\marg{long}\marg{label}\marg{insert}
\end{definition}
checks if \meta{insert} starts with a hyphen. If it does, then
\cs{glsxtrwordsep} is locally redefined to a hyphen. This command
only uses \meta{insert} to test if it starts with a hyphen.
The actual insertion code isn't typeset until the post-link
hook and it's also localised, which means that you can use 
commands like \cs{gls} in \meta{insert} for this style without
causing nested hyperlinks, but only for commands like \cs{gls}.

\begin{important}
Don't use \cs{gls} in the \meta{insert} part for commands
like \cs{glsxtrfull}, \cs{glsxtrshort} or \cs{glsxtrlong}.
\end{important}

The inline full display format used by commands like \cs{glsxtrfull}
behaves differently to the first use of \cs{gls} with this style.
It's better to use \cs{glsreset}\marg{label}\cs{gls}\marg{label}
if you want to ensure the full format.

\item[\abbrstyledef{long-hyphen-postshort-hyphen-desc}]

New to version 1.17.
This is similar to \abbrstyle{long-hyphen-postshort-hyphen} but the
user supplies the description. The \gloskey{name} is obtained from
\cs{glsxtrlongshortdescname}.


\item[\abbrstyledef{short-hyphen-long-hyphen}]
This style (new to v1.17) is like
\abbrstyle{long-hyphen-short-hyphen}, except that the short form
is displayed first followed by the long form in parentheses. The full form is formatted using
\begin{definition}[\DescribeMacro\glsxtrshorthyphenlong]
\cs{glsxtrshorthyphenlong}\marg{label}\marg{short}\marg{long}\marg{insert}
\end{definition}
which behaves in an analogous way to \cs{glsxtrlonghyphenshort}.
The \gloskey{name} is obtained from \cs{glsxtrshortlongname}.

\item[\abbrstyledef{short-hyphen-long-hyphen-desc}]

New to version 1.17.
This is similar to \abbrstyle{short-hyphen-long-hyphen} but the
user supplies the description.
The \gloskey{name} is obtained from \cs{glsxtrshortlongdescname}.

\item[\abbrstyledef{short-hyphen-postlong-hyphen}]
This style (new to v1.17) is like
\abbrstyle{long-hyphen-postshort-hyphen}, but the 
short form is displayed first followed by the long form in 
parentheses.
On first use, \cs{gls}\marg{label}\oarg{insert} will do
\begin{display}
\cs{glsxtrshorthyphen}\marg{short}\marg{label}\marg{insert}\cs{glsxtrposthyphenlong}\marg{label}\meta{insert}
\end{display}
where
\begin{definition}[\DescribeMacro\glsxtrposthyphenlong]
\cs{glsxtrposthyphenlong}\marg{label}\meta{insert}
\end{definition}
is in the post-link hook. These commands behave in an analogous
manner to those used with \abbrstyle{long-hyphen-postshort-hyphen}.
The \gloskey{name} is obtained from \cs{glsxtrshortlongname}.

\begin{important}
Don't use \cs{gls} in the \meta{insert} part for commands
like \cs{glsxtrfull}, \cs{glsxtrshort} or \cs{glsxtrlong}.
\end{important}

The inline full display format used by commands like \cs{glsxtrfull}
behaves differently to the first use of \cs{gls} with this style.
It's better to use \cs{glsreset}\marg{label}\cs{gls}\marg{label}
if you want to ensure the full format.

\item[\abbrstyledef{short-hyphen-postlong-hyphen-desc}]

New to version 1.17.
This is similar to \abbrstyle{short-hyphen-postlong-hyphen} but the
user supplies the description. The \gloskey{name} is obtained from
\cs{glsxtrshortlongdescname}.

\end{description}

\section{Defining New Abbreviation Styles}
\label{sec:newabbrvstyle}

New abbreviation styles may be defined using:
\begin{definition}[\DescribeMacro\newabbreviationstyle]
\cs{newabbreviationstyle}\marg{name}\marg{setup}\marg{fmts}
\end{definition}
\begin{sloppypar}\noindent
where \meta{name} is the name of the new style (as used in the
mandatory argument of \ics{setabbreviationstyle}). This is similar
but not identical to the \styfmt{glossaries} package's
\ics{newacronymstyle} command.
\end{sloppypar}

\begin{important}
You can't use styles defined by
\cs{newacronymstyle} with \styfmt{glossaries-extra} unless you have
reverted \ics{newacronym} back to its generic definition from \styfmt{glossaries}
(using \ics{RestoreAcronyms}). The acronym styles from the
\styfmt{glossaries} package can't be used with abbreviations defined
with \cs{newabbreviation}.
\end{important}

The \meta{setup} argument deals with the way the entry is defined
and may set attributes for the given abbreviation 
category. This argument should redefine
\begin{definition}[\DescribeMacro\CustomAbbreviationFields]
\cs{CustomAbbreviationFields}
\end{definition}
to set the entry fields including the \gloskey{name} (defaults to
the short form if omitted),
\gloskey{sort}, \gloskey{first}, \gloskey{firstplural}. Other fields
may also be set, such as \gloskey{text}, \gloskey{plural} and
\gloskey{description}.

\begin{important}
\cs{CustomAbbreviationFields} is expanded by \cs{newabbreviation}
so take care to protect commands that shouldn't be expanded.
\end{important}

For example, the \abbrstyle{long-short} style has the following in
\meta{setup}:
\begin{verbatim}
  \renewcommand*{\CustomAbbreviationFields}{%
    name={\protect\glsabbrvfont{\the\glsshorttok}},
    sort={\the\glsshorttok},
    first={\protect\glsfirstlongfont{\the\glslongtok}%
     \protect\glsxtrfullsep{\the\glslabeltok}%
     \glsxtrparen{\protect\glsfirstabbrvfont{\the\glsshorttok}}},%
    firstplural={\protect\glsfirstlongfont{\the\glslongpltok}%
     \protect\glsxtrfullsep{\the\glslabeltok}%
     \glsxtrparen{\protect\glsfirstabbrvfont{\the\glsshortpltok}}},%
    plural={\protect\glsabbrvfont{\the\glsshortpltok}},%
    description={\the\glslongtok}}%
\end{verbatim}
Note that the \gloskey{first} and \gloskey{firstplural} are
set even though they're not used by \cs{gls}. 

The basic styles, such as \abbrstyle{long-short}, use commands like
\cs{glsabbrvfont} (which are redefined whenever the style formatting
is set) within \cs{CustomAbbreviationFields}. Other styles, such as
\abbrstyle{long-em-short-em} directly use their own custom commands,
such as \cs{glsabbrvemfont}. With these styles, commands like
\cs{glsabbrvfont} still need to be defined as appropriate in the
\meta{fmts} argument even if they're not used within 
\cs{CustomAbbreviationFields}.

The \meta{setup} argument may also redefine
\begin{definition}[\DescribeMacro\GlsXtrPostNewAbbreviation]
\cs{GlsXtrPostNewAbbreviation}
\end{definition}
which can be used to assign attributes. (This will automatically
be initialised to do nothing.)

For example, the \abbrstyle{short-footnote} includes the following in
\meta{setup}:
\begin{verbatim}
  \renewcommand*{\GlsXtrPostNewAbbreviation}{%
    \glssetattribute{\the\glslabeltok}{nohyperfirst}{true}%
    \glshasattribute{\the\glslabeltok}{regular}%
    {%
      \glssetattribute{\the\glslabeltok}{regular}{false}%
    }%
    {}%
  }%
\end{verbatim}
This sets the \catattr{nohyperfirst} attribute to \qt{true}.
It also unsets the \catattr{regular} attribute if it has
previously been set. Note that the \catattr{nohyperfirst}
attribute doesn't get unset by other styles, so take care
not to switch styles for the same category.

You can access the short, long, short plural and long plural
values through the following token registers.

Short value (defined by \styfmt{glossaries}):
\begin{definition}[\DescribeMacro\glsshorttok]
\cs{glsshorttok}
\end{definition}

Short plural value (defined by \styfmt{glossaries-extra}):
\begin{definition}[\DescribeMacro\glsshortpltok]
\cs{glsshortpltok}
\end{definition}
(This may be the default value or, if provided, the value provided by the user
through the \gloskey{shortplural} key in the optional argument
of \ics{newabbreviation}.)

Long value (defined by \styfmt{glossaries}):
\begin{definition}[\DescribeMacro\glslongtok]
\cs{glslongtok}
\end{definition}

Long plural value (defined by \styfmt{glossaries-extra}):
\begin{definition}[\DescribeMacro\glslongpltok]
\cs{glslongpltok}
\end{definition}
(This may be the default value or, if provided, the value provided by the user
through the \gloskey{longplural} key in the optional argument
of \ics{newabbreviation}.)

The short or long values may be modified by attributes (such as
\catattr{markwords}). The above registers reflect the modification.
If you want to access the original (unmodified) short or long form
(as provided in the final two arguments of \cs{newabbreviation}),
then use the commands:
\begin{definition}[\DescribeMacro\glsxtrorgshort]
\cs{glsxtrorgshort}
\end{definition}
for the short form and
\begin{definition}[\DescribeMacro\glsxtrorglong]
\cs{glsxtrorglong}
\end{definition}
for the long form. (These may be useful for the \gloskey{sort} key
to avoid any formatting that may be added by the attribute setting.)

There are two other registers available that are defined by
\styfmt{glossaries}:

\begin{definition}[\DescribeMacro\glslabeltok]
\cs{glslabeltok}
\end{definition}
which contains the entry's label and
\begin{definition}[\DescribeMacro\glskeylisttok]
\cs{glskeylisttok}
\end{definition}
which contains the values provided in the optional argument
of \ics{newabbreviation}.

Remember put \cs{the} in front of the register command as in the
examples above. The category label can be access through
the command (not a register):
\begin{definition}[\DescribeMacro\glscategorylabel]
\cs{glscategorylabel}
\end{definition}
This may be used inside the definition of
\cs{GlsXtrPostNewAbbreviation}.

If you want to base a style on an existing style, you can use
\begin{definition}[\DescribeMacro\GlsXtrUseAbbrStyleSetup]
\cs{GlsXtrUseAbbrStyleSetup}\marg{name}
\end{definition}
where \meta{name} is the name of the existing style.
For example, the \abbrstyle{long-noshort-sc-desc} style simply does
\begin{verbatim}
\GlsXtrUseAbbrStyleSetup{long-noshort-desc}
\end{verbatim}
within \meta{setup}.

The \meta{fmts} argument deals with the way the entry is displayed
in the document. This argument should redefine the following
commands.

The default suffix for the plural short form (if not overridden by
the \gloskey{shortplural} key):
\begin{definition}[\DescribeMacro\abbrvpluralsuffix]
\cs{abbrvpluralsuffix}
\end{definition}
(Note that this isn't used for the plural long form, which
just uses the regular \cs{glspluralsuffix}.)

The font used for the short form on \gls{firstuse} or in the full forms:
\begin{definition}[\DescribeMacro\glsfirstabbrvfont]
\cs{glsfirstabbrvfont}\marg{text}
\end{definition}

The font used for the short form on subsequent use or through
commands like \ics{glsxtrshort}:
\begin{definition}[\DescribeMacro\glsabbrvfont]
\cs{glsabbrvfont}\marg{text}
\end{definition}

The font used for the long form on \gls{firstuse} or in the full forms:
\begin{definition}[\DescribeMacro\glsfirstlongfont]
\cs{glsfirstlongfont}\marg{text}
\end{definition}

The font used for the long form in commands like \cs{glsxtrlong}
use:
\begin{definition}[\DescribeMacro\glslongfont]
\cs{glslongfont}\marg{text}
\end{definition}

Display full form singular no case-change (used by \ics{gls} on
\gls{firstuse} for abbreviations without the \catattr{regular} attribute
set):
\begin{definition}[\DescribeMacro\glsxtrfullformat]
\cs{glsxtrfullformat}\marg{label}\marg{insert}
\end{definition}

Display full form singular first letter converted to upper case 
(used by \ics{Gls} on \gls{firstuse} for abbreviations without the 
\catattr{regular} attribute set):
\begin{definition}[\DescribeMacro\Glsxtrfullformat]
\cs{Glsxtrfullformat}\marg{label}\marg{insert}
\end{definition}

Display full form plural no case-change 
(used by \ics{glspl} on \gls{firstuse} for abbreviations without the 
\catattr{regular} attribute set):
\begin{definition}[\DescribeMacro\glsxtrfullplformat]
\cs{glsxtrfullplformat}\marg{label}\marg{insert}
\end{definition}

Display full form plural first letter converted to upper case
(used by \ics{Glspl} on \gls{firstuse} for abbreviations without the 
\catattr{regular} attribute set):
\begin{definition}[\DescribeMacro\Glsxtrfullplformat]
\cs{Glsxtrfullplformat}\marg{label}\marg{insert}
\end{definition}

In addition \meta{fmts} may also redefine the following commands that 
govern the inline full formats. If the style doesn't redefine them,
they will default to the same as the display full forms.

Inline singular no case-change (used by 
\ics{glsentryfull}, \ics{glsxtrfull} and \ics{GLSxtrfull}):
\begin{definition}[\DescribeMacro\glsxtrinlinefullformat]
\cs{glsxtrinlinefullformat}\marg{label}\marg{insert}
\end{definition}

Inline singular first letter converted to upper case (used by 
\ics{Glsentryfull} and \ics{Glsxtrfull}):
\begin{definition}[\DescribeMacro\Glsxtrinlinefullformat]
\cs{Glsxtrinlinefullformat}\marg{label}\marg{insert}
\end{definition}

Inline plural no case-change (used by 
\ics{glsentryfullpl}, \ics{glsxtrfullpl} and \ics{GLSxtrfullpl}):
\begin{definition}[\DescribeMacro\glsxtrinlinefullplformat]
\cs{glsxtrinlinefullplformat}\marg{label}\marg{insert}
\end{definition}

Inline plural first letter converted to upper case (used by 
\ics{Glsentryfullpl} and \ics{Glsxtrfullpl}):
\begin{definition}[\DescribeMacro\Glsxtrinlinefullplformat]
\cs{Glsxtrinlinefullplformat}\marg{label}\marg{insert}
\end{definition}

(New to version 1.17.)
You can also modify the way the subsequent use is formatted by
redefining the following four commands, but these won't be used for
abbreviations with the \catattr{regular} attribute set. If the style
doesn't redefine these commands, the default values are used.

Singular with no case-change:
\begin{definition}[\DescribeMacro\glsxtrsubsequentfmt]
\cs{glsxtrsubsequentfmt}\marg{label}\marg{insert}
\end{definition}

Singular with first letter upper case:
\begin{definition}[\DescribeMacro\Glsxtrsubsequentfmt]
\cs{Glsxtrsubsequentfmt}\marg{label}\marg{insert}
\end{definition}

Plural with no case-change:
\begin{definition}[\DescribeMacro\glsxtrsubsequentplfmt]
\cs{glsxtrsubsequentplfmt}\marg{label}\marg{insert}
\end{definition}

Plural with first letter upper case:
\begin{definition}[\DescribeMacro\Glsxtrsubsequentplfmt]
\cs{Glsxtrsubsequentplfmt}\marg{label}\marg{insert}
\end{definition}

If you want to provide support for \sty{glossaries-accsupp}
use the following \cs{glsaccess\meta{xxx}} commands
(\sectionref{sec:accsupp}) within the definitions of
\cs{glsxtrfullformat} etc instead of the analogous
\cs{glsentry}\meta{xxx} commands. (If you don't use
\sty{glossaries-accsupp}, they will just do the corresponding
\cs{glsentry}\meta{xxx} command.)

For example, the \abbrstyle{short-long} style has the following in
\meta{fmts}:
\begin{verbatim}
  \renewcommand*{\abbrvpluralsuffix}{\glsxtrabbrvpluralsuffix}%
  \renewcommand*{\glsabbrvfont}[1]{\glsabbrvdefaultfont{##1}}%
  \renewcommand*{\glsfirstabbrvfont}[1]{\glsfirstabbrvdefaultfont{##1}}%
  \renewcommand*{\glsfirstlongfont}[1]{\glsfirstlongdefaultfont{##1}}%
  \renewcommand*{\glslongfont}[1]{\glslongdefaultfont{##1}}%
  \renewcommand*{\glsxtrfullformat}[2]{%
    \glsfirstabbrvfont{\glsaccessshort{##1}\ifglsxtrinsertinside##2\fi}%
    \ifglsxtrinsertinside\else##2\fi
    \glsxtrfullsep{##1}%
    \glsxtrparen{\glsfirstlongfont{\glsaccesslong{##1}}}%
  }%
  \renewcommand*{\glsxtrfullplformat}[2]{%
    \glsfirstabbrvfont{\glsaccessshortpl{##1}\ifglsxtrinsertinside##2\fi}%
    \ifglsxtrinsertinside\else##2\fi
    \glsxtrfullsep{##1}%
    \glsxtrparen{\glsfirstlongfont{\glsaccesslongpl{##1}}}%
  }%
  \renewcommand*{\Glsxtrfullformat}[2]{%
    \glsfirstabbrvfont{\Glsaccessshort{##1}\ifglsxtrinsertinside##2\fi}%
    \ifglsxtrinsertinside\else##2\fi\glsxtrfullsep{##1}%
    \glsxtrparen{\glsfirstlongfont{\glsaccesslong{##1}}}%
  }%
  \renewcommand*{\Glsxtrfullplformat}[2]{%
    \glsfirstabbrvfont{\Glsaccessshortpl{##1}\ifglsxtrinsertinside##2\fi}%
     \ifglsxtrinsertinside\else##2\fi\glsxtrfullsep{##1}%
    \glsxtrparen{\glsfirstlongfont{\glsaccesslongpl{##1}}}%
  }%
\end{verbatim}
Since the inline full commands aren't redefined, they default
to the same as the display versions.

If you want to base a style on an existing style, you can use
\begin{definition}[\DescribeMacro\GlsXtrUseAbbrStyleFmts]
\cs{GlsXtrUseAbbrStyleFmts}\marg{name}
\end{definition}
within \meta{fmts}, where \meta{name} is the name of the existing
style. For example, the \abbrstyle{long-short-desc} style has the
following in \meta{fmts}:
\begin{verbatim}
  \GlsXtrUseAbbrStyleFmts{long-short}%
\end{verbatim}

Here's an example of an abbreviation style that's based on
\abbrstyle{long-short} that displays the short form within
\cs{textsf}:
\begin{verbatim}
\newabbreviationstyle
{custom-sf}% label
{% setup
  \GlsXtrUseAbbrStyleSetup{short-long}%
}%
{% fmts
  \GlsXtrUseAbbrStyleFmts{short-long}%
  \renewcommand*{\glsabbrvfont}[1]{\textsf{##1}}%
}
\end{verbatim}
Note that this wouldn't work if it was instead based on one of the
modified versions such as \abbrstyle{short-sc-long} as they
explicitly use their own formatting commands, such as
\cs{glsabbrvemfont}. The base styles, such as \abbrstyle{short-long},
use the more generic \cs{glsabbrvfont} etc which makes them easier
to adapt than the modified styles.

For further details, see the \qt{Abbreviations} section in the documented code
(\nolinkurl{glossaries-extra-code.pdf}).

\chapter{Entries in Sectioning Titles, Headers, Captions and Contents}
\label{sec:headtitle}

The \styfmt{glossaries} user manual cautions against using commands
like \cs{gls} in chapter or section titles. The principle problems
are:
\begin{itemize}
\item if you have a table of contents, the \gls{firstuseflag} 
will be unset in the contents rather than later in the document;
\item if you have the location lists displayed in the glossary,
unwanted locations will be added to it corresponding to the table of
contents (if present) and every page that contains the entry in the
page header (if the page style in use adds the chapter or section
title to the header);
\item if the page style in use adds the chapter or section title to
the header and attempts to convert it to upper case, the entry label
(in the argument of \cs{gls} etc) will be converted to upper case
and the entry won't be recognised;
\item if you use \sty{hyperref}, commands like \cs{gls} can't be
expanded to a simple string and only the label will appear in the
PDF bookmark (with a warning from \sty{hyperref});
\item if you use \sty{hyperref}, you will end up with nested hyperlinks
in the table of contents.
\end{itemize}
Similar problems can also occur with captions (except for the page
header and bookmark issues).

\section{Simplistic Approach}

To get around all these problems, the \styfmt{glossaries} user manual
recommends using the expandable non-hyperlink commands, such as 
\ics{glsentrytext} (for regular entries) or \ics{glsentryshort}
(for abbreviations). This is the simplest solution, but doesn't
allow for special formatting that's applied to the entry through
commands like \cs{glstext} or \cs{glsxtrshort}. This means that if,
for example, you are using one of the abbreviation styles that uses
\ics{textsc} then the short form displayed with \cs{glsentryshort} won't use small
caps. If you only have one abbreviation style in use, you can
explicitly enclose \cs{glsentryshort}\marg{label} in the argument of
\cs{glsabbrvfont}, like this:
\begin{verbatim}
\chapter{A Chapter about \glsabbrvfont{\glsentryshort{html}}}
\end{verbatim}
Or, if you are using \sty{hyperref}:
\begin{verbatim}
\chapter{A Chapter about 
\texorpdfstring{\glsabbrvfont{\glsentryshort{html}}}{\glsentryshort{html}}}
\end{verbatim}

Since this is a bit cumbersome, you might want to define a new
command to do this for you. However, if you have mixed styles this
won't work as commands like \cs{gls} and \cs{glsxtrshort} redefine
\cs{glsabbrvfont} to match the entry's style before displaying it.
In this case, the above example doesn't take into account the
shifting definitions of \cs{glsabbrvfont} and will use whatever
happens to be the last abbreviation style in use. More complicated
solutions interfere with the upper casing used by the standard
page styles that display the chapter or section title in the page
header using \ics{MakeUppercase}.

The \styfmt{glossaries-extra} package tries to resolve this by
modifying \ics{markright} and \ics{markboth} and \cs{@starttoc}. If you don't like this
change, you can restore their former definitions using
\begin{definition}[\DescribeMacro\glsxtrRevertMarks]
\cs{glsxtrRevertMarks}
\end{definition}
If you only want to restore \cs{@starttoc} you can use:
\begin{definition}[\DescribeMacro\glsxtrRevertTocMarks]
\cs{glsxtrRevertTocMarks}
\end{definition}
If you restore the header or table of contents commands, you'll have to use the \styfmt{glossaries} manual's
recommendations of either simply using \cs{glsentryshort} (as above)
or use the sectioning command's option argument to provide an
alternative for the table of contents and page header. For example:
\begin{verbatim}
\chapter[A Chapter about \glsentryshort{html}]{A Chapter about \gls{html}}
\end{verbatim}
Alternatively, you need to find a way to insert \cs{glsxtrmarkhook} and
\cs{@glsxtrinmark} at the start of the header or table of contents
either scoped or afterwards cancelled with \cs{@glsxtrnotinmark} and
\cs{glsxtrrestoremarkhook}.

If you don't revert the mark commands back with
\cs{glsxtrRevertMarks}, you can use the commands described below in
the argument of sectioning commands. You can still use them even
if the mark commands have been reverted, but only where they don't
conflict with the page style.

\section{New Commands Designed for Chapter/Section Headings}

The commands listed below are designed for use in chapter or section
headings. There are still limitations, but they provide a better
solution. They all use \ics{texorpdfstring} if \sty{hyperref} has
been loaded so that the expandable non-formatted version is added to
the PDF bookmarks. Note that since the commands that convert the
first letter to upper case aren't expandable, the non-case-changing
version is used for the bookmarks. If the required field contains
non-expandable (robust or fragile) commands that cause a problem for
the bookmarks then \ics{texorpdfstring} will be needed as
appropriate in that field.  (Take care if the field has its value
expanded before being assigned.)

These commands essentially behave as though you have used
\cs{glsxtrshort}, \cs{glstext} etc with the options
\gloskey[glslink]{noindex} and \gloskey[glslink]{hyper\eq false}.
The text produced won't be converted to upper case in the page
headings by default. If you want the text converted to upper case
you need to set the \catattr{headuc} attribute to \qt{true}
for the appropriate category.

\begin{important}
If you use one of the \ics{textsc} styles, be aware that the default
fonts don't provide bold small-caps or italic small-caps. This
means that if the chapter or section title style uses bold, this
may override the small-caps setting, in which case the abbreviation
will just appear as lower case bold. If the heading style uses
italic, the abbreviation may appear in upright small-caps,
\emph{even if you have set the \catattr{headuc} attribute} since
the all-capitals form still uses \cs{glsabbrvfont}.
You may want to consider using the \sty{slantsc} package in this
case.
\end{important}

Display the short form:
\begin{definition}[\DescribeMacro\glsfmtshort]
\cs{glsfmtshort}\marg{label}
\end{definition}

Display the plural short form:
\begin{definition}[\DescribeMacro\glsfmtshortpl]
\cs{glsfmtshortpl}\marg{label}
\end{definition}

First letter upper case singular short form:
\begin{definition}[\DescribeMacro\Glsfmtshort]
\cs{Glsfmtshort}\marg{label}
\end{definition}
(No case-change applied to PDF bookmarks.)

First letter upper case plural short form:
\begin{definition}[\DescribeMacro\Glsfmtshortpl]
\cs{Glsfmtshortpl}\marg{label}
\end{definition}
(No case-change applied to PDF bookmarks.)

All caps singular short form:
\begin{definition}[\DescribeMacro\GLSfmtshort]
\cs{GLSfmtshort}\marg{label}
\end{definition}
(No case-change applied to PDF bookmarks.)

All caps plural short form:
\begin{definition}[\DescribeMacro\Glsfmtshortpl]
\cs{Glsfmtshortpl}\marg{label}
\end{definition}
(No case-change applied to PDF bookmarks.)

Display the long form:
\begin{definition}[\DescribeMacro\glsfmtlong]
\cs{glsfmtlong}\marg{label}
\end{definition}

Display the plural long form:
\begin{definition}[\DescribeMacro\glsfmtlongpl]
\cs{glsfmtlongpl}\marg{label}
\end{definition}

First letter upper case singular long form:
\begin{definition}[\DescribeMacro\Glsfmtlong]
\cs{Glsfmtlong}\marg{label}
\end{definition}
(No case-change applied to PDF bookmarks.)

First letter upper case plural long form:
\begin{definition}[\DescribeMacro\Glsfmtlongpl]
\cs{Glsfmtlongpl}\marg{label}
\end{definition}
(No case-change applied to PDF bookmarks.)

All caps singular long form:
\begin{definition}[\DescribeMacro\GLSfmtlong]
\cs{GLSfmtlong}\marg{label}
\end{definition}
(No case-change applied to PDF bookmarks.)

All caps plural long form:
\begin{definition}[\DescribeMacro\GLSfmtlongpl]
\cs{GLSfmtlongpl}\marg{label}
\end{definition}
(No case-change applied to PDF bookmarks.)

There are similar commands for the full form, but note that these
use the \emph{inline} full form, which may be different from the
full form used by \cs{gls}. The PDF version has to be a simple
fully expandable command, so one of two commands that are unrelated to the style
is used instead:
\begin{definition}[\DescribeMacro\glspdffmtfull]
\cs{glspdffmtfull}\marg{label}
\end{definition}
for the singular form or:
\begin{definition}[\DescribeMacro\glspdffmtfullpl]
\cs{glspdffmtfullpl}\marg{label}
\end{definition}
for the full form. These simply do the long form followed by the
short form in parentheses.

Display the full form:
\begin{definition}[\DescribeMacro\glsfmtfull]
\cs{glsfmtfull}\marg{label}
\end{definition}

Display the plural full form:
\begin{definition}[\DescribeMacro\glsfmtfullpl]
\cs{glsfmtfullpl}\marg{label}
\end{definition}

First letter upper case singular full form:
\begin{definition}[\DescribeMacro\Glsfmtfull]
\cs{Glsfmtfull}\marg{label}
\end{definition}
(No case-change applied to PDF bookmarks.)

First letter upper case plural full form:
\begin{definition}[\DescribeMacro\Glsfmtfullpl]
\cs{Glsfmtfullpl}\marg{label}
\end{definition}
(No case-change applied to PDF bookmarks.)

All caps singular full form:
\begin{definition}[\DescribeMacro\GLSfmtfull]
\cs{GLSfmtfull}\marg{label}
\end{definition}
(No case-change applied to PDF bookmarks.)

All caps plural full form:
\begin{definition}[\DescribeMacro\GLSfmtfullpl]
\cs{GLSfmtfullpl}\marg{label}
\end{definition}
(No case-change applied to PDF bookmarks.)

There are also equivalent commands for the value of the
\gloskey{text} field:
\begin{definition}[\DescribeMacro\glsfmttext]
\cs{glsfmttext}\marg{label}
\end{definition}

First letter converted to upper case:
\begin{definition}[\DescribeMacro\Glsfmttext]
\cs{Glsfmttext}\marg{label}
\end{definition}
(No case-change applied to PDF bookmarks.)

All caps:
\begin{definition}[\DescribeMacro\GLSfmttext]
\cs{GLSfmttext}\marg{label}
\end{definition}
(No case-change applied to PDF bookmarks.)

The plural equivalents:
\begin{definition}[\DescribeMacro\glsfmtplural]
\cs{glsfmtplural}\marg{label}
\end{definition}
First letter upper case:
\begin{definition}[\DescribeMacro\Glsfmtplural]
\cs{Glsfmtplural}\marg{label}
\end{definition}
and all caps:
\begin{definition}[\DescribeMacro\GLSfmtplural]
\cs{GLSfmtplural}\marg{label}
\end{definition}

Likewise for the value of the
\gloskey{name} field:
\begin{definition}[\DescribeMacro\glsfmtname]
\cs{glsfmtname}\marg{label}
\end{definition}

First letter converted to upper case:
\begin{definition}[\DescribeMacro\Glsfmtname]
\cs{Glsfmtname}\marg{label}
\end{definition}
(No case-change applied to PDF bookmarks.)

All caps:
\begin{definition}[\DescribeMacro\GLSfmtname]
\cs{GLSfmtname}\marg{label}
\end{definition}
(No case-change applied to PDF bookmarks.)

Similarly for the value of the
\gloskey{first} field:
\begin{definition}[\DescribeMacro\glsfmtfirst]
\cs{glsfmtfirst}\marg{label}
\end{definition}

First letter converted to upper case:
\begin{definition}[\DescribeMacro\Glsfmtfirst]
\cs{Glsfmtfirst}\marg{label}
\end{definition}
(No case-change applied to PDF bookmarks.)

All caps:
\begin{definition}[\DescribeMacro\GLSfmtfirst]
\cs{GLSfmtfirst}\marg{label}
\end{definition}
(No case-change applied to PDF bookmarks.)

The plural equivalents:
\begin{definition}[\DescribeMacro\glsfmtfirstpl]
\cs{glsfmtfirstpl}\marg{label}
\end{definition}
First letter upper case:
\begin{definition}[\DescribeMacro\Glsfmtfirstpl]
\cs{Glsfmtfirstpl}\marg{label}
\end{definition}
and all caps:
\begin{definition}[\DescribeMacro\GLSfmtfirstpl]
\cs{GLSfmtfirstpl}\marg{label}
\end{definition}

\chapter{Categories}
\label{sec:categories}

Each entry defined by \ics{newglossaryentry} (or commands that
internally use it such as \ics{newabbreviation}) 
is assigned a category
through the \gloskey{category} key. You may add any category that
you like, but since the category is a label used in the creation
of some control sequences, avoid problematic characters within the
category label. (So take care if you have \sty{babel} shorthands on
that make some characters active.)

The use of categories can give you more control over the way entries
are displayed in the text or glossary. Note that an entry's category
is independent of the glossary type. Be careful not to confuse
\gloskey{category} with \gloskey{type}.

The default category assumed by
\cs{newglossaryentry} is labelled \category{general}. Abbreviations
defined with \cs{newabbreviation} have the category set to
\category{abbreviation} by default. Abbreviations defined with
\cs{newacronym} have the category set to \category{acronym} by
default.

Additionally, if you have enabled \ics{newterm} with the
\pkgopt{index} package option that command
will set the category to \category{index} by default. If you have
enabled \ics{glsxtrnewsymbol} with the \pkgopt{symbols} package
option, that command will set the category to \category{symbol}. If you have
enabled \ics{glsxtrnewnumber} with the \pkgopt{numbers} package
option, that command will set the category to \category{number}.

You can obtain the category label for a given entry using
\begin{definition}[\DescribeMacro\glscategory]
\cs{glscategory}\marg{label}
\end{definition}
This is equivalent to commands like \cs{glsentryname} and so may be
used in an expandable context. No error is generated if the entry
doesn't exist.

You can test the category for a given entry using
\begin{definition}[\DescribeMacro\glsifcategory]
\cs{glsifcategory}\marg{entry-label}\marg{category-label}\marg{true
part}\marg{false part}
\end{definition}
This is equivalent to
\begin{alltt}
\ics{ifglsfieldeq}\marg{entry-label}\{category\}\marg{category-label}\marg{true
part}\marg{false part}
\end{alltt}
so any restrictions that apply to \cs{ifglsfieldeq} also apply to
\cs{glsifcategory}.

Each category may have a set of attributes. For example, the
\category{general} and \category{acronym} categories have the
attribute \catattr{regular} set to \qt{true} to indicate that all
entries with either of those categories are regular entries (as
opposed to abbreviations). This attribute is accessed by
\cs{glsentryfmt} to determine whether to use \cs{glsgenentryfmt} or
\cs{glsxtrgenabbrvfmt}.

Other attributes recognised by \styfmt{glossaries-extra} are:

\begin{description}

\item[\catattrdef{nohyperfirst}] When using commands like \ics{gls}
this will automatically suppress the hyperlink on \gls{firstuse} for entries with a
category that has this attribute set to \qt{true}.
(This settings can be overridden by explicitly setting
the \gloskey[glslink]{hyper} key on or off in the optional argument of
commands like \cs{gls}.) As from version 1.07, \cs{glsfirst},
\cs{Glsfirst}, \cs{GLSfirst} and their plural versions (which should ideally
behave in a similar way to the \gls{firstuse} of \cs{gls} or
\cs{glspl}) now honour
this attribute (but not the package-wide \pkgopt[false]{hyperfirst}
option, which matches the behaviour of \sty{glossaries}). If you
want commands like \cs{glsfirst} to ignore the
\catattr{nohyperfirst} attribute then just redefine
\begin{definition}[\DescribeMacro\glsxtrchecknohyperfirst]
\cs{glsxtrchecknohyperfirst}\marg{label}
\end{definition}
to do nothing.

\item[\catattrdef{nohyper}] When using commands like \ics{gls}
this will automatically suppress the hyperlink for entries with a
category that has this attribute set to \qt{true}.
(This settings can be overridden by explicitly setting
the \gloskey[glslink]{hyper} key on or off in the optional argument of
commands like \cs{gls}.)

\item[\catattrdef{indexonlyfirst}] This is similar to the 
\pkgopt{indexonlyfirst} package option but only for entries
that have a category with this attribute set to \qt{true}.

\item[\catattrdef{wrgloss}] When using commands like \ics{gls},
if this attribute is set to \qt{after}, it will automatically
implement \gloskey[glslink]{wrgloss}\optfmt{=after}. (New to v1.14.)

\item[\catattrdef{discardperiod}] If set to \qt{true}, the
post-\gls{linktext} hook will
discard a~full stop (period) that follows \emph{non-plural} commands like \cs{gls} or
\cs{glstext}. (Provided for entries such as abbreviations that end
with a full stop.) This attribute is only applicable to entries
defined using \cs{newabbreviation} (or \cs{newacronym} if it's using
\cs{newabbreviation}.) This attribute
doesn't apply to the accessibility fields.

Note that this can cause a problem if you access a field that
doesn't end with a full stop. For example:
\begin{verbatim}
\newabbreviation
 [user1={German Speaking \TeX\ User Group}]
 {dante}{DANTE e.V.}{Deutschsprachige Anwendervereinigung \TeX\
e.V.}
\end{verbatim}
Here the \gloskey{short} and \gloskey{long} fields end with a full stop, but the
\gloskey{user1} field doesn't. The simplest solution in this
situation is to put the sentence terminator in the final optional
argument. For example:
\begin{verbatim}
\glsuseri{dante}[.]
\end{verbatim}
This will bring the punctuation character inside the \gls{linktext}
and it won't be discarded.

\item[\catattrdef{pluraldiscardperiod}] If this attribute is set to 
\qt{true} \emph{and} the \catattr{discardperiod} attribute is set to
\qt{true}, this will behave as above for the plural commands like
\cs{glspl} or \cs{glsplural}. This attribute is only applicable to entries
defined using \cs{newabbreviation} (or \cs{newacronym} if it's using
\cs{newabbreviation}.)

\item[\catattrdef{retainfirstuseperiod}] If this attribute is set
to \qt{true} then the full stop won't be discarded for
\gls{firstuse} instances, even if \catattr{discardperiod} or
\catattr{pluraldiscardperiod} are set. This is useful for
\meta{short} (\meta{long}) abbreviation styles where only the short
form has a trailing full stop. This attribute is only applicable to entries
defined using \cs{newabbreviation} (or \cs{newacronym} if it's using
\cs{newabbreviation}.) This attribute doesn't apply to the
accessibility fields.

\item[\catattrdef{markwords}] If this attribute is set to \qt{true}
any entry defined using \ics{newabbreviation} will automatically
have spaces in the long form replaced with
\begin{definition}[\DescribeMacro\glsxtrwordsep]
\cs{glsxtrwordsep}
\end{definition}
and each word is encapsulated with
\begin{definition}[\DescribeMacro\glsxtrword]
\cs{glsxtrword}\marg{word}
\end{definition}
For example:
\begin{verbatim}
\glssetcategoryattribute{abbreviation}{markwords}{true}
\newabbreviation{ip}{IP}{Internet Protocol}
\end{verbatim}
is essentially the same as
\begin{verbatim}
\newabbreviation{ip}{IP}
{\glsxtrword{Internet}\glsxtrwordsep\glsxtrword{Protocol}}
\end{verbatim}
The \qt{hyphen} styles, such as
\abbrstyle{long-hyphen-short-hyphen}, take advantage of this
markup. If the inserted material (provided in the final argument
of commands like \ics{gls}) starts with a hyphen then
\cs{glsxtrwordsep} is locally redefined to a hyphen. (The default
value is a space). Note that this only applies to commands
like \cs{gls} and not like \cs{glsxtrlong}. You can provide your own
localised switch, if required. For example:
\begin{verbatim}
\newcommand{\hyplong}[2][]{%
 {\def\glsxtrwordsep{-}\glsxtrlong[#1]{#2}}}
\end{verbatim}

This setting will also adjust the long plural. This attribute is only applicable to entries
defined using \cs{newabbreviation} (or \cs{newacronym} if it's using
\cs{newabbreviation}.)

\begin{important}
This setting may result in the 
\cs{glsxtrword} and \cs{glsxtrwordsep} markup ending up in the
\gloskey{sort} field, depending on the style in use.
\end{important}

\item[\catattrdef{markshortwords}] This is similar to
\catattr{markwords} but applies to the short form. (Only useful for
abbreviations that contain spaces.) This attribute is only applicable to entries
defined using \cs{newabbreviation} (or \cs{newacronym} if it's using
\cs{newabbreviation}.)

This setting will only adjust the short plural if the
\gloskey{shortplural} key isn't used. This setting will
take precedence over \catattr{insertdots}.

\begin{important}
This setting may result in the 
\cs{glsxtrword} and \cs{glsxtrwordsep} markup ending up in the
\gloskey{sort} field, depending on the style in use.
\end{important}

\item[\catattrdef{insertdots}] If this attribute is set to \qt{true}
any entry defined using \ics{newabbreviation} will automatically
have full stops (periods) inserted after each letter. The entry will
be defined with those dots present as though they had been present
in the \meta{short} argument of \cs{newabbreviation} (rather than
inserting them every time the entry is used). The short plural
form defaults to the new dotted version of the original \meta{short}
form with the plural suffix appended. \emph{This setting is incompatible
with \catattr{markshortwords}.} This attribute is only applicable to entries
defined using \cs{newabbreviation} (or \cs{newacronym} if it's using
\cs{newabbreviation}.)

\begin{important}
If you explicitly override
the short plural using the \gloskey{shortplural} key, you must
explicitly insert the dots yourself (since there's no way for the
code to determine if the plural has a suffix that shouldn't be
followed by a dot).
\end{important}

This attribute is best used with the \catattr{discardperiod}
attribute set to \qt{true}.

\item[\catattrdef{aposplural}] If this attribute is set to \qt{true},
\ics{newabbreviation} will insert an apostrophe (') before
the plural suffix for the \emph{short} plural form (unless explicitly
overridden with the \gloskey{shortplural} key). The long plural form
is unaffected by this setting. This setting overrides
\catattr{noshortplural}. This attribute is only applicable to entries
defined using \cs{newabbreviation} (or \cs{newacronym} if it's using
\cs{newabbreviation}.) Check with your supervisor, publisher or
editor if you want to use this attribute as this 
usage is controversial.


\item[\catattrdef{noshortplural}] If this attribute is set to
\qt{true}, \ics{newabbreviation} won't append the plural suffix for
the short plural form. This means the \gloskey{short} and
\gloskey{shortplural} values will be the same unless explicitly
overridden. \emph{This setting is incompatible with \catattr{aposplural}.} This attribute is only applicable to entries
defined using \cs{newabbreviation} (or \cs{newacronym} if it's using
\cs{newabbreviation}.)

\item[\catattrdef{headuc}] If this attribute is set to \qt{true},
commands like \ics{glsfmtshort} will use the upper case version
in the page headers.

\item[\catattrdef{tagging}] If this attribute is set to \qt{true},
the tagging command defined by \ics{GlsXtrEnableInitialTagging}
will be activated to use \ics{glsxtrtagfont} in the glossary
(see \sectionref{sec:tagging}).

\item[\catattrdef{entrycount}] Unlike the above attributes,
this attribute isn't boolean but instead must be an integer 
value and is used in combination with \ics{glsenableentrycount}
(see \sectionref{sec:entrycountmods}). Leave blank or undefined
for categories that shouldn't have this facility enabled. The
value of this attribute is used by \ics{glsxtrifcounttrigger}
to determine how commands such as \ics{cgls} should behave.

With \styfmt{glossaries}, commands like \cs{cgls} use \cs{cglsformat}
only if the previous usage count for that entry was equal to~1.
With \styfmt{glossaries-extra} the test is now for entries that
have the \catattr{entrycount} attribute set and
where the previous usage count for that entry is less than or
equal to the value of that attribute.

\item[\catattrdef{linkcount}] This attribute is set to
\texttt{true} by \ics{GlsXtrEnableLinkCounting} (see
\sectionref{sec:linkcount}).

\item[\catattrdef{linkcountmaster}] This attribute is set to
the name of the master counter by \ics{GlsXtrEnableLinkCounting}
if the optional argument is provided (see
\sectionref{sec:linkcount}).

\item[\catattrdef{glossdesc}] The \ics{glossentrydesc} command
(used in the predefined glossary styles) is modified by
\styfmt{glossaries-extra} to check for this attribute.
The attribute may have one of the following values:
\begin{itemize}
\item \texttt{firstuc}: the first letter of the description will be
converted to upper case (using \ics{Glsentrydesc}).

\item \texttt{title}: the description will be used in the argument of the
title casing command \ics{capitalisewords} (provided by
\sty{mfirstuc}). If you want to use a different command
you can redefine:
\begin{definition}[\DescribeMacro\glsxtrfieldtitlecasecs]
\cs{glsxtrfieldtitlecasecs}\marg{phrase cs}
\end{definition}
For example:
\begin{verbatim}
\newcommand*{\glsxtrfieldtitlecasecs}[1]{\xcapitalisefmtwords*{#1}}
\end{verbatim}
(Note that the argument to \cs{glsxtrfieldtitlecasecs}
will be a control sequence whose replacement text is the 
entry's description, which is why \cs{xcapitalisefmtwords}
is needed instead of \cs{capitalisefmtwords}.)

\end{itemize}
Any other values of this attribute are ignored. Remember
that there are design limitations for both the first letter
uppercasing and the title casing commands. See the \sty{mfirstuc}
user manual for further details.

\item[\catattrdef{glossdescfont}] (New to version 1.04) 
In addition to the above, the modified \ics{glossentrydesc} 
command also checks this attribute. If set, it should be the name of
a control sequence (without the leading backslash) that takes one
argument. This control sequence will be applied to the description
text. For example:
\begin{verbatim}
\glssetcategoryattribute{general}{glossdescfont}{emph}
\end{verbatim}


\item[\catattrdef{glossname}] As \catattr{glossdesc} but applies
to \ics{glossentryname}. Additionally, if this attribute is
set to \qt{uc} the name is converted to all capitals.

\item[\catattrdef{indexname}] If set, the \cs{glsxtrpostnamehook} hook
used at the end of \ics{glossentryname}
will index the entry using \ics{index}. See
\sectionref{sec:autoindex} for further details.

\item[\catattrdef{glossnamefont}] (New to version 1.04) 
In addition to the above, the modified \ics{glossentryname} 
command also checks this attribute. If set, it should be the name of
a control sequence (without the leading backslash) that takes one
argument. This control sequence will be applied to the name
text. For example:
\begin{verbatim}
\glssetcategoryattribute{general}{glossnamefont}{emph}
\end{verbatim}
Note that this overrides \cs{glsnamefont} which will only
be used if this attribute hasn't been set.

Remember that glossary styles may additionally apply a font change,
such as the list styles which put the name in the optional argument
of \cs{item}.

\item[\catattrdef{glosssymbolfont}] (New to version 1.42.)
This is similar to \catattr{glossnamefont} and
\catattr{glossdescfont} but is used by \ics{glossentrysymbol}. 

\item[\catattrdef{textformat}] (New to version 1.21.)
Commands like \cs{gls} and \cs{glstext} have the link text
encapsulated in the argument of \cs{glstextformat} by default. If
this attribute is set, the control sequence given by the attribute
value will be used instead. As with the above, the attribute value
should be the name (without the leading backslash) of a command that
takes a single argument (the link text). Remember that the
abbreviation styles may apply an additional font change.

\item[\catattrdef{hyperoutside}] (New to v1.21.) The attribute value
may be \texttt{false}, \texttt{true} or unset. If unset,
\texttt{true} is assumed. This indicates the default setting
of the \gloskey[glslink]{hyperoutside} key, described in 
\sectionref{sec:wrglossary}.

\item[\catattrdef{dualindex}] If set, whenever a glossary entry
has information written to the external glossary file
through commands like \ics{gls} and \ics{glsadd}, a~corresponding
line will be written to the indexing file using \ics{index}. See
\sectionref{sec:autoindex} for further details.

\item[\catattrdef{targeturl}] If set, the hyperlink generated by
commands like \cs{gls} will be set to the URL provided by this
attributes value. For example:
\begin{verbatim}
\glssetcategoryattribute{general}{targeturl}{master-doc.pdf}
\end{verbatim}
(See also the accompanying sample file
\texttt{sample-external.tex}.) If the URL contains awkward
characters (such as \verb|%| or \verb|~|) remember that the base
\styfmt{glossaries} package provides commands like
\ics{glspercentchar} and \cs{glstildechar} that expand to literal
characters.

If you want to a named anchor within the target URL (notionally
adding \texttt{\#}\meta{name} to the URL), then you also
need to set \catattr{targetname} to the anchor \meta{name}. You may
use \cs{glslabel} within \meta{name} which is set by commands
like \cs{gls} to the entry's label.

All the predefined glossary styles start each entry
listing with \cs{glstarget} which sets the anchor to 
\verb|\glolinkprefix\glslabel|, so if you want entries to link
to glossaries in the URL given by \catattr{targeturl}, you
can just do:
\begin{verbatim}
\glssetcategoryattribute{general}{targetname}{\glolinkprefix\glslabel}
\end{verbatim}
(If the target document changed \cs{glolinkprefix} then you will
need to adjust the above as appropriate.)

If the anchor is in the form \meta{name1}\texttt{.}\meta{name2}
then use \catattr{targetname} for the \meta{name2} part and
\catattr{targetcategory} for the \meta{name1} part.

For example:
\begin{verbatim}
\glssetcategoryattribute{general}{targeturl}{master-doc.pdf}
\glssetcategoryattribute{general}{targetcategory}{page}
\glssetcategoryattribute{general}{targetname}{7}
\end{verbatim}
will cause all link text for \texttt{general} entries to
link to \texttt{master-doc.pdf\#page.7} (page 7 of that PDF).

If you want a mixture in your document of entries that link to 
an internal glossary and entries that link to an external URL
then you can use the starred form of \ics{newignoredglossary}
for the external list. For example:
\begin{verbatim}
\newignoredglossary*{external}

\glssetcategoryattribute{external}{targeturl}{master-doc.pdf}
\glssetcategoryattribute{general}{targetname}{\glolinkprefix\glslabel}

\newglossaryentry{sample}{name={sample},description={local example}}

\newglossaryentry{sample2}{name={sample2},
  type=external,
  category=external,
  description={external example}}
\end{verbatim}

\item[\catattrdef{externallocation}] The value should be the
file name of the target document when manually indexing an
external location (see \sectionref{sec:wrglossary}). In general,
it's better to use \gls{bib2gls} v1.7+ which can handle multiple
external sources and doesn't require this attribute.

\end{description}

If a term is defined using \ics{newabbreviation} and accessibility
support has been supplied via the \pkgopt{accsupp} package option
then the following attributes are also available. If
\gloskey{shortaccess} isn't set its value will be obtained from:
\begin{definition}[\DescribeMacro\glsdefaultshortaccess]
\cs{glsdefaultshortaccess}\marg{short}\marg{long}
\end{definition}
This is defined by \sty{glossaries-accsupp} to just do \meta{long}
but \sty{glossaries-extra} redefines it to do \meta{long}
(\meta{short}). The \catattr{accessinsertdots},
\catattr{accessaposplural} and \catattr{accessnoshortplural}
attributes below refer to the \meta{short} form that's passed to
this command. 

The following attributes are only applicable to entries
defined using \cs{newabbreviation} (or \cs{newacronym} if it's using
\cs{newabbreviation}). They have no effect if the \pkgopt{accsupp}
package option hasn't been used.

\begin{description}
\item[\catattrdef{accessinsertdots}] If this attribute is set to
\qt{true} and the \sty{glossaries-accsupp} package has been loaded
(with the \pkgopt{accsupp} option), then this behaves like
\catattr{insertdots} but for the \meta{short} part used in the \gloskey{shortaccess} field instead
of the \gloskey{short} field.
This setting will be overridden by an explicit use of
the \gloskey{shortaccess} key in the optional argument
of \cs{newabbreviation} (or \cs{newacronym}). 

\item[\catattrdef{accessaposplural}] This boolean attribute overrides
\catattr{aposplural} for the \gloskey{shortpluralaccess} key.
Has no effect if there's no accessibility support or if
the \gloskey{shortaccess} key hasn't been set or if the
\gloskey{shortpluralaccess} key is explicitly set. If
the \catattr{aposplural} is set and this attribute isn't set
and the \gloskey{shortaccess} key is set, then the
\catattr{aposplural} setting governs the default
\gloskey{shortpluralaccess} setting. If you want
\catattr{aposplural} but don't want it applied to the accessibility
support, set the \catattr{accessaposplural} attribute to \qt{false}. This attribute is only applicable to entries
defined using \cs{newabbreviation} (or \cs{newacronym} if it's using
\cs{newabbreviation}.)

\item[\catattrdef{accessnoshortplural}] This boolean attribute overrides
\catattr{noshortplural} for the \gloskey{shortpluralaccess} key.
The same conditions apply as for \catattr{accessaposplural}. This attribute is only applicable to entries
defined using \cs{newabbreviation} (or \cs{newacronym} if it's using
\cs{newabbreviation}.)

\item[\catattrdef{nameshortaccess}] If this attribute is set to
\qt{true}, the \gloskey{access} field (used for the \gloskey{name}
field's accessibility support) automatically assigned if it's not
provided.

\item[\catattrdef{textshortaccess}] Like \gloskey{nameshortaccess}
but applies to the \gloskey{textaccess} field automatically assigned if it's not
provided.

\item[\catattrdef{firstshortaccess}] Like \gloskey{nameshortaccess}
but applies to the \gloskey{firstaccess} field automatically assigned if it's not
provided.

\end{description}

An attribute can be set using:
\begin{definition}[\DescribeMacro\glssetcategoryattribute]
\cs{glssetcategoryattribute}\marg{category-label}\marg{attribute-label}\marg{value}
\end{definition}
where \meta{category-label} is the category label,
\meta{attribute-label} is the attribute label and \meta{value} is
the new value for the attribute.

There is a shortcut version to set the \catattr{regular} attribute
to \qt{true}:
\begin{definition}[\DescribeMacro\glssetregularcategory]
\cs{glssetregularcategory}\marg{category-label}
\end{definition}

If you need to lookup the category label for a particular entry, you
can use the shortcut command:
\begin{definition}[\DescribeMacro\glssetattribute]
\cs{glssetattribute}\marg{entry-label}\marg{attribute-label}\marg{value}
\end{definition}
This uses \cs{glssetcategoryattribute} with \cs{glscategory} to set
the attribute. Note that this will affect all other entries that
share this entry's category.

You can fetch the value of an attribute for a particular category
using:
\begin{definition}[\DescribeMacro\glsgetcategoryattribute]
\cs{glsgetcategoryattribute}\marg{category-label}\marg{attribute-label}
\end{definition}
Again there is a shortcut if you need to lookup the category label
for a given entry:
\begin{definition}[\DescribeMacro\glsgetattribute]
\cs{glsgetattribute}\marg{entry-label}\marg{attribute-label}
\end{definition}

You can test if an attribute has been assigned to a given
category using:
\begin{definition}[\DescribeMacro\glshascategoryattribute]
\cs{glshascategoryattribute}\marg{category-label}\marg{attribute-label}\marg{true
code}\marg{false code}
\end{definition}
This uses \sty{etoolbox}'s 
\cs{ifcsvoid} and does \meta{true code} if the attribute has been
set and isn't blank and isn't \cs{relax}. The shortcut if you need
to lookup the category label from an entry is:
\begin{definition}[\DescribeMacro\glshasattribute]
\cs{glshasattribute}\marg{entry-label}\marg{attribute-label}\marg{true
code}\marg{false code}
\end{definition}

You can test the value of an attribute for a particular category
using:
\begin{definition}[\DescribeMacro\glsifcategoryattribute]
\cs{glsifcategoryattribute}\marg{category-label}\marg{attribute-label}\marg{value} \marg{true-part}\marg{false-part}
\end{definition}
This tests if the attribute (given by \meta{attribute-label}) for
the category (given by \meta{category-label}) is set and equal to
\meta{value}. If true, \meta{true-part} is done. If the attribute
isn't set or is set but isn't equal to \meta{value}, \meta{false
part} is done.

For example:
\begin{verbatim}
\glsifcategoryattribute{general}{nohyper}{true}{NO HYPER}{HYPER}
\end{verbatim}
This does \qt{NO HYPER} if the \category{general} category has the
\catattr{nohyper} attribute set to \texttt{true} otherwise if
does \qt{HYPER}.

With boolean-style attributes like \catattr{nohyper}, make sure you
always test for \texttt{true} not \texttt{false} in case the attribute hasn't been set.

Again there's a shortcut if you need to lookup the category label
from a particular entry:
\begin{definition}[\DescribeMacro\glsifattribute]
\cs{glsifattribute}\marg{entry-label}\marg{attribute-label}\marg{value}\marg{true-part} \marg{false-part}
\end{definition}

There's also a shortcut to determine if a particular category has
the \catattr{regular} attribute set to \qt{true}:
\begin{definition}[\DescribeMacro\glsifregularcategory]
\cs{glsifregularcategory}\marg{category-label}\marg{true-part}\marg{false-part}
\end{definition}
Alternatively, if you need to lookup the category for a particular
entry:
\begin{definition}[\DescribeMacro\glsifregular]
\cs{glsifregular}\marg{entry-label}\marg{true-part}\marg{false-part}
\end{definition}
Note that if the \catattr{regular} attribute hasn't be set, the
above do \meta{false-part}. There are also reverse commands that
test if the \catattr{regular} attribute has been set to \qt{false}:
\begin{definition}[\DescribeMacro\glsifnotregularcategory]
\cs{glsifnotregularcategory}\marg{category-label}\marg{true-part}\marg{false-part}
\end{definition}
or for a particular entry:
\begin{definition}[\DescribeMacro\glsifnotregular]
\cs{glsifnotregular}\marg{entry-label}\marg{true-part}\marg{false-part}
\end{definition}
Again, if the \catattr{regular} attribute hasn't been set, the above
do \meta{false-part}, so these reverse commands aren't logically
opposite in the strict sense.

You can iterate through all entries with a given category using:
\begin{definition}
\cs{glsforeachincategory}\oarg{glossary-labels}\marg{category-label} \marg{glossary-cs}\marg{label-cs}\marg{body}
\end{definition}
This iterates through all entries in the glossaries identified by
the comma-separated list \meta{glossary-labels} that have the
category given by \meta{category-label} and performs \meta{body} for
each match. Within \meta{body}, you can use \meta{glossary-cs} and
\meta{label-cs}
(which much be control sequences) to access the current glossary and
entry label. If \meta{glossary-labels} is omitted, all glossaries
are assumed.

Similarly, you can iterate through all entries that have a category
with a given attribute using:
\begin{definition}[\DescribeMacro\glsforeachwithattribute]
\cs{glsforeachwithattribute}\oarg{glossary-labels}\marg{attribute-label} \marg{attribute-value}\marg{glossary-cs}\marg{label-cs}\marg{body}
\end{definition}
This will do \meta{body} for each entry that has a category with the
attribute \meta{attribute-label} set to \meta{attribute-value}.
The remaining arguments are as the previous command.

You can change the category for a particular entry using the
standard glossary field changing commands, such as
\ics{glsfielddef}. Alternatively, you can use
\begin{definition}[\DescribeMacro\glsxtrsetcategory]
\cs{glsxtrsetcategory}\marg{entry-labels}\marg{category-label}
\end{definition}
This will change the category to \meta{category-label} for each
entry listed in the comma-separated list \meta{entry-labels}. This
command uses \ics{glsfieldxdef} so it will expand
\meta{category-label} and make the change global.

You can also change the category for all entries with a glossary or
glossaries using:
\begin{definition}[\DescribeMacro\glsxtrsetcategoryforall]
\cs{glsxtrsetcategoryforall}\marg{glossary-labels}\marg{category-label}
\end{definition}
where \meta{glossary-labels} is a comma-separated list of glossary
labels.

\chapter{Counting References}
\label{sec:countref}

There are three basic ways of counting entry references:
\begin{enumerate}
\item Counting the total number of times \cs{glsunset} is used
(\cs{glsreset} resets the count and is best avoided).
This is provided by the base \styfmt{glossaries} package and is
intended for documents where the term should be displayed
differently if it's only been used a certain number of times.
The information has to be written to the \texttt{.aux} file
so that it's available on the next \LaTeX\ run.

This method is extended by \styfmt{glossaries-extra} and is
described in \sectionref{sec:entrycount}.

\item Counting the total number of records. This method is
only available with \gls{bib2gls} and is intended
for documents where the term should be displayed differently
if it's only been recorded (indexed) a certain number of times.
See \sectionref{sec:recordcount} for further details.

\item Counting the number of times the \cs{gls}-like or
\cs{glstext}-like commands are used. (That is, those commands
that internally use \cs{@gls@link}.) Unlike the other two methods,
this just provides a running total rather than the total from the
previous \LaTeX\ run. This method is intended to make it more
convenient to work with hooks like \cs{glslinkcheckfirsthyperhook},
\cs{glslinkpostsetkeys} or \cs{glslinkpresetkeys}.
See \sectionref{sec:linkcount} for further details.
\end{enumerate}

\section{Entry Counting (First Use Flag)}
\label{sec:entrycount}

As mentioned in \sectionref{sec:entrycountmods},
\styfmt{glossaries-extra} modifies the \ics{glsenableentrycount} command to allow
for the \catattr{entrycount} attribute. This means that
you not only need to enable entry counting with
\ics{glsenableentrycount}, but you also need
to set the appropriate attribute (see \sectionref{sec:categories}).

\begin{important}
Remember that entry counting only counts the number of times an
entry is used by commands that change the \gls{firstuseflag}. (That
is, all those commands that mark the entry as having been used.)
There are many commands that don't modify this flag and they won't
contribute to the entry use count.
\end{important}

With \styfmt{glossaries-extra}, you may use \ics{cgls} instead of
\ics{gls} even if you haven't enabled entry counting. You will only
get a warning if you use \cs{glsenableentrycount} without setting
the \catattr{entrycount} attribute.  (With \styfmt{glossaries},
commands like \ics{cgls} will generate a warning if
\cs{glsenableentrycount} hasn't been used.) The abbreviation
shortcut \ics{ab} uses \cs{cgls} (see
\sectionref{sec:abbrshortcuts}). The acronym shortcut
\ics{ac} uses \cs{cgls} if it's been defined with
\pkgopt[ac]{shortcuts} (or \pkgopt[all]{shortcuts})
but uses \cs{gls} if it's been defined with
\pkgopt[acronyms]{shortcuts} (or \pkgopt[acro]{shortcuts}).

All upper case versions (not provided by \styfmt{glossaries}) are
also available:
\begin{definition}[\DescribeMacro\cGLS]
\cs{cGLS}\oarg{options}\marg{label}\oarg{insert}
\end{definition}
and
\begin{definition}[\DescribeMacro\cGLSpl]
\cs{cGLSpl}\oarg{options}\marg{label}\oarg{insert}
\end{definition}
These are analogous to \cs{cgls} and \cs{cglspl} but they
use
\begin{definition}[\DescribeMacro\cGLSformat]
\cs{cGLSformat}\marg{label}\marg{insert}
\end{definition}
and
\begin{definition}[\DescribeMacro\cGLSplformat]
\cs{cGLSplformat}\marg{label}\marg{insert}
\end{definition}
which convert the analogous \cs{cglsformat} and
\cs{cglsplformat} to upper case.

Just using \styfmt{glossaries}:
\begin{verbatim}
\documentclass{article}

\usepackage{glossaries}

\makeglossaries

\glsenableentrycount

\newacronym{html}{HTML}{hypertext markup language}
\newacronym{xml}{XML}{extensible markup language}

\begin{document}

Used once: \cgls{html}.

Used twice: \cgls{xml} and \cgls{xml}.

\printglossaries

\end{document}
\end{verbatim}

If you switch to \styfmt{glossaries-extra} you must set the
\catattr{entrycount} attribute:
\begin{verbatim}
\documentclass{article}

\usepackage{glossaries-extra}

\makeglossaries

\glsenableentrycount

\glssetcategoryattribute{abbreviation}{entrycount}{1}

\newabbreviation{html}{HTML}{hypertext markup language}
\newabbreviation{xml}{XML}{extensible markup language}

\begin{document}

Used once: \cgls{html}.

Used twice: \cgls{xml} and \cgls{xml}.

\printglossaries

\end{document}
\end{verbatim}

When activated with \cs{glsenableentrycount}, commands such as 
\ics{cgls} now use
\begin{definition}[\DescribeMacro\glsxtrifcounttrigger]
\cs{glsxtrifcounttrigger}\marg{label}\marg{trigger code}\marg{normal
code}
\end{definition}
to determine if the entry trips the entry count trigger.
The \meta{trigger code} uses commands like \cs{cglsformat} and 
unsets the \gls{firstuseflag}. The \meta{normal code} is the code that
would ordinarily be performed by whatever the equivalent
command is (for example, \cs{cgls} will use \cs{cglsformat} in 
\meta{trigger code} but the usual \cs{gls} behaviour in \meta{normal
code}).

The default definition is:
\begin{verbatim}
\newcommand*{\glsxtrifcounttrigger}[3]{%
 \glshasattribute{#1}{entrycount}%
 {%
   \ifnum\glsentryprevcount{#1}>\glsgetattribute{#1}{entrycount}\relax
    #3%
   \else
    #2%
   \fi 
 }%
 {#3}% 
}
\end{verbatim}
This means that if an entry is assigned to a category that has
the \catattr{entrycount} attribute then the \meta{trigger code}
will be used if the previous count value 
(the number of times the entry was used on the last run) is greater
than the value of the attribute.

For example, to trigger normal use if the previous count value
is greater than four:
\begin{verbatim}
\glssetcategoryattribute{abbreviation}{entrycount}{4}
\end{verbatim}

There is a convenient command provided to enable entry counting,
set the \catattr{entrycount} attribute and redefine \ics{gls},
etc to use \ics{cgls} etc:
\begin{definition}[\DescribeMacro\GlsXtrEnableEntryCounting]
\cs{GlsXtrEnableEntryCounting}\marg{categories}\marg{value}
\end{definition}
The first argument \meta{categories} is a comma-separated list
of categories. For each category, the \catattr{entrycount}
attribute is set to \meta{value}. In addition, this does:
\begin{verbatim}
\renewcommand*{\gls}{\cgls}%
\renewcommand*{\Gls}{\cGls}%
\renewcommand*{\glspl}{\cglspl}%
\renewcommand*{\Glspl}{\cGlspl}%
\renewcommand*{\GLS}{\cGLS}%
\renewcommand*{\GLSpl}{\cGLSpl}%
\end{verbatim}
This makes it easier to enable entry-counting on existing
documents.

If you use \cs{GlsXtrEnableEntryCounting} more than once, subsequent uses will
just set the \catattr{entrycount} attribute for each listed
category.

The above example document can then become:
\begin{verbatim}
\documentclass{article}

\usepackage{glossaries-extra}

\makeglossaries

\GlsXtrEnableEntryCounting{abbreviation}{1}

\newabbreviation{html}{HTML}{hypertext markup language}
\newabbreviation{xml}{XML}{extensible markup language}

\begin{document}

Used once: \gls{html}.

Used twice: \gls{xml} and \gls{xml}.

\printglossaries

\end{document}
\end{verbatim}

The standard entry-counting function describe above counts
the number of times an entry has been marked as used throughout
the document. (The reset commands will reset the total back to 
zero.) If you prefer to count per sectional-unit, you can
use
\begin{definition}[\DescribeMacro\GlsXtrEnableEntryUnitCounting]
\cs{GlsXtrEnableEntryUnitCounting}\marg{categories}\marg{value}\marg{counter-name}
\end{definition}
where \meta{categories} is a comma-separated list of categories
to which this feature should be applied, \meta{value} is the
trigger value and \meta{counter-name} is the name of the counter
used by the sectional unit.

\begin{important}
Due to the asynchronous nature of \TeX's output routine,
discrepancies will occur in page spanning paragraphs if you
use the \ctr{page} counter.
\end{important}

Note that you can't use both the document-wide counting and
the per-unit counting in the same document.

The counter value is used as part of a label, which means
that \cs{the}\meta{counter-name} needs to be expandable.
Since \sty{hyperref} also has a similar requirement and provides
\cs{theH}\meta{counter-name} as an expandable alternative,
\styfmt{glossaries-extra} will use \cs{theH}\meta{counter-name}
if it exists otherwise it will use \cs{the}\meta{counter-name}.

The per-unit counting function uses two attributes: \catattr{entrycount}
(as before) and \catattr{unitcount} (the name of the counter).

Both the original document-wide counting mechanism and the
per-unit counting mechanism provide a command that can be
used to access the current count value for this run:
\begin{definition}[\DescribeMacro\glsentrycurrcount]
\cs{glsentrycurrcount}\marg{label}
\end{definition}
and the final value from the previous run:
\begin{definition}[\DescribeMacro\glsentryprevcount]
\cs{glsentryprevcount}\marg{label}
\end{definition}
In the case of the per-unit counting, this is the final value
\emph{for the current unit}. In both commands \meta{label}
is the entry's label.

The per-unit counting mechanism additionally provides:
\begin{definition}[\DescribeMacro\glsentryprevtotalcount]
\cs{glsentryprevtotalcount}\marg{label}
\end{definition}
which gives the sum of all the per-unit totals from the previous run
for the entry given by \meta{label}, and
\begin{definition}[\DescribeMacro\glsentryprevmaxcount]
\cs{glsentryprevmaxcount}\marg{label}
\end{definition}
which gives the maximum per-unit total from the previous run.

The above two commands are unavailable for the document-wide counting.

Example of per-unit counting, where the unit is the chapter:
\begin{verbatim}
\documentclass{report}
\usepackage{glossaries-extra}

\GlsXtrEnableEntryUnitCounting{abbreviation}{2}{chapter}

\makeglossaries

\newabbreviation{html}{HTML}{hypertext markup language}
\newabbreviation{css}{CSS}{cascading style sheet}

\newglossaryentry{sample}{name={sample},description={sample}}

\begin{document}
\chapter{Sample}

Used once: \gls{html}.

Used three times: \gls{css} and \gls{css} and \gls{css}.

Used once: \gls{sample}.

\chapter{Another Sample}

Used once: \gls{css}.

Used twice: \gls{html} and \gls{html}.

\printglossaries
\end{document}
\end{verbatim}
In this document, the \texttt{css} entry is used three times in the
first chapter. This is more than the trigger value of 2, so
\verb|\gls{css}| is expanded on \gls{firstuse} with the short
form used on subsequent use, and the \texttt{css} entries in
that chapter are added to the glossary. In the second chapter,
the \texttt{css} entry is only used once, which trips the 
suppression trigger, so in that chapter, the long form
is used and \verb|\gls{css}| doesn't get a line added to
the glossary file.

The \texttt{html} is used a total of three times, but the
expansion and indexing suppression trigger is tripped 
in both chapters because the per-unit total (1 for the first
chapter and 2 for the second chapter) is less than or equal
to the trigger value.

The \texttt{sample} entry has only been used once, but it doesn't
trip the indexing suppression because it's in the \category{general}
category, which hasn't been listed in
\cs{GlsXtrEnableEntryUnitCounting}.

The per-unit entry counting can be used for other purposes.
In the following example document the trigger value is set
to zero, which means the index suppression won't be triggered,
but the unit entry count is used to automatically suppress the
hyperlink for commands like \ics{gls} by modifying the
hook
\begin{definition}[\DescribeMacro\glslinkcheckfirsthyperhook]
\cs{glslinkcheckfirsthyperhook}
\end{definition}
which is used at the end of the macro the determines whether
or not to suppress the hyperlink.

\begin{verbatim}
\documentclass{article}

\usepackage[colorlinks]{hyperref}
\usepackage{glossaries-extra}

\makeglossaries

\GlsXtrEnableEntryUnitCounting{general}{0}{page}

\newglossaryentry{sample}{name={sample},description={an example}}

\renewcommand*{\glslinkcheckfirsthyperhook}{%
  \ifnum\glsentrycurrcount\glslabel>0
   \setkeys{glslink}{hyper=false}%
  \fi
}

\begin{document}

A \gls{sample} entry.
Next use: \gls{sample}.

\newpage

Next page: \gls{sample}.
Again: \gls{sample}.

\printglossaries

\end{document}
\end{verbatim}
This only produces a hyperlink for the first instance of 
\verb|\gls{sample}| on each page.

The earlier warning about using the \ctr{page} counter still 
applies. If the first instance of \cs{gls} occurs at the top of the
page within a paragraph that started on the previous page, then
the count will continue from the previous page.

\section{Link Counting}
\label{sec:linkcount}

As from version 1.26, an alternative method of entry counting
is to count the number of times the \cs{gls}-like or
\cs{glstext}-like commands are used. (The \qt{link} in this method's
name refers to the use of the internal command \cs{@gls@link}
not to \cs{hyperlink} although \cs{@gls@link} may use
\cs{hyperlink} when displaying the \gls{linktext}.)

To enable link counting use the preamble-only command:
\begin{definition}[\DescribeMacro\GlsXtrEnableLinkCounting]
\cs{GlsXtrEnableLinkCounting}\oarg{master counter}\marg{categories}
\end{definition}
where \meta{categories} is a list of category labels. The optional
argument \meta{master counter} may be used to identify a master
counter (which must be defined). If present, the associated
link counter will be reset when the master counter is incremented.
This command automatically sets the \catattr{linkcount} attribute
for the given categories. If the optional argument is
present, it also sets the \catattr{linkcountmaster} attribute.

When enabled, commands like \cs{gls} and \cs{glstext} 
increment the associated counter using
\begin{definition}[\DescribeMacro\glsxtrinclinkcounter]
\cs{glsxtrinclinkcounter}\marg{counter name}
\end{definition}
This just does 
\texttt{\cs{stepcounter}\marg{counter name}} by default but if you
need \cs{refstepcounter} instead, just redefine this command:
\begin{verbatim}
\renewcommand*{\glsxtrinclinkcounter}[1]{\refstepcounter{#1}}
\end{verbatim}

You can access the internal count register using
\begin{definition}[\DescribeMacro\GlsXtrLinkCounterValue]
\cs{GlsXtrLinkCounterValue}\marg{label}
\end{definition}
where \meta{label} is the entry's label. This will expand
to 0 if the counter hasn't been defined.

It's also possible to access the display value
(\cs{the}\meta{counter}) using
\begin{definition}[\DescribeMacro\GlsXtrTheLinkCounter]
\cs{GlsXtrTheLinkCounter}\marg{counter}
\end{definition}
(This will expand to 0 if the counter hasn't been defined.)

\begin{important}
In order to conserve resources, the counter is only defined
when it first needs to be incremented so terms that have
been defined but haven't been used in the document
won't have the associated count register allocated.
\end{important}

You can test if the counter has been defined using:
\begin{definition}[\DescribeMacro\GlsXtrIfLinkCounterDef]
\cs{GlsXtrIfLinkCounterDef}\marg{label}\marg{true}\marg{false}
\end{definition}
where \meta{label} is the entry's label.

The counter name can be obtained using 
\begin{definition}[\DescribeMacro\GlsXtrLinkCounterName]
\cs{GlsXtrLinkCounterName}\marg{label}
\end{definition}
This simply expands to the counter name associated with the
entry given by \meta{label} without any check for
existence. For example, to change the display command 
(\cs{the}\meta{counter}) using \sty{etoolbox}:
\begin{verbatim}
\csdef{the\GlsXtrLinkCounterName{duck}}{\Roman{\GlsXtrLinkCounterName{duck}}}
\end{verbatim}
This is useful if you just want to change the display for
specific entries but isn't convenient if you want to change the
display for all entries. Instead, it's simpler to redefine
\cs{GlsXtrTheLinkCounter}. For example:
\begin{verbatim}
\renewcommand*{\GlsXtrTheLinkCounter}[1]{%
 \GlsXtrIfLinkCounterDef{#1}%
 {\Roman{\GlsXtrLinkCounterName{#1}}}%
 {0}%
}
\end{verbatim}
In both cases, the redefinition should be implemented
after \cs{GlsXtrEnableLinkCounting}.

Here's an example document that uses link counting to disable
the hyperlink after the first reference. This redefines
\ics{glslinkpresetkeys} (which is used by both \cs{gls}
and \cs{glstext}) instead of
\cs{glslinkcheckfirsthyperhook} (which is used by \cs{gls}
but not by \cs{glstext}).

\begin{verbatim}
\documentclass{article}

\usepackage[colorlinks]{hyperref}
\usepackage{glossaries-extra}

\makeglossaries

\renewcommand*{\glslinkpresetkeys}{%
 \ifnum\GlsXtrLinkCounterValue{\glslabel}>1
  \setkeys{glslink}{hyper=false}%
 \fi
}

\GlsXtrEnableLinkCounting{general}

\newglossaryentry{sample1}{name={sample1},description={an example}}
\newglossaryentry{sample2}{name={sample2},description={another example}}

\newabbreviation{ex}{ex}{example}

\begin{document}

\section{Sample Section}

\Gls{sample1}, \gls{sample2} and \gls{ex}.
\Glstext{sample1} and \gls{ex} again.

\section{Another Sample Section}

\Gls{sample1}, \gls{sample2} and \gls{ex}.

\printglossaries

\end{document}
\end{verbatim}
The use of \ics{glslinkpresetkeys}
means that the options can override this. For example
\begin{verbatim}
\gls[hyper=true]{sample1}
\end{verbatim}
will override the \texttt{hyper=false} setting in
\ics{glslinkpresetkeys}. If \ics{glslinkpostsetkeys} is used
instead, the \texttt{hyper=false} setting will override
the setting provided in the optional argument.

The \category{abbreviation} category doesn't have the
\catattr{linkcount} attribute set (since it's not
listed in the argument of \cs{GlsXtrEnableLinkCounting}). This means that
\cs{GlsXtrLinkCounterValue}
always expands to 0 for the abbreviation (\texttt{ex}), so the 
inequality test
\begin{verbatim}
\ifnum\GlsXtrLinkCounterValue{\glslabel}>1
\end{verbatim}
will always be false. This means that the abbreviation won't
have \texttt{hyper=false} applied. If the test is changed to
\begin{verbatim}
\ifnum\GlsXtrLinkCounterValue{\glslabel}=1
\else
 \setkeys{glslink}{hyper=false}%
\fi
\end{verbatim}
Then the abbreviation will always have \texttt{hyper=false}
applied.

To reset the counter every section use the optional argument to set
the master counter:
\begin{verbatim}
\GlsXtrEnableLinkCounting[section]{general}
\end{verbatim}

\chapter{Auto-Indexing}
\label{sec:autoindex}

It's possible that you may also want a normal index as well as
the glossary, and you may want entries to automatically be
added to the index (as in this document).
There are two attributes that govern this: \catattr{indexname}
and \catattr{dualindex}.

\begin{important}
The auto-indexing is designed for \gls{makeindex} syntax. If you've
used the \pkgopt{xindy} package option, the automatic escaping of
\gls{xindy} special characters in the \gloskey{sort} field may
result in an incorrect sort value for the \cs{index} command used by
the auto-indexing. Note also that \app{texindy} has a fixed set of
special characters (corresponding to \gls{makeindex}['s] defaults)
that can't be customized. You may want to consider using \gls{bib2gls} and
its dual entries as an alternative approach.
\end{important}

\begin{sloppypar}
The \ics{glsxtrpostnamehook} macro, used
at the end of \ics{glossentryname} and \ics{Glossentryname},
checks the \catattr{indexname} attribute for the category
associated with that entry.
Since \cs{glossentryname} is used in the default glossary
styles, this makes a convenient way of automatically indexing
each entry name at its location in the glossary without
fiddling around with the value of the \gloskey{name} key.
\end{sloppypar}

The internal macro used by the \styfmt{glossaries} package to
write the information to the external glossary file is
modified to check for the \catattr{dualindex} attribute.

In both cases, the indexing is done through
\begin{definition}[\DescribeMacro\glsxtrdoautoindexname]
\cs{glsxtrdoautoindexname}\marg{label}\marg{attribute-label}
\end{definition}
This uses the standard \ics{index} command with the sort value 
taken from the entry's \gloskey{sort} key and the actual value 
set to \cs{glsentryname}\marg{label}. As from v1.16, there are
user-level commands available to change the sort and actual
value used by the automated index.

The actual value is given by
\begin{definition}[\DescribeMacro\glsxtrautoindexentry]
\cs{glsxtrautoindexentry}\marg{label}
\end{definition}
where \meta{label} is the entry's label. The default definition
is:
\begin{verbatim}
\newcommand*{\glsxtrautoindexentry}[1]{\string\glsentryname{#1}}
\end{verbatim}
Note the use of \cs{string} to prevent \cs{glsentryname} from
being expanded as it's written to the index file.

The sort value is assigned using:
\begin{definition}[\DescribeMacro\glsxtrautoindexassignsort]
\cs{glsxtrautoindexassignsort}\marg{cs}\marg{label}
\end{definition}
where \meta{label} is the entry label and \meta{cs} is the command
which needs to be set to the sort value. The default definition is:
\begin{verbatim}
\newcommand*{\glsxtrautoindexassignsort}[2]{%
  \glsletentryfield{#1}{#2}{sort}%
}
\end{verbatim}
After this macro is called, \meta{cs} is then processed to escape
any of \app{makeindex}'s special characters. Note that this 
escaping is only performed on the sort not on the actual value.
The escaping of the sort value is performed by
\begin{definition}[\DescribeMacro\glsxtrautoindexesc]
\cs{glsxtrautoindexesc}
\end{definition}
You can redefine this to do nothing if you want to omit the
escaping. You may want to consider providing another field to obtain
the appropriate sort value if the one provided in the \gloskey{sort}
field isn't suitable (because it may already have had special
characters escaped or it may be a numeric value in the case of
sort by use or definition).

The command used to perform the actual indexing is:
\begin{definition}[\DescribeMacro\glsxtrautoindex]
\cs{glsxtrautoindex}\marg{text}
\end{definition}
This just does \cs{index}\marg{text} by default.

\begin{important}
The entry's \gloskey{parent} field isn't referenced in this 
automated indexing.
\end{important}

For example, to index the value of the \gloskey{first} key,
instead of the \gloskey{name} key:
\begin{verbatim}
\renewcommand*{\glsxtrautoindexentry}[1]{\string\glsentryfirst{#1}}
\end{verbatim}
and if the sort value also needs to be set to the \gloskey{long}
field, if present, otherwise the \gloskey{sort} field:
\begin{verbatim}
\renewcommand*{\glsxtrautoindexassignsort}[2]{%
  \ifglshaslong{#2}%
  {\glsletentryfield{#1}{#2}{long}}%
  {\glsletentryfield{#1}{#2}{sort}}%
}
\end{verbatim}

If the value of the attribute given by \meta{attribute-label} 
is \qt{true}, no encap 
will be added, otherwise the encap will be the
attribute value. For example:
\begin{verbatim}
\glssetcategoryattribute{general}{indexname}{textbf}
\end{verbatim}
will set the encap to \texttt{textbf} which will display the
relevant page number in bold whereas
\begin{verbatim}
\glssetcategoryattribute{general}{dualindex}{true}
\end{verbatim}
won't apply any formatting to the page number in the index.

\begin{important}
The location used in the index will always be the page number
not the counter used in the glossary. (Unless some other loaded
package has modified the definition of \cs{index} to use
some thing else.)
\end{important}

By default the \gloskey[glslink]{format} key won't be used with 
the \catattr{dualindex} attribute. You can allow the 
\gloskey[glslink]{format} key to override the attribute value
by using the preamble-only command:
\begin{definition}[\DescribeMacro\GlsXtrEnableIndexFormatOverride]
\cs{GlsXtrEnableIndexFormatOverride}
\end{definition}
If you use this command and \sty{hyperref} has been loaded, 
then the \env{theindex} environment will be modified to redefine 
\ics{glshypernumber} to allow formats that use that command.

\begin{important}
The \catattr{dualindex} attribute will still be used on 
subsequent use even if the \catattr{indexonlyfirst} attribute
(or \pkgopt{indexonlyfirst} package option) is set. However,
the \catattr{dualindex} attribute will honour the 
\gloskey[glslink]{noindex} key.
\end{important}

The \cs{glsxtrdoautoindexname} command will attempt to escape any of
\ics{makeindex}'s special characters, but there may be special cases
where it fails, so take care.  This assumes the default \gls{makeindex} actual,
level, quote and encap values (unless any of the commands
\ics{actualchar}, \ics{levelchar}, \ics{quotechar} or
\ics{encapchar} have been defined before \styfmt{glossaries-extra}
is loaded).

If this isn't the case, you can use the following preamble-only
commands to set the correct characters.
\begin{important}
Be very careful of possible shifting category codes!
\end{important}

\begin{definition}[\DescribeMacro\GlsXtrSetActualChar]
\cs{GlsXtrSetActualChar}\marg{char}
\end{definition}
Set the actual character to \meta{char}.

\begin{definition}[\DescribeMacro\GlsXtrSetLevelChar]
\cs{GlsXtrSetLevelChar}\marg{char}
\end{definition}
Set the level character to \meta{char}.

\begin{definition}[\DescribeMacro\GlsXtrSetEscChar]
\cs{GlsXtrSetEscChar}\marg{char}
\end{definition}
Set the escape (quote) character to \meta{char}.

\begin{definition}[\DescribeMacro\GlsXtrSetEncapChar]
\cs{GlsXtrSetEncapChar}\marg{char}
\end{definition}
Set the encap character to \meta{char}.

\chapter{\appfmt{bib2gls}: Managing Reference Databases}
\label{sec:bib2gls}

There is a new command line application called
\gls{bib2gls}, which works in much the same way as 
a combination of \appfmt{bibtex} and
\appfmt{makeindex}\slash\appfmt{xindy}.
Instead of storing all your entry definitions in a \texttt{.tex} and
loading them using \cs{input} or \cs{loadglsentries}, the entries
can instead be stored in a \texttt{.bib} file and \gls{bib2gls} can
selectively write the appropriate commands to a \texttt{.glstex}
file which is loaded using \cs{glsxtrresourcefile} (or
\cs{GlsXtrLoadResources}).

This means that you can use a reference managing system, such as
JabRef, to maintain the database and it reduces the \TeX\ overhead
by only defining the entries that are actually required in the
document. If you currently have a \texttt{.tex} file that contains
hundreds of definitions, but you only use a dozen or so in your
document, then the build time is needlessly slowed by the unrequired
definitions that occur when the file is input. (You can convert an
existing \texttt{.tex} file containing glossary definitions to a
\texttt{.bib} file using \app{convertgls2bib}, supplied with
\gls{bib2gls}.)

There are some new commands and options added to
\styfmt{glossaries-extra} to help assist the integration of
\gls{bib2gls} into the document build process.

This chapter just provides a general overview of \gls{bib2gls}.
The full details and some sample documents are provided
in the \gls{bib2gls} \href{http://mirrors.ctan.org/support/bib2gls/bib2gls.pdf}{manual}.

An example of the contents of \texttt{.bib} file that stores
glossary entries that can be extracted with \gls{bib2gls}:
\begin{verbatim}
@entry{bird,
  name={bird},
  description = {feathered animal},
  see={[see also]{duck,goose}}
}

@entry{duck,
  name={duck},
  description = {a waterbird with short legs}
}

@entry{goose,
  name="goose",
  plural="geese",
  description={a waterbird with a long neck}
}
\end{verbatim}

The follow provides some abbreviations:
\begin{verbatim}
@string{ssi={server-side includes}}
@string{html={hypertext markup language}}

@abbreviation{shtml,
  short="shtml",
  long= ssi # " enabled " # html,
  description={a combination of \gls{html} and \gls{ssi}}
}

@abbreviation{html,
  short ="html",
  long  = html,
  description={a markup language for creating web pages}
}

@abbreviation{ssi,
  short="ssi",
  long = ssi,
  description={a simple interpreted server-side scripting language}
}
\end{verbatim}

Here are some symbols:
\begin{verbatim}
preamble{"\providecommand{\mtx}[1]{\boldsymbol{#1}}"}

@symbol{M,
  name={$\mtx{M}$},
  text={\mtx{M}},
  description={a matrix}
}

@symbol{v,
  name={$\vec{v}$},
  text={\vec{v}},
  description={a vector}
}

@symbol{S,
  name={$\mathcal{S}$},
  text={\mathcal{S}},
  description={a set}
}
\end{verbatim}

To ensure that \gls{bib2gls} can find out which entries have been
used in the document, you need the \pkgopt{record} package option:
\begin{verbatim}
\usepackage[record]{glossaries-extra}
\end{verbatim}
If this option's value is omitted (as above), the normal indexing
will be switched off, since \gls{bib2gls} can also sort the entries and 
collate the locations.

If you still want to use an indexing application (for example, you
need a custom \gls{xindy} rule), then just use
\pkgopt[alsoindex]{record} and continue to use \cs{makeglossaries}
and \cs{printglossary} (or \cs{printglossaries}), but you also need
to instruct \gls{bib2gls} to omit sorting to save time and
to prevent the \gloskey{sort} key from being set.

The \texttt{.glstex} file created by \gls{bib2gls} is loaded using:
\begin{definition}[\DescribeMacro\glsxtrresourcefile]
\cs{glsxtrresourcefile}\oarg{options}\marg{filename}
\end{definition}
(Don't include the file extension in \meta{filename}.)
There's a shortcut version (recommended over the above) that sets \meta{filename} to use \cs{jobname}:
\begin{definition}[\DescribeMacro\GlsXtrLoadResources]
\cs{GlsXtrLoadResources}\oarg{options}
\end{definition}
On the first use, this command is a shortcut for
\begin{alltt}
\cs{glsxtrresourcefile}\oarg{options}\{\cs{jobname}\}
\end{alltt}
On subsequent use,\footnote{Version 1.11 only allowed one use
of \cs{GlsXtrLoadResources} per document.}\ this command is a shortcut for
\begin{alltt}
\cs{glsxtrresourcefile}\oarg{options}\{\cs{jobname}-\meta{n}\}
\end{alltt}
where \meta{n} is the current value of
\begin{definition}
\cs{glsxtrresourcecount}
\end{definition}
which is incremented at the end of \cs{GlsXtrLoadResources}.
Any advisory notes regarding \cs{glsxtrresourcefile} also
apply to \cs{GlsXtrLoadResources}.

The \cs{glsxtrresourcefile} command writes the line
\begin{alltt}
\cs{glsxtr@resource}\marg{options}\marg{filename}
\end{alltt}
to the \texttt{.aux} file and will input
\meta{filename}\texttt{.glstex} if it exists.\footnote{v1.08 assumed
\meta{filename}\texttt{.tex} but that's potentially dangerous if,
for example, \meta{filename} happens to be the same as \cs{jobname}.
The \texttt{.glstex} extension was enforced by version 1.11.}

Since the \texttt{.glstex} file won't exist on the first \LaTeX\ run, the
\pkgopt{record} package option additionally switches on
\pkgopt[warn]{undefaction}. Any use of commands like \cs{gls} or
\cs{glstext} will produce ?? in the document, since the entries are
undefined at this point. Once \gls{bib2gls} has created the
\texttt{.glstex} file the references should be resolved.
This may cause a shift in the locations if the actual text produced
once the entry is defined is significantly larger than the
placeholder ?? (as this can alter the page breaking).

Note that as from v1.12, \cs{glsxtrresourcefile} temporarily
switches the category code of \texttt{@} to 11 (letter) while it
reads the file to allow for any internal commands stored in the
location field.

\section{Selection}

The default behaviour is for \gls{bib2gls} to select all entries
that have a record in the \texttt{.aux} file, and any dependent
entries (including parent and cross-references). The 
\texttt{glsignore} format (for example, 
\verb|\gls[format=glsignore]{duck}|) is recognised by \gls{bib2gls}
as a special ignored record. This means that it will match
the selection criteria but the record won't be added to
the location list. This means that you won't get spurious
commas in the location list (as can happen with the other indexing
methods), so you can do, for example,
\begin{verbatim}
\GlsXtrSetDefaultNumberFormat{glsignore}
\end{verbatim}
at the start of the front matter and
\begin{verbatim}
\GlsXtrSetDefaultNumberFormat{glsnumberformat}
\end{verbatim}
at the start of the main matter to prevent any records in the
front matter from occurring in the location lists.

\begin{important}
Note that commands like \cs{glsaddall} and \cs{glsaddallunused} don't work with \gls{bib2gls} as the command 
has to iterate over the internal lists of defined entry labels, which will be empty on the first
run and on subsequent runs will only contain those entries that have
been selected by \gls{bib2gls}.
\end{important}

If you want to add all
entries to the glossary, you need to tell \gls{bib2gls} this in the
options list. For example:
\begin{verbatim}
\GlsXtrLoadResources[src={terms},selection={all}]
\end{verbatim}
This will add all entries, regardless of whether or not they
have any records in the \texttt{.aux} file. Those that don't
have any records will have an empty location list.
See the \gls{bib2gls} user manual for more details of this option.

\section{Sorting and Displaying the Glossary}

There are many sorting options provided by \gls{bib2gls}.
The default is to sort according to the system locale.
If the document has a language setting, you can use
\texttt{sort=doc} to instruct \gls{bib2gls} to sort according to
that. (The language tag obtained from \sty{tracklang}'s
interface is written to the \texttt{.aux} file.) For a multilingual
document you need to explicitly set the locale using a well-formed
language tag. For example:
\begin{verbatim}
\GlsXtrLoadResources[
  src=terms, % data in terms.bib
  sort=de-DE-1996 % sort according to this locale
]
\end{verbatim}
The locale-sensitive sort methods usually ignore most 
punctuation so for lists of symbols you may find it more
appropriate to use one of the letter-base sort methods
that sort according to the Unicode value of each character.
Alternatively you can provide a custom rule.
See the \gls{bib2gls} manual for full details of all the available
sort methods.

Since the \texttt{.glstex} file only defines those references
required
within the document (selected according to the \texttt{selection}
option) and the definitions have been written in the
order corresponding to \gls{bib2gls}['s] sorted list, the glossaries can
simply be displayed using \cs{printunsrtglossary} (or
\cs{printunsrtglossaries}), described in \sectionref{sec:printunsrt}.

Suppose the \texttt{.bib} examples shown above have been stored in
the files \texttt{terms.bib}, \texttt{abbrvs.bib} and
\texttt{symbols.bib} which may either be in the current directory or
on \TeX's path. Then the document might look like:
\begin{verbatim}
\documentclass{article}

\usepackage[record]{glossaries-extra}

\setabbreviationstyle{long-short-desc}

\GlsXtrLoadResources[src={terms,abbrvs,symbols}]

\begin{document}
\gls{bird}

\gls{shtml}

\gls{M}

\printunsrtglossaries
\end{document}
\end{verbatim}
The document build process (assuming the document is called
\texttt{mydoc}) is:
\begin{verbatim}
pdflatex mydoc
bib2gls mydoc
pdflatex mydoc
\end{verbatim}
This creates a single glossary containing the entries:
\texttt{bird}, \texttt{duck}, \texttt{goose},
\texttt{html}, \texttt{M}, \texttt{shtml} and \texttt{ssi} (in that
order). The \texttt{bird}, \texttt{shtml} and \texttt{M} entries
were added because \gls{bib2gls} detected (from the \texttt{.aux}
file) that they had been used in the document. The other entries
were added because \gls{bib2gls} detected (from the \texttt{.bib}
files) that they are referenced by the used entries. In the case of
\texttt{duck} and \texttt{goose}, they are in the \gloskey{see}
field for \texttt{bird}. In the case of \texttt{ssi} and
\texttt{html}, they are referenced in the \gloskey{description}
field of \texttt{shtml}. These cross-referenced entries won't have a
location list when the glossary is first displayed, but depending on
how they are referenced, they may pick up a location list on the
next document build.

The entries can be separated into different glossaries with
different sort methods:
\begin{verbatim}
\documentclass{article}

\usepackage[record,abbreviations,symbols]{glossaries-extra}

\setabbreviationstyle{long-short-desc}

\GlsXtrLoadResources[src={terms},sort={en-GB},type=main]

\GlsXtrLoadResources
 [src={abbrvs},sort={letter-nocase},type=abbreviations]

\GlsXtrLoadResources
 [src={symbols},sort={use},type={symbols}]

\begin{document}
\gls{bird}

\gls{shtml}

\gls{M}

\printunsrtglossaries
\end{document}
\end{verbatim}

Or you can have multiple instance of \cs{GlsXtrLoadResources}
with the same \texttt{type}, which will produce a glossary
with ordered sub-blocks. For example:
\begin{verbatim}
\documentclass{article}

\usepackage[record,style=indexgroup]{glossaries-extra}

\setabbreviationstyle{long-short-desc}

\GlsXtrLoadResources
 [src={abbrvs},sort={letter-nocase},type=main,
  group={Abbreviations}]

\GlsXtrLoadResources
 [src={symbols},sort={use},type=main,
  group={Symbols}]

\GlsXtrLoadResources[src={terms},sort={en-GB},type=main]

\begin{document}
\gls{bird}

\gls{shtml}

\gls{M}

\printunsrtglossaries
\end{document}
\end{verbatim}
This will result in a glossary where the first group has
the title \qt{Abbreviations}, the second group has the
title \qt{Symbols} and then follow the usual letter groups.
Note that for this example to work, you must run \gls{bib2gls}
with the \verb|--group| (or \verb|-g|) switch. For example,
if the document is called \texttt{myDoc.tex}:
\begin{verbatim}
pdflatex myDoc
bib2gls -g myDoc
pdflatex myDoc
\end{verbatim}
The value of the \gloskey{group} field must always be a label. You
can set the corresponding title with \cs{glsxtrsetgrouptitle}
(see \sectionref{sec:glosstylemods}). If no title is set then the
label is used as the group title.

You can provide your own custom sort rule. For example,
if you are using \XeLaTeX\ or \LuaLaTeX:
\begin{alltt}
\cs{GlsXtrLoadResources}[
  src=terms, % entries in terms.bib
  sort=custom, % custom sort rule
  sort-rule=\{% required with sort=custom
  < \ae;\AE\ < a;\'a;\aa;\"a,\"A;\'A;\AA;\"A < b,B 
  < c;\'c,C;\'C < d,D < e;\'e,E;\'E < f,F < g,G
  < h,H < i;\'i,I;\'I < j,J < l;\l,L;\L < m,M < n,N 
  < o;\"o;\o,O;\"O;\O\ < p,P < q,Q < r,R < s;\'s,S;\'S 
  < t,T < u;\'u,U;\'U < v,V < w,W < x,X < y,Y <
  z;\.z,Z;\.Z 
  \}
]
\end{alltt}
Remember that if you are using \sty{inputenc} then extended
characters, such as \'e or \o, are active and will expand when
written to the \texttt{.aux} file. So with PDF\LaTeX\ the
above would have to be changed to protect against expansion.
Some of the options, including \texttt{sort-rule},
allow Unicode characters to be indicated in the format
\cs{u}\meta{hex} (or \cs{u}~\meta{hex}) \textbf{in the \texttt{.aux} file}. \gls{bib2gls}
will recognise this as the character
given by the hexadecimal value \meta{hex}. The \cs{u} also needs
protection from expansion, so with a non-Unicode aware
engine, the character \ae\ needs to be written as
\texttt{\cs{string}\cs{u}E6} and so on. This is quite cumbersome,
but you can use the shortcut \texttt{\ics{glshex} E6} instead, so
the above needs to be written as:
\begin{alltt}
\cs{GlsXtrLoadResources}[
  src=terms, % entries in terms.bib
  sort=custom, % custom sort rule
  sort-rule=\{% required with sort=custom
  < \cs{glshex} E6;\cs{glshex} C6
  < a;\cs{glshex} E1;\cs{glshex} E5,\cs{glshex} E4,A;\cs{glshex} C1;\cs{glshex} C5;\cs{glshex} C4
  < b,B < c;\cs{glshex} 0107,C;\cs{glshex} 0106 < d,D
  < e;\cs{glshex} E9,E;\cs{glshex} C9 < f,F < g,G
  < h,H < i;\cs{glshex} ED,I;\cs{glshex} CD < j,J
  < l;\cs{glshex} 0142,L;\cs{glshex} 0141 < m,M < n,N 
  < o;\cs{glshex} F6;\cs{glshex} F8,O;\cs{glshex} D6;\cs{glshex} D8 
  < p,P < q,Q < r,R < s;\cs{glshex} 013F,S;\cs{glshex} 015A 
  < t,T < u;\cs{glshex} FA,U;\cs{glshex} DA < v,V < w,W < x,X < y,Y
  < z;\cs{glshex} 017C,Z;\cs{glshex} 017B 
  \}]
\end{alltt}

\section{The \styfmt{glossaries-extra-bib2gls} package}
\label{sec:bib2glssty}

The package option \pkgopt[only]{record} (or simply \pkgopt{record})
automatically loads the supplementary package
\sty{glossaries-extra-bib2gls}, which provides some commands
that are specific to \gls{bib2gls}. The package isn't
loaded by \pkgopt[alsoindex]{record} as that option is intended
for sorting with \gls{makeindex} or \gls{xindy} and it is expected
that the sorting will be switched off (with the resource option
\texttt{sort=none}).

If \sty{glossaries-extra-bib2gls} is loaded via the
\pkgopt{record} package option then the check for associated
language resource files (see \sectionref{sec:lang}) will also
search for the existence of
\texttt{glossariesxtr-}\meta{script}\texttt{.ldf} for each
document dialect (where \meta{script} is the four letter
script identifier, such as \texttt{Latn}).

This package provides some shortcut commands that use
\ics{printunsrtglossary} if the relevant package option has defined
the associated glossary: \ics{printunsrtabbreviations}, 
\ics{printunsrtacronyms}, \ics{printunsrtsymbols},
\ics{printunsrtnumbers} and \ics{printunsrtindex}.

The \pkgopt{savenumberlist} package option doesn't have any effect
with \gls{bib2gls}. The location lists are saved by default, so
\sty{glossaries-extra-bib2gls} patches \ics{glsentrynumberlist} and
\ics{glsdisplaynumberlist} to work with the \gloskey{location} field

\begin{definition}[\DescribeMacro\glsxtrprovidecommand]
\cs{glsxtrprovidecommand}\marg{cs}\oarg{n}\oarg{default}\marg{definition}
\end{definition}
This command is intended for use in \texttt{@preamble}.
It's simply defined to \cs{providecommand} in
\sty{glossaries-extra-bib2gls} but \gls{bib2gls}['s] interpreter
treats it as \cs{renewcommand}. This means that you can 
override \gls{bib2gls}['s] internal definition of a command
without overriding the command definition in the document
(if it's already defined before the resource file is input).
For example
\begin{verbatim}
@preamble{"\glsxtrprovidecommand{\int}{integral}"}
\end{verbatim}
This will force \gls{bib2gls} to treat \cs{int} as the word
\qt{integral} to assist sorting but if this preamble code is written
to the \texttt{.glstex} file (as it is by default) then it won't
override the current definition (provided by the kernel or redefined
by a package).

The helper commands in the resource files are defined using
\cs{providecommand}. For many of them, if you want to provide an
alternative definition then you need to define the command before
the resource file is loaded. There are a few that may be redefined
afterwards but if you use \cs{renewcommand} then you will get an
error on the first \LaTeX\ run when the \texttt{.glstex} file
doesn't exist. In this case, you may prefer to use:
\begin{definition}[\DescribeMacro\glsrenewcommand]
\cs{glsrenewcommand}\marg{cs}\oarg{n}\oarg{default}\marg{definition}
\end{definition}
This behaves like \cs{renewcommand} but only generates a warning
rather than an error if the command isn't already defined so it
won't interrupt the document build.

\begin{definition}[\DescribeMacro\GlsXtrIndexCounterLink]
\cs{GlsXtrIndexCounterLink}\marg{text}\marg{label}
\end{definition}
If the \ics{hyperref} command has been defined (that is,
\sty{hyperref} has been loaded before \styfmt{glossaries-extra})
then this command checks for the existence of the
\texttt{indexcounter} field. If this field is set for the entry
given by \meta{label}, this command does
\cs{hyperref}\texttt{[wrglossary.\meta{value}]\marg{text}}, where
\meta{value} is the value of the \texttt{indexcounter} field. If the
field isn't set or if \cs{hyperref} hasn't been defined, this just
does \meta{text}. This command is provided for use with the
\pkgopt{indexcounter} package option combined with \gls{bib2gls}['s]
\texttt{save-index-counter} resource option. See the \gls{bib2gls}
manual for further details (at least version 1.4).

If you use the \texttt{set-widest} resource option, \gls{bib2gls}
v1.8+ will now use:
\begin{definition}[\DescribeMacro\glsxtrSetWidest]
\cs{glsxtrSetWidest}\marg{type}\marg{level}\marg{text}
\end{definition}
(if it has been defined) to set the widest name for the given
glossary type and level. This allows for both the \glostyle{alttree}
style and the styles provided by \sty{glossary-longextra}, which
need to know the widest name.

If \gls{bib2gls} can't determine the widest name (typically because
the name field consists of commands that aren't recognised by the
interpreter) then \gls{bib2gls} v1.8+ will now use:
\begin{definition}[\DescribeMacro\glsxtrSetWidestFallback]
\cs{glsxtrSetWidestFallback}\marg{max depth}\marg{list}
\end{definition}
(if defined). Currently the maximum hierarchical depth \meta{max
depth} may only be 0 or 2. This command requires commands provided
by the \sty{glossaries-extra-stylemods} package with the
\glostyle{alttree} style enabled.  In this case, it may be simpler
to just use \ics{glssetwidest}.

\subsection{Supplemental Locations}
\label{sec:supplocations}

\begin{definition}[\DescribeMacro\glsxtrdisplaysupploc]
\cs{glsxtrdisplaysupploc}\marg{prefix}\marg{counter}\marg{format}\marg{src}\marg{location}
\end{definition}
This is used by \gls{bib2gls} version 1.7+ for supplemental
locations, instead of using \cs{glsxtrsupphypernumber} with the
\catattr{externallocation} attribute. This command sets up the
location counter and prefix (used in the formation of hyperlinks)
and then uses
\begin{definition}[\DescribeMacro\glsxtrmultisupplocation]
\cs{glsxtrmultisupplocation}\marg{location}\marg{src}\marg{format}
\end{definition}
to format the actual location (with an external hyperlink, if
supported).

\subsection{Nameref Record}
\label{sec:recordnameref}

Normally locations are recorded in the \texttt{.aux} file
in the form:
\begin{definition}
\cs{glsxtr@record}\marg{label}\marg{prefix}\marg{counter}\marg{format}\marg{location}
\end{definition}
The \pkgopt[nameref]{record} option, which requires at least
\gls{bib2gls} v1.8, instead uses:
\begin{definition}
\cs{glsxtr@record@nameref}\marg{label}\marg{prefix}\marg{counter}\marg{format}\marg{location}\marg{title}\marg{href}\marg{hcounter}
\end{definition}
where \meta{title} is obtained from \cs{@currentlabelname}
and \meta{href} is obtained from \cs{@currentHref}. These
commands require \sty{hyperref}. If they are undefined,
\meta{title} and \meta{href} will be left empty and \gls{bib2gls}
will treat it as a regular record. 

\begin{important}
Be careful with this option as \meta{href} will globally change on every
instance of \cs{refstepcounter} but \meta{title} won't necessarily
change. It can therefore cause unexpected behaviour.
\end{important}

The final argument \meta{hcounter} is obtained from
\cs{theH}\meta{counter} which provides the partial target name associated
with the indexing counter. With the original
\gls{makeindex}\slash \gls{xindy} approach, it's not possible to
include this information in the location, so the base
\sty{glossaries} package attempts to derive a prefix from which the
\meta{hcounter} value can be reconstituted by appending the prefix.
Unfortunately, not all definitions of \cs{theH}\meta{counter} are in
the form \meta{prefix}\cs{the}\meta{counter} (most notably the
\ctr{equation} counter with chapters) so this can fail.

Since \gls{bib2gls} is customized specifically for use with
\styfmt{glossaries-extra}, it's now possible to save
\meta{hcounter}, so the \pkgopt[nameref]{record} option does this.
By providing both \meta{href} and \meta{hcounter}, you can determine
which target you would rather use. The default is to use
\meta{hcounter}, which will take you to the place where the
corresponding counter was incremented with \cs{refstepcounter}.
However, you may choose to switch to using the \meta{href} target,
which will take you to the nearest target before the indexing took
place.

With \gls{bib2gls} v1.8+, normal locations are displayed using:
\begin{definition}
\cs{glsnoidxdisplayloc}\marg{prefix}\marg{counter}\marg{format}\marg{location}
\end{definition}
This is provided by the base \sty{glossaries} package and is simply
defined to do:
\begin{definition}
\cs{setentrycounter}\oarg{prefix}\marg{counter}\cs{csuse}\marg{format}\marg{location}
\end{definition}
Earlier versions of \gls{bib2gls} only used this in the
\gloskey{loclist} field and explicitly used \cs{setentrycounter} in
the \gloskey{location} field followed by
\cs{}\meta{format}\marg{location}, which follows the code that's created
with the default \pkgopt{makeindex} setting.
The \cs{setentrycounter} command sets up the prefix needed for
\cs{glshypernumber} to reform the target name from the given
location.

The locations identified by \cs{glsxtr@record@nameref} are written
by \gls{bib2gls} to the location list using:
\begin{definition}[\DescribeMacro\glsxtrdisplaylocnameref]
\cs{glsxtrdisplaylocnameref}\marg{prefix}\marg{counter}\marg{format}\marg{location}\marg{title}\marg{href}\marg{hcounter}\marg{file}
\end{definition}
With normal internal locations, \meta{file} will always be empty.
With supplemental locations, \meta{file} will be the external file
reference.

The default definition is:
\begin{verbatim}
\newcommand*{\glsxtrdisplaylocnameref}[8]{%
  \ifstrequal{#2}{equation}%
  {\glsxtrnamereflink{#3}{(#4)}{#2.#7}{#8}}%
  {%
     \ifstrempty{#5}%
     {%
       \glsxtrnamereflink{#3}{#4}{#2.#7}{#8}%
     }%
     {%
       \ifstrequal{#2}{page}%
       {\glsxtrnamereflink{#3}{#4}{#2.#7}{#8}}%
       {\glsxtrnamereflink{#3}{#5}{#2.#7}{#8}}%
     }%
  }%
}
\end{verbatim}
which uses:
\begin{definition}[\DescribeMacro\glsxtrnamereflink]
\cs{glsxtrnamereflink}\marg{format}\marg{title}\marg{target}\marg{file}
\end{definition}
This ignores the \meta{prefix}, \meta{counter} and \marg{location}
arguments and instead creates a hyperlink with the target name obtained from \meta{target} (and \meta{file}, if not empty).

Since pages and equations typically don't have titles, the default
definition of \cs{glsxtrdisplaylocnameref} checks the counter was
used as the location. If it's \ctr{page} or if \meta{title} is empty, then just the location is
used as the hyperlink text. If the counter \ctr{equation}, then the
text is the location in parentheses. Otherwise the text is obtained
from \meta{title}.

If \meta{file} is empty an internal link is created with:
\begin{definition}[\DescribeMacro\glsxtrfmtinternalnameref]
\cs{glsxtrfmtinternalnameref}\marg{target}\marg{format}\marg{title}
\end{definition}
otherwise an external link is created with:
\begin{definition}[\DescribeMacro\glsxtrfmtexternalnameref]
\cs{glsxtrfmtexternalnameref}\marg{target}\marg{format}\marg{title}\marg{file}
\end{definition}
The \meta{file} argument is set by \gls{bib2gls} for supplemental
locations.

Here's alternative definition that uses the \meta{prefix} and
\meta{counter} to reform the target name (as
\cs{glsnoidxdisplayloc}) but uses the \meta{title} as the hyperlink text:
\begin{verbatim}
\renewcommand*{\glsxtrdisplaylocnameref}[7]{%
 \glsxtrnamereflink{#3}{#5}{#6}{#7}%
}
\end{verbatim}
which uses:
\begin{definition}[\DescribeMacro\glsxtrnameloclink]
\cs{glsxtrnameloclink}\marg{prefix}\marg{counter}\marg{format}\marg{location}\marg{title}\marg{file}
\end{definition}
This uses the same commands as \cs{glsxtrnamereflink}
to produce the hyperlinks.

In both cases, the link is encapsulated with the text-block command
whose name is given by \meta{format}, but \cs{glshypernumber} is
first locally redefined to \cs{@firstofone} to prevent a conflict with the
usual location hyperlink formation. This means that if the
\meta{format} is \texttt{hyperbf} then it will simply behave like
\texttt{textbf}.

For compactness, \gls{bib2gls} merges normal records if the
\meta{prefix}, \meta{counter} and \meta{location} all match. (An
order of precedence can be provided for format conflicts.) With
nameref records, you can use the \verb|--merge-nameref-on| switch 
provided by \gls{bib2gls} v1.8+ to determine how to merge nameref
records. This switch must be followed by one of the following
keywords: \texttt{hcounter} (merge on \meta{hcounter}, default)
\texttt{href} (merge on \meta{href}), \texttt{title} (merge
on \meta{title}) and \texttt{location} (merge on \meta{location}, as
regular records).  In all cases, the \meta{counter} must also match.

\subsection{Helper Commands for Resource Options}
\label{sec:resourcecommands}

\begin{definition}[\DescribeMacro\glshex]
\cs{glshex}
\end{definition}
This is just defined as \verb|\string\u|, which is required
when you need to indicate a Unicode character in the form
\cs{u}\meta{hex} in some of
the resource options (as illustrated above).

\begin{definition}[\DescribeMacro\glscapturedgroup]
\cs{glscapturedgroup}
\end{definition}
This is just defined as \verb|\string\$| and is used for
the captured group reference in a replacement part of a regular
expression substitution (requires at least \gls{bib2gls} version 1.5). For example:
\begin{verbatim}
sort-replace={{([a-zA-Z])\string\.}{\glscapturedgroup1}}
\end{verbatim}
This only removes a full stop that follows any of the characters
a,\ldots,z or A,\ldots,Z.

If you use the \texttt{save-child-count} resource option,
you can test if the \texttt{childcount} field is non-zero using:
\begin{definition}[\DescribeMacro{\GlsXtrIfHasNonZeroChildCount}]
\cs{GlsXtrIfHasNonZeroChildCount}\marg{label}\marg{true}\marg{false}
\end{definition}
This internally uses \cs{GlsXtrIfFieldNonZero} and will do
\meta{false} if the field isn't set. Within \meta{true} and
\meta{false} you can use \cs{glscurrentfieldvalue} to access the
value. (It will be 0 in \meta{false} if the field isn't set.)


A convenient shortcut for use in the \texttt{entry-type-aliases}
setting: 
\begin{definition}[\DescribeMacro\GlsXtrBibTeXEntryAliases]
\cs{GlsXtrBibTeXEntryAliases}
\end{definition}
This provides aliases for Bib\TeX's standard entry types to
\gls{bib2gls}['s] \texttt{@bibtexentry} entry type (requires at
least \gls{bib2gls} version 1.4).

You may also want to provide storage keys for Bib\TeX's standard
fields rather than having to alias them all. This can be done with:
\begin{definition}[\DescribeMacro\GlsXtrProvideBibTeXFields]
\cs{GlsXtrProvideBibTeXFields}
\end{definition}
Note that Bib\TeX's \texttt{type} field clashes with the \sty{glossaries}
package's \gloskey{type} key, so this command provides the key
\texttt{bibtextype} instead. You can alias it with
\texttt{field-aliases={type=bibtextype}} in the resource options.
Each storage key is provided with a convenient command to access the
value in the form \cs{glsxtrbib}\meta{field}. For example,
\cs{glsxtrbibaddress}. The \texttt{bibtextype} field can be accessed
with \cs{glsxtrbibtype}. Each of these commands takes the entry
label as the sole argument.

The \sty{glossaries-extra-bib2gls} package also provides definitions
of the missing mathematical Greek commands: \ics{Alpha},
\ics{Beta}, \ics{Epsilon}, \ics{Zeta}, \ics{Eta}, \ics{Iota},
\ics{Kappa}, \ics{Mu}, \ics{Nu}, \ics{Omicron}, \ics{Rho},
\ics{Tau}, \ics{Chi}, \ics{Digamma}, \ics{omicron}.
These are all defined with \cs{providecommand}, so they won't override
any definitions provided by any package loaded before
\sty{glossaries-extra}. Since \gls{bib2gls}['s] interpreter
recognises these commands, using them instead of explicitly using
the Latin characters with the same shape helps to keep the Greek
symbols together when sorting.
Similarly, if \sty{upgreek} has been loaded, the missing upright
Greek commands are also provided.

The remaining commands provide common rule blocks for use in
the \texttt{sort-rule} resource option. If you want a rule for
a specific locale, you can provide similar commands in a file called
\texttt{glossariesxtr-}\meta{tag}\texttt{.ldf}, where \meta{tag}
identifies the dialect, locale, region or root language. See the
description of \cs{IfTrackedLanguageFileExists} in the 
\sty{tracklang} documentation for further details. If this file is
on \TeX's path and the \sty{tracklang} package (automatically loaded
by \styfmt{glossaries}) detects that the document has requested that
language or locale, then the file will automatically be loaded.
For example, if you want to provide a rule block for Welsh, then
create a file called \texttt{glossariesxtr-welsh.ldf} that contains:
\begin{verbatim}
\ProvidesGlossariesExtraLang{welsh}[2018/02/23 v1.0]

\@ifpackageloaded{glossaries-extra-bib2gls}
{
  \newcommand{\glsxtrWelshRules}{%
   \glsxtrLatinA
   \string<b,B
   \string<c,C
   \string<ch,CH
   \string<d,D
   \string<dd,DD
   \string<\glsxtrLatinE
   \string<f,F
   \string<ff,FF
   \string<g,G
   \string<ng,NG
   \string<\glsxtrLatinH
   \string<\glsxtrLatinI
   \string<j,J
   \string<\glsxtrLatinL
   \string<ll,LL
   \string<\glsxtrLatinM
   \string<\glsxtrLatinN
   \string<\glsxtrLatinO
   \string<\glsxtrLatinP
   \string<ph,PH
   \string<r,R
   \string<rh,RH
   \string<\glsxtrLatinS 
   \string<\glsxtrLatinT
   \string<th,TH
   \string<u,U
   \string<w,W
   \string<y,Y
  }
}
{}% glossaries-extra-bib2gls.sty not loaded
\end{verbatim}
(The use of \cs{string} is in case the \texttt{<} character
has been made active.) You can provide more than one rule-block
per local, to allow for loanwords or foreign words. For example, 
you could provide \cs{glsxtrWelshIRules}, \cs{glsxtrWelshIIRules}
etc.

If the rules are for a particular script (independent of language
or region) then they can be provided in a file given by
\texttt{glossariesxtr-}\meta{script}\texttt{.ldf} instead. For
example, the file \texttt{glossariesxtr-Cyrl.ldf} could contain:
\begin{verbatim}
\ProvidesGlossariesExtraLang{Cyrl}[2018/02/23 v1.0]
\newcommand*{\glsxtrGeneralCyrillicIRules}{%
  % Cyrillic rules
}
\newcommand*{\glsxtrGeneralCyrillicIIRules}{%
  % an alternative set of Cyrillic rules
}
\end{verbatim}
(Remember that the required document language scripts need to be tracked
through the \sty{tracklang} package, in order for these files to be
automatically loaded. This essentially means ensuring you load the
appropriate language package before \sty{tracklang} is loaded by
the base \sty{glossaries} package or any other package that uses it.
See the \sty{tracklang} documentation for further details.)

Alternatively, if the rules are specific to a subject rather than
a region or language, then you can provide a supplementary
package. For example, if you have a package called, say, 
\styfmt{mapsymbols} that provides map symbols, then the file
\texttt{mapsymbols.sty} might contain:
\begin{verbatim}
\NeedsTeXFormat{LaTeX2e}
\ProvidesPackage{mapsymbols}
% some package or font loading stuff here to provide
% the appropriate symbols
\newcommand{\Stadium}{...}
\newcommand{\Battlefield}{...}
\newcommand{\Harbour}{...}
% etc

% Provide a rule block:
\newcommand{\MapSymbolOrder}{%
 \glshex 2694 % crossed-swords 0x2694
 \string< \glshex 2693 % anchor 0x2693
 \string< \glshex 26BD % football 0x26BD
}
\end{verbatim}
and the supplementary file \texttt{mapsymbols.bib} can provide
the appropriate definitions for \gls{bib2gls}:
\begin{verbatim}
@preamble{"\glsxtrprovidecommand{\Harbour}{\char"2693}
\glsxtrprovidecommand{\Battlefield}{\char"2694}
\glsxtrprovidecommand{\Stadium}{\char"26BD}"}
\end{verbatim}
Now both the preamble and rule block can be used in the resource
set:
\begin{verbatim}
\usepackage{mapsymbols}% my custom package
\usepackage[record]{glossaries-extra}

\GlsXtrLoadResources[
  src={mapsymbols,% <--- my custom mapsymbols.bib
   entries% data in entries.bib
  },
  sort={custom},
  sort-rule={\glsxtrcontrolrules % control codes
 ;\glsxtrspacerules % space characters
 ;\glsxtrnonprintablerules % non-printable characters
 ;\glsxtrcombiningdiacriticrules % combining diacritics
 ,\glsxtrhyphenrules % hyphens
 <\glsxtrgeneralpuncrules % general punctuation
 <\glsxtrdigitrules % 0, ..., 9
 <\glsxtrfractionrules % fraction symbols
 <\MapSymbolOrder % <--- custom map symbols
 <\glsxtrMathItalicGreekIrules % math-greek symbols
 <\glsxtrGeneralLatinIrules % Latin letters
 }
]
\end{verbatim}

The following commands are provided by
\sty{glossaries-extra-bib2gls}. They should be separated by
the rule separator characters \texttt{;}~(semi-colon) or \texttt{,}~(comma) 
or \verb|&|~(ampersand) or \texttt{<}~(less than). See Java's
\href{http://docs.oracle.com/javase/8/docs/api/java/text/RuleBasedCollator.html}{Rule\-Based\-Collator}
documentation for details of the rule syntax.

For example, the following will place the mathematical Greek
symbols (\ics{alpha}, \ics{Alpha}, \ics{beta}, \ics{Beta} etc)
in a block before Latin characters:
\begin{verbatim}
sort-rule={\glsxtrcontrolrules
 ;\glsxtrspacerules
 ;\glsxtrnonprintablerules
 ;\glsxtrcombiningdiacriticrules
 ,\glsxtrhyphenrules
 <\glsxtrgeneralpuncrules
 <\glsxtrdigitrules
 <\glsxtrfractionrules
 <\glsxtrMathItalicGreekIrules
 <\glsxtrGeneralLatinIVrules
 <\glsxtrLatinAA
 <\glsxtrLatinOslash
}
\end{verbatim}

\begin{definition}[\DescribeMacro\glsxtrcontrolrules]
\cs{glsxtrcontrolrules}
\end{definition}
These are control characters that are usually placed at the start of
a rule in the ignored section. They typically won't occur in
any sort values, but if they do they should normally be ignored.

\begin{definition}[\DescribeMacro\glsxtrspacerules]
\cs{glsxtrspacerules}
\end{definition}
These are space characters. They typically come after the control
characters with the two blocks separated by a \texttt{;}~(semi-colon).

\begin{definition}[\DescribeMacro\glsxtrnonprintablerules]
\cs{glsxtrnonprintablerules}
\end{definition}
These are non-printable characters (BOM, tabs, line feed and
carriage return). They typically come after the spaces separated by
a \texttt{;}~(semi-colon). These characters aren't checked for by
\gls{bib2gls} when it determines whether or not to use the
interpreter, so a~TAB or newline character may end up in the sort
value if it wasn't interpreted.

\begin{definition}[\DescribeMacro\glsxtrcombiningdiacriticrules]
\cs{glsxtrcombiningdiacriticrules}
\end{definition}
These are combining diacritic marks which typically follow
the space and non-printable blocks (separated by a semi-colon). This command is 
defined in terms of sub-block commands:
\begin{verbatim}
\newcommand*{\glsxtrcombiningdiacriticrules}{%
 \glsxtrcombiningdiacriticIrules\string;
 \glsxtrcombiningdiacriticIIrules\string;
 \glsxtrcombiningdiacriticIIIrules\string;
 \glsxtrcombiningdiacriticIVrules
}
\end{verbatim}
If you prefer, you can use the sub-blocks directly in your
required ordered. 

\begin{definition}[\DescribeMacro\glsxtrcombiningdiacriticIrules]
\cs{glsxtrcombiningdiacriticIrules}
\end{definition}
This contains the combining diacritics: acute, grave, breve,
circumflex, caron, ring, vertical line above, diaeresis (umlaut), 
double acute, tilde, dot above, combining macron.

\begin{definition}[\DescribeMacro\glsxtrcombiningdiacriticIIrules]
\cs{glsxtrcombiningdiacriticIIrules}
\end{definition}
This contains the combining diacritics: short solidus overlay,
cedilla, ogonek, dot below, low line, overline, hook above,
double vertical line above, double grave accent, candrabindu,
inverted breve, turned comma above, comma above, reversed comma
above, comma above right, grave accent below, acute accent below.

\begin{definition}[\DescribeMacro\glsxtrcombiningdiacriticIIIrules]
\cs{glsxtrcombiningdiacriticIIIrules}
\end{definition}
This contains the combining diacritics: left tack below, right tack
below, left angle above, horn, left half ring below, up tack below,
down tack below, plus sign below, minus sign below, palatalized hook
below, retroflex hook below, diaresis below, ring below, comma
below, vertical line below, bridge below, inverted double arch
below, caron below, circumflex accent below, breve below, inverted
breve below, tilde below, macron below, double low line, tilde
overlay, short stroke overlay, long stroke overlay, long solidus
overlay, right half ring below, inverted bridge below, square below,
seagull below, x above, vertical tilde, double overline, Greek
perispomeni, Greek dialytika tonos, Greek ypogegrammeni, double
tilde, double inverted breve, Cyrillic titlo, Cyrillic
palatalization, Cyrillic dasia pneumata, Cyrillic psili pneumata.

\begin{definition}[\DescribeMacro\glsxtrcombiningdiacriticIVrules]
\cs{glsxtrcombiningdiacriticIVrules}
\end{definition}
This contains the combining diacritics: 
left harpoon above, right harpoon above, long vertical line overlay,
short vertical line overlay, anticlockwise arrow above, clockwise
arrow above, left arrow above, right arrow above, ring overlay,
clockwise ring overlay, anticlockwise ring overlay, three dots
above, four dots above, enclosing circle, enclosing square,
enclosing diamond, enclosing circle backslash, left right arrow
above.

\begin{definition}[\DescribeMacro\glsxtrhyphenrules]
\cs{glsxtrhyphenrules}
\end{definition}
This contains hyphens (including the minus sign 0x2212). This rule
block typically comes after the diacritic rules separated by a
comma.

\begin{definition}[\DescribeMacro\glsxtrgeneralpuncrules]
\cs{glsxtrgeneralpuncrules}
\end{definition}
This contains punctuation characters. This rule
block typically comes after the hyphen rules separated by a
less than (\texttt{<}). As with the combining diacritics, this
command is defined in terms of sub-blocks which may be used directly
instead if a different order is required:
\begin{verbatim}
\newcommand*{\glsxtrgeneralpuncrules}{%
  \glsxtrgeneralpuncIrules
  \string<\glsxtrcurrencyrules
  \string<\glsxtrgeneralpuncIIrules
}
\end{verbatim}

\begin{definition}[\DescribeMacro\glsxtrgeneralpuncIrules]
\cs{glsxtrgeneralpuncIrules}
\end{definition}
This is the first punctuation sub-block containing:
underscore, macron, comma, semi-colon, colon, exclamation mark,
inverted exclamation mark, question mark, inverted question mark,
solidus, full stop, acute accent, grave accent, circumflex accent,
diaersis, tilde, middle dot, cedilla, straight apostrophe, straight
double quote, left guillemet, right guillemet, left parenthesis,
right parenthesis, left square bracket, right square bracket, left
curly bracket, right curly bracket, section sign, pilcrow sign,
copyright sign, registered sign, at sign.

\begin{definition}[\DescribeMacro\glsxtrcurrencyrules]
\cs{glsxtrcurrencyrules}
\end{definition}
This sub-block contains some currency symbols:
currency sign, Thai currency symbol baht, cent sign, colon sign,
cruzeiro sign, dollar sign, dong sign, euro sign, French franc sign,
lira sign, mill sign, naira sign, peseta sign, pound sign, rupee
sign, new sheqel sign, won sign, yen sign.

\begin{definition}[\DescribeMacro\glsxtrgeneralpuncIIrules]
\cs{glsxtrgeneralpuncIIrules}
\end{definition}
This sub-block contains some other punctuation symbols:
asterisk, backslash, ampersand, hash sign, percent sign, plus sign,
plus-minus sign, division sign, multiplication sign, less-than sign,
equals sign, greater-than sign, not sign, vertical bar (pipe),
broken bar, degree sign, micron sign.

\begin{definition}[\DescribeMacro\glsxtrdigitrules]
\cs{glsxtrdigitrules}
\end{definition}
This rule block contains the Basic Latin digits (0, \ldots, 9)
and the subscript and superscript digits (\textsubscript{0}
\textsuperscript{0} etc) made equivalent to the
corresponding Basic Latin digit. The digit block typically
comes after the punctuation rules separated by a less than
(\texttt{<}).

\begin{definition}[\DescribeMacro\glsxtrBasicDigitrules]
\cs{glsxtrBasicDigitrules}
\end{definition}
This rule block contains just the Basic Latin digits (0, \ldots, 9).

\begin{definition}[\DescribeMacro\glsxtrSubScriptDigitrules]
\cs{glsxtrSubScriptDigitrules}
\end{definition}
This rule block contains just the subscript digits
(\textsubscript{0} \ldots\ \textsubscript{9}).

\begin{definition}[\DescribeMacro\glsxtrSuperScriptDigitrules]
\cs{glsxtrSuperScriptDigitrules}
\end{definition}
This rule block contains just the superscript digits
(\textsuperscript{0} \ldots\ \textsuperscript{9}).

\begin{definition}[\DescribeMacro\glsxtrfractionrules]
\cs{glsxtrfractionrules}
\end{definition}
This rule block contains vulgar fraction characters. The digit block
typically comes after the digit rules separated by a less than
(\texttt{<}).

There are a number of Latin rule blocks. Some of these included
extended characters or ligatures (such as \ss\ or \oe) but they don't
include accented characters. If you require a Latin rule block
that includes accented characters, digraphs, trigraphs or 
other extended characters, then it's best to provide
similar commands in a \texttt{glossariesxtr-}\meta{tag}\texttt{.ldf}
file for the particular language or region.

\begin{definition}[\DescribeMacro\glsxtrGeneralLatinIrules]
\cs{glsxtrGeneralLatinIrules}
\end{definition}
This is just the basic (non-extended) Latin alphabet with the superscript
and subscript Latin letters (\textsuperscript{a} \textsubscript{a}
etc) treated as the equivalent basic Latin letter. (If you don't
want the subscripts and superscripts included you can redefine
\cs{glsxtrLatinA} etc to omit them.)

\begin{definition}[\DescribeMacro\glsxtrGeneralLatinIIrules]
\cs{glsxtrGeneralLatinIIrules}
\end{definition}
This is like \cs{glsxtrGeneralLatinIrules} but it includes eth
(Ð) between `D' and `E' and eszett (\ss) treated as `ss'.

\begin{definition}[\DescribeMacro\glsxtrGeneralLatinIIIrules]
\cs{glsxtrGeneralLatinIIIrules}
\end{definition}
This is like \cs{glsxtrGeneralLatinIrules} but it includes eth
(Ð) between `D' and `E' and eszett (\ss) treated as `sz'.

\begin{definition}[\DescribeMacro\glsxtrGeneralLatinIVrules]
\cs{glsxtrGeneralLatinIVrules}
\end{definition}
This is like \cs{glsxtrGeneralLatinIrules} but it includes eth
(Ð) between `D' and `E', ae-ligature (\ae) is treated as `ae',
oe-ligature (\oe) is treated as `oe', eszett (\ss) treated as `ss'
and thorn (þ) is treated as `th'.

\begin{definition}[\DescribeMacro\glsxtrGeneralLatinVrules]
\cs{glsxtrGeneralLatinVrules}
\end{definition}
This is like \cs{glsxtrGeneralLatinIrules} but it includes eth
(Ð) between `D' and `E', eszett (\ss) treated as `ss'
and thorn (þ) treated as `th'.

\begin{definition}[\DescribeMacro\glsxtrGeneralLatinVIrules]
\cs{glsxtrGeneralLatinVIrules}
\end{definition}
This is like \cs{glsxtrGeneralLatinIrules} but it includes eth
(Ð) between `D' and `E', eszett (\ss) treated as `sz'
and thorn (þ) treated as `th'.

\begin{definition}[\DescribeMacro\glsxtrGeneralLatinVIIrules]
\cs{glsxtrGeneralLatinVIIrules}
\end{definition}
This is like \cs{glsxtrGeneralLatinIrules} but it includes
ae-ligature (\ae) between `A' and `B', eth (Ð) between `D' and
`E', insular G (Ᵹ) instead of `G', oe-ligature between `O' and `P', 
long~s (ſ) equivalent to `s', thorn (þ) between `T' and `U' and wynn (Ƿ)
instead of W.

\begin{definition}[\DescribeMacro\glsxtrGeneralLatinVIIIrules]
\cs{glsxtrGeneralLatinVIIIrules}
\end{definition}
This is like \cs{glsxtrGeneralLatinIrules} but ae-ligature (\ae) is treated as `ae',
oe-ligature (\oe) is treated as `oe', eszett (\ss) treated as `ss',
thorn (þ) is treated as `th', \O\ is treated as `O' and
`\L' is treated as `L'.

\begin{definition}[\DescribeMacro\glsxtrLatinA]
\cs{glsxtrLatinA}
\end{definition}
A mini-rule that just covers `A' but includes the sub- and
superscript A.

\begin{definition}[\DescribeMacro\glsxtrLatinE]
\cs{glsxtrLatinE}
\end{definition}
A mini-rule that just covers `E' but includes the 
subscript E.

\begin{definition}[\DescribeMacro\glsxtrLatinH]
\cs{glsxtrLatinH}
\end{definition}
A mini-rule that just covers `H' but includes the 
subscript H.

\begin{definition}[\DescribeMacro\glsxtrLatinK]
\cs{glsxtrLatinK}
\end{definition}
A mini-rule that just covers `K' but includes the 
subscript K.

\begin{definition}[\DescribeMacro\glsxtrLatinI]
\cs{glsxtrLatinI}
\end{definition}
A mini-rule that just covers `I' but includes the 
superscript I.

\begin{definition}[\DescribeMacro\glsxtrLatinL]
\cs{glsxtrLatinL}
\end{definition}
A mini-rule that just covers `L' but includes the 
subscript L.

\begin{definition}[\DescribeMacro\glsxtrLatinM]
\cs{glsxtrLatinM}
\end{definition}
A mini-rule that just covers `M' but includes the 
subscript M.

\begin{definition}[\DescribeMacro\glsxtrLatinN]
\cs{glsxtrLatinN}
\end{definition}
A mini-rule that just covers `N' but includes the 
sub- and superscript N.

\begin{definition}[\DescribeMacro\glsxtrLatinO]
\cs{glsxtrLatinO}
\end{definition}
A mini-rule that just covers `O' but includes the 
sub- and superscript O.

\begin{definition}[\DescribeMacro\glsxtrLatinP]
\cs{glsxtrLatinP}
\end{definition}
A mini-rule that just covers `P' but includes the 
subscript P.

\begin{definition}[\DescribeMacro\glsxtrLatinS]
\cs{glsxtrLatinS}
\end{definition}
A mini-rule that just covers `S' but includes the 
subscript S.

\begin{definition}[\DescribeMacro\glsxtrLatinT]
\cs{glsxtrLatinT}
\end{definition}
A mini-rule that just covers `T' but includes the 
subscript T.

\begin{definition}[\DescribeMacro\glsxtrLatinX]
\cs{glsxtrLatinX}
\end{definition}
A mini-rule that just covers `X' but includes the 
subscript X.

\begin{definition}[\DescribeMacro\glsxtrLatinEszettSs]
\cs{glsxtrLatinEszettSs}
\end{definition}
A mini-rule that just covers eszett (\ss) and makes long~s (ſ) followed
by short `s' equivalent to `\ss'. (This is used in the above blocks
that treat `\ss' as `ss'.)

\begin{definition}[\DescribeMacro\glsxtrLatinEszettSz]
\cs{glsxtrLatinEszettSz}
\end{definition}
A mini-rule that just covers eszett (\ss) and makes long~s (ſ) followed
by `z' equivalent to `\ss'. (This is used in the above blocks
that treat `\ss' as `sz'.)

\begin{definition}[\DescribeMacro\glsxtrLatinEth]
\cs{glsxtrLatinEth}
\end{definition}
A mini-rule for eth (Ð) so you don't need to remember the
Unicode values.

\begin{definition}[\DescribeMacro\glsxtrLatinThorn]
\cs{glsxtrLatinThorn}
\end{definition}
A mini-rule for thorn (þ) so you don't need to remember the
Unicode values.

\begin{definition}[\DescribeMacro\glsxtrLatinAELigature]
\cs{glsxtrLatinAELigature}
\end{definition}
A mini-rule for ae-ligature (\ae) so you don't need to remember the
Unicode values.

\begin{definition}[\DescribeMacro\glsxtrLatinOELigature]
\cs{glsxtrLatinOELigature}
\end{definition}
A mini-rule for oe-ligature (\oe) so you don't need to remember the
Unicode values.

\begin{definition}[\DescribeMacro\glsxtrLatinOslash]
\cs{glsxtrLatinOslash}
\end{definition}
A mini-rule for `\O' so you don't need to remember the
Unicode values.

\begin{definition}[\DescribeMacro\glsxtrLatinLslash]
\cs{glsxtrLatinLslash}
\end{definition}
A mini-rule for `\L' so you don't need to remember the
Unicode values.

\begin{definition}[\DescribeMacro\glsxtrLatinWynn]
\cs{glsxtrLatinWynn}
\end{definition}
A mini-rule for wynn (Ƿ) so you don't need to remember the
Unicode values.

\begin{definition}[\DescribeMacro\glsxtrLatinInsularG]
\cs{glsxtrLatinInsularG}
\end{definition}
A mini-rule for insular G (Ᵹ) so you don't need to remember the
Unicode values.

\begin{definition}[\DescribeMacro\glsxtrLatinSchwa]
\cs{glsxtrLatinSchwa}
\end{definition}
A mini-rule that just covers schwa (Ə) but includes the 
subscript schwa. (Not used in any of the provided Latin rule blocks
described above.)

\begin{definition}[\DescribeMacro\glsxtrLatinAA]
\cs{glsxtrLatinAA}
\end{definition}
A mini-rule for `\AA' so you don't need to remember the
Unicode values. (Not used in any of the provided Latin rule blocks
described above.)

\begin{definition}[\DescribeMacro\glsxtrMathGreekIrules]
\cs{glsxtrMathGreekIrules}
\end{definition}
A rule block for mathematical Greek (\ics{alpha}, \ics{beta} etc)
and upright Greek (\ics{upalpha}, etc, from the \sty{upgreek}
package) characters that includes digamma (\ics{digamma} and
\ics{Digamma}) between epsilon and zeta. The upright and italic
versions are gathered together into the same letter group.

\begin{definition}[\DescribeMacro\glsxtrMathGreekIIrules]
\cs{glsxtrMathGreekIIrules}
\end{definition}
As \cs{glsxtrMathGreekIrules} but doesn't include digamma.

\begin{definition}[\DescribeMacro\glsxtrMathUpGreekIrules]
\cs{glsxtrMathUpGreekIrules}
\end{definition}
A rule block for upright Greek (\ics{upalpha}, etc, from
the \sty{upgreek} package)
characters that includes digamma (\ics{digamma} and \ics{Digamma})
between epsilon and zeta.

\begin{definition}[\DescribeMacro\glsxtrMathUpGreekIIrules]
\cs{glsxtrMathUpGreekIIrules}
\end{definition}
A rule block for upright Greek (\ics{upalpha}, etc, from
the \sty{upgreek} package) that doesn't include digamma.

\begin{definition}[\DescribeMacro\glsxtrMathItalicGreekIrules]
\cs{glsxtrMathItalicGreekIrules}
\end{definition}
A rule block for mathematical Greek (\ics{alpha}, \ics{Alpha}, etc)
characters that includes digamma (\ics{diagamma} and \ics{Digamma})
between epsilon and zeta. Note that even though the upper case
\cs{Delta} etc are actually rendered upright by \LaTeX,
\gls{bib2gls}['s] interpreter treats them as italic to help keep
them close to the lower case versions.

\begin{definition}[\DescribeMacro\glsxtrMathItalicGreekIIrules]
\cs{glsxtrMathItalicGreekIIrules}
\end{definition}
A rule block for mathematical Greek (\ics{alpha}, \ics{Alpha}, etc)
characters that doesn't include digamma.

\begin{definition}[\DescribeMacro\glsxtrMathItalicUpperGreekIrules]
\cs{glsxtrMathItalicUpperGreekIrules}
\end{definition}
A rule block for upper case mathematical Greek (\ics{Alpha},
\ics{Beta}, etc) characters that includes digamma (\ics{Digamma})
between epsilon and zeta.

\begin{definition}[\DescribeMacro\glsxtrMathItalicUpperGreekIIrules]
\cs{glsxtrMathItalicUpperGreekIIrules}
\end{definition}
A rule block for upper case mathematical Greek (\ics{Alpha},
\ics{Beta}, etc) characters that doesn't include digamma.

\begin{definition}[\DescribeMacro\glsxtrMathItalicLowerGreekIrules]
\cs{glsxtrMathItalicLowerGreekIrules}
\end{definition}
A rule block for lower case mathematical Greek (\ics{alpha},
\ics{beta}, etc) characters that includes digamma (\ics{digamma})
between epsilon and zeta.

\begin{definition}[\DescribeMacro\glsxtrMathItalicLowerGreekIIrules]
\cs{glsxtrMathItalicLowerGreekIIrules}
\end{definition}
A rule block for lower case mathematical Greek (\ics{alpha},
\ics{beta}, etc) characters that doesn't include digamma.

Additionally, there are commands in the form \cs{glsxtrUpAlpha},
\cs{glsxtrUpBeta} etc and \cs{glsxtrMathItalicAlpha},
\cs{glsxtrMathItalicBeta} etc
that just cover the upper and lower case forms of a special
Greek character (\cs{Upalpha}, \cs{upalpha} etc and \cs{Alpha},
\cs{alpha} etc) as well as the following:

\begin{definition}[\DescribeMacro\glsxtrMathItalicPartial]
\cs{glsxtrMathItalicPartial}
\end{definition}
The partial derivative symbol ($\partial$).

\begin{definition}[\DescribeMacro\glsxtrMathItalicNabla]
\cs{glsxtrMathItalicNabla}
\end{definition}
The nabla symbol ($\nabla$).


\section{Supplementary Commands}

These commands are provided by \styfmt{glossaries-extra}
for use with \gls{bib2gls}.

The information provided with \cs{GlsXtrLoadResources}
is written to the \texttt{.aux} file using
\begin{alltt}
\cs{protected@write}\cs{@auxout}\{\cs{glsxtrresourceinit}\}\marg{information}
\end{alltt}
where \meta{information} is the information to pass to
\gls{bib2gls}. The command in the second argument
\begin{definition}[\DescribeMacro\glsxtrresourceinit]
\cs{glsxtrresourceinit}
\end{definition}
may be used to temporarily redefine commands before the
information is written to the file. This does nothing
by default, but may be redefined to allow the use of
short commands for convenience. For example, with:
\begin{verbatim}
\renewcommand{\glsxtrresourceinit}{\let\u\glshex}
\end{verbatim}
you can just use, for example, \verb|\u E6| instead of
\verb|\string\uE6| in the custom rule. This redefinition of \cs{u}
is scoped so its original definition is restored after the write operation.

If you have multiple resource commands and you want a default set of
options you can supply them in the definition of:
\begin{definition}[\DescribeMacro\GlsXtrDefaultResourceOptions]
\cs{GlsXtrDefaultResourceOptions}
\end{definition}
For example:
\begin{verbatim}
\renewcommand{\GlsXtrDefaultResourceOptions}{selection=all}
\end{verbatim}
This should be done before the resource commands.

It's possible to specify label prefixes. For example, modifying
the earlier example:
\begin{verbatim}
\documentclass{article}

\usepackage[record,style=indexgroup]{glossaries-extra}

\setabbreviationstyle{long-short-desc}

\GlsXtrLoadResources
 [src={abbrvs},sort={letter-nocase},type=main,
  label-prefix={abr.},
  group={Abbreviations}]

\GlsXtrLoadResources
 [src={symbols},sort={use},type=main,
  label-prefix={sym.},
  group={Symbols}]

\GlsXtrLoadResources[src={terms},sort={en-GB},type=main
  label-prefix={trm.}]

\begin{document}
\gls{trm.bird}

\gls{abr.shtml}

\gls{sym.M}

\printunsrtglossaries
\end{document}
\end{verbatim}
If you do something like this, you may find it more convenient
to define custom commands that set the prefix. For example:
\begin{verbatim}
\newcommand*{\sym}[2][]{\gls[#1]{sym.#2}}
\end{verbatim}
The problem with this is that the custom command \cs{sym}
doesn't allow for modifiers (such as \cs{gls*} or \cs{gls+}).
Instead you can use:
\begin{definition}[\DescribeMacro\glsxtrnewgls]
\cs{glsxtrnewgls}\oarg{default options}\marg{prefix}\marg{cs}
\end{definition}
which defines the command \meta{cs} that behaves like
\begin{alltt}
\cs{gls}\meta{modifier}[\meta{default options},\meta{options}]\{\meta{prefix}\meta{label}\}\oarg{insert}
\end{alltt}
For example:
\begin{verbatim}
\glsxtrnewgls{sym.}{\sym}
\end{verbatim}
or (to default to no hyperlinks)
\begin{verbatim}
\glsxtrnewgls[hyper=false]{sym.}{\sym}
\end{verbatim}
now you can use \verb|\sym+{M}| to behave like \verb|\gls+{sym.M}|.

If you also want the plural and first letter upper case
versions you can use
\begin{definition}[\DescribeMacro\glsxtrnewglslike]
\cs{glsxtrnewglslike}\oarg{default
options}\marg{prefix}\marg{\cs{gls}-like cs}\marg{\cs{glspl}-like cs}\marg{\cs{Gls}-like cs}\marg{\cs{Glspl}-like cs}
\end{definition}
For example:
\begin{verbatim}
\glsxtrnewglslike[hyper=false]{idx.}{\idx}{\idxpl}{\Idx}{\Idxpl}
\end{verbatim}
For the all caps versions:
\begin{definition}[\DescribeMacro\glsxtrnewGLSlike]
\cs{glsxtrnewGLSlike}\oarg{default
options}\marg{prefix}\marg{\cs{GLS}-like cs}\marg{\cs{GLSpl}-like cs}
\end{definition}
For example:
\begin{verbatim}
\glsxtrnewGLSlike[hyper=false]{idx.}{\IDX}{\IDXpl}
\end{verbatim}

There's an analogous command for \cs{rgls}:
\begin{definition}[\DescribeMacro\glsxtrnewrgls]
\cs{glsxtrnewrgls}\oarg{default options}\marg{prefix}\marg{cs}
\end{definition}
and for \cs{rgls}, \cs{rglspl}, \cs{rGls}
and \cs{rGlspl}:
\begin{definition}[\DescribeMacro\glsxtrnewrglslike]
\cs{glsxtrnewrglslike}\oarg{default
options}\marg{prefix}\marg{\cs{rgls}-like cs}\marg{\cs{rglspl}-like
cs}\marg{\cs{rGls}-like cs}\marg{\cs{rGlspl}-like cs}
\end{definition}
and for the all caps:
\begin{definition}[\DescribeMacro\glsxtrnewrGLSlike]
\cs{glsxtrnewrGLSlike}\oarg{default
options}\marg{prefix}\marg{\cs{rGLS}-like cs}\marg{\cs{rGLSpl}-like cs}
\end{definition}

If you provide your own custom command with just 
\cs{newcommand} that has syntax that starts with
\oarg{options}\marg{label}, then you can notify \gls{bib2gls} using:
\begin{definition}[\DescribeMacro\glsxtridentifyglslike]
\cs{glsxtridentifyglslike}\marg{prefix}\marg{command}
\end{definition}
This writes information to the \texttt{.aux} file so that
\gls{bib2gls} can search for the given command when looking
for dependencies.

Another possibility is to set up known label prefixes, with each
one identified by:
\begin{definition}[\DescribeMacro\glsxtraddlabelprefix]
\cs{glsxtraddlabelprefix}\marg{prefix}
\end{definition}
These should be listed in order of precedence. Since no entries
are defined on the first \LaTeX\ run, the final prefix should
be the fallback. You can prepend a prefix to the list using:
\begin{definition}[\DescribeMacro\glsxtrprependlabelprefix]
\cs{glsxtrprependlabelprefix}\marg{prefix}
\end{definition}
which gives it the highest order of precedence.

The \meta{prefix} argument may be empty. You can clear the list
with:
\begin{definition}[\DescribeMacro\glsxtrclearlabelprefixes]
\cs{glsxtrclearlabelprefixes}
\end{definition}

You can test if a prefix is already in the list with:
\begin{definition}[\DescribeMacro\glsxtrifinlabelprefixlist]
\cs{glsxtrifinlabelprefixlist}\marg{prefix}\marg{true}\marg{false}
\end{definition}
In general it's best to avoid adding multiple instances of the same
prefix, so you can check with this command before adding a prefix to
the list. However, it can be useful to repeat a prefix at
the end of the list so that it can be used as a fallback for entries
that haven't yet been defined.

With the list of possible prefixes set up (including an empty
prefix if necessary), you can use:
\begin{definition}[\DescribeMacro\dgls]
\cs{dgls}\oarg{options}\marg{label}\oarg{insert}
\end{definition}
which behaves like
\begin{alltt}
\cs{gls}\oarg{options}\{\meta{prefix}\meta{label}\}\oarg{insert}
\end{alltt}
where \meta{prefix} is the first prefix in the list such that
\meta{prefix}\meta{label} matches a defined entry (or the last entry
in the prefix list if none match).

There are also analogous commands for the plural and case-changing
versions:
\begin{definition}[\DescribeMacro\dglspl]
\cs{dglspl}\oarg{options}\marg{label}\oarg{insert}
\end{definition}
(uses \cs{glspl}),
\begin{definition}[\DescribeMacro\dGls]
\cs{dGls}\oarg{options}\marg{label}\oarg{insert}
\end{definition}
(uses \cs{Gls}),
\begin{definition}[\DescribeMacro\dGlspl]
\cs{dGlspl}\oarg{options}\marg{label}\oarg{insert}
\end{definition}
(uses \cs{Glspl}),
\begin{definition}[\DescribeMacro\dGLS]
\cs{dGLS}\oarg{options}\marg{label}\oarg{insert}
\end{definition}
(uses \cs{GLS}),
\begin{definition}[\DescribeMacro\dGLSpl]
\cs{dGLSpl}\oarg{options}\marg{label}\oarg{insert}
\end{definition}
(uses \cs{GLSpl}),
\begin{definition}[\DescribeMacro\dglslink]
\cs{dglslink}\oarg{options}\marg{label}\oarg{insert}
\end{definition}
(uses \cs{glslink}),
\begin{definition}[\DescribeMacro\dglsdisp]
\cs{dglsdisp}\oarg{options}\marg{label}\oarg{insert}
\end{definition}
(uses \cs{glsdisp}).

These commands are essentially provided for a mixture of single and
dual entries. Suppose the file \texttt{entries.bib} contains:
\begin{verbatim}
@index{duck}
@dualindexabbreviation{svm,
  short={SVM},
  long={support vector machine}
}
@dualindexsymbol{pi,
  symbol={\ensuremath{\pi}},
  description={ratio of a circle's circumference to its 
    diameter}
}
\end{verbatim}
and suppose the document code is:
\begin{verbatim}
\documentclass{article}
\usepackage{hyperref}
\usepackage[record,abbreviations,symbols]{glossaries-extra}

\newcommand{\bibglsnewdualindexsymbolsecondary}[5]{%
 \longnewglossaryentry*{#1}{name={#3},category={symbol},
   type={symbols},symbol={#4},#2}{#5}%
}

\GlsXtrLoadResources[src=entries]

\begin{document}
First use: \gls{duck}, \gls{svm}, \gls{pi}.
Next use: \gls{duck}, \gls{svm}, \gls{pi}.
\printunsrtglossaries
\end{document}
\end{verbatim}
This uses the default empty primary prefix and \texttt{dual.}\ for
the dual prefix, so \verb|\gls{svm}| is referencing the primary
entry, which is (essentially) an \texttt{@index} type not an
abbreviation. It therefore doesn't follow the abbreviation style, and
it also hyperlinks to the index not to the list of abbreviations.
Similarly for \verb|\gls{pi}|, which references the primary 
\texttt{@index} entry rather than the symbol.

What's really needed is:
\begin{verbatim}
\gls{duck}, \gls{dual.svm}, \gls{dual.pi}.
\end{verbatim}
or with
\begin{verbatim}
\GlsXtrLoadResources[src=entries,
 label-prefix={idx.},
 dual-prefix={}
]
\end{verbatim}
then only the entries without a dual need a prefix:
\begin{verbatim}
\gls{idx.duck}, \gls{svm}, \gls{pi}.
\end{verbatim}
Using \cs{glsxtrnewglslike}, as earlier, this can be simplified to:
\begin{verbatim}
\idx{duck}, \gls{svm}, \gls{pi}.
\end{verbatim}
but this requires remembering which terms have duals.

An alternative is to use \cs{dgls} instead:
\begin{verbatim}
\documentclass{article}
\usepackage{hyperref}
\usepackage[record,abbreviations,symbols]{glossaries-extra}

\newcommand{\bibglsnewdualindexsymbolsecondary}[5]{%
 \longnewglossaryentry*{#1}{name={#3},category={symbol},
   type={symbols},symbol={#4},#2}{#5}%
}

\GlsXtrLoadResources[src=entries]

\glsxtraddlabelprefix{dual.}
\glsxtraddlabelprefix{}

\begin{document}
First use: \dgls{duck}, \dgls{svm}, \dgls{pi}.
Next use: \dgls{duck}, \dgls{svm}, \dgls{pi}.
\printunsrtglossaries
\end{document}
\end{verbatim}
On the first \LaTeX\ call (when the \texttt{.glstex} file doesn't exist),
neither \texttt{dual.svm} nor \texttt{svm} exists, so \cs{dgls}
uses the last prefix (which is empty in this case). This means
that on the first \LaTeX\ run, \verb|\dgls{svm}| behaves like
\verb|\gls{svm}|, which adds a record for the primary 
\texttt{svm} entry. The default primary-dual dependency means
that this will cause both the primary (\texttt{svm}) and dual
(\texttt{dual.svm}) entry to be selected. The location will be
added to the primary entry's location list, unless overridden
by resource options, such as \texttt{combine-dual-locations}.

Once \gls{bib2gls} has been run and the \texttt{.glstex} file
exists, then \texttt{dual.svm} exists. So \verb|\dgls{svm}| will
again first try \texttt{dual.svm} (as \texttt{dual.} is the
first in the list of label prefixes). That now exists, so
\verb|\dgls{svm}| now behaves like \verb|\gls{dual.svm}|, which
follows the abbreviation style and hyperlinks to the list
of abbreviations.

Similarly for the index-symbol combination \texttt{dual.pi} and
\texttt{pi}. In the case of \verb|\dgls{duck}|, the label
\texttt{dual.duck} never exists, so that's never selected.
The empty prefix is the only one that matches, so
\verb|\dgls{duck}| always behaves like \verb|\gls{duck}|.

If you haven't used \texttt{combine-dual-locations} an extra 
\gls{bib2gls}+\LaTeX\ run may be required to correct the location
lists.

If you change the label prefixes, remember to update the
corresponding \cs{glsxtraddlabelprefix}\marg{prefix}.
If no prefixes have been added to the list (or if the list is
cleared), just an empty prefix is assumed.

As from version 1.8, \gls{bib2gls} provides hooks that identify
the label prefixes in the \texttt{.glstex} file:
\begin{alltt}
\ics{bibglstertiaryprefixlabel}\marg{prefix}
\ics{bibglsdualprefixlabel}\marg{prefix}
\ics{bibglsprimaryprefixlabel}\marg{prefix}
\end{alltt}
These do nothing by default, but they can be defined before
the resource file is loaded to set up the prefix list. For
example:
\begin{verbatim}
\newcommand{\bibglsprimaryprefixlabel}[1]{\glsxtraddlabelprefix{#1}}
\newcommand{\bibglsdualprefixlabel}[1]{\glsxtrprependlabelprefix{#1}}
\GlsXtrLoadResources[src=entries]
\end{verbatim}
Remember that this will only have an effect once the
\texttt{.glstex} file has been created. The prefix list will be empty
on the first run (which is treated as a single empty prefix).
If this isn't a suitable fallback, it may be necessary to add one
after all the resource commands:
\begin{verbatim}
\newcommand{\bibglsprimaryprefixlabel}[1]{\glsxtraddlabelprefix{#1}}
\newcommand{\bibglsdualprefixlabel}[1]{\glsxtrprependlabelprefix{#1}}
\GlsXtrLoadResources[src=entries,label-prefix={idx.}]
\glsxtraddlabelprefix{idx.}
\end{verbatim}
Although this rather defeats the purpose of using the hooks as you
still have to keep track of the fallback prefix.

\section{Record Counting}
\label{sec:recordcount}

As from version 1.1 of \gls{bib2gls}, you can save the total
record count for each entry by invoking \gls{bib2gls}
with the \texttt{-{}-record-count} or \texttt{-{}-record-count-unit}
switches. These options will ensure that when each entry
is written to the \texttt{.glstex} file \gls{bib2gls} will
additionally set the following internal fields for that entry:
\begin{itemize}
\item \texttt{recordcount}: set to the total
number of records found for the entry;
\item \texttt{recordcount.\meta{counter}}: set to the total
number of records found for the entry for the given counter.
\end{itemize}
If \texttt{-{}-record-count-unit} is used then additionally:
\begin{itemize}
\item \texttt{recordcount.\meta{counter}.\meta{location}}: set to the total
number of records found for the entry for the given counter with the
given location.
\end{itemize}
Only use the unit counting option if the locations don't contain any special
characters. If you really need it with locations that may contain
formatting commands, then you can try redefining:
\begin{definition}[\DescribeMacro\glsxtrdetoklocation]
\cs{glsxtrdetoklocation}\marg{location}
\end{definition}
so that it detokenizes \meta{location} but take care when
using \cs{GlsXtrLocationRecordCount} with commands like
\cs{thepage} as they can end up becoming detokenized
too early.

Note that the record count includes locations that \gls{bib2gls}
discards, such as ignored records, duplicates and partial
duplicates. It doesn't include cross-reference records.  For
example, if \verb|\gls{bird}| is used twice on page~1, once on
page~2 and fours times on page~3, and
\verb|\gls[counter=section]{bird}| is used once in section~3, then
the total record count (stored in the \texttt{recordcount} field) is
$2+1+4+1=8$, the total for the \texttt{page} counter (stored in the
\texttt{recordcount.page} field) is $2+1+4=7$, and the total for the
\texttt{section} counter (stored in the \texttt{recordcount.section}
field) is 1.

With the unit counting on as well, the field
\texttt{recordcount.page.1} is set to 2, \texttt{recordcount.page.2}
is set to 1, \texttt{recordcount.page.3} is set to 4 and
\texttt{recordcount.section.3} is set to 1.

You can access these fields using the following commands which
will expand to the field value if set or to 0 if unset:
\begin{definition}[\DescribeMacro\GlsXtrTotalRecordCount]
\cs{GlsXtrTotalRecordCount}\marg{label}
\end{definition}
This expands to the total record count for the entry given by
\meta{label}.
\begin{verbatim}
\GlsXtrTotalRecordCount{bird}
\end{verbatim}
expands to 8.

\begin{definition}[\DescribeMacro\GlsXtrRecordCount]
\cs{GlsXtrRecordCount}\marg{label}\marg{counter}
\end{definition}
This expands to the counter total for the entry given by
\meta{label} where \meta{counter} is the counter name.
For example:
\begin{verbatim}
\GlsXtrRecordCount{bird}{page}
\end{verbatim}
expands to 7 and 
\begin{verbatim}
\GlsXtrRecordCount{bird}{section}
\end{verbatim}
expands to 1.

\begin{definition}[\DescribeMacro\GlsXtrLocationRecordCount]
\cs{GlsXtrLocationRecordCount}\marg{label}\marg{counter}\marg{location}
\end{definition}
This expands to the total for the given location. For example
\begin{verbatim}
\GlsXtrLocationRecordCount{bird}{page}{3}
\end{verbatim}
expands to 4. Be careful about using \cs{thepage} in the
\meta{location} part. Remember that due to \TeX's asynchronous
output routine, \cs{thepage} may not be correct.

There are commands analogous to the entry counting commands like
\cs{cgls} and \cs{cglsformat} that are triggered by the record count. 
These are listed below.

\begin{definition}[\DescribeMacro\rgls]
\cs{rgls}\meta{modifier}\oarg{options}\marg{label}\oarg{insert}
\end{definition}
\begin{definition}[\DescribeMacro\rglspl]
\cs{rglspl}\meta{modifier}\oarg{options}\marg{label}\oarg{insert}
\end{definition}
\begin{definition}[\DescribeMacro\rGls]
\cs{rGls}\meta{modifier}\oarg{options}\marg{label}\oarg{insert}
\end{definition}
\begin{definition}[\DescribeMacro\rGlspl]
\cs{rGlspl}\meta{modifier}\oarg{options}\marg{label}\oarg{insert}
\end{definition}
\begin{definition}[\DescribeMacro\rGLS]
\cs{rGLS}\meta{modifier}\oarg{options}\marg{label}\oarg{insert}
\end{definition}
\begin{definition}[\DescribeMacro\rGLSpl]
\cs{rGLSpl}\meta{modifier}\oarg{options}\marg{label}\oarg{insert}
\end{definition}
These commands check the \catattr{recordcount} attribute
which, if set, should be a number. For example:
\begin{verbatim}
\glssetcategoryattribute{abbreviation}{recordcount}{1}
\end{verbatim}
For convenience, you can use
\begin{definition}[\DescribeMacro\GlsXtrSetRecordCountAttribute]
\cs{GlsXtrSetRecordCountAttribute}\marg{category list}\marg{n}
\end{definition}
to set the \catattr{recordcount} attribute to \meta{n}
for all the categories listed in \meta{category list}.

The \cs{rgls}-like commands use
\begin{definition}[\DescribeMacro\glsxtrifrecordtrigger]
\cs{glsxtrifrecordtrigger}\marg{label}\marg{trigger
code}\marg{normal}
\end{definition}
to determine whether the \cs{rgls}-like command should behave like
its \cs{gls} counterpart (\meta{normal}) or whether it should
instead use \meta{trigger code}.

This command checks if the \catattr{recordcount} attribute is set.
If not is just does \meta{normal}, otherwise it tests
if
\begin{definition}[\DescribeMacro\glsxtrrecordtriggervalue]
\cs{glsxtrrecordtriggervalue}\marg{label}
\end{definition}
is greater than the value given in the \catattr{recordcount}
attribute for that entry's category. If true, this does
\meta{normal} otherwise it does \meta{trigger code}.
The default definition of the trigger value command is: 
\begin{verbatim}
\newcommand*{\glsxtrrecordtriggervalue}[1]{%
 \GlsXtrTotalRecordCount{#1}%
}
\end{verbatim}
The \meta{trigger code} part writes a record with
the \gloskey[glslink]{format} set to \texttt{glstriggerrecordformat} (which
\gls{bib2gls} v1.1+ recognises as a special type of ignored location
format) and does \meta{trigger format}. Then it
unsets the \gls{firstuseflag}. Note that it doesn't
implement the post-link hook. This ensures that the record
count is correct on the next run.

The \meta{trigger format} depends on the \cs{rgls}-like
command used and will be one of the following:
\begin{definition}[\DescribeMacro\rglsformat]
\cs{rglsformat}\marg{label}\marg{insert}
\end{definition}
\begin{definition}[\DescribeMacro\rglsplformat]
\cs{rglsplformat}\marg{label}\marg{insert}
\end{definition}
\begin{definition}[\DescribeMacro\rGlsformat]
\cs{rGlsformat}\marg{label}\marg{insert}
\end{definition}
\begin{definition}[\DescribeMacro\rGlsplformat]
\cs{rGlsplformat}\marg{label}\marg{insert}
\end{definition}
\begin{definition}[\DescribeMacro\rGLSformat]
\cs{rGLSformat}\marg{label}\marg{insert}
\end{definition}
\begin{definition}[\DescribeMacro\rGLSplformat]
\cs{rGLSplformat}\marg{label}\marg{insert}
\end{definition}
These all behave much like their \cs{cglsformat} counterparts.  If
the entry's \catattr{regular} attribute is set or if the entry
doesn't have a long form, the first or first plural is used,
otherwise the long or long plural form is used.

You can use
\begin{definition}[\DescribeMacro\glsxtrenablerecordcount]
\cs{glsxtrenablerecordcount}
\end{definition}
to redefine \cs{gls}, \cs{glspl}, \cs{Gls}, \cs{Glspl},
\cs{GLS}, \cs{GLSpl} to \cs{rgls}, \cs{rglspl}, \cs{rGls}, \cs{rGlspl},
\cs{rGLS}, \cs{rGLSpl}, respectively, for convenience.

If you don't want the entries that use \meta{trigger code}
to appear in the glossary, you need to use the resource option
\texttt{trigger-type} to assign them to another glossary.
For example:
\begin{verbatim}
\documentclass{article}

\usepackage[record]{glossaries-extra}

\newignoredglossary{ignored}

\GlsXtrLoadResources[
 src=example-glossaries-acronym,
 trigger-type=ignored,
 category=abbreviation
]

\glsxtrenablerecordcount
\GlsXtrSetRecordCountAttribute{abbreviation}{1}

\begin{document}
\gls{lid}.  \gls{stc}. \gls{lid}.  \gls{aeu}.
\gls{upa}. \gls{aeu}.

\printunsrtglossaries
\end{document}
\end{verbatim}
In the above, \texttt{stc} and \texttt{upa} both only have
one record, so they are assigned to the \texttt{ignored}
glossary instead of the \texttt{main} one.

\chapter{Miscellaneous New Commands}
\label{sec:miscnew}

The \styfmt{glossaries} package provides \ics{glsrefentry} entry to
cross-reference entries when used with the \pkgopt{entrycounter} or
\pkgopt{subentrycounter} options. The \styfmt{glossaries-extra}
package provides a supplementary command
\begin{definition}[\DescribeMacro\glsxtrpageref]
\cs{glsxtrpageref}\marg{label}
\end{definition}
that works in the same way except that it uses \ics{pageref}
instead of \ics{ref}.

You can copy an entry to another glossary using
\begin{definition}[\DescribeMacro\glsxtrcopytoglossary]
\cs{glsxtrcopytoglossary}\marg{entry-label}\marg{glossary-type}
\end{definition}
This appends \meta{entry-label} to the end of the internal
list for the glossary given by \meta{glossary-type}.
Be careful if you use the \sty{hyperref} package as this 
may cause duplicate hypertargets. You will need to change
\cs{glolinkprefix} to another value or disable hyperlinks
when you display the duplicate. Alternatively, use the new
\gloskey[printglossary]{target} key to switch off the targets:
\begin{verbatim}
\printunsrtglossary[target=false]
\end{verbatim}

The \styfmt{glossaries} package allows you to set preamble code
for a given glossary type using \cs{setglossarypreamble}. This
overrides any previous setting. With \styfmt{glossaries-extra}
(as from v1.12) you can instead append to the preamble
using
\begin{definition}[\DescribeMacro\apptoglossarypreamble]
\cs{apptoglossarypreamble}\oarg{type}\marg{code}
\end{definition}
or prepend using
\begin{definition}[\DescribeMacro\pretoglossarypreamble]
\cs{pretoglossarypreamble}\oarg{type}\marg{code}
\end{definition}

\section{Entry Fields}
\label{sec:fields}

A field may now be used to store the name of a text-block command
that takes a single argument. The field is given by
\begin{definition}[\DescribeMacro\GlsXtrFmtField]
\cs{GlsXtrFmtField}
\end{definition}
The default value is \texttt{\GlsXtrFmtField}. Note that the
value must be the control sequence name \emph{without the initial
backslash}.

For example:
\begin{verbatim}
\newcommand*{\mtx}[1]{\boldsymbol{#1}}
\newcommand*{\mtxinv}[1]{\mtx{#1}\sp{-1}}

\newglossaryentry{matrix}{%
  name={matrix},
  symbol={\ensuremath{\mtx{M}}},
  plural={matrices},
  user1={mtx},
  description={rectangular array of values}
}

\newglossaryentry{identitymatrix}{%
  name={identity matrix},
  symbol={\ensuremath{\mtx{I}}},
  plural={identity matrices},
  description={a diagonal matrix with all diagonal elements equal to
1 and all other elements equal to 0}
}

\newglossaryentry{matrixinv}{%
  name={matrix inverse},
  symbol={\ensuremath{\mtxinv{M}}},
  user1={mtxinv},
  description={a square \gls{matrix} such that
   $\mtx{M}\mtxinv{M}=\glssymbol{identitymatrix}$}
}
\end{verbatim}

There are two commands provided that allow you to apply the
command to an argument:
\begin{definition}[\DescribeMacro\glsxtrfmt]
\cs{glsxtrfmt}\oarg{options}\marg{label}\marg{text}
\end{definition}
This effectively does
\begin{alltt}
\cs{glslink}[\meta{default-options},\meta{options}]\marg{label}\{\cs{glsxtrfmtdisplay}\marg{cs name}\marg{text}\{\}\}
\end{alltt}
where \meta{cs name} is the control
sequence name supplied in the provided field, which must be defined
to take a single required argument. Although it effectively
acts like \cs{glslink} it misses out the post-link hook.

The default \meta{default-options} are given by
\begin{definition}[\DescribeMacro\GlsXtrFmtDefaultOptions]
\cs{GlsXtrFmtDefaultOptions}
\end{definition}
This is defined as \texttt{\GlsXtrFmtDefaultOptions} but may
be redefined as appropriate. Note that the replacement text of
\cs{GlsXtrFmtDefaultOptions} is prepended to the optional
argument of \cs{glslink}. 

As from version 1.23, there's also a starred version of
this command that has a final optional argument:
\begin{definition}[\DescribeMacro{\glsxtrfmt*}]
\cs{glsxtrfmt*}\oarg{options}\marg{label}\marg{text}\oarg{insert}
\end{definition}
Both the starred and unstarred versions use:
\begin{definition}[\DescribeMacro\glsxtrfmtdisplay]
\cs{glsxtrfmtdisplay}\marg{cs name}\marg{text}\marg{insert}
\end{definition}
within the link text.
In the case of the unstarred version \meta{insert} will be empty.
It will also be empty if the final option argument is
missing from the starred form. If the entry given by \meta{label}
is undefined \cs{glsxtrfmt} and \cs{glsxtrfmt*} will issue
an error (or warning if \pkgopt[warn]{undefaction}).
There won't be a warning or error if the entry is defined by
the given field is missing. In either case, 
(the entry is undefined or the field is missing) \meta{cs name}
will be \texttt{@firstofone} otherwise it will be the
field value. The default definition is:
\begin{verbatim}
\newcommand{\glsxtrfmtdisplay}[3]{\csuse{#1}{#2}#3}
\end{verbatim}
which puts \meta{text} inside the argument of the control sequence
and \meta{insert} outside it (but it will still be inside the
link text).

\begin{important}
Remember that nested links cause a problem so avoid using
commands like \cs{gls} or \cs{glsxtrfmt} within \meta{text}.
\end{important}

For example:
\begin{verbatim}
\[
  \glsxtrfmt{matrix}{A}
  \glsxtrfmt{matrixinv}{A}
  =
  \glssymbol{identitymatrix}
\]
\end{verbatim}
If the default options are set to \texttt{noindex} then 
\cs{glsxtrfmt} won't index, but will create a hyperlink (if
\sty{hyperref} has been loaded). This can be changed so that
it also suppresses the hyperlink:
\begin{verbatim}
\renewcommand{\GlsXtrFmtDefaultOptions}{hyper=false,noindex}
\end{verbatim}

Note that \cs{glsxtrfmt} won't work with PDF bookmarks. Instead
you can use
\begin{definition}[\DescribeMacro\glsxtrentryfmt]
\cs{glsxtrentryfmt}\marg{label}\marg{text}
\end{definition}
If \sty{hyperref} is used, this uses \cs{texorpdfstring} and will expand to 
\begin{definition}[\DescribeMacro\glsxtrpdfentryfmt]
\cs{glsxtrpdfentryfmt}\marg{label}\marg{text}
\end{definition}
within the PDF bookmarks, but in the document it will do
\meta{cs}\marg{text} if a control sequence name has been provided
or just \meta{text} otherwise. The PDF bookmark version simply does
\meta{text}. It may be redefined, but remember that it needs to
expand fully.

The \styfmt{glossaries} package provides \cs{glsaddstoragekey} to
add new keys. This command will cause an error if the key has
already been defined. The \styfmt{glossaries-extra} package provides
a supplementary command that will only define the key if it doesn't
already exist:
\begin{definition}[\DescribeMacro\glsxtrprovidestoragekey]
\cs{glsxtrprovidestoragekey}\marg{key}\marg{default}\marg{cs}
\end{definition}
If the key has already been defined, it will still provide the command given in
the third argument \meta{cs} (if it hasn't already been defined). Unlike
\cs{glsaddstoragekey}, \meta{cs} may be left empty if you're happy
to just use \cs{glsfieldfetch} to fetch the value of this new key.

You can test if a key has been provided with:
\begin{definition}[\DescribeMacro\glsxtrifkeydefined]
\cs{glsxtrifkeydefined}\marg{key}\marg{true}\marg{false}
\end{definition}
This tests if \meta{key} is available for use in the
\meta{key}=\value{value} list in the second argument of
\cs{newglossaryentry} (or the optional argument of
commands like \cs{newabbreviation}). The corresponding
field may not have been set for any of the entries if no
default was provided.

There are now commands provided to set individual fields. Note that
these only change the specified field, not any related values. For
example, changing the value of the \gloskey{text} field won't update
the \gloskey{plural} field. There are also some fields that
should really only be set when entries are defined (such
as the \gloskey{parent} field). Unexpected results may occur
if they are subsequently changed.

\begin{definition}[\DescribeMacro\GlsXtrSetField]
\cs{GlsXtrSetField}\marg{label}\marg{field}\marg{value}
\end{definition}
Sets the field given by \meta{field} to \meta{value} for the entry
given by \meta{label}. No expansion is performed. It's not
necessary for the field to have been defined as a key. You
can access the value later with \cs{glsxtrusefield}. Note that
\cs{glsxtrifkeydefined} only tests if a key has been defined for use
with commands like \cs{newglossaryentry}. If a field without a
corresponding key is assigned a value, the key remains undefined.
This command is robust.

\cs{GlsXtrSetField} uses
\begin{definition}[\DescribeMacro\glsxtrsetfieldifexists]
\cs{glsxtrsetfieldifexists}\marg{label}\marg{field}\marg{code}
\end{definition}
where \meta{label} is the entry label and \meta{code} is the
assignment code.

This command just uses \ics{glsdoifexists}\marg{label}\marg{code}
(ignoring the \meta{field} argument), so by default it causes an
error if the entry doesn't exist.  This can be changed to a warning
with \pkgopt[warn]{undefaction}.  You can redefine
\cs{glsxtrsetfieldifexists} to simply do \meta{code} if you want to
skip the existence check.  Alternatively you can instead use
\begin{definition}[\DescribeMacro\glsxtrdeffield]
\cs{glsxtrdeffield}\marg{label}\marg{field}\meta{arguments}\marg{replacement text}
\end{definition}
This simply uses \sty{etoolbox}'s \cs{csdef} without any checks.
This command isn't robust. There is also a version that uses
\cs{protected@csedef} instead:\footnote{Pre version 1.28 used
\cs{csedef}.}
\begin{definition}[\DescribeMacro\glsxtredeffield]
\cs{glsxtredeffield}\marg{label}\marg{field}\meta{arguments}\marg{replacement text}
\end{definition}

\begin{definition}[\DescribeMacro\gGlsXtrSetField]
\cs{gGlsXtrSetField}\marg{label}\marg{field}\marg{value}
\end{definition}
As \cs{GlsXtrSetField} but globally.

\begin{definition}[\DescribeMacro\eGlsXtrSetField]
\cs{eGlsXtrSetField}\marg{label}\marg{field}\marg{value}
\end{definition}
As \cs{GlsXtrSetField} but uses protected expansion.

\begin{definition}[\DescribeMacro\xGlsXtrSetField]
\cs{xGlsXtrSetField}\marg{label}\marg{field}\marg{value}
\end{definition}
As \cs{gGlsXtrSetField} but uses protected expansion.

\begin{definition}[\DescribeMacro\GlsXtrLetField]
\cs{GlsXtrLetField}\marg{label}\marg{field}\marg{cs}
\end{definition}
Sets the field given by \meta{field} to the replacement text of \meta{cs} 
for the entry given by \meta{label} (using \cs{let}).

\begin{definition}[\DescribeMacro\csGlsXtrLetField]
\cs{csGlsXtrLetField}\marg{label}\marg{field}\marg{cs name}
\end{definition}
As \cs{GlsXtrLetField} but the control sequence name is supplied
instead.

\begin{definition}[\DescribeMacro\GlsXtrLetFieldToField]
\cs{GlsXtrLetFieldToField}\marg{label-1}\marg{field-1}\marg{label-2}\marg{field-2}
\end{definition}
Sets the field given by \meta{field-1} for the entry given by
\meta{label-1} to the field given by \meta{field-2} for the entry
given by \meta{label-2}. There's no check for the existence of
\meta{label-2}, but
\cs{glsxtrsetfieldifexists}\marg{label-1}\marg{field-1}\marg{code}
is still used, as for \cs{GlsXtrSetField}.

The \styfmt{glossaries} package provides \ics{ifglshasfield}
to determine if a field has been set. The \styfmt{glossaries-extra}
package provides a simpler version:
\begin{definition}[\DescribeMacro\glsxtrifhasfield]
\cs{glsxtrifhasfield}\marg{field}\marg{label}\marg{true}\marg{false}
\end{definition}
(New to v1.19.) Note that in this case the \meta{field} must be the
\emph{internal} field label (for example, \texttt{useri} rather than
\texttt{user1}).
Unlike \cs{ifglshasfield}, this version doesn't complain if the
entry (given by \meta{label}) or the field don't exist and
will simply do \meta{false}. If the field does exist for the given
entry and it's not empty, the \meta{true} part is done otherwise it
does \meta{false}. Within \meta{true} you may use
\begin{definition}[\DescribeMacro\glscurrentfieldvalue]
\cs{glscurrentfieldvalue}
\end{definition}
to access the field value. This command includes grouping
which scopes the \meta{true} and \meta{false} parts.
The starred version
\begin{definition}[\DescribeMacro{\glsxtrifhasfield*}]
\cs{glsxtrifhasfield*}\marg{field}\marg{label}\marg{true}\marg{false}
\end{definition}
omits the implicit grouping.

\begin{important}
Be careful of keys that use \cs{relax} as the default value
(such as the \gloskey{symbol}). Use \cs{ifglshassymbol} instead.
\end{important}

There is also a version that simply uses \cs{ifcsundef}. It
doesn't save the field value, but can be used if you only
need to check if the field is defined without accessing it:
\begin{definition}[\DescribeMacro\GlsXtrIfFieldUndef]
\cs{GlsXtrIfFieldUndef}\marg{field}\marg{label}\marg{true}\marg{false}
\end{definition}

\begin{important}
There's a difference between an undefined field and an unset field.
An undefined field hasn't been assigned any value (no
associated internal control sequence has been defined). If a defined
field has been defined to empty, then it's considered unset.
\cs{GlsXtrIfFieldUndef} implement \meta{false} for a defined 
but empty field whereas \cs{glsxtrifhasfield} and \cs{ifglshasfield}
will implement \meta{false} a defined but empty field.
Remember that any keys that may be used in \cs{newglossaryentry}
will have a default value if not provided. In many cases, the
default value is empty, so only use \cs{GlsXtrIfFieldUndef}
for fields that can only be defined through commands like 
\cs{GlsXtrSetField}.
\end{important}

You can test if a field value equals a string using
\begin{definition}[\DescribeMacro{\GlsXtrIfFieldEqStr}]
\cs{GlsXtrIfFieldEqStr}\marg{field}\marg{label}\marg{text}\marg{true}\marg{false}
\end{definition}
If the entry exists and has the given field set to the
given text then this does \meta{true} otherwise it does
\meta{false}. This uses \cs{glsxtrifhasfield}
to test if the field exists and then compares the replacement
text of \cs{glscurrentfieldvalue} with \meta{text} using
\sty{etoolbox}'s \cs{ifdefstring}. Version 1.39 introduced
a starred form of this command, which uses the starred form
of \cs{glsxtrifhasfield}.

As from version 1.31, there's a similar command:
\begin{definition}[\DescribeMacro{\GlsXtrIfFieldEqXpStr}]
\cs{GlsXtrIfFieldEqXpStr}\marg{field}\marg{label}\marg{text}\marg{true}\marg{false}
\end{definition}
This is like \cs{GlsXtrIfFieldEqStr} but first (protected) fully
expands \meta{text} (but not the field value). If you want to
compare the (protected) full expansion of both the field value and \meta{text}
use:
\begin{definition}[\DescribeMacro{\GlsXtrIfXpFieldEqXpStr}]
\cs{GlsXtrIfXpFieldEqXpStr}\marg{field}\marg{label}\marg{text}\marg{true}\marg{false}
\end{definition}
Again, version 1.39 introduced a starred form of these commands,
which use the starred form of \cs{glsxtrifhasfield}.

As from v1.42, you can test if the field value is contained in a
comma-separated list with:
\begin{definition}[\DescribeMacro\GlsXtrIfFieldValueInCsvList]
\cs{GlsXtrIfFieldValueInCsvList}\marg{label}\marg{field}\marg{list}\marg{true}\marg{false}
\end{definition}
The unstarred version uses the unstarred form of
\cs{glsxtrifhasfield} (which adds grouping so \meta{true} and
\meta{false} will be localised). The starred version uses
\cs{glsxtrifhasfield*}. If the field value (without expansion) is
contained in \meta{list} then \meta{true} is done otherwise
\meta{false} is done. If the field hasn't been set \meta{false} is
done. This internally uses \cs{DTLifinlist} provided by
\sty{datatool-base} which performs a one level expansion on
\meta{list}. See the \sty{datatool} documentation for further
details.

This command is intended for fields that contain a label. For
example, to test if the category is one of a set of labels:
\begin{verbatim}
\GlsXtrIfFieldValueInCsvList{examplelabel}{category}%
 {abbreviation,acronym}% set of labels
 {true part}{false part}
\end{verbatim}

As from v1.31, if a field represents a numeric (integer) value, you
can use the following two numeric tests. If the field is set, it must
expand to an integer. You may use \cs{glscurrentfieldvalue}
within \meta{true} or \meta{false} to access the actual value. Both \meta{true}
and \meta{false} are scoped. If the field is undefined or empty, 
the value is assumed to be 0, and \cs{glscurrentfieldvalue} is set
accordingly. As from version 1.39, these numeric tests have
starred versions. The unstarred versions add implicit grouping.
The starred versions don't.

To test if the value is non-zero:
\begin{definition}[\DescribeMacro{\GlsXtrIfFieldNonZero}]
\cs{GlsXtrIfFieldNonZero}\marg{field}\marg{label}\marg{true}\marg{false}
\end{definition}
Alternatively, you can test if the field expands to a specific
number using:
\begin{definition}[\DescribeMacro{\GlsXtrIfFieldEqNum}]
\cs{GlsXtrIfFieldEqNum}\marg{field}\marg{label}\marg{n}\marg{true}\marg{false}
\end{definition}
This does \meta{true} if the field value equals \meta{n}
(using \cs{ifnum} for the comparison) otherwise it does
\meta{false}. For a more general numeric comparison you
can use:
\begin{definition}[\DescribeMacro{\GlsXtrIfFieldCmpNum}]
\cs{GlsXtrIfFieldCmpNum}\marg{field}\marg{label}\marg{comparison}\marg{n}\marg{true}\marg{false}
\end{definition}
where \meta{comparison} is one of \texttt{=}, \texttt{<} or
\texttt{>}.

The \styfmt{glossaries} package provides \ics{glsfieldfetch} which
can be used to fetch the value of the given field and store it in a
control sequence. The \styfmt{glossaries-extra} package provides 
another way of accessing the field value:
\begin{definition}[\DescribeMacro\glsxtrusefield]
\cs{glsxtrusefield}\marg{entry-label}\marg{field-label}
\end{definition}
This works in the same way as commands like \cs{glsentrytext} but
the field label is specified in the first argument. Note that the
\meta{field-label} corresponds to the internal field tag, which
isn't always the same as the key name. See Table~4.1 of the
\sty{glossaries} manual. No error occurs if the entry or field
haven't been defined. This command is not robust.

There is also a version that converts the first letter to uppercase
(analogous to \cs{Glsentrytext}):
\begin{definition}[\DescribeMacro\Glsxtrusefield]
\cs{Glsxtrusefield}\marg{entry-label}\marg{field-label}
\end{definition}
and, as from v1.37, a command that converts the entire value to upper case:
\begin{definition}[\DescribeMacro\GLSxtrusefield]
\cs{GLSxtrusefield}\marg{entry-label}\marg{field-label}
\end{definition}

If you want to use a field to store a list that can be used
as an \sty{etoolbox} internal list, you can use the following
command that adds an item to the field using \sty{etoolbox}'s
\cs{listcsadd}:
\begin{definition}[\DescribeMacro\glsxtrfieldlistadd]
\cs{glsxtrfieldlistadd}\marg{label}\marg{field}\marg{item}
\end{definition}
where \meta{label} is the entry's label, \meta{field} is
the entry's field and \meta{item} is the item to add. There
are analogous commands that use \cs{listgadd}, \cs{listeadd}
and \cs{listxadd}:
\begin{definition}[\DescribeMacro\glsxtrfieldlistgadd]
\cs{glsxtrfieldlistgadd}\marg{label}\marg{field}\marg{item}
\end{definition}
\begin{definition}[\DescribeMacro\glsxtrfieldlisteadd]
\cs{glsxtrfieldlisteadd}\marg{label}\marg{field}\marg{item}
\end{definition}
\begin{definition}[\DescribeMacro\glsxtrfieldlistxadd]
\cs{glsxtrfieldlistxadd}\marg{label}\marg{field}\marg{item}
\end{definition}
You can then iterate over the list using:
\begin{definition}[\DescribeMacro\glsxtrfielddolistloop]
\cs{glsxtrfielddolistloop}\marg{label}\marg{field}
\end{definition}
or
\begin{definition}[\DescribeMacro\glsxtrfieldforlistloop]
\cs{glsxtrfieldforlistloop}\marg{label}\marg{field}\marg{handler}
\end{definition}
that internally use \cs{dolistcsloop} and \cs{forlistloop},
respectively. You can use \ics{listbreak} to break out of the loop
(see the \sty{etoolbox} manual for further details).

For fields that contain comma-separated lists use
\cs{glsxtrforcsvfield} instead.

New to v1.42:
\begin{definition}[\DescribeMacro\glsxtrfieldformatlist]
\cs{glsxtrfieldformatlist}\marg{label}\marg{field}
\end{definition}
Like \sty{datatool-base}'s \ics{DTLformatlist}, this iterates over
an (\sty{etoolbox}) internal list and formats each item. Uses all 
the same helper commands as \cs{DTLformatlist} (the same list handler
macro is used). Unlike the \sty{datatool-base} command, this command
has no starred version. Grouping is automatically applied.

For example, with \gls{bib2gls}['s] \texttt{save-child-count}
resource option:
\begin{verbatim}
\newcommand*{\listchildren}[1]{%
  \GlsXtrIfFieldNonZero{childcount}{#1}%
  {child count: \glscurrentfieldvalue; children:
   \let\DTLlistformatitem\glsfmtname
   \glsxtrfieldformatlist{#1}{childlist}%
  }%
  {no children}%
}
\end{verbatim}
For fields that contain a comma-separated list use \cs{glsxtrfieldformatcsvlist}
instead.

There are also commands that use \cs{ifinlistcs}:
\begin{definition}[\DescribeMacro\glsxtrfieldifinlist]
\cs{glsxtrfieldifinlist}\marg{label}\marg{field}\marg{item}\marg{true}\marg{false}
\end{definition}
and \cs{xifinlistcs}
\begin{definition}[\DescribeMacro\glsxtrfieldxifinlist]
\cs{glsxtrfieldxifinlist}\marg{label}\marg{field}\marg{item}\marg{true}\marg{false}
\end{definition}

See the \sty{etoolbox}'s user manual for further
details of these commands, in particular the limitations
of \cs{ifinlist}.

If the field has a comma-separated list value instead, you
can iterate over it using:
\begin{definition}[\DescribeMacro\glsxtrforcsvfield]
\cs{glsxtrforcsvfield}\marg{label}\marg{field}\marg{handler}
\end{definition}
where again \meta{handler} is a control sequence that takes
a single argument. Unlike the \sty{etoolbox} loops, this
doesn't ignore empty elements nor does it discard leading / trailing
spaces. Internally it uses \cs{@for} (modified by \sty{xfor}
which is automatically loaded by \sty{glossaries}). The
\sty{xfor} package modifies the behaviour of \cs{@for} to allow
the loop to be broken prematurely using \cs{@endfortrue}.
The \cs{glsxtrforcsvfield} command locally defines
a user level command:
\begin{definition}[\DescribeMacro\glsxtrendfor]
\cs{glsxtrendfor}
\end{definition}
which is just a synonym for \cs{@endfortrue}.

The loop is performed within the true part of \cs{glsxtrifhasfield}
so scoping is automatically applied.

New to v1.42:
\begin{definition}[\DescribeMacro\glsxtrfieldformatcsvlist]
\cs{glsxtrfieldformatcsvlist}\marg{label}\marg{field}
\end{definition}
Like \cs{glsxtrfieldformatcsvlist} but for fields that contain
comma-separated lists.

As from version 1.32, if the field given by
\begin{definition}[\DescribeMacro\GlsXtrForeignTextField]
\cs{GlsXtrForeignTextField}
\end{definition}
(which defaults to \texttt{userii}) contains a locale tag, then
\begin{definition}[\DescribeMacro\GlsXtrForeignText]
\cs{GlsXtrForeignText}\marg{entry label}\marg{text}
\end{definition}
can be used to encapsulate \meta{text} in
\cs{foreignlanguage}\marg{dialect}\marg{text} where \meta{dialect}
is obtained from the locale tag through \sty{tracklang}'s
\ics{GetTrackedDialectFromLanguageTag} command. You need at least
\sty{tracklang} v1.3.6 for this to work properly. The \meta{dialect}
must be one that's tracked (which typically means that \sty{babel}
or \sty{polyglossia} has been loaded with the appropriate setting
for that language). If \cs{foreignlanguage} hasn't been defined,
this just does \meta{text}.
For example:
\begin{verbatim}
\documentclass{article}

\usepackage[main=british,brazilian,ngerman]{babel}
\usepackage{glossaries-extra}

\setabbreviationstyle{long-short-user}
\newabbreviation
 [user1={Associa\c{c}\~ao Brasileria de Normas T\'ecnicas},
  user2= {pt-BR}
 ]
 {abnt}{ABNT}{Brazilian National Standards Organization}

\newabbreviation
 [user1={Deutsches Institut f\"ur Normung e.V.},
  user2={de-DE-1996}]
 {din}{DIN}{German Institute for Standardization}

\renewcommand*{\glsxtruserparen}[2]{%
  \glsxtrfullsep{#2}%
  \glsxtrparen
  {#1%
   \ifglshasfield{\glsxtruserfield}{#2}%
   {, \emph{\GlsXtrForeignText{#2}{\glscurrentfieldvalue}}}%
   {}%
  }%
}

\begin{document}
\gls{abnt}, \gls{din}.
\end{document}
\end{verbatim}


When using the \pkgopt{record} option, in addition to recording the
usual location, you can also record the current value
of another counter at the same time using the preamble-only command:
\begin{definition}[\DescribeMacro\GlsXtrRecordCounter]
\cs{GlsXtrRecordCounter}\marg{counter name}
\end{definition}
For example:
\begin{verbatim}
\usepackage[record]{glossaries-extra}
\GlsXtrRecordCounter{section}
\end{verbatim}
Each time an entry is referenced with commands like \cs{gls}
or \cs{glstext}, the \texttt{.aux} file will not only contain
the \cs{glsxtr@record} command but also
\begin{alltt}
\cs{glsxtr@counterrecord}\marg{label}\{section\}\marg{n}
\end{alltt}
where \meta{n} is the current expansion of \cs{thesection}
and \meta{label} is the entry's label. On the next run, when the
\texttt{.aux} file is run, this command will do
\begin{alltt}
\cs{glsxtrfieldlistgadd}\marg{label}\{record.\meta{counter}\}\marg{n}
\end{alltt}
In the above example, if \verb|\gls{bird}| is used in section~1.2
this would be
\begin{verbatim}
\glsxtrfieldlistgadd{bird}{record.section}{1.2}
\end{verbatim}
Note that there's no key corresponding to this new
\texttt{record.section} field, but its value can be
accessed with \cs{glsxtrfielduse} or the list can be
iterated over with \cs{glsxtrfielddolistloop} etc.

\section{Display All Entries Without Sorting or Indexing}
\label{sec:printunsrt}

\begin{definition}[\DescribeMacro\printunsrtglossary]
\cs{printunsrtglossary}\oarg{options}
\end{definition}
This behaves like \cs{printnoidxglossary} but never sorts the
entries and always lists all the defined entries for the given
glossary (and doesn't require \cs{makenoidxglossaries}). 
If you want to use one of the \env{tabular}-like 
styles with \cs{printunsrtglossary}, make sure you load
\sty{glossaries-extra-stylemods} which modifies
the definition of \cs{glsgroupskip} to avoid the
\qt{Incomplete \cs{iftrue}} error that may otherwise occur.

There's also a starred form
\begin{definition}[\DescribeMacro{\printunsrtglossary*}]
\cs{printunsrtglossary}*\oarg{options}\marg{code}
\end{definition}
which is equivalent to
\begin{alltt}
\cs{begingroup}
 \meta{code}\cs{printunsrtglossary}\oarg{options}\%
\cs{endgroup}
\end{alltt}
Note that unlike \cs{glossarypreamble}, the supplied \meta{code} is
done before the glossary header.

This means you now have the option to simply list all entries on the
first \LaTeX\ run without the need for a post-processor, however
there will be no \gls{numberlist} in this case, as that has to be
set by a post-processor such as \gls{bib2gls} (see
\sectionref{sec:bib2gls}).

\begin{important}
No attempt is made to gather hierarchical elements.
If child entries aren't defined immediately after their parent
entry, they won't be together in the glossary when using
\cs{printunsrtglossary}. 
\end{important}

There's a difference in behaviour depending on whether or not the
\gloskey{group} key is defined. If not defined (default), the
group label will be formed from the first letter of the
\gloskey{name} field. The corresponding group title will be obtained
as discussed in \sectionref{sec:glosstylemods}. This can lead to an
odd effect if you are using a style that separates letter groups
when the ordering isn't alphabetical.

If the \gloskey{group} key is defined (which it is with the
\pkgopt{record} option) then the group label will be obtained from
the value of that field. If the field is empty, \emph{no grouping is
performed}, even if the style supports it. (That is, there won't be
a header or a vertical separation.) If the \gloskey{group} field is
non-empty, then the corresponding title is obtained as described
earlier.

In either case, if the \gloskey[printunsrtglossary]{groups} option is set
to false then no group formation will be performed.

If you want to use a different field for the group label, you can redefine
\begin{definition}[\DescribeMacro\glsxtrgroupfield]
\cs{glsxtrgroupfield}
\end{definition}
to the relevant internal field label, but the \gloskey{group}
\emph{key} must still be defined (through the \pkgopt{record} option
or with commands like \cs{glsaddstoragekey}) to ensure that
\cs{printunsrtglossary} uses \cs{glsxtrgroupfield}. (This method is
used by \gls{bib2gls} for secondary entries, which have the group
label stored in the \texttt{secondarygroup} internal field.)

If you have any entries with the \gloskey{see} key set, you will
need the \sty{glossaries} package option \pkgopt[ignore]{seenoindex}
or \pkgopt[warn]{seenoindex} to prevent an error occurring from the
automated \cs{glssee} normally triggered by this key. The
\pkgopt[only]{record} package option will automatically deal with
this.

For example:
\begin{verbatim}
\documentclass{article}

\usepackage{glossaries-extra}

\newglossaryentry{zebra}{name={zebra},description={stripy animal}}
\newglossaryentry{ant}{name={ant},description={an insect}}

\begin{document}
\gls{ant} and \gls{zebra}

\printunsrtglossaries
\end{document}
\end{verbatim}
In the above, zebra will be listed before ant as it was defined
first.

If you allow document definitions with the \pkgopt{docdefs} option, 
the document will require a second \LaTeX\ run if the entries are
defined after \cs{printunsrtglossary}.

The optional argument is as for \cs{printnoidxglossary} (except for
the \gloskey[printnoidxglossary]{sort} key, which isn't available).

All glossaries may be displayed in the order of their definition
using:
\begin{definition}[\DescribeMacro\printunsrtglossaries]
\cs{printunsrtglossaries}
\end{definition}
which is analogous to \cs{printnoidxglossaries}. This just
iterates over all defined glossaries (that aren't on the ignored
list) and does \cs{printunsrtglossary}[type=\meta{type}].

\subsection{Hooks}
\label{sec:printunsrthooks}

To avoid complications caused by \env{tabular}-like glossary
styles, \cs{printunsrtglossary} iterates over all entries
in the selected glossary and appends the appropriate code
to an internal command. Once the construction of this
command is complete, then it's performed to display the
glossary. This puts the loop outside the style code.
For convenience, there's a hook used within the loop:
\begin{definition}[\DescribeMacro\printunsrtglossaryentryprocesshook]
\cs{printunsrtglossaryentryprocesshook}\marg{label}
\end{definition}
This hook should not display any content, but may be used
to perform calculations. For example, to calculate widths.
Within this hook you can use:
\begin{definition}[\DescribeMacro\printunsrtglossaryskipentry]
\cs{printunsrtglossaryskipentry}
\end{definition}
to skip the current entry. This will prevent the entry from
being added to the internal command.

There's another hook immediately before the internal command
containing the glossary code is performed:
\begin{definition}[\DescribeMacro\printunsrtglossarypredoglossary]
\cs{printunsrtglossarypredoglossary}
\end{definition}

The internal command uses
\begin{definition}[\DescribeMacro\printunsrtglossaryhandler]
\cs{printunsrtglossaryhandler}\marg{label}
\end{definition}
to display each item in the list, where \meta{label} is the current label.

By default the handler just does
\begin{definition}[\DescribeMacro\glsxtrunsrtdo]
\cs{glsxtrunsrtdo}\marg{label}
\end{definition}
which determines whether to use \cs{glossentry} or
\cs{subglossentry} and checks the \gloskey{location} and
\gloskey{loclist} fields for the \gls{numberlist}. If you want to
use a different field to be used instead of \gloskey{location}
then redefine:
\begin{definition}[\DescribeMacro\GlsXtrLocationField]
\cs{GlsXtrLocationField}
\end{definition}
to the internal name of the desired field. For example:
\begin{verbatim}
\renewcommand{\GlsXtrLocationField}{primarylocation}
\end{verbatim}
\begin{important}
Note that if the given location field is undefined or unset, 
the handler will fallback on the \gloskey{loclist} field.
\end{important}
You can instruct \gls{bib2gls} to omit setting the \gloskey{loclist}
field with the resource option \texttt{--save-loclist=false}
to prevent it from being used as a fallback.

\subsection{Filtering}
\label{sec:printunsrtfilter}

You can redefine the handler used by \cs{printunsrtglossary} if required. For example,
you may want to filter entries according to the category
label. You can test if a label is contained in a comma-separated
list of labels using:
\begin{definition}[\DescribeMacro\glsxtriflabelinlist]
\cs{glsxtriflabelinlist}\marg{label}\marg{label list}\marg{true}\marg{false}
\end{definition}
The \meta{label} and \meta{label list} will be fully expanded.

\begin{important}
If you redefine the handler to exclude entries, you may end
up with an empty glossary. This could cause a problem for
the list-based styles.
\end{important}

For example, if the preamble includes:
\begin{verbatim}
\usepackage[record,style=index]{glossaries-extra}
\GlsXtrRecordCounter{section}
\end{verbatim}
then you can print the glossary but first redefine the handler
to only select entries that include the current section number
in the \texttt{record.section} field:
\begin{verbatim}
\renewcommand{\printunsrtglossaryhandler}[1]{%
  \glsxtrfieldxifinlist{#1}{record.section}{\thesection}
  {\glsxtrunsrtdo{#1}}%
  {}%
}
\end{verbatim}

Alternatively you can use the starred form of
\cs{printunsrtglossary} which will localise the change:

\begin{verbatim}
\printunsrtglossary*{%
  \renewcommand{\printunsrtglossaryhandler}[1]{%
    \glsxtrfieldxifinlist{#1}{record.section}{\thesection}
    {\glsxtrunsrtdo{#1}}%
    {}%
  }%
}
\end{verbatim}

If you are using the \sty{hyperref} package and want to 
display the same glossary more than once, you can also
add a temporary redefinition of \cs{glolinkprefix} to
avoid duplicate hypertarget names. For example:

\begin{verbatim}
\printunsrtglossary*{%
  \renewcommand{\printunsrtglossaryhandler}[1]{%
    \glsxtrfieldxifinlist{#1}{record.section}{\thesection}
    {\glsxtrunsrtdo{#1}}%
    {}%
  }%
  \ifcsundef{theHsection}%
  {%
    \renewcommand*{\glolinkprefix}{record.\csuse{thesection}.}%
  }%
  {%
    \renewcommand*{\glolinkprefix}{record.\csuse{theHsection}.}%
  }%
}
\end{verbatim}
Note that this will cause a problem if your descriptions
contain commands like \cs{gls} that need to link an entry that
doesn't appear in the summary. In this case, it's a better approach
to use:
\begin{verbatim}
\printunsrtglossary*{%
  \renewcommand{\printunsrtglossaryhandler}[1]{%
    \glsxtrfieldxifinlist{#1}{record.section}{\thesection}
    {\glsxtrunsrtdo{#1}}%
    {}%
  }%
  \ifcsundef{theHsection}%
  {%
    \setkeys{printgloss}{targetnameprefix={record.\csuse{thesection}.}}%
  }%
  {%
    \setkeys{printgloss}{targetnameprefix={record.\csuse{theHsection}.}}%
  }%
}
\end{verbatim}

If it's a short summary at the start of a section, you might
also want to suppress the glossary header and add some vertical
space afterwards:
\begin{verbatim}
\printunsrtglossary*{%
  \renewcommand{\printunsrtglossaryhandler}[1]{%
    \glsxtrfieldxifinlist{#1}{record.section}{\thesection}
    {\glsxtrunsrtdo{#1}}%
    {}%
  }%
  \ifcsundef{theHsection}%
  {%
    \renewcommand*{\glolinkprefix}{record.\csuse{thesection}.}%
  }%
  {%
    \renewcommand*{\glolinkprefix}{record.\csuse{theHsection}.}%
  }%
  \renewcommand*{\glossarysection}[2][]{}%
  \appto\glossarypostamble{\glspar\medskip\glspar}%
}
\end{verbatim}

There's a shortcut command that essentially does this:
\begin{definition}[\DescribeMacro\printunsrtglossaryunit]
\cs{printunsrtglossaryunit}\oarg{options}\marg{counter name}
\end{definition}
The above example can simply be replaced with:
\begin{verbatim}
\printunsrtglossaryunit{section}
\end{verbatim}

This shortcut command is actually defined to use \cs{printunsrtglossary*} with
\begin{definition}[\DescribeMacro\printunsrtglossaryunitsetup]
\cs{printunsrtglossaryunitsetup}\marg{counter name}
\end{definition}
so if you want to just make some minor modifications you can
do
\begin{verbatim}
\printunsrtglossary*{\printunsrtglossaryunitsetup{section}%
  \renewcommand*{\glossarysection}[2][]{\subsection*{Summary}}%
}
\end{verbatim}
which will start the list with a subsection header with the
title \qt{Summary} (overriding the glossary's title).

Note that this shortcut command is only available with the
\pkgopt{record} (or \pkgopt[alsoindex]{record}) package option.

This temporary change in the hypertarget prefix means you
need to explicitly use \cs{hyperlink} to create a link to it
as commands like \cs{gls} will try to link to the target
created with the default definition of \cs{gloslinkprefix}.
This isn't a problem if you want a primary glossary of all terms
produced using just \cs{printunsrtglossary} (in the front or back
matter) which can be the target for all glossary references
and then just use \cs{printunsrtglossaryunit} for a quick 
summary at the start of a section etc.

\section{Partial Glossaries}
\label{sec:printunsrtinner}

It's possible that you may require greater customisation over the
way the glossary is displayed that can't be simply achieved with the
hooks or glossary styles. This section describes a command and
environment provided to assist with this, but note that they aren't
compatible with tabular-like glossary styles, such as
\glostyle{long} or \glostyle{super}, due to their complexity and
internal scoping that interferes with alignment. You will also need
to take care with list styles (provided in the \sty{glossary-list}
package).

To get a better understanding of how this works, it's useful to
consider how a glossary with \gls{makeindex} works. Here's a simple
example document:
\begin{verbatim}
\documentclass{article}

\usepackage{glossaries}

\makeglossaries

\newglossaryentry{cabbage}{name={cabbage}, 
 description={vegetable with thick green or purple leaves}}

\newglossaryentry{broccoli}{name={broccoli},
 description={vegetable with heads of small green or purple flower buds}}

\begin{document}
\Gls{cabbage} and \gls{broccoli}.

\printglossary
\end{document}
\end{verbatim}
When the document is built, \gls{makeindex} creates a file that
contains:
\begin{alltt}
\ics{glossarysection}[\ics{glossarytoctitle}]\mgroup{\ics{glossarytitle}}\ics{glossarypreamble}
\cs{begin}\mgroup{theglossary}\ics{glossaryheader}
\ics{glsgroupheading}\mgroup{B}\cs{relax} \ics{glsresetentrylist} 
\ics{glossentry}\mgroup{broccoli}\marg{location list}\ics{glsgroupskip}
\ics{glsgroupheading}\mgroup{C}\cs{relax} \ics{glsresetentrylist} 
\ics{glossentry}\mgroup{cabbage}\marg{location list}\%
\cs{end}\mgroup{theglossary}\ics{glossarypostamble}
\end{alltt}
The \cs{printglossary} command defines the commands shown
above after it's detected the glossary type, the title and style. It
then inputs the file.

The \cs{printunsrtglossary} command performs the same preliminary
setup but there's no file to input. Instead it then does the section
heading (\cs{glossarysection}) and preamble (\cs{glossarypreamble})
and constructs a control sequence that contains
\texttt{\cs{begin}\mgroup{theglossary}} \meta{content}
\texttt{\cs{end}\mgroup{theglossary}}.

The letter group markup (\cs{glsgroupheading} and \cs{glsgroupskip})
is inserted whenever the group label changes between top-level
entries. In this case, \meta{content} doesn't explicitly contain
\cs{glossentry} but uses a handler function instead. For example,
instead of:
\begin{alltt}
\ics{glossentry}\mgroup{broccoli}\marg{location list}
\end{alltt}
the \meta{content} will contain:
\begin{alltt}
\meta{internal handler}\mgroup{broccoli}
\end{alltt}
This internal handler uses \ics{printunsrtglossaryhandler} to
determine the required information (location list, hierarchical
level) that's needed to typeset the entry (using \cs{glossentry} or
\cs{subglossentry}).

\begin{definition}
\cs{begin}\mgroup{printunsrtglossarywrap}\oarg{options}\newline
\meta{content}
\cs{end}\mgroup{printunsrtglossarywrap}
\end{definition}
The \env{printunsrtglossarywrap} environment provides a wrapper in
which you can place the actual glossary code. The optional argument
is the same as for \cs{printunsrtglossary}, but note that in this
case the \gloskey[printglossary]{type} key simply provides the
glossary title and doesn't identify the content.

This environment essentially does:
\begin{alltt}
\meta{initial setup}
\ics{glossarysection}[\ics{glossarytoctitle}]\mgroup{\ics{glossarytitle}}\ics{glossarypreamble}
\cs{begin}\mgroup{theglossary}\ics{glossaryheader}\cs{glsresetentrylist}
\meta{content}
\cs{end}\mgroup{theglossary}\cs{glossarypreamble}
\end{alltt}
where \meta{initial setup} is much the same code as used by
\cs{printglossary} and \cs{printunsrtglossary}. Most importantly,
\meta{initial setup} sets the current glossary style, which
determines the definition of \env{theglossary}, \cs{glossentry} and
\cs{subglossentry}.

Whilst it is possible to explicitly use the commands that create
letter group headings and the handler within \meta{content}, it
would be quite laborious and prone to error to do so for anything
other than a very short list. This can be illustrated with the
following silly example:
\begin{verbatim}
\begin{printunsrtglossarywrap}[style=indexgroup]%
\glsgroupheading{W}%
\gdef\glscurrententrylabel{cabbage}%
\printunsrtglossaryhandler\glscurrententrylabel
\glsgroupskip
\glsgroupheading{Z}%
\gdef\glscurrententrylabel{broccoli}%
\printunsrtglossaryhandler\glscurrententrylabel
\end{printunsrtglossarywrap}
\end{verbatim}
This trivial example will work with a tabular-like style, such as
\glostyle{long} (although the group headings will be ignored). A
more practical example that uses a loop within \meta{content} won't.

\begin{definition}
\ics{printunsrtinnerglossary}\oarg{options}\marg{pre-code}\marg{post-code}
\end{definition}
This command is provided for use within
\env{printunsrtglossarywrap}. The optional argument \meta{options}
is similar to \cs{printunsrtglossary} but the following keys are
unavailable: \gloskey[printglossary]{title},
\gloskey[printglossary]{toctitle}, \gloskey[printglossary]{style},
\gloskey[printglossary]{numberedsection} and \gloskey[printglossary]{label}.

This will do:
\begin{alltt}
 \cs{begingroup}
  \meta{initial setup}
  \meta{pre-code}
  \meta{content}
  \meta{post-code}
 \cs{endgroup}
\end{alltt}
where \meta{initial setup} processes the options and assigns
\ics{currentglossary}, \meta{pre-code} is some preliminary code and
\meta{post-code} is some final code before the end of the scoping.
The \meta{content} is the sequence of code that
\cs{printunsrtglossary} would normally insert within the
\env{theglossary} environment.

For example:
\begin{verbatim}
\begin{printunsrtglossarywrap}[style=indexgroup]
\printunsrtinnerglossary[type=main]{}{}
\end{printunsrtglossarywrap}
\end{verbatim}
This produces a result very similar to that obtained with just:
\begin{verbatim}
\printunsrtglossary[style=indexgroup,type=main]
\end{verbatim}
The first case is unsuitable for use with a tabular-style. It will
also cause a problem with a list style when used with \gls{bib2gls}
(where the inner content will be empty on the first \LaTeX\ run
which will cause a missing \cs{item} error).

This partial glossary command is more useful when you need to apply
filtering (which can be set up in the \meta{pre-code} argument as in
\cs{printunsrtglossary*}) or if you have multiple glossaries. For
example:
\begin{verbatim}
\begin{printunsrtglossarywrap}[style=list]
\item[Animals]
\printunsrtinnerglossary[type=animal]{}{}
\item[Vegetables]
\printunsrtinnerglossary[type=vegetable]{}{}
\item[Minerals]
\printunsrtinnerglossary[type=mineral]{}{}
\end{printunsrtglossarywrap}
\end{verbatim}
In this case, the \glostyle{list} style doesn't cause a problem as
there will be three instances of \cs{item} on the first \LaTeX\ run.

Here's another example:
\begin{verbatim}
\begin{printunsrtglossarywrap}[style=indexgroup]
\glsgroupheading{animal}%
\printunsrtinnerglossary[type=animal]{}{}
\glsgroupskip
\glsgroupheading{vegetable}%
\printunsrtinnerglossary[type=vegetable]{}{}
\glsgroupskip
\glsgroupheading{mineral}%
\printunsrtinnerglossary[type=mineral]{}{}
\end{printunsrtglossarywrap}
\end{verbatim}
Note that in both of the above cases, the inner glossaries have been
arranged manually (animal, then vegetable and then mineral).

It's also possible to use \cs{printunsrtinnerglossary} within the
handler function used by \cs{printunsrtglossary}. The internal
scoping within \cs{printunsrtinnerglossary} helps to reduce
interference.

This is a rather more complicated example that requires
\gls{bib2gls}. Suppose I have a file called
\nolinkurl{topictitles.bib} that contains:
\begin{verbatim}
@indexplural{animal}
@indexplural{mineral}
@indexplural{vegetable}
\end{verbatim}
And also a file called \nolinkurl{topics.bib} that contains entries
like:
\begin{verbatim}
@entry{zebra,
  name={zebra},
  description={striped African horse},
  category={animal}
}

@entry{broccoli,
  name={broccoli},
  description={vegetable with heads of small green or purple flower buds},
  category={vegetable}
}

@entry{quartz,
  name={quartz},
  description={hard mineral consisting of silica},
  category={mineral}
}
\end{verbatim}
Note that the category labels in the second file match the entry
labels in the first file.

The simplest way of creating a hierarchical glossary from this data
is to input both files and copy the \gloskey{category} field to the
\gloskey{parent} field:
\begin{verbatim}
\GlsXtrLoadResources[src={topictitles,topics},
 replicate-fields={category=parent}]
\end{verbatim}
The glossary can then simply be displayed with:
\begin{verbatim}
\printunsrtglossary[style=indexgroup]
\end{verbatim}
This will now be ordered: animal (followed by animal child entries),
mineral (followed by mineral child entries), vegetable (followed by
vegetable child entries).

Suppose (for some strange reason) that I now need the ordering to be
vegetable, mineral, animal (that is, reverse alphabetical) but the
child entries still need to be in the normal alphabetical order.

It's not possible to use different sort methods for different
hierarchical levels with \gls{bib2gls}, but it is possible to
simulate it.

Instead of making the entries within \nolinkurl{topics.bib} children
of the entries defined in \nolinkurl{topictitles.bib}, I'm now going
to create separate glossaries for each type:
\begin{verbatim}
\newignoredglossary*{animal}
\newignoredglossary*{mineral}
\newignoredglossary*{vegetable}
\end{verbatim}
The entries in \nolinkurl{topictitles.bib} go in the default main
glossary and are sorted in reverse:
\begin{verbatim}
\GlsXtrLoadResources[src=topictitles,type=main,selection=all,
 sort=en-GB-reverse]
\end{verbatim}
Note that I've selected all entries in this example. It becomes more
complicated with the default selection criteria. (See the
\nolinkurl{sample-nested.tex} example provided with \gls{bib2gls}
v2.3+.)

Now the entries in \nolinkurl{topics.bib} are selected but the
\gloskey{type} field is set to the \gloskey{category} field:
\begin{verbatim}
\GlsXtrLoadResources[src=topics,type={same as category},selection=all,
 sort=en-GB]
\end{verbatim}
The aim here is to use \cs{printunsrtglossary} to list all the
entries in the main glossary (that is, all the topic titles) and use
a handler that tests if there is a glossary that has the same label
as the current entry label. If one exists, it's then listed using
\cs{printunsrtinnerglossary} with the level offset shifted to give a
hierarchical appearance:
\begin{verbatim}
\newcommand{\nestedhandler}[1]{%
  \glsxtrunsrtdo{#1}%
  \ifglossaryexists*{#1}%
  {%
    \printunsrtinnerglossary[type={#1},leveloffset=++1]{}{}%
  }%
  {}%
}

\printunsrtglossary*[type=main,style=index]{%
  \let\printunsrtglossaryhandler\nestedhandler
}
\end{verbatim}

I've used \cs{newignoredglossary*} as I don't need to specify a
title for any of those glossaries nor do I need to use those glossaries with
\cs{printunsrtglossaries}. With \gls{bib2gls} v2.3+, I can omit the
three \cs{newignoredglossary*} lines and use the
\texttt{-{}-provide-glossaries} switch which will make \gls{bib2gls}
automatically provide any unknown glossaries (with
\cs{provideignoredglossary}) in the \texttt{.glstex} file.

\section{Standalone Entry Items}
\label{sec:glossentry}

It may be that you don't want a list but would rather display
entry details throughout the document. You can simply
do \cs{glsentryname} followed by \cs{glsentrydesc}.
(Remember that if you don't want a sorted list, use
\pkgopt[none]{sort} to skip the construction of the
\gloskey{sort} field.)
For example, in the preamble provide a custom command:
\begin{verbatim}
\newcommand{\displayterm}[1]{%
 \par\medskip\par\noindent
 Definition: \glsentryname{#1}.\par
 \glsentrydesc{#1}
 \par\medskip
}
\end{verbatim}
define your entries
\begin{verbatim}
\newglossaryentry{function}{name={function},
  description={a relation or expression involving variables}
}
\end{verbatim}
and then later in the text:
\begin{verbatim}
\displayterm{function}
\end{verbatim}

However, if may be that you want to use \sty{hyperref} and
have commands like \cs{gls} link back to the place where
the term is described. Instead of using \cs{glsentryname}
use
\begin{definition}[\DescribeMacro\glsxtrglossentry]
\cs{glsxtrglossentry}\marg{label}
\end{definition}
where \meta{label} is the entry's label.

This is designed to behave much like the way the name is displayed
in the glossary. It performs the following:
\begin{itemize}
\item Defines \cs{glscurrententrylabel} to the entry's label.
This is usually done at the start of the glossary style commands 
\cs{glossentry} and \cs{subglossentry} and may be used by hooks,
such as the post-name hooks. Here the definition is localised
so that it's only available for use in \cs{glossentryname}.

\item Defines \cs{currentglossary} to the entry's glossary type.
This is usually done at the start of commands like
\cs{printglossary} and may be used by style hooks.
Here the definition is localised so that it's only available for use
in \cs{glsentryitem} and \cs{glssubentryitem}. The value is obtained
by fully expanding:
\begin{definition}[\DescribeMacro\GlsXtrStandaloneGlossaryType]
\cs{GlsXtrStandaloneGlossaryType}
\end{definition}
which defaults to the value of the \gloskey{type} field for the
current entry.

\item Increments and display the entry counters 
  if the \pkgopt{entrycounter} or \pkgopt{subentrycounter}
  package options are set.  If the entry doesn't have a parent, then 
  this does
  \begin{itemize}
  \item \cs{glsentryitem}\marg{label}
  \end{itemize}
  otherwise it does (as from v1.31)
  \begin{itemize}
  \item \ics{GlsXtrStandaloneSubEntryItem}\marg{label}
which defaults to \cs{glssubentryitem}\marg{label} if the entry
has a parent but not a grandparent.
  \end{itemize}
  This reflects the behaviour of the predefined hierarchical styles.
A bug in pre-version~1.31 used \cs{glsentryitem} for all child levels,
which doesn't match the hierarchical glossary styles. If you want to
restore this behaviour, just do:
\begin{verbatim}
\renewcommand*{\GlsXtrStandaloneSubEntryItem}[1]{\glssubentryitem{#1}}
\end{verbatim}

\item Sets the hyper-target if supported (using \cs{glstarget})
and displays the entry name using:
\begin{definition}[\DescribeMacro\GlsXtrStandaloneEntryName]
\cs{GlsXtrStandaloneEntryName}\marg{label}
\end{definition}
  which uses
  \cs{glstarget}\marg{label}\{\cs{glossentryname}\marg{label}\} 
  by default.
  Remember that \cs{glossentryname} uses \cs{glsnamefont} or picks up
  the style from category attributes such as 
  \catattr{glossnamefont}.
\end{itemize}
If you have used \ics{nopostdesc} or \ics{glsxtrnopostpunc}
in any of your description fields, you can use
\begin{definition}[\DescribeMacro\glsxtractivatenopost]
\cs{glsxtractivatenopost}
\end{definition}
to make these commands behave as they normally do within a glossary.
This needs to be placed before
\begin{alltt}
\cs{glossentrydesc}\marg{label}\cs{glspostdescription}
\end{alltt}
and scoped.  Note that \cs{glsnonextpages} and \cs{glsnextpages} have no
effect outside of the glossary and are not intended for use in a
standalone context.

It's also possible to select a different field (rather than
using \gloskey{name}):
\begin{definition}[\DescribeMacro\glsxtrglossentryother]
\cs{glsxtrglossentryother}\marg{header}\marg{label}\marg{field}
\end{definition}
The \meta{field} must be given using its internal field label
which may not be the same as the key used to set the field.
See the key to field mappings table in the \styfmt{glossaries}
user manual. The \meta{header} argument is the code
to pass to the third argument of \cs{glsxtrtitleorpdforheading}.
It may be left empty in which case the default is determined as follows:
\begin{itemize}
\item If \cs{glsxtrhead\meta{field}} is defined, then \meta{header}
is \cs{glsxtrhead\meta{field}}\marg{label}.
\item Otherwise \meta{header} is simply the field value.
\end{itemize}
The \cs{glsxtrglossentryother} command internally uses
\begin{definition}[\DescribeMacro\GlsXtrStandaloneEntryOther]
\cs{GlsXtrStandaloneEntryOther}\marg{label}\marg{field}
\end{definition}
instead of \cs{GlsXtrStandaloneEntryName}, which uses
\cs{glossentrynameother}\marg{label}\marg{field} instead of
\cs{glossentryname}\marg{label}.

If you have loaded the
\sty{glossaries-accsupp} package (through the \pkgopt{accsupp}
option) then accessibility support will be provided if
there's a corresponding command
\begin{alltt}
\cs{gls\meta{field}accessdisplay}\marg{text}\marg{label}
\end{alltt}
(for example, \cs{glssymbolaccessdisplay}).

This means that my custom command can be changed to:
\begin{verbatim}
\newcommand{\displayterm}[1]{%
 \par\medskip\par\noindent
 Definition: \glsxtrglossentry{#1}.\par
 \glsentrydesc{#1}
 \par\medskip
}
\end{verbatim}
If I want numbered definitions, then I can use the
package options \sty{entrycounter} or \sty{subentrycounter} 
and remove the colon:
\begin{verbatim}
\newcommand{\displayterm}[1]{%
 \par\medskip\par\noindent
 Definition \glsxtrglossentry{#1}.\par
 \glsentrydesc{#1}
 \par\medskip
}
\end{verbatim}
The counter label uses a dot after the number by default
but this can be changed to a colon:
\begin{verbatim}
\renewcommand*{\glsentrycounterlabel}{\theglossaryentry:\space}
\end{verbatim}
It's now possible to not only use \cs{gls} to link back to the
definition but also use \cs{glsrefentry} to reference
the counter and \cs{glsxtrpageref} to reference the page number.

If I want the description to behave more like it does
in a glossary in need to make the following modification:
\begin{verbatim}
\newcommand{\displayterm}[1]{%
 \par\medskip\par\noindent
 Definition \glsxtrglossentry{#1}.\par
 \begingroup
  \glsxtractivatenopost
  \glossentrydesc{#1}\glspostdescription
 \endgroup
 \par\medskip
}
\end{verbatim}
(Note the grouping to localise \cs{glsxtractivatenopost}.)

You can also use \cs{glsxtrglossentry} within section headings.
For example:
\begin{verbatim}
\section{\glsxtrglossentry{function}}
\end{verbatim}
This will use \cs{glsentryname} in PDF bookmarks (if
\cs{texorpdfstring} is defined) and will use \cs{glsxtrheadname} in
page headers and table of contents. (If you're using a page style or
table of contents that doesn't use \cs{markright} or \cs{markbook}
or \cs{@starttoc} then you need to insert \cs{glsxtrmarkhook} and
\cs{@glsxtrinmark} at the start of the header or table of contents
either scoped or afterwards cancelled with \cs{@glsxtrnotinmark} and
\cs{glsxtrrestoremarkhook}.)

\section{Entry Aliases}
\label{sec:alias}

An entry can be made an alias of another entry using the
\gloskey{alias} key. The value should be the label of the other
term. There's no check for the other's existence when the aliased
entry is defined. This is to allow the possibility of defining the
other entry after the aliased entry. (For example, when used with
\gls{bib2gls}.)

If an entry \meta{entry-1} is made an alias of \meta{entry-2} then:
\begin{itemize}
\item If the \gloskey{see} field wasn't provided when \meta{entry-1}
was defined, the \gloskey{alias} key will automatically trigger
\begin{alltt}
\cs{glssee}\marg{entry-1}\marg{entry-2}
\end{alltt}
\item If the \sty{hyperref} package has been loaded then
\cs{gls}\marg{entry-1} will link to \meta{entry-2}'s target. (Unless
the \catattr{targeturl} attribute has been set for \meta{entry-1}'s
category.)
\item With \pkgopt[off]{record} or \pkgopt[alsoindex]{record}, the \gloskey[glslink]{noindex} setting will automatically be triggered
when referencing \meta{entry-1} with commands like \cs{gls} or
\cs{glstext}. This prevents \meta{entry-1} from have a
\gls{locationlist} (aside from the cross-reference added with
\cs{glssee}) unless it's been explicitly indexed with \cs{glsadd} or
if the indexing has been explicitly set using
\texttt{noindex=false}.

Note that with \pkgopt[only]{record}, the \gls{locationlist}
for aliased entries is controlled with \gls{bib2gls}['s] settings.
\end{itemize}

The index suppression trigger is performed by
\begin{definition}[\DescribeMacro\glsxtrsetaliasnoindex]
\cs{glsxtrsetaliasnoindex}
\end{definition}
This is performed after the default options provided by
\ics{GlsXtrSetDefaultGlsOpts} have been set.
With \pkgopt[only]{record}, \cs{glsxtrsetaliasnoindex} will 
default to do nothing.

Within the definition of \cs{glsxtrsetaliasnoindex} you can use
\begin{definition}[\DescribeMacro\glsxtrindexaliased]
\cs{glsxtrindexaliased}
\end{definition}
to index \meta{entry-2}. 

The index suppression command can be redefined to index the main
term instead. For example:
\begin{verbatim}
\renewcommand{\glsxtrsetaliasnoindex}{%
 \glsxtrindexaliased
 \setkeys{glslink}{noindex}%
}
\end{verbatim}

The value of the \gloskey{alias} field can be accessed using
\begin{definition}[\DescribeMacro\glsxtralias]
\cs{glsxtralias}\marg{label}
\end{definition}

\chapter{On-the-Fly Document Definitions}
\label{sec:onthefly}

\begin{important}
The commands described here may superficially look like
\meta{word}\cs{index}\marg{word}, but they behave rather 
differently. If you want to use \cs{index} then just use
\cs{index}.
\end{important}

The \styfmt{glossaries} package advises against defining entries
in the \env{document} environment. As mentioned in
\sectionref{sec:pkgopts} above, this ability is disabled by
default with \styfmt{glossaries-extra} but can be enabled using
the \pkgopt{docdefs} package options.

Although this can be problematic, the \styfmt{glossaries-extra}
package provides a way of defining and using entries within
the \env{document} environment without the tricks used with the
\pkgopt{docdefs} option. \emph{There are limitations with this
approach, so take care with it.} This function is disabled by
default, but can be enabled using the preamble-only command:
\begin{definition}[\DescribeMacro\GlsXtrEnableOnTheFly]
\cs{GlsXtrEnableOnTheFly}
\end{definition}
When used, this defines the commands described below.

\begin{important}
The commands \cs{glsxtr}, \cs{glsxtrpl}, \cs{Glsxtr}
and \cs{Glsxtrpl} can't be used after the glossaries have been
displayed (through \ics{printglossary} etc). It's best not to mix
these commands with the standard glossary commands, such
as \cs{gls} or there may be unexpected results.
\end{important}

\begin{definition}[\DescribeMacro\glsxtr]
\cs{glsxtr}\oarg{gls-options}\oarg{dfn-options}\marg{label}
\end{definition}
If an entry with the label \meta{label} has already been defined,
this just does \cs{gls}\oarg{gls-options}\marg{label}. If
\meta{label} hasn't been defined, this will define the entry
using:
\begin{alltt}
\cs{newglossaryentry}\marg{label}\{name=\marg{label},
 category=\cs{glsxtrcat},
 description=\{\cs{nopostdesc}\},
 \meta{dfn-options}\}
\end{alltt}

\begin{important}
The \meta{label} must contain any non-expandable commands, such as
formatting commands or problematic characters.
If the term requires any of these, they must be omitted from the
\meta{label} and placed in the \gloskey{name} key must be provided
in the optional argument \meta{dfn-options}.
\end{important}

The second optional argument \meta{dfn-options} should be empty
if the entry has already been defined, since it's too late for
them. If it's not empty, a~warning will be generated with
\begin{definition}[\DescribeMacro\GlsXtrWarning]
\cs{GlsXtrWarning}\marg{dfn-options}\marg{label}
\end{definition}

For example, this warning will be generated on the second instance
of \cs{glsxtr} below:
\begin{verbatim}
\glsxtr[][plural=geese]{goose}
% ... later
\glsxtr[][plural=geese]{goose}
\end{verbatim}

If you are considering doing something like:
\begin{verbatim}
\newcommand*{\goose}{\glsxtr[][plural=geese]{goose}}
\renewcommand*{\GlsXtrWarning}[2]{}
% ... later
\goose\ some more text here
\end{verbatim}
then don't bother. It's simpler and less problematic to just
define the entries in the preamble with \ics{newglossaryentry}
and then use \cs{gls} in the document.

There are plural and case-changing alternatives to \cs{glsxtr}:
\begin{definition}[\DescribeMacro\glsxtrpl]
\cs{glsxtrpl}\oarg{gls-options}\oarg{dfn-options}\marg{label}
\end{definition}
This is like \cs{glsxtr} but uses \cs{glspl} instead of \cs{gls}.

\begin{definition}[\DescribeMacro\Glsxtr]
\cs{Glsxtr}\oarg{gls-options}\oarg{dfn-options}\marg{label}
\end{definition}
This is like \cs{glsxtr} but uses \cs{Gls} instead of \cs{gls}.

\begin{definition}[\DescribeMacro\Glsxtrpl]
\cs{Glsxtrpl}\oarg{gls-options}\oarg{dfn-options}\marg{label}
\end{definition}
This is like \cs{glsxtr} but uses \cs{Glspl} instead of \cs{gls}.

If you use UTF-8 and don't want the inconvenient of needing
to use an ASCII-only label, then it's better to use
\XeLaTeX\ or \LuaLaTeX\ instead of \LaTeX\ (or \pdfLaTeX).
If you really desperately want to use UTF-8 entry labels without
switching to \XeLaTeX\ or \LuaLaTeX\ then there is a starred
version of \cs{GlsXtrEnableOnTheFly} that allows you to
use UTF-8 characters in \meta{label}, but it's experimental and 
may not work in some cases.

\begin{important}
If you use the starred version of \cs{GlsXtrEnableOnTheFly}
don't use any commands in the \meta{label}, even if they expand
to just text.
\end{important}

\chapter{Supplemental Packages}
\label{sec:supplemental}

The \styfmt{glossaries} bundle provides additional support packages
\sty{glossaries-prefix} (for prefixing) and \sty{glossaries-accsupp}
(for accessibility support). These packages aren't automatically
loaded.

\section{Prefixes or Determiners}
\label{sec:prefix}

If prefixing is required, you can simply load
\sty{glossaries-prefix} after \styfmt{glossaries-extra}. For example:
\begin{verbatim}
\documentclass{article}

\usepackage{glossaries-extra}
\usepackage{glossaries-prefix}

\makeglossaries

\newabbreviation
 [prefix={an\space},
 prefixfirst={a~}]
 {svm}{SVM}{support vector machine}

\begin{document}

First use: \pgls{svm}.
Next use: \pgls{svm}.

\printglossaries

\end{document}
\end{verbatim}

\section{Accessibility Support}
\label{sec:accsupp}

\begin{important}
The \sty{glossaries-accsupp} package is still experimental and so
accessibility features are liable to change.
\end{important}

The \sty{glossaries-accsupp} package needs to be loaded before
\styfmt{glossaries-extra} or through the \pkgopt{accsupp} package
option:
\begin{verbatim}
\usepackage[accsupp]{glossaries-extra}
\end{verbatim}
If you don't load \sty{glossaries-accsupp} or you load
\sty{glossaries-accsupp} after \styfmt{glossaries-extra}
the new \cs{glsaccess}\meta{xxx} commands described below will
simply be equivalent to the corresponding \cs{glsentry}\meta{xxx}
commands. Other accessibility features, such as the attributes used
by \ics{newabbreviation} also won't work if \sty{glossaries-accsupp}
is loaded too late.

The following \cs{glsaccess}\meta{xxx} commands add accessibility information wrapped around
the corresponding \cs{glsentry}\meta{xxx} commands. There is
no check for existence of the entry nor do any of these commands
add formatting, hyperlinks or indexing information.

\begin{definition}[\DescribeMacro\glsaccessname]
\cs{glsaccessname}\marg{label}
\end{definition}
This displays the value of the \gloskey{name} field for the entry
identified by \meta{label}.

If the \sty{glossaries-accsupp} package isn't loaded, this is 
simply defined as:
\begin{verbatim}
\newcommand*{\glsaccessname}[1]{\glsentryname{#1}}
\end{verbatim}
otherwise it's defined as:
\begin{verbatim}
\newcommand*{\glsaccessname}[1]{%
  \glsnameaccessdisplay
  {%
    \glsentryname{#1}%
  }%
  {#1}%
}
\end{verbatim}
(\ics{glsnameaccessdisplay} is defined by the
\sty{glossaries-accsupp} package.) The first letter upper case
version is:
\begin{definition}[\DescribeMacro\Glsaccessname]
\cs{Glsaccessname}\marg{label}
\end{definition}
Without the \sty{glossaries-accsupp} package this is just defined 
as:
\begin{verbatim}
\newcommand*{\Glsaccessname}[1]{\Glsentryname{#1}}
\end{verbatim}
With the \sty{glossaries-accsupp} package this is defined as:
\begin{verbatim}
\newcommand*{\Glsaccessname}[1]{%
  \glsnameaccessdisplay
  {%
    \Glsentryname{#1}%
  }%
  {#1}%
}
\end{verbatim}

The following commands are all defined in an analogous manner.
\begin{definition}[\DescribeMacro\glsaccesstext]
\cs{glsaccesstext}\marg{label}
\end{definition}
This displays the value of the \gloskey{text} field.

\begin{definition}[\DescribeMacro\Glsaccesstext]
\cs{Glsaccesstext}\marg{label}
\end{definition}
This displays the value of the \gloskey{text} field with the first
letter converted to upper case.

\begin{definition}[\DescribeMacro\glsaccessplural]
\cs{glsaccessplural}\marg{label}
\end{definition}
This displays the value of the \gloskey{plural} field.

\begin{definition}[\DescribeMacro\Glsaccessplural]
\cs{Glsaccessplural}\marg{label}
\end{definition}
This displays the value of the \gloskey{plural} field
with the first letter converted to upper case.

\begin{definition}[\DescribeMacro\glsaccessfirst]
\cs{glsaccessfirst}\marg{label}
\end{definition}
This displays the value of the \gloskey{first} field.

\begin{definition}[\DescribeMacro\Glsaccessfirst]
\cs{Glsaccessfirst}\marg{label}
\end{definition}
This displays the value of the \gloskey{first} field
with the first letter converted to upper case.

\begin{definition}[\DescribeMacro\glsaccessfirstplural]
\cs{glsaccessfirstplural}\marg{label}
\end{definition}
This displays the value of the \gloskey{firstplural} field.

\begin{definition}[\DescribeMacro\Glsaccessfirstplural]
\cs{Glsaccessfirstplural}\marg{label}
\end{definition}
This displays the value of the \gloskey{firstplural} field
with the first letter converted to upper case.

\begin{definition}[\DescribeMacro\glsaccesssymbol]
\cs{glsaccesssymbol}\marg{label}
\end{definition}
This displays the value of the \gloskey{symbol} field.

\begin{definition}[\DescribeMacro\Glsaccesssymbol]
\cs{Glsaccesssymbol}\marg{label}
\end{definition}
This displays the value of the \gloskey{symbol} field
with the first letter converted to upper case.

\begin{definition}[\DescribeMacro\glsaccesssymbolplural]
\cs{glsaccesssymbolplural}\marg{label}
\end{definition}
This displays the value of the \gloskey{symbolplural} field.

\begin{definition}[\DescribeMacro\Glsaccesssymbolplural]
\cs{Glsaccesssymbolplural}\marg{label}
\end{definition}
This displays the value of the \gloskey{symbolplural} field
with the first letter converted to upper case.

\begin{definition}[\DescribeMacro\glsaccessdesc]
\cs{glsaccessdesc}\marg{label}
\end{definition}
This displays the value of the \gloskey{description} field.

\begin{definition}[\DescribeMacro\Glsaccessdesc]
\cs{Glsaccessdesc}\marg{label}
\end{definition}
This displays the value of the \gloskey{description} field
with the first letter converted to upper case.

\begin{definition}[\DescribeMacro\glsaccessdescplural]
\cs{glsaccessdescplural}\marg{label}
\end{definition}
This displays the value of the \gloskey{descriptionplural} field.

\begin{definition}[\DescribeMacro\Glsaccessdescplural]
\cs{Glsaccessdescplural}\marg{label}
\end{definition}
This displays the value of the \gloskey{descriptionplural} field
with the first letter converted to upper case.

\begin{definition}[\DescribeMacro\glsaccessshort]
\cs{glsaccessshort}\marg{label}
\end{definition}
This displays the value of the \gloskey{short} field.

\begin{definition}[\DescribeMacro\Glsaccessshort]
\cs{Glsaccessshort}\marg{label}
\end{definition}
This displays the value of the \gloskey{short} field with the first
letter converted to upper case.

\begin{definition}[\DescribeMacro\glsaccessshortpl]
\cs{glsaccessshortpl}\marg{label}
\end{definition}
This displays the value of the \gloskey{shortplural} field.

\begin{definition}[\DescribeMacro\Glsaccessshortpl]
\cs{Glsaccessshortpl}\marg{label}
\end{definition}
This displays the value of the \gloskey{shortplural} field with the first
letter converted to upper case.

\begin{definition}[\DescribeMacro\glsaccesslong]
\cs{glsaccesslong}\marg{label}
\end{definition}
This displays the value of the \gloskey{long} field.

\begin{definition}[\DescribeMacro\Glsaccesslong]
\cs{Glsaccesslong}\marg{label}
\end{definition}
This displays the value of the \gloskey{long} field with the first
letter converted to upper case.

\begin{definition}[\DescribeMacro\glsaccesslongpl]
\cs{glsaccesslongpl}\marg{label}
\end{definition}
This displays the value of the \gloskey{longplural} field.

\begin{definition}[\DescribeMacro\Glsaccesslongpl]
\cs{Glsaccesslongpl}\marg{label}
\end{definition}
This displays the value of the \gloskey{longplural} field with the first
letter converted to upper case.

\chapter{Sample Files}
\label{sec:samples}

The following sample files are provided with this package:
\begin{description}
\item[sample.tex] Simple sample file that uses one of the dummy
files provided by the \styfmt{glossaries} package for testing.

\item[sample-abbr-styles.tex] Demonstrates all predefined
abbreviation styles.

\item[sample-mixture.tex] General entries, acronyms and initialisms
all treated differently.

\item[sample-name-font] Categories and attributes are used to
customize the way different entries appear.

\item[sample-abbrv.tex] General abbreviations.

\item[sample-acronym.tex] Acronyms aren't initialisms and don't
expand on \gls{firstuse}.

\item[sample-acronym-desc.tex] Acronyms that have a separate long
form and description.

\item[sample-crossref.tex] Unused entries that have been
cross-referenced automatically are added at the end of the document.

\item[sample-indexhook.tex] Use the index hook to track
which entries have been indexed (and therefore find out
which ones haven't been indexed).

\item[sample-footnote.tex] Footnote abbreviation style that moves
the footnote marker outside of the hyperlink generated by \cs{gls} 
and moves it after certain punctuation characters for neatness.

\item[sample-undef.tex] Warn on undefined entries instead of
generating an error.

\item[sample-mixed-abbrv-styles.tex] Different abbreviation styles
for different entries.

\item[sample-initialisms.tex] Automatically insert dots into
initialisms.

\item[sample-postdot.tex] Another initialisms example.

\item[sample-postlink.tex] Automatically inserting text after
the \gls{linktext} produced by commands like \cs{gls} (outside 
of hyperlink, if present).

\item[sample-header.tex] Using entries in section/chapter headings.

\item[sample-autoindex.tex] Using the \catattr{dualindex} and
\catattr{indexname} attributes to automatically add glossary
entries to the index (in addition to the glossary \gls{locationlist}).

\item[sample-autoindex-hyp.tex] As previous but uses \sty{hyperref}.

\item[sample-nested.tex] Using \ics{gls} within the
value of the \gloskey{name} key.

\item[sample-entrycount.tex] Enable entry-use counting (only index
if used more than $n$ times, see \sectionref{sec:entrycount}).

\item[sample-unitentrycount.tex] Enable use of per-unit entry-use counting
(\sectionref{sec:entrycount}).

\item[sample-onelink.tex] Using the per-unit entry counting
(\sectionref{sec:entrycount})
to only have one hyperlink per entry per page.

\item[sample-linkcount.tex] Using link counting
(\sectionref{sec:linkcount})
to only have one hyperlink per entry.

\item[sample-pages.tex] Insert \qt{page} or \qt{pages} before the
location list.

\item[sample-altmodifier.tex] Set the default options for
commands like \cs{gls} and add an alternative modifier.

\item[sample-mixedsort.tex] Uses the optional argument
of \cs{makeglossaries} to allow a mixture of \cs{printglossary}
and \cs{printnoidxglossary}.

\item[sample-external.tex] Uses the \catattr{targeturl} attribute
to allow for entries that should link to an external URL
rather than to an internal glossary.

\item[sample-fmt.tex] Provides text-block commands associated
with entries in order to use \cs{glsxtrfmt}.

\item[sample-alias.tex] Uses the \gloskey{alias} key.
(See \sectionref{sec:alias}.)

\item[sample-alttree.tex] Uses the \sty{glossaries-extra-stylemods}
package with the \glostyle{alttree} style (see \sectionref{sec:stylemods}).

\item[sample-alttree-sym.tex] Another \glostyle{alttree} example
that measures the symbol widths.

\item[sample-alttree-marginpar.tex] Another \glostyle{alttree} example
that puts the \gls{numberlist} in the margin.

\item[sample-onthefly.tex] Using on-the-fly commands. Terms with
accents must have the \gloskey{name} key explicitly set.

\item[sample-onthefly-xetex.tex] Using on-the-fly commands
with \XeLaTeX. Terms with UTF-8 characters don't need to
have the \gloskey{name} key explicitly set. Terms that contain
commands must have the \gloskey{name} key explicitly set
with the commands removed from the label.

\item[sample-onthefly-utf8.tex] Tries to emulate the previous
sample file for use with \LaTeX\ through the starred version
of \ics{GlsXtrEnableOnTheFly}. This is a bit iffy and may not
always work. Terms that contain commands must have the 
\gloskey{name} key explicitly set with the commands removed from 
the label.

\item[sample-accsupp.tex] Integrate \sty{glossaries-accsupp}.

\item[sample-prefix.tex] Integrate \sty{glossaries-prefix}.

\item[sample-suppl-main.tex] Uses \gloskey[glsadd]{thevalue} to 
reference a location in the supplementary file
\texttt{sample-suppl.tex}.

\item[sample-suppl-main-hyp.tex] A more complicated version to the
above that uses the \sty{hyperref} package to reference a location
in the supplementary file \texttt{sample-suppl-hyp.tex}.

\end{description}

\chapter{Multi-Lingual Support}
\label{sec:lang}

There's only one command provided by \styfmt{glossaries-extra} 
that you're likely to want to change in your document and that's
\ics{abbreviationsname} (\sectionref{sec:pkgopts}) if you use
the \pkgopt{abbreviations} package option to automatically
create the glossary labelled \texttt{abbreviations}. If this
command doesn't already exist, it will be defined to
\qt{Abbreviations} if \sty{babel} hasn't been loaded, otherwise
it will be defined as \cs{acronymname} (provided by
\styfmt{glossaries}).

You can redefine it in the usual way. For example:
\begin{verbatim}
\renewcommand*{\abbreviationsname}{List of Abbreviations}
\end{verbatim}
Or using \sty{babel} or \sty{polyglossia} captions hook:
\begin{verbatim}
\appto\captionsenglish{%
 \renewcommand*{\abbreviationsname}{List of Abbreviations}%
}
\end{verbatim}

Alternatively you can use the \gloskey[printglossary]{title}
key when you print the list of abbreviations. For example:
\begin{verbatim}
\printabbreviations[title={List of Abbreviations}]
\end{verbatim}
or
\begin{verbatim}
\printglossary[type=abbreviations,title={List of Abbreviations}]
\end{verbatim}

The other fixed text commands are the diagnostic messages, which
shouldn't appear in the final draft of your document.

The \styfmt{glossaries-extra} package has the facility to load
language modules (whose filename is in the form
\texttt{glossariesxtr\dhyphen}\meta{language}\texttt{.ldf}) if they exist,
but won't warn if they don't.  If \sty{glossaries-extra-bib2gls} is
loaded via the \pkgopt{record} package option then the check for
language resource files will additionally search for an associated
language script file given by
\texttt{glossariesxtr-}\meta{script}\texttt{.ldf} where
\meta{script} is the four letter script identifier, such as
\texttt{Latn}, associated with the given dialect. There's no warning if the
associated file isn't found. The script file is loaded after the
dialect file.

If you want to write your own language module, you just need to
create a file called
\texttt{glossariesxtr\dhyphen}\meta{lang}\texttt{.ldf}, where \meta{lang}
identifies the language or dialect (see the \sty{tracklang}
package). For example, \nolinkurl{glossariesxtr-french.ldf}.

The simplest code for this file is:
\begin{verbatim}
\ProvidesGlossariesExtraLang{french}[2015/12/09 v1.0]

\newcommand*{\glossariesxtrcaptionsfrench}{%
 \def\abbreviationsname{Abr\'eviations}%
}
\glossariesxtrcaptionsfrench

\ifcsdef{captions\CurrentTrackedDialect}
{%
  \csappto{captions\CurrentTrackedDialect}%
  {%
    \glossariesxtrcaptionsfrench
  }%
}%
{%
  \ifcsdef{captions\CurrentTrackedLanguage}
  {%
    \csappto{captions\CurrentTrackedLanguage}%
    {%
      \glossariesxtrcaptionsfrench
    }%
  }%
  {}%
}
\end{verbatim}

You can adapt this for other languages by replacing 
all instances of the language identifier \texttt{french} and
the translated text \verb|Abr\'eviations| as appropriate.
You can also use the \texttt{.ldf} file to provide rule
blocks for a particular language for use with \gls{bib2gls}['s] custom
sort rule. See \sectionref{sec:bib2glssty} for further details.

This \texttt{.ldf} file then needs to be put somewhere on \TeX's
path so that it can be found by \styfmt{glossaries-extra}.
You might also want to consider uploading it to CTAN so that
it can be useful to others. (Please don't send it to me. I already
have more packages than I am able to maintain.)

If you additionally want to provide translations for the diagnostic
messages used when a glossary is missing, you need to redefine
the following commands:
\begin{definition}[\DescribeMacro\GlsXtrNoGlsWarningHead]
\cs{GlsXtrNoGlsWarningHead}\marg{label}\marg{file}
\end{definition}
This produces the following text in English:
\begin{quote}
\GlsXtrNoGlsWarningHead{\meta{label}}{\meta{file}}
\end{quote}

\begin{definition}[\DescribeMacro\GlsXtrNoGlsWarningEmptyStart]
\cs{GlsXtrNoGlsWarningEmptyStart}
\end{definition}
This produces the following text in English:
\begin{quote}
\GlsXtrNoGlsWarningEmptyStart
\end{quote}

\begin{definition}[\DescribeMacro\GlsXtrNoGlsWarningEmptyMain]
\cs{GlsXtrNoGlsWarningEmptyMain}
\end{definition}
This produces the following text in English:
\begin{quote}
\GlsXtrNoGlsWarningEmptyMain
\end{quote}

\begin{definition}[\DescribeMacro\GlsXtrNoGlsWarningEmptyNotMain]
\cs{GlsXtrNoGlsWarningEmptyNotMain}\marg{label}
\end{definition}
This produces the following text in English:
\begin{quote}
\GlsXtrNoGlsWarningEmptyNotMain{\meta{label}}
\end{quote}

\begin{definition}[\DescribeMacro\GlsXtrNoGlsWarningCheckFile]
\cs{GlsXtrNoGlsWarningCheckFile}\marg{file}
\end{definition}
This produces the following text in English:
\begin{quote}
\GlsXtrNoGlsWarningCheckFile{\meta{file}}
\end{quote}

\begin{definition}[\DescribeMacro\GlsXtrNoGlsWarningMisMatch]
\cs{GlsXtrNoGlsWarningMisMatch}
\end{definition}
This produces the following text in English:
\begin{quote}
\GlsXtrNoGlsWarningMisMatch
\end{quote}

\begin{definition}[\DescribeMacro\GlsXtrNoGlsWarningNoOut]
\cs{GlsXtrNoGlsWarningNoOut}\marg{file}
\end{definition}
This produces the following text in English:
\begin{quote}
\GlsXtrNoGlsWarningNoOut{\meta{file}}
\end{quote}

\begin{definition}[\DescribeMacro\GlsXtrNoGlsWarningTail]
\cs{GlsXtrNoGlsWarningTail}
\end{definition}
This produces the following text in English:
\begin{quote}
\GlsXtrNoGlsWarningTail
\end{quote}

\begin{definition}[\DescribeMacro\GlsXtrNoGlsWarningBuildInfo]
\cs{GlsXtrNoGlsWarningBuildInfo}
\end{definition}
This is advice on how to generate the glossary files.

\begin{definition}[\DescribeMacro\GlsXtrNoGlsWarningAutoMake]
\cs{GlsXtrNoGlsWarningAutoMake}\marg{label}
\end{definition}
This is the message produced when the \pkgopt{automake} option
is used, but the document needs a rerun or the shell escape
setting doesn't permit the execution of the external application.
This command also generates a warning in the transcript file.

See the documented code (\nolinkurl{glossaries-extra-code.pdf})
for further details.

\printglossaries
\PrintIndex

\end{document}

\newcommand*\Q[2]{\frac{\partial #1}{\partial #2}}

\section[slide=false]{Übersicht}
\begin{slide}[toc=,bm=]{Übersicht}
\tableofcontents
\end{slide}

\section[slide=false]{Forschung und Studium}
\begin{slide}[toc=Das Integral]{Das Integral und seine geometrischen Anwendungen.} 
Die erste Gleichung von Green:
\begin{align}\label{green}
\underset{\mathcal{G}\quad}\iiint\!
	\left[u\nabla^{2}v+\left(\nabla u,\nabla v\right)\right]d^{3}V
	=\underset{\mathcal{S}\quad}\oiint u\Q{v}{n}d^{2}A
\end{align}

\onslide{1-4}{Die Gleichung von Green (\ref{green}) wird später überpüft.}

\begin{itemize}
  \item<1-> Eine Zeile mit \texttt{itemize}.
  \begin{itemize}
    \item<2> Eine Zeile mit \texttt{itemize}.
    \begin{enumerate}
      \item<1> Eine Zeile mit \texttt{enumerate}.
      \item<-3> Noch eine \ldots
    \end{enumerate}
    \item<3-> Eine Zeile mit \texttt{itemize}.
  \end{itemize}
  \item<4-> Eine Zeile mit \texttt{itemize}.
\end{itemize}
\end{slide}
\subsection{Intervall}
\begin{slide}{Definition}
Das \emph{Intervall} $\langle a,b\rangle$ besteht aus allen Zahlen $x$, die den 
Bedingungen $<\le x\le b$  genügen.
\end{slide}
\subsection{Zahlenfolge}
\begin{slide}{Definition der Folge}
Eine \emph{Zahlenfolge} oder \emph{Folge} entsteht, wenn man sich jedes Glied der 
unendlichen Nummernreihe $1,2,3,\ldots$ durch irgend eine (rationale oder irrationale) 
Zahl ersetzt denkt, also jedes $n$ durch eine Zahl $x_n$.
\end{slide}
\section[slide=false]{Konvergenz und Limes}
\begin{slide}{Definition Limes}
$\lim x_n=g$ bedeutet, daß in jeder Umgebung von $g$ fast alle Glieder der Folge liegen.
\end{slide}
\subsection{Konvergenzkriterium}
\begin{slide}{Definition der Konvergenz}
\onslide{1-}{\textbf{Konvergenzkriterium}.} \onslide{2}{Die Folge $x_1,x_2,x_3,\ldots$ ist dann und nur dann 
konvergent, wenn \textbf{jede} Teilfolge $x^\prime_1,x^\prime_2, x^\prime_3,\ldots$ die 
Relation $\lim(x_n-x^\prime_n)=0$}
\end{slide}

\endinput
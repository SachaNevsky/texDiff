%%	This is file 'mensa-tex-doc.tex', Version 2017-12-03
%%	Copyright 2017 Sebastian Friedl <sfr682k@t-online.de>
%% 
%%	This work may be distributed and/or modified under the conditions of the LaTeX Project
%%	Public License, either version 1.3c of this license or (at your option) any later version.
%%	The latest version of this license is available at
%%		http://www.latex-project.org/lppl.txt
%%	and version 1.3c or later is part of all distributions of LaTeX version 2008-05-04 or later.
%%
%%	This work has the LPPL maintenance status 'maintained'.
%%	Author: Sebastian Friedl
%%	Current maintainer of this work is Sebastian Friedl
%%
%%	This work consists of the files mensa-tex.cls, mensa-tex-doc.tex and mensa-tex-example.tex
%%
%%	---------------------------------------------------------------------------------------------------------------------------------------------
%%
%%	A LaTeX class for typesetting school cafeteria menus consisting of two lunches (with dessert) and dinner
%%
%%	---------------------------------------------------------------------------------------------------------------------------------------------
%%
%%	Please report bugs and other problems as well as suggestions for improvements to the following email address: sfr682k@t-online.de
%%
%%	--------------------------------------------------------------------------------------------------------------------------------------------- 


% !TeX spellcheck = en_US
% !TeX program=lualatex



\documentclass[11pt]{ltxdoc}

\usepackage[utopia]{mathdesign}
\usepackage[no-math]{fontspec}

\usepackage{polyglossia}
\setdefaultlanguage{english}
\usepackage[english]{selnolig}

\usepackage{amsmath}
\usepackage{array}
\usepackage{csquotes}
\usepackage{graphicx}
\usepackage{hyperref}
\usepackage{multicol}
\usepackage{textcomp}
\usepackage{xcolor}


\parindent 0pt

\setmainfont[Numbers=OldStyle]{erewhon}
\setsansfont[Numbers=OldStyle,Scale=MatchLowercase]{Source Sans Pro}
\setmonofont[Numbers=OldStyle,Scale=MatchLowercase]{Source Code Pro}

\usepackage[left=4.50cm,right=2.75cm,top=3.25cm,bottom=2.75cm,nohead]{geometry}

\hyphenation{}

\title{The \texttt{mensa-tex} class \\ {\large\url{https://github.com/SFr682k/mensa-tex}}}
\author{Sebastian Friedl \\ \href{mailto:sfr682k@t-online.de}{\ttfamily sfr682k@t-online.de}}
\date{2017/12/03}

\hypersetup{pdftitle={The mensa-tex class},pdfauthor={Sebastian Friedl}}

\begin{document}
	\maketitle
	\thispagestyle{empty}
	
	\begin{center} \itshape
		\enquote{I can't go to a restaurant and order food \\ because I keep looking at the fonts on the menu} \\
		--- \textsc{\upshape Donald E. Knuth} ---
	\end{center}
	
	\medskip
	\begin{abstract}
		\noindent%
		A \LaTeX\ class for typesetting school cafeteria menus consisting of two lunches (with dessert) and dinner
	\end{abstract}
	
	
	\tableofcontents
	
	\clearpage
	
	

	\subsection*{Dependencies and other requirements}
	\addcontentsline{toc}{subsection}{Dependencies and other requirements}
	The \texttt{mensa-tex} class requires \LaTeXe\ and the following packages:
	\begin{multicols}{3}\ttfamily\centering
		array \\ colortbl \\ datetime2 \\ datetime2-calc \\ geometry \\ graphicx \\ lmodern \\ textcomp \\ xcolor
	\end{multicols}
	
	
	\subsection*{License}
	\begin{small}
		\addcontentsline{toc}{subsection}{License}
		\textcopyright\ 2017 Sebastian Friedl
		
		\smallskip
		This work may be distributed and/or modified under the conditions of the \LaTeX\ Project Public License, either version 1.3c of this license or (at your option) any later version.
		
		\smallskip
		The latest version of this license is available at \url{http://www.latex-project.org/lppl.txt} and version 1.3c or later is part of all distributions of \LaTeX\ version 2008-05-04 or later.
		
		\smallskip
		This work has the LPPL maintenance status \enquote*{maintained}. The current maintainer of this work is Sebastian Friedl. \\
		This work consists of the following files:
		\begin{itemize} \itemsep 0pt
			\item \texttt{mensa-tex.cls},
			\item \texttt{mensa-tex-doc.tex} and
			\item \texttt{mensa-tex-example.tex}
		\end{itemize}
	\end{small}


	\subsection*{Call for cooperation}
	\addcontentsline{toc}{subsection}{Call for cooperation}
	Please report bugs and other problems as well as suggestions for improvements by using the \href{https://github.com/SFr682k/mensa-tex/issues}{issue tracker on GitHub} or sending an email to \href{mailto:sfr682k@t-online.de}{\texttt{sfr682k@t-online.de}}.


	\clearpage
	
		
		
	% DOCUMENTATION PART ----------------------------------------------------------------------
	
	\section{Using the \texttt{\textbackslash documentclass} command} \label{sec:docclass}
	Using this class is as easy as using the \verb|\documentclass{mensa-tex}| command.
	
	\bigskip
	Following class options are available:
	\begin{itemize}
		\item[\texttt{app}] Use a layout optimized for small screens using DIN/ISO A6 paper
		\item[\texttt{en-GB}] Use an English localization, British variant \textit{(default)}
		\item[\texttt{en-US}] Use an English localization, American variant
		\item[\texttt{german}] Use a German localization
	\end{itemize}



	\section{Creating a menu}
	This section deals with creating a menu using \texttt{mensa-tex}. \\
	\textit{\color{red} All the commands described in this section are to be used inside the preamble} since the menu gets created instantly when using \verb|\begin{document}|.
	
	\subsection{Setting up the basic information}
	The basic information consists of the name of the cafeteria, the institute (or school) it is located at and the image used on the single pages. \\
	It can be set by using the following commands:
	
	\medskip
	\DescribeMacro{\mensaname}
	This command is used to declare the name of the cafeteria -- maybe something like \verb|\mensaname{Food Corner}| \textit{(default is the plain old boring \enquote{Mensa})}. \\
	You may want to change the font size by using arbitrary \LaTeX\ font size commands.
	
	\medskip
	\DescribeMacro{\institute}
	Sets the name of the institute the cafeteria is located at -- for example, you can insert \verb|\institute{University of LOL}| in your preamble if your cafeteria is located at some institute called the \enquote{University of LOL}. \\
	Note that information about the institute is only printed when using the normal layout.

	\medskip
	\DescribeMacro{\setimage}
	Add an image to your diet plans using this command. \\
	Note that you \textit{have} to declare the image by using \verb|\includegraphics| inside \verb|\setimage| (e.~g.~\verb|\setimage{\includegraphics[width=8cm]{path/to/picture}}|). \\
	The space available for the image depends on the used layout (see table \ref{tab:image-sizes}).
	
	\begin{table}[b]\centering\sffamily\small\renewcommand{\arraystretch}{1.25}
		\begin{tabular}{r*{2}{|>{\centering\ttfamily}m{.25\textwidth}<{\arraybackslash}}}
			       & \textsf{normal layout}                                & app \textsf{layout}                                   \tabularnewline\hline\hline
			 width & \verb|.50\textwidth|  \newline $\approx$ \verb|9.0cm| & \verb|.58\textwidth|  \newline $\approx$ \verb|4.9cm| \tabularnewline\hline
			height & \verb|.15\textheight| \newline $\approx$ \verb|4.0cm| & \verb|.15\textheight| \newline $\approx$ \verb|1.9cm|
		\end{tabular}
	
		\rmfamily
		\caption{Available space for the header image}
		\label{tab:image-sizes}
	\end{table}
	
	
	\subsection{Adding food}
	For adding information about the food, the commands described in this subsection are provided.
	
	\medskip
	\DescribeMacro{\startdate}
	This commands defines the \enquote{start date}, the date of the first entry \textit{(the start date's weekday mostly happens to be a Monday)}. The start date has to be present in \textsf{YYYY-MM-DD} format, e.~g.~\verb|\startdate{2007-01-01}| \textit{(Default is 2001-01-01)}.
	
	\medskip
	\DescribeMacro{\monday}\DescribeMacro{\tuesday}\DescribeMacro{\wednesday}\DescribeMacro{\thursday}\DescribeMacro{\friday}
	Use these five commands to insert food into the empty diet. \\
	Every single command requires the same four arguments:
	\begin{multicols}{4}
		\begin{enumerate}\itemsep0pt
			\item Menu I
			\item Menu II
			\item Dessert
			\item Dinner
		\end{enumerate}
	\end{multicols}

	For example, to obtain Monday's menu consisting of
	\begin{multicols}{2}
		\begin{tabbing}
			\hspace{2cm}\=\kill
			Menu I:		\> Fish and chips \\
			Menu II:	\> Crispy fried chicken \\
			Dessert:	\> Chocolate fudge \\
			Dinner:		\> DIY hamburgers
		\end{tabbing}
	\end{multicols}
	you have to write \\
	\verb|\monday{Fish and chips}%          <-- % is required when| \\\nopagebreak[4]
	\verb|       {Crispy fried chicken}%        commands are continued| \\\nopagebreak[4]
	\verb|       {Chocolate fudge}%             in the following line| \\\nopagebreak[4]
	\verb|       {DIY hamburgers}|
	
	\smallskip
	It is possible to insert the command listed above without line breaks, however, doing so will result in the source being less human-readable.
	
	\smallskip
	Due to the menu being implemented in a \texttt{tabular} environment, you have to use \verb|\linebreak| instead of \,\verb|\\|\, to produce additional lines. \\
	Possible, additional hyphenations not found by \LaTeX\ can be marked by manually inserting discretionary hyphens (\verb|\-|) (e.~g.~\verb|hyphen\-ation|).
	
	
	\subsection{Adding additional information}
	\subsubsection{Remarks at the bottom of the page} \label{sec:remarks}
	Some people may want to insert some remarks or annotations at the bottom of the page. \\
	These remarks are defined using two commands, depending on the used layout.
	
	\medskip
	\DescribeMacro{\longremarks}
	This command defines the remarks used for the normal layout. \\
	They may be quite short (like \verb|\longremarks{Try it!}|) or very, very long.
	
	\medskip
	\DescribeMacro{\shortremarks}
	This command defines the remarks used for the \texttt{app}--Layout \\
	Maybe, you should reduce the font size by inserting arbitrary \LaTeX\ font size commands (e.~g.~\verb|\shortremarks{\footnotesize Now using a smaller font size}|)
	
	
	\subsubsection{Remarks for additive ingredients}
	When food contains additive ingredients, these additive ingredients are usually indicated by adding superscript figures. \\
	You have to insert the legend manually by using remarks (see section \ref{sec:remarks}).
	
	\medskip
	\DescribeMacro{\sup}
	This command is a shortcut providing access to the \verb|\textsuperscript| command. \\
	It can be used to produce a superscript 4 with \verb|\sup{4}|.
	
	
	\subsubsection{Symbols for vegetarian and vegan food}
	Due to the recent development I decided to declare symbols for labeling vegetarian and vegan food. \\
	However, you have to insert the legend manually by using remarks (see section \ref{sec:remarks}).
	
	\medskip
	\DescribeMacro{\vgt}
	Produces a symbol consisting of one green leaf ({\color{green!50!black}\textleaf}) for labeling vegetarian food
	
	\medskip
	\DescribeMacro{\vgn}
	Produces a symbol consisting of two green leaves ({\color{green!50!black}\textleaf\textleaf}) for labeling vegan food
	
	\medskip
	Since the leaf symbol is obtained by loading the \texttt{textcomp} package and using the \verb|{\rmfamily\textleaf}| command, please avoid packages loading another roman font not supporting this symbol. This should not be a big limitation since the class only uses sans--serif fonts. \\
	Otherwise, \verb|\vgt| and \verb|\vgn| have to be redefined.
	
	
	\subsection{Using fancy colors}
	To avoid a very boring look of the menu, some parts can be highlighted by using colors. \\
	The class defines three colors: one for the background of certain boxes, another one for the text inside these boxes and a third one for structure text.
	
	\subsubsection{Changing colors}
	Colors can be changed globally by using three commands. The colors themselves are described using the normal syntax of the \texttt{xcolor} package.
	
	\medskip
	\DescribeMacro{\setbgcolor}
	Changes the background color of certain boxes (e.~g. \verb|\setbgcolor{blue}|)
	
	\medskip
	\DescribeMacro{\setcolorfg}
	Changes the color of text inside these colored boxes (e.~g. \verb|\setcolorfg{white}|)
	
	\medskip
	\DescribeMacro{\setctextcolor}
	Changes the color of structure text (e.~g. \verb|\setctextcolor{red}|)
	
	
	\subsubsection{A short note about color selection}
	Please ensure, that the colors set with \verb|\setbgcolor| and \verb|\setcolorfg| are distinctive enough from each other. \verb|\setbgcolor{black}| and \verb|\setcolorfg{black!85}| are usually a very, very bad choice when being used together. \\
	Also, a clear difference between \verb|\setctextcolor| and the background of the sheet of paper is recommended.
	
	
	\subsubsection{Using class--defined colors in normal text}
	The colors defined with \verb|\setbgcolor|, \verb|\setcolorfg| and \verb|\setctextcolor| can be used with \verb|\color| and other color commands. \\
	The corresponding defined colors are called \verb|\bgcolor|, \verb|\colorfg| and \verb|\ctextcolor|.
	
	\medskip
	Examples: \verb|\color{\bgcolor}|, \verb|\color{\colorfg}| and \verb|\color{\ctextcolor}|
	
	
	
	\section{How to add support for other languages}
	Currently, the class natively supports English (GB/US) and German. \\
	However, with the instructions in this section, you are able to define additional localizations yourself. \\
	It is recommended to select the localization closest to your localization by using the class options described in section \ref{sec:docclass} \textit{before} redefining commands (see the examples in table \ref{tab:locals-example}). \\
	Code described in this section is to be placed \textit{before} the \verb|\begin{document}| command. \\
	Only redefine commands when necessary.
	
	% TODO: Table with example of pre-defined localizations
	\begin{table}\centering\sffamily\renewcommand{\arraystretch}{1.25}
		\begin{tabular}{r*{3}{|>{\centering}p{.2\textwidth}<{\arraybackslash}}}
			                     & \texttt{en-GB}         & \texttt{en-US}         & \texttt{german}         \tabularnewline \hline\hline
			    \verb|\menuname| & \multicolumn{2}{c|}{Menu}                       & Menü                    \tabularnewline \hline
			 \verb|\dessertname| & \multicolumn{2}{c|}{Dessert}                    & Dessert                 \tabularnewline \hline
			  \verb|\dinnername| & \multicolumn{2}{c|}{Dinner}                     & Abendessen              \tabularnewline \hline
			    \verb|\dietname| & \multicolumn{2}{c|}{Weekly menu}                & Speiseplan vom          \tabularnewline \hline
			   \verb|\shortdate| & 18/09/17               & 09/18/17               & 18.09.                  \tabularnewline \hline
			\verb|\dowshortdate| & Monday, 18/09/17       & Monday, 09/18/17       & Montag, 18.09.17        \tabularnewline \hline
			    \verb|\longdate| & 18/09/2017             & 09/18/2017             & 18.09.2017              \tabularnewline \hline
			   \verb|\daterange| & 18/09/17\,--\,22/09/17 & 09/18/17\,--\,09/22/17 & 18.09.\,--\,22.09.2017
		\end{tabular}
	
		\rmfamily
		\caption{Examples for natively supported localizations}
		\label{tab:locals-example}
	\end{table}
	
	
	
	
	\subsection{Weekday names}
	Weekday names are stored in the \verb|\wdayname| and \verb|\swdayname| commands. \\
	To modify them, copy the code printed below into your preamble and replace the English weekday names (and their abbreviations) with the appropriate form of your localization (but leave the \verb|%|s untouched)
	
	\medskip
	\verb|% Weekday names| \\
	\verb|\renewcommand{\wdayname}[1]{%| \\
	\verb|    \ifcase\DTMfetchdow{#1}| \\
	\verb|    Monday%| \\
	\verb|    \or| \\
	\verb|    Tuesday%| \\
	\verb|    \or| \\
	\verb|    Wednesday%| \\
	\verb|    \or| \\
	\verb|    Thursday%| \\
	\verb|    \or| \\
	\verb|    Friday%| \\
	\verb|    \or| \\
	\verb|    Saturday%| \\
	\verb|    \or| \\
	\verb|    Sunday%| \\
	\verb|    \fi| \\
	\verb|}|
	
	\bigskip
	\verb|% Short weekday names| \\
	\verb|\renewcommand{\swdayname}[1]{%| \\
	\verb|    \ifcase\DTMfetchdow{#1}| \\
	\verb|    Mon%| \\
	\verb|    \or| \\
	\verb|    Tue%| \\
	\verb|    \or| \\
	\verb|    Wed%| \\
	\verb|    \or| \\
	\verb|    Thu%| \\
	\verb|    \or| \\
	\verb|    Fri%| \\
	\verb|    \or| \\
	\verb|    Sat%| \\
	\verb|    \or| \\
	\verb|    Sun%| \\
	\verb|    \fi| \\
	\verb|}|
	
	
	\subsection{Keywords}
	There are four keywords stored in separate commands. \\
	To modify them, copy the code printed below into your preamble and replace the English words by vocabulary appropriate for your localization.
	
	\medskip
	\verb|\def\menuname{Menu}| \\
	\verb|\def\dessertname{Dessert}| \\
	\verb|\def\dinnername{Dinner}| \\
	\verb|\def\dietname{Weekly menu}|
	
	
	\subsection{Date formats}
	This part is probably the most complicated one when defining own localizations. \\
	Basically, there are four commands that may be redefined:
	\begin{itemize}
		\item \verb|\shortdate| \\
		The short form of the date (e.~g. 10/11)
		
		\item \verb|\dowshortdate| \\
		The short form of the date, including the weekday (e.~g. Sat., 10/11)
		
		\item \verb|\longdate| \\
		The long form of the date (e.~g. 10/11/2012)
		
		\item \verb|\daterange| \\
		A range between two dates (e.~g. 10/11--14/11/2012)
	\end{itemize}
	
	\medskip
	When redefining these commands, you have to assemble the templates available for day, month and year in an order matching the localization.
	
	\subsubsection*{Date templates}
	\begin{itemize}
		\item[\sffamily\bfseries DD]
		\verb|\DTMtwodigits{\DTMfetchday{#1}}| \\
		Prints the day using two digits \\
		If the day consists of only one digit, a zero is inserted  (e.~g. 01 instead of 1)
		
		\item[\sffamily\bfseries D]
		\verb|\DTMfetchday{#1}| \\
		Prints the day using one or two digits
	\end{itemize}
	
	\subsubsection*{Month templates}
	\begin{itemize}
		\item[\sffamily\bfseries MM]
		\verb|\DTMtwodigits{\DTMfetchmonth{#1}}| \\
		Prints the month using two digits

		\item[\sffamily\bfseries M]
		\verb|\DTMfetchmonth{#1}| \\
		Prints the month using one or two digits
	\end{itemize}
	
	\subsubsection*{Year templates}
	\begin{itemize}
		\item[\sffamily\bfseries YYYY]
		\verb|\DTMfetchyear{#1}| \\
		Prints the year using as many digits as required
		
		\item[\sffamily\bfseries YY]
		\verb|\DTMtwodigits{\DTMfetchyear{#1}}| \\
		Prints the year using two digits
	\end{itemize}
	
	\subsubsection*{Inserting weekday names}
	\begin{tabular}{>{\sffamily\bfseries}rl}
		\enquote{normal} weekday names: & \verb|\wdayname{#1}|  \\
		           short weekday names: & \verb|\swdayname{#1}|
	\end{tabular}
	
	\subsubsection*{Redefining commands using templates}
	Now, you only have to redefine the commands. Use the following basic structure: \\
	\verb|\renewcommand{ %% COMMAND %% }[1]{%| \\
	\verb|    %% INSERT THE TEMPLATE COMBINATIONS HERE %%| \\
	\verb|}|
	
	\bigskip
	The templates listed above can be combined suitable. \\
	For example, if you want the long date to be displayed in the \textsf{YYYY-MM-DD} format, the following code does the trick: \\
	\verb|\renewcommand{\longdate}[1]{%| \\\nopagebreak[4]
	\verb|    \DTMfetchyear{#1}%                      The YYYY template| \\\nopagebreak[4]
	\verb|    -%                                      Year/month separator| \\\nopagebreak[4]
	\verb|    \DTMtwodigits{\DTMfetchmonth{#1}}%      The MM template| \\\nopagebreak[4]
	\verb|    -%                                      Month/day separator| \\\nopagebreak[4]
	\verb|    \DTMtwodigits{\DTMfetchday{#1}}%        The DD template| \\\nopagebreak[4]
	\verb|}| \pagebreak[0]
	
	\medskip
	The same principle applies to redefinitions of \verb|\shortdate| and \verb|\dowshortdate|. \\
	The \verb|%|s avoid spaces after templates and separators when inserting line breaks.
	
	\medskip
	If you have already redefined \verb|\shortdate| and \verb|\longdate|, you may reuse these definitions when redefining \verb|\dowshortdate|, for example: \\
	\verb|\renewcommand{\dowshortdate}[1]{%| \\\nopagebreak[4]
	\verb|    \wdayname{#1}%            Insert the weekday| \\\nopagebreak[4]
	\verb|    ,~%                       Weekday/day seperator (~ = space)| \\\nopagebreak[4]
	\verb|    \shortdate{#1}%           Use the short date template| \\\nopagebreak[4]
	\verb|}| \pagebreak[0]
	
	\bigskip
	\textit{Redefining the} \verb|\daterange| \textit{command is special} since it requires two arguments. \\
	Here, you have to use \dots
	\begin{itemize}
		\item
		{\ttfamily\#1} for commands and templates referring to the start date and
		
		\item
		{\ttfamily\#2} for commands and templates referring to the end date
	\end{itemize}
	
	For example, a working redefinition of \verb|\daterange| can be achieved with this code: \\
	\verb|\renewcommand{\daterange}[2]{%  <-- »[2]« instead of »[1]«!!| \\\nopagebreak[4]
	\verb|    \shortdate{#1}%                 Start date| \\\nopagebreak[4]
	\verb|    \,--\,%                         Seperator: -- with spaces| \\\nopagebreak[4]
	\verb|    \shortdate{#2}%                 End date: #2 inst. of #1| \\\nopagebreak[4]
	\verb|}|



	\section{A working example}
	\verb|\documentclass[en-US]{mensa-tex}| \\
	
	\verb|\usepackage[american]{babel}| \\
	\verb|\usepackage[utf8]{inputenc}| \\
	
	\verb|\setbgcolor{blue}| \\
	\verb|\setcolorfg{white}| \\
	\verb|\setctextcolor{red}| \\
	
	
	
	\verb|\institute{Some university far, far away}| \\
	\verb|\mensaname{Café}| \\
	
	\verb|\setimage{\includegraphics[height=.125\textheight]{cafe-logo}}| \\[\bigskipamount]
	
	
	\verb|\startdate{2017-09-18}| \\
	
	\verb|\monday{Scrambled Eggs \linebreak\vgt}%| \\
	\verb|       {Curry Potato Salad with Peas, Mint \& Red Onion \linebreak\vgn}%| \\
	\verb|       {Ice Cream}%| \\
	\verb|       {Turkey Burger}| \\
	
	\verb|\tuesday{Philly Cheese Steak\sup{1)}}%| \\
	\verb|        {Sesame Noodles \linebreak\vgn}%| \\
	\verb|        {Donuts}%| \\
	\verb|        {Orzo Pasta}| \\
	
	\verb|\wednesday{-/-}{-/-}{-/-}{-/-}| \\
	
	\verb|\thursday{Buffalo Wings}%| \\
	\verb|         {Vegetarian Eggrolls \linebreak\vgt}%| \\
	\verb|         {Sacher Cake}%| \\
	\verb|         {Chicken Tortilla}| \\
	
	\verb|\friday{Pastrami Melt}%| \\
	\verb|       {Grilled Cheese \linebreak\vgt}%| \\
	\verb|       {Tiramisu}%| \\
	\verb|       {Salmon Burger}| \\[\bigskipamount]
	
	\verb|\longremarks{%| \\
	\verb|    {\color{\ctextcolor}| \\
	\verb|    Due to a training course of our staff, the café is closed| \\
	\verb|    on Wednesday, 09/20/17.}| \\
		
	\verb|    \medskip| \\
	\verb|    In our efforts to sustain a seasonal menu, sometimes| \\
	\verb|    substitutions may be required, and menu items may change| \\
	\verb|    without notice.| \\
		
	\verb|    \bigskip| \\
	\verb|    \textbf{Key:} \quad| \\
	\verb|    \vgt: vegetarian \quad| \\
	\verb|    \vgn: vegan \quad| \\
	\verb|    \sup{1)}: spicy| \\
	\verb|}| \\
	
	\verb|\begin{document}| \\
	\verb|\end{document}|
	
	\bigskip
	Output: \\[\medskipamount]\nopagebreak[4]
	$\boxed{\text{\includegraphics[width=.95\textwidth]{mensa-tex-example}}}$
\end{document}
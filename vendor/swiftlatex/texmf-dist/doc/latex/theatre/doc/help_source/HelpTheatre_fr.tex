% % % % % % % % % % % % % % % % % % % % % % % % % % % % % % % % % % % % % % % % % % % % % % % % % % % % % % %
% % % % % % % % % % % % % % % % % % % % % % % % % % % % % % % % % % % % % % % % % % % % % % % % % % % % % % %
% % %                                                                                                   % % %
\section{Extension theatre.sty - Aide en Français}%                                                     % % %
% % %                                                                                                   % % %
% % % % % % % % % % % % % % % % % % % % % % % % % % % % % % % % % % % % % % % % % % % % % % % % % % % % % % %
% % % % % % % % % % % % % % % % % % % % % % % % % % % % % % % % % % % % % % % % % % % % % % % % % % % % % % %
%
\subsection{Objet}%
Le but de cette extension est de pouvoir insérer des dialogues dans un texte en personnalisant les impressions par acteur.

Pour ce faire, il faut préalablement créer les rôles et les sauver dans un fichier séparé. Pour chaque rôle, on définit son nom, la couleur d'impression et deux tailles de caractère (en vision claire ou proche et en vision sombre ou éloignée) .

Ensuite, un environnement \og{}theatre\fg{} permet d'écrire les répliques. Elles seront numérotées automatiquement, ce qui facilite le suivi, vu que plusieurs mises en page peuvent être générées selon les demandes des lecteurs.

\bigskip{}

\subsection{Les commandes}
Cette extension prévoit qu'on puisse créer autant de rôles que nécessaires, via la commande de création de rôle.
De plus deux commandes sont prévues, une pour ajouter des commentaires, l'autre pour décrire des mouvements. Elles permettent également de placer un commentaire avant le texte proprement-dit.

\subsubsection{Préparer les rôles}
Dans un fichier séparé, on crée le nombre nécessaire de rôles.\\ La commande \chicom{TheatreCreerUnRole} est prévue à cet effet.

\medskip{}
Sa structure est la suivante :\\
\begin{chicitecommandelarge}
\TheatreCreerUnRole{nomcourt}{nomlong}{petit}{grand}{couleur}
\end{chicitecommandelarge}
\medskip{}
\begin{enumerate}
  \item \textbf{nomcourt} sera utilisé pour créer le nom des macros des rôles, un \emph{\textbf{r}} (pour réplique) puis deux lettres précéderont le nom court
    \begin{itemize}
      \item \textbf{pt} pour petite taille de texte -> donc macro créée \chicom{rptnomcourt},
      \item \textbf{gd} pour grande taille de texte -> donc macro créée \chicom{rgdnomcourt}.
    \end{itemize}
  \item \textbf{nomlong} sera imprimé au-dessus des répliques.   
  \item \textbf{petit} et \textbf{grand} définissent la taille du texte pour chacune des deux macros créées. Ces tailles sont à choisir de préférence parmi les grandes tailles (large, Large, LARGE, huge et HUGE) ou rester à la taille standard (normalsize). Mais rien n'empêche de choisir un taille plus petite que le standard (small, footnotesize, scriptsize ou tiny).
  \item \textbf{couleur} pour personnaliser un rôle, à choisir parmi black, white, red, green, blue, cyan, magenta et yellow. Cependant, avec des packages plus spécialisés en couleur, on peut avoir un choix plus grand.
\end{enumerate}

\medskip{}

Deux variables décident d'autres tailles. Elles peuvent être redéfinies quand on veut (\chivar{def}\chivar{variable\dots}).
\begin{itemize}
\item \chivar{TheatreTailleRoleCpte} décide de la taille du compteur qui s'imprime devant la réplique. Par défaut : small.
\item \chivar{TheatreTailleRoleNom} décide de la taille du rôle qui s'imprime au-dessus de la réplique. Par défaut : small.
\end{itemize}

\medskip{}

Deux longueurs peuvent être redéfinies, en n'oubliant pas l'unité de longueur, par la commande \chicom[nom\}\{Nex]{setlength} (1ex, 2ex \dots)
\begin{itemize}
\item \chivar{TheatreEntreNomEtReplique}  augmente l'espace entre le nom et la réplique.%
\item \chivar{TheatreEcarterRepliques} augmente l'espace entre les répliques.%
\end{itemize}
%-------------------------------------
\subsubsection{Utiliser les rôles créés}
Les commandes de \og{}theatre\fg{} permettent d'écrire un commentaire sur la ligne du nom du rôle. Ce commentaire n'est pas obligatoire, il s'écrira donc entre \og{}[\,]\fg{}, comme il se doit pour les variables optionnelles.

\bigskip{}

\begin{center}
\begin{minipage}[h]{0.9\textwidth}
\def\TheatreEcarterRepliques{3mm}
\begin{SideBySideExample}[gobble=0,fontfamily=lmss,fontseries=b,fontsize=\footnotesize,xrightmargin=0.5\textwidth,baselinestretch=2]
\CommandeEcrite
\begin{theatre}
\rptflore[\emph{commentaire}]{Quittez,\dots}
\rptclimdaph{Berger,\dots}
\rgdtirdor{Mais\dots}
\end{theatre}
\end{SideBySideExample}
\end{minipage}
\end{center}

\bigskip{}

Il suffit donc de se souvenir des noms donnés aux différents rôles.

%---------------
\subsubsection{\chicom{TheatreMvt} : pour décrire les mouvements.}
\begin{itemize}
\item \chivar{TheatreMvtHt} détermine la hauteur du texte,
\item \chivar{TheatreMvtCouleur} détermine sa couleur.
\end{itemize}

\bigskip{}

\begin{center}
\begin{minipage}[h]{0.9\textwidth}
\begin{SideBySideExample}[gobble=0,fontfamily=lmss,fontseries=b,fontsize=\footnotesize,xrightmargin=0.5\textwidth,baselinestretch=2]
\CommandeEcrite

\TheatreMvt[commentaire]{Le texte\dots}

\TheatreMvt{Le texte\dots}
\end{SideBySideExample}
\end{minipage}
\end{center}

\bigskip{}

%----------------------------
\subsubsection{\chicom{TheatreComment} : pour placer des commentaires dans le texte}
\begin{itemize}
\item \chivar{TheatreCommentHt} détermine la hauteur du texte,
\item \chivar{TheatreCommentCouleur} détermine sa couleur.
\end{itemize}

\bigskip{}

\begin{center}
\begin{minipage}[h]{0.9\textwidth}
\begin{SideBySideExample}[gobble=0,fontfamily=lmss,fontseries=b,fontsize=\footnotesize,xrightmargin=0.5\textwidth,baselinestretch=2]
\CommandeEcrite

\TheatreComment[commentaire]{Le texte\dots}

\TheatreComment{Le texte\dots}
\end{SideBySideExample}
\end{minipage}
\end{center}

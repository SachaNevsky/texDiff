% % % % % % % % % % % % % % % % % % % % % % % % % % % % % % % % % % % % % % % % % % % % % % % % % % % % % % %
% % % % % % % % % % % % % % % % % % % % % % % % % % % % % % % % % % % % % % % % % % % % % % % % % % % % % % %
% % %                                                                                                   % % %
\section{Le malade imaginaire}%                                                                           % % %
\subsection{Acte 1 - Scène 1}%                                                                              % % %
\TheatreIniCpteReplique{1}% Choisir le premier numéro des répliques                                      % % %
% % %                                                                                                   % % %
% % % % % % % % % % % % % % % % % % % % % % % % % % % % % % % % % % % % % % % % % % % % % % % % % % % % % % %
% % % % % % % % % % % % % % % % % % % % % % % % % % % % % % % % % % % % % % % % % % % % % % % % % % % % % % %
\TheatreComment{Après les glorieuses fatigues, et les exploits victorieux de notre auguste monarque ; il est bien juste que tous ceux qui se mêlent d'écrire, travaillent ou à ses louanges, ou à son divertissement. C'est ce qu'ici l'on a voulu faire, et ce prologue est un essai des louanges de ce grand prince, qui donne entrée à la comédie du Malade imaginaire, dont le projet a été fait pour le délasser de ses nobles travaux.\newline
La décoration représente un lieu champêtre, et néanmoins fort agréable.}%

\bigskip{}

\textbf{\textsc{\florent{}, \climdaphnt, \tirdornt, \pannt, \newline deux zéphirs, une troupe de bergères et bergers}}\newline

%\renewcommand{\separvers}{\newline}%
\def\separvers{\newline}%

\begin{theatre}
\rptflore{Quittez, quittez vos troupeaux,\separvers{}Venez Bergers, venez Bergères,\separvers{}Accourez, accourez sous ces tendres ormeaux ;\separvers{}Je viens vous annoncer des nouvelles bien chères,\separvers{}Et réjouir tous ces hameaux.\separvers{}Quittez, quittez vos troupeaux,\separvers{}Venez bergers, venez Bergères,\separvers{}Accourez, accourez sous ces tendres ormeaux.}%
\rptclimdaph{Berger, laissons là tes feux,\separvers{}Voilà Florequi nous appelle.}%
\rpttirdor{Mais au moins dis-moi, cruelle,}%
\rpttircis{Si d'un peu d'amitié tu payeras mes vœux ?}%
\rptdorilas{Si tu seras sensible à mon ardeur fidèle ?}%
\rptclimdaph{Voilà Flore qui nous appelle.}%
\rpttirdor{Ce n'est qu'un mot, un mot, un seul mot que je veux.}%
\rpttircis{Languirai-je toujours dans ma peine mortelle ?}%
\rptdorilas{Puis-je espérer qu'un jour tu me rendras heureux ?}%
\rptclimdaph{Voilà Flore qui nous appelle.}%
%
\TheatreMvt[entrée de ballet]{Toute la troupe des Bergers et des Bergères va se placer en cadence autour de Flore.}%
%
\rptclimene{Quelle nouvelle parmi nous,\separvers{}Déesse, doit jeter tant de réjouissance ?}%
\rptdaphne{Nous brûlons d'apprendre de vous\separvers{}Cette nouvelle d'importance.}%
\rptdorilas{D'ardeur nous en soupirons tous.}%
\rpttous[ensemble]{Nous en mourons d'impatience.}%
\rptflore{La voici, silence, silence !\separvers{}Vos vœux sont exaucés, LOUIS est de retour,\separvers{}Il ramène en ces lieux les plaisirs et l'amour,\separvers{}Et vous voyez finir vos mortelles alarmes,\separvers{}Par ses vastes exploits son bras voit tout soumis,\separvers{}Il quitte les armes,\separvers{}Faute d'ennemis.}%
\rpttous{Ah quelle douce nouvelle !\separvers{}Qu'elle est grande ! qu'elle est belle !\separvers{}Que de plaisirs ! que de ris ! que de jeux !\separvers{}Que de succès heureux !\separvers{}Et que le Ciel a bien rempli nos vœux,\separvers{}Ah quelle douce nouvelle !\separvers{}Qu'elle est grande ! qu'elle est belle !}%
%
\TheatreMvt[autre entrée de ballet]{Tous les bergers et bergères expriment par des danses les transports de leur joie.}%
%
\rptflore{De vos flûtes bocagères\separvers{}Réveillez les plus beaux sons ;\separvers{}LOUIS offre à vos chansons\separvers{}La plus belle des matières.\separvers{}Après cent combats,\separvers{}Où cueille son bras\separvers{}Une ample victoire :\separvers{}Formez entre vous\separvers{}Cent combats plus doux,\separvers{}Pour chanter sa gloire.}%
\rpttous{Formons entre nous Cent combats plus doux, Pour chanter sa gloire.}%
\rptflore{Mon jeune amant dans ce bois,\separvers{}Des présents de mon empire\separvers{}Prépare un prix à la voix,\separvers{}Qui saura le mieux nous dire\separvers{}Les vertus et les exploits\separvers{}Du plus auguste des rois.}%
\rptclimene{Si Tircis a l'avantage,}%
\rptdaphne{Si Dorilas est vainqueur,}%
\rptclimene{À le chérir je m'engage.}%
\rptdaphne{Je me donne à son ardeur.}%
\rpttircis{Ô trop chère espérance !}%
\rptdorilas{Ô mot plein de douceur !}%
\rpttirdor{Plus beau sujet, plus belle récompense\separvers{}Peuvent-ils animer un cœur ?}%
%
\TheatreComment{Les violons jouent un air pour animer les deux Bergers au combat, tandis que Flore comme juge va se placer au pied d'un bel arbre, qui est au milieu du théâtre, avec deux Zéphirs, et que le reste comme spectateurs va occuper les deux côtés de la scène.}%
%
\rpttirdor{Plus beau sujet, plus belle récompense\separvers{}Peuvent-ils animer un cœur ?}%
%
\TheatreComment{Les violons jouent un air pour animer les deux Bergers au combat, tandis que Flore comme juge va se placer au pied d'un bel arbre, qui est au milieu du théâtre, avec deux Zéphirs, et que le reste comme spectateurs va occuper les deux côtés de la scène.}%
%
\rpttircis{Quand la neige fondue enfle un torrent fameux,\separvers{}Contre l'effort soudain de ses flots écumeux\separvers{}Il n'est rien d'assez solide ;\separvers{}Digues, châteaux, villes, et bois,\separvers{}Hommes, et troupeaux à la fois,\separvers{}Tout cède au courant qui le guide,\separvers{}Tel, et plus fier, et plus rapide,\separvers{}Marche LOUIS dans ses exploits.}%
%
\TheatreMvt[ballet]{Les Bergers et Bergères du côté de Tircis, dansent autour de lui sur une ritournelle, pour exprimer leurs applaudissements.}%
%
\rptdorilas{Le foudre menaçant qui perce avec fureur\separvers{}L'affreuse obscurité de la nue enflammée,\separvers{}Fait d'épouvante et d'horreur\separvers{}Trembler le plus ferme cœur :\separvers{}Mais à la tête d'une armée\separvers{}LOUIS jette plus de terreur.}%
%
\TheatreMvt[ballet]{Les Bergers et Bergères du côté de Dorilas font de même que les autres.}%
%
\rpttircis{Des fabuleux exploits que la Grèce a chantés,\separvers{}Par un brillant amas de belles vérités\separvers{}Nous voyons la gloire effacée,\separvers{}Et tous ces fameux demi-dieux,\separvers{}Que vante l'histoire passée\separvers{}Ne sont point à notre pensée,\separvers{}Ce que LOUIS est à nos yeux.}%
%
\TheatreMvt[ballet]{Les Bergers et Bergères de son côté, font encore la même chose.}%
%
\rptdorilas{LOUIS fait à nos temps, par ses faits inouïs\separvers{}Croire tous les beaux faits que nous chante l'histoire\separvers{}Des siècles évanouis :\separvers{}Mais nos neveux dans leur gloire,\separvers{}N'auront rien qui fasse croire\separvers{}Tous les beaux faits de LOUIS.}%
%
\TheatreMvt[ballet]{Les Bergers et Bergères de son côté font encore de même, après quoi les deux partis se mêlent.}%
%
\rptpan[\emph{suivi de six Faunes}]{Laissez, laissez, Bergers, ce dessein téméraire,\separvers{}Hé, que voulez-vous faire ?\separvers{}Chanter sur vos chalumeaux,\separvers{}Ce qu'Apollon sur sa lyre\separvers{}Avec ses chants les plus beaux,\separvers{}N'entreprendrait pas de dire ?\separvers{}C'est donner trop d'essor au feu qui vous inspire,\separvers{}C'est monter vers les cieux sur des ailes de cire,\separvers{}Pour tomber dans le fond des eaux.\separvers{}Pour chanter de LOUIS l'intrépide courage ;\separvers{}Il n'est point d'assez docte voix,\separvers{}Point de mots assez grands pour en tracer l'image ;\separvers{}Le silence est le langage\separvers{}Qui doit louer ses exploits.\separvers{}Consacrez d'autres soins à sa pleine victoire,\separvers{}Vos louanges n'ont rien qui flatte ses désirs,\separvers{}Laissez, laissez là sa gloire\separvers{}Ne songez qu'à ses plaisirs.}%
\rpttous{Laissons, laissons là sa gloire\separvers{}Ne songeons qu'à ses plaisirs.}%
\rptflore{Bien que pour étaler ses vertus immortelles\separvers{}La force manque à vos esprits,\separvers{}Ne laissez pas tous deux de recevoir le prix.\separvers{}Dans les choses grandes et belles\separvers{}Il suffit d'avoir entrepris.}%
%
\TheatreMvt[entrée de ballet]{Les deux Zéphirs dansent avec deux couronnes de fleurs à la main, qu'ils viennent donner ensuite aux deux Bergers.}%
%
\rptclimdaph[\emph{en leur donnant la main}]{Dans les choses grandes et belles\separvers{}Il suffit d'avoir entrepris.}%
\rpttirdor{Ha ! que d'un doux succès notre audace est suivie.}%
\rptflorpan{Ce qu'on fait pour LOUIS, on ne le perd jamais.}%
\rptquatre{Au soin de ses plaisirs donnons-nous désormais.}%
\rptflorpan{Heureux, heureux qui peut lui consacrer sa vie.}%
\rpttous{Joignons tous dans ces bois\separvers{}Nos flûtes et nos voix,\separvers{}Ce jour nous y convie,\separvers{}Et faisons aux échos redire mille fois,\separvers{}\og{}LOUIS est le plus grand des rois.\separvers{}Heureux, heureux, qui peut lui consacrer sa vie !\fg{}}%
%
\TheatreMvt[dernière et grande entrée de ballet]{Faunes, Bergers et Bergères, tous se mêlent, et il se fait entre eux des jeux de danse, après quoi ils se vont préparer pour la Comédie}%
\end{theatre}%

% % % % % % % % % % % % % % % % % % % % % % % % % % % % % % % % % % % % % % % % % % % % % % % % % % % % % % %
% % % % % % % % % % % % % % % % % % % % % % % % % % % % % % % % % % % % % % % % % % % % % % % % % % % % % % %
% % % % % % % % % % % % % % % % % % % % % % % % % % % % % % % % % % % % % % % % % % % % % % % % % % % % % % %
% % %                                                                                                   % % %
% % % Classe du document à créer                                                                        % % %
%\documentclass[11pt,a4paper,twoside]{memoir}%  Nettoyer les fichiers après l'utilisation de memoir     % % %
\documentclass[11pt,a4paper,twoside]{book}%                                                             % % %
%\documentclass[11pt,a4paper,twoside]{article}%  attention au niveau chapitre manquant                  % % %
% % %---------------------------------------------------------------------------------------------------% % %
%% % % % % % % % % % % % % % % % % % % % % % % % % % % % % % % % % % % % % % % % % % % % % % % % % % % % % %
% % % % % % % % % % % % % % % % % % % % % % % % % % % % % % % % % % % % % % % % % % % % % % % % % % % % % %
% % %                                                                                                 % % %
% % %  Paramètres pour les fontes classiques de LaTeX (un autre fichier est fait pour XeTeX)          % % % 
% % %  Encodage des fichiers, fontes, langue                                                          % % %
% % %                                                                                                 % % %
% % %                                                                                                 % % %
\usepackage[utf8]{inputenc}%            Accents dans le fichier tex                                   % % %
\usepackage[T1]{fontenc}%               Demande d'utiliser des fontes avec le codage T1.              % % %
\usepackage{libertine}%                 Une fonte libre et opensource                                 % % %
\usepackage[francais]{babel}%           Les règles de typo. françaises                                % % %
\newcommand{\cc}{\c{c}}%                Problème du c cédille                                         % % %
\usepackage{eurosym}%                   pour le symbole Euro                                          % % %
\usepackage{amsmath,amsfonts,amssymb}%  spécial mathématique                                          % % %
\usepackage{harmony}%                   pour les notes de musiques                                    % % %
\setlength{\parindent}{0pt}%            pour éliminer le retrait de la première ligne des paragraphes % % %
\sloppy%                                Ne pas faire déborder les lignes dans la marge                % % %
% % %                                                                                                 % % %
% % % % % % % % % % % % % % % % % % % % % % % % % % % % % % % % % % % % % % % % % % % % % % % % % % % % % %
% % % % % % % % % % % % % % % % % % % % % % % % % % % % % % % % % % % % % % % % % % % % % % % % % % % % % %% le fichier pour utiliser LaTeX et non pas XeTeX                     % % %
% % % % % % % % % % % % % % % % % % % % % % % % % % % % % % % % % % % % % % % % % % % % % % % % % % % % % %
% % % % % % % % % % % % % % % % % % % % % % % % % % % % % % % % % % % % % % % % % % % % % % % % % % % % % %
% % %                                                                                                 % % %
% % %  Paramètres pour les fontes classiques de LaTeX (un autre fichier est fait pour XeTeX)          % % % 
% % %  Encodage des fichiers, fontes, langue                                                          % % %
% % %                                                                                                 % % %
% % %                                                                                                 % % %
\usepackage[francais]{babel}%           Les règles de typo. françaises                                % % %
\usepackage{fontspec}
\fontspec{Linux Libertine O}%
\setmainfont{Linux Libertine O}[Numbers={OldStyle}, Ligatures={Historic}, Alternate=0]% déclare rm family
\setsansfont{Linux Biolinum O}[Numbers={OldStyle}, Ligatures={Historic}, Alternate=0]% déclare sffamily
\usepackage{sectsty}\allsectionsfont{\normalfont\sffamily\bfseries}% Pour choisir la fonte des titres
%\setmonofont{⟨font name⟩}[⟨font features⟩]
%\newfontfamily⟨cmd⟩{⟨font name⟩}[⟨font features⟩]
\usepackage{setspace}%
\newcommand{\cc}{\c{c}}%                Problème du c cédille                                         % % %
\usepackage{eurosym}%                   pour le symbole Euro                                          % % %
%\usepackage{amsmath,amsfonts,amssymb}%  spécial mathématique                                         % % %
\usepackage{harmony}%                   pour les notes de musiques                                    % % %
\setlength{\parindent}{0pt}%            pour éliminer le retrait de la première ligne des paragraphes % % %
\sloppy%                                Ne pas faire déborder les lignes dans la marge                % % %
% % % % % % % % % % % % % % % % % % % % % % % % % % % % % % % % % % % % % % % % % % % % % % % % % % % % % %
% % % % % % % % % % % % % % % % % % % % % % % % % % % % % % % % % % % % % % % % % % % % % % % % % % % % % %% le fichier pour utiliser XeTeX pour les fontes linux libertine...    % % %
% % %---------------------------------------------------------------------------------------------------% % %
% % % Fichiers avec des commandes personnalisées                                                        % % %
% % %                                                                                                   % % %
\usepackage{util/chimakehelps}%  Série de fonctions pour standadiser mes fichiers helps                 % % %
\savegeometry{base}%            les valeurs par défaut de l'extension                                   % % %
\chihelppages{}% cére les mises en pages                                                                % % %
% % %                                                                                                   % % %
% % %---------------------------------------------------------------------------------------------------% % %
% % % L'extension pour les dialogues                                                                    % % %
\usepackage{theatre}%                        extension de ce package                                    % % %
% % % % % % % % % % % % % % % % % % % % % % % % % % % % % % % % % % % % % % % % % % % % % % % % % % % % % %
% % % % % % % % % % % % % % % % % % % % % % % % % % % % % % % % % % % % % % % % % % % % % % % % % % % % % %
% % %                                                                                                 % % %
% % %  Personnaliser les textes                                                                       % % %
% % %                                                                                                 % % %
% % % Pour rappel, la taille normale du texte s'appelle : normalsize                                  % % % 
% % % Les tailles plus petites que la normale sont : small, footnotesize, scriptsize et tiny          % % %
% % % Les tailles plus grandes que la normale sont : large, Large, LARGE, huge et Huge                % % %
% % %                                                                                                 % % %
% % % Les couleurs prédéfinies sont les suivantes                                                     % % %
% % % black, white, red, green, blue, cyan, magenta, yellow                                           % % % 
% % %                                                                                                 % % %
% % % - - - - - - - - - - - - - - - - - - - - - - - - - - - - - - - - - - - - - - - - - - - - - - - - % % %
% % % - - - - - - - - - - - - - - - - - - - - - - - - - - - - - - - - - - - - - - - - - - - - - - - - % % %
% % %                                                                                                 % % %
% % % - - - - - - - - - - - - - - - - - - - - - - - - - - - - - - - - - - - - - - - - - - - - - - - - % % %
% % % Personnalisation des textes de mouvement d'acteur                                               % % %
% % % - - - - - - - - - - - - - - - - - - - - - - - - - - - - - - - - - - - - - - - - - - - - - - - - % % %
% % %                                                                                                 % % %
\def\TheatreMvtHt{normalsize}%                                                                        % % %
\def\TheatreMvtCouleur{blue}%                                                                         % % %
% % %                                                                                                 % % %
% % %                                                                                                 % % %
% % % - - - - - - - - - - - - - - - - - - - - - - - - - - - - - - - - - - - - - - - - - - - - - - - - % % %
% % % Personnalisation des textes des commentaires                                                    % % %
% % % - - - - - - - - - - - - - - - - - - - - - - - - - - - - - - - - - - - - - - - - - - - - - - - - % % %
% % %                                                                                                 % % %
\def\TheatreCommentHt{normalsize}%                                                                    % % %
\def\TheatreCommentCouleur{red}%                                                                      % % %
% % %                                                                                                 % % %
% % %                                                                                                 % % %
% % % - - - - - - - - - - - - - - - - - - - - - - - - - - - - - - - - - - - - - - - - - - - - - - - - % % %
% % % Personnalisation de la présentation des noms des rôles                                          % % %
% % % - - - - - - - - - - - - - - - - - - - - - - - - - - - - - - - - - - - - - - - - - - - - - - - - % % %
% % %                                                                                                 % % %
\setlength{\TheatreEcarterRepliques}{0ex}%     écarter les répliques                                  % % %
\setlength{\TheatreEntreNomEtReplique}{0.5ex}% écarter le nom du role et la réplique                  % % %
\def\TheatreTailleRoleCpte{small}%       taille du compteur                                           % % %
\def\TheatreTailleRoleNom{small}%        taille du nom du rôle au-dessus de la réplique               % % %
\setcounter{TheatreCpteReplique}{1}%     initialisation du compteur, on peut l'utiliser dans le texte % % %
% % %                                                                                                 % % %
% % % % % % % % % % % % % % % % % % % % % % % % % % % % % % % % % % % % % % % % % % % % % % % % % % % % % %
% % % % % % % % % % % % % % % % % % % % % % % % % % % % % % % % % % % % % % % % % % % % % % % % % % % % % %
%
%
%
% % % % % % % % % % % % % % % % % % % % % % % % % % % % % % % % % % % % % % % % % % % % % % % % % % % % % %
% % % % % % % % % % % % % % % % % % % % % % % % % % % % % % % % % % % % % % % % % % % % % % % % % % % % % %
% % %                                                                                                 % % %
% % %                 C'est ici qu'il faut créer les différents rôles de la pièce                     % % %
% % %                                                                                                 % % %
% % %                                                                                                 % % %
% % % % % % % % % % % % % % % % % % % % % % % % % % % % % % % % % % % % % % % % % % % % % % % % % % % % % %
% % %                                                                                                 % % %
% % % Structure :  \TheatreCreerUnRole{nomcourt}{nomlong}{petitetaille}{grandetaille}{couleur}        % % %
% % % ----------------------------------------------------------------------------------------------- % % %
% % %   nomcourt : sera utilisé pour créer le nom des macros des rôles, avec                          % % %
% % %               r pour réplique                                                                   % % %
% % %               pt pour petite taille de texte -> donc macro créée \rptnomcourt                   % % %
% % %               gd pour grande taille de texte -> donc macro créée \rgdnomcourt                   % % %
% % % ----------------------------------------------------------------------------------------------- % % %
% % %   nomlong : ce sera ce nom qui sera imprimé au-dessus des répliques                             % % %
% % % ----------------------------------------------------------------------------------------------- % % %
% % %   petitetaille et grandetaille : nécessaire si différents éclairages ou                         % % %
% % %                                  différentes distances lors de la lecture                       % % %
% % %               à choisir parmi normalsize - large - Large - LARGE - huge - HUGE                  % % %
% % %                                                                                                 % % %
% % % ----------------------------------------------------------------------------------------------- % % %
% % %                                                                                                 % % %
% % %   couleur : pour personnaliser un rôle, à choisir parmi                                         % % %
% % %             black, white, red, green, blue, cyan, magenta, yellow                               % % %
% % %                                                                                                 % % %
% % % - - - - - - - - - - - - - - - - - - - - - - - - - - - - - - - - - - - - - - - - - - - - - - - - % % %
% % % - - - - - - - - - - - - - - - - - - - - - - - - - - - - - - - - - - - - - - - - - - - - - - - - % % %
% % %                                                                                                 % % %
% % %   À vous de créer vos rôles                                                                     % % %
% % %                                                                                                 % % %
% % % - - - - - - - - - - - - - - - - - - - - - - - - - - - - - - - - - - - - - - - - - - - - - - - - % % %
% % % - - - - - - - - - - - - - - - - - - - - - - - - - - - - - - - - - - - - - - - - - - - - - - - - % % %
%
\TheatreCreerUnRole{flore}{Flore}{normalsize}{LARGE}{black}%
\TheatreCreerUnRole{pan}{Pan}{normalsize}{LARGE}{black}%
\TheatreCreerUnRole{florpan}{Flore et Pan}{normalsize}{LARGE}{black}%
%
\TheatreCreerUnRole{climene}{Climène}{normalsize}{LARGE}{black}%
%
\TheatreCreerUnRole{daphne}{Daphné}{normalsize}{LARGE}{black}%
\TheatreCreerUnRole{climdaph}{Climène et Daphné}{normalsize}{LARGE}{black}%
%
\TheatreCreerUnRole{tircis}{Tircis}{Large}{LARGE}{blue}%
%
\TheatreCreerUnRole{dorilas}{Dorilas}{normalsize}{LARGE}{black}%
\TheatreCreerUnRole{tirdor}{Tircis et Dorilas}{normalsize}{LARGE}{black}%
%
\TheatreCreerUnRole{zephirs}{Deux Zéphirs}{normalsize}{LARGE}{black}%
%
\TheatreCreerUnRole{troupe}{Troupe de bergères et de bergers}{normalsize}{LARGE}{black}%
\TheatreCreerUnRole{tous}{Tous}{normalsize}{LARGE}{black}%
\TheatreCreerUnRole{quatre}{Les quatre amants}{normalsize}{LARGE}{black}%
%     un fichier préparé contenant les définitions des rôles, exemple français     % % %
% % % % % % % % % % % % % % % % % % % % % % % % % % % % % % % % % % % % % % % % % % % % % % % % % % % % % %
% % % % % % % % % % % % % % % % % % % % % % % % % % % % % % % % % % % % % % % % % % % % % % % % % % % % % %
% % %                                                                                                 % % %
% % %  Customization of the texts                                                                     % % %
% % %                                                                                                 % % %
% % % As a reminder, the normal size of the text is called: normalsize                                % % % 
% % % The sizes smaller than the normal are: small, footnotesize, scriptsize et tiny                  % % %
% % % The sizes bigger than the normal are: large, Large, LARGE, huge et Huge                         % % %
% % %                                                                                                 % % %
% % % The predefined colors are the following ones                                                    % % %
% % % black, white, red, green, blue, cyan, magenta, yellow                                           % % % 
% % %                                                                                                 % % %
% % % - - - - - - - - - - - - - - - - - - - - - - - - - - - - - - - - - - - - - - - - - - - - - - - - % % %
% % % - - - - - - - - - - - - - - - - - - - - - - - - - - - - - - - - - - - - - - - - - - - - - - - - % % %
% % %                                                                                                 % % %
% % % - - - - - - - - - - - - - - - - - - - - - - - - - - - - - - - - - - - - - - - - - - - - - - - - % % %
% % % Customization of the actor's movements                                                          % % %
% % % - - - - - - - - - - - - - - - - - - - - - - - - - - - - - - - - - - - - - - - - - - - - - - - - % % %
% % %                                                                                                 % % %
\def\TheatreMvtHt{normalsize}%                                                                        % % %
\def\TheatreMvtCouleur{blue}%                                                                         % % %
% % %                                                                                                 % % %
% % %                                                                                                 % % %
% % % - - - - - - - - - - - - - - - - - - - - - - - - - - - - - - - - - - - - - - - - - - - - - - - - % % %
% % % Customization of the comments                                                                   % % %
% % % - - - - - - - - - - - - - - - - - - - - - - - - - - - - - - - - - - - - - - - - - - - - - - - - % % %
% % %                                                                                                 % % %
\def\TheatreCommentHt{normalsize}%                                                                    % % %
\def\TheatreCommentCouleur{red}%                                                                      % % %
% % %                                                                                                 % % %
% % %                                                                                                 % % %
% % % - - - - - - - - - - - - - - - - - - - - - - - - - - - - - - - - - - - - - - - - - - - - - - - - % % %
% % % Customization of the display of the names of the roles.                                         % % %
% % % - - - - - - - - - - - - - - - - - - - - - - - - - - - - - - - - - - - - - - - - - - - - - - - - % % %
% % %                                                                                                 % % %
\setlength{\TheatreEcarterRepliques}{0ex}%    Separate the retorts                                    % % %
\setlength{\TheatreEntreNomEtReplique}{0.5ex}%Separate the name of the role and the retort            % % %
\def\TheatreTailleRoleCpte{small}%            Size of the counter                                     % % %
\def\TheatreTailleRoleNom{small}%             Size of the name of the role over the retort            % % %
\setcounter{TheatreCpteReplique}{1}%          Initialization of the counter, we can use it in the text% % %
% % %                                                                                                 % % %
% % % % % % % % % % % % % % % % % % % % % % % % % % % % % % % % % % % % % % % % % % % % % % % % % % % % % %
% % % % % % % % % % % % % % % % % % % % % % % % % % % % % % % % % % % % % % % % % % % % % % % % % % % % % %
%
%
%
% % % % % % % % % % % % % % % % % % % % % % % % % % % % % % % % % % % % % % % % % % % % % % % % % % % % % %
% % % % % % % % % % % % % % % % % % % % % % % % % % % % % % % % % % % % % % % % % % % % % % % % % % % % % %
% % %                                                                                                 % % %
% % %                      Create here the various roles of the play.                                 % % %
% % %                                                                                                 % % %
% % %                                                                                                 % % %
% % % % % % % % % % % % % % % % % % % % % % % % % % % % % % % % % % % % % % % % % % % % % % % % % % % % % %
% % %                                                                                                 % % %
% % % Structure :  \TheatreCreerUnRole{nomcourt}{nomlong}{petitetaille}{grandetaille}{couleur}        % % %
% % % ----------------------------------------------------------------------------------------------- % % %
% % %   nomcourt : will be used to create the name of the macro of the roles, with                    % % %
% % %               r for retort                                                                      % % %
% % %               pt for small (petite) text size -> command created \rptnomcourt                   % % %
% % %               gd for large (grande) text size -> command created \rgdnomcourt                   % % %
% % % ----------------------------------------------------------------------------------------------- % % %
% % %   nomlong : this name will be printed over the dialogues                                        % % %
% % % ----------------------------------------------------------------------------------------------- % % %
% % %   petitetaille et grandetaille : usefull if various lightings or                                % % %
% % %                                  carious distances when reading                                 % % %
% % %               à choisir parmi normalsize - large - Large - LARGE - huge - HUGE                  % % %
% % %                                                                                                 % % %
% % % ----------------------------------------------------------------------------------------------- % % %
% % %                                                                                                 % % %
% % %   Color: to personalize a role, choose in                                                       % % %
% % %             black, white, red, green, blue, cyan, magenta, yellow                               % % %
% % %                                                                                                 % % %
% % % - - - - - - - - - - - - - - - - - - - - - - - - - - - - - - - - - - - - - - - - - - - - - - - - % % %
% % % - - - - - - - - - - - - - - - - - - - - - - - - - - - - - - - - - - - - - - - - - - - - - - - - % % %
% % %                                                                                                 % % %
% % %   Create your roles, now.                                                                       % % %
% % %                                                                                                 % % %
% % % - - - - - - - - - - - - - - - - - - - - - - - - - - - - - - - - - - - - - - - - - - - - - - - - % % %
% % % - - - - - - - - - - - - - - - - - - - - - - - - - - - - - - - - - - - - - - - - - - - - - - - - % % %
%
\TheatreCreerUnRole{ducan}{Ducan}{normalsize}{LARGE}{magenta}%
\TheatreCreerUnRole{malcolm}{Malcolm}{normalsize}{LARGE}{black}%
\TheatreCreerUnRole{sergeant}{Sergeant}{normalsize}{LARGE}{black}%
\TheatreCreerUnRole{ross}{Ross}{normalsize}{LARGE}{black}%
\TheatreCreerUnRole{lennox}{Lennox}{normalsize}{LARGE}{black}%

%       un fichier préparé contenant les définitions des rôles, exemple anglais      % % %
% % %                                                                                                   % % %
% % % % % % % % % % % % % % % % % % % % % % % % % % % % % % % % % % % % % % % % % % % % % % % % % % % % % % %
% % % % % % % % % % % % % % % % % % % % % % % % % % % % % % % % % % % % % % % % % % % % % % % % % % % % % % %
% % % % % % % % % % % % % % % % % % % % % % % % % % % % % % % % % % % % % % % % % % % % % % % % % % % % % % %
%
% % % % % % % % % % % % % % % % % % % % % % % % % % % % % % % % % % % % % % % % % % % % % % % % % % % % % % %
% % % % % % % % % % % % % % % % % % % % % % % % % % % % % % % % % % % % % % % % % % % % % % % % % % % % % % %
% % % % % % % % % % % % % % % % % % % % % % % % % % % % % % % % % % % % % % % % % % % % % % % % % % % % % % %
% % %                                                                                                   % % %
% % % Renseignements sur le fichier                                                                     % % %
% % %                                                                                                   % % %
\def\titre{Aide à l'utilisation de l'extension theatre.sty}%                                            % % %
\def\auteur{Christian Hinque - LeChi}%                                                                  % % %
\def\traducteur{Virgil Hinque}%                                                                         % % %
\def\jour{\today}%                                                                                      % % %
\def\sujet{Manuel Extension LaTeX}%                                                                     % % %
\def\motsclefs{}%                                                                                       % % %
\chihelpentetpied{}% crée les pieds de page et entêtes avec ces données                                 % % %
% % %---------------------------------------------------------------------------------------------------% % %
% % % Paramètres pour la création du pdf                                                                % % %
% % %                                                                                                   % % %
% % %                                                                                                   % % %
\RequirePackage[colorlinks=true,urlcolor=black,linkcolor=black]{hyperref}%                              % % %
% % %                                                                                                   % % %
\hypersetup{%                     Mo difiez la valeur des champs suivants                               % % %
    pdfauthor = {\auteur{}},%       attention de bien définir \auteur au début du fichier compilateur   % % %
    pdftitle  = {\titre{}},%        attention de bien définir \titre au début du fichier compilateur    % % %
    pdfsubject = {\sujet{}},%       attention de bien définir \sujet au début du fichier compilateur    % % %
    pdfkeywords = {\motsclefs{}},%  attention de bien définir \motsclefs au début du fichier compilateur% % %
    pdfcreator  = {TexMaker},%                                                                          % % %
    pdfproducer = {PDFLaTeX},%                                                                          % % %
%   pdfpagemode = {FullScreen}%   ouvre le pdf directement en plein écran                               % % %
    bookmarks = true%             pour exporter les signets.                                            % % %
}%                                                                                                      % % %
% % % % % % % % % % % % % % % % % % % % % % % % % % % % % % % % % % % % % % % % % % % % % % % % % % % % % % %
% % % % % % % % % % % % % % % % % % % % % % % % % % % % % % % % % % % % % % % % % % % % % % % % % % % % % % %
% % % % % % % % % % % % % % % % % % % % % % % % % % % % % % % % % % % % % % % % % % % % % % % % % % % % % % %
%
%

%
%
%
% $ $ $ $ $ $ $ $ $ $ $ $ $ $ $ $ $ $ $ $ $ $ $ $ $ $ $ $ $ $ $ $ $ $ $ $ $ $ $ $ $ $ $ $ $ $ $ $ $ $ $ $ $ $
% $ $                                                                                                   $ $ $
% $ $    DÉBUT DU DOCUMENT DÉBUT DU DOCUMENT DÉBUT DU DOCUMENT DÉBUT DU DOCUMENT DÉBUT DU DOCUMENT      $ $ $
\begin{document}%                                                                                       $ $ $
% $ $ $ $ $ $ $ $ $ $ $ $ $ $ $ $ $ $ $ $ $ $ $ $ $ $ $ $ $ $ $ $ $ $ $ $ $ $ $ $ $ $ $ $ $ $ $ $ $ $ $ $ $ $
% - - - - - - - - - - - - - - - - - - - - - - - - - - - - - - - - - - - - - - - - - - - - - - - - - - - - - -
% - -    Titre de l'article ou du livre                                                                 - - -
% - - - - - - - - - - - - - - - - - - - - - - - - - - - - - - - - - - - - - - - - - - - - - - - - - - - - - -
\loadchihelpgeometry{helplaune}%
\title{\titre}\author{\auteur{}\\ English version : \traducteur{}}\date{\jour}%
\maketitle{}\thispagestyle{empty}%                    crée une page de couverture
%
%\chirappelmiseenpage{sanszonenotes}%
\loadchihelpgeometry{helpsanszonenotes}
\frontmatter{}%
% - - - - - - - - - - - - - - - - - - - - - - - - - - - - - - - - - - - - - - - - - - - - - - - - - - - - - -
% - -    Table des matières                                                                             - - -
% - - - - - - - - - - - - - - - - - - - - - - - - - - - - - - - - - - - - - - - - - - - - - - - - - - - - - -
\tableofcontents%
\mainmatter{}%
% - - - - - - - - - - - - - - - - - - - - - - - - - - - - - - - - - - - - - - - - - - - - - - - - - - - - - -
% - -    Insertion du fichier Help                                                                      - - -
% - - - - - - - - - - - - - - - - - - - - - - - - - - - - - - - - - - - - - - - - - - - - - - - - - - - - - -
\chapter{Présentations - Helps}
%\def\chiniveauins{chapter}%  niveau d'insertion des textes
\def\chiniveauins{section}% pour l'exemple article, car il n'y a pas de niveau chapitre en classe article.
%
% % % % % % % % % % % % % % % % % % % % % % % % % % % % % % % % % % % % % % % % % % % % % % % % % % % % % % %
% % % % % % % % % % % % % % % % % % % % % % % % % % % % % % % % % % % % % % % % % % % % % % % % % % % % % % %
% % %                                                                                                   % % %
\section{Extension theatre.sty - Aide en Français}%                                                     % % %
% % %                                                                                                   % % %
% % % % % % % % % % % % % % % % % % % % % % % % % % % % % % % % % % % % % % % % % % % % % % % % % % % % % % %
% % % % % % % % % % % % % % % % % % % % % % % % % % % % % % % % % % % % % % % % % % % % % % % % % % % % % % %
%
\subsection{Objet}%
Le but de cette extension est de pouvoir insérer des dialogues dans un texte en personnalisant les impressions par acteur.

Pour ce faire, il faut préalablement créer les rôles et les sauver dans un fichier séparé. Pour chaque rôle, on définit son nom, la couleur d'impression et deux tailles de caractère (en vision claire ou proche et en vision sombre ou éloignée) .

Ensuite, un environnement \og{}theatre\fg{} permet d'écrire les répliques. Elles seront numérotées automatiquement, ce qui facilite le suivi, vu que plusieurs mises en page peuvent être générées selon les demandes des lecteurs.

\bigskip{}

\subsection{Les commandes}
Cette extension prévoit qu'on puisse créer autant de rôles que nécessaires, via la commande de création de rôle.
De plus deux commandes sont prévues, une pour ajouter des commentaires, l'autre pour décrire des mouvements. Elles permettent également de placer un commentaire avant le texte proprement-dit.

\subsubsection{Préparer les rôles}
Dans un fichier séparé, on crée le nombre nécessaire de rôles.\\ La commande \chicom{TheatreCreerUnRole} est prévue à cet effet.

\medskip{}
Sa structure est la suivante :\\
\begin{chicitecommandelarge}
\TheatreCreerUnRole{nomcourt}{nomlong}{petit}{grand}{couleur}
\end{chicitecommandelarge}
\medskip{}
\begin{enumerate}
  \item \textbf{nomcourt} sera utilisé pour créer le nom des macros des rôles, un \emph{\textbf{r}} (pour réplique) puis deux lettres précéderont le nom court
    \begin{itemize}
      \item \textbf{pt} pour petite taille de texte -> donc macro créée \chicom{rptnomcourt},
      \item \textbf{gd} pour grande taille de texte -> donc macro créée \chicom{rgdnomcourt}.
    \end{itemize}
  \item \textbf{nomlong} sera imprimé au-dessus des répliques.   
  \item \textbf{petit} et \textbf{grand} définissent la taille du texte pour chacune des deux macros créées. Ces tailles sont à choisir de préférence parmi les grandes tailles (large, Large, LARGE, huge et HUGE) ou rester à la taille standard (normalsize). Mais rien n'empêche de choisir un taille plus petite que le standard (small, footnotesize, scriptsize ou tiny).
  \item \textbf{couleur} pour personnaliser un rôle, à choisir parmi black, white, red, green, blue, cyan, magenta et yellow. Cependant, avec des packages plus spécialisés en couleur, on peut avoir un choix plus grand.
\end{enumerate}

\medskip{}

Deux variables décident d'autres tailles. Elles peuvent être redéfinies quand on veut (\chivar{def}\chivar{variable\dots}).
\begin{itemize}
\item \chivar{TheatreTailleRoleCpte} décide de la taille du compteur qui s'imprime devant la réplique. Par défaut : small.
\item \chivar{TheatreTailleRoleNom} décide de la taille du rôle qui s'imprime au-dessus de la réplique. Par défaut : small.
\end{itemize}

\medskip{}

Deux longueurs peuvent être redéfinies, en n'oubliant pas l'unité de longueur, par la commande \chicom[nom\}\{Nex]{setlength} (1ex, 2ex \dots)
\begin{itemize}
\item \chivar{TheatreEntreNomEtReplique}  augmente l'espace entre le nom et la réplique.%
\item \chivar{TheatreEcarterRepliques} augmente l'espace entre les répliques.%
\end{itemize}
%-------------------------------------
\subsubsection{Utiliser les rôles créés}
Les commandes de \og{}theatre\fg{} permettent d'écrire un commentaire sur la ligne du nom du rôle. Ce commentaire n'est pas obligatoire, il s'écrira donc entre \og{}[\,]\fg{}, comme il se doit pour les variables optionnelles.

\bigskip{}

\begin{center}
\begin{minipage}[h]{0.9\textwidth}
\def\TheatreEcarterRepliques{3mm}
\begin{SideBySideExample}[gobble=0,fontfamily=lmss,fontseries=b,fontsize=\footnotesize,xrightmargin=0.5\textwidth,baselinestretch=2]
\CommandeEcrite
\begin{theatre}
\rptflore[\emph{commentaire}]{Quittez,\dots}
\rptclimdaph{Berger,\dots}
\rgdtirdor{Mais\dots}
\end{theatre}
\end{SideBySideExample}
\end{minipage}
\end{center}

\bigskip{}

Il suffit donc de se souvenir des noms donnés aux différents rôles.

%---------------
\subsubsection{\chicom{TheatreMvt} : pour décrire les mouvements.}
\begin{itemize}
\item \chivar{TheatreMvtHt} détermine la hauteur du texte,
\item \chivar{TheatreMvtCouleur} détermine sa couleur.
\end{itemize}

\bigskip{}

\begin{center}
\begin{minipage}[h]{0.9\textwidth}
\begin{SideBySideExample}[gobble=0,fontfamily=lmss,fontseries=b,fontsize=\footnotesize,xrightmargin=0.5\textwidth,baselinestretch=2]
\CommandeEcrite

\TheatreMvt[commentaire]{Le texte\dots}

\TheatreMvt{Le texte\dots}
\end{SideBySideExample}
\end{minipage}
\end{center}

\bigskip{}

%----------------------------
\subsubsection{\chicom{TheatreComment} : pour placer des commentaires dans le texte}
\begin{itemize}
\item \chivar{TheatreCommentHt} détermine la hauteur du texte,
\item \chivar{TheatreCommentCouleur} détermine sa couleur.
\end{itemize}

\bigskip{}

\begin{center}
\begin{minipage}[h]{0.9\textwidth}
\begin{SideBySideExample}[gobble=0,fontfamily=lmss,fontseries=b,fontsize=\footnotesize,xrightmargin=0.5\textwidth,baselinestretch=2]
\CommandeEcrite

\TheatreComment[commentaire]{Le texte\dots}

\TheatreComment{Le texte\dots}
\end{SideBySideExample}
\end{minipage}
\end{center}
%
\newpage{}%
% % % % % % % % % % % % % % % % % % % % % % % % % % % % % % % % % % % % % % % % % % % % % % % % % % % % % % %
% % % % % % % % % % % % % % % % % % % % % % % % % % % % % % % % % % % % % % % % % % % % % % % % % % % % % % %
% % %                                                                                                   % % %
\section{About the theatre.sty extension}%                                                              % % %
% % %                                                                                                   % % %
% % % % % % % % % % % % % % % % % % % % % % % % % % % % % % % % % % % % % % % % % % % % % % % % % % % % % % %
% % % % % % % % % % % % % % % % % % % % % % % % % % % % % % % % % % % % % % % % % % % % % % % % % % % % % % %
%
\subsection{Purpose}%

The point of this extension is to insert dialogues in a text while being able to customize the printout for each actor.

To do so, the roles must be defined and saved in a separated file. Each role is parametrized by its name, its printout color and two font sizes (one for short-range reading and one for long-range reading).

The lines can then be written in an environment named \textit{theatre}. These lines are automatically numbered in order to ease the reading as several layouts can be generated according to the demands of readers.

\bigskip{}

\subsection{Commands}
This extension provides one can create as many roles as necessary, by using the command \chicom{TheatreCreerUnRole}.
Two other commands are available. One adds comments, the other describes motions.

\subsubsection{Preparing the roles}
The roles are created in a separated file. This is done using the \chicom{chicreerunrole} command and the following arguments:

\begin{chicitecommandelarge}
\TheatreCreerUnRole{nomcourt}{nomlong}{petit}{grand}{couleur}
\end{chicitecommandelarge}
%\begin{chicitecommande}\chicom[nomcourt\}\{nomlong\}\{petit\}\{grand\}\{couleur]{chicreerunrole}\end{chicitecommande}

\begin{enumerate}
  \item \chivar{nomcourt} is used to generate the name of the macro of this role. This name is composed of a \emph{\textbf{r}}, two others letters and finally the \textbf{nomcourt}. The \emph{\textbf{r}} refers to \textit{répliques}, which is the french word for \textit{line}. The two other letters depend on the size of the text:
  
    \begin{itemize}
      \item pt for a small font size -> the macro name is then \chicom{rptnomcourt},
      \item gd for a large font size -> the macro name is then \chicom{rgdnomcourt}.
    \end{itemize}
  \item \chivar{nomlong} is printed above each line.   
  \item \chivar{petit} and \chivar{grand} define the text size for both created macro. These sizes must be either large (large, Large, LARGE, huge or HUGE), standard (normalsize) or small (small, footnotesize, scriptsize or tiny).
  \item \chivar{color} allows to customize a role with a color. It can be black, white, red, green, blue, cyan, magenta or yellow. Dedicated packages can provide a larger choice.
\end{enumerate}

\medskip{}

Two variables respectively define two other sizes.
\begin{itemize}
\item \chivar{chitaillerolecpte} defines the size of the counter printed before the line. Default value: small.
\item \chivar{chitaillenomrole} defines the size of the role name printed above the line. Default value: small.
\end{itemize}

\medskip{}

Two lengths can be redefined with the command \chicom[nom\}\{1ex]{setlength}. The length unit must be specified!

\begin{itemize}
\item \chivar{chientrenometreplique}  increases the space between the name and the line.%
\item \chivar{chiecarterlesrepliques} increases the space between each line.%
\end{itemize}
%-------------------------------------
\subsubsection{Using the roles}
The commands of \textit{theatre} allow to write a comment on the name line of the role. This comment is optional, so it will be written between \og{}[\,]\fg{}.

\bigskip{}

\begin{center}
\begin{minipage}[h]{0.9\textwidth}
\def\TheatreEcarterRepliques{3mm}
\begin{SideBySideExample}[gobble=0,fontfamily=lmss,fontseries=b,fontsize=\footnotesize,xrightmargin=0.5\textwidth,baselinestretch=2]
\CommandeEcrite
\begin{theatre}
\rptflore[\emph{commentaire}]{Quittez,\dots}
\rptclimdaph{Berger,\dots}
\rgdtirdor{Mais\dots}
\end{theatre}
\end{SideBySideExample}
\end{minipage}
\end{center}

\bigskip{}

The only challenge is to remember the name given to each role.


%---------------
\subsubsection{\chicom{TheatreMvt} : to describe the motions.}
\begin{itemize}
\item \chivar{TheatreMvtHt} defines the text height,
\item \chivar{TheatreMvtCouleur} fixes its color.
\end{itemize}

\bigskip{}

\begin{center}
\begin{minipage}[h]{0.9\textwidth}
\begin{SideBySideExample}[gobble=0,fontfamily=lmss,fontseries=b,fontsize=\footnotesize,xrightmargin=0.5\textwidth,baselinestretch=2]
\CommandeEcrite

\TheatreMvt[commentaire]{Le texte\dots}

\TheatreMvt{Le texte\dots}
\end{SideBySideExample}
\end{minipage}
\end{center}

\bigskip{}

%----------------------------
\subsubsection{\chicom{TheatreComment} : to add comments in the text}
\begin{itemize}
\item \chivar{TheatreCommentHt} defines the text height,
\item \chivar{TheatreCommentCouleur} fixes its color.
\end{itemize}

\bigskip{}

\begin{center}
\begin{minipage}[h]{0.9\textwidth}
\begin{SideBySideExample}[gobble=0,fontfamily=lmss,fontseries=b,fontsize=\footnotesize,xrightmargin=0.5\textwidth,baselinestretch=2]
\CommandeEcrite

\TheatreComment[commentaire]{Le texte\dots}

\TheatreComment{Le texte\dots}
\end{SideBySideExample}
\end{minipage}
\end{center}
%
\newpage{}
\section{Liste des commandes, variables et extensions}
(List of available commands, variables and extensions)\\
\def\chinompackage{theatre}%
%^^^^^^^^^^^^^^^^^^^^^^^^^^^^^^^^^^^^^^^
\def\chitheme{Packages utilisés}%
%^^^^^^^^^^^^^^^^^^^^^^^^^^^^^^^
\begin{chitableaucommandes}%
\chilstcom{enumitem}{\chitheme}{\chinompackage}%
\end{chitableaucommandes}%
%
\\
\def\chitheme{Commandes de l'extension}%
%^^^^^^^^^^^^^^^^^^^^^^^^^^^^^^^^^^
\begin{chitableaucommandes}%
\chilstcom{TheatreCreerUnRole[5]}{\chitheme}{\chinompackage}%
\chilstcom{TheatreNomDuRole[1]}{\chitheme}{\chinompackage}%
\chilstcom{TheatreIniCpteReplique[1]}{\chitheme}{\chinompackage}%
\chilstcom{TheatreMvt[1]}{\chitheme}{\chinompackage}%
\chilstcom{TheatreComment[1]}{\chitheme}{\chinompackage}%
\chilstcom{TheatreXparagraph}{\chitheme}{\chinompackage}%
\end{chitableaucommandes}%
%
\\
\def\chitheme{Commandes redéfinies}%
%^^^^^^^^^^^^^^^^^^^^^^^^^^^^^^^^^^
\begin{chitableaucommandes}%
\chilstcom{paragraph[2]}{\chitheme}{\chinompackage}%
\end{chitableaucommandes}%
\\
\def\chitheme{Environnements de l'extension}%
%^^^^^^^^^^^^^^^^^^^^^^^^^^^^^^^^^^
\begin{chitableaucommandes}%
\chilstcom{theatre}{\chitheme}{\chinompackage}%
\end{chitableaucommandes}%
\\
\def\chitheme{Variables}%
%^^^^^^^^^^^^^^^^^^^^^^^^
\begin{chitableaucommandes}%
\chilstcom{TheatreTailleRoleCpte}{\chitheme}{\chinompackage}%
\chilstcom{TheatreTailleRoleNom}{\chitheme}{\chinompackage}%
\chilstcom{TheatreMvtHt}{\chitheme}{\chinompackage}%
\chilstcom{TheatreMvtCouleur}{\chitheme}{\chinompackage}%
\chilstcom{TheatreCommentHt}{\chitheme}{\chinompackage}%
\chilstcom{TheatreCommentCouleur}{\chitheme}{\chinompackage}%
\end{chitableaucommandes}%
\\%
\def\chitheme{Nouvelles Longueurs}%
%^^^^^^^^^^^^^^^^^^^^^^^^^^^^^^^^
\begin{chitableaucommandes}%
\chilstcom{TheatreSerrerRepliques}{\chitheme}{\chinompackage}%
\chilstcom{TheatreEntreNomEtReplique}{\chitheme}{\chinompackage}%
\chilstcom{TheatreEcarterRepliques}{\chitheme}{\chinompackage}%
\chilstcom{TheatreXparindent}{\chitheme}{\chinompackage}%
\chilstcom{TheatreXparskip}{\chitheme}{\chinompackage}%
\end{chitableaucommandes}%
\\
\def\chitheme{Compteurs}%
%^^^^^^^^^^^^^^^^^^^^^^^
\begin{chitableaucommandes}%
\chilstcom{TheatreCpteReplique}{\chitheme}{\chinompackage}%
\end{chitableaucommandes}%
%
\newpage{}
\chapter{Exemple}\def\chiniveauins{section}%
\loadchihelpgeometry{helpfinezonenotes}%
\setlength{\TheatreEcarterRepliques}{1ex}%  écarter les répliques
\setlength{\TheatreEntreNomEtReplique}{0ex}% écarter le nom du role et la réplique
% % % % % % % % % % % % % % % % % % % % % % % % % % % % % % % % % % % % % % % % % % % % % % % % % % % % % % %
% % % % % % % % % % % % % % % % % % % % % % % % % % % % % % % % % % % % % % % % % % % % % % % % % % % % % % %
% % %                                                                                                   % % %
\section{Le malade imaginaire}%                                                                           % % %
\subsection{Acte 1 - Scène 1}%                                                                              % % %
\TheatreIniCpteReplique{1}% Choisir le premier numéro des répliques                                      % % %
% % %                                                                                                   % % %
% % % % % % % % % % % % % % % % % % % % % % % % % % % % % % % % % % % % % % % % % % % % % % % % % % % % % % %
% % % % % % % % % % % % % % % % % % % % % % % % % % % % % % % % % % % % % % % % % % % % % % % % % % % % % % %
\TheatreComment{Après les glorieuses fatigues, et les exploits victorieux de notre auguste monarque ; il est bien juste que tous ceux qui se mêlent d'écrire, travaillent ou à ses louanges, ou à son divertissement. C'est ce qu'ici l'on a voulu faire, et ce prologue est un essai des louanges de ce grand prince, qui donne entrée à la comédie du Malade imaginaire, dont le projet a été fait pour le délasser de ses nobles travaux.\newline
La décoration représente un lieu champêtre, et néanmoins fort agréable.}%

\bigskip{}

\textbf{\textsc{\florent{}, \climdaphnt, \tirdornt, \pannt, \newline deux zéphirs, une troupe de bergères et bergers}}\newline

%\renewcommand{\separvers}{\newline}%
\def\separvers{\newline}%

\begin{theatre}
\rptflore{Quittez, quittez vos troupeaux,\separvers{}Venez Bergers, venez Bergères,\separvers{}Accourez, accourez sous ces tendres ormeaux ;\separvers{}Je viens vous annoncer des nouvelles bien chères,\separvers{}Et réjouir tous ces hameaux.\separvers{}Quittez, quittez vos troupeaux,\separvers{}Venez bergers, venez Bergères,\separvers{}Accourez, accourez sous ces tendres ormeaux.}%
\rptclimdaph{Berger, laissons là tes feux,\separvers{}Voilà Florequi nous appelle.}%
\rpttirdor{Mais au moins dis-moi, cruelle,}%
\rpttircis{Si d'un peu d'amitié tu payeras mes vœux ?}%
\rptdorilas{Si tu seras sensible à mon ardeur fidèle ?}%
\rptclimdaph{Voilà Flore qui nous appelle.}%
\rpttirdor{Ce n'est qu'un mot, un mot, un seul mot que je veux.}%
\rpttircis{Languirai-je toujours dans ma peine mortelle ?}%
\rptdorilas{Puis-je espérer qu'un jour tu me rendras heureux ?}%
\rptclimdaph{Voilà Flore qui nous appelle.}%
%
\TheatreMvt[entrée de ballet]{Toute la troupe des Bergers et des Bergères va se placer en cadence autour de Flore.}%
%
\rptclimene{Quelle nouvelle parmi nous,\separvers{}Déesse, doit jeter tant de réjouissance ?}%
\rptdaphne{Nous brûlons d'apprendre de vous\separvers{}Cette nouvelle d'importance.}%
\rptdorilas{D'ardeur nous en soupirons tous.}%
\rpttous[ensemble]{Nous en mourons d'impatience.}%
\rptflore{La voici, silence, silence !\separvers{}Vos vœux sont exaucés, LOUIS est de retour,\separvers{}Il ramène en ces lieux les plaisirs et l'amour,\separvers{}Et vous voyez finir vos mortelles alarmes,\separvers{}Par ses vastes exploits son bras voit tout soumis,\separvers{}Il quitte les armes,\separvers{}Faute d'ennemis.}%
\rpttous{Ah quelle douce nouvelle !\separvers{}Qu'elle est grande ! qu'elle est belle !\separvers{}Que de plaisirs ! que de ris ! que de jeux !\separvers{}Que de succès heureux !\separvers{}Et que le Ciel a bien rempli nos vœux,\separvers{}Ah quelle douce nouvelle !\separvers{}Qu'elle est grande ! qu'elle est belle !}%
%
\TheatreMvt[autre entrée de ballet]{Tous les bergers et bergères expriment par des danses les transports de leur joie.}%
%
\rptflore{De vos flûtes bocagères\separvers{}Réveillez les plus beaux sons ;\separvers{}LOUIS offre à vos chansons\separvers{}La plus belle des matières.\separvers{}Après cent combats,\separvers{}Où cueille son bras\separvers{}Une ample victoire :\separvers{}Formez entre vous\separvers{}Cent combats plus doux,\separvers{}Pour chanter sa gloire.}%
\rpttous{Formons entre nous Cent combats plus doux, Pour chanter sa gloire.}%
\rptflore{Mon jeune amant dans ce bois,\separvers{}Des présents de mon empire\separvers{}Prépare un prix à la voix,\separvers{}Qui saura le mieux nous dire\separvers{}Les vertus et les exploits\separvers{}Du plus auguste des rois.}%
\rptclimene{Si Tircis a l'avantage,}%
\rptdaphne{Si Dorilas est vainqueur,}%
\rptclimene{À le chérir je m'engage.}%
\rptdaphne{Je me donne à son ardeur.}%
\rpttircis{Ô trop chère espérance !}%
\rptdorilas{Ô mot plein de douceur !}%
\rpttirdor{Plus beau sujet, plus belle récompense\separvers{}Peuvent-ils animer un cœur ?}%
%
\TheatreComment{Les violons jouent un air pour animer les deux Bergers au combat, tandis que Flore comme juge va se placer au pied d'un bel arbre, qui est au milieu du théâtre, avec deux Zéphirs, et que le reste comme spectateurs va occuper les deux côtés de la scène.}%
%
\rpttirdor{Plus beau sujet, plus belle récompense\separvers{}Peuvent-ils animer un cœur ?}%
%
\TheatreComment{Les violons jouent un air pour animer les deux Bergers au combat, tandis que Flore comme juge va se placer au pied d'un bel arbre, qui est au milieu du théâtre, avec deux Zéphirs, et que le reste comme spectateurs va occuper les deux côtés de la scène.}%
%
\rpttircis{Quand la neige fondue enfle un torrent fameux,\separvers{}Contre l'effort soudain de ses flots écumeux\separvers{}Il n'est rien d'assez solide ;\separvers{}Digues, châteaux, villes, et bois,\separvers{}Hommes, et troupeaux à la fois,\separvers{}Tout cède au courant qui le guide,\separvers{}Tel, et plus fier, et plus rapide,\separvers{}Marche LOUIS dans ses exploits.}%
%
\TheatreMvt[ballet]{Les Bergers et Bergères du côté de Tircis, dansent autour de lui sur une ritournelle, pour exprimer leurs applaudissements.}%
%
\rptdorilas{Le foudre menaçant qui perce avec fureur\separvers{}L'affreuse obscurité de la nue enflammée,\separvers{}Fait d'épouvante et d'horreur\separvers{}Trembler le plus ferme cœur :\separvers{}Mais à la tête d'une armée\separvers{}LOUIS jette plus de terreur.}%
%
\TheatreMvt[ballet]{Les Bergers et Bergères du côté de Dorilas font de même que les autres.}%
%
\rpttircis{Des fabuleux exploits que la Grèce a chantés,\separvers{}Par un brillant amas de belles vérités\separvers{}Nous voyons la gloire effacée,\separvers{}Et tous ces fameux demi-dieux,\separvers{}Que vante l'histoire passée\separvers{}Ne sont point à notre pensée,\separvers{}Ce que LOUIS est à nos yeux.}%
%
\TheatreMvt[ballet]{Les Bergers et Bergères de son côté, font encore la même chose.}%
%
\rptdorilas{LOUIS fait à nos temps, par ses faits inouïs\separvers{}Croire tous les beaux faits que nous chante l'histoire\separvers{}Des siècles évanouis :\separvers{}Mais nos neveux dans leur gloire,\separvers{}N'auront rien qui fasse croire\separvers{}Tous les beaux faits de LOUIS.}%
%
\TheatreMvt[ballet]{Les Bergers et Bergères de son côté font encore de même, après quoi les deux partis se mêlent.}%
%
\rptpan[\emph{suivi de six Faunes}]{Laissez, laissez, Bergers, ce dessein téméraire,\separvers{}Hé, que voulez-vous faire ?\separvers{}Chanter sur vos chalumeaux,\separvers{}Ce qu'Apollon sur sa lyre\separvers{}Avec ses chants les plus beaux,\separvers{}N'entreprendrait pas de dire ?\separvers{}C'est donner trop d'essor au feu qui vous inspire,\separvers{}C'est monter vers les cieux sur des ailes de cire,\separvers{}Pour tomber dans le fond des eaux.\separvers{}Pour chanter de LOUIS l'intrépide courage ;\separvers{}Il n'est point d'assez docte voix,\separvers{}Point de mots assez grands pour en tracer l'image ;\separvers{}Le silence est le langage\separvers{}Qui doit louer ses exploits.\separvers{}Consacrez d'autres soins à sa pleine victoire,\separvers{}Vos louanges n'ont rien qui flatte ses désirs,\separvers{}Laissez, laissez là sa gloire\separvers{}Ne songez qu'à ses plaisirs.}%
\rpttous{Laissons, laissons là sa gloire\separvers{}Ne songeons qu'à ses plaisirs.}%
\rptflore{Bien que pour étaler ses vertus immortelles\separvers{}La force manque à vos esprits,\separvers{}Ne laissez pas tous deux de recevoir le prix.\separvers{}Dans les choses grandes et belles\separvers{}Il suffit d'avoir entrepris.}%
%
\TheatreMvt[entrée de ballet]{Les deux Zéphirs dansent avec deux couronnes de fleurs à la main, qu'ils viennent donner ensuite aux deux Bergers.}%
%
\rptclimdaph[\emph{en leur donnant la main}]{Dans les choses grandes et belles\separvers{}Il suffit d'avoir entrepris.}%
\rpttirdor{Ha ! que d'un doux succès notre audace est suivie.}%
\rptflorpan{Ce qu'on fait pour LOUIS, on ne le perd jamais.}%
\rptquatre{Au soin de ses plaisirs donnons-nous désormais.}%
\rptflorpan{Heureux, heureux qui peut lui consacrer sa vie.}%
\rpttous{Joignons tous dans ces bois\separvers{}Nos flûtes et nos voix,\separvers{}Ce jour nous y convie,\separvers{}Et faisons aux échos redire mille fois,\separvers{}\og{}LOUIS est le plus grand des rois.\separvers{}Heureux, heureux, qui peut lui consacrer sa vie !\fg{}}%
%
\TheatreMvt[dernière et grande entrée de ballet]{Faunes, Bergers et Bergères, tous se mêlent, et il se fait entre eux des jeux de danse, après quoi ils se vont préparer pour la Comédie}%
\end{theatre}%
%
\clearpage{}
% % % % % % % % % % % % % % % % % % % % % % % % % % % % % % % % % % % % % % % % % % % % % % % % % % % % % % %
% % % % % % % % % % % % % % % % % % % % % % % % % % % % % % % % % % % % % % % % % % % % % % % % % % % % % % %
% % %                                                                                                   % % %
\section{MacBeth}%                                                                           % % %
\subsection{Acte 1 - Scene 2 Acamp near Forres.}%                                                                              % % %
\TheatreIniCpteReplique{1}% Choisir le premier numéro des répliques                                      % % %
% % %                                                                                                   % % %
% % % % % % % % % % % % % % % % % % % % % % % % % % % % % % % % % % % % % % % % % % % % % % % % % % % % % % %
% % % % % % % % % % % % % % % % % % % % % % % % % % % % % % % % % % % % % % % % % % % % % % % % % % % % % % %
\TheatreComment{Alarum within. Enter DUNCAN, MALCOLM, DONALBAIN, LENNOX, with Attendants, meeting a bleeding Sergeant}%

\bigskip{}

%\renewcommand{\separvers}{\newline}%
\def\separvers{\newline}%

\begin{theatre}
\rgdducan{What bloody man is that? He can report,\separvers{}As seemeth by his plight, of the revolt\separvers{}The newest state.}
\rptmalcolm{This is the sergeant\separvers{}Who like a good and hardy soldier fought\separvers{}'Gainst my captivity. Hail, brave friend!\separvers{}Say to the king the knowledge of the broil\separvers{}As thou didst leave it.}
\rptsergeant{Doubtful it stood;\separvers{}As two spent swimmers, that do cling together\separvers{}And choke their art. The merciless Macdonwald--\separvers{}Worthy to be a rebel, for to that\separvers{}The multiplying villanies of nature\separvers{}Do swarm upon him--from the western isles\separvers{}Of kerns and gallowglasses is supplied;\separvers{}And fortune, on his damned quarrel smiling,\separvers{}Show'd like a rebel's whore: but all's too weak:\separvers{}For brave Macbeth--well he deserves that name--\separvers{}Disdaining fortune, with his brandish'd steel,\separvers{}Which smoked with bloody execution,\separvers{}Like valour's minion carved out his passage\separvers{}Till he faced the slave;\separvers{}Which ne'er shook hands, nor bade farewell to him,\separvers{}Till he unseam'd him from the nave to the chaps,\separvers{}And fix'd his head upon our battlements.}
\rgdducan{O valiant cousin! worthy gentleman!}
\rptsergeant{As whence the sun 'gins his reflection\separvers{}Shipwrecking storms and direful thunders break,\separvers{}So from that spring whence comfort seem'd to come\separvers{}Discomfort swells. Mark, king of Scotland, mark:\separvers{}No sooner justice had with valour arm'd\separvers{}Compell'd these skipping kerns to trust their heels,\separvers{}But the Norweyan lord surveying vantage,\separvers{}With furbish'd arms and new supplies of men\separvers{}Began a fresh assault.}
\rgdducan{Dismay'd not this\separvers{}Our captains, Macbeth and Banquo?}
\rptsergeant{Yes;\separvers{}As sparrows eagles, or the hare the lion.\separvers{}If I say sooth, I must report they were\separvers{}As cannons overcharged with double cracks, so they\separvers{}Doubly redoubled strokes upon the foe:\separvers{}Except they meant to bathe in reeking wounds,\separvers{}Or memorise another Golgotha,\separvers{}I cannot tell.\separvers{}But I am faint, my gashes cry for help.}
\rgdducan{So well thy words become thee as thy wounds;\separvers{}They smack of honour both. Go get him surgeons.}
\TheatreMvt{Exit Sergeant, attended}
\rgdducan{Who comes here?}
\TheatreMvt{Enter ROSS}
\rptmalcolm{The worthy thane of Ross.}
\rptlennox{What a haste looks through his eyes! So should he look\separvers{}That seems to speak things strange.}
\rptross{God save the king!}
\rgdducan{Whence camest thou, worthy thane?}
\rptross{From Fife, great king;\separvers{}Where the Norweyan banners flout the sky\separvers{}And fan our people cold. Norway himself,\separvers{}With terrible numbers,\separvers{}Assisted by that most disloyal traitor\separvers{}The thane of Cawdor, began a dismal conflict;\separvers{}Till that Bellona's bridegroom, lapp'd in proof,\separvers{}Confronted him with self-comparisons,\separvers{}Point against point rebellious, arm 'gainst arm.\separvers{}Curbing his lavish spirit: and, to conclude,\separvers{}The victory fell on us.}
\rgdducan{Great happiness!}
\rptross{That now\separvers{}Sweno, the Norways' king, craves composition:\separvers{}Nor would we deign him burial of his men\separvers{}Till he disbursed at Saint Colme's inch\separvers{}Ten thousand dollars to our general use.}
\rgdducan{No more that thane of Cawdor shall deceive\separvers{}Our bosom interest: go pronounce his present death,\separvers{}And with his former title greet Macbeth.}
\rptross{I'll see it done.}
\rgdducan{What he hath lost noble Macbeth hath won.}
\TheatreComment{Exeunt}
\end{theatre}%
%
%
\backmatter{}%
% - - - - - - - - - - - - - - - - - - - - - - - - - - - - - - - - - - - - - - - - - - - - - - - - - - - - - -
% - -    Index                                                                                          - - -
% - - - - - - - - - - - - - - - - - - - - - - - - - - - - - - - - - - - - - - - - - - - - - - - - - - - - - -
\printindex{}%                    pour imprimer la table des index insérés
%\chapter{Index}
%\addcontentsline{toc}{\chiniveauins}{Index}
% - - - - - - - - - - - - - - - - - - - - - - - - - - - - - - - - - - - - - - - - - - - - - - - - - - - - - -
\cleardoublepage{}
\loadchihelpgeometry{helplisting}%
\chapter{Code source}
\lstinputlisting{theatre.sty}
\end{document}

%!TEX encoding = UTF-8 Unicode
%!TEX TS-program = lualatex
% !BIB TS-program = biber
% biblatex-fiwi,v 1.7 2017/11/21

\documentclass{ltxdockit}
\usepackage{luatex85}
\usepackage{fontspec}
\usepackage{hyperref}
\usepackage{zref-xr}

\setmainfont[Extension=.otf,UprightFont= CharterITCStd-Regular,
Path = /Users/simi/FontExplorerX/C/Charter ITC Std/,
BoldFont = *-Bold,
ItalicFont = *-Italic,
BoldItalicFont = *-BoldItalic,
             Ligatures = TeX] 
            {Charter ITC Std}
\setsansfont[Extension=.ttf,UprightFont=nimbus-sans-l_regular,BoldFont=nimbus-sans-l_bold,Path=/Users/simi/FontExplorerX/N/Nimbus Sans L/]{Nimbus Sans L}	
\setmonofont[]{Courier New}
\newfontfamily\cyrillicfont[Extension=.ttf,
	UprightFont= LinLibertine_Re-4.4.1,
	Path=/Users/simi/FontExplorerX/L/Linux Libertine/]
	{Linux Libertine}
%\usepackage[base]{babel}
\usepackage{polyglossia}
\setdefaultlanguage[latesthyphen,spelling=new,babelshorthands=true]{german}
\setotherlanguage{russian}
\usepackage[style=fiwi,series=true,publisher=true,fullcitefilm,translatedas=true,partofcited=true]{biblatex}
\newbibmacro*{altscript:cyrillic-font}[1]{%
{\cyrillicfont{\textsc{#1}}}}
\ExecuteBibliographyOptions{bibencoding=utf8}
\ExecuteBibliographyOptions{backref=false}
\usepackage[strict,style=german,german=guillemets]{csquotes}
\usepackage{btxdockit}
\usepackage{tabularx}
\usepackage{longtable}
\usepackage{booktabs}
\usepackage{graphicx}
\usepackage{shortvrb}
\usepackage{needspace}
\usepackage{pifont}
%\usepackage{dtklogos}
\usepackage{microtype}
\usepackage{typearea}
\usepackage{mdframed}
\areaset[current]{370pt}{700pt}
\lstset{
    basicstyle=\ttfamily,
    keepspaces=true,
    upquote=true,
    frame=single,
    breaklines=true,
    postbreak=\raisebox{0ex}[0ex][0ex]{\ensuremath{\color{red}\hookrightarrow\space}}
}
\KOMAoptions{numbers=noenddot}
\addtokomafont{title}{\sffamily}
\addtokomafont{paragraph}{\spotcolor}
\addtokomafont{section}{\spotcolor}
\addtokomafont{subsection}{\spotcolor}
\addtokomafont{subsubsection}{\spotcolor}
\addtokomafont{descriptionlabel}{\spotcolor}
\setkomafont{caption}{\bfseries\sffamily\spotcolor}
\setkomafont{captionlabel}{\bfseries\sffamily\spotcolor}
\pretocmd{\cmd}{\sloppy}{}{}
\pretocmd{\bibfield}{\sloppy}{}{}
\pretocmd{\bibtype}{\sloppy}{}{}
\MakeAutoQuote{»}{«}
\MakeAutoQuote*{<}{>}
\MakeShortVerb{\|}
\addbibresource{examples.bib}

\deffootnote{2em}{2em}{\makebox[2em][l]{\thefootnotemark}}

\newcommand*{\biber}{\sty{biber}\xspace}
\newcommand*{\biblatex}{\sty{biblatex}\xspace}
\newcommand*{\biblatexml}{\sty{biblatexml}\xspace}

\titlepage{%
  title={Der \sty{biblatex-fiwi}-Stil},
  subtitle={Ein \sty{biblatex}-Zitierstil für Filmwissenschaftler},
  url={http://www.ctan.org/pkg/biblatex-fiwi},
  author={Simon Spiegel},
  email={simon@simifilm.ch},
  revision={1.7},
  date={\rcstoday}}

\hypersetup{%
  pdftitle={Das biblatex-fiwi-Paket},
  pdfsubject={Ein biblatex-Zitierstile für Filmwissenschaftler},
  pdfauthor={Simon Spiegel},
  citecolor=black,
  unicode,
  pdfkeywords={tex, e-tex, latex, biblatex, film studies, humanities}}

\newcolumntype{H}{>{\sffamily\bfseries\spotcolor}l}
\newcolumntype{L}{>{\raggedright\let\\=\tabularnewline}p}
\newcolumntype{R}{>{\raggedleft\let\\=\tabularnewline}p}
\newcolumntype{C}{>{\centering\let\\=\tabularnewline}p}
\newcolumntype{V}{>{\raggedright\let\\=\tabularnewline\ttfamily}p}

\newcommand*{\sorttablesetup}{%
  \tablesetup
  \ttfamily
  \def\new{\makebox[1.25em][r]{\ensuremath\rightarrow}\,}%
  \def\alt{\par\makebox[1.25em][r]{\ensuremath\hookrightarrow}\,}%
  \def\note##1{\textrm{##1}}}

\newcommand{\tickmarkyes}{\Pisymbol{psy}{183}}
\newcommand{\tickmarkno}{\textendash}
\providecommand*{\textln}[1]{#1}
\providecommand*{\lnstyle}{}

% markup and misc

\setcounter{secnumdepth}{4}

\makeatletter


\newenvironment{namedelims}
  {\trivlist\item
   \tabularx{\textwidth}{@{}c@{=}l>{\raggedright\let\\=\tabularnewline}X@{}}}
  {\endtabularx\endtrivlist}

\newenvironment{namesample}
  {\def\delim##1##2{\@delim{##1}{\normalfont\tiny\bfseries##2}}%
   \def\@delim##1##2{{%
     \setbox\@tempboxa\hbox{##1}%
     \@tempdima=\wd\@tempboxa
     \wd\@tempboxa=\z@
     \box\@tempboxa
     \begingroup\spotcolor
     \setbox\@tempboxa\hb@xt@\@tempdima{\hss##2\hss}%
     \vrule\lower1.25ex\box\@tempboxa
     \endgroup}}%
   \ttfamily\trivlist
   \setlength\itemsep{0.5\baselineskip}}
  {\endtrivlist}

\makeatother

\newrobustcmd*{\Deprecated}{%
  \textcolor{spot}{\margnotefont Deprecated}}
\newrobustcmd*{\DeprecatedMark}{%
  \leavevmode\marginpar{\Deprecated}}
\newrobustcmd*{\LF}{%
  \textcolor{spot}{\margnotefont Label field}}
\newrobustcmd*{\LFMark}{%
  \leavevmode\marginpar{\LF}}
\newrobustcmd*{\CSdelim}{%
  \textcolor{spot}{\margnotefont Context Sensitive}}
\newrobustcmd*{\CSdelimMark}{%
  \leavevmode\marginpar{\CSdelim}}


% following snippet is based on code by Michael Ummels (TeX Stack Exchange)
% <http://tex.stackexchange.com/a/13073/8305>
\makeatletter
  \newcommand\fnurl@[1]{\footnote{\url@{#1}}}
  \DeclareRobustCommand{\fnurl}{\hyper@normalise\fnurl@}
\makeatother

% hyphenation

\hyphenation{%
 }

\begin{document}

\printtitlepage
\tableofcontents

\section{Einleitung}
\label{int}
\subsection[About]{Über \sty{biblatex-fiwi}}

Ich habe den \sty{biblatex-fiwi}"=Stil ursprünglich für meine Dissertation in Filmwissenschaft geschrieben und seither  kontinuierlich weiterentwickelt. Mittlerweile haben diverse Studentinnen und Doktoranden den Stil erfolgreich eingesetzt, ich selbst habe \sty{biblatex-fiwi} für zwei Bücher\footnote{\fullcite{Spiegel.S:2007b,Spiegel.S:2010b}.} benutzt -- der Stil ist also definitiv praxiserprobt.

Es handelt sich bei \sty{biblatex-fiwi} nicht um die Implementierung eines definierten Standards, sondern um einen Zitierstil, der für verschiedene geisteswissenschaftliche Disziplinen geeignet sein sollte. Dabei wird dem Zitieren von Filmen und Fernsehserien besonderes Augenmerk geschenkt. Zudem liegt ein Fokus auf vertrackten Fällen, z.\,B. Artikel in Sammelbänden innerhalb eines mehrbändigen Werks mit vielen Mitwirkenden.

Bug"=Reports oder Fragen und Anregungen zur Weiterentwicklung des Stils sind stets willkommen. Falls sich jemand, der den Stil erfolgreich einsetzt, erkenntlich zeigen möchte, sei hier auf meine Amazon"=Wunschliste\fnurl{http://www.amazon.de/registry/wishlist/2HW0Y8SWX4QDU} verwiesen.

\subsection{License}

Copyright © 2003--2017 Simon Spiegel. Permission is granted to copy, distribute and\slash or modify this software under the terms of the \lppl, version 1.3c.\fnurl{http://www.ctan.org/tex-archive/macros/latex/base/lppl.txt} This package is author"=maintained.

\subsection{Voraussetzungen}

\sty{biblatex-fiwi} benötigt \sty{biblatex} ab Version 3.5. Die Verwendung des  \bibtex-Ersatzes \bin{biber}\fnurl{http://biblatex-biber.sourceforge.net/} in der Version 2.6 oder höher wird dringend empfohlen. Zwar sollten diverse Funktionen auch mit \bibtex funktionieren, ich selber verwende aber seit längerem ausschließlich \bin{biber} und kann deshalb für nichts garantieren. Da sowohl \bin{biber} als auch \sty{biblatex} Bestandteil der \TeX{}"=Distributionen  \TeX{}~Live\fnurl{http://www.tug.org/texlive} und MiK\TeX{}\fnurl{http://www.miktex.org} sind, sollten beide den meisten Benutzern ohnehin zur Verfügung stehen.

\subsection{Zu dieser Anleitung}
Diese Anleitung behandelt nur die spezifischen Anpassungen und Eigenheiten von \sty{biblatex-fiwi} und ist nicht als allgemeine Einführung in \sty{biblatex} gedacht. Für Informationen zu \sty{biblatex} empfehle ich das exzellente \sty{biblatex}"=Manual sowie die deutschsprachige Einführung von Dominik Waßenhoven\fnurl{http://biblatex.dominik-wassenhoven.de/dtk.shtml}.
\section{Benutzung}\label{benutzung}

\sty{biblatex-fiwi} wird wie alle anderen \sty{biblatex}"=Stile aufgerufen:

\begin{ltxexample}
\usepackage[style=fiwi]{biblatex}
\end{ltxexample}
%
oder 
%
\begin{ltxexample}
\usepackage[style=fiwi2]{biblatex}
\end{ltxexample}
%
Der Stil \sty{fiwi2} entspricht der Option \opt{yearatbeginning} (\secref{opt}); hier wird das Erscheinungsjahr unmittelbar nach dem Autor/Herausgeber in Klammern ausgegeben.

\subsection{Paketoptionen}\label{opt}

\sty{biblatex-fiwi} unterstützt alle gängigen Lade"=Optionen von \sty{biblatex}, bietet zusätzlich aber noch eigene an. Auch diese werden wie gewohnt aktiviert:
\begin{ltxexample}
\usepackage[bibstyle=fiwi,citestyle=fiwi,dashed=true]{biblatex}
\end{ltxexample}

\begin{optionlist}

\optitem[false]{actor}{\opt{true}, \opt{false}}
Aktiviert die Ausgabe der Schauspieler im Feld \bibfield{actor} bei Filmen (\secref{sub:filme}).

\optitem[normal]{citefilm}{\opt{normal}, \opt{country}, \opt{full}, \opt{complete}}

Mit dieser Option wird die Ausgabe von Filmen bei Zitaten im Text definiert. Dies betrifft nur das Aussehen bei der Ernstnennung respektive den Befehl \cmd{citefullfilm} (\secref{zitbefehl}). Die Kurzform \opt{citefilm} entspricht \kvopt{citefilm}{normal}.

\begin{valuelist}
\item[normal] Im Normalfall wird der volle Filmtitel und das Jahr der Veröffentlichung in Klammern ausgegeben; zB. \citefilm{Kubrick.S:1968}.
\item[country] Gibt den Filmtitel sowie -- in Klammern -- das Produktionsland und das Jahr der Veröffentlichung aus; zB. \toggletrue{cbx:citefilmcountry} \citefullfilm{Kubrick.S:1968}.\togglefalse{cbx:citefilmcountry}
\item[full] Gibt den Filmtitel sowie -- in Klammern -- den Regisseur, das Produktionsland und das Jahr der Veröffentlichung aus; zB. \toggletrue{citefullfilm}\citefullfilm{Kubrick.S:1968}.\togglefalse{citefullfilm}
\item[complete] Gibt den Filmtitel sowie  -- in Klammern --  den deutschen Verleihtitel, Produktionsland und das Jahr der Veröffentlichung aus; zB. \toggletrue{citecompletefilm}\citefullfilm{Kubrick.S:1968}.\togglefalse{citecompletefilm}
\end{valuelist}

\optitem[false]{titleindex}{\opt{true}, \opt{subitem}, \opt{both}}

Mit dieser Option wird beim Zitieren mit \cmd{citetitle} der Titel des Werks indexiert. Diese Option setzt voraus, dass \opt{indexing} auf \opt{cite} gesetzt ist, was bei \sty{biblatex-fiwi} aber ohnehin standardmäßig der Fall ist.

\begin{valuelist}
\item[true] Im Normalfall wird, falls vorhanden, \bibfield{indextitle} indexiert, andernfalls \bibfield{title}.
\item[subitem] In dieser Variante wird der Werktitel im Index unter dem Titel des Autors geführt; dies entspricht einem Index-Eintrag in der Form von |\index{index{Autor!Werk}}|. Hat ein Werk mehr als einen Autoren, wird der Titel unter jedem Namen geführt.
\item[both] In dieser Variante wird wie bei \opt{subitem} der Werktitel im Index unter dem Titel des Autors geführt; zusätzlich wird ein Index-Eintrag unter dem Titel des Werkes erstellt, der auf den Autoren verweist. Dies entspricht zwei Index-Einträgen in der Form von |\index{Autor!Werk}| und |\index{Werk!see{Autor}}|. Hat ein Werk mehr als einen Autoren, wird der Titel unter jedem Namen geführt; der Werktitel verweist aber nur auf den ersten Autor.
\end{valuelist}

\optitem[false]{filmindex}{\opt{true}, \opt{complete}}

Mit dieser Option wird beim Zitieren von Filmen mit \cmd{filmcite} der Titel der Filmtitel indexiert. Diese Option setzt voraus, dass \opt{indexing} auf \opt{cite} gesetzt ist, was bei \sty{biblatex-fiwi} aber ohnehin standardmäßig der Fall ist.

\begin{valuelist}
\item[true] Im Normalfall wird, falls vorhanden, \bibfield{indextitle} indexiert, andernfalls \bibfield{title}; werden zwei Filme mit identischem Titel indexiert, wird zur Unterscheidung zusätzlich das Jahr in Klammern ausgegeben.
\item[complete] Ist \opt{filmindex} auf \opt{complete} gesetzt, werden die komplette filmographische Angaben ausgegeben, wie sie mit \cmd{printbibliography} erscheinen würden. Damit kann auf eine eigenständige Filmographie verzichtet werden.
\end{valuelist}

Siehe zum Indexieren von Filmen auch \secref{sub:indexfilm}.

\optitem[false]{splitfilmindex}{\opt{true}, \opt{false}}

Mit Paketen wie \sty{imakeidx}\fnurl{http://ctan.org/pkg/imakeidx} resp. \sty{splitidx} (ist Teil von koma-script\fnurl{http://ctan.org/pkg/koma-script}) ist es möglich, mehrere Indizes in einem Dokument zu setzen, z. B. einen für Literatur und einen für Filme. \sty{biblatex-fiwi} bietet die Möglichkeit, Filmtitel automatisch in ein separates Index-File zu schreiben, so dass am Ende ein eigener Filmindex erzeugt werden. Ist diese Option aktiviert, werden Filmtitel mit dem Befehl \cmd{index[film]} zitiert und in ein entsprechendes .idx-File geschrieben. Um diese Funktion zu nutzen, muss \opt{filmindex} aktiviert und eines der genannten Pakete geladen werden; zudem müssen die passenden Indizes mit \cmd{makeindex} erzeugt werden. Der entsprechende Code sieht so aus:

\begin{ltxexample}
...
\ExecuteBibliographyOptions{filmindex=true}
\ExecuteBibliographyOptions{splitfilmindex=true}...
\usepackage{imakeidx}
\makeindex[...]
\makeindex[name=film, ...]
...
\begin{document}
...
\printindex
\printindex[film]
\end{document}
\end{ltxexample}

Siehe zum Indexieren von Filmen auch \secref{sub:indexfilm}.

\boolitem[true]{citeprefix}

Das korrekte Schreiben und Sortieren von Nachnamen mit Zusätzen wie <von> oder <van> ist eine Wissenschaft für sich. Im Deutschen gilt grundsätzlich folgende Regel: Ein Name wie »Ursula von Keitz« muss in der Bibliographie als »Keitz, Ursula von« erscheinen. Dies gilt aber nicht für Namen in anderen Sprachen; im Prinzip müssen hier jeweils die Regeln der Sprache des Herkunftslandes befolgt werden. Und streng genommen gibt es auch einen Unterschied zwischen der Deutschen »Ursula von Keitz« und dem Schweizer »Peter von Matt«. Im ersten Fall handelt es sich ursprünglich um einen Adelstitel, im zweiten Fall um einen Flurnamen, der eigentlich als »von Matt, Peter« in der Bibliographie stehen müsste.

\sty{biblatex-fiwi} setzt die \opt{citeprefix}-Option standardmäßig auf true. Steht unter \bibfield{name} einfach \opt{von Keitz, Ursula}, dann erfolgt die Ausgabe somit als »von Keitz, Ursula« und wird  unter »von« einsortiert. Möchte man diese Ausgabe aber eine Einsortierung unter »Keitz«, dann müsste ein entsprechender Eintrag in das \bibfield{sortkey}"=Feld.

Setzt man dagegen in das Feld \bibfield{options} den Eintrag \bibfield{useprefix=false}, dann lautet die Ausgabe »Keitz, Ursula von« und wird unter »Keitz« einsortiert.

Unabhängig von diesem Feld lautet die Ausgabe im Text in jedem Fall standardmäßig »von Keitz« resp. »von Matt«. Für Zitate im Text wird \option{useprefix=false} also normalerweise ignoriert. Mit der Option \opt{citeprefix} kann dies geändert werden, dann  steht auch beim Zitat  im Text nur noch »Matt« bzw. »Keitz«. Die Kurzform \opt{citeprefix} entspricht \kvopt{citeprefix}{true}.

\boolitem[false]{dashed}

Bei aufeinander folgenden identischen Autoren wird in der Bibliographie statt des Namens ein Spiegelstrich (--) gesetzt. Die Kurzform \opt{dashed} entspricht \kvopt{dashed}{true}.

\boolitem[false]{filmruntime}

Bestimmt, ob in der Bibliographie bei Filmen die in \bibfield{duration} resp. \bibfield{pagetotal} angegebene Laufzeit angegeben wird (vgl. \secref{sub:filme}). Die Kurzform \opt{filmruntime} entspricht \kvopt{filmruntime}{true}.

\boolitem[false]{ibidpage}

Steuert wie »ebd.« bei aufeinander folgenden Verweisen gehandhabt wird. Ist \cmd{ibidpage} auf \opt{true} gesetzt, wird bei identischem Autor, Titel und Seitenzahl nur »ebd.«, ausgegeben. Bei \opt{false} wird in jedem Fall die Seitenzahl ausgegeben.

\optitem[false]{ignoreforeword}{\opt{true}, \opt{false}, \opt{cite}, \opt{bib}, \opt{both}}

Bestimmt, ob der Inhalt von \bibfield{foreword} ausgegeben wird oder nicht.

\begin{valuelist}
\item[true] Die Ausgabe von \bibfield{foreword} wird sowohl bei \cmd{fullcite} im Lauftext als auch in der Bibliographie unterdrückt.
\item[both] Ist mit \opt{true} identisch.
\item[false] \bibfield{foreword} wird überall ausgegeben.
\item[cite] Die Ausgabe von \bibfield{foreword} bei \cmd{fullcite} wird unterdrückt.
\item[bib] Die Ausgabe von \bibfield{foreword} in der Bibliographie wird unterdrückt.
\end{valuelist}
Die Kurzform \opt{ignoreforeword} entspricht \kvopt{ignoreforeword}{true}

Diese Option kann innerhalb eines Dokuments mit den Befehlen \cmd{ignoreforewordtrue} und \cmd{ignoreforewordfalse} gesteuert werden.

\optitem[false]{ignoreafterword}{\opt{true}, \opt{false}, \opt{cite}, \opt{bib}, \opt{both}}

Bestimmt, ob der Inhalt von \bibfield{afterword} ausgegeben wird oder nicht.

\begin{valuelist}
\item[true] Die Ausgabe von \bibfield{afterword} wird sowohl bei \cmd{fullcite} im Lauftext als auch in der Bibliographie unterdrückt.
\item[both] Ist mit \opt{true} identisch.
\item[false] \bibfield{afterword} wird überall ausgegeben.
\item[cite] Die Ausgabe von \bibfield{afterword} bei \cmd{fullcite} wird unterdrückt.
\item[bib] Die Ausgabe von \bibfield{afterword} in der Bibliographie wird unterdrückt.
\end{valuelist}
Die Kurzform \opt{ignoreafterword} entspricht \kvopt{ignoreafterword}{true}

Diese Option kann innerhalb eines Dokuments mit den Befehlen \cmd{ignoreafterwordtrue} und \cmd{ignoreafterwordfalse} gesteuert werden.

\optitem[false]{ignoreintroduction}{\opt{true}, \opt{false}, \opt{cite}, \opt{bib}, \opt{both}}

Bestimmt, ob der Inhalt von \bibfield{introduction} ausgegeben wird oder nicht.

\begin{valuelist}
\item[true] Die Ausgabe von \bibfield{introduction} wird sowohl bei \cmd{fullcite} im Lauftext als auch in der Bibliographie unterdrückt.
\item[both] Ist mit \opt{true} identisch.
\item[false] \bibfield{introduction} wird überall ausgegeben.
\item[cite] Die Ausgabe von \bibfield{introduction} bei \cmd{fullcite} wird unterdrückt.
\item[bib] Die Ausgabe von \bibfield{introduction} in der Bibliographie wird unterdrückt.
\end{valuelist}
Die Kurzform \opt{ignoreintroduction} entspricht \kvopt{ignoreintroduction}{true}

Diese Option kann innerhalb eines Dokuments mit den Befehlen \cmd{ignoreintroductiontrue} und \cmd{ignoreintroductionfalse} gesteuert werden.

\optitem[false]{ignoreaddendum}{\opt{true}, \opt{false}, \opt{cite}, \opt{bib}, \opt{both}}

Bestimmt, ob der Inhalt von \bibfield{addendum} ausgegeben wird oder nicht.

\begin{valuelist}
\item[true] Die Ausgabe von \bibfield{addendum} wird sowohl bei \cmd{fullcite} im Lauftext als auch in der Bibliographie unterdrückt.
\item[both] Ist mit \opt{true} identisch.
\item[false] \bibfield{addendum} wird überall ausgegeben.
\item[cite] Die Ausgabe von \bibfield{addendum} bei \cmd{fullcite} wird unterdrückt.
\item[bib] Die Ausgabe von \bibfield{addendum} in der Bibliographie wird unterdrückt.
\end{valuelist}
Die Kurzform \opt{ignoreaddendum} entspricht \kvopt{ignoreaddendum}{true}

Diese Option kann innerhalb eines Dokuments mit den Befehlen \cmd{ignoreaddendumtrue} und \cmd{ignoreaddendumfalse} gesteuert werden.

\optitem[false]{ignoreparatext}{\opt{true}, \opt{false}, \opt{cite}, \opt{bib}, \opt{both}}
Kombination der vorhergehenden Befehle, der bestimmt, ob der Inhalt von \bibfield{foreword}, \bibfield{afterword}, \bibfield{introduction} und \bibfield{addendum} ausgegeben wird oder nicht.  Die Kurzform \opt{ignoreparatext} entspricht \kvopt{ignoreparatext}{true}

\boolitem[true]{ignorearticle}

\bin{biber} bietet die Möglichkeit, Teile eines Felds beim Sortieren zu ignorieren. Mit \opt{ignorearticle} werden die gängigen Artikel bei den Feldern \bibfield{title} und \bibfield{maintitle} ignoriert (\secref{sub:sort}). Die Kurzform \opt{ignorearticle} entspricht \kvopt{ignorearticle}{true}.

\boolitem[false]{isbn}

Aktiviert die Ausgabe der Felder \bibfield{isbn/issn/isrn/isan}.

\boolitem[false]{compactcite}
Wenn mit einem \cmd{cite}- resp. \cmd{parencite}-Befehl zwei Werke des gleichen Autors zitiert werden, so geschieht das normalerweise in Form von »Autor Jahr, Autor Jahr«. Ist diese Option aktiviert, werden die beiden Verweise als »Autor Jahr und Jahr« zusammengezogen, z.\,B. »Müller 1997 und 1998«. Die Funktion hat allerdings einen Haken: Standardmäßig ist <und> als Verbindung zwischen den beiden Jahreszahlen definiert. Da \sty{biblatex} nicht unterscheiden kann, ob mehr als zwei Quellen des gleichen Autors zitiert werden, würde ein Beispiel mit drei Quellen des gleichen Autors so aussehen: »Müller 1997 und 1998 und 1999«. Das verbindende Partikel, normalerweise <und>, wird in \cmd{compcitedelim} definiert.

\optitem[true]{mergedate}{\opt{maximum}, \opt{compact}, \opt{basic}, \opt{minimum}, \opt{true (=basic)}, \opt{false}}

Bestimmt, wie kompakt die Datumsangabe in der Bibliographie ausfällt. Dies betrifft den Stil \sty{fiwi2} resp. die Option \opt{yearatbeginning}. Die Darstellungsweise weicht etwas von der \opt{mergedate}-Option des Standard-Stils \opt{authoryear} ab.

\begin{valuelist}
\item[false] Es wird strikt zwischen dem Zitationsjahr zu Beginn und der Datumsangabe am Ende unterschieden.

Müller, Hans (2000): \emph{Buch}. Location 2000.\\
Müller, Hans (2003a): \emph{Buch 2}. Location 2003.\\
Müller, Hans (2003b): \emph{Buch 3}. Location: 2003.\\
Müller, Hans (2006): »Artikel 1«. In: \emph{Monatliche Zeitschrift} Jg. 28, Nr. 5, Juni 2006, 70--85.
Müller, Hans (2007): »Artikel 2«. In: \emph{ Zeitschrift} Jg. 28, Nr. 5, Herbst 2007, 70--85.

\item[minimum] Wenn das Zitationsjahr und die Datumsangabe identisch sind, werden sie zusammengezogen. Sobald sie sich unterscheiden, werden sie beide ausgegeben.

Müller, Hans (2000): \emph{Buch}. Location.\\
Müller, Hans (2003a): \emph{Buch 2}. Location 2003.\\
Müller, Hans (2003b): \emph{Buch 3}. Location 2003.\\
Müller, Hans (2006): »Artikel 1«. In: \emph{Monatliche Zeitschrift} Jg. 28, Nr. 5, Juni 2006, 70--85.
Müller, Hans (2007): »Artikel 2«. In: \emph{ Zeitschrift} Jg. 28, Nr. 5, Herbst 2007, 70--85.

\item[basic] Funktioniert ähnlich wie \opt{minimum}, allerdings werden Jahre auch dann zusammengezogen, wenn \bibfield{extrayear}-Label generiert wird.

Müller, Hans (2000): \emph{Buch}. Location.\\
Müller, Hans (2003a): \emph{Buch 2}. Location.\\
Müller, Hans (2003b): \emph{Buch 3}. Location.\\
Müller, Hans (2006): »Artikel 1«. In: \emph{Monatliche Zeitschrift} Jg. 28, Nr. 5, Juni 2006, 70--85.
Müller, Hans (2007): »Artikel 2«. In: \emph{ Zeitschrift} Jg. 28, Nr. 5, Herbst 2007, 70--85.

\item[compact] Funktioniert ähnlich wie \opt{minimum}, allerdings werden Jahre auch dann zusammengezogen, wenn \bibfield{extrayear}-Label generiert wird.

Müller, Hans (2000): \emph{Buch}. Location.\\
Müller, Hans (2003a): \emph{Buch 2}. Location.\\
Müller, Hans (2003b): \emph{Buch 3}. Location.\\
Müller, Hans (Juni 2006): »Artikel 1«. In: \emph{Monatliche Zeitschrift} Jg. 28, Nr. 5, 70--85.
Müller, Hans (2007): »Artikel 2«. In: \emph{ Zeitschrift} Jg. 28, Nr. 5, Herbst 2007, 70--85.

\item[maximum] Funktioniert ähnlich wie \opt{minimum}, allerdings werden Jahre auch dann zusammengezogen, wenn \bibfield{extrayear}-Label generiert wird.

Müller, Hans (2000): \emph{Buch}. Location.\\
Müller, Hans (2003a): \emph{Buch 2}. Location.\\
Müller, Hans (2003b): \emph{Buch 3}. Location.\\
Müller, Hans (Juni 2006): »Artikel 1«. In: \emph{Monatliche Zeitschrift} Jg. 28, Nr. 5, 70--85.
Müller, Hans (Herbst 2007): »Artikel 2«. In: \emph{ Zeitschrift} Jg. 28, Nr. 5, 70--85.
\end{valuelist}

Die Kurzform \opt{mergedate} entspricht \kvopt{mergedate}{true}

\boolitem[false]{parensvolume}

Setzt den Inhalte des Felds \bibfield{volume} in Klammern.

\boolitem[false]{translatedas}

Aktiviert die Ausgabe der deutschen Übersetzung am Ende des Eintrags, siehe dazu \secref{sub:germ}. Die Kurzform \opt{translatedas} entspricht \kvopt{translatedas}{true}.\footnote{In früheren Versionen hiess diese Option \opt{germ}.}

\boolitem[false]{origyearsuperscript}
Ist diese Option aktiviert, wird bei Neuauflagen vor \bibfield{origdate} immer eine hochgestellte \textsuperscript{1} ausgegeben. Ansonsten geschieht dies nur bei Büchern.

\optitem[false]{origyearwithyear}{\opt{true}, \opt{false}, \opt{brackets}}

Mit dieser Option wird definiert, wie das Feld \bibfield{origdate} bei Neuauflagen ausgegeben wird.

\begin{valuelist}
\item[false] \bibfield{origdate} wird nicht zusammen mit \bibfield{date} respektive \bibfield{year} ausgegeben, sondern bei Aufsätzen in Klammern unmittelbar nach dem Aufsatzitel und bei den übrigen Dokumenttypen am Ende des Eintrags in Klammern.
\item[true] \bibfield{origdate} wird unmittelbar nach \bibfield{date} respektive \bibfield{year} ausgegeben; die beiden Felder sind durch einen Schrägstrich getrennt.
\item[brackets] \bibfield{origdate} wird ebenfalls unmittelbar nach \bibfield{date} respektive \bibfield{year} ausgegeben, steht aber in eckigen Klammern.
\end{valuelist}

\begingroup
  \emergencystretch=3em %
Die verschiedenen Varianten von \opt{origyearwithyear} und \opt{origyearsuperscript} lassen sich frei kombinieren.
\par\endgroup
\opt{origyearwithyear} hat nur einen Einfluss auf die Darstellung von Neuauflagen und auf Übersetzungen; wenn das Feld \bibfield{origtitle} existiert, wird \bibfield{oridate} weiterhin mit den Angaben zum Original ausgegeben.

\optitem[false]{origcite}{\opt{true}, \opt{false}, \opt{superscript}} Analog zu den Optionen \opt{origyearsuperscript} und \opt{origyearwithyear} kann  \bibfield{origyear} auch bei der Quellenangabe im Lauftext angegeben werden. Ist diese Option auf \opt{true} gesetzt, wird das Jahr der Erstveröffentlichung in eckigen Klammern ausgegeben, mit \opt{superscript} wird zusätzlich eine hochgestellte \textsuperscript{1} gesetzt.

\boolitem[false]{partofcited} Mit den \bibfield{crossref}- und \bibfield{xref}"=Feldern können in \sty{biblatex} Kinder- und Elterneinträge miteinander verknüpft werden; in der Regel sind dies Aufsätze und die entsprechenden Sammelbände. Die \opt{partofcited}"=Option sorgt dafür, dass bei der Ausgabe des Kindereintrages nicht die ganzen Angaben des Sammelbandes, sondern nur ein Verweis auf diesen ausgegeben wird; siehe dazu \secref{sub:kurz}. Die Kurzform \opt{partofcited} entspricht \kvopt{partofcited}{true}.

\optitem[false]{pages}{\opt{true}, \opt{false}, \opt{cite}, \opt{bib}, \opt{both}}

Diese Option steuert, ob bei vor die Seitenzahlen <S. > vorangestellt wird. Dabei können die Ausgabe im Lauftext und in der Bibliographie getrennt definiert werden. Mögliche Optionen sind:

\begin{valuelist}
\item[true] Sowohl im Lauftext als auch in der Bibliographie steht <S. > vor der Seitenzahl.
\item[both] Ist mit \opt{true} identisch.
\item[false] Die Seitenzahlen werden ohne Zusatz ausgegeben.
\item[cite] Bei Zitaten im Lauftext werden die Seitenzahlen mit dem Zusatz <S. > ausgegeben.
\item[bib] In der Bibliographie werden die Seitenzahlen mit dem Zusatz <S. > ausgegeben.
\end{valuelist}

\boolitem[false]{publisher}

Bestimmt, ob in der Bibliographie der Verlag -- respektive bei Übersetzungen der Verlag des Originals -- ausgegeben wird. Die Kurzform \opt{publisher} entspricht \kvopt{publisher}{true}.

Diese Option kann innerhalb eines Dokuments mit den Befehlen \cmd{ignorepublishertrue} und \cmd{ignorepublisherfalse} gesteuert werden. Zusätzlich gibt es noch die Befehle \cmd{ignoreaddresstrue} und \cmd{ignoreaddressfalse} respektive \cmd{ignorelocationtrue} und \cmd{ignorelocationfalse}, die die Ausgabe des Verlagortes steuern.

\optitem[false]{script}{\opt{true}, \opt{false}}
Aktiviert die Ausgabe der Drehbuchautoren im Feld \bibfield{scriptwriter} resp. \bibfield{editor} bei Filmen (\secref{sub:filme}).

\boolitem[false]{series}

Bestimmt, ob in der Bibliographie die Reihe, in der der zitierte Titel erschienenen ist, ausgegeben wird. Die Kurzform \opt{publisher} entspricht \kvopt{publisher}{true}.

\boolitem[false]{directorreplace}

Normalerweise wird bei Filmen, bei denen kein Regisseur angegeben ist, der Bibstring \texttt{notavailable}, also <[K.\,A.]> ausgegeben. Ist diese Option gesetzt, wird stattdessen der Inhalt des Felds \bibfield{production} angezeigt. Die Kurzform \opt{directorreplace} entspricht \kvopt{directorreplace}{true}.

\boolitem[false]{parensfilmnote}

Im Normalfall wird das Feld \bibfield{note} bei Filmen am Ende des Eintrags mit einem Punkt abgetrennt. Ist \opt{parensfilmnote} gesetzt, wird das Feld in Klammern ausgegeben; dies kann beispielsweise für die Angabe der benutzten DVD o.\,ä. genutzt werden. Die Kurzform \opt{directorreplace} entspricht \kvopt{directorreplace}{true}.

\boolitem[false]{yearatbeginning}

Während im Normalfall das Jahr am Ende des Eintrags ausgegeben wird, steht es mit der aktivierten Option \opt{yearatbeginning} nach dem Autor/Herausgeber zu Beginn in Klammern. Diese Option entspricht dem Aufruf des Stils \sty{fiwi2} (\secref{benutzung}). Die Kurzform \opt{yearatbeginning} entspricht \kvopt{yearatbeginning}{true}.

\boolitem[false]{xindy}

Standardmäßig aktiviert \sty{biblatex-fiwi} die Option \opt{indexing=cite}, das heißt, \sty{biblatex} schreibt bei allen \cmd{cite}-Befehlen einen Index-Eintrag in die \file{.idx}"=Datei. Dieser Eintrag hat normalerweise folgende Form
\begin{ltxcode}
\indexentry{Index-Eintrag@Index-Eintrag}{Seitenzahl}.
\end{ltxcode}
%
Der zweite Teil des Eintrags nach dem @-Zeichen zeigt dem Programm \sty{makeindex} an, wie der Eintrag sortiert werden muss und kann bei Bedarf über das Feld \bibfield{indextitle} definiert werden (ist \bibfield{indextitle} nicht definiert, wird \bibfield{title} verwendet). Das weitaus mächtigere Index"=Programm \bin{xindy}\fnurl{http://xindy.sourceforge.net/} ermöglicht viel komplexere Sortiervorgänge und benötigt den zweiten Teil des Index"=Eintrags nicht -- im Gegenteil: Der so genannte Actual-Operator @ macht die Verarbeitung mit \bin{xindy} viel aufwendiger. Die Option \opt{xindy} unterbindet deshalb, dass der zweite Teil ausgegeben wird; der Eintrag in die \sty{.idx}"=Datei sieht dann folgendermaßen aus:

\begin{ltxcode}
\indexentry{Index-Eintrag}{Seitenzahl}.
\end{ltxcode}
%
In diesem Fall hat das Feld \bibfield{indexsorttitle}  auch keinen Einfluss auf den Index"=Eintrag. Die Kurzform \opt{xindy} entspricht \kvopt{xindy}{true}. 

Beim Zitieren eines Films mittels \cmd{citefilm} wird der Titel des Films und nicht der Name des Regisseurs indexiert. Wie beim \cmd{citefilm}-Befehl überprüft \sty{biblatex-fiwi} dabei, ob innerhalb des Dokuments mehrere Filme mit gleichen Titels zitiert werden. Ist dies der Fall, wird zur Unterscheidung automatisch das Jahr im Index angegeben (siehe zum Indexieren von Filmen \secref{sub:indexfilm}).

In jedem Fall muss  die Index"=Erzeugung mittels eines Paket wie \sty{makedidx} oder \sty{imakedidx} sowie dem Befehl \cmd{makeindex} aktiviert werden.
\end{optionlist}

%\subsection{Optionen für die Bibliography}\label{check}
%
%\sty{biblatex-fiwi} stellt Filter für die Bibliographie bereit, die beim Aufrufen des Befehls \cmd{printbibliography} eingesetzt werden können. Die Filter entsprechen den gleichnamigen Paketoptionen (\secref{opt}).
%
%\begin{ltxcode}
%\printbibliography[check=<<Filter>>]
%\end{ltxcode}
%
%\begin{keymarglist}
%\item[ignoreforeword]
%Bestimmt, ob der Inhalt von \bibfield{foreword} ausgegeben wird oder nicht. 
%
%\item[ignoreafterword]
%Bestimmt, ob der Inhalt von \bibfield{afterword} ausgegeben wird oder nicht. 
%
%\item[ignoreintroduction]
%Bestimmt, ob der Inhalt von \bibfield{introduction} ausgegeben wird oder nicht. 
%
%\item[ignoreparatext]
%Kombination der beiden vorhergehenden Filter, der bestimmt, ob der Inhalt von \bibfield{foreword}, \bibfield{afterword} und \bibfield{introduction} ausgegeben wird oder nicht.
%
%\item[printpublisher]
%Bestimmt unabhängig von der  globalen Option \opt{publisher}, ob in der Bibliographie der Verlag -- respektive bei übersetzungen der Verlag des Originals -- ausgegeben wird.
%
%\item[ignorepublisher]
%Bestimmt unabhängig von der  globalen Option \opt{publisher}, ob in der Bibliographie der Verlag -- respektive bei übersetzungen der Verlag des Originals -- ausgegeben wird.
%
%\item[ignoreaddress]
%Bestimmt unabhängig, ob in der Bibliographie der Verlagsort -- respektive bei übersetzungen der Verlagsort des Originals -- ausgegeben wird.
%
%\end{keymarglist}

\section{Befehle}
\label{bas}

\subsection{Zitierbefehle}
\label{zitbefehl}

\sty{biblatex-fiwi} unterstützt alle gängigen \sty{biblatex}"=Befehle. Auch die \cmd{footcite}"=Befehle sollten grundsätzlich funktionieren. Da es sich bei \sty{biblatex-fiwi} aber um ein Format für Zitate im Lauftext handelt, ist deren Einsatz wenig sinnvoll und wird von mir auch nicht getestet.

\begin{ltxsyntax}
\cmditem{inparencite}[prenote][postnote]{key}

Neben den normalen \sty{biblatex}"=Befehlen unterstützt \sty{biblatexf-fiwi} den durch den \sty{biblatex"=luh"=ipw}"=Stil\fnurl{http://www.tex.ac.uk/ctan/macros/latex/exptl/biblatex-contrib/biblatex-luh-ipw/} eingeführten \cmd{inparencite}"=Befehl. Dieser ähnelt dem Befehl \cmd{parencite}, erlaubt aber Zitatangaben im Fließtext nach dem Schema \enquote{Wie \inparencite[28]{Lavery.D:2008a} erläutert}. \sty{inparencite} entspricht im Wesentlichen dem \sty{natbib}-Befehl \cmd{citet}, der durch die \opt{natbib}-Option aktiviert werden kann. Im Gegensatz zu diesem und den meisten anderen Zitierbefehlen wird bei \cmd{inparencite} eine Zitatwiederholung nie durch \enquote{ebd.} ersetzt.

\end{ltxsyntax}

\noindent Neben den Standardbefehlen bietet \sty{biblatex-fiwi} zudem noch spezielle Befehle für die Dokumenttypen \bibtype{movie}, \bibtype{video} und \bibtype{misc} (siehe \secref{supp:types}):

\begin{ltxsyntax}

\cmditem{citefilm}{key}

Zum Zitieren von Filmen. Die Form des Zitates hängt von der Option \opt{citefilm} ab (\secref{opt}). Im Normalfall werden der volle Titel und das Jahr ausgegeben, zB.: \citefilm{Kubrick.S:1975}. Bei allen weiteren Nennungen wird nur das Feld \sty{shorttitle} (falls vorhanden) resp. \sty{title} ausgegeben, zB: \citefilm{Kubrick.S:1975}. Existieren zwei Filme mit dem gleichen Titel, wird auch bei späteren Nennungen des Films das Jahr in Klammern ausgegeben, so dass der Film eindeutig identifiziert werden kann (das Gleiche geschieht auch beim Index-Eintrag). 

\cmditem{citefullfilm}{key}

Gibt  immer die kompletten Angaben zum Film aus, auch wenn der Film bereits zitiert wurde, die Form der Ausgabe hängt von der Option \sty{citefilm} ab.

\cmditem{citecfilm}{key}

Gibt -- unabhängig von der Option \sty{fullcitefilm} -- den vollen Filmtitel sowie Produktionsland und -jahr aus; zB.: \citecfilm{Kubrick.S:1975}. 
\end{ltxsyntax}

\subsection{Weitere Befehle}
\begin{ltxsyntax}
\cmditem{film}{text}

Dieser Befehl markiert Filmtitel und ist standardmäßig auf \cmd{textsc}, also auf Kapitälchen, eingestellt.
\end{ltxsyntax}

\section{Unterstütze Dokumenttypen}\label{supp:types}

\sty{biblatex} unterstützt eine Fülle von Dokumenttypen, von denen man -- je nach Wissenschaftsgebiet -- in der Praxis allerdings nur einen Bruchteil tatsächlich einsetzt. \sty{biblatex-fiwi} bietet  nur für Dokumenttypen, die im geisteswissenschaftlichen Bereich relevant sind,  spezifische Anpassungen. Alle nicht unterstützen Typen sind mir in der Praxis bislang nicht begegnet; bei Bedarf können in späteren Versionen aber weitere Dokumenttypen hinzukommen.

In der folgenden Liste sind die Dokumenttypen, für die \sty{biblatex-fiwi} spezielle Anpassungen mitbringt, aufgeführt und allfällige Besonderheiten beschrieben. Typen, die nicht in der Liste stehen, können dennoch eingesetzt werden -- \sty{biblatex-fiwi} bietet für sie einfach keine besonderen Anpassungen.

\begin{typelist}

\typeitem{article}
Entspricht dem normalen \sty{biblatex}"=Typ. Besonderes Augenmerk liegt auf  Kombinationen von \bibfield{volume}, \bibfield{number}, \bibfield{issue} und \bibfield{issuetitle}.

\typeitem{periodical}
Entspricht dem normalen \sty{biblatex}"=Typ.

\typeitem{book}
Entspricht dem normalen \sty{biblatex}"=Typ. Besonderes Augenmerk liegt auf mehrbändigen Werken mit \bibfield{title}, \bibfield{maintitle}, \bibfield{parttitle} u.a. sowie auf Kombinationen von Autoren, Herausgeber nund Bearbeitern.

\typeitem{incollection/inbook}

Die Unterscheidung dieser beiden Typen hat mir nie eingeleuchtet. Zumindest ist mir bislang noch kein Fall begegnet, bei dem sie mir sinnvoll erschienen wäre. Deshalb werden die beiden Dokumenttypen von \sty{biblatex-fiwi} gleich behandelt.

\typeitem{online} 

Entspricht dem normalen \sty{biblatex}"=Typ.

\typeitem{movie, video,misc}

Diese drei Typen werden alle für Filme verwendet, wobei \bibtype{movie} und \bibtype{video} gleich behandelt werden sollten; \bibtype{misc} ist eher ein historisches Überbleibsel. Die Standardfelder haben bei diesen Dokumenttypen eine leicht andere Bedeutung als normal (siehe dazu \secref{sub:filme} und \secref{zitbefehl}).

\typeitem{thesis}
Entspricht dem normalen \sty{biblatex}"=Typ.

\typeitem{phdthesis}
Entspricht dem normalen \sty{biblatex}"=Typ.

\typeitem{collection, reference}
Diese beiden Typen sind für Nachschlagewerke gedacht. In der Bibliographie wird zuerst der Name des Buches ausgegeben und anschließend die Autoren oder Herausgeber. Die Sortierung erfolgt anhand des Titels.

\typeitem{review}
Dieser Typ ist für Rezensionen gedacht. In den Standard"=Stilen von \sty{biblatex} ist \bibtype{review} mit \bibtype{article} identisch, \sty{biblatex-fiwi} dagegen stellt eine spezielle Anpassung zur Verfügung, bei der die Rezension mit dem Eintrag des rezensierten Werks verknüpft wird. Dabei wird das rezensierte Werk  im Feld \bibfield{related} angegeben, eine Angabe von \bibfield{title} ist fakultativ (\secref{sub:rezis}).

\typeitem{archival} Für Dokumente aus Archiven. Dieser Dokumenttyp wird nur von \sty{biblatex-fiwi} unterstützt (\secref{sub:zusatzdok}).
\end{typelist}

\subsection{Zusätzlich Dokumenttypen}\label{sub:zusatzdok}

Derzeit bietet \sty{biblatex-fiwi} mit \bibtype{archival} nur einen Dokumenttypen, den \sty{biblatex} nicht von Haus aus unterstützt. \bibtype{archival} ist für Dokumente aus Archivbeständen gedacht. Da ich im Rahmen eines Forschungsprojekts mit Dokumenten aus dem Bundesarchiv gearbeitet habe und \sty{biblatex} dafür keine brauchbare Lösung bot, habe ich diesen Typus hinzugefügt. 

Die Anforderungen beim Zitieren von Archivalien variieren je nach Fachbereich stark. Für meine Zwecke reichten die beiden Felder \bibfield{library} (wird von \sty{biblatex} definiert, aber nicht genutzt) und \bibfield{librarylocation}. Obwohl ich \bibtype{archival} bereits nutze, ist der Dokumenttype derzeit noch als experimentell zu betrachten. Falls Historiker noch zusätzliche Wünsche haben sollten, mögen sie sich bitte bei mir melden.

\bibtype{archival} verfügt über die üblichen Felder wie \bibfield{author, title, year}, \bibfield{date} etc. Im Vergleich zu den gängigen Dokumenttypen gibt es nur zwei Zusätze.

\begin{fieldlist}

\listitem{library}{literal}
Das Archiv, aus dem das Dokument stammt.

\listitem{librarylocation}{literal}
Die Archivsignatur.
\end{fieldlist}
%
Das Ergebnis sieht dann so aus:

\begin{ltxcode}
@archival{Hellwig.J:1976b,
	Author = {Hellwig, Joachim},
	Date = {1976-05-03},
	<<Library = {BArch},>>
	<<Librarylocation = {DR 118/3558},>>
	Title = {Protokoll-Auszüge aus der Gesprächsrunde \enquote{Werkstatt Zukunft 7} über eine Vorführung des Films \enquote{Werkstatt Zukunft 1}},
	Year = {1976}}
\end{ltxcode}%
\fullcite{Hellwig.J:1976b}
\bigskip

\noindent Wie gesagt: Für meine Zwecke haben diese beiden zusätzlichen Felder ausgereicht, bei Bedarf kann der Dokumenttyp aber noch weiter ausgebaut werden.

\subsection{Zusätzliche Felder}\label{sub:zusatzfeld}

\sty{biblatex} stellt eine Reihe von Feldern bereit, um das Erscheinungsjahr der Erstausgabe oder -- bei Übersetzungen -- das Original anzugeben. In den Standard"=Stilen wird nur das \bibfield{origlanguage}"=Feld unterstützt. \sty{biblatex-fiwi} dagegen nutzt alle  \bibfield{orig*}"=Felder; siehe dazu auch die Beispiele in \secref{beisp} (Alternativ kann hierfür auch der \bibfield{related}-Mechanismus eingesetzt werden \secref{sub:uebersetzungen}). Ausserdem wird \bibfield{nameaddon} unterstützt.

\begin{fieldlist}

\fielditem{origdate}{date}
Das Datum -- normalerweise nur das Jahr -- der Erstveröffentlichung respektive des Originals im Falle einer Übersetzung. Bei Aufsätzen, bei denen es sich nicht um Übersetzungen handelt, wird dieses Datum unmittelbar nach \bibfield{title} in Klammern ausgegeben, bei Büchern, die keine Übersetzungen sind, dagegen am Ende des Eintrags in Klammern mit einer hochgestellten \textsuperscript{1}. Handelt es sich um eine Übersetzung -- ist der \bibfield{origtitle} angegeben -- erfolgt die Ausgabe zusammen mit den Angaben zum Original ebenfalls am Ende des Eintrags in Klammern.

\fielditem{origlanguage}{key}
Die Sprache des Originals. Normalerweise wird dieses Feld gemeinsam mit \bibfield{translator}, \bibfield{origtitle}, \bibfield{origdate} und \bibfield{origlocation} angegeben, keine dieser Angaben ist aber zwingend nötig. 

\fielditem{origtitle}{literal}
Bei Übersetzungen der Titel des Originals. Wenn dieses Feld leer ist, werden auch \bibfield{origlocation} und \bibfield{origpublisher} nicht angegeben und \bibfield{origyear} nur mit einer hochgestellten \textsuperscript{1}.

\listitem{origlocation}{literal}
Bei Übersetzungen der Publikationsort des Originals. Wird nur ausgegeben, wenn \bibfield{origtitle} nicht leer ist.

\listitem{origpublisher}{literal}
Bei Übersetzungen der Verlag des Originals. Wie \bibfield{publisher} wird die Ausgabe über die Option \opt{publisher} gesteuert (\secref{opt}). Wird nur ausgegeben, wenn \bibfield{origtitle} nicht leer ist.

\fielditem{nameaddon}{literal}
Diente ursprünglich zur Angabe von Pseudonymen bei Autoren und Herausgebern; dabei wird Inhalt dieses Feldes  unverändert unmittelbar nach dem Namen in eckigen Klammern ausgegeben. Mit der Möglichkeit, für jeden Namen gesondert ein Pseudonym anzugeben (\secref{sub:pseudo}), dürfte dieses Feld nicht mehr allzu wichtig sein. Im Gegensatz zur Pseudonymfunktion bezieht sich \bibfield{nameaddon} nicht auf einen bestimmten Namen, sondern auf das jeweilige Feld als Ganzes.

\fielditem{parttitle}{literal}
Dient für den -- zugegebenermaßen nicht sehr häufigen -- Fall eines Bandes eines mehrbändigen Werks, das innerhalb einer Werkausgabe o.\,ä. erscheint.

\begin{ltxcode}
@book{Cassirer.E:2010c,
	Author = {Cassirer, Ernst},
	Editora = {Clemens, Julia},
	Editoratype = {redactor},
	Location = {Hamburg},
	Maintitle = {Gesammelte Werke},
	Number = {609},
	Origdate = {1929},
	Part = {3},
	<<Parttitle = {Phänomenologie der Erkenntnis},>>
	Publisher = {Meiner},
	Series = {Philosophische Bibliothek},
	Title = {Philosophie der symbolischen Formen},
	Volume = {13},
	Year = {2010}}
\end{ltxcode}%
\fullcite{Cassirer.E:2010c}

\textbf{Achtung}: \bibfield{parttitle} ist ein Zusatz von \sty{biblatex-fiwi}, der von keinem anderen Stil unterstützt wird.
\end{fieldlist}





\subsection{Filme}\label{sub:filme}

Filme werden mit den \cmd{citefilm}"=Befehlen zitiert (siehe \secref{zitbefehl}). Da \bibtex ursprünglich nicht für audiovisuelle Medien vorgesehen war, fehlen in den Standardtypen spezielle Felder. Bei \sty{biblatex-fiwi} kamen hierfür zu Beginn Standardfelder wie \bibfield{author} oder \bibfield{pagetotal} zum Einsatz, mittlerweile gibt es aber spezielle Felder mit passenden Namen. In der Funktionalität unterscheiden sich die Felder nicht; zu beachten ist einzig, dass Felder wie \bibfield{director} oder \bibfield{scriptwriter} spezifisch für \sty{biblatex-fiwi} sind, andere \sty{biblatex}-Stile können mit ihnen somit nichts anfangen. Da aber ohnehin kaum ein anderer Stil das Zitieren von audiovisuellen Medien so umfassend unterstützt, spielt das auch keine große Rolle.

\begin{fieldlist}

\listitem{director, author}{name}
Der Regisseur des Films.

\fielditem{filmmakertype}{literal}
Es kommt manchmal vor, dass ein Film keinen identifizierbaren Regisseur hat und man stattdeßen den Produzenten, Kameramann, Drehbuchautor o.\,ä. ausgeben will -- dafür ist \bibfield{filmmakertype} gedacht. Der Inhalt dieses Feldes wird in eckigen Klammern anschließend unmittelbar nach \bibfield{director} ausgegeben (der <Macher> des Films wird also nach wie vor mit \bibfield{director} angegeben, \bibfield{filmmakertype} macht aber deutlich, dass es nicht der Regisseur ist.

\listitem{scriptwriter, editor}{name}
Der/die Drehbuchautor(en) des Films, wird nur ausgegeben, wenn \opt{script} aktiviert ist (vgl. \secref{opt}).

\listitem{actor}{name}
Die Darsteller des Films, wird nur ausgegeben, wenn \opt{actor} aktiviert ist (vgl. \secref{opt}).

\listitem{production}{literal}
Die Produktionsfirma; wird nur ausgegeben, wenn \bibfield{director} nicht definiert und \opt{directorreplace} aktiviert ist (vgl. \secref{opt}).

\listitem{address, location}{literal}
Das Produktionsland.

\fielditem{title}{literal}
Der internationale Verleihtitel (siehe auch \secref{alternatetitle}).

\fielditem{subtitle}{literal}
Der deutsche Verleihtitel. 

\fielditem{duration, pagetotal}{literal}
Die Laufzeit des Films in Minuten, wird nur ausgegeben, wenn \opt{filmruntime} aktiviert ist (vgl. \secref{opt}). Die Einheit, in der die Laufzeit angegeben wird, kann über \bibfield{durationtype} bestimmt werden.

\fielditem{durationtype}{key}

Normalerweise wird die Laufzeit eines Films in Minuten angegeben; bei Stummfilmen ist es allerdings üblich, die Länge in Metern zu vermerken. Der Wert dieses Feldes bestimmt die Ausgabe in der Filmographie, wenn  \opt{filmruntime} aktiviert ist (vgl. \secref{opt}). Mögliche Werte sind \texttt{minute} (Standard) und \texttt{meter}.

\listitem{note}{literal}
Angaben zur zitierten DVD o.ä.; wird, falls vorhanden, unverändert am Ende des Eintrags in Klammern ausgegeben.

\end{fieldlist}

\subsection{Alternative Filmtitel}\label{alternatetitle}

Für Filme, die zwei gebräuchliche internationale Verleihtitel haben, gibt es das Feld \bibfield{alternattitle}. Ist dieses Feld vorhanden , wird es in der Filmographie sowie bei der ersten Nennung im Text automatisch ausgegeben. Ist \opt{xindy} aktiviert, wird zudem ein Index"=Eintrag für \bibfield{alternatetitle} erstellt, der auf \bibfield{title} verweist.

Falls der alternative Verleihtitel in einem  Alphabet ausgegeben werden soll, das von der Hauptschrift des  Dokuments nicht unterstützt wird (z.B. Kyrillisch oder Griechisch), kann dieses im Feld \bibfield{alternattitlescript} angegeben werden. Ist dieses Feld vorhanden, steht automatisch das Makro \sty{altscript:\prm{alternatetitlescript}-font} zur Verfügung. Dieses wird bei der Ausgabe von \bibfield{alternattitle} ausgeführt und kann entsprechend so definiert werden, dass eine passende Schrift geladen wird.
%
\begin{ltxcode}
@movie{Protazanov.Y:1924,
	Address = {SU},
	<<Alternatetitle = Аэлита,>>
	<<Alternatetitlescript = {cyrillic},	>>
	Author = {Protazanov, Yakov},
	Read = {Yes},
	Title = {Aelita},
	Year = {1924}}
@movie{Protazanov.Y:1924,
	Address = {SU},
	<<Alternatetitle = {Аэлита},>>
	<<Alternatetitlescript = {cyrillic},	>>
	Author = {Protazanov, Yakov},
	Read = {Yes},
	Title = {Aelita},
	Year = {1924}}
\end{ltxcode}
%
%
In diesem Beispiel steht das Makro \sty{altscript:cyrillic-font} bereit,\footnote{Das in dem Beispiel bei Аэлита die geschweiften Klammern fehlen, liegt daran, dass ich es noch nicht geschafft habe, die für diese Dokumentation verwendete Dokumentklasse vollständig mit nichtlateinischen Schriften zum Laufen zu bringen.}	 das in \luatex z.B. folgendermaßen definieren kann (der Befehl \cmd{newbibmacro*} ist Teil von\sty{biblatex} und kann somit erst nach dem Laden des Pakets verwendet werden):

\begin{ltxexample}
\newbibmacro*{altscript:cyrillic-font}[1]%
{\newfontfamily\cyrfont[]{Linux Libertine}{\cyrfont{\textsc{#1}}}}
\end{ltxexample}
%
Das Ergebnis sieht dann wie folgt aus: \citefilm{Protazanov.Y:1924}.

\subsection{Serien}\label{sub:serien}

\sty{biblatex-fiwi} stellt sowohl Zitierstile für Filme als auch für Fernsehserien bereit; bei Fernsehserien werden zwei Varianten unterschieden:

\subsubsection*{Zitieren einer einzelnen Folge}
In diesem Fall bleibt die Bedeutung der bisherigen Felder gleich; es kommen allerdings neue hinzu:

\begin{fieldlist}
\fielditem{maintitle}{literal}
Der Titel der Serie; für den Titel der zitierten Folge wird \bibfield{title} verwendet.

\fielditem{volume}{literal}
Die Staffel der zitierten Folge.

\fielditem{volume}{number}
Die Nummer der zitierten Folge innerhalb der Staffel.
\end{fieldlist}
%
\begin{ltxcode}
@movie{Reardon.J:1994a,
	Author = {Reardon, Jim},
	Location = {USA},
	Maintitle = {The Simpsons},
	Number = {6},
	Rating = {5},
	Title = {Treehouse of Horror V},
	Volume = {6},
	Year = {1994}}
\end{ltxcode}
\fullcite{Reardon.J:1994a}.
\\ \\
\noindent Um Fernsehserien innerhalb des Textes zu zitieren, gibt es zwei verschiedene Befehle: Mit \cmd{citefilm} wird der Titel der Serie (also \bibfield{maintitle}), die Angaben zu Staffel und Folge, sowie der Titel der spezifischen Folge und das Jahr ausgegeben. Bei \cmd{citeepisode} dagegen werden nur der Titel der Folge (also \bibfield{title}) und das Jahr ausgegeben.

\subsubsection*{Zitieren einer ganzen Serie}
Wenn eine ganze Serie und nicht eine einzelne Folge zitiert werden soll, gelten andere Anforderungen; so ist nun der Sender, der die Serie ursprünglich ausgestrahlt hat, oft wichtiger als das Land. Dagegen sind die Regisseure der Sendung weniger wichtig als ihre <Erfinder> . Als Zitierbefehle kommen weiterhin die Befehle \cmd{citefilm} und \cmd{citefullfilm} zum Einsatz. Um anzuzeigen, dass es sich bei dem zitierten Film um eine ganze Serie handelt, wird das Feld \bibfield{entrysubtype} auf \prm{serial} gesetzt. Zudem werden einige bestehende Felder nun anders verwendet.

\begin{fieldlist}
\fielditem{entrysubtype}{literal}
Um eine ganze Serie zu zitieren, muss in diesem Feld \prm{serial} stehen.

\fielditem{organization}{literal}
Der Fernsehsender, der die Sendung zuerst ausgestrahlt hat.

\fielditem{title}{literal}
Ist \bibfield{entrysubtype} auf \prm{serial} gesetzt, wird hier der Name der Serie angegeben und nicht der der Folge.

\fielditem{author}{name}
Wird hier für den Schöpfer/Erfinder der Serie verwendet.
\end{fieldlist}

\begin{ltxcode}
@movie{Ball.A:2003a,
	Author = {Ball, Alan},
	Date = {2003/},
	Entrysubtype = {serial},
	Location = {USA},	
	Organization = {HBO},
	Title = {True Blood}}
\end{ltxcode}
\fullcite{Ball.A:2003a}

\subsection{Zitieren von Fernsehsendungen}\label{movie:tv}

Im Gegensatz zum vorangegangen Kapitel geht es hier um das Zitieren einer zu einem bestimmten Zeitpunkt ausgestrahlten Sendung. In der Regel handelt es sich dabei gerade nicht um Filme, Serien o.ä., sondern um die Eigenproduktionen -- meist Infomations- oder Live"=Sendungen -- eines Senders.

Fernsehsendungen wie z.\,B.\  Talkshows oder Nachrichtensendungen haben keinen Regisseur im eigentlichen Sinn, dafür aber ein präzises Ausstrahlungsdatum. Statt des Landes wird hier der Fernsehsender angegeben.
\begin{fieldlist}
\fielditem{entrysubtype}{literal}
Um eine Sendung zu zitieren, muss in diesem Feld \prm{tv} stehen.

\fielditem{organization}{literal}
Der Fernsehsender, auf dem die Sendung  ausgestrahlt wurde.

\fielditem{date}{date}
Das Datum der Ausstrahlung.

\fielditem{maintitle}{literal}
Handelt es sich um ein wiederkehrendes Format, bei dem die einzelnen Folgen eigene Titel haben, wird hier der Name der Sendung angegeben.

\fielditem{title}{literal}
Handelt es sich um ein wiederkehrendes Format, bei dem die einzelnen Folgen eigene Titel haben, wird hier der Name zitieren Folge angegeben. Tragen die einzelnen Folgen dagegen keinen Namen, bleibt \bibfield{maintitle} leer und der Name der Sendung wird mit \bibfield{title} angegeben.
\end{fieldlist}
%
\begin{ltxcode}
@movie{Kassensturz:2012-05-12,
	Date = {2012-05-12},
	Entrysubtype = {tv},
	Organization = {SF1},
	Pagetotal = {23},
	Title = {Kassensturz},
	Year = {2012}}
\end{ltxcode}
\fullcite{Wuergel.H:2012d}.
\\ \\
\noindent Auch hier gibt es die Möglichkeit den Titel der Sendung und jenen der spezifischen Folge mit \cmd{citefilm} zu zitieren oder mittels \cmd{citeepisode} nur den Titel der gewünschten Folge.

Für den Fall, dass verschiedene Ausgaben einer Sendung zitiert werden, bei der eine einzelne Folge keinen eigenen Titel hat (man danke an \film{Tagesschau}, \film{Sportstudio} oder dergleichen) wird das Ausstrahlungsdatum der Sendung auch bei Folgezitaten ausgegeben. Werden also zum Beispiel \citefilm{Wuergel.H:2012d} und \citefilm{Wuergel.H:2012b} zitiert, so so sieht ein Wiederholungszitat folgedermaßen aus: \citefilm{Wuergel.H:2012d}.

\subsection{Sortieren von Filmen}\label{sub:sort}

Filme werden normalerweise in einer gesonderten Filmographie aufgeführt. Da \sty{biblatex} die Möglichkeit bietet, mehrere Bibliographien auszugeben, kann auf einfache Weise eine Filmographie erstellt werden:

\begin{ltxexample}
\documentclass{...}
\usepackage{biblatex}
\addbibresource{...}
\begin{document}
...
\printbibliography[<<nottype=movie,nottype=video,nottype=misc>>,heading=subbibliography, title={Literatur}]
\printbibliography[<<type=movie,type=video,type=misc>>,heading=subbibliography,title={Filmographie}]
\end{document}
\end{ltxexample}
%
Normalerweise werden Filmographien nicht nach dem Namen des Regisseurs, sondern nach dem Titel des Films geordnet. Seit Version 2.0 bietet \sty{biblatex} die Möglichkeit, das Sortier"=Schema nicht nur global, sondern gesondert für jede Bibliographie zu setzen. \sty{biblatex-fiwi} bietet zusätzlich zu den Standard"=Schemata noch das Sortierschema \opt{title}, das ausschliesslich nach Titeln ordnet:

\begin{ltxexample}
<<\newrefcontext[sorting=title]>>
\printbibliography[ype=movie,type=video,type=misc,  heading=subbibliography,title={Filmographie}]
\end{ltxexample}
%
 Bei einzelnen Folgen von Fernsehserien wird zuerst nach \bibfield{maintitle} und anschließend nach \bibfield{title} sortiert.\footnote{In früheren Versionen wurde mittels \cmd{DeclareSortExclusion}  das Sortieren nach Autoren, Herausgebern etc. für \bibtype{misc,movie} und \bibtype{video} generell deaktiviert. Dies ist nun nicht mehr der Fall. Das kann zur Folge haben, dass bestehende Dokumente angepasst werden müssen.}

Zudem bietet \bin{biber} die Möglichkeit, beim Sortieren von Filmtiteln die Artikel zu ignorieren, so dass beispielsweise \textsc{The Godfather} unter <G> eingeordnet wird. Standardmäßig werden bei \sty{biblatex-fiwi} so die gängigen Artikel in den Feldern \bibfield{title} und \bibfield{maintitle} ignoriert. Mit der Option \opt{ignorearticle=false} kann diese Funktion deaktiviert werden. Wenn alle Stricke reißen, kann man am Ende immer noch auf die Felder \bibfield{sorttitle} respektive \bibfield{sortkey} ausweichen.

\subsection{Indexieren von Filmen}\label{sub:indexfilm}
\sty{biblatex-fiwi} bietet mit \opt{filmindex} und \opt{splitfilmindex} zwei Optionen zum Indexieren von Filmen (\secref{opt}). Ist \opt{filmindex} aktiviert, führt \cmd{citefilm} zu folgendem Eintrag in das .idx-File:

\begin{ltxcode}
@movie{Kubrick.S:1968,
        Location = {GB and USA},
        Director = {Kubrick, Stanley},
        Subtitle = {2001: Odyssee im Weltraum},
        Title = {2001: A Space Odyssey},
        Year = {1968}}
\end{ltxcode}

\begin{ltxcode}
\indexentry{2001: A Space Odyssey@\textsc {2001: A Space Odyssey}}{1}
\end{ltxcode}
%
Ist \opt{filmindex} auf \opt{complete} gesetzt, erzeugt \sty{biblatex-fiwi} folgenden Eintrag:

\begin{ltxcode}
\indexentry{2001: A Space Odyssey @ \fullcite {Kubrick.S:1968}}{1}
\end{ltxcode}
%
\sty{biblatex-fiwi} schreibt also einfach einen \cmd{fullcite}-Befehl in den Index, entsprechend sieht das Ergebnis aus. Falls nun zwei Filme mit gleichem Titel indexiert werden, wird automaitsch das Jahr in das .idx-File geschrieben:

\begin{ltxcode}
@movie{Siegel.D:1956,
        Location = {USA},
        Director = {Siegel, Don},
        Subtitle = {Die Dämonischen},
        Title = {Invasion of the Body Snatchers},
        Year = {1956}}
@movie{Kaufman.P:1978,
        Location = {USA},
        Director = {Kaufman, Philip},
        Subtitle = {Die Körperfresser kommen},
        Title = {Invasion of the Body Snatchers},
        Year = {1978}}
\end{ltxcode}

\begin{ltxcode}
\indexentry{Invasion of the Body Snatchers (1978)@\textsc {Invasion of the Body Snatchers} (1978)}{1}
\indexentry{Invasion of the Body Snatchers (1956)@\textsc {Invasion of the Body Snatchers} (1956)}{1}
\end{ltxcode}
%
Ist \opt{filmindex=complete} gesetzt, sieht der Eintrag etwas anders aus:

\begin{ltxcode}
\indexentry{Invasion of the Body Snatchers1978 @ \fullcite {Kaufman.P:1978}}{1}
\indexentry{Invasion of the Body Snatchers1956 @ \fullcite {Siegel.D:1956}}{1}
\end{ltxcode}
%
In beiden Fällen ist also sicher gestellt, dass \bin{makeindex} die Einträge unterscheiden kann.

Die Option \opt{xindy} sorgt dafür, dass der Actual-Operator @ und der zweite Teil des Eintrags nicht ausgegeben wird. In diesem Fall sieht ein Eintrag folgendermaßen aus:

\begin{ltxcode}
\indexentry{\textsc  {2001: A Space Odyssey}}{1}
\end{ltxcode}
%
Ein Eintrag dieser Form führt bei der Kombination von \opt{filmindex=complete} und \bin{xindy} zu falschen Sortierungen; denn in diesem Fall würde nur ein \cmd{fullcite}-Befehl im .idx-File stehen und \bin{xindy} hätte keine Möglichkeit, den Filmtitel zu ermitteln. Deshalb ist hier ein bisschen Gebastel nötig. Bei \opt{filmindex=complete} wird zusätzlich der Filmtitel (resp. \bibfield{indextitle}) in das .idx-File geschrieben.

\begin{ltxcode}
 \indexentry{\sortentry {2001: A Space Odyssey} \fullcite{Kubrick.S:1968}}{1}
\end{ltxcode}
%
Der Befehl \cmd{sortentry} ist so definiert, dass sein Inhalt nicht ausgegeben wird. Dieser Teil des Index-Eintrags erscheint also nicht im Index, sondern dient dazu, \bin{xindy} anzuzeigenden, welcher Teil des Eintrags zum Sortieren gedacht ist. Mit einer Merge-Rule dieser Form kann man dann dafür sorgen, dass der Eintrag richtig sortiert wird:

\begin{ltxcode}
(merge-rule "\\sortentry \{(.*)\}\\fullcite(.*)" "\1" :eregexp :again)
\end{ltxcode}
%
Zur Erklärung: Diese Regel sortiert den Eintrag unter dem Inhalt von \cmd{sortentry}. Da \cmd{sortentry} nichts anzeigt, erscheint nur der zweite Teil, also das Ergebnis von \cmd{fullcite}, im Index.

Und hier schließlich noch ein Beispiel für einen Index-Eintrag eines Films, bei dem \bibfield{alternatetitlescript} (\secref{alternatetitle}) definiert ist:

\begin{ltxcode}
@movie{Protazanov.Y:1924,
        Location = {SU},
        Alternatetitle = Аэлита,
        Alternatetitlescript = {cyrillic},
        Director = {Protazanov, Yakov},
        Year = {1924}}
\end{ltxcode}
%
Ist \opt{filmindex=complete} gesetzt, sieht der Eintrag etwas anders aus (in der Variante mit \opt{filmindex=complete} und \opt{xindy}):

\begin{ltxcode}
\indexentry{\sortentry {Aelita} \fullcite {Protazanov.Y:1924}}{1}
\indexentry{\usebibmacro *{altscript:cyrillic-font}Аэлита}{{}}\rmfamily |see{\textsc {Aelita}}}{1}
\end{ltxcode}
%
Hier werden zwei Einträge erzeugt. Zum einen Eintrag mit dem vollen Titel, der unter ›Aelita‹ sortiert werden kann, sowie ein zweiter Eintrag mit dem kyrillischen Titel, der auf den lateinischen verweist. Damit dieser korrekt sortiert wird, ist wieder eine Merge-Rule nötig.

\begin{ltxcode}
(merge-rule "altscript\:(.*)-font(.*)" "\2" :eregexp :again)
\end{ltxcode}
%
Unschlüssig bin ich hinsichtlich der Frage, wie Einträge in nicht-lateinischem Alphabet sortiert werden müssen. Sinnvolle Vorschläge sind erwünscht.

\section{Verbundene Einträge}\label{sub:verbund}

Seit Version 2.0 bietet \sty{biblatex} die Möglichkeit, zwei Einträge miteinander zu verknüpfen. Dabei wird im Feld \bibfield{related} der Citekey des verknüpften Eintrags und im Feld \bibfield{relatedtype} die Art der Verbindung angegeben. Mit \sty{biblatex-fiwi} sind verschiedene Varianten von Verknüpfungen möglich.

\subsection{Übersetzungen und Neuauflagen}\label{sub:uebersetzungen}

Übersetzungen und Neuauflagen können nicht nur mittels den \bibfield{orig*}-Feldern (\secref{sub:zusatzfeld}), sondern auch über verknüpfte Einträge angegeben werden.

Diese beiden Einträge
\begin{ltxcode}
@book{Todorov.T:1970,
	Author = {Todorov, Tzvetan},	
	Location = {Paris},
	Publisher = {Éditions du Seuil},
	Title = {Introduction à la littérature fantastique},
	Year = {1970}}

@book{Todorov.T:1992,
	Author = {Todorov, Tzvetan},
	Location = {Frankfurt a. M.},
	Origlanguage = {french},
	Publisher = {Fischer},
	<<Related = {Todorov.T:1970},
	Relatedtype = {translationof},>>
	Title = {Einführung in die fantastische Literatur},
	Translator = {Kersten, Karin and Metz, Senta and Neubaur, Caroline},
	Year = {1992}}
\end{ltxcode}
%
führen zur gleichen Ausgabe wie dieser:

\begin{ltxcode}
@book{Todorov.T:1992,
	Author = {Todorov, Tzvetan},
	Location = {Frankfurt a. M.},	
	<<Origdate = {1970},
	Origlanguage = {french},
	Origlocation = {Paris},
	Origpublisher = {Éditions du Seuil},
	Origtitle = {Introduction à la littérature fantastique},>>
	Publisher = {Fischer},
	Title = {Einführung in die fantastische Literatur},
	Translator = {Kersten, Karin and Metz, Senta and Neubaur, Caroline},
	Year = {1992}}
\end{ltxcode}
%
\fullcite{Todorov.T:1992}\\

\noindent Welche Variante man im konkreten Fall wählt, ist nur eine Frage des persönlichen Geschmacks respektive der Bequemlichkeit.\\

\noindent Analog funktioniert \bibfield{related} auch bei Neuauflagen. Diese beiden Einträge
\begin{ltxcode}
@book{Bordwell.D:1979a,
	Author = {Bordwell, David and Thompson, Kristin},
	Location = {Reading},
	Publisher = {Addison-Wesley},
	Title = {Film Art: An Introduction},
	Year = {1979}}

@book{Bordwell.D:2004a,
	Author = {Bordwell, David and Thompson, Kristin},
	Edition = {7},
	Location = {Boston and Burridge and Dubuque and Madison and New York},
	Publisher = {McGraw-Hill},
	<<Related = {Bordwell.D:1979a},
	Relatedtype = {origpubin},>>
	Title = {Film Art: An Introduction},
	Year = {2004}}
\end{ltxcode}
%
führen zur gleichen Ausgabe wie dieser:
\begin{ltxcode}  
@book{Bordwell.D:2004a,
	Author = {Bordwell, David and Thompson, Kristin},
	Edition = {7},
	<<Origdate = {1979},>>
	Location = {Boston and Burridge and Dubuque and Madison and New York},	
	Publisher = {McGraw-Hill},
	Title = {Film Art: An Introduction},
	Year = {2004}}
\end{ltxcode}
\fullcite{Bordwell.D:2004a}\\

\noindent Wie man an diesem Beispiel sieht, lohnt sich die Verknüpfung bei Neuauflagen in der Regel nicht. Die Funktion ist vor allem dann nützlich, wenn man -- aus welchen Gründen auch immer -- ohnehin zwei verschiedene Varianten einer Publikation bibliographiert hat.

\subsection{Deutsche Übersetzungen}\label{sub:germ}

Über den gleichen Mechanismus kann bei einer fremdsprachigen Publikation die deutsche Übersetzung angegeben werden: Dazu kommt die Option \opt{translatedas} (siehe \secref{opt}) in Kombination mit den Feldern \bibfield{related} und \bibfield{relatedtype} zum Einsatz.\footnote{In früheren Versionen hiess diese Option \opt{germ} wurde in Kombination mit dem \bibfield{usera} benutzt; dieses wird nun nicht mehr unterstützt. Alte .bib-Dateien müssen entsprechend angepasst werden.} Diese Funktion habe ich für ein eigenes Buchprojekt geschrieben, im Normalfall dürfte sie kaum benötigt werden (in wissenschaftlichen Arbeiten ist es üblich, bei Übersetzungen das fremdsprachige Original anzugeben. Der gegenteilige Fall kommt dagegen kaum vor). Dabei wird ein Eintrag mittels Angabe des Citekeys in \bibfield{related} mit dem Eintrag, der die Übersetzung enthält, verknüpft und in \bibfield{relatedtype} die Art der Verknüpfung angegeben -- in diesem Fall \opt{translatedas} -- angegeben:

\begin{ltxcode}
@incollection{Poe.E:1982c,
	Author = {Poe, Edgar Allan},
	Bookauthor = {Poe, Edgar Allan},
	Booktitle = {The Complete Tales and Poems of Edgar Allan Poe},
	Location = {London and New York},
	Origdate = {1843},
	Pages = {223--230},
	Title = {The Black Cat},
	<<Related = {Poe.E:1999a},
	Relatedtype = {translatedas},>>
	Year = {1982}}

@incollection{Poe.E:1999a,
	Author = {Poe, Edgar Allan},
	Bookauthor = {Poe, Edgar Allan},
	Booktitle = {Der schwarze Kater. Erzählungen},
	Location = {Zürich},
	Maintitle = {Gesammelte Werke in 5 Bänden},
	Origdate = {1843},
	Origlanguage = {american},
	Origtitle = {The Black Cat},
	Pages = {139--152},
	Publisher = {Haffmans Verlag},
	Title = {Der schwarze Kater},
	Translator = {Wollschläger, Hans},
	Volume = {3},
	Year = {1999}}
\end{ltxcode}
\fullcite{Poe.E:1982c}
\bigskip

\noindent Wie das Beispiel zeigt, wird die deutsche Übersetzung am Ende in eckigen Klammern ausgegeben. Der Autor sowie die Angaben zum englischsprachigen Original, die im Eintrag der deutschen Fassung enthalten sind, werden nicht ausgegeben, da sie in diesem Fall redundant wären.

\subsection{Rezensionen}\label{sub:rezis}
Der Dokumenttyp \bibtype{review} ist für Rezensionen gedacht und funktioniert analog zu den vorher beschriebenen Verknüpfunge. Auch hier werden zwei Einträge miteinander verbunden, wobei der Citekey des rezensierten Werks im Feld \bibfield{related} angegeben wird; da eine Rezension eigentlich immer mit dem rezensierten Werk verknüpft ist, entfällt die Angaben von \bibfield{relatedtype}.

Das Feld \bibfield{title} ist bei \bibtype{review} fakultativ. Ist es nicht vorhanden, wird einfach nur »Rezension von \ldots« als Titel ausgegeben.

\begin{ltxcode}
@movie{Cameron.J:2009b,
	Author = {Cameron, James},
	Location = {USA and GB},
	Pagetotal = {162},
	Subtitle = {Avatar -- Aufbruch nach Pandora},
	Title = {Avatar},
	Year = {2009}}
	
@review{Spiegel.S:2009a,
	Author = {Spiegel, Simon},
	Date = {2009-12-17},
	Journal = {Basler Zeitung},
	Pages = {21--23},
	Title = {Welcome to Pandora},
	<<Related = {Cameron.J:2009b}>>}
\end{ltxcode}
\fullcite{Spiegel.S:2009a}

\subsection{Verkürzte Bibliographie"=Einträge mit \bibfield{crossref} und \bibfield{xref}}\label{sub:kurz}

\sty{biblatex} bietet zwei verschiedene Möglichkeiten an, um Eltern"= und Kindeinträge zu verbinden -- das \bibfield{crossref}"= und das \bibfield{xref}"=Feld. Mittels der \opt{mincrossrefs}"=Option von \sty{biblatex} kann bestimmt werden, ab wie vielen Nennungen von Kindereinträgen der Elterneintrag automatisch ausgegeben wird, selbst wenn dieser nicht explizit zitiert wird (siehe Kapitel 2.4.1 in der \sty{biblatex}"=Anleitung). Die Option \opt{partofcited} dagegen ist für einen anderen Fall vorgesehen: Wenn sowohl Eltern- wie auch Kindeintrag explizit zitiert werden, werden mit dieser Option bei der Ausgabe des Kindereintrages nicht die ganzen Angaben des Sammelbandes, sondern nur ein Verweis auf diesen ausgegeben. Auf das Verhalten von \opt{mincrossrefs} hat \opt{partofcited} keinen Einfluss.

\begin{ltxcode}
@book{Telotte.J:2008a,
	Editor = {Telotte, J. P.},
	Location = {Lexington},
	Publisher = {University Press of Kentucky},
	Series = {Essential Readers in Contemporary Media and Culture},
	Title = {The Essential Science Fiction Television Reader},
	Year = {2008}}

@incollection{Lavery.D:2008a,
	Author = {Lavery, David},
	<<Crossref = {Telotte.J:2008a},>>
	Pages = {283--245},
	Subtitle = {Is \film{Lost} Science Fiction?},
	Title = {The Island's Greatest Mystery}}
\end{ltxcode}
Diese  Einträge führen, wenn sie beide zitiert werden und \opt{partofcited} aktiviert ist, zu folgender Ausgabe in der Bibliographie:\\ \\
\fullcite{Telotte.J:2008a}\\
\fullcite{Lavery.D:2008a} \\

%\noindent Es gibt bei dieser Option derzeit noch zwei Einschränkungen zu beachten: Die Nennung des Elterneintrags erfolgt übereinen \cmd{textcite}"=Befehl, der direkt in die \file{.bbl}"=Datei geschrieben wird. Dies hat zur Folge, dass der Elterneintrag und die verkürzte Variante des Kindereintrages auch dann in der Bibliographie ausgegeben bleiben, wenn der Elterneintrag bereits wieder gelöscht worden ist. Daran ändert auch ein erneuter \bin{biber}"=Durchlauf nichts. Um die korrekte Ausgabe zu erhalten, muss in dieser Situation die \file{.bbl}"=Datei von Hand gelöscht werden. Auch diese Beschränkungen sollte mit \sty{biblatex} 2.x fallen.

%Derzeit ist diese Option nur beim \opt{authoryear}"=Format wirklich sinnvoll; beim Standardformat mit der Angabe des Jahres am Ende des Eintrags fehlt nun die Jahresangabe des Kindereintrags, was nicht optimal ist. Ideen für eine bessere Darstellungsform werden gerne entgegengenommen.

\section{Pseudonyme}\label{sub:pseudo}

Seit Version 3.5 bietet \biblatex die Möglichkeit, beliebige Schemata für Namen festzulegen, so können auch nicht-westliche Namenskonventionen unterstützt werden. Standardmäßig werden die folgenden, vom klassischen \bibtex vorgegebenen Namensteile definiert:

\begin{itemize}
\item Familienname (<last> part)
\item Vorname (<first> part)
\item Namen-Präfix (<von> part)
\item Namen-Suffix (<Jr> part)
\end{itemize}
%
\sty{biblatex-fiwi} nutzt dieses Feature für Pseudonyme. Dabei werden die klassischen Namenslemente, also der Name, unter dem das Werk veröffentlicht wurde, als Pseudonym behandelt, zu dem bei Bedarf ein Realname angegeben werden kann. Analog zu den bisherigen Namenselementen kommen somit also noch die folgenden hinzu:

\begin{itemize}
\item Realer Familienname (<last> part)
\item Realer Vorname (<first> part)
\item Reales Namen-Präfix (<von> part)
\item Reales Namen-Suffix (<Jr> part)
\end{itemize}
%
Diese neue Funktion hat allerdings einen Haken: Das \bibtex-Format ist sehr alt und ursprünglich war eine Erweiterung des Namensschemas nicht vorgesehen. Diese Einschränkung kann auf zwei Arten umgangen werden. Zum einen unterstützt \biblatex ein neues, auf XML basiertes Dateiformat namens \biblatexml. Ein Name mit Pseudonym und Realname kann damit folgendermaßen angegeben werden:

\begin{lstlisting}[language=xml]
<bltx:entry id="Blish.J:1973a" entrytype="book">
    <bltx:names type="author">
      <bltx:name>
        <bltx:namepart type="family" initial="A">Atheling</bltx:namepart>
        <bltx:namepart type="suffix" initial="J">Jr.</bltx:namepart>
        <bltx:namepart type="given" initial="W">William</bltx:namepart>
        <bltx:namepart type="truefamily" initial="B">Blish</bltx:namepart>
        <bltx:namepart type="truegiven" initial="J">James</bltx:namepart>
      </bltx:name>
    </bltx:names>
\end{lstlisting}
%    
Dies wäre zwar die saubere Variante, sie ist aber nicht wirklich praktikabel. Zum einen wird \biblatexml noch immer als experimentell eingestuft, zum anderen wird es von keinem Programm ausser \biber~-- insbesondere von keiner Bibliographieverwaltung -- unterstützt. Und XML-Daten von Hand zu bearbeiten, ist keine sonderlich lustige Angelegenheit. Wirklich praxistauglich ist \biblatexml somit (noch) nicht. 

Daneben gibt es aber noch die Möglichkeit, das bestehende \bibtex-Format zurechtzubiegen. Der oben stehende Eintrag kann in traditionellem \bibtex so angegeben werden:
    
\begin{ltxcode}
@book{Blish.J:1973a,
	Address = {Chicago},
	<<Author = {given=William, family=Atheling, suffix=Jr., truefamily=Blish, truegiven=James},>>
	Edition = {2},
	Origdate = {1964},
	Publisher = {Advent Publishers},
	Subtitle = {Studies in Contemporary Magazine Science Fiction},
	Title = {The Issue at Hand},
	Year = {1973}}
\end{ltxcode}
%
Diese Variante ist nicht sehr elegant und wahrscheinlich haben nicht alle Bibliographieverwaltungen Freude an ihr, aber sie stellt derzeit den bestmöglichen Kompromiss dar. Der oben stehende Eintrag wird dann so ausgegeben:\\
\fullcite{Blish.J:1973a}.

Bei Pseudonymen werden beide Namen indexiert, wobei der Realname auf das Pseudonym verweist.

\section{Kleinkram}\label{sub:besond}
\subsection{Neue Localization Strings}
Localization Strings sind vordefinierte Begriffe wie »Aufl.« oder »Hg.«, die von \sty{biblatex} verwendet und bei Bedarf an die jeweilige Sprache angepasst werden. \sty{biblatex-fiwi} definiert eine Reihe zusätzlicher Strings -- allerdings vorläufig nur auf Deutsch. Anpassungen an andere Sprachen müssen mittels \cmd{DefineBibliographyStrings} vorgenommen werden.

\begin{keymarglist}
\item[fromjapanese] Der Ausdruck <aus dem Japanischen>.
\item[fromhebrew] Der Ausdruck <aus dem Hebräischen>.
\item[minutes] Der Ausdruck <min>, für Längenangaben bei Filmen.
\item[meters] Der Ausdruck <m>, für Längenangaben bei Filmen.
\item[tvseason] Der Ausdruck <Staffel>, für Fernsehserien.
\item[tvepisode] Der Ausdruck <Folge>, für Fernsehserien.
\item[translatedto] Der Ausdruck <dt.>, wird verwendet, wenn die deutschsprachige Übersetzung eines fremdsprachigen Textes angegeben werden soll (siehe \opt{translatedas} und \bibfield{related}  \secref{sub:germ}).
\item[prepublished] Der Ausdruck <Online-Vorveröffentlichung>. Ein zusätzlicher Publikationsstand für das Feld \bibfield{pubstate}.
\item[reviewof] Der Ausdruck <Rezension von>, wird bei \bibtype{review} verwendet.
%\item[undated] Der Ausdruck <Undatiert>, wird bei Einträgen ohne Datumsangabe verwendet.
\item[actors] Der Ausdruck <Darsteller>, wird bei der Anzeigen von Schauspielern verwendet, wenn die Option \opt{actor} aktiviert ist (vgl. \secref{opt}).

\item[directedby] Der Ausdruck <Regie>, wird bei der Anzeigen von Regisseuren verwendet, wenn die Option \opt{script} aktiviert ist (vgl. \secref{opt}).

\item[writtenby] Der Ausdruck <Buch>, wird bei der Anzeigen von Drehbuchautoren verwendet, wenn die Option \opt{script} aktiviert ist (vgl. \secref{opt}).

\item[writtendirectedby] Der Ausdruck <Buch und Regie>, wird bei der Anzeigen von Regisseuren und Drehbuchautoren verwendet, wenn die Option \opt{script} aktiviert ist (vgl. \secref{opt}).
\item[notavailable] Der Ausdruck <[K.\,A.]>, wird verwendet, wenn \bibfield{director} nicht gesetzt und die Option \option{directorreplace} nicht aktiviert ist (vgl. \secref{opt}).
\end{keymarglist}



\subsection{Aktivierte Optionen}
Wenn \sty{biblatex-fiwi} ohne weitere Optionen geladen wird, werden folgende \sty{biblatex}-Optionen gesetzt: 

\begin{ltxexample}
\ExecuteBibliographyOptions{indexing=cite,maxnames=3,minnames=3,maxitems=9,useprefix=true,hyperref=auto,sorting=nyt,origdate=long,datelabel=long,urldate=long,citetracker=context,labeldate=true,autolang=hyphen,datezeros=false,dateera=simple,dateuncertain=true,datecirca=true,pagetracker=true,ibidtracker=context,isbn=false,partofcited=true}

\ExecuteBibliographyOptions[misc,movie,video]{uniquename=false,labelyear=false,labeltitle=true}
\end{ltxexample}


\section{Beispiele}\label{beisp}
Die folgenden Beispiele sollen einen Einblick geben, was mit \sty{biblatex-fiwi} alles möglich ist. 

Für die Beispiele wurden folgende Optionen  zusätzlich zu den Standardeinstellungen aktiviert: \kvopt{publisher}{true}, \kvopt{series}{true} und \kvopt{partofcited}{true}. Weitere Beispiele mit anderen Einstellungen finden sich in den separaten PDF"=Dateien.%

\begin{ltxcode}
@collection{clute.j:1999,
	Edition = {2},
	Editor = {Clute, John and Nicholls, Peter},
	Location = {London},
	Origdate = {1979},
	Publisher = {Orbit},
	Title = {The Encyclopedia of Science Fiction},
	Year = {1999}}
\end{ltxcode}
\fullcite{clute.j:1999}.%\pagebreak

\begin{ltxcode}
@incollection{Wells.HG:1980,
	Author = {Wells, H. G.},
	Bookauthor = {Wells, H. G.},
	Booktitle = {H.{\,}G. Wells's Literary Criticism},
	Editor = {Philmus, Robert M. and Parrinder, Patrick},
	Location = {Sussex},
	Origdate = {1938},
	Pages = {248--249},
	Publisher = {Harvester},
	Title = {Fiction about the Future},
	Year = {1980}}
\end{ltxcode}
\fullcite{Wells.HG:1980}.

\begin{ltxcode}
@book{Kant.I:2004a,
	Author = {Kant, Immanuel},
	Editor = {Weischedel, Wilhelm},
	Location = {Frankfurt a. M.},
	Maintitle = {Werkausgabe},
	Origdate = {1790},
	Publisher = {Suhrkamp},
	Title = {Kritik der Urteilskraft},
	Volume = {10},
	Year = {1996}}
\end{ltxcode}
\fullcite{Kant.I:2004a}.%

\begin{ltxcode}
@book{Kuhn.T:1976,
	Author = {Kuhn, Thomas Samuel},
	Edition = {2., rev. u. um das Postskriptum von 1969 ergänzte Aufl.},
	Location = {Frankfurt a. M.},
	Origdate = {1962},
	Origlanguage = {english},
	Origlocation = {Chicago},
	Origpublisher = {University of Chicago Press},
	Origtitle = {The Structure of Scientific Revolutions},
	Publisher = {Suhrkamp},
	Title = {Die Struktur wissenschaftlicher Revolutionen},
	Translator = {Vetter, Hermann},
	Year = {1976}}
\end{ltxcode}
\fullcite{Kuhn.T:1976}.

\begin{ltxcode}
@incollection{Kepler.J:1993,
	Author = {Kepler, Johannes},
	Bookauthor = {Kepler, Johannes},
	Booktitle = {Calendaria et prognostica. Astronomica minora. Somnium},
	Editora = {Bialas, Volker and Grössing, Helmuth},
	Editoratype = {redactor},
	Location = {München},
	Maintitle = {Gesammelte Werke},
	Origdate = {1634},
	Origlocation = {Frankfurt a. M},
	Origtitle = {Somnium sive De Astronomia Lunari},
	Pages = {317--379},
	Part = {2},
	Publisher = {Beck},
	Title = {Somnium},
	Volume = {11},
	Year = {1993}}
\end{ltxcode}
\fullcite{Kepler.J:1993}.
\begin{ltxcode}
@incollection{sklovskij.v:1969a,
	Author = {Šklovskij, Viktor},
	Booktitle = {Texte zur allgemeinen Literaturtheorie und zur Theorie der Prosa},
	Editor = {Striedter, Jurij},
	Location = {München},
	Maintitle = {Texte der Russsischen Formalisten},
	Number = {6},
	Origdate = {1929},
	Origlanguage = {russian},
	Origlocation = {Moskau},
	Origtitle = {Iskusstvo kak priem},
	Pages = {3--35},
	Publisher = {Wilhelm Fink Verlag},
	Series = {Theorie und Geschichte der Literatur und der der schönen Künste. Texte und Abhandlungen},
	Title = {Die Kunst als Verfahren},
	Translator = {Fieguth, Rolf},
	Volume = {1},
	Year = {1969}}
\end{ltxcode}%
\fullcite{sklovskij.v:1969a}.

\begin{ltxcode}
@movie{Kubrick.S:1968,
	Author = {Kubrick, Stanley},
	Location = {GB and USA},
	Subtitle = {2001: Odyssee im Weltraum},
	Title = {2001: A Space Odyssey},
	Year = {1968}}
\end{ltxcode}
\fullcite{Kubrick.S:1968}.

\begin{ltxcode}
@movie{Lidelof.D:2004a,
	Author = {Lidelof, Damon and Cuse, Carlton and Abrams, J. J.},
	Date = {2004/2010},
	Entrysubtype = {serial},
	Location = {USA},	
	Organization = {ABC},
	Title = {Lost}}
\end{ltxcode}
\fullcite{Lidelof.D:2004a}.
\newpage
\begin{ltxcode}
@book{Oeuver.A:2010a,
	Editor = {van den Oever, Annie},
	Location = {Amsterdam},
	Number = {1},
	Publisher = {Amsterdam University Press},
	Series = {The Key Debates},
	Subtitle = {On \enquote{Strangeness} and the Moving Image. 
	The History, Reception, and Relevance of a Concept},
	Title = {Ostrannenie},
	Year = {2010}}

@review{Spiegel.S:2011b,
	Author = {Spiegel, Simon},
	Doi = {10.3167/proj.2011.050209},
	Journal = {Projections},
	Number = {2},
	Pages = {128--134},
	Related = {Oeuver.A:2010a},
	Volume = {5},
	Year = {2011}}
\end{ltxcode}
\fullcite{Spiegel.S:2011b}

\begin{ltxcode}
@book{Pearson.R:2009a,
	Editor = {Pearson, Roberta E.},
	Location = {London and New York},	
	Publisher = {I. B. Tauris},
	Subtitle = {Perspectives on a Hit Television Show},
	Title = {Reading Lost},
	Year = {2009}}

@incollection{Ndalianis.A:2009b,
	Author = {Ndalianis, Angela},
	Booksubtitle = {Perspectives on a Hit Television Show},
	Booktitle = {Reading Lost},
	Editor = {Pearson, Roberta E.},
	Location = {London and New York},
	Pages = {181--197},
	Publisher = {I. B. Tauris},
	Title = {Lost in Genre: Chasing the White Rabbit to Find a 	White Polar Bear},
	Xref = {Pearson.R:2009a},
	Year = {2009}}
\end{ltxcode}
\fullcite{Pearson.R:2009a}\\
\fullcite{Ndalianis.A:2009b}

\section{Versionsgeschichte}

\begin{changelog}
\begin{release}{1.7}{2017-11-21}
\item Setzt \sty{biblatex} ab Version 3.8 resp. \bin{biber} ab Version 2.8 voraus.
\item Anpassung an interne Änderungen von \sty{biblatex}
\item Neuer Dokumenttyp \bibtype{archival} \see{sub:zusatzdok}
\item Neue Option \opt{directorreplace} \see{opt}
\item Neue Option \opt{parensfilmnote} \see{opt}
\item Kleine Fehlerkorrekturen
\end{release}

\begin{release}{1.6c}{2017-01-28}
\item Korrektur eines dummen Tippfehlers, der den Stil unbrauchbar machte
\end{release}

\begin{release}{1.6b}{2017-01-27}
\item Bugfixes für die Option \opt{filmindex} \see{opt}
\item Neue Option \opt{splitfilmindex} zum getrennten Indexieren von Filmtiteln \see{opt}
\item Bessere Implementierung der Option \opt{xindy} \see{opt}
\item Neues Unterkapitel zur Indexierung von Filmen \see{sub:indexfilm}
\end{release}


\begin{release}{1.6a}{2017-01-20}
\item Die Option \opt{citefilm} kennt nun die Variante \opt{country}, bei der Produktionsland und -jahr ausgegeben werden \see{opt}
\item Ist kein Regisseur definiert, wird nun die Produktionsfirma angegeben oder »[K.\,A.]« ausgegeben \see{sub:filme}
\item Neue Option \opt{compactcite} zur kompakten Ausgabe von mehreren Zitaten des gleichen Autors \see{opt}
\item Neue Option\opt{titleindex} zum Indexieren von Werktiteln \see{opt}
\item Neue Option \opt{filmindex} zum Indexieren von Filmtiteln \see{opt}
\item Bei Pseudonymen werden nun Pseudonym und Realname indexiert \see{sub:pseudo}
\end{release}

\begin{release}{1.6}{2016-09-09}
\item Setzt \sty{biblatex} ab Version 3.5 resp, \bin{biber} ab Version 2.6 voraus.
\item Unterstützung der neuen Datumsfunktionen von \sty{biblatex} (Uhrzeiten werden derzeit nicht unterstützt, da diese Funktion kaum praktische Relevanz besitzt)
\item Unterstützung von Pseudonymen \see{sub:pseudo}
\item Neue Option \opt{origcite}, welche die Ausgabe des Originaljahrs im Lauftext erlaubt \see{opt}
\item Aufräumen und Vereinheitlichung bei der Ausgabe des Jahrs der Erstveröffentlichung
\end{release}

\begin{release}{1.5}{2016-05-17}
\item Setzt \sty{biblatex} ab Version 3.4 resp. \bin{biber} ab Version 2.5 voraus.
\item Anpassung an interne änderungen von \sty{biblatex}
\item Neue Felder \bibfield{alternatitle} und \bibfield{alternatetitlescript} \see{alternatetitle}
\item Beim Zitieren von zwei Filmen mit identischem Titel wird immer das Jahr ausgegeben \see{zitbefehl}
\item Beim Indexieren von zwei Filmen mit identischem Titel wird immer das Jahr ausgegeben \see{zitbefehl}
\item Es wird nun überprüft, ob das Paket \sty{ragged2e} vorhanden ist
\item Bei \cmd{multicitedelim} wird nun standardmäßig immer ein Komma ausgegeben
\item Die Dokumentation wird nun mit \luatex gesetzt
\end{release}

\begin{release}{1.4}{2016-03-09}
\item Setzt \sty{biblatex} ab Version 3.4 resp. \bin{biber} ab Version 2.4 voraus.
\item Anpassung an interne änderungen von \sty{biblatex} im Umgang mit Namen
\item Verbesserte Ausgabe von \cmd{footcite}-Befehlen
\item Korrigiert mehrere kleine Fehler
\end{release}

\begin{release}{1.3}{2015-06-30}
\item Setzt \sty{biblatex} ab Version 3.0. resp. \bin{biber} ab Version 2.0 voraus.
\item Korrigiert mehrere kleine Fehler
\end{release}

\begin{release}{1.2e}{2015-01-10}
\item Neue Option \opt{ibidpage} \see{opt}
\item Neue Option \opt{mergedate} \see{opt}
\item Neue Option \opt{parensvolume} \see{opt}
\item \bibtype{review} stellt nun auch  Artikel in Büchern korrekt dar
\item Bei fehlender Datumsangabe wird  der der Bibstring \texttt{nodate} statt \texttt{undated} verwendet
\item Interne Reorganisation
\item Korrigiert mehrere kleine Fehler
\end{release}

\begin{release}{1.2d}{2013-05-13}
\item Setzt \sty{biblatex} ab Version 2.6. resp \bin{biber} ab Version 1.6 voraus.
\item Korrigiert mehrere kleine Fehler
\end{release}

\begin{release}{1.2c}{2013-03-29}
\item Neue Option \opt{isbn} \see{opt}
\item Korrigiert einen dummen Fehler mit dem den Dokumenttyp \bibtype{misc}
\end{release}

\begin{release}{1.2b}{2012-12-09}
\item Korrigiert zwei dumme Fehler der vorhergehen Version
\end{release}

\begin{release}{1.2a}{2012-12-03}
\item Neue Optionen \opt{ignoreforeword}, \opt{ignoreafterword}, \opt{ignoreintroduction} und \opt{ignoreparatext} (Anregung von Rolf Niepraschk) \see{opt}
\item Dazu passende Befehle für Wechsel innerhalb des Dokuments \see{opt}
\item Neues Feld \bibfield{parttitle} \see{sub:zusatzfeld}
\item Leicht geänderte Ausgabe des Typs \bibtype{review}
\item Kleine Fehlerkorrekturen (Dank an Rolf Niepraschk)
\end{release}

\begin{release}{1.2}{2012-07-20}
\item Unterstützung der \bibfield{related}-Funktionen, die \sty{biblatex} seit Version 2.0 anbietet:
\subitem übersetzungen und Neuauflagen \see{sub:uebersetzungen}
\subitem Anpassung der Option \opt{partofcited} \see{sub:kurz}
\subitem Anpassung des Typs \bibtype{review} \see{sub:rezis}
\subitem Option \opt{germ}  überarbeitet und in \opt{translatedas} umbenannt \see{sub:germ}
\subitem In allen Fällen ist nun nur noch ein Durchlauf von \bin{biber} nötig
\item Neuer \bibfield{entrysubtype} für Fernsehsendungen \see{movie:tv}
\item \cmd{citefilm} gibt bei Serien und Fernsehendungen nun \bibfield{maintitle} aus
\item Neuer Befehl \cmd{citeepisode} \see{sub:serien}
\item Sortierschema \opt{title} hinzugefügt \see{sub:sort}
\item Option \opt{ignorearticle} hinzugefügt \see{sub:sort}
\item Bei fehlender Datumsangabe wird nun der der Bibstring \texttt{undated} ausgegeben
\item Weitere Detailkorrekturen
\end{release}

\begin{release}{1.1e}{2011-12-16} 
\item Neuer Befehl \cmd{inparencite} (Anregung von Christian Erll) \see{zitbefehl}
\item Unterstützung für den Dokumenttyp \bibtype{periodical} \see{supp:types}
\item Neue Paketoption \opt{citefilm} \see{opt}
\item Neue Paketoption \opt{pages} \see{opt}
\item Zusätzliche Localization Strings \bibfield{page} und \bibfield{pages} \see{sub:besond}
\item Die Option \opt{origyearwithyear=brackets} hinzugefügt \see{opt}
\item Unterstützung für das Feld \bibfield{nameaddon} \see{sub:zusatzfeld}
\item Mehrere Fehler bei der Option \opt{dashed} korrigiert
\item Fehlende Seitenzahlen bei der Option \opt{partofcited} ergänzt

\item Die Ausgabeform für \bibtype{article} wurde leicht geändert
\end{release}

\begin{release}{1.1d}{2011-12-13}
\item Separate Beispiele mit unterschiedlichen Einstellungen hinzugefügt
\item Viel Detail"=Aufräumarbeiten, zahlreiche kleine Unregelmäßigkeiten korrigiert
\item Zukünftige Versionen sollten dank eines neuen Testverfahrens seltener neue Fehler einführen
\end{release}

\begin{release}{1.1c}{2011-12-06}
\item Die Option \opt{partofcited hinzugeführt} \see{sub:kurz}
\item Das Format des Beispiels für \file{biber.conf} angepasst \see{sub:sort}
\end{release}

\begin{release}{1.1b}{2011-11-22}
\item Die Option \opt{origyearsuperscript} hinzugefügt \see{opt}
\item Die Option \opt{origyearwithyear} hinzugefügt \see{opt}
\item Die Option \opt{yearatbeginning} hinzugefügt, entspricht dem Stil \sty{fiwi2} \see{opt}
\item Die Datei \file{fiwi2.cbx} hinzugefügt, ermöglicht den Aufruf \opt{style=fiwi2}
\item Kleine Fehlerkorrekturen (Dank an Dominik Waßenhoven)
\end{release}

\begin{release}{1.1a}{2011-11-18}
\item Kleine Fehlerkorrekturen (Dank an j mach wust)
\end{release}

\begin{release}{1.1}{2011-11-15}
\item Erste Veröffentlichung
\end{release}

\end{changelog}
%\printbibliography
\end{document}

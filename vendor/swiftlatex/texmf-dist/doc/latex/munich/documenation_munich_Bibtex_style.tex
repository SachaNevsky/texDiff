
\documentclass[12pt,a4paper]{article}
\usepackage[latin1]{inputenc}
\usepackage[T1]{fontenc}
\usepackage[ngerman]{babel}
\usepackage{url}
\usepackage{parskip}

\title{munich.bst -- A Name (Year) Bibtex Style \newline --Documentation--}

\author{Martin Brixner}
\date{10. August 2006}
\begin{document}

\maketitle%
\newpage%
\section{Introduction}
Munich.bst was created as a Bibtex style that complements the existing Bibtex styles for name
(year) citations. In most cases the popular Harvard Bibtex style
(\url{http://www.tug.org/tex-archive/help/Catalogue/entries/harvard-latex.html}) or the Oxford
Bibtex style (\url{http://www.tug.org/tex-archive/help/Catalogue/entries/oxford.html}) lead to good
results. Yet I found them not entirely suitable to create a bibliography for a German/continental
Europe academic paper. Namely three aspects were not satisfactory:
%
\begin{itemize}
\item Multiple authors are to be separated with a slash only; no additional spaces should be
inserted. %
\item With the existing styles the city, in which the work was published, is omitted or not
positioned at the end of the biography entry.%
\item The existing styles use punctuation and italics/bold face in rather untypical way for a
German/continental Europe bibliography.
\end{itemize}
%
\section{Description}
Munich.bst is based on the result of a custom-bib
(\url{http://www.tug.org/tex-archive/help/Catalogue/entries/custom-bib.html}) query. Various
changes have been made to the resulting Bibtex style. Munich.bst is designed to work with the natbib package.
(\url{http://www.tug.org/tex-archive/help/Catalogue/entries/natbib.html})\\
For further reference you should find in the same directory where you found this document a .tex
file example\_munich\_bibtex\_style.tex and the respective .pdf document
example\_munich\_bibtex\_style.pdf. These files will give you an idea how to work with munich.bst
and what to expect from using this Bibtex style. It also contains a set of new commands that work
well in conjunction with natbib and offer new citation formats. In particular, citations with a
page reference are not completely put between brackets but only the page reference itself. Please
note that the ifthen package
(\url{http://www.tug.org/tex-archive/help/Catalogue/entries/ifthen.html}) must be installed to use
these customized commands.

\section{Contact}
Feel free to use munich.bst and the customized commands with your LaTex document. For comments you
can contact me at: \url{brixner_m@yahoo.de}.

Please note that I will not be able to offer support for your particular document or to debug your
biography.

I hope you enjoy using munich.bst.\\[0.5cm]%
%
\hspace{1cm}M.\,Brixner\\%
%
\section{Reference -- Customized commands}
\begin{verbatim}
% Customized commands*****
% Additional commands, not needed but they come in
% handy if you want to include page numbers and do
% not want the whole entry to be put between brackets.
%*************************
\newcommand{\ct}[2][]
{%
\citet{#2}%
\ifthenelse{\equal{}{#1}}{.}{, S.\,#1.}%
}
%*************************
\newcommand{\fct}[2][]
{%
\footnote{
\citet{#2}%
\ifthenelse{\equal{}{#1}}{.}{, S.\,#1.}%
}}
%*************************
\newcommand{\fctv}[2][]
{%
\footnote{Vgl.\ %
\citet{#2}%
\ifthenelse{\equal{}{#1}}{.}{, S.\,#1.}%
}}
%*************************
\end{verbatim}
\end{document}

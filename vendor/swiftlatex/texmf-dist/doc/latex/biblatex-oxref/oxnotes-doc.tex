%%
%% This is file `oxnotes-doc.tex',
%% generated with the docstrip utility.
%%
%% The original source files were:
%%
%% oxref.dtx  (with options: `doc,n')
%% ----------------------------------------------------------------
%% biblatex-oxref --- Biblatex styles inspired by the Oxford Guide to Style
%% Author:  Alex Ball
%% E-mail:  a.j.ball@bath.ac.uk
%% License: Released under the LaTeX Project Public License v1.3c or later
%% See:     http://www.latex-project.org/lppl.txt
%% ----------------------------------------------------------------
%% 
\def\Version{2020/01/27 v2.0.1}
\ProvidesFile{oxnotes-doc.tex}
    [\Version\space Footnote-based biblatex style inspired by the Oxford Guide to Style]
\PassOptionsToPackage{style=oxnotes,scnames,varissuedate,anon}{biblatex}
\documentclass[extrafontsizes,11pt,a4paper,oneside]{memoir}
\setlrmarginsandblock{3cm}{3cm}{*}
\setulmarginsandblock{2.5cm}{2.5cm}{*}
\checkandfixthelayout
\chapterstyle{ell}
\renewcommand{\prechapterprecis}{%
  \vspace*{\prechapterprecisshift}%
  \begin{flushright}\precisfont
}
\renewcommand{\postchapterprecis}{\end{flushright}}
\hangsecnum
\setsecheadstyle{\Large\bfseries\raggedright}
\setsubsecheadstyle{\large\bfseries\scshape\raggedright}
\setsecnumdepth{subsection}
\nouppercaseheads
\makeoddhead{myheadings}{\textsc{\leftmark}}{}{\thepage}
\makepsmarks{myheadings}{%
  \def\chaptermark##1{\markboth{##1}{##1}}%
  \def\sectionmark##1{\markright{##1}}%
}
\pagestyle{myheadings}
\aliaspagestyle{title}{empty}
\setlength{\parindent}{0pt}\nonzeroparskip
\firmlists

\usepackage[british]{babel}
\usepackage[mono=false,defaultfeatures={SmallCapsFeatures={Letters=SmallCaps,Renderer=Basic,Ligatures=NoCommon}}]{libertine}
\usepackage{fontawesome}[2015/07/07]
\newcommand{\booksym}{\makebox[1em][c]{\faicon{book}}}
\newcommand{\cogsym}{\makebox[1em][c]{\faicon{cog}}}
\makeatletter
\@ifpackageloaded{fontspec}{%
  \setmonofont[Scale=MatchLowercase,StylisticSet=1,AutoFakeSlant]{Inconsolatazi4}
}{%
  \usepackage[utf8]{inputenc}
  \usepackage[varl]{zi4}
}
\makeatother

\usepackage{xcolor,xparse}
\definecolor{Green}{rgb}{0,.5,0}
\colorlet{ok}{Green}
\colorlet{todo}{red}
\colorlet{hacked}{orange}
\colorlet{manual}{purple}

\usepackage{tcolorbox}
\tcbuselibrary{skins,xparse,documentation,breakable,minted}
\colorlet{Option}{violet}
\colorlet{Command}{red!75!black}
\colorlet{Environment}{blue!75!black}
\colorlet{Value}{olive!75!black}
\colorlet{Color}{cyan!75!black}
\tcbset
  { enhanced
  , listing engine=minted
  , minted options=
    { breaklines
    , fontsize=\footnotesize
    }
  , index format=pgf
  , color command=Command
  , color environment=Environment
  , color key=Option
  , color value=Value
  , color color=Color
  }
\renewcommand{\theFancyVerbLine}{\footnotesize\itshape\color{gray}\arabic{FancyVerbLine}}
\let\tcbcs=\cs
\renewcommand*{\cs}[1]{\textcolor{Command}{\tcbcs{#1}}}
\def\sqbrackets#1{%
  \texttt{\textcolor{Option}{[}#1\textcolor{Option}{]}}}
\def\brackets#1{%
  \texttt{\textcolor{Environment}{\char`\{}#1\textcolor{Environment}{\char`\}}}}
\def\marg#1{%
  \textcolor{Environment}{\ttfamily\char`\{}\meta{#1}\textcolor{Environment}{\ttfamily\char`\}}}
\newcommand*{\env}[1]{\textcolor{Environment}{\ttfamily #1}}
\newcommand*{\key}[1]{\textcolor{Option}{\ttfamily #1}}
\newcommand*{\val}[1]{\textcolor{Value}{\ttfamily #1}}
\makeatletter
\newcommand{\resetmintedformat}{%
  % Comments
  \expandafter\def\csname PYGdefault@tok@c\endcsname{\let\PYGdefault@it=\textit\def\PYGdefault@tc####1{\textcolor{gray}{####1}}}
  % Command sequences
  \expandafter\def\csname PYGdefault@tok@k\endcsname{\def\PYGdefault@tc####1{\textcolor{Command}{####1}}}
  % Optional arguments
  \expandafter\def\csname PYGdefault@tok@na\endcsname{\def\PYGdefault@tc####1{\textcolor{Option}{####1}}}
  % Braces
  \expandafter\def\csname PYGdefault@tok@nb\endcsname{\def\PYGdefault@tc####1{\textcolor{Environment}{####1}}}
}
\apptocmd{\minted@checkstyle}{\resetmintedformat}{}{}
\makeatother

\RecustomVerbatimEnvironment
  {Verbatim}{Verbatim}
  {commentchar=\%}

\usepackage{xpatch,csquotes,multicol}

\usepackage
[backend=biber%
,hyperref=false%
]{biblatex}
\addbibresource{oxref.bib}
\setcounter{biburllcpenalty}{900}
\setcounter{biburlucpenalty}{800}
\setcounter{biburlnumpenalty}{700}
\DeclareBibliographyCategory{hidden}
\makeatletter
\DeclareCiteCommand{\fullcite}
  {\usebibmacro{prenote}}
  {\usedriver
     {\defcounter{maxnames}{\blx@maxbibnames}}
     {\thefield{entrytype}}}
  {\multicitedelim}
  {\usebibmacro{postnote}\finentrypunct}
\makeatother
\DeclareSourcemap{%
  \maps[datatype=bibtex]{%
    \map[overwrite=true]{%
      \pernottype{legslation}
      \pernottype{legal}
      \step[fieldset=shorthand, null=true]
    }%
  }%
}

\hypersetup{pdfborder={0 0 0},pdfencoding=auto}
\usepackage[noabbrev,capitalize,nameinlink]{cleveref}
\crefname{page}{page}{pages}

\NewDocumentCommand{\pkg}{om}{%
  \IfNoValueTF{#1}
    {\lowercase{\href{http://www.ctan.org/pkg/#2}}{\textsf{#2}}}
    {\lowercase{\href{http://www.ctan.org/pkg/#1-#2}}{\textsf{#2}}}%
}
\newcommand*{\lit}[1]{\textsf{#1}}
\newcommand*{\code}[1]{\texttt{#1}}
\newcommand*{\aside}[1]{\textcolor{violet}{[\textsc{tip:} #1]}}
\newcommand{\tip}[1]{\hangfrom{\makebox[2em][c]{\faLightbulbO}}#1\par}
\newcommand{\info}[1]{\hangfrom{\makebox[2em][c]{\faInfoCircle}}#1\par}
\newcommand{\hack}[1]{\hangfrom{\makebox[2em][c]{\faWrench}}#1\par}

\makeatletter
\def\CiteStatus{todo}
\newcommand{\dbgcolor}[2]{%
  \bgroup
  \blx@citecmdinit
  \blx@citeinit
  \def\blx@precode{}%
  \def\blx@postcode{}%
  \def\blx@loopcode{%
    \iffieldundef{userd}
    {\xdef\CiteStatus{ok}}
    {\xdef\CiteStatus{\abx@field@userd}}}%
  \blx@citeloop{#1}%
  \textcolor{\CiteStatus}{#2}%
  \egroup
}
\makeatother
\NewTColorBox{bibexbox}{D(){ok}d<>omo}%
  {bicolor
  ,colframe = #1
  ,colback = #1!5!white
  ,colbacklower = white
  ,fontlower = \footnotesize
  ,before upper = {\hangfrom{\booksym\space}\biburlsetup}
  ,IfNoValueTF={#3}%
    {after upper = {\par\hangfrom{\cogsym\space}\fullcite{#4}}
    }%
    {after upper = {\par\hangfrom{\cogsym\space}\fullcite[#3]{#4}}
    ,title = {\texttt{\string\fullcite[#3]\{#4\}}}
    }
  ,IfNoValueTF={#2}{}%
    {overlay = {
      \node[anchor=south east,text=teal] at (frame.south east) {#2};
      }
    }
  ,phantomlabel={ex:#4}
  ,IfNoValueTF={#5}{}{#5}
  }
\NewTotalTColorBox{\spec}{m}%
  {enhanced
  ,sharp corners = west
  ,colframe = teal
  ,colback = teal!5!white
  ,toprule = 0pt
  ,bottomrule = 0pt
  ,rightrule = 0pt
  }{#1}
\NewTCBListing{egcite}{D(){ok} o m !o}%
  {colframe = #1
  ,colback = #1!5!white
  ,listing side text
  ,lefthand width = 14em
  ,IfNoValueTF={#2}{}{title = #2}
  ,before lower = {\raggedright\ifblank{#3}{}{\hangfrom{\booksym\space}#3\par}\hangfrom{\cogsym\space}}
  ,IfNoValueTF={#4}{}{#4}}
\NewTCBListing{egcite*}{D(){ok} o m !o}%
  {colframe = #1
  ,colback = #1!5!white
  ,IfNoValueTF={#2}{}{title = #2}
  ,before lower = {\raggedright\ifblank{#3}{}{\hangfrom{\booksym\space}#3\par}\hangfrom{\cogsym\space}}
  ,IfNoValueTF={#4}{}{#4}}

\frenchspacing

\usepackage[british]{isodate}
\usepackage{readprov}
\GetFileInfo{\jobname.tex}
\title{OXNOTES -- A notes-based style for Biblatex}
\author{Alex Ball}
\date{\printdateTeX{\filedate}}
\begin{document}
\thispagestyle{empty}
\begin{adjustwidth}{.2\textwidth}{0pt}
  \sffamily\setlength{\parindent}{0pt}%
  \LARGE\textsc{oxref bundle}

  \vspace{\stretch{1}}
  \LARGE\thetitle

  \bigskip
  \Large\theauthor

  \bigskip
  \Large\thedate

  \bigskip
  \Large\fileversion
\end{adjustwidth}

\vspace{\stretch{3}}
\noindent
\hspace*{.1\textwidth}\raisebox{0pt}[0pt][0pt]{\rule{\normalrulethickness}{\textheight}}

\newpage
\tableofcontents*

\chapter{Introduction}

\section{Loading the style}\label{sec:loading}

The style is self-contained, so you can load it with \pkg{biblatex}:
\begin{tcblisting}{listing only}
\usepackage[style=oxnotes]{biblatex}
\end{tcblisting}

The style has some options additional to the regular \pkg{biblatex} ones:

\begin{docKey}{anon}{=\val{literal}|\val{long}|\val{short}}{default \val{short}, initially \val{literal}}
  Affects what happens if the author name matches the value of \cs{oxrefanon}.
  By default, this is \enquote{Anonymous}, but you could change it a different word (such as \enquote{Anonimo}) instead.
  \begin{itemize}
  \item\docValue{literal}
    means no special handling is used.
  \item\docValue{long}
    will print the unabbreviated localization string \code{anon} (\enquote{Anonymous}) instead of the author name
    in the bibliography, but neither are printed in citations.
  \item\docValue{short}
    will print the abbreviated localization string \code{anon} (\enquote{Anon.\@}) instead of the author name
    in the bibliography, but neither are printed in citations.
  \end{itemize}
\end{docKey}

\begin{docKey}{bookseries}{=\val{in}|\val{out}}{default \val{in}, initially \val{in}}
  Puts the series information for a book \docValue{in}side or \docValue{out}side the parenthetical publication block.
\end{docKey}

\begin{docKey}{court-plain}{=\val{true}|\val{false}}{default \val{true}, initially \val{false}}
  Prints courts of decision without parentheses.
\end{docKey}

\begin{docKey}{dashed}{=\val{true}|\val{false}}{default \val{true}, initially \val{false}}
  In the biblography, replaces recurring author\slash editor names with a dash.
\end{docKey}

\begin{docKey}{ecli}{=\val{yes}|\val{only}|\val{no}}{default \val{yes}, initially \val{yes}}
  Determines when ECLI numbers for EU legal cases are printed (if provided).
  \begin{itemize}
  \item\docValue{yes}
    prints the ECLI number in addition to the official report.
  \item\docValue{only}
    prints the ECLI number instead of the official report.
  \item\docValue{no}
    only prints the ECLI number if the case is otherwise unreported.
  \end{itemize}
\end{docKey}

\begin{docKey}{isourls}{=\val{true}|\val{false}}{default \val{true}, initially \val{false}}
  Surrounds URLs with angle brackets.
\end{docKey}

\begin{docKey}{issuedate-plain}{=\val{true}|\val{false}}{default \val{true}, initially \val{false}}
  Removes the parentheses around the date of a periodical without a volume number.
  This can also be set on a per-type and per-entry basis.
  (This option was previously called \key{varissuedate}.)
\end{docKey}

\begin{docKey}{issuestyle}{=\val{slash}|\val{colon}|\val{comma}|\val{parens}}{default \val{slash}, initially \val{slash}}
  Affects how journal volumes and numbers are printed.
  \begin{itemize}
  \item\docValue{slash}
    separates the two with a solidus, e.g. \enquote{23/2}.
  \item\docValue{colon}
    separates the two with a colon and space, e.g. \enquote{23: 2}.
  \item\docValue{comma}
    separates the two with a comma and space, e.g. \enquote{23, 2}.
  \item\docValue{parens}
    sets off the issue number in parentheses, e.g. \enquote{23 (2)}.
    It is intended for use with \textsf{oxalph}\slash\textsf{oxyear} and not recommended for this style.
  \end{itemize}
\end{docKey}


\begin{docKey}{nolocation}{}{no value, initially unset}
  Replaces missing locations with \enquote{n.p.} or the localized equivalent in books, collections, reference works, proceedings, and similar entry types. Once set, this option cannot be overridden.
  Alternatively, it may be set on a per-entry basis.
\end{docKey}


\begin{docKey}{nopublisher}{}{no value, initially unset}
  Removes publisher name from all entries. Once set, this option cannot be overridden.
\end{docKey}

\begin{docKey}{relationpunct}{=\val{period}|\val{comma}|\val{semicolon}|\val{colon}|\val{space}}{default \val{semicolon}, initially \val{semicolon}}
  Sets the punctuation that precedes the \code{relatedtype} localization string.
  An additional space is assumed unless the value is \val{space}.
  This can also be set on a per-type and per-entry basis.
  Note that the general and per-type settings are ignored for some values of \code{relatedtype},
  but the per-entry setting is always effective.
\end{docKey}

\begin{docKey}{scnames}{=\val{true}|\val{false}}{default \val{true}, initially \val{false}}
  Prints initial author or editor names in bibliography entries in small capitals.
\end{docKey}

\begin{docKey}{thesis}{=\val{in}|\val{out}|\val{plain}}{default \val{out}, initially \val{out}}
  Puts the thesis type \docValue{in}side or \docValue{out}side the parenthetical publication block,
  while \docValue{plain} removes the parentheses entirely.
\end{docKey}

\begin{docKey}{timefirst}{=\val{true}|\val{false}}{default \val{true}, initially \val{false}}
  Prints the time (if provided) before the date instead of after it.
  This can also be set on a per-type and per-entry basis.
\end{docKey}

\begin{docKey}{usenametitles}{=\val{true}|\val{false}}{default \val{true}, initially \val{true}}
  Controls whether titular prefixes such as Revd, Dr, Mrs, Sir, and so on are printed.
  (See \cref{sec:nametitles} for how to supply titles as part of a name.)
  This can also be set on a per-type and per-entry basis.
\end{docKey}

\pkg[biblatex]{Oxref} makes use of Biber-specific techniques to solve some of the challenges presented by Oxford style.
It will not stop you from using Bib\TeX\ instead but, if you do, only simple and standard entries will work.
In particular, manuscripts, legal references and anything involving related entries will be adversely affected.

\section{How to use this document}

Bibliographical items are given throughout this document, and serve three purposes:

\begin{itemize}
  \item
  To demonstrate which of the variations discussed by the
  \emph{Oxford Guide to Style} and \emph{New Hart's Rules} have been chosen
  in this style.
  \item
  To help me, as package author, check that the style is working as intended.
  \item
  To show you, as document author, how to use the style to get the effect you want.
\end{itemize}

Examples that follow standard \textsf{biblatex} semantics, and are therefore
(mostly) portable to other styles, are in green:

\begin{tcolorbox}%
  [bicolor
  ,colframe = ok
  ,colback = ok!5!white
  ,colbacklower = white
  ,fontlower = \footnotesize\ttfamily
  ,overlay = {\node[anchor=south east,text=teal] at (frame.south east) {Source};}
  ]
  \hangfrom{\booksym\space} Reference text as it should look.\par
  \hangfrom{\cogsym\space} Reference text as generated by \textsf{biblatex}.
  \tcblower
  Code used in bibliography file.
\end{tcolorbox}

Examples that have been \enquote{hacked} in some way, and are therefore not portable
to other styles, are in amber:

\begin{tcolorbox}%
  [bicolor
  ,colframe = hacked
  ,colback = hacked!5!white
  ,colbacklower = white
  ,fontlower = \footnotesize\ttfamily
  ,overlay = {\node[anchor=south east,text=teal] at (frame.south east) {Source};}
  ]
  \hangfrom{\booksym\space} Reference text as it should look.\par
  \hangfrom{\cogsym\space} Reference text as generated by \textsf{biblatex}.
  \tcblower
  Code used in bibliography file.
\end{tcolorbox}

Where a source is provided, it refers to a section from one of the reference works below:

\begin{description}
\item[OGS]
\fullcite{ritter2002ogs}

\item[NHR]
\fullcite{waddingham2014nhr}
\end{description}

Where the source is starred (*), this indicates the example is not quite how it appears in the book,
usually because the original is demonstrating an option that is not the \pkg[biblatex]{oxref} default.

\section{Design philosophy}

The stipulations given by the \emph{Oxford Guide to Style} regarding citations and references
amount not so much to a consistent style as a body of advice for creating one.
Unlike the style manuals published by the American Psychological Association,
the Modern Language Association, and the University of Chicago to name but three,
the emphasis of the Oxford guide is to describe good practice rather than prescribe a particular style.
This means that when it comes to \pkg[biblatex]{oxref}, there are decisions to be made
as to which variations to support by default, which to support as options,
and which to ignore quietly.

The situation is complicated further by the nature of the available versions.
The 2002 guide remains the most comprehensive in terms of rules, principles and examples.
The succeeding versions, under the title \emph{New Hart's Rules},
update the aspects of the 2002 guide that now seem somewhat dated,
such as its handling of URLs and DOIs,
and have more of an eye on machine processing of bibliographies.
They do, however, introduce additional variations with less of a steer on what is preferred,
and are considerably shorter with fewer examples.

The approach of \pkg[biblatex]{oxref} is to follow the 2014 \emph{New Hart's Rules} as much as possible,
but where variations are given without strong preference, or where guidance is lacking,
to follow the preferences of the 2002 guide.
Where the practices of the humanities and the sciences are contrasted,
the former are followed for \textsf{oxnotes}\slash\textsf{oxnum}
and the latter for \textsf{oxalph}\slash\textsf{oxyear}.
Where neither version gives explicit guidance on citing a resource supported by \textsf{biblatex},
\pkg[biblatex]{oxref} extrapolates from what is provided, guided by standard \textsf{biblatex}
and other major referencing styles.

Where it is practical to do so,
ways and means of achieving the variations defined by the two guides are provided,
but by no means all of them will be supported.

\section{Technical documentation}

For information on installing the styles, and for the documented source code,
see the separate documentation file \href{./oxref.pdf}{\texttt{oxref.pdf}}

\section{Stability}

The \pkg[biblatex]{oxref} family of styles is currently at \fileversion.

I have no plans to change the default (expected) behaviour of the styles,
and will do my best to avoid backwards-incompatible changes, though
alternative behaviour may be added. If changes are introduced that alter
the advertised output, this will be signalled by a change in major
version number.

Please report any bugs you discover on the
\href{https://github.com/alex-ball/biblatex-oxref/issues}{GitHub issue tracker}.%
\footnote{Issue tracker for \pkg[biblatex]{oxref}:
\url{https://github.com/alex-ball/biblatex-oxref/issues}}
You are also welcome to leave your thoughts there on how the styles could
be improved, especially for the cases not covered by the two style manuals.

\chapter{Citations and common formatting}\label{sec:citing}

The \textsf{oxnotes} bibliography style is intended for use with a footnote or endnote citation style,
and indeed loads a standard \textsf{biblatex} one.
While the note text generated by the style is very similar to that which appears in the bibliography,
there is a difference in how author names are printed.

\section{Test citations}

Following the advice of \emph{New Hart's Rules}, \textsf{oxnotes} loads by default a tweaked version of the  standard \textsf{verbose} citation style, in which repeated citations are abbreviated to the author surname and a short title.

\begin{egcite}{}
Test\footcite{faith1997epg}
Test\footcite[49-50]{faith1997epg}
\end{egcite}
\begin{egcite}{}
\cites[80 (Westminster), 66
  (Glastonbury), 149 (Osney), 128
  (Bolton)]{knowles.hadcock1953mrh}%
  [186]{kershaw1973bp}[609]{cobban}
\end{egcite}
\addtocategory{hidden}{cobban}

Both the \emph{Oxford Guide to Style} and \emph{New Hart's Rules} describe an alternative system that uses the abbreviations \enquote{ibid.\@}, \enquote{op.\@ cit.\@}, \enquote{loc.\@ cit.\@}, and \enquote{id.\@} and friends, though with underwhelming enthusiasm. If you would like to use these abbreviations, you can use the \textsf{oxnotes-trad1} citation style instead (and remember to use the \code{gender} field). As with the standard styles, a bibliographic style alias has been provided so you can just do this:

\begin{tcblisting}{listing only}
\usepackage[style=oxnotes-trad1]{biblatex}
\end{tcblisting}

If you are particularly keen to use the other variants that \pkg{biblatex} provides, then parallel \textsf{oxnotes} variants have been provided for your convenience: \textsf{oxnotes-ibid}, \textsf{oxnotes-note}, \textsf{oxnotes-inote}, \textsf{oxnotes-trad2}, and \textsf{oxnotes-trad3}.


\section{Missing or inferred attribution}

\subsection{Missing attribution}
Where works have no attribution, both the \emph{Oxford Guide to Style} and \emph{New Hart's Rules} suggest printing them without further adornment in notes, but listing them under \enquote{Anonymous} (or \enquote{Anon.\@} in the latter case) in the bibliography.

\tip{\pkg[biblatex]{Oxref} will not automatically generate anonymous labels for you, but if you use the \key{anon} option (see \cref{sec:loading}), you can transform an author name of \enquote{Anonymous} (or whatever \cs{oxrefanon} is set to) in your .bib file to either the long or short localization string \code{anon}.
  Doing this will automatically suppress the printing of the anonymous author in notes.
}

\begin{egcite}{}
\cite{anon1822san}.
\textcite{anon1956lu}.
\end{egcite}

\begin{bibexbox}
<OGS \S15.2.1>
{anon1822san}
\textsc{Anon.}, \emph{Stories after Nature} (London: Allman, 1822).
\toggletrue{blx@bibliography}
\tcblower
\begin{Verbatim}
@book{anon1822san,
  author = {Anonymous},
  title = {Stories after Nature},
  location = {London},
  publisher = {Allman},
  date = {1822}}
\end{Verbatim}
\end{bibexbox}

\begin{bibexbox}
<OGS \S15.2.1>
{anon1956lu}
\textsc{Anon.}, \emph{Liber usualis} (Tournai: Desclée, 1956).
\toggletrue{blx@bibliography}
\tcblower
\begin{Verbatim}
@book{anon1956lu,
  author = {Anonymous},
  title = {Liber usualis},
  location = {Tournai},
  publisher = {Desclée},
  year = {1956}}
\end{Verbatim}
\end{bibexbox}


\subsection{Pseudepigraphy}
Some older works are known to have been written pseudepigraphically, that is, falsely attributed to a more famous author. The way of indicating this in Oxford style, if desired, is
to add the prefix \enquote{Pseudo-} before the name in notes, and
to put \enquote{(Ps.-)} after the name in the bibliography.

\tip{With \pkg[biblatex]{oxref}, you can achieve this by annotating the name with the keyword \code{pseudo}.
Since this mechanism is aimed at ancient texts, it has only been designed to work with single-part names
(i.e. either a single word, or a whole name wrapped in braces).
}

\begin{egcite}{}
\textcite{boethius1976dds}
\end{egcite}

\begin{bibexbox}
<OGS \S15.2.1>
{boethius1976dds}
\textsc{Boethius (Ps.-)}, \emph{De disciplina scolarium: Édition critique, introduction et notes}, ed. Olga Weijers (Leiden, 1976).
\toggletrue{blx@bibliography}
\tcblower
\begin{Verbatim}
@book{boethius1976dds,
  author = {Boethius},
  author+an = {1=pseudo},
  title = {De disciplina scolarium},
  subtitle = {Édition critique, introduction et notes},
  editor = {Olga Weijers},
  location = {Leiden},
  date = {1976}}
\end{Verbatim}
\end{bibexbox}

\subsection{Inferred attribution}

If the attribution is missing from the work but may be inferred from other sources,
Oxford style is to give the attribution in square brackets.

\tip{With \pkg[biblatex]{oxref}, you can achieve this by annotating the whole name field
  (usually \code{author} or \code{editor}) with the keyword \code{inferred}.
  You can also annotate names individually within the list, if only some of them should be taken as inferred.}

\begin{bibexbox}
<OGS \S15.2.1>
{balfour1768pe}
{[James Balfour]}, \emph{Philosophical Essays} (Edinburgh, 1768).
\tcblower
\begin{Verbatim}
@book{balfour1768pe,
  author = {James Balfour},
  author+an = {=inferred},
  title = {Philosophical Essays},
  location = {Edinburgh},
  date = {1768}}
\end{Verbatim}
\end{bibexbox}

\tip{You can also use the syntax from \textsf{biblatex-realauthor}.
  If you do not specify the \code{author}, then \code{realauthor} is treated as an alias for \code{author} annotated with the keyword \code{inferred}.
  The equivalent is true for \code{realeditor}.
  Note, however, that \pkg[biblatex]{oxref} does not recognize the \key{userealauthor} and \key{userealeditor} options.}

\subsection{Pseudonyms}

If an author publishes under a pen name, and you want to link the names in the bibliography,
the name as given in the work should be given first, immediately followed by the other name
in parentheses (\emph{Oxford Guide to Style}) or brackets (\emph{New Hart's Rules}).

\tip{If you don't mind the second name being printed as-is,
  the canonical place to include it is the \code{nameaddon} field.}

\begin{bibexbox}
<NHR \S18.2.2>
{dodgson1896sl}
C. L. Dodgson [Lewis Carroll], \emph{Symbolic Logic} (Oxford, 1896).
\tcblower
\begin{Verbatim}
@book{dodgson1896sl,
  author = {C. L. Dodgson},
  nameaddon = {Lewis Carroll},
  title = {Symbolic Logic},
  location = {Oxford},
  date = {1896}}
\end{Verbatim}
\end{bibexbox}

\tip{You can switch to using parentheses for name addons by changing the \code{nameaddon} field format.}

\begin{tcblisting}{listing only}
\DeclareFieldFormat{nameaddon}{\mkbibparens{#1}}
\end{tcblisting}

\tip{If you do want the second name to be normalized, or you are dealing with a list of names,
  you can use the (non-standard) \code{authoraddon} and \code{editoraddon} fields. The addon
  name will be printed after the corresponding name in the regular name list, using the
  \code{nameaddon} field format, but only if it is different.}

\begin{bibexbox}
<NHR \S18.2.2>
{lauder1965lss}
Afferbeck Lauder [Alistair Morrison], \emph{Let Stalk Strine} (Sydney, 1965).
\tcblower
\begin{Verbatim}
@book{lauder1965lss,
  author = {Afferbeck Lauder},
  authoraddon = {Alistair Morrison},
  title = {Let Stalk Strine},
  location = {Sydney},
  date = {1965}}
\end{Verbatim}
\end{bibexbox}

\tip{Again, you can also use the syntax from \textsf{biblatex-realauthor}.
  If you specify the \code{author}, then \code{realauthor} is treated as an alias for \code{authoraddon}.
  The equivalent is true for \code{realeditor}.}

\section{Name variants}

In cases where an author changes the name under which they publish (e.g.\ due to changes of marital status), both the \emph{Oxford Guide to Style} and \emph{New Hart's Rules} suggest putting the later form of the name first, followed by parentheses containing an equals sign and the earier form of the name.
This is only supposed to to printed in the bibliography, not in notes.

\tip{To trigger this formatting, annotate the relevant name in \code{authoraddon} or \code{editoraddon} with the keyword \code{variant}.}

\begin{egcite*}{}
\cite{joukovsky1967gdd}
\end{egcite*}

\begin{bibexbox}
<OGS \S15.17.4>
{joukovsky1967gdd}
\textsc{Joukovsky, F.} (= \textsc{Joukovsky-Micha, F.}), \enquote{La Guerre des dieux et des géants chez les poètes francais du XVI\textsuperscript{e} siècle (1500–1585)}, \emph{Bibliothèque d'Humanisme et Renaissance}, 29 (1967), 55–92.
\toggletrue{blx@bibliography}
\tcblower
\begin{Verbatim}
@article{joukovsky1967gdd,
  author = {F. Joukovsky},
  authoraddon = {F. Joukovsky-Micha},
  authoraddon+an = {1=variant},
  title = {La Guerre des dieux et des géants chez les poètes francais du XVI\textsuperscript{e}
    siècle (1500--1585)},
  journaltitle = {Bibliothèque d'Humanisme et Renaissance},
  volume = {29},
  date = {1967},
  langid = {french},
  pages = {55-92}}
\end{Verbatim}
\end{bibexbox}

\section{Author-translators and author-revisers}

If the contribution of the translators or revisers is so great they could be joint authors, Oxford style is to print them immediately after the actual author. The motivation comes from textbooks like this one:

\begin{egcite*}{}
\textcite{kuehner.blass1890ef}
\end{egcite*}

\begin{bibexbox}
{kuehner.blass1890ef}
\textsc{Kühner, Raphael}, rev. \textsc{Blass, Friedrich}, \emph{Ausführliche grammatik der griechischen sprache}, i: \emph{Elementar- und Formenlehre} (Hannover: Hahnsche Buchhandlung, 1890–2).
\toggletrue{blx@bibliography}
\tcblower
\begin{Verbatim}
@mvbook{kuehner.blass1890ef,
  author = {Raphael Kühner},
  editor = {Friedrich Blass},
  editor+an = {=jointauthor},
  editortype = {reviser},
  shortauthor = {Kühner--Blass},
  title = {Elementar- und Formenlehre},
  maintitle = {Ausführliche grammatik der griechischen sprache},
  volume = {1},
  location = {Hannover},
  publisher = {Hahnsche Buchhandlung},
  date = {1890/1892}}
\end{Verbatim}
\end{bibexbox}

\tip{Annotate \emph{either} the \code{editor} \emph{or} \code{translator} field with the keyword \code{jointauthor} to promote the name to the joint author position (you cannot do it for both). Internally, what this does is move the names to the \pkg[biblatex]{oxref}-specific \code{jointauthor} field; you can use this and \code{jointauthortype} directly if you like, at the cost of portability to other styles. Note that you have to supply a suitable \code{shortauthor} value yourself; \pkg[biblatex]{oxref} does not calculate it for you.}

\section{Titular prefixes}\label{sec:nametitles}

Generally speaking, using titular prefixes like Revd, Dr, Mrs, Sir, and so on as part of an author's name is unnecessary,
but there are occasions when removing them can cause problems, so with \pkg[biblatex]{oxref} you can supply them if you need to.
I have not added special rules for recognizing such titles, so if you want to include one you have to label each part of the name explicitly, as in the following example (the keyword for the titular prefix is \code{title}).

\begin{bibexbox}
<NHR \S18.2.2*>
{wood1861el}
Mrs Henry Wood, \emph{East Lynne}, 3 vols. (London, 1861).
\tcblower
\begin{Verbatim}
@mvbook{wood1861el,
  author = {title=Mrs, given=Henry, family=Wood},
  title = {East Lynne},
  volumes = {3},
  location = {London},
  date = {1861}}
\end{Verbatim}
\end{bibexbox}

You can switch the display of titular prefixes on or off using the \key{usenametitles} option.
This can be set globally or on a per-type or per-entry basis.

\section{Works in foreign languages}

If you used a foreign language work, you might want to recommend a good translation.

\tip{Add the translation in \code{related}, and set \code{relatedtype} to \code{translationas}.}

\begin{bibexbox}
<NHR \S18.2.5*>
{tschichold1955tg}
J. Tschichold, \emph{Typographische Gestaltung} (Basle, 1955); Eng. trans. as \emph{Asymmetric Typography} (London, 1967).
\tcblower
\begin{Verbatim}
@book{tschichold1955tg,
  author = {J. Tschichold},
  title = {Typographische Gestaltung},
  location = {Basle},
  date = {1955},
  related = {tschichold1967tg},
  relatedtype = {translationas}}
@book{tschichold1967tg,
  title = {Asymmetric Typography},
  location = {London},
  date = {1967}}
\end{Verbatim}
\end{bibexbox}

\tip{To specify the translator up front, set the \code{relatedtype} to \code{bytranslator} instead.}

\begin{bibexbox}
<NHR \S18.2.14>
{sarrau1975ta}
José Sarrau, \emph{Tapas y aperitivos} (Madrid, 1975); trans. Francesca Piemonte Slesinger as \emph{Tapas and Appetizers} (New York, 1987).
\tcblower
\begin{Verbatim}
@book{sarrau1975ta,
  author = {José Sarrau},
  title = {Tapas y aperitivos},
  location = {Madrid},
  date = {1975},
  related = {sarrau1987ta},
  relatedtype = {bytranslator}}
@book{sarrau1987ta,
  author = {José Sarrau},
  translator = {Francesca Piemonte Slesinger},
  title = {Tapas and Appetizers},
  location = {New York},
  date = {1987}}
\end{Verbatim}
\end{bibexbox}

If you used the translation, you might want to give the original publication as well.

\tip{Add the original in \code{related}, and set \code{relatedtype} to \code{translationof}.}

\begin{bibexbox}
<NHR \S18.2.5>
{metz1938hyb}
R. Metz, \emph{A Hundred Years of British Philosophy}, ed. J. H. Muirhead, trans. J. W. Harvey (1938) [Ger. orig., \emph{Die philosophischen Strömungen der Gegenwart in Grossbritannien} (1935)]
\tcblower
\begin{Verbatim}
@book{metz1938hyb,
  author = {R. Metz},
  title = {A Hundred Years of British Philosophy},
  editor = {J. H. Muirhead},
  translator = {J. W. Harvey},
  date = {1938},
  related = {metz1935psg},
  relatedtype = {translationof}}
@book{metz1935psg,
  title = {Die philosophischen Strömungen der Gegenwart in Grossbritannien},
  date = {1935},
  language = {german}}
\end{Verbatim}
\end{bibexbox}

It is often helpful to provide an informative translation of foreign language titles.

\tip{Put the translation in \code{titleaddon}.}

\begin{bibexbox}
<NHR \S18.2.5>
{nissan1965hnm}
Nissan Motor Corporation, \emph{Nissan Jidosha 30nen shi} [A 30-year history of Nissan Motors] (1965).
\tcblower
\begin{Verbatim}
@book{nissan1965hnm,
  author = {{Nissan Motor Corporation}},
  title = {Nissan Jidosha 30nen shi},
  titleaddon = {A 30-year history of Nissan Motors},
  date = {1965}}
\end{Verbatim}
\end{bibexbox}

Conversely, it might be helpful to provide the English original of a translated title.

\begin{bibexbox}
<OGS \S15.2.1>
{milne1938pb}
A. A. Milne, \emph{Pu der Bär} [Ger. trans. of \emph{Winnie the Pooh}] (Potsdam: Williams, 1938).
\tcblower
\begin{Verbatim}
@book{milne1938pb,
  author = {A. A. Milne},
  title = {Pu der Bär},
  language = {german},
  origtitle = {Winnie the Pooh},
  location = {Potsdam},
  publisher = {Williams},
  date = {1938}}
\end{Verbatim}
\end{bibexbox}

\section{Missing place of publication}

For periodicals, grey literature, audiovisual and online material, the lack of a place of publication is not surprising; indeed it may be expected. For books, however, it may be remarkable and deserve marking in the bibliography with \enquote{n.p.\@} (for \enquote{no place}).

\tip{To have \pkg[biblatex]{oxref} automatically insert \enquote{n.p.\@} or the localized equivalent for books, collections, reference works, proceedings and similar entry types, use the \key{nolocation} bibliography option.}

\tip{To have \pkg[biblatex]{oxref} automatically insert \enquote{n.p.\@} or the localized equivalent for only a specific entry, use the \key{nolocation} entry option instead.
  The advantage of doing this over simply giving \enquote{n.p.\@} as the publisher is that it hides it from other styles that do not follow the same convention.}

\begin{bibexbox}
<NHR \S18.2.9>
{padua1961p}
Marchetto of Padua, \emph{Pomerium}, ed. Guiseppe Vecchi (n.p., 1961).
\tcblower
\begin{Verbatim}
@book{padua1961p,
  author = {{Marchetto of Padua}},
  title = {Pomerium},
  editor = {Guiseppe Vecchi},
  date = {1961},
  options = {nolocation}}
\end{Verbatim}
\end{bibexbox}

\section{Missing or inferred date of publication}

If the date is missing from a work, you can sometimes make an educated guess
what it should be. It is usual practice to enclose such guesses in square
brackets.

\tip{With \pkg[biblatex]{oxref}, you can achieve this by annotating the
  \code{date}, \code{origdate}, or \code{eventdate} field with the keyword
  \code{inferred}.}

\begin{tcblisting}{listing only}
  date+an = {=inferred},
\end{tcblisting}

\chapter{Articles and periodicals}\label{sec:article}
\chapterprecis{article, periodical, suppperiodical, review}

\section{Articles in periodicals with volumes/numbers}

The \emph{Oxford Guide to Style} consistently prefers volume and part numbers to be written like \enquote{23/2}, but also discusses formats such as \enquote{23: 2} and \enquote{23 (2)}. \emph{New Hart's Rules} adds \enquote{23, 2} as a further possibility. The latter two are better suited to \textsf{oxalph}\slash\textsf{oxyear} which use a colon to demarcate the page numbers. These variations are implemented as the option \key{issuestyle}; see \autoref{sec:loading} for details.

\spec{Author, \enquote{Title}, \emph{JournalTitle}, Vol/Number (Year), Pages.}

\begin{bibexbox}<OGS \S15.4.1>
{goldblatt1973dmm}
Robert Goldblatt, \enquote{Diodorean Modality in Minkowski Space-Time}, \emph{Studia Logica}, 39/3 (1973), 219--36.
\tcblower
\begin{Verbatim}
@article{goldblatt1973dmm,
  author = {Robert Goldblatt},
  title = {Diodorean Modality in Minkowski Space-Time},
  journaltitle = {Studia Logica},
  volume = {39},
  number = {3},
  date = {1973},
  pages = {219-236}}
\end{Verbatim}
\end{bibexbox}

\begin{bibexbox}<OGS \S15.4.1>
{inalcik1992csm}
Halil Inalcik, \enquote{Comments on \enquote{Sultanism}: Max Weber's Typification of the Ottoman Polity}, \emph{Princeton Papers in Near Eastern Studies}, 1 (1992), 49--72.
\tcblower
\begin{Verbatim}
@article{inalcik1992csm,
  author = {Halil Inalcik},
  title = {Comments on \enquote{Sultanism}},
  subtitle = {Max Weber's Typification of the Ottoman Polity},
  journaltitle = {Princeton Papers in Near Eastern Studies},
  volume = {1},
  date = {1992},
  pages = {49-72}}
\end{Verbatim}
\end{bibexbox}


\begin{bibexbox}
<OGS \S15.2.1>
{vaucouleurs1975nmn}
Gerald de Vaucouleurs et al., \enquote{The New Martian Nomenclature of the International Astronomical Union}, \emph{Icarus}, 26 (1975), 85--98.
\tcblower
\begin{Verbatim}
@article{vaucouleurs1975nmn,
  author = {Gerald de Vaucouleurs and J. Blunck and M. Davies and A. Dollfus and I. Koval and
    G. Kuiper and H. Masursky and S. Miyamoto and V Moroz and Carl Sagan},
  title = {The New {Martian} Nomenclature of the {International} {Astronomical} {Union}},
  journaltitle = {Icarus},
  volume = {26},
  date = {1975},
  pages = {85-98}}
\end{Verbatim}
\end{bibexbox}

\begin{bibexbox}
<NHR \S18.8.5>
{druin2002rcd}
A. Druin, \enquote{The Role of Children in the Design of New Technology}, \emph{Behaviour \& Information Technology}, 21/1 (2002), 1–25. doi: \path{10.1080/01449290110108659}
\tcblower
\begin{Verbatim}
@article{druin2002rcd,
  author = {A. Druin},
  title = {The Role of Children in the Design of New Technology},
  journaltitle = {Behaviour \& Information Technology},
  volume = {21},
  number = {1},
  date = {2002},
  pages = {1-25},
  doi = {10.1080/01449290110108659}}
\end{Verbatim}
\end{bibexbox}

\begin{bibexbox}(hacked)
<NHR \S18.8.5*>
{li.etal2013flh}
Shu Li et al., \enquote{Forever Love: The Hitherto Earliest Record of Copulating Insects from the Middle Jurassic of China}, \emph{PLoS ONE}, 8/11 (2013), e78188. doi: \path{10.1371/journal.pone.0078188}
\tcblower
\begin{Verbatim}
@article{li.etal2013flh,
  author = {Shu Li and Chungkun Shih and Chen Wang and Hong Pang and Dong Ren},
  title = {Forever Love},
  subtitle = {The Hitherto Earliest Record of Copulating Insects from the Middle Jurassic of China},
  journaltitle = {{PLoS ONE}},
  volume = {8},
  number = {11},
  date = {2013},
  eid = {e78188},
  doi = {10.1371/journal.pone.0078188}}
\end{Verbatim}
\end{bibexbox}

\hack{If the last\slash only word of the journal title is abbreviated (indicated by point or capital letter), it does not need a comma after it. \pkg[biblatex]{Oxref} will try to detect this; if it gets it wrong, you can suppress the comma by adding \cs{nopunct} to the end of the journal (sub)title, or restore it by adding a pair of braces.}

\begin{bibexbox}<OGS \S15.4.1>
{lindars1965eir}
B. Lindars, \enquote{Ezechiel and Individual Responsibility}, \emph{VT} 15 (1965), 452--67.
\tcblower
\begin{Verbatim}
@article{lindars1965eir,
  author = {B. Lindars},
  title = {Ezechiel and Individual Responsibility},
  journaltitle = {VT},
  volume = {15},
  date = {1965},
  pages = {452-467}}
\end{Verbatim}
\end{bibexbox}

\begin{bibexbox}
<OGS \S15.4.1*>
{jsr1990spt}
\enquote{Solar Photon Thruster}, \emph{Journal of Spacecraft and Rockets}, 28/4 (July–Aug. 1990), 411–6.
\tcblower
\begin{Verbatim}
@article{jsr1990spt,
  title = {Solar Photon Thruster},
  journaltitle = {Journal of Spacecraft and Rockets},
  volume = {28},
  number = {4},
  date = {1990-07/1990-08},
  pages = {411-416}}
\end{Verbatim}
\end{bibexbox}

\section{Articles in periodicals with series}

\spec{Author, \enquote{Title}, \emph{JournalTitle}, nth \lit{ser}., Vol/Number (Year), Pages.}

\begin{bibexbox}
<NHR \S18.3.3>
{moody1953mdb}
T. W. Moody, \enquote{Michael Davitt and the British Labour Movement, 1882--1906}, \emph{Transactions of the Royal Historical Society}, 5th ser., 3 (1953), 53–76.
\tcblower
\begin{Verbatim}
@article{moody1953mdb,
  author = {T. W. Moody},
  title = {Michael Davitt and the British Labour Movement, 1882--1906},
  journaltitle = {Transactions of the Royal Historical Society},
  series = {5},
  volume = {3},
  date = {1953},
  pages = {53-76}}
\end{Verbatim}
\end{bibexbox}

\spec{Author, \enquote{Title}, \emph{JournalTitle}, \textsc{ns} Vol/Number (Year), Pages.}

\begin{bibexbox}<OGS \S15.4.3>
{barnes1971has}
J. Barnes, \enquote{Homonymy in Aristotle and Speusippus}, \emph{Classical Quarterly}, \textsc{ns} 21 (1971), 65--80.
\tcblower
\begin{Verbatim}
@article{barnes1971has,
  author = {J. Barnes},
  title = {Homonymy in Aristotle and Speusippus},
  journaltitle = {Classical Quarterly},
  series = {newseries},
  volume = {21},
  date = {1971},
  pages = {65-80}}
\end{Verbatim}
\end{bibexbox}


\section{Articles in issues identified by date alone}

\spec{Author, \enquote{Title}, \emph{JournalTitle}, Day Month Year, Pages.}

\begin{bibexbox}<OGS \S15.4.1>
{bw1984wen}
\enquote{Who's Excellent Now?}, \emph{Business Week}, 5 Nov. 1984, 76--86.
\tcblower
\begin{Verbatim}
@article{bw1984wen,
  title = {Who's Excellent Now?},
  journaltitle = {Business Week},
  date = {1984-11-05},
  pages = {76-86}}
\end{Verbatim}
\end{bibexbox}

\begin{bibexbox}<OGS \S15.4.1>
{lee1995ehf}
Alan Lee, \enquote{England Haunted by Familiar Failings}, \emph{The Times}, 23 June 1995.
\tcblower
\begin{Verbatim}
@article{lee1995ehf,
  author = {Alan Lee},
  title = {England Haunted by Familiar Failings},
  journaltitle = {The Times},
  date = {1995-06-23}}
\end{Verbatim}
\end{bibexbox}


\begin{bibexbox}
<OGS \S15.3>
{boyce1957pgi}
M. Boyce, \enquote{The Parthian \emph{Gsn} and Iranian Minstrel Tradition}, \emph{Journal of the Royal Asiatic Society}, 1957, 10--45.
\tcblower
\begin{Verbatim}
@article{boyce1957pgi,
  author = {M. Boyce},
  title = {The Parthian \emph{Gsn} and Iranian Minstrel Tradition},
  journaltitle = {Journal of the Royal Asiatic Society},
  date = {1957},
  pages = {10-45}}
\end{Verbatim}
\end{bibexbox}

\section{Works published as an issue}

These examples illustrate where a book is also published as a whole
issue of a periodical, and show how you may reference both at once.
You can either add the periodical details to a book entry
or add the book details to a periodical entry.

\spec{Author, \emph{Title} = \emph{JournalTitle}, Vol/Number (Location: Publisher, Year), pages.}

\tip{Use \code{equals} as the \code{relatedtype}.}

\begin{bibexbox}
<OGS \S15.4.1>
{bec1976isc}
C. Bec (ed.), \emph{Italie 1500--1550: Une situation de crise?} = \emph{Annales de l'Université Jean Moulin}, 1975/2 (Langues étrangères, 2; Lyon, 1976), 99--109.
\tcblower
\begin{Verbatim}
@collection{bec1976isc,
  editor = {C. Bec},
  title = {Italie 1500--1550},
  subtitle = {Une situation de crise?},
  related = {aujm1975.2},
  relatedtype = {equals},
  series = {Langues étrangères},
  number = {2},
  location = {Lyon},
  date = {1976},
  pages = {99-109}}
@periodical{aujm1975.2,
  title = {Annales de l'Université Jean Moulin},
  volume = {1975},
  number = {2}}
\end{Verbatim}
\end{bibexbox}

\spec{Author, \emph{Title} = \emph{JournalTitle}, Vol/Number (Year).}

\tip{Use \code{issuetitle} instead of \code{title} to get the right formatting.}

\begin{bibexbox}
<OGS \S15.4.1>
{trisoglio1973gnq}
F. Trisoglio, \emph{Gregorio di Nazianzo in un quarentennio di recherche (1925--1965)} = \emph{Rivista Iasalliana}, 40 (1973).
\tcblower
\begin{Verbatim}
@article{trisoglio1973gnq,
  author = {F. Trisoglio},
  issuetitle = {Gregorio di Nazianzo in un quarentennio di recherche (1925--1965)},
  journaltitle = {Rivista Iasalliana},
  volume = {40},
  date = {1973}}
\end{Verbatim}
\end{bibexbox}

\section{Articles in an issue that is a supplement to another issue}

\DeclareNumChars*{S}

\spec{Author, \enquote{Title}, \emph{JournalTitle}, Vol/Number (Year); \lit{Supplement to} \emph{MainJournalTitle} Vol/Number, Pages.}

\tip{To get this format, use a separate entry for the parent issue, reference it in the \code{related} field, and use the key \code{suppto} as the \code{relatedtype}.}

\begin{bibexbox}
<OGS \S15.4.1>
{zhentao.etal1989ars}
X. Zhentao, K. K. C. Yau, and F. R. Stephenson, \enquote{Astronomical Records on the Shang Dynasty Oracle Bones}, \emph{Archaeoastronomy}, 14 (1989); Supplement to \emph{Journal for the History of Astronomy}, 20, pp. S61--S72.
\tcblower
\begin{Verbatim}
@article{zhentao.etal1989ars,
  author = {X. Zhentao and K. K. C. Yau and F. R. Stephenson},
  title = {Astronomical Records on the Shang Dynasty Oracle Bones},
  journaltitle = {Archaeoastronomy},
  volume = {14},
  date = {1989},
  related = {jha1989},
  relatedtype = {suppto},
  pages = {S61-S72},
  bookpagination = {page}}
@periodical{jha1989,
  title = {Journal for the History of Astronomy},
  volume = {20}}
\end{Verbatim}
\end{bibexbox}

\tip{Use \code{bookpagination} to force the display of \enquote{pp.} (since the number format is odd), and \cs{DeclareNumChars*}\brackets{S} to ensure that \code{S61} is recognized as a number.}%

\section{Articles that span multiple issues}

\tip{Use the key \code{serialarticle} as the \code{relatedtype}.}



\section{Accepted journal articles, pre-publication}

\spec{Author, \enquote{Title}, \lit{to be published in} \emph{JournalTitle}, Year.}

\begin{bibexbox}
{briscoe2008esp}
(not in book)
\tcblower
\begin{Verbatim}
@article{briscoe2008esp,
  author = {Robert Briscoe},
  title = {Egocentric Spatial Representation in Action and Perception},
  journaltitle = {Philosophy and Phenomenological Research},
  url = {http://cogprints.org/5780/1/ECSRAP.F07.pdf},
  pubstate = {inpress}}
\end{Verbatim}
\end{bibexbox}

\section{Editorials and other regular features}

Strictly speaking, when a work is headed \enquote{Editorial} or \enquote{Letter to the Editor}
in a periodical, this is a descriptor rather than a title.

\tip{If a piece has a true title, the descriptor goes in the \code{note} field.
  If it doesn't, the descriptor goes in the \code{title} field; annotate the field with the keyword \code{descriptor} to remove the quote marks.
  Alternatively, use the (non-standard) \code{descriptor} field for the descriptor in all cases, and \pkg[biblatex]{oxref} will take care of this for you.}

\begin{bibexbox}
{ball2015ed}
(not in book)
\tcblower
\begin{Verbatim}
@suppperiodical{ball2015ed,
  author = {Alexander Ball},
  title = {Editorial},
  title+an = {=descriptor},
  date = {2015},
  journaltitle = {International Journal of Digital Curation},
  volume = {10},
  number = {1},
  pages = {i-v},
  doi = {10.2218/ijdc.v10i1.376}}
\end{Verbatim}
\end{bibexbox}

\begin{bibexbox}
{jccm2006ed}
(not in book)
\tcblower
\begin{Verbatim}
@suppperiodical{jccm2006ed,
  title = {\enquote{What a Disaster} and Why Does This Question Matter?},
  date = {2006},
  note = {Editorial},
  journaltitle   = {Journal of Contingencies and Crisis Management},
  volume = {14},
  pages = {1-2}}
\end{Verbatim}
\end{bibexbox}

\section{Reviews}

\spec{Author, \enquote{Title}, \lit{review of} ReviewedWork, \lit{in} \emph{JournalTitle}, Vol/Number (Year), Pages.}

\tip{To get this format, use a separate entry for the reviewed item, reference it in the \enquote{related} field, and use the key \enquote{reviewof} as the \enquote{relatedtype}.}

\begin{bibexbox}
<OGS \S15.4.4>
{dean1995rpb}
J. Dean, review of Philippe Basiron, \emph{My Early Life} (Bourges, 1994), in \emph{Res facta}, 17 (1995), 56--9.
\tcblower
\begin{Verbatim}
@review{dean1995rpb,
  author = {J. Dean},
  related = {basiron1994mel},
  relatedtype = {reviewof},
  journaltitle = {Res facta},
  volume = {17},
  date = {1995},
  pages = {56-59}}
@book{basiron1994mel,
  author = {Philippe Basiron},
  title = {My Early Life},
  location = {Bourges},
  date = {1994}}
\end{Verbatim}
\end{bibexbox}

\begin{bibexbox}
<OGS \S15.4.4>
{jocelyn1989pav}
H. D. Jocelyn, \enquote{Probus and Virgil}, review of Maria Luisa Delvigo, \emph{Testo virgiliano e tradizione indiretta} (Pisa, 1987), in \emph{CR}, \textsc{ns} 39 (1989), 27--8.
\tcblower
\begin{Verbatim}
@review{jocelyn1989pav,
  author = {H. D. Jocelyn},
  title = {Probus and Virgil},
  related = {delvigo1987tvt},
  relatedtype = {reviewof},
  journaltitle = {CR},
  series = {newseries},
  volume = {39},
  date = {1989},
  pages = {27-28}}
@book{delvigo1987tvt,
  author = {Maria Luisa Delvigo},
  title = {Testo virgiliano e tradizione indiretta},
  location = {Pisa},
  date = {1987}}
\end{Verbatim}
\end{bibexbox}

\chapter{Books}\label{sec:book}
\chapterprecis{book, mvbook, collection, mvcollection, reference, mvreference}

\section{Monographs}

Biblatex defines \code{book} as a monograph, written either by a single author, or by several authors who have joint responsibility for the entire work. An editor in this context has a modest role selecting or annotating the content, and therefore in Oxford style is given after the title even if the author is not given.

\subsection{Basic form}

\spec{Author, \emph{Title} (Location: Publisher, Year).}

\begin{bibexbox}<OGS \S15.2.1>
{eliot1977m}
George Eliot, \emph{Middlemarch} (New York: W. W. Norton, 1977).
\tcblower
\begin{Verbatim}
@book{eliot1977m,
  author = {George Eliot},
  title = {Middlemarch},
  location = {New York},
  publisher = {W. W. Norton},
  date = {1977}}
\end{Verbatim}
\end{bibexbox}

\begin{bibexbox}<OGS \S15.2.2>
{benvenuti1986op}
Antonia Tissoni Benvenuti, \emph{L'\emph{Orfeo} del Poliziano} (Padua: Editrice Antenore, 1986).
\tcblower
\begin{Verbatim}
@book{benvenuti1986op,
  author = {Antonia Tissoni Benvenuti},
  title = {L'\emph{Orfeo} del Poliziano},
  location = {Padua},
  publisher = {Editrice Antenore},
  date = {1986}}
\end{Verbatim}
\end{bibexbox}

\begin{bibexbox}<OGS \S15.2.2>
{oconor1977sjf}
Roderick O'Conor, \emph{A Sentimental Journal through \enquote{Finnegan's Wake}, with a Map of the Liffey} (Dublin: HCE Press, 1977).
\tcblower
\begin{Verbatim}
@book{oconor1977sjf,
  author = {Roderick O'Conor},
  title = {A Sentimental Journal through \enquote{Finnegan's Wake}, with a Map of the {Liffey}},
  location = {Dublin},
  publisher = {HCE Press},
  date = {1977}}
\end{Verbatim}
\end{bibexbox}


\subsection{No publisher}

\spec{Author, \emph{Title} (Location, Year).}

\begin{bibexbox}<NHR \S18.1.3>
{rogers1986tmp}
C.~D.~Rogers, \emph{Tracing Missing Persons} (Manchester, 1986).
\tcblower
\begin{Verbatim}
@book{rogers1986tmp,
  author = {C. D. Rogers},
  title = {Tracing Missing Persons},
  location = {Manchester},
  date = {1986}}
\end{Verbatim}
\end{bibexbox}


\subsection{Edited book}

\spec{Author, \emph{Title}, \lit{ed.} Editor(s) (Location: Publisher, Year).}

\spec{\emph{Title}, \lit{ed.} Editor(s) (Location: Publisher, Year).}

\begin{bibexbox}
<OGS \S15.2.1>
{boas.botschuyver1952dc}
\emph{Distichia Catonis}, ed. Marcus Boas and Henricus Johannes Botschuyver (Amsterdam, 1952).
\tcblower
\begin{Verbatim}
@book{boas.botschuyver1952dc,
  title = {Distichia Catonis},
  editor = {Marcus Boas and Henricus Johannes Botschuyver},
  location = {Amsterdam},
  date = {1952}}
\end{Verbatim}
\end{bibexbox}

\section{Collections}

Biblatex defines \code{collection} as a book made up of multiple self-contained contributions from distinct authors. There is no overall author: use the regular \code{book} entry type for collections of a single author’s work. The editor in this case has a more active role and therefore comes before the title.

\spec{Editor (\lit{ed.}), \emph{Title} (Location: Publisher, Year).}

\begin{bibexbox}<OGS \S15.2.1>
{stewart.etal1994mb}
Rosemary Stewart et al. (eds.), \emph{Managing in Britain} (London, 1994).
\tcblower
\begin{Verbatim}
@collection{stewart.etal1994mb,
  editor = {Rosemary Stewart and Jean-Louis Barsoux and Alfred Kieser and Hans-Dieter Ganter and Peter Walgenbach},
  title = {Managing in {Britain}},
  location = {London},
  date = {1994}}
\end{Verbatim}
\end{bibexbox}



\section{Reference works}

Biblatex uses \code{reference} for encyclopaedias and dictionaries, which are typically made up of many small contributions by distinct authors and compiled by an editorial board whose membership might change between successive editions. From a database perspective, reference works are like collections in that they have no single author (a reference work written by a single author should use the \code{book} entry type instead).

\tip{The \emph{Oxford Guide to Style} considers the titles of reference works to be more important and memorable than those of the editor, and so lists the title first, but \emph{New Hart's Rules} doesn't. \pkg[biblatex]{Oxref} takes the former approach, but you can switch to the latter by setting the \key{useeditor} option back to true for this type.}

\begin{tcblisting}{listing only}
\ExecuteBibliographyOptions[reference,mvreference]{useeditor=true}
\end{tcblisting}

\spec{\emph{Title}, \lit{ed.} Editor(s) (Location: Publisher, Year).}

\begin{bibexbox}
<OGS \S15.2.1>
{fortenbaugh.etal1991tes}
\emph{Theophrastus of Eresus: Sources for his Life, Writings, Thought, and Influence}, ed. William W. Fortenbaugh et al., 2 vols. (Philosophia Antiqua, 54; Leiden, 1991).
\tcblower
\begin{Verbatim}
@mvreference{fortenbaugh.etal1991tes,
  title = {{Theophrastus} of {Eresus}},
  subtitle = {Sources for his Life, Writings, Thought, and Influence},
  editor = {William W. Fortenbaugh and Pamela M Huby and Robert W. Sharples and Dimitri Gutas
    and others},
  volumes = {2},
  series = {Philosophia Antiqua},
  number = {54},
  location = {Leiden},
  date = {1991}}
\end{Verbatim}
\end{bibexbox}

\section{Multi-volume works}

Biblatex provides additional entry types for multi-volume works: \code{mvbook}, \code{mvcollection} and \code{mvreference} respectively. These can be used to cite all the volumes at once, or just one volume from the set.

\subsection{All volumes}

Where all volumes were published consistently, the following form is used:

\spec{Author, \emph{Title}, n \lit{vols.} (Location: Publisher, Year), VolNo. Pages.}

\begin{egcite}{\dots(Brussels, 1867--88), ii. 367--8}
\cite[ii.~\mkcomprange{367-368}]%
  {straeten1867lmp}
\end{egcite}

\begin{bibexbox}
<OGS \S15.2.6>
{straeten1867lmp}%
Edmond Vander Straeten, \emph{La Musique aux Pays-Bas avant le XIX\textsuperscript{e} siècle}, 8 vols. (Brussels, 1867--88).
\tcblower
\begin{Verbatim}
@mvbook{straeten1867lmp,
    author = {Edmond {Vander Straeten}},
    title = {La Musique aux Pays-Bas avant le XIX\textsuperscript{e} siècle},
    volumes = {8},
    location = {Brussels},
    date = {1867/1888}}
\end{Verbatim}
\end{bibexbox}

Where the publisher changed between volumes, the following form is used:

\spec{Author, \emph{Title}, VolNo, n \lit{vols.} (Location: Publisher and Location: Publisher, Year–Year).}

\begin{bibexbox}
<OGS \S15.2.6*>
{ritter1838hap}
Heinrich Ritter, \emph{The History of Ancient Philosophy}, trans. Alexander J. W. Morrison, 4~vols. (Oxford: Talboys and London: Bohn, 1838--46).
\tcblower
\begin{Verbatim}
@mvbook{ritter1838hap,
  author = {Heinrich Ritter},
  title = {The History of Ancient Philosophy},
  translator = {Alexander J. W. Morrison},
  volumes = {4},
  location = {Oxford and London},
  publisher = {Talboys and Bohn},
  date = {1838/1846}}
\end{Verbatim}
\end{bibexbox}

\info{In the \emph{Oxford Style Manual}, the translator in the above reference
  is in parentheses; this occurs in one other reference (\S15.2.2, Lawrence 1992),
  but in many more does not (\S15.2.15, Bischoff 1990; \S15.8, Auden 1990;
  all in \S13.11.1). I am therefore choosing to ignore this unnecessary
  complication.}

\tip{You can alternatively use the standard \textsf{biblatex} relation type \code{multivolume}.
Note that when you do so, the overall date is removed, so be sure that individual volume dates are provided.
}

\begin{bibexbox}
<OGS \S15.2.6*>
{ritter1838hap:mv}
Heinrich Ritter, \emph{The History of Ancient Philosophy}, trans. Alexander J. W. Morrison, 4~vols., i–iii (Oxford: Talboys, 1838–9), iv (London: Bohn, 1846).
\tcblower
\begin{Verbatim}
@mvbook{ritter1838hap:mv,
  author = {Heinrich Ritter},
  title = {The History of Ancient Philosophy},
  translator = {Alexander J. W. Morrison},
  volumes = {4},
  date = {1838/1846},
  related = {ritter1838hap1-3,ritter1838hap4},
  relatedtype = {multivolume}}
@book{ritter1838hap1-3,
  volume = {1-3},
  location = {Oxford},
  publisher = {Talboys},
  date = {1838/1839}}
@book{ritter1838hap4,
  volume = {4},
  location = {London},
  publisher = {Bohn},
  date = {1846}}
\end{Verbatim}
\end{bibexbox}

\subsection{One volume from several}

Where the volumes are merely numbered, the following form is used:

\spec{Author, \emph{Title}, VolNo (Location: Publisher, Year), Pages.}

\tip{To get this format, do not use \code{maintitle}: put the title of the whole work in \code{title}.}

\begin{egcite}{\dots ii (Brussels, 1867--88), 367--8}
\cite[367--368]{straeten1867lmp.ii}
\end{egcite}

\begin{bibexbox}<OGS \S15.2.6>
{straeten1867lmp.ii}
Edmond Vander Straeten, \emph{La Musique aux Pays-Bas avant le XIX\textsuperscript{e} siècle}, ii (Brussels, 1867--88).
\tcblower
\begin{Verbatim}
@mvbook{straeten1867lmp.ii,
  author = {Edmond {Vander Straeten}},
  title = {La Musique aux Pays-Bas avant le XIX\textsuperscript{e} siècle},
  volume = {2},
  location = {Brussels},
  date = {1867/1888}}
\end{Verbatim}
\end{bibexbox}

Where the volumes each have their own (sub)title,
and they were all published at once (more or less),
the following form is used:

\spec{Author, \emph{Title}, VolNo: \emph{VolTitle} (Location: Publisher, Year), Pages.}

\tip{To get this format, use \code{title} for the volume title and \code{maintitle} for the whole work.}

\begin{bibexbox}
<OGS \S15.2.2>
{brady.etal1994hehsa}
Thomas A. Brady, Jr., Heiko A. Oberman, and James D. Tracy (eds.), \emph{Handbook of European History, 1400--1600: Late Middle Ages, Renaissance and Reformation}, i: \emph{Structures and Assertions} (Leiden: E. J. Brill, 1994).
\tcblower
\begin{Verbatim}
@mvcollection{brady.etal1994hehsa,
  editor = {Brady, Jr., Thomas A. and Heiko A. Oberman and James D. Tracy},
  maintitle = {Handbook of European History, 1400--1600},
  mainsubtitle = {{Late} {Middle} {Ages,} {Renaissance} and {Reformation}},
  volume = {1},
  title = {Structures and Assertions},
  location = {Leiden},
  publisher = {E. J. Brill},
  date = {1994}}
\end{Verbatim}
\end{bibexbox}

\begin{egcite}{\dots (Cambridge: CUP, 1932), 42--56}
\cite[42-56]{ward.waller1932che}
\end{egcite}

\begin{bibexbox}
<OGS \S15.2.6>
{ward.waller1932che}
A. W. Ward and A. E. Waller (eds.), \emph{The Cambridge History of English Literature}, xii: \emph{The Nineteenth Century} (Cambridge: CUP, 1932).
\tcblower
\begin{Verbatim}
@mvcollection{ward.waller1932che,
  editor = {A. W. Ward and A. E. Waller},
  maintitle = {The Cambridge History of English Literature},
  volume = {12},
  title = {The Nineteenth Century},
  location = {Cambridge},
  publisher = {CUP},
  date = {1932}}
\end{Verbatim}
\end{bibexbox}

Where the volumes each have their own (sub)title,
and there are many volumes spanning years and possibly publishers,
the following form is used:

\spec{Author, \emph{VolTitle}, [\lit{vol.} VolNo \lit{of} Title] (Location: Publisher, Year), Pages.}

\tip{To get this format, use \code{book} instead of \code{mvbook}.}

\begin{bibexbox}
<NHR \S18.2.7>
{fischer1989asf}
David Hackett Fischer, \emph{Albion’s Seed: Four British Folkways in America}, [vol.\ i of \emph{America: A Cultural History}] (New York: Oxford University Press, 1989).
\tcblower
\begin{Verbatim}
@book{fischer1989asf,
  author = {David Hackett Fischer},
  title = {Albion’s Seed},
  subtitle = {Four British Folkways in America},
  volume = {1},
  maintitle = {America},
  mainsubtitle = {A Cultural History},
  location = {New York},
  publisher = {Oxford University Press},
  date = {1989}}
\end{Verbatim}
\end{bibexbox}

\section{Additions, translations, and revisions}

Where works have significant introductions, forewords, afterwords, illustrations, etc.\ this may be noted as follows:

\spec{Author, \emph{Title}, \lit{with an} Addition \lit{by} Contributor (Location: Publisher, Year).}

\begin{bibexbox}
<OGS \S15.2.1>
{twain1971cyk}
Mark Twain, \emph{A Connecticut Yankee at King Arthur's Court}, with an introduction by Justin Kaplan (Harmondsworth: Penguin, 1971).
\tcblower
\begin{Verbatim}
@book{twain1971cyk,
  author = {Mark Twain},
  title = {A {Connecticut} Yankee at {King} {Arthur's} Court},
  introduction = {Justin Kaplan},
  location = {Harmondsworth},
  publisher = {Penguin},
  date = {1971}}
\end{Verbatim}
\end{bibexbox}

Translators are credited as follows:

\spec{Author, \emph{Title}, \lit{trans.} Translator(s) (Location: Publisher, Year).}

\begin{bibexbox}
<NHR \S18.2.14>
{bischoff1990lpa}
Bernhard Bischoff, \emph{Latin Palaeography: Antiquity and the Middle Ages}, trans. Dáibhi Ó Cróinín and David Ganz (Cambridge, 1990).
\tcblower
\begin{Verbatim}
@book{bischoff1990lpa,
  author = {Bernhard Bischoff},
  title = {Latin Palaeography},
  subtitle = {Antiquity and the Middle Ages},
  translator = {Dáibhi Ó Cróinín and David Ganz},
  location = {Cambridge},
  date = {1990}}
\end{Verbatim}
\end{bibexbox}

\spec{Author, \emph{Title}, \lit{trans.\ with} Addition Translator(s) (Location: Publisher, Year).}

\begin{bibexbox}
<NHR \S18.2.14>
{martorell1984tlb}
Joanat Martorell, \emph{Tirant lo Blanc}, trans. with foreword David H. Rosenthal (London, 1984).
\tcblower
\begin{Verbatim}
@book{martorell1984tlb,
  author = {Joanat Martorell},
  title = {Tirant lo Blanc},
  translator = {David H. Rosenthal},
  foreword = {David H. Rosenthal},
  location = {London},
  date = {1984}}
\end{Verbatim}
\end{bibexbox}

\section{Editions}

\subsection{Later edition only}

Where the author\slash editor is common to both, the following form is used:

\spec{Author, \emph{Title} (nth \lit{edn.}, Location: Publisher, Year).}

\tip{This is the format used by the entry type \code{book}}

\begin{egcite}{\dots (3rd edn., 1990), 419--21}
\cite[419-421]{baker1990iel}
\end{egcite}

\begin{bibexbox}<OGS \S15.2.3>
{baker1990iel}
J. H. Baker, \emph{An Introduction to English Legal History} (3rd edn., 1990).
\tcblower
\begin{Verbatim}
@book{baker1990iel,
  author = {J. H. Baker},
  title = {An Introduction to {English} Legal History},
  edition = {3},
  date = {1990}}
\end{Verbatim}
\end{bibexbox}

\begin{bibexbox}
<NHR \S18.8.5>
{beckford1823v}
William Beckford, \emph{Vathek} (4th edn., London, 1823) [online facsimile], \url{http://beckford.c18.net/wbvathek1823.html}, accessed 5 Nov. 2013.
\tcblower
\begin{Verbatim}
@book{beckford1823v,
  author = {William Beckford},
  title = {Vathek},
  edition = {4},
  location = {London},
  date = {1823},
  howpublished = {online facsimile},
  url = {http://beckford.c18.net/wbvathek1823.html},
  urldate = {2013-11-05}}
\end{Verbatim}
\end{bibexbox}

Where the editor has changed, the following form is used:

\spec{\emph{Title}, nth \lit{edn.}, \lit{ed.} Editor (Location: Publisher, Year).}

\tip{This format is used by entry type \code{reference}}

\begin{bibexbox}
<OGS \S15.2.3>
{knowles1999odq}
\emph{The Oxford Dictionary of Quotations}, 5th edn., ed. Elizabeth Knowles (Oxford, 1999).
\tcblower
\begin{Verbatim}
@reference{knowles1999odq,
  title = {The {Oxford} Dictionary of Quotations},
  edition = {5},
  editor = {Elizabeth Knowles},
  publisher = {Oxford},
  date = {1999}}
\end{Verbatim}
\end{bibexbox}

\subsection{Both first and later edition}

\spec{Author, \emph{Title} (Year; nth \lit{edn.}, Location: Publisher, Year).}

\tip{Use the \code{orig} fields for the earlier edition.}

\begin{bibexbox}<OGS \S15.2.3>
{denniston1934gp}
J. D. Denniston, \emph{The Greek Particles} (1934; 2nd edn., Oxford, 1954).
\tcblower
\begin{Verbatim}
@book{denniston1934gp,
  author = {J. D. Denniston},
  title = {The Greek Particles},
  origdate = {1934},
  edition = {2},
  location = {Oxford},
  date = {1954}}
\end{Verbatim}
\end{bibexbox}

\tip{Use the localization key \code{revised} as the edition to get \enquote{revised edition}. Use the localization key \code{revisedenlarged} as the edition to get \enquote{revised and enlarged edition}}

\begin{bibexbox}<OGS \S15.2.1*>
{gibbon1686dfs}
{[John Gibbon]}, \emph{Day-Fatality: Or Some Observations on Days Lucky and Unlucky}, (London, 1678; rev.~edn., 1686).
\tcblower
\begin{Verbatim}
@book{gibbon1686dfs,
  author = {John Gibbon},
  author+an = {=inferred},
  title = {Day-Fatality},
  subtitle = {Or Some Observations on Days Lucky and Unlucky},
  edition = {revised},
  origlocation = {London},
  origdate = {1678},
  date = {1686}}
\end{Verbatim}
\end{bibexbox}

\info{In the \emph{Oxford Guide to Style}, there is no comma after \enquote{rev. edn.} in the above reference; this may be an error.}

\subsection{Republications}

If the publication has not been revised, but has been re-typeset by a different publisher,
the \emph{Oxford Guide to Style} suggests putting the original date in a note at the end,
rather than within the publication block%
.

\spec{Author, \emph{Title} (Location: Publisher, Year) (\lit{originally pub.} Year).}

\tip{This format is triggered if you specify an original date but do not specify the original publisher or location, nor the edition of the later publication.}
\info{In the following example, you will see the style uses \enquote{originally published} instead of \enquote{first published}. This is because \enquote{originally} is more common among the available examples.}

\begin{bibexbox}
<OGS \S15.2.3>
{bettelheim1976uem}
Bruno Bettelheim, \emph{The Uses of Enchantment: The Meaning and Importance of Fairy Tales} (Harmondsworth: Penguin Books, 1988) (first pub. 1976).
\tcblower
\begin{Verbatim}
@book{bettelheim1976uem,
  author = {Bruno Bettelheim},
  title = {The Uses of Enchantment},
  subtitle = {The Meaning and Importance of Fairy Tales},
  location = {Harmondsworth},
  publisher = {Penguin Books},
  date = {1988},
  origdate = {1976}}
\end{Verbatim}
\end{bibexbox}

\subsection{Reprints}

\spec{Author, \emph{Title} (Location: Publisher, Year; \lit{repr.} Location: Publisher, Year).}

\tip{Use localization key \code{reprint} as the edition. For a revised reprint, use \code{revisedreprint}.}

\begin{bibexbox}
<OGS \S15.2.4>
{adam.tannery1897odd}
C. Adam and D. Tannery (eds.), \emph{Œuvres de Descartes} (Paris: Cerf, 1897--1913; repr. Paris: J. Vrin, CNRS, 1964--76).
\tcblower
\begin{Verbatim}
@collection{adam.tannery1897odd,
  editor = {C. Adam and D. Tannery},
  title = {Œuvres de Descartes},
  origlocation = {Paris},
  origpublisher = {Cerf},
  origdate = {1897/1913},
  edition = {reprint},
  location = {Paris},
  publisher = {J. Vrin, CNRS},
  date = {1964/1976}}
\end{Verbatim}
\end{bibexbox}

\begin{bibexbox}
<OGS \S15.2.4>
{southern1991sap}
R. W. Southern, \emph{Saint Anselm: A Portrait in a Landscape} (rev. repr., Cambridge: Cambridge University Press, 1991).
\tcblower
\begin{Verbatim}
@book{southern1991sap,
  author = {R. W. Southern},
  title = {Saint Anselm},
  subtitle = {A Portrait in a Landscape},
  edition = {revisedreprint},
  location = {Cambridge},
  publisher = {Cambridge University Press},
  date = {1991}}
\end{Verbatim}
\end{bibexbox}

\spec{Author, \emph{Title} (Location: Publisher, Year; \lit{facs. edn.}, Location: Publisher, Year).}

\tip{Use localization key \code{facsimile} as the edition.}

\begin{bibexbox}
<OGS \S15.2.4>
{allen1594kkk}
E. Allen, \emph{A Knack to Know a Knave} (London, 1594; facs. edn., Oxford: Malone Society Reprints, 1963).
\tcblower
\begin{Verbatim}
@book{allen1594kkk,
  author = {E. Allen},
  title = {A Knack to Know a Knave},
  origlocation = {London},
  origdate = {1594},
  edition = {facsimile},
  location = {Oxford},
  publisher = {Malone Society Reprints},
  date = {1963}}
\end{Verbatim}
\end{bibexbox}

\subsection{Title changes}

\emph{New Hart's Rules} recommends giving the original publication details first,
then the new title and its details after a semicolon.

\tip{The standard \code{relatedtype} value \code{reprintas} is supported.}

\begin{bibexbox}
<NHR \S18.2.13>
{hare1949wwb}
Cyril Hare, \emph{When the Wind Blows} (London, 1949); repr. as \emph{The Wind Blows Death} (London, 1987).
\tcblower
\begin{Verbatim}
@book{hare1949wwb,
  author = {Cyril Hare},
  title = {When the Wind Blows},
  location = {London},
  date = {1949},
  related = {hare1987wbd},
  relatedtype = {reprintas}}
@book{hare1987wbd,
  author = {Cyril Hare},
  title = {The Wind Blows Death},
  location = {London},
  date = {1987}}
\end{Verbatim}
\end{bibexbox}

\tip{A more generalized version of this relation is available.
  If you set the \code{relatedtype} to the \pkg[biblatex]{oxref}-specific keyword \code{editedas},
  the edition and any editorial contributions will be printed in the linking text.}

\begin{bibexbox}
<NHR \S18.2.12*>
{berkenhout1769onh}
John Berkenhout, \emph{Outlines of the Natural History of Great Britain}, 3 vols. (London, 1769–72); rev. edn., as \emph{A Synopsis of the Natural History of Great Britain}, 2 vols. (London, 1789).
\tcblower
\begin{Verbatim}
@mvbook{berkenhout1769onh,
  author = {John Berkenhout},
  title = {Outlines of the Natural History of Great Britain},
  volumes = {3},
  location = {London},
  date = {1769/1772},
  related = {berkenhout1789snh},
  relatedtype = {editedas}}
@mvbook{berkenhout1789snh,
  author = {John Berkenhout},
  title = {A Synopsis of the Natural History of Great Britain},
  volumes = {2},
  edition = {revised},
  location = {London},
  date = {1789}}
\end{Verbatim}
\end{bibexbox}

\begin{bibexbox}
<NHR \S18.2.13>
{lower1665dtw}
Richard Lower, \emph{Diatribæ Thomæ Willisii Doct. Med. \& Profess. Oxon. De febribus Vindicatio adversus Edmundum De Meara Ormoniensem Hibernum M.D.} (London, 1665); facs. edn. with introduction, ed. and trans. Kenneth Dewhurst, as \emph{Richard Lower's \enquote{Vindicatio}: A Defence of the Experimental Method} (Oxford, 1983).
\tcblower
\begin{Verbatim}
@book{lower1665dtw,
  author = {Richard Lower},
  title = {Diatribæ Thomæ Willisii Doct. Med. \& Profess. Oxon. De febribus Vindicatio adversus
    Edmundum De Meara Ormoniensem Hibernum M.D.},
  location = {London},
  date = {1665},
  related = {dewhurst1983rlv},
  relatedtype = {editedas}}
@book{dewhurst1983rlv,
  author = {Richard Lower},
  introduction = {Kenneth Dewhurst},
  editor = {Kenneth Dewhurst},
  translator = {Kenneth Dewhurst},
  title = {Richard Lower's \enquote{Vindicatio}},
  subtitle = {A Defence of the Experimental Method},
  edition = {facsimile},
  location = {Oxford},
  date = {1983}}
\end{Verbatim}
\end{bibexbox}

\subsection{Co-publications\slash co-editions}

If the same book has been co-published by multiple publishers at approximately the same time,
you can express this by putting multiple sets of details in the publication block.
You should not do this, however, if you might want to switch to using the \textsf{oxyear} style at some point,
since it does not really work for author–year styles.

\spec{Author, \emph{Title} (Location: Publisher, Year; Location: Publisher, Year).}

\tip{Put each publication in the bib file separately. In the one you plan to cite, put the keys of the others in \code{related} and give \code{copub} as the \code{relatedtype}.}

\begin{bibexbox}
<OGS \S15.2.5>{holfordstrevens1988ag1}
L. A. Holford-Strevens, \emph{Aulus Gellius} (London: Duckworth, 1988; Chapel Hill: University of North Carolina Press, 1989).
\tcblower
\begin{Verbatim}
@book{holfordstrevens1988ag1,
  author = {L. A. Holford-Strevens},
  title = {Aulus Gellius},
  location = {London},
  publisher = {Duckworth},
  date = {1988},
  related = {holfordstrevens1988ag2},
  relatedtype = {copub}}
@book{holfordstrevens1988ag2,
  location = {Chapel Hill},
  publisher = {University of North Carolina Press},
  date = {1989}}
\end{Verbatim}
\end{bibexbox}

If the work is published under two different titles, it is a good idea to
provide both to make it easier to locate.

\tip{Use a generic relation and specify the country details in the \code{relatedstring}.}

\begin{bibexbox}
<OGS \S15.2.5>
{salinger1953ns}
J. D. Salinger, \emph{Nine Stories} (Boston: Little, Brown, 1953), published in the UK as \emph{For Esmé---With Love and Squalor, and Other Stories} (London: Hamish Hamilton, 1953).
\tcblower
\begin{Verbatim}
@book{salinger1953ns,
  author = {J. D. Salinger},
  title = {Nine Stories},
  location = {Boston},
  publisher = {Little, Brown},
  date = {1953},
  related = {salinger1953few},
  relatedstring = {published in the UK as},
  options = {relationpunct=comma}}
@book{salinger1953few,
  title = {For {Esmé}---With Love and Squalor, and Other Stories},
  location = {London},
  publisher = {Hamish Hamilton},
  date = {1953}}
\end{Verbatim}
\end{bibexbox}

\section{Works from a series}

\spec{Author, \emph{Title} (Series; Location: Publisher, Year).}
\spec{Author, \emph{Title} (Series, Number; Location: Publisher, Year).}

\begin{bibexbox}<OGS \S15.2.7>
{garlandia1972dmm}
Johannes de Garlandia, \emph{De mensurabili musica}, ed. Erich Reimer, 2 vols. (Beihefte zum Archiv für Musikwissenschaft, 10--11; Wiesbaden, 1972).
\tcblower
\begin{Verbatim}
@mvbook{garlandia1972dmm,
  author = {Johannes de Garlandia},
  title = {De mensurabili musica},
  editor = {Erich Reimer},
  volumes = {2},
  series = {Beihefte zum Archiv für Musikwissenschaft},
  number = {10--11},
  publisher = {Wiesbaden},
  date = {1972}}
\end{Verbatim}
\end{bibexbox}


\spec{Author, \emph{Title} (nth \lit{ser.}, Year).}

\section{Pre-publication book}

\spec{Author, \emph{Title} (Location: Publisher, \lit{forthcoming}).}


\chapter{Works within books}\label{sec:inbook}
\chapterprecis{inbook, bookinbook, suppbook, incollection, suppcollection, inreference}

\section{Works in collections of a single author's works}

The \code{inbook} entry type is intended for books that consist of several self-contained works by the same author, for citing one of these works.

\spec{Author, \enquote{Title}, \lit{in id.}, \emph{BookTitle}, \lit{ed.} Editor(s) (Location: Publisher, Year), Pages.}

\tip{If you provide \code{bookauthor}, and it is the same as the \code{author}, the second instance of the name will replaced by \enquote{id.}\slash \enquote{ead.}\slash \enquote{eid.}\slash \enquote{eaed.} Use the \code{gender} field to select which is used.}

\begin{bibexbox}
<OGS \S15.2.11*>
{kristeller1979thm}
Paul Oskar Kristeller, \enquote{The Aristotelian Tradition}, in id., \emph{Renaissance Thought and Its Sources}, ed. Michael Mooney (New York: Columbia University Press, 1979).
\tcblower
\begin{Verbatim}
@inbook{kristeller1979thm,
  author = {Paul Oskar Kristeller},
  gender = {sm},
  title = {The Aristotelian Tradition},
  bookauthor = {Paul Oskar Kristeller},
  booktitle = {Renaissance Thought and Its Sources},
  editor = {Michael Mooney},
  location = {New York},
  publisher = {Columbia University Press},
  date = {1979}}
\end{Verbatim}
\end{bibexbox}

\spec{Author, \enquote{Title}, \lit{in} \emph{BookTitle}, \lit{ed.} Editor(s) (Location: Publisher, Year), Pages.}

\begin{bibexbox}
<NHR \S18.2.6*>
{ashton1991d}
John Ashton, \enquote{Dualism}, in \emph{Understanding the Fourth Gospel} (Oxford, 1991), 205–37.
\tcblower
\begin{Verbatim}
@inbook{ashton1991d,
  author = {John Ashton},
  title = {Dualism},
  booktitle = {Understanding the Fourth Gospel},
  location = {Oxford},
  date = {1991},
  pages = {205-237}}
\end{Verbatim}
\end{bibexbox}

\section{Works in collections}

The \code{incollection} entry type is intended for citing one of a collection of self-contained works by different authors.

\subsection{Works in a mixed collection}

\spec{Author, \enquote{Title}, \lit{in} Editors (\lit{eds.}), \emph{BookTitle} (Location: Publisher, Year), Pages.}

\begin{bibexbox}<OGS \S15.3>
{shearman1993vsf}
John Shearman, \enquote{The Vatican Stanze: Functions and Decoration}, in George Holmes (ed.), \emph{Art and Politics in Renaissance Italy: British Academy Lectures} (Oxford: Clarendon Press, 1993), 185--240.
\tcblower
\begin{Verbatim}
@incollection{shearman1993vsf,
  author = {John Shearman},
  title = {The Vatican Stanze},
  subtitle = {Functions and Decoration},
  editor = {George Holmes},
  booktitle = {Art and Politics in Renaissance Italy},
  booksubtitle = {British Academy Lectures},
  location = {Oxford},
  publisher = {Clarendon Press},
  date = {1993},
  pages = {185-240}}
\end{Verbatim}
\end{bibexbox}


\subsection{Works by the editor in a mixed collection}

\spec{Author, \enquote{Title}, \lit{in id.} (\lit{ed.}), \emph{BookTitle} (Location: Publisher, Year), Pages.}

\tip{If you provide the same values for \code{author} and \code{editor}, the editor name will replaced by \enquote{id.}\slash \enquote{ead.}\slash \enquote{eid.}\slash \enquote{eaed.} Use the \code{gender} field to select which is used.}

\begin{bibexbox}
<OGS \S15.3>
{todd1974dhp}
W. B. Todd, \enquote{David Hume: A Preliminary Bibliography}, in id. (ed.), \emph{Hume and the Enlightenment: Essays Presented to Ernest Campbell Mossner} (Edinburgh: Edinburgh University Press, 1974).
\tcblower
\begin{Verbatim}
@incollection{todd1974dhp,
  author = {W. B. Todd},
  gender = {sm},
  title = {David Hume},
  subtitle = {A Preliminary Bibliography},
  editor = {W. B. Todd},
  booktitle = {Hume and the Enlightenment},
  booksubtitle = {Essays Presented to Ernest Campbell Mossner},
  location = {Edinburgh},
  publisher = {Edinburgh University Press},
  date = {1974}}
\end{Verbatim}
\end{bibexbox}

\section{Anthologies of independently published works}

Biblatex provides the \code{bookinbook} entry type for citing a part of an anthology that has previously been published as a book in its own right. The main difference from \code{inbook} is how the original publication information is handled.

\subsection{Later version more accessible}

\tip{Use \code{origdate} to give the date of original publication.}

\begin{bibexbox}
<OGS \S15.3>
{frege1892osr}
G. Frege, \enquote{On Sense and Reference}, in id., \emph{Philosophical Writings}, trans. and ed. P. T. Geach and M. Black (Oxford: Blackwell, 1952) (originally pub. 1892).
\tcblower
\begin{Verbatim}
@bookinbook{frege1892osr,
  author = {G. Frege},
  gender = {sm},
  title = {On Sense and Reference},
  bookauthor = {G. Frege},
  booktitle = {Philosophical Writings},
  translator = {P. T. Geach and M. Black},
  editor = {P. T. Geach and M. Black},
  location = {Oxford},
  publisher = {Blackwell},
  date = {1952},
  origdate = {1892}}
\end{Verbatim}
\end{bibexbox}

\subsection{Both versions equally accessible}

\tip{Put the key of the original work in \code{related}, and in \code{relatedtype} use the keyword \code{reprintfrom}.}

\begin{bibexbox}
<OGS \S15.3>
{owen1986pi}
G. E. L. Owen, \enquote{Philosophical Invective}, in id., \emph{Logic, Science and Dialectic}, ed. M. Nussbaum (Ithaca, NY: Cornell University Press, 1986), 347--64. From \emph{Oxford Studies in Ancient Philosophy}, 1 (1983), 1--25.
\tcblower
\begin{Verbatim}
@bookinbook{owen1986pi,
  author = {G. E. L. Owen},
  gender = {sm},
  title = {Philosophical Invective},
  booktitle = {Logic, Science and Dialectic},
  bookauthor = {G. E. L. Owen},
  editor = {M. Nussbaum},
  location = {Ithaca, NY},
  publisher = {Cornell University Press},
  date = {1986},
  pages = {347-364},
  related = {owen1983pi},
  relatedtype = {reprintfrom}}
@article{owen1983pi,
  author = {G. E. L. Owen},
  title = {Philosophical Invective},
  journaltitle = {Oxford Studies in Ancient Philosophy},
  volume = {1},
  date = {1983},
  pages = {1-25}}
\end{Verbatim}
\end{bibexbox}

\section{Articles in yearbooks and works of reference}

Yearbooks may be treated either as collections or periodicals; the distinction boils down to whether you want to print the location and publisher.

The \emph{Oxford Guide to Style} suggests citing entries in dictionaries and encyclopaedias using a postnote like \code{[s.v.\@ `Tawdry']}, but for online works gives a format rather similar to yearbooks, that is, without \enquote{in}.

\tip{To omit the \enquote{in} before the book title, either use the \code{inreference} entry type without specifying an editor, or use the \code{incollection} type with an \code{entrysubtype} of \code{yearbook}.}

\begin{bibexbox}
<OGS \S15.3>
{dahlhaus1971mmf}
Carl Dahlhaus, \enquote{Miszellen zur Musiktheorie des 15.\@ Jahrhunderts}, \emph{Jahrbuch des Staatlichen Instituts für Musikforschung Preußischer Kulturbesitz, 1970} (Berlin, 1971), 21–33.
\tcblower
\begin{Verbatim}
@inreference{dahlhaus1971mmf,
  author = {Carl Dahlhaus},
  title = {Miszellen zur Musiktheorie des 15.\@ Jahrhunderts},
  booktitle = {Jahrbuch des Staatlichen Instituts für Musikforschung Preußischer Kulturbesitz,
    1970},
  location = {Berlin},
  date = {1971},
  pages = {21-33}}
\end{Verbatim}
\end{bibexbox}

\section{Supplementary works in books and collections}

If the work is headed \enquote{Introduction}, \enquote{Foreword}, \enquote{Afterword}, or similar,
this is technically a descriptor rather than a title.

\tip{If the work has a true title, the descriptor goes in the \code{note} field.
  If it doesn't, the descriptor goes in the \code{title} field; annotate the field with the keyword \code{descriptor} to correct the formatting.
  Alternatively, use the (non-standard) \code{descriptor} field for the descriptor in all cases, and \pkg[biblatex]{oxref} will take care of this for you.}

\begin{bibexbox}
<NHR \S18.2.6*>
{gill1987intro}
Roma Gill, introduction in \emph{The Complete Works of Christopher Marlow}, i (Oxford, 1987; repr.\@ 2001).
\tcblower
\begin{Verbatim}
@suppbook{gill1987intro,
  author = {Roma Gill},
  title = {introduction},
  title+an = {=descriptor},
  booktitle = {The Complete Works of Christopher Marlow},
  volume = {1},
  origlocation = {Oxford},
  origdate = {1987},
  edition = {reprint},
  date = {2001}}
\end{Verbatim}
\end{bibexbox}

\begin{bibexbox}
{atwan2000fw}
(not in book)
\tcblower
\begin{Verbatim}
@suppcollection{atwan2000fw,
  author = {Robert Atwan},
  title = {foreword},
  title+an = {=descriptor},
  address = {Boston},
  booktitle = {The Best American Essays of the Century},
  date = {2000},
  editor = {Joyce Carol Oates and Robert Atwan},
  publisher = {Houghton},
  pages = {x-xvi}}
\end{Verbatim}
\end{bibexbox}

It is quite rare for such items to have a true title in addition, so previous
versions of \pkg[biblatex]{oxref} treated all titles of \texttt{suppbook}
entries as descriptors.

\tip{To restore the previous behaviour, include the following code:}

\begin{tcblisting}{listing only}
\DeclareFieldFormat[suppbook]{title}{#1}
\end{tcblisting}

\chapter{Works presented at meetings}\label{sec:proceedings}
\chapterprecis{proceedings, mvproceedings, inproceedings, unpublished}

\section{Single volume proceedings}

The \code{proceedings} entry type is intended for a set of conference papers that have been collected together into a single volume.

\spec{\emph{Title}, \lit{ed.} Editor (Location: Publisher, Year).}

\begin{bibexbox}
<OGS \S15.2.12>
{icp1975pic}
\emph{Proceedings of the XIV International Congress of Papyrologists: Oxford, 24--31 July 1974} (London, 1975).
\tcblower
\begin{Verbatim}
@proceedings{icp1975pic,
  title = {Proceedings of the {XIV} {International} {Congress} of {Papyrologists}},
  subtitle = {{Oxford}, 24--31 {July} 1974},
  location = {London},
  date = {1975}}
\end{Verbatim}
\end{bibexbox}

\info{The \code{organization} is automatically converted into an \code{author}.}

\begin{egcite}{\dots Imperial College Bookstall, 1922), 1.52–3}
\cite[1.52-3]{iau1922tia}
\end{egcite}

\begin{bibexbox}
<OGS \S15.2.13>
{iau1922tia}
International Astronomical Union, \emph{Transactions of the International Astronomical Union, Rome}, 12--20 May 1922 (London: Imperial College Bookstall, 1922).
\tcblower
\begin{Verbatim}
@proceedings{iau1922tia,
  organization = {{International Astronomical Union}},
  title = {Transactions of the {International} {Astronomical} {Union}, {Rome}},
  eventdate = {1922-05-12/1922-05-20},
  location = {London},
  publisher = {Imperial College Bookstall},
  date = {1922}}
\end{Verbatim}
\end{bibexbox}

Sometimes the title of the proceedings does not relate to the conference; in which case the following form might be used:

\spec{\emph{Title}, EventTitle, Venue, ConfDate, \lit{ed.} Editor (Series, Number; Location: Publisher, Year).}

\begin{bibexbox}
{ecdl2009}
(not in book)
\tcblower
\begin{Verbatim}
@proceedings{ecdl2009,
  editor = {Maristella Agosti and José Borbinha and Sarantos Kapidakis and Christos
    Papatheodorou and Giannis Tsakonas},
  title = {Research and Advanced Technology for Digital Libraries},
  eventtitle = {13th European Conference, ECDL 2009},
  venue = {Corfu, Greece},
  eventdate = {2009-09-27/2009-10-02},
  series = {Lecture Notes in Computer Science},
  number = {5714},
  location = {Berlin},
  publisher = {Springer},
  date = {2009}}
\end{Verbatim}
\end{bibexbox}

\section{Multi-volume proceedings}

\textsf{Biblatex} provides \code{mvproceedings} for multi-volume proceedings.
The examples here are a combination of Oxford style for proceedings and multi-volume books.

\subsection{All volumes}

\spec{\emph{Title}, EventTitle, Venue, ConfDate, \lit{ed.} Editor, n \lit{vols.} (Series, Number; Location: Publisher, Year).}

\begin{bibexbox}
{iced2009}
(not in book)
\tcblower
\begin{Verbatim}
@mvproceedings{iced2009,
  title = {Proceedings of the 17th International Conference on Engineering Design (ICED ’09)},
  volumes = {10},
  editor = {Margareta Norell Bergendahl and Martin Grimheden and Larry Leifer},
  venue = {Stanford, CA},
  eventdate = {2009-08-24/2009-08-27},
  location = {Glasgow},
  publisher = {Design Society},
  date = {2009},
  isbn = {978-1-904670-12-4}}
\end{Verbatim}
\end{bibexbox}

\subsection{One volume}

\spec{\emph{Title}, VolNo: \emph{VolTitle}, EventTitle, Venue, ConfDate, \lit{ed.} Editor (Series, Number; Location: Publisher, Year).}

\begin{bibexbox}
{iced2009dik}
(not in book)
\tcblower
\begin{Verbatim}
@mvproceedings{iced2009dik,
  maintitle = {Proceedings of the 17th International Conference on Engineering Design (ICED ’09)},
  title = {Design Information and Knowledge},
  volume = {8},
  editor = {Margareta Norell Bergendahl and Martin Grimheden and Larry Leifer},
  venue = {Stanford, CA},
  eventdate = {2009-08-24/2009-08-27},
  location = {Glasgow},
  publisher = {Design Society},
  date = {2009},
  isbn = {978-1-904670-12-4}}
\end{Verbatim}
\end{bibexbox}

\section{Published conference paper}

The \code{inproceedings} entry type is intended for a conference paper published as part of a proceedings volume. For papers that have not been collected into a proceedings volume, or presentation slide sets, use the \code{unpublished} entry type instead (see below).

\spec{Author, \enquote{Title}, \lit{in} \emph{BookTitle}, EventTitle, Venue, ConfDate, \lit{ed.} Editor (Series, Number; Location: Publisher, Year).}

\begin{bibexbox}
{tonkin.strelnikov2009iem}
(not in book)
\tcblower
\begin{Verbatim}
@inproceedings{tonkin.strelnikov2009iem,
  author = {Emma Tonkin and Alexey Strelnikov},
  title = {Information Environment Metadata Schema Registry},
  editor = {Maristella Agosti and José Borbinha and Sarantos Kapidakis and Christos Papatheodorou and Giannis Tsakonas},
  booktitle = {Research and Advanced Technology for Digital Libraries},
  eventtitle = {13th European Conference, ECDL 2009},
  venue = {Corfu, Greece},
  eventdate = {2009-09-27/2009-10-02},
  series = {Lecture Notes in Computer Science},
  number = {5714},
  location = {Berlin},
  publisher = {Springer},
  date = {2009},
  pages = {487-488},
  isbn = {978-3-642-04345-1},
  issn = {0302-9743}}
\end{Verbatim}
\end{bibexbox}

\spec{Author, \enquote{Title}, \lit{in} \emph{BookTitle}, EventTitle, Venue, ConfDate, \lit{ed.} Editor, n \lit{vols.} (Series, Number; Location: Publisher, Year), VolNo. Pages.}

\begin{bibexbox}
{ding.etal2009sfc}
(not in book)
\tcblower
\begin{Verbatim}
@inproceedings{ding.etal2009sfc,
  author = {Lian Ding and Alex Ball and Manjula Patel and Jason Matthews and Glen Mullineux},
  title = {Strategies for the Collaborative Use of CAD Product Models},
  maintitle = {Proceedings of the 17th International Conference on Engineering Design (ICED ’09)},
  booktitle = {Design Information and Knowledge},
  volume = {8},
  editor = {Margareta Norell Bergendahl and Martin Grimheden and Larry Leifer},
  venue = {Stanford, CA},
  eventdate = {2009-08-24/2009-08-27},
  location = {Glasgow},
  publisher = {Design Society},
  date = {2009},
  pages = {123-134},
  isbn = {978-1-904670-12-4},
  url = {http://opus.bath.ac.uk/14285}}
\end{Verbatim}
\end{bibexbox}

\section{Published orations, addresses, lectures, and speeches}

\spec{Author, \emph{Title}, Type, Venue, Date (Location: Publisher, Year).}

\begin{bibexbox}
<OGS \S15.6>
{gombrich1957as}
E. H. Gombrich, \emph{Art and Scholarship}, Inaugural Lecture, University College London, 14 Feb. 1957 (London: H. K. Lewis, 1957).
\tcblower
\begin{Verbatim}
@proceedings{gombrich1957as,
  author = {E. H. Gombrich},
  title = {Art and Scholarship},
  eventtitle = {Inaugural Lecture},
  venue = {University College London},
  eventdate = {1957-02-14},
  location = {London},
  publisher = {H. K. Lewis},
  date = {1957}}
\end{Verbatim}
\end{bibexbox}

\tip{Use the \texttt{eventtitle} field to indicate the type of the oration.}

\section{Unpublished conference paper}

\spec{Author, \enquote{Title}, \lit{paper given at the} ConfTitle, Venue, EventDate.}

\begin{bibexbox}<OGS \S15.6>
{holfordstrevens1999hlm}
Leofranc Holford-Strevens, \enquote{Humanism and the Language of Music Theory Treatises}, paper given at the 65th Annual Meeting of the American Musicological Society, Kansas City, MO, 4--7 Nov. 1999.
\tcblower
\begin{Verbatim}
@unpublished{holfordstrevens1999hlm,
  author = {Leofranc Holford-Strevens},
  title = {Humanism and the Language of Music Theory Treatises},
  howpublished = {paper given at the 65th Annual Meeting of the American Musicological Society},
  location = {Kansas City, MO},
  date = {1999-11-04/1999-11-07}}
\end{Verbatim}
\end{bibexbox}


\section{Unpublished orations, addresses, lectures, and speeches}

\spec{Author, \enquote{Title}, Description.}


\chapter{Grey literature}\label{sec:report}
\chapterprecis{booklet, manual, patent, report, thesis, standard}

\section{One-off reports}

\spec{Author, \emph{Title} (Location: Institution, Year).}

\begin{bibexbox}
<OGS \S15.2.13>
{botswana1980hdr}
Government of Botswana, \emph{A Human Drought Relief Programme for Botswana} (Gabarone: Ministry of Local Government and Lands, 1980).
\tcblower
\begin{Verbatim}
@report{botswana1980hdr,
  author = {{Government of Botswana}},
  title = {A Human Drought Relief Programme for Botswana},
  location = {Gabarone},
  institution = {Ministry of Local Government and Lands},
  date = {1980}}
\end{Verbatim}
\end{bibexbox}

\tip{If the report does not specify an author, give the publishing institution in the \texttt{author} field.
  Some examples in the \emph{Oxford Guide to Style} repeat the institution in the publisher slot
  (i.e.\@ the \texttt{institution} field) and some do not.}

\begin{bibexbox}<OGS \S15.2.13>
{pac1988upt}
Penal Affairs Consortium, \emph{An Unsuitable Place for Treatment: Diverting Mentally Disordered Offenders from Custody} (London, 1988).
\tcblower
\begin{Verbatim}
@report{pac1988upt,
  author = {{Penal Affairs Consortium}},
  title = {An Unsuitable Place for Treatment},
  subtitle = {Diverting Mentally Disordered Offenders from Custody},
  location = {London},
  date = {1988}}
\end{Verbatim}
\end{bibexbox}

\begin{bibexbox}<NHR \S18.8.5>
{unesco2012unw}
UNESCO, \emph{The United Nations World Water Development Report 4}, vol. 1: \emph{Managing Water under Uncertainty and Risk} (Paris: UNESCO, 2012), \url{http://unesdoc.unesco.org/images/0021/002156/215644e.pdf}, accessed 9 Nov. 2013.
\tcblower
\begin{Verbatim}
@report{unesco2012unw,
  author = {UNESCO},
  maintitle = {The United Nations World Water Development Report 4},
  volume = {1},
  title = {Managing Water under Uncertainty and Risk},
  location = {Paris},
  institution = {UNESCO},
  date = {2012},
  url = {http://unesdoc.unesco.org/images/0021/002156/215644e.pdf},
  urldate = {2013-11-09}}
\end{Verbatim}
\end{bibexbox}


\section{Reports with a type or series}

\tip{The number of a report will only be printed if you also supply a type or
  series, or both. The difference between the two is that the series will be
  separated from the (type and) number by a comma, while only a space
  separates the type and number. In addition, if you supply a series but no
  type, the number will be prefixed by `No.\@' or the localized equivalent.}

\spec{Author, \emph{Title}, Series, Type Number (Location: Institution, Year).}


\section{Jointly published reports}

It is more common with reports than with books for a work to be branded
jointly by several issuing organizations. Neither the \emph{Oxford Guide to
Style} nor \emph{New Hart's Rules} give explicit guidance on this, but I
suggest handling it by giving corresponding lists as the \code{location}
and \code{institution}:

\begin{tcblisting}{listing only}
  location = {Place 1 and Place 2 and Place 3},
  institution = {Organization 1 and Organization 2 and Organization 3},
\end{tcblisting}

This will be printed as follows:

\begin{tcolorbox}
(Place 1: Organization 1, Place 2: Organization 2, and Place 3: Organization 3)
\end{tcolorbox}

\section{Theses}

\spec{Author, \enquote{Title}, Type (Institution, Year).}

\begin{bibexbox}
<OGS \S15.5.1>
{blackburn1970tlp}
Bonnie J. Blackburn, \enquote{The Lupus Problem}, Ph.D. diss. (University of Chicago, 1970).
\tcblower
\begin{Verbatim}
@thesis{blackburn1970tlp,
  author = {Bonnie J. Blackburn},
  title = {The Lupus Problem},
  type = {Ph.D. diss.},
  institution = {University of Chicago},
  date = {1970}}
\end{Verbatim}
\end{bibexbox}

The following forms are also supported; to use them, pass the \key{thesis} option:

\begin{itemize}
  \item \key{thesis}\texttt{=}\val{in}\par
\spec{Author, \enquote{Title} (Type, Institution, Year).}
  \item \key{thesis}\texttt{=}\val{plain}\par
\spec{Author, \enquote{Title}, Type, Institution, Year.}
\end{itemize}

\section{Booklets, leaflets, and other formal but unpublished texts}

There are no obvious examples of this kind of material in the \emph{Oxford Guide
to Style} or \emph{New Hart's Rules}. I welcome suggestions for how to make the
generated references more Oxford-like.

\begin{bibexbox}
{msbh1921fvd}
(not in book)
\tcblower
\begin{Verbatim}
@booklet{msbh1921fvd,
  title = {The Facts about Venereal Diseases},
  date = {1921},
  howpublished = {distributed by the Missouri State Board of Health, Division of Venereal Diseases},
  location = {Jefferson City, MO}}
\end{Verbatim}
\end{bibexbox}

\begin{bibexbox}
{lloyds2015m}
(not in book)
\tcblower
\begin{Verbatim}
@booklet{lloyds2015m,
  author = {{Lloyds TSB Bank plc}},
  title = {Mortgages},
  date = {2015},
  howpublished = {obtained in Paisley branch}}
\end{Verbatim}
\end{bibexbox}

\section{Patents}

There are no examples of patents in the \emph{Oxford Guide to Style} or
\emph{New Hart's Rules}. I welcome suggestions for how to make the
generated references more Oxford-like.

Following the conventions from \pkg{biblatex-chicago}, use \texttt{origdate} for
the date the patent was filed (or the application was published), and
\texttt{date} for the date the patent was finally issued or published.

\tip{You can override the default date descriptions with the (non-standard)
\code{origdatetype} and \code{datetype} fields.}

\begin{bibexbox}
{petroff.stapelbroek1980bib}
(not in book)
\tcblower
\begin{Verbatim}
@patent{petroff.stapelbroek1980bib,
  author = {Petroff, M.~D. and Stapelbroek, M.~G.},
  title = {Blocked impurity band detectors},
  date = {1986-02-04},
  origdate = {1980-10-23},
  type = {patentus},
  number = {4,586,960},
  location = {Washington, DC},
  publisher = {U.S. Patent and Trademark Office}}
\end{Verbatim}
\end{bibexbox}

\begin{bibexbox}
{arduengo.etal2001pmi}
(not in book)
\tcblower
\begin{Verbatim}
@patent{arduengo.etal2001pmi,
  author = {Arduengo, III, Anthony J. and Gentry, Jr., Frederick P.
    and Taverkere,  Prakash~K. and Simmons, III, Howard E.},
  title = {Process for manufacture of imidazoles},
  year = {2001},
  type = {patentus},
  holder = {{E.~I. DuPont}},
  number = {6177575}}
\end{Verbatim}
\end{bibexbox}

\begin{bibexbox}
{pm1981opa}
(not in book)
\tcblower
\begin{Verbatim}
@patent{pm1981opa,
  author = {{Phillipp Morris Inc.}},
  title = {Optical perforating apparatus and system},
  origdate = {1981-01-07},
  date = {1985-05-15},
  type = {patenteu},
  number = {EP0021165}}
\end{Verbatim}
\end{bibexbox}

\begin{bibexbox}
{hideki1992qmc}
(not in book)
\tcblower
\begin{Verbatim}
@patent{hideki1992qmc,
  author = {U. Hideki},
  title = {Quadrature Modulation Circuit},
  type = {patentjp},
  number = {152932/92},
  date = {1992-05-20}}
\end{Verbatim}
\end{bibexbox}

\section{Standards}

There are no examples of technical standards in the \emph{Oxford Guide
to Style} or \emph{New Hart's Rules}. I welcome suggestions for how to make the
generated references more Oxford-like.

The particular quirk with standards entries is that the number is promoted to
the head of the reference if an author is not specified (or \key{useauthor} is
\val{false}). If this happens, the number is copied to \code{sortkey} so the
entry appears in the right place in the reference list, unless you get there
first with your own value of \code{sortkey}. You may want to do this if, for
example, you want \enquote{ISO 1000} to come after \enquote{ISO 999}.

\tip{You are encouraged to put the standards body in \code{organization}, but
  you can use \code{publisher} instead.}

\tip{The examples don't demonstrate it, but you can use the \code{version} field
  for the version of the standard.}

\begin{bibexbox}
{bs5605:1990}
(not in book)
\tcblower
\begin{Verbatim}
@standard{bs5605:1990,
  sortyear = {1990},
  title = {Recommendations for citing and referencing published material},
  number = {BS~5605:1990},
  organization = {British Standards Institute}}
\end{Verbatim}
\end{bibexbox}


\begin{bibexbox}
{w3c2017html5.2}
(not in book)
\tcblower
\begin{Verbatim}
@standard{w3c2017html5.2,
  title = {{HTML} 5.2},
  editor = {Steve Faulkner and Arron Eicholz and Travis Leithead and
    Alex Danilo and Sangwhan Moon},
  date = {2017-12-14},
  type = {W3C Recommendation},
  organization = {World Wide Web Consortium},
  url = {https://www.w3.org/TR/html52/},
  urldate = {2018-02-17}}
\end{Verbatim}
\end{bibexbox}

\begin{bibexbox}
{ietf.rfc1155}
(not in book)
\tcblower
\begin{Verbatim}
@standard{ietf.rfc1155,
  author = {M. T. Rose and K. McCloghrie},
  title = {Structure and identification of management information for {TCP/IP}-based internets},
  date = {1990-05},
  organization = {Internet Engineering Task Force},
  series = {Request for Comments},
  type = {Internet Standard},
  number = {RFC~1155/STD~16},
  doi = {10.17487/RFC1155}}
\end{Verbatim}
\end{bibexbox}

\chapter{Audiovisual materials}\label{sec:audiovideo}
\chapterprecis{artwork, audio, inaudio, image, movie, music, inmusic, performance, video}

There are typically many contributors to an audiovisual work, and the priority given to each role is fluid. \pkg[biblatex]{Oxref} uses \code{author} for contributors that should go at the head of the reference, and \code{editor} for others. As well as the normal ones you can use the author\slash editor types \code{performer}, \code{conductor}, \code{director} and \code{reader}.

\section{Audio recordings}

The basic structure of audio recording entries is as follows:

\spec{Composer, \emph{Title}, Artist, RecordingDate (RecordingCompany Type Number, PublicationDate) [Medium].}

If the composer's name is not of interest, it may be omitted or replaced by the artist's name.

\tip{Use \code{origdate} to specify the date of recording. If you want it to be introduced by something other than the localized string \enquote{recorded}, use the (non-standard) \code{origdatetype} field%
.}

The \emph{Oxford Guide to Style} offers several ways of expressing the recording company:

\begin{itemize}
\item
  a simple label name – e.g.\@ \enquote{Columbia}, \enquote{Hyperion}, \enquote{Archiv} – followed immediately by the type or number
\item
  a label name separated from the number by a comma, e.g.\@ \enquote{Valois,}
  \enquote{EMI Electrola,} \enquote{EMI Blues Series,}
\item
  a compiling organization, semicolon, publishing company, no comma, e.g.\@ \enquote{Smithsonian Institution; Columbia}
\item
  a publishing company, colon, label name, comma, e.g.\@ \enquote{Decca Record Company, Ltd.: Éditions de l'Oiseau-lyre,}
\item
  a publishing company, comma, location, comma, e.g.\@ \enquote{Christophorus-Verlag Herder, Freiburg im Breisgau,}
\end{itemize}

\emph{New Hart's Rules} sticks firmly with the first of these, so \pkg[biblatex]{oxref} makes this the easiest choice but provides workarounds to allow you to achieve most of the others if you really need to.

\tip{To get the normal format, put the label name in \code{publisher}, the type of release (e.g.\@ compact disc, audio cassette) in \code{type}, and the catalogue number in \code{number}. In the examples from \emph{New Hart's Rules}, the type is uniformly omitted.}

\tip{To put a comma between the label and the type\slash number, put the label name in \code{series} instead.}

\tip{If you provide both a \code{publisher} and a \code{series}, they will be separated by \cs{recordseriespunct}, which is initially set to \cs{addcomma}\cs{space}. You could set it to \cs{addcolon}\cs{space} instead to match the example.}

\tip{If the compiling organization is distinct from the publisher, put the former in \code{origpublisher}; it will be separated from the publisher by a semicolon.}

\tip{Contrary to the OGS example, but in common with other entry types, the location if provided will be printed before the publisher, separated from it by a colon. (This is why a comma is the default for \cs{recordseriespunct}.) In the absence of a publisher it will be followed by a comma.}

\begin{bibexbox}
<NHR \S18.7.2*>
{carter1991fsq}
Elliott Carter, \emph{The Four String Quartets}, Juilliard String Quartet (Sony S2K 47229, 1991).
\tcblower
\begin{Verbatim}
@music{carter1991fsq,
  author = {Elliott Carter},
  title = {The Four String Quartets},
  editor = {{Juilliard String Quartet}},
  editortype = {performer},
  publisher = {Sony},
  number = {S2K 47229},
  date = {1991}}
\end{Verbatim}
\end{bibexbox}

\begin{bibexbox}
<OGS \S15.14.1>
{hillier1989p}
Paul Hillier, \emph{Proensa} (ECM Records compact disc ECM 1368, 1989).
\tcblower
\begin{Verbatim}
@music{hillier1989p,
  author = {Paul Hillier},
  title = {Proensa},
  publisher = {ECM Records},
  type = {compact disc},
  number = {ECM 1368},
  date = {1989}}
\end{Verbatim}
\end{bibexbox}

\begin{bibexbox}
<OGS \S15.14.1>
{couperin1970pdc}
Francois Couperin, \emph{Pièces de clavecin: Huit préludes de L'Art de toucher le clavecin. Livre I. Troisième et quatrième ordres}, Huguette Dreyfus (Valois, MB 797, 1970).
\tcblower
\begin{Verbatim}
@music{couperin1970pdc,
  author = {Francois Couperin},
  title = {Pièces de clavecin},
  subtitle = {Huit préludes de L'Art de toucher le clavecin. Livre I. Troisième et quatrième
    ordres},
  editor = {Huguette Dreyfus},
  editortype = {performer},
  series = {Valois},
  number = {MB 797},
  date = {1970}}
\end{Verbatim}
\end{bibexbox}

\begin{bibexbox}
<OGS \S15.14.1>
{pageNDmn}
\emph{The Mirror of Narcissus: Songs by Guillaume de Machaut}, Gothic Voices, dir. Christopher Page (Hyperion compact disc CDA 66087).
\tcblower
\begin{Verbatim}
@music{pageNDmn,
  title = {The Mirror of Narcissus},
  subtitle = {Songs by Guillaume do Machaut},
  editor = {{Gothic Voices}},
  editortype = {performer},
  editora = {Christopher Page},
  editoratype = {director},
  publisher = {Hyperion},
  type = {compact disc},
  number = {CDA 66087}}
\end{Verbatim}
\end{bibexbox}

\tip{To describe the medium in a separate note, use the \texttt{howpublished} field.}

\begin{bibexbox}
<NHR \S18.7.2>
{hopkinsNDcar}
Lightnin' Hopkins, \emph{The Complete Aladdin Recordings} (EMI Blues Series, CDP-7-96843-2) [2-vol.\@ CD set].
\tcblower
\begin{Verbatim}
@music{hopkinsNDcar,
  author = {{Lightnin’ Hopkins}},
  title = {The Complete Aladdin Recordings},
  series = {EMI Blues Series},
  number = {CDP-7-96843-2},
  howpublished = {2-vol.\@ CD set}}
\end{Verbatim}
\end{bibexbox}

\tip{To describe the medium as part of the publication statement, use the \texttt{type} field.}

\begin{bibexbox}
<OGS \S15.14.1>
{lewis1981lww}
C. S. Lewis, \emph{The Lion, the Witch, and the Wardrobe}, read by Sir Michael Hordern (2 audio cassettes, TO1611, 1981).
\tcblower
\begin{Verbatim}
@audio{lewis1981lww,
  author = {C. S. Lewis},
  title = {The Lion, the Witch, and the Wardrobe},
  editor = {title=Sir, given=Michael, family=Hordern},
  editortype = {reader},
  type = {2 audio cassettes},
  number = {TO1611},
  date = {1981}}
\end{Verbatim}
\end{bibexbox}

\subsection{Combination releases}

Sometimes publishers will put several works on a single release without giving it an overall title.

\tip{Put the subsequent composer\slash title combinations in new entries, and relate them with a \code{relatedtype} of \code{includes}.}

\begin{bibexbox}
<NHR \S18.7.2*>
{dutilleux1987as}
Henri Dutilleux, \emph{L'Arbre des songes}, and Peter Maxwell Davies, \emph{Concerto for Violin and Orchestra}, Royal Philharmonic Orchestra, cond. André Previn, violin Isaac Stern (CBS MK 42449, 1987).
\tcblower
\begin{Verbatim}
@audio{dutilleux1987as,
  author = {Henri Dutilleux},
  title = {L'Arbre des songes},
  related = {maxwelldavies1987},
  relatedtype = {includes},
  editor = {{Royal Philharmonic Orchestra}},
  editortype = {performer},
  editora = {André Previn},
  editoratype = {conductor},
  editorb = {Isaac Stern},
  editorbtype = {violin},
  publisher = {CBS},
  number = {MK 42449},
  date = {1987}}
@audio{maxwelldavies1987,
  author = {Maxwell Davies, Peter},
  title = {Concerto for Violin and Orchestra},
  date = {1987}}
\end{Verbatim}
\end{bibexbox}

\hack{Only the primary work will show up in citations; the subsequent ones will not.
  If this is not what you want, you will need to use \code{shortauthor}\slash\code{shorttitle} or \code{shorthand}, or build up the citation manually with \cs{citeauthor} and the like.}

\info{Depending on the thrust of your document, you could remove the complication by shifting the composers into the title and putting the performers in the author position.}

\info{If you are only interested in one of the works in the release, you might be better off using the \code{inaudio} entry type instead (see below).}

\subsection{Tracks from an album}

\tip{Use the non-standard entry types \code{inaudio} or \code{inmusic} to reference a single work within an album. Use \code{booktitle} (somewhat loosely) for the album title.}

\begin{bibexbox}
<NHR \S18.7.2*>
{davis1997sw}
Miles Davis et al., \enquote{So What}, in \emph{Kind of Blue}, rec. 1959 (Columbia CK 64935, 1997) [CD].
\tcblower
\begin{Verbatim}
@inaudio{davis1997sw,
  author = {Miles Davis and others},
  title = {So What},
  booktitle = {Kind of Blue},
  origdate = {1959},
  origdatetype = {rec\adddot},
  publisher = {Columbia},
  number = {CK 64935},
  date = {1997},
  howpublished = {CD}}
\end{Verbatim}
\end{bibexbox}

\info{Note the use of \code{origdate} for the date of recording in this example.}

\section{Video recordings}

\spec{\emph{Title} (Publisher, Number, Year).}

\begin{bibexbox}
<OGS \S15.14.2>
{bbc1987ava}
\emph{The Ashes: Victory in Australia} (BBCV 4040, 1987).
\tcblower
\begin{Verbatim}
@video{bbc1987ava,
  title = {The {Ashes}},
  subtitle = {Victory in {Australia}},
  publisher = {BBCV},
  number = {4040},
  date = {1987}}
\end{Verbatim}
\end{bibexbox}

\spec{\enquote{EpisodeTitle}, \emph{SeriesTitle} \lit{series} (Publisher, Number, Year).}

\tip{For an episode in a series, set the \code{entrysubtype} to \code{episode} and put the series name in \code{maintitle}.}

\begin{bibexbox}
<OGS \S15.14.2>
{britt1986pho}
\enquote{Percy and Harold and Other Stories}, \emph{Thomas the Tank Engine and Friends} series (Britt Allcroft 5-014861-100224, 1986).
\tcblower
\begin{Verbatim}
@video{britt1986pho,
  entrysubtype = {episode},
  title = {{Percy} and {Harold} and Other Stories},
  maintitle = {Thomas the Tank Engine and Friends \textup{series}},
  publisher = {Britt Allcroft},
  number = {5-014861-100224},
  date = {1986}}
\end{Verbatim}
\end{bibexbox}

\spec{Director (\lit{dir.}), \emph{Title} (Publisher, Number, Year).}

\tip{Use the \texttt{author} field to place a credit at the head of the reference.}

\begin{bibexbox}
<OGS \S15.14.2>
{chaplin1936mt}
Charles Chaplin (dir.), \emph{Modern Times} (United Artists, 1936).
\tcblower
\begin{Verbatim}
@movie{chaplin1936mt,
  author = {Charles Chaplin},
  authortype = {director},
  title = {Modern Times},
  publisher = {United Artists},
  date = {1936}}
\end{Verbatim}
\end{bibexbox}

\spec{\emph{Title}, \lit{dir.} Director (Publisher, Number, Year).}

\tip{Use the \texttt{editor} field to place a credit after the title.}

\begin{bibexbox}
<OGS \S15.14.2>
{reiner1983tst}
\emph{This is Spinal Tap: A Rockumentary by Marti Di Bergi}, dir.\ Rob Reiner (Embassy Pictures, 1983).
\tcblower
\begin{Verbatim}
@movie{reiner1983tst,
  title = {This is {Spinal} {Tap}},
  subtitle = {A Rockumentary by {Marti} {Di} {Bergi}},
  editor = {Rob Reiner},
  editortype = {director},
  publisher = {Embassy Pictures},
  date = {1983}}
\end{Verbatim}
\end{bibexbox}

\spec{\emph{Title} (Publisher, Number, Year), Credit.}

\tip{You can place up to four credits in the editor position, more if any of them correspond to the built-in editor types like \texttt{commentator} or \texttt{translator}.
  If you do supply a lot of credits, the \emph{Oxford Guide to Style} recommends you shift them to the end of the reference.
  You can do this with the \key{endeditor} option.}

\begin{bibexbox}
<OGS \S15.14.2>
{kaurismaki1989lcg}
\emph{Leningrad Cowboys Go America} (Villealfa Filmproductions, 1989), dir.\@ and screenplay by Aki Kaurismäki, story by Sakke Järvenpää, Aki Kaurismäki, Mato Valtonen.
\tcblower
\begin{Verbatim}
@movie{kaurismaki1989lcg,
  title = {Leningrad Cowboys Go America},
  publisher = {Villealfa Filmproductions},
  date = {1989},
  editortype = {dir.\ and screenplay by},
  editor = {Aki Kaurismäki},
  editoratype = {story by},
  editora = {Sakke Järvenpää and Aki Kaurismäki and Mato Valtonen},
  options = {endeditor}}
\end{Verbatim}
\end{bibexbox}

\tip{For an online video, set the \code{entrysubtype} to \code{clip}.}

\begin{bibexbox}
<NHR \S18.8.5>
{bbc2013iim}
BBC News, \enquote{Inside India's Mars Mission HQ} [video] (5 Nov. 2013), \url{http://www.bbc.co.uk/news/world-24826253}, accessed 5 Nov. 2013.
\tcblower
\begin{Verbatim}
@video{bbc2013iim,
  entrysubtype = {clip},
  author = {{BBC News}},
  title = {Inside India's Mars Mission HQ},
  date = {2013-11-05},
  url = {http://www.bbc.co.uk/news/world-24826253},
  urldate = {2013-11-05}}
\end{Verbatim}
\end{bibexbox}

\begin{bibexbox}
<NHR \S18.8.5>
{rubinstein1956rpc}
Arthur Rubinstein, \enquote{Rachmaninoff Piano Concerto No. 2, Op. 18, I Moderato, Allegro (Fritz Reiner)} [video], YouTube (recorded 9 Jan. 1956, uploaded 8 Nov. 2011), \url{http://www.youtube.com/watch?v=0Vv0Sy9FJrc&list=PLDB11C4F39E09047F}, accessed 9 Nov. 2013.
\tcblower
\begin{Verbatim}
@video{rubinstein1956rpc,
  entrysubtype = {clip},
  author = {Arthur Rubinstein},
  title = {Rachmaninoff Piano Concerto No. 2, Op. 18, I Moderato, Allegro (Fritz Reiner)},
  organization = {YouTube},
  origdate = {1956-01-09},
  date = {2011-11-08},
  datetype = {uploaded},
  url = {http://www.youtube.com/watch?v=0Vv0Sy9FJrc&list=PLDB11C4F39E09047F},
  urldate = {2013-11-09}}
\end{Verbatim}
\end{bibexbox}

\tip{Use \texttt{origdate} and, if necessary, the (non-standard) \texttt{origdatetype} field to specify the date of recording or original broadcast. Use the (non-standard) \texttt{datetype} field to clarify the event represented by the publication date.}

\section{Broadcasts}\label{sec:broadcasts}

The \emph{Oxford Guide to Style} prints all information about the transmission, including the date, bare. \emph{New Hart's Rules} tends to put the transmittion date in parentheses, perhaps because (\emph{a}) broadcast is a type of publication, (\emph{b}) episodes in some very long-running series are identified by date alone, and (\emph{c}) it may be important to know which broadcast is referenced since the edits might be different. \pkg[biblatex]{Oxref} sides with the latter.

\spec{\emph{Title}, Channel (Location, Date, Time).}

\tip{Put the channel in the \code{organization} field.}

\begin{bibexbox}
<OGS \S15.14.3*>
{canal2001ch}
\emph{Un Cœur in Hiver}, Canal+ (Paris, 15 May 2001, 11.40 p.m.).
\tcblower
\begin{Verbatim}
@video{canal2001ch,
  title = {Un Cœur in Hiver},
  organization = {Canal+},
  location = {Paris},
  date = {2001-05-15T23:40:00+02:00}}
\end{Verbatim}
\end{bibexbox}

\spec{\enquote{EpisodeTitle}, \emph{SeriesTitle}, \lit{Episode} EpisodeNumber, Channel, Date.}

\tip{If the episode has a true title, the episide descriptor (e.g. \enquote{Episode 1}) goes in the \code{note} field.
  If it doesn't, the episode descriptor goes in the \code{title} field; annotate the field with the keyword \code{descriptor} to remove the quote marks.
  Alternatively, use the (non-standard) \code{descriptor} field for the episode descriptor in all cases, and \pkg[biblatex]{oxref} will take care of this for you.}

\begin{bibexbox}
<NHR \S18.8.5>
{berger1972ws}
John Berger, Episode 1, \emph{Ways of Seeing}, BBC (1972), \url{https://www.youtube.com/watch?v=0pDE4VX_9Kk}, accessed 9 Nov. 2013.
\tcblower
\begin{Verbatim}
@video{berger1972ws,
  entrysubtype = {episode},
  author = {John Berger},
  title = {Episode 1},
  title+an = {=descriptor},
  maintitle = {Ways of Seeing},
  organization = {BBC},
  date = {1972},
  url = {https://www.youtube.com/watch?v=0pDE4VX_9Kk},
  urldate = {2013-11-09}}
\end{Verbatim}
\end{bibexbox}

\section{Podcasts, video podcasts and webcasts}

\tip{Set the \code{entrysubtype} to \code{podcast} for a podcast.}

\begin{bibexbox}
<NHR \S18.7.3>
{weldon2010did}
Fay Weldon, interview with Kirsty Young, \emph{Desert Island Discs Archive} [podcast], BBC Radio 4 (9 May 2010), \url{http://www.bbc.co.uk/podcasts/series/dida05/all}.
\tcblower
\begin{Verbatim}
@audio{weldon2010did,
  entrysubtype = {podcast},
  author = {Fay Weldon},
  descriptor = {interview with {Kirsty Young}},
  maintitle = {Desert Island Discs Archive},
  organization = {BBC Radio 4},
  date = {2010-05-09},
  url = {http://www.bbc.co.uk/podcasts/series/dida05/all}}
\end{Verbatim}
\end{bibexbox}

\begin{bibexbox}
<NHR \S18.8.5*>
{perry2013ifm}
Grayson Perry, \enquote{I Found Myself in the Art World} [podcast], Reith Lecture, BBC Radio 4  (5 Nov. 2013), \url{http://downloads.bbc.co.uk/podcasts/radio4/reith/reith_20131105-0940b.mp3}, accessed 5 Nov. 2013.
\tcblower
\begin{Verbatim}
@audio{perry2013ifm,
  entrysubtype = {podcast},
  author = {Grayson Perry},
  title = {I Found Myself in the Art World},
  note = {Reith Lecture},
  date = {2013-11-05},
  organization = {BBC Radio 4},
  url = {http://downloads.bbc.co.uk/podcasts/radio4/reith/reith_20131105-0940b.mp3},
  urldate = {2013-11-05}}
\end{Verbatim}
\end{bibexbox}

\info{I am not sure why \emph{New Hart's Rules} places the channel after the date in the above case; it is probably a mistake or an odd variation, so I have adjusted it for consistency.}

\tip{As for other online videos, set the \code{entrysubtype} to \code{clip} for a video podcast.}

\begin{bibexbox}
<NHR \S18.8.5*>
{nicholson2011qsm}
Christie Nicholson, \enquote{A Quirk of Speech May Become a New Vocal Style} [video], \emph{Scientific American} (17 Dec. 2011), \url{http://www.scientificamerican.com/podcast/episode.cfm?id=a-quirk-of-speech-may-become-a-new-11-12-17}, accessed 4 Nov. 2013.
\tcblower
\begin{Verbatim}
@video{nicholson2011qsm,
  entrysubtype = {clip},
  author = {Christie Nicholson},
  title = {A Quirk of Speech May Become a New Vocal Style},
  organization = {\emph{Scientific American}},
  date = {2011-12-17},
  url = {http://www.scientificamerican.com/podcast/episode.cfm?id=a-quirk-of-speech-may-become-a-new-11-12-17},
  urldate = {2013-11-04}}
\end{Verbatim}
\end{bibexbox}

\info{The above is actually an episode of the audio podcast \emph{60-Second Mind} (which would be the \code{maintitle}).
  \emph{Scientific American} is in this instance the \code{organization} that is hosting it, but because it is also the name of a journal, it has been manually formatted to match.}

\tip{Perhaps a better way of achieiving consistency in this case would be to treat it like an article, and clarify the medium (i.e.\@ video, erroneously) using the \texttt{titleaddon} field.}

\tip{Set the \code{entrysubtype} to \code{webcast} for a webcast.}

\begin{bibexbox}
<NHR \S18.8.5*>
{yousafzai2013mwa}
Malala Yousafzai, \enquote{Making a Wish for Action on Global Education: Malala Yousafzai Addresses Youth Assembly at UN on her 16th Birthday, 12 July 2013} [webcast], UN Web TV (12 July 2013), \url{http://webtv.un.org/search/malala-yousafzai-un-youth-assembly/2542094251001?term=malala}, accessed 15 Feb. 2015.
\tcblower
\begin{Verbatim}
@video{yousafzai2013mwa,
  entrysubtype = {webcast},
  author = {Malala Yousafzai},
  title = {Making a Wish for Action on Global Education},
  subtitle = {Malala Yousafzai Addresses Youth Assembly at UN on her 16th Birthday, 12 July 2013},
  organization = {UN Web TV},
  date = {2013-07-12},
  url = {http://webtv.un.org/search/malala-yousafzai-un-youth-assembly/2542094251001?term=malala},
  urldate = {2015-02-15}}
\end{Verbatim}
\end{bibexbox}

\section{Images and works of art}

\tip{If you need the type of image or arrtwork to appear in square brackets,
  put it in \code{titleaddon}. Otherwise, put it in \code{type}.}

\tip{Use \code{institution} (a list) or \code{venue} (a literal) for the
  organization, museum, gallery or building physically hosting the image, and
  \code{location} for where it is.}

\begin{bibexbox}
<NHR \S18.8.5>
{clarkeNDeci}
M. Clarke, \enquote{Exports of Coal to the IFS} [poster], Manchester Art Gallery, \url{http://www.machestergalleries.org/the-collections/search-the-collection/display.php?EMUSESSID=70bd7f1a388d79a82f52ea9aae713ef2&irn=4128}, accessed 5 Nov. 2013.
\tcblower
\begin{Verbatim}
@image{clarkeNDeci,
  author = {M. Clarke},
  title = {Exports of Coal to the IFS},
  titleaddon = {poster},
  institution = {Manchester Art Gallery},
  url = {http://www.machestergalleries.org/the-collections/search-the-collection/display.php?EMUSESSID=70bd7f1a388d79a82f52ea9aae713ef2&irn=4128},
  urldate = {2013-11-05}}
\end{Verbatim}
\end{bibexbox}

\tip{Use \code{organization} for the website or online location of the image.}

\begin{bibexbox}
<NHR \S18.8.5>
{ibwNDcgs}
\enquote{Christ the Good Shepherd}, stained glass window, Church of St Erfyl, Llanerfyl, Powys, Imaging the Bible in Wales Database, \url{http://imagingthebible.llgc.org.uk/object/1884}, accessed 10 Nov. 2013.
\tcblower
\begin{Verbatim}
@image{ibwNDcgs,
  title = {Christ the Good Shepherd},
  type = {stained glass window},
  venue = {Church of St Erfyl},
  location = {Llanerfyl, Powys},
  organization = {Imaging the Bible in Wales Database},
  url = {http://imagingthebible.llgc.org.uk/object/1884},
  urldate = {2013-11-10}}
\end{Verbatim}
\end{bibexbox}

\begin{bibexbox}
{davinci1480mr}
(not in book)
\tcblower
\begin{Verbatim}
@artwork{davinci1480mr,
  author = {given=Leonardo, family=da Vinci},
  title = {Madonna of the Rocks},
  type = {oil on canvas},
  note = {78 x 48.5 in\adddot},
  year = {1480s},
  date+an = {=inferred},
  institution = {Louvre},
  location = {Paris}}
\end{Verbatim}
\end{bibexbox}

\begin{bibexbox}
{gormley1998an}
(not in book)
\tcblower
\begin{Verbatim}
@artwork{gormley1998an,
  author = {Anthony Gormley},
  title = {Angel of the {North}},
  date = {1998},
  type = {sculpture},
  location = {Low Fell, Gateshead}}
\end{Verbatim}
\end{bibexbox}

\begin{bibexbox}
{rodin1882k}
(not in book)
\tcblower
\begin{Verbatim}
@artwork{rodin1882k,
  author = {Auguste Rodin},
  title = {The Kiss},
  date = {1882},
  type = {marble},
  institution = {Musée Rodin},
  location = {Paris}}
\end{Verbatim}
\end{bibexbox}

\section{Performances}

\tip{Use \code{origdate} for the date when the play, ballet, opera, or whatever
  was written or premièred, and \code{date} for the date of the performance you
  are referencing. You can specify what these dates signify with the
  (non-standard) \code{origdatetype} and \code{datetype} fields. You can also
  use \code{eventdate} in place of \code{date}, but then you won't be able to
  specify the type.}

\tip{As for recordings, you can use the \code{author} and \code{editor} fields
  to provide various credits. As for artworks, use \code{venue} and
  \code{location} for where the performance occurred, and \code{organization}
  for a website or organization hosting a recording of the performance online.}

\begin{bibexbox}
{ashton1937wb}
(not in book)
\tcblower
\begin{Verbatim}
@performance{ashton1937wb,
  author = {Frederick Ashton},
  origdate = {1937},
  title = {A Wedding Bouquet},
  venue = {Royal Opera House},
  location = {London},
  date = {2004-10-22}}
\end{Verbatim}
\end{bibexbox}

\begin{bibexbox}
{lord2007dc}
(not in book)
\tcblower
\begin{Verbatim}
@performance{lord2007dc,
  author = {Jon Lord},
  date = {2007-10-20},
  title = {Durham Concerto},
  editor = {{Liverpool Philharmonic Orchestra}},
  editortype = {performer},
  editora = {Mischa Damev},
  editoratype = {conductor},
  venue = {Durham Cathedral},
  location = {Durham}}
\end{Verbatim}
\end{bibexbox}

\begin{bibexbox}
{judge1995mnd}
(not in book)
\tcblower
\begin{Verbatim}
@performance{judge1995mnd,
  options = {useauthor=false,useeditor=false},
  author = {William Shakespeare},
  title = {A Midsummer Night's Dream},
  editor = {Ian Judge},
  editortype = {director},
  date = {1995-02-26},
  venue = {Theatre Royal},
  location = {Newcastle upon Tyne}}
\end{Verbatim}
\end{bibexbox}

\chapter{Digital media}\label{sec:electronic}
\chapterprecis{online, software, dataset}

In Section 18.8.5 of \emph{New Hart's Rules}, some publication dates are printed in parentheses and some are left bare.
In theory, the difference should be that parentheses indicate \enquote{proper} publication (bare dates are when the resource was created or issued). In practice, it seems to be random. I have therefore decided to ignore the variation and leave all such dates in parentheses.

\section{Website articles}

Standard \textsf{biblatex} only provides for one title for online resources, which is fine for whole websites but not if you want to cite one page or article within a website.

\tip{If the piece is unsigned, put the site name in the \code{author} field.}

\begin{bibexbox}
<NHR \S18.8.5>
{bbc2013cgh}
BBC News, \enquote{Colchester General Hospital: Police Probe Cancer Treatment} (5 Nov. 2013), \url{http://www.bbc.co.uk/news/uk-england-essex-24819973}, accessed 5 Nov. 2013.
\tcblower
\begin{Verbatim}
@online{bbc2013cgh,
  author = {{BBC News}},
  title = {Colchester General Hospital},
  subtitle = {Police Probe Cancer Treatment},
  date = {2013-11-05},
  url = {http://www.bbc.co.uk/news/uk-england-essex-24819973},
  urldate = {2013-11-05}}
\end{Verbatim}
\end{bibexbox}

\tip{If the piece is signed, put the site name in the \code{organization} field.
  The rationale is that \textsf{biblatex} uses this for the organization hosting the site,
  and it can get a bit blurry between that and the name of the site itself.}

\begin{bibexbox}
<NHR \S18.8.5>
{hooper2013lfs}
Richard Hooper, \enquote{Lebanon's Forgotten Space Programme}, BBC News Magazine (14 Nov. 2013), \url{http://www.bbc.co.uk/news/magazine-24735423}, accessed 14 Nov. 2013.
\tcblower
\begin{Verbatim}
@online{hooper2013lfs,
  author = {Richard Hooper},
  title = {Lebanon's Forgotten Space Programme},
  organization = {BBC News Magazine},
  date = {2013-11-14},
  url = {http://www.bbc.co.uk/news/magazine-24735423},
  urldate = {2013-11-14}}
\end{Verbatim}
\end{bibexbox}

\begin{bibexbox}
<NHR \S18.8.5>
{tan2013wdt}
Siu-Lan Tan, \enquote{Why does this Baby Cry when her Mother Sings?} [including video], OUPblog (5 Nov. 2013), \url{http://blog.oup.com/2013/11/why-does-this-baby-cry-when-her-mother-sings-viral-video/}, accessed 9 Nov. 2013.
\tcblower
\begin{Verbatim}
@online{tan2013wdt,
  author = {Siu-Lan Tan},
  title = {Why does this Baby Cry when her Mother Sings?},
  titleaddon = {including video},
  organization = {OUPblog},
  date = {2013-11-05},
  url = {http://blog.oup.com/2013/11/why-does-this-baby-cry-when-her-mother-sings-viral-video/},
  urldate = {2013-11-09}}
\end{Verbatim}
\end{bibexbox}

\begin{bibexbox}
<NHR \S18.8.5>
{allaby2013fll}
Michael Allaby, \enquote{Feathers and Lava Lamps}, Oxford Reference (2013), \url{http://www.oxfordreference.com/page/featherslavalamps}, accessed 9 Nov. 2013.
\tcblower
\begin{Verbatim}
@online{allaby2013fll,
  author = {Michael Allaby},
  title = {Feathers and Lava Lamps},
  organization = {Oxford Reference},
  date = {2013},
  url = {http://www.oxfordreference.com/page/featherslavalamps},
  urldate = {2013-11-09}}
\end{Verbatim}
\end{bibexbox}

\tip{If the site looks more like a traditional journal (e.g.\@ with an ISSN), use the \code{article} entry type instead of \code{online}.}

\begin{bibexbox}
<NHR \S18.8.5>
{mcewen2013tte}
Stephen McEwen, \enquote{Tan Twan Eng Interview: \enquote{I Have No Alternative but to Write in English}}, \emph{The Spectator} (20 May 2013), \url{http://blogs.spectator.co.uk/books/2013/05/tan-twang-eng-interview-i-have-no-alternative-but-to-write-in-english/}, accessed 9 Nov. 2013.
\tcblower
\begin{Verbatim}
@article{mcewen2013tte,
  author = {Stephen McEwen},
  title = {Tan Twan Eng Interview: \enquote{I Have No Alternative but to Write in English}},
  journaltitle = {The Spectator},
  date = {2013-05-20},
  options = {varissuedate=false},
  url = {http://blogs.spectator.co.uk/books/2013/05/tan-twang-eng-interview-i-have-no-alternative-but-to-write-in-english/},
  urldate = {2013-11-09}}
\end{Verbatim}
\end{bibexbox}

\section{Online reference article}

\begin{bibexbox}
<NHR \S18.8.5>
{eb2013gp}
\enquote{Gunpowder Plot}, \emph{Encyclopaedia Britannica}, \url{http://www.britannica.com/EBchecked/topic/249505/Gunpowder-Plot}, accessed 5 Nov. 2013.
\tcblower
\begin{Verbatim}
@online{eb2013gp,
  title = {Gunpowder Plot},
  maintitle = {Encyclopaedia Britannica},
  url = {http://www.britannica.com/EBchecked/topic/249505/Gunpowder-Plot},
  urldate = {2013-11-05}}
\end{Verbatim}
\end{bibexbox}

\tip{Unlike standard \textsf{biblatex}, \pkg[biblatex]{oxref} supports \code{maintitle} for online entries and prints the result in italics. For more portability, you could either hard-code the italics into the \code{organization} field or use the \code{article} entry type.}

\info{It is not made explicit why \emph{Encyclopaedia Britannica} is in italics and \emph{Wikipedia} is not.
  It might be historical or customary, because \emph{Encyclopaedia Britannica} was established as a print publication and \emph{Wikipedia} has only been online;
  it might be because \emph{Encyclopaedia Britannica} is a Latin title;
  it might reflect an opinion on the two resources;
  or it may just be illustrative of variant practice.}

\begin{bibexbox}
<NHR \S18.8.5>
{wp2013oup}
\enquote{Oxford University Press}, Wikipedia (last modified 5 Nov. 2013), \url{http://en.wikipedia.org/wiki/Oxford_University_Press}, accessed 5 Nov. 2013.
\tcblower
\begin{Verbatim}
@online{wp2013oup,
  title = {Oxford University Press},
  organization = {Wikipedia},
  date = {2013-11-05},
  datetype = {modified},
  url = {http://en.wikipedia.org/wiki/Oxford_University_Press},
  urldate = {2013-11-05}}
\end{Verbatim}
\end{bibexbox}

\tip{Use the (non-standard) \texttt{datetype} field to clarify the event represented by the main date.}

\section{Social media}

\begin{bibexbox}
<NHR \S18.8.5>
{obama2013tvd}
Barack Obama, \enquote{Tomorrow is Veterans Day} [Facebook post] (10 Nov. 2013), \url{https://www.facebook.com/barackobama/photos/a.53081056748.66806.6815841748/10151936988101749/}, accessed 13 Nov. 2013.
\tcblower
\begin{Verbatim}
@online{obama2013tvd,
  entrysubtype = {facebook},
  author = {Barack Obama},
  title = {Tomorrow is Veterans Day},
  date = {2013-11-10},
  url = {https://www.facebook.com/barackobama/photos/a.53081056748.66806.6815841748/10151936988101749/},
  urldate = {2013-11-13}}
\end{Verbatim}
\end{bibexbox}

\begin{bibexbox}
<NHR \S18.8.5>
{harvey2013tfm}
John Harvey, \enquote{\enquote{These are a Few of My Favourite Things}, No.~28} [Facebook post] (13 Nov. 2013), \url{https://www.facebook.com/photo.php?fbid=229786530530896&set=a.108896335953250.15125.100004986510149&type=1&theatre}, accessed 13 Nov. 2013.
\tcblower
\begin{Verbatim}
@online{harvey2013tfm,
  entrysubtype = {facebook},
  author = {John Harvey},
  title = {\enquote{These are a Few of My Favourite Things}, No.~28},
  date = {2013-11-13},
  url = {https://www.facebook.com/photo.php?fbid=229786530530896&set=a.108896335953250.15125.100004986510149&type=1&theatre},
  urldate = {2013-11-13}}
\end{Verbatim}
\end{bibexbox}

\begin{bibexbox}
<NHR \S18.8.5*>
{globe2013otd}
Shakespeare's Globe, \enquote{On this day in 1611 first production of The Tempest was performed by King's Men at Whitehall Palace before James I} [Twitter post] (5.48~a.m., 1~Nov. 2013), \url{https://twitter.com/The_Globe/status/396257422928400385}, accessed 5 Nov. 2013.
\tcblower
\begin{Verbatim}
@online{globe2013otd,
  entrysubtype = {tweet},
  author = {{Shakespeare's Globe}},
  title = {On this day in 1611 first production of The Tempest was performed by King's Men at Whitehall Palace before James I},
  date = {2013-11-01T05:48:00},
  options = {timefirst},
  url = {https://twitter.com/The_Globe/status/396257422928400385},
  urldate = {2013-11-05}}
\end{Verbatim}
\end{bibexbox}

\tip{If it feels odd putting the entire content of a tweet in your reference, \emph{New Hart's Rules} suggests using a descriptor like \enquote{Twitter post} instead of the title. As described in \cref{sec:broadcasts}, you can do this either by annotating the title field with the keyword \code{descriptor} or by using the (non-standard) \code{descriptor} field in place of \code{title}.}

\begin{bibexbox}
{oup2015tweet}
(not in book)
\tcblower
\begin{Verbatim}
@online{oup2015tweet,
  author = {{Oxford University Press}},
  shortauthor = {OUP},
  descriptor = {Twitter post},
  date = {2015-11-16T01:07:00},
  url = {https://twitter.com/OxUniPress/status/666180787251843072},
  urldate = {2015-12-25}}
\end{Verbatim}
\end{bibexbox}

\section{Software}

\begin{bibexbox}
<NHR \S18.8.5>
{simoga1.1d6}
Simoga, \emph{Device 6} (version 1.1) [mobile application for iPhone and iPad], downloaded 9 Nov. 2013.
\tcblower
\begin{Verbatim}
@software{simoga1.1d6,
  author = {Simoga},
  title = {Device 6},
  version = {1.1},
  titleaddon = {mobile application for iPhone and iPad},
  urldate = {2013-11-09}}
\end{Verbatim}
\end{bibexbox}

\begin{bibexbox}
<NHR \S18.8.5>
{eliot1.1.1twl}
T. S. Eliot, \emph{The Waste Land} (version 1.1.1) [mobile application for iPad] (London: Touch Press, 2013), downloaded 9 Nov. 2013.
\tcblower
\begin{Verbatim}
@software{eliot1.1.1twl,
  author = {T. S. Eliot},
  title = {The Waste Land},
  version = {1.1.1},
  titleaddon = {mobile application for iPad},
  location = {London},
  publisher = {Touch Press},
  date = {2013},
  urldate = {2013-11-09}}
\end{Verbatim}
\end{bibexbox}

\section{Database}

\begin{bibexbox}
<NHR \S18.8.5>
{un2011wpp}
United Nations, \emph{World Population Prospects: The 2010 Revision} [CD-ROM] (New York: United Nations Department of Economic and Social Affairs, Population Division, 2011).
\tcblower
\begin{Verbatim}
@dataset{un2011wpp,
  author = {{United Nations}},
  title = {World Population Prospects},
  subtitle = {The 2010 Revision},
  titleaddon = {CD-ROM},
  location = {New York},
  publisher = {{United Nations Department of Economic and Social Affairs, Population Division}},
  date = {2011}}
\end{Verbatim}
\end{bibexbox}

\chapter{Legal references}\label{sec:legal}
\chapterprecis{commentary, jurisdiction, legal, legislation}

\pkg[biblatex]{Oxref} provides only a basic level of support for legal references, in case you need to use some in a mainly non-legal text.
For a more thorough and robust treatment, intended for a specialist readership, I recommend you use the \pkg{oscola} style instead.

Since there is a family resemblence between the \emph{Oxford Guide to Style\slash New Hart's Rules} and the \emph{Oxford Standard for the Citation of Legal Authorities}, \pkg[biblatex]{oxref} largely mimics \pkg{oscola} and its data model, though there are some small formatting differences. What you should \emph{not} expect from \pkg[biblatex]{oxref} are facilities for compiling specialist indices of sources, or anything special regarding the handling of postnotes.


\section{Cases}

\subsection{Reported cases}

Use the \code{jurisdiction} entry type for citing cases.

\tip{Use the \code{keyword} field to specify the jurisdiction. The following are recognized:
  \begin{itemize}
  \item\val{gb}
    United Kingdom
  \item\val{en}:
    England (default)
  \item\val{cy}:
    Wales
  \item\val{sc}:
    Scotland
  \item\val{ni}:
    Northern Ireland
  \item\val{eu}:
    European Union (including the EEC, EC, ECSC and EURATOM)
  \item\val{echr}:
    Organs of the Council of Europe dealing with the European Convention on Human Rights
  \item\val{int}:
    (Public) international law cases and materials
  \item\val{us}:
    United States of America
  \item\val{ca}:
    Canada
  \item\val{aus}:
    Australia
  \item\val{nz}:
    New Zealand
  \end{itemize}
}

\tip{Scottish cases heard at the House of Lords or the Supreme Court should technically be given the keyword \val{gb}, therefore you can also identify them with the \key{scottish-style} option.}

\tip{Use either the (non-standard) \code{reporter} or \code{journaltitle} field for the (abbreviated) name of the report series. Use the \code{series} field for a numeric sub-series.}

\tip{Several additional pagination types are defined for use with legal citations:
  \val{article}, \val{clause}, \val{regulation}, and \val{rule} all work as normal.
  The default is a bare number (\val{none}), indicating a page reference, though EU and ECHR cases default to using the \enquote{para} (\val{paragraph}) prefix.
  There is also an alternative pagination type for paragraphs, \val{[]}, which prints the number in square brackets instead of giving it a prefix.
}

\tip{If you want to include a parallel citation, you can use the non-standard \code{pardate}, \code{parreporter}, \code{parseries}, \code{parvolume}, and \code{parpages} fields.}

\tip{If you need to, use either the (non-standard) \code{court} or \code{institution} field for the court that decided the case.}

\tip{Use the \code{location} field for the location of the court in American, Australian or Canadian cases.}

\begin{egcite}{\emph{Ridge v Baldwin} [1964] AC 40 at 78–9}
\cite[78-79]{ridge1964}
\end{egcite}

\begin{bibexbox}
<NHR \S13.4.1>
{ridge1964}
\emph{Ridge v Baldwin} [1964] AC 40
\tcblower
\begin{Verbatim}
@jurisdiction{ridge1964,
  title = {Ridge v. Baldwin},
  keywords = {gb},
  date = {1964},
  journaltitle = {A.C.},
  pages = {40}}
\end{Verbatim}
\end{bibexbox}

\tip{Use the \key{year-essential} Boolean entry option to specify whether the year is essential for locating the report, in cases where automatic detection (based on the presence or otherwise of a volume number) fails.
  There is a similar \key{paryear-essential} option for the parallel year.}

\begin{bibexbox}
<NHR \S13.4.1>
{lambert2001}
\emph{R v Lambert} [2001] 2 WLR 211 (QBD)
\tcblower
\begin{Verbatim}
@jurisdiction{lambert2001,
  title = {R v Lambert},
  keywords = {gb},
  date = {2001},
  volume = {2},
  journaltitle = {WLR},
  pages = {211},
  institution = {QBD},
  options = {year-essential=true}}
\end{Verbatim}
\end{bibexbox}

\begin{bibexbox}
<NHR \S13.4.1>
{badische1897}
\emph{Badische v Soda-Fabrics} (1897) 14 RPC 919 (HL)
\tcblower
\begin{Verbatim}
@jurisdiction{badische1897,
  title = {Badische v. Soda-Fabrics},
  date = {1897},
  journaltitle = {RPC},
  volume = {14},
  pages = {919},
  institution = {HL}}
\end{Verbatim}
\end{bibexbox}

\tip{Use the (non-standard) \code{neutralcite} or \code{number} field for the neutral citation or case number.}

\begin{bibexbox}
<NHR \S13.4.1>
{rvg2004}
\emph{R v G} [2003] UKHL 50, [2004] 1 AC 1034
\tcblower
\begin{Verbatim}
@jurisdiction{rvg2004,
  title = {R. v. G.},
  number = {[2003] UKHL 50},
  date = {2004},
  journaltitle = {A.C.},
  volume = {1},
  pages = {1034},
  options = {year-essential},
  pagination = {[]}}
\end{Verbatim}
\end{bibexbox}

\begin{egcite}{}
Test\footcite[13]{rvg2004}
\end{egcite}


\tip{If a report is published a long time after the decision, you can put the decision date in \code{origdate} to clarify the situation.}

\begin{bibexbox}
<NHR \S13.4.3>
{smith2001}
\emph{Smith v Jones} [2001] (1948) 2 All ER 431
\tcblower
\begin{Verbatim}
@jurisdiction{smith2001,
  title = {Smith v Jones},
  date = {2001},
  origdate = {1948},
  journaltitle = {All ER},
  volume = {2},
  pages = {431},
  options = {year-essential}}
\end{Verbatim}
\end{bibexbox}

Both the \emph{Oxford Guide to Style} and \emph{New Hart's Rules} are inconsistent on whether the court of decision should be printed plain or in parentheses at the end of a reference to a reported case. The default chosen by \pkg[biblatex]{oxref} is to use parentheses as per the \emph{Oxford Standard for the Citation of Legal Authorities}.

\tip{To print the court of decision plain, you can use the \key{court-plain} option. You can set this globally at the style level or on a per-entry basis.}

\begin{bibexbox}
<NHR \S13.4.3>
{bowman1978}
\emph{Bowman v Fussy} [1978] RPC 545, HL
\tcblower
\begin{Verbatim}
@jurisdiction{bowman1978,
  title = {Bowman v Fussy},
  keywords = {gb},
  date = {1978},
  journaltitle = {RPC},
  pages = {545},
  institution = {HL},
  options = {court-plain}}
\end{Verbatim}
\end{bibexbox}

\begin{bibexbox}
<NHR \S13.4.1>
{hughes1907}
\emph{Hughes v Stewart}, 1907 SC 791
\tcblower
\begin{Verbatim}
@jurisdiction{hughes1907,
  title = {Hughes v. Stewart},
  date = {1907},
  reporter = {SC},
  pages = {791},
  keywords = {sc}}
\end{Verbatim}
\end{bibexbox}

\begin{bibexbox}
<NHR \S13.4.1>
{corcoran1932}
\emph{Corcoran v HM Advocate}, 1932 JC 42
\tcblower
\begin{Verbatim}
@jurisdiction{corcoran1932,
  title = {Corcoran v. H.M. Advocate},
  date = {1932},
  reporter = {JC},
  pages = {42},
  keywords = {sc}}
\end{Verbatim}
\end{bibexbox}

\begin{bibexbox}
<NHR \S13.4.1>
{michael1976}
\emph{Michael v Johnson}, 426 US 346 (1976)
\tcblower
\begin{Verbatim}
@jurisdiction{michael1976,
  title = {Michael v. Johnson},
  volume = {426},
  reporter = {U.S.},
  pages = {346},
  date = {1976},
  keywords = {us}}
\end{Verbatim}
\end{bibexbox}

\subsection{Unreported cases}

\tip{For cases reported in newspapers, set the \code{entrysubtype} to \code{newspaper}.}

\begin{bibexbox}
<NHR \S13.4.2*>
{powick1993}
\emph{Powick v Malvern Wells Water Co}, The Times, 28 Sept.\@ 1993
\tcblower
\begin{Verbatim}
@jurisdiction{powick1993,
  entrysubtype = {newspaper},
  title = {Powick v. Malvern Wells Water Co},
  date = {1993-09-28},
  journaltitle = {The Times}}
\end{Verbatim}
\end{bibexbox}

\begin{bibexbox}
<NHR \S13.4.2*>
{marianishi1965}
\emph{R v Marianishi, ex p London Borough of Camden} (CA, 13 Apr.\@ 1965)
\tcblower
\begin{Verbatim}
@jurisdiction{marianishi1965,
  title = {R v Marianishi, ex p London Borough of Camden},
  date = {1965-04-13},
  institution = {C.A.}}
\end{Verbatim}
\end{bibexbox}

\subsection{European cases}

\tip{You can record the European Case Law Identifier for a case using the \code{ecli} field.
  The \key{ecli} option determines under what circumstances the identifier is printed.}

\tip{European cases still get sorted by title, even if the entry begins with the case number.
  Use \code{sorttitle} if you would rather they were sorted under \enquote{Case} or \enquote{Joined Cases}.}

\tip{The number will be classed as \enquote{Joined Cases} if it contains a comma, a double hyphen (\code{--}),
  or the string \cs{oxrefand}, which defaults to \enquote{\code{\textvisiblespace and\textvisiblespace}}.}

\tip{For ECR cases, you can use either \code{number} or the non-standard \code{casenumber} field
  for the case number.}

\begin{bibexbox}
<OGS \S13.4.4>
{C118/07}
Case C–118/07 \emph{Commission of the European Communities v Finland} [2002] ECR I–10889
\tcblower
\begin{Verbatim}
@jurisdiction{C118/07,
  keywords = {eu},
  title = {Commission of the European Communities v. Finland},
  sorttitle = {Case C0118/0007},
  number = {C\textendash118/07},
  journaltitle = {ECR},
  volume = {I},
  pages = {10889},
  date = {2002}}
\end{Verbatim}
\end{bibexbox}

\tip{Indicate Commission Decisions by putting \code{Commission} in the \code{institution} field. If a decision has a formal decision number, put this in the \code{number} field, and put the case number in the (non-standard) \code{casenumber} field or (\pkg{oscola}-supported) \code{userb} field. If wider portability is an issue, you can put the case number, including the word \enquote{Case} if needed, in the \code{titleaddon} field instead.}

\begin{bibexbox}
<OGS \S13.2.6>
{alcatel}
\emph{Alcatel/Telettra} (Case IV/M042) [1991] OJ L122/48, [1991] 4 CLMR 391
\tcblower
\begin{Verbatim}
@jurisdiction{alcatel,
  keywords = {eu},
  title = {Alcatel/Telettra},
  casenumber = {IV/M042},
  institution = {Commission},
  date = {1991},
  journaltitle = {OJ},
  series = {L},
  volume = {122},
  pages = {48},
  pardate = {1991},
  parreporter = {CLMR},
  parvolume = {4},
  parpages = {391},
  options = {paryear-essential}}
\end{Verbatim}
\end{bibexbox}

\tip{For ECHR cases, put the application number in the \code{number} field.}

\begin{bibexbox}
<NHR \S13.4.5>
{young1982}
\emph{Young, James and Webster v UK} (App no 7601/76) (1982) 4 EHRR 38
\tcblower
\begin{Verbatim}
@jurisdiction{young1982,
  title = {Young, James and Webster v. UK},
  number = {7601/76},
  reporter = {EHRR},
  volume = {4},
  date = {1982},
  pages = {38},
  keywords = {echr}}
\end{Verbatim}
\end{bibexbox}

\tip{Special formatting is triggered if you specify \code{Series A} or \code{ECHR} as the \code{journaltitle} or \code{reporter}.}

\begin{bibexbox}
<NHR \S13.4.5>
{plattform1988}
\emph{Plattform \enquote{Artze für das Leben} v Austria} (App no 10126/82) (1988) Series A no 139
\tcblower
\begin{Verbatim}
@jurisdiction{plattform1988,
  title = {Plattform \enquote{Artze für das Leben} v. Austria},
  number = {10126/82},
  reporter = {Series A},
  date = {1988},
  pages = {139},
  keywords = {echr}}
\end{Verbatim}
\end{bibexbox}

\begin{bibexbox}
<NHR \S13.4.5>
{osman1998}
\emph{Osman v UK} (App no 23452/94) ECHR 1998–VIII 3124
\tcblower
\begin{Verbatim}
@jurisdiction{osman1998,
  title = {Osman v UK},
  number = {23452/94},
  reporter = {ECHR},
  date = {1998},
  volume = {8},
  pages = {3124},
  institution = {ECtHR},
  keywords = {echr}}
\end{Verbatim}
\end{bibexbox}

\section{Legislation}

Use the \code{legislation} entry type for citing Acts, Bills, Orders, Measures, and the like.

\subsection{UK legislation}

\tip{Specify an \code{entrysubtype} of \code{primary} for primary legislation.
  This ensures only the title and year are printed.}

\begin{bibexbox}
<OGS \S13.2.3>
{fwa1891}
Factory and Workshop Act 1891
\tcblower
\begin{Verbatim}
@legislation{fwa1891,
  entrysubtype = {primary},
  title = {Factory and Workshop Act},
  shorttitle = {FWA 91},
  date = {1891},
  pagination = {section},
  keywords = {en},
  number = {54 \& 55 Vict, c. 75}}
\end{Verbatim}
\end{bibexbox}

\tip{Specify an \code{entrysubtype} of \code{secondary} for secondary legislation.
  The number is then printed as well.}

\begin{bibexbox}
<NHR \S13.5.1>
{lap1897}
Local Authority Precepts Order 1897, SR~\&~O 1897/201
\tcblower
\begin{Verbatim}
@legislation{lap1897,
  entrysubtype = {secondary},
  title = {Local Authority Precepts Order},
  date = {1897},
  number = {SR~\&~O 1897\slash 201},
  pagination = {rule}}
\end{Verbatim}
\end{bibexbox}

\begin{bibexbox}
<NHR \S13.5.1>
{ccsga1987}
Community Charge Support Grant (Abolition) Order 1987, SI 1987/466
\tcblower
\begin{Verbatim}
@legislation{ccsga1987,
  entrysubtype = {secondary},
  title = {Community Charge Support Grant (Abolition) Order},
  date = {1987},
  number = {SI 1987\slash 466},
  pagination = {regulation}}
\end{Verbatim}
\end{bibexbox}

\subsection{European legislation}

As with European cases, European legislation should be given the \val{eu} keyword.
This is currently the only keyword that makes a difference to the formatting.

According to both the \emph{Oxford Guide to Style} and \emph{New Hart's Rules},
you can provide an even sparser reference for primary European legislation.

\begin{egcite}{EC Treaty (Treaty of Rome, as amended), art.\@ 3b}
\cite[\pno~3b]{ECT:Rome}
\end{egcite}

\begin{bibexbox}
<OGS \S13.2.7*>
{ECT:Rome}%
EC Treaty
\tcblower
\begin{Verbatim}
@legislation{ECT:Rome,
  entrysubtype = {eu-treaty},
  title = {EC Treaty},
  shorthandintro = {Treaty of Rome, as amended},
  shorthand = {EC Treaty},
  pagination = {article},
  keywords = {eu}}
\end{Verbatim}
\end{bibexbox}

If this seems scandalously short, you may prefer to provide a full reference (including the relevant entry in the OJ or OJ Spec Ed) for the bibliography and use, say, \cs{citetitle} for citations.

Secondary legislation should always have a full reference.

\begin{bibexbox}
<NHR \S13.5.2>
{EC:97/1}
Council Directive (EC) 97/1 on banking practice [1997] OJ L234/3
\tcblower
\begin{Verbatim}
@legislation{EC:97/1,
  title = {Council Directive (EC) 97/1 on banking practice},
  date = {1997},
  journaltitle = {OJ},
  series = {L},
  issue = {234},
  pages = {3},
  pagination = {article},
  keywords = {eu}}
\end{Verbatim}
\end{bibexbox}

\begin{bibexbox}
<NHR \S13.5.2>
{EEC:1017/68}
Council Regulation (EEC) 1017/68 applying rules of competition to transport [1968] OJ Spec Ed 302
\tcblower
\begin{Verbatim}
@legislation{EEC:1017/68,
  title = {Council Regulation (EEC) 1017/68 applying rules of competition to transport},
  date = {1968},
  journaltitle = {OJ Spec Ed},
  pages = {302},
  pagination = {article},
  keywords = {eu}}
\end{Verbatim}
\end{bibexbox}

\tip{Add amendment notices to the \code{note} field; \pkg[biblatex]{oxref} does not, sadly, provide a clever way of constructing these from their component parts so you have to write them verbatim.}

\begin{bibexbox}
<NHR \S13.5.2>
{ECN:1986}
Commission Notice on agreements of minor importance [1986] OJ C231/2, as amended [1994] OJ C368/20
\tcblower
\begin{Verbatim}
@legislation{ECN:1986,
  title = {Commission Notice on agreements of minor importance},
  date = {1986},
  journaltitle = {OJ},
  series = {C},
  issue = {231},
  pages = {2},
  note = {as amended [1994] OJ C368/20},
  keywords = {eu}}
\end{Verbatim}
\end{bibexbox}

\section{Treaties}

Treaties use the \code{legal} entry type, with \code{piltreaty} as the \code{entrysubtype}.

\tip{Don't use the \code{date} field for treaties; instead, use the \code{execution} field to give a list of dates. Each date should be given in the form \meta{type}=\meta{year-month-day}, where \meta{type} is one of \key{opened}, \key{signed}, \key{adopted}, or \key{inforce}. The dates won't be sorted; they will be printed in the order given.}

\begin{bibexbox}
<NHR \S13.6*>
{refugees}
Convention Relating to the Status of Refugees (adopted 28 July 1951, entered into force 22 Apr.\@ 1954) 189 UNTS 137
\tcblower
\begin{Verbatim}
@legal{refugees,
  entrysubtype = {piltreaty},
  title = {Convention Relating to the Status of Refugees},
  shorthand = {Refugee Convention},
  execution = {adopted=1951-07-28 and inforce=1954-04-22},
  pagination = {article},
  volume = {189},
  journaltitle = {UNTS},
  pages = {137}}
\end{Verbatim}
\end{bibexbox}


\section{Hansard}

References to Hansard use the \code{legal} entry type, with \code{parliamentary} as the \code{entrysubtype}.

\tip{For debates since 1909, give \code{Hansard} as the \code{title} and put either \code{HC} or \code{HL} in the \code{type} field.}

\tip{Put the column numbers in the \code{pages} field, and put \code{column} in the \code{bookpagination} field.}

\begin{bibexbox}
<NHR \S13.7.3*>
{hc357}
Hansard, HC vol.\@ 357, cols.\@ 234–45 (13 Apr.\@ 1965)
\tcblower
\begin{Verbatim}
@legal{hc357,
  entrysubtype = {parliamentary},
  title = {Hansard},
  type = {HC},
  volume = {357},
  pages = {234-245},
  date = {1965-04-13},
  bookpagination = {column}}
\end{Verbatim}
\end{bibexbox}

\tip{For debates before 1909, give \code{Parl. Deb.} as the \code{title}.}

\begin{bibexbox}
<NHR \S13.7.3*>
{pd4/24}
Parl.\@ Deb.\@ (series 4) vol.\@ 24, col.\@ 234 (24 Mar.\@ 1895)
\tcblower
\begin{Verbatim}
@legal{pd4/24,
  entrysubtype = {parliamentary},
  title = {Parl. Deb.},
  series = {4},
  volume = {24},
  pages = {234},
  date = {1895-03-24},
  bookpagination = {column}}
\end{Verbatim}
\end{bibexbox}

\section{Legal reports}

Reports of Parliamentary select committees and the Law Commission should be entered using the \code{report} entry type, with the \code{entrysubtype} set to \code{legal}.

\tip{To print the identifying codes within the publication block, put them all in \code{number}; do not specify a \code{type} or \code{series}.}

\begin{bibexbox}
<NHR \S13.7.2>
{lc2009icl}
Law Commission, \emph{Intoxication and Criminal Liability} (Law Comm No 314, Cm 7526, 2009) para 1.15
\tcblower
\begin{Verbatim}
@report{lc2009icl,
  entrysubtype = {legal},
  author = {{Law Commission}},
  title = {Intoxication and Criminal Liability},
  number = {Law Comm No 314, Cm 7526},
  date = {2009},
  pages = {1.15},
  bookpagination = {paragraph}}
\end{Verbatim}
\end{bibexbox}

\section{Commentaries}

\pkg[biblatex]{Oxref} does not provide any special formatting for legal commentaries;
it provides the \code{commentary} entry type simply as an alias for \code{book}.

\chapter{Specialist materials}\label{sec:special}
\chapterprecis{misc, unpublished, letter, manuscript}

\section{Poems}\label{sec:poem}

\subsection{Short poems}

\tip{Use the \code{incollection} entry type for a poem in a collection.
  If it appears in an article or a work that is itself in a collection,
  use the \code{misc} entry type for the poem, with a relation of type \code{in} pointing to the containing work.}

\begin{bibexbox}
<OGS \S15.8>
{auden1990era}
W. H. Auden, \enquote{Es regnet auf mir in den Schottische Lände} [\emph{sic}], in \enquote{The German Auden: Six Early Poems}, trans. David Constantine, in Katherine Bucknell and Nicholas Jenkins (eds.), \emph{W. H. Auden, \enquote{The Map of All my Youth}: Early Works, Friends, and Influences} (Auden Studies, 1; Oxford, 1990), 1--15 at 6.
\tcblower
\begin{Verbatim}
@misc{auden1990era,
  author = {W. H. Auden},
  title = {Es regnet auf mir in den Schottische Lände},
  titleaddon = {\emph{sic}},
  related = {constantine1990gas},
  relatedtype = {in},
  pages = {6}}
@incollection{constantine1990gas,
  title = {The {German} {Auden}},
  subtitle = {Six Early Poems},
  author = {David Constantine},
  authortype = {translator},
  options = {useauthor=false},
  editor = {Katherine Bucknell and Nicholas Jenkins},
  booktitle = {W. H. Auden, \enquote{The Map of All my Youth}},
  booksubtitle = {Early Works, Friends, and Influences},
  series = {Auden Studies},
  number = {1},
  location = {Oxford},
  date = {1990},
  pages = {1-15}}
\end{Verbatim}
\end{bibexbox}

\begin{bibexbox}
<OGS \S15.8>
{blois1993qff}
William of Blois, \enquote{The Quarrel of the Flea and the Fly} (\emph{Pulicis et musce iurgia}), trans. in Jan M. Ziolkowski, \emph{Talking Animals: Medieval Latin Beast Poetry, 750-1150} (Middle Ages Series, ed. Edward Peters; Philadelphia: University of Pennsylvania Press, 1993), 274--8.
\tcblower
\begin{Verbatim}
@misc{blois1993qff,
  author = {{William of Blois}},
  title = {The Quarrel of the Flea and the Fly},
  origtitle = {Pulicis et musce iurgia},
  related = {ziolkowski1993tam},
  relatedtype = {in},
  relatedstring = {trans.\@ in},
  pages = {274-278}}
@book{ziolkowski1993tam,
  author = {Jan M. Ziolkowski},
  title = {Talking Animals},
  subtitle = {Medieval Latin Beast Poetry, 750-1150},
  series = {Middle Ages Series},
  serieseditor = {Edward Peters},
  location = {Philadelphia},
  publisher = {University of Pennsylvania Press},
  date = {1993}}
\end{Verbatim}
\end{bibexbox}

\subsection{Long poems}

If a poem is divided into cantos, the title is set in italics rather than quoted.

\tip{To achieve this, use the entry type \code{book} if the poem is published on its own,
  and \code{bookinbook} with entry subtype \code{poem} if the poem is published in an anthology.}

\tip{To assist with quoting passages, \pkg[biblatex]{oxref} provides the additional pagination types
  \code{book}, \code{canto}, and \code{stanza} (\code{line} is already defined),
  though in reality you are probably better off doing it by hand.}

\begin{egcite}{\dots 1590–6), bk. ii, canto vi, stanza iii}
\cite[\pno~ii, canto vi, stanza iii]{spenser1965fq}
\end{egcite}

\begin{bibexbox}
<OGS \S15.8*>
{spenser1965fq}%
Edmund Spenser, \emph{The Faerie Queene} (Everyman's Library, 443--4; London: Dent, 1965–6) (originally pub. 1590–6).
\tcblower
\begin{Verbatim}
@book{spenser1965fq,
  author = {Edmund Spenser},
  title = {The Faerie Queene},
  shorttitle = {Faerie Queene},
  series = {Everyman's Library},
  number = {443--4},
  location = {London},
  publisher = {Dent},
  date = {1965/1966},
  origdate = {1590/1596},
  pagination = {book}}
\end{Verbatim}
\end{bibexbox}

\section{Plays}\label{sec:play}

Play titles, like those of epic poems, are set in italics rather than quoted.

\tip{To achieve this, use the entry type \code{book} if the play is published on its own,
  and \code{bookinbook} with entry subtype \code{play} if the play is published in an anthology.}

\tip{To assist with quoting passages, \pkg[biblatex]{oxref} provides the additional pagination types
  \code{act} and \code{scene} (\code{line} is already defined),
  though in reality you are probably better off doing it by hand.}

\section{Manuscripts}\label{sec:ms}

\spec{Author, \enquote{Title}/Descriptor, Date, Archive, Location, Collection Series, Shelfmark, Folios.}

With \pkg[biblatex]{oxref}, you can enter manuscripts in either of two ways. The first uses the \code{unpublished} entry type.

\tip{If the manuscript has a descriptor but no title, put the descriptor in the \code{title} field and annotate the field with the term \enquote{\code{descriptor}}. If you want to supply both a title and a descriptor, put the descriptor in the \code{note} field. Alternatively, you can use the (\pkg[biblatex]{oxref}-specific) \code{descriptor} field, and \pkg[biblatex]{oxref} will handle all this for you.}

\tip{Use \code{author} and \code{date} as normal. Note that the \code{date} will not be printed if both the \code{author} and \code{title}\slash \code{descriptor} fields are left blank.}

\tip{Put the library or archive where the manuscript is kept in the \code{library} field, and the city or place name in the \code{location} field. Note that the \code{library} field is required to switch on the special support for manuscripts.}

\tip{Put the collection name in the \code{series} field and the shelfmark in the \code{number} field.}

\tip{Put the page range studied (or some other subdivision) in the \code{pages} field. You can use \cs{recto} and \cs{verso} for the respective sides of a folio, and you can also specify a \code{pagetotal}.}

\tip{You can use the \code{folio} key in the \code{pagination} and \code{bookpagination} fields.}

\begin{bibexbox}
<NHR 18.6.3*>
{smithMS23116}
Francis Smith, travel diaries, 1912–7, British Library, Add. MS 23116.
\tcblower
\begin{Verbatim}
@unpublished{smithMS23116,
  author = {Francis Smith},
  descriptor = {travel diaries},
  date = {1912/1917},
  library = {British Library},
  series = {Add. MS},
  number = {23116}}
\end{Verbatim}
\end{bibexbox}

The second uses the dedicated \code{manuscript} entry type from \textsf{biblatex-manuscripts-philology}.
\pkg[biblatex]{Oxref} does not have all the bells and whistles of that style, but it understands a subset of its data model.

\tip{You can use \code{author}, \code{title}\slash \code{note}\slash \code{descriptor} and \code{date} as described above. (This is a departure from \textsf{biblatex-manuscripts-philology}.)}

\tip{Put the library or archive where the manuscript is kept in the \code{library} field, and the city or place name in the \code{location} field}

\tip{Put the collection name in the \code{collection} field and the shelfmark in the \code{shelfmark} field.}

\tip{If the date is vague (e.g. a century), you can put this in the \code{dating} field. Note that it will only be printed if the \code{date} is missing\slash not printed.}

\tip{Put the page range studied (or some other subdivision) in the \code{pages} field. You can use \cs{recto} and \cs{verso} for the respective sides of a folio, and you can also specify a \code{pagetotal}.}

\tip{You can specify the number of columns (either \code{1} or \code{2}) in the \code{columns} field.}

\tip{You can specify the layer of a palimpsest in the \code{layer} field. Use \code{inf} for the inferior layer and \code{sup} for the superior layer.}

\tip{You can specify the writing support material in the \code{support} field. Three special keys – \code{papyrus}, \code{paper}, and \code{pergament} – are recognized and (potentially) translated, but other values will be printed as-is.}

\begin{bibexbox}
<NHR 18.6.2>
{chaundlerMS288}
Thomas Chaundler, \enquote{Collocutiones}, Balliol College, Oxford, MS288.
\tcblower
\begin{Verbatim}
@manuscript{chaundlerMS288,
  author = {Thomas Chaundler},
  title = {Collocutiones},
  library = {Balliol College},
  location = {Oxford},
  shelfmark = {MS288}}
\end{Verbatim}
\end{bibexbox}

\begin{bibexbox}
<NHR 18.6.2>
{exchequerE311}
exchequer accounts, Dec. 1798, Cheshire Record Office, E311.
\tcblower
\begin{Verbatim}
@manuscript{exchequerE311,
  descriptor = {exchequer accounts},
  date = {1798-12},
  library = {Cheshire Record Office},
  shelfmark = {E311}}
\end{Verbatim}
\end{bibexbox}

\begin{bibexbox}
<NHR 18.6.3>
{blcBOX19d}
Bearsden Ladies' Club minutes, 12 June 1949, Bearsden and Milngavie District Libraries, box 19/d.
\tcblower
\begin{Verbatim}
@manuscript{blcBOX19d,
  title = {Bearsden Ladies' Club minutes},
  title+an = {=descriptor},
  date = {1949-06-12},
  library = {Bearsden and Milngavie District Libraries},
  shelfmark = {box 19/d}}
\end{Verbatim}
\end{bibexbox}

\begin{bibexbox}
<NHR 18.6.5>
{bodMSrawl-d520}
Bodleian Library, Oxford, MS Rawlinson D. 520, fo. 7.
\tcblower
\begin{Verbatim}
@manuscript{bodMSrawl-d520,
  library = {Bodleian Library},
  location = {Oxford},
  collection = {MS Rawlinson D.},
  shelfmark = {520},
  pages = {7},
  bookpagination = {folio}}
\end{Verbatim}
\end{bibexbox}

\begin{bibexbox}
{cantabAGM4429}
(not in book)
\tcblower
\begin{Verbatim}
@manuscript{cantabAGM4429,
  library = {University Library},
  location = {Cambridge},
  collection = {Add. Greek MS},
  shelfmark = {4489},
  support = {pergament},
  dating = {8th--9th c.},
  pagetotal = {16},
  columns = {1},
  pages = {11\recto-11\verso},
  layer = {inf},
  bookpagination = {folio}}
\end{Verbatim}
\end{bibexbox}


\section{Letters}

These examples relate to letters held personally or online. For letters that
form part of archival collections, see the section on manuscripts above.

Note that it is usually acceptable to exclude personal communications from
the reference section, and simply cite them in the text (in which case you may
not need the \pkg{biblatex} machinery).

\spec{Author, \enquote{Title}/Descriptor, Date.}

\begin{bibexbox}
<OGS \S15.11.1*>
{anon2001pl}
Personal letter to the author, 2 May 2001.
\tcblower
\begin{Verbatim}
@letter{anon2001pl,
  title = {Personal letter to the author},
  title+an = {=descriptor},
  date = {2001-05-02}}
\end{Verbatim}
\end{bibexbox}

In the absence of any explicit rationale governing whether the date is
parenthetical, in \pkg[biblatex]{oxref} the presence of a URL
will trigger the use of parentheses.

\spec{Author, \enquote{Title}/Descriptor [HowPublished], (Date), OnlineAccess.}

\begin{bibexbox}
<OGS \S15.15.3*>
{ritter2001rp}%
R. M. Ritter, \enquote{Revised proofs} [email to H. E. Cox], (1 Aug.\@ 2001), \url{ogs@ritter.org.uk}, accessed 3 Aug.\@ 2001.
\tcblower
\begin{Verbatim}
@letter{ritter2001rp,
  author = {R. M. Ritter},
  title = {Revised proofs},
  howpublished = {email to H. E. Cox},
  date = {2001-08-01},
  url = {ogs@ritter.org.uk},
  urldate = {2001-08-03}}
\end{Verbatim}
\end{bibexbox}

\printbibliography[notcategory=hidden]
\end{document}
%% 
%% Copyright (C) 2016–2020 Alex Ball
%%
%% End of file `oxnotes-doc.tex'.

%%
%% This is file `testidx-glossaries-samples.tex',
%% generated with the docstrip utility.
%%
%% The original source files were:
%%
%% testidx.dtx  (with options: `testidx-glossaries-samples.tex,package')
%% 
%%  testidx.dtx
%%  Copyright 2019 Nicola Talbot
%% 
%%  This work may be distributed and/or modified under the
%%  conditions of the LaTeX Project Public License, either version 1.3
%%  of this license or (at your option) any later version.
%%  The latest version of this license is in
%%    http://www.latex-project.org/lppl.txt
%%  and version 1.3 or later is part of all distributions of LaTeX
%%  version 2005/12/01 or later.
%% 
%%  This work has the LPPL maintenance status `maintained'.
%% 
%%  The Current Maintainer of this work is Nicola Talbot.
%% 
%%  This work consists of the files testidx.dtx and testidx.ins and the derived files testidx.sty, testidx-glossaries.sty, testidx-glossaries-diglyphs.tex, testidx-glossaries-nodiglyphs.tex, testidx-glossaries-samples.tex, testidx-glossaries-diglyphs-utf8.bib, testidx-glossaries-markers.bib, testidx-glossaries-mathsym.bib, testidx-glossaries-nodiglyphs-utf8.bib, testidx-glossaries-nodiglyphs.bib, testidx-glossaries-numbers.bib, testidx-glossaries-samples-ascii.bib, testidx-glossaries-samples-utf8.bib, testidx-glossaries-samples.bib.
%% 
%% \CharacterTable
%%  {Upper-case    \A\B\C\D\E\F\G\H\I\J\K\L\M\N\O\P\Q\R\S\T\U\V\W\X\Y\Z
%%   Lower-case    \a\b\c\d\e\f\g\h\i\j\k\l\m\n\o\p\q\r\s\t\u\v\w\x\y\z
%%   Digits        \0\1\2\3\4\5\6\7\8\9
%%   Exclamation   \!     Double quote  \"     Hash (number) \#
%%   Dollar        \$     Percent       \%     Ampersand     \&
%%   Acute accent  \'     Left paren    \(     Right paren   \)
%%   Asterisk      \*     Plus          \+     Comma         \,
%%   Minus         \-     Point         \.     Solidus       \/
%%   Colon         \:     Semicolon     \;     Less than     \<
%%   Equals        \=     Greater than  \>     Question mark \?
%%   Commercial at \@     Left bracket  \[     Backslash     \\
%%   Right bracket \]     Circumflex    \^     Underscore    \_
%%   Grave accent  \`     Left brace    \{     Vertical bar  \|
%%   Right brace   \}     Tilde         \~}
\tstidxnewword{packages}{files provided to adjust the document
design or provide new commands}
\tstidxnewword{books}{written or printed works}
\tstidxnewword{films}{stories recorded by camera}
\tstidxnewword{people}{human individuals (or fictional
anthropomorphisms)}
\tstidxnewword{places}{particular areas or locations}
\tstidxnewword{sub-items}{items that have a parent entry}
\tstidxnewword{applications}{computer programs}
\tstidxnewword{environments}{a marked-up block within the document
that requires special action or formatting}
\tstidxnewword{attributes}{qualities or features that characterise
something}
\tstidxnewword{parse}{divide a sentence into its component parts for
analysis}
\tstidxnewword{interpret}{translate or explain the meaning of}
\tstidxnewword{oak}{large tree that produces acorns}
\tstidxnewword{daft}{silly or foolish}
\tstidxnewword{rhubarb}{thick reddish or green leaf stalks
of a plant from the dock family}
\tstidxnewword{rhinoceros}{a large plant-eating mammal
with one or two horns on its nose}
\tstidxnewword{aardvark}{an African mammal}
\tstidxnewword{check}{make sure something is the case}
\tstidxnewword{chalk}{soft limestone substance made into sticks to
use for writing}
\tstidxnewword{index}{an alphabetical list of names, terms or
subjects with references to where they occur in the document}
\tstidxnewword{phrase}{group of words forming a unit}
\tstidxnewword{glossary}{an alphabetical list of words with
explanations}
\tstidxnewword{layout}{the way something is laid out (for example,
on a page)}
\tstidxnewword{filler}{something used to fill in gaps or increase bulk}
\tstidxnewwordseealso{padding}{something used
to pad out material to make it longer}{filler}
\tstidxnewword{example}{thing used to illustrate something}
\tstidxnewword{purpose}{the reason for doing something}
\tstidxnewword{whatsit}{In \TeX\ terms, a mechanism to allow
extensions to \TeX, including writing to files or providing
special instructions to printing devices. More generally,
a whatsit is an unnamed person or thing}
\tstidxnewword{test}{a means of testing something}
\tstidxnewword{design}{plan or purpose}
\tstidxnewword{document}{a piece of written, printed or electronic
matter that provides information}
\tstidxnewword{book}{written or printed work}
\tstidxnewword{range}{area of variation or scope}
\tstidxnewword{word}{single unit of language which has meaning}
\tstidxnewword{marker}{an identifying symbol}
\tstidxnewword{footnote}{additional information written at the
bottom of a page}
\tstidxnewword{encap}{the format used to encapsulate the location}
\tstidxnewword{text}{written or printed or electronically
represented words}
\tstidxnewword{argument}{parameter passed to a command or
application}
\tstidxnewword{predefined}{already defined}
\tstidxnewword{excessive}{more than normal or desired}
\tstidxnewword{block}{large quantity of things viewed as a unit}
\tstidxnewword{paragraph}{block of writing, beginning on a new line}
\tstidxnewword{waffle}{speak or write vague or trivial information in a lengthy way}
\tstidxnewword{gibberish}{meaningless or hard to comprehend speech
or writing}
\tstidxnewwordsee{gobbledegook}{language that is
difficult to understand}{gibberish}
\tstidxnewword{clarify}{make it easier to understand}
\tstidxnewword{confidential}{private or secret information}
\tstidxnewword{roundabout}{not direct}
\tstidxnewword{fashion}{a way of doing something}
\tstidxnewwordseealso{illustration}{an example to explain something}{example}
\tstidxnewword{wit}{natural talent for saying or writing things in
an amusing or clever way}
\tstidxnewword{eloquence}{fluent or persuasive speech or writing}
\tstidxnewword{adage}{popular saying}
\tstidxnewword{motto}{short sentence of phrase expressing a belief
or aim}
\tstidxnewword{verbiage}{excessively length speech or writing}
\tstidxnewword{drivel}{nonsense}
\tstidxnewword{tutor}{teacher}
\tstidxnewword{grass}{in addition to the plant, also slang for
reporting criminal activity}
\tstidxnewword{distress}{cause anxiety}
\tstidxnewword{concordance}{an alphabetical list of important words
used in a document}
\tstidxnewword{ogling}{staring at someone in a lecherous manner}
\tstidxnewwordsee{gawping}{staring in a rude or stupid manner}{ogling}
\tstidxnewword{passim}{referencing various places in a document}
\tstidxnewword{localisation}{place or position}
\tstidxnewword{digraph}{two letters representing a single sound}
\tstidxnewword{trigraph}{three letters representing a single sound}
\tstidxnewwordseealso{lyuk}{hole (Hungarian)}{digraph}
\tstidxnewwordseealso{dzsungel}{jungle (Hungarian)}{trigraph}
\tstidxnewword{nonsense}{something that doesn't make sense}
\tstidxnewword{vol-au-vent}{small round puff pastry containing
savoury food}
\tstidxnewword{two}{number following one}
\tstidxnewword{sculpture}{artwork made of wood, metal or stone}
\tstidxnewword{Venus}{Roman goddess (or planet named after her)}
\tstidxnewword{zoo}{place containing animals}
\tstidxnewword{alphabet}{ordered set of letters used to form words}
\tstidxnewword{resume}{continue after an interruption}
\tstidxnewword{fly}{travel by air}
\tstidxnewword{aeroplane}{fixed wing flying vehicle}
\tstidxnewword{window}{opening in wall or window to view out or in}
\tstidxnewword{ogonek}{a diacritic hook placed under the lower right
corner of a vowel}
\tstidxnewword{sail}{travel by or navigate a boat}
\tstidxnewword{ship}{large boat}
\tstidxnewword{OT1}{one of the original font encodings provided with
\TeX}
\tstidxnewword{UTF-8}{character encoding that uses 8-bit code units}
\tstidxnewword{life}{existence}
\tstidxnewword{universe}{all existing space and matter}
\tstidxnewword{even}{multiple of two}
\tstidxnewword{century}{period of 100 years or a score of 100 runs}
\tstidxnewword{cricket}{bat and ball game played between two teams
with eleven players on either side}
\tstidxnewword{calendar}{chart or list showing a period of time}
\tstidxnewword{prefix}{word or letter placed at the start of another
word}
\tstidxnewword{gobsmacked}{astonished}
\tstidxnewword{astounded}{shocked or very surprised}
\tstidxnewword{quaint}{old-fashioned or unusual}
\tstidxnewword{yatter}{chatter or talk non-stop}
\tstidxnewword{yawn}{open one's mouth due to tiredness or boredom}
\tstidxnewword{quirky}{peculiar or unexpected habits}
\tstidxnewword{badinage}{witty conversation}
\tstidxnewword{expire}{come to an end}
\tstidxnewword{story}{account or description of imaginary or real
events}
\tstidxnewword{begin}{start something}
\tstidxnewword{forget}{fail to remember something}
\tstidxnewword{tale}{a story}
\tstidxnewword{adventure}{unusual or exciting or daring experience}
\tstidxnewword{derring-do}{heroic action}
\tstidxnewword{hero}{principle male character in a story or a person
admired for their achievements}
\tstidxnewword{title}{name of a document or work, or a name that
describes a position}
\tstidxnewword{knight}{a man raised to military rank after service
as a page or squire, or a man entitled to use \tstidxqt{Sir} before
his name}
\tstidxnewword{handsome}{good-looking}
\tstidxnewword{bold}{confident or brave}
\tstidxnewword{brave}{prepared to face danger or difficulties}
\tstidxnewword{duck}{a waterbird with a blunt bill, short legs and webbed feet}
\tstidxnewword{name}{word or words by which something is known}
\tstidxnewword{plight}{dangerous situation}
\tstidxnewword{land}{area of ground}
\tstidxnewword{OgRe}{an ogre is a man-eating giant or a terrifying
person. The silly capitalisation is just a reference to \TeX's output
routine}
\tstidxnewword{reference}{referring to something}
\tstidxnewword{perilous}{dangerous or full of risk}
\tstidxnewword{quest}{long or difficult search for something}
\tstidxnewword{xor}{\LaTeX3 experimental output routine}
\tstidxnewword{continuation}{state of continuing}
\tstidxnewword{newcomers}{recent arrivals}
\tstidxnewword{nod}{in addition to the action of moving one's head
up and down, also indicates a reference of acknowledgement of
something}
\tstidxnewword{centre}{point in the middle (UK spelling)}
\tstidxnewword{center}{point in the middle (US spelling)}
\tstidxnewword{lore}{tradition or knowledge of a particular subject}
\tstidxnewword{raft}{flat water vessel made from pieces of wood
tied together}
\tstidxnewword{external}{belonging outside of something}
\tstidxnewword{pun}{play on words}
\tstidxnewword{joke}{something said to amuse others}
\tstidxnewword{witty}{having the ability to say clever or
amusing things}
\tstidxnewword{cameo}{small part in a story for a distinguished
actor}
\tstidxnewword{exclamation}{sudden cry}
\tstidxnewword{bog}{area of soft, wet, muddy ground}
\tstidxnewword{leviathans}{very large, powerful thing (such as a sea
monster)}
\tstidxnewword{vale}{valley}
\tstidxnewword{doom}{fate}
\tstidxnewword{chaos}{complete confusion or disorder}
\tstidxnewword{butterfly}{winged insect}
\tstidxnewword{motif}{repeated theme}
\tstidxnewword{sentence}{set of words forming a complete statement,
instruction or question}
\tstidxnewword{conjunction}{word used to connect words or clauses}
\tstidxnewword{naughty}{badly behaved}
\tstidxnewword{leap}{jump far or high across something}
\tstidxnewword{bound}{leaping movement}
\tstidxnewword{sword}{weapon with a long metal blade}
\tstidxnewword{myriad}{countless or a very great number}
\tstidxnewword{clones}{identical copy}
\tstidxnewword{repetition}{something that's been repeated}
\tstidxnewword{lair}{hiding place or den}
\tstidxnewword{roar}{loud, deep sound}
\tstidxnewword{peace}{freedom from war or anxiety}
\tstidxnewword{harmony}{arranged well or peacefully together}
\tstidxnewword{span}{length of time or full extent or extend across
something}
\tstidxnewword{fear}{anxiety about something unpleasant}
\tstidxnewword{thrilling}{exciting}
\tstidxnewword{quixotic}{impractically unselfish and idealist}
\tstidxnewword{sea}{large area of salt water}
\tstidxnewword{seal}{sea-dwelling fish-eating mammal with flippers}
\tstidxnewword{seaborne}{transported or travelling by sea}
\tstidxnewword{zither}{a type of stringed musical instrument}
\tstidxnewword{zealous}{having great enthusiasm for something}
\tstidxnewword{fan}{an admirer of something}
\tstidxnewword{youthful}{seeming young}
\tstidxnewword{magic}{having the apparent power of supernatural or
mysterious forces}
\tstidxnewword{magical}{relating to or using magic}
\tstidxnewword{yo-yo}{a round toy consisting of two discs and a
piece of string}
\tstidxnewword{wily}{using cunning or crafty methods to gain an
advantage}
\tstidxnewword{wombat}{a type of small marsupial}
\tstidxnewword{warrior}{brave or experienced fighter}
\tstidxnewword{laser-guided}{guided by a laser}
\tstidxnewword{villainous}{characteristic of a villain}
\tstidxnewword{zany}{amusingly unconventional}
\tstidxnewword{zoologist}{someone who studies animals}
\tstidxnewword{xebec}{a type of small sailing ship}
\tstidxnewword{xenon}{a type of inert gas}
\tstidxnewword{xylem}{a plant tissue}
\tstidxnewword{xylene}{a type of liquid hydrocarbon}
\tstidxnewword{zounds}{an exclamation}
\tstidxnewword{Ooh}{an exclamation}
\tstidxnewword{zucchini}{a type of long, green summer squash (called
a courgette in British English)}
\tstidxnewword{xylophone}{a type of musical instrument}
\tstidxnewword{exhilarating}{pleasing or energetic}
\tstidxnewword{yuppie}{urban well-paid young middle-class professional}
\tstidxnewword{yoghurt}{thick, liquid food made from milk}
\tstidxnewword{yummy}{delicious}
\tstidxnewword{yuck}{disgusting}
\tstidxnewword{Viking}{a member of Scandinavian seafaring people
between the eighth and eleventh centuries}
\tstidxnewword{vignette}{brief episode}
\tstidxnewword{viceroy}{a person who governs a colony on behalf of
the sovereign}
\tstidxnewword{vichyssoise}{a type of soup}
\tstidxnewword{viceregal}{relating to a viceroy}
\tstidxnewword{quiz}{game or competition}
\tstidxnewword{glyph}{small graphic symbol}
\tstidxnewword{asleep}{in or into a state of sleep}
\tstidxnewword{ashore}{to or on shore or land}
\tstidxnewword{aspire}{to have strong ambitions to be or do something}
\tstidxnewword{assailed}{past tense of assail}
\tstidxnewword{recover}{get well again}
\tstidxnewword[name={re-cover}]{reecover}{to cover again}
\tstidxnewdigraph{cz}{digraph in some languages, such as Polish}
\tstidxnewdigraph{dd}{digraph in some languages, such as Welsh}
\tstidxnewdigraph{ff}{digraph in some languages, such as Welsh}
\tstidxnewdigraph{ng}{digraph in some languages, such as Welsh}
\tstidxnewdigraph{ly}{digraph in some languages, such as Hungarian}
\tstidxnewtrigraph{dzs}{Hungarian \glshyperlink{trigraph} considered a separate letter}
\tstidxnewword{ddisgynedig}{descending (Welsh)}
\tstidxnewword{ddyrchafedig}{advanced (Welsh)}
\tstidxnewword{ffotograff}{photo (Welsh)}

\tstidxnewutfword{dzsoker}{dzs\'oker}{dzsóker}{joker (Hungarian)}
\tstidxnewutfword{czesc}{cze\'s\'c}{cześć}{hello (Polish)}
\tstidxnewutfword{elite}{\'elite}{élite}{group of people regarded as
the best of a particular society or organisation}
\tstidxnewutfword{aesthetic}{\ae sthetic}{æsthetic}{concerning
beauty}
\tstidxnewutfword{blase}{blas\'e}{blasé}{unimpressed or indifferent
due to familiarity}
\tstidxnewutfword{protege}{prot\'eg\'e}{protégé}{person guided by an
older more experienced person}
\tstidxnewutfword{clientele}{client\`ele}{clientèle}{all the clients
of a particular business}
\tstidxnewutfword{resumee}{r\'esum\'e}{résumé}{summary of something
or curriculum vitae}
\tstidxnewutfword{soiree}{soir\'ee}{soirée}{an evening social
gathering involving conversation or music}
\tstidxnewutfword{phoenix}{ph\oe nix}{phœnix}{mythical bird that
periodically burned itself and was reborn from the ashes}
\tstidxnewutfword{decor}{d\'ecor}{décor}{the furnishings and
decorations of a room}
\tstidxnewutfword{faerie}{f\ae rie}{færie}{fairyland}
\tstidxnewutfword{facade}{fa\c{c}ade}{façade}{the front face of a
building}
\tstidxnewutfword{aethereal}{\ae thereal}{æthereal}{light, airy or
tenuous}
\tstidxnewutfword{debutante}{d\'ebutante}{débutante}{a young
upper-class woman making her first appearance in society}
\tstidxnewutfword{naive}{na\"{\i}ve}{naïve}{lacking experience or
wisdom}
\tstidxnewutfword{foetid}{f\oe tid}{fœtid}{smelling very unpleasant}
\tstidxnewutfword{cliche}{clich\'e}{cliché}{overused phrase or idea}
\tstidxnewutfword{deshabille}{d\'eshabill\'e}{déshabillé}{the state
of being only partially clothed}
\tstidxnewutfword{negligee}{n\'eglig\'ee}{négligée}{a woman's very
thin dressing gown}
\tstidxnewutfword{cafe}{caf\'e}{café}{small restaurant that sells
light meals}
\tstidxnewutfword{anaemic}{an\ae mic}{anæmic}{suffering from anaemia}

\tstidxnewutfwordsee{thornletter}{\th}{þ}{thorn}{thorn}
\tstidxnewutfwordsee{ethletter}{\dh}{ð}{eth}{eth}
\tstidxnewutfword{thorn}{thorn (\th)}{thorn (þ)}{Old English and
Icelandic runic letter. In English, eventually replaced by the
digraph \tstidxqt{th}}
\tstidxnewutfword{eth}{eth (\dh)}{eth (ð)}{Old English letter eventually
superseded by the digraph \tstidxqt{th}, but still in use in
some other languages}
\tstidxnewutfword{oesophagus}{\oe sophagus}{œsophagus}{part of the
alimentary canal}
\tstidxnewplace{Poland}{an Eastern European country}
\tstidxnewplace{Glasgow}{a Scottish city}
\tstidxnewplace{Iceland}{a Nordic island nation}
\tstidxnewplace{Nghaerdydd}{Cardiff}
\tstidxnewplace{Nghymru}{Wales}
\tstidxnewplace{Ffestiniog}{a place in Wales}
\tstidxnewutfplace{Ostergotland}{\"Osterg\"otland}{Östergötland}{a
county in Sweden}
\tstidxnewutfplace{Angelholm}{\"Angelholm}{Ängelholm}{a place in
Sweden}
\tstidxnewutfplace{Oresund}{\O resund}{Øresund}{a strait which
separates Denmark and Sweden}
\tstidxnewutfplace{Tarnby}{T\r{a}rnby}{Tårnby}{a town in Denmark}
\tstidxnewutfplace{Rodovre}{R\o dovre}{Rødovre}{a Danish town}
\tstidxnewutfplace{Naestved}{N\ae stved}{Næstved}{a Danish town on
the island of Zealand}
\tstidxnewutfplace{OlstykkeStenlose}{\O lstykke-Stenl\o se}%
{Ølstykke-Stenløse}{a Danish city in North Zealand}
\tstidxnewutfplace{Asslar}{A\ss lar}{Aßlar}{German town}
\tstidxnewutfplace{BadGottleubaBerggiesshubel}%
{Bad Gottleuba-Berggie\ss h\"ubel}{Bad Gottleuba-Berggießhübel}%
{a town in the Free State of Saxony, Germany}
\tstidxnewutfplace{Lodz}{\L\'od\'z}{Łódź}{Polish city}
\tstidxnewutfplace{Swietokrzyskie}{\'Swi\k{e}tokrzyskie}%
{Świętokrzyskie}{Polish province}
\tstidxnewutfplace{Zory}{\.Zory}{Żory}{Polish town and city county}
\tstidxnewutfplace{Zelechow}{\.Zelech\'ow}{Żelechów}%
{Polish town}
\tstidxnewutfplace{Lobez}{\L obez}{Łobez}{Polish town}
\tstidxnewutfplace{Glogow}{G\l og\'ow}{Głogów}{Polish town}
\tstidxnewutfplace{Cmielow}{\'Cmiel\'ow}{Ćmielów}{Polish town}
\tstidxnewutfplace{Scinawa}{\'Scinawa}{Ścinawa}{Polish town}
\tstidxnewutfplace{Swidnica}{\'Swidnica}{Świdnica}{Polish town}
\tstidxnewutfplace{Olvesvatn}{\"Olvesvatn}{Ölvesvatn}{Icelandic lake}
\tstidxnewutfplace{Ulfsvatn}{\'Ulfsvatn}{Úlfsvatn}{Icelandic lake}
\tstidxnewutfplace{Anavatn}{\'Anavatn}{Ánavatn}{Icelandic lake}
\tstidxnewutfplace{Masvatn}{M\'asvatn}{Másvatn}{Icelandic lake}
\tstidxnewutfplace{Thrihyrningsvatn}{\TH r\'{\i}hyrningsvatn}%
{Þríhyrningsvatn}{Icelandic lake}
\tstidxnewutfplace{Sigridharstadhavatn}%
{Sigr\'{\i}\dh arsta\dh avatn}{Sigríðarstaðavatn}%
{Icelandic lagoon}
\tstidxnewutfplace{Graenavatn}{Gr\ae navatn}{Grænavatn}%
{Icelandic lake}
\tstidxnewutfplace{Arneslon}{\'Arnesl\'on}{Árneslón}%
{Icelandic lake}
\tstidxnewutfplace{Isholsvatn}{\'Ish\'olsvatn}{Íshólsvatn}%
{Icelandic lake}
\tstidxnewartplace{the}{Bog of Eternal Stench}{place in the film
\glshyperlink{Labyrinth}}
\tstidxnewphrase{dado rail}{waist-high moulding around the
wall of a room}
\tstidxnewphrase{indexing application}{an application that generates
a document index}
\tstidxnewphrase{visual effects}{use of imagery to create an effect}
\tstidxnewphrase{dummy text}{sample text used for demonstration not
for its content}
\tstidxnewphrase{page break}{the point where document text is broken
across two pages}
\tstidxnewphrase{link text}{for the \glshyperlink{glossariespackage}
package, this refers to the text inserted into the document
through commands like \glshyperlink{cs.gls}}
\tstidxnewphrase{cross-reference}{reference to another part of the
document or to a part of another document}
\tstidxnewphrase{marginal note}{text that's placed in the page
margin}
\tstidxnewphrase{overfull lines}{lines where the text extends into
the margin because of a formatting failure}
\tstidxnewphrase{lorem ipsum}{dummy text}
\tstidxnewphrase{between you, me and the gatepost}{an expression
meaning you're telling someone a secret that shouldn't be passed on
(common in some British dialects)}
\tstidxnewphrase{way with words}{have a particular talent with words}
\tstidxnewphrase{creative writing}{writing typically identified
by narrative craft, character development and use of literary tropes}
\tstidxnewphrase{cut to the chase}{get to the point}
\tstidxnewphrase{get to the point}{state something directly}
\tstidxnewphraseseealso{keep mum}{be silent about something}{confidential}
\tstidxnewphrase{output routine}{\TeX's method of outputting a page}
\tstidxnewphrase{out of whack}{out of order or not working}
\tstidxnewphrase{page dimensions}{the dimensions of a page (such as
the page width and page height)}
\tstidxnewphrase{font family}{the name of a font}
\tstidxnewphrase{font size}{the size of a font}
\tstidxnewphraseseealso{location list}%
{the list of locations used in an index to indicate where
the term being referenced was used in the document}{crossreference}
\tstidxnewphrasesee{range separator}{the symbol
used between the start and end location to indicate a range}{locationlist}
\tstidxnewphrase{page number}{the number identifying a
particular page}
\tstidxnewphrase{multiple encaps}{a warning issued by
\glshyperlink{makeindex} when the same page number is indexed
with different encap values}
\tstidxnewphrase{input encoding}{the character encoding used
in the document source code}
\tstidxnewphrase{font encoding}{the encoding used by the document
font}
\tstidxnewphrase{extended Latin characters}{Latin characters outside
the basic ASCII set}
\tstidxnewphrase{cup of tea}{an expression indicating what one likes
or is interested in}
\tstidxnewphrase{whistle-stop tour}{a series of short visits to
different places}
\tstidxnewphrase{number group}{a group associated with numbers}
\tstidxnewphrase{prime number}{a number that is only divisible by
itself and 1}
\tstidxnewphrase{hold my breath}{stop breathing temporarily, also
used as an expression to indicate a state of anticipation or
suspense}
\tstidxnewphrase{letter groups}{groups associated with letters}
\tstidxnewphrase{Once upon a time}{an expression commonly used at
the start of fairy tales}
\tstidxnewphrase{across the pond}{colloquial expression indicating
the other side of the Atlantic}
\tstidxnewphrase{bad form}{an offence against accepted behaviour}
\tstidxnewphrase{Monty Python}{a British surreal comedy group}
\tstidxnewphrase{magic incantation}{words used to create a magical
effect}
\tstidxnewphrase{common knowledge}{something widely known}
\tstidxnewphrase{intrepid hero}{a hero known for his boldness
and bravery}
\tstidxnewphrase{sea lion}{a type of large seal}
\tstidxnewphrase{sealant gun}{a device used for applying sealant}
\tstidxnewphrase{zoot suit}{a suit typically having a long loose
jacket and high-waisted trousers}
\tstidxnewphrase{anonymous reviewer}{an unnamed reviewer}
\tstidxnewphrase{yule log}{a large log traditionally burnt on
Christmas Eve or a log-shaped chocolate cake}
\tstidxnewphrase{vice-president}{a president's deputy}
\tstidxnewphrase{vice admiral}{a high rank of naval officer}
\tstidxnewphrase{Victoria plum}{a large, red, dessert plum}
\tstidxnewphrase{Victoria sponge}{a sponge cake consisting of two
layers with jam filling in between}
\tstidxnewphrase{vice versa}{reversing the order of the items just
mentioned}
\tstidxnewphrase{vice chancellor}{a deputy chancellor of a
British university in charge of its administration}
\tstidxnewphrase{letter ordering}{ordering according to the
individual characters}
\tstidxnewphrase{word ordering}{ordering according to the
language or locale's definition of words}
\tstidxnewphrase{mot juste}{the most appropriate word}
\tstidxnewutfphrase{attachecase}{attach\'e case}{attaché case}%
{small, flat briefcase for carrying documents}
\tstidxnewutfphrase{piedaterre}{pied-\`a-terre}{pied-à-terre}%
{small flat or house kept for occasional use}
\tstidxnewutfphrase{bergerehat}{berg\`ere hat}{bergère hat}%
{a type of wide-brimmed straw hat}
\tstidxnewartphrase{the}{commercial world}{pertaining to commerce}
\tstidxnewartphrase{a}{far away land}{somewhere that's far away; a
term often used in fairy tales}
\tstidxnewartphrase{the}{Golden Arara}{a made-up item in the dummy
text}
\tstidxnewartphrase{the}{Mighty Helm of Knuth}{a made-up item in the dummy
text}
\tstidxnewartphrase{the}{Legendary Sword}{a made-up item in the dummy
text}
\tstidxnewartphrase{the}{Bog of Eternal Glossaries}{a made-up place in the dummy
text}
\tstidxnewartphrase{the}{Dread Vale of the Editors}{a made-up place in the dummy
text}
\tstidxnewartphrase{the}{butterflies of chaos}{the butterfly effect
is a popular method of describing aspects of chaos theory}
\tstidxnewartphrase{The}{End}{denotes the end of a
story, especially fairy tales}
\tstidxnewsubword{sub-items}{lonely}{a sub-item that doesn't have
any siblings}
\tstidxnewsubword{document}{properties}{attributes such as page size}
\tstidxnewsubword{font encoding}{OT1}{one of the original font
encodings supplied with \TeX}
\tstidxnewsubwordsee{hero}{intrepid}{a hero known for his boldness
and bravery}{intrepidhero}
\tstidxnewsubphrase{location list}{page separator}%
{symbol used to separate page references}
\tstidxnewsubphrase{location list}{range separator}%
{symbol used to mark page range references}
\tstidxnewperson{James}{Joyce}{an author}
\tstidxnewperson{Donald}{Knuth}{creator of \TeX}
\tstidxnewperson{Paulo}{Cereda}{creator of \glshyperlink{arara}}
\tstidxnewperson{Sir}{Quackalot}{fictitious character}
\tstidxnewperson{the}{Fairy Goose}{fictitious character}
\tstidxnewperson{David}{Carlisle}{member of the \LaTeX3 team}
\tstidxnewperson{Joseph}{Wright}{member of the \LaTeX3 team}
\tstidxnewutfperson{Anders Jonas}{\AA ngstr\"om}%
[AndersJonasAngstrom]{Anders Jonas}{Ångström}%
{Swedish physicist}
\tstidxnewbook{Ulysses}{a modernist novel by James Joyce}
\tstidxnewbook{Sir Quackalot and the Golden Arara}{a fictitious book}
\tstidxnewbook{Sir Quackalot and the Hyper Lake of Doom}{a fictitious book}
\tstidxnewbook{Compact Oxford English Dictionary}{a dictionary}
\tstidxnewartbook{The}{Hitchhiker's Guide to the Galaxy}{a comedy
series by Douglas Adams, originally created for radio but later
adapted to book, TV and film}
\tstidxnewartbook{The}{Adventures of Sir Quackalot}{a fictitious book}
\tstidxnewfilm{Labyrinth}{a musical fantasy film}
\tstidxnewartfilm{The}{Third Man}{a British film noir}
\tstidxnewsym{TeX}{\TeX}{a typesetting system created by Donald
Knuth}
\tstidxnewmath{f(x)}[fx]{f(\protect\vec{x})}{a function of $x$}
\tstidxnewmath{n}{n}{an integer}
\tstidxnewmath{E}{E}{energy}
\tstidxnewmathsym{alpha}{\protect\alpha}{Greek letter alpha}
\tstidxnewmathsym{beta}{\protect\beta}{Greek letter beta}
\tstidxnewmathsym{gamma}{\protect\gamma}{Greek letter gamma}
\tstidxnewmathsym{sum}{\protect\sum}{summation}
\tstidxnewmathsym{partial}{\protect\partial}{partial derivative}
\tstidxnewmathsym{eth}[spinderiv]{\protect\eth}{spin-weighted partial derivative}
\tstidxnewsty{testidx}{package that produces dummy text for testing
index styles and indexing applications}
\tstidxnewsty{testidx-glossaries}{package that produces dummy text
for testing glossary styles and indexing applications that integrate
with the \glshyperlink{glossariespackage} or
\glshyperlink{glossariesextrapackage} packages}
\tstidxnewsty{glossaries}{a package for creating glossaries or lists
of terms, symbols or abbreviations}
\tstidxnewsty{glossaries-extra}{an extension to the
\glshyperlink{glossariespackage} package}
\tstidxnewsty{hyperref}{a package that provides extensive support
for hypertext}
\tstidxnewsty{lipsum}{a package that generates dummy text}
\tstidxnewsty{inputenc}{a package that can be used to identify the document encoding}
\tstidxnewstyseealso{fontenc}{a package that can be used to set the font
encoding}{inputencpackage}
\tstidxnewsty{amsmath}{a package that provides AMS mathematical
facilities}
\tstidxnewsty{amssymb}{a package that provides mathematical
symbols}
\tstidxnewsty{longtable}{a package that allows tables to flow over
page boundaries}
\tstidxnewsty{makeidx}{a package that provides indexing
facilities}
\tstidxnewsty{imakeidx}{a sophisticated package that provides indexing
facilities}
\tstidxnewstyopt{testidx-glossaries}{extra}{load the
\glshyperlink{glossariesextrapackage} package}
\tstidxnewstyopt{testidx-glossaries}{noextra}{don't load the
\glshyperlink{glossariesextrapackage} package (only load
\glshyperlink{glossariespackage})}
\tstidxnewstyopt{testidx-glossaries}{noseekey}{don't use the
\tstidxqt{see} key to implement the cross-referencing (use
\glshyperlink{cs.glssee} instead)}
\tstidxnewstyopt{testidx-glossaries}{seekey}{use the
\tstidxqt{see} key to implement the cross-referencing}
\tstidxnewstyopt{testidx-glossaries}{xindy}{set up the
\glshyperlink{glossariespackage} package to use \glshyperlink{xindy}
as the indexing application}
\tstidxnewstyopt{testidx-glossaries}{tex}{set up the
\glshyperlink{glossariespackage} package to use \TeX\ to
sort and collate the entries}
\tstidxnewstyopt{testidx-glossaries}{bib2gls}{set up the
\glshyperlink{glossariesextrapackage} package to use
\glshyperlink{bib2gls} as the indexing application}
\tstidxnewstyopt{testidx-glossaries}{noglsnumbers}{pass
the \tstidxstyoptfmt{glsnumbers=false} option to the
\glshyperlink{glossariespackage} package}
\tstidxnewstyopt{testidx-glossaries}{glsnumbers}{pass
the \tstidxstyoptfmt{glsnumbers=true} option to the
\glshyperlink{glossariespackage} package}
\tstidxnewstyopt{testidx-glossaries}{verbose}{write
information about the test entries in the transcript file}
\tstidxnewstyopt{testidx-glossaries}{noverbose}{don't write
information about the test entries in the transcript file}
\tstidxnewstyopt{testidx-glossaries}{desc}{add descriptions
to the dummy entries}
\tstidxnewstyopt{testidx-glossaries}{sanitize}{sanitize the sort value}
\tstidxnewstyopt{testidx-glossaries}{nosanitize}{don't sanitize the
sort value}
\tstidxnewstyopt{testidxglossaries}{stripaccents}{in ASCII mode, strip accent
commands from the sort value}
\tstidxnewstyopt{testidxglossaries}{nostripaccents}{in ASCII mode, strip accent
commands from the sort value}
\tstidxnewstyopt{testidx}{hidemarks}{hide the marks showing
where the indexing is occurring}
\tstidxnewstyopt{testidx}{showmarks}{mark
where the indexing is occurring}
\tstidxnewstyopt{testidx}{notestencaps}{don't use the
test encaps}
\tstidxnewstyopt{testidx}{verbose}{show the index
commands in the document text}
\tstidxnewstyopt{testidx}{noverbose}{don't show the index
commands in the document text}
\tstidxnewstyopt{testidx}{digraphs}{use glyphs instead of
the two-character digraphs for certain words}
\tstidxnewstyopt{testidx}{german}{change the
\glshyperlink{makeindex} quote character to allow the
double-quote character to indicate an umlaut}
\tstidxnewstyopt{testidx}{ngerman}{change the
\glshyperlink{makeindex} quote character to allow the
double-quote character to indicate an umlaut}
\tstidxnewstyopt{testidx}{sanitize}{sanitize the sort value
before passing it to the indexing command}
\tstidxnewstyopt{testidx}{nosanitize}{don't sanitize the sort value
before passing it to the indexing command}
\tstidxnewstyopt{testidx}{stripaccents}{in ASCII mode, strip accent
commands from the sort value}
\tstidxnewstyopt{testidx}{nostripaccents}{in ASCII mode, strip accent
commands from the sort value}
\tstidxnewstyopt{testidx}{prefix}{insert a prefix before the sort
value for certain symbols}
\tstidxnewstyopt{testidx}{noprefix}{don't insert a prefix before the sort
value for certain symbols}
\tstidxnewstyopt{hyperref}{hidelinks}{don't use a visual effect to
show the hyperlinks}
\tstidxnewstyopt{fontenc}{T1}{set the font encoding to T1}
\tstidxnewenv{theindex}{environment used to display an index}
\tstidxnewenv{align}{environment provided by the
\glshyperlink{amsmathpackage} package to align equations}
\tstidxnewenv{eqnarray}{environment provided by the \LaTeX\
kernel to align equations}
\tstidxnewapp{bib2gls}{an indexing application designed
to work with the \glshyperlink{glossariesextrapackage} package}
\tstidxnewapp{makeindex}{an indexing application}
\tstidxnewapp{xindy}{a highly-configurable indexing application with
localisation support}
\tstidxnewapp{texdoc}{an application for viewing documentation
installed in a \TeX\ distribution}
\tstidxnewapp{arara}{an automation tool for building documents}
\tstidxnewapp{Perl}{a scripting language}
\tstidxnewapp{makeglossaries}{a Perl script provided with the
\glshyperlink{glossariespackage} package that automatically runs
either \glshyperlink{makeindex} or \glshyperlink{xindy}
according to the document settings}
\tstidxnewapp{makeglossaries-lite}{a light-weight Lua alternative to
\glshyperlink{makeglossaries}}
\tstidxnewapp{Emacs}{a text editor}
\tstidxnewapp{Vi}{a text editor}
\tstidxnewappopt{xindy}{-L swedish}{use the Swedish language module}
\tstidxnewappopt{xindy}{-L danish}{use the Danish language module}
\tstidxnewappopt{xindy}{-L polish}{use the Polish language module}
\tstidxnewappopt{xindy}{-L icelandic}{use the Icelandic language module}
\tstidxnewappopt{xindy}{-L german-duden}{use the German language
module with the duden setting}
\tstidxnewappopt{xindy}{-L german-din5007}{use the German language
module with the din5007 setting}
\tstidxnewappopt{xindy}{-M ord/letorder}{use the letter ordering
module}
\tstidxnewappopt{makeindex}{-g}{use the German setting that
recognises the double-quote character as an umlaut command}
\tstidxnewappopt{makeindex}{-l}{use letter ordering}
\tstidxnewindexmarker{tstidxmarker}{indicates where the indexing
command was used for a top-level (level~0) entry}
\tstidxnewindexmarker{tstidxsubmarker}{indicates where the indexing
command was used for a level~1 entry}
\tstidxnewindexmarker{tstidxsubsubmarker}{indicates where the indexing
command was used for a level~2 entry}
\tstidxnewindexmarker{tstidxopenmarker}{indicates where
the start of a range was indexed for a top-level (level~0) entry}
\tstidxnewindexmarker{tstidxclosemarker}{indicates where
the end of a range was indexed for a top-level (level~0) entry}
\tstidxnewindexmarker{tstidxopensubmarker}{indicates where
the start of a range was indexed for a level~1 entry}
\tstidxnewindexmarker{tstidxclosesubmarker}{indicates where
the end of a range was indexed for a level~1 entry}
\tstidxnewindexmarker{tstidxopensubsubmarker}{indicates where
the start of a range was indexed for a level~2 entry}
\tstidxnewindexmarker{tstidxclosesubsubmarker}{indicates where
the end of a range was indexed for a level~2 entry}
\tstidxnewindexmarker{tstidxseemarker}{indicates where the indexing
command was used to cross-reference a top-level (level~0) entry}
\tstidxnewindexmarker{tstidxsubseemarker}{indicates where the indexing
command was used to cross-reference a level~1 entry}
\tstidxnewencapcsn{tstidxencapi}{first test encap}
\tstidxnewencapcsn{tstidxencapii}{second test encap}
\tstidxnewencapcsn{tstidxencapiii}{third test encap}
\tstidxnewcs{index}{write information to the external index file
that will be processed by an indexing application (defined by the
\LaTeX\ kernel)}

\tstidxnewcs{testidx}{produce the dummy text (defined in the
\glshyperlink{testidxpackage} package)}

\tstidxnewcs{tstidxtoidx}{switch back to the original definitions
provided by the base \glshyperlink{testidxpackage} package (defined
in the \glshyperlink{testidxglossariespackage} package)}

\tstidxnewcs{gls}{reference a term defined by the
\glshyperlink{glossariespackage} package (displays text and performs
indexing)}

\tstidxnewcs{glspl}{as \glshyperlink{cs.gls} but displays the
plural form}

\tstidxnewcs{glsadd}{indexes a term defined by the
\glshyperlink{glossariespackage} package (but doesn't display any text)}

\tstidxnewcs{glssee}{indexes a cross-referenced term or terms defined by the
\glshyperlink{glossariespackage} package (but doesn't display any text)}

\tstidxnewcs{glsxtrindexseealso}{indexes a \tstidxqt{see also}
cross-referenced term or terms defined by the
\glshyperlink{glossariesextrapackage} package (but doesn't display any text)}

\tstidxnewcs{glshyperlink}{displays the text associated with a
term (with a hyperlink if enabled) but doesn't perform
any indexing (defined by the \glshyperlink{glossariespackage} package)}

\tstidxnewcs{setglossarystyle}{sets the glossary style
(defined by the \glshyperlink{glossariespackage} package)}

\tstidxnewcs{delimN}{page delimiter used in location lists
(defined by the \glshyperlink{glossariespackage} package)}

\tstidxnewcs{delimR}{page range delimiter used in location lists
(defined by the \glshyperlink{glossariespackage} package)}

\tstidxnewcs{tstidxfootnote}{produces a footnote
(defined by the \glshyperlink{testidxpackage} package)}

\tstidxnewcs{footnote}{produces a footnote
(defined by the \LaTeX\ kernel)}

\tstidxnewcs{tstidxtext}{used to mark the sample text being indexed
(defined by the \glshyperlink{testidxpackage} package)}

\tstidxnewcs{textcolor}{displays the given text in the given colour
(a colour package is required to enable this command)}

\tstidxnewcs{glstreenamefmt}{used to set the font for
the name field in the tree-like glossary styles}

\tstidxnewcs{tstindex}{used to index the sample text
for the base \glshyperlink{testidxpackage} package (not for the
\glshyperlink{testidxglossariespackage} package)}

\tstidxnewcs{GlsAddXdyAttribute}{adds a \glshyperlink{xindy}
attribute (provided by the \glshyperlink{glossariespackage} package)}

\tstidxnewcs{IeC}{used internally by the
\glshyperlink{inputencpackage}
package}

\tstidxnewcs{tstidxindexmarkerprefix}{prefix used in the
sort key for markers if the \tstidxstyoptfmt{prefix}
option is used (but not with the \tstidxstyoptfmt{bib2gls}
option)}

\tstidxnewcs{tstidxindexmathsymprefix}{prefix used in the
sort key for mathematical symbols if the \tstidxstyoptfmt{prefix}
option is used (but not with the \tstidxstyoptfmt{bib2gls}
option)}

\tstidxnewcs{tstidxmakegloss}{command used to load
the files containing the sample glossary definitions
and also use the appropriate command to initialise the indexing,
depending on the package options}
\tstidxnewnumber{42}{forty-two}
\tstidxnewnumber{10}{ten}
\tstidxnewnumber{16}{sixteen}
\tstidxnewnumber{2}{two}
\tstidxnewnumber{100}{one hundred}
\endinput
%%
%% End of file `testidx-glossaries-samples.tex'.

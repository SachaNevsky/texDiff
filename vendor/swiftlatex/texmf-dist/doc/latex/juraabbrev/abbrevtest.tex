%%
%% This is file `abbrevtest.tex',
%% generated with the docstrip utility.
%%
%% The original source files were:
%%
%% juraabbrev.dtx  (with options: `example')
%% 
%% IMPORTANT NOTICE:
%% 
%% For the copyright see the source file.
%% 
%% Any modified versions of this file must be renamed
%% with new filenames distinct from abbrevtest.tex.
%% 
%% For distribution of the original source see the terms
%% for copying and modification in the file juraabbrev.dtx.
%% 
%% This generated file may be distributed as long as the
%% original source files, as listed above, are part of the
%% same distribution. (The sources need not necessarily be
%% in the same archive or directory.)
%% Written by Lucas Wartenburger
%%
%%
%% \CharacterTable
%%  {Upper-case    \A\B\C\D\E\F\G\H\I\J\K\L\M\N\O\P\Q\R\S\T\U\V\W\X\Y\Z
%%   Lower-case    \a\b\c\d\e\f\g\h\i\j\k\l\m\n\o\p\q\r\s\t\u\v\w\x\y\z
%%   Digits        \0\1\2\3\4\5\6\7\8\9
%%   Exclamation   \!     Double quote  \"     Hash (number) \#
%%   Dollar        \$     Percent       \%     Ampersand     \&
%%   Acute accent  \'     Left paren    \(     Right paren   \)
%%   Asterisk      \*     Plus          \+     Comma         \,
%%   Minus         \-     Point         \.     Solidus       \/
%%   Colon         \:     Semicolon     \;     Less than     \<
%%   Equals        \=     Greater than  \>     Question mark \?
%%   Commercial at \@     Left bracket  \[     Backslash     \\
%%   Right bracket \]     Circumflex    \^     Underscore    \_
%%   Grave accent  \`     Left brace    \{     Vertical bar  \|
%%   Right brace   \}     Tilde         \~}
%%
\documentclass{article}
\usepackage{german}
\usepackage[T1]{fontenc}
\usepackage[latin1]{inputenc}
\usepackage{index}
\usepackage{juraabbrev}
\newindex{laws}{ldx}{lnd}{Zitierte Gesetze}
\begin{document}

"`So steht es in der \abbnjw."'
"`In \abbdb habe ich aber was anderes gelesen."'
"`Noch anders jetzt Woopen in \abbdstr."'

\abbnjw heisst ausgeschrieben n"amlich \abbnjw+.
\abbbb+ erscheint nicht im Verzeichnis, da er nur in der
Langform verwendet wird, nicht aber als Abkuerzung.
Dagegen erscheint \abbistr+, obwohl nicht verwendet.
Das liegt an dem * im Aufruf.

Nun sollen noch ein paar Gesetze zitiert werden,
damit sich das Verzeichnis f�llt.

Paragraphen: \citepar*{242}{bgb}
\citepar*{323}{bgb} \S\S~242, 323 \abbbgb
\citepar{17}[Abs. 2][S. 1 Nr. 1]{estg}

Artikel: \citeart{33}[Abs.~2][S. 1]{gg}

Sonstiges: \citelaw[art.]{23}[al. 2][]{cciv}

Damit hier noch ein Verzeichnis erscheint, sollte
\verb|makeindex| mit folgenden Optionen aufgerufen werden:
\begin{verbatim}
makeindex -o "abbrevtest.lnd" "abbrevtest.ldx" -s laws.ist
\end{verbatim}

\begin{longabbreviations}
\printindex[laws]
\end{longabbreviations}

\begin{abbreviations}

\abbrev{bgb}{BGB}{B�rgerliches Gesetzbuch}
\abbrev{gg}{GG}{Grundgesetz}
\abbrev{estg}{EStG}{Einkommensteuergesetz}
\abbrev{cciv}{CCiv}{Code Civil}
\abbrev{njw}{NJW}{Neue Juristische Wochenschrift}
\abbrev{dstr}{DStR}{Deutsches Steuerrecht}
\abbrev{db}{DB}{Der Betrieb}
\abbrev{bb}{BB}{Der Betriebsberater}
\abbrev*{istr}{IStR}{Internationales Steuerrecht}

\end{abbreviations}

\end{document}
\endinput
%%
%% End of file `abbrevtest.tex'.

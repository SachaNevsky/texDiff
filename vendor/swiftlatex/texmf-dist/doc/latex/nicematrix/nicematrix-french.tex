% -*- coding: utf-8 ; -*-
\documentclass[dvipsnames]{article}% dvipsnames is for xcolor (loaded by Tikz, loaded by nicematrix)
\usepackage{xltxtra}
\usepackage[french]{babel}
\frenchsetup{og = « , fg = »}

\usepackage{geometry}
\geometry{left=2.8cm,right=2.8cm,top=2.5cm,bottom=2.5cm,papersize={21cm,29.7cm}}

\usepackage{amsmath}

\usepackage{array}
\usepackage{colortbl}
\usepackage{nicematrix}
\usepackage{tikz}
\usetikzlibrary{fit}


\usepackage{enumitem}

\usepackage{siunitx}


\usepackage{arydshln}
\usepackage{verbatim}

% We use \MakeShortVerb of shortvrb and not \DefineShortVerb of fancyvrb
% because we don't want the contents of short verbatim colored in gray
\usepackage{shortvrb}
\MakeShortVerb{\|}


\usepackage{fancyvrb}
\fvset{commandchars=\~\#\@,formatcom=\color{gray}}

\usepackage{titlesec}
\titlespacing*{\section}{0pt}{6.5ex plus 1ex minus .2ex}{4.3ex plus .2ex}
\titlespacing*{\subsection}{0pt}{4.5ex plus 1ex minus .2ex}{2ex plus .2ex}

\usepackage{multicol}
\setlength{\columnseprule}{0.4pt}

\def\interitem{\vspace{7mm plus 2 mm minus 3mm}}          
\def\emphase{\bgroup\color{RoyalPurple}\let\next=}

\usepackage{footnote}

\usepackage{booktabs}

\usepackage{varwidth}

\usepackage[hyperfootnotes = false]{hyperref}

\hypersetup
  {
    pdfinfo = 
      {
        Title = L’extension nicematrix ,
        Subject = Une extension LaTeX ,
        Author = F. Pantigny 
      }
  } 


\NewDocumentEnvironment {scope} {} {} {}

\NewDocumentCommand {\pkg} {m} {\textsf{#1}}
\NewDocumentCommand {\cls} {m} {\textsf{#1}}

\setlength{\parindent}{0pt}


\begin{document}

\VerbatimFootnotes

\title{L'extension \pkg{nicematrix}
       \thanks{Ce document correspond à la version~\myfileversion\space of \pkg{nicematrix},
               en date du~\myfiledate.}} 
\author{F. Pantigny \\ \texttt{fpantigny@wanadoo.fr}}


\maketitle

\begin{abstract}
L'extension LaTeX \pkg{nicematrix} fournit de nouveaux environnements similaires aux environnements classiques
|{array}| et |{matrix}| mais avec des fonctionnalités supplémentaires. Parmi ces fonctionnalités figurent la
possibilité de fixer la largeur des colonnes et de tracer des traits en pointillés continus entre les cases du
tableau.
\end{abstract}

\vspace{1cm}
\section{Présentation}

Cette extension peut être utilisée avec |xelatex|, |lualatex| et |pdflatex| mais aussi avec le cheminement
classique |latex|-|dvips|-|ps2pdf| (ou Adobe Distiller). Deux ou trois compilations successives peuvent être
nécessaires. Cette extension nécessite et charge les extensions \pkg{expl3}, \pkg{l3keys2e}, \pkg{xparse},
\pkg{array}, \pkg{amsmath} et \pkg{pgfcore} ainsi que le module \pkg{shapes} de \textsc{pgf} (l'extension
\pkg{tikz} n'est \emph{pas} chargée). L'utilisateur final n'a qu'à charger l'extension \pkg{nicematrix} avec
l'instruction habituelle : |\usepackage{nicematrix}|.



\medskip
\begin{savenotes}
\begin{minipage}{0.6\linewidth}
Cette extension fournit quelques outils supplémentaires pour dessiner des matrices (au sens mathématique). Les
principales caractéristiques sont les suivantes :
\begin{itemize}\setlength{\itemsep}{0pt}
\item des lignes en pointillés continues\footnote{Si l'option de classe
\texttt{draft} est utilisée, ces lignes en pointillés ne sont pas tracées pour
accélérer la compilation.} ;
\item des rangées et colonnes extérieures pour les labels ;
\item un contrôle sur la largeur des colonnes.
\end{itemize}
\end{minipage}
\end{savenotes}
\hspace{1.4cm}
$\begin{bNiceArray}{CCCC}[first-row,first-col,
                          code-for-first-col=\color{blue}\scriptstyle,
                          code-for-first-row=\color{blue}\scriptstyle,
                          columns-width = auto]
       & C_1     & C_2    & \Cdots  & C_n    \\
L_1    & a_{11}  & a_{12}  & \Cdots  & a_{1n} \\
L_2    & a_{21}  & a_{22}  & \Cdots  & a_{2n} \\
\Vdots & \Vdots & \Vdots  & \Ddots  & \Vdots \\
L_n    & a_{n1}  & a_{n2}  & \Cdots  & a_{nn} 
\end{bNiceArray}$




\medskip
Une commande|\NiceMatrixOptions| est fournie pour régler les options (la portée des options fixées par cette
commande est le groupe TeX courant).




\bigskip
\textbf{Un exemple d'utilisation pour les lignes en pointillés continues}

\medskip
\begin{minipage}{10cm}
Considérons par exemple le code suivant qui utilise un environnement |{pmatrix}| de l'extension \pkg{amsmath}.

\smallskip
\begin{BVerbatim}
$A = \begin{pmatrix}
1      & \cdots & \cdots & 1      \\
0      & \ddots &        & \vdots \\
\vdots & \ddots & \ddots & \vdots \\
0      & \cdots & 0      & 1
\end{pmatrix}$
\end{BVerbatim}

\smallskip
Ce code compose la matrice $A$ représentée à droite.
\end{minipage}\hspace{1cm}
$A = \begin{pmatrix}
1      &\cdots &\cdots &1      \\
0      &\ddots &       &\vdots \\
\vdots &\ddots &\ddots &\vdots \\
0      &\cdots &0      &1
\end{pmatrix}$




\bigskip
\begin{scope}
\NiceMatrixOptions{transparent}
\begin{minipage}{10cm}
Maintenant, si nous utilisons l'extension \pkg{nicematrix} avec l'option |transparent|, le même code va donner le
résultat ci-contre à droite.
\end{minipage}\hspace{1cm}
$A = \begin{pmatrix}
1      & \cdots & \cdots & 1      \\
0      & \ddots &        & \vdots \\
\vdots & \ddots & \ddots & \vdots \\
0      & \cdots & 0      & 1
\end{pmatrix}$
\end{scope}




\section{Les environnements de cette extension} 

L'extension \pkg{nicematrix} définit les nouveaux environnements suivants :

\medskip
\begin{ttfamily}
\setlength{\tabcolsep}{3mm}
\begin{tabular}{llll}
\{NiceMatrix\}  & \{NiceArray\} & \{pNiceArray\}          \\
\{pNiceMatrix\} &               & \{bNiceArray\}          \\
\{bNiceMatrix\} &               & \{BNiceArray\}          \\
\{BNiceMatrix\} &               & \{vNiceArray\}          \\
\{vNiceMatrix\} &               & \{VNiceArray\}          \\
\{VNiceMatrix\} &               & \{NiceArrayWithDelims\} \\
\end{tabular}
\end{ttfamily}



\medskip
Par défaut, les environnements |{NiceMatrix}|, |{pNiceMatrix}|, |{bNiceMatrix}|, |{BNiceMatrix}|, |{vNiceMatrix}|
et |{VNiceMatrix}| se comportent quasiment comme les environnements correspondants de \pkg{amsmath} : |{matrix}|,
|{pmatrix}|, |{bmatrix}|, |{Bmatrix}|, |{vmatrix}| et |{Vmatrix}|.


\medskip
L'environnement |{NiceArray}| est similaire à l'environnement |{array}| de l'extension |{array}|. Néanmoins, pour
des raisons techniques, dans le préambule de l'environnement |{NiceArray}|, l'utilisateur doit utiliser les lettres
|L|, |C| et~|R| au lieu de |l|, |c| et |r|. Il est possible d'utiliser les constructions |w{...}{...}|,
|W{...}{...}|\footnote{Pour les colonnes de type |w| et |W|, les cases sont composées en mode mathématique (dans
  les environnements de \pkg{nicematrix}) alors que dans |{array}| de \pkg{array}, elles sont composées en mode
  texte.}, \verb+|+, |>{...}|, |<{...}|, |@{...}|, |!{...}| et |*{n}{...}| mais les lettres |p|, |m| et |b| ne
doivent pas être employées. Voir p.~\pageref{NiceArray} la partie concernant |{NiceArray}|.

\section{Les lignes en pointillés continues}

À l'intérieur des environnements de l'extension \pkg{nicematrix}, de nouvelles commandes sont définies : |\Ldots|,
|\Cdots|, |\Vdots|, |\Ddots|, et |\Iddots|. Ces commandes sont conçues pour être utilisées à la place de |\dots|,
|\cdots|, |\vdots|, |\ddots| et |\iddots|.\footnote{La commande |\iddots|, définie dans \pkg{nicematrix}, est une
  variante de |\ddots| avec les points allant vers le haut. Si \pkg{mathdots} est chargée, la
  version de \pkg{mathdots} est utilisée. Elle correspond à la commande |\adots| de \pkg{unicode-math}.}
\newcounter{fniddots} 
\setcounter{fniddots}{\thefootnote}

\smallskip
Chacune de ces commandes doit être utilisée seule dans la case du tableau et elle trace une ligne en pointillés
entre les premières cases non vides\footnote{La définition précise de ce qui est considéré comme une «case vide»
  est donnée plus loin (cf. p.~\pageref{empty-cells}).} situées de part et d'autre de la case courante. Bien
entendu, pour |\Ldots| et |\Cdots|, c'est une ligne horizontale ; pour |\Vdots|, c'est une ligne verticale et pour
|\Ddots| et |\Iddots|, ce sont des lignes diagonales. On peut changer la couleur d'une ligne avec
l'option~|color|.\footnote{Il est aussi possible de changer la couleur de toutes ces lignes pointillées avec
  l'option |xdots/color| (\textsl{xdots} pour rappeler que cela s'applique à |\Cdots|, |\Ldots|, |Vdots|, etc.) :
  cf. p.~\pageref{customization}).
}\par\nobreak   

\bigskip
\begin{BVerbatim}[baseline=c,boxwidth=10cm]
\begin{bNiceMatrix}
a_1      & \Cdots &        & & a_1     \\
\Vdots   & a_2    & \Cdots & & a_2     \\
         & \Vdots & \Ddots[color=red] \\
\\
a_1      & a_2    &        & & a_n 
\end{bNiceMatrix}
\end{BVerbatim}
$\begin{bNiceMatrix}
a_1      & \Cdots &        & & a_1   \\
\Vdots   & a_2    & \Cdots & & a_2   \\
         & \Vdots & \Ddots[color=red] \\
\\
a_1      & a_2    &        & & a_n 
\end{bNiceMatrix}$




\interitem
Pour représenter la matrice nulle, on peut choisir d'utiliser le codage suivant :\par\nobreak

\bigskip
\begin{BVerbatim}[baseline=c,boxwidth=10cm]
\begin{bNiceMatrix}
0      & \Cdots & 0      \\
\Vdots &        & \Vdots \\
0      & \Cdots & 0 
\end{bNiceMatrix}
\end{BVerbatim}
$\begin{bNiceMatrix}
0      & \Cdots & 0      \\
\Vdots &        & \Vdots \\
0      & \Cdots & 0 
\end{bNiceMatrix}$

\bigskip
On peut néanmoins souhaiter une matrice plus grande. Habituellement, dans un tel cas, les utilisateurs de LaTeX
ajoutent une nouvelle ligne et une nouvelle colonne. Il est possible d'utiliser la même méthode avec
\pkg{nicematrix} :\par\nobreak

\bigskip
\begin{BVerbatim}[baseline=c,boxwidth=10cm]
\begin{bNiceMatrix}
0      & \Cdots & \Cdots & 0      \\
\Vdots &        &        & \Vdots \\
\Vdots &        &        & \Vdots \\
0      & \Cdots & \Cdots & 0 
\end{bNiceMatrix}
\end{BVerbatim}
$\begin{bNiceMatrix}
0      & \Cdots & \Cdots & 0      \\
\Vdots &        &        & \Vdots \\
\Vdots &        &        & \Vdots \\
0      & \Cdots & \Cdots & 0 
\end{bNiceMatrix}$

\bigskip
Dans la première colonne de cet exemple, il y a deux instructions |\Vdots| mais une seule ligne en pointillés sera
tracée (il n'y a pas d'objets qui se superposent dans le fichier \textsc{pdf} résultant\footnote{Et il n'est pas
  possible de tracer une ligne |\Ldots| et une ligne |\Cdots| entre les mêmes cases.}).

\bigskip
En fait, dans cet exemple, il aurait été possible de tracer la même matrice plus rapidement avec le codage suivant
:\par\nobreak

\bigskip
\begin{BVerbatim}[baseline=c,boxwidth=10cm]
\begin{bNiceMatrix}
0       & \Cdots &        & 0      \\
\Vdots  &        &        &        \\
        &        &        & \Vdots \\
0       &        & \Cdots & 0 
\end{bNiceMatrix}
\end{BVerbatim}
$\begin{bNiceMatrix}
0       & \Cdots &        & 0      \\
\Vdots  &        &        &        \\
        &        &        & \Vdots \\
0       &        & \Cdots & 0 
\end{bNiceMatrix}$

\bigskip
Il y a aussi d'autres moyens de changer la taille d'une matrice. On pourrait vouloir utiliser l'argument optionnel
de la commande~|\\| pour l'espacement vertical et la commande~|\hspace*| dans une case pour l'espacement
horizontal.\footnote{Dans \pkg{nicematrix}, il faut utiliser |\hspace*| et non |\hspace| car \pkg{nicematrix}
  utilise \pkg{array}. Remarquons aussi que l'on peut également régler la largeur des colonnes en utilisant l'environnement |{NiceArray}| (ou
  une de ses variantes) avec une colonne de type~|w| ou |W|: cf. p.~\pageref{width}}

Toutefois, une commande~|\hspace*| pourrait interférer dans la construction des lignes en pointillés. C'est
pourquoi l'extension \pkg{nicematrix} fournit une commande~|\Hspace| qui est une variante de |\hspace| transparente
pour la construction des lignes en pointillés de \pkg{nicematrix}.\par\nobreak

\bigskip
\begin{BVerbatim}[baseline=c,boxwidth=10cm]
\begin{bNiceMatrix}
0      & \Cdots & ~emphase#\Hspace*{1cm}@ & 0      \\
\Vdots &        &               & \Vdots \\~emphase#[1cm]@
0      & \Cdots &               & 0    
\end{bNiceMatrix}
\end{BVerbatim}
$\begin{bNiceMatrix}
0      & \Cdots & \Hspace*{1cm} & 0      \\
\Vdots &        &               & \Vdots \\[1cm]
0      & \Cdots &               & 0    
\end{bNiceMatrix}$

\subsection{L'option nullify-dots}

Considérons la matrice suivante qui a été composée classiquement avec l'environnement |{pmatrix}| de
\pkg{amsmath}.\par\nobreak

\medskip
\begin{BVerbatim}[baseline=c,boxwidth=10cm]
$A = \begin{pmatrix}
h & i & j & k & l & m \\
x &   &   &   &   & x \\
\end{pmatrix}$
\end{BVerbatim}
$A = \begin{pmatrix}
h & i & j & k & l & m \\
x   &     &     &    &     & x \\
\end{pmatrix}$


\bigskip
Si nous ajoutons des instructions |\ldots| dans la seconde rangée, la géométrie de la matrice est
modifiée.\par\nobreak

\medskip
\begin{BVerbatim}[baseline=c,boxwidth=10cm]
$B = \begin{pmatrix}
h & i & j & k & l & m \\
x & \ldots & \ldots  & \ldots & \ldots & x \\
\end{pmatrix}$
\end{BVerbatim}
$B = \begin{pmatrix}
h & i & j & k & l & m \\
x   & \ldots   & \ldots  & \ldots & \ldots & x \\
\end{pmatrix}$

\bigskip
Par défaut, avec \pkg{nicematrix}, si nous remplaçons |{pmatrix}| par |{pNiceMatrix}| et |\ldots| par |\Ldots|, la
géométrie de la matrice n'est pas changée.\par\nobreak

\medskip
\begin{BVerbatim}[baseline=c,boxwidth=10cm]
$C = \begin{pNiceMatrix}
h & i & j & k & l & m \\
x & \Ldots & \Ldots & \Ldots & \Ldots & x \\
\end{pNiceMatrix}$
\end{BVerbatim}
$C = \begin{pNiceMatrix}
h & i & j & k & l & m \\
x & \Ldots & \Ldots  & \Ldots & \Ldots & x \\
\end{pNiceMatrix}$

\bigskip
On pourrait toutefois préférer la géométrie de la première matrice $A$ et vouloir avoir la même géométrie avec une
ligne en pointillés continue dans la seconde rangée. C'est possible en utilisant l'option |nullify-dots| (et une
seule instruction |\Ldots| suffit).\par\nobreak

\medskip
\begin{BVerbatim}[baseline=c,boxwidth=10cm]
$D = \begin{pNiceMatrix}[~emphase#nullify-dots@]
h & i & j & k & l & m \\
x & \Ldots & & & & x \\
\end{pNiceMatrix}$
\end{BVerbatim}
$D = \begin{pNiceMatrix}[nullify-dots]
h & i & j & k & l & m \\
x & \Ldots & &  &  & x \\
\end{pNiceMatrix}$

\medskip
L'option |nullify-dots| «smashe» les instructions |\Ldots| (et ses variantes) horizontalement mais aussi
verticalement.

\medskip
{\bfseries Il doit n'y avoir aucun espace devant le crochet ouvrant (|[|) des options de l'environnement.}

\subsection{La commande \textbackslash Hdotsfor}

Certaines personnes utilisent habituellement la commande |\hdotsfor| de l'extension \pkg{amsmath} pour tracer des
lignes en pointillés horizontales dans une matrice. Dans les environnements de \pkg{nicematrix}, il convient
d'utiliser |\Hdotsfor| à la place pour avoir les lignes en pointillés similaires à toutes celles tracées par
l'extension \pkg{nicematrix}.

Comme avec les autres commandes de \pkg{nicematrix} (comme |\Cdots|, |\Ldots|, |\Vdots|, etc.), la ligne en
pointillés tracée par |\Hdotsfor| s'étend jusqu'au contenu des cases de part et d'autre.

\medskip
\begin{BVerbatim}[baseline=c,boxwidth=7cm]
$\begin{pNiceMatrix}
1 & 2 & 3 & 4 & 5 \\
1 & ~emphase#\Hdotsfor{3}@ & 5 \\
1 & 2 & 3 & 4 & 5 \\
1 & 2 & 3 & 4 & 5 
\end{pNiceMatrix}$
\end{BVerbatim}
$\begin{pNiceMatrix}
1 & 2 & 3 & 4 & 5 \\
1 & \Hdotsfor{3} & 5 \\
1 & 2 & 3 & 4 & 5 \\
1 & 2 & 3 & 4 & 5 
\end{pNiceMatrix}$

\bigskip
Néanmoins, si ces cases sont vides, la ligne en pointillés s'étend seulement dans les cases spécifiées par
l'argument de |\Hdotsfor| (par conception).

\medskip
\begin{BVerbatim}[baseline=c,boxwidth=7cm]
$\begin{pNiceMatrix}
1 & 2 & 3 & 4 & 5 \\
  & ~emphase#\Hdotsfor{3}@ \\
1 & 2 & 3 & 4 & 5 \\
1 & 2 & 3 & 4 & 5 \\
\end{pNiceMatrix}$
\end{BVerbatim}
$\begin{pNiceMatrix}
1 & 2 & 3 & 4 & 5 \\
  & \Hdotsfor{3} \\
1 & 2 & 3 & 4 & 5 \\
1 & 2 & 3 & 4 & 5 
\end{pNiceMatrix}$

\bigskip
La commande |\hdotsfor| de \pkg{amsmath} prend un argument optionnel (entre crochets) qui est utilisé pour un
réglage fin de l'espace entre deux points consécutifs. Par homogénéité, |\Hdotsfor| prend aussi un argument
optionnel mais cet argument est écarté silencieusement.

Remarque : Contrairement à la commande |\hdotsfor| de \pkg{amsmath}, la commande |\Hdotsfor| est utilisable lorsque
l'extension \pkg{colortbl} est chargée (mais vous risquez d'avoir des problèmes si vous utilisez |\rowcolor| sur la
même rangée que |\Hdotsfor|).

\subsection{Comment créer les lignes en pointillés de manière transparente}

L'extension \pkg{nicematrix} fournit une option appelée |transparent| qui permet d'utiliser du code existant de
manière transparente dans les environnements de l'\pkg{amsmath} : |{matrix}|, |{pmatrix}|, etc. En fait, cette
option est un alias pour la conjonction de deux options : |renew-dots| et |renew-matrix|.\footnote{Comme toutes les
  autres options, les options |renew-dots|, |renew-matrix| et |transparent| peuvent être fixées avec la commande
  |\NiceMatrixOptions|, mais elles peuvent aussi être passées en option du |\usepackage| (ce sont les trois
  seules).}

\smallskip

\begin{itemize}
\item L'option |renew-dots|\par\nobreak 

Avec cette option, les commandes |\ldots|, |\cdots|, |\vdots|, |\ddots|, |\iddots|\footnotemark[\thefniddots] et
|\hdotsfor| sont redéfinies dans les environnements de \pkg{nicematrix} et agissent alors comme |\Ldots|, |\Cdots|,
|\Vdots|, |\Ddots|, |\Iddots| et |\Hdotsfor| ; la commande |\dots| (points de suspension «automatiques» de
|amsmath|) est aussi redéfinie et se comporte comme |\Ldots|.

\item L'option |renew-matrix|\par\nobreak 

Avec cette option, l'environnement |{matrix}| est redéfini et se comporte comme |{NiceMatrix}| et il en est de même
pour les cinq variantes.
\end{itemize}

\bigskip 
Par conséquent, avec l'option |transparent|, un code classique donne directement le résultat fourni par
\pkg{nicematrix}.\par\nobreak

\bigskip
\begin{scope}
\NiceMatrixOptions{transparent}
\begin{BVerbatim}[baseline=c,boxwidth=10cm]
~emphase#\NiceMatrixOptions{transparent}@
\begin{pmatrix}
1      & \cdots & \cdots & 1      \\
0      & \ddots &        & \vdots \\
\vdots & \ddots & \ddots & \vdots \\
0      & \cdots & 0      & 1
\end{pmatrix}
\end{BVerbatim}
$\begin{pmatrix}
1      & \cdots & \cdots & 1      \\
0      & \ddots &        & \vdots \\
\vdots & \ddots & \ddots & \vdots \\
0      & \cdots & 0      & 1
\end{pmatrix}$
\end{scope}


\subsection{Personnalisation des lignes en pointillés}


\label{customization}
Les lignes pointillées tracées par |\Ldots|, |\Cdots|, |\Vdots|, |\Ddots|, |\Idots| et |\Hdotsfor| (ainsi que par la
commande |\line| dans le |code-after| décrite p.~\pageref{line-in-code-after}) peuvent être
paramétrées par trois options (que l'on met entre crochets après la commande) :
%
\begin{itemize}
\item |color| ;
\item |shorten| ; 
\item |line-style|.
\end{itemize}

Ces options peuvent aussi être fixées avec |\NiceMatrixOption| ou bien au niveau d'un environnement mais elles
doivent alors être préfixées par |xdots|, ce qui fait que leurs noms deviennent :
%
\begin{itemize}
\item |xdots/color| ;
\item |xdots/shorten| ; 
\item |xdots/line-style|.
\end{itemize}
%
Pour la clarté, dans la suite, on utilisera ces noms-là.

\bigskip
\textbf{L'option xdots/color}\par\nobreak

\smallskip
L'option |xdots/color| indique bien entendu la couleur de la ligne tracée. On remarquera néanmoins que les lignes
tracées dans les rangées et colonnes extérieures (décrites plus loin) bénificient d'un régime spécial : cf.
p.~\pageref{exterior}.


\bigskip
\textbf{L'option xdots/shorten}\par\nobreak

\smallskip
L'option |xdots/shorten| indique la marge qui est laissée aux deux extrémités de la ligne. Le nom s'inspire des options
«|shorten >|» et «|shorten <|» de Tikz, mais il faut remarquer que \pkg{nicematrix} ne propose que |xdots/shorten|.
La valeur initiale de ce paramètre est de $0.3$~em (il est conseillé d'utiliser une unité de mesure dépendante de
la fonte courante).


\bigskip
\textbf{L'option xdots/line-style}\par\nobreak

\smallskip
Il faut savoir que, par défaut, les lignes de Tikz tracées avec le paramètre |dotted| sont composées de points
carrés et non pas ronds.\footnote{La raison de départ est que le format \textsc{pdf} comporte un système de description
  de lignes en tiretés, qui, puisqu'il est incorporé dans le \textsc{pdf}, est affiché très rapidement par les
  lecteurs de \textsc{pdf}. Il est facile à partir de ce type de ligne de créer des lignes de points carrés alors
  qu'une ligne de points ronds doit être construite explicitement point par point.}

\begin{BVerbatim}[baseline=c,boxwidth=9cm]
\tikz \draw [dotted] (0,0) -- (5,0) ;
\end{BVerbatim}
\tikz \draw [dotted] (0,0) -- (5,0) ;

\bigskip
Voulant proposer des lignes avec des points ronds dans le style de celui de |\ldots| (au moins celui des fontes
\emph{Computer Modern}), l'extension \pkg{nicematrix} contient en interne son propre système de ligne en pointillés
(qui, au passage n'utilise que \pkg{pgf} et non \pkg{tikz}). Ce style est appelé le style |standard|. Cette valeur
est la valeur initiale du paramètre |xdots/line-style|.

\bigskip
Néanmoins (quand Tikz est chargé), on peut utiliser pour |xdots/line-style| n'importe quel style proposé par Tikz,
c'est-à-dire n'importe quelle suite d'options Tikz applicables à un chemin (à l'exception de «|color|», 
«|shorten >|» et «|shorten <|»). 

\medskip
Voici par exemple une matrice tridiagonale avec le style |loosely dotted| :\par\nobreak

\medskip
\begin{BVerbatim}[baseline=c]
$\begin{pNiceMatrix}[nullify-dots,~emphase#xdots/line-style=loosely dotted@]
a      & b      & 0      &        & \Cdots & 0      \\ 
b      & a      & b      & \Ddots &        & \Vdots \\
0      & b      & a      & \Ddots &        &        \\
       & \Ddots & \Ddots & \Ddots &        & 0      \\
\Vdots &        &        &        &        & b      \\
0      & \Cdots &        & 0      & b      & a
\end{pNiceMatrix}$
\end{BVerbatim}


\[\begin{pNiceMatrix}[nullify-dots,xdots/line-style=loosely dotted]
a      & b      & 0      &        & \Cdots & 0      \\ 
b      & a      & b      & \Ddots &        & \Vdots \\
0      & b      & a      & \Ddots &        &        \\
       & \Ddots & \Ddots & \Ddots &        & 0      \\
\Vdots &        &        &        &        & b      \\
0      & \Cdots &        & 0      & b      & a
\end{pNiceMatrix}\]


\section{Les nœuds PGF-Tikz créés par l'extension nicematrix}

\label{name}

L'extension \pkg{nicematrix} crée un nœud PGF-Tikz pour chaque case (non vide) du tableau considéré. Ces nœuds sont
utilisés pour tracer les lignes en pointillés entre les cases du tableau. Toutefois, l'utilisateur peut aussi
utiliser directement ces nœuds (s'il a chargé Tikz\footnote{On rappelle que depuis la version 3.13,
  \pkg{nicematrix} ne charge plus Tikz par défaut, mais seulement \textsc{pgf} (Tikz est une surcouche de
  \textsc{pgf}).}). On commence par donner un nom au tableau (avec l'option |name|). Cela étant fait, les nœuds
sont accessibles à travers les noms «\textsl{nom}-$i$-$j$» où \textsl{nom} est le nom donné au tableau et $i$ et
$j$ les numéros de rangée et de colonne de la case considérée.

\medskip
\begin{BVerbatim}[baseline=c,boxwidth=11cm]
$\begin{pNiceMatrix}[name=~emphase#ma-matrice@]
1 & 2 & 3 \\
4 & 5 & 6 \\
7 & 8 & 9 
\end{pNiceMatrix}$
\tikz[remember picture,overlay] 
     \draw ~emphase#(ma-matrice-2-2)@ circle (2mm) ; 
\end{BVerbatim}
$\begin{pNiceMatrix}[name=ma-matrice]
1 & 2 & 3 \\
4 & 5 & 6 \\
7 & 8 & 9 
\end{pNiceMatrix}$
\tikz[remember picture,overlay] 
     \draw (ma-matrice-2-2) circle (2mm) ; 

\medskip
Ne pas oublier les options |remember picture| et |overlay|.

\bigskip
Dans l'exemple suivant, nous avons surligné toutes les cases de la matrice.
\[\begin{pNiceMatrix}[
 code-after = {\begin{tikzpicture}
                  [every node/.style = {blend mode = multiply,
                                        inner sep = 0 pt ,
                                        fill = red!15}]
               \node [fit = (1-1)] {} ;
               \node [fit = (1-3)] {} ;
               \node [fit = (2-2)] {} ;
               \node [fit = (3-1)] {} ;
               \node [fit = (3-3)] {} ;
               \node [fit = (1-2)] {} ;
               \node [fit = (2-1)] {} ;
               \node [fit = (2-3)] {} ;
               \node [fit = (3-2)] {} ;
               \end{tikzpicture}}]
a & a + b & a + b + c \\
a & a     & a + b  \\
a & a     & a 
\end{pNiceMatrix}\]


\bigskip
En fait, l'extension \pkg{nicematrix} peut créer deux séries de nœuds supplémentaires (\emph{extra nodes} en
anglais) : les «nœuds moyens» (\emph{medium nodes} en anglais) et les «nœuds larges» (\emph{large nodes} en
anglais). Les premiers sont créés avec l'option |create-medium-nodes| et les seconds avec l'option
|create-large-nodes|.\footnote{Il existe aussi l'option |create-extra-nodes| qui est un alias pour la conjonction
de |create-medium-nodes| et |create-large-nodes|.}

\medskip
Les noms des «nœuds moyens» s'obtiennent en ajoutant le suffixe «|-medium|» au nom des nœuds normaux. Dans
l'exemple suivant, on a surligné tous les «nœuds moyens». Nous considérons que cet exemple se suffit à lui-même
comme définition de ces nœuds.
\[\begin{pNiceMatrix}[
 create-medium-nodes,
 code-after = {\begin{tikzpicture}
                  [every node/.style = {fill = red!15,
                                        blend mode = multiply,
                                        inner sep = 0 pt},
                   name suffix = -medium]
               \node [fit = (1-1)] {} ;
               \node [fit = (1-2)] {} ;
               \node [fit = (1-3)] {} ;
               \node [fit = (2-1)] {} ;
               \node [fit = (2-2)] {} ;
               \node [fit = (2-3)] {} ;
               \node [fit = (3-1)] {} ;
               \node [fit = (3-2)] {} ;
               \node [fit = (3-3)] {} ;
               \end{tikzpicture}}]
a & a + b & a + b + c \\
a & a     & a + b  \\
a & a     & a 
\end{pNiceMatrix}\]


\medskip
Les noms des «nœuds larges» s'obtiennent en ajoutant le suffixe «|-large|» au nom des nœuds normaux. Dans l'exemple
suivant, on a surligné tous les «nœuds larges». Nous considérons que cet exemple se suffit à lui-même comme
définition de ces nœuds.\footnote{Il n'y a pas de «nœuds larges» créés dans les rangées et colonnes extérieures
  (pour ces rangées et colonnes, voir p.~\pageref{exterior}).}

\[\begin{pNiceMatrix}[
 create-large-nodes,
 code-after = {\begin{tikzpicture}
                  [every node/.style = {blend mode = multiply,
                                        inner sep = 0 pt},
                   name suffix = -large]
               \node [fit = (1-1),fill = red!15] {} ;
               \node [fit = (1-3),fill = red!15] {} ;
               \node [fit = (2-2),fill = red!15] {} ;
               \node [fit = (3-1),fill = red!15] {} ;
               \node [fit = (3-3),fill = red!15] {} ;
               \node [fit = (1-2),fill = blue!15] {} ;
               \node [fit = (2-1),fill = blue!15] {} ;
               \node [fit = (2-3),fill = blue!15] {} ;
               \node [fit = (3-2),fill = blue!15] {} ;
               \end{tikzpicture}}]
a & a + b & a + b + c \\
a & a     & a + b  \\
a & a     & a 
\end{pNiceMatrix}\]


\medskip
Les «nœuds larges» de la première colonne et de la dernière colonne peuvent apparaître trop petits pour certains
usages. C'est pourquoi il est possible d'utiliser les options |left-margin| et |right-margin| pour ajouter de
l'espace des deux côtés du tableau et aussi de l'espace dans les «nœuds larges» de la première colonne et de la
dernière colonne. Dans l'exemple suivant, nous avons utilisé les options |left-margin| et
|right-margin|.\footnote{Les options |left-margin| et |right-margin| prennent des dimensions comme valeurs mais, si
  aucune valeur n'est donnée, c'est la valeur par défaut qui est utilisée et elle est égale à |\arraycolsep| (par
  défaut, 5~pt).Il existe aussi une option |margin| pour fixer à la fois |left-margin| et |right-margin|.}
\[\begin{pNiceMatrix}[
 create-large-nodes,left-margin,right-margin,
 code-after = {\begin{tikzpicture}
                  [every node/.style = {blend mode = multiply,
                                        inner sep = 0 pt},
                   name suffix = -large]
               \node [fit = (1-1),fill = red!15] {} ;
               \node [fit = (1-3),fill = red!15] {} ;
               \node [fit = (2-2),fill = red!15] {} ;
               \node [fit = (3-1),fill = red!15] {} ;
               \node [fit = (3-3),fill = red!15] {} ;
               \node [fit = (1-2),fill = blue!15] {} ;
               \node [fit = (2-1),fill = blue!15] {} ;
               \node [fit = (2-3),fill = blue!15] {} ;
               \node [fit = (3-2),fill = blue!15] {} ;
               \end{tikzpicture}}]
a & a + b & a + b + c \\
a & a     & a + b  \\
a & a     & a 
\end{pNiceMatrix}\]

\medskip
Il est aussi possible d'ajouter de l'espace sur les côtés du tableau avec les options |extra-left-margin| et
|extra-right-margin|. Ces marges ne sont pas incorporées dans les «nœuds larges». Dans l'exemple suivant, nous
avons utilisé |extra-left-margin| et |extra-right-margin| avec la valeur $3$~pt.
\[\begin{pNiceMatrix}[
 create-large-nodes,left-margin,right-margin,extra-right-margin=3pt,extra-left-margin=3pt,
 code-after = {\begin{tikzpicture}
                  [every node/.style = {blend mode = multiply,
                                        inner sep = 0 pt},
                   name suffix = -large]
               \node [fit = (1-1),fill = red!15] {} ;
               \node [fit = (1-3),fill = red!15] {} ;
               \node [fit = (2-2),fill = red!15] {} ;
               \node [fit = (3-1),fill = red!15] {} ;
               \node [fit = (3-3),fill = red!15] {} ;
               \node [fit = (1-2),fill = blue!15] {} ;
               \node [fit = (2-1),fill = blue!15] {} ;
               \node [fit = (2-3),fill = blue!15] {} ;
               \node [fit = (3-2),fill = blue!15] {} ;
               \end{tikzpicture}}]
a & a + b & a + b + c \\
a & a     & a + b  \\
a & a     & a 
\end{pNiceMatrix}\]

\medskip
Dans le cas présent, si on veut un contrôle sur la hauteur des rangées, on peut ajouter un |\strut| dans chaque
rangée du tableau.
\[\begin{pNiceMatrix}[
 create-large-nodes,left-margin,right-margin,extra-right-margin=3pt,extra-left-margin=3pt,
 code-after = {\begin{tikzpicture}
                  [every node/.style = {blend mode = multiply,
                                        inner sep = 0 pt},
                   name suffix = -large]
               \node [fit = (1-1),fill = red!15] {} ;
               \node [fit = (1-3),fill = red!15] {} ;
               \node [fit = (2-2),fill = red!15] {} ;
               \node [fit = (3-1),fill = red!15] {} ;
               \node [fit = (3-3),fill = red!15] {} ;
               \node [fit = (1-2),fill = blue!15] {} ;
               \node [fit = (2-1),fill = blue!15] {} ;
               \node [fit = (2-3),fill = blue!15] {} ;
               \node [fit = (3-2),fill = blue!15] {} ;
               \end{tikzpicture}}]
\strut a & a + b & a + b + c \\
\strut a & a     & a + b  \\
\strut a & a     & a 
\end{pNiceMatrix}\]

\bigskip
On explique plus loin comment surligner les nœuds créés par Tikz (cf. p. \pageref{highlight}).

\section{Le code-after}

\label{code-after}

L'option |code-after| peut être utilisée pour indiquer du code qui sera exécuté après la construction de la matrice,
et donc, en particulier, après la construction de tous les nœuds.

\textbf{Si on a chargé Tikz}\footnote{On rappelle que depuis la version 3.13, \pkg{nicematrix} ne charge plus Tikz
  par défaut, mais seulement \textsc{pgf} (Tikz est une surcouche de \textsc{pgf}).}, on peut accéder à ces nœuds
avec des instructions Tikz classiques. Les nœuds devront être désignés sous la forme $i$-$j$ (sans le préfixe
correspondant au nom de l'environnement).

De plus, une commande spéciale, nommée |\line| est disponible pour tracer directement des lignes en pointillés
entre les nœuds).
\label{line-in-code-after}

\medskip
\begin{BVerbatim}[baseline=c,boxwidth=13cm]
$\begin{pNiceMatrix}[~emphase#code-after = {\line{1-1}{3-3}[color=blue]}@]
0 & 0 & 0 \\
0 &   & 0 \\
0 & 0 & 0 
\end{pNiceMatrix}$
\end{BVerbatim}
$\begin{pNiceMatrix}[code-after = {\line{1-1}{3-3}[color=blue]}]
0 & 0 & 0 \\
0 &   & 0 \\
0 & 0 & 0 
\end{pNiceMatrix}$

\section{L'environnement \{NiceArray\}}
\label{NiceArray}

L'environnement |{NiceArray}| est similaire à l'environnement |{array}|. Comme pour |{array}|, l'argument
obligatoire est le préambule du tableau. Néanmoins, pour des raisons techniques, l'utilisateur doit utiliser les
lettres |L|, |C| et |R|\footnote{Les types de colonnes |L|, |C| et |R| sont définis localement à l'intérieur de
  |{NiceArray}| avec la commande |\newcolumntype| de \pkg{array}. Cette définition masque une éventuelle définition
  précédente. En fait, les types de colonnes |w| et |W| sont également redéfinis.} au lieu de |l|, |c| et |r|.

Il est possible d'utiliser les constructions |w{...}{...}|, |W{...}{...}|, \verb+|+, |>{...}|, |<{...}|, |@{...}|,
|!{...}| et |*{n}{...}| mais les lettres |p|, |m| et |b| ne doivent pas être employées.\footnote{Dans une commande
  |\multicolumn|, on doit également utiliser les lettres |L|, |C| et |R|.}

\medskip
En plus des options proposées pour |{pNiceMatrix}| et ses variantes, l'environnement |{NiceArray}| propose une
option |baseline| qui prend en argument un entier qui indique le numéro de rangée dont la ligne de base servira de
ligne de base pour l'environnement |{NiceArray}|.

\medskip
\begin{BVerbatim}[baseline=c,boxwidth=11cm]
$A = 
\begin{NiceArray}{CCCC}[hvlines,~emphase#baseline=2@]
1 & 2 & 3 & 4 \\
1 & 2 & 3 & 4 \\
1 & 2 & 3 & 4 \\
\end{NiceArray}$
\end{BVerbatim}
%
$A = 
\begin{NiceArray}{CCCC}[hvlines,baseline=2]
1 & 2 & 3 & 4 \\
1 & 2 & 3 & 4 \\
1 & 2 & 3 & 4 \\
\end{NiceArray}$

\smallskip
L'option |hvlines| utilisée dans cet exemple est présentée plus loin (cf. p. \pageref{hvlines}).

\medskip
L'option |baseline| peut aussi prendre les trois valeurs spéciales |t|, |c| et |b|. Ces trois lettres peuvent aussi
être utilisées de manière absolue comme pour l'option de l'environnement |{array}| de \pkg{array}. La valeur
initiale de |baseline| est~|c|.


\medskip
Dans l'exemple suivant, on utilise l'option |t| (synonyme de |baseline=t|) immédiatement après un |\item| de liste.
On remarquera que la présence d'un |\hline| initial n'empêche pas l'alignement sur la ligne de base de la première
rangée (avec |{array}| de {array}, il faut utiliser |\firsthline|\footnote{On peut aussi utiliser |\firsthline|
  avec |{NiceArray}|.}).

\smallskip
\begin{BVerbatim}[baseline=c,boxwidth=9cm]
\begin{enumerate}
\item un item
\smallskip
\item \renewcommand{\arraystretch}{1.2}
$\begin{NiceArray}[t]{LCCCCCC}
\hline
n   & 0 & 1 & 2 & 3 & 4  & 5 \\
u_n & 1 & 2 & 4 & 8 & 16 & 32 
\hline
\end{NiceArray}$
\end{enumerate}
\end{BVerbatim}
%
\begin{minipage}{5cm}
\begin{enumerate}
\item un item
\smallskip
\item \renewcommand{\arraystretch}{1.2}
$\begin{NiceArray}[t]{LCCCCCC}
\hline
n   & 0 & 1 & 2 & 3 & 4  & 5  \\
u_n & 1 & 2 & 4 & 8 & 16 & 32 \\
\hline
\end{NiceArray}$
\end{enumerate}
\end{minipage}


\medskip
Il est également possible d'utiliser les outils de \pkg{booktabs}: |\toprule|,
|\bottomrule| et |\midrule|.\par\nobreak

\smallskip
\begin{BVerbatim}[baseline=c,boxwidth=9cm]
\begin{enumerate}
\item an item
\smallskip
\item 
$\begin{NiceArray}[t]{LCCCCCC}
~emphase#\toprule@
n   & 0 & 1 & 2 & 3 & 4  & 5 \\
~emphase#\midrule@
u_n & 1 & 2 & 4 & 8 & 16 & 32 
~emphase#\bottomrule@
\end{NiceArray}$
\end{enumerate}
\end{BVerbatim}
\begin{minipage}{5cm}
\begin{enumerate}
\item an item
\smallskip
\item 
$\begin{NiceArray}[t]{LCCCCCC}
\toprule
n   & 0 & 1 & 2 & 3 & 4  & 5  \\
\midrule
u_n & 1 & 2 & 4 & 8 & 16 & 32 \\
\bottomrule
\end{NiceArray}$
\end{enumerate}
\end{minipage}




\vspace{1cm}
L'utilisation de |{NiceArray}| permet de tracer des filets verticaux :\par\nobreak

\bigskip
\begin{BVerbatim}[baseline=c,boxwidth=10cm]
$\left[\begin{NiceArray}{CCCC|C}
a_1    & ?      & \Cdots & ?       & ?     \\
0      &        & \Ddots & \Vdots  & \Vdots\\
\Vdots & \Ddots & \Ddots & ? \\ 
0      & \Cdots & 0      & a_n     & ?     
\end{NiceArray}\right]$
\end{BVerbatim} 
$\left[\begin{NiceArray}{CCCC|C}
a_1    & ?      & \Cdots & ?       & ?     \\
0      &        & \Ddots & \Vdots  & \Vdots\\
\Vdots & \Ddots & \Ddots & ? \\ 
0      & \Cdots & 0      & a_n     & ?     
\end{NiceArray}\right]$

\vspace{1cm}
Il existe également des variantes pour l'environnement |{NiceArray}| : |{pNiceArray}|, |{bNiceArray}|,
|{BNiceArray}|, |{vNiceArray}| et |{VNiceArray}|. La clé |baseline| n'est \emph{pas} disponible pour ces
environnements. 


Dans l'exemple suivant, on utilise un environnement |{pNiceArray}| (on n'utilise pas |{pNiceMatrix}| car on
souhaite utiliser les types de colonne |L| et |R| --- avec |{pNiceMatrix}|, toutes les colonnes sont de type |C|).


\bigskip
\begin{BVerbatim}[baseline=c,boxwidth=10cm]
$\begin{pNiceArray}{LCR}
a_{11}    & \Cdots & a_{1n} \\
a_{21}    &        & a_{2n} \\
\Vdots    &        & \Vdots \\
a_{n-1,1} & \Cdots & a_{n-1,n} 
\end{pNiceArray}$
\end{BVerbatim}
$\begin{pNiceArray}{LCR}
a_{11}    & \Cdots & a_{1n} \\
a_{21}    &        & a_{2n} \\
\Vdots    &        & \Vdots \\
a_{n-1,1} & \Cdots & a_{n-1,n} 
\end{pNiceArray}$


\bigskip
\bigskip
En fait, l'environnement |{pNiceArray}| et ses variantes sont fondés sur un environnement plus général, appelé
|{NiceArrayWithDelims}|. Les deux premiers arguments obligatoires de cet environnement sont les délimiteurs gauche
et droit qui seront utilisés dans la construction de la matrice. Il est possible d'utiliser |{NiceArrayWithDelims}|
si on a besoin de délimiteurs atypiques ou asymétriques.

\medskip
\begin{BVerbatim}[baseline=c,boxwidth=11cm]
$\begin{~emphase#NiceArrayWithDelims@}
   {\downarrow}{\uparrow}{CCC}[margin]
1 & 2 & 3 \\
4 & 5 & 6 \\
7 & 8 & 9 \\
\end{~emphase#NiceArrayWithDelims@}$
\end{BVerbatim}
$\begin{NiceArrayWithDelims}
   {\downarrow}{\uparrow}{CCC}[margin]
1 & 2 & 3 \\
4 & 5 & 6 \\
7 & 8 & 9 \\
\end{NiceArrayWithDelims}$



\bigskip
\section{Les rangées et colonnes extérieures}
Les environnements de \pkg{nicematrix} permettent de composer des rangées et des colonnes «extérieures» grâce aux options
|first-row|, |last-row|, |first-col| et |last-col|.
\label{exterior}

Si elle est présente, la «première rangée» (extérieure) est numérotée par $0$ (et non $1$). Il en est de même pour
la «première rangée». 

\begin{Verbatim}
$\begin{pNiceMatrix}[~emphase#first-row,last-row,first-col,last-col@,nullify-dots]
       & C_1    & \Cdots &        & C_4    &        \\
L_1    & a_{11} & a_{12} & a_{13} & a_{14} & L_1    \\
\Vdots & a_{21} & a_{22} & a_{23} & a_{24} & \Vdots \\
       & a_{31} & a_{32} & a_{33} & a_{34} &        \\
L_4    & a_{41} & a_{42} & a_{43} & a_{44} & L_4    \\
       & C_1    & \Cdots &        & C_4    &     
\end{pNiceMatrix}$
\end{Verbatim}
%
\[\begin{pNiceMatrix}[first-row,last-row,first-col,last-col,nullify-dots]
       & C_1    & \Cdots &        & C_4    &        \\
L_1    & a_{11} & a_{12} & a_{13} & a_{14} & L_1    \\
\Vdots & a_{21} & a_{22} & a_{23} & a_{24} & \Vdots \\
       & a_{31} & a_{32} & a_{33} & a_{34} &        \\
L_4    & a_{41} & a_{42} & a_{43} & a_{44} & L_4    \\
       & C_1    & \Cdots &        & C_4    &     
\end{pNiceMatrix}\]

\bigskip
Il y a plusieurs remarques à formuler.
%
\begin{itemize}[beginpenalty=10000]
\item Si on utilise un environnement avec préambule explicite (c'est-à-dire |{NiceArray}| ou l'une de ses
variantes), on ne doit pas mettre dans ce préambule de spécification de colonne pour les éventuelles première et
dernière colonne : ce sera automatiquement (et nécessairement) une colonne |R| pour la première colonne et une
colonne |L| pour la dernière.

\item On peut se demander comment \pkg{nicematrix} détermine le nombre de rangées et de colonnes nécessaires à la
composition de la «dernière rangée» et de la «dernière colonne».

\begin{itemize}
\item Dans le cas d'un environnement avec préambule, comme |{NiceArray}| ou |{pNiceArray}|, le nombre de colonnes
se déduit évidemment du préambule.

\item Dans le cas où l'option |light-syntax| (cf. p. \pageref{light-syntax}) est utilisée, \pkg{nicematrix} profite
du fait que cette option nécessite de toutes manières le chargement complet du contenu de l'environnement (d'où
l'impossibilité de mettre du verbatim dans ce cas-là) avant composition du tableau. L'analyse du contenu de
l'environnement donne le nombre de rangées (mais pas le nombre de colonnes).

\item Dans les autres cas, \pkg{nicematrix} détermine le nombre de rangées et de colonnes à la première compilation
et l'écrit dans le fichier |.aux| pour pouvoir l'utiliser à la compilation suivante.

\textsl{Néanmoins, il est possible de donner le numéro de la dernière rangée et le numéro de la dernière colonne en
arguments des options |last-row| et |last-col| pour ce qui permettra d'accélérer le processus complet de
compilation.} C'est ce que nous ferons dans la suite.
\end{itemize}

\end{itemize}



\bigskip
On peut contrôler l'apparence de ces rangées et colonnes avec les options |code-for-first-row|,
|code-for-last-row|, |code-for-first-col| et |code-for-last-col|. Ces options sont des listes de tokens qui seront
insérées au début de chaque case de la rangée ou de la colonne considérée.

\begin{Verbatim}
\NiceMatrixOptions{~emphase#code-for-first-row@ = \color{red},
                   ~emphase#code-for-first-col@ = \color{blue},
                   ~emphase#code-for-last-row@ = \color{green},
                   ~emphase#code-for-last-col@ = \color{magenta}}
$\begin{pNiceArray}{CC|CC}[first-row,last-row=6,first-col,last-col,nullify-dots]
       & C_1    & \Cdots &        & C_4    &        \\
L_1    & a_{11} & a_{12} & a_{13} & a_{14} & L_1    \\
\Vdots & a_{21} & a_{22} & a_{23} & a_{24} & \Vdots \\
\hline
       & a_{31} & a_{32} & a_{33} & a_{34} &        \\
L_4    & a_{41} & a_{42} & a_{43} & a_{44} & L_4    \\
       & C_1    & \Cdots &        & C_4    &     
\end{pNiceArray}$
\end{Verbatim}
%
\begin{scope}
\NiceMatrixOptions{code-for-first-row = \color{red},
                   code-for-first-col = \color{blue},
                   code-for-last-row = \color{green},
                   code-for-last-col = \color{magenta}}
\begin{displaymath}
\begin{pNiceArray}{CC|CC}[first-row,last-row=5,first-col,last-col,nullify-dots]
       & C_1    & \multicolumn1C{\Cdots} &        & C_4    &        \\
L_1    & a_{11} & a_{12} & a_{13} & a_{14} & L_1    \\
\Vdots & a_{21} & a_{22} & a_{23} & a_{24} & \Vdots \\
\hline
       & a_{31} & a_{32} & a_{33} & a_{34} &        \\
L_4    & a_{41} & a_{42} & a_{43} & a_{44} & L_4    \\
       & C_1    & \multicolumn1C{\Cdots} &        & C_4    &     
\end{pNiceArray}
\end{displaymath}
\end{scope}



\emph{Remarques}
\begin{itemize}[beginpenalty=10000]
\item Comme on peut le voir dans l'exemple précédent, un filet horizontal (tracé avec |\hline|) ne s'étend pas
dans les colonnes extérieures et un filet vertical (spécifié par un caractère «\verb+|+» dans le préambule du
tableau) ne s'étend pas dans les rangées extérieures.\footnote{Ce dernier point n'est pas valable si on a chargé,
  en plus de \pkg{nicematrix}, l'extension \pkg{arydshln}. Les extensions \pkg{nicematrix} et \pkg{arydshln} ne
  sont pas parfaitement compatibles car \pkg{arydshln} redéfinit beaucoup de structures internes à \pkg{array}. Par
ailleurs, si on veut vraiment un filet qui s'étende dans la première et la dernière rangée, on peut utiliser
|!{\vline}| dans le préambule à la place de \verb+|+.}

Si on veut définir de nouveaux spécificateurs de colonnes pour des filets (par exemple plus épais), on aura
peut-être intérêt à utiliser la commande |\OnlyMainNiceMatrix| décrite p.~\pageref{OnlyMainNiceMatrix}.
\item Une spécification de couleur présente dans |code-for-first-row| s'applique à une ligne pointillée tracée
dans cette «première rangée» (sauf si une valeur a été donnée à |xdots/color|). Idem pour les autres.
\item Sans surprise, une éventuelle option |columns-width| (décrite p.~\pageref{width}) ne s'applique pas à la
«première colonne» ni à la «dernière colonne».
\item Pour des raisons techniques, il n'est pas possible d'utiliser l'option de la commande |\\| après la
«première rangée» ou avant la «dernière rangée» (le placement des délimiteurs serait erroné).
\end{itemize}





\section[Les lignes en pointillés pour séparer les rangées et les colonnes]{Les lignes en pointillés pour séparer les rangées\\ et les colonnes}


Dans les environnements de \pkg{nicematrix}, il est possible d'utiliser la commande |\hdottedline| (fournie
par \pkg{nicematrix}) qui est l'équivalent pour les pointillés des commandes |\hline| et |\hdashline|
(cette dernière étant une commande de \pkg{arydshln}).

\medskip
\begin{BVerbatim}[baseline=c,boxwidth=9.5cm]
\begin{pNiceMatrix}
1 & 2 & 3 & 4 & 5 \\
~emphase#\hdottedline@
6 & 7 & 8 & 9 & 10 \\
11 & 12 & 13 & 14 & 15 
\end{pNiceMatrix}
\end{BVerbatim}
$\begin{pNiceMatrix}
1 & 2 & 3 & 4 & 5 \\
\hdottedline
6 & 7 & 8 & 9 & 10 \\
11 & 12 & 13 & 14 & 15 
\end{pNiceMatrix}$


\bigskip
Dans les environnements avec un préambule explicite (comme |{NiceArray}|, |{pNiceArray}|, etc.), il est possible de
dessiner un trait vertical en pointillés avec le spécificateur «|:|».

\medskip
\begin{BVerbatim}[baseline=c,boxwidth=9.5cm]
\begin{pNiceArray}{CCCC~emphase#:@C}
1 & 2 & 3 & 4 & 5 \\
6 & 7 & 8 & 9 & 10 \\
11 & 12 & 13 & 14 & 15 
\end{pNiceArray}
\end{BVerbatim}
$\begin{pNiceArray}{CCCC:C}
1 & 2 & 3 & 4 & 5 \\
6 & 7 & 8 & 9 & 10 \\
11 & 12 & 13 & 14 & 15 
\end{pNiceArray}$


\bigskip
Ces lignes en pointillés ne s'étendent pas dans les rangées et colonnes extérieures.\par\nobreak


\medskip
\begin{BVerbatim}[baseline=c,boxwidth=9.5cm]
$\begin{pNiceArray}{CCC:C}%
   [first-row,last-col, 
    code-for-first-row = \color{blue}\scriptstyle,
    code-for-last-col = \color{blue}\scriptstyle ]
C_1 & C_2 & C_3 & C_4 \\
1 & 2 & 3 & 4 & L_1 \\
5 & 6 & 7 & 8 & L_2 \\
9 & 10 & 11 & 12 & L_3 \\
\hdottedline
13 & 14 & 15 & 16 & L_4 
\end{pNiceArray}$
\end{BVerbatim}
$\begin{pNiceArray}{CCC:C}%
   [first-row,last-col, 
    code-for-first-row = \color{blue}\scriptstyle,
    code-for-last-col = \color{blue}\scriptstyle ]
C_1 & C_2 & C_3 & C_4 \\
1 & 2 & 3 & 4 & L_1 \\
5 & 6 & 7 & 8 & L_2 \\
9 & 10 & 11 & 12 & L_3 \\
\hdottedline
13 & 14 & 15 & 16 & L_4 
\end{pNiceArray}$

\bigskip
Il est possible de changer dans \pkg{nicematrix} la lettre utilisée pour indiquer dans le préambule un trait
vertical en pointillés avec l'option |letter-for-dotted-lines| disponible dans |\NiceMatrixOptions|. Par exemple,
dans ce document, nous avons chargé l'extension \pkg{arydshln} qui utilise la lettre «|:|» pour indiquer un trait
vertical en tiretés. Par conséquent, en utilisant l'option |letter-for-dotted-lines|, on peut utiliser les traits
verticaux fournis à la fois par \pkg{arydshln} et par \pkg{nicematrix}.


\medskip
\begin{BVerbatim}[baseline=c,boxwidth=9.5cm]
\NiceMatrixOptions{letter-for-dotted-lines = I}
\arrayrulecolor{blue}
\begin{pNiceArray}{~emphase#C|C:CIC@}
1 & 2 & 3 & 4 \\
5 & 6 & 7 & 8 \\
9 & 10 & 11 & 12
\end{pNiceArray}
\arrayrulecolor{black}
\end{BVerbatim}
\begin{scope}
\arrayrulecolor{blue}
\NiceMatrixOptions{letter-for-dotted-lines = I}
$\begin{pNiceArray}{C|C:CIC}
1 & 2 & 3 & 4 \\
5 & 6 & 7 & 8 \\
9 & 10 & 11 & 12
\end{pNiceArray}$
\arrayrulecolor{black}
\end{scope}

\smallskip
On a utilisé |\arrayrulecolor| de \pkg{colortbl} pour colorier les trois traits.


\bigskip
\emph{Remarque} : Quand l'extension \pkg{array} (sur laquelle s'appuie \pkg{nicematrix}) est chargée, les traits
verticaux et horizontaux que l'on insère rendent le tableau plus large ou plus long d'une quantité égale à la
largeur du trait\footnote{En fait, cela est vrai pour |\hline| et «\verb+|+» mais pas pour |\cline|.}. Avec
\pkg{nicematrix}, les lignes en pointillés tracées par |\hdottedline| et «|:|» ont le même effet.

\section{La largeur des colonnes}
\label{width}

Dans les environnements avec un préambule explicite (comme |{NiceArray}|, |{pNiceArray}|, etc.), il est possible de
fixer la largeur d'une colonne avec les lettres classiques |w| et |W| de l'extension \pkg{array}. Dans les
environnements de \pkg{nicematrix}, les cases des colonnes de ce type sont composées en mode mathématique (alors
que dans |{array}| de \pkg{array}, elles sont composées en mode texte).

\medskip
\begin{BVerbatim}[baseline=c,boxwidth=10cm]
$\begin{pNiceArray}{~emphase#Wc{1cm}@CC}
1  & 12 & -123 \\
12 & 0  & 0    \\
4  & 1  & 2 
\end{pNiceArray}$
\end{BVerbatim}
$\begin{pNiceArray}{Wc{1cm}CC}
1  & 12 & -123 \\
12 & 0  & 0    \\
4  & 1  & 2 
\end{pNiceArray}$


\bigskip
Dans les environnements de \pkg{nicematrix}, il est aussi possible de fixer la largeur \emph{minimale} de toutes
les colonnes de la matrice directement avec l'option |columns-width|.

\medskip
\begin{BVerbatim}[baseline=c,boxwidth=10cm]
$\begin{pNiceMatrix}[~emphase#columns-width = 1cm@]
1  & 12 & -123 \\
12 & 0  & 0    \\
4  & 1  & 2 
\end{pNiceMatrix}$
\end{BVerbatim}
$\begin{pNiceMatrix}[columns-width = 1cm]
1  & 12 & -123 \\
12 & 0  & 0    \\
4  & 1  & 2 
\end{pNiceMatrix}$

\medskip
Noter que l'espace inséré entre deux colonnes (égal à 2 |\arraycolsep|) n'est pas supprimé (il est évidemment
possible de le supprimer en mettant |\arraycolsep| à~$0$ avant).

\bigskip
Il est possible de donner la valeur spéciale |auto| à l'option |columns-width|: toutes les colonnes du tableau
auront alors une largeur égale à la largeur de la case la plus large du tableau.\footnote{Le résultat est atteint
  dès la première compilation (mais Tikz écrivant des informations dans le fichier |.aux|, un message demandant une
  deuxième compilation apparaîtra).}\par\nobreak

\medskip
\begin{BVerbatim}[baseline=c,boxwidth=10cm]
$\begin{pNiceMatrix}[~emphase#columns-width = auto@]
1  & 12 & -123 \\
12 & 0  & 0    \\
4  & 1  & 2 
\end{pNiceMatrix}$
\end{BVerbatim}
$\begin{pNiceMatrix}[columns-width = auto]
1  & 12 & -123 \\
12 & 0  & 0    \\
4  & 1  & 2 
\end{pNiceMatrix}$

\bigskip
Sans surprise, il est possible de fixer la largeur minimale de toutes les colonnes de toutes les matrices dans une
certaine portion de document avec la commande |\NiceMatrixOptions|.\par\nobreak

\medskip
\begin{BVerbatim}[baseline=c,boxwidth=8.5cm]
~emphase#\NiceMatrixOptions{columns-width=10mm}@
$\begin{pNiceMatrix}
a & b \\ c & d 
\end{pNiceMatrix}
= 
\begin{pNiceMatrix}
1   & 1245 \\ 345 & 2 
\end{pNiceMatrix}$
\end{BVerbatim}
\begin{scope}
\NiceMatrixOptions{columns-width=10mm}
$\begin{pNiceMatrix}
a & b \\
c & d 
\end{pNiceMatrix}
= 
\begin{pNiceMatrix}
1   & 1245 \\
345 & 2 
\end{pNiceMatrix}$
\end{scope}


\bigskip
Mais il est aussi possible de fixer une zone dans laquelle toutes les matrices auront leurs colonnes de la même
largeur, égale à la largeur de la case la plus large de toutes les matrices de la zone. Cette construction utilise
l'environnement |{NiceMatrixBlock}| avec l'option |auto-columns-width|\footnote{Pour le moment, c'est le seul
  usage de l'environnement |{NiceMatrixBlock}| mais il pourrait y en avoir davantage dans le futur.}.
L'environnement |{NiceMatrixBlock}| n'a pas de rapport direct avec la commande |\Block| présentée juste ci-dessous
(cf.~p.~\pageref{Block}). 

\medskip
\begin{BVerbatim}[baseline=c,boxwidth=8.5cm]
~emphase#\begin{NiceMatrixBlock}[auto-columns-width]@
$\begin{pNiceMatrix}
a & b \\ c & d 
\end{pNiceMatrix}
= 
\begin{pNiceMatrix}
1   & 1245 \\ 345 & 2 
\end{pNiceMatrix}$
~emphase#\end{NiceMatrixBlock}@
\end{BVerbatim}
\begin{NiceMatrixBlock}[auto-columns-width]
$\begin{pNiceMatrix}
 a & b \\ c & d 
 \end{pNiceMatrix}
= 
 \begin{pNiceMatrix}
 1   & 1245 \\  345 & 2 
\end{pNiceMatrix}$
\end{NiceMatrixBlock}

\medskip
\textbf{Plusieurs compilations peuvent être nécessaires pour obtenir le résultat désiré.} 

\section{Les matrices par blocs}

\label{Block}

Cette partie, qui introduit une commande |\Block|, n'a pas de rapport direct avec l'environnement
|{NiceMatrixBlock}| présenté dans la section précédente.

\smallskip
Dans les environnements de \pkg{nicematrix}, on peut utiliser la commande |\Block| pour placer un élément au centre
d'un rectangle de cases fusionnées.

La commande |\Block| doit être utilisée dans la case supérieure gauche du bloc avec deux arguments. Le premier
argument est la taille de ce bloc avec la syntaxe $i$-$j$ où $i$ est le nombre de rangées et $j$ le nombre de
colonnes du bloc. Le deuxième argument, est, sans surprise, le contenu du bloc (en mode mathématique). Un nœud Tikz
correspondant à l'ensemble des cases fusionnées est créé sous le nom «$i$-$j$-block» où \textsl{nom} est le
nom donné au tableau. Si on a demandé la création des nœuds |medium|, alors un nœud de ce type est aussi créé pour
ce bloc avec un nom suffixé par |-medium|.

\medskip
Dans les exemples qui suivent, on utilise la commande |\arrayrulecolor| de \pkg{colortbl}.

\medskip
\begin{BVerbatim}[baseline=c,boxwidth=10.6cm]
\arrayrulecolor{cyan}
$\begin{bNiceArray}{CCC|C}[margin]
~emphase#\Block{3-3}{A}@ & & & 0 \\
& \hspace*{1cm} & & \Vdots \\
& & & 0 \\
\hline
0 & \Cdots& 0 & 0
\end{bNiceArray}$
\arrayrulecolor{black}
\end{BVerbatim}
\begin{scope}
\arrayrulecolor{cyan}
$\begin{bNiceArray}{CCC|C}[margin]
\Block{3-3}{A} & & & 0 \\
& \hspace*{1cm} & & \Vdots \\
& & & 0 \\
\hline
0 & \Cdots& 0 & 0
\end{bNiceArray}$
\arrayrulecolor{black}
\end{scope}

\bigskip
On peut souhaiter agrandir la taille du «$A$» placé dans le bloc de l'exemple précédent. Comme il est composé en
mode mathématique, on ne peut pas directement utiliser une commande comme |\large|, |\Large| ou |\LARGE|. C'est
pourquoi une option à mettre entre chevrons est proposée par |\Block| pour spécifier du code LaTeX qui sera inséré
\emph{avant} le début du mode mathématique.

\medskip
\begin{BVerbatim}[baseline=c,boxwidth=10.6cm]
\arrayrulecolor{cyan}
$\begin{bNiceArray}{CCC|C}[margin]
\Block{3-3}~emphase#<\Large>@{A} & & & 0 \\
& \hspace*{1cm} & & \Vdots \\
& & & 0 \\
\hline
0 & \Cdots& 0 & 0
\end{bNiceArray}$
\arrayrulecolor{black}
\end{BVerbatim}
\begin{scope}
\arrayrulecolor{cyan}
$\begin{bNiceArray}{CCC|C}[margin]
\Block{3-3}<\Large>{A} & & & 0 \\
& \hspace*{1cm} & & \Vdots \\
& & & 0 \\
\hline
0 & \Cdots& 0 & 0
\end{bNiceArray}$
\arrayrulecolor{black}
\end{scope}


\medskip
Pour des raisons techniques, il n'est pas possible d'écrire |\Block{|$i$|-|$j$|}{<}|  
(mais on peut écrire |\Block{|$i$|-|$j$|}<>{<}| avec le résultat attendu).


\section{Fonctionnalités avancées}

\subsection{Option d'alignement dans NiceMatrix}

Les environnements sans préambule (|{NiceMatrix}|, |{pNiceMatrix}|, |{bNiceMatrix}|, etc.) proposent les options
|l| et |r| (possédant |L| et |R| comme alias) qui imposent des colonnes alignées à gauche ou à
droite.\footnote{Cela reprend une partie des fonctionnalités proposées par les environnements |{pmatrix*}|,
  |{bmatrix*}|, etc. de \pkg{mathtools}.}

\medskip
\begin{BVerbatim}[baseline=c,boxwidth=10cm]
$\begin{bNiceMatrix}[R]
\cos x & - \sin x \\
\sin x & \cos x 
\end{bNiceMatrix}$
\end{BVerbatim}
$\begin{bNiceMatrix}[R]
\cos x & - \sin x \\
\sin x & \cos x 
\end{bNiceMatrix}$




\subsection{La commande \textbackslash rotate}

Utilisée au début d'une case, la commande |\rotate| (fournie par \pkg{nicematrix}) compose le contenu après une 
rotation de 90° dans le sens direct. 

Dans l'exemple suivant, on l'utilise dans le |code-for-first-row|.

\bigskip

\begin{BVerbatim}[baseline=c,boxwidth=12cm]
\NiceMatrixOptions%
 {code-for-first-row = \scriptstyle ~emphase#\rotate@ \text{image de },
  code-for-last-col = \scriptstyle }
$A = \begin{pNiceMatrix}[first-row,last-col=4]
e_1 & e_2 & e_3       \\
1   & 2   & 3   & e_1 \\
4   & 5   & 6   & e_2 \\
7   & 8   & 9   & e_3 \\
\end{pNiceMatrix}$
\end{BVerbatim}
\begin{varwidth}{10cm}
\NiceMatrixOptions%
 {code-for-first-row = \scriptstyle\rotate \text{image de },
  code-for-last-col = \scriptstyle }
$ A = \begin{pNiceMatrix}[first-row,last-col=4]
e_1 & e_2 & e_3 \\
1   & 2   & 3  & e_1 \\
4   & 5   & 6  & e_2 \\
7   & 8   & 9  & e_3 \\
\end{pNiceMatrix}$
\end{varwidth}

\bigskip
Si la commande |\rotate| est utilisée dans la ``dernière rangée'' (extérieure à la matrice), les éléments qui
subissent cette rotation sont alignés vers le haut.

\bigskip
\begin{BVerbatim}[baseline=c,boxwidth=12cm]
\NiceMatrixOptions%
 {code-for-last-row = \scriptstyle ~emphase#\rotate@ ,
  code-for-last-col = \scriptstyle }
$A = \begin{pNiceMatrix}[last-row,last-col=4]
1   & 2   & 3   & e_1 \\
4   & 5   & 6   & e_2 \\
7   & 8   & 9   & e_3 \\
\text{image de } e_1 & e_2 & e_3 \\
\end{pNiceMatrix}$
\end{BVerbatim}
\begin{varwidth}{10cm}
\NiceMatrixOptions%
 {code-for-last-row = \scriptstyle\rotate ,
  code-for-last-col = \scriptstyle }%
$A = \begin{pNiceMatrix}[last-row,last-col=4]
1   & 2   & 3  & e_1 \\
4   & 5   & 6  & e_2 \\
7   & 8   & 9  & e_3 \\
\text{image de } e_1 & e_2 & e_3 \\
\end{pNiceMatrix}$
\end{varwidth}


\subsection{L'option small}

\label{small}

Avec l'option |small|, les environnements de l'extension \pkg{nicematrix} sont composés d'une manière proche de ce
que propose l'environnement |{smallmatrix}| de l'\pkg{amsmath} (et les environnements |{psmallmatrix}|,
|{bsmallmatrix}|, etc. de \pkg{mathtools}).

\bigskip
\begin{Verbatim}
$\begin{bNiceArray}{CCCC|C}[~emphase#small@, 
                            last-col, 
                            code-for-last-col = \scriptscriptstyle, 
                            columns-width = 3mm ] 
1 & -2 & 3 & 4 & 5 \\
0 & 3  & 2 & 1 & 2 & L_2 \gets 2 L_1 - L_2 \\
0 & 1  & 1 & 2 & 3 & L_3 \gets L_1 + L_3 \\
\end{bNiceArray}$
\end{Verbatim}
%
\[\begin{bNiceArray}{CCCC|C}[small, last-col, code-for-last-col = \scriptscriptstyle, columns-width=3mm]
1 & -2 & 3 & 4 & 5 \\
0 & 3  & 2 & 1 & 2 & L_2 \gets 2 L_1 - L_2 \\
0 & 1  & 1 & 2 & 3 & L_3 \gets L_1 + L_3 \\
\end{bNiceArray}\]



\bigskip
On remarquera néanmoins que l'environnement |{NiceMatrix}| avec l'option |small| ne prétend pas être composé
exactement comme l'environnement |{smallmatrix}|. C'est que les environnements de \pkg{nicematrix} sont tous fondés
sur |{array}| (de \pkg{array}) alors que ce n'est pas le cas de |{smallmatrix}| (fondé directement sur un |\halign|
de TeX).

\medskip
En fait, l'option |small| correspond aux réglages suivants :
\begin{itemize}
\item les composantes du tableau sont composées en |\scriptstyle| ; 
\item |\arraystretch| est fixé à $0.47$ ; 
\item |\arraycolsep| est fixé à $1.45$~pt ; 
\item les caractéristiques des lignes en pointillés sont également modifiées.
\end{itemize}

\subsection{Les compteurs iRow et jCol}

Dans les cases du tableau, il est possible d'utiliser les compteurs LaTeX |iRow| et |jCol| qui représentent le
numéro de la rangée courante et le numéro de la colonne courante\footnote{On rappelle que le numéro de la «première
  rangée» (si elle existe) est $0$ et que le numéro de la «première colonne» (si elle existe) est $0$ également.}.
Bien entendu, l'utilisateur ne doit pas modifier les valeurs de ces compteurs qui sont utilisés en interne par
\pkg{nicematrix}.

Dans le |code-after| (cf. p. \pageref{code-after}), |iRow| représente le nombre total de rangées (hors éventuelles
rangées extérieures) et |jCol| le nombre total de colonnes (hors potentielles colonnes extérieures).

\medskip
\begin{BVerbatim}[baseline=c,boxwidth=10.6cm]
$\begin{pNiceMatrix}%
    [first-row,
     first-col,
     code-for-first-row = \mathbf{~emphase#\alph{jCol}@} ,
     code-for-first-col = \mathbf{~emphase#\arabic{iRow}@} ]
&   &    &    &   \\
& 1 & 2  & 3  & 4 \\
& 5 & 6  & 7  & 8 \\
& 9 & 10 & 11 & 12
\end{pNiceMatrix}$
\end{BVerbatim}
$\begin{pNiceMatrix}[first-row,
                   first-col,
                   code-for-first-row = \mathbf{\alph{jCol}} ,
                   code-for-first-col = \mathbf{\arabic{iRow}} ]
&   &    &    &   \\
& 1 & 2  & 3  & 4 \\
& 5 & 6  & 7  & 8 \\
& 9 & 10 & 11 & 12
\end{pNiceMatrix}$

\medskip
Si des compteurs LaTeX nommés |iRow| ou |jCol| sont créés dans le document par d'autres extensions que
\pkg{nicematrix} (ou tout simplement par l'utilisateur), ces compteurs sont masqués dans les environnements de
\pkg{nicematrix}.


\bigskip
L'extension \pkg{nicematrix} propose aussi des commandes pour composer automatiquement des matrices à partir d'un
motif général. Ces commandes sont nommées |\pAutoNiceMatrix|, |\bAutoNiceMatrix|, |\vAutoNiceMatrix|,
|\VAutoNiceMatrix| et |\BAutoNiceMatrix|.

Chacune de ces commandes prend deux arguments obligatoires : le premier est la taille de la matrice, sous la forme 
$n$-$p$, où $n$ est le nombre de rangées et $p$ est le nombre de colonnes et le deuxième est le motif (c'est-à-dire
simplement des tokens qui seront insérés dans chaque case de la matrice, exceptées celles des éventuelles rangées et
colonnes extérieures). 

\medskip
\begin{Verbatim}
$C = ~emphase#\pAutoNiceMatrix@{3-3}{C_{\arabic{iRow},\arabic{jCol}}}$
\end{Verbatim}


$C = \pAutoNiceMatrix{3-3}{C_{\arabic{iRow},\arabic{jCol}}}$

\subsection{Les options hlines, vlines et hvlines}

\label{hvlines}

Dans les environnements de \pkg{nicematrix}, on peut bien entendu ajouter des filets horizontaux entre les rangées
avec la commande~|\hline| et des filets verticaux avec le spécificateur ``\verb+|+'' dans le préambule de
l'environnement. Par souci de commodité, l'extension \pkg{nicematrix} fournit aussi l'option |hlines| (resp.
|vlines|) qui impose directement que tous les filets horizontaux (resp. verticaux) soient tracés (à l'exception,
très naturelle, des filets extérieurs aux rangées et colonnes extérieures). L'option |lines| est la conjonction des
options |hlines| et |vlines|.

\medskip
Dans l'exemple suivant, on utilise la commande |\arrayrulecolor| de \pkg{colortbl}.

\medskip
\begin{BVerbatim}[baseline=c,boxwidth=11cm]
\arrayrulecolor{cyan}
$\begin{NiceArray}{CCCC}%
 [~emphase#hvlines@,first-row,first-col]
  & e & a & b & c \\
e & e & a & b & c \\
a & a & e & c & b \\
b & b & c & e & a \\
c & c & b & a & e 
\end{NiceArray}$
\arrayrulecolor{black}
\end{BVerbatim}
%
\arrayrulecolor{cyan}
$\begin{NiceArray}{CCCC}[hvlines,first-row,first-col]
  & e & a & b & c \\
e & e & a & b & c \\
a & a & e & c & b \\
b & b & c & e & a \\
c & c & b & a & e 
\end{NiceArray}$
\arrayrulecolor{black}


\bigskip
Il y a néanmoins une différence entre l'utilisation de l'option |vlines| et du spécificateur ``\verb+|+'' dans le
préambule de l'environnement : les filets tracés par |vlines| traversent les double-filets horizontaux tracés par
|\hline\hline|. 

\medskip
\begin{BVerbatim}[baseline=c,boxwidth=11.5cm]
$\begin{NiceArray}{CCCC}[vlines] \hline
a & b & c & d \\ \hline \hline
1 & 2 & 3 & 4 \\
1 & 2 & 3 & 4 \\ \hline
\end{NiceArray}$
\end{BVerbatim}
%
$\begin{NiceArray}{CCCC}[vlines]
\hline
a & b & c & d \\
\hline \hline
1 & 2 & 3 & 4 \\
1 & 2 & 3 & 4 \\
\hline
\end{NiceArray}$

\bigskip
Dans le cas d'un environnement avec délimiteurs (par exemple |{pNiceArray}| ou |{pNiceMatrix}|), l'option |vlines| ne
trace pas de filets verticaux au niveau des deux délimiteurs (bien entendu).

\medskip
\begin{BVerbatim}[baseline=c,boxwidth=10.6cm]
\setlength{\arrayrulewidth}{0.2pt}
$\begin{pNiceMatrix}[vlines]
1 & 2 & 3 & 4 & 5 & 6 \\
1 & 2 & 3 & 4 & 5 & 6 \\
1 & 2 & 3 & 4 & 5 & 6 \\
\end{pNiceMatrix}$
\end{BVerbatim}
%
\begin{scope}
\setlength{\arrayrulewidth}{0.2pt}
$\begin{pNiceMatrix}[vlines]
1 & 2 & 3 & 4 & 5 & 6 \\
1 & 2 & 3 & 4 & 5 & 6 \\
1 & 2 & 3 & 4 & 5 & 6 \\
\end{pNiceMatrix}$
\end{scope}




\subsection{L'option light-syntax}

\label{light-syntax}

L'option |light-syntax|\footnote{Cette option est inspirée de l'extension \pkg{spalign} de Joseph Rabinoff.} permet
d'alléger la saisie des matrices, ainsi que leur lisibilité dans le source TeX. Lorsque cette option est activée,
on doit utiliser le point-virgule comme marqueur de fin de rangée et séparer les colonnes par des espaces ou des
tabulations. On remarquera toutefois que, comme souvent dans le monde TeX, les espaces après les séquences de
contrôle ne sont pas comptées et que les éléments entre accolades sont considérés comme un tout.


\medskip
\begin{scope}
\begin{BVerbatim}[baseline=c,boxwidth=10cm]
$\begin{bNiceMatrix}[~emphase#light-syntax@,first-row,first-col]
{} a             b                 ;
a  2\cos a       {\cos a + \cos b} ;
b \cos a+\cos b  { 2 \cos b }
\end{bNiceMatrix}$
\end{BVerbatim}
\end{scope}
% 
$\begin{bNiceMatrix}[light-syntax,first-row,first-col]
{} a             b                 ;
a  2\cos a       {\cos a + \cos b} ;
b \cos a+\cos b  { 2 \cos b }
\end{bNiceMatrix}$

\medskip
On peut changer le caractère utilisé pour indiquer les fins de rangées avec l'option |end-of-row|. Comme dit
précédemment, la valeur initiale de ce paramètre est un point-virgule.

\medskip
Lorsque l'option |light-syntax| est utilisée, il n'est pas possible de mettre d'éléments en verbatim (avec par
exemple la commande |\verb|) dans les cases du tableau.\footnote{La raison en est que lorsque l'option
  |light-syntax| est utilisée, le contenu complet de l'environnement est chargé comme un argument de commande TeX.
  L'environnement ne se comporte plus comme un «vrai» environnement de LaTeX qui se contente d'insérer des
  commandes avant et après.}

\subsection{Utilisation du type de colonne S de siunitx}

Si l'extension \pkg{siunitx} est chargée (avant ou après \pkg{nicematrix}), il est possible d'utiliser les colonnes
de type |S| de \pkg{siunitx} dans les environnements de \pkg{nicematrix}. L'implémentation n'utilise explicitement
aucune macro privée de \pkg{siunitx}. 

\medskip
\begin{BVerbatim}[baseline = c, boxwidth = 10.5cm]
$\begin{pNiceArray}{~emphase#S@CWc{1cm}C}[nullify-dots,first-row]
{C_1} & \Cdots &  & C_n \\
2.3  & 0 & \Cdots & 0 \\
12.4 & \Vdots & & \Vdots \\
1.45 \\
7.2  & 0 & \Cdots & 0 
\end{pNiceArray}$
\end{BVerbatim}
$\begin{pNiceArray}{SCWc{1cm}C}[nullify-dots,first-row]
{C_1} & \Cdots &  & C_n \\
2.3  & 0 & \Cdots & 0 \\
12.4 & \Vdots & & \Vdots \\
1.45 \\
7.2  & 0 & \Cdots & 0 
\end{pNiceArray}$

\medskip
En revanche, les colonnes |d| de l'extension \pkg{dcolumn} ne sont pas prises en charge par \pkg{nicematrix}.



\section{Remarques techniques}

\subsection{Pour définir de nouveaux types de colonnes}

\label{OnlyMainNiceMatrix}

L'extension \pkg{nicematrix} fournit la commande |\OnlyMainNiceMatrix| qui est destinée à être utilisée dans des
définitions de nouveaux types de colonnes. Son argument n'est exécuté que si on se place dans la partie principale
du tableau, c'est-à-dire que l'on n'est pas dans l'une des éventuelles rangées extérieures.

Par exemple, si on souhaite définir un type de colonne |?| pour tracer un trait fort (noir) d'épaisseur 1~pt, on
pourra écrire\footnote{La commande |\vrule| est une commande de TeX (et non de LaTeX).} :
\begin{Verbatim}
\newcolumntype{?}{!{\OnlyMainNiceMatrix{\vrule width 1 pt}}}
\end{Verbatim}

Le trait fort correspondant ne s'étendra pas dans les rangées extérieures :

\medskip
\begin{scope}
\newcolumntype{?}{!{\OnlyMainNiceMatrix{\vrule width 1 pt}}}

\begin{BVerbatim}[baseline = c, boxwidth = 10.5cm]
$\begin{pNiceArray}{CC?CC}[first-row,last-row=3]
C_1 & C_2 & C_3 & C_4 \\
a & b & c & d \\
e & f & g & h \\
C_1 & C_2 & C_3 & C_4
\end{pNiceArray}$
\end{BVerbatim}
$\begin{pNiceArray}{CC?CC}[first-row,last-row=3]
C_1 & C_2 & C_3 & C_4 \\
a & b & c & d \\
e & f & g & h \\
C_1 & C_2 & C_3 & C_4
\end{pNiceArray}$
\end{scope}

\medskip
Le spécificateur |?| ainsi créé est aussi utilisable dans les environnements |{array}| (de \pkg{array}) et, dans ce
cas, |\OnlyMainNiceMatrix| est sans effet.


\subsection{Intersection des lignes pointillées}

Depuis la version 3.1 de \pkg{nicematrix}, les lignes en pointillées créées par |\Cdots|, |\Ldots|, |\Vdots|, etc.
ne peuvent pas se croiser entre elles.\footnote{En revanche, les lignes créées par |\hdottedline|, la lettre
  ``|:|'' dans le préambule de la matrice et la commande |\line| dans le |code-after| peuvent croiser une autre
  ligne en pointillés.}

Cela signifie qu'une ligne en pointillés créée par l'une des ces commandes s'arrête automatiquement quand elle
arrive à une autre ligne pointillée déjà tracée par l'une de ces commandes. Par conséquent, l'ordre dans lequel les
lignes sont tracées a son importance pour le résultat final. Voici cet ordre (c'est à dessein qu'il a été choisi
ainsi) : |\Hdotsfor|, |\Vdots|, |\Ddots|, |\Iddots|, |\Cdots| et |\Ldots|.

\medskip
De ce fait, on peut tracer la matrice suivante :\par\nobreak

\smallskip
\begin{BVerbatim}[baseline = c, boxwidth = 10.5cm]
$\begin{pNiceMatrix}[nullify-dots]
1 & 2 & 3 & \Cdots & n \\
1 & 2 & 3 & \Cdots & n \\
\Vdots & \Cdots & & \Hspace*{15mm} & \Vdots \\
& \Cdots & & &  \\
& \Cdots & & &  \\
& \Cdots & & &  \\
\end{pNiceMatrix}$
\end{BVerbatim}
$\begin{pNiceMatrix}[nullify-dots]
1 & 2 & 3 & \Cdots & n \\
1 & 2 & 3 & \Cdots & n \\
\Vdots & \Cdots & & \Hspace*{15mm} & \Vdots \\
& \Cdots & & &  \\
& \Cdots & & &  \\
& \Cdots & & &  \\
\end{pNiceMatrix}$



\subsection{Le nom des nœuds PGF créés par nicematrix}

Les nœuds PGF-Tikz créés par \pkg{nicematrix} peuvent être utilisés hors des environnements de \pkg{nicematrix}
après avoir nommé l'environnement concerné avec l'option |name| (cf. p.~\pageref{name}). Il s'agit là de la méthode
conseillée mais on décrit néanmoins maintenant le nom Tikz interne de ces nœuds.

Les environnements créés par \pkg{nicematrix} sont numérotés par un compteur global interne. La commande
|\NiceMatrixLastEnv| donne le numéro du dernier de ces environnements (pour LaTeX, il s'agit d'une commande —
complètement développable — et non d'un compteur).

Si l'environnement concerné a le numéro $n$, alors le nœud de la rangée~$i$ et de la colonne~$j$ a pour nom
|nm-|$n$|-|$i$|-|$j$. Les noms des nœuds |medium| et |large| correspondants s'obtiennent en suffixant par |-medium|
et |-large|. 

\subsection{Lignes diagonales} 

Par défaut, toutes les lignes diagonales\footnote{On parle des lignes créées par |\Ddots| et non des lignes créées
  par une commande |\line| dans le |code-after|.} d'un même tableau sont «parallélisées». Cela signifie que la
première diagonale est tracée et que, ensuite, les autres lignes sont tracées parallèlement à la première (par
rotation autour de l'extrémité la plus à gauche de la ligne). C'est pourquoi la position des instructions |\Ddots|
dans un tableau peut avoir un effet marqué sur le résultat final.

\medskip
Dans les exemples suivants, la première instruction |\Ddots| est marquée en couleur :

\medskip
\begin{scope}
\begin{minipage}{9.5cm}
Exemple avec parallélisation (comportement par défaut):
\begin{Verbatim}
$A = \begin{pNiceMatrix}
1      & \Cdots &        & 1      \\
a+b    & ~emphase#\Ddots@~ &        & \Vdots \\
\Vdots & \Ddots &        &        \\
a+b    & \Cdots & a+b    & 1
\end{pNiceMatrix}$
\end{Verbatim}
\end{minipage}
$A = \begin{pNiceMatrix}
1      & \Cdots &     & 1      \\
a+b    & \Ddots &     & \Vdots \\
\Vdots & \Ddots &     &        \\
a+b    & \Cdots & a+b & 1
\end{pNiceMatrix}$

\bigskip
\NiceMatrixOptions{parallelize-diags=true}%
\begin{minipage}{9.5cm}
\begin{Verbatim}
$A = \begin{pNiceMatrix}
1      & \Cdots &        & 1      \\
a+b    &        &        & \Vdots \\
\Vdots & ~emphase#\Ddots@~ & \Ddots &        \\
a+b    & \Cdots & a+b    & 1
\end{pNiceMatrix}$
\end{Verbatim}
\end{minipage}
$A = \begin{pNiceMatrix}
1      & \Cdots &        & 1      \\
a+b    &        &        & \Vdots \\
\Vdots & \Ddots & \Ddots &        \\
a+b    & \Cdots & a+b    & 1
\end{pNiceMatrix}$

\bigskip
Il est possible de désactiver la parallélisation avec l'option |parallelize-diags| mise à |false|: \par\nobreak

\medskip
\NiceMatrixOptions{parallelize-diags=false}%
\begin{minipage}{9.5cm}
Le même exemple sans parallélisation :
\end{minipage}
$A = \begin{pNiceMatrix}
1      & \Cdots  &     & 1      \\
a+b    & \Ddots  &     & \Vdots \\
\Vdots & \Ddots  &     &        \\
a+b    & \Cdots  & a+b & 1
\end{pNiceMatrix}$



\end{scope}

\subsection{Les cases «vides»}

\label{empty-cells}
Une instruction comme |\Ldots|, |\Cdots|, etc. essaye de déterminer la première case vide de part et d'autre de la
case considérée. Néanmoins, une case vide n'est pas nécessairement sans contenu dans le codage TeX (c'est-à-dire
sans aucun token entre les deux esperluettes~|&|). En effet, une case dont le contenu est |\hspace*{1cm}| peut être
considérée comme vide.

\interitem
Pour \pkg{nicematrix}, les règles précises sont les suivantes :

\begin{itemize}
\item Une case implicite est vide. Par exemple, dans la matrice suivante

\begin{Verbatim}
\begin{pmatrix}
a & b \\
c \\
\end{pmatrix}
\end{Verbatim}

la dernière case (deuxième rangée et deuxième colonne) est vide.

\medskip
\item Chaque case avec un rendu par TeX de largeur nulle est vide.

\medskip
\item Une case avec une commande |\Hspace| (ou |\Hspace*|) est vide. Cette commande |\Hspace| est une commande
définie par l'extension \pkg{nicematrix} avec la même signification que |\hspace| excepté que la case où cette
commande est utilisée est considérée comme vide. Cette commande peut être utilisée pour fixer la largeur des
colonnes sans interférer avec le tracé des lignes en pointillés par \pkg{nicematrix}.

\end{itemize}


\subsection{L'option exterior-arraycolsep}

L'environnement |{array}| insère un espace horizontal égal à |\arraycolsep| avant et après chaque colonne. En
particulier, il y a un espace égal à |\arraycolsep| avant et après le tableau. Cette caractéristique de
l'environnement |{array}| n'était probablement pas une bonne idée\footnote{Dans la documentation de |{amsmath}|, on
  peut lire : {\itshape The extra space of |\arraycolsep| that \pkg{array} adds on each side is a waste so we
    remove it [in |{matrix}|] (perhaps we should instead remove it from array in general, but that's a harder
    task).}}. L'environnement |{matrix}| et ses variantes (|{pmatrix}|,
|{vmatrix}|, etc.) de \pkg{amsmath} préfèrent supprimer ces espaces avec des instructions explicites 
|\hskip -\arraycolsep|\footnote{Et non en insérant |@{}| de part et d'autre du préambule, ce qui fait que la
  longueur des |\hline| n'est pas modifiée et elle peut paraître trop longue, surtout avec des crochets.}.
L'extension \pkg{nicematrix} fait de même dans \emph{tous} ses environnements y compris l'environnement
|{NiceArray}|. Néanmoins, si l'utilisateur souhaite que l'environnement |{NiceArray}| se comporte par défaut comme
l'environnement |{array}| de \pkg{array} (par exemple pour faciliter l'adaptation d'un document existant), il peut
contrôler ce comportement avec l'option |exterior-arraycolsep| accessible via la commande |\NiceMatrixOptions|.
Avec cette option, des espaces extérieurs de longueur |\arraycolsep| seront insérés dans les environnements
|{NiceArray}| (les autres environnements de l'extension \pkg{nicematrix} ne sont pas affectés).


\subsection{L'option de classe draft}

Quand l'option de classe |draft| est utilisée, les lignes en pointillés ne sont pas tracées, pour accélérer la
compilation.


\subsection{Un problème technique avec l'argument de \textbackslash\textbackslash}


Pour des raisons techniques, si vous utilisez l'argument optionnel de la commande |\\|, l'espace vertical sera
aussi ajouté au nœud («normal») correspondant à la case précédente.


\medskip
\begin{BVerbatim}[baseline=c,boxwidth=11cm]
     \begin{pNiceMatrix}
     a & \frac AB \\~emphase#[2mm]@
     b & c
     \end{pNiceMatrix}
\end{BVerbatim}
$\begin{pNiceMatrix}[
   code-after = {\tikz \node [inner sep = 0pt,
                              fill = red!15,
                              blend mode = multiply,
                              fit = (1-2) ] {} ; } ]
a & \frac AB \\[2mm]
b & c
\end{pNiceMatrix}$

\bigskip
Il y a deux solutions pour résoudre ce problème. La première solution est d'utiliser une commande TeX pour insérer
l'espace entre les deux rangées.

\medskip
\begin{BVerbatim}[baseline=c,boxwidth=11cm]
     \begin{pNiceMatrix}
     a & \frac AB \\
     ~emphase#\noalign{\kern2mm}@
     b & c
     \end{pNiceMatrix}
\end{BVerbatim}
$\begin{pNiceMatrix}[
   code-after = {\tikz \node [inner sep = 0pt,
                              fill = red!15,
                              blend mode = multiply,
                              fit = (1-2) ] {} ; } ]
a & \frac AB \\
\noalign{\kern2mm}
b & c
\end{pNiceMatrix}$


\bigskip
L'autre solution est d'utiliser la commande |\multicolumn| dans la case précédente :

\medskip
\begin{BVerbatim}[baseline=c,boxwidth=11cm]
     \begin{pNiceMatrix}
     a & ~emphase#\multicolumn1C{\frac AB}@ \\[2mm]
     b & c
     \end{pNiceMatrix}
\end{BVerbatim}
$\begin{pNiceMatrix}[
   code-after = {\tikz \node [inner sep = 0pt,
                              fill = red!15,
                              blend mode = multiply,
                              fit = (1-2) ] {} ; } ]
a & \multicolumn1C{\frac AB} \\[2mm]
b & c
\end{pNiceMatrix}$

\subsection{Environnements obsolètes}      

La version 3.0 de \pkg{nicematrix} a introduit l'environnement |{pNiceArray}| (et ses variantes) avec les options
|first-row|, |last-row|, |first-col| et |last-col|.

Par conséquent, les environnements suivants, présents dans les versions précédentes de \pkg{nicematrix} sont devenus
obsolètes :
%
\begin{itemize}
\item |{NiceArrayCwithDelims}| ;
\item |{pNiceArrayC}|, |{bNiceArrayC}|, |{BNiceArrayC}|, |{vNiceArrayC}|,
|{VNiceArrayC}| ;
\item |{NiceArrayRCwithDelims}| ;
\item |{pNiceArrayRC}|, |{bNiceArrayRC}|, |{BNiceArrayRC}|, |{vNiceArrayRC}|,
|{VNiceArrayRC}|.
\end{itemize}

\medskip
Depuis la version 3.12 de \pkg{nicematrix}, on ne peut utiliser ces environnements que si on a chargé l'extension
\pkg{nicematrix} avec l'option |obsolete-environments|. Il faut toutefois avoir conscience que ces environnements
seront certainement supprimés dans une prochaine version de \pkg{nicematrix}.



\section{Exemples}

\subsection{Lignes en pointillés}


Une matrice de permutation.\par\nobreak

À titre d'exemple, on a augmenté la valeur du paramètre |xdots/shorten|.\par\nobreak 


\bigskip
\begin{BVerbatim}[baseline=c]
$\begin{pNiceMatrix}[~emphase#xdots/shorten=6em@]
0       & 1 & 0 &        & \Cdots &   0    \\
\Vdots  &   &   & \Ddots &        & \Vdots \\
        &   &   & \Ddots &        &        \\
        &   &   & \Ddots &        &   0    \\
0       & 0 &   &        &        &   1    \\
1       & 0 &   & \Cdots &        &   0    
\end{pNiceMatrix}$
\end{BVerbatim}
\hspace{2.5cm}
$\begin{pNiceMatrix}[xdots/shorten=0.6em]
0       & 1 & 0 &        & \Cdots &   0    \\
\Vdots  &   &   & \Ddots &        & \Vdots \\
        &   &   & \Ddots &        &        \\
        &   &   & \Ddots &        &   0    \\
0       & 0 &   &        &        &   1    \\
1       & 0 &   & \Cdots &        &   0    
\end{pNiceMatrix}$

\vspace{2cm}

Un exemple avec |\Iddots|. On a augmenté encore davantage la valeur de |xdots/shorten|.\par\nobreak
\bigskip
\begin{BVerbatim}[baseline=c]
$\begin{pNiceMatrix}[~emphase#xdots/shorten = 0.9em@]
1       & \Cdots  &         & 1      \\
\Vdots  &         &         & 0      \\
        & ~emphase#\Iddots@ & ~emphase#\Iddots@ & \Vdots \\
1       & 0       & \Cdots  & 0 
\end{pNiceMatrix}$
\end{BVerbatim}
\hspace{4cm}
$\begin{pNiceMatrix}[xdots/shorten = 0.9em]
1       & \Cdots  &         & 1      \\
\Vdots  &         &         & 0      \\
        & \Iddots & \Iddots & \Vdots \\
1       & 0       & \Cdots  & 0 
\end{pNiceMatrix}$


\vspace{2cm}
Un exemple avec |\multicolumn|:\par\nobreak
\bigskip
\begin{BVerbatim}
\begin{BNiceMatrix}[nullify-dots]
1 & 2 & 3 & 4 & 5 & 6 & 7 & 8 & 9 & 10\\
1 & 2 & 3 & 4 & 5 & 6 & 7 & 8 & 9 & 10\\
\Cdots &  & ~emphase#\multicolumn{6}{C}{10 \text{ autres lignes}}@ & \Cdots \\
1 & 2 & 3 & 4 & 5 & 6 & 7 & 8 & 9 & 10
\end{BNiceMatrix}
\end{BVerbatim}

\bigskip
\[\begin{BNiceMatrix}[nullify-dots]
1 & 2 & 3 & 4 & 5 & 6 & 7 & 8 & 9 & 10\\
1 & 2 & 3 & 4 & 5 & 6 & 7 & 8 & 9 & 10\\
\Cdots &  & \multicolumn{6}{C}{10 \text{ autres lignes}} & \Cdots \\
1 & 2 & 3 & 4 & 5 & 6 & 7 & 8 & 9 & 10
\end{BNiceMatrix}\]

\vspace{2cm}
Un exemple avec |\Hdotsfor|:\par\nobreak

\bigskip
\begin{BVerbatim}[baseline=c,boxwidth=11cm]
\begin{pNiceMatrix}[nullify-dots]
0 & 1 & 1 & 1 & 1 & 0 \\
0 & 1 & 1 & 1 & 1 & 0 \\
\Vdots  & ~emphase#\Hdotsfor{4}@ & \Vdots \\
 & ~emphase#\Hdotsfor{4}@ & \\
 & ~emphase#\Hdotsfor{4}@ & \\
 & ~emphase#\Hdotsfor{4}@ & \\
0 & 1 & 1 & 1 & 1 & 0 
\end{pNiceMatrix}
\end{BVerbatim}
$\begin{pNiceMatrix}[nullify-dots]
0 & 1 & 1 & 1 & 1 & 0 \\
0 & 1 & 1 & 1 & 1 & 0 \\
\Vdots  & \Hdotsfor{4} & \Vdots \\
 & \Hdotsfor{4} & \\
 & \Hdotsfor{4} & \\
 & \Hdotsfor{4} & \\
0 & 1 & 1 & 1 & 1 & 0 
\end{pNiceMatrix}$

\vspace{2cm}
Un exemple pour le résultant de deux polynômes :
\par\nobreak
\bigskip
\begin{BVerbatim}
\setlength{\extrarowheight}{1mm}
\begin{vNiceArray}{CCCC:CCC}[columns-width=6mm]
a_0   &      &&       &b_0    &      &     \\
a_1   &\Ddots&&       &b_1    &\Ddots&     \\
\Vdots&\Ddots&&       &\Vdots &\Ddots&b_0  \\
a_p   &      &&a_0    &       &      &b_1   \\
      &\Ddots&&a_1    &b_q    &      &\Vdots\\
      &      &&\Vdots &       &\Ddots&      \\
      &      &&a_p     &       &      &b_q    
\end{vNiceArray}
\end{BVerbatim}

\bigskip

\begin{scope}
\setlength{\extrarowheight}{1mm}
\[\begin{vNiceArray}{CCCC:CCC}[columns-width=6mm]
a_0   &      &&       &b_0    &      &     \\
a_1   &\Ddots&&       &b_1    &\Ddots&     \\
\Vdots&\Ddots&&       &\Vdots &\Ddots&b_0  \\
a_p   &      &&a_0    &       &      &b_1   \\
      &\Ddots&&a_1    &b_q    &      &\Vdots\\
      &      &&\Vdots &       &\Ddots&      \\
      &      &&a_p     &       &      &b_q    
\end{vNiceArray}\]
\end{scope}   

\vspace{2cm}
Un exemple avec un système linéaire (le trait vertical a été tracé en cyan avec les outils de
\pkg{colortbl}):\par\nobreak 

\begin{Verbatim}
\arrayrulecolor{cyan}
$\begin{pNiceArray}{*6C|C}[nullify-dots,last-col,code-for-last-col={\scriptstyle}]
1      & 1 & 1 &\Cdots &   & 1      & 0      & \\
0      & 1 & 0 &\Cdots &   & 0      &        & L_2 \gets L_2-L_1 \\
0      & 0 & 1 &\Ddots &   & \Vdots &        & L_3 \gets L_3-L_1 \\
       &   &   &\Ddots &   &        & \Vdots & \Vdots \\
\Vdots &   &   &\Ddots &   & 0      & \\
0      &   &   &\Cdots & 0 & 1      & 0      & L_n \gets L_n-L_1 
\end{pNiceArray}$
\arrayrulecolor{black}
\end{Verbatim}

\arrayrulecolor{cyan}
\[\begin{pNiceArray}{*6C|C}[nullify-dots,last-col,code-for-last-col={\scriptstyle}]
1      & 1 & 1 &\Cdots &   & 1      & 0      & \\
0      & 1 & 0 &\Cdots &   & 0      &        & L_2 \gets L_2-L_1 \\
0      & 0 & 1 &\Ddots &   & \Vdots &        & L_3 \gets L_3-L_1 \\
       &   &   &\Ddots &   &        & \Vdots & \Vdots \\
\Vdots &   &   &\Ddots &   & 0      & \\
0      &   &   &\Cdots & 0 & 1      & 0      & L_n \gets L_n-L_1 
\end{pNiceArray}\]
\arrayrulecolor{black}


\subsection{Largeur des colonnes}

\medskip
Dans l'exemple suivant, nous utilisons |{NiceMatrixBlock}| avec l'option |auto-columns-width| parce que nous
voulons la même largeur (automatique) pour toutes les colonnes.

\bigskip
\begin{BVerbatim}
~emphase#\begin{NiceMatrixBlock}[auto-columns-width]@
\NiceMatrixOptions{code-for-last-col = \color{blue}\scriptstyle}
\setlength{\extrarowheight}{1mm}
$\begin{pNiceArray}{CCCC:C}[last-col]
1&1&1&1&1&\\
2&4&8&16&9&\\
3&9&27&81&36&\\
4&16&64&256&100&
\end{pNiceArray}$
...
~emphase#\end{NiceMatrixBlock}@
\end{BVerbatim}

\bigskip

\begin{multicols}{2}
\begin{NiceMatrixBlock}[auto-columns-width]
\NiceMatrixOptions{code-for-last-col = \color{blue}\scriptstyle}
\setlength{\extrarowheight}{1mm}

\enskip $\begin{pNiceArray}{CCCC:C}[last-col]
1&1&1&1&1&\\
2&4&8&16&9&\\
3&9&27&81&36&\\
4&16&64&256&100&
\end{pNiceArray}$

\medskip

\enskip $\begin{pNiceArray}{CCCC:C}[last-col]
1&1&1&1&1&\\
0&2&6&14&7&L_2\gets-2L_1+L_2 \\
0&6&24&78&33&L_3\gets-3L_1+L_3 \\
0&12&60&252&96&L_4\gets-4L_1+L_4 
\end{pNiceArray}$

\medskip

\enskip $\begin{pNiceArray}{CCCC:C}[last-col]
1&1&1&1&1&\\
0&1&3&7&\frac72&L_2\gets\frac12L_2\\
0&3&12&39&\frac{33}2&L_3\gets\frac12L_3 \\
0&1&5&21&8&L_4\gets\frac1{12}L_4 
\end{pNiceArray}$

\medskip

\enskip $\begin{pNiceArray}{CCCC:C}[last-col]
1&1&1&1&1&\\
0&1&3&7&\frac72&\\
0&0&3&18&6&L_3 \gets -3L_2+L_3 \\
0&0&-2&-14&-\frac92&L_4 \gets L_2-L_4 
\end{pNiceArray}$

\medskip

\enskip $\begin{pNiceArray}{CCCC:C}[last-col]
1&1&1&1&1&\\
0&1&3&7&\frac72&\\
0&0&1&6&2&L_3 \gets \frac13L_3\\
0&0&-2&-14&-\frac92&
\end{pNiceArray}$

\medskip

\enskip $\begin{pNiceArray}{CCCC:C}[last-col]
1&1&1&1&1&\\
0&1&3&7&\frac72&\\
0&0&1&6&2& \\
0&0&0&-2&-\frac12 & L_4 \gets 2L_3+L_4 
\end{pNiceArray}$
\end{NiceMatrixBlock}
\end{multicols}


\subsection{Comment surligner les cases}

\label{highlight}

\medskip
Les exemples suivants nécessitent d'avoir chargé Tikz (\pkg{nicematrix} ne charge que \textsc{pgf}) ainsi que la
bibliothèque Tikz |fit|, ce qui peut se faire avec les deux instructions suivantes dans le préambule du document :

\begin{verbatim}
\usepackage{tikz}
\usetikzlibrary{fit}
\end{verbatim}



\medskip
Pour mettre en évidence une case, il est possible de «dessiner» l'un des nœuds (le «nœud normal», le «nœud moyen»
ou le «nœud large»). Dans l'exemple suivant, on utilise les «nœuds larges» de la diagonale de la matrice (avec la
clé de Tikz «|name suffix|», il est facile d'utiliser les «nœuds larges»). 

Nous redessinons les nœuds avec de nouveaux nœuds en utilisant la bibliothèque \pkg{fit} de Tikz. Comme nous
voulons recréer des nœuds identiques aux premiers, nous devons fixer |inner sep = 0pt| (si on ne fait
pas cela, les nouveaux nœuds seront plus grands que les nœuds d'origine créés par \pkg{nicematrix}).

\begin{Verbatim}
$\begin{pNiceArray}{>{\strut}CCCC}%
   [create-large-nodes,margin,extra-margin=2pt,
    code-after = {\begin{tikzpicture}
                     [~emphase#name suffix = -large@,
                      every node/.style = {draw,
                                           ~emphase#inner sep = 0pt@}]
                     \node [fit = (1-1)] {} ; 
                     \node [fit = (2-2)] {} ; 
                     \node [fit = (3-3)] {} ; 
                     \node [fit = (4-4)] {} ; 
                  \end{tikzpicture}}]
a_{11} & a_{12} & a_{13} & a_{14} \\
a_{21} & a_{22} & a_{23} & a_{24} \\
a_{31} & a_{32} & a_{33} & a_{34} \\
a_{41} & a_{42} & a_{43} & a_{44} 
\end{pNiceArray}$
\end{Verbatim}


\[\begin{pNiceArray}{>{\strut}CCCC}[
    create-large-nodes,margin,extra-margin=2pt,
    code-after = {\begin{tikzpicture}[name suffix = -large,
                                      every node/.style = {draw,
                                                           inner sep = 0pt}]
                     \node [fit = (1-1)] {} ; 
                     \node [fit = (2-2)] {} ; 
                     \node [fit = (3-3)] {} ; 
                     \node [fit = (4-4)] {} ; 
                  \end{tikzpicture}}]
a_{11} & a_{12} & a_{13} & a_{14} \\
a_{21} & a_{22} & a_{23} & a_{24} \\
a_{31} & a_{32} & a_{33} & a_{34} \\
a_{41} & a_{42} & a_{43} & a_{44} 
\end{pNiceArray}\]
%
On remarquera que les traits que l'on vient de tracer sont dessinés \emph{après} la matrice sans modifier la
position des composantes de celle-ci. En revanche, les traits tracés par |\hline|, le spécificateur «\verb+|+» ou
les options |hlines| et |vlines| «écartent» les composantes de la matrice (quand l'extension \pkg{array} est
chargée, ce qui est toujours le cas avec \pkg{nicematrix}).\footnote{En revanche les traits tracés par |\ncline|
  n'écartent pas les lignes de la matrice.}

\vspace{1cm}

L'extension \pkg{nicematrix} est construite au-dessus de l'environnement |{array}| et, par conséquent, il est
possible d'utiliser l'extension \pkg{colortbl} dans les environnements de \pkg{nicematrix}. Les possibilités de
réglage de \pkg{colortbl} sont néanmoins assez limitées. C'est pourquoi nous proposons une autre méthode pour
surligner une rangée de la matrice. Nous créons un nœud Tikz rectangulaire qui englobe les nœuds de la deuxième
rangée en utilisant les outils de la bibliothèque Tikz \pkg{fit}. Ce nœud est rempli après la construction de la
matrice. Pour que l'on puisse voir le texte \emph{sous} le nœud, nous devons utiliser la transparence avec le
|blend mode| égal à |multiply|. 


\tikzset{highlight/.style={rectangle,
                           fill=red!15,
                           blend mode = multiply, 
                           rounded corners = 0.5 mm, 
                           inner sep=1pt,
                           fit=#1}}

\medskip
\begin{Verbatim}
\tikzset{highlight/.style={rectangle,
                           fill=red!15,
                           ~emphase#blend mode = multiply@,
                           rounded corners = 0.5 mm, 
                           inner sep=1pt,
                           fit=~#1}}

$\begin{bNiceMatrix}[~emphase#code-after = {\tikz \node [highlight = (2-1) (2-3)] {} ;}@]
0 & \Cdots & 0 \\
1 & \Cdots & 1 \\
0 & \Cdots & 0 
\end{bNiceMatrix}$
\end{Verbatim}

\[\begin{bNiceMatrix}[code-after = {\tikz \node [highlight = (2-1) (2-3)] {} ;}]
0 & \Cdots & 0 \\
1 & \Cdots & 1 \\
0 & \Cdots & 0 
\end{bNiceMatrix}\]


\bigskip
Ce code échoue avec |latex|-|dvips|-|ps2pdf| parce que Tikz pour |dvips|, pour le moment, ne prend pas en charge
les \emph{blend modes}. Néanmoins, le code suivant, dans le préambule du document LaTeX, devrait activer les
\emph{blend modes} pour ce mode de compilation.


\begin{scope} \small
|\ExplSyntaxOn|

|\makeatletter|

|\tl_set:Nn \l_tmpa_tl {pgfsys-dvips.def}|

|\tl_if_eq:NNT \l_tmpa_tl \pgfsysdriver|

|  {\cs_set:Npn\pgfsys@blend@mode#1{\special{ps:~/\tl_upper_case:n #1~.setblendmode}}}|

|\makeatother|

|\ExplSyntaxOff|
\end{scope}

\vspace{1cm}
On rappelle que dans le cas d'un ensemble de cases fusionnées (avec la commande |\Block|), un nœud Tikz est créé
pour l'ensemble des cases avec pour nom $i$|-|$j$|-block| où $i$ et $j$ sont les numéros de ligne et de colonne de
la case en haut à gauche (où a été utilisée la commande |\Block|). Si on a demandé la création des nœuds |medium|,
alors un nœud de ce type est aussi créé pour ce bloc avec un nom suffixé par |-medium|.

\medskip
\begin{BVerbatim}[baseline=c,boxwidth=11.6cm]
$\begin{pNiceMatrix}%
  [
    margin,
    create-medium-nodes,
    code-after = 
     { \tikz \node [~emphase#highlight = (1-1-block-medium)@] {} ; } 
  ]
\Block{3-3}<\Large>{A} & & & 0 \\
& \hspace*{1cm} & & \Vdots \\
& & & 0 \\
0 & \Cdots& 0 & 0
\end{pNiceMatrix}$
\end{BVerbatim}
$\begin{pNiceMatrix}%
  [
   margin,
   create-medium-nodes,
   code-after = 
    { \tikz \node [highlight = (1-1-block-medium)] {} ; } 
  ]
\Block{3-3}<\Large>{A} & & & 0 \\
& \hspace*{1cm} & & \Vdots \\
& & & 0 \\
0 & \Cdots& 0 & 0
\end{pNiceMatrix}$





\vspace{1cm}
On considère maintenant la matrice suivante que l'on a appelée |exemple|.

\medskip
\begin{Verbatim}
$\begin{pNiceArray}{CCC}[~emphase#name=exemple@,last-col,create-medium-nodes]
a & a + b & a + b + c & L_1 \\
a & a     & a + b     & L_2 \\
a & a     & a         & L_3
\end{pNiceArray}$
\end{Verbatim}
                          
\[\begin{pNiceArray}{CCC}[last-col]
a & a + b & a + b + c & L_1 \\
a & a     & a + b     & L_2 \\
a & a     & a         & L_3
\end{pNiceArray}\]

\bigskip
Si on veut surligner chaque rangée de la matrice, on peut utiliser la technique précédente trois fois.

\begin{Verbatim}
\tikzset{mes-options/.style={remember picture, 
                             overlay,
                             name prefix = exemple-,
                             highlight/.style = {fill = red!15,
                                                 blend mode = multiply,
                                                 inner sep = 0pt,
                                                 fit = ~#1}}}
\end{Verbatim}


\tikzset{mes-options/.style={remember picture, 
                             overlay,
                             name prefix = exemple-,
                             highlight/.style = {fill = red!15,
                                                 blend mode = multiply,
                                                 inner sep = 0pt,
                                                 fit = #1}}}


\begin{Verbatim}
\begin{tikzpicture}[mes-options]
\node [highlight = (1-1) (1-3)] {} ;
\node [highlight = (2-1) (2-3)] {} ;
\node [highlight = (3-1) (3-3)] {} ;
\end{tikzpicture}
\end{Verbatim}

\medskip
On obtient la matrice suivante.

\[\begin{pNiceArray}{CCC}[
     last-col,
     code-after = {\begin{tikzpicture}[highlight/.style = {fill = red!15,
                                                           blend mode = multiply,
                                                           inner sep = 0pt,
                                                           fit = #1}]
                   \node [highlight = (1-1) (1-3)] {} ;
                   \node [highlight = (2-1) (2-3)] {} ;
                   \node [highlight = (3-1) (3-3)] {} ;
                   \end{tikzpicture}}]
a & a + b & a + b + c & L_1 \\
a & a     & a + b     & L_2 \\
a & a     & a         & L_3
\end{pNiceArray}\]

\medskip
Le résultat peut paraître décevant. On peut l'améliorer en utilisant les «nœuds moyens» au lieu des «nœuds normaux».

\begin{Verbatim}
\begin{tikzpicture}[mes-options, ~emphase#name suffix = -medium@]
\node [highlight = (1-1) (1-3)] {} ;
\node [highlight = (2-1) (2-3)] {} ;
\node [highlight = (3-1) (3-3)] {} ;
\end{tikzpicture}
\end{Verbatim}

\medskip
On obtient la matrice suivante.

\[\begin{pNiceArray}{CCC}[
     last-col,
     create-medium-nodes,
     code-after = {\begin{tikzpicture}[highlight/.style = {fill = red!15,
                                                           blend mode = multiply,
                                                           inner sep = 0pt,
                                                           fit = #1},
                                       name suffix = -medium]
                   \node [highlight = (1-1) (1-3)] {} ;
                   \node [highlight = (2-1) (2-3)] {} ;
                   \node [highlight = (3-1) (3-3)] {} ;
                   \end{tikzpicture}}]
a & a + b & a + b + c & L_1 \\
a & a     & a + b     & L_2 \\
a & a     & a         & L_3
\end{pNiceArray}\]

                          
\vspace{1cm}

Dans l'exemple suivant, on utilise les «nœuds larges» pour surligner une zone de la matrice.\par\nobreak
\begin{Verbatim}
\left(\,\begin{NiceArray}{>{\strut}CCCC}%
   [create-large-nodes,left-margin,right-margin,
    code-after = {\tikz \path [~emphase#name suffix = -large@,
                               fill = red!15, 
                               blend mode = multiply]
                        (1-1.north west)
                     |- (2-2.north west)
                     |- (3-3.north west)
                     |- (4-4.north west)
                     |- (4-4.south east)
                     |- (1-1.north west) ; } ]
A_{11} & A_{12} & A_{13} & A_{14} \\
A_{21} & A_{22} & A_{23} & A_{24} \\
A_{31} & A_{32} & A_{33} & A_{34} \\
A_{41} & A_{42} & A_{43} & A_{44}  
\end{NiceArray}\,\right)
\end{Verbatim}
                             
\[\left(\,\begin{NiceArray}{>{\strut}CCCC}[
    create-large-nodes,left-margin,right-margin,
    code-after = {\tikz \path [name suffix = -large,
                               fill = red!15, 
                               blend mode = multiply]
                        (1-1.north west)
                     |- (2-2.north west)
                     |- (3-3.north west)
                     |- (4-4.north west)
                     |- (4-4.south east)
                     |- (1-1.north west) ; } ]
A_{11} & A_{12} & A_{13} & A_{14} \\
A_{21} & A_{22} & A_{23} & A_{24} \\
A_{31} & A_{32} & A_{33} & A_{34} \\
A_{41} & A_{42} & A_{43} & A_{44}  
\end{NiceArray}\,\right)\]


\subsection{Utilisation directe des nœuds Tikz}

Dans l'exemple suivant, on souhaite illustrer le produit mathématique de deux matrices.

\medskip
L'utilisation de |{NiceMatrixBlock}| avec l'option |auto-columns-width| va permettre que toutes les colonnes aient 
la même largeur ce qui permettra un alignement des deux matrices superposées.
\begin{Verbatim}
\begin{NiceMatrixBlock}[auto-columns-width]
\end{Verbatim}

\begin{Verbatim}
\NiceMatrixOptions{nullify-dots}
\end{Verbatim}

Les trois matrices vont être disposées les unes par rapport aux autres grâce à un tableau de LaTeX.
\begin{Verbatim}
$\begin{array}{cc}
& 
\end{Verbatim}

La matrice $B$ a une «première rangée» (pour $C_j$) d'où l'option |first-row|.
\begin{Verbatim}
\begin{bNiceArray}{C>{\strut}CCCC}[name=B,first-row]
       &          & ~emphase#C_j@                      \\
b_{11} & \Cdots   & b_{1j} & \Cdots & b_{1n} \\
\Vdots &          & \Vdots &        & \Vdots \\
       &          & b_{kj}                   \\
       &          & \Vdots                   \\
 b_{n1}  & \Cdots & b_{nj} & \Cdots & b_{nn} 
\end{bNiceArray} \\ \\
\end{Verbatim}

La matrice $A$ a une «première colonne» (pour $L_i$) d'où l'option |first-col|.
\begin{Verbatim}
\begin{bNiceArray}{CC>{\strut}CCC}[name=A,first-col]
    & a_{11} & \Cdots &        &        & a_{1n} \\
    & \Vdots &        &        &        & \Vdots \\
~emphase#L_i@ & a_{i1} & \Cdots & a_{ik} & \Cdots & a_{in} \\
    & \Vdots &        &        &        & \Vdots \\
    & a_{n1} & \Cdots &        &        & a_{nn} \\
\end{bNiceArray}
& 
\end{Verbatim}

Dans la matrice produit, on remarquera que les lignes en pointillés sont «semi-ouvertes».
\begin{Verbatim}
\begin{bNiceArray}{CC>{\strut}CCC}
       & &        & & \\
       & & \Vdots     \\
\Cdots & & c_{ij}     \\
\\
\\
\end{bNiceArray} 
\end{array}$

\end{NiceMatrixBlock}
\end{Verbatim}

\begin{Verbatim}                             
\begin{tikzpicture}[remember picture, overlay]
 \node [highlight = (A-3-1) (A-3-5) ] {} ; 
 \node [highlight = (B-1-3) (B-5-3) ] {} ; 
 \draw [color = gray] (A-3-3) to [bend left] (B-3-3) ; 
\end{tikzpicture}
\end{Verbatim}


\begin{NiceMatrixBlock}[auto-columns-width]
\NiceMatrixOptions{nullify-dots}
$\begin{array}{cc}
& 
\begin{bNiceArray}{C>{\strut}CCCC}[name=B,first-row]
      &        & C_j \\
b_{11} & \Cdots & b_{1j} & \Cdots & b_{1n} \\
\Vdots &       & \Vdots &       & \Vdots \\
       &       & b_{kj}  \\
       &       & \Vdots \\
 b_{n1}  & \Cdots & b_{nj} & \Cdots & b_{nn} 
\end{bNiceArray} \\ \\
\begin{bNiceArray}{CC>{\strut}CCC}[name=A,first-col]
    & a_{11} & \Cdots &  &  & a_{1n} \\
    & \Vdots &       &  &  & \Vdots \\
L_i & a_{i1} & \Cdots & a_{ik} & \Cdots & a_{in} \\
    & \Vdots &       &  &  & \Vdots \\
    & a_{n1} & \Cdots &  &  & a_{nn} \\
\end{bNiceArray}
& 
\begin{bNiceArray}{CC>{\strut}CCC}
       &  &        & & \\
       &  & \Vdots     \\
\Cdots &  & c_{ij}      \\
\\
\\
\end{bNiceArray} 
\end{array}$

\end{NiceMatrixBlock}

\begin{tikzpicture}[remember picture, overlay]
 \node [highlight = (A-3-1) (A-3-5) ] {} ; 
 \node [highlight = (B-1-3) (B-5-3) ] {} ; 
 \draw [color = gray] (A-3-3) to [bend left] (B-3-3) ; 
\end{tikzpicture}

\section*{Autre documentation}

Le document |nicematrix.pdf| (fourni avec l'extension \pkg{nicematrix}) contient une traduction anglaise de la
documentation ici présente, ainsi que le code source commenté et un historique des versions.


\end{document}


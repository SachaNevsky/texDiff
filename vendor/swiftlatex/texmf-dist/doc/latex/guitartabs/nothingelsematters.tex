%NAME: nothingelsematters.tex
%
% Andrey Babushkin, 2018/05/01
% babusand@fit.cvut.cz
%
% An example of guitartabs class usage.
% 
% Released under the The LaTeX Project Public License 1.3c.

\documentclass{guitartabs}

\artistname{Metallica}
\albumtitle{Metallica}
\songname{Nothing Else Matters}

\begin{document}
	\maketabheader
	
	Refrain:
	
	\begin{tabline}{3}{6}{8}{E,A,D,G,B,E}
		% 1 bar
		\notel{1}{2}{1}{0}{4.5}
		\note{1}{2}{2}{1}
		\note{1}{2}{3}{0}
		\note{1}{2}{4}{2}
		\note{1}{2}{5}{3}
		
		\notel{2}{2}{2}{2}{4.5}
		\note{2}{2}{3}{2}
		\note{2}{2}{4}{2}
		\note{2}{2}{5}{0}
		
		\nextbar
		
		% 2 bar
		\notel{1}{12}{1}{2}{8.5}
		\note{1}{12}{2}{3}
		\note{1}{12}{3}{2}
		\note{1}{12}{4}{0}
		
		\notel{4}{12}{1}{2}{16}
		\note{4}{12}{2}{3}
		\note{4}{12}{3}{2}
		\note{4}{12}{4}{0}
		
		\notel{5}{12}{1}{2}{16}
		\note{5}{12}{2}{3}
		\note{5}{12}{3}{2}
		\note{5}{12}{4}{0}
		
		\notel{6}{12}{1}{2}{16}
		\note{6}{12}{2}{3}
		\note{6}{12}{3}{2}
		\note{6}{12}{4}{0}
		
		\notel{7}{12}{1}{3}{16}
		\note{7}{12}{2}{3}
		\note{7}{12}{3}{2}
		\note{7}{12}{4}{0}
		
		\notel{8}{12}{1}{2}{8}
		\note{8}{12}{2}{3}
		\note{8}{12}{3}{2}
		\note{8}{12}{4}{0}
		
		\notel{10}{12}{1}{0}{8}
		\note{10}{12}{2}{3}
		\note{10}{12}{3}{2}
		\note{10}{12}{4}{0}
		
		\notel{12}{12}{1}{2}{16}
		\note{12}{12}{2}{3}
		\note{12}{12}{3}{2}
		\note{12}{12}{4}{0}
		
		\nextbar
		
		% 3 bar
		\notel{1}{2}{1}{0}{4.5}
		\note{1}{2}{2}{1}
		\note{1}{2}{3}{0}
		\note{1}{2}{4}{2}
		\note{1}{2}{5}{3}
		
		\notel{2}{2}{2}{2}{4.5}
		\note{2}{2}{3}{2}
		\note{2}{2}{4}{2}
		\note{2}{2}{5}{0}
	\end{tabline}

	\begin{tabline}{3}{6}{8}{E,A,D,G,B,E}
		% 1 bar
		\notel{1}{12}{1}{2}{8.5}
		\note{1}{12}{2}{3}
		\note{1}{12}{3}{2}
		\note{1}{12}{4}{0}
		
		\notel{4}{12}{1}{2}{16}
		\note{4}{12}{2}{3}
		\note{4}{12}{3}{2}
		\note{4}{12}{4}{0}
		
		\notel{5}{12}{1}{2}{16}
		\note{5}{12}{2}{3}
		\note{5}{12}{3}{2}
		\note{5}{12}{4}{0}
		
		\notel{6}{12}{1}{2}{16}
		\note{6}{12}{2}{3}
		\note{6}{12}{3}{2}
		\note{6}{12}{4}{0}
		
		\notel{7}{12}{1}{3}{16}
		\note{7}{12}{2}{3}
		\note{7}{12}{3}{2}
		\note{7}{12}{4}{0}
		
		\notel{8}{12}{1}{2}{8}
		\note{8}{12}{2}{3}
		\note{8}{12}{3}{2}
		\note{8}{12}{4}{0}
		
		\notel{10}{12}{1}{0}{8}
		\note{10}{12}{2}{3}
		\note{10}{12}{3}{2}
		\note{10}{12}{4}{0}
		
		\notel{12}{12}{1}{2}{16}
		\note{12}{12}{2}{3}
		\note{12}{12}{3}{2}
		\note{12}{12}{4}{0}
		
		\nextbar
		
		% 2 bar
		\notel{1}{2}{1}{0}{4.5}
		\note{1}{2}{2}{1}
		\note{1}{2}{3}{0}
		\note{1}{2}{4}{2}
		\note{1}{2}{5}{3}
		
		\notel{2}{2}{2}{2}{4.5}
		\note{2}{2}{3}{2}
		\note{2}{2}{4}{2}
		\note{2}{2}{5}{0}
		
		\nextbar
		
		% 3 bar
		\notel{1}{6}{1}{2}{8}
		\note{1}{6}{2}{3}
		\note{1}{6}{3}{2}
		\note{1}{6}{4}{0}
		
		\notel{3}{6}{1}{2}{16}
		\note{3}{6}{2}{3}
		\note{3}{6}{3}{2}
		\note{3}{6}{4}{0}
		
		\notel{4}{6}{1}{2}{16}
		\note{4}{6}{2}{3}
		\note{4}{6}{3}{2}
		\note{4}{6}{4}{0}
		
		\notel{5}{6}{1}{2}{8}
		\note{5}{6}{2}{3}
		\note{5}{6}{3}{2}
		\note{5}{6}{4}{0}
	\end{tabline}

	\begin{tabline}{3}{6}{8}{E,A,D,G,B,E}
		% 1 bar
		\notel{1}{6}{6}{0}{8}
		\notel{2}{6}{3}{0}{8}
		\notel{3}{6}{2}{0}{8}
		\notel{4}{6}{1}{0}{8}
		\notel{5}{6}{2}{0}{8}
		\notel{6}{6}{3}{0}{8}
		
		\nextbar
		
		% 2 bar
		\notel{1}{12}{6}{0}{8}
		\notel{3}{12}{3}{0}{8}
		\notel{5}{12}{2}{0}{8}
		\notel{7}{12}{1}{0}{8}
		\notel{9}{12}{2}{0}{8}
		\notel{11}{12}{6}{3}{16}
		\notel{12}{12}{6}{2}{16}
		
		\nextbar
		
		% 3 bar
		\notel{1}{12}{6}{0}{8}
		\notel{3}{12}{3}{0}{8}
		\notel{5}{12}{2}{0}{8}
		\notel{7}{12}{1}{3}{8}
		\notel{9}{12}{3}{0}{16}
		\notel{10}{12}{1}{0}{16}
		\notel{11}{12}{2}{0}{16}
		\notel{12}{12}{3}{0}{16}
	\end{tabline}

	\vspace{2cm}
	
	Here I demonstrate the functionality of drawing pauses:

	\begin{tabline}{1}{}{}{E,A,D,G,B,E}
		% Pauses
		\restwhole{1}{5}
		\resthalf{2}{5}
		\restquarter{3}{5}
		\resteighth{4}{5}
		\restsixteenth{5}{5}
		
		
	\end{tabline}

	

\end{document}
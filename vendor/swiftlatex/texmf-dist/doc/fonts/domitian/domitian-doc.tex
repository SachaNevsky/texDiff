\documentclass[12pt]{article}
\usepackage[utf8]{luainputenc}\usepackage[T1]{fontenc}
\usepackage{amsmath,amsthm,mathpazo,parskip,hologo,booktabs}
\usepackage[oldstyle]{domitian}
\usepackage{microtype}

\title{\textit{The} Domitian \textit{font family}\\\large \textsc{version 1.0c}}
\author{Daniel Benjamin Miller\thanks{\texttt{dbmiller@dbmiller.org}}}
\begin{document}
\maketitle
\section{Description}
The Domitian fonts are a free and open-source OpenType font family, based on the Palatino design by Hermann Zapf (1918-2015), as implemented in Palladio (also known as \textsc{p052}), the version distributed as part of \textsc{urw}'s free Core 35 PostScript fonts (version 2.0, 2015).

Domitian is meant as a drop-in replacement for Adobe's version of Palatino. It extends Palladio with small capitals, old-style figures and scientific inferiors. The metrics have been adjusted to more closely match Adobe Palatino, and hinting has been improved.

\section{Usage}
\subsection{Basic usage}
To use Domitian with \hologo{LaTeX}, include in the preamble:

\begin{verbatim}
\usepackage[T1]{fontenc} %of OT1, T1, LGR, T2A/B/C
\usepackage{domitian}
\end{verbatim}

\subsection{Package Options}
\begin{tabular}{@{} ll @{}}
\toprule
Package Option & Feature \\
\midrule
\texttt{lining} & Use lining figures (default) \lining{1234567890}\\
\texttt{oldstyle} & Use old-style figures \oldstyle{1234567890}\\
\bottomrule
\end{tabular}
\subsection{Text Commands}
\begin{tabular}{@{} ll @{}}
\toprule
Text Command & Feature \\
\midrule
\texttt{\textbackslash lining\{\}} & Use lining figures \lining{1234567890}\\
\texttt{\textbackslash oldstyle\{\}} & Use old-style figures \oldstyle{1234567890}\\
\texttt{\textbackslash textsc\{\}} & Use small caps (regular font only) \textsc{abcde\$\&}\\
\texttt{\textbackslash textsu\{\}} & Use superior figures \textsu{1234567890}\\
\texttt{\textbackslash textin\{\}} & Use inferior figures \textin{1234567890}\\
\bottomrule
\end{tabular}

\subsection{Using superior figures for footnotes}
To use superior figures for footnotes, include the following line in your preamble:

\begin{verbatim}
\usepackage[supstfm=Domitian-Roman-sup-t1]{superiors}
\end{verbatim}

\subsection{OpenType with \texttt{fontspec}}

To use the OpenType version of Domitian in \hologo{LuaLaTeX} or \hologo{XeLaTeX}, include in the preamble:

\begin{verbatim}
\usepackage{fontspec}
\setmainfont{Domitian}
\end{verbatim}

\subsection{Basic samples}
The quick brown fox jumped over the lazy dog.\\
\textit{The quick brown fox jumped over the lazy dog.}\\
\textbf{The quick brown fox jumped over the lazy dog.}\\
\textbf{\textit{The quick brown fox jumped over the lazy dog.}}\\
\textsc{the quick brown fox jumped over the lazy dog.}\\
\lining{1234567890}\oldstyle{1234567890}\textsu{1234567890}\textin{1234567890}

\section{Mathematics support}

No math font is provided, but Domitian works well with \texttt{mathpazo}, like so:

\begin{verbatim}
\usepackage{mathpazo,domitian}
\end{verbatim}

Note that when Domitian is used with \texttt{mathpazo}, \texttt{mathpazo} must be loaded first, or else Domitian will not be used as the text font.

With \hologo{LuaLaTeX} or \hologo{XeTeX}, it is possible to replace the Pazo alphabets with Domitian's, although this will reduce the quality of spacing in mathematics. To do so, do the following, again being sure to load \texttt{mathpazo} first:

\begin{verbatim}
\usepackage{mathpazo}
\usepackage{mathfont} %Loads fontspec automatically
\mathfont[upper,lower,diacritics,greekupper,greeklower,
agreekupper,agreeklower,digits]{Domitian}
\setmainfont[Numbers=OldStyle]{Domitian}
\end{verbatim}


(This example will use Domitian for text and for math alphabets, and will use
the mathematical symbols from \texttt{mathpazo}; \texttt{newpxmath} also works well.)



For more usage details, see the \texttt{fontspec} and \texttt{mathfont} packages' documentation.

\section{Copyrights and licensing}
Copyright © 2014, 2015 by (\textsc{urw})++ Design \& Development.\\
Copyright © 2019--2020 by Daniel Benjamin Miller.

Domitian is made available under your choice of the \textsc{gnu} Affero \textsc{gpl} 3.0 (with a font
exception), the \hologo{LaTeX} Project Public License 1.3c or the \textsc{sil} Open Font License
1.1. For more details, see \texttt{COPYING}.

\end{document}
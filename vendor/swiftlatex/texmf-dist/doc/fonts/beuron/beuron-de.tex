% beuron package: German manual
% Version 1.3
% Datum: 17. Januar 2018
\documentclass[numbers=noenddot]{scrartcl}
\PassOptionsToPackage{hyphens}{url}
\usepackage{fontspec}
\usepackage[latin,czech,ngerman]{babel}
\usepackage[ngerman]{isodate}
\usepackage[style=authortitle,language=auto,autolang=hyphen]{biblatex}
\usepackage{csquotes}
\usepackage{cnltx-tools}
\usepackage{cnltx-example}
\usepackage{beuron}
\usepackage{scrlayer-scrpage}
\usepackage{graphicx}
\usepackage{metalogo}
\usepackage{mflogo}
\usepackage{booktabs}
\usepackage{tcolorbox}
\usepackage{multirow}
\usepackage[toc]{multitoc}
\usepackage[justification=centering]{caption}
\usepackage[colorlinks=true,
   allcolors=black,
   bookmarksnumbered=true,
   pdfencoding=auto,
   pdftitle={Die Schrift der Beuroner Kunst},
   pdfsubject={Anleitung zum Paket BEURON},
   pdfkeywords={latex font Beuron Beuroner Kunst},
   pdfauthor={K. Wehr}]{hyperref}

\addbibresource{Literatur.bib}

\renewcommand{\labelnamepunct}{\addcolon\space}
\renewcommand{\nametitledelim}{\addcolon\space}

% "Hrsg." in Klammern setzen, Komma davor entfernen
\DeclareFieldFormat{editortype}{\mkbibparens{#1}}
\renewbibmacro*{editor+others}{%
  \ifboolexpr{
    test \ifuseeditor
    and
    not test {\ifnameundef{editor}}
  }
    {\printnames{editor}%
    \space%
     \usebibmacro{editor+othersstrg}%
     \clearname{editor}}
    {}}

\setmainfont{GilliusADF}
\setsansfont{GilliusADF}

\setlogokern{Xe}{-0,03em}
\setlogokern{eL}{-0,01em}
\setlogokern{La}{-0,2em}
\setlogokern{aT}{-0,05em}
\setlogokern{Te}{-0,15em}
\setlogokern{eX}{-0,03em}

\definecolor{Beuronblau}{rgb}{0.15,0.2,0.35}
\definecolor{Beurongelb}{rgb}{0.95,0.65,0.15}

\tcbset{colframe=Beurongelb,colback=Beuronblau,fontupper=\color{Beurongelb}}

\ExplSyntaxOn

\NewDocumentCommand \AutorVorname {s}
   {
      \IfBooleanTF{#1}{k}{K}
      eno
   }

\NewDocumentCommand \AutorNachname {s}
   {
      \IfBooleanTF{#1}{w}{W}
      ehr
   }

\NewDocumentCommand \AutorEMailDomain {}
   {
      abgol
   }

\NewDocumentCommand \tschechB {m}
   {
      \group_begin:
      \fontspec{Latin Modern Sans}
      #1
      \group_end:
   }

\NewDocumentEnvironment {Befehlsliste} { }
   {
      \begin{list}{ }
         {
            \setlength{\leftmargin}{0pt}
            \setlength{\itemindent}{-1em}
            \setlength{\parsep}{0pt}
            \setlength{\listparindent}{\parindent}
            \setlength{\itemsep}{\topsep}
         }
   }
   {
      \end{list}
   }

\NewDocumentCommand \Befehlsbeschreibung {m o}
   {
      \item
      \cs{#1}
      \IfValueT{#2}{#2}
      \\
   }

\NewDocumentEnvironment {Symboltabelle} { }
   {
      \begin{tabular}{rl@{\hspace{6mm}}l}
   }
   {
      \end{tabular}
   }

\NewDocumentCommand \Symboltabellenzeile {mo}
   {
      \small\texttt{#1} & \texttt{\symbol{#1}}
      & \large
      \IfValueTF{#2}
         {
            \textbeuron{\symbol{#2}}
         }
         {
            \textbeuron{\symbol{#1}}
         }
   }

\NewDocumentCommand \Symboltabellenbereich {mm}
   {
      \int_step_inline:nnnn {#1} {1} {#2}
         {
            \Symboltabellenzeile {##1}
            \int_compare:nNnF {##1} = {#2} {\\}
         }
   }

\NewDocumentCommand \Kunstort {m}
   {
      \smallskip\par\noindent\textbf{#1}
   }

\NewDocumentCommand \tschechKunstort {mmmm}
   {
      \Kunstort{#1/#2~(\emph{#3})}
   }

\NewDocumentCommand \tschechKunststaette {mmm}
   {
      #1~(\foreignlanguage{czech}{\emph{#3}})
   }

\NewDocumentEnvironment {Kunststaettenliste} { }
   {
      \begin{list}{ }
         {
            \setlength{\topsep}{1ex plus0.4ex minus0.2ex}
            \setlength{\itemsep}{\topsep}
            \setlength{\parsep}{0pt}
            \setlength{\leftmargin}{1,5em}
         }
   }
   {
      \end{list}
   }

\NewDocumentEnvironment {Kunstliteraturliste} { }
   {
      \begin{list}{ }
         {
            \setlength{\leftmargin}{1,5em}
            \small
         }
   }
   {
      \end{list}
   }

\ExplSyntaxOff

\setlength{\columnsep}{2em}
\setlength{\columnseprule}{0,4pt}
\setlength{\floatsep}{2\bigskipamount}
\setlength{\bibitemsep}{\smallskipamount}

\newlength{\Boxbreite}
\setlength{\Boxbreite}{\paperwidth}
\addtolength{\Boxbreite}{-2\fboxsep}

\DeclareNewLayer[background,topmargin,voffset={5mm},contents={\colorbox{Beuronblau}{\parbox[c][6mm][c]{\Boxbreite}{\centering\color{Beurongelb}\Large
\ifcase\thepage
\or\textbeuronc{+Beata es Virgo Maria Dei genetrix quae credidisti Domino+}
\or\textbeuronx{+Perfecta sunt in te quae dicta sunt tibi+}
\or\textbeuronx{+Ecce exaltata es super choros angelorum+}
\or\textbeuron{+Intercede pro nobis ad Dominum Deum nostrum+}
\or\textbeuronx{+Ecce virgo concipiet et pariet filium+}
\or\textbeuronx{+Beata progenies unde Christus natus est+}
\or\textbeuron{+O quam gloriosa est Virgo quae coeli regem genuit+}
\or\textbeuron{+Congratulamini mihi omnes qui diligitis Dominum+}
\or\textbeuronx{+Stemus iuxta crucem cum Maria matre Iesu+}
\or\textbeuronx{+Gaude et laetare virgo Maria alleluia+}
\fi
}}}]{Oberband}

\DeclareNewLayer[background,bottommargin,mode=picture,addheight={-1cm},contents={\includegraphics{Band}}]{Unterband}

\AddLayersToPageStyle{plain}{Oberband,Unterband}

\pagestyle{plain}

\newcommand{\PL}{Peter Lenz (1832--1928)}
\newcommand{\JW}{Jakob Wüger (1829--1892)}

\newcommand{\beur}{\textcolor{Beuronblau}{\texttt{beuron}}}

\begin{document}

\begin{center}
\Huge
\textcolor{Beuronblau}{\textbeuronx{BEURON}}

\smallskip
\LARGE
Die Schrift der Beuroner Kunst

\medskip
\Large
Version 1.3

\medskip
\normalsize
\today

\vspace{2\bigskipamount}
\large
\textit{Paketautor}

\medskip
\AutorVorname\ \AutorNachname

\normalsize
\smallskip
\texttt{\AutorVorname*.\AutorNachname*@\AutorEMailDomain.de}
\end{center}

\medskip
\begin{abstract}
\noindent Das Paket \beur\ macht die in den Kunstwerken der Beuroner Schule verwendete Schrift zur Benutzung mit \TeX\ und \LaTeX\ verfügbar. Es handelt sich um eine nur aus Großbuchstaben bestehende Monumentalschrift.
\end{abstract}

\medskip
\tableofcontents

\medskip
\section{Die Beuroner Kunst}
\subsection{Kunsthistorische Einordnung}
Die Beuroner Kunst war eine Erneuerungsbewegung der christlichen Kunst, die von den beiden vom Studium in München her befreundeten Künstlern \PL\ und \JW\ während ihres Romaufenthaltes in den 1860er Jahren begründet wurde.

Sie entwickelte sich einerseits im Anschluss an die Kunst der sogenannten Nazarener, wandte sich aber andererseits vom Naturalismus der Romantik ab und strebte eine stärker geometrisch stilisierte Darstellung christlicher Bildthemen an. Entscheidende Impulse gab dabei die Auseinandersetzung mit der altägyptischen Kunst, was sich vor allem durch einen weitgehenden Verzicht auf Tiefenräumlichkeit in der bildlichen Darstellung bemerkbar macht.

Lenz und Wüger traten 1872 bzw. 1870 in die Benediktinerabtei Beuron (bei Sigmaringen) ein, wo sie als Pater Desiderius und Pater Gabriel wirkten. Die Beuroner Kunst wurde in den folgenden Jahrzehnten bis in die 1930er Jahre im Wesentlichen von ihrem Schülerkreis aus Beuroner Mönchen getragen.

Die Beuroner Künstler wurden nicht nur mit der Ausstattung des erst 1863 wiedererrichteten Klosters Beuron, sondern auch etlicher weiterer Kirchen und Klöster in mehreren Ländern Europas beauftragt. Ihren Höhepunkt erreichte die Beuroner Kunstschule um das Jahr 1900, als sie durch mehrere Ausstellungsbeteiligungen auch über das benediktinisch-kirchliche Milieu hinaus Aufmerksamkeit in der Kunstwelt erlangte.

Durch den Zweiten Weltkrieg sowie durch »Kirchenumgestaltungen« der Folgezeit wurden zahlreiche Werke der Beuroner Kunst ganz oder teilweise zerstört. Erhaltene Werke sind heute u.\,a. in Beuron (Mauruskapelle und Erzabtei St. Martin), Rüdesheim am Rhein (Abtei St. Hildegard), Prag (Kirchen der ehemaligen Abteien Emaus und St. Gabriel), aber auch in Amerika in Conception/Missouri (Basilika der Unbefleckten Empfängnis) zu sehen (siehe auch Anhang
\ref{Staettenliste}).

\subsection{Die Schrift}
Die von den Künstlern der Beuroner Schule ausgeführten Wandmalereien wurden mit monumentalen Inschriften versehen, die der Heiligen Schrift oder der Gebetstradition der Kirche entnommen sind und den lehrhaften Charakter der Bildwerke unterstützen. Für diese Inschriften wurde eine Schrift mit einigen markanten Merkmalen verwendet, die mit nur kleinen Variationen in den meisten Wandbemalungen der Beuroner Kunst, daneben auch auf kunstgewerblichen Arbeiten der Schule anzutreffen ist.

Leider schweigt sich die mit der Beuroner Schule befasste kunsthistorische Literatur über diese Schrift weitgehend aus, obwohl sie offenkundig einen integralen Bestandteil der Kunstrichtung darstellt. In einem Führer zur Unterkapelle der Beuroner Gnadenkapelle wird lediglich auf die Verwendung von „Beuroner Versalien“ hingewiesen\footcite[\ppno\ 19 u. 23]{Groeger}, ein Führer zur Gnadenkapelle selbst erwähnt „ein eigenes, aus der Antiqua entwickeltes Alphabet“ und nennt einige Charakteristika.\footcite[21]{KrinsB}
Über den Ursprung der Schrift kann daher nur gemutmaßt werden. Naheliegend erscheint eine Beeinflussung durch die Inschriften frühchristlicher Basiliken in Italien. Das erste dem Paketautor bekannte Vorkommen der Schrift stellt die Kreuzesinschrift in Wügers Kreuzigungsgemälde von 1868 dar.\footnote{\url{https://commons.wikimedia.org/wiki/File:W\%C3\%BCger_Kreuzigung.jpg}}

Das Paket \beur\ macht die Beuroner Schrift mit Hilfe von \MF\ zur Benutzung mit \TeX\ zugänglich.

\begin{center}
\Large\textbeuron{ABCDEFGHIJKLMNOPQRSTUVWXYZ}
\end{center}

Die Schrift zeigt folgende Besonderheiten, von denen einige auf antike Vorbilder zurückzuführen sind:
\begin{itemize}
\item Es werden nur Großbuchstaben verwendet.
\item Das E und das G weisen eine außergewöhnliche gerundete Form auf.
\item Das A hat in vielen Fällen (so auch in der mit diesem Paket verfügbar gemachten Form) einen gewinkelten Querbalken.
\item Bei den Buchstaben B und R bleibt in der Mitte eine Lücke zwischen Bogen und Stamm.
\item Zwischen U und V wird nicht unterschieden.
\item Statt J wird in der Regel I verwendet. Allerdings taucht gelegentlich ein J am Wortanfang auf.
\item Es gibt keinerlei Satzzeichen. In einigen Fällen wird ein Doppelpunkt zur Trennung von Satzteilen verwendet.
\item Die Wörter werden nicht durch Leerschritte, sondern durch Mittepunkte voneinander getrennt.
\item Die Verse werden oftmals durch zwei griechische Kreuze eingefasst.
\item Die Schrift wird überwiegend für lateinische, hier und da aber auch für deutsche Texte verwendet.
\end{itemize}

\section{Schriften}
\subsection{Zeichenvorrat}
Die Schriften des Pakets \beur\ stellen gemäß dem historischen Vorbild nur die sechsundzwanzig Großbuchstaben des lateinischen Alphabets zur Verfügung; dabei unterscheiden sich U und V graphisch nicht. Um die Verwendung zu vereinfachen, stellen die Schriften auch Kleinbuchstaben zur Verfügung; diese haben allerdings die gleichen Glyphen wie die Großbuchstaben. Hinzu kommt ein großes griechisches Omega. Umlaute und Akzentbuchstaben sowie Satzzeichen sind nicht vorhanden.

\begin{table}
\setlength{\columnseprule}{0pt}
\begin{multicols}{4}
\begin{Symboltabelle}
\Symboltabellenzeile{43} \\
\Symboltabellenzeile{45} \\
\Symboltabellenzeile{47} \\
\Symboltabellenzeile{58} \\
\Symboltabellenbereich{65}{75}
\end{Symboltabelle}

\begin{Symboltabelle}
\Symboltabellenbereich{76}{90}
\end{Symboltabelle}

\begin{Symboltabelle}
\Symboltabellenbereich{97}{111}
\end{Symboltabelle}

\begin{Symboltabelle}
\Symboltabellenbereich{112}{122} \\
\Symboltabellenzeile{141}[937]
\end{Symboltabelle}
\end{multicols}
\caption{Zeichenvorrat der Beuroner Schriften. In der Opentype-Version hat das Omega abweichend den Code \texttt{937}
(Unicode: \emph{Greek capital letter Omega}).}
\label{Zeichenvorrat}
\end{table}

Als Sonderzeichen stehen der Mittepunkt, der bei Bedarf als Schrägstrich eingegeben werden kann, der Doppelpunkt und das griechische Kreuz, das als Pluszeichen einzugeben ist, zur Verfügung. Daneben gibt es noch einen Bindestrich, wodurch prinzipiell auch eine Silbentrennung möglich ist. Die Tabelle \ref{Zeichenvorrat} zeigt alle vorhandenen Zeichen und ihre Codierung.

\subsection{Schriftschnitte}

Die Beuroner Schrift wird in drei Schnitten angeboten, die sich nur in der Breite der Buchstaben unterscheiden (vgl. Abb. \ref{Breiten}). Dies orientiert sich am historischen Gebrauch der Schrift, die je nach Bedarf von den Künstlern unterschiedlich breit gestaltet wurde.

\begin{figure}
\begin{tcolorbox}[left=2mm,right=2mm]
\centering
\textbeuronc{abcdefghijklmnopqrstuvwxyz\beuronOmega}

\smallskip
\textbeuronc{Laetifica nos pro diebus quibus nos humiliasti}

\medskip
\textbeuron{abcdefghijklmnopqrstuvwxyz\beuronOmega}

\smallskip
\textbeuron{Laetifica nos pro diebus quibus nos humiliasti}

\medskip
\textbeuronx{abcdefghijklmnopqrstuvwxyz\beuronOmega}

\smallskip
\textbeuronx{Laetifica nos pro diebus quibus nos humiliasti}
\end{tcolorbox}
\caption{Die Beuroner Schrift in der schmalen, normalen und breiten Ausführung \\ \emph{Beispielvers:} Ps~90 (89),\,15}
\label{Breiten}
\end{figure}

Die Zeichen der drei Schriften sowie die Unterschneidungspaare (Buchstabenpaare mit verändertem Abstand) wurden durch \MF-Dateien definiert, mit Hilfe des Programms \emph{Mftrace}\footnote{\url{https://ctan.org/pkg/mftrace}} vektorisiert und mit Hilfe des Programms \emph{Fontforge}\footnote{\url{http://fontforge.github.io/en-US/}} konvertiert. Sie stehen in den Formaten \emph{Type~1} und \emph{Opentype} zur Verfügung.

Die \TeX-Schriftnamen und die \LaTeX-Schriftattribute sind der Tabelle \ref{Namen} zu entnehmen. Hieraus ist ersichtlich, dass beispielsweise die breite Variante der Beuroner Schrift in \LaTeX\ mit der Befehlsfolge
\verbcode:\fontfamily{beuron}\fontseries{x}\selectfont: ausgewählt werden kann.

\begin{table}
\centering
\begin{tabular}{lccccc}
\toprule
& \multirow{2}*{\TeX-Schriftname} & \multicolumn{4}{c}{\LaTeX-Schriftattribute} \\
& & Codierung & Familie & Serie & Form \\
\midrule
schmal (condensed) & \texttt{beuronc} & \texttt{T1} & \texttt{beuron} & \texttt{c} & \texttt{n} \\
normal (medium) & \texttt{beuron} & \texttt{T1} & \texttt{beuron} & \texttt{m} & \texttt{n} \\
breit (extended) & \texttt{beuronx} & \texttt{T1} & \texttt{beuron} & \texttt{x} & \texttt{n} \\
\bottomrule
\end{tabular}
\caption{Schriftnamen und Schriftattribute}
\label{Namen}
\end{table}

Bei Verwendung der modernen \LaTeX-Varianten \XeLaTeX\ und \LuaLaTeX\ kann mit Hilfe des Pakets \texttt{fontspec} auf die Opentype-Version der Schriften zurückgegriffen werden. Der Tabelle~\ref{Opentype}
entnimmt man beispielsweise, dass bei Verwendung von \LuaLaTeX\ die breite Variante der Beuroner Schrift mit dem Befehl
\verbcode:\fontspec{Beuron Extended}:
gewählt werden kann.

\begin{table}
\centering
\begin{tabular}{lcc}
\toprule
& Opentype-Schriftfamilie & Opentype-Schriftdatei \\
\midrule
schmal (condensed) & \texttt{Beuron Condensed} & \texttt{BeuronCondensed-Regular.otf} \\
normal (medium) & \texttt{Beuron} & \texttt{Beuron-Regular.otf} \\
breit (extended) & \texttt{Beuron Extended} & \texttt{BeuronExtended-Regular.otf} \\
\bottomrule
\end{tabular}
\caption{Opentype-Schriftfamilien und -Dateien}
\label{Opentype}
\end{table}

Schnitte in unterschiedlichen Schriftgrößen existieren nicht; die Schrift wird auf die jeweils ausgewählte Schriftgröße skaliert. Wie in der Typographie üblich ist die tatsächliche Höhe der Buchstaben kleiner als die nominelle Schriftgröße (Kegelhöhe). Bei den Beuroner Schriften beträgt die Buchstabenhöhe zwei Drittel der Kegelhöhe.

\subsection{Schriftauswahlbefehle}
Um die Verwendung der Beuroner Schriften zu erleichtern, stellt das Paket \beur\ passende Befehle zur Verwendung mit \LaTeX\ zur Verfügung. Die Voraussetzung zu ihrer Verwendung ist das Laden des Pakets mit
\verbcode:\usepackage{beuron}:
in der Präambel des Dokuments.

\begin{Befehlsliste}
\Befehlsbeschreibung{textbeuron}[\marg{text}]
Der Befehl setzt einen Text in normaler Beuroner Schrift. Dabei werden Leerzeichen durch Mittepunkte ersetzt. Außerdem werden Umlaute durch nachgestelltes E aufgelöst und das ß wird durch SS ersetzt. Das Textargument darf nicht mehrere Absätze umfassen. Beispiele zeigt die Abbildung~\ref{textbeuron}.

\begin{figure}
\begin{tcolorbox}
\selectlanguage{ngerman}
\noindent{\color{gray}\fontspec{Latin Modern Mono Prop}\textbackslash textbeuron\{Du legst mir größere Freude ins Herz als andere haben bei Korn und Wein in Fülle\}}

\smallskip
\noindent\textbeuron{Du legst mir größere Freude ins Herz als andere haben bei Korn und Wein in Fülle}

\selectlanguage{latin}
\medskip
\noindent{\color{gray}\fontspec{Latin Modern Mono Prop}\textbackslash textbeuron\{Pater noster qui es in caelis sanctificetur nomen tuum adveniat regnum tuum fiat voluntas tua sicut in caelo et in terra panem nostrum cotidianum da nobis hodie et dimitte nobis debita nostra sicut et nos dimittimus debitoribus nostris et ne nos inducas in tentationem sed libera nos a malo\}}

\smallskip
\noindent\textbeuron{Pater noster qui es in caelis sanctificetur nomen tuum adveniat regnum tuum fiat voluntas tua sicut in caelo et in terra panem nostrum cotidianum da nobis hodie et dimitte nobis debita nostra sicut et nos dimittimus debitoribus nostris et ne nos inducas in tentationem sed libera nos a malo}
\end{tcolorbox}
\caption{Beispiele für die Verwendung des Befehls \verbcode:textbeuron:.\\ Beispieltexte: Ps 4,\,8 und Mt 6,\,9--13}
\label{textbeuron}
\end{figure}

\Befehlsbeschreibung{textbeuronc}[\marg{text}]
Das Gleiche mit schmaler Beuroner Schrift.
\Befehlsbeschreibung{textbeuronx}[\marg{text}]
Das Gleiche mit breiter Beuroner Schrift.
\Befehlsbeschreibung{textbeuron*}[\marg{text}]
Der Befehl wirkt wie \cs{textbeuron}, wobei Leerzeichen jedoch erhalten bleiben.
\Befehlsbeschreibung{textbeuronc*}[\marg{text}]
Das Gleiche mit schmaler Beuroner Schrift.
\Befehlsbeschreibung{textbeuronx*}[\marg{text}]
Das Gleiche mit breiter Beuroner Schrift.
\Befehlsbeschreibung{beuronOmega}
Gibt den griechischen Großbuchstaben Omega aus.
\end{Befehlsliste}

\noindent \emph{Technischer Hinweis:} Bei Verwendung von \XeLaTeX\ oder \LuaLaTeX\ lädt das Paket \beur\ das Paket \texttt{fontspec}. Falls Sie \texttt{fontspec} mit Optionen laden wollen, muss dies vor dem Laden von \beur\ erfolgen.

\section{Anwendungsgebiete}
Neben der Verwendung für monumentale Inschriften nach historischem Vorbild bietet sich der Gebrauch der Beuroner Schrift für Überschriften und Zierinschriften in Gebet- und Gesangbüchern etc. an. Für längere Texte sowie für profane Zwecke ist die Schrift nicht geeignet.

\section{Lizenz}
Das Paket \beur\ unterliegt der
\emph{\LaTeX\ Project Public License},
Version 1.3 oder Nachfolgeversion.\footnote{\url{http://www.latex-project.org/lppl.txt}}
Die Opentype-Schriften können auch unabhängig vom Paket unter der
\emph{SIL Open Font License}, Version 1.1, verwendet werden.\footnote{\url{http://scripts.sil.org/OFL}}

\appendix
\nocite{*}
\printbibliography[heading=bibnumbered]

\section{Unvollständiges Verzeichnis der Stätten der Beuroner Kunst}\label{Staettenliste}
\Kunstort{Berlin}
\begin{Kunststaettenliste}
\item\emph{Stadtteil Wedding:} Kirche St. Joseph\footnote{In dieser Kirche wurde die Beuroner Schrift nicht verwandt.}
\begin{Kunstliteraturliste}
\item\url{http://www.berlin.de/landesdenkmalamt/denkmalpflege/erkennen-und-erhalten/sakralbauten/st-josephs-kirche-639576.php}
\end{Kunstliteraturliste}
\end{Kunststaettenliste}
\Kunstort{Beuron/Landkreis Sigmaringen}
\begin{Kunststaettenliste}
\item Erzabtei St. Martin (Hauptaltar der Abteikirche, Gnadenkapelle mit Unterkapelle)
\begin{Kunstliteraturliste}
\item\cite{KrinsB}
\item\cite{Groeger}
\item\url{http://www.erzabtei-beuron.de/kloster/kultur/kirche/index.html}
\item\url{http://www.erzabtei-beuron.de/kloster/kultur/gnadenkapelle/index.html}
\item\url{http://www.erzabtei-beuron.de/kloster/kultur/krypta/index.html}
\end{Kunstliteraturliste}
\item Mauruskapelle
\begin{Kunstliteraturliste}
\item\cite{KrinsB}
\item\url{http://www.erzabtei-beuron.de/kloster/kultur/mauruskapelle/index.html}
\end{Kunstliteraturliste}
\end{Kunststaettenliste}
\tschechKunstort{Budweis}{Südböhmen}{\tschechB{\v{C}}eské Bud\tschechB{\v{e}}jovice}{South Bohemia}
\begin{Kunststaettenliste}
\item \tschechKunststaette{Maria-Rosenkranz-Kirche}{Church of Our Lady of the Rosary}{Kostel Panny Marie R\tschechB{ů}žencové}
\begin{Kunstliteraturliste}
\item\url{http://www.petrini.cz/clanky/komunity/ceske_budejovice/kostel}
\item\url{http://www.bbkult.net/kulturdatenbank/adressen:sehenswuerdigkeit:gerettete-baudenkmaeler::5/13294761189848.html}
\end{Kunstliteraturliste}
\end{Kunststaettenliste}
\Kunstort{Collegeville/Minnesota}
\begin{Kunststaettenliste}
\item Große Halle der Universität St. Johannes, ehemals Kirche der Abtei St. Johannes
\begin{Kunstliteraturliste}
\item\url{http://www.newliturgicalmovement.org/2012/03/church-of-st-johns-abbey-collegeville.html}
\end{Kunstliteraturliste}
\end{Kunststaettenliste}
\Kunstort{Conception/Missouri}
\begin{Kunststaettenliste}
\item Basilika der Unbefleckten Empfängnis
\begin{Kunstliteraturliste}
\item\url{https://www.conceptionabbey.org/monastery/basilica/}
\item\url{https://www.conceptionabbey.org/monastery/beuronese-murals/}
\end{Kunstliteraturliste}
\end{Kunststaettenliste}
\Kunstort{Gaußig/Landkreis Bautzen}
\begin{Kunststaettenliste}
\item Schlosskapelle
\end{Kunststaettenliste}
\tschechKunstort{Königgrätz}{Nordostböhmen}{Hradec Králové}{North-East Bohemia}
\begin{Kunststaettenliste}
\item \tschechKunststaette{Städtische Musikhalle}{Municipal Music Hall}{M\tschechB{\v{e}}stská hudební sí\tschechB{\v{n}}}, \tschechKunststaette{ehemals Kirche St. Johannes Nepomuk}{formerly Church of St John of Nepomuk}{Kostel sv. Jana Nepomuckého}
\end{Kunststaettenliste}
\Kunstort{Konstanz}
\begin{Kunststaettenliste}
\item Konradskapelle im Münster
\end{Kunststaettenliste}
\Kunstort{Meßkirch/Landkreis Sigmaringen}
\begin{Kunststaettenliste}
\item Herz-Jesu-Heim, ehemals Herz-Jesu-Kirche
\end{Kunststaettenliste}
\Kunstort{Montecassino/Latium}
\begin{Kunststaettenliste}
\item Krypta der Abteikathedrale
\end{Kunststaettenliste}
\Kunstort{Prag}
\begin{Kunststaettenliste}
\item \tschechKunststaette{\emph{Neustadt:} Emauskloster}{\emph{New Town:} Emmaus monastery}{Emauzský klášter}
\begin{Kunstliteraturliste}
\item\cite{Emaus}
\end{Kunstliteraturliste}
\item \tschechKunststaette{\emph{Stadtteil Smíchov:} Kirche der ehem. Abtei St. Gabriel}{\emph{Smíchov district:} Church of the former abbey of St Gabriel}{Klá\v{s}ter sv. Gabriela}
\begin{Kunstliteraturliste}
\item\cite{Pieta}
\item\cite{Evangelistar}
\item\cite{Anschriften}
\item\url{http://malakim.cz/}
\item\url{http://www.radio.cz/de/rubrik/spazier/juwel-der-beuroner-kunst-kloster-st-gabriel}
\end{Kunstliteraturliste}
\item \tschechKunststaette{\emph{Stadtteil \tschechB{\v{R}}epy:} Kirche der Heiligen Familie}{\emph{\tschechB{\v{R}}epy district:} Church of the Holy Family}{Kostel sv. Rodiny}
\item \tschechKunststaette{\emph{Stadtteil \v{Z}i\v{z}kov:} Kirche St. Anna}{\emph{\v{Z}i\v{z}kov district:} Church of St Anne}{Kostel sv. Anny}
\end{Kunststaettenliste}
\Kunstort{Räckelwitz/Landkreis Bautzen}
\begin{Kunststaettenliste}
\item Schlosskapelle
\end{Kunststaettenliste}
\Kunstort{Rüdesheim am Rhein}
\begin{Kunststaettenliste}
\item Abtei St. Hildegard
\begin{Kunstliteraturliste}
\item\cite{Kappel}
\item\url{http://www.abtei-st-hildegard.de/?p=4537}
\item\url{http://www.abtei-st-hildegard.de/?p=4603}
\item\url{http://www.abtei-st-hildegard.de/?p=4574}
\item\emph{englisch:} \url{http://www.abtei-st-hildegard.de/?page_id=1632}
\end{Kunstliteraturliste}
\end{Kunststaettenliste}
\tschechKunstort{Teplitz}{Nordböhmen}{Teplice}{North Bohemia}
\begin{Kunststaettenliste}
\item \tschechKunststaette{Kapelle St. Karl Borromäus}{Chapel of St Charles Borromeo}{Kaple sv. Karla Boromejského}
\end{Kunststaettenliste}
\tschechKunstort{Troppau}{Schlesien}{Opava}{Silesia}
\begin{Kunststaettenliste}
\item \tschechKunststaette{Kapelle des hlst. Herzens Jesu im Marianum}{Chapel of the Most Sacred Heart of Jesus in the Marianum}{Kaple Božského Srdce Pán\tschechB{\v{e}}}
\end{Kunststaettenliste}

\section{Versionsprotokoll}
\begin{description}
\item[1.0] \printdate{2016-04-09}
\item[1.1] \printdate{2016-09-11}
\begin{itemize}
\item Erweiterung des Zeichenvorrats um den griechischen Großbuchstaben Omega
\item Verbesserung des Paketcodes
\item Erweiterung der Literaturhinweise in der Anleitung
\item Ergänzung einer map-Datei zur Benutzung der Type-1-Version der Schriften
\end{itemize}
\item[1.2] \printdate{2018-01-01}
\begin{itemize}
\item Ergänzung einer Opentype-Version der Schriften
\item Unterstützung von Blocksatz bei Verwendung von Mittepunkten
\item Erweiterung des Zeichenvorrats um die lateinischen Kleinbuchstaben (mit gleichen Glyphen wie die Großbuchstaben)
\item Erweiterung der Literaturhinweise in der Anleitung
\item Erweiterung der Anleitung um eine unvollständige Liste mit den Stätten der Beuroner Kunst
\item Ergänzung der Kerning-Paare QU, QV, QW, UZ, VZ, WZ und ZT
\end{itemize}
\item[1.3] \printdate{2018-01-17}
\begin{itemize}
\item Korrektur fehlerhafter Umrisslinien
\end{itemize}
\end{description}

\vfill
\noindent\small
Die Verse in den Kopfzeilen dieser Anleitung stammen aus der Beuroner Gnadenkapelle. Das Muster in der Fußzeile ist der Deckenbemalung dieser Kapelle entnommen.
\end{document}

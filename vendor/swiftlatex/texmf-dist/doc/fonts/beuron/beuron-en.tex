% beuron package: English manual
% Version 1.3
% Datum: 17. Januar 2018
\documentclass[numbers=noenddot]{scrartcl}
\PassOptionsToPackage{hyphens}{url}
\usepackage{fontspec}
\usepackage[latin,czech,ngerman,UKenglish]{babel}
\usepackage[UKenglish]{isodate}
\usepackage[style=authortitle,language=auto,autolang=hyphen]{biblatex}
\usepackage{csquotes}
\usepackage{cnltx-tools}
\usepackage{cnltx-example}
\usepackage{beuron}
\usepackage{scrlayer-scrpage}
\usepackage{graphicx}
\usepackage{metalogo}
\usepackage{mflogo}
\usepackage{booktabs}
\usepackage{tcolorbox}
\usepackage{multirow}
\usepackage[toc]{multitoc}
\usepackage[justification=centering]{caption}
\usepackage[colorlinks=true,
   allcolors=black,
   bookmarksnumbered=true,
   pdfencoding=auto,
   pdftitle={The script of Beuronese art},
   pdfsubject={Manual for the BEURON package},
   pdfkeywords={latex font Beuron Beuronese art},
   pdfauthor={K. Wehr}]{hyperref}

\addbibresource{Literatur.bib}

\renewcommand{\labelnamepunct}{\addcolon\space}
\renewcommand{\nametitledelim}{\addcolon\space}

% "Hrsg." in Klammern setzen, Komma davor entfernen
\DeclareFieldFormat{editortype}{\mkbibparens{#1}}
\renewbibmacro*{editor+others}{%
  \ifboolexpr{
    test \ifuseeditor
    and
    not test {\ifnameundef{editor}}
  }
    {\printnames{editor}%
    \space%
     \usebibmacro{editor+othersstrg}%
     \clearname{editor}}
    {}}

\setmainfont{GilliusADF}
\setsansfont{GilliusADF}

\setlogokern{Xe}{-0,03em}
\setlogokern{eL}{-0,01em}
\setlogokern{La}{-0,2em}
\setlogokern{aT}{-0,05em}
\setlogokern{Te}{-0,15em}
\setlogokern{eX}{-0,03em}

\definecolor{Beuronblau}{rgb}{0.15,0.2,0.35}
\definecolor{Beurongelb}{rgb}{0.95,0.65,0.15}

\tcbset{colframe=Beurongelb,colback=Beuronblau,fontupper=\color{Beurongelb}}

\ExplSyntaxOn

\NewDocumentCommand \AutorVorname {s}
   {
      \IfBooleanTF{#1}{k}{K}
      eno
   }

\NewDocumentCommand \AutorNachname {s}
   {
      \IfBooleanTF{#1}{w}{W}
      ehr
   }

\NewDocumentCommand \AutorEMailDomain {}
   {
      abgol
   }

\NewDocumentCommand \tschechB {m}
   {
      \group_begin:
      \fontspec{Latin Modern Sans}
      #1
      \group_end:
   }

\NewDocumentEnvironment {Befehlsliste} { }
   {
      \begin{list}{ }
         {
            \setlength{\leftmargin}{0pt}
            \setlength{\itemindent}{-1em}
            \setlength{\parsep}{0pt}
            \setlength{\listparindent}{\parindent}
            \setlength{\itemsep}{\topsep}
         }
   }
   {
      \end{list}
   }

\NewDocumentCommand \Befehlsbeschreibung {m o}
   {
      \item
      \cs{#1}
      \IfValueT{#2}{#2}
      \\
   }

\NewDocumentEnvironment {Symboltabelle} { }
   {
      \begin{tabular}{rl@{\hspace{6mm}}l}
   }
   {
      \end{tabular}
   }

\NewDocumentCommand \Symboltabellenzeile {mo}
   {
      \small\texttt{#1} & \texttt{\symbol{#1}}
      & \large
      \IfValueTF{#2}
         {
            \textbeuron{\symbol{#2}}
         }
         {
            \textbeuron{\symbol{#1}}
         }
   }

\NewDocumentCommand \Symboltabellenbereich {mm}
   {
      \int_step_inline:nnnn {#1} {1} {#2}
         {
            \Symboltabellenzeile {##1}
            \int_compare:nNnF {##1} = {#2} {\\}
         }
   }

\NewDocumentCommand \Kunstort {m}
   {
      \smallskip\par\noindent\textbf{#1}
   }

\NewDocumentCommand \tschechKunstort {mmmm}
   {
      \Kunstort{#3/#4}
   }

\NewDocumentCommand \tschechKunststaette {mmm}
   {
      #2~(\foreignlanguage{czech}{\emph{#3}})
   }

\NewDocumentEnvironment {Kunststaettenliste} { }
   {
      \begin{list}{ }
         {
            \setlength{\topsep}{1ex plus0.4ex minus0.2ex}
            \setlength{\itemsep}{\topsep}
            \setlength{\parsep}{0pt}
            \setlength{\leftmargin}{1,5em}
         }
   }
   {
      \end{list}
   }

\NewDocumentEnvironment {Kunstliteraturliste} { }
   {
      \begin{list}{ }
         {
            \setlength{\leftmargin}{1,5em}
            \small
         }
   }
   {
      \end{list}
   }

\ExplSyntaxOff

\setlength{\columnsep}{2em}
\setlength{\columnseprule}{0,4pt}
\setlength{\floatsep}{2\bigskipamount}
\setlength{\bibitemsep}{\smallskipamount}

\newlength{\Boxbreite}
\setlength{\Boxbreite}{\paperwidth}
\addtolength{\Boxbreite}{-2\fboxsep}

\DeclareNewLayer[background,topmargin,voffset={5mm},contents={\colorbox{Beuronblau}{\parbox[c][6mm][c]{\Boxbreite}{\centering\color{Beurongelb}\Large
\ifcase\thepage
\or\textbeuronc{+Beata es Virgo Maria Dei genetrix quae credidisti Domino+}
\or\textbeuronx{+Perfecta sunt in te quae dicta sunt tibi+}
\or\textbeuronx{+Ecce exaltata es super choros angelorum+}
\or\textbeuron{+Intercede pro nobis ad Dominum Deum nostrum+}
\or\textbeuronx{+Ecce virgo concipiet et pariet filium+}
\or\textbeuronx{+Beata progenies unde Christus natus est+}
\or\textbeuron{+O quam gloriosa est Virgo quae coeli regem genuit+}
\or\textbeuron{+Congratulamini mihi omnes qui diligitis Dominum+}
\or\textbeuronx{+Stemus iuxta crucem cum Maria matre Iesu+}
\or\textbeuronx{+Gaude et laetare virgo Maria alleluia+}
\fi
}}}]{Oberband}

\DeclareNewLayer[background,bottommargin,mode=picture,addheight={-1cm},contents={\includegraphics{Band}}]{Unterband}

\AddLayersToPageStyle{plain}{Oberband,Unterband}

\pagestyle{plain}

\newcommand{\PL}{Peter Lenz (1832--1928)}
\newcommand{\JW}{Jakob Wüger (1829--1892)}

\newcommand{\beur}{\textcolor{Beuronblau}{\texttt{beuron}}}

\begin{document}

\begin{center}
\Huge
\textcolor{Beuronblau}{\textbeuronx{BEURON}}

\smallskip
\LARGE
The script of Beuronese art

\medskip
\Large
Version 1.3

\medskip
\normalsize
\today

\vspace{2\bigskipamount}
\large
\textit{Package author}

\medskip
\AutorVorname\ \AutorNachname

\normalsize
\smallskip
\texttt{\AutorVorname*.\AutorNachname*@\AutorEMailDomain.de}
\end{center}

\medskip
\begin{abstract}
\noindent The \beur\ package provides the script used in the works of the Beuron art school for use with \TeX\ and \LaTeX. It is a monumental script consisting of capital letters only.
\end{abstract}

\medskip
\tableofcontents

\medskip
\section{Beuronese Art}
\subsection{Position in the History of Art}
Beuronese art was a reform movement of Christian art, established by \PL\ and \JW, who were friends from their studies in Munich, during their stay in Rome in the 1860s.

On the one hand, it arose from the art of the Nazarene movement, but on the other hand, it turned away from the naturalism of the Romantic period and strove for a more geometrically stylized depiction of Christian themes. An important impact on this had the examination of ancient Egyptian art, which becomes noticable especially by a far-reaching renunciation of spatial depth in depiction.

Lenz and Wüger entered the Benedictine abbey of Beuron (near Sigmaringen in Southern Germany) in 1872 and 1870 respectively, where they worked as Pater Desiderius and Pater Gabriel. Beuronese art was essentially carried by the circle of their pupils from the monastery in the following decades up to the 1930s.

The Beuronese artists were not only commissioned to paint and furnish the monastery of Beuron itself, reestablished in 1863, but also quite a lot of other churches and monasteries in several countries of Europe. The Beuron art school reached its summit about 1900, when it received attention by the world of art beyond the religious milieu through the participation in various exhibitions.

Due to the Second World War and church “renovations” in the following period many works of Beuronese art were partially or totally destroyed. Today remaining works can be seen for instance in Beuron (Chapel of St Maurus and Archabbey of St Martin), Rüdesheim am Rhein (Abbey of St Hildegard), Prague (churches of the former abbeys of Emaus and St Gabriel), but also in America in Conception/Missouri (Basilica of the Immaculate Conception), see also appendix
\ref{Staettenliste}.

\subsection{The Script}
The murals painted by the artists of the Beuron school were provided with monumental inscriptions, taken from the Holy Bible or the prayer tradition of the Church, which support the didactic character of the paintings. For these paintings a script with some striking features was used, recuring in the most murals and also craft objects of the school with only minor variations.

Unfortunately the art-historic literature dealing with Beuronese art says nearly nothing about this script, although it constitutes obviously an integral part of that art. A guide to the crypt of the Beuron Gnadenkapelle only points to the use of ``Beuronese capital letters''\kern1pt\footcite[\ppno\ 19 and 23]{Groeger}, a guide to the Gnadenkapelle itself mentions ``an own alphabet, developed from roman type'' and names some characteristic features.\footcite[21]{KrinsB}
So the origin of the script is a matter of conjecture. Possibly it is influenced by the inscriptions of early Christian basilicas in Italy. The first use of the script known to the package author is the inscription of the cross in Wüger's painting of the crucifixion from 1868.\footnote{\url{https://commons.wikimedia.org/wiki/File:W\%C3\%BCger_Kreuzigung.jpg}}

The \beur\ package provides the Beuronese script for use with \TeX\ and \LaTeX\ by means of \MF.

\begin{center}
\Large\textbeuron{ABCDEFGHIJKLMNOPQRSTUVWXYZ}
\end{center}

The script has the following pecularities, some of whom are based on ancient models:
\begin{itemize}
\item Only capital letters are used.
\item The E and the G have an unusual rounded form.
\item The A has an angled bar in many cases (and so it has in the form provided by this package).
\item The letters B and R have a gap between the bowl an the stem in the middle.
\item No difference is made between U and V.
\item The letter I is used instead of J as a rule, though in some cases a J occurs at the beginning of a word.
\item There are no punctuation marks. In some cases a colon is used to seperate the parts of a sentence.
\item The words are not seperated by spaces, but by interpuncts (middle dots).
\item The verses are often surrounded by two Greek crosses.
\item The script is mainly used for Latin texts, but occasionally also for German ones.
\end{itemize}

\section{Fonts}
\subsection{Characters}
In accordance with the historical model the fonts of the \beur\ package provide only the twenty-six capital letters of the latin alphabet, though there is no graphical difference between U and V. To simplify the use, the fonts also provide lower case letters; but these have the same glyphs as the upper case ones. Additionally there is a Greek capital letter Omega. Umlauts, accented letters and puncutation marks are not available.

\begin{table}
\setlength{\columnseprule}{0pt}
\begin{multicols}{4}
\begin{Symboltabelle}
\Symboltabellenzeile{43} \\
\Symboltabellenzeile{45} \\
\Symboltabellenzeile{47} \\
\Symboltabellenzeile{58} \\
\Symboltabellenbereich{65}{75}
\end{Symboltabelle}

\begin{Symboltabelle}
\Symboltabellenbereich{76}{90}
\end{Symboltabelle}

\begin{Symboltabelle}
\Symboltabellenbereich{97}{111}
\end{Symboltabelle}

\begin{Symboltabelle}
\Symboltabellenbereich{112}{122} \\
\Symboltabellenzeile{141}[937]
\end{Symboltabelle}
\end{multicols}
\caption{Characters of the Beuron fonts. In the Opentype version the Omega has the code \texttt{937} deviantly
(Unicode: \emph{Greek capital letter Omega}).}
\label{Zeichenvorrat}
\end{table}

As special characters the package provides a middle dot, which can be typed in as slash if necessary, a colon and a Greek cross, which has to be typed in as plus sign. Besides that there is a hyphen, so hyphenation will be possible. Table \ref{Zeichenvorrat} shows all available characters and their encoding.

\subsection{Type Styles}

The Beuronese script is offered in three type styles, differing only in the width of the characters (\cf\ fig.\@ \ref{Breiten}). This is based on the historical use of the script, which was drawn by the artists in different widths according to the concrete need.

\begin{figure}
\begin{tcolorbox}[left=2mm,right=2mm]
\centering
\textbeuronc{abcdefghijklmnopqrstuvwxyz\beuronOmega}

\smallskip
\textbeuronc{Laetifica nos pro diebus quibus nos humiliasti}

\medskip
\textbeuron{abcdefghijklmnopqrstuvwxyz\beuronOmega}

\smallskip
\textbeuron{Laetifica nos pro diebus quibus nos humiliasti}

\medskip
\textbeuronx{abcdefghijklmnopqrstuvwxyz\beuronOmega}

\smallskip
\textbeuronx{Laetifica nos pro diebus quibus nos humiliasti}
\end{tcolorbox}
\caption{The condensed, normal and extended Beuron font \\ \emph{Example verse:} Ps~90 (89)\kern1pt:\kern1pt15}
\label{Breiten}
\end{figure}

The characters of the three fonts as well as the kerning pairs (character pairs with changed spacing) have been defined by \MF\ files, vectorized with the aid of the \emph{Mftrace} program\footnote{\url{https://ctan.org/pkg/mftrace}}, and converted with the aid of the \emph{Fontforge} program\footnote{\url{http://fontforge.github.io/en-US/}}. They are available in the \emph{Type~1} and in the \emph{Opentype} format.

The \TeX\ font names and the \LaTeX\ font attributes are given in table \ref{Namen}. From this you can see for instance that the extended Beuron typeface can be selected by the command sequence
\verbcode:\fontfamily{beuron}\fontseries{x}\selectfont:.

\begin{table}
\centering
\begin{tabular}{lccccc}
\toprule
& \multirow{2}*{\TeX\ font name} & \multicolumn{4}{c}{\LaTeX\ font attributes} \\
& & encoding & family & series & shape \\
\midrule
condensed & \texttt{beuronc} & \texttt{T1} & \texttt{beuron} & \texttt{c} & \texttt{n} \\
normal (medium) & \texttt{beuron} & \texttt{T1} & \texttt{beuron} & \texttt{m} & \texttt{n} \\
extended & \texttt{beuronx} & \texttt{T1} & \texttt{beuron} & \texttt{x} & \texttt{n} \\
\bottomrule
\end{tabular}
\caption{Font names and font attributes}
\label{Namen}
\end{table}

When using the modern \LaTeX\ variants \XeLaTeX\ and \LuaLaTeX\ the Opentype versions of the fonts may be loaded with the aid of the \texttt{fontspec} package. From table~\ref{Opentype}
you can see for example that the extended Beuron typeface can be chosen by
\verbcode:\fontspec{Beuron Extended}:
using \LuaLaTeX.

\begin{table}
\centering
\begin{tabular}{lcc}
\toprule
& Opentype font family & Opentype font file \\
\midrule
condensed & \texttt{Beuron Condensed} & \texttt{BeuronCondensed-Regular.otf} \\
normal (medium) & \texttt{Beuron} & \texttt{Beuron-Regular.otf} \\
extended & \texttt{Beuron Extended} & \texttt{BeuronExtended-Regular.otf} \\
\bottomrule
\end{tabular}
\caption{Opentype font families and font files}
\label{Opentype}
\end{table}

The Beuron fonts are provided in one size only and will be scaled to the selected font size. As usual in typography, the real height of the characters is less than the nominal font size (point size). For the Beuron fonts the height of the characters is two thirds of the point size.

\subsection{Font Selection Commands}
To make the use of the Beuron fonts easier, the \beur\ package provides appropriate commands for use with \LaTeX. Their use requires loading the package with
\verbcode:\usepackage{beuron}:
in the preamble of the document.

\begin{Befehlsliste}
\Befehlsbeschreibung{textbeuron}[\marg{text}]
The command typesets a text in the normal Beuron font. Spaces are replaced by interpuncts. In accordance with a rule of German typography, umlauts are replaced by the respective vowel followed by an E, the German letter ß by SS. The text argument must not be longer than one paragraph. Figure~\ref{textbeuron} shows examples.

\begin{figure}
\begin{tcolorbox}
\selectlanguage{ngerman}
\noindent{\color{gray}\fontspec{Latin Modern Mono Prop}\textbackslash textbeuron\{Du legst mir größere Freude ins Herz als andere haben bei Korn und Wein in Fülle\}}

\smallskip
\noindent\textbeuron{Du legst mir größere Freude ins Herz als andere haben bei Korn und Wein in Fülle}

\selectlanguage{latin}
\medskip
\noindent{\color{gray}\fontspec{Latin Modern Mono Prop}\textbackslash textbeuron\{Pater noster qui es in caelis sanctificetur nomen tuum adveniat regnum tuum fiat voluntas tua sicut in caelo et in terra panem nostrum cotidianum da nobis hodie et dimitte nobis debita nostra sicut et nos dimittimus debitoribus nostris et ne nos inducas in tentationem sed libera nos a malo\}}

\smallskip
\noindent\textbeuron{Pater noster qui es in caelis sanctificetur nomen tuum adveniat regnum tuum fiat voluntas tua sicut in caelo et in terra panem nostrum cotidianum da nobis hodie et dimitte nobis debita nostra sicut et nos dimittimus debitoribus nostris et ne nos inducas in tentationem sed libera nos a malo}
\end{tcolorbox}
\caption{Examples for the use of the \verbcode:textbeuron: command.\\ Example texts: Ps 4\kern1pt:\kern1pt8 and Mt 6\kern1pt:\kern1pt9--13}
\label{textbeuron}
\end{figure}

\Befehlsbeschreibung{textbeuronc}[\marg{text}]
The same with the condensed Beuron font.
\Befehlsbeschreibung{textbeuronx}[\marg{text}]
The same with the extended Beuron font.
\Befehlsbeschreibung{textbeuron*}[\marg{text}]
The command acts like \cs{textbeuron}, but keeps the spaces.
\Befehlsbeschreibung{textbeuronc*}[\marg{text}]
The same with the condensed Beuron font.
\Befehlsbeschreibung{textbeuronx*}[\marg{text}]
The same with the extended Beuron font.
\Befehlsbeschreibung{beuronOmega}
Outputs the Greek capital letter Omega.
\end{Befehlsliste}

\noindent \emph{Technical remark:} The \beur\ package loads the \texttt{fontspec} package when using \XeLaTeX\ or \LuaLaTeX. If you want to load \texttt{fontspec} with options, you have to do it before loading \beur.

\section{Areas of Use}
Besides the use for monumental inscriptions following the historical example the Beuron typeface is recommended for headings and ornaments in prayer books, hymnals and the like. The typeface is not suitable for longer texts and for profane use.

\section{License}
The \beur\ package is subject to the
\emph{\LaTeX\ Project Public License},
version 1.3 or later.\footnote{\url{http://www.latex-project.org/lppl.txt}}
The Opentype fonts may also be used independantly of the package under the
\emph{SIL Open Font License}, version 1.1.\footnote{\url{http://scripts.sil.org/OFL}}

\appendix
\nocite{*}
\printbibliography[heading=bibnumbered]

\section{Incomplete list of the places of Beuronese art}\label{Staettenliste}
\Kunstort{Berlin}
\begin{Kunststaettenliste}
\item\emph{Wedding district:} Church of St Joseph\footnote{In this church the Beuronese script has not been used.}
\begin{Kunstliteraturliste}
\item\url{http://www.berlin.de/landesdenkmalamt/denkmalpflege/erkennen-und-erhalten/sakralbauten/st-josephs-kirche-639576.php}
\end{Kunstliteraturliste}
\end{Kunststaettenliste}
\Kunstort{Beuron/District of Sigmaringen}
\begin{Kunststaettenliste}
\item Archabbey of St Martin (Main altar of the abbey church, \emph{Gnadenkapelle} with under chapel)
\begin{Kunstliteraturliste}
\item\cite{KrinsB}
\item\cite{Groeger}
\item\url{http://www.erzabtei-beuron.de/kloster/kultur/kirche/index.html}
\item\url{http://www.erzabtei-beuron.de/kloster/kultur/gnadenkapelle/index.html}
\item\url{http://www.erzabtei-beuron.de/kloster/kultur/krypta/index.html}
\end{Kunstliteraturliste}
\item Chapel of St Maurus
\begin{Kunstliteraturliste}
\item\cite{KrinsB}
\item\url{http://www.erzabtei-beuron.de/kloster/kultur/mauruskapelle/index.html}
\end{Kunstliteraturliste}
\end{Kunststaettenliste}
\tschechKunstort{Budweis}{Südböhmen}{\tschechB{\v{C}}eské Bud\tschechB{\v{e}}jovice}{South Bohemia}
\begin{Kunststaettenliste}
\item \tschechKunststaette{Maria-Rosenkranz-Kirche}{Church of Our Lady of the Rosary}{Kostel Panny Marie R\tschechB{ů}žencové}
\begin{Kunstliteraturliste}
\item\url{http://www.petrini.cz/clanky/komunity/ceske_budejovice/kostel}
\item\url{http://www.bbkult.net/kulturdatenbank/adressen:sehenswuerdigkeit:gerettete-baudenkmaeler::5/13294761189848.html}
\end{Kunstliteraturliste}
\end{Kunststaettenliste}
\Kunstort{Collegeville/Minnesota}
\begin{Kunststaettenliste}
\item Great Hall of Saint John's University, former church of Saint John's Abbey
\begin{Kunstliteraturliste}
\item\url{http://www.newliturgicalmovement.org/2012/03/church-of-st-johns-abbey-collegeville.html}
\end{Kunstliteraturliste}
\end{Kunststaettenliste}
\Kunstort{Conception/Missouri}
\begin{Kunststaettenliste}
\item Basilica of the Immaculate Conception
\begin{Kunstliteraturliste}
\item\url{https://www.conceptionabbey.org/monastery/basilica/}
\item\url{https://www.conceptionabbey.org/monastery/beuronese-murals/}
\end{Kunstliteraturliste}
\end{Kunststaettenliste}
\Kunstort{Gaußig/District of Bautzen}
\begin{Kunststaettenliste}
\item Castle chapel
\end{Kunststaettenliste}
\tschechKunstort{Königgrätz}{Nordostböhmen}{Hradec Králové}{North-East Bohemia}
\begin{Kunststaettenliste}
\item \tschechKunststaette{Städtische Musikhalle}{Municipal Music Hall}{M\tschechB{\v{e}}stská hudební sí\tschechB{\v{n}}}, \tschechKunststaette{ehemals Kirche St. Johannes Nepomuk}{formerly Church of St John of Nepomuk}{Kostel sv. Jana Nepomuckého}
\end{Kunststaettenliste}
\Kunstort{Constance}
\begin{Kunststaettenliste}
\item Chapel of St Conrad in the Minster
\end{Kunststaettenliste}
\Kunstort{Meßkirch/District of Sigmaringen}
\begin{Kunststaettenliste}
\item Parish hall of the Sacred Heart (\emph{Herz-Jesu-Heim}), formerly Church of the Sacred Heart
\end{Kunststaettenliste}
\Kunstort{Montecassino/Lazio}
\begin{Kunststaettenliste}
\item Crypt of the abbey cathedral
\end{Kunststaettenliste}
\Kunstort{Prague}
\begin{Kunststaettenliste}
\item \tschechKunststaette{\emph{Neustadt:} Emauskloster}{\emph{New Town:} Emmaus monastery}{Emauzský klášter}
\begin{Kunstliteraturliste}
\item\cite{Emaus}
\end{Kunstliteraturliste}
\item \tschechKunststaette{\emph{Stadtteil Smíchov:} Kirche der ehem. Abtei St. Gabriel}{\emph{Smíchov district:} Church of the former abbey of St Gabriel}{Klá\v{s}ter sv. Gabriela}
\begin{Kunstliteraturliste}
\item\cite{Pieta}
\item\cite{Evangelistar}
\item\cite{Anschriften}
\item\url{http://malakim.cz/}
\item\url{http://www.radio.cz/de/rubrik/spazier/juwel-der-beuroner-kunst-kloster-st-gabriel}
\end{Kunstliteraturliste}
\item \tschechKunststaette{\emph{Stadtteil \tschechB{\v{R}}epy:} Kirche der Heiligen Familie}{\emph{\tschechB{\v{R}}epy district:} Church of the Holy Family}{Kostel sv. Rodiny}
\item \tschechKunststaette{\emph{Stadtteil \v{Z}i\v{z}kov:} Kirche St. Anna}{\emph{\v{Z}i\v{z}kov district:} Church of St Anne}{Kostel sv. Anny}
\end{Kunststaettenliste}
\Kunstort{Räckelwitz/District of Bautzen}
\begin{Kunststaettenliste}
\item Castle chapel
\end{Kunststaettenliste}
\Kunstort{Rüdesheim am Rhein}
\begin{Kunststaettenliste}
\item Abbey of St Hildegard
\begin{Kunstliteraturliste}
\item\cite{Kappel}
\item\url{http://www.abtei-st-hildegard.de/?p=4537}
\item\url{http://www.abtei-st-hildegard.de/?p=4603}
\item\url{http://www.abtei-st-hildegard.de/?p=4574}
\item\emph{English:} \url{http://www.abtei-st-hildegard.de/?page_id=1632}
\end{Kunstliteraturliste}
\end{Kunststaettenliste}
\tschechKunstort{Teplitz}{Nordböhmen}{Teplice}{North Bohemia}
\begin{Kunststaettenliste}
\item \tschechKunststaette{Kapelle St. Karl Borromäus}{Chapel of St Charles Borromeo}{Kaple sv. Karla Boromejského}
\end{Kunststaettenliste}
\tschechKunstort{Troppau}{Schlesien}{Opava}{Silesia}
\begin{Kunststaettenliste}
\item \tschechKunststaette{Kapelle des hlst. Herzens Jesu im Marianum}{Chapel of the Most Sacred Heart of Jesus in the Marianum}{Kaple Božského Srdce Pán\tschechB{\v{e}}}
\end{Kunststaettenliste}

\section{Version History}
\begin{description}
\item[1.0] \printdate{2016-04-09}
\item[1.1] \printdate{2016-09-11}
\begin{itemize}
\item Addition of the Greek capital letter Omega
\item Revision of the package code
\item Extension of the bibliographical references in the manual
\item Addition of a map file for the use of the Type~1 version of the fonts
\end{itemize}
\item[1.2] \printdate{2018-01-01}
\begin{itemize}
\item Addition of an Opentype version of the fonts
\item Support of justification when using interpuncts
\item Addition of the latin lower case letters (with the same glyphs as the upper case ones)
\item Extension of the bibliographical references in the manual
\item Addition of an incomplete list of the places of Beuronese art to the manual
\item Addition of the kerning pairs QU, QV, QW, UZ, VZ, WZ, and ZT
\end{itemize}
\item[1.3] \printdate{2018-01-17}
\begin{itemize}
\item Correction of corrupted outlines
\end{itemize}
\end{description}

\vfill
\noindent\small
The verses in the page header of this manual come from the Beuron Gnadenkapelle. The pattern in the footer is taken from the ceiling painting of this chapel.
\end{document}

\documentclass{article}
\usepackage{amsmath}
\usepackage[displaylimits]{cmupint}
\usepackage[T1]{fontenc}
\usepackage[utf8]{inputenc}
\usepackage[a4paper,top=2.0cm,left=3cm,right=2.5cm,bottom=2cm,includefoot,includehead]{geometry}
\input glyphtounicode.tex
\pdfgentounicode=1
\usepackage{fancyvrb}
\usepackage{tabularx}

\title{The \texttt{cmupint} package}
\author{Uroš Stefanović\footnote{\texttt{urostajms@gmail.com}}}
\date{\today{} v1.0}

\newcommand\tbview[1]{ \verb"#1" & $\textstyle #1$ & $\displaystyle #1$ }

\begin{document}
\maketitle

\section{Why \textsf{cmupint}?}

The shape of integral symbol in some languages differs slightly from the shape commonly seen in English-language textbooks.
While the English integral symbol leans to the right, the integral symbol used throughout Central Europe (such as in German or Serbian language) is upright.

Another difference is in the placement of limits for definite integrals. Generally, in English-language books, limits go to the right of the integral symbol, while in Central European languages the limits are placed above and below the integral symbol.

\[ \intop\nolimits^U_L x\,\mathrm{d}x\quad \text{(English language)} \]
\[ \int\limits^U_L x\,\mathrm{d}x\quad \text{(Central European languages)} \]

The \textsf{cmupint} package contains various upright integral symbols to match Computer Modern font (default \LaTeX{} font).

\section{Usage}

This package is very simple to use: just put
\begin{verbatim}
\usepackage{cmupint}
\end{verbatim}
in preamble of the document.

\section{Options}

The available options are \texttt{displaylimits} (default option), \texttt{limits} and \texttt{nolimits} (see Table~\ref{tabl1}).

\begin{table}[h!]
	\newcolumntype{C}{>{\centering\arraybackslash}X}%
	\setlength{\extrarowheight}{10pt}
	\centering
	\begin{tabularx}{.7\textwidth}{|C|C|C|}
		\hline
		Option & Text style & Display style \\
		\hline
		\hline
		\texttt{displaylimits} & $\textstyle\int\displaylimits^U_L x\,\mathrm{d}x$ & $\displaystyle \int\displaylimits^U_L x\,\mathrm{d}x$ \\
		\hline
		\texttt{limits} & $\textstyle\int\limits^U_L x\,\mathrm{d}x$ & $\displaystyle \int\limits^U_L x\,\mathrm{d}x$ \\
		\hline
		\texttt{nolimits} & $\textstyle\int\nolimits^U_L x\,\mathrm{d}x$ & $\displaystyle \int\nolimits^U_L x\,\mathrm{d}x$ \\
		\hline
	\end{tabularx}
\caption{Package options.}\label{tabl1}
\end{table}

\section{Integral symbols}

For the avaible integral symbols see Table~\ref{tabl2}.

\begin{table}[h!]
	\newcolumntype{C}{>{\centering\arraybackslash\hsize=.1\hsize}X}%
	\newcolumntype{S}{>{\centering\hsize=.3\hsize}X}
	\setlength{\extrarowheight}{10pt}
	\centering
	\begin{tabularx}{\textwidth}{|SCC|SCC|}
	\hline
	Command & Text style & Display style & Command & Text style & Display style \\
	\hline\hline
	\tbview{\int} & \tbview{\iint} \\
	\tbview{\iiint} & \tbview{\iiiint} \\
	\tbview{\oint} & \tbview{\oiint} \\
	\tbview{\oiiint} & \tbview{\ointctrclockwise} \\
	\tbview{\ointclockwise} & \tbview{\varointclockwise} \\
	\tbview{\varointctrclockwise} & \tbview{\sqint} \\
	\tbview{\sqiint} & \tbview{\pointint} \\
	\tbview{\npolint} & \tbview{\scpolint} \\
	\tbview{\rppolint} & \tbview{\cirfnint} \\
	\tbview{\intclockwise} & \tbview{\awint} \\
	\tbview{\fint} & \tbview{\barint} \\
	\tbview{\doublebarint} & \tbview{\xint} \\
	\tbview{\landupint} & \tbview{\landdownint} \\
	\tbview{\intlarhk} & \tbview{\upint} \\
	\tbview{\downint} & \tbview{\varidotsint} \\
	\verb"\idotsint"\footnotemark[1] & $\textstyle \idotsint$ & $\displaystyle \idotsint$ & \verb"\idotsint"\footnotemark[2] & $\textstyle \varidotsint$ & $\displaystyle \varidotsint$ \\
	\hline
	\end{tabularx}
	\caption{Integral symbols.}\label{tabl2}
\end{table}
\footnotetext[1]{If package \texttt{amsmath} is loaded.}
\footnotetext[2]{If package \texttt{amsmath} is not loaded.}

\end{document}
%% missaali.tex
%% Copyright 2016 Tommi Syrjönen 
%
% This work may be distributed and/or modified under the
% conditions of the LaTeX Project Public License, either version 1.3
% of this license or (at your option) any later version.
% The latest version of this license is in
%   http://www.latex-project.org/lppl.txt
% and version 1.3 or later is part of all distributions of LaTeX
% version 2005/12/01 or later.
%
% This work has the LPPL maintenance status `maintained'.
% 
% The Current Maintainer of this work is Tommi Syrjänen 
%
% See file MANIFEST-Missaali.txt to a list of files that make up this
% work.
\documentclass[a4paper, 12pt]{article}
%% Author: Tommi Syrjänen (tssyrjan@iki.fi)
%%
%% This software is distributed under the LPPL license

%% This is the documentation file for Missaali OpenType textura font
%% and its related XeLaTeX style file missaali.sty

\usepackage{xltxtra}
\usepackage{longtable}
\usepackage{xcolor}
\usepackage{graphics}
\usepackage{multicol}
\usepackage{missaali}
\usepackage{subfigure}

\newcommand{\feature}[1]{\texttt{#1}}

\newcommand{\sample}[1]{{\samplefont #1}}
\newcommand{\sampleabbr}[1]{{\fontsize{26}{12}\fontspec{Missaali}%
    \hbox{\vbox to 10pt{\hbox{#1}}}}}

\newcommand{\initsample}[1]{{\initsamplefont #1}}
\newcommand{\topalign}[2]{\hbox{\vbox to \ht#1
    {\hbox{#2}\vfill}}}
\newcommand{\midalign}[2]{\hbox{\vbox to \ht#1
    {\vfill\hbox{#2}\vfill}}}
\newcommand{\samplefont}{\fontsize{18}{18}\fontspec{Missaali}\spaceskip0.2em}
\newcommand{\initsamplefont}{\fontsize{36}{36}\fontspec{Missaali}\spaceskip0.2em}

\newbox\tmpbox
\newcommand{\showlettertitle}[2]{%
  \fbox{\hbox to 1.8em{\vbox to 2.5em{\hbox to%
        1.8em{\tiny\hfill#2\hfill}\hbox to 1.8em{\vbox to 2em{\vfill{\hbox%
              to 1.8em{\hfill\samplefont{#1}\hfill}\vfill}}}}}}}


\title{\fontsize{36}{36}\fontspec[BoldFont=Missaali]{Missaali}Missaali}

\author{Tommi Syrjänen \\ \texttt{tssyrjan@iki.fi}}

\date{Version 1.004, 25 December, 2016}

\setcounter{secnumdepth}{1}
\setcounter{tocdepth}{1}

\begin{document}

 \maketitle

\mbox{}

\thispagestyle{empty}

\begin{center}
{
  \mssetsizes{18}{6.65cm}{20}
  \begin{mstexturablock}
    \msaltpunctuation%
    \msstartchapter{Q}{Uo usque tandem abu-}{%
      tere, Catilina, pati\mstdacc{e}{en}tia}
    nostra? \msabbr{quam} diu eti{am} furor iste
    tuus nos eludet? qu{em} ad fin\mstdacc{e}{em}
    se\msabbr{se} effr\mstdacc{e}{en}ata iactabit audacia?
    Nihilne te noc\msabbr{tur}num \msabbr{prae}sidium
    Palati, nihil urbis vigiliae,
    nihil timor populi, nihil con-
    cur\msabbr{sus} bonorum omnium, nihil
    hic m\mstdacc{u}{un}itissi\msabbr{mus} hab\mstdacc{e}{en}di senatus
    locus, nihil ho\msabbr{rum} ora voltus\msabbr{que}
    moverunt? Patere tua \msabbr{con}silia 
    non sentis, \msabbr{con}strictam iam ho\msabbr{rum} 
    omnium scientia teneri \msabbr{con}iura- 
    tionem tu\mstdacc{a}{am} non vides? Quid
    \msabbr{pro}x\mstdacc{i}{im}a, \msabbr{quid} su\msabbr{per}iore nocte egeris,
    ubi fueris, \msabbr{qvo}s {con}vocaveris,
    {quid} consilii ceperis, quem 
    nostr{um} ignorare arbitraris? 
  \end{mstexturablock}
}
\end{center}

\newpage

\thispagestyle{empty}


The text in the previous page is the beginning of Marcus Tullius
Cicero's \emph{Oratio in catilinam primo} that has been used to set
typographical speciments at least since the days of Gianbattista Bodoni
(1740–1813). In modern fonts its text reads:

\begin{quote}
  Quo usque tandem abutere, Catilina, patientia nostra? quam diu etiam
  furor iste tuus nos eludet? quem ad finem sese effrenata iactabit
  audacia? Nihilne te nocturnum praesidium Palati, nihil urbis
  vigiliae, nihil timor populi, nihil concursus bonorum omnium, nihil
  hic munitissimus habendi senatus locus, nihil horum ora voltusque
  moverunt? Patere tua consilia non sentis, constrictam iam horum
  omnium scientia teneri coniurationem tuam non vides? Quid proxima,
  quid superiore nocte egeris, ubi fueris, quos convocaveris, quid
  consilii ceperis, quem nostrum ignorare arbitraris?
\end{quote}

C.\,D. Yonge~\cite{yonge} translated the paragraph as:

 \begin{quote}
   When, O Catiline, do you mean to cease abusing our patience? How
   long is that madness of yours still to mock us? When is there to be
   an end of that unbridled audacity of yours, swaggering about as it
   does now? Do not the nightly guards placed on the Palatine Hill—do
   not the watches posted throughout the city—does not the alarm of the
   people, and the union of all good men—does not the precaution taken
   of assembling the senate in this most defensible place—do not the
   looks and countenances of this venerable body here present, have any
   effect upon you? Do you not feel that your plans are detected? Do
   you not see that your conspiracy is already arrested and rendered
   powerless by the knowledge which every one here possesses of it?
   What is there that you did last night, what the night before—where
   is it that you were—who was there that you summoned to meet
   you—what design was there which was adopted by you, with which you
   think that any one of us is unacquainted?
 \end{quote}


\vfill


The font Missaali is copyrighted by Tommi Syrjänen, 2016. It may be
used and distributed under the Open Font License 1.1 (see
page~\pageref{pg_license} for details).

The documentation and LaTeX style files are also copyrighted by Tommi
Syrjänen, 2016, and they may be distributed under the terms of Latex
Project Public License 1.3c (see the accompanying file
\texttt{lppl.txt} for details or refer to
\texttt{http://www.latex-project.org/lppl.txt}). 

\newpage


\begin{center}
\fbox{
  \begin{minipage}{0.8\linewidth}
    \begin{center}
      \textbf{The Short Version of Important Stuff}
    \end{center}
    \begin{itemize}
    \item \Missaali\ is a free OpenType/CFF font with two application
      areas:
    \begin{enumerate}
    \item as a Gothic display font; and
    \item for emulating late-medieval manuscripts.
    \end{enumerate}
    
  \item It should be used in a large size (18+ points).

  \item It is licensed under Open Font License (see the end of the document
    for details).

  \item It is a 15th century typeface. It contains a large number of
    symbols (see
    pages~\pageref{tab:ghota_abbr}–\pageref{tab:ghota_abbr_general},
    \pageref{tab:other_abbrs}, and
    \pageref{pg:symbols_start}–\pageref{pg:symbols_end}) that are no
    longer in common use but it lacks many symbols (like Arabic
    numbers) that are essential for setting longer modern texts.

  \item It has a large number of OpenType features that control its
    behavior. See pages~\pageref{tab:feature_summary} and
    \pageref{pg:features_start}–\pageref{pg:features_end} for details. 

  \item It is designed to be used with XeLaTeX. See pages
    \pageref{pg:latex_start}–\pageref{pg:latex_end} for details.
 
  \item \Missaali\ works in recentish programs. Some old font
    renderers will mess up the spacing between letters. This happens
    at least in some old versions of MS Word. Newer programs shouldn't
    have problems. However, some care is necessary when using
    abbreviation symbols with XeLaTeX, see
    page~\pageref{pg:fontspec_problems} for details. 

\end{itemize}
  \end{minipage}

  
}
  
\end{center}



\newpage
\tableofcontents
\reversemarginpar

\section{Introduction}

\Missaali\ is a free unicode OpenType font that has two main purposes:
it is a blackletter display font that can also be used to emulate
late-Medieval formal manuscripts.

The letterforms are based on \emph{Missale
  Aboense}~\cite{missale_aboense} that Bartholomew
Ghotan\index{Ghotan, Bartholomew} printed in Lübeck in 1488. It was
the first book printed for Finland and was commissioned by Konrad Bitz
who was at the time the bishop of Turku diocese. His aim was to
harmonize the church services of the parishes of the diocese and the
missal contains texts and instructions for all masses that are held
over a year.

Ghotan followed the standard practice of the time and set the missal
using \emph{textura}, a type based on \emph{textualis formata} that
was the prevalent late-Medieval script in Germany for the most
valuable manuscripts. Thirty years earlier Gutenberg had used the same
script for his famous Bibles.

\begin{figure}[t]
  \centering
  \sample{ABCDEFGHIKLMNOPQRSTUVWXYZ} \\
  \sample{{\addfontfeature{RawFeature=-liga;-calt;}abcdefghijklmnopqr\rotundar{}ſstuvwxyz}} \\
  \caption{The basic alphabet. Capital letters \emph{W}, \emph{Y}, and
    \emph{Z} do not occur in \emph{Missale Aboense} so they have been
    made from scratch.}
  \label{fig:alpha}
\end{figure}

\emph{Missale} also has a number of large initials that are mostly set
in the Lombardic style. This font includes the initials that were
printed using leaden slugs among its symbols, but leaves out the large
woodcuts as well as most of the hand-written ones.

Old printing inks spread more and more randomly than the modern ones.
This means that two impressions created by the same slug are not identical.
In the design I aimed for a look that is somewhat spread. Most corners
are rounded and there are few sharp edges.

\begin{figure}
  \centering
  \begin{tabular}{cccccc}
    \initsample{\msinit{A}} & \initsample{\msinita{A}}  &
    \initsample{\msinit{B}} & \initsample{\msinit{C}} &
    \initsample{\msinit{D}} & \initsample{\msinita{D}} \\
    A & A & B & C & D & D \\
    \initsample{\msinit{E}} & \initsample{\msinit{F}} &
    \initsample{\msinit{G}} & \initsample{\msinit{H}} &
    \initsample{\msinit{I}} &\initsample{\msinita{I}} \\
    E & F & G & H & I & I \\
    \initsample{\msinit{J}}  &
    \initsample{\msinit{K}} & \initsample{\msinit{L}} &
    \initsample{\msinit{M}} & \initsample{\msinit{N}} &
    \initsample{\msinit{O}} \\
    J & K & L & M & N & O  \\
     \initsample{\msinit{P}} & \initsample{\msinit{Q}} & \initsample{\msinit{R}} &
    \initsample{\msinit{S}} & \initsample{\msinita{S}} &
    \initsample{\msinit{T}} \\
    P &Q & R & S & S & T  \\
    \initsample{\msinit{U}}  & \initsample{\msinit{V}} & \initsample{\msinit{W}} &
    \initsample{\msinit{X}} & \initsample{\msinit{Y}} &
    \initsample{\msinit{Z}} \\
    U & V & W & X & Y & Z \\
  \end{tabular}
  \caption{Lombardic initials. Several letters have two alternate
    forms. These correspond with the larger size of printed
    \emph{Missale Aboense} initials except for \emph{X} that occurs
    only in the smaller size and \emph{W}, \emph{Y}, and \emph{Z} that
    do not occur at all. The variant form for \emph{I} is based on a
    hand-written initial and it needs to be used in a very large size.}
  \label{fig:lombardic_initials}
\end{figure}

The set of symbols in the 15th century textura is very different from
what we use nowadays. It retained many of the abbreviations that were
in common use during middle ages while many modern symbols are
missing. Most notably Ghotan didn't use number symbols preferring
instead to either spell the numbers out or write them using Roman
numberals. While I designed a few missing glyphs for the font (most
important being \emph{W}, \emph{Y}, and \emph{Z} that don't occur in
the book), I decided against including numbers.\footnote{Largely
  because of lack of suitable samples for 15th century textura
  numbers.} This means that Missaali is even less suitable for setting
body texts of ordinary documents than blackletter in general. However,
the limitation is not serious for the intended uses of the font.

Most of the symbols in the font cannot be found from modern keyboard
layout charts. They can be accessed by turning on a suitable
combination of OpenType features that are present in Missaali. They
may also be entered directly if the program used supports inputting
arbitrary unicode code points. All symbols in the font have code
points assigned to them, and those that are not present in the unicode
standard are put into the private use area.\footnote{Some symbols have
  duplicates that are not assigned code points to make searching
  resulting text files easier. See page~\pageref{pg:symbols_start} for
  details.} Where possible, I have put the symbols to the same places
as \emph{MUFI character recommendation 3.0}.\footnote{MUFI is the
  Medieval Unicode Font Initiative (\texttt{http://www.mufi.info})
  whose purpose is to bring Medieval scribal symbols into unicode.}
However, Missaali contains only a small part of the full
recommendation and it has some symbols that are not present in it.


\subsection{OpenType}\index{OpenType}

OpenType is a technology that combines two different approaches of
creating fonts under one package: Microsoft's \emph{true type}
(OpenType/TT) and Adobe's \emph{type 1} (OpenType/PS or OpenType/CFF)
fonts. The PS technology is older and it originated in the 80s for
defining digital fonts for printing. The true type fonts were designed
primarily for high-quality screen fonts for Windows. 

Missaali is a PS font that has been designed explicitly for printing
and it is not very usable a screen font. The lack of numbers is one
obstacle, but more seriously it is difficult to distinguish between
strings of \emph{m}, \emph{n} and \emph{u} letters in small size. It
works best in a large size.

\paragraph{OpenType features}\index{feature}

OpenType fonts may contain a number of \emph{features} that alter
their behavior. The most common ones that are used with European
languages are ligature substitutions where two individual characters
are replaced by a glyph that contains both of them. For example, in
textura the letters 'd' and 'o' are fused together when they occur as
a pair:

\newbox\foobox
\setbox\foobox\hbox{\sample{o}}
\newcommand{\aqsample}[1]{\midalign{\foobox}{#1}}

\begin{center}
  \sample{d} \aqsample{+}  \sample{o} \aqsample{=} \sample{do}
\end{center}

Different programs use different conventions for selecting features.
For example, in Photoshop the features are specified in a well-hidden
menu that's shown in figure~\ref{fig_photoshop}.

\begin{figure}
  \centering
    \includegraphics[width=8cm]{pics/pic-1.png} \\
    \includegraphics[width=8cm]{pics/pic-2.png} \\
  \caption{Photoshop OpenType menu}\label{fig_photoshop}
\end{figure}

Unfortunately many programs offer only a limited support for
features.\footnote{I myself use XeLaTeX that supports arbitrary
  combinations of features and Photoshop that doesn't.} In particular,
they often allow the user to choose them from a predefined set of
feature identifiers (called \emph{tags}) and different programs have
different sets of allowed tags. For this reason Missaali implements
most of the features with two different tags: as a stylistic set that
can be used with MS Word and as a randomly selected unrelated tag that
Photoshop recognizes. Figure~\ref{fig:word} shows the advanced tab of
Word 2010 that can be used to change the set of features in
use.\footnote{I have very little experience in using Word and I
  haven't debugged \Missaali\ with it except to note that it doesn't
  work in the legacy document modes. So, if you have problems with
  Word start by ensuring that the document mode is at least Word 2010
  and that the "use contextual alternates" checkbox is marked. Other
  than that I can't help.} Table~\ref{tab:feature_summary} has a
summary of all available features and they will be described in detail
later.
\begin{table}
  \centering
  \begin{tabular}{l|c|c|c|l}
    \textbf{Feature} & \textbf{All?} & \textbf{Tag} & \textbf{Alt.} & \textbf{Photoshop menu} \\
    \hline
    Long \emph{s} and rotunda \emph{r} & $\times$ & \feature{calt} &
    \feature{calt} & Contextual alternates \\
    Textura ligatures & $\times$ & \feature{liga} & \feature{liga} & Standard ligatures \\
    Old æ and œ & $\times$ & \feature{hist} & \feature{swsh} & Swash
    \\
    Ghotan's abbreviations & & \feature{dlig} & \feature{dlig} &
    Discretionary ligatures \\
    Scandinavian ligatures &  & \feature{ss01} & \feature{subs} &
    Subscript \\
    \hline
    Additional abbreviations & & \feature{ss02} & \feature{ornm} &
    Ornaments \\
    Alternate abbreviations & & \feature{ss03} & \feature{frac} &
    Fractions \\
    Abbreviation marks & & \feature{ss04} & \feature{salt} &
    Stylistic alternates \\
    Add tilde & & \feature{ss05}  & — &  \\
    Add dieresis & & \feature{ss06} & — &  \\
    \hline
    Add r-abbreviation & & \feature{ss07} & — & \\
    Add shorter tilde & & \feature{ss08} & — & \\
    Add a-abbreviation & & \feature{ss09} & \feature{onum} & Oldstyle
    \\
    Add ring & & \feature{ss10} & — & \\
    Add diagonal slash & & \feature{ss11} & \feature{sups} &
    Superscript \\
    \hline
    Add ur-abbreviation & & \feature{ss12} & — & \\
    Add horizontal strike & & \feature{ss13} & — & \\ 
    Add loop & & \feature{ss14} & — & \\
    Use liturgical symbols & & \feature{ss15} & \feature{smcp} & Small
    caps \\
    Old Finnish ortography & $\times$ & \feature{ss16} &
    \feature{ordn} & Ordinals \\
    \hline
    Alternate punctuation & $\times$ & \feature{ss17} & — &
    \\
    Use po-ligature & $\times$ & \feature{ss18} & — &\\
    Alternate form for z & $\times$ & \feature{ss19} & — & \\
    Initial capitals & & \feature{ss20} & \feature{titl} &     Titling alternates \\
    Add dot accent & & \feature{ss21} & — & \\
    \hline
    Alternate form for G & $\times$ & \feature{ss22} & — & \\
    Alternate initials & & \feature{ss23} &  — & \\
  \end{tabular}
  \caption{OpenType features. Those marked with $\times$ may be
    enabled for the whole document, the others should be turned on
    only for affected characters.\index{feature!table of}}
  \label{tab:feature_summary}
\end{table}


\begin{figure}
  \centering
  \includegraphics[width=12cm]{pics/word_features.png}
  \caption{MS Word 2010 font feature dialog where selecting
    'Historical and Discretionary' option for ligatures turns on
    abbreviation for \emph{que}. For proper working of the font you
    should always leave the 'use contextual alternates' mark selected.
  }
  \label{fig:word}
\end{figure}

\subsection{Open Font License v.1.1}\index{Open Font License}

\Missaali\ may be used and distributed under the conditions of the
Open Font License v.1.1. The full text of the license is at the end of
this document. It and answers to frequently asked questions about it
are available at \texttt{http://scripts.sil.org/OFL}. The most
important features of OFL are:
\begin{enumerate}
\item You may freely use the font in your documents, including
  embedding it in it.

\item You may freely give the font to whoever you want, but you should
  give the whole package (including the documentation files). You may
  also sell it as long as you comply with the few restrictions that
  are enumerated in the text of the license.

\item You may freely modify the font, but if you distribute the
  modified font you should give it a different name and release it
  under OFL.
\end{enumerate}

\subsection{Latex Project Public License version 1.3c} 

The documentation files and the LaTeX style file are distributed under
the Latex Project Public License (LPPL version 1.3c). The exact
conditions of the license are defined in the file \texttt{lppl.txt}
that should been included in the font package but is also available at
\texttt{http://www.latex-project.org/lppl.txt}.

The terms of LPPL are similar to those of the OFL. The short and
incomplete version is:
\begin{enumerate}
\item You may freely use the files however you want. 

\item You may freely distribute the package in its original form. 

\item If you modify the files, you may distribute them but you must
  document what changes you have made in the files, make it clear that
  the changed version is not the original, and distribute either the
  original version or information on how it can be obtained with it.
\end{enumerate}

\newpage
\section{Missale Aboense}

\begin{figure}
  \centering
  \includegraphics[width=12cm]{pics/missale-preface.png}
  \caption{Hand-colored woodcut from the preface of \emph{Missale
      Aboense} (reduced size). In the middle is Saint Henry treading
    on his killer Lalli. He is surrounded by bishop Konrad Bitz and
    dean Magnus Nicolai Särkilahti. Back left is the printer
    Bartholomew Ghotan. The identity of the priest back right is not
    certain, but he might be the proofreader, Daniel de Egher. The
    text starts: \emph{Reverendus in cristo pater et dominus, dominus
      Conradus Bystz dei et apostolice sedis grande praesul ecclesie
      Aboense.} — Reverend Father in Christ and Lord, Lord Konrad Bitz
    of the Apostolic See, bishop of the Turku diocese.}
  \label{fig:titlepage}
\end{figure}

\paragraph{Historical background}

About 1000 years ago the area of modern Finland was essentially a
sparsely-populated forest between Sweden and Novgorod. The inhabitants
belonged to three tribes: \emph{Suomalaiset} (Finns),
\emph{Hämäläiset} (Tavasts) and \emph{Karjalaiset} (Karelians) that
each had several subgroups. At the time the name 'Finland' meant only
the South West part where Finns lived, and it took hundreds of years
before the word acquired its modern sense and the term \emph{Finn}
came to encompass all people living in the area.

It is impossible to know exactly when Christianity first arrived in
the area as there are no reliable written sources. The first surviving
accounts about the history of Church in Finland seem to have been
composed in the late 13th century and their historical value is not
great. According to them and later medieval and early modern texts,
King Eric IX of Sweden conducted a crusade to the land of Finns
sometime around 1150 to baptize them and that he was accompanied by
bishop Henry of Uppsala. The modern view is that Eric may have
conducted a campaign in Finland—it is not certain—, but the aim was
more to obtain tribute than to spread Christianity.

Archealogical evidence suggests that first Christians were baptized in
the 6th century and that by Eric's time most people in South West
Finland were already Christian while the people living inlands were
still pagan. There were infuences from both Western and Eastern
Churches, but Finland wasn't integrated with either, yet.

What happened in the mid 12th century was that the Swedish crown
extended its rule over the Finns and during the next century also over
the Tavasts. The area of Karelians was divided between Sweden and
Novgorod with Novgorod getting the larger share.

\paragraph{The Legend of Saint Henry} The legend of Saint Henry dates
from the late 13th century. According to it, Henry was an English
clergyman who had moved to Sweden and became the bishop of Uppsala. He
urged King Eric to go on to the crusade and joined the venture
himself, baptizing the first Finns and then staying behind in Finland
to organize the church when the king returned to Sweden.

The legend further tells that he was killed the next winter. The
killer was a murderer who was enraged at Henry because he had imposed
'church discipline' on him in addition to the penalty that the law
decreed from murder. The text doesn't detail the exact natures of
either punishment, but at the time killers typically paid weregild to
the family of the victim. The legend tells that Lalli got a divine
punishment for his deed: he put the bishop's cap on his head, and when
he removed it, his whole scalp came off with it. This was considered
to be the first miracle attributed with St Henry. 

Later sources add details to the short legend. They make the mass
baptism to happen at the spring of Kupittaa that nowadays lies inside
the borders of the city of Turku, and also give the name of the killer
and the place of the murder. The murder happened on the ice of Lake
Köyliö and the killer was called Lalli.\footnote{It is interesting to
  note that \emph{Lalli} is a form of \emph{Laurentius} so it is a
  Christian name, which speaks against those versions of the story
  that describe him as a pagan.} They also give two alternate
explanations for the motive: he either didn't want to pay tithes to
the church or he got angry when Henry requested food from his farm.
The divine retribution against Lalli was also extended, and the
stories tell that he was eaten alive by mice on an island of
Hiirijärvi lake.

Almost nothing is known about the historical bishop Henry. Even his
whole existence is uncertain as it has proven impossible to identify
him from contemporary sources. Whether he was a historical person or a
completely invented saint is an open question. If he was real, it is
very unlikely that he was a bishop at Uppsala because in that case he
should have left a written trace of himself.

There is an alternative explanation for his origins. The English
cardinal Nicolaus Albanus\footnote{Known also by his English name
  Nicholas Breakspear. He later became Pope Adrianus IV.} was a papal
legate in Scandinavia in early 1150s. His mission was to integrate the
Scandinavian episcopal sees tighter with the Catholic church. It is
possible that Henry was one of his assistants who stayed behind when
Nicolaus returned to Rome. It is plausible but far from certain that
he left Henry behind to organize the churches in Finland.

What is certain is that St. Henry's cult existed in the latter half of
the 13th century and in 1300 or shortly before bones that were
believed to belong to him were moved from the village of Nousiainen to
the new Turku cathedral. Even though he was venerated as a saint in
Finland and Sweden, he was never officially canonized by the Vatican.

Henry's skeleton vanished during the Great Northern War of 1700–21. It
had been moved to a side chamber of the cathedral after the
reformation, and Czar Peter~I gave an order to send the bones to
St.\,Petersburg. After that their whereabouts are unknown. Today only
a small piece of tibia can certainly be attributed to him. It survived
by being buried with Blessed Bishop Hemmingus (c.1290–1366). Hemmingus
would probably have been canonized as a saint but the reformation
stopped the process.

There is also a skull that was dated to the 12th century using
radiocarbon dating that was found hidden inside the cathedral. It is
possible that the person who hid it believed that it was Henry's
skull. The skull is over 100 years older than the cathedral so it
cannot belong to anyone whose first burial was inside it.

\paragraph{Turku diocese} It is not known for certain when the Turku
diocese was really established, but it existed at the beginning of the
13th century. During the middle ages the diocese was a part of the
archdiocese of Uppsala but it did not follow the same liturgical
tradition. Instead, Turku diocese was heavyly influenced by the
Dominican tradition starting from the time of bishop Thomas
(c.1220–45) who belonged to the order. The influence was strengthened
by the Dominican convent that was built in Turku in 1249.

Finland was far from the learned centers of Europe. While several
dozens of Finnish clergymen studied in the continental universities
over the years, most priests were trained at the Turku cathedral
school and their education was not very deep and the church could not
root out all old pagan traditions. There was a scriptorium that
produced many illuminated manuscripts, but they couldn't write enough
for all parishes in the diocese and imported books were excessively
expensive.

In the 1480s bishop Konrad Bitz\footnote{A bishop between 1460–89.}
and dean Magnus Nicolai Särkilahti\footnote{A dean in 1465–1489, then
  bishop 1489–1500.} started a project to improve the literary
situation of the parishes in the diocese. They started by organizing a
group order for \emph{Psalterium cum Canticis} that was printed by
Bartholomew Ghotan, and next started to plan for a missal.


\paragraph{Bartholomäus Ghotan}\index{Ghotan, Bartholomew}
Ghotan was the foremost authority on printing liturgical texts at the
time. He had previously worked as a priest at Magdeburg where he
learned about book printing and decided to start a new career. His
clerical training made him an expert on the subject of liturgical
books and he printed the very first missal, \emph{Missale
  Praemonstratense} there in 1479. 

Ghotan moved to Lübeck that had become the center of bookprinting in
Northern Europe and continued to make liturgical books. In 1486 he was
invited to Stockholm where he printed several books for Swedish
dioceses, including \emph{Missale Strengnense}. In 1487 he had a
disagreement with the government led by regent Sten Sture and returned
to Lübeck where he printed \emph{Missale Aboense} the next year.

Bartholomew Ghotan continued printing books in Lübeck for several
years before deciding to open a printing shop at Novgorod in 1493. His
timing proved to be unfortunate as a short while later the conflict
between the Hanseatic League and Czar Ivan the Great intensified and
culminated in the desctruction of the Hansa office Peterhof. Ghotan
vanishes from historical records and he probably died in Novgorod in
1494. Some suspect that he was killed in the disturbances but the real
cause of death is unknowable.

\paragraph{Printing \emph{Missale Aboense}}

It seems likely that the project for Turku missal was started during
Ghotan's time in Stockholm. Even though Bitz was the official head of
the project, the practical side was handled mostly by Magnus Nicolai.
From the beginning Ghotan intended to print a multi-purpose book. The
liturgy of Turku was close enough to the standard Dominican tradition
that it would be possible to sell the book also to the order.

A manuscript for the book was prepared in Turku and it was sent to
Daniel de Egher for review and corrections. He was a Dominican and the
Professor of Holy Theology at the University of Paris. In his proofing
he used as a source a Dominican missal printed in Venice in 1484,
which explains how the feast days of several Hungarian saints ended up
in a Finnish calendar. Though the text was checked, the quotations
from the Bible have some small differences from the text of official
version of the Latin \emph{Vulgate} Bible.

The printing started in 1488. Ghotan apparently first printed the
general Dominican missal and then he customized it for Turku by adding
a prologue, a different calendar, and a section of texts specific to
the diocese. The prologue was dated on August 17, 1488.

Copies were printed on both parchament and paper. The total size of
the printing is not known but was probably in the range of 120–200.
Twenty copies still exist and most are incomplete. Only one parchament
copy is still in the original bindings. It is the copy of the Halikko
parish that was stolen by Danes in 1509 and it is now at the Kongelike
Biliotek in Copenhagen. Three complete parchament copies have been
reconstructed from separated pages that had been torn out and used as
covers for archival documents in the late 16th century. The rest
sixteen books are incomplete copies printed on paper.

\begin{figure}
  \centering
  \rule{9.92cm}{0.1pt}
  \includegraphics[width=9.82cm]{pics/missale-normal.png}
  \rule{9.92cm}{0.1pt}
  \caption{Sample from \emph{Missale Aboense} containing text in
    normal and small sizes, several abbreviations, and four initials.
    The red ones are two sizes of printed Lombardian initials while
    the two blue ones are drawn by hand. Note how the other initials
    are inside the text block but the flourished initial \emph{I} in
    \emph{In} is drawn completely in the gutter between the columns.
    Also note now the abbreviated rubric \emph{Oratio}
    (\missaali{\rmstdt{Or}o}) is set at
    the right end of the first line of the section.
    \label{fig:normalfont}}
\end{figure}

The book was printed in large folio size with large letters. There are
four different text sizes in use:
\begin{itemize}
\item The main body of text that is approximately 26 points in modern
  terms;

\item Smaller main body text approximately 22 points in size. This was
  set with the same line height as the main font;

\item Small text in a simpler typeface (\emph{textus preacissus}) of
  about 14 points; and

\item Large text for the initial part of the mass (\emph{Te igitur})
  approximately 48 points in size. 
\end{itemize}

\begin{figure}
  \centering
  \rule{8.84cm}{0.1pt}
  \includegraphics[width=8.84cm]{pics/missale-tinyfont.png}
  \rule{8.84cm}{0.1pt}
  \caption{The smallest \emph{Missale Aboense} font that is based on
    \emph{textus preacissus} instead of \emph{textus quadrata}.}
  \label{fig:smallfont}
\end{figure}

\begin{figure}
  \centering
  \rule{10.26cm}{0.1pt}
  \includegraphics[width=10.26cm]{pics/missale-largefont.png}
  \rule{10.26cm}{0.1pt}
  \caption{A large woodcut initial and the start of the \emph{Te
      igitur} in large font size. The initial is not colored in the
    facsimile reprint of \emph{Missale} but it probably was
    hand-colored in many copies.}
  \label{fig:largefont}
\end{figure}


The main text was printed with black ink with red captions,
instructions, and initials. Ghotan used four different types of
initials:
\begin{itemize}
\item Large woodcuts of initials;

\item Two lines high printed Lombardic initials;

\item One line high printed Lombardic initials;

\item Leaving a two or three-line square empty space for hand-drawn
  initials.

\end{itemize}

Many of the surviving copies have blue hand-drawn initials that are
very close in style with each other, so it is likely that there was a
workshop — probably in Lübeck — that initialed some of the copies
before they were sent to their recipients. 

The woodcut initials as well as the few woodcut illustrations were
colored by hand, and some copies had also a bit of burnished gold
added to illustrations.


\section{Textura}

In late medieval times different types of texts were written using
different scripts. The most refined hand that was used for the most
valuable — and usually religious — German manuscripts was
\emph{textualis formata} in its \emph{textus quadrata} form. I will be
calling it \emph{textura} in this document. Since it was considered
the highest form of writing, it was natural that Johannes Gutenberg
modeled his Bible typefaces on it and Bartholomew Ghotan followed his
example for his liturgical books.

Textura is a very angular script that has high contrast between parts of
the letter. There are wide usually vertical strikes and thin usually
diagonal hairlines. The letters are spaced closely together and while
the word spaces are not very wide, they are wide enough to clearly
distinguish the words.
\begin{center}
  \sample{Quick red fox jumped et cetera.}
\end{center}

One of the distinguishing features of textura is that the letter
\emph{f} does not have a descender like it has in cursive Gothic
scripts. Another feature is that the vertical strikes of lowercase
letters have small quadrangles at their feet. This is most notable in
the letter \emph{m}.

\paragraph{Text body} Textura was set in justified rectangular blocks.
The text was set in continuous lines and there was no line break
between paragraphs. A new paragraph was marked by either a colored
initial or by the paragraph symbol. The last line was also justified
if at all possible. Justification was done by a combination of
increasing word spaces and by using abbreviations. When a word was
split at the end of the line, the hyphen was often left out. For
example, in figure~\ref{fig:normalfont} \emph{leticia} has a hyphen
but \emph{hereditate} in the following row doesn't. When used, a
hyphen is two short diagonal strokes.

\paragraph{Fusions} 

When writing textura two adjacent bowls of letters were fused together
into one stroke. Ghotan fused together the following combinations:
\begin{center}
  \begin{tabular}{ccccccc}
  \sample{be} & \sample{bo} & \sample{de} & \sample{do} & \sample{pe}
   & \sample{ve} \\    
  be & bo & de & do & pe & ve \\
  \end{tabular}
\end{center}

The surprising thing is that Ghotan didn't use the \emph{po} fusion
systematically. It is used in some places of the text, but usually
they are kept separate. He also fused together double \emph{p}:
\begin{center}
  \sample{pp}
\end{center}


\paragraph{Letter variant forms}

Like all Gothic scripts, textura has two forms of the lower case
letter \emph{s}: 'short' \missaali{s} and 'long' \missaali{ſ}. It also
has two forms of \emph{r}: the 'straight' \missaali{r} and the 'round'
\missaali{\rotundar} that is usually called \emph{r}-rotunda.

\begin{center}
  \begin{tabular}{cccc}
    \sample{s} & \sample{ſ} & \sample{r} & \sample{\rotundar} \\
    Short \emph{s} & Long \emph{s}  & Straight \emph{r} & Rotunda
    \emph{r} \\
  \end{tabular}
\end{center}

The rules for the use of long and short \emph{s} became quite complex
over centuries, but Ghotan followed the simple medieval rules: short
\emph{s} is used only if it is the last letter of the word and the
long \emph{s} is used in all other places.\footnote{\Missaali\ uses
  the opentype feature \feature{calt} to automatically insert long
  \emph{s} and \emph{r}-rotunda in places where Ghotan would have used
  them. If you want to follow a later convention, you need to turn the
  feature off.} For example, the Latin word \emph{servus} was written
as:
\begin{center}
  \sample{servus}
\end{center}

The round \emph{r} was used after letters whose right edge had a bowl
except when the \emph{r} was at the end of the word. In textura those
letters were traditionally \emph{b}, \emph{d}, \emph{h}, \emph{o},
\emph{p}, \emph{v}, and \emph{y},\footnote{Other Gothic scripts used
  different set of letters for \emph{r}-rotunda selection.} but Ghotan
used a straight \emph{r} after \emph{h} and Missaali follows that
convention. He also used rotunda \emph{r} at the end of the word if
the next to the last letter was not \emph{u} or \emph{v}:
\begin{center}
  \begin{tabular}{ccccc}
    \sample{soror} & \sample{quodra} & \sample{igitur}\\
  \end{tabular}
\end{center}
Ghotan wasn't completely consistent in his usage and you occasionally
find a rotunda \emph{r} in a place where you would expect a straight
\emph{r} or vice versa. 

\paragraph{The case of \emph{u} and \emph{i}}

Textura had two other letters with variant forms. At the time \emph{v}
and \emph{u} were considered to be the same letter and similarly for
\emph{i} and \emph{j}. Ghotan used \emph{u} and \emph{v}
interchangeably and there is no clear rule for which letter he
selected except that \emph{U} was used almost always for the capital
letter. The capital \emph{V} was quite rare and it doesn't fit the
other capitals so well. 
\begin{center}
  \begin{tabular}{cccc}
    \sample{U} & \sample{u} & \sample{V} & \sample{v} \\
    U & u & V & v \\
  \end{tabular}
\end{center}

Symbols \emph{i} and \emph{j} were also considered to be forms of the
same letter but their use was more consistent. Ghotan used \emph{j}
for the last letter in a sequence of \emph{i}s. This occurred most
often in Roman numerals:
\begin{center}
  \begin{tabular}{ccccc}
    \sample{i} & \sample{ii} & \sample{iii} & \sample{iiii}
  \end{tabular}
\end{center}
For an added complexity, some scribes used \emph{ÿ} in place of
\emph{ij}. However, Ghotan didn't do that. 


\paragraph{Punctuation} There were no strict rules for applying
punctuation in the middle ages. Texts were usually read out loud and
punctuation was used to mark out pauses of different lengths. In the
body of text Ghotan used three different lengths of pauses and a
question mark:
\begin{center}
  \begin{tabular}{ccccc}
  \sample{:} & \sample{\addfontfeature{RawFeature=+ss17;}.} &
  \sample{.} & \sample{?}
  \end{tabular}
\end{center}
Ghotan used a colon for a shortish pause and centered dot for where we
would normally write an ordinary period. Both colon and dot were
usually placed so that there was as much space on both sides of it. He
used an ordinary period usually at the end of paragraphs. In several
places he used also the virgule \missaali{/} or a vertical bar
\missaali{|} for punctuation.

The font has also few other medieval punctuation symbols that were in
common use. They are placed in the positions of different-length dashes: 
\begin{center}
  \begin{tabular}{cccccccccc}
    \sample{‒} & \sample{–} & \sample{—}
  \end{tabular}
\end{center}


\newpage

\section{Latin Abbreviations}

Medieval scribes used many abbreviation symbols and there were
different abbreviation systems in different places and for different
scripts. They were particularly prominent in monastic and academic
texts, but also liturgical works had them. They fell slowly out of use
with the increase of literacy when the writers could not rely on their
non-specialist audience to understand them and also because many of
the symbols were specifically tailored for Latin and did not have much
use in vernacular texts. Early printed books used old abbreviations
but they fell out of use in the 16th century. Some holdouts survived
longer: using tilde to mark missing nasal consonants is relatively
common even in the 17th century in some areas and the ampersand is
still in use.

Bartholomew Ghotan used two different systems of abbreviations in
\emph{Mis\-sale Aboense}: one for the main body of text and another
for the small 14~pt font. There is not much text in the small font and
it isn't possible to reconstruct his full system from it, but the
system for the larger font is quite complete. Some of the symbols that
he used were used in a different sense in different abbreviation
systems. I will first describe the system that he used and then add to
that some commonly-used symbols that are not found in \emph{Missale}.

\begin{table}
  \centering
  \begin{tabular}{clcllccccc}
     \multicolumn{5}{l}{\textbf{Abbreviations with specific meanings}}\\
     Symbol & Meaning & Feat & Entering & Notes \\
    \hline
    \sample{\msabbr{bis}} & bis & \feature{dlig} & bis &
    occasionally general \\
    \sample{\msabbr{con}} & con & \feature{dlig} & con \\
    \sample{\msabbr{et}} & et & \feature{dlig} & et \\
    \sample{\msabbr{etc}} & etc & \feature{dlig} & etc & et cetera\\
    \sample{\msabbr{hoc}} & hoc & \feature{dlig} & hoc & also
    general \\
    \hline
    \sample{\msabbra{m}} & m & \feature{dlig} & m & occasional final
    form \\
    \sample{\msabbr{mur}} & mur & \feature{dlig} & mur \\
    \sample{\msabbr{Per}} & Per & \feature{dlig} & Per \\
    \sample{\msabbr{per}} & per & \feature{dlig} & per & occasionally
    \emph{por} \\
    \sample{\msabbr{prae}} & prae & \feature{dlig} & prae & also general \\
    \hline
    \sample{\msabbr{præ}} & prae & \feature{dlig} & præ & entered
    with æ-ligature \\
    \sample{\msabbr{Pro}} & Pro & \feature{dlig} & Pro \\
    \sample{\msabbr{pro}} & pro & \feature{dlig} & pro \\

    \sample{\msabbr{quae}} & quae & \feature{dlig} & quae \\
    \sample{\msabbr{quam}} & quam & \feature{dlig} & quam \\
    \hline
    \sample{\msabbr{que}} & que & \feature{dlig} & que \\
    \sample{\msabbr{qui}} & qui & \feature{dlig} & qui \\
    \sample{\msabbr{quid}} & quid & \feature{dlig} & quid \\
    \sample{\msabbr{quia}} & quia & \feature{dlig} & quia \\
    \sample{\msabbr{quo}} & quo & \feature{dlig} & quo \\
     \hline
    \sample{\msabbr{qvo}} & quo & \feature{dlig} & qvo \\
     \sample{\msabbr{quod}} & quod & \feature{dlig} & quod \\
     \sample{\msabbr{qvod}} & quod & \feature{dlig} & qvod \\
     \sample{\msabbr{quoque}} & quoque & \feature{dlig} & quoque \\
     \sample{\msabbr{rum}} & rum & \feature{dlig} & rum \\
     \hline
     \sample{\msabbr{se}} & se & \feature{dlig} & se \\
     \sample{\msabbr{sed}} & sed & \feature{dlig} & sed & also other     syllables with \emph{se}. \\

     \sample{\msabbr{tur}} & tur & \feature{dlig} & tur \\
     \sample{\msabbr{us}} & us & \feature{dlig} & us & \\
     \sample{\msabbr{vs}} & us & \feature{dlig} & vs \\
     \hline
     \sample{\msabbr{ver}} & ver & \feature{dlig} & ver & occasionally
     \emph{vir} \\
  \end{tabular}
  \caption{Bartholomew Ghotan's specific abbreviations in \protect\Missaali.}
  \label{tab:ghota_abbr}
\end{table}


\begin{table}
  \centering
  \begin{tabular}{llllp{6cm}llccccc}
    \multicolumn{5}{l}{\textbf{General abbreviations}}\\
    Symbol & Example &  Meaning & Feat & Notes \\
    \hline
    Tilde & \sample{\mstd{a}} & General & \feature{ss04} & usually \emph{m} or \emph{n} \\
    Dieresis & \sample{\rmstd{r}} & General & \feature{ss06} &
    general abbreviation, an alternative to tilde \\
    Dot & \sample{\mstd{h}} & General & \feature{ss21} & 
    general abbreviation, an alternative to tilde \\
    Loop & \sample{\mstd{l}} & General & \feature{ss14} & 
    general abbreviation, an alternative to tilde. The looped l
    (\missaali{l}) is available also as \feature{dlig} ligature
    \emph{lum}.  \\
    Squashed \emph{a} & \sample{\mstda{g}} & A-abbr
    & \feature{ss09} &  Something with \emph{a} \\
    Tailed dot & \sample{\mstdr{e}} & R-abbr & \feature{ss07} &
    Something with \emph{r} \\
    Ring & \sample{\mstdo{q}} & O-abbr & \feature{ss10} &
    Something with \emph{o} \\
     \end{tabular}
  \caption{Bartholomew Ghotan's general abbreviations in \protect\Missaali.}
  \label{tab:ghota_abbr_general}
\end{table}

There are two basic ways to mark abbreviations: either by adding small
marks over the letters to signify places where things are left out or
to add new symbols to the running text. 

\begin{table}
  \centering
  \begin{tabular}{clcllccccc}
     \multicolumn{5}{l}{\textbf{Alternate set of abbreviations}}\\
     Symbol & Meaning & Feat & Entering & Notes \\
    \hline
    \sample{\msabbralt{bis}} & bis & \feature{ss03} & bis &
    occasionally general \\
    \sample{\msabbralt{con}} & con & \feature{ss03} & con \\
    \sample{\msabbralt{et}} & et & \feature{ss03} & et \\
    \sample{\char`^^^^a770} & et & \feature{ss03} & \emph{X}et &
    in middle of a word \\
    \sample{\msabbralt{hoc}} & hoc & \feature{ss03} & hoc & \\
    \hline
    \sample{\msabbralt{lum}} & lum & \feature{ss03} & lum & \\
    \sample{\msabbralt{mur}} & mur & \feature{ss03} & mur & \\
    \sample{\msabbralt{par}} & par & \feature{ss03} & par & \\
    \sample{\msabbralt{Per}} & Per & \feature{ss03} & Per \\
    \sample{\msabbralt{per}} & per & \feature{ss03} & per & \\
    \hline
    \sample{\msabbralt{prae}} & prae & \feature{ss03} & prae & \\
    \sample{\msabbralt{Pro}} & Pro & \feature{ss03} & Pro \\
    \sample{\msabbralt{pro}} & pro & \feature{ss03} & pro \\
    \sample{\msabbralt{prop}} & prop & \feature{ss03} & prop \\
    \sample{\msabbralt{qua}} & qua & \feature{ss03} & quaa \\
    \hline
    \sample{\msabbralt{quae}} & quae & \feature{ss03} & quae \\
    \sample{\msabbralt{quam}} & quam & \feature{ss03} & quam \\
    \sample{\msabbralt{que}} & que & \feature{ss03} & que \\
    \sample{\msabbralt{qui}} & qui & \feature{ss03} & qui \\
    \sample{\msabbralt{quo}} & quo & \feature{ss03} & quo \\
    \hline
    \sample{\msabbralt{ris}} & ris & \feature{ss03} & ris \\
    \sample{\msabbralt{rum}} & rum & \feature{ss03} & rum \\
    \sample{\msabbralt{sec}} & sec & \feature{ss03} & sec & also sed  \\
    \sample{\msabbralt{sed}} & sed & \feature{ss03} & sed & also sec  \\
    \sample{\msabbralt{ter}} & ter & \feature{ss03} & ter  & \\
    \hline
    \sample{\msabbralt{tur}} & tur & \feature{ss03} & tur & \\
    \sample{\msabbralt{uer}} & uer & \feature{ss03} & uer \\
    \sample{\msabbralt{us}} & us & \feature{ss03} & us & \\
    \sample{\msabbralt{ver}} & ver & \feature{ss03} & ver 
  \end{tabular}
  \caption{Alternate set of abbreviations.}
  \label{tab:alternate_abbr}
\end{table}



\subsection{Abbreviation marks}

Ghotan used at least five semantically different abbreviation marks
that were written over letters and a few of the symbols have also
variant forms that are used over some specific letters. The basic
marks are:
\begin{center}
  \begin{tabular}{cccccccc}
    \sampleabbr{\char`\^^^^02dc} & \sampleabbr{\char`\^^^^02c6} & \sampleabbr{\char`\^^^^00b4} &
    \sampleabbr{\char`\^^^^02c7} & \sampleabbr{\char`\^^^^030a}  \\
    abbreviation & \emph{a}-abbr. & \emph{r}-abbr. &
    \emph{ur}-abbr. & \emph{o}-abbr. \\
  \end{tabular}
\end{center}

\paragraph{Tilde} The tilde is used in three different senses:
\begin{enumerate}
\item to denote that a nasal consonant (\emph{m} or \emph{n}) is left
  out;
\item over \emph{p} and \emph{q} it denotes syllables \emph{prae} and
  \emph{quae}; and 
\item as a general abbreviation mark that is used when the
  abbreviation doesn't have a specific notation. 
\end{enumerate}

\begin{center}
  \begin{tabular}{lcl}
    \sample{omnes} & \aqsample{$\Longrightarrow$} &
    \sample{o\mstd{m}es} \\
    \sample{praesul} & \aqsample{$\Longrightarrow$} &
    \sample{\msabbr{prae}sul} \\
    \sample{dominus} & \aqsample{$\Longrightarrow$}&
    \sample{d\mstd{n}s}
  \end{tabular}
\end{center}

Here the tilde is used in the first sense in \emph{omnes}, the second
in \emph{praesul} and in the third in \emph{dominus}. The mark has a
different form over some letters such as \emph{d} and \emph{r}.
Letters with ascenders use different conventions to represent the
general abbreviation. In \emph{h} and \emph{b} it is replaced by a dot
placed on the right side of the letter, in \emph{l} it forms a
loop at the right top, and in \emph{S} it is a diagonal stroke:
\begin{center}
  \begin{tabular}{cccccccc}
    \sample{\mstd{d}} & \sample{\rmstd{r}} &  \sample{\mstd{b}} &
    \sample{\mstd{h}} & \sample{\mstd{l}} & \sample{\mstd{S}}
  \end{tabular}
\end{center}

\paragraph{A-abbreviation} The symbol for \emph{a}-abbreviation is
very close to the tilde but it has two strokes. It represents an
\emph{a} that is left open and squashed flat. It is used to signify
that the abbreviated portion contains an \emph{a} somewhere in it. For
example, 
\begin{center}
\begin{tabular}{lcl}
  \sample{magna} & \aqsample{$\Longrightarrow$} &
  \sample{m\mstda{g}a} \\
  \sample{psalmis} & \aqsample{$\Longrightarrow$} &
  \sample{p\mstda{s}}
\end{tabular}
\end{center}

\paragraph{R-abbreviation} A dot that ends in a short diagonal tail is
used to signify that the abbreviated part contains a letter \emph{r}.
Ghotan used it most often for denoting the \emph{er}-syllable. For
example, \emph{propter} could be use the \emph{r}-abbreviation in two
different ways:
\begin{center}
\begin{tabular}{lcl}
  \sample{propter} & \aqsample{$\Longrightarrow$} &
  \sample{prop\mstdr{t}} \\
  \sample{propter} & \aqsample{$\Longrightarrow$} &
  \sample{propt\mstdr{e}}
\end{tabular}
\end{center}

The dot has a different form over \emph{u} and \emph{v} where it looks
like a squished and turned tilde and in both cases it denotes the
syllable \emph{ver}:
\begin{center}
  \begin{tabular}{lclcccccccccc}
    \sample{universitas} & \aqsample{$\Longrightarrow$} & \sample{uni{\msabbr{ver}sitas}}
  \end{tabular}
\end{center}

In the instructions for mass \emph{ver} set in red denotes a versicle
that the officiant sings and that is answered by the choir in
response.

\paragraph{Ur-abbreviation} Ghotan used a sideways hook over \emph{t}
and \emph{m} to denote the syllables \emph{tur} and \emph{mur}:

\begin{center}
\begin{tabular}{lcl}
  \sample{nocturnus} & \aqsample{$\Longrightarrow$} &
  \sample{noc\msabbr{tur}nus} 
\end{tabular}
\end{center}

\paragraph{O-abbreviation} The least used abbreviation mark in
\emph{Missale Aboense}\footnote{That I have noticed.} is a small ring
that usually marks that something with an \emph{o} has been left out.
Ghotan used it in two different \emph{q}-related abbreviations:

\begin{center}
\begin{tabular}{ccc}
  \sample{\msabbr{qvo}} & \sample{\msabbr{quoque}} \\
  quo & quoque 
\end{tabular}
\end{center}

\subsection{Abbreviation glyphs}

Ghotan had about a dozen special symbols for abbreviations that he
then combined with abbreviation marks and ligatures to create a
large amount of abbreviations for the text. By far the largest
group is the symbols for Latin syllables starting with a \emph{q} with
the next largest being symbols for \emph{p}. I will now go through his
abbreviations in alphabetical order by the base symbol. 

\paragraph{C} The syllable for \emph{con} is abbreviated with a turned
\emph{c} with a cedilla:
\begin{center}
  \begin{tabular}{lcl}
    \sample{contra} & \aqsample{$\Longrightarrow$} & \sample{\msabbr{con}tra}
  \end{tabular}
\end{center}

This occurs almost exclusively in the first syllable.

\paragraph{E} Missaali uses the Tironian \emph{et} symbol in place of
an ampersand. The symbol was named so after its inventor Marcus
Tullius Tiro who was Cicero's scribe.
\begin{center}
  \begin{tabular}{ccc}
    \sample{\&} \\
    \emph{et} 
  \end{tabular}
\end{center}

Ghotan used the symbol only for the word \emph{et} and it does not
occur as a part of longer words even though many scribes used it that
way.

\paragraph{M} There are no special symbols for abbreviating syllables
containing \emph{m} but Ghotan occasionally set \emph{z} in place of
\emph{m}, mimicking the custom of scribes to write the \emph{m}
sideways if it didn't otherwise fit. This form is used only at the end
of the word and it is quite rare. For example, in one place
\emph{alligatam} is written as:
\begin{center}
  \sample{alligataz}
\end{center}

In manuscripts this form was used almost exclusively at the end of a
line, but Ghotan used it also in the middle of line in places where
the word was followed by punctuation.

\paragraph{P} Three symbols based on \emph{p} occur in \emph{Missale}.
Two of them are very common and they correspond with syllables
\emph{per} and \emph{pro} while the 'squirrel tailed' \emph{prae} is
uncommon as the syllable is usually represented by \emph{p} tilde. Two
of the symbols have capital letter versions:
\begin{center}
\begin{tabular}{cccccccc}
  \sample{\msabbr{Per}} & \sample{\msabbr{per}} &
  \sample{\msabbr{Pro}} & \sample{\msabbr{pro}} &
  \sample{\msabbr{præ}/\msabbr{prae}} \\
  Per/Por & per/por & Pro & pro & prae \\
\end{tabular}
\end{center}

There's a ligature for \emph{pro} and \emph{p}. For example, the word
\emph{prophete} is often abbreviated as:
\begin{center}
  \begin{tabular}{lclcccccccc}
    \sample{prophete} & \aqsample{$\Longrightarrow$} &
    \sample{\msabbr{prop}\mstd{h}e} 
  \end{tabular}
\end{center}


\paragraph{Q} There are three different abbreviation symbols based on
the letter \emph{q} and they are combined with abbreviation marks to
create more. In addition to the three that Ghotan used, I added a
fourth symbol as a ligature even though Ghotan set it with two
separate slugs of type.\footnote{Handling ligatures in OpenType
  features is a bit simpler than handling separate glyphs.} The
\emph{q}-based symbols are:\label{pg_que}

\begin{center}
  \begin{tabular}{cccccccccccc}
    \sample{\msabbr{que}} & & \sample{\msabbr{quod}} & &
    \sample{\msabbr{quo}} & & \sample{\msabbr{quia}} \\
    que & & quod & & quo & & quia 
  \end{tabular}
\end{center}
Here \emph{quia} is my addition and Ghotan used ordinary \emph{q} and
\emph{r} rotunda to set it. The composed \emph{q}-based abbreviations
are:
\begin{center}
\begin{tabular}{cccccccccccc}
  \sample{\msabbr{quae}} &  \sample{\msabbr{qvo}} &
  \sample{\msabbr{qua}} & 
\sample{\msabbr{qui}} & \sample{\msabbr{quoque}} &
  \sample{\msabbr{quam}} & \sample{\msabbr{quid}} \\
quae & quo & qua & qui & quoque & quam & quid \\
\end{tabular}
\end{center}

It is very easy to confuse \emph{quae} with \emph{qua} and \emph{quam}
with \emph{quid} as in both cases the only difference is the shape of
the tilde.

Ghotan had two different ways to abbreviate \emph{quo} and \emph{quod}
and it is not clear to me how he chose which form to use:
\begin{center}
  \begin{tabular}{lcl}
    \sample{quo} & \aqsample{$\Longrightarrow$} &
    \sample{\msabbr{quo}/\msabbr{qvo}} \\
    \sample{quod} & \aqsample{$\Longrightarrow$} & \sample{\msabbr{quod}/\msabbr{qvod}}
  \end{tabular}
\end{center}

\paragraph{R} There are two \emph{r}-based abbreviations but one of
them occurs amost exclusively in the instructions for the mass:\label{pg_response}
\begin{center}
  \begin{tabular}{cccccccccc}
    \sample{\msabbr{rum}} & & \sample{\mstdl{r}} \\
    -rum &  & response
  \end{tabular}
\end{center}

Both of them are essentially the same symbol that denotes the suffix
\emph{rum}, but one of them is based on the straight \emph{r} and the
other in small caps rotunda. A red response symbol is very common in
the instructions, but it occurs in the text set in black in a handful
of places where it means \emph{respondit}. Ghotan did not use lower
case \emph{rum} rotunda and instead set the straight \emph{rum} after
all letters.

The straight \emph{rum} was used also in some other abbreviations. For
example, the Easter litany of saints uses it to ask each saint to pray
for us:
\begin{center}
  \begin{tabular}{cccccccccc}
    \sample{ora pro nobis} & \aqsample{$\Longrightarrow$} & \sample{O\msabbr{rum}}
  \end{tabular}
\end{center}

\paragraph{S} The word \emph{sed} is denoted by the eszet \emph{ß}
ligature:
\begin{center}
  \begin{tabular}{cccccccccc}
    \sample{\msnoliga{se}d} & \aqsample{$\Longrightarrow$} & \sample{\msabbr{sed}}
  \end{tabular}
\end{center}

In some places the same symbol is used also for other syllables
starting with \emph{se-}. It is notable that Ghotan did not use it to
represent double \emph{s} in any place.

He had also an \emph{se} ligature that he occasionally used that I
have decided to interpret as an abbreviation symbol as it is not used
consistently:
\begin{center}
  \begin{tabular}{cccccccccc}
    \sample{se} & \aqsample{$\Longrightarrow$} & \sample{\msabbr{se}}
  \end{tabular}
\end{center}



\paragraph{U} There are two different symbols for the syllable
\emph{-us} that are used for different grammatical endings:
\begin{center}
\begin{tabular}{cccccccccc}
  \sample{\msabbr{vs}} &&   \sample{\msabbr{us}} \\
  Dative   &&   Other \\
  suffix \emph{us} && suffix \emph{us}.
\end{tabular}
\end{center}

The dative suffix \missaali{\msabbr{vs}} occurs also in words that end
in \emph{-bus} even if they are in a different case. 

\begin{center}
  \begin{tabular}{cccccccccc}
    \sample{{deus}} & \aqsample{$\Longrightarrow$} &
    \sample{\msabbr{deus}} \\
    \sample{{omnibus}} & \aqsample{$\Longrightarrow$} &
    \sample{omni\msabbr{bus}} \\
    \sample{{tribus}} & \aqsample{$\Longrightarrow$} &
    \sample{tri\msabbr{bus}} \\
  \end{tabular}
\end{center}


\paragraph{V} The only specific symbol based on~\emph{V} is used only
to denote versicle and it is set in red in all places:
\begin{center}
  \begin{tabular}{cccccccccc}
    \sample{\mstdl{v}} \\
    versicle
  \end{tabular}
\end{center}

\subsection{Other abbreviation symbols}

Missaali has several abbreviation symbols that don't occur in
\emph{Missale Aboense} but that were in use when it was printed. These
abbreviations are enabled using other OpenType features and are shown
in table~\ref{tab:other_abbrs}. Most of them represent the suffix
\emph{-um} that is denoted by adding a diagonal stroke to a base letter.

\begin{table}
  \centering
  \begin{tabular}{clcllccccc}
     \multicolumn{5}{l}{\textbf{Abbreviations with specific meanings}}\\
     Symbol & Meaning & Feat & Entering & Notes \\
    \hline
    \sample{\msabbra{dum}} & dum & \feature{ss02} & dum & \\
    \sample{\msabbra{is}} & is & \feature{ss02} & is & \\
    \sample{\msabbra{mum}} & mum & \feature{ss02} & mum & \\
    \sample{\msabbra{num}} & num & \feature{ss02} & num & \\
    \sample{\msabbra{par}} & par & \feature{ss02} & par \\
    \sample{\msabbra{que}} & que & \feature{ss02} & que & \\
    \sample{\msabbra{qve}} & que & \feature{ss02} & qve & \\
    \sample{\msabbra{rum}} & rum & \feature{ss02} & rum & rum
    rotunda \\
    \sample{\msabbra{sunt}} & sunt & \feature{ss02} & sunt \\
    \sample{\msabbra{sm}} & sm & \feature{ss02} & sm \\
  \end{tabular}
  \caption{Other abbreviation symbols}
  \label{tab:other_abbrs}
\end{table}

\paragraph{D} The syllable \emph{dum} at the end of a word was
occasionally abbreviated by adding a diagonal stroke to \emph{d}:

\begin{center}
  \begin{tabular}{c}
    \sample{\msabbra{dum}} \\
    \emph{dum}
  \end{tabular}
\end{center}


\paragraph{M} The syllable \emph{mum} was abbreviated in the same way
by adding a diagonal stroke:

\begin{center}
  \begin{tabular}{c}
    \sample{\msabbra{mum}} \\
    \emph{mum}
  \end{tabular}
\end{center}


\paragraph{N} The syllable \emph{num} was abbreviated in the same way
by adding a diagonal stroke:

\begin{center}
  \begin{tabular}{c}
    \sample{\msabbra{num}} \\
    \emph{num}
  \end{tabular}
\end{center}


\paragraph{P} Adding a dieresis on top of \emph{p} was a common way to
represent \emph{par}. It is a bit surprising that it does not occur in
\emph{Missale Aboense}. 

\begin{center}
  \begin{tabular}{c}
    \sample{\msabbra{par}} \\
    \emph{par}
  \end{tabular}
\end{center}

\paragraph{Q} The syllable \emph{que} could be abbreviated in many
different ways. Missaali contains two additional ways in addition to
Ghotan's convention shown on page~\pageref{pg_que}. Simpler gothic
scripts as well as early books printed in the roman type often used a
semicolon after \emph{q} for the syllable, and some abbreviation
conventions (including Ghotan's one for small size script) used a
diaresis over~\emph{q}:

\begin{center}
  \begin{tabular}{ccc}
    \sample{\msabbra{que}} & & \sample{\msabbra{qve}} \\
    \emph{que} & & \emph{que}
  \end{tabular}
\end{center}

\paragraph{R} Ghotan didn't use rotunda version of \emph{rum} except
for the response\footnote{See page \pageref{pg_response}.} but it was
in a common use at the time.  

\begin{center}
  \begin{tabular}{ccc}
    \sample{\msabbra{rum}}  \\
    \emph{rum}
  \end{tabular}
\end{center}

\paragraph{S} There are three additional symbols based on the long
\emph{s}. The syllable \emph{is} adds a loop to it, \emph{sm} a high
horizontal stroke, and \emph{sunt} a diagonal stroke:

\begin{center}
  \begin{tabular}{ccccccc}
    \sample{\msabbra{is}} & &
    \sample{\msabbra{sm}} & &
    \sample{\msabbra{sunt}} \\
    \emph{is} & &
    \emph{sm} & &
    \emph{sunt} 
  \end{tabular}
\end{center}


\section{Old Finnish Ortography}

Missaali has support for typesetting Finnish using late 17th century
ortographical conventions. It is anachronistic as textura was no
longer in use at the time as it had been replaced by swabacher and
fraktur during the 16th century.

Finnish was not a written language when \emph{Missale Aboense} was
printed. Latin was the educated language, and everyday matters were
written in Swedish or German. Surviving Finnish texts are few and
almost all are translations of prayers.

The first long texts in Finnish were written by Mikael Agricola in the
16th century. He was the reformer of Finland and he translated the New
Testament and large parts of the Old to Finnish. His texts are very
difficult to read nowadays, as he based his ortography on Latin,
Swedish, and German, all languages very different from Finnish. Over
time it became evident that the written language should be
systematized. This happened with the 1642 Bible translation, \emph{Se
  Coco Pyhä Ramatu}. This translation fixed the way Finnish was
written for 150 years, and there were only small changes until the
early 19th century.

Missaali has an OpenType feature (\feature{ss16}, \feature{ordn}) that
implements a number of contextual substitutions for cases where the
1642 Bible systematically used characters that are different from
modern Finnish.

\paragraph{\emph{K}} 

The most complex phoneme in 1642 was \emph{K}. Mikael Agricola had
eight different ways to represent it in different contexts, and the
translators cut it down to five. The rules for its use were:

\begin{enumerate}
\item In most cases it is written with \emph{c}:
  {\fontspec{Missaali:+ss16}kansa}, {\fontspec{Missaali:+ss16}katko};

\item Before a front vowel (\emph{e}, \emph{i}, \emph{y}, \emph{ä},
  \emph{ö}) it is \emph{k}: {\fontspec{Missaali:+ss16} käpy},
  {\fontspec{Missaali:+ss16} kiire};

\item Double \emph{k} is written \emph{ck}: {\fontspec{Missaali:+ss16} sakko},
  {\fontspec{Missaali:+ss16} takki};

\item Combination \emph{ks} is written with \emph{x}:
  {\fontspec{Missaali:+ss16} yksi}, {\fontspec{Missaali:+ss16} suksi};

\item After letters \emph{n} and \emph{l} it is replaced by \emph{g}:
  {\fontspec{Missaali:+ss16} lanka}, {\fontspec{Missaali:+ss16} jalka};

\end{enumerate}

There were a few combinations where these rules were not always
followed. If a word started with \emph{Sk}, it was occasionally
written with \emph{k} even if the next letter was a back vowel:
\missaali{Skåne}, \missaali{Skoulu}. Some writers continued the
medieval tradition of using \emph{q} in front of the diftong
\emph{ui}: \missaali{quin}, \missaali{quinga}.

\paragraph{\emph{S}} The short \missaali{s} was used according to the
word-final convention, so it was short only at the end of the word or
at the end of the first part of a compound word. This is the default
behavior of Missaali but it differs from the 19th century black letter
conventions. 

Marking the double \emph{s} was inconsistent and a text might use
different ways to the same word in different parts of the text. The
three options were:

\begin{enumerate}
\item using the eszet ligature: \missaali{kißa};

\item using the long \missaali{ſ} for both: \missaali{kissa}; or

\item using only one long \missaali{ſ}: \missaali{kisa}.
\end{enumerate}

The combination \emph{ts} was usually written using the \emph{tz}
ligature. Missaali does not have it in its set of characters, but the
feature still replaces \emph{ts} with \emph{tz}:
{\fontspec{Missaali:+ss16} katse}, {\fontspec{Missaali:+ss16} Ruotsi}.

\paragraph{Ä and Ö} The letters \emph{ä} and \emph{ö} were written
with \emph{æ} and \emph{œ} ligatatures, except that the \emph{e} was
written over the original letter: {\fontspec{Missaali:+ss16}ä},
{\fontspec{Missaali:+ss16}ö}. The modern convention of using two dots
gained prevalence in the early decades of the 19th century.
 

\paragraph{Consonant softening} It was mentioned above that \emph{k}
was replaced with \emph{g} in combinations \emph{nk} and \emph{lk}.
The letters \emph{t} and \emph{p} were similarly softened in some
contexts:

\begin{enumerate}
\item Combinations \emph{lt} and \emph{nt} were written as \emph{ld}
  and \emph{nd}: {\fontspec{Missaali:+ss16} kulta},
{\fontspec{Missaali:+ss16} lintu}; and 

\item The combination \emph{mp} was written \emph{mb}:
  {\fontspec{Missaali:+ss16} ompi}.

\end{enumerate}

\paragraph{\emph{V} and \emph{W}} The letter \emph{v} is never used
when setting Finnish in black letter and a \emph{w} is always used in
its place, even in names: {\MsOldFinnishStyle \missaali{kasvi}},
{\MsOldFinnishStyle \missaali{Vaasa}},
{\MsOldFinnishStyle\missaali{Olavi}}. However, if the name is in a
different language, it is written as in the original language:
\missaali{Verdun}, \missaali{Riviera}.

\paragraph{Comma}

A virgule was used instead of a comma: {\fontspec{Missaali:+ss16},}.

\paragraph{Sound length}

In principle the sound length was marked the same way as in modern
Finnish, that is, a long sound is written with a doubled letter, but
its use was inconsistent. Missaali does not try to alter the sound
lengths, but if you want to emulate the 17th century conventions by
hand, you can follow the following three rules:

\begin{enumerate}
\item if the sound length changes the meaning of the word, then it is
  marked. For example, \missaali{tuli} and \missaali{tuuli} are
  always written with the correct vowel length; 

\item a long vowel that occurs in the first syllable is always marked:
  \missaali{päällä}. 

\item otherwise, if there is no possibility of confusion, the sound is
  often written with only one letter. For example,
  {\fontspec{Missaali:+ss16}viisaks} instead of
  {\fontspec{Missaali:+ss16}viisaaks} and
  {\fontspec{Missaali:+ss16}puhtaks} instead of
  {\fontspec{Missaali:+ss16}puhtaaks}.\footnote{The final \emph{i} is
    missing from both words as the most writers used the Turku dialect
    and the dialect has lost it from the transitive suffix.}
\end{enumerate}


\section{OpenType Features}
\label{pg:features_start}

This section describes the OpenType features of Missaali. Many of the
features have secondary names and these names do not usually
correspond with the standard behavior of those features. This is
because this font is intended for special uses and there are no
standard tags for describing its behaviors. Abusing the existing
standard tags is the only way to get the features to work in programs
that do not support arbitrary tags.

The features implementing abbreviation marks have two additional names
defined for them. This is because of how the current version of the
\texttt{fontspec} package of XeLaTeX works. The problem is that
turning features on and off blocks the contexts of contextual
substitutions, which means that the rules for \emph{r}-rotunda do not
work correctly. I had to split the abbreviation mark features into two
because of this, one feature that adds the symbols to everything
except \emph{r} and another that adds it just to \emph{r}. See
page~\pageref{pg:fontspec_problems} for details.


\paragraph{\feature{aalt} — All alternatives}

This tag maps all letters to all variant forms and abbreviations that
make use of the symbol. This is a special feature that is supported by
some programs. However, none of those that I have easy access to
supports it, this isn't debugged and it may or may not work.

\paragraph{\feature{calt} — Contextual alternatives}

The \feature{calt} tag defines the substitution rules for long
\emph{s} and rotunda \emph{r}. The tag implements the rules in the
form that Bartholomew Ghotan used them that differ from the standard
rules of later typography. This feature should be turned on unless you
specifically want to follow more modern conventions. 
\begin{center}
  \begin{tabular}{cccccccccc}
    \sample{{\addfontfeature{RawFeature=-calt}Missaali}} &
    \sample{Missaali} \\
    \feature{-calt} &\feature{+calt}  \\
  \end{tabular}
\end{center}

There is a duplicate version of the short \emph{s} glyph in the font
that you can use if you want to insert one in the middle of a word
without turning \feature{calt} off. It is at the position
\texttt{U+F509}, and similarly there is a straight \emph{r} at
\texttt{U+F4EB}. 

Some old versions of MS Word provide only partial support for
\feature{calt}: they substitute the characters but do not change the
spacing within the word, so a long \emph{s} is shown with the spacing
of a short \emph{s}.

\paragraph{\feature{dlig} — Ghotan's abbreviations}

Most of Bartholomew Ghotan's abbreviations for Latin are enabled with
\feature{dlig}. The table~\ref{tab:ghota_abbr} on
page~\pageref{tab:ghota_abbr} contains a list of them.

\begin{center}
  \begin{tabular}{cccccccccc}
    \sample{que, hoc, urbis} & \sample{\msabbr{que, hoc, urbis}} \\
    \feature{-dlig} & \feature{+dlig}
  \end{tabular}
\end{center}

\paragraph{\feature{hist}, \feature{swsh} — Historical forms}


This form changes the \emph{æ}, \emph{œ}, \emph{ä}, and \emph{ö}
characters to medieval forms where \emph{e} is set over the top of
\emph{a} or \emph{o}.

\begin{center}
  \begin{tabular}{cccccccccc}
    \sample{ä, æ, ö, œ} & \sample{{\addfontfeature{RawFeature=+hist;}
        ä, æ, ö, œ}} \\
    \feature{-hist} & \feature{+hist}
  \end{tabular}
\end{center}

\paragraph{\feature{liga} — Standard ligatures}

The feature \feature{liga} implements the standard ligature
substitution. This implements the fusion of bowls of adjacent letters
as well as several substitutions for \emph{f} and long \emph{s}. This
should usually be turned on for the whole document.

\begin{center}
  \begin{tabular}{cccccccccc}
    \sample{\msnoliga{deus}} &    \sample{deus}   \\
    \feature{-liga} & \feature{+liga}
  \end{tabular}
\end{center}

\paragraph{\feature{ss01}, \feature{subs} — Scandinavian vowel
  ligatures}

Nordic texts often used fused vowels to denote either long vowels or
diphtongs. These fusions are not present in \emph{Missale Aboense},
but I added them to the font:
\begin{center}
  \begin{tabular}{cccccccccc}
    \sample{aa, ao} &
    \sample{{\addfontfeature{RawFeature=+ss01}aa, ao}} \\
    \feature{-ss01} & \feature{+ss01}
  \end{tabular}
\end{center}

\paragraph{\feature{ss02}, \feature{ornm} — Additional abbreviations}

This feature enables some abbreviations that are not in \emph{Missale
  Aboense}. The details are shown in table~\ref{tab:other_abbrs}.

\begin{center}
  \begin{tabular}{cccccccccc}
    \sample{mum, sm, que} &
    \sample{{\addfontfeature{RawFeature=+ss02}mum, sm, que}} \\
    \feature{-ss02} & \feature{+ss02}
  \end{tabular}
\end{center}

\paragraph{\feature{ss03}, \feature{frac} — Alternate abbreviations}

This set works like \feature{dlig} but it uses a different
abbreviations for some letter combinations.

\begin{center}
  \begin{tabular}{cccccccccc}
    \sample{que, con, habet} & \sample{\addfontfeature{RawFeature=+ss03}{que, con, habet}} \\
    \feature{-ss03} & \feature{+ss03}
  \end{tabular}
\end{center}


\paragraph{\feature{ss04}, \feature{salt}, \feature{lt04}, \feature{lr04} — Generic abbreviations}

This feature adds a generic abbreviation symbol on the top of a letter
or ligature. For most of the letters it uses a tilde, but some letters
use something different. 

\begin{center}
  \begin{tabular}{cccccccccc}
    \sample{a, h, l, do, r, d} &
    \sample{\addfontfeature{RawFeature=+ss04}{a, h, l, do, r,d }} \\
    \feature{-ss04} & \feature{+ss04}
  \end{tabular}
\end{center}

\paragraph{\feature{ss05}, \feature{lt05}, \feature{lr05} — Tilde} Adds a tilde on top of most
letters.
\begin{center}
  \begin{tabular}{cccccccccc}
    \sample{a, n, q, r} &
    \sample{\addfontfeature{RawFeature=+ss05}{a, n, q, r}} \\
    \feature{-ss05} & \feature{+ss05}
  \end{tabular}
\end{center}


\paragraph{\feature{ss06}, \feature{lt06}, \feature{lr06} — Dieresis} Adds a dieresis on top of most
letters.
\begin{center}
  \begin{tabular}{cccccccccc}
    \sample{a, x, r, z} &
    \sample{\addfontfeature{RawFeature=+ss06}{a, x, r, z}} \\
    \feature{-ss06} & \feature{+ss06}
  \end{tabular}
\end{center}


\paragraph{\feature{ss07}, \feature{lt07}, \feature{lr07} —
  R-abbreviation} Adds an \emph{r}-abbreviation over most letters.
Note that in this font the acute accent is used to mark the
\emph{r}-abbreviation tailed dot.
\begin{center}
  \begin{tabular}{cccccccccc}
    \sample{a, e, t} &
    \sample{\addfontfeature{RawFeature=+ss07}{a, e, t}} \\
    \feature{-ss07} & \feature{+ss07}
  \end{tabular}
\end{center}

\paragraph{\feature{ss08}, \feature{lt08}, \feature{lr08} — Shorter
  tilde} Adds a shorter tilde over most letters. Ghotan used this
occasionally in places where a normal tilde would have not fit well,
most often over \emph{i}. In this font this is implemented as a grave
accent:
\begin{center}
  \begin{tabular}{cccccccccc}
    \sample{e, i, u} &
    \sample{\addfontfeature{RawFeature=+ss08;}{e, i, u}} \\
    \feature{-ss08} & \feature{+ss08}
  \end{tabular}
\end{center}

Note that in case of letters where the normal tilde is already short,
it uses the same symbol as \feature{ss05}



\paragraph{\feature{ss09}, \feature{onum}, \feature{lt09}, \feature{lr09} — A-abbreviation} Adds a
squashed open \emph{a} that looks like a two-part tilde over most
letters. This marks that something containing an \emph{a} has been
abbreviated out:
\begin{center}
  \begin{tabular}{cccccccccc}
    \sample{g, s, q} &
    \sample{\addfontfeature{RawFeature=+ss09;}{g, s, q}} \\
    \feature{-ss09} & \feature{+ss09}
  \end{tabular}
\end{center}
This mark is implemented as the circumflex accent.


\paragraph{\feature{ss10}, \feature{lt10}, \feature{lr10} — O-abbreviation}

Adds a ring over most letters. This denotes that something containing
an \emph{o} is abbreviated out. This is implemented as the ring accent.

\begin{center}
  \begin{tabular}{cccccccccc}
    \sample{q, u} &
    \sample{\addfontfeature{RawFeature=+ss10;}{q, u}} \\
    \feature{-ss10} & \feature{+ss10}
  \end{tabular}
\end{center}


\paragraph{\feature{ss11}, \feature{sups} — Um-abbreviation}

This feature adds a diagonal slash to some letters that is in most
contexts used for abbreviating \emph{um}. The only one of these that
occurs in \emph{Missale Aboense} is straight \emph{rum}. 

\begin{center}
  \begin{tabular}{cccccccccc}
    \sample{k, d, r, R} &
    \sample{\addfontfeature{RawFeature=+ss11;}{k, d, r, R}} \\
    \feature{-ss11} & \feature{+ss11}
  \end{tabular}
\end{center}

As you can see from the example, this feature turns capital \emph{R}
into \emph{rum} rotunda. 

\paragraph{\feature{ss12}, \feature{lt10}, \feature{lr10}  — Ur-abbreviation} Adds a hook that denotes
\emph{ur} over most letters. This is implemented as the caron accent.
\begin{center}
  \begin{tabular}{cccccccccc}
    \sample{m, t} &
    \sample{\addfontfeature{RawFeature=+ss12;}{m, t}} \\
    \feature{-ss12} & \feature{+ss12}
  \end{tabular}
\end{center}


\paragraph{\feature{ss13} — Horizontal strike} Adds a horizontal stroke
to some letters. 
\begin{center}
  \begin{tabular}{cccccccccc}
    \sample{k, p, q} &
    \sample{\addfontfeature{RawFeature=+ss13;}{k, p, q}} \\
    \feature{-ss13} & \feature{+ss13}
  \end{tabular}
\end{center}


\paragraph{\feature{ss14} — Loop} Adds a loop to some letters. 
\begin{center}
  \begin{tabular}{cccccccccc}
    \sample{s l o} &
    \sample{\addfontfeature{RawFeature=+ss14;}{s l o}} \\
    \feature{-ss14} & \feature{+ss14}
  \end{tabular}
\end{center}


\paragraph{\feature{ss15}, \feature{smcp} — Liturgical symbols}
Changes some of the letters into liturgical symbols used in
\emph{Missale Aboense}. 
\begin{center}
  \begin{tabular}{ccccccccc}
    \sample{. r v V } & \sample{\addfontfeature{RawFeature=+ss15;}{.
      r v V}} \\
    \feature{-ss15} & \feature{+ss15} 
  \end{tabular}
\end{center}


\paragraph{\feature{ss16}, \feature{ordn}, \feature{ofin} — Old Finnish ortography}

This changes text to correspond to syntactic conventions that were in
use in late 17th century Finnish texts (the so-called \emph{1642 Bible
  Translation} ortograhpy).\footnote{The Finnish translation of
  \emph{In Catalinam} in the following example is by K.~Laurila \cite{laurila}.}

\begin{center}
  \begin{tabular}{lp{8cm}}
    Input &
      Kuinka pitkälle aijot sinä, Catilina, käyttää väärin meidän
      kärsivällisyyttämme? Kuinka kauvan luulet sinä vielä voivasi
      hurjuudessasi pitää meitä pilkkana? Mihin määrään saakka kohoo
      sinun rajaton röyhkeytesi? \\
      \feature{+ss16}&
      \begin{minipage}{8.5cm}
        {\sample{\addfontfeature{RawFeature=+ss16}
            \spaceskip0.2em plus 0.1 em minus 0.05em
            Kuinka pitkälle
            aijot sinä, Catilina, käyttää väärin meidän
            kärsivälli\-syyttämme? Kuinka kauvan luulet sinä vielä
            voivasi \mbox{hurjuudessasi pi-}
            tää meitä pilkkana? Mihin
            mää\-rään saakka kohoo sinun rajaton röyhkeytesi?}}
      \end{minipage}
  \end{tabular}
\end{center}

\paragraph{\feature{ss17} — Alternate punctuation}

Replaces comma by colon, period by centered period and the exclamation
sign by period. 
\begin{center}
  \begin{tabular}{cccccccccc}
    \sample{, . ! } & \sample{\addfontfeature{RawFeature=+ss17;}{, .
        !}} \\
    \feature{-ss17} & \feature{+ss17} 
  \end{tabular}
\end{center}

\paragraph{\feature{ss18} — Fuse po}
Enables the \emph{po} fusion. 
\begin{center}
  \begin{tabular}{cccccccccc}
    \sample{po} & \sample{\addfontfeature{RawFeature=+ss18;}{po}} \\
    \feature{-ss18} & \feature{+ss18} 
  \end{tabular}
\end{center}

\paragraph{\feature{ss19} — Alternative z}

Use alternative form for \emph{z}.

\begin{center}
  \begin{tabular}{cccccccccc}
    \sample{z} & \sample{\addfontfeature{RawFeature=+ss19;}{z}}\\
    \feature{-ss19} & \feature{+ss19} 
  \end{tabular}
\end{center}

\paragraph{\feature{ss20}, \feature{titl} — Initials}

Turn capital letters into Lombardic initials

\begin{center}
  \begin{tabular}{cccccccccc}
    \sample{A B C} & \sample{\addfontfeature{RawFeature=+ss20;}{A B
        C}} \\
    \feature{-ss20} & \feature{+ss20} 
  \end{tabular}
\end{center}

\paragraph{\feature{ss21}, \feature{lt21}, \feature{lr21} — Dotaccent}
Adds a dot to most letters.

\begin{center}
  \begin{tabular}{cccccccccc}
    \sample{b o v} &
    \sample{\addfontfeature{RawFeature=+ss21;}{b o v}} \\
    \feature{-ss21} & \feature{+ss21}
  \end{tabular}
\end{center}


\paragraph{\feature{ss22} — Alternate G} 
Use alternative form for \emph{G}.
\begin{center}
  \begin{tabular}{cccccccccc}
    \sample{G} &
    \sample{\addfontfeature{RawFeature=+ss22;}{G}} \\
    \feature{-ss22} & \feature{+ss22}
  \end{tabular}
\end{center}

\paragraph{\feature{ss23} — Alternate initials}

Changes some of the Lombardic initials into their alternate forms:

\begin{center}
  \begin{tabular}{cccccccccc}
    \sample{A D S} & \sample{\addfontfeature{RawFeature=+ss23;}{A D S}} \\
    \feature{-ss23} & \feature{+ss23} 
  \end{tabular}
\end{center}



\label{pg:features_end}

\section{LaTeX interface}

\label{pg:latex_start}

\newcommand{\ltxenv}[1]{\texttt{#1}}
\newcommand{\ltxncmd}[1]{\texttt{\textbackslash{}#1}}
\newcommand{\ltxcmd}[2]{\texttt{\textbackslash{}#1\{#2\}}}
\newcommand{\ltxcmdtwo}[3]{\texttt{\textbackslash{}#1\{#2\}\{#3\}}}
\newcommand{\ltxcmdthree}[4]{\texttt{\textbackslash{}#1\{#2\}\{#3\}\{#4\}}}
\newcommand{\ltxcmdfour}[5]{\texttt{\textbackslash{}#1\{#2\}\{#3\}\{#4\}\{#5\}}}


The file \texttt{missaali.sty} defines the package \texttt{missaali}
that contains a number of convenience environments and macros for
using the font together with XeLaTeX. All features of the font are
available with the system, but accessing some of them is a bit
cumbersome. This is partly because of how \texttt{fontspec} works and
partly because the TeX typesetting algorihms are intended for modern
typesetting conventions.

\paragraph{The problem with \texttt{fontspec} contexts}

\begin{figure}
  \centering \fbox{\begin{tabular}{llll}
      \multicolumn{2}{l}{Problem} & \multicolumn{2}{l}{Solution} \\
      \hline \multicolumn{2}{l}{Short \missaali{s} in place of
        \missaali{ſ}} &
      \multicolumn{2}{l}{Move \emph{s} inside the macro:} \\
      &\sample{concurs\msabbr{us}} & &\sample{concur\msabbr{sus}} \\
      &\texttt{concurs\ltxcmd{msabbr}{us}} &
      & \texttt{concur\ltxcmd{msabbr}{sus}} \\
      \\
      \multicolumn{2}{l}{Straight \emph{r} after initial} &
      \multicolumn{2}{l}{Use
        \ltxcmd{rotundar}{}} \\
      & \sample{\msparinitial{P}ro} & & \sample{\msparinitial{P}\rotundar{}o}
      \\
      & \texttt{\ltxcmd{msparinitial}{P}ro} & & \texttt{\ltxcmd{msparinitial}{P}\ltxcmd{rotundar}{}o}\\
      \\
      \multicolumn{2}{l}{Straight \emph{r} after abbreviation} &
      \multicolumn{2}{l}{Move \emph{r} inside macro} \\
      & \sample{\msabbr{per}ry c\mstdr{u}ru} & &
      \sample{\msabbr{perr}y c\mstdr{ur}u} \\
      &\texttt{\ltxcmd{msabbr}{per}ry c\ltxcmd{mstdr}{u}ru} & &
      \texttt{\ltxcmd{msabbr}{perr}y c\ltxcmd{mstdr}{ur}u} \\
      \\
      \multicolumn{2}{l}{Missing mark over \emph{r}} & 
      \multicolumn{2}{l}{Use the \emph{r} versions of macros}
      \\
      & \sample{m\mstd{r}is co\mstd{r}s} & &
        \sample{m\rmstd{r}is c\rmstd{or}s} \\
      & \texttt{m\ltxcmd{mstd}{r}is co\ltxcmd{mstd}{r}s} & &
        \texttt{m\ltxcmd{rmstd}{r}is c\ltxcmd{rmstd}{or}s} \\
  \end{tabular}}
  \caption{Solving the LaTeX \texttt{fontspec} problems.}
  \label{pg:fontspec_problems}
\end{figure}


Missaali has contextual substitutions where different forms of letters
are substituted based on where in the word they occur. The most
important of these are the rotunda \emph{r} and the long \emph{s}. The
current version of \texttt{fontspec}\footnote{Version 2.3c in Texlive
  2013} has the problem that the feature boundaries break the
substitution contexts. This happens with abbreviation symbols as well
as when using colored initials. For example, here are three different
ways to write \emph{usque}:\footnote{Command \ltxcmd{msabbr}{}{} is a
  macro that enables the abbreviation ligatures defined by
  \feature{dlig}.}

\begin{center}
  \begin{tabular}{lcl}
    \sample{usque} & & \texttt{usque} \\
    \sample{us\msabbr{que}} & & \texttt{us\ltxcmd{msabbr}{que}} \\
    \sample{u\msabbr{sque}} & & \texttt{u\ltxcmd{msabbr}{sque}} 
  \end{tabular}
\end{center}

Here the middle line has short \emph{s} because \texttt{fontspec}
broke the context at the point where the abbreviation starts so it
thinks that the \emph{s} is occurring at the end of the word. The last
line has the correct long \emph{s} because it is moved inside the
\ltxcmd{msabbr}{} command so it belongs to the same context. Since
\emph{s} does not have any abbreviation marks defined for it and
\emph{sed} is the only abbreviation ligature starting with it, moving
it inside the macros don't cause any problems.

However, doing the same for \emph{r} does cause problems because it
has abbreviation marks. Moving~\emph{r} to the same context will force
both symbols to have the same mark. The solution that Missaali uses is
to define two additional features for each abbreviation mark feature,
one that abbreviates only~\emph{r} and another for all other letters:

\begin{itemize}
\item \feature{ssXX} — the base feature that adds abbreviation symbol
  to all letters;
\item \feature{ltXX} — adds abbreviation to all letters except \emph{r}
\item \feature{lrXX} — adds abbreviation only to \emph{r}.
\end{itemize}

For example, the behavior of the standard abbreviation macros is:
\begin{center}
  \begin{tabular}{ccccccc}
    \sample{or} & & \sample{\addfontfeature{RawFeature=+ss04}or} & &
    \sample{\mstd{or}} & & \sample{\rmstd{or}} \\
    Normal & & \feature{ss04} & & \feature{lt04} & & \feature{lr04}
  \end{tabular}
\end{center}

The package has two versions of for all abbreviation marks, one using
the \feature{ltXX} feature and another using \feature{lrXX}.

\subsection{Typesetting Textura Blocks}

Textura was set using two justified columns with only 24–26 characters
per line. Automatic typesetting algorithms, including the one used by
TeX, work very poorly with justified short lines. It is necessary to
help the algorithm by selecting suitable line break points and
hyphenating words by hand. Using abbreviations makes setting the lines
a bit easier, but the cost is that the modern audience will not
understand them.

The style file defines four environments: \ltxenv{mstextura},
\ltxenv{mstexturablocks}, and \ltxenv{mstexturablock} for setting the
text. The first sets complete two-column pages in Missaali, the second
sets two-column Missaali that is intended to fit inside one page that
may contain also modern typography, and the last sets one column of
text that is intended to fit inside one page.

The environments make the following changes to the ordinary TeX
layout:\label{pag:environments}
\begin{enumerate}
\item they turn on the \ltxncmd{obeylines} so that LaTeX will force a
  line break at newlines of the text;
\item they add infinitely stretchable glue to the word space to force
  justification; and 
\item they add space also to the left side of punctuation symbols.
\end{enumerate}

When \ltxncmd{obeylines} is in effect, LaTeX doesn't hyphenate any
words automatically so all hyphenations have to be added by hand. 

When setting text with Missaali, I've used the following approach:

\begin{enumerate}
\item Set the font and textura block sizes.

\item Set the beginning of the text by placing possible rubrics in red
  at the start as well as the two first lines that contain the
  initial. 

\item Go down through the text block adding line breaks to good places
  if XeLaTeX puts them in ugly places.

\item Further adjust the text by adding hyphenation points manually as
  well as abbreviation symbols. 

\item If the text ends with a partial line, go back few lines altering
  them so that the last line is also full. 
\end{enumerate}

\subsection{Defining Font and Block Sizes}

There are four parameters that alter the layout of textura blocks.

\begin{itemize}
\item font size (default 18 pt);  
\item text block width (default 6.4 cm);
\item text block height (default 34 lines); and 
\item width of the gutter between columns (default 1 cm). 
\end{itemize}

The first two affect all three textura environments, the third only
\ltxenv{mstextura} and the last affects environments with two columns.
Each parameter has a macro for defining its value, and there is a
combined macro for setting the first three at one time. See
figure~\ref{fig:text_sizes} for examples. 

\begin{figure}
  \centering
  \begin{tabular}{ll}
    & \begin{minipage}{15cm}
      {\mssetfontsize{22}
      \mssetblockwidth{7.70cm}
      \begin{mstexturablock}
        Nihilne te noc\msabbr{tur}num \msabbr{prae}sidium
        Palati, nihil urbis vigiliae,
        nihil timor populi, nihil con-
        cur\msabbr{sus} bonorum omnium, nihil
      \end{mstexturablock}}
      \end{minipage}\\
      \\
&      \begin{minipage}{15cm}
\begin{verbatim}
      \mssetfontsize{22}
      \mssetblockwidth{7.70cm}
      \begin{mstexturablock}
        Nihilne te noc\msabbr{tur}num \msabbr{prae}sidium
        Palati, nihil urbis vigiliae,
        nihil timor populi, nihil con-
        cur\msabbr{sus} bonorum omnium, nihil
      \end{mstexturablock}
\end{verbatim} 
      \end{minipage}\\
      \\
    & \begin{minipage}{15cm}
      {\mssetfontsize{18}
      \mssetblockwidth{6.40cm}
      \begin{mstexturablock}
Nihilne te noc\msabbr{tur}num \msabbr{prae}sidium
        Palati, nihil urbis vigiliae,
        nihil timor populi, nihil con-
        cur\msabbr{sus} bonorum omnium, nihil
      \end{mstexturablock}}
      \end{minipage} \\
      \\
&      \begin{minipage}{15cm}
\begin{verbatim}
      \mssetfontsize{18}
      \mssetblockwidth{6.40cm}
      \begin{mstexturablock}
        Nihilne te noc\msabbr{tur}num \msabbr{prae}sidium
        Palati, nihil urbis vigiliae,
        nihil timor populi, nihil con-
        cur\msabbr{sus} bonorum omnium, nihil
      \end{mstexturablock}
\end{verbatim} 
      \end{minipage}\hfill\\
  \end{tabular}
  \caption{Examples of text size macros}
  \label{fig:text_sizes}
\end{figure}

\paragraph{\ltxcmd{mssetfontsize}{size in points}}

This sets the font size. This is implemented as an independent
variable because it is used for calculating the height of the text
block.  

\paragraph{\ltxcmd{mssetblockwidth}{width as a dimension}}

This sets the width of the textura block to a given dimension. 

\paragraph{\ltxcmd{mssetblocklines}{number of lines}}

This sets how many lines of text there are on each page of the
\ltxenv{mstextura} environment. The height of the text block is
calculated by multiplying the number of lines with the font size.

\paragraph{\ltxcmd{mssetgutterwidth}{width as a dimension}}

This sets the gutter width between the columns to its argument.

\paragraph{\ltxcmdthree{mssetsizes}{font size}{block width}{block
    lines}}

This sets the font size, text block width, and the number of lines on
a page in one call.

\subsection{Initials}

\begin{figure}
  \centering
  \begin{tabular}{ll}
    & \begin{minipage}{6.65cm}
        \fontsize{18}{18}\fontspec[RawFeature=+ss17;]{Missaali}\msmanualspacing
        \msstartchapter{Q}{Uo usque tandem abu-}{%
          tere, Catilina, pati\mstdacc{e}{en}tia}
        nostra? \msabbr{quam} diu eti\mstdacc{a}{am} furor iste
      \end{minipage} \\
      \\
&      \begin{minipage}{10cm}
\begin{verbatim}
 \msstartchapter{Q}{Uo usque tandem abu-}{%
   tere, Catilina, pati\mstdacc{e}{en}tia}
   nostra? \msabbr{quam} diu  eti\mstdacc{a}{am} furor iste
\end{verbatim} 
      \end{minipage}\\
    \multicolumn{2}{l}{\hspace{3cm}\textbf{(a)}} \\
        \\
&      \begin{minipage}{6.65cm}
        \fontsize{18}{18}\fontspec[RawFeature=+ss17;]{Missaali}\msmanualspacing
        \msstartchapterblue{Q}{Uo usque tandem abu-}{%
          tere, Catilina, pati\mstdacc{e}{en}tia}
        nostra? \msabbr{quam} diu eti\mstdacc{a}{am} furor iste

      \end{minipage} \\
      \\
&      \begin{minipage}{10cm}
\begin{verbatim}
 \msstartchapterblue{Q}{Uo usque tandem abu-}{%
   tere,Catilina,pati\mstdacc{e}{en}tia}
   nostra? \msabbr{quam} diu  eti\mstdacc{a}{am} furor iste
\end{verbatim} 
      \end{minipage}\\
      \multicolumn{2}{l}{\hspace{3cm}\textbf{(b)}} \\
      \\
&      \begin{minipage}{6.3cm}
 \fontsize{18}{18}\fontspec[RawFeature=+ss17;]{Missaali}\msmanualspacing
\msstartchapterwithrubric{Lectio}{R}{Egnante illus-\ }{trissimo rege sancto}
Erico in Svecia, venerabilis
      \end{minipage} \\
      \\
&      \begin{minipage}{10cm}
\begin{verbatim}
\msstartchapterwithrubric{Lectio}{R}{%
  Egnante illus-\ }{trissimo rege sancto}
  Erico in Svecia, venerabilis
\end{verbatim} 
  \end{minipage}\\
      \multicolumn{2}{l}{\hspace{3cm}\textbf{(c)}} \\
      \\
&      \begin{minipage}{6.3cm}
 \fontsize{18}{18}\fontspec[RawFeature=+ss17;]{Missaali}\msmanualspacing
\msrubric{De vita sancto henrici e\mstdacc{p}{piscop}i}
\msstartchapterwithrubricblue{Lectio}{R}{Egnante illus-\ }{trissimo rege sancto}
Erico in Svecia, venerabilis
      \end{minipage} \\
      \\
&      \begin{minipage}{10cm}
\begin{verbatim}
\msrubric{De vita sancto henrici e\mstdacc{p}{piscop}i}
\msstartchapterwithrubricblue{Lectio}{R}{%
  Egnante illus-\ }{trissimo rege sancto}
  Erico in Svecia, venerabilis
\end{verbatim} 
  \end{minipage}\\
  \multicolumn{2}{l}{\hspace{3cm}\textbf{(d)}} \\
    \end{tabular}
  \caption{Examples for chapter start macros}\label{fig:chapterstarts}

\end{figure}

The style file contains four macros for setting the first two lines of
a text. They set a Lombardic initial in either red or blue, and
possibly an additional partial line rubric in red at the end of the
first line. See figure~\ref{fig:chapterstarts} for examples.

\paragraph{\ltxcmdthree{msstartchapter}{initial}{first line}{second
    line}}\mbox{}\\

This sets a two-line \texttt{initial} in red, and then the
contents of the \texttt{first line} and the \texttt{second line} to
its right indented by the proper abount. This is shown in
figure~\ref{fig:chapterstarts}a. 

\paragraph{\ltxcmdthree{msstartchapterblue}{initial}{first line}{second line}}\mbox{}\\

This sets a two-line \texttt{initial} in blue, and then the
contents of the \texttt{first line} and the \texttt{second line} to
its right indented by the proper abount. This is shown in
figure~\ref{fig:chapterstarts}b. 


\paragraph{\ltxcmdfour{msstartchapterwithrubric}{rubric}{initial}{first}{second}}\mbox{}\\

This sets a two-line \texttt{initial} in red, and the
contents of the \texttt{first line} and at its end the \texttt{rubric}
in red. The second line is set as in the previous commands. 
See figure~\ref{fig:chapterstarts}c. 


\paragraph{\ltxcmdfour{msstartchapterwithrubricblue}{rubric}{initial}{first}{second}}\mbox{}\\

This sets a two-line \texttt{initial} in blue, and the contents of the
\texttt{first line} and at its end the \texttt{rubric} in red. The
second line is set as in the previous commands. See
figure~\ref{fig:chapterstarts}d. That figure also shows how to add a
rubric in red in front above the chapter start using the command
\ltxncmd{msrubric}.

\paragraph{Alternate initial I} The macro \ltxncmd{initialI} sets the
tall version of initial I in blue into the margin or gutter using a
very large size. As the letter is not in the text block, the next line
doesn't have to indented. If the line has a partial rubric on it, then
the text has to be entered with macros \ltxncmd{mspartialrubric} and
\ltxncmd{mspartialline}. For example, the blue initial on
page~\pageref{pg_carmina_sample} is created with:

\begin{verbatim}
\mspartialrubric{Incognit\msabbr{\msrubric{us}}}
\mspartialline{\initialI{}N taberna qu\mstdacc{a}{an}do }
  sumus: non curam{us} {quid} sit 
\end{verbatim}

The \ltxncmd{mspartialrubric} and \ltxncmd{mspartialline} have to be
always used as a pair where rubric is first and the line is next.

\subsection{Abbreviations}

There are two different ways how abbreviations are handled in
\texttt{missaali.sty}. The abbreviations with specific meanings are
handled as ligatures while abbreviation marks are simple letter
substitutions. There are three different commands for ligature
substitutions and each abbreviation mark has four different commands
defined for it. 

Different commands for the abbreviation marks are needed because
otherwise r-rotunda would not work correctly. Thus, there is a command
that puts the mark over all letters except for \emph{r}, and one that
puts it only on \emph{r}. This makes it possible to put \emph{r}
inside the abbreviation context to get a correct form for it. There
are also two variants on whether an additional unabbreviated text is
added also to the pdf using the actual text mechanism. This makes it
easier to search text on advanced pdf viewers. The commands for
abbreviation marks are shown in table~\ref{tab:abbr_commands}.

\begin{table}
  \centering
  \begin{tabular}{llccll}
    &     Command & Marks \emph{r} & Actual text  & Sample & Code \\
    \hline
    \multicolumn{2}{l}{Generic abbreviation} \\
    &    \ltxncmd{mstd}       & & &  \sample{\mstd{or}} & \ltxcmd{mstd}{or} \\
    &    \ltxncmd{mstdacc}    & & $\times$ & \\
    &    \ltxncmd{rmstd}      & $\times$ & & \sample{\rmstd{or}} & \ltxcmd{rmstd}{or} \\
    &    \ltxncmd{rmstdacc}   & $\times$ & $\times$ & \\
    \hline
    \multicolumn{2}{l}{A-abbreviation} \\
    &    \ltxncmd{mstda}      & & &  \sample{\mstda{or}} & \ltxcmd{mstda}{or} \\
    &    \ltxncmd{mstdaacc}   & & $\times$ &     \\    
    &    \ltxncmd{rmstda}     & $\times$ & & \sample{\rmstda{or}} & \ltxcmd{rmstda}{or} \\          
    &    \ltxncmd{rmstdaacc}  & $\times$ & $\times$ &\\
    \hline
    \multicolumn{2}{l}{R-abbreviation} \\
    &    \ltxncmd{mstdr}      & & & \sample{\mstdr{or}} & \ltxcmd{mstdr}{or} \\ 
    &    \ltxncmd{mstdracc}   & & $\times$ &           \\
    &    \ltxncmd{rmstdr}     & $\times$ & & \sample{\rmstdr{or}} & \ltxcmd{rmstdr}{or} \\           
    &    \ltxncmd{rmstdracc}  & $\times$ & $\times$ & \\
    \hline
    \multicolumn{2}{l}{O-abbreviation} \\
    &    \ltxncmd{mstdo}      & & & \sample{\mstdo{or}} & \ltxcmd{mstdo}{or} \\ 
    &    \ltxncmd{mstdoacc}   & & $\times$ &           \\
    &    \ltxncmd{rmstdo}     & $\times$ & & \sample{\rmstdo{or}} & \ltxcmd{rmstdo}{or} \\ 
    &    \ltxncmd{rmstdoacc}  & $\times$ & $\times$ &\\
    \hline
    \multicolumn{2}{l}{Specifically tilde} \\
    &    \ltxncmd{mstdt}     & & &   \sample{\mstdt{or}} & \ltxcmd{mstdt}{or} \\ 
    &    \ltxncmd{mstdtacc}  & & $\times$ &           \\
    &    \ltxncmd{rmstdt}    & $\times$ & &\sample{\rmstdt{or}} & \ltxcmd{rmstdt}{or} \\  
    &    \ltxncmd{rmstdtacc} & $\times$ & $\times$ & \\
    \hline
    \multicolumn{2}{l}{Short tilde accent} \\
    &    \ltxncmd{mstdst}     & & &  \sample{\mstdst{or}} & \ltxcmd{mstdst}{or} \\  
    &    \ltxncmd{mstdstacc}  & & $\times$ &        \\
    &    \ltxncmd{rmstdst}    & $\times$ & &\sample{\rmstdst{or}} & \ltxcmd{rmstdst}{or} \\   
    &    \ltxncmd{rmstdstacc} & $\times$ & $\times$ & \\
    \hline
    \multicolumn{2}{l}{Specifically dieresis} \\
    &    \ltxncmd{mstdd}     & & &  \sample{\mstdd{or}} & \ltxcmd{mstdd}{or} \\                     
    &    \ltxncmd{mstddacc}  & & $\times$ &           \\
    &    \ltxncmd{rmstdd}    & $\times$ &  &\sample{\rmstdd{or}} & \ltxcmd{rmstdd}{or} \\                     
    &    \ltxncmd{rmstdacc} & $\times$ & $\times$ &\\
    \hline
    \multicolumn{2}{l}{Diagonal stroke} \\
    &    \ltxncmd{mstddg}     & & &  \sample{\mstddg{Sr}} &
    \ltxcmd{mstddg}{Sr} \\                       
    &    \ltxncmd{mstddgacc}  & & $\times$ &           \\
    &    \ltxncmd{rmstddg}    & $\times$ &  &\sample{\rmstddg{Sr}} & \ltxcmd{rmstddg}{Sr} \\                     
    &    \ltxncmd{rmstdgacc} & $\times$ & $\times$ &\\
    
  \end{tabular}
  \caption{LaTeX commands for abbreviation marks}\label{tab:abbr_commands}
\end{table}

The commands are named systematically so that if the base name of the
macro is \texttt{X}, then the commands have the form:

\begin{enumerate}
\item \ltxncmd{X}: adds the mark to all other letters except for
  \emph{r}, does not add the actual text to the pdf.
\item \ltxncmd{rX}: adds the mark only to \emph{r}, does not add the
  actual text to the pdf.
\item \ltxncmd{Xacc}: adds the mark to all other letters except for
  \emph{r}, adds the second argument as the actual text to the pdf;
  and 
\item \ltxncmd{Xacc}: adds the mark only to \emph{r}, adds the second
  argument as the actual text to the pdf. 
\end{enumerate}

\subsection{Other macros}

There are several additional macros for accessing other symbols of the
font. 

\paragraph{\ltxncmd{msmanualspacing}} Turns on the manually-adjusted
spacing for text. See page~\pageref{pag:environments}. 

\paragraph{\ltxncmd{msaltpunctuation}} Switches to use alternate
punctuation marks. 

\paragraph{\ltxcmd{msparinitial}{character}} 
Sets a one-line high Lombardic initial in red. 

\paragraph{\ltxcmd{msparinitialblue}{character}} 
Sets a one-line high Lombardic initial in blue. 

\paragraph{\ltxcmd{msinit}{letter}}

Inserts a Lombardic initial using the current size and color. 

\paragraph{\ltxcmd{msinita}{letter}}

Inserts an alternate form of a Lombardic initial using the current
size and color.

\paragraph{\ltxcmd{msrubric}{text}} 

Inserts \texttt{text} in red. 

\paragraph{\ltxcmd{msrubricblue}{text}} 

Inserts \texttt{text} in blue. 

\paragraph{\ltxncmd{mspara}} 

Inserts a red paragraph symbol.

\paragraph{\ltxcmd{mstdl}{character}} 
Turns on the liturgical variants of the symbols. Namely, it turns
\emph{r} to the response symbol, \emph{v} and \emph{V} to two
different versicle symbols, and the period to the cross.

\paragraph{\ltxncmd{rotundar}}

Inserts a forced \emph{r}-rotunda to a place where ordinary contextual
substitution rules would place a straight \emph{r}.

\paragraph{\ltxncmd{straighr}} 

Inserts a forced straight \emph{r} in a place where the contextual
substitution rules would place a \emph{r}-rotunda.

\paragraph{\ltxncmd{shorts}}

Inserts a forced short \emph{s} in a place where the contextual
substitution rules would place a long \emph{s}.

\paragraph{\ltxncmd{longs}}

Inserts a forced long \emph{s} in a place where the contextual
substitution fules would place a short \emph{s}.


\paragraph{\ltxcmd{msnoliga}{text}} Turns off the standard ligature
substitution for its argument. 


\paragraph{\ltxcmd{mspartialrubric}{text}} Sets \texttt{text} at the
right end of the current line in red. This has to be used together
with \ltxncmd{mspartialline} and in the correct order with rubric
coming before the line.

\paragraph{\ltxcmd{mspartialline}{text}} Sets \texttt{text} 
right end of the current line in red. This has to be used together
with \ltxncmd{mspartialrubric} and in the correct order with rubric
coming before the line.
 


\subsection{Package options}


When \texttt{missaali.sty} is inlcuded in the document, there are some
package-level options available:

\paragraph{\feature{OldAE}} Use old variant forms for \emph{ä} and
\emph{ö} where there is a small \emph{e} letter in place of dieresis
(\feature{hist}).

\paragraph{AltG} Use variant form of \emph{G} (\feature{ss22}).

\paragraph{AltZ} Use variant form of \emph{Z} (\feature{ss19}).

\paragraph{FusePo} Use fused \emph{po}-ligature.

\paragraph{\feature{OldFinnish}} Use 1642 Finnish ortographic
conventions (\feature{ss16}).

After setting the options, you can change to Missaali using the
command \ltxncmd{texturafamily}. This command also changes the
\ltxncmd{emph} command to use red color for emphasis instead of
italics and sets paragraph indentation to zero.

\subsection{Style Definition Macros}

There are several style definition options for enabling some features.
This should be placed before the \ltxncmd{fontspec} command or the
macro or environment that calls it. 

\paragraph{\ltxncmd{MsNormalStyle}}

Revert back to the standard style, cancels the effect of other style
definition macros. 

\paragraph{\ltxncmd{MsAltAEStyle}}

Changes the \emph{æ} and \emph{œ} ligatures to the form where the
\emph{e} is set above the letter instead of being fused into it. This
also converts \emph{ä} and \emph{ö} to the same form. 

\paragraph{\ltxncmd{MsAltPunctuationStyle}}

Changes the comma, the period, and the explanation mark to symbols
that Ghotan used in \emph{Missale Aboense} for similar purposes. His
punctuation conventions don't match the modern usage so this only a
convenient approximation.

\paragraph{\ltxncmd{MsAltZStyle}}

Changes the shape of the lower case \emph{z} to a slightly different
form.

\paragraph{\ltxncmd{MsAltGStyle}}

Changes the shape of the upper case \emph{G} to a different form.

\paragraph{\ltxncmd{MsOldFinnishStyle}}

Turns on the emulation for late 17th century Finnish ortography.

\paragraph{\ltxncmd{MsPoFusionStyle}}

Enables the {\MsPoFusionStyle \missaali{po}} fusion ligature for
\emph{po}. 



\label{pg:latex_end}

\section{How This Font Was Made}

I started the work by looking through the reproduction of
\emph{Missale Aboense} searching for good examples for all different
symbols that were used there. Next, I scanned the symbols in and
created a separate 1000x1000 pixel image for each, making sure to not
alter their relative sizes.

At this point I also consulted Derolez~\cite{derolez},
Cappelli~\cite{cappelli}, and various internet sources trying to
identify the various abbreviation symbols. A very successful strategy
was to search the internet for words next to the abbreviated part as
in most cases the results would contain a page with the full text of
the section. I ended on many different pages and didn't keep track on
what I used.

The first phase of making the actual font was to draw the glyphs based
on the scanned images. I did the tracing by hand as my experiences
with automatic tracing haven't been very encouraging. I used Glyphs
for making the font as it has a good support for OpenType features and
it is much easier to use than FontForge. The version that I used
(1.4.2 (554)) had an annoying bug where it occasionally deleted the
undo history, but on the whole it is the best of all font editors that
I've tried.

After making the basic alphabet and the basic ligatures, I turned to
creating the missing letters \emph{W, Y}, and \emph{Z}. The \emph{Z}
is essentially an enlarged \emph{z} while the other two letters are
based on \emph{U}. 

The next step was to run the \texttt{checkOutlines} script of the
\emph{Adobe FDK} took suite to check that the glyphs were valid. Most
weren't and had to be fixed. At this point I also created the
BlueValues and stem width arrays that are used for hinting the
letters. However, I didn't pay very much attention for them as textura
shouldn't be used at font sizes where hinting is needed.

Creating glyphs with abbreviation marks was a large task, but Glyphs
has a good support for them so it wasn't as overwhelming as it would
have been with TypeTool that I have used for my previous fonts. 

Because there isn't any existing font with exactly the same initials
that Ghotan used, I decided to add them to the font also. Again, some
of them were missing and had to be created from scratch. The
initial~\emph{X} is enlarged from a half-size initial, \emph{Y} is
based on \emph{U}, \emph{Z} is a mirrored alternate form of \emph{S},
and \emph{W} is a rotated \emph{M}. Rotated initials were relatively
common in incunabulas. Ghotan himself used a rotated~\emph{C} in place
of a \emph{D} in several places, and I've seen a text where the
typesetter used a rotated~\emph{Q} for a \emph{D}.

I created the features using a combination of copy-pasting and a few
auxiliary scripts. I found out the \texttt{fontspec} problem with
contextual substitutions by accident and the result was that I had to
almost triple the number of features so that all would work correctly
with XeLaTeX. 

The anachronistic Old Finnish feature was originally made for the font
Fraktuura that is based on late 19th century newspaper fraktur fonts.
That font is still unpublished as it needs one more editing pass
before it is ready for wider audience. 

The last major change for the font was a complete renaming of most of
the glyphs. I didn't follow the recommended conventions for glyph
names from the start as I wasn't aware of how important they really
are for pdf usability. At this point I also created a large number of
duplicate glyphs so that one copy has no code point assigned to it and
the other has. This makes it possible to use the one without the point
in software that supports the relevant features and so get searchable
pdfs, while making the symbol available also in software with no
support for that feature, though, with the cost that the typeset file
might not be searchable. 

The very final step was to run the \emph{autohint} tool of the
\emph{Adobe FDK} to create the font hints. The tool gave a number of
suggetstions how the glyphs could be changed for better hinting. I
followed some but not all of the suggestions as hinting is not so
important for a large-size font as it is for body text fonts.


\section{Symbols in Missaali}

\newcommand{\thd}[1]{\vbox to 4em{\vspace{1em}\hbox{\textbf{#1}}}}

There is a dilemma in creating OpenType fonts that have symbols that
are not included in Unicode: whether to assign code points from the
private use area to them or not and rely completely on OpenType
features to access them. The advantage of assigning code points is
that then the symbols can be used in programs that don't support the
features by entering them via Character Map or some similar utility
program. 

The disadvantage is that using the private use area breaks the search
functionality of most if not all pdf viewers. If the fusion
\missaali{do} is placed into the private use area, then searching for
\emph{door} does not work because the viewer doesn't recognize the
fusion as the letters \emph{d} and \emph{o}.

I resolved this dilemma by making two identical copies of a number of
symbols: one that has a code point and thus can be entered with
Character Map and one that hasn't that can be searched. The
\missaali{do}-fusion that is automatically created as a ligature
substitution can be found with search but the one entered with
Character Map cannot. 

I added duplicates for symbols that are substituted into text using
features \feature{calt}, \feature{liga}, \feature{dlig}, and
\feature{ss01}. The long~\emph{s} and rotunda~\emph{r} are there even
though they have actual unicode code points because not all viewers
can recognize that they are variants of the respective letters.  


\label{pg:symbols_start}

\begin{longtable}{lcccccccccccc}
\thd{A} &
\showlettertitle{\char`\^^^^0041}{0041} &
\showlettertitle{\char`\^^^^0061}{0061} &
\showlettertitle{\char`\^^^^00c4}{00C4} & 
\showlettertitle{\char`\^^^^00c5}{00C5} & 
\showlettertitle{\char`\^^^^00c6}{00C6} &
\showlettertitle{\char`\^^^^00e0}{00E0} &
\showlettertitle{\char`\^^^^00e1}{00E1} &
\showlettertitle{\char`\^^^^00e2}{00E2} \\ &
\showlettertitle{\char`\^^^^00e3}{00E3} &
\showlettertitle{\char`\^^^^00e4}{00E4} &
\showlettertitle{\char`\^^^^00e5}{00E5} &
\showlettertitle{\char`\^^^^00e6}{00E6} &
\showlettertitle{\char`\^^^^01ce}{01CE} &
\showlettertitle{\char`\^^^^01e3}{01E3} &
\showlettertitle{\char`\^^^^01fd}{01FD} & 
\showlettertitle{\char`\^^^^0227}{0227} \\ &
\showlettertitle{\char`\^^^^a733}{A733} & 
\showlettertitle{\char`\^^^^a735}{A735} & 
\showlettertitle{\char`\^^^^a737}{A737} &
\showlettertitle{\char`\^^^^a738}{A738} &
\showlettertitle{\char`\^^^^a73d}{A73D} &
\showlettertitle{\char`\^^^^efe1}{EFE1} &
\showlettertitle{\char`\^^^^efe2}{EFE2} &
\showlettertitle{\char`\^^^^efe5}{EFE5} \\ &
\showlettertitle{\char`\^^^^efe7}{EFE7} &
\showlettertitle{\char`\^^^^efe8}{EFE8} &
\showlettertitle{\char`\^^^^efef}{EFEF} &
\showlettertitle{\char`\^^^^eff1}{EFF1} &
\showlettertitle{\char`\^^^^efff}{EFFF} &
\showlettertitle{\char`\^^^^f400}{F400} &
\showlettertitle{\char`\^^^^f401}{F401} &
\showlettertitle{\char`\^^^^f402}{F402} \\ &
\showlettertitle{\char`\^^^^f403}{F403} & 
\showlettertitle{\char`\^^^^f404}{F404} &
\showlettertitle{\char`\^^^^f405}{F405} &
\showlettertitle{\char`\^^^^f406}{F406} &
\showlettertitle{\char`\^^^^f407}{F407} &
\showlettertitle{\char`\^^^^f408}{F408} &
\showlettertitle{\char`\^^^^f409}{F409} &
\showlettertitle{\char`\^^^^f40a}{F40A} \\ & 
\showlettertitle{\char`\^^^^f40b}{F40B} &
\showlettertitle{\char`\^^^^f40c}{F40C} &
\showlettertitle{\char`\^^^^f40d}{F40D} &
\showlettertitle{\char`\^^^^f40e}{F40E} &
\showlettertitle{\char`\^^^^f40f}{F40F} &
\showlettertitle{\char`\^^^^f410}{F410} &
\showlettertitle{\char`\^^^^f411}{F411} &
\showlettertitle{\char`\^^^^f412}{F412} \\ &
\showlettertitle{\char`\^^^^f413}{F413} &
\showlettertitle{\char`\^^^^f414}{F414} &
\showlettertitle{\char`\^^^^f415}{F415} &
\showlettertitle{\char`\^^^^f416}{F416} &
\showlettertitle{\char`\^^^^f417}{F417} &
\showlettertitle{\char`\^^^^f418}{F418} &
\showlettertitle{\char`\^^^^f419}{F419} &
\showlettertitle{\char`\^^^^f41a}{F41A} \\ &
\showlettertitle{\char`\^^^^f41b}{F41B} &
\showlettertitle{\char`\^^^^f41c}{F41C} &
\showlettertitle{\char`\^^^^f41d}{F41D} &
\showlettertitle{\char`\^^^^f41e}{F41E} &
\showlettertitle{\char`\^^^^f41f}{F41F} &
\showlettertitle{\char`\^^^^f420}{F420} &
\showlettertitle{\char`\^^^^f421}{F421} &
\showlettertitle{\char`\^^^^f422}{F422} \\ &
\showlettertitle{\char`\^^^^f423}{F423} &
\showlettertitle{\char`\^^^^f424}{F424} &
\showlettertitle{\char`\^^^^f425}{F425} &
\showlettertitle{\char`\^^^^f42e}{F42E} &
\showlettertitle{\char`\^^^^f42f}{F42F} &
\showlettertitle{\char`\^^^^f430}{F430} &
\showlettertitle{\char`\^^^^f431}{F431} &
\showlettertitle{\char`\^^^^f432}{F432} \\ &
\showlettertitle{\char`\^^^^f433}{F433} &
\showlettertitle{\char`\^^^^f434}{F434} &
\showlettertitle{\char`\^^^^f435}{F435} &
\showlettertitle{\char`\^^^^f606}{F606} &
\showlettertitle{\char`\^^^^f607}{F607} &
\showlettertitle{\char`\^^^^f608}{F608} &
\showlettertitle{\char`\^^^^f609}{F609} \\
 
\thd{B} &
 
\showlettertitle{\char`\^^^^0042}{0042} &
\showlettertitle{\char`\^^^^0062}{0062} &
\showlettertitle{\char`\^^^^f436}{F436} &
\showlettertitle{\char`\^^^^f437}{F437} &
\showlettertitle{\char`\^^^^f438}{F438} &
\showlettertitle{\char`\^^^^f439}{F439} &
\showlettertitle{\char`\^^^^f43a}{F43A} &
\showlettertitle{\char`\^^^^f43b}{F43B} \\ &
\showlettertitle{\char`\^^^^f43c}{F43C} &
\showlettertitle{\char`\^^^^f43d}{F43D} &
\showlettertitle{\char`\^^^^f43e}{F43E} &
\showlettertitle{\char`\^^^^f43f}{F43F} &
\showlettertitle{\char`\^^^^f440}{F440} &
\showlettertitle{\char`\^^^^f441}{F441} &
\showlettertitle{\char`\^^^^f442}{F442} & 
\showlettertitle{\char`\^^^^f443}{F443} \\ &
\showlettertitle{\char`\^^^^f444}{F444} &
\showlettertitle{\char`\^^^^f445}{F445} &
\showlettertitle{\char`\^^^^f446}{F446} &
\showlettertitle{\char`\^^^^f447}{F447} &
\showlettertitle{\char`\^^^^f60a}{F60A} \\
 
\thd{C} &

\showlettertitle{\char`\^^^^0043}{0043} &
\showlettertitle{\char`\^^^^0063}{0063} &
\showlettertitle{\char`\^^^^0107}{0107} &
\showlettertitle{\char`\^^^^0109}{0109} &
\showlettertitle{\char`\^^^^010b}{010B} &
\showlettertitle{\char`\^^^^010d}{010D} &
\showlettertitle{\char`\^^^^a76f}{A76F} & 
\showlettertitle{\char`\^^^^f448}{F448}  \\ &
\showlettertitle{\char`\^^^^f449}{F449} &
\showlettertitle{\char`\^^^^f44a}{F44A} &
\showlettertitle{\char`\^^^^f44b}{F44B} &
\showlettertitle{\char`\^^^^f44c}{F44C} &
\showlettertitle{\char`\^^^^f44d}{F44D} &
\showlettertitle{\char`\^^^^f44e}{F44E} & 
\showlettertitle{\char`\^^^^f44f}{F44F}  \\ &
\showlettertitle{\char`\^^^^f450}{F450} &
\showlettertitle{\char`\^^^^f451}{F451} &
\showlettertitle{\char`\^^^^f452}{F452} &
\showlettertitle{\char`\^^^^f539}{F539} &
\showlettertitle{\char`\^^^^f60b}{F60B} \\

\thd{D} &

\showlettertitle{\char`\^^^^0044}{0044} &
\showlettertitle{\char`\^^^^0064}{0064} &
\showlettertitle{\char`\^^^^010f}{010F} &
\showlettertitle{\char`\^^^^1e0b}{1E0B} &
\showlettertitle{\char`\^^^^1ebd}{1EBD} &
\showlettertitle{\char`\^^^^a771}{A771} &
\showlettertitle{\char`\^^^^f1a6}{F1A6} &
\showlettertitle{\char`\^^^^f453}{F453} \\ &
\showlettertitle{\char`\^^^^f454}{F454} &
\showlettertitle{\char`\^^^^f455}{F455} &
\showlettertitle{\char`\^^^^f456}{F456} &
\showlettertitle{\char`\^^^^f457}{F457} &
\showlettertitle{\char`\^^^^f458}{F458} & 
\showlettertitle{\char`\^^^^f459}{F459} &
\showlettertitle{\char`\^^^^f45a}{F45A} & 
\showlettertitle{\char`\^^^^f45b}{F45B} \\ &
\showlettertitle{\char`\^^^^f45c}{F45C} &
\showlettertitle{\char`\^^^^f45d}{F45D} &
\showlettertitle{\char`\^^^^f45e}{F45E} &
\showlettertitle{\char`\^^^^f45f}{F45F} &
\showlettertitle{\char`\^^^^f460}{F460} &
\showlettertitle{\char`\^^^^f461}{F461} &
\showlettertitle{\char`\^^^^f462}{F462} &
\showlettertitle{\char`\^^^^f463}{F463} \\ &
\showlettertitle{\char`\^^^^f464}{F464} &
\showlettertitle{\char`\^^^^f465}{F465} & 
\showlettertitle{\char`\^^^^f466}{F466} &
\showlettertitle{\char`\^^^^f467}{F467} &
\showlettertitle{\char`\^^^^f468}{F468} &
\showlettertitle{\char`\^^^^f469}{F469} &
\showlettertitle{\char`\^^^^f46a}{F46A} &
\showlettertitle{\char`\^^^^f60c}{F60C} \\ &
\showlettertitle{\char`\^^^^f60d}{F60D} \\

\thd{E} &

\showlettertitle{\char`\^^^^0045}{0045} &
\showlettertitle{\char`\^^^^0065}{0065} &
\showlettertitle{\char`\^^^^00e8}{00E8} &
\showlettertitle{\char`\^^^^00e9}{00E9} &
\showlettertitle{\char`\^^^^00ea}{00EA} &
\showlettertitle{\char`\^^^^00eb}{00EB} &
\showlettertitle{\char`\^^^^0117}{0117} & 
\showlettertitle{\char`\^^^^011b}{011B} \\ &
\showlettertitle{\char`\^^^^f158}{F158} & 
\showlettertitle{\char`\^^^^f46b}{F46B} &
\showlettertitle{\char`\^^^^f53b}{F53B} &
\showlettertitle{\char`\^^^^f60e}{F60E} \\
 
\thd{F} &

\showlettertitle{\char`\^^^^0046}{0046} &
\showlettertitle{\char`\^^^^0066}{0066} &
\showlettertitle{\char`\^^^^eecb}{EECB} &
\showlettertitle{\char`\^^^^fb00}{FB00} &
\showlettertitle{\char`\^^^^fb01}{FB01} &
\showlettertitle{\char`\^^^^fb02}{FB02} & 
\showlettertitle{\char`\^^^^f60f}{F60F} \\

\thd{G} &
\showlettertitle{\char`\^^^^0047}{0047} &
\showlettertitle{\char`\^^^^0067}{0067} &
\showlettertitle{\char`\^^^^011d}{011D} &
\showlettertitle{\char`\^^^^0121}{0121} &
\showlettertitle{\char`\^^^^01e7}{01E7} &
\showlettertitle{\char`\^^^^01f5}{01F5} & 
\showlettertitle{\char`\^^^^f42d}{F42D} &
\showlettertitle{\char`\^^^^f46c}{F46C} \\ &
\showlettertitle{\char`\^^^^f46d}{F46D} &
\showlettertitle{\char`\^^^^f46e}{F46E} &
\showlettertitle{\char`\^^^^f46f}{F46F} &
\showlettertitle{\char`\^^^^f610}{F610} \\
 

\thd{H} &

\showlettertitle{\char`\^^^^0048}{0048} &
\showlettertitle{\char`\^^^^0068}{0068} &
\showlettertitle{\char`\^^^^f53d}{F53D} &
\showlettertitle{\char`\^^^^f611}{F611} \\

\thd{I} &

\showlettertitle{\char`\^^^^0049}{0049} &
\showlettertitle{\char`\^^^^0069}{0069} &
\showlettertitle{\char`\^^^^00ec}{00EC} &
\showlettertitle{\char`\^^^^00ed}{00ED} &
\showlettertitle{\char`\^^^^00ee}{00EE} &
\showlettertitle{\char`\^^^^00ef}{00EF} & 
\showlettertitle{\char`\^^^^0129}{0129} & 
\showlettertitle{\char`\^^^^0131}{0131} \\ &
\showlettertitle{\char`\^^^^0133}{0133} &
\showlettertitle{\char`\^^^^a76d}{A76D} &
\showlettertitle{\char`\^^^^f470}{F470} &
\showlettertitle{\char`\^^^^f612}{F612} &
\showlettertitle{\char`\^^^^f613}{F613} \\

\thd{J} & 

\showlettertitle{\char`\^^^^004a}{004A} &
\showlettertitle{\char`\^^^^006a}{006A} &
\showlettertitle{\char`\^^^^0135}{0135} &
\showlettertitle{\char`\^^^^01f0}{01F0} &
\showlettertitle{\char`\^^^^0237}{0237} &
\showlettertitle{\char`\^^^^e554}{E554} &
\showlettertitle{\char`\^^^^f471}{F471} & 
\showlettertitle{\char`\^^^^f472}{F472} \\ &
\showlettertitle{\char`\^^^^f473}{F473} &
\showlettertitle{\char`\^^^^f474}{F474} &
\showlettertitle{\char`\^^^^f614}{F614} \\


\thd{K} &

\showlettertitle{\char`\^^^^004b}{004B} &
\showlettertitle{\char`\^^^^006b}{006B} &
\showlettertitle{\char`\^^^^a740}{A740} &
\showlettertitle{\char`\^^^^a741}{A741} &
\showlettertitle{\char`\^^^^a742}{A742} &
\showlettertitle{\char`\^^^^a743}{A743} &
\showlettertitle{\char`\^^^^f615}{F615} \\

\thd{L} &

\showlettertitle{\char`\^^^^004c}{004C} &
\showlettertitle{\char`\^^^^006c}{006C} &
\showlettertitle{\char`\^^^^a772}{A772} & 
\showlettertitle{\char`\^^^^f616}{F616} \\




\thd{M} & 

\showlettertitle{\char`\^^^^004d}{004D} & 
\showlettertitle{\char`\^^^^006d}{006D} &
\showlettertitle{\char`\^^^^1e3f}{1E3F} &
\showlettertitle{\char`\^^^^a773}{A773} &
\showlettertitle{\char`\^^^^f476}{F476} &
\showlettertitle{\char`\^^^^f477}{F477} &
\showlettertitle{\char`\^^^^f478}{F478} &
\showlettertitle{\char`\^^^^f479}{F479} \\ &
\showlettertitle{\char`\^^^^f47a}{F47A} &   
\showlettertitle{\char`\^^^^f47b}{F47B} &    
\showlettertitle{\char`\^^^^f47c}{F47C} &   
\showlettertitle{\char`\^^^^f617}{F617} \\   
                                          
\thd{N} &                                   
                                           
\showlettertitle{\char`\^^^^004e}{004E} &   
\showlettertitle{\char`\^^^^006e}{006E} &   
\showlettertitle{\char`\^^^^00f1}{00F1} &   
\showlettertitle{\char`\^^^^0144}{0144} &   
\showlettertitle{\char`\^^^^0148}{0148} &   
\showlettertitle{\char`\^^^^01f9}{01F9} &   
\showlettertitle{\char`\^^^^a774}{A774} &   
\showlettertitle{\char`\^^^^f47d}{F47D} \\ &
\showlettertitle{\char`\^^^^f47e}{F47E} &
\showlettertitle{\char`\^^^^f47f}{F47F} &
\showlettertitle{\char`\^^^^f618}{F618} \\

\thd{O} & 

\showlettertitle{\char`\^^^^004f}{004F} &   
\showlettertitle{\char`\^^^^006f}{006F} &   
\showlettertitle{\char`\^^^^00d6}{00D6} &   
\showlettertitle{\char`\^^^^00f2}{00F2} &   
\showlettertitle{\char`\^^^^00f3}{00F3} &   
\showlettertitle{\char`\^^^^00f4}{00F4} &   
\showlettertitle{\char`\^^^^00f5}{00F5} &   
\showlettertitle{\char`\^^^^00f6}{00F6} \\ &
\showlettertitle{\char`\^^^^0153}{0153} &   
\showlettertitle{\char`\^^^^01d2}{01D2} &   
\showlettertitle{\char`\^^^^022f}{022F} &   
\showlettertitle{\char`\^^^^a74d}{A74D} &   
\showlettertitle{\char`\^^^^a74f}{A74F} &   
\showlettertitle{\char`\^^^^ebe5}{EBE5} &   
\showlettertitle{\char`\^^^^f426}{F426} &   
\showlettertitle{\char`\^^^^f427}{F427} \\ &
\showlettertitle{\char`\^^^^f428}{F428} &   
\showlettertitle{\char`\^^^^f429}{F429} &   
\showlettertitle{\char`\^^^^f42a}{F42A} &   
\showlettertitle{\char`\^^^^f42b}{F42B} &   
\showlettertitle{\char`\^^^^f42c}{F42C} &   
\showlettertitle{\char`\^^^^f480}{F480} &   
\showlettertitle{\char`\^^^^f481}{F481} &   
\showlettertitle{\char`\^^^^f482}{F482} \\ &
\showlettertitle{\char`\^^^^f483}{F483} &   
\showlettertitle{\char`\^^^^f484}{F484} &   
\showlettertitle{\char`\^^^^f485}{F485} &   
\showlettertitle{\char`\^^^^f486}{F486} &   
\showlettertitle{\char`\^^^^f487}{F487} &   
\showlettertitle{\char`\^^^^f488}{F488} &   
\showlettertitle{\char`\^^^^f489}{F489} &   
\showlettertitle{\char`\^^^^f48a}{F48A} \\ &
\showlettertitle{\char`\^^^^f619}{F619} &
\showlettertitle{\char`\^^^^f61a}{F61A} \\

\thd{P} &
\showlettertitle{\char`\^^^^0050}{0050} &   
\showlettertitle{\char`\^^^^0070}{0070} &   
\showlettertitle{\char`\^^^^1e55}{1E55} &   
\showlettertitle{\char`\^^^^1e57}{1E57} &   
\showlettertitle{\char`\^^^^a750}{A750} &   
\showlettertitle{\char`\^^^^a751}{A751} &   
\showlettertitle{\char`\^^^^a752}{A752} &   
\showlettertitle{\char`\^^^^a753}{A753} \\ &
\showlettertitle{\char`\^^^^a755}{A755} &   
\showlettertitle{\char`\^^^^eed6}{EED6} &   
\showlettertitle{\char`\^^^^eed7}{EED7} &   
\showlettertitle{\char`\^^^^f48b}{F48B} &   
\showlettertitle{\char`\^^^^f48c}{F48C} &   
\showlettertitle{\char`\^^^^f48d}{F48D} &   
\showlettertitle{\char`\^^^^f48e}{F48E} &   
\showlettertitle{\char`\^^^^f48f}{F48F} \\ &
\showlettertitle{\char`\^^^^f490}{F490} &   
\showlettertitle{\char`\^^^^f491}{F491} &   
\showlettertitle{\char`\^^^^f492}{F492} &   
\showlettertitle{\char`\^^^^f493}{F493} &   
\showlettertitle{\char`\^^^^f494}{F494} &   
\showlettertitle{\char`\^^^^f495}{F495} &   
\showlettertitle{\char`\^^^^f496}{F496} &   
\showlettertitle{\char`\^^^^f497}{F497} \\ &
\showlettertitle{\char`\^^^^f498}{F498} &   
\showlettertitle{\char`\^^^^f499}{F499} &   
\showlettertitle{\char`\^^^^f49a}{F49A} &   
\showlettertitle{\char`\^^^^f49b}{F49B} &   
\showlettertitle{\char`\^^^^f49c}{F49C} &   
\showlettertitle{\char`\^^^^f49d}{F49D} &   
\showlettertitle{\char`\^^^^f49e}{F49E} &   
\showlettertitle{\char`\^^^^f49f}{F49F} \\ &
\showlettertitle{\char`\^^^^f4a0}{F4A0} &   
\showlettertitle{\char`\^^^^f4a1}{F4A1} &   
\showlettertitle{\char`\^^^^f4a2}{F4A2} &   
\showlettertitle{\char`\^^^^f4a3}{F4A3} &   
\showlettertitle{\char`\^^^^f4a4}{F4A4} &   
\showlettertitle{\char`\^^^^f4a5}{F4A5} &   
\showlettertitle{\char`\^^^^f4a6}{F4A6} &   
\showlettertitle{\char`\^^^^f4a7}{F4A7} \\ &
\showlettertitle{\char`\^^^^f4a8}{F4A8} &   
\showlettertitle{\char`\^^^^f4a9}{F4A9} &   
\showlettertitle{\char`\^^^^f4aa}{F4AA} &   
\showlettertitle{\char`\^^^^f4ab}{F4AB} &   
\showlettertitle{\char`\^^^^f4ac}{F4AC} &   
\showlettertitle{\char`\^^^^f4ad}{F4AD} &   
\showlettertitle{\char`\^^^^f4ae}{F4AE} &   
\showlettertitle{\char`\^^^^f4af}{F4AF} \\ &
\showlettertitle{\char`\^^^^f4b0}{F4B0} &   
\showlettertitle{\char`\^^^^f4b1}{F4B1} &   
\showlettertitle{\char`\^^^^f4b2}{F4B2} &   
\showlettertitle{\char`\^^^^f4b3}{F4B3} &   
\showlettertitle{\char`\^^^^f4b4}{F4B4} &   
\showlettertitle{\char`\^^^^f4b5}{F4B5} &   
\showlettertitle{\char`\^^^^f4b6}{F4B6} &   
\showlettertitle{\char`\^^^^f4b7}{F4B7} \\ &
\showlettertitle{\char`\^^^^f4b8}{F4B8} &   
\showlettertitle{\char`\^^^^f4b9}{F4B9} &   
\showlettertitle{\char`\^^^^f4ba}{F4BA} &   
\showlettertitle{\char`\^^^^f4bb}{F4BB} &   
\showlettertitle{\char`\^^^^f4bc}{F4BC} &   
\showlettertitle{\char`\^^^^f4bd}{F4BD} &   
\showlettertitle{\char`\^^^^f4be}{F4BE} &   
\showlettertitle{\char`\^^^^f4bf}{F4BF} \\ &
\showlettertitle{\char`\^^^^f4c0}{F4C0} &   
\showlettertitle{\char`\^^^^f4c1}{F4C1} &   
\showlettertitle{\char`\^^^^f4c2}{F4C2} &   
\showlettertitle{\char`\^^^^f4c3}{F4C3} &   
\showlettertitle{\char`\^^^^f4c4}{F4C4} &   
\showlettertitle{\char`\^^^^f4c5}{F4C5} &   
\showlettertitle{\char`\^^^^f4c6}{F4C6} &   
\showlettertitle{\char`\^^^^f4c7}{F4C7} \\ &
\showlettertitle{\char`\^^^^f4c8}{F4C8}  &
\showlettertitle{\char`\^^^^f4c9}{F4C9} &
\showlettertitle{\char`\^^^^f4ca}{F4CA} &
\showlettertitle{\char`\^^^^f61b}{F61B} \\

\thd{Q} &

\showlettertitle{\char`\^^^^0051}{0051} &   
\showlettertitle{\char`\^^^^0071}{0071} &   
\showlettertitle{\char`\^^^^a757}{A757} &   
\showlettertitle{\char`\^^^^a759}{A759} &   
\showlettertitle{\char`\^^^^e8b3}{E8B3} &   
\showlettertitle{\char`\^^^^e8bf}{E8BF} &   
\showlettertitle{\char`\^^^^f4cb}{F4CB} &   
\showlettertitle{\char`\^^^^f4cc}{F4CC} \\ &
\showlettertitle{\char`\^^^^f4cd}{F4CD} &   
\showlettertitle{\char`\^^^^f4ce}{F4CE} &   
\showlettertitle{\char`\^^^^f4cf}{F4CF} &   
\showlettertitle{\char`\^^^^f4d0}{F4D0} &   
\showlettertitle{\char`\^^^^f4d1}{F4D1} &   
\showlettertitle{\char`\^^^^f4d2}{F4D2} &   
\showlettertitle{\char`\^^^^f4d3}{F4D3} &   
\showlettertitle{\char`\^^^^f4d4}{F4D4} \\ &
\showlettertitle{\char`\^^^^f4d5}{F4D5} &   
\showlettertitle{\char`\^^^^f4d6}{F4D6} &   
\showlettertitle{\char`\^^^^f4d7}{F4D7} &   
\showlettertitle{\char`\^^^^f4d8}{F4D8} &   
\showlettertitle{\char`\^^^^f4d9}{F4D9} &   
\showlettertitle{\char`\^^^^f4da}{F4DA} &   
\showlettertitle{\char`\^^^^f4db}{F4DB} &   
\showlettertitle{\char`\^^^^f4dc}{F4DC} \\ &
\showlettertitle{\char`\^^^^f4dd}{F4DD} &   
\showlettertitle{\char`\^^^^f4de}{F4DE} &   
\showlettertitle{\char`\^^^^f4df}{F4DF} &   
\showlettertitle{\char`\^^^^f4e0}{F4E0} &   
\showlettertitle{\char`\^^^^f4e1}{F4E1} &   
\showlettertitle{\char`\^^^^f4e2}{F4E2} &   
\showlettertitle{\char`\^^^^f4e3}{F4E3} &   
\showlettertitle{\char`\^^^^f4e4}{F4E4} \\ &
\showlettertitle{\char`\^^^^f4e5}{F4E5} &   
\showlettertitle{\char`\^^^^f4e6}{F4E6} &   
\showlettertitle{\char`\^^^^f4e7}{F4E7} &   
\showlettertitle{\char`\^^^^f4e8}{F4E8} &   
\showlettertitle{\char`\^^^^f4e9}{F4E9} &   
\showlettertitle{\char`\^^^^f4ea}{F4EA} &   
\showlettertitle{\char`\^^^^f61c}{F61C} \\
                                        
\thd{R} &

\showlettertitle{\char`\^^^^0052}{0052} &   
\showlettertitle{\char`\^^^^0072}{0072} &   
\showlettertitle{\char`\^^^^0155}{0155} &   
\showlettertitle{\char`\^^^^0159}{0159} &   
\showlettertitle{\char`\^^^^1e59}{1E59} &   
\showlettertitle{\char`\^^^^211e}{211E} &   
\showlettertitle{\char`\^^^^a75b}{A75B} &   
\showlettertitle{\char`\^^^^a75d}{A75D} \\ &
\showlettertitle{\char`\^^^^a775}{A775} &   
\showlettertitle{\char`\^^^^f4eb}{F4EB} &   
\showlettertitle{\char`\^^^^f4ec}{F4EC} &   
\showlettertitle{\char`\^^^^f4ed}{F4ED} &   
\showlettertitle{\char`\^^^^f4ee}{F4EE} &   
\showlettertitle{\char`\^^^^f4ef}{F4EF} &   
\showlettertitle{\char`\^^^^f4f0}{F4F0} &   
\showlettertitle{\char`\^^^^f4f1}{F4F1} \\ &
\showlettertitle{\char`\^^^^f4f2}{F4F2} &   
\showlettertitle{\char`\^^^^f4f3}{F4F3} &   
\showlettertitle{\char`\^^^^f4f4}{F4F4} &   
\showlettertitle{\char`\^^^^f4f5}{F4F5} &   
\showlettertitle{\char`\^^^^f4f6}{F4F6} &   
\showlettertitle{\char`\^^^^f4f7}{F4F7} &   
\showlettertitle{\char`\^^^^f4f8}{F4F8} &   
\showlettertitle{\char`\^^^^f4f9}{F4F9} \\ &
\showlettertitle{\char`\^^^^f4fa}{F4FA} &   
\showlettertitle{\char`\^^^^f4fb}{F4FB} &   
\showlettertitle{\char`\^^^^f4fc}{F4FC} &   
\showlettertitle{\char`\^^^^f4fd}{F4FD} &   
\showlettertitle{\char`\^^^^f4fe}{F4FE} &   
\showlettertitle{\char`\^^^^f4ff}{F4FF} &   
\showlettertitle{\char`\^^^^f500}{F500} &   
\showlettertitle{\char`\^^^^f501}{F501} \\ &
\showlettertitle{\char`\^^^^f502}{F502} &   
\showlettertitle{\char`\^^^^f503}{F503} &   
\showlettertitle{\char`\^^^^f504}{F504} &   
\showlettertitle{\char`\^^^^f505}{F505} &   
\showlettertitle{\char`\^^^^f506}{F506} &   
\showlettertitle{\char`\^^^^f507}{F507} &   
\showlettertitle{\char`\^^^^f508}{F508} &   
\showlettertitle{\char`\^^^^f61d}{F61D} \\ 

\thd{S} & 

\showlettertitle{\char`\^^^^0053}{0053} &   
\showlettertitle{\char`\^^^^0073}{0073} &   
\showlettertitle{\char`\^^^^00df}{00DF} &   
\showlettertitle{\char`\^^^^015d}{015D} &   
\showlettertitle{\char`\^^^^017f}{017F} &   
\showlettertitle{\char`\^^^^eba2}{EBA2} &   
\showlettertitle{\char`\^^^^eba3}{EBA3} &   
\showlettertitle{\char`\^^^^eba6}{EBA6} \\ &
\showlettertitle{\char`\^^^^f475}{F475} &   
\showlettertitle{\char`\^^^^f509}{F509} &   
\showlettertitle{\char`\^^^^f50a}{F50A} &   
\showlettertitle{\char`\^^^^f50b}{F50B} &   
\showlettertitle{\char`\^^^^f53c}{F53C} &   
\showlettertitle{\char`\^^^^fb05}{FB05} &   
\showlettertitle{\char`\^^^^f61e}{F61E} &   
\showlettertitle{\char`\^^^^f61f}{F61F} \\ 

\thd{T} &

\showlettertitle{\char`\^^^^0054}{0054} &   
\showlettertitle{\char`\^^^^0074}{0074} &   
\showlettertitle{\char`\^^^^0165}{0165} &   
\showlettertitle{\char`\^^^^1e6b}{1E6B} &   
\showlettertitle{\char`\^^^^1e97}{1E97} &   
\showlettertitle{\char`\^^^^f50c}{F50C} &   
\showlettertitle{\char`\^^^^f50d}{F50D} &   
\showlettertitle{\char`\^^^^f50e}{F50E} \\ &
\showlettertitle{\char`\^^^^f53a}{F53A} &
\showlettertitle{\char`\^^^^f620}{F620} \\

\thd{U} &

\showlettertitle{\char`\^^^^0055}{0055} &   
\showlettertitle{\char`\^^^^0075}{0075} &   
\showlettertitle{\char`\^^^^00dc}{00DC} &   
\showlettertitle{\char`\^^^^00f9}{00F9} &   
\showlettertitle{\char`\^^^^00fa}{00FA} &   
\showlettertitle{\char`\^^^^00fb}{00FB} &   
\showlettertitle{\char`\^^^^00fc}{00FC} &   
\showlettertitle{\char`\^^^^0169}{0169} \\ &
\showlettertitle{\char`\^^^^016f}{016F} &   
\showlettertitle{\char`\^^^^01d4}{01D4} &   
\showlettertitle{\char`\^^^^a770}{A770} &   
\showlettertitle{\char`\^^^^f510}{F510} &   
\showlettertitle{\char`\^^^^f511}{F511} &   
\showlettertitle{\char`\^^^^f621}{F621} \\
                                           
                                        
\thd{V} & 

\showlettertitle{\char`\^^^^0056}{0056} &   
\showlettertitle{\char`\^^^^0076}{0076} &   
\showlettertitle{\char`\^^^^1e7d}{1E7D} &   
\showlettertitle{\char`\^^^^2123}{2123} &   
\showlettertitle{\char`\^^^^a75f}{A75F} &   
\showlettertitle{\char`\^^^^f512}{F512} &   
\showlettertitle{\char`\^^^^f513}{F513} &   
\showlettertitle{\char`\^^^^f514}{F514} \\ &
\showlettertitle{\char`\^^^^f515}{F515} &   
\showlettertitle{\char`\^^^^f516}{F516} &   
\showlettertitle{\char`\^^^^f517}{F517} &   
\showlettertitle{\char`\^^^^f518}{F518} &   
\showlettertitle{\char`\^^^^f519}{F519} &   
\showlettertitle{\char`\^^^^f51a}{F51A} &   
\showlettertitle{\char`\^^^^f51b}{F51B} &   
\showlettertitle{\char`\^^^^f51c}{F51C} \\ &
\showlettertitle{\char`\^^^^f51d}{F51D} &   
\showlettertitle{\char`\^^^^f51e}{F51E} &   
\showlettertitle{\char`\^^^^f51f}{F51F} &   
\showlettertitle{\char`\^^^^f520}{F520} &   
\showlettertitle{\char`\^^^^f521}{F521} &   
\showlettertitle{\char`\^^^^f522}{F522} &   
\showlettertitle{\char`\^^^^f622}{F622} \\   
                                        

\thd{W} &

\showlettertitle{\char`\^^^^0057}{0057} &   
\showlettertitle{\char`\^^^^0077}{0077} &   
\showlettertitle{\char`\^^^^0175}{0175} &   
\showlettertitle{\char`\^^^^1e81}{1E81} &   
\showlettertitle{\char`\^^^^1e83}{1E83} &   
\showlettertitle{\char`\^^^^1e85}{1E85} &   
\showlettertitle{\char`\^^^^1e87}{1E87} &   
\showlettertitle{\char`\^^^^1e98}{1E98} \\ &
\showlettertitle{\char`\^^^^f523}{F523} &
\showlettertitle{\char`\^^^^f524}{F524} &
\showlettertitle{\char`\^^^^f525}{F525} & 
\showlettertitle{\char`\^^^^f623}{F623} \\

\thd{X} & 

\showlettertitle{\char`\^^^^0058}{0058} &   
\showlettertitle{\char`\^^^^0078}{0078} &   
\showlettertitle{\char`\^^^^1e8b}{1E8B} &   
\showlettertitle{\char`\^^^^1e8d}{1E8D} &   
\showlettertitle{\char`\^^^^f526}{F526} &   
\showlettertitle{\char`\^^^^f527}{F527} &   
\showlettertitle{\char`\^^^^f528}{F528} &   
\showlettertitle{\char`\^^^^f529}{F529} \\ &
\showlettertitle{\char`\^^^^f52a}{F52A} &
\showlettertitle{\char`\^^^^f52b}{F52B} & 
\showlettertitle{\char`\^^^^f624}{F624} \\

\thd{Y} &

\showlettertitle{\char`\^^^^0059}{0059} &   
\showlettertitle{\char`\^^^^0079}{0079} &   
\showlettertitle{\char`\^^^^00fd}{00FD} &   
\showlettertitle{\char`\^^^^00ff}{00FF} &   
\showlettertitle{\char`\^^^^0177}{0177} &   
\showlettertitle{\char`\^^^^1e8f}{1E8F} &   
\showlettertitle{\char`\^^^^1e99}{1E99} &   
\showlettertitle{\char`\^^^^1ef3}{1EF3} \\ &
\showlettertitle{\char`\^^^^1ef9}{1EF9} &
\showlettertitle{\char`\^^^^f52c}{F52C} &
\showlettertitle{\char`\^^^^f625}{F625} \\

\thd{Z} &

\showlettertitle{\char`\^^^^005a}{005A} &   
\showlettertitle{\char`\^^^^007a}{007A} &   
\showlettertitle{\char`\^^^^017a}{017A} &   
\showlettertitle{\char`\^^^^017c}{017C} &   
\showlettertitle{\char`\^^^^017e}{017E} &   
\showlettertitle{\char`\^^^^01ef}{01EF} &   
\showlettertitle{\char`\^^^^1e91}{1E91} &   
\showlettertitle{\char`\^^^^f50f}{F50F} \\ &
\showlettertitle{\char`\^^^^f52d}{F52D} &   
\showlettertitle{\char`\^^^^f52e}{F52E} &   
\showlettertitle{\char`\^^^^f52f}{F52F} &   
\showlettertitle{\char`\^^^^f530}{F530} &   
\showlettertitle{\char`\^^^^f531}{F531} &   
\showlettertitle{\char`\^^^^f532}{F532} &   
\showlettertitle{\char`\^^^^f533}{F533} &   
\showlettertitle{\char`\^^^^f534}{F534} \\ &
\showlettertitle{\char`\^^^^f535}{F535} &
\showlettertitle{\char`\^^^^f536}{F536} &
\showlettertitle{\char`\^^^^f537}{F537} &
\showlettertitle{\char`\^^^^f538}{F538} &
\showlettertitle{\char`\^^^^f626}{F626} \\

\multicolumn{4}{l}{\thd{Other symbols}} \\ &
\showlettertitle{\char`\^^^^0021}{0021} &   
\showlettertitle{\char`\^^^^0026}{0026} &   
\showlettertitle{\char`\^^^^002a}{002A} &   
\showlettertitle{\char`\^^^^002c}{002C} &   
\showlettertitle{\char`\^^^^002d}{002D} &   
\showlettertitle{\char`\^^^^002e}{002E} &   
\showlettertitle{\char`\^^^^002f}{002F} &   
\showlettertitle{\char`\^^^^003a}{003A} \\ &
\showlettertitle{\char`\^^^^003b}{003B} &   
\showlettertitle{\char`\^^^^003f}{003F} &   
\showlettertitle{\char`\^^^^005c}{005C} &   
\showlettertitle{\char`\^^^^0060}{0060} &   
\showlettertitle{\char`\^^^^007c}{007C} &   
\showlettertitle{\char`\^^^^00a8}{00A8} &   
\showlettertitle{\char`\^^^^00af}{00AF} &   
\showlettertitle{\char`\^^^^00b4}{00B4} \\ &
\showlettertitle{\char`\^^^^00b7}{00B7} &   
\showlettertitle{\char`\^^^^00b6}{00B6} &   
\showlettertitle{\char`\^^^^02c6}{02C6} &   
\showlettertitle{\char`\^^^^02c7}{02C7} &   
\showlettertitle{\char`\^^^^02d9}{02D9} &   
\showlettertitle{\char`\^^^^02da}{02DA} &   
\showlettertitle{\char`\^^^^02dc}{02DC} &   
\showlettertitle{\char`\^^^^2012}{2012} \\ &
\showlettertitle{\char`\^^^^2013}{2013} &   
\showlettertitle{\char`\^^^^2014}{2014} \\   

\end{longtable}




\label{pg:symbols_end}


\section{Examples}

This section has two longish examples of how medieval latin text can
be set using Missaali. The first example is a page containing one poem
and parts of two others, and the second is the first half of the
Legend of St. Henry.

The complete poem is Quintus Horatius Flaccus's \emph{Integer
  vitae}~\cite{horatius}, from his first book of Odes. The text has
divided opinions on whether it is Horace intended it to be taken as a
serious exultation of honest life or a satire of such poems. The text
has been a popular students' song since Friedrich Flemming composed a
melody for it in the early 19th century.

The first partial poem are three verses from \emph{Estuans
  intrinsecus} by The Archpoet. He was a poet who was active around
the mid-12th century and was somehow associated with Rainald of
Dassel, Archbischop of Cologne. \emph{Estuans intrinsecus} was very
popular during the middle ages and it survives on over 30 manuscripts,
including the famous \emph{Carmina Burana}~\cite{carmina}. The verses
in this sample belong to the \emph{Meum est propositum} segment of the
poem. 

The final partial poem is a snippet from \emph{In taberna} that is
also from \emph{Carmina Burana}. 

The second example contains first five of the nine lections of the
legend of Saint Henry. They contain the \emph{vita} or the biography
of the saint as well as the first of the miracles attributed to him.
The text follows the critical edition by Tuomas
Heikkilä~\cite{legenda} that is based on a large number of surviving
manuscripts.


\newpage
\label{pg_carmina_sample}

\MsAltPunctiationStyle
\mssetsizes{18}{6.4cm}{33}
\begin{mstexturablocks}
\msrubric{Q Horatius fl carm{in}a \rmstdacc{pr}{prim}o}
\msstartchapterwithrubric{xxii}{I}{Nteger vitae scele- }{%
  leris{que} pur\msabbr{us} n{on} eget}
mauris iaculis ne\msabbr{que} arcu nec 
venenatis gravida sagittis, 
Fusce, pharetra, sive \msabbr{per} syrtis 
iter aestuosas sive factur\msabbr{us} {per} 
in hospital{em} caucasum vel \msabbr{quae} 
loca fabulos{us} lambit hyda- 
spes. \msparinitial{N}am\msabbr{que} me silva lupus 
in sabina, d{um} meam canto 
lalagen et ultra termin\mstdacc{u}{um} cu- 
ris vagor expeditis, fugit in-
ermem: \msabbr{qua}le portentum ne\msabbr{que} 
militaris daunias latis alit 
aesculetisnec Jubae tell\msabbr{us} ge- 
nerat, leonum arida nutrix. 
\msparinitial{P}one me pigris ubi nulla 
campis arbor aestiva recrea\msabbr{tur} 
aura, quod latus mundi ne- 
bulae malusque Iuppiter 
urget: p{on}e sub curru nimi- 
um {pro}pin{qui} solis in terra 
{dom}i\msabbr{bus} negata: dulce riden-
tem lalag{en} amabo, dulce lo-

\msstartchapterwithrubric{Archipoeta}{M}{quentem. }{%
Eum est propositum}
in tabe{rn}a mori, ubi vina {pro}
xima  morientis ori. tunc can-
tabunt leti{us} angelo\msabbr{rum} chori: 
sit {deu}s \msabbr{prop}itius isti potatori.
\msparinitial{P}oculis accenditur animi 
lucerna, cor imbutum necta- 
re volat ad superna. michi 
sapit dulcius vin{um} de taber- 
na, {quam} quod aqua miscuit
{prae}sulis p\mstdacc{i}{in}cerna. \msparinitial{T}ales ver-
sus facio, {qua}le vinum bibo,
nichil possum fac\mstdracc{e}{er}e nisi s\mstdacc{u}{um}p- 
to cibo: nichil valent \mstdacc{pe}{pen}itus,
{que} ieiunus scribo, Nasonem 
post calices carmine preibo.

\mspartialrubric{Incognit\msabbr{\msrubric{us}}}
\mspartialline{\initialI{}N taberna qu\mstdacc{a}{an}do  }
sumus: non curam{us} {quid} sit 
humus, sed ad ludum \msabbr{prop}era- 
mus, cui semper insudamus,
mus, quid aga{tur} in taberna
, ubi nu{mm}us est pincerna,
hoc est opus, ut queratur,
sed {quid} loquar audia{tur}.
\msparinitial{Q}uidam ludunt, {qui}dam
bib\mstdacc{u}{un}t, {qui}dam indiscrete vi-
vunt, {sed} in ludo {qui} moran-
tur, ex his {qui}dam denudan- 
tur, quid{am} ibi vestiun{tur},
quidam saccis induuntur, ibi 
nullus timet mortem, sed
pro Baccho mittunt sortem.
\end{mstexturablocks}

\thispagestyle{empty}

\newpage


\pagestyle{empty}

\mssetsizes{18}{6.4cm}{33}
\begin{mstextura}
\msrubric{Incipit vita {\mstddgacc{S}{sancto}} henrici e\mstdacc{p}{piscop}i}
\msstartchapterwithrubric{Lectio}{R}{Egnante illus-\ }{trissimo rege sancto}
Erico in Svecia, venerabilis
pontifex beatus Henricus, in 
Anglia ori\mstdacc{u}{un}d\msabbr{us}, \msabbr{con}spicuus vite
sanctiate et mo\msabbr{rum} honestate
preclarus, Upsalensem rege- 
bat ecclesiam. Quibus quasi duobus magnis luminari\msabbr{bus}
populus terre ilius, ad veri
dei noticiam \msabbr{et} cultum magis 
magis\msabbr{que} illustrabatur iugi\mstdracc{t}{ter}
ac informabatur. Sanctus 
autem rex insignem ponti- 
ficem, vita, moribus et eccle- siastice dignitatis culmine 
prepollentem, intimi amoris
complectebatur affectu \msabbr{et} sue 
specialis f\mstdacc{a}{am}iliaritatis gratia 
plurimum honorabat. Felix 
patria, \msabbr{quam} divina maiestas 
concesserat talium ac tanto\msabbr{rum} 
recto\msabbr{rum} eodem tempore mode- 
ramine gubernari. Non erat 
tunc formidini, ut regnum in 
se divisum desolaretur, dum 
ad gloriam dei, et iust\mstdacc{u}{um} tran- 
quillumque regimen sub- dito{rum} capita populorum 
tam concordit\mstdracc{e}{er} convenerunt. 

\msstartchapterwithrubric{Lectio}{E}{Dificabatur illo}{%
in t\mstdacc{e}{em}{por}e e{ccles}ia cres-}
cens in timore dei, rectifica- 
bantur leges, {quas} simples 
anti\msabbr{qui}tas vel minus recte c{on}-
diderat vel obli{qua}verat per-
versitas maligno{rum}. Stude-
ba\msabbr{tur} paci \msabbr{et} iusticie subdito\msabbr{rum}; 
lupi rapaces dentes venena-
tos in insontes acuere non 
audebant, cum rex {se}dens in 
solio intuitu suo o\mstdacc{m}{mn}e malum 
dissiparet {et} bonus pastor vi- 
rili\mstdracc{t}{ter} vigilaret super custodi\mstdacc{a}{am} 
gregis sui\mspara{}Cum vero plebs 
Finlandie, tunc ceca et cru-
delis gentilitas, habitanti\msabbr{bus} 
in Svecia gravia dampna 
frequenter inferret, sanctus
rex Ericus, assumpto secum 
ab ecclesia Upsalensi beato 
Henrico collecto exercitu, 
\msabbr{con}tra nominis Christi et po- 
puli sui inimicos expedicion\mstdacc{e}{em}
dirigit. Qui\msabbr{bus} potenter fidei 
Christi et suo subiugatis do- 
minio, baptizatis plurimis 
et fundatis in partibus illis ecclesiis, ad Sveciam cum gloriosa victoria remeavit. 

\msstartchapterwithrubric{L\mstdtacc{c}{ecti}o}{B}{%
  Eatus aut\mstdacc{e}{em} Hen- }{%
  ricus vinee domini cul-}
  tor{em} \msabbr{et} custodem se divinit\msabbr{us}
  constitutum considerans, ad 
irrigandum ymbre celestis 
doctrine novellas neophito\msabbr{rum} 
plantulas et ad corroboran-
dum dei cultum in partibus
illis audacter remansit, non 
metuens se \msabbr{qui}buslibet adversis 
exponere, ut posset gloriam 
dei dilatare. O quantus fidei 
fervor, quantus ardor divini 
amoris altare aure\mstdacc{u}{um} cordis 
devotissimi p\mstdacc{o}{on}tificis accende- 
rat, \msabbr{qui}, postpositis opulenciis 
rerum {et} amico{rum} solaciis 
et Upsalensis presulatus \msabbr{se}de 
sublimi, {pro} paucarum et\,
pauperum ovium salvacione 
multis se mortis periculis 
exponebat. Illus imitacione 
pastoris, \msabbr{qui} relictis nonag\mstdacc{i}{in}ta 
novem ovi{bus} in deserto uni-
cam errantem sollicite re\msabbr{qui}-
vit et invent\mstdacc{a}{am} ad ovile \msabbr{prop}riis imposit\mstdacc{a}{am} h\mstdacc{u}{um}eris reportavit.


\msstartchapterwithrubric{L\mstdtacc{c}{ecti}o}{C}{%
  Um \msabbr{ver} edificaci\mstdacc{o}{on}i\ }{%
  et confirmacioni Fin- }
landensis ecclesie prudenter \msabbr{et} 
fideliter insudaret, accidit, ut 
homicidam \mstddacc{q}{que}ndam ob ipsius 
facinoris i\mstdacc{m}{mm}anitatem eccle- 
siastica disciplina vellet cor- rigere, ne facilitas nimia 
venie incentivum delinquendi 
preberet. \msabbr{quod} salutis remedium 
infelix vir ille sanguinum 
contempsit et in sue vertit 
dampnacionis augmentum, 
odio habens se salubriter 
arguentem. In ministrum 
ita\msabbr{que} iusticie \msabbr{et} in sue salutis 
zelatorem funestus insiluit, 
ipsum\msabbr{que} crudeliter trucidavit
Sic sacredos domini, acc- 
eptabilis hostia divinis ob-
lata {con}specti{bus}, occumbens 
\msabbr{pro} iusticia, templum superne 
Iherusalem c\mstdacc{u}{um} g\mstdacc{l}{orio}si palma 
triumphi feliciter introivit. 

\msstartchapterwithrubric{Lectio}{P}{%
  Ostmodum ille}{%
scelest\msabbr{us} occisor, tollens} 
de capite sancti p\mstdacc{o}{on}tificis bir- 
retum, \msabbr{quod} gestare \msabbr{con}sueverat, 
imposuit suo capiti, {et} re- 
gressus domum, de commisso 
flagicio se iactans \msabbr{quod} ursum 
prostrasset, \msabbr{per} hoc significans 
occisionem sancti viri, leta- 
batur cum male fecisset, et 
exultabat in rebus pessimis. 
Dum autem birretum, quod 
in suo posuerat capite, 
amovere temptaret, cutis et 
caro birreto coheserunt, et 
illa insimul a testa capitis 
amovebat: congrua profecto 
divine ulcionis vindicta, ut 
tali pena miserabiliter tor- 
queretur, \msabbr{qui} non est veritus 
christum domini truculenter 
impetere et occisum spoliare. 
\end{mstextura}


\newpage
\pagestyle{plain}

\nocite{cappelli}
\nocite{derolez}
\nocite{opentype}
\nocite{type1}
\nocite{carter}
\nocite{laurila}
\nocite{yonge}
\nocite{kpt_historia}
\nocite{rapola}
\nocite{hakkinen}
\nocite{gutenberg}
\nocite{carmina}
\nocite{horatius}
\nocite{suomalainen_kirjallisuus}
\nocite{legenda}
\nocite{kirjallinen_kulttuuri_keskiajan_suomessa}

\newpage

\bibliographystyle{plain}

\bibliography{./missaali}

\newpage


\section{License}

This font is licensed under the SIL Open Font License. The text of the
license has been written by Summer Institute of Linguistics
International (\texttt{http://www.sil.org}). I am not affiliated with
SIL and this font is not endorsed by SIL, so please don't bug them
about it. 

This user's manual and the XeLaTeX style file are licensed under the
LaTeX Project Public License.

\subsection{SIL OPEN FONT LICENSE Version 1.1}\index{Open Font License}

\label{pg_license}

\subsubsection{PREAMBLE}

The goals of the Open Font License (OFL) are to stimulate worldwide
development of collaborative font projects, to support the font
creation efforts of academic and linguistic communities, and to
provide a free and open framework in which fonts may be shared and
improved in partnership with others.

The OFL allows the licensed fonts to be used, studied, modified and
redistributed freely as long as they are not sold by themselves. The
fonts, including any derivative works, can be bundled, embedded, 
redistributed and/or sold with any software provided that any reserved
names are not used by derivative works. The fonts and derivatives,
however, cannot be released under any other type of license. The
requirement for fonts to remain under this license does not apply
to any document created using the fonts or their derivatives.

\subsubsection{DEFINITIONS}

"Font Software" refers to the set of files released by the Copyright
Holder(s) under this license and clearly marked as such. This may
include source files, build scripts and documentation.

"Reserved Font Name" refers to any names specified as such after the
copyright statement(s).

"Original Version" refers to the collection of Font Software components as
distributed by the Copyright Holder(s).

"Modified Version" refers to any derivative made by adding to, deleting,
or substituting -- in part or in whole -- any of the components of the
Original Version, by changing formats or by porting the Font Software to a
new environment.

"Author" refers to any designer, engineer, programmer, technical
writer or other person who contributed to the Font Software.

\subsubsection{PERMISSION \& CONDITIONS}

Permission is hereby granted, free of charge, to any person obtaining
a copy of the Font Software, to use, study, copy, merge, embed, modify,
redistribute, and sell modified and unmodified copies of the Font
Software, subject to the following conditions:

\begin{enumerate}
\item Neither the Font Software nor any of its individual components,
in Original or Modified Versions, may be sold by itself.

\item Original or Modified Versions of the Font Software may be bundled,
redistributed and/or sold with any software, provided that each copy
contains the above copyright notice and this license. These can be
included either as stand-alone text files, human-readable headers or
in the appropriate machine-readable metadata fields within text or
binary files as long as those fields can be easily viewed by the user.

\item  No Modified Version of the Font Software may use the Reserved Font
Name(s) unless explicit written permission is granted by the corresponding
Copyright Holder. This restriction only applies to the primary font name as
presented to the users.

\item  The name(s) of the Copyright Holder(s) or the Author(s) of the Font
Software shall not be used to promote, endorse or advertise any
Modified Version, except to acknowledge the contribution(s) of the
Copyright Holder(s) and the Author(s) or with their explicit written
permission.

\item The Font Software, modified or unmodified, in part or in whole,
must be distributed entirely under this license, and must not be
distributed under any other license. The requirement for fonts to
remain under this license does not apply to any document created
using the Font Software.


\end{enumerate}


\subsubsection{TERMINATION}

This license becomes null and void if any of the above conditions are
not met.

\subsubsection{DISCLAIMER}

THE FONT SOFTWARE IS PROVIDED "AS IS", WITHOUT WARRANTY OF ANY KIND,
EXPRESS OR IMPLIED, INCLUDING BUT NOT LIMITED TO ANY WARRANTIES OF
MERCHANTABILITY, FITNESS FOR A PARTICULAR PURPOSE AND NONINFRINGEMENT
OF COPYRIGHT, PATENT, TRADEMARK, OR OTHER RIGHT. IN NO EVENT SHALL THE
COPYRIGHT HOLDER BE LIABLE FOR ANY CLAIM, DAMAGES OR OTHER LIABILITY,
INCLUDING ANY GENERAL, SPECIAL, INDIRECT, INCIDENTAL, OR CONSEQUENTIAL
DAMAGES, WHETHER IN AN ACTION OF CONTRACT, TORT OR OTHERWISE, ARISING
FROM, OUT OF THE USE OR INABILITY TO USE THE FONT SOFTWARE OR FROM
OTHER DEALINGS IN THE FONT SOFTWARE.


\end{document}


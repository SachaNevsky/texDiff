\documentstyle[12pt]{article}
\newfont{\suet}{suet14}
\newfont{\schwell}{schwell}
\newfont{\schwellv}{schwell scaled \magstep5}
\newfont{\suetv}{suet14 scaled \magstep5}
\parindent=0pt\addtolength{\textwidth}{2cm}\addtolength{\textheight}{2cm}
\setlength{\hoffset}{-1cm}\setlength{\voffset}{-1cm}
\sloppy\frenchspacing
\pagestyle{empty}

\begin{document}

\begin{center}
{\Large Deutsche Schreibschrift in \TeX{}}
\end{center}

Der deutsche P\"adagoge und Grafiker Ludwig S\"utterlin (1865-1917) schuf
die nach ihm benannte ``S\"utterlin-Schrift''.

 Die Schrift wurde 1915 in Preu\ss{}en und sp\"ater auch in anderen
deutschen L\"andern an den Schulen eingef\"uhrt. Die kurze Bl\"utezeit der
S\"utterlin-Schrift endete schon 1942 mit der Einf\"uhrung der Lateinischen
Schreibschrift an den deutschen Schulen.

 Heute wird  die S\"utterlin-Schrift an unseren Schulen bestenfalls
im Kunstunterricht gelehrt.

 F\"ur den \TeX{}-Freund ist diese Schrift, wie auch andere
Schreibschriften,  eine interessante Anwendung der erweiterten Ligaturen von
\TeX{} 3.x bzw. METAFONT 2.x. Buchstabenverbindungen und Sonderbehandlungen am
Ende eines Wortes lassen sich relativ einfach meistern.
Lediglich das lange s am Ende einer Silbe erfordert einen ``Eingriff von Hand''.

 Die Schrift ``SUET14'' ist die eigentliche Grundform der deutschen
Schreibschrift. Hier wurde eine Feder mit einer runden Spitze verwendet;
die grobe Richtung der Schrift ist fast aufrecht.

 Bei der Schrift ``SCHWELL'' wurde der Versuch gemacht, die Schrift durch
die Ver\"anderung einiger Winkel zu neigen, und Unter- und Oberl\"angen
mehr zu betonen. Zus\"atzlich wurde eine breite schr\"aggestellte Feder
benutzt, um den sogenannten Schwellzug nachzuahmen.

 In der Hoffnung, da\ss{} der Zeichensatzvorschlag von Cork recht bald
zum neuen Standard wird, sind die Sonderzeichen nach diesem Schema kodiert. Das
hat zur Folge, da\ss{} die Eingabe zur Zeit noch sehr gew\"ohnungsbed\"urftig
ist (z.B. \verb+^^ff+ f\"ur \ss{} und \verb+^^fc+ f\"ur \"u).
Der \TeX{}-Kenner hat hier aber auch elegantere L\"osungen in seiner Trickkiste.

\bigskip
 Hier noch einige Beispiele:
\bigskip


\baselineskip=20pt

SUET14, magstep 0

{\suet Schrift Fahne F^^e4hnchen b^^f6se Boot Dach Au^^1cgang Stadt Katze
Aschenkasten ^^d6ffentlichkeit ^^c4rgernis ^^dcbelkeit B^^fcttenpapier}

\medskip



SCHWELL, magstep 0


{\schwell Schrift Fahne F^^e4hnchen b^^f6se Boot Dach Au^^1cgang Stadt Katze
Aschenkasten ^^d6ffentlichkeit ^^c4rgernis ^^dcbelkeit B^^fcttenpapier}


\newpage

SUET14, magstep 5

\baselineskip=54pt

{\suetv Schrift Fahne F^^e4hnchen b^^f6se Boot Dach Au^^1cgang Stadt Katze
Aschenkasten ^^d6ffentlichkeit ^^c4rgernis ^^dcbelkeit B^^fcttenpapier}

\medskip


SCHWELL, magstep 5

{\schwellv Schrift Fahne F^^e4hnchen b^^f6se Boot Dach Au^^1cgang Stadt Katze
Aschenkasten ^^d6ffentlichkeit ^^c4rgernis ^^dcbelkeit B^^fcttenpapier}

\end{document}


% !TEX TS-program = pdflatexmk 
% Template file for TeXShop by Michael Sharpe, LPPL
\documentclass[11pt]{article}
\usepackage[margin=1in]{geometry} 
\usepackage[parfill]{parskip}% Begin paragraphs with an empty line rather than an indent
\usepackage{graphicx}
%\pdfmapfile{=algolrevived.map}
%\pdfmapline{+AlgolRevived-tlf-ts1 AlgolRevived " zalts1Enc ReEncodeFont " <zalts1.enc <AlgolRevived.pfb}
%SetFonts
% algolrevived
\usepackage{trace}
\usepackage{algolrevived} % use sb in place of bold
\usepackage[T1]{fontenc}
\usepackage{textcomp}
\usepackage[varqu,varl]{zi4}% inconsolata
%\usepackage[tt]{algolrevived}
%\usepackage{amsmath,amsthm}
%\usepackage[libertine,bigdelims,vvarbb]{newtxmath}
% option vvarbb gives you stix blackboard bold
%SetFonts
\usepackage{fonttable}
\title{Algol Revived\footnote{Creation of this package was spurred by Barbara Beeton's column in a recent TUGBoat, conveying a request from Jacques Andr\'e for someone to digitize Frutiger's Algol alphabet.}}
\author{Michael Sharpe}
\date{\today}  % Activate to display a given date or no date

\begin{document}
\maketitle
\emph{AlgolRevived} is a revival of the (photo)font \emph{Algol} by Adrian Frutiger  whose sole use was for printing ALGOL code in a manual.  It  is not meant to be a general purpose text font---the spacing is not optimized for that, being designed instead for printing computer code, where each letter should be distinct and text ligatures are banished. It seems to work well with the {\tt listings} package, designed for exactly that purpose. Unusually for such a font, it is not monospaced, though perhaps this is no longer the issue it was in the days of FORTRAN. 

Nonetheless, if you don't object to a typewritten appearance, I think the font doesn't really look as bad as you might think it should. (This document uses it as its main text font.)

Both opentype and type1 fonts are provided, along with LaTeX support files. Most characters in the T1 encoding are provided, except for f-ligatures and the Sami characters Eng, eng.

\textbf{Use with fontspec}

The package provides algolrevived.fontspec, with contents:

\begin{verbatim}
\defaultfontfeatures[algolrevived]
{
	Extension = .otf,
	UprightFont = AlgolRevived ,
	BoldFont = AlgolRevived-Medium ,
	BoldItalicFont = AlgolRevived-MediumSlanted ,
	ItalicFont = AlgolRevived-Slanted ,
}
\end{verbatim}
which allows you to set up your preamble using simply
\begin{verbatim}
...
\usepackage{fontspec}
\setmainfont{algolrevived} % for use as main font
%\setmonofont{algolrevived} % for use as typewriter font only
...
\end{verbatim}
\newpage

\textbf{Use with LaTeX}

The package offers OT1, LY1, T1 and TS1 encodings, and sets T1 as the default. To change to LY1, you will need something like

\begin{verbatim}
\usepackage{algolrevived}
%\usepackage[tt]{algolrevived} for just typewriter
\usepackage[LY1]{fontenc}
\end{verbatim}

AlgolRevived-tlf-t1.tfm:
\fonttable{AlgolRevived-tlf-t1}

The package has a few options and macros. The option \texttt{scaled=.95} or \texttt{scale=.95} renders at 95\% of the default size, and option \texttt{medium} makes medium weight LaTeX's regular weight. Option \texttt{tt} species typewriter. The macros \verb|{\sufigures 9}| (same effect as \verb|\textsu{9}|) render the figure as a superscript, \textsu{9}, and similarly with \verb|\infigures|, \verb|\textin| for inferior figures.

The sty file requires textcomp so there is no need to load it separately. Textcomp adds the following glyphs. (The mathematical symbols in the otherwise vacant slots in positions 192 and up were mostly borrowed from the STIX math fonts, which use the same SIL OFL as this package. The names below were in those cases are the same as the STIX names, prefixed by "text".)

AlgolRevived-tlf-ts1.tfm:
\fonttable{AlgolRevived-tlf-ts1}
\newpage

\textbf{List of LaTeX macros to access the TS1 symbols in text mode:}

\begin{verbatim}
 11 \capitalcedilla
 12 \capitalogonek
 24 \textleftarrow
 25 \textrightarrow
 36 \textdollar
 39 \textquotesingle
 42 \textasteriskcentered
 47 \textfractionsolidus
 61 \textminus
 77 \textmho
 79 \textbigcircle
 87 \textohm
 91 \textlbrackdbl
 93 \textrbrackdbl
 94 \textuparrow
 95 \textdownarrow
 96 \textasciigrave
132 \textdagger
133 \textdaggerdbl
134 \textbardbl
136 \textbullet
151 \texttrademark
162 \textcent
163 \textsterling
164 \textcurrency
165 \textyen
166 \textbrokenbar
167 \textsection
169 \textcopyright
170 \textordfeminine
172 \textlnot
174 \textregistered
176 \textdegree
177 \textpm
178 \texttwosuperior
179 \textthreesuperior
181 \textmu
182 \textparagraph
183 \textperiodcentered
184 \textreferencemark
185 \textonesuperior
186 \textordmasculine
191 \texteuro
192 \textprime
193 \textdprime
196 \textleftrightarrow
197 \textupdownarrow
198 \textLeftarrow
199 \textUparrow
200 \textRightarrow
201 \textDownarrow
202 \textLeftrightarrow
203 \textUpdownarrow
204 \textforall
205 \textcomplement
206 \textpartial
207 \textexists
208 \textnexists
209 \textvarnothing
210 \textincrement
211 \textnabla
212 \textin
213 \textnotin
214 \texttimes
215 \textsmallin
216 \textni
217 \textnni
218 \textsmallni
219 \textsmallsetminus
220 \textlargebullet
221 \textland
222 \textlor
224 \textcap
225 \textcup
226 \textcoloneq
227 \texteqcolon
228 \textneq
229 \textequiv
230 \textneqiv
231 \textleq
232 \textgeq
233 \textsubset
234 \textsupset
235 \textnsubset
236 \textnsupset
237 \textsubseteq
238 \textsupseteq
239 \textnsubseteq
240 \textnsupseteq
241 \textsqsubset
242 \textsqsubset
243 \textsqsupset
244 \textsqsubseteq
245 \textsqcap
246 \textdiv
247 \textsqcup
\end{verbatim}
For example, typing in \verb|A\textcoloneq B| results in A\textcoloneq B.
\end{document}  
% !TEX TS-program = lualatex
\documentclass{article}
\usepackage{scrextend}
\setkomafont{disposition}{\bfseries}
\KOMAoptions{fontsize=12pt}
\usepackage[letterpaper]{geometry}
\usepackage{fontspec}
\usepackage{url}
\usepackage{realscripts}
\setmainfont[Numbers={OldStyle,Proportional}]{Garamond Libre}
\usepackage[protrusion=true,expansion=true]{microtype}
\usepackage[russian,greek.ancient,latin.clasical,main=english]{babel}
\usepackage[autostyle=false,babel=true,english=american]{csquotes}
\deffootnote[1.5em]{1.5em}{1em}{\makebox[1.5em][l]{\thefootnotemark.}}
\usepackage{setspace}\usepackage{titling}\usepackage{titlesec}
\usepackage[pdfusetitle,pdfa,hidelinks]{hyperref}\hypersetup{pdfencoding=unicode}
\usepackage{parskip}
\usepackage{hologo}
\title{Garamond Libre}
\date{version 1.1\\July 21, 2019}
\author{Daniel~Benjamin Miller\thanks{\texttt{dbmiller@dbmiller.org}}}
\thanksmarkseries{alph}
\renewcommand{\thanksfootmark}{\makebox[1.5em][l]{a.}}
\begin{document}
\maketitle
Garamond Libre is a free and open-source old-style font family. It is a ``true Garamond,'' i.e., it is based off the designs of 16th-century French engraver Claude Garamond (also spelled Garamont). The Roman design is Garamond's; the italics are from a design by Robert Granjon. The upright Greek font is after a design by Firmin Didot; the ``italic'' Greek font is after a design by Alexander Wilson. The font family includes support for Latin, Greek (monotonic and polytonic) and Cyrillic scripts, as well as small capitals, old-style figures, superior and inferior figures, historical ligatures, Byzantine musical symbols, the \textsc{ipa} and swash capitals. Currently, regular, italic and bold fonts are provided; there is no set timeframe for the completion of a bold italic. The fonts are provided in OpenType format, and are intended to be used with \hologo{LuaLaTeX} or \hologo{XeLaTeX} via \texttt{fontspec}.\footnote{\url{https://ctan.org/pkg/fontspec}}
\section*{Basic Usage}
To use, simply set the font using \texttt{fontspec} in your preamble, as follows:
\begin{verbatim}
\usepackage{fontspec}
\setmainfont{Garamond Libre}
\end{verbatim}
\section*{Sample Texts}
\subsection*{Commentarii de bello Gallico I, Chapter One}
\begin{otherlanguage}{latin}
Gallia est omnis divisa in partes tres, quarum unam incolunt Belgae, aliam Aquitani, tertiam qui ipsorum lingua Celtae, nostra Galli appellantur. Hi omnes lingua, institutis, legibus inter se differunt. Gallos ab Aquitanis Garumna flumen, a Belgis Matrona et Sequana dividit. Horum omnium fortissimi sunt Belgae, propterea quod a cultu atque humanitate provinciae longissime absunt, minimeque ad eos mercatores saepe commeant atque ea quae ad effeminandos animos pertinent important, proximique sunt Germanis, qui trans Rhenum incolunt, quibuscum continenter bellum gerunt. Qua de causa Helvetii quoque reliquos Gallos virtute praecedunt, quod fere cotidianis proeliis cum Germanis contendunt, cum aut suis finibus eos prohibent aut ipsi in eorum finibus bellum gerunt. Eorum una pars, quam Gallos obtinere dictum est, initium capit a flumine Rhodano, continetur Garumna flumine, Oceano, finibus Belgarum, attingit etiam ab Sequanis et Helvetiis flumen Rhenum, vergit ad septentriones. Belgae ab extremis Galliae finibus oriuntur, pertinent ad inferiorem partem fluminis Rheni, spectant in septentrionem et orientem solem. Aquitania a Garumna flumine ad Pyrenaeos montes et eam partem Oceani quae est ad Hispaniam pertinet; spectat inter occasum solis et septentriones. 

{\itshape
Gallia est omnis divisa in partes tres, quarum unam incolunt Belgae, aliam Aquitani, tertiam qui ipsorum lingua Celtae, nostra Galli appellantur. Hi omnes lingua, institutis, legibus inter se differunt. Gallos ab Aquitanis Garumna flumen, a Belgis Matrona et Sequana dividit. Horum omnium fortissimi sunt Belgae, propterea quod a cultu atque humanitate provinciae longissime absunt, minimeque ad eos mercatores saepe commeant atque ea quae ad effeminandos animos pertinent important, proximique sunt Germanis, qui trans Rhenum incolunt, quibuscum continenter bellum gerunt. Qua de causa Helvetii quoque reliquos Gallos virtute praecedunt, quod fere cotidianis proeliis cum Germanis contendunt, cum aut suis finibus eos prohibent aut ipsi in eorum finibus bellum gerunt. Eorum una pars, quam Gallos obtinere dictum est, initium capit a flumine Rhodano, continetur Garumna flumine, Oceano, finibus Belgarum, attingit etiam ab Sequanis et Helvetiis flumen Rhenum, vergit ad septentriones. Belgae ab extremis Galliae finibus oriuntur, pertinent ad inferiorem partem fluminis Rheni, spectant in septentrionem et orientem solem. Aquitania a Garumna flumine ad Pyrenaeos montes et eam partem Oceani quae est ad Hispaniam pertinet; spectat inter occasum solis et septentriones. 
}

{\bfseries
Gallia est omnis divisa in partes tres, quarum unam incolunt Belgae, aliam Aquitani, tertiam qui ipsorum lingua Celtae, nostra Galli appellantur. Hi omnes lingua, institutis, legibus inter se differunt. Gallos ab Aquitanis Garumna flumen, a Belgis Matrona et Sequana dividit. Horum omnium fortissimi sunt Belgae, propterea quod a cultu atque humanitate provinciae longissime absunt, minimeque ad eos mercatores saepe commeant atque ea quae ad effeminandos animos pertinent important, proximique sunt Germanis, qui trans Rhenum incolunt, quibuscum continenter bellum gerunt. Qua de causa Helvetii quoque reliquos Gallos virtute praecedunt, quod fere cotidianis proeliis cum Germanis contendunt, cum aut suis finibus eos prohibent aut ipsi in eorum finibus bellum gerunt. Eorum una pars, quam Gallos obtinere dictum est, initium capit a flumine Rhodano, continetur Garumna flumine, Oceano, finibus Belgarum, attingit etiam ab Sequanis et Helvetiis flumen Rhenum, vergit ad septentriones. Belgae ab extremis Galliae finibus oriuntur, pertinent ad inferiorem partem fluminis Rheni, spectant in septentrionem et orientem solem. Aquitania a Garumna flumine ad Pyrenaeos montes et eam partem Oceani quae est ad Hispaniam pertinet; spectat inter occasum solis et septentriones. 
}
\end{otherlanguage}
\subsection*{Genesis~1:1--5 (\textsc{lxx})}
\begin{otherlanguage}{greek}
Ἐν ἀρχῇ ἐποίησεν ὁ Θεὸς τὸν οὐρανὸν καὶ τὴν γῆν. ἡ δὲ γῆ ἦν ἀόρατος καὶ ἀκατασκεύαστος, καὶ σκότος ἐπάνω τῆς ἀβύσσου, καὶ πνεῦμα Θεοῦ ἐπεφέρετο ἐπάνω τοῦ ὕδατος. καὶ εἶπεν ὁ Θεός· γενηθήτω φῶς· καὶ ἐγένετο φῶς.καὶ εἶδεν ὁ Θεὸς τὸ φῶς, ὅτι καλόν· καὶ διεχώρισεν ὁ Θεὸς ἀνὰ μέσον τοῦ φωτὸς καὶ ἀνὰ μέσον τοῦ σκότους.καὶ ἐκάλεσεν ὁ Θεὸς τὸ φῶς ἡμέραν καὶ τὸ σκότος ἐκάλεσε νύκτα. καὶ ἐγένετο ἑσπέρα καὶ ἐγένετο πρωΐ, ἡμέρα μία.

{\itshape Ἐν ἀρχῇ ἐποίησεν ὁ Θεὸς τὸν οὐρανὸν καὶ τὴν γῆν. ἡ δὲ γῆ ἦν ἀόρατος καὶ ἀκατασκεύαστος, καὶ σκότος ἐπάνω τῆς ἀβύσσου, καὶ πνεῦμα Θεοῦ ἐπεφέρετο ἐπάνω τοῦ ὕδατος.καὶ εἶπεν ὁ Θεός· γενηθήτω φῶς· καὶ ἐγένετο φῶς. καὶ εἶδεν ὁ Θεὸς τὸ φῶς, ὅτι καλόν· καὶ διεχώρισεν ὁ Θεὸς ἀνὰ μέσον τοῦ φωτὸς καὶ ἀνὰ μέσον τοῦ σκότους.καὶ ἐκάλεσεν ὁ Θεὸς τὸ φῶς ἡμέραν καὶ τὸ σκότος ἐκάλεσε νύκτα. καὶ ἐγένετο ἑσπέρα καὶ ἐγένετο πρωΐ, ἡμέρα μία.}

{\bfseries Ἐν ἀρχῇ ἐποίησεν ὁ Θεὸς τὸν οὐρανὸν καὶ τὴν γῆν. ἡ δὲ γῆ ἦν ἀόρατος καὶ ἀκατασκεύαστος, καὶ σκότος ἐπάνω τῆς ἀβύσσου, καὶ πνεῦμα Θεοῦ ἐπεφέρετο ἐπάνω τοῦ ὕδατος.καὶ εἶπεν ὁ Θεός· γενηθήτω φῶς· καὶ ἐγένετο φῶς.καὶ εἶδεν ὁ Θεὸς τὸ φῶς, ὅτι καλόν· καὶ διεχώρισεν ὁ Θεὸς ἀνὰ μέσον τοῦ φωτὸς καὶ ἀνὰ μέσον τοῦ σκότους.καὶ ἐκάλεσεν ὁ Θεὸς τὸ φῶς ἡμέραν καὶ τὸ σκότος ἐκάλεσε νύκτα. καὶ ἐγένετο ἑσπέρα καὶ ἐγένετο πρωΐ, ἡμέρα μία.}
\end{otherlanguage}
\subsection*{Cyrillic Lorem Ipsum}
\begin{otherlanguage}{russian}
Лорем ипсум долор сит амет, ет лаборе медиоцрем еум, сингулис цонституто яуо ид. Елит цонсулату ид ест, еос цу яуод фабулас. Волуптуа оцурререт ет вис, те чоро аппареат форенсибус пер. Те иуварет витуператорибус яуи, нам амет апеириан детрахит ад, яуидам вереар атоморум еи еам. Те ребум фугит промпта усу, иус ет сумо алияуандо. Семпер поссим еа яуи, дуис адиписцинг сусципиантур сеа ех. Перципитур адверсариум еам ех, меи ех суас видит утамур, ад ест доценди перфецто. Ут лаборе импетус про, ферри харум пертинах дуо ет, меа ид дебет лаборе. Те ерат темпор меа, ан нибх фацилис яуо, сеа те адиписцинг цонцлудатуряуе. Цум дисцере улламцорпер ид, цонсететур сцрипторем яуи ут. Перфецто лобортис вис ад, еи уллум инермис еам, яуис постулант про ан. Нец фастидии сусципиантур ет, меи цу атяуи десерунт.

{\itshape
Лорем ипсум долор сит амет, ет лаборе медиоцрем еум, сингулис цонституто яуо ид. Елит цонсулату ид ест, еос цу яуод фабулас. Волуптуа оцурререт ет вис, те чоро аппареат форенсибус пер. Те иуварет витуператорибус яуи, нам амет апеириан детрахит ад, яуидам вереар атоморум еи еам. Те ребум фугит промпта усу, иус ет сумо алияуандо. Семпер поссим еа яуи, дуис адиписцинг сусципиантур сеа ех. Перципитур адверсариум еам ех, меи ех суас видит утамур, ад ест доценди перфецто. Ут лаборе импетус про, ферри харум пертинах дуо ет, меа ид дебет лаборе. Те ерат темпор меа, ан нибх фацилис яуо, сеа те адиписцинг цонцлудатуряуе. Цум дисцере улламцорпер ид, цонсететур сцрипторем яуи ут. Перфецто лобортис вис ад, еи уллум инермис еам, яуис постулант про ан. Нец фастидии сусципиантур ет, меи цу атяуи десерунт.}

{\bfseries
Лорем ипсум долор сит амет, ет лаборе медиоцрем еум, сингулис цонституто яуо ид. Елит цонсулату ид ест, еос цу яуод фабулас. Волуптуа оцурререт ет вис, те чоро аппареат форенсибус пер. Те иуварет витуператорибус яуи, нам амет апеириан детрахит ад, яуидам вереар атоморум еи еам. Те ребум фугит промпта усу, иус ет сумо алияуандо. Семпер поссим еа яуи, дуис адиписцинг сусципиантур сеа ех. Перципитур адверсариум еам ех, меи ех суас видит утамур, ад ест доценди перфецто. Ут лаборе импетус про, ферри харум пертинах дуо ет, меа ид дебет лаборе. Те ерат темпор меа, ан нибх фацилис яуо, сеа те адиписцинг цонцлудатуряуе. Цум дисцере улламцорпер ид, цонсететур сцрипторем яуи ут. Перфецто лобортис вис ад, еи уллум инермис еам, яуис постулант про ан. Нец фастидии сусципиантур ет, меи цу атяуи десерунт.}
\end{otherlanguage}
\newpage
\section*{Licensing}
These fonts are based on an older version of George Douros' Textfonts project\footnote{\url{http://users.teilar.gr/~g1951d/}} which was released as free for any use. The following license applies to the Garamond Libre fonts:
\begin{quote}
Copyright © 2019 Daniel Benjamin Miller.

Permission is hereby granted, free of charge, to any person obtaining a copy of this software and associated documentation files (the ``Software"), to deal in the Software without restriction, including without limitation the rights to use, copy, modify, merge, publish, distribute, sublicense, and/or sell copies of the Software, and to permit persons to whom the Software is furnished to do so, subject to the following conditions:

The above copyright notice and this permission notice shall be included in all copies or substantial portions of the Software.
\end{quote}
\end{document}
\documentclass[12pt]{article}
\usepackage{a4,mxedruli,xucuri,tipa}
\usepackage{times}

% Designed by Johannes Heinecke                                             %
%             <johannes.heinecke@wanadoo.fr>                                %
% This software is under the LaTeX Project Public License                   %

%\newif\ifpdf\ifx\pdfoutput\undefined\pdffalse\else\pdftrue\fi

%\ifpdf
\usepackage[pdftex,colorlinks=true,
  urlcolor=urlcol, % URL: http://
  citecolor=bibcol, % bibTeX
  linkcolor=linkcol, % sections, footnotes, ...
  filecolor=filecol, % URL: file:/
  pdftitle={The Georgian Alphabets},
  pdfauthor={Johannes Heinecke},
  pdfsubject={},pdfkeywords={},pagebackref,
  pdfpagemode=UseNone,bookmarksopen=true]{hyperref}
 \usepackage{color}
 \definecolor{linkcol}{rgb}{0.75,0,0}
 \definecolor{bibcol}{rgb}{0,0.5,0}
 \definecolor{urlcol}{rgb}{0,0,0.75}
 \definecolor{filecol}{rgb}{0,0,0.75}
 \urlstyle{same}

%\else
%\fi


\parskip7mm
\parindent 0pt
\addtolength{\topmargin}{-1ex}
\addtolength{\textheight}{2ex}

\font\logo=logo10 scaled \magstep1
\font\logogr=logobf10 scaled \magstep3
%\font\phon=wsuipa12

\title{{\mxedc mxedruli} --- \it Mxedruli,\\ 
       {\xucr XUCURI, xucuri} --- Xucuri\\
The Georgian Alphabets\thanks{With improvements by 
Jan De Lameillieure, Berlin and Mark Leisher, Las Cruces, New Mexico}}

\author{Johannes Heinecke (\href{mailto:johannes.heinecke@wanadoo.fr}{johannes.heinecke@wanadoo.fr})\\
Lannion, France}

\def\mxedversion{3.4}

\date{Version \mxedversion, 5th April 2013}

\begin{document}
\maketitle

\thispagestyle{empty}

\section{Introduction}
This is a short documentation of the two alphabets
used by Georgian and some of its neighbouring languages from
the Kartvelian language family. The first alphabet is called 
{\it Mxedruli}.  Some letters used by Old Georgian or 
other languages such as Ossetian are also included.
The second alphabet is called {\it Xucuri\/}. Whereas {\it Mxedruli\/} 
does differentiate majuscules and minuscules, {\it Xucuri\/} distinguishes
between majuscules (also called {\it Mrg(v)lovani\/}) and minuscules
({\it \d Kutxovani\/}). However, in opposition to the Roman, Greek and
Cyrillic alphabets in a text either majuscules or minuscules are
used. They cannot be combined. {\it Xucuri\/} is now restricted to
religious use.

It is implemented using {\logo METAFONT} and can be used in
\LaTeX\ or \TeX-documents.
The font are of a rather simple design (cf. section \ref{Examples}) and
surely may be improved.  I would be very happy if any suggestions,
improvements, corrections, extensions, enhancements etc.  could be
forwarded to me\footnote{By e-mail: \href{mailto:johannes.heinecke@wanadoo.fr}{johannes.heinecke@wanadoo.fr}.  I try to realize suggestions etc. as
quick as possible. So please check my WWW-Homepage for the newest
$\beta$-release: \url{http://pagesperso-orange.fr/heinecke/mxedruli/}}.  Thank
you very much!

%%%%%%%%%%%%%%%%%%%%%%%%%%%%%%%%%%%%%%%%%%%%%%%%%%%%%%%%%%%%%%%%%%%%%%%%
%%%%%%%%%%%%%%%%%%%%%%%%%%%% USAGE %%%%%%%%%%%%%%%%%%%%%%%%%%%%%%%%%%%%%
%%%%%%%%%%%%%%%%%%%%%%%%%%%%%%%%%%%%%%%%%%%%%%%%%%%%%%%%%%%%%%%%%%%%%%%%

\section{The fonts and their usage}
\subsection{\emph{Mxedruli}}
The tables below show the names of the {\it Mxedruli\/} letters,
followed by the letter in normal and bold shape and in a
``capital''\footnote{As there are no capital letters in {\it mxedruli\/}
the letters are scaled to the same height for headlines etc.} mode.
Within a \LaTeX-document they can be activated by \verb+\mxedr+
(normal), \verb+\mxedb+ (bold)
\verb+\mxedi+ (italics) or \verb+\mxedc+
analogously to standard \LaTeX\ font commands such as \verb+\rm+,
\verb+\bf+, \verb+\it+ or \verb+\sc+ respectively. Therefore the
\verb|mxedruli.sty| stylefile must be loaded for standard font
selection and NFSS2.\footnote{NFSS1 is supported too. 
Due to the lack of NFSS1 however, I
could not test, whether the NFSS1 routines work satisfactorily.}.
Please refer to section
\ref{Install} in order to install the fonts. They are provided for
300dpi printers, if you need different solutions section \ref{MF}
describes how you can produce the {\tt .pk}-files required.

Within \LaTeX\ the font size commands have also effect on the {\it
Mxedruli\/} letters.  Finally, in the last column, the required
\LaTeX\ or \TeX-input to produce the {\it Mxedruli\/} letter is shown.


(As I designed some new letters I had to reposition the
`{\mxedr .+c}' (\verb-.+c-) from its original position\footnote{As
coded in versions anterior to 2.1} {\it '171\/}
to {\it '014\/}. On {\it '171\/} ({\it y\/}) the new
letter `{\mxedr y}' ({\it schwa\/}) is installed now. This is
mainly important for those users, who address the letters
directly by using the \TeX\ \verb+\char+{\it nnn\/} or the
\LaTeX\ \verb+\symbol{+{\it nnn\/}\verb+}+ commands respectively.
If you address the letter as described here, nothing has changed
compared to previous versions.)

For those (Kartvelian) languages, which use {\it Mxedruli\/} together
with diacritics\footnote{Sometimes also used in Kartvelian
dialectology or descriptive linguistics of Kaukasian languages in
Georgian, in order to provide a phonetic transcription without recurring to the
Internal Phonetic Alphabet.} an umlaut (`\char127'), a circumflex (`\char94')
and a macron (`\char22') have
been defined. They can be used as in normal \LaTeX-documents by the
\verb+\"+{\it char\/} (umlaut on {\it char\/}), \verb+\^+{\it char\/} (circonflex)
 or \verb+\=+{\it char\/} (macron)
command respectively, e.g. \verb+{\mxedr \"o}+ yields `{\mxedr \"o}',
\verb+{\mxedr+ \verb+\^a}+ yields `{\mxedr \^a}', 
and \verb+{\mxedr \=u}+ yields `{\mxedr \=u}'.

The table in sections \ref{mxedletters} and \ref{xucletters}
contains in the leftmost column the corresponding 
Unicode-code\footnote{Cf. \url{http://www.unicode.org/charts/}.} Please note that the
letter-names do slightly differ to the names used in the Unicode name table. This is due
to different conventions.


\vfill
\subsubsection{Standard letters}\label{mxedletters}

\begin{center}
\begin{tabular}{|l |c |c |c |c |c |c|c|}
\hline
           & \multicolumn{4}{c|}{\it Mxedruli} &  &  &\\
\cline{2-5}
\raisebox{1.5ex}[1.5ex]{Name}  & normal  & bold  & italics & capital &
\raisebox{1.5ex}[1.5ex]{Transcription} & \raisebox{1.5ex}[1.5ex]{input} & \raisebox{1.5ex}[1.5ex]{Unicode} \\
\hline\hline
an         & \mxedr a   & \mxedb a   & \mxedi a   & \mxedc a   & a       & \verb|a| & 10D0  \\ \hline
ban        & \mxedr b   & \mxedb b   & \mxedi b   & \mxedc b   & b       & \verb|b| & 10D1  \\ 
\hfill \it or & \mxedr b1   & \mxedb b1  &  \mxedi b1   &   &     & \verb|b1| &  \\ \hline
gan        & \mxedr g   & \mxedb g   & \mxedi g   & \mxedc g   & g       & \verb|g| & 10D2 \\ \hline
don        & \mxedr d   & \mxedb d   & \mxedi d   & \mxedc d   & d       & \verb|d| & 10D3 \\ \hline
en         & \mxedr e   & \mxedb e   & \mxedi e   & \mxedc e   & e       & \verb|e| & 10D4 \\ \hline
vin        & \mxedr v   & \mxedb v   & \mxedi v   & \mxedc v   & v       & \verb|v| & 10D5 \\ \hline
zen        & \mxedr z   & \mxedb z   & \mxedi z   & \mxedc z   & z       & \verb|z| & 10D6  \\ \hline
tan        & \mxedr t   & \mxedb t   & \mxedi t   & \mxedc t   & t       & \verb|t| & 10D7  \\ \hline
in         & \mxedr i   & \mxedb i   & \mxedi i   & \mxedc i   & i       & \verb|i| & 10D8  \\ \hline
\d kan     & \mxedr .k  & \mxedb .k  & \mxedi .k  & \mxedc .k  & \d k    & \verb|.k| & 10D9 \\ \hline
las        & \mxedr l   & \mxedb l   & \mxedi l   & \mxedc l   & l       & \verb|l|  & 10DA \\ \hline
man        & \mxedr m   & \mxedb m   & \mxedi m   & \mxedc m   & m       & \verb|m|  & 10DB \\ \hline
nar        & \mxedr n   & \mxedb n   & \mxedi n   & \mxedc n   & n       & \verb|n|  & 10DC \\ \hline
on         & \mxedr o   & \mxedb o   & \mxedi o   & \mxedc o   & o       & \verb|o|  & 10DD \\ \hline
\.par      & \mxedr .p  & \mxedb .p  & \mxedi .p  & \mxedc .p  & \.p     & \verb|.p| & 10DE \\ \hline
\v zan     & \mxedr +z  & \mxedb +z  & \mxedi +z  & \mxedc +z  & \v z    & \verb|+z| & 10DF \\ \hline
ran        & \mxedr r   & \mxedb r   & \mxedi r   & \mxedc r   & r       & \verb|r| & 10E0  \\ \hline
san        & \mxedr s   & \mxedb s   & \mxedi s   & \mxedc s   & s       & \verb|s|  & 10E1 \\ \hline
\d tar     & \mxedr .t  & \mxedb .t  & \mxedi .t  & \mxedc .t  & \d t    & \verb|.t| & 10E2 \\ \hline
un         & \mxedr u   & \mxedb u   & \mxedi u   & \mxedc u   & u       & \verb|u|  & 10E3 \\ \hline
par        & \mxedr p   & \mxedb p   & \mxedi p   & \mxedc p   & p       & \verb|p|  & 10E4 \\ \hline
kan        & \mxedr k   & \mxedb k   & \mxedi k   & \mxedc k   & k       & \verb|k|  & 10E5 \\ \hline
\.gan      & \mxedr .g  & \mxedb .g  & \mxedi .g  & \mxedc .g  & \.g     & \verb|.g| & 10E6 \\ \hline
\.qar      & \mxedr q   & \mxedb q   & \mxedi q   & \mxedc q   & \.q     & \verb|q|\footnotemark & 10E7  \\ \hline
\v san     & \mxedr +s  & \mxedb +s  & \mxedi +s  & \mxedc +s  & \v s    & \verb|+s| & 10E8 \\ \hline
\v cin     & \mxedr +c  & \mxedb +c  & \mxedi +c  & \mxedc +c  & \v c    & \verb|+c| & 10E9 \\ \hline
can        & \mxedr c   & \mxedb c   & \mxedi c   & \mxedc c   & c       & \verb|c|  & 10EA \\ \hline
dzil       & \mxedr j   & \mxedb j   & \mxedi j   & \mxedc j   & j/dz    & \verb|j|  & 10EB \\ \hline
\d cil     & \mxedr .c  & \mxedb .c  & \mxedi .c  & \mxedc .c  & \d c    & \verb|.c| & 10EC \\ \hline
\d{\v c}ar & \mxedr .+c & \mxedb .+c & \mxedi .+c & \mxedc .+c & \d{\v c}& \verb|.+c| & 10ED \\ \hline
xan        & \mxedr x   & \mxedb x   & \mxedi x   & \mxedc x   & x       & \verb|x| & 10EE  \\ \hline
d\v zan    & \mxedr +j  & \mxedb +j  & \mxedi +j  & \mxedc +j  & d\v z   & \verb|+j| & 10EF \\ \hline
ha         & \mxedr h   & \mxedb h   & \mxedi h   & \mxedc h   & h       & \verb|h|  & 10F0 \\ \hline
\end{tabular}
\end{center}

\footnotetext{Although the transcription of `{\mxedr q}' is {\it \.q\/} the
\TeX-input will remain `{\tt q}' for the letter {\it \.qar\/} as
it appears with a far more frequent than {\it qar\/} (`{\mxedr q1}')
which is coded as `{\tt q1}'.}

\subsubsection{Other letters}

\begin{center}
\begin{tabular}{|l |c |c |c |c |c |c| c|}
\hline
           & \multicolumn{4}{c|}{\it Mxedruli} &  &  & \\
\cline{2-5}
\raisebox{1.5ex}[1.5ex]{Name}  & normal  & bold  & italics & capital &
\raisebox{1.5ex}[1.5ex]{Transcription} & \raisebox{1.5ex}[1.5ex]{input} & \raisebox{1.5ex}[1.5ex]{Unicode} \\ \hline\hline
qar        & \mxedr q1  & \mxedb q1  & \mxedi q1  & \mxedc q1  & q       & \verb|q1| & 10F4 \\ \hline
h\=e       & \mxedr e0  & \mxedb e0  & \mxedi e0  & \mxedc e0  & \=e, e\u\i     & \verb|e0| & 10F1 \\ \hline
ho         & \mxedr o1  & \mxedb o1  & \mxedi o1  & \mxedc o1  & \=o, oy & \verb|o1| & 10F5 \\ \hline
jo         & \mxedr i1  & \mxedb i1  & \mxedi i1  & \mxedc i1  & \u\i    & \verb|i1| & 10F2 \\ \hline
wi         & \mxedr w   & \mxedb w   & \mxedi w   & \mxedc w   & w       & \verb|w| & 10F3  \\ \hline
fi         & \mxedr f   & \mxedb f   & \mxedi f   & \mxedc f   & f       & \verb|f|  & 10F6  \\ \hline
schwa      & \mxedr y   & \mxedb y   & \mxedi y   & \mxedc y   & \textschwa, y       & \verb|y| & 10F7  \\ \hline
elifi      & \mxedr a1  & \mxedb a1  & \mxedi a1  & \mxedc a1  & '       & \verb|a1| & 10F8 \\ \hline
\ae        & \mxedr e1  & \mxedb e1  & \mxedi e1  & \mxedc e1  & \ae     & \verb|e1| & \\ \hline
\end{tabular}
\end{center}

The letters \emph{qar}, \emph{h\=e}, \emph{ho}, \emph{jo}, \emph{wi} and \emph{fi}
are seldomly used or archaic letters (\emph{fi} may be used to transcribe
latin \emph{f}). \emph{Schwa} and \emph{elifi} are mainly used in Mingrelian or Svan, whereas \emph{\ae} is used to write the Ossetian
language using Mxedruli.

\subsubsection{Digits}


\begin{center}
\begin{tabular}{|c |c |c | c|}
\hline
\multicolumn{3}{|c|}{\it Mxedruli}  & \\
\cline{1-3}
normal & bold & capital & \raisebox{1.5ex}[1.5ex]{input} \\
\hline\hline
 \mxedr 1   & \mxedb 1 & \mxedc 1   &  \verb|1|  \\ \hline
 \mxedr 2   & \mxedb 2 & \mxedc 2   &  \verb|2|  \\ \hline
 \mxedr 3   & \mxedb 3 & \mxedc 3   &  \verb|3|  \\ \hline
 \mxedr 4   & \mxedb 4 & \mxedc 4   &  \verb|4|  \\ \hline
 \mxedr 5   & \mxedb 5 & \mxedc 5   &  \verb|5|  \\ \hline
\end{tabular} \hspace{10mm}
%
\begin{tabular}{|c |c |c | c|}
\hline
\multicolumn{3}{|c|}{\it Mxedruli} & \\
\cline{1-3}
normal & bold & capital & \raisebox{1.5ex}[1.5ex]{input} \\
\hline\hline
\mxedr 6  & \mxedb 6 &  \mxedc6  & \verb|6|  \\ \hline
\mxedr 7  & \mxedb 7 &  \mxedc7  & \verb|7|  \\ \hline
\mxedr 8  & \mxedb 8 &  \mxedc8  & \verb|8|  \\ \hline
\mxedr 9  & \mxedb 9 &  \mxedc9  & \verb|9|  \\ \hline
\mxedr 0  & \mxedb 0 &  \mxedc0  & \verb|0|  \\ \hline
\end{tabular}
\end{center}


\subsubsection{Punctuation}\label{punkte}
\begin{center}


\begin{tabular}[t]{|c |c |c | c|}
\hline
\multicolumn{3}{|c|}{\it Mxedruli} & \\
\cline{1-3}
normal & bold & capital & \raisebox{1.5ex}[1.5ex]{input} \\
\hline\hline
 \mxedb .   & \mxedr .  & \mxedc .   &   \verb|.|   \\ \hline
 \mxedr ,   & \mxedb ,  & \mxedc ,   &   \verb|,|   \\ \hline
 \mxedr !   & \mxedb !  & \mxedc !   &   \verb|!|   \\ \hline
 \mxedr ?   & \mxedb ?  & \mxedc ?   &   \verb|?|   \\ \hline
 \mxedr =   & \mxedb =  & \mxedc =   &   \verb|=|   \\ \hline
 \mxedr -   & \mxedb -  & \mxedc -   &   \verb|-|   \\ \hline
 \mxedr --  & \mxedb -- & \mxedc --    &  \verb|--|  \\ \hline
 \mxedr --- & \mxedb --- & \mxedc ---  &  \verb|---| \\ \hline
 \mxedr /   & \mxedb /   & \mxedc /    &  \verb|/|   \\ \hline
 \mxedr \char92 & \mxedb \char92 & \mxedc \char92   &  \verb|\char92| \\ \hline
\end{tabular} \hspace{10mm}
%
\begin{tabular}[t]{|c |c |c | c|}
\hline
\multicolumn{3}{|c|}{\it Mxedruli} & \\
\cline{1-3}
normal & bold & capital & \raisebox{1.5ex}[1.5ex]{input} \\
\hline\hline
 \mxedr :  & \mxedb :  & \mxedc :  &   \verb|:|   \\ \hline
 \mxedr ;  & \mxedb ;  & \mxedc ;  &   \verb|;|   \\ \hline
 \mxedr +  & \mxedb +  & \mxedc +  &   \verb|+|   \\ \hline
 \mxedr (  & \mxedb (  & \mxedc (  &   \verb|(|   \\ \hline
 \mxedr )  & \mxedb )  & \mxedc )  &   \verb|)|   \\ \hline
 \mxedr [  & \mxedb [  & \mxedc [  &   \verb|[|   \\ \hline
 \mxedr ]  & \mxedb ]  & \mxedc ]  &   \verb|]|   \\ \hline
 \mxedr '' & \mxedb '' & \mxedc '' &   \verb|''|  \\ \hline
 \mxedr ,, & \mxedb ,, & \mxedc ,, &   \verb|,,|  \\ \hline
\end{tabular}
\end{center}

%%%%%%%%%%%%%%%%%%%%%%%%%%%%%%%%%%%%%%%%%%%%%%%%%%%%%%%%%%%%%%%%

\subsection{\emph{Xucuri}}
The following tables show the names and the coding of the {\it
Xucuri\/} alphabets ({\it Mrgvlovani\/} and {\it \d
Kutxovani\/}). These fonts do not have any 
diacritics. Further, for the time being, it only consists of the normal
series.\footnote{If there is a demand for italics and bold versions, I
will gladly provide the driver files required. Please contact me as
indicated on page 1.}

Within a \LaTeX-document they can be activated by \verb+\xucr+.
If used within a \LaTeX-document together with the style file provided
({\tt xucuri.sty}) the \LaTeX\ size commands will also work. 
(\LaTeXe\ is supported together with NFSS2, as is \LaTeX 2.09)

\subsubsection{Letters}\label{xucletters}

\begin{center}
\begin{tabular}{|l |c |c| c| c|c|c|}
\hline
%           & \it Xucuri &   \\
%\cline{2-2}
%\raisebox{1.5ex}[1.5ex]{Name}  & normal  &
%\raisebox{1.5ex}[1.5ex]{Transcription} & 
%raisebox{1.5ex}[1.5ex]{\TeX-input} \\
Name & \multicolumn{2}{c|}{\it Xucuri} & Transcription & input & \multicolumn{2}{c|}{Unicode}\\
     & majuscules & minuscules & & & & \\
     & \it Mrg(v)lovani & \it \d Kutxovani & & & \it Mrg(v)lovani & \it \d Kutxovani \\
\hline\hline
an         & \xucr A   & \xucr a   & a       & \verb|A|, \verb|a| & 10A0 & 2D00 \\ \hline
ban        & \xucr B   & \xucr b   & b       & \verb|B|, \verb|b| & 10A1 & 2D01 \\ \hline
gan        & \xucr G   & \xucr g   & g       & \verb|G|, \verb|g| & 10A2 & 2D02 \\ \hline
don        & \xucr D   & \xucr d   & d       & \verb|D|, \verb|d| & 10A3 & 2D03 \\ \hline
en         & \xucr E   & \xucr e   & e       & \verb|E|, \verb|e| & 10A4 & 2D04 \\ \hline
vin        & \xucr V   & \xucr v   & v       & \verb|V|, \verb|v| & 10A5 & 2D05 \\ \hline
zen        & \xucr Z   & \xucr z   & z       & \verb|Z|, \verb|z| & 10A6 & 2D06 \\ \hline
tan        & \xucr T   & \xucr t   & t       & \verb|T|, \verb|t| & 10A7 & 2D07 \\ \hline
in         & \xucr I   & \xucr i   & i       & \verb|I|, \verb|i| & 10A8 & 2D08 \\ \hline
\d kan     & \xucr .K  & \xucr .k  & \d k    & \verb|.K|, \verb|.k| & 10A9 & 2D09\\ \hline
las        & \xucr L   & \xucr l   & l       & \verb|L|, \verb|l| & 10AA & 2D0A \\ \hline
man        & \xucr M   & \xucr m   & m       & \verb|M|, \verb|m| & 10AB & 2D)B \\ \hline
nar        & \xucr N   & \xucr n   & n       & \verb|N|, \verb|n| & 10AC & 2D0C \\ \hline
on         & \xucr O   & \xucr o   & o       & \verb|O|, \verb|o| & 10AD & 2D0D \\ \hline
\.par      & \xucr .P  & \xucr .p  & \.p     & \verb|.P|, \verb|.p| & 10AE & 2D0E\\ \hline
\v zan     & \xucr +Z  & \xucr +z  & \v z    & \verb|+Z|, \verb|+z| & 10AF & 2D0F\\ \hline
ran        & \xucr R   & \xucr r   & r       & \verb|R|, \verb|r| & 10B0 & 2D10 \\ \hline
san        & \xucr S   & \xucr s   & s       & \verb|S|, \verb|s| & 10B1 & 2D11 \\ \hline
\d tar     & \xucr .T  & \xucr .t  & \d t    & \verb|.T|, \verb|.t| & 10B2 & 2D12\\ \hline
un         & \xucr U   & \xucr u   & u       & \verb|U|, \verb|u| & 10B3 & 2D13  \\ \hline
par        & \xucr P   & \xucr p   & p       & \verb|P|, \verb|p| & 10B4 & 2D14 \\ \hline
kan        & \xucr K   & \xucr k   & k       & \verb|K|, \verb|k| & 10B5 & 2D15 \\ \hline
\.gan      & \xucr .G  & \xucr .g  & \.g     & \verb|.G|, \verb|.g| & 10B6 & 2D16 \\ \hline
\.qar      & \xucr Q   & \xucr q   & \.q     & \verb|Q|, \verb|q|\footnotemark & 10B7 & 2D17 \\ \hline
\v san     & \xucr +S  & \xucr +s  & \v s    & \verb|+S|, \verb|+s| & 10B8 & 2D18 \\ \hline
\v cin     & \xucr +C  & \xucr +c  & \v c    & \verb|+C|, \verb|+c| & 10B9 & 2D19 \\ \hline
can        & \xucr C   & \xucr c   & c       & \verb|C|, \verb|c|  & 10BA & 2D1A \\ \hline
dzil       & \xucr J   & \xucr j   & j/dz    & \verb|J|, \verb|j|  & 10BB & 2D1B \\ \hline
\d cil     & \xucr .C  & \xucr .c  & \d c    & \verb|.C|, \verb|.c| & 10BC & 2D1C \\ \hline
\d{\v c}ar & \xucr .+C & \xucr .+c & \d{\v c}& \verb|.+C|, \verb|.+c| & 10BD & 2D1D \\ \hline
xan        & \xucr X   & \xucr x   & x       & \verb|X|, \verb|x| & 10BE & 2D1E  \\ \hline
d\v zan    & \xucr +J  & \xucr +j  & d\v z   & \verb|+J|, \verb|+j| & 10BF & 2D1F \\ \hline
ha         & \xucr H   & \xucr h   & h       & \verb|H|, \verb|h|  & 10C0 & 2D20 \\ \hline
\end{tabular}
\end{center}

\footnotetext{Although the transcription of `{\xucr Q}' and
`{\xucr q}' is {\it \.Q\/} and {\it \.q\/} respectively, the
\TeX-input will remain `{\tt Q}' and `{\tt q}' for the letter {\it \.qar\/} 
as it appears with a far more frequent than {\it qar\/}
(`{\xucr Q1}', `{\xucr q1}') which is coded as `{\tt Q1}' and `{\tt q1}'.}


\begin{center}
\begin{tabular}{|l |c| c |c| c|}
\hline
%           & \it Xucuri &  & \\
%\cline{2-2}
%\raisebox{1.5ex}[1.5ex]{Name}  & normal  &
%\raisebox{1.5ex}[1.5ex]{Transcription} & 
%\raisebox{1.5ex}[1.5ex]{\TeX-input} \\
Name & \multicolumn{2}{c|}{\it Xucuri} & Transcription & \TeX-input \\
     & majuscules & minuscules & & \\
     & \it Mrg(v)lovani & \it \d Kutxovani & & \\
\hline\hline
qar    & \xucr Q1  & \xucr q1  & q       & \verb|Q1|, \verb|q1|  \\ \hline
h\=e   & \xucr E0  & \xucr e0  & \=e, e\u\i & \verb|E0|, \verb|e0|  \\ \hline
ho     & \xucr O1  & ---\footnotemark       & \=o, oy & \verb|O1|       \\ \hline
jo     & \xucr I1  & \xucr i1  & \u\i    & \verb|I1|, \verb|i1|  \\ \hline
wi     & \xucr W   & \xucr w   & w       & \verb|W|, \verb|w|    \\ \hline
\end{tabular}
\end{center}

\footnotetext{The letter {\it ho\/} does only have a majuscule form.}

\subsubsection{Punctuation}
Please refer to section \ref{punkte} on page
\pageref{punkte}. Punctuation for {\it Xucuri\/} is encoded exactly as
for {\it Mxedruli\/}

\subsection{Correspondant letters}

This table shows, which character of {\it Mxedruli\/} corresponds to
which {\it Xucuri\/} character. Characters not found here do only occur in
one of the alphabets.

\begin{center}
\begin{tabular}{|l |c |c | c|}\hline
           & \it Mxedruli & \multicolumn{2}{c|}{\it Xucuri}  \\ 
           &              & majuscules & minuscules \\
           &              & \it Mrg(v)lovani & \it \d Kutxovani \\
\hline
an         & \mxedr a   & \xucr A   & \xucr a   \\ \hline
ban        & \mxedr b   & \xucr B   & \xucr b   \\ \hline
gan        & \mxedr g   & \xucr G   & \xucr g   \\ \hline
don        & \mxedr d   & \xucr D   & \xucr d   \\ \hline
en         & \mxedr e   & \xucr E   & \xucr e   \\ \hline
vin        & \mxedr v   & \xucr V   & \xucr v   \\ \hline
zen        & \mxedr z   & \xucr Z   & \xucr z   \\ \hline
tan        & \mxedr t   & \xucr T   & \xucr t   \\ \hline
in         & \mxedr i   & \xucr I   & \xucr i   \\ \hline
\d kan     & \mxedr .k  & \xucr .K  & \xucr .k  \\ \hline
las        & \mxedr l   & \xucr L   & \xucr l   \\ \hline
man        & \mxedr m   & \xucr M   & \xucr m   \\ \hline
nar        & \mxedr n   & \xucr N   & \xucr n   \\ \hline
\end{tabular}

\begin{tabular}{|l |c |c | c|}\hline
           & \it Mxedruli & \multicolumn{2}{c|}{\it Xucuri}  \\ 
           &              & majuscules & minuscules \\
           &              & \it Mrg(v)lovani & \it \d Kutxovani \\
\hline
on         & \mxedr o   & \xucr O   & \xucr o   \\ \hline
\.par      & \mxedr .p  & \xucr .P  & \xucr .p  \\ \hline
\v zan     & \mxedr +z  & \xucr +Z  & \xucr +z  \\ \hline
ran        & \mxedr r   & \xucr R   & \xucr r   \\ \hline
san        & \mxedr s   & \xucr S   & \xucr s   \\ \hline
\d tar     & \mxedr .t  & \xucr .T  & \xucr .t  \\ \hline
un         & \mxedr u   & \xucr U   & \xucr u   \\ \hline
par        & \mxedr p   & \xucr P   & \xucr p   \\ \hline
kan        & \mxedr k   & \xucr K   & \xucr k   \\ \hline
\.gan      & \mxedr .g  & \xucr .G  & \xucr .g  \\ \hline
\.qar      & \mxedr q   & \xucr Q   & \xucr q   \\ \hline
\v san     & \mxedr +s  & \xucr +S  & \xucr +s  \\ \hline
\v cin     & \mxedr +c  & \xucr +C  & \xucr +c  \\ \hline
can        & \mxedr c   & \xucr C   & \xucr c   \\ \hline
dzil       & \mxedr j   & \xucr J   & \xucr j   \\ \hline
\d cil     & \mxedr .c  & \xucr .C  & \xucr .c  \\ \hline
\d{\v c}ar & \mxedr .+c & \xucr .+C & \xucr .+c \\ \hline
xan        & \mxedr x   & \xucr X   & \xucr x   \\ \hline
d\v zan    & \mxedr +j  & \xucr +J  & \xucr +j  \\ \hline
ha         & \mxedr h   & \xucr H   & \xucr h   \\ \hline
%\hline
%% \end{tabular}
%% \end{center}
%% 
%% \begin{center}
%% \begin{tabular}{|l |c |c|}
%% \hline
%%            & \it Mxedruli & \it Xucuri  \\ \hline
qar        & \mxedr  q1 & \xucr Q1  & \xucr q1  \\ \hline
h\=e       & \mxedr  e0 & \xucr E0  & \xucr e0  \\ \hline
ho         & \mxedr  o1 & \xucr O1  & \xucr ---  \\ \hline
jo         & \mxedr  i1 & \xucr I1  & \xucr i1  \\ \hline
wi         & \mxedr  w  & \xucr W   & \xucr w   \\ \hline
\end{tabular}
\end{center}


%%%%%%%%%%%%%%%%%%%%%%%%%%%%%%%%%%%%%%%%%%%%%%%%%%%%%%%%%%%%%%%%%%%%%%%%
%%%%%%%%%%%%%%%%%%%%%%%%%%%% FIRST EXAMPLE %%%%%%%%%%%%%%%%%%%%%%%%%%%%%
%%%%%%%%%%%%%%%%%%%%%%%%%%%%%%%%%%%%%%%%%%%%%%%%%%%%%%%%%%%%%%%%%%%%%%%%



\section{Examples}\label{Examples}
\subsection{{\mxedb  ve.pxis .tqaosani} -- {The Knight in the Tiger's Skin}}
The following example are two stanzas from {\it \v Sota Rustaveli\/}'s
{\it Ve\.pxis \d T\.qaosani\/} ``The Knight in the Tiger's Skin''.%
\footnote{This example you can find slightly changed in the example
file {\tt vepxis.tex}.}

\begin{center}
{\Large\mxedc ve.pxis .tqaosani}

\medskip
{\large\mxedc +sota rustaveli}
\end{center}

\bigskip

\begin{verse}
\begin{mxedr}
.gmertsa +semvedre, nutu .kvla damxsnas soplisa +sromasa,\\
cecxls, .cqalsa da mi.casa, haerta tana +sromasa;\\
momcnes prteni da a.gvprinde, mivhxvde mas +cemsa ndomasa,\\
d.gisit da .gamit vhxedvide mzisa elvata .krtomasa.

\medskip
mze u+senod ver ikmdebis, ratgan +sen xar masa .cili,\\
gana.gamca mas eaxel misi e.tli, ar tu .cbili!\\
muna gnaxo, madve gsaxo, ganminatlo guli +crdili,\\
tu sicocxle m.care mkonda, si.kvdilimca mkonda .t.kbili!
\end{mxedr}
\end{verse}

\bigskip
This was set with the following:


\begin{small}
\begin{verbatim}
   \documentclass[12pt]{article}
   \usepackage{mxedruli}
   \begin{document}
   \begin{center}
   {\Large\mxedc ve.pxis .tqaosani}

   \medskip
   {\large\mxedc +sota rustaveli}
   \end{center}

   \bigskip

   \begin{verse}
   \begin{mxedr}
   .gmertsa +semvedre, nutu .kvla damxsnas soplisa 
         +sromasa,\\
   cecxls, .cqalsa da mi.casa, haerta tana +sromasa;\\
   momcnes prteni da a.gvprinde, mivhxvde mas +cemsa 
         ndomasa,\\
   d.gisit da .gamit vhxedvide mzisa elvata .krtomasa.

   \medskip
   mze u+senod ver ikmdebis, ratgan +sen xar masa .cili,\\
   gana.gamca mas eaxel misi e.tli, ar tu .cbili!\\
   muna gnaxo, madve gsaxo, ganminatlo guli +crdili,\\
   tu sicocxle m.care mkonda, si.kvdilimca mkonda .t.kbili!
   \end{mxedr}
   \end{verse}
   \end{document}
\end{verbatim}
\end{small}


%%%%%%%%%%%%%%%%%%%%%%%%%%%%%%%%%%%%%%%%%%%%%%%%%%%%%%%%%%%%%%%%%%%%%%%%
%%%%%%%%%%%%%%%%%%%%%%%%%%%% SECOND EXAMPLE %%%%%%%%%%%%%%%%%%%%%%%%%%%%
%%%%%%%%%%%%%%%%%%%%%%%%%%%%%%%%%%%%%%%%%%%%%%%%%%%%%%%%%%%%%%%%%%%%%%%%


\subsection{Another example with different font sizes}
The second sample text is due to the lack of some proper text%
\footnote{If you would send me a sample text, you prepared with
these fonts to be included in this documentation, I would be most grateful.}
when I prepared this file just taken from the preface of a small
Georgian-German dictionary, but it should give another impression how
this font -- using different sizes and the ``capitals''-- looks like as well:

\begin{center}
{\LARGE\mxedc kartul--germanuli 
leksi.koni}

\medskip
\mxedi +sedgenili: \\
{\large\mxedb o. xuci+svilis da \\
t. xa.tia+svilis mier}

\bigskip
\bigskip
{\mxedc gamomcemloba ,,ganatleba''\\
tbilisi ---} 1977
\end{center}

\bigskip
\mxedr
.cinamdebare kartul--germanul leksi.kon+si +ser+ceulia kartuli
enis leksi\-.ki\-dan si.tqvebi, romlebic qvelaze me.tad
ixmareba sala.para.ko ena+si, agretve mxa.tvrul, .poli.ti.kur,
mecnierul da sas.cavlo li.tera.tura+si. ganmar.tebulia si.tqvis
qvela jiritadi mni+svneloba da ilus.trirebulia nimu\-+se\-bit.
vcdilobdit, satanado prazeologizmebis mo+sveleibit gang\-ve\-mar\-.ta
si.tqvis niuansobrivi gagebanic.

\rm
The input was:


\begin{small}
\begin{verbatim}
  \documentclass[12pt]{article}
  \usepackage{mxedruli}
  \begin{document}
  \begin{center}
  {\LARGE\mxedc kartul--germanuli 
  leksi.koni}

  \medskip
  \mxedi +sedgenili: \\
  {\large\mxedb o. xuci+svilis da \\
  t. xa.tia+svilis mier}

  \bigskip
  \bigskip
  {\mxedc gamomcemloba ,,ganatleba''\\
  tbilisi ---} 1977
  \end{center}

  \bigskip \mxedr 
  .cinamdebare kartul--germanul leksi.kon+si
  +ser+ceulia kartuli enis lek\-si.ki\-dan si.tqvebi,
  romlebic qvelize me.tad ixmareba sala.para.ko eni+si,
  abretve mxa.tvrul, .poli.ti.kur, mecnierul da sas.cavlo
  li.tera.turi+si. ganmar.tbulia si.tqvis qvela jiritadi
  mni+svneloba da ilus.trarebulia nimu\-+se\-bit.
  vcdilobdit, satanado prazeologizsegis mo+svelibit
  gang\-ve\-mar\-.ta si.tqvis niuansobrivi gagebanic.
  \end{document}
\end{verbatim}
\end{small}


\subsection{An example using \emph{Xucuri}}
The following example is taken from {\it N. Marr and M. Bri\`ere, 
La Langue G\'eor\-gienne, Paris 1931\/}, p. 595:

\begin{xucr}
\begin{center}
SAXAREBAI1 MATEES TAVISAI1.

B
\end{center}

{\rm 1.} --- XOLO IESOW KRIS.TEES +SUBASA BETLEMS
HOWRIAS.TANISASA. D.GETA HERODE MEPISATA. AHA
MOGOWNI A.GMOSAVALIT MOVIDES IEROWSALEEMD DA
I.TQODES:

{\rm 2.} --- SADA ARS ROMELI IGI I+SVA. MEOWPEE
HOWRIATAI1? RAI1\-RETOW VIXILET VARS.KOWLAVI MISI
A.GMOSAVALIT DA MOVEDIT TAV\-QOWA\-NIS-CEMAD MISA.

{\rm 3.} --- VITARCA ESMA ESE HERODES
MEPESA. +SEJR.COWNDA DA +SOVELI IEROWSALEEMI
MISTANA.

\end{xucr}

This was generated using the following input:


\begin{small}
\begin{verbatim}
   \documentclass[12pt]{article}
   \usepackage{xucuri}

   \begin{xucr}
   \begin{center}
   SAXAREBAI1 MATEES TAVISAI1.
   
   B
   \end{center}
   
   {\rm 1.} --- XOLO IESOW KRIS.TEES +SUBASA BETLEMS
   HOWRIAS.TANISASA. D.GETA HERODE MEPISATA. AHA
   MOGOWNI A.GMOSAVALIT MOVIDES IEROWSALEEMD DA
   I.TQODES:
   
   {\rm 2.} --- SADA ARS ROMELI IGI I+SVA. MEOWPEE
   HOWRIATAI1? RAI1RETOW VIXILET VARS.KOWLAVI MISI
   A.GMOSAVALIT DA MOVEDIT TAVQOWANIS-CEMAD MISA.
   
   {\rm 3.} --- VITARCA ESMA ESE HERODES
   MEPESA. +SEJR.COWNDA DA +SOVELI IEROWSALEEMI
   MISTANA.
   \end{xucr}
\end{verbatim}
\end{small}


The example which illustrates the {\it Xucuri\/} minuscules is again
taken from {\it N. Marr and M. Bri\`ere, La Langue G\'eor\-gienne,
Paris 1931\/}, p. 599: 

\begin{xucr}
\begin{center}\Large
saxarebai1 lu.kai1s tavisai1.

ie
\end{center}

{\rm 11.} --- merme i.tqoda da tkua: .kacsa visme escnes
or je.

{\rm 12.} --- da hrkua umr.cemesman man mamasa twssa:
mamao. momec me romeli mxudebis na.cilidam.kw drebelisa. da ganuqo
mat sacxovrebeli igi.
\end{xucr}


The input was


\begin{small}
\begin{verbatim}
  \begin{xucr}
  \begin{center}\Large
  saxarebai1 lu.kai1s tavisai1.

  ie
  \end{center}

  {\rm 11.} --- merme i.tqoda da tkua: .kacsa visme escnes
  or je.

  {\rm 12.} --- da hrkua umr.cemesman man mamasa twssa:
  mamao. momec me romeli mxudebis na.cilidam.kw drebelisa. 
  da ganuqo mat sacxovrebeli igi.
\end{xucr}
\end{verbatim}
\end{small}

%%%%%%%%%%%%%%%%%%%%%%%%%%%%%%%%%%%%%%%%%%%%%%%%%%%%%%%%%%%%%%%%%%%%%%%%
%%%%%%%%%%%%%%%%%%%%%%%%%%%% FONT CODING %%%%%%%%%%%%%%%%%%%%%%%%%%%%%%%
%%%%%%%%%%%%%%%%%%%%%%%%%%%%%%%%%%%%%%%%%%%%%%%%%%%%%%%%%%%%%%%%%%%%%%%%

\section{Font coding}
\def\cell#1{\char'#1}

\def\cellrow#1{
      &  \cell{#10} & \cell{#11} & \cell{#12} & \cell{#13} &
         \cell{#14} & \cell{#15} & \cell{#16} & \cell{#17} \\ \hline}

\def\reihe#1{\it '#10 \cellrow{#1}}

The following table shows the internal encoding of the 
defined letters of {\it Mxedruli\/} and {\it Xucuri\/}:


\begin{center}
{\mxedr
\begin{tabular}{ r || c | c | c | c | c | c | c | c | }
  & \it 0 & \it 1 & \it 2 & \it 3 & \it 4 & \it 5 & \it 6 & \it 7  \\
\hline\hline
\reihe{00}\reihe{01}\reihe{02}\reihe{03}\reihe{04}
\reihe{05}\reihe{06}\reihe{07}\reihe{10}\reihe{11}
\reihe{12}\reihe{13}\reihe{14}\reihe{15}\reihe{16}
\reihe{17}
\end{tabular}}
%\end{center}

%\begin{center}
{\xucr
\begin{tabular}{ r || c | c | c | c | c | c | c | c | }
  & \it 0 & \it 1 & \it 2 & \it 3 & \it 4 & \it 5 & \it 6 & \it 7  \\
\hline\hline
\reihe{00}\reihe{01}\reihe{02}\reihe{03}\reihe{04}
\reihe{05}\reihe{06}\reihe{07}\reihe{10}\reihe{11}
\reihe{12}\reihe{13}\reihe{14}\reihe{15}\reihe{16}
\reihe{17}
\end{tabular}}
\end{center}





%%%%%%%%%%%%%%%%%%%%%%%%%%%%%%%%%%%%%%%%%%%%%%%%%%%%%%%%%%%%%%%%%%%%%%%%
%%%%%%%%%%%%%%%%%%%%%%%%%%%% FILES %%%%%%%%%%%%%%%%%%%%%%%%%%%%%%%%%%%%%
%%%%%%%%%%%%%%%%%%%%%%%%%%%%%%%%%%%%%%%%%%%%%%%%%%%%%%%%%%%%%%%%%%%%%%%%

\section{Files}\label{Install}
The package consists of the following files:

\begin{list}{}{%
  \labelwidth55mm
  \labelsep2mm
  \itemindent0mm
  \parsep0mm
  \leftmargin57mm
  \topsep0mm
  \itemsep0pt
 }
\renewcommand{\makelabel}[1]{#1\hfill}


\item[\tt README.txt] A short information file.
%\item[\tt fonts/pk300/mxed10.*pk] The {\tt .pk}-Files for standard 
%    {\it Mxedruli\/} generated for use with 300 dpi printers.
%\item[\tt fonts/pk300/mxedbf10.*pk] The {\tt .pk}-Files for bold 
%    {\it Mxedruli\/} generated for use with 300 dpi printers.
%\item[\tt fonts/pk300/mxedi10.*pk] The {\tt .pk}-Files for italics
%    {\it Mxedruli\/} generated for use with 300 dpi printers.
%\item[\tt fonts/pk300/mxedc10.*pk] The {\tt .pk}-Files for capital
%    {\it Mxedruli\/} generated for use with 300 dpi printers.
%\item[\tt fonts/pk300/xuc10.*pk] The {\tt .pk}-Files for 
%    {\it Xucuri\/} generated for use with 300 dpi printers.
%\item [\tt fonts/tfm/*.tfm] \TeX\ Font Metrics Files.
 
\item[\tt inputs/umxed.fd] Font definition files for use with NFSS2.
\item[\tt inputs/uxuc.fd]  %Font definition file for use with NFSS2.
\item[\tt inputs/mxedruli.sty] \LaTeX\ Style File.
\item[\tt inputs/xucuri.sty] % \LaTeX\ Style File.

\item[\tt mf/mxed.mf] The standard generation file for {\logo METAFONT}.
\item[\tt mf/mxed10.mf] Driver file for normal {\it Mxedruli\/}.
     Run {\logo METAFONT} on this file to generate {\it Mxedruli\/} 
     for any other resolution than 300 dpi.
\item[\tt mf/mxedbf10.mf] As before, but bold {\it Mxedruli\/}.
\item[\tt mf/mxedi10.mf] As before, but italics {\it Mxedruli\/}.
\item[\tt mf/mxedc10.mf] As before, but capital {\it Mxedruli\/}.
\item[\tt mf/mxedacc.mf] Accents for other Kartvelian languages (Svan).
\item[\tt mf/mxedbase.mf] {\logo METAFONT} macros etc.
\item[\tt mf/mxedcaps.mf] ``Capital'' letters.
\item[\tt mf/mxedd.mf] Digits (can be replaced by Computer Modern digits,
     cf. below).
\item[\tt mf/mxedfont.mf] Letters.
\item[\tt mf/mxedp.mf] Punctuation (can be replaced by Computer Modern
     punctuation, cf. below).

\item[\tt mf/xuc.mf] The standard generation file for {\logo METAFONT}.
\item[\tt mf/xuc10.mf] Driver file for {\it Xucuri\/}.
     Run {\logo METAFONT} on this file to generate {\it Xucuri\/} 
     for any other resolution than 300 dpi.
\item[\tt mf/xucbase.mf] {\logo METAFONT} macros etc.
\item[\tt mf/xucfont.mf] {\it Xucuri\/} majuscules.
\item[\tt mf/xucd.mf] Digits (can be replaced by Computer Modern digits,
     cf. below).
\item[\tt mf/xucl.mf] {\it Xucuri\/} minuscules.
\item[\tt mf/xucp.mf] Punctuation (can be replaced by Computer Modern
     punctuation, cf. below).
\item[\tt afm/*.afm] afm files for use with type1 fonts
\item[\tt type1/*.pbf] type1 vector fonts
\item[\tt alphabets.tex] An example showing both Xucuri and Mxedruli
\item[\tt mxeddoc.ps] This documentation.
\item[\tt ossetic.tex] A short example of Ossetic text, written in
     {\it Mxedruli\/}.
\item[\tt vepxis.tex] The first {\it Mxedruli\/}-example of this 
     documentation.

\end{list}

To install the fonts etc., please copy all {\tt .mf}-files from the 
{\tt mf}-directory to the appropriate directory of
your \TeX-system. 
%The {\tt .tfm}-files must be copied into
%the {\tt tfm}-directory of your \TeX-system. 
Further, the files
from the {\tt inputs}-directory must be copied into the directory
where \LaTeX\ can find them.

\subsection{Debian package}

If you are on a Ubuntu platform (at least Ubuntu 10.04 or 12.04), you can install
the necessary files (style-files and type1 fonts) via the debian package

{\tt sudo dpkg -i mxedruli-\mxedversion.deb}


%%%%%%%%%%%%%%%%%%%%%%%%%%%%%%%%%%%%%%%%%%%%%%%%%%%%%%%%%%%%%%%%%%%%%%%%
%%%%%%%%%%%%%%%%%%%%%%%%%%%% METAFONT %%%%%%%%%%%%%%%%%%%%%%%%%%%%%%%%%%
%%%%%%%%%%%%%%%%%%%%%%%%%%%%%%%%%%%%%%%%%%%%%%%%%%%%%%%%%%%%%%%%%%%%%%%%

\section{Using {\logogr METAFONT}} % on {\tt mxed[bf|c|i]10.mf}/{\tt xuc10.mf}}
\label{MF}
If you are going to regenerate the fonts, there are two possibilities,
using the Computer Modern digits and punctuation (which are far nicer)
or the digits and punctuation provided in \verb|mxedd.mf| and
\verb|mxedp.mf| or \verb|xucp.mf| respectively.

In the first case you should use the following command in a UN*X 
environment:\footnote{Do so analogously for {\tt mxedbf10}, 
{\tt mxedi10}, {\tt mxedc10} and {\tt xuc10} respectively.}\\
\verb|  mf '&cmmf \mode=|{\it your\_mode\/}\verb|;' input mxed10|


Else you use plain {\logo METAFONT} by:\\
\verb|  mf '\mode=|{\it your\_mode\/}\verb|;' input mxed10|

(Other sizes are generated by:\\
\verb|  mf '&cmmf \mode=|{\it your\_mode\/}\verb|; mag=magstep|{\it
n\/}\verb|' input mxed10|

or\\
\verb|  mf '\mode=|{\it your\_mode\/}\verb|; mag=magstep|{\it
n\/}\verb|' input mxed10|

respectively.) {\it your\_mode\/} has to be replaced by the mode your
printer requires, e.g. \verb|localmode| or \verb|laserjet|, {\it n\/} by
a valid magstep ({\tt 1}, {\tt 2}, {\tt 3}, {\tt 4}, {\tt 5}
or {\tt half}).

In either case \verb|mf| must be followed by \verb|gftopk| to 
generate the \verb|.pk|-files. Please refer to the documentation
of \verb|gftopk| for further information on postprocessing
{\logo METAFONT} output.

\section{License}

This Material is subjec to the LaTeX Project Public Li­cense 1.3 
(\url{http://ctan.org/license/lppl1.3}).

\section{Changelog}

\begin{itemize}
\item 6th April 2013: Version 3.4

    there is now a debian package which installs type1 fonts, tfm, afm files as well
    as style files on Ubuntu 10.04, Ubuntu 12.04 platforms and probably (untested) other 
    debian and derived platforms 

\item 18th January 2009: Version 3.3c

    10 years, and the only thing changed is the doc (reference to Unicode codes and some minor details)

\item 1st September 1999: Version 3.3
    
Changed Fontcoding U to u

\item 15th April 1997: Version 3.0

    Added Xucuri characters (upper and lower case)

\item 15th July 1996: Version 2.4

    Added italics, minor corrections

\item 20th May 1996: Version 2.3

    Added new letter: Glottal Stop (reverse `q')

\item 5th March 1996: Version 2.2

    Added two more letters: `qhar' and `ee'

\item 26th June 1995: Version 2.1

    Minor corrections (not distributed via CTAN)

\item 16th June 1995: Version 2.0

    `Capital' fonts included

    Introduced some letters for Old Georgian and Ossetian resp.

    The letter .+c is now at position oct(014), not at
       postion oct(171) where it used to be. This was necessary
       as a letter being transcribed by "y" (oct(171)) was introduced.

\item 8th August 1994: Version 1.0
   
 Some Corrections on too mishaped letters.
    Introduced some letters for Old Georgian.

\item September 1993:
    
First Release.
\end{itemize}

\end{document}

%%%%%%%%%%%%%%%%%%%%%%%%%%% EOF %%%%%%%%%%%%%%%%%%%%%%%%%%%%%%%%%%%

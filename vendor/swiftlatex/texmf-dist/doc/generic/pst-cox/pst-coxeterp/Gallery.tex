%%%%%%%%%%%%%%%%%%%%%%%%%%%%%%%%%%%%%%%%%%%%%%%%%%%%%%%%%%%%%%%%%%
%    pst-coxeter_parameter\Gallery.tex
%    Authors:      J.-G. Luque and M. Luque
%    Purpose:      Demonstration of the library pst-coxeterp
%    Created:      02/02/2008
%    License:      LGPL
%    Project:      PST-Cox V1.00
%%%%%%%%%%%%%%%%%%%%%%%%%%%%%%%%%%%%%%%%%%%%%%%%%%%%%%%%%%%%%%%%%%%
% Copyright � 2008 Jean-Gabriel Luque, Manuel Luque.
% This work may be distributed and/or modified  under the condition of
% the Lesser GPL.
%%%%%%%%%%%%%%%%%%%%%%%%%%%%%%%%%%%%%%%%%%%%%%%%%%%%%%%%%%%%%%%%%%%
% This file is part of PST-Cox V1.00.
%
%    PST-Cox V1.00 is free software: you can redistribute it and/or modify
%    it under the terms of the Lesser GNU General Public License as published by
%    the Free Software Foundation, either version 3 of the License, or
%    (at your option) any later version.
%
%    PST-Cox V1.00 is distributed in the hope that it will be useful,
%    but WITHOUT ANY WARRANTY; without even the implied warranty of
%    MERCHANTABILITY or FITNESS FOR A PARTICULAR PURPOSE.  See the
%    Lesser GNU General Public License for more details.
%
%    You should have received a copy of the Lesser GNU General Public License
%    along with PST-Cox V1.00.  If not, see <http://www.gnu.org/licenses/>.
%%%%%%%%%%%%%%%%%%%%%%%%%%%%%%%%%%%%%%%%%%%%%%%%%%%%%%%%%%%%%%%%%%%%%%%%
\documentclass[a4paper]{article}
\usepackage[latin1]{inputenc}%
\usepackage[margin=2cm]{geometry}
\usepackage{pst-coxeterp}
\usepackage{multido}
\usepackage{amssymb}
\usepackage{amsfonts}
\usepackage{amsmath}
\usepackage{graphics}
% d\'emonstration
% JG Luque 12 ao�t 2003
\newcount\ChoicePolytope
\def\S{\mbox{\goth S}}
\def\Sym{{\bf Sym}}
\def\sym{{\sl Sym}}
\def\QSym{{QSym}}
\def\N{{\mathbb N}}\def\L{{\mathbb L}}
\def\C{{\mathbb C}}
\def\Z{{\mathbb Z}}
\def\R{{\mathbb R}}
\def\Q{{\mathbb Q}}
\def\demoPolytopes#1{%}
\begin{center}
\ifcase\ichoice\or  \def\polname{$2\{3\}3$}\def\ep{0.5mm}
 \or \def\polname{$3\{3\}2$}\def\ep{0.3mm}\or
\def\polname{$3\{3\}3$}\def\ep{0.3mm}\or
 \def\polname{$3\{4\}2$}\def\ep{0.3mm}\or \def\polname{$3\{4\}4$}\def\ep{0.1mm}
 \or \def\polname{$3\{4\}3$}\def\ep{0.1mm}\or \def\polname{$4\{3\}4$}\def\ep{0.1mm}\or
\def\polname{$2\{4\}3\{3\}3$}\def\ep{0.1mm}\or \def\polname{ Hessien}\def\ep{0.1mm}
 \or  \def\polname{$3\{3\}3\{4\}2$}\def\ep{0.1mm}
 \or \def\polname{de Witting} \def\ep{0.01mm} \or
 \def\polname{$3\{8\}2$} \def\ep{0.1mm} \or
 \def\polname{$2\{8\}3$} \def\ep{0.1mm}  \or
  \def\polname{$3\{5\}3$} \def\ep{0.1mm}
 \or\def\polname{$4\{4\}3$} \def\ep{0.1mm}
 \or\def\polname{$4\{3\}2$} \def\ep{0.1mm}
 \or\def\polname{$2\{3\}4$} \def\ep{0.1mm}
 \or\def\polname{$2\{6\}4$} \def\ep{0.1mm}
 \or\def\polname{$4\{6\}2$} \def\ep{0.1mm}
  \or\def\polname{$5\{3\}5$} \def\ep{0.1mm}
  \or\def\polname{$2\{10\}3$} \def\ep{0.1mm}
  \or\def\polname{$3\{10\}2$} \def\ep{0.1mm}
  \or\def\polname{$2\{5\}3$} \def\ep{0.1mm}
  \or\def\polname{$3\{5\}2$} \def\ep{0.1mm}
  \or\def\polname{$2\{4\}3$} \def\ep{0.1mm}
  \or\def\polname{$2\{3\}2\{4\}3$} \def\ep{0.1mm}
   \or\def\polname{$3\{4\}2\{3\}2$} \def\ep{0.1mm}
   \or\def\polname{$3\{4\}2\{3\}2\{3\}2$} \def\ep{0.1mm}
   \or\def\polname{$2\{3\}2\{3\}2\{4\}3$} \def\ep{0.1mm}
   \fi {\Huge Polytope \polname}

\begin{pspicture}(-9,-9)(9,9)
\psset{unit=3cm,linewidth=0.01mm}
\CoxeterCoordinates[choice=#1,linewidth=\ep] % par d�faut choice=1 (332)
\end{pspicture}

$\backslash$\texttt{CoxeterCoordinates[choice=#1]}
\end{center}
\begin{center}
\begin{tabular}{ccc}
\begin{pspicture}(-2,-2)(2,2)
\psset{unit=0.7cm}
\CoxeterCoordinates[drawvertices=false,choice=#1,linewidth=0.01mm] %
\end{pspicture}
&
\begin{pspicture}(-2,-2)(2,2)
\psset{unit=0.7cm}
\CoxeterCoordinates[drawcenters=false,choice=#1,linewidth=0.01mm] %
\end{pspicture}
&
\begin{pspicture}(-2,-2)(2,2)
\psset{unit=0.7cm}
\CoxeterCoordinates[drawedges=false,choice=#1,linewidth=0.01mm] %
\end{pspicture}\\
\texttt{[drawvertices=false,choice=#1]}
&
\texttt{[drawcenters=false,choice=#1]}
&
\texttt{[drawedges=false,choice=#1]}
\end{tabular}
\end{center}}
%
\title{The Gallery of Infinite Series}
\author{Jean-Gabriel \textsc{Luque}\footnote{Jean-Gabriel.Luque@univ-mlv.fr},
Manuel \textsc{Luque}\footnote{manuel.luque27@gmail.com}}
\begin{document}
\maketitle
\newpage
\section{Real polygons}
There are the polytopes $2\{\frac pq\}2$ (with $p$ and $q$ in $\N$)
in the notation of Coxeter. Use the command:
\begin{verbatim}
\psset{unit=1.5cm}\Polygon[P=p,Q=q]
\end{verbatim}
\[\begin{array}{|c|c|c|}
\hline 2&3&4\\
\hline \begin{pspicture}(-1.5,-3)(1.5,3)
\psset{unit=1.5cm}\Polygon[P=2,Q=1]
\end{pspicture}&\begin{pspicture}(-3,-3)(3,3)
\psset{unit=1.5cm}\Polygon[P=3]
\end{pspicture}&\begin{pspicture}(-3,-3)(3,3)
\psset{unit=1.5cm}\Polygon[P=4]
\end{pspicture}\\
\hline 5&\frac52&6\\
\hline \begin{pspicture}(-1.5,-3)(1.5,3)
\psset{unit=1.5cm}\Polygon[P=5,Q=1]
\end{pspicture}&\begin{pspicture}(-3,-3)(3,3)
\psset{unit=1.5cm}\Polygon[P=5,Q=2]
\end{pspicture}&\begin{pspicture}(-3,-3)(3,3)
\psset{unit=1.5cm}\Polygon[P=6]
\end{pspicture}\\
\hline 7&\frac72&\frac73\\
\hline \begin{pspicture}(-1.5,-3)(1.5,3)
\psset{unit=1.5cm}\Polygon[P=7]
\end{pspicture}&\begin{pspicture}(-3,-3)(3,3)
\psset{unit=1.5cm}\Polygon[P=7,Q=2]
\end{pspicture}&\begin{pspicture}(-3,-3)(3,3)
\psset{unit=1.5cm}\Polygon[P=10,Q=3]
\end{pspicture}\\
\hline
\end{array}
\]
\newpage
\section{Simplices }
There are the real polytopes $2\{3\}2\cdots2\{3\}2$ in dimension $n$
(tetrahedron, pentatope, sextatope etc...) in the notation of
Coxeter. Use the command:
\begin{verbatim}
\psset{unit=1.5cm}\Simplex[dimension=n]
\end{verbatim}
\[\begin{array}{|c|c|c|}
\hline 2&3&4\\
\hline \begin{pspicture}(-1.5,-3)(1.5,3)
\psset{unit=1.5cm}\Simplex[dimension=2]
\end{pspicture}&\begin{pspicture}(-3,-3)(3,3)
\psset{unit=1.5cm}\Simplex[dimension=3]
\end{pspicture}&\begin{pspicture}(-3,-3)(3,3)
\psset{unit=1.5cm}\Simplex[dimension=4]
\end{pspicture}\\
\hline 5&6&7\\
\hline \begin{pspicture}(-1.5,-3)(1.5,3)
\psset{unit=1.5cm}\Simplex[dimension=5]
\end{pspicture}&\begin{pspicture}(-3,-3)(3,3)
\psset{unit=1.5cm}\Simplex[dimension=6]
\end{pspicture}&\begin{pspicture}(-3,-3)(3,3)
\psset{unit=1.5cm}\Simplex[dimension=7]
\end{pspicture}\\
\hline 8&9&10\\
\hline \begin{pspicture}(-1.5,-3)(1.5,3)
\psset{unit=1.5cm}\Simplex[dimension=8]
\end{pspicture}&\begin{pspicture}(-3,-3)(3,3)
\psset{unit=1.5cm}\Simplex[dimension=9]
\end{pspicture}&\begin{pspicture}(-3,-3)(3,3)
\psset{unit=1.5cm}\Simplex[dimension=10]
\end{pspicture}\\
\hline
\end{array}
\]\newpage
\section{The infinite series $\gamma_n^p$}
It is an infinite series of polytopes with two parameters $p$ and
$n$. The parameter $n$ is the dimension of the polytope. In the
notation of Coxeter, its name reads $p\{4\}2\{3\}\dots\{3\}2$. In
the case $p=2$, we recovers the family of the hypercubes. Use the
command:
 \begin{verbatim}
 \gammapn[P=p,dimension=n]
 \end{verbatim}
\[\begin{array}{|c|c|c|}
\hline \gamma_2^2&\gamma_2^3&\gamma_2^4\\
\hline \begin{pspicture}(-2,-3)(2,3)
\psset{unit=1.2cm}\gammapn[dimension=2,P=2,linewidth=0.01mm]
\end{pspicture}&\begin{pspicture}(-2,-3)(2,3)
\psset{unit=1.2cm}\gammapn[P=3,dimension=2,linewidth=0.01mm]
\end{pspicture}&\begin{pspicture}(-2,-3)(2,3)
\psset{unit=1cm}\gammapn[P=4,dimension=2,linewidth=0.01mm]
\end{pspicture}\\
\hline \gamma_3^2&\gamma_3^3&\gamma_3^4\\ \hline
\begin{pspicture}(-2,-3)(2,3)
\psset{unit=1cm}\gammapn[P=2,dimension=3,linewidth=0.01mm]
\end{pspicture}&\begin{pspicture}(-2,-3)(2,3)
\psset{unit=0.8cm}\gammapn[P=3,dimension=3,linewidth=0.01mm]
\end{pspicture}&\begin{pspicture}(-2,-3)(2,3)
\psset{unit=0.7cm}\gammapn[P=4,dimension=3,linewidth=0.01mm]
\end{pspicture}\\
\hline \gamma_4^2&\gamma_4^3&\gamma_4^4\\
\hline \begin{pspicture}(-2,-3)(2,3)
\psset{unit=0.8cm}\gammapn[P=2,dimension=4,linewidth=0.01mm]
\end{pspicture}&\begin{pspicture}(-2,-3)(2,3)
\psset{unit=0.6cm}\gammapn[P=3,dimension=4,linewidth=0.01mm]
\end{pspicture}&\begin{pspicture}(-2,-3)(2,3)
\psset{unit=0.55cm}\gammapn[P=4,dimension=4,linewidth=0.01mm]
\end{pspicture}\\
\hline
\end{array}
\]%
\newpage
\section{The infinite series $\beta_n^p$}
It is an infinite series of polytopes with two parameters $p$ and
$n$ reciprocals of $\gamma_n^p$. The parameter $n$ is the dimension
of the polytope. In the notation of Coxeter, its name reads
$2\{3\}2\{3\}\dots\{3\}2\{4\}p$. In the case $p=2$, we recovers the
family of the $2^n$-topes which generalizes the tetrahedron for
higher dimension. Use the command:
 \begin{verbatim}
 \betapn[P=p,dimension=n]
 \end{verbatim}
\[\begin{array}{|c|c|c|}
\hline \beta_2^2&\beta_2^3&\beta_2^4\\
\hline \begin{pspicture}(-2,-3)(2,3)
\psset{unit=2cm}\betapn[dimension=2,P=2]
\end{pspicture}&\begin{pspicture}(-2,-3)(2,3)
\psset{unit=1.5cm}\betapn[P=3,dimension=2,linewidth=0.01mm]
\end{pspicture}&\begin{pspicture}(-2,-3)(2,3)
\psset{unit=1.4cm}\betapn[P=4,dimension=2,linewidth=0.01mm]
\end{pspicture}\\
\hline \beta_3^2&\beta_3^3&\beta_3^4\\ \hline
\begin{pspicture}(-2,-3)(2,3)
\psset{unit=2cm}\betapn[P=2,dimension=3,linewidth=0.01mm]
\end{pspicture}&\begin{pspicture}(-2,-3)(2,3)
\psset{unit=1.5cm}\betapn[P=3,dimension=3,linewidth=0.01mm]
\end{pspicture}&\begin{pspicture}(-2,-3)(2,3)
\psset{unit=1.4cm}\betapn[P=4,dimension=3,linewidth=0.01mm]
\end{pspicture}\\
\hline \beta_4^2&\beta_4^3&\beta_4^4\\
\hline \begin{pspicture}(-2,-3)(2,3)
\psset{unit=2cm}\betapn[P=2,dimension=4,linewidth=0.01mm]
\end{pspicture}&\begin{pspicture}(-2,-3)(2,3)
\psset{unit=1.5cm}\betapn[P=3,dimension=4,linewidth=0.01mm]
\end{pspicture}&\begin{pspicture}(-2,-3)(2,3)
\psset{unit=1.4cm}\betapn[P=4,dimension=4,linewidth=0.01mm]
\end{pspicture}\\
\hline
\end{array}
\]%
\newpage
\section{The infinite series $\gamma_2^p$}
It is a special case of the series $\gamma_n^p$ for $n=2$. In this
case, the polytopes are complex polygons. The projection used here
is different than the projection used with {\tt gammapn}. Use the
command:
\begin{verbatim}
\gammaptwo[P=p]
\end{verbatim}
\[\begin{array}{|c|c|c|}
\hline \gamma_2^3&\gamma_2^4&\gamma_2^5\\
\hline \begin{pspicture}(-2,-3)(2,3) \psset{unit=1cm}\gammaptwo[P=3]
\end{pspicture}&\begin{pspicture}(-2,-3)(2,3)
\psset{unit=1cm}\gammaptwo[P=4,linewidth=0.01mm]
\end{pspicture}&\begin{pspicture}(-2,-3)(2,3)
\psset{unit=1cm}\gammaptwo[P=5,linewidth=0.01mm]
\end{pspicture}\\
\hline \gamma_2^6&\gamma_2^7&\gamma_2^8\\ \hline
\begin{pspicture}(-2,-3)(2,3)
\psset{unit=1cm}\gammaptwo[P=6,linewidth=0.01mm]
\end{pspicture}&\begin{pspicture}(-2,-3)(2,3)
\psset{unit=0.8cm}\gammaptwo[P=7,linewidth=0.01mm]
\end{pspicture}&\begin{pspicture}(-2,-3)(2,3)
\psset{unit=0.7cm}\gammaptwo[P=8,linewidth=0.01mm]
\end{pspicture}\\
\hline \gamma_2^9&\gamma_2^{10}&\gamma_2^{11}\\
\hline \begin{pspicture}(-2,-3)(2,3)
\psset{unit=0.8cm}\gammaptwo[P=9,linewidth=0.01mm]
\end{pspicture}&\begin{pspicture}(-2,-3)(2,3)
\psset{unit=0.7cm}\gammaptwo[P=10,linewidth=0.01mm]
\end{pspicture}&\begin{pspicture}(-2,-3)(2,3)
\psset{unit=0.7cm}\gammaptwo[P=11,linewidth=0.01mm]
\end{pspicture}\\
\hline
\end{array}
\]%
\newpage
\section{The infinite series $\beta_2^p$}
It is a special case of the series $\beta_n^p$ for $n=2$. In this
case, the polytopes are complex polygons. The projection used here
is different than the projection used with {\tt betapn}. Use the
command:
\begin{verbatim}
\betaptwo[P=p]
\end{verbatim}
\[\begin{array}{|c|c|c|}
\hline \beta_2^3&\beta_2^4&\beta_2^5\\
\hline \begin{pspicture}(-2,-3)(2,3)
\psset{unit=1.5cm}\betaptwo[P=3]
\end{pspicture}&\begin{pspicture}(-2,-3)(2,3)
\psset{unit=1.5cm}\betaptwo[P=4,linewidth=0.01mm]
\end{pspicture}&\begin{pspicture}(-2,-3)(2,3)
\psset{unit=1.5cm}\betaptwo[P=5,linewidth=0.01mm]
\end{pspicture}\\
\hline \beta_2^6&\beta_2^7&\beta_2^8\\ \hline
\begin{pspicture}(-2,-3)(2,3)
\psset{unit=1.5cm}\betaptwo[P=6,linewidth=0.01mm]
\end{pspicture}&\begin{pspicture}(-2,-3)(2,3)
\psset{unit=1.5cm}\betaptwo[P=7,linewidth=0.01mm]
\end{pspicture}&\begin{pspicture}(-2,-3)(2,3)
\psset{unit=1.5cm}\betaptwo[P=8,linewidth=0.01mm]
\end{pspicture}\\
\hline \beta_2^9&\beta_2^{10}&\beta_2^{11}\\
\hline \begin{pspicture}(-2,-3)(2,3)
\psset{unit=1.5cm}\betaptwo[P=9,linewidth=0.01mm]
\end{pspicture}&\begin{pspicture}(-2,-3)(2,3)
\psset{unit=1.5cm}\betaptwo[P=10,linewidth=0.01mm]
\end{pspicture}&\begin{pspicture}(-2,-3)(2,3)
\psset{unit=1.5cm}\betaptwo[P=11,linewidth=0.01mm]
\end{pspicture}\\
\hline
\end{array}
\]%
\begin{thebibliography}{ABC}
%
\bibitem{Cox1}
H. S. M. Coxeter, {\em Regular Complex Polytopes}, Second Edition,
Cambridge University Press, 1991 .
%
\end{thebibliography}
\end{document}

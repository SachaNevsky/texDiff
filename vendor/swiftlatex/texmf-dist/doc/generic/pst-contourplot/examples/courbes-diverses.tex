\documentclass[pstricks]{standalone}
\usepackage{pst-contourplot,pst-plot}%,multido
\begin{document}
\begin{pspicture}[showgrid](-6,-4)(6,4)
\psContourPlot[unit=0.5,algebraic,a=0.4,linecolor=blue,Fill=false,fillcolor=red,function=x*(x^2+y^2)-10*(x^2-y^2)](-6,-8)(12,8)
\psContourPlot[unit=0.5,algebraic,a=0.4,linecolor=blue,Fill=false,fillcolor=red,function=x*(x^2+y^2)-10*(x^2-y^2)-20](-6,-8)(12,8)
\psline{<->}(0,3.5)(0,0)(5.5,0)
\uput[d](0,0){$O$}
\uput[u](0,3.5){$y$}
\uput[r](5.5,0){$x$}
\end{pspicture}

\begin{pspicture}(-6,-4)(6,4)
\psframe*(-6,-4)(6,4)
\psContourPlot[unit=0.5,algebraic,a=0.4,linecolor=-red,Fill,fillcolor={[rgb]{0.5 0.5 1}},function=x*(x^2+y^2)-10*(x^2-y^2)-50](-6,-8)(12,8)
\psContourPlot[unit=0.5,algebraic,a=0.4,linecolor=-red,Fill,fillcolor=-blue,function=x*(x^2+y^2)-10*(x^2-y^2)-20](-6,-8)(12,8)
\psContourPlot[unit=0.5,algebraic,a=0.4,linecolor=-red,Fill,fillcolor=-green,function=x*(x^2+y^2)-10*(x^2-y^2)+10](-6,-8)(12,8)
\psgrid[subgriddiv=0,gridcolor=white,griddots=10,gridlabels=5pt]
\psline[linecolor=white]{<->}(0,3.5)(0,0)(5.5,0)
\uput[d](0,0){\white$O$}
\uput[u](0,3.5){\white$y$}
\uput[r](5.5,0){\white$x$}
\end{pspicture}

\begin{pspicture}(-3,-3)(3,4)
\psContourPlot[unit=2,a=0.02,linecolor=yellow,Fill,fillcolor=red,function=x dup mul y dup mul add 1 sub 3 exp
		       x dup mul y 3 exp mul sub](-2,-2)(2,2)
\psgrid[subgriddiv=0,gridcolor=black,griddots=10,gridlabels=5pt]
\psline{<->}(0,3.5)(0,0)(2.5,0)
\uput[d](0,0){$O$}
\uput[l](0,3.5){$y$}
\uput[d](2.5,0){$x$}
\end{pspicture}

\begin{pspicture}(-5,-5)(5,5)
\psset{unit=0.8333}%
% https://www.maplesoft.com/applications/view.aspx?sid=1582&view=html
\psContourPlot[algebraic,a=0.1,linecolor=red,Fill,fillcolor=yellow,ReverseColors,function=x*y*cos(x^2 + y^2)-1](-6,-6)(6,6)
\psaxes[labelFontSize=\scriptstyle]{->}(0,0)(-6,-6)(6,6)
\end{pspicture}

\begin{pspicture}(-5,-5)(5,5)
% https://www.maplesoft.com/applications/view.aspx?sid=1582&view=html
\psset{unit=0.5}%
\psContourPlot[algebraic,a=0.1,linecolor=red,function=sin(x + 2*sin(y))-cos(y + 3*cos(x))](-10,-10)(10,10)
\psgrid[subgriddiv=0,gridcolor=black,griddots=10,gridlabels=0pt](-10,-10)(10,10)
\psaxes[labelFontSize=\scriptstyle]{->}(0,0)(-10,-10)(10,10)%}
\end{pspicture}

\begin{pspicture}(-5,-5)(5,5)
% https://www.maplesoft.com/applications/view.aspx?sid=1582&view=html
\psframe*[linecolor=cyan](-5,-5)(5,5)
\psset{unit=0.5}%
\psContourPlot[algebraic,a=0.1,linecolor=red,Fill,fillcolor=yellow,ReverseColors,function=sin(x + 2*sin(y))-cos(y + 3*cos(x))](-10,-10)(10,10)
%\psgrid[subgriddiv=0,gridcolor=black,griddots=10,gridlabels=0pt](-10,-10)(10,10)
\psaxes[labelFontSize=\scriptstyle]{->}(0,0)(-10,-10)(10,10)%}
\end{pspicture}

\begin{pspicture}(-5,-5)(5,5)
\psframe*[linecolor=cyan](-5,-5)(5,5)
% https://www.maplesoft.com/applications/view.aspx?sid=1582&view=html
\psset{unit=0.5}%
\psContourPlot[algebraic,a=0.1,linecolor=blue,Fill,fillcolor=orange,ReverseColors,function=ln((x + 7*sin(y))^2)- EXP(y + 2*cos(x))](-10,-10)(10,10)
\psaxes[labelFontSize=\scriptstyle]{->}(0,0)(-10,-10)(10,10)%}
\end{pspicture}

\begin{pspicture}(-4,-4)(4.1,4.1)
\psframe*[linecolor=cyan](-4,-4)(4.1,4.1)
% Courbe d�duite de 8 droites
% page 124 : Revue du Palais de la D�couverte
% Courbes math�matiques
% Num�ro sp�cial 8 . Juillet 1976
\psContourPlot[algebraic,a=0.1,linecolor=blue,Fill,fillcolor=orange,ReverseColors,function=(x^4-5*x^2+4)*(y^4-5*y^2+4)+1](-4,-4)(4,4)
\psaxes[labelFontSize=\scriptstyle]{->}(0,0)(-4,-4)(4,4)
\end{pspicture}

\multido{\r=4+-0.05}{25}{%
\begin{pspicture}(-4,-4)(4,4)
\psframe*[linecolor=orange](-4,-4)(4,4)
\pstVerb{/rayon 1 def}%
\psContourPlot[unit=2,a=0.02,linecolor={[rgb]{0 0 0.5}},Fill,fillcolor=cyan,ReverseColors,
               function=
               1 x rayon 30 cos mul sub dup mul y rayon 30 sin mul add dup mul add div
               1 x rayon 30 cos mul add dup mul y rayon 30 sin mul add dup mul add div add
               1 x dup mul y rayon sub dup mul add div add
               \r\space sub](-4,-4)(4,4)
\psgrid[subgriddiv=0,gridcolor=black,griddots=10,gridlabels=0pt]
\end{pspicture}}
\end{document}


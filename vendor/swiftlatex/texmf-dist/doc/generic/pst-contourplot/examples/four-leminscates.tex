\documentclass[11pt]{article}
\usepackage[a4paper,margin=2cm]{geometry}
\usepackage[latin1]{inputenc}
\usepackage[T1]{fontenc}
\usepackage[garamond]{mathdesign}
\usepackage{pst-contourplot,pst-plot}
\title{Exemples avec pst-contourplot : \\ courbe d�duite de quatre lemniscates}
\date{27 mai 2018}
\author{manuel.luque27@gmail.com}
\begin{document}
\maketitle
Cette courbe est � la page 126 du num�ro sp�cial 8 (Juillet 1976) `\textit{Courbes math�matiques}' de la revue du Palais de la D�couverte.
% Courbe d�duite de quatre lemniscates
% page 126 : Revue du Palais de la D�couverte
% Courbes math�matiques
% Num�ro sp�cial 8 . Juillet 1976

Les �quations des  lemniscates sont :
\[
\left\{
\begin{array}[m]{l}
f_1(x,y)=\sqrt{[(a+x)^2+y^2][x^2+(a-y)^2]}-\frac{a^2}{2}\\[1em]
f_2(x,y)=\sqrt{[(a-x)^2+y^2][x^2+(a-y)^2]}-\frac{a^2}{2}\\[1em]
f_3(x,y)=\sqrt{[(a-x)^2+y^2][x^2+(a+y)^2]}-\frac{a^2}{2}\\[1em]
f_4(x,y)=\sqrt{[(a+x)^2+y^2][x^2+(a+y)^2]}-\frac{a^2}{2}\\[1em]
\end{array}
\right.
\]
Ils sont repr�sent�s ci-dessous :
%\def\lemniscateA{sqrt(((ai+x)^2+y^2)*(x^2+(ai-y)^2))-AI}
%\def\lemniscateB{sqrt(((ai-x)^2+y^2)*(x^2+(ai-y)^2))-AI}
%\def\lemniscateC{sqrt(((ai-x)^2+y^2)*(x^2+(ai+y)^2))-AI}
%\def\lemniscateD{sqrt(((ai+x)^2+y^2)*(x^2+(ai+y)^2))-AI}
\def\lemniscateA{ai x add dup mul y dup mul add 
                 x dup mul ai y sub dup mul add 
                 mul sqrt AI sub }
\def\lemniscateB{ai x sub dup mul y dup mul add
                 x dup mul ai y sub dup mul add
                 mul sqrt AI sub }
\def\lemniscateC{ai x sub dup mul y dup mul add
                 x dup mul ai y add dup mul add
                 mul sqrt AI sub }
\def\lemniscateD{ai x add dup mul y dup mul add
                 x dup mul ai y add dup mul add
                 mul sqrt AI sub }
\begin{center}
\begin{pspicture}(-4,-4)(4,4)
\pstVerb{/ai 2 def /AI ai dup mul 2 div def}%
\psContourPlot[a=0.1,linecolor=blue,function=\lemniscateA](-4,-4)(4,4)
\psContourPlot[,a=0.1,linecolor=red,function=\lemniscateB](-4,-4)(4,4)
\psContourPlot[a=0.1,linecolor=green,function=\lemniscateC](-4,-4)(4,4)
\psContourPlot[a=0.1,linecolor=cyan,function=\lemniscateD](-4,-4)(4,4)
\psaxes[labelFontSize=\scriptstyle]{->}(0,0)(-4,-4)(4,4)
\end{pspicture}
\end{center}

On repr�sente ensuite la courbe d�finie par :
\[
f_1(x,y)f_2(x,y)f_3(x,y)f_4(x,y)+K=0
\]
\newpage
Suivant les valeurs de $K$ on obtient :
\begin{center}
$K=0$

\begin{pspicture}(-4,-4)(4,4)
\pstVerb{/ai 2 def /AI ai dup mul 2 div def}%
\psContourPlot[a=0.04,linecolor=blue,Fill,fillcolor=orange,function=\lemniscateA \lemniscateB mul \lemniscateC mul \lemniscateD mul ](-4,-4)(4,4)
\psaxes[labelFontSize=\scriptstyle]{->}(0,0)(-4,-4)(4,4)
\end{pspicture}
\end{center}

\begin{center}
$K=-5$

\begin{pspicture}(-4,-4)(4,4)
\pstVerb{/ai 2 def /AI ai dup mul 2 div def}%
\psContourPlot[a=0.04,linecolor=blue,Fill,fillcolor=orange,function=\lemniscateA \lemniscateB mul \lemniscateC mul \lemniscateD mul 5 sub](-4,-4)(4,4)
\psaxes[labelFontSize=\scriptstyle]{->}(0,0)(-4,-4)(4,4)
\end{pspicture}
\end{center}
\newpage
\begin{center}
$K=5$

\begin{pspicture}(-4,-4)(4,4)
\pstVerb{/ai 2 def /AI ai dup mul 2 div def}%
\psContourPlot[a=0.04,linecolor=blue,Fill,fillcolor=orange,function=\lemniscateA \lemniscateB mul \lemniscateC mul \lemniscateD mul 5 add](-4,-4)(4,4)
\psaxes[labelFontSize=\scriptstyle]{->}(0,0)(-4,-4)(4,4)
\end{pspicture}
\end{center}
\end{document} 
%% $Id: pst-circ-doc.tex 918 2019-01-22 16:41:03Z herbert $
\documentclass[fontsize=11pt,english,BCOR=10mm,DIV=12,bibliography=totoc,parskip=false,
   headings=small, headinclude=false,footinclude=false,oneside,abstract=on]{pst-doc}
\usepackage{pst-circ}
\let\pstCircFV\fileversion
\usepackage{amsmath,siunitx}

\addbibresource{\jobname.bib}
\DeclareSIUnit\dbm{dBm} %% Définition du dBm
\lstset{explpreset={pos=l,wide=false,rframe=},language=PSTricks,
    morekeywords={multidipole,parallel},basicstyle=\footnotesize\ttfamily}
%
\newcommand\CircPackage{\LPack{pst-circ}}
\makeatletter
\renewenvironment{description}
  {\list{}{\labelwidth\z@ \itemindent-\leftmargin
    \itemsep0pt \parsep0pt
    \let\makelabel\descriptionlabel}}
  {\endlist}
\makeatother

\def\bgImage{\resizebox{0.75\linewidth}{!}{%
\begin{pspicture}(1,2)(19,9)
  \pnode(2,8){A}  \antenna{90}(A)
  \rput(4,8){\rnode{B}{\psframebox{\begin{tabular}{c}Ferrite\\Switch\end{tabular}}}}
  \ncline{A}{B}
  %%% Branche Calibration
  \pnode(4,6){C} \pnode(4,4){D} \pnode(5,5){E}
  \circulator[tripolestyle=isolator,tripoleconfig=right]{90}(C)(D)(E){Isolator}{}
  \ncline{B}{C}
  \pnode(3,3){F} \pnode(5,3){G}
  \resistor[unit=0.5,dipolestyle=zigzag,variable=true](F)(G){}
  \pnode(4,3){H}
  \ncline{D}{H}
  \rput[t](4,2.75){%
    \begin{tabular}{c}
        Hot and Cold\\
        loads for calibration
        \end{tabular}}
  %%% Branche reception
  \pnode(6,8){R1}  \pnode(8,8){R2}  \pnode(7,7){X1}
  \circulator[tripolestyle=isolator,tripoleconfig=right]{180}(R1)(R2)(X1){Isolator}{}
  \ncline{B}{R1}
  \pnode(10,8){R3}  \pnode(9,7){X2}
  \mixer[inputarrow,fillcolor=blue,fillstyle=solid, labeloffset=0.8](R2)(R3)(X2){Mixer}{}
  \pnode(9,6){X3}
  \oscillator[output=top](X3){LO}{}
  \pnode(12,8){R4}
  \ncline{R3}{R4}
  \naput{0.5~GHZ}
  \pnode(14,8){R5}
  \filter[labeloffset=0.8](R4)(R5){BPF}%
  \pnode(16,8){R6}
  \amplifier[inputarrow=true,fillcolor=red,fillstyle=solid, labeloffset=0.8](R5)(R6){IF~Amp}
  \pnode(18,8){R7}
  \detector[inputarrow=true, labeloffset=0.8](R6)(R7){Detector}
  \pnode(18,4){R8}
  \amplifier[inputarrow=true,labeloffset=-1](R7)(R8){Amp}
  \pscircle[fillstyle=solid,fillcolor=white](18,4){0.1}
  \rput[t](18,3.9){%
    \begin{tabular}{c}
        Output\\
        for processing
    \end{tabular}}
\end{pspicture}}
}
\lstset{preset={\centering},vsep=5mm}


\begin{document}

\title{\texttt{pst-circ}}
\subtitle{A PSTricks package for drawing electric circuits; v.\pstCircFV}
\author{%Christoph Jorssen \\
Herbert Vo\ss}
%\docauthor{Herbert Vo\ss}
\date{\today}
\maketitle

\tableofcontents

\clearpage

\begin{abstract}
\noindent
The package \LPack{pst-circ} is a collection of graphical elements based
on PStricks that can be used to facilitate display of electronic
circuit elements. For example, an equivalent circuit of a voltage
source, its source impedance, and a connected load can easily be
constructed along with arrows indicating current flow and potential
differences. The emphasis is upon the circuit elements and the
details of the exact placement are hidden as much as possible so the
author can focus on the circuitry without the distraction of sorting
out the underlying vector graphics.

\LPack{pst-circ} loads by default the following packages: \LPack{pst-node}, 
\LPack{multido}, \LPack{pst-xkey}, and, of course \LPack{pstricks}.
All should be already part of your local \TeX\ installation. If not, or in case
of having older versions, go to \url{http://www.CTAN.org/} and load the newest version.

\vfill\noindent
\raggedright
\begin{sloppypar}
Thanks to: \\
\mbox{Rafal Bartczuk},
\mbox{Christoph Bersch}, 
\mbox{Fran\c{c}ois Boone}, 
\mbox{Vincent Breton},
    \mbox{Jean-C\^ome Charpentier},
    \mbox{Patrick Drechsler},
    \mbox{Amit Finkler},
    \mbox{Felix Gottwald},
    \mbox{Markus Graube},
    \mbox{Henning Heinze},
    \mbox{Christophe Jorssen},
    \mbox{Bernd Landwehr},
    \mbox{Michael Lauterbach},
    \mbox{Manuel Luque}, 
    \mbox{Steven P. McPherson}, 
    \mbox{Patrice Mégret},
    \mbox{Ted Pavlic}, 
    \mbox{Alan Ristow},
    \mbox{Uwe Siart},
    \mbox{Carlos Marcelo de Oliveira Stein}, 
    \mbox{Pierre Vivegnis},
\mbox{Douglas Waud}, 
\mbox{Richard Weissnar}, and
\mbox{Felix Wienker}.
\end{sloppypar}
\end{abstract}

\clearpage

\section{The basic system}

\subsection{Parameters}

There are specific parameters defined to change easily the behaviour of the \LPack{pst-circ}
objects you are drawing. You'll find a list in Section~\ref{sec:para} on p.~\pageref{sec:para}.

\iffalse
\begin{longtable}{@{}>{\ttfamily}l l l@{}}
\textrm{\emph{name}} & \emph{type} & \emph{default}\\\hline
\endhead
\Lkeyword{intensity} & boolean &  \emph{false} \\
\Lkeyword{intensitylabel} & string &  \emph{ } \\
\Lkeyword{intensitylabeloffset} & dimension &  \emph{ 0.5} \\
\Lkeyword{intensitycolor} & color &  \emph{ black} \\
\Lkeyword{intensitylabelcolor} & color &  \emph{ black} \\
\Lkeyword{intensitywidth} & dimension &  \emph{ \texttt{\Lcs{pslinewidth}}} \\
\Lkeyword{tension} & boolean &  \emph{ false} \\
\Lkeyword{tensionstyle} & string&  \emph{line} \\
\Lkeyword{tensionlabel} & string &  \emph{ } \\
\Lkeyword{tensionoffset} & dimension &  \emph{ 1} \\
\Lkeyword{tensionlabeloffset} & dimension &  \emph{ 1.2} \\
\Lkeyword{tensioncolor} & color &  \emph{ black} \\
\Lkeyword{tensionlabelcolor} & color &  \emph{ black} \\
\Lkeyword{tensionwidth} & dimension &  \emph{ \texttt{\Lcs{pslinewidth}}} \\
\Lkeyword{labeloffset} & dimension &  \emph{ 0.7} \\
\Lkeyword{labelangle} & label angle &  \emph{ 0} \\
\Lkeyword{labelInside} & integer &  \emph{ 0} \\
\Lkeyword{dipoleconvention} & & \emph{ receptor} \\
\Lkeyword{directconvetion} & boolean &  \emph{ true} \\
\Lkeyword{dipolestyle} & string &  \emph{ normal} \\
\Lkeyword{variable} & boolean &  \emph{ false} \\
\Lkeyword{parallel} & boolean &  \emph{ false} \\
\Lkeyword{parallelarm} & dimension &  \emph{ 1.5} \\
\Lkeyword{parallelsep} & real &  \emph{ 0} \\
\Lkeyword{parallelnode} & boolean &  \emph{ false} \\
\Lkeyword{intersect} & boolean &  \emph{ false} \\
\Lkeyword{intersectA} & node & \\
\Lkeyword{intersectB} & node & \\
\Lkeyword{OAinvert} & boolean &  \emph{ true} \\
\Lkeyword{OAperfect} & boolean &  \emph{ true} \\
\Lkeyword{OAiplus} & boolean &  \emph{ false} \\
\Lkeyword{OAiminus} & boolean &  \emph{ false} \\
\Lkeyword{OAiout} & boolean &  \emph{ false} \\
\Lkeyword{OAipluslabel} & string &  \emph{ } \\
\Lkeyword{OAiminuslabel} & string &  \emph{ } \\
\Lkeyword{OAioutlabel} & string &  \emph{ } \\
\Lkeyword{transistorcircle} & boolean &  \emph{ true} \\
\Lkeyword{transistorinvert} & boolean &  \emph{ false} \\
\Lkeyword{transistoribase} & boolean &  \emph{ false} \\
\Lkeyword{transistoricollector} & boolean &  \emph{ false} \\
\Lkeyword{transistoriemitter} & boolean &  \emph{ false} \\
\Lkeyword{transistoribaselabel} & string &  \emph{ } \\
\Lkeyword{transistoricollectorlabel} & string &  \emph{ } \\
\Lkeyword{transistoriemitterlabel} & string &  \emph{ } \\
\Lkeyword{TRot} & angle &  \emph{ 0} \\
\Lkeyword{edge} & macro &  \emph{ \texttt{\textbackslash ncangles}} \\
\Lkeyword{transistortype} & string &  \emph{ NPN} \\
\Lkeyword{FETchanneltype} & string &  \emph{ N} \\
\Lkeyword{FETmemory} & boolean &  \emph{ false} \\
\Lkeyword{primarylabel} & string &  \emph{ } \\
\Lkeyword{secondarylabel} & string &  \emph{ } \\
\Lkeyword{transformeriprimary} & boolean &  \emph{ false} \\
\Lkeyword{transformerisecondary} & boolean &  \emph{ false} \\
\Lkeyword{transformeriprimarylabel} & string &  \emph{ } \\
\Lkeyword{transformerisecondarylabel} & string &  \emph{ } \\
\Lkeyword{tripolestyle} & string &  \emph{ normal}
\end{longtable}
\fi

\subsection{Macros}

\subsubsection{Dipole macros}
\xLcs{resistor}
\begin{LTXexample}[width=3.5cm]
\begin{pspicture}[showgrid=true](3,2)
  \pnodes(0,1){A}(3,1){B}
  \resistor(A)(B){$R$}
\end{pspicture}
\end{LTXexample}

\xLcs{RFLine}
\begin{LTXexample}[width=3.5cm]
\begin{pspicture}[showgrid=true](3,2)
  \pnodes(0,1){A}(3,1){B}
  \RFLine(A)(B){R}
\end{pspicture}
\end{LTXexample}

\xLcs{capacitor}
\begin{LTXexample}[width=3.5cm]
\begin{pspicture}[showgrid=true](3,2)
  \pnodes(0,1){A}(3,1){B}
  \capacitor(A)(B){$C$}
\end{pspicture}
\end{LTXexample}

\xLcs{battery}
\begin{LTXexample}[width=3.5cm]
\begin{pspicture}(3,2)
  \pnodes(0,1){A}(3,1){B}
  \battery(A)(B){$E$}
\end{pspicture}
\end{LTXexample}

\xLcs{coil}
\begin{LTXexample}[width=3.5cm]
\begin{pspicture}(3,2)
  \pnodes(0,1){A}(3,1){B}
  \coil(A)(B){$L$}
\end{pspicture}
\end{LTXexample}

\xLcs{Ucc}
\begin{LTXexample}[width=3.5cm]
\begin{pspicture}(3,2)
 \pnodes(0,1){A}(3,1){B}
 \Ucc[dipolestyle=normal](A)(B){$E$}
\end{pspicture}
\end{LTXexample}

\begin{LTXexample}[width=3.5cm]
\begin{pspicture}(3,2)
 \pnodes(0,1){A}(3,1){B}
 \Ucc[dipolestyle=diamond](A)(B){$E$}
\end{pspicture}
\end{LTXexample}

\begin{LTXexample}[width=3.5cm]
\begin{pspicture}(3,2)
 \pnodes(0,1){A}(3,1){B}
 \Ucc[dipolestyle=normalCei](A)(B){$E$}
\end{pspicture}
\end{LTXexample}

\begin{LTXexample}[width=3.5cm]
\begin{pspicture}(3,2)
 \pnodes(0,1){A}(3,1){B}
 \Ucc[dipolestyle=diamondCei](A)(B){$E$}
\end{pspicture}
\end{LTXexample}


\xLcs{Icc}
\begin{LTXexample}[width=3.5cm]
\begin{pspicture}(3,2)
 \pnodes(0,1){A}(3,1){B}
 \Icc[dipolestyle=normal](A)(B){$\eta$}
\end{pspicture}
\end{LTXexample}

\begin{LTXexample}[width=3.5cm]
\begin{pspicture}(3,2)
 \pnodes(0,1){A}(3,1){B}
 \Icc[dipolestyle=twoCircles](A)(B){$\eta$}
\end{pspicture}
\end{LTXexample}

\begin{LTXexample}[width=3.5cm]
\begin{pspicture}(3,2)
 \pnodes(0,1){A}(3,1){B}
\Icc[dipolestyle=diamond](A)(B){$\eta$}
\end{pspicture}
\end{LTXexample}







\xLcs{switch}
\begin{LTXexample}[width=3.5cm]
\begin{pspicture}(3,2)
  \pnodes(0,1){A}(3,1){B}
  \switch(A)(B){$K$}
\end{pspicture}
\end{LTXexample}

\begin{LTXexample}[width=3.5cm]
\begin{pspicture}(3,2)
  \pnodes(0,1){A}(3,1){B}
  \switch[dipolestyle=close](A)(B){$K$}
\end{pspicture}
\end{LTXexample}

\xLcs{arrowswitch}
\begin{LTXexample}[width=3.5cm]
\begin{pspicture}(3,2)
  \pnodes(0,1){A}(3,1){B}
  \arrowswitch(A)(B){$K$}
\end{pspicture}
\end{LTXexample}

\begin{LTXexample}[width=3.5cm]
\begin{pspicture}(3,2)
  \pnodes(0,1){A}(3,1){B}
  \arrowswitch[dipolestyle=close](A)(B){$K$}
\end{pspicture}
\end{LTXexample}



\xLcs{diode}
\begin{LTXexample}[width=3.5cm]
\begin{pspicture}(3,2)
  \pnodes(0,1){A}(3,1){B}
  \diode(A)(B){$D$}
\end{pspicture}
\end{LTXexample}

\xLcs{Zener}
\begin{LTXexample}[width=3.5cm]
\begin{pspicture}(3,2)
  \pnodes(0,1){A}(3,1){B}
  \Zener(A)(B){$D$}
\end{pspicture}
\end{LTXexample}

\xLcs{lamp}
\begin{LTXexample}[width=3.5cm]
\begin{pspicture}(3,2)
  \pnodes(0,1){A}(3,1){B}
  \lamp(A)(B){$\mathcal L$}
\end{pspicture}
\end{LTXexample}

\xLcs{circledipole}
\begin{LTXexample}[width=3.5cm]
\begin{pspicture}(3,2)
  \pnodes(0,1){A}(3,1){B}
  \circledipole(A)(B){$\mathcal G$}
\end{pspicture}
\end{LTXexample}

\xLkeyword{labeloffset}
\begin{LTXexample}[width=3.5cm]
\begin{pspicture}(3,2)
  \pnodes(0,1){A}(3,1){B}
  \circledipole[labeloffset=0](A)(B){\Large\textbf{A}}
\end{pspicture}
\end{LTXexample}

\xLcs{LED}
\begin{LTXexample}[width=3.5cm]
\begin{pspicture}(3,2)
  \pnodes(0,1){A}(3,1){B}
  \LED(A)(B){$\mathcal D$}
\end{pspicture}
\end{LTXexample}

\begin{LTXexample}[width=3.5cm]
\begin{pspicture}(3,2)
  \pnodes(0,1){A}(3,1){B}
  \SQUID(A)(B){S}
\end{pspicture}
\end{LTXexample}

\xLcs{RelayNOP}
\begin{LTXexample}[width=3.5cm]
\begin{pspicture}(3,3)
  \pnodes(0,0){A}(3,0){B}%Relay normally open
  \RelayNOP[labeloffset=1.6](A)(B){RelayNOP}
\end{pspicture}
\end{LTXexample}

\xLcs{Suppressor}
\begin{LTXexample}[width=3.5cm]
\begin{pspicture}(3,2)
  \pnodes(0,1){A}(3,1){B}% Suppressor (Diode)
  \Suppressor[labeloffset=0.5](A)(B){Supressor}
\end{pspicture}
\end{LTXexample}

\xLcs{Arrestor}
\begin{LTXexample}[width=3.5cm]
\begin{pspicture}(3,2)
  \pnodes(0,1){A}(3,1){B}
  % Arrestor (Lightning protection)
  \Arrestor(A)(B){Arrestor}
\end{pspicture}
\end{LTXexample}

\begin{LTXexample}[width=3.5cm,rframe={}]
\begin{pspicture}(3,2)
  \pnode(0,1){A}  \pnode(3,1){B}
  \cell[labeloffset=1cm](A)(B){Cell}
\end{pspicture}
\end{LTXexample}

\begin{LTXexample}[width=3.5cm,rframe={}]
\begin{pspicture}(3,2)
  \pnode(0,1){A}  \pnode(3,1){B}
  \igbt[labeloffset=0.7cm, IGBTinvert=false](A)(B){IGBT}
\end{pspicture}
\end{LTXexample}


\bigskip
\subsubsection{Tripole macros}

Obviously, \Index{tripole}s are not node connections. So \LPack{pst-circ} tries its best to adjust the
position of the tripole regarding the three nodes. Internally, the connections are done by the
\Lcs{ncangle} pst-node macro. However, the auto-positionning and the auto-connections are not always
well chosen, so don't try to use tripole macros in strange situations!


\xLcs{OA}
\begin{LTXexample}[width=5.5cm]
\begin{pspicture}(5,2)
  \pnodes(0,0){A}(0,2){B}(5,1){C}
  \OA(B)(A)(C)
\end{pspicture}
\end{LTXexample}

\xLkeyword{OApower}
\begin{LTXexample}[width=5.5cm]
\begin{pspicture}(5,2)
  \pnodes(0,0){A}(0,2){B}(5,1){C}
  \OA[OApower=true](B)(A)(C)
\end{pspicture}
\end{LTXexample}

\xLcs{GM}
\begin{LTXexample}[width=5.5cm]
\begin{pspicture}(5,2)
  \pnodes(0,0){A}(0,2){B}(5,1){C}
  \GM[GMperfect=true](B)(A)(C)
\end{pspicture}
\end{LTXexample}

\xLkeyword{GMpower}
\begin{LTXexample}[width=5.5cm]
\begin{pspicture}(5,2)
  \pnodes(0,0){A}(0,2){B}(5,1){C}
  \GM[GMpower=true](B)(A)(C)
\end{pspicture}
\end{LTXexample}

There are 5 types of transistors included : NPN, PNP, FET, NMOS and PMOS. It's the macro \Lcs{transistortype} that determines which transistor will be drawn.

\xLcs{transistor}
\begin{LTXexample}[width=5.5cm]
\begin{pspicture}(3,4)
\pnodes(0,2){A}(3,1){B}(3,3){C}
\transistor(A)(B)(C)
\end{pspicture}
\end{LTXexample}

\xLcs{transistorFET}
\begin{LTXexample}[width=5.5cm]
\begin{pspicture}(3,4)
\pnodes(0,2){A}(3,1){B}(3,3){C}
\transistor[basesep=1cm, transistortype=FET](A)(B)(C)
\end{pspicture}
\end{LTXexample}

\xLcs{transistorNMOS}
\begin{LTXexample}[width=5.5cm]
\begin{pspicture}(3,4)
\pnodes(0,2){A}(3,1){B}(3,3){C}
\transistor[basesep=1cm, transistortype=NMOS, transistorcircle=false](A)(B)(C)
\end{pspicture}
\end{LTXexample}

\xLcs{transistor}\xLkeyword{TRot}
\begin{LTXexample}[width=5.5cm]
\begin{pspicture}[showgrid](3,4)
\pnodes(3,2){A}(0,1){B}(0,3){C}
\transistor[TRot=180](A)(B)(C)
\end{pspicture}
\end{LTXexample}




\xLkeyword{TRot}
\begin{LTXexample}[width=5.5cm]
\begin{pspicture}[showgrid=true](5,5)
\pnode(1,3){b}
\transistor[TRot=90](b){emitter}{collector}
\transistor[TRot=45](4,4){emitter}{collector}
\transistor[TRot=180](1,1){emitter}{collector}
\transistor[TRot=180,transistorinvert=true]%
  (4,1){emitter}{collector}
\end{pspicture}
\end{LTXexample}

\xLkeyset{transistortype=PNP}
\begin{LTXexample}[width=5.5cm]
\begin{pspicture}(3,4)
\pnodes(0,2){A}(3,1){B}(3,3){C}
\transistor[transistortype=PNP](A)(B)(C)
\end{pspicture}
\end{LTXexample}

\xLkeyword{basesep}\xLkeyword{arrows}
\begin{LTXexample}[width=5.5cm]
\begin{pspicture}(5,3)
  \pnodes(0,1.5){A}(5,0){B}(5,3){C}
  \transistor[basesep=2cm,arrows=o-o](A)(B)(C)
\end{pspicture}
\end{LTXexample}

\xLkeyword{basesep}\xLkeyword{transistoriemitter}
\begin{LTXexample}[width=5.5cm]
\begin{pspicture}(3,4)
\pnode(0,2){A}\pnode(3,0.5){B}
\pnode(3,3.5){C}
\transistor[transistoriemitter=true,
  basesep=1cm](A)(B)(C)
\end{pspicture}
\end{LTXexample}

\xLkeyword{basesep}\xLkeyword{transistorinvert}
\begin{LTXexample}[width=5.5cm]
\begin{pspicture}(3,4)
\pnode(0,2){A}\pnode(3,1){B}
\pnode(3,3){C}
\transistor[transistorinvert,
  basesep=1cm](A)(B)(C)
\end{pspicture}
\end{LTXexample}

\xLkeyset{transistortype=PNP}
\xLkeyword{basesep}\xLnotation{Emitter}\xLkeyword{nodesep}\xLnotation{Collector}
\begin{LTXexample}[width=5.5cm]
\begin{pspicture}(5,3)
  \pnode(0,1.5){A}\psset{linewidth=1pt}
  \transistor[transistortype=PNP,basesep=2cm,
     arrows=o-o](A){Emitter}{Collector}
  \psline{o-}(5,3)(3,3)(3,3|Collector)(Collector)
  \psline{o-}(5,0)(3,0)(3,3|Emitter)(Emitter)
  \psline{o-}(A)([nodesep=2]A)
\end{pspicture}
\end{LTXexample}

\xLcs{Tswitch}
\begin{LTXexample}[width=5.5cm]
\begin{pspicture}(5,2)
  \pnodes(0,2){A}(5,2){B}(0,0){C}
  \Tswitch(A)(B)(C){$K$}
\end{pspicture}
\end{LTXexample}

\xLcs{potentiometer}
\begin{LTXexample}[width=5.5cm]
\begin{pspicture}(3,3)
  \pnodes(0,1){A}(3,1){B}(3,2.25){C}
  \potentiometer[labeloffset=0pt](A)(B)(C){$P$}
\end{pspicture}
\end{LTXexample}

\xLcs{transistorFET}
\begin{LTXexample}[width=5.5cm]
\begin{pspicture}(3,4)
\pnodes(0,2){A}(3,1){B}(3,3){C}
\transistor[basesep=1cm, transistortype=FET, FETchanneltype=P](A)(B)(C)
\end{pspicture}
\end{LTXexample}


\begin{LTXexample}[pos=b]
\psset{mathlabel}
\def\pcTran(#1)(#2){\psline(#1)(#2|#1)(#2)}% only 2 segements
\psset{circedge=\pcTran,connectingdot=false}

\begin{pspicture}(10,10)
\pnodes(1,1){G1}(6,1){G2}(7.5,1){G3}
\newground[arrows=o](G1)\newground(G2)\newground(G3)
\pnodes(1,3){D1u}(7,3){T1B}(0,3){IB}(4,4){T2B}
\newdiode(G1)(D1u){D1}\qdisk(D1u){2pt}
\transistor[TRot=270,arrows=-o](T2B)(IB)(T1B)
\pnode(8,7){O1}%junction to out
\transistor(T1B)(G3)(O1)
\pnodes(1,6){D2u}(1,4){G4}
\newground(G4) 
\newdiode(G4)(D2u){D2}\qdisk(D2u){2pt}
\pnodes(2.5,7){T4B}(0,6){IA}(5.5,5){T3B}(6,7){R3d}
\transistor[TRot=270,arrows=-o](T4B)(IA)(T3B)\uput[90](IA){$\mathtt{A}$}
\transistor(T3B)(G2)(R3d)\uput[90](IB){$\mathtt{B}$}
\pnodes(2.5,10){VCC1}(4,10){VCC2}(6,10){VCC3}
\resistor[arrows=o-](VCC1)(T4B){4{,}7k\Omega}
\resistor[arrows=o-](VCC2)(T2B){4{,}7k\Omega}
\resistor[arrows=o-](VCC3)(R3d){100\Omega}
\wire[arrows=*-o](R3d)(O1)
\uput[90](O1){$\mathtt{OUT}$} \qdisk(7.5,7){2pt}
\end{pspicture}
\end{LTXexample}



\clearpage
\subsubsection{Quadrupole macros}

\xLcs{quadripole}
\begin{LTXexample}[width=5.5cm]
\begin{pspicture}(5,3)
  \pnodes(0,2.5){A}(0,0.5){B}%
         (4,2.5){C}(4,0.5){D}
  \quadripole(A)(B)(C)(D){Text}
\end{pspicture}
\end{LTXexample}

\xLcs{transformer}
\begin{LTXexample}[width=5.5cm]
\begin{pspicture}(5,5)
  \pnodes(0,5){A}(0,0){B}%
         (5,5){C}(5,0){D}
  \transformer(A)(B)(C)(D){$\mathcal T$}
\end{pspicture}
\end{LTXexample}


\xLcs{newtransformer}
\begin{LTXexample}[width=5.5cm]
\begin{pspicture}(3,2)
 \pnodes(0,1){A}(3,1){B}
 \newtransformer(A)(B){$\mathcal T$}
\end{pspicture}
\end{LTXexample}


\xLcs{newtransformerquad}
\begin{LTXexample}[width=5.5cm]
\begin{pspicture}(5,3)
 \pnodes(0,3){A}(0,0){B}%
        (5,3){C}(5,0){D}
 \newtransformerquad(A)(B)(C)(D)%
   {$\mathcal T$}
\end{pspicture}
\end{LTXexample}

\xLcs{optoCoupler}
\begin{LTXexample}[width=5.5cm]
\begin{pspicture}(5,3)
  \pnodes(0,2.5){A}(0,0.5){B}%
         (4,2.5){C}(4,0.5){D}
  \optoCoupler(A)(B)(C)(D){$OC$}
\end{pspicture}
\end{LTXexample}




\subsubsection{Multidipole}

\Lcs{multidipole} is a macro that allows multiple dipoles to be drawn between two specified nodes.
\Lcs{multidipole} takes as many arguments as you want. Note the \rnode{Dot}{dot} that is
after the last dipole.

\bigskip
\xLcs{diode}\xLcs{resistor}\xLcs{caoacitor}
\begin{minipage}{7cm}
\begin{pspicture}(7,7)
  \pnodes(0,0){A}(7,7){B}
  \multidipole(A)(B)\resistor{$R$}%
    \capacitor[linecolor=red]{$C$}%
    \diode{$D$}{}\rnode{Dot2}{}.
\end{pspicture}
\end{minipage}\hfill
\begin{minipage}{6cm}
\verb+\begin{pspicture}(7,7)+\\
\verb+  \pnodes(0,0){A}(7,7){B}+\\
\verb+  \multidipole(A)(B)\resistor{$R$}%+\\
\verb+    \capacitor[linecolor=red]{$C$}%+\\
\verb+    \diode{$D$}{}+\rnode{Dot2}{}.\\
\verb+\end{pspicture}+
\end{minipage}


\bigskip
\ncangles[linestyle=dashed,linecolor=gray,nodesep=3pt,armA=.5cm,angleA=-90,armB=4cm,angleB=0]{->}{Dot}{Dot2} 
Important: for the time being, \Lcs{multidipole} takes optional arguments but does not 
restore original values. We recommand not using it.


\bigskip
\subsubsection{Wire}

\xLcs{wire}
\begin{LTXexample}[width=3.5cm]
\begin{pspicture}(3,2)
 \pnodes(0,1){A}(3,1){B}\wire(A)(B)
 \pnodes(0,0){A}(3,0){B}\wire[arrows=o-*](A)(B)
\end{pspicture}
\end{LTXexample}

\bigskip
\subsubsection{Potential}

\xLcs{tension}
\begin{LTXexample}[width=3.5cm]
\begin{pspicture}(3,2)
  \pnodes(0,1){A}(3,1){B}
  \tension(A)(B){$u$}
\end{pspicture}
\end{LTXexample}

\bigskip
\subsubsection{ground}

\xLcs{ground}
\begin{LTXexample}[width=3.5cm]
\begin{pspicture}(3,2)
  \pnodes(0.5,1){A}(1,1){B}(2.5,1){C}
  \ground(A)
  \ground{135}(B)
  \ground[linecolor=blue]{180}(C)
\end{pspicture}
\end{LTXexample}

\bigskip
\subsubsection{Open dipol and open tripol}

\xLcs{OpenDipol}\xLcs{OpenTripol}
\begin{LTXexample}[width=4.5cm]
\def\Wave{\psscalebox{3}{$\approx$}}
\def\PM{\psscalebox{2}{$+\,\,-$}}
\begin{pspicture}(4,3)
\pnodes(0,0){A}(2,3){B}(4,3){C}(4,0){D}
\OpenDipol[radius=3pt,labelangle=:U,
  labeloffset=-0.5](A)(B){\Wave}
\OpenDipol[radius=3pt,labelangle=:U](B)(C){\PM}
\OpenTripol(A)(D){}
\end{pspicture}
\end{LTXexample}


\bigskip
\subsection{Parameters}

\subsubsection{Label parameters}


\xLcs{resitor}\xLkeyword{labeloffset}
\begin{LTXexample}[width=3.5cm]
\begin{pspicture}(3,1)
  \pnodes(0,.5){A}(3,.5){B}
  \resistor[labeloffset=0](A)(B){$R$}
\end{pspicture}
\end{LTXexample}

\xLcs{resitor}\xLkeyset{labelangle=:U}
\begin{LTXexample}[width=3.5cm]
\begin{pspicture}(3,2)
  \pnodes(0,0){A}(3,2){B}
  \resistor[labelangle=:U](A)(B){$R$}
\end{pspicture}
\end{LTXexample}

\begin{LTXexample}[width=3.5cm]
\begin{pspicture}(3,2)
  \pnodes(0,0){A}(3,2){B}
  \resistor[labelangle=0](A)(B){$R$}
\end{pspicture}
\end{LTXexample}

\xLcs{transformer}\xLkeyword{primarylabel}\xLkeyword{secondarylabel}
\begin{LTXexample}[width=5.5cm]
\begin{pspicture}(5,5)
  \pnodes(0,5){A}(0,0){B}(5,5){C}(5,0){D}
  \transformer[primarylabel=$n_1$,
    secondarylabel=$n_2$](A)(B)(C)(D){$\mathcal T$}
\end{pspicture}
\end{LTXexample}

\xLkeyword{labelInside}
\begin{LTXexample}[width=3.5cm]
\begin{pspicture}(3,4.5)
  \pnodes(0,.5){A}(3,.5){B}
  \Ucc[labelInside=1](A)(B){$V$}
  \pnodes(0,2){A}(3,2){B}
  \Ucc[labelInside=2](A)(B){$V$}
  \pnodes(0,3.5){A}(3,3.5){B}
  \Ucc[labelInside=3](A)(B){$V$}
\end{pspicture}
\end{LTXexample}

\bigskip
\subsubsection{Current intensity and electrical potential parameters}

If the \Lkeyword{intensity} parameter is set to \true, an arrow is drawn on the wire
connecting one of the nodes to the dipole. If the \Lkeyword{tension} parameter is set to \true,
an arrow is drawn parallel to the dipole.

The way those \Index{arrows} are drawn is set by \Lkeyword{dipoleconvention} and \Lkeyword{directconvention}
parameters. \Lkeyword{dipoleconvention} can take two values~: \Lkeyval{generator} or \Lkeyval{receptor}.
\Lkeyword{directconvention} is a boolean.


\xLkeyword{intensity}\xLkeyword{tension}
\begin{LTXexample}[width=3.5cm]
\begin{pspicture}(3,2)
  \pnodes(0,.5){A}(3,.5){B}
  \resistor[intensity,tension](A)(B){}
\end{pspicture}
\end{LTXexample}

\xLkeyword{intensity}\xLkeyword{tensionstyle}
\begin{LTXexample}[width=3.5cm]
\begin{pspicture}(3,2)
  \pnodes(0,.5){A}(3,.5){B}
  \resistor[intensity,tension,tensionstyle=pm](A)(B){}
\end{pspicture}
\end{LTXexample}

\xLkeyword{intensity}\xLkeyword{tension}\xLkeyset{dipoleconvention=generator}
\begin{LTXexample}[width=3.5cm]
\begin{pspicture}(3,2)
  \pnodes(0,.5){A}(3,.5){B}
  \resistor[intensity,tension,dipoleconvention=generator](A)(B){}
\end{pspicture}
\end{LTXexample}

\xLkeyword{intensity}\xLkeyword{tension}\xLkeyword{directconvention}
\begin{LTXexample}[width=3.5cm]
\begin{pspicture}(3,2)
  \pnodes(0,.5){A}(3,.5){B}
  \resistor[intensity,tension,directconvention=false](A)(B){}
\end{pspicture}
\end{LTXexample}

\xLkeyword{intensity}\xLkeyword{tension}\xLkeyset{dipoleconvention=generator}\xLkeyword{directconvention}
\begin{LTXexample}[width=3.5cm]
\begin{pspicture}(3,2)
  \pnodes(0,.5){A}(3,.5){B}
  \resistor[intensity,tension,
    dipoleconvention=generator,directconvention=false](A)(B){}
\end{pspicture}
\end{LTXexample}

If \Lkeyword{intensitylabel} is set to an non empty argument, then \Lkeyword{intensity} is automatically
set to true. If \Lkeyword{tensionlabel} is set to an non empty argument, then \Lkeyword{tension} is
automatically set to true.

\begin{LTXexample}[width=3.5cm]
\begin{pspicture}(3,2)
  \pnodes(0,.5){A}(3,.5){B}
  \resistor[intensitylabel=$i$,tensionlabel=$u$](A)(B){}
\end{pspicture}
\end{LTXexample}

\xLkeyword{intensitylabel}\xLkeyword{intensitylabeloffset}
\xLkeyword{tensionlabel}\xLkeyword{tensionoffset}\xLkeyword{tensionlabeloffset}
\begin{LTXexample}[width=3.5cm]
\begin{pspicture}(3,2)
  \pnodes(0,1.5){A}(3,1.5){B}
  \resistor[intensitylabel=$i$,intensitylabeloffset=-0.5,
    tensionlabel=$u$,tensionlabeloffset=-1.2,
    tensionoffset=-1](A)(B){}
\end{pspicture}
\end{LTXexample}

\xLkeyword{intensitycolor}\xLkeyword{intensitywidth}\xLkeyword{intensitylabel}\xLkeyword{intensitylabelcolor}
\xLkeyword{tensionwidth}\xLkeyword{tensionlabel}\xLkeyword{tensioncolor}\xLkeyword{tensionlabelcolor}
\begin{LTXexample}[width=3.5cm]
\begin{pspicture}(3,2)
  \pnodes(0,.5){A}(3,.5){B}
  \resistor[intensitylabel=$i$,intensitywidth=3\pslinewidth,
    intensitycolor=red,intensitylabelcolor=yellow,
    tensionlabel=$u$,tensionwidth=2\pslinewidth,
    tensioncolor=green,tensionlabelcolor=blue](A)(B){}
\end{pspicture}
\end{LTXexample}

Some specific intensity parameters are available for \Index{tripole}s and \Index{quadrupole}s.

\xLkeyword{OAiminuslabel}\xLkeyword{OAipluslabel}\xLkeyword{OAioutlabel}
\begin{LTXexample}[width=5.5cm]
\begin{pspicture}(5,3)
  \pnodes(0,0){A}(0,3){B}(5,1.5){C}
  \OA[OAipluslabel=$i_+$,
    OAiminuslabel=$i_-$,
    OAioutlabel=$i_o$](B)(A)(C)
\end{pspicture}
\end{LTXexample}

\xLkeyword{basesep}\xLkeyword{transistoribaselabel}\xLkeyword{transistoricollectorlabel}\xLkeyword{transistoriemitterlabel}
\begin{LTXexample}[width=5.5cm]
\begin{pspicture}(5,3)
  \pnodes(0,1.5){A}(5,0){B}(5,3){C}
  \transistor[basesep=2cm,transistoribaselabel=$i_B$,
    transistoricollectorlabel=$i_C$,
    transistoriemitterlabel=$i_E$](A)(B)(C)
\end{pspicture}
\end{LTXexample}

\xLkeyword{transformerisecondarylabel}
\xLkeyword{transformeriprimarylabel}
\begin{LTXexample}[width=5.5cm]
\begin{pspicture}(5,5)
  \pnodes(0,5){A}(0,0){B}(5,5){C}(5,0){D}
  \transformer[transformeriprimarylabel=$i_1$,
    transformerisecondarylabel=$i_2$]%
    (A)(B)(C)(D){$\mathcal T$}
\end{pspicture}
\end{LTXexample}


\subsubsection{Parallel parameters}

If the \Lkeyword{parallel} parameter is set to \true, the dipole is drawn parallel to the line
connecting the nodes.

\begin{LTXexample}[width=3.5cm]
\begin{pspicture}(3,3)
  \pnodes(0,.5){A}(3,.5){B}
  \resistor(A)(B){}
  \resistor[parallel](A)(B){}
\end{pspicture}
\end{LTXexample}

\begin{LTXexample}[width=3.5cm]
\begin{pspicture}(3,3)
  \pnodes(0,.5){A}(3,.5){B}
  \resistor(A)(B){}
  \resistor[parallel,parallelsep=.5](A)(B){}
\end{pspicture}
\end{LTXexample}

\begin{LTXexample}[width=3.5cm]
\begin{pspicture}(3,3)
  \pnodes(0,.5){A}(3,.5){B}
  \resistor(A)(B){}
  \resistor[parallel,parallelsep=.3,
    parallelarm=2](A)(B){}
\end{pspicture}
\end{LTXexample}

\begin{LTXexample}[width=3.5cm]
\begin{pspicture}(3,3)
  \pnodes(0,.5){A}(3,.5){B}
  \resistor(A)(B){}
  \resistor[parallel,parallelsep=.3,
    parallelarm=2,parallelnode](A)(B){}
\end{pspicture}
\end{LTXexample}

\begin{LTXexample}[width=8.5cm]
\begin{pspicture}(8,8)
  \pnodes(0,0){A}(8,8){B}
  \multidipole(A)(B)\resistor{$R$}%
    \capacitor[linecolor=red]{$C$}%
    \coil[parallel,parallelsep=.1]{$L$}%
    \diode{$D$}.
\end{pspicture}
\end{LTXexample}

Note: When used with \Lcs{multidipole}, the  \Lkeyword{parallel} parameter
must not be set for the first dipole.



\subsubsection{Wire intersections}

\begin{LTXexample}[width=3.5cm]
\begin{pspicture}(3,3)
  \pnodes(0,0){A}(3,3){B}(0,3){C}(3,0){D}
  \wire(A)(B)
  \wire[intersect,intersectA=A,intersectB=B](C)(D)
\end{pspicture}
\end{LTXexample}

Wire intersect parameters work also with \Lcs{multidipole}.

\begin{LTXexample}[width=6.5cm]
\begin{pspicture}(7,7)
  \pnodes(0,0){A}(6,6){B}(0,6){C}(6,0){D}
  \wire(A)(B)
  \multidipole(C)(D)\resistor{$R$}%
    \wire[intersect,intersectA=A,intersectB=B]%
    \capacitor{$C$}.
\end{pspicture}
\end{LTXexample}


\bigskip
\subsubsection{Dipole style parameters}

\xLkeyset{dipolestyle=twoCircles}\xLcs{ICC}
\begin{LTXexample}[width=3.5cm]
\begin{pspicture}(3,2)
  \pnodes(0,1){A}(3,1){B}
  \Icc[dipolestyle=twoCircles](A)(B){$I$}
\end{pspicture}
\end{LTXexample}

\xLkeyset{dipolestyle=zigzag}\xLcs{resistor}
\begin{LTXexample}[width=3.5cm]
\begin{pspicture}(3,2)
  \pnodes(0,1){A}(3,1){B}
  \resistor[dipolestyle=zigzag](A)(B){$R$}
\end{pspicture}
\end{LTXexample}

\xLkeyset{dipolestyle=varistor}\xLcs{resistor}
\begin{LTXexample}[width=3.5cm]
\begin{pspicture}(3,2)
  \pnodes(0,1){A}(3,1){B}
  \resistor[dipolestyle=varistor](A)(B){U}
\end{pspicture}
\end{LTXexample}

\xLkeyset{dipolestyle=chemical}\xLcs{capacitor}
\begin{LTXexample}[width=3.5cm]
\begin{pspicture}(3,2)
  \pnodes(0,1){A}(3,1){B}
  \capacitor[dipolestyle=chemical](A)(B){$C$}
\end{pspicture}
\end{LTXexample}

\xLkeyset{dipolestyle=elektor}\xLcs{capacitor}
\begin{LTXexample}[width=3.5cm]
\begin{pspicture}(3,2)
  \pnodes(0,1){A}(3,1){B}
  \capacitor[dipolestyle=elektor](A)(B){$C$}
\end{pspicture}
\end{LTXexample}

\xLkeyset{dipolestyle=elektorchemical}\xLcs{capacitor}
\begin{LTXexample}[width=3.5cm]
\begin{pspicture}(3,2)
  \pnodes(0,1){A}(3,1){B}
  \capacitor[dipolestyle=elektorchemical](A)(B){$C$}
\end{pspicture}
\end{LTXexample}

\xLkeyset{dipolestyle=crystal}\xLcs{capacitor}\index{Quartz}
\begin{LTXexample}[width=3.5cm]
\begin{pspicture}(3,2)
  \pnodes(0,1){A}(3,1){B}
  \capacitor[dipolestyle=crystal](A)(B){$Q$}
\end{pspicture}
\end{LTXexample}

\xLkeyset{dipolestyle=rectangle}\xLcs{coil}
\begin{LTXexample}[width=3.5cm]
\begin{pspicture}(3,2)
  \pnodes(0,1){A}(3,1){B}
  \coil[dipolestyle=rectangle](A)(B){$L$}
\end{pspicture}
\end{LTXexample}

\xLkeyset{dipolestyle=curved}\xLcs{coil}
\begin{LTXexample}[width=3.5cm]
\begin{pspicture}(3,2)
  \pnodes(0,1){A}(3,1){B}
  \coil[dipolestyle=curved](A)(B){$L$}
\end{pspicture}
\end{LTXexample}

\xLkeyset{dipolestyle=elektor}\xLcs{coil}
\begin{LTXexample}[width=3.5cm]
\begin{pspicture}(3,2)
  \pnodes(0,1){A}(3,1){B}
  \coil[dipolestyle=elektor](A)(B){$L$}
\end{pspicture}
\end{LTXexample}

\xLkeyset{dipolestyle=elektorcurved}\xLcs{coil}
\begin{LTXexample}[width=3.5cm]
\begin{pspicture}(3,2)
  \pnodes(0,1){A}(3,1){B}
  \coil[dipolestyle=elektorcurved](A)(B){$L$}
\end{pspicture}
\end{LTXexample}

\xLkeyset{dipolestyle=thyristor}\xLcs{diode}
\begin{LTXexample}[width=3.5cm]
\begin{pspicture}(3,2)
  \pnodes(0,1){A}(3,1){B}
  \diode[dipolestyle=thyristor](A)(B){$T$}
\end{pspicture}
\end{LTXexample}

\xLkeyset{dipolestyle=GTO}\xLcs{diode}
\begin{LTXexample}[width=3.5cm]
\begin{pspicture}(3,2)
  \pnodes(0,1){A}(3,1){B}
  \diode[dipolestyle=GTO](A)(B){$T$}
\end{pspicture}
\end{LTXexample}

\xLkeyset{dipolestyle=triac}\xLcs{diode}
\begin{LTXexample}[width=3.5cm]
\begin{pspicture}(3,2)
  \pnodes(0,1){A}(3,1){B}
  \diode[dipolestyle=triac](A)(B){$T$}
\end{pspicture}
\end{LTXexample}

\xLkeyset{dipolestyle=schottky}\xLcs{diode}
\begin{LTXexample}[width=3.5cm]
\begin{pspicture}(3,2)
  \pnodes(0,1){A}(3,1){B}
  \diode[dipolestyle=schottky](A)(B){$T$}
\end{pspicture}
\end{LTXexample}

\xLkeyword{variable}\xLcs{resistor}
\begin{LTXexample}[width=3.5cm]
\begin{pspicture}(3,2)
  \pnodes(0,1){A}(3,1){B}
  \resistor[variable](A)(B){$R$}
\end{pspicture}
\end{LTXexample}

\xLkeyword{variable}\xLcs{capacitor}
\begin{LTXexample}[width=3.5cm]
\begin{pspicture}(3,2)
  \pnodes(0,1){A}(3,1){B}
  \capacitor[variable](A)(B){$C$}
\end{pspicture}
\end{LTXexample}

\xLkeyword{variable}\xLcs{coil}
\begin{LTXexample}[width=3.5cm]
\begin{pspicture}(3,2)
  \pnodes(0,1){A}(3,1){B}
  \coil[variable](A)(B){$L$}
\end{pspicture}
\end{LTXexample}

\xLkeyword{variable}\xLcs{battery}
\begin{LTXexample}[width=3.5cm]
\begin{pspicture}(3,2)
  \pnodes(0,1){A}(3,1){B}
  \battery[variable](A)(B){$U$}
\end{pspicture}
\end{LTXexample}

\xLkeyword{variable}\xLcs{coil}\xLkeyset{dipolestyle=elektor}
\begin{LTXexample}[width=3.5cm]
\begin{pspicture}(3,2)
  \pnodes(0,1){A}(3,1){B}
  \coil[dipolestyle=elektor,variable](A)(B){$L$}
\end{pspicture}
\end{LTXexample}

In the following example the parameter \Lkeyword{dipolestyle} is used for a tripole and quadrupole, because
the coils are drawn as rectangles and the resistor as a \Lkeyword{zigzag}.

\xLkeyset{labelangle=:U}\xLcs{potentiometer}\xLkeyset{dipolestyle=zigzag}
\begin{LTXexample}[width=3.5cm]
\begin{pspicture}(3,3)
  \pnodes(0,0){A}(3,3){B}(3,1.5){C}
  \potentiometer[dipolestyle=zigzag,%
  	labelangle=:U](A)(B)(C){$P$}
\end{pspicture}
\end{LTXexample}

\xLcs{transformer}\xLkeyset{dipolestyle=rectangle}
\begin{LTXexample}[width=4.5cm]
\begin{pspicture}(4,4)
  \pnodes(0,4){A}(0,0){B}(4,4){C}(4,0){D}
  \transformer[dipolestyle=rectangle](A)(B)(C)(D){$\mathcal T$}
\end{pspicture}
\end{LTXexample}


\subsubsection{Tripole style parameters}

\xLcs{Tswitch}\xLkeyset{tripolestyle=left}
\begin{LTXexample}[width=5.5cm]
\begin{pspicture}(5,3)
  \pnodes(0,2){A}(5,2){B}(0,0){C}
  \Tswitch[tripolestyle=left](A)(B)(C){$K$}
\end{pspicture}
\end{LTXexample}

\xLcs{Tswitch}\xLkeyset{tripolestyle=right}
\begin{LTXexample}[width=5.5cm]
\begin{pspicture}(5,3)
  \pnodes(0,2){A}(5,2){B}(0,0){C}
  \Tswitch[tripolestyle=right](A)(B)(C){$K$}
\end{pspicture}
\end{LTXexample}

\xLcs{OA}\xLkeyset{tripolestyle=french}
\begin{LTXexample}[width=5.5cm]
\begin{pspicture}(5,3)
  \pnodes(0,3){A}(0,0){B}(5,1.5){C}
  \OA[tripolestyle=french](A)(B)(C)
\end{pspicture}
\end{LTXexample}

\subsubsection{Tripoles}

\xLcs{potentiometer}\xLkeyword{labeloffset}
\begin{pspicture}(3,3)
  \pnodes(0,1){A}(3,1){B}(3,2){C}
  \potentiometer[labeloffset=0pt](A)(B)(C){P}
\end{pspicture}
\hfill
\begin{pspicture}(3,3)
  \pnodes(0,2.5){A}(3,2.5){B}(0,1){C}
  \potentiometer[labeloffset=0pt](A)(B)(C){P}
\end{pspicture}
\hfill
\xLcs{potentiometer}\xLkeyword{labeloffset}\xLkeyset{labelangle=:U}
\begin{pspicture}(3,3)
  \pnodes(0,0){A}(3,2){B}(2.5,3){C}
  \potentiometer[labeloffset=0pt,labelangle=:U](A)(B)(C){P}
\end{pspicture}

\vspace{1cm}
\noindent
\xLcs{potentiometer}\xLkeyword{labeloffset}
\begin{pspicture}(3,3)
  \pnodes(1,0){A}(1,3){B}(2.5,0){C}
  \potentiometer[labeloffset=0pt](A)(B)(C){P}
\end{pspicture}
\hfill
\begin{pspicture}(3,3)
  \pnodes(0,3){A}(3,0){B}(2,0){C}
  \potentiometer[labeloffset=0pt,labelangle=:U](A)(B)(C){P}
\end{pspicture}
\hfill
\begin{pspicture}(3,3)
  \pnodes(0,2){A}(3,2){B}(1.5,0){C}
  \potentiometer[labeloffset=0pt](A)(B)(C){P}
\end{pspicture}


\vspace{1cm}
\noindent
\begin{pspicture}(3,3)
  \pnodes(1,0){A}(1,3){B}(2.5,0){C}
  \potentiometer[dipolestyle=zigzag](A)(B)(C){P}
\end{pspicture}
\hfill
\begin{pspicture}(3,3)
  \pnodes(0,3){A}(3,0){B}(2,0){C}
  \potentiometer[dipolestyle=zigzag,labelangle=:U](A)(B)(C){P}
\end{pspicture}
\hfill
\begin{pspicture}(3,3)
  \pnodes(0,2){A}(3,2){B}(1.5,0){C}
  \potentiometer[dipolestyle=zigzag](A)(B)(C){P}
\end{pspicture}

\begin{pspicture}[showgrid=true](9.5,10)
\dotnode(0,2){N1}
\dotnode(4,2){N2}
\dotnode(2,0){N3}
\powermeter[tripolestyle=bottom,tripoleconfig=left](N1)(N2)(N3){WR}

\rput(5,0){
\dotnode(0,2){N1}
\dotnode(4,2){N2}
\dotnode(2,0){N3}
\powermeter[tripolestyle=bottom,tripoleconfig=right](N1)(N2)(N3){WR}
}

\rput(0,3){
\dotnode(0,2){N1}
\dotnode(4,2){N2}
\dotnode(2,4){N3}
\powermeter[tripolestyle=top,tripoleconfig=left](N1)(N2)(N3){WR}
}

\rput(5,3){
\dotnode(0,2){N1}
\dotnode(4,2){N2}
\dotnode(2,4){N3}
\powermeter[tripolestyle=top,tripoleconfig=right](N1)(N2)(N3){WR}
}

\end{pspicture}



\begin{pspicture}[showgrid=true](-1,-1)(9.5,5)
%\psgrid

{%\psset{showNode}
\dotnode(0,4){NR}\nput{180}{NR}{R}
\dotnode(0,2.5){NS}\nput{180}{NS}{S}
\dotnode(0,1){NT}\nput{180}{NT}{T}
\dotnode(0,0){NNS}\nput{180}{NNS}{N}

\dotnode(1.5,0){NWRM}
\dotnode(2.5,0){NWSM}
\dotnode(3.5,0){NWTM}

\pnode(3,4){NRC}
\pnode(5,2.5){NSC}
\pnode(7,1){NTC}
\pnode(4,0){NNC}

\dotnode(7,4){NU}\nput{0}{NU}{U}
\dotnode(7,2.5){NV}\nput{0}{NV}{V}
\dotnode(7,1){NW}\nput{0}{NW}{W}
\dotnode(7,0){NN}\nput{0}{NN}{N}
}

\powermeter[tripolestyle=bottom,tripoleconfig=left](NR)(NRC)(NWRM){WR}
\powermeter[tripolestyle=bottom,tripoleconfig=left](NS)(NSC)(NWSM){WS}
\powermeter[tripolestyle=bottom,tripoleconfig=left](NT)(NTC)(NWTM){WT}

\psframe(7,-0.5)(9,4.5)
\wire(NRC)(NU)
\wire(NSC)(NV)
\wire(NTC)(NW)
\wire(NNS)(NN)

\end{pspicture}



\subsubsection{Other Parameters}

\xLkeyword{OAinvert}\xLcs{OA}
\begin{LTXexample}[width=5.5cm]
\begin{pspicture}(5,3)
  \pnodes(0,0){A}(0,3){B}(5,1.5){C}
  \OA[OAinvert=false](B)(A)(C)
\end{pspicture}
\end{LTXexample}

\xLkeyword{OAperfect}\xLcs{OA}
\begin{LTXexample}[width=5.5cm]
\begin{pspicture}(5,3)
  \pnodes(0,0){A}(0,3){B}(5,1.5){C}
  \OA[OAperfect=false](B)(A)(C)
\end{pspicture}
\end{LTXexample}

\begin{LTXexample}[width=5.5cm]
\begin{pspicture}(5,3)
  \pnodes(0,1.5){A}(5,0){B}(5,3){C}
  \transistor[basesep=2cm,%
    transistorinvert,transistorcircle=false](A)(B)(C)
\end{pspicture}
\end{LTXexample}

\xLkeyset{transistortype=FET}\xLcs{transistor}
\begin{LTXexample}[width=5.5cm]
\begin{pspicture}(5,3)
  \pnode(0,1.5){A}\psset{linewidth=1pt}
  \transistor[basesep=2cm,arrows=o-o,
    transistortype=FET](A){Emitter}{Collector}
  \psline{o-}(5,3)(3,3)(3,3|Collector)(Collector)
  \psline{o-}(5,0)(3,0)(3,3|Emitter)(Emitter)
  \psline{o-}(A)([nodesep=2]A)
\end{pspicture}
\end{LTXexample}

\xLkeyset{transistortype=FET}\xLcs{transistor}\xLkeyset{FETchannel=P}
\begin{LTXexample}[width=5.5cm]
\begin{pspicture}(5,3)
  \pnode(0,1.5){A}\psset{linewidth=1pt}
  \transistor[basesep=2cm,arrows=o-o,
    transistortype=FET,
    FETchanneltype=P](A){Emitter}{Collector}
  \psline{o-}(5,3)(3,3)(3,3|Collector)(Collector)
  \psline{o-}(5,0)(3,0)(3,3|Emitter)(Emitter)
  \psline{o-}(A)([nodesep=2]A)
\end{pspicture}
\end{LTXexample}

\xLkeyset{transistortype=FET}\xLcs{transistor}\xLkeyword{FETmemory}
\begin{LTXexample}[width=5.5cm]
\begin{pspicture}(5,3)
\transistor[basesep=2cm,transistortype=FET,
  FETmemory=true](0,1.5)(5,0)(5,3)
\end{pspicture}
\end{LTXexample}

\clearpage
\subsection{Special objects}

\subsubsection{\nxLcs{dashpot}}


\begin{LTXexample}[pos=t]
\newcommand*\pswall[3]{% ll ur lr
  \psframe[linecolor=white,fillstyle=hlines,hatchcolor=black](#1)(#2)% (ll)(ur)
  \psline[linecolor=black](#1)(#3)}
\begin{pspicture}(0.5,1)(8,10)
  \rput(3,9.5){\sffamily \textbf{Viscoelasticity}}
  % Kelvin-Voigt model (spring and dashpot parallel): ===========
  \rput[c](1.75,8.85){\sffamily Kelvin-Voigt}
  \pswall{1,8}{2.5,8.5}{2.5,8}% top
  \psline(1.75,8)(1.75,7)% top vertical line
  % node definitions:
  \pnodes(1,7){ul1}(2.5,7){ur1}(1,3){ll1}(2.5,3){lr1}%
  \psline(ul1)(ur1)% top line 
  \psline(ll1)(lr1)% bottom line
  \resistor[dipolestyle=zigzag,linewidth=0.5pt](ul1)(ll1){}% spring
  \dashpot[linewidth=0.5pt](ur1)(lr1){}% dashpot
  \psline[arrowscale=3]{->}(1.75,3)(1.75,2)% force
  % Maxwell model (spring and dashpot serial): ==================
  \rput[c](4.5,8.85){\sffamily Maxwell}
  \pswall{4,8}{5,8.5}{5,8}% top
  \pnodes(4.5,8){t}(4.5,4){b}% node definitions
  \resistor[dipolestyle=zigzag,linewidth=0.5pt,labeloffset=1.8](t)(b)% spring
  {\sffamily\small\begin{tabular}{c}\textbf{elasticity}\\(Hookean solid)\end{tabular}}% end spring
  \dashpot[linewidth=0.5pt,labeloffset=1.8](4.5,5)(4.5,3)% dashpot
  {\sffamily\small\begin{tabular}{c}\textbf{viscosity}\\(Newtonian fluid)\end{tabular}
  }% end dashpot
  \psline[arrowscale=3]{->}(4.5,3)(4.5,2)% force
\end{pspicture}
\end{LTXexample}

%
% Modified pst-circ Components
%
\section{Modified default symbols}


\subsection{Dipole}

%
% New Diode
%
\subsubsection{New Diode}
\xLcs{NewDiode}
\begin{LTXexample}[width=3.5cm]
\begin{pspicture}[showgrid=false](3,4)
  \pnodes(0,1){A}(3,1){B}(0,3){C}(3,3){D}
  \newdiode(C)(D){$D_1$}
  \newdiode[ison=false](A)(B){$D_2$}
\end{pspicture}
\end{LTXexample}

%
% New Zener
%
\subsubsection{New Zener}
\xLcs{NewZener}
\begin{LTXexample}[width=3.5cm]
\begin{pspicture}[showgrid=false](3,4)
  \pnodes(0,1){A}(3,1){B}(0,3){C}(3,3){D}
  \newZener(C)(D){$D_1$}
  \newZener[ison=false](A)(B){$D_2$}
\end{pspicture}
\end{LTXexample}

%
% New LED
%
\subsubsection{New LED}
\xLcs{NewLED}
\begin{LTXexample}[width=3.5cm]
\begin{pspicture}[showgrid=false](3,4)
  \pnodes(0,1){A}(3,1){B}(0,3){C}(3,3){D}
  \newLED(C)(D){$D_1$}
  \newLED[ison=false](A)(B){$D_2$}
\end{pspicture}
\end{LTXexample}

%
% New Ideal Switch
%
\subsubsection{New Ideal Switch}
\xLcs{NewSwitch}
\begin{LTXexample}[width=3.5cm]
\begin{pspicture}[showgrid=false](3,4)
  \pnodes(0,1){A}(3,1){B}(0,3){C}(3,3){D}
  \newSwitch(C)(D){$S_1$}
  \newSwitch[ison=false](A)(B){$S_2$}
\end{pspicture}
\end{LTXexample}

%
% New Capacitor
%
\subsubsection{New Capacitor}
\xLcs{RFLine}
\begin{LTXexample}[width=3.5cm]
\begin{pspicture}[showgrid=false](3,2)
  \pnodes(0,1){A}(3,1){B}
  \newcapacitor(A)(B){$C_1$}
\end{pspicture}
\end{LTXexample}

%
% New Armature
%
\subsubsection{New Armature (motor or generator)}
\xLcs{RFLine}
\begin{LTXexample}[width=3.5cm]
\begin{pspicture}[showgrid=false](3,4)
  \pnodes(0,1){A}(3,1){B}(0,3){C}(3,3){D}
  \newarmature[labelInside=1](C)(D){$M_{CC}$}
  \newarmature[labelInside=2](A)(B){$G_{CC}$}
\end{pspicture}
\end{LTXexample}

%
% VDC
%
\subsubsection{V DC}
\xLcs{RFLine}
\begin{LTXexample}[width=3.5cm]
\begin{pspicture}[showgrid=false](3,2)
  \pnodes(0,1){A}(3,1){B}
  \vdc(A)(B){$V_{DC}$}
\end{pspicture}
\end{LTXexample}

%
% VAC
%
\subsubsection{V AC}
\xLcs{RFLine}
\begin{LTXexample}[width=3.5cm]
\begin{pspicture}[showgrid=false](3,2)
  \pnodes(0,1){A}(3,1){B}
  \vac(A)(B){$V_{AC}$}
\end{pspicture}
\end{LTXexample}

\clearpage


\section{Examples}

\begin{LTXexample}[pos=t]
  \begin{pspicture}(-1.5,-1)(6,5)
  \pnodes(0,0){A}(0,3){B}(4.5,3){C}(4.5,0){D}
  \Ucc[tension,dipoleconvention=generator](A)(B){$E$}
  \multidipole(B)(C)%
    \switch[intensitylabel=$i$]{$K$}%
    \resistor[labeloffset=0,tensionlabel=$u_R$]{$R$}.
  \capacitor[tensionlabel={$u_C$},tensionlabeloffset=-1.2,
    tensionoffset=-1,directconvention=false](D)(C){$C$}
  \wire(A)(D)
  \ground(D)
  \end{pspicture}
\end{LTXexample}

\begin{LTXexample}[pos=t]
  \begin{pspicture}(-0.5,0)(7,8)
  \pnodes(0.5,1){A}(3.5,1){B}(6.5,1){C}(0.5,4){D}(3.5,4){Minus}
         (3.5,3){Plus}(6.5,5){S}(3.5,5){E}
  \resistor(D)(Minus){$R_2$}
  \capacitor(E)(S){$C$}
  \resistor[parallel,parallelarm=2](E)(S){$R_1$}
  \OA[intensity](Minus)(Plus)(S)
  \wire(Minus)(E)
  \wire(Plus)(B)
  \tension(A)(D){$u_E$}
  \makeatletter % (special tricks see below)
  \tension(C)(S@@){$u_S$}
  \tension[linecolor=blue](Plus@@)(Minus@@){$\epsilon$}
  \makeatother
  \ground(A) \ground(B) \ground(C)
  \end{pspicture}
\end{LTXexample}

\begin{LTXexample}[pos=t]
  \begin{pspicture}(-1,0)(7,8)
  \pnodes(1,1){A}(1,7){B}(3,1){C}(3,7){D}
  \Ucc[tensionlabel=$E$](A)(B){}
  \resistor(B)(D){$R$}
  \coil(D)(C){$L$}
  \capacitor[parallel,parallelarm=2.5](D)(C){$C$}
  \wire(A)(C)
  \end{pspicture}
\end{LTXexample}




\begin{LTXexample}[pos=t]
% \usepackage{amsmath}  % example by Markus Graube
\begin{pspicture}(0,.5)(13,4)
  \pnodes(1,1){I_U}(1,3){I_O}(2.5,1){C}(2.5,3){D}(4,1){K_LU}(4,3){K_LO}(7,1){K_RU}%
         (7,3){K_RO}(9,3){E}(7.3,3){K_RO1}(7.3,1){K_RU1}(11,3){F}(12,1){O_U}(12,3){O_O}
  \tension[labeloffset=-0.5](I_O)(I_U){$\underline{u}$}
  \wire[arrows=o-](I_U)(C)
  \wire[intensitylabel=$\underline{i}$, arrows=o-](I_O)(D)
  \capacitor[labeloffset=.9](C)(D){$C_B$}
  \qdisk(C){2pt} \qdisk(D){2pt}
  \wire(C)(K_LU)
  \wire[intensitylabel=$\underline{i}_W$](D)(K_LO)
  \quadripole(K_LO)(K_LU)(K_RO)(K_RU){\parbox{3cm}{%
    \begin{align*} 
      \underline{u}   &= \frac{\underline{p}}{X} \\[2ex]
      \underline{i}_W &= X \underline{q}_U
    \end{align*}}}
  \wire(K_RO)(K_RO1)
  \tension[labeloffset=0.5](K_RO1)(K_RU1){$\underline{p}$}
  \coil[dipolestyle=rectangle](K_RO)(E){$M_{a,K}$}
  \capacitor(E)(F){$N_{a,K}$}
  \wire[intensitylabel=$\underline{q}_U$,arrows=-o](F)(O_O)
  \wire[arrows=-o](K_RU)(O_U)
  \tension[labeloffset=0.5](O_O)(O_U){$\underline{p}_U$}
\end{pspicture}
\end{LTXexample}




\begin{LTXexample}[width=8.5cm]
  \begin{pspicture}(-0.25,-0.25)(6,6)
  \pnodes(0,3){A}(3,3){B}(6,3){C}
  % Dipole node connections
  \coil[intensitylabel=$i$](A)(B){$L$}
  \coil[intensitylabel=$i'$,intensitycolor=green,%
    parallel,parallelarm=2](B)(C){$L'$}
  \capacitor[parallel,parallelarm=-2](B)(C){$C$}
  \end{pspicture}
\end{LTXexample}

\begin{LTXexample}[pos=t]
  \begin{pspicture}(6,6)
  \pnodes(0,0){A}(6,0){B}(0.3,4){Cprime}(5.7,4){Dprime}(2.5,4){Gprime}%
         (2.5,0){Hprime}(0,4){C}(6,4){D}(0.3,6){E}(5.7,6){F}(4,6){G}(4,0){H}
  \multidipole(G)(H)%
    \wire[intersect,
      intersectA=C,intersectB=D]
    \resistor{$R'_3$}.
  \resistor(E)(G){$R'_1$}
  \resistor(G)(F){$R'_2$}
  \multidipole(C)(D)\resistor{$R_1$}%
    \wire\resistor{$R_2$}.
  \wire(A)(B)\wire(Cprime)(E)
  \wire(Dprime)(F)
  \resistor(Hprime)(Gprime){$R_3$}
  \end{pspicture}
\end{LTXexample}



\begin{LTXexample}[pos=t]
  \begin{pspicture}(0,-0.25)(9,11)
  \pnodes(0,0){A}(9,0){B}(0,6){C}(9,6){D}(4.5,1){E}(4.5,10.5){F}
  \switch(A)(C){$K$}
  \multidipole(A)(B)\resistor{$R$}\battery[intensitylabel=$i$]{$V$}.
  \wire(B)(D)
  \multidipole(C)(D)\diode{$D$}\wire.
  \resistor[tensionlabel=$U_1$](C)(F){$R_1$} \resistor(C)(E){$R_4$}
  \capacitor[parallel,parallelarm=1.2,parallelsep=1.5](C)(E){$C_2$}
  \coil(E)(D){$L$}
  \capacitor[parallel,parallelarm=1.2,parallelsep=1.5](E)(D){$C_3$}
  \capacitor[tensionlabel=$U_2$](F)(D){$C_1$}
  \multidipole(E)(F)\wire\wire[intersect,intersectA=C,intersectB=D]%
    \circledipole[labeloffset=-0.7]{$E$}%
    \resistor[parallel,parallelsep=.6,parallelarm=.8]{$R$}.
  \end{pspicture}
\end{LTXexample}

\begin{LTXexample}[pos=t]
\begin{pspicture}(0,-0.2)(13,8)
  \psset{intensitycolor=red,intensitylabelcolor=red,tensioncolor=green,
    tensionlabelcolor=green, intensitywidth=3pt}
  \circledipole[tension,tensionlabel=$U_0$,
    tensionoffset=0.75,labeloffset=0](0,0)(0,6){\LARGE\textbf{=}}
  \wire[intensity,intensitylabel=$i_0$](0,6)(2.5,6)
  \diode[dipolestyle=thyristor](2.5,6)(4.5,6){$T_1$}
  \wire[intensity,intensitylabel=$i_1$](4.5,6)(6.5,6)
  \multidipole(6.5,7.5)(2.5,7.5)%
        \coil[dipolestyle=rectangle,labeloffset=-0.75]{$L_5$}%
        \diode[labeloffset=-0.75]{$D_5$}.
  \wire[intensity,intensitylabel=$i_5$](6.5,6)(6.5,7.5)
  \wire(2.5,7.5)(2.5,3)
  \wire[intensity,intensitylabel=$i_c$](2.5,4.5)(2.5,6)
  \qdisk(2.5,6){2pt}\qdisk(6.5,6){2pt}
  \diode[dipolestyle=thyristor](2.5,4.5)(4.5,4.5){$T_2$}
  \wire[intensity,intensitylabel=$i_2$](4.5,4.5)(6.5,4.5)
  \capacitor[tension,tensionlabel=$u_c$,tensionoffset=-0.75,
    tensionlabeloffset=-1](6.5,4.5)(6.5,6){$C_k$}
  \qdisk(2.5,4.5){2pt}\qdisk(6.5,4.5){2pt}
  \wire[intensity,intensitylabel=$i_3$](6.5,4.5)(6.5,3)
  \multidipole(6.5,3)(2.5,3)%
    \coil[dipolestyle=rectangle,labeloffset=-0.75]{$L_3$}%
    \diode[labeloffset=-0.75]{$D_3$}.
  \wire(6.5,6)(9,6)\qdisk(9,6){2pt}
  \diode(9,0)(9,6){$D_4$}
  \wire[intensity,intensitylabel=$i_4$](9,3.25)(9,6)
  \wire[intensity,intensitylabel=$i_a$](9,6)(11,6)
  \multidipole(11,6)(11,0)%
    \resistor{$R_L$}
    \coil[dipolestyle=rectangle]{$L_L$}
    \circledipole[labeloffset=0,tension,tensionoffset=0.7,tensionlabel=$U_B$]{\LARGE\textbf{=}}.
  \wire(0,0)(11,0)\qdisk(9,0){2pt}
  \pnode(12.5,5.5){A}\pnode(12.5,0.5){B}
  \tension(A)(B){$u_a$}
\end{pspicture}
\end{LTXexample}


\makeatletter
%
\def\REG{\@ifnextchar[{\pst@REG}{\pst@REG[]}}
%
\def\pst@REG[#1](#2)(#3)(#4)#5{{%
  \psset{dimen=middle,arm=0}%
  \psset{#1}
  \pst@getcoor{#2}\pst@tempa
  \pst@getcoor{#3}\pst@tempb
  \pst@getcoor{#4}\pst@tempc
  \pnode(!%
    \pst@tempa /Y1 exch \pst@number\psyunit div def
    /X1 exch \pst@number\psxunit div def
    \pst@tempb /Y2 exch \pst@number\psyunit div def
    /X2 exch \pst@number\psxunit div def
    \pst@tempc /Y3 exch \pst@number\psyunit div def
    /X3 exch \pst@number\psxunit div def
    /XC X1 X2 add 2 div def
    /YC Y1 2 mul Y3 add 3 div def
    /Xin XC 1 sub def
    /Yin YC 0.5 add def
    /Xout XC 1 add def
    /Yout Yin def
    /Xref XC def
    /Yref YC 1 sub def
    XC YC){C@}
  \pnode(! Xin Yin){in@}
  \pnode(! Xout Yout){out@}
  \pnode(! Xref Yref){ref@}
  \rput(C@){\pst@draw@REG}
  \ncangle{#2}{in@}
  \ncangle{#3}{out@}
  \ncangle{#4}{ref@}
  \rput(C@){#5}
  }\ignorespaces}
%
\def\pst@draw@REG{%
  \begingroup
  \psset{linewidth=1.5\pslinewidth}%
  \psframe(-1,-0.5)(1,0.75)
  \psline(-1.5,0.5)(-1,0.5)
  \psline(1.5,0.5)(1,0.5)
  \psline(0,-0.5)(0,-1)
  \endgroup
  }
%
\makeatother

The following example was written by Manuel Luque.

\begin{LTXexample}[pos=t]
\begin{pspicture}(0,-0.5)(14,4)
  \pnodes(0,-0.50){B}(0,3){A}(2.5,3.5){C}(2.5,-0.5){D}(5,3){E}(6.5,1.5){F}(5,0){G}%
         (3.5,1.5){H}(8,2.5){I}(8,1){J}(10,2.5){K}(10,1){L}(14,2.5){M}(12,1){N}%
         (3,1){H'}(14,2.5){O}(14,1){P}(13.5,1){Q}
  \transformer[transformeriprimarylabel=$i_1$,transformerisecondarylabel=$i_2$,
    primarylabel=$n_1$,secondarylabel=$n_2$](A)(B)(C)(D){$T_1$}
  {\psset{fillstyle=solid,fillcolor=black}
  \diode(H)(E){}\diode(H)(G){} \diode(E)(F){}\diode(G)(F){}}
  \capacitor[dipolestyle=chemical](I)(J){}  \capacitor(K)(L){}
  \REG(K)(M)(N){\shortstack{\textsf{\textbf{\large LM7805}}\\\textbf{+5V}}}
  \ncangle{I}{F}\psline(I)(K)  \ncangle{E}{C}\ncangle{G}{D}
  \ncangle[arm=0]{P}{Q}        \ncangle[arm=0]{H}{H'}
  \ground(H')\ground(J)\ground(L)\ground(N)\ground(Q)
  \psdots(A)(B)(P)(O)(G)(H)(F)(I)(K)(E)
\end{pspicture}
\end{LTXexample}



The following example was written by Lionel Cordesses.


\begin{LTXexample}[pos=t]
\begin{pspicture}(11,3)
  \psset{dipolestyle=elektor}
  \pnodes(1,2){Vin}(0.5,2){S}(0.5,0){Sm}(2.5,2){A}(4.5,2){B}(6.5,2){C}(8,2){Cd}%
         (8.5,2){D}(9.5,2){E}(2.5,0){Am}(4.5,0){Bm}(6.5,0){Cm}(8.5,0){Dm}(9.5,0){Em}
  \Ucc[labeloffset=0.9](Sm)(S){$V_{in}$}\resistor(Vin)(A){$R$}
  \capacitor(A)(Am){$C_1$} \capacitor(B)(Bm){$C_3$}
  \capacitor[labeloffset=-0.7](D)(Dm){$C_n$}\resistor(E)(Em){$R$}
  \coil(A)(B){$L_2$}\coil(B)(C){$L_4$}
  \wire(Am)(Bm)\wire(Bm)(Cm)\wire(Cm)(Dm)\wire(Dm)(Em)\wire(D)(E)
  \wire(Cd)(D)\psline[linestyle=dashed](C)(Cd)
  \wire(S)(Vin)\wire(Sm)(Am)
  \psdots(D)(Dm)(A)(Am)(B)(Bm)
\end{pspicture}
\end{LTXexample}

The following example was written by Christian Hoffmann.


\begin{LTXexample}[pos=t]
  \SpecialCoor
  \begin{pspicture}(0,-1)(7,6.5)%
  \pnodes(0,6){plus}(3,3){basis}([nodesep=-2] basis){schalter}(0,0){masse}
  \wire[arrows=o-*](plus)(basis|plus)
  \uput[l](plus){$U_0$}
  \resistor[labeloffset=.8](basis|plus)(basis){$R_1$}
  \transistor[basesep=2cm](basis){emitter}{kollektor}
  \wire[arrows=-*](schalter)(basis)
%  \wire(basis)([nodesep=2] basis)
  \wire(TBaseNode)(basis)
  \switch(schalter|masse)(schalter){S}
  \lamp(kollektor|plus)(kollektor){L}
  \resistor(kollektor|plus)(basis|plus){$R_2$}
  \wire(emitter)(emitter|masse)
  \wire(emitter|masse)(basis|masse)
  \capacitor(basis)(basis|masse){$C_1$}
  \wire[arrows=*-](basis|masse)(schalter|masse)
  \wire[arrows=*-o](schalter|masse)(masse)
  \end{pspicture}
\end{LTXexample}


Variable radius for \xLcs{circledipole}

\begin{LTXexample}[pos=t]
\begin{pspicture}(\linewidth,3)
\circledipole(0,1)(3,1){} \pnodes(4,1){A}(7,1){B}
\circledipole[radius=7mm,labeloffset=1cm](A)(B){Strommesser}\rput(5.5,1){\Huge I}
\circledipole[radius=4mm,fillstyle=solid,fillcolor=blue!30](10,1)(13,1){C}
\end{pspicture}
\end{LTXexample}




\section{Microwave symbols}
Since for microwave signal, the direction in which the signal spreads is very important, 
There are  dipoleinput or tripoleinput or quadripoleinput and arrowinput parameters. 
The value of theses parameters are left or right for the first one and true or false for second one. 

\xLcs{ifPst@inputarrow}\xLcs{pcline}
\begin{lstlisting}[style=code]
\ifPst@inputarrow
   \ifx\psk@Dinput\pst@Dinput@right
       \pcline[arrows=-C](#2)(dipole@1)
       \pcline[arrows=->,arrowinset=0](#3)(dipole@2)
    \else
       \pcline[arrows=->,arrowinset=0](#2)(dipole@1)
       \pcline[arrows=C-](dipole@2)(#3)
   \fi
\else
   \pcline[arrows=-C](#2)(dipole@1)
   \pcline[arrows=C-](dipole@2)(#3)
\fi
\pcline[fillstyle=none,linestyle=none](#2)(#3)
\end{lstlisting}

The last line is to correct some problems when I use colors (see example2)
To add color in components (Monopole, tripole and Quadripole), there is a new 
argument. 
 \Lcs{multidipole} also works:

\begin{LTXexample}[width=3.5cm,rframe={}]
\begin{pspicture}(4,2)
  \pnodes(0.5,1){A}(3.5,1){B}
  \multidipole(A)(B)\filter{BPF}%
    \resistor{$R$}.
\end{pspicture}
\end{LTXexample}

\bigskip
\begin{LTXexample}[width=3.5cm,rframe={}]
\begin{pspicture}(4,2)
  \pnodes(0.5,1){A}(3.5,1){B}
  \multidipole(A)(B)\amplifier{LNA}%
    \resistor{$R$}.
\end{pspicture}
\end{LTXexample}


\subsection{New monopole components}
\subsubsection{New ground}
\begin{description}
  \item[\Lkeyword{groundstyle}:]  \Lkeyval{ads} | \Lkeyval{old} | \Lkeyval{triangle}
\end{description}

\begin{LTXexample}[width=3.5cm,rframe={}]
\begin{pspicture}(3,2)
  \pnodes(0.5,1){A}(1,1){B}(2.5,1){C}
  \newground(A)
  \newground[groundstyle=old]{135}(B)
  \newground[linecolor=blue,groundstyle=triangle]{180}(C)
\end{pspicture}
\end{LTXexample}


\subsubsection{Antenna}
\begin{description}
  \item[\Lkeyword{antennastyle}:]  \Lkeyval{two} | \Lkeyval{three} | \Lkeyval{triangle}
\end{description}

\begin{LTXexample}[width=3.5cm,rframe={}]
\begin{pspicture}(3,2)
  \pnode(1,0.5){A}
  \antenna[antennastyle=three](A)
\end{pspicture}
\end{LTXexample}

\begin{LTXexample}[width=3.5cm,rframe={}]
\begin{pspicture}(3,2)
  \pnode(1,0.5){A}
  \antenna(A)
\end{pspicture}
\end{LTXexample}

\begin{LTXexample}[width=3.5cm,rframe={}]
\begin{pspicture}(3,2)
  \pnode(1,0.5){A}
  \antenna[antennastyle=triangle](A)
\end{pspicture}
\end{LTXexample}


\subsection{New monopole macro-components}
\subsubsection{Oscillator}
\begin{description}
  \item[\Lkeyword{output}:]  \Lkeyval{top} | \Lkeyval{right} | \Lkeyval{bottom} | \Lkeyval{left}
  \item[\Lkeyword{inputarrow}:] \false | \true
  \item[\Lkeyword{LOstyle}:] -- | \Lkeyval{crystal}
\end{description}

\begin{LTXexample}[width=3.5cm,rframe={}]
\begin{pspicture}(3,2)
  \pnode(1,1){A}
  \oscillator[output=left,inputarrow=false](A)%
    {$f_{LO}$}{}
\end{pspicture}
\end{LTXexample}

\begin{LTXexample}[width=3.5cm,rframe={}]
\begin{pspicture}(3,2)
  \pnode(1,1){A}
  \oscillator[output=top,inputarrow=true,LOstyle=crystal](A)%
    {f$_{\textrm{LO}}$}{}
\end{pspicture}
\end{LTXexample}

\begin{LTXexample}[width=3.5cm,rframe={}]
\begin{pspicture}(3,2)
  \pnode(1,1){A}
  \oscillator[output=right,inputarrow=false](A)%
    {$f_{LO}$}{fillstyle=solid,fillcolor=blue}
\end{pspicture}
\end{LTXexample}

\begin{LTXexample}[width=3.5cm,rframe={}]
\begin{pspicture}(3,2)
  \pnode(1,1){A}
  \oscillator[output=bottom,inputarrow=false](A)%
    {$f_{LO}$}{}
\end{pspicture}
\end{LTXexample}

\subsection{New dipole macro-components}
\subsubsection{Filters}
\begin{description}
  \item[\Lkeyword{dipolestyle}:]  \Lkeyval{bandpass} | \Lkeyval{lowpass} | \Lkeyval{highpass}
  \item[\Lkeyword{inputarrow}:] \false | \true
  \item[\Lkeyword{dipoleinput}:] \Lkeyval{left} | \Lkeyval{right}
\end{description}

\xLcs{filter}
\begin{LTXexample}[width=3.5cm,rframe={}]
\begin{pspicture}(3,2)
  \pnode(0,1){A}  \pnode(3,1){B}
  \filter(A)(B){BPF}
\end{pspicture}
\end{LTXexample}

\xLcs{filter}
\begin{LTXexample}[width=3.5cm,rframe={}]
\begin{pspicture}(3,2)
  \pnode(0,1){A}  \pnode(3,1){B}
  \filter[dipolestyle=lowpass,fillstyle=solid,%
    fillcolor=red](A)(B){LPF}
\end{pspicture}
\end{LTXexample}

\xLcs{filter}\xLkeyset{dipolestyle=highpass}
\begin{LTXexample}[width=3.5cm,rframe={}]
\begin{pspicture}(3,2)
  \pnode(0,1){A}  \pnode(3,1){B}
  \filter[dipolestyle=highpass,dipoleinput=right,
    inputarrow=true](A)(B){HPF}
\end{pspicture}
\end{LTXexample}

\begin{LTXexample}[width=3.5cm,rframe={}]
\begin{pspicture}(3,2)
  \pnode(0,1){A}  \pnode(3,1){B}
  \filter[dipolestyle=highpass,inputarrow=true](A)(B){BPF}
\end{pspicture}
\end{LTXexample}

\subsubsection{Isolator}
\begin{description}
  \item[\Lkeyword{inputarrow}:] \false | \true
  \item[\Lkeyword{dipoleinput}:] \Lkeyval{left} | \Lkeyval{right}
\end{description}

\begin{LTXexample}[width=3.5cm,rframe={}]
\begin{pspicture}(3,2)
  \pnode(0,1){A}  \pnode(3,1){B}
  \isolator[inputarrow=true](A)(B){}
\end{pspicture}
\end{LTXexample}

\begin{LTXexample}[width=3.5cm,rframe={}]
\begin{pspicture}(3,2)
\pnode(0,1){A}  \pnode(3,1){B}
\isolator[dipoleinput=right,inputarrow=true,
  fillstyle=solid,fillcolor=yellow](A)(B){Isolator}
\end{pspicture}
\end{LTXexample}

\begin{LTXexample}[width=3.5cm,rframe={}]
\begin{pspicture}(3,2)
  \pnode(0,1){A}\pnode(3,1){B}
  \isolator[dipoleinput=left](A)(B){}
\end{pspicture}
\end{LTXexample}

\subsubsection{Frequency multiplier/divider}
\begin{description}
  \item[\Lkeyword{dipolestyle}:] \Lkeyval{multiplier} | \Lkeyval{divider}
  \item[\Lkeyword{value}:] \Lkeyval{N} | $n\in N$
  \item[\Lkeyword{programmable}:] \false | \true
  \item[\Lkeyword{inputarrow}:] \false | \true
  \item[\Lkeyword{dipoleinput}:] \Lkeyval{left} | \Lkeyval{right}
\end{description}

\begin{LTXexample}[width=3.5cm,rframe={}]
\begin{pspicture}(3,2)
  \pnode(0,1){A}\pnode(3,1){B}
  \freqmult[dipolestyle=divider,inputarrow=true](A)(B){}
\end{pspicture}
\end{LTXexample}

\begin{LTXexample}[width=3.5cm,rframe={}]
\begin{pspicture}(3,2)
  \pnode(0,1){A}\pnode(3,1){B}
  \freqmult[dipolestyle=multiplier,value=10](A)(B){}
\end{pspicture}
\end{LTXexample}

\begin{LTXexample}[width=3.5cm,rframe={}]
\begin{pspicture}(3,3)
\pnode(0,1.5){A}\pnode(3,1.5){B}
\freqmult[dipolestyle=multiplier,programmable=true,
  labeloffset=-1,dipoleinput=right,inputarrow=true,
  fillstyle=solid,fillcolor=green](A)(B){10<N<35}
\end{pspicture}
\end{LTXexample}

\subsubsection{Phase shifter}
\begin{description}
  \item[\Lkeyword{inputarrow}:] \false | \true
  \item[\Lkeyword{dipoleinput}:] \Lkeyval{left} | \Lkeyval{right}
\end{description}

\begin{LTXexample}[width=3.5cm,rframe={}]
\begin{pspicture}(3,2)
  \pnode(0,1){A1}  \pnode(3,1){A2}
  \phaseshifter(A1)(A2){}
\end{pspicture}
\end{LTXexample}

\begin{LTXexample}[width=3.5cm,rframe={}]
\begin{pspicture}(3,2)
\pnode(0,1){B1}  \pnode(3,1){B2}
\phaseshifter[inputarrow=true,dipoleinput=right,
  fillstyle=solid,fillcolor=red](B1)(B2){90$^\circ$}
\end{pspicture}
\end{LTXexample}

\subsubsection{VCO}
\begin{description}
  \item[\Lkeyword{inputarrow}:] \false | \true
  \item[\Lkeyword{dipoleinput}:] \Lkeyval{left} | \Lkeyval{right}
\end{description}

\begin{LTXexample}[width=3.5cm,rframe={}]
\begin{pspicture}(3,2)
  \pnode(0,1){A1}  \pnode(3,1){A2}
  \vco[fillstyle=solid,fillcolor=yellow](A1)(A2){}
\end{pspicture}
\end{LTXexample}

\begin{LTXexample}[width=3.5cm,rframe={}]
\begin{pspicture}(3,2)
  \pnode(0,1){B1}  \pnode(3,1){B2}
  \vco[dipoleinput=right,inputarrow=true](B1)(B2){VCO}
\end{pspicture}
\end{LTXexample}

\subsubsection{Amplifier}
\begin{description}
  \item[\Lkeyword{inputarrow}:] \false | \true
  \item[\Lkeyword{dipoleinput}:] \Lkeyval{left} | \Lkeyval{right}
\end{description}

\begin{LTXexample}[width=3.5cm,rframe={}]
\begin{pspicture}(3,2)
  \pnode(0,1){A}  \pnode(3,1){B}
  \amplifier[inputarrow=true](A)(B){}
\end{pspicture}
\end{LTXexample}

\begin{LTXexample}[width=3.5cm,rframe={}]
\begin{pspicture}(3,2)
  \pnode(0,1){A}  \pnode(3,1){B}
  \amplifier[dipoleinput=right,inputarrow=true](A)(B){PA}
\end{pspicture}
\end{LTXexample}

\begin{LTXexample}[width=3.5cm,rframe={}]
\begin{pspicture}(3,2)
  \pnode(0,1){A}  \pnode(3,1){B}
  \amplifier[dipoleinput=left](A)(B){LNA}
\end{pspicture}
\end{LTXexample}

\subsubsection{Detector}
\begin{description}
  \item[\Lkeyword{inputarrow}:] \false | \true
  \item[\Lkeyword{dipoleinput}:] \Lkeyval{left} | \Lkeyval{right}
\end{description}

\begin{LTXexample}[width=3.5cm,rframe={}]
\begin{pspicture}(3,2)
  \pnode(0,1){A}  \pnode(3,1){B}
  \detector[inputarrow=true](A)(B){}
\end{pspicture}
\end{LTXexample}

\begin{LTXexample}[width=3.5cm,rframe={}]
\begin{pspicture}(3,2)
  \pnode(0,1){A}  \pnode(3,1){B}
  \detector[dipoleinput=right,inputarrow=true](A)(B){}
\end{pspicture}
\end{LTXexample}

\begin{LTXexample}[width=3.5cm,rframe={}]
\begin{pspicture}(3,2)
  \pnode(0,1){A}  \pnode(3,1){B}
  \detector[dipoleinput=left](A)(B){}
\end{pspicture}
\end{LTXexample}

\begin{LTXexample}[width=3.5cm,rframe={}]
\begin{pspicture}(3,2)
  \pnode(0,1){A}  \pnode(3,1){B}
  \attenuator[inputarrow,labeloffset=0.7cm, 
    dipoleinput=left](A)(B){Attentuator}
\end{pspicture}
\end{LTXexample}


\subsection{New tripole macro-components}
\subsubsection{Mixer}
\begin{description}
  \item[\Lkeyword{tripolestyle}:] \Lkeyval{bottom} | \Lkeyval{top}
  \item[\Lkeyword{tripoleconfig}:] \Lkeyval{left} | \Lkeyval{right}
  \item[\Lkeyword{inputarrow}:] \false | \true
\end{description}

\begin{LTXexample}[width=3.5cm,rframe={}]
\begin{pspicture}(3,2)
  \pnode(0.5,1){A}\pnode(2.5,1){B}\pnode(1.5,2){C}
  \mixer[tripolestyle=top,inputarrow=true](A)(B)(C)%
    {Mixer}{}
\end{pspicture}
\end{LTXexample}

\begin{LTXexample}[width=3.5cm,rframe={}]
\begin{pspicture}(3,2)
  \pnode(0.5,1){A}\pnode(2.5,1){B}\pnode(1.5,0){C}
  \mixer[inputarrow=true,tripoleinput=right](A)(B)(C)
    {Mixer}{fillstyle=solid,fillcolor=yellow}
\end{pspicture}
\end{LTXexample}

\subsubsection{Splitter}
\begin{description}
  \item[\Lkeyword{tripolestyle}:] \Lkeyval{bottom} | \Lkeyval{top}
  \item[\Lkeyword{tripoleconfig}:] \Lkeyval{left} | \Lkeyval{right}
  \item[\Lkeyword{inputarrow}:] \false | \true
\end{description}


\begin{LTXexample}[width=3.5cm,rframe={}]
\begin{pspicture}(3,2)
\pnode(0.5,1){A}\pnode(2.5,1){B}\pnode(1.5,2){C}
  \splitter[inputarrow,
    tripolestyle=top](A)(B)(C){Splitter}{}
\end{pspicture}
\end{LTXexample}

\begin{LTXexample}[width=3.5cm,rframe={}]
\begin{pspicture}(3,2)
\pnode(0.5,1){A}\pnode(2.5,1){B}\pnode(1.5,0){C}
  \splitter[inputarrow,
    tripolestyle=bottom,tripoleinput=right, fillstyle=solid, fillcolor=ForestGreen](A)(B)(C){Splitter}{}
\end{pspicture}
\end{LTXexample}


\subsubsection{Circulator}
\begin{description}
  \item[\Lkeyword{tripolestyle}:] \Lkeyval{circulator} | \Lkeyval{isolator}
  \item[\Lkeyword{inputarrow}:] \false | \true
  \item[\Lkeyword{tripoleinput}:] \Lkeyval{left} | \Lkeyval{right}
\end{description}

\begin{LTXexample}[width=3.5cm,rframe={}]
\begin{pspicture}(3,2)
  \pnode(0.5,1){A}\pnode(2.5,1){B}\pnode(1.5,0){C}
  \circulator{0}(A)(B)(C){Circulator}{}
\end{pspicture}
\end{LTXexample}

\begin{LTXexample}[width=3.5cm,rframe={}]
\begin{pspicture}(3,3)
  \pnode(1.5,0.5){A}\pnode(1.5,2.5){B}\pnode(0.5,1.5){C}
  \circulator[tripolestyle=isolator,inputarrow=true]{90}%
    (A)(B)(C){Isolator}{}
\end{pspicture}
\end{LTXexample}

\begin{LTXexample}[width=3.5cm,rframe={}]
\begin{pspicture}(3,2)
\pnode(0.5,1){A}\pnode(2.5,1){B}\pnode(1.5,0){C}
\circulator[tripoleconfig=right,tripolestyle=isolator,
    inputarrow=true,tripoleinput=right]{0}%
    (B)(A)(C){Isolator}{}
\end{pspicture}
\end{LTXexample}

\begin{LTXexample}[width=3.5cm,rframe={}]
\begin{pspicture}(3,2)
\pnode(0.5,1){A}\pnode(2.5,1){B}\pnode(1.5,2){C}
\circulator[tripoleconfig=right,
  inputarrow=true]{180}(A)(B)(C){Isolator}%
    {fillstyle=solid,fillcolor=red}
\end{pspicture}
\end{LTXexample}

\subsubsection{Agc}
\begin{description}
  \item[\Lkeyword{inputarrow}:] \false | \true
  \item[\Lkeyword{tripoleinput}:] \Lkeyval{left} | \Lkeyval{right}
\end{description}

\begin{LTXexample}[width=3.5cm,rframe={}]
\begin{pspicture}(3,2)
  \pnode(0.5,1){A}\pnode(2.5,1){B}\pnode(1.5,0){C}
  \agc(A)(B)(C){AGC}{fillstyle=solid,fillcolor=yellow}
\end{pspicture}
\end{LTXexample}

\begin{LTXexample}[width=3.5cm,rframe={}]
\begin{pspicture}(3,2)
  \pnode(0.5,1){A}\pnode(2.5,1){B}\pnode(1.5,0){C}
  \agc[tripoleinput=right,inputarrow=true](A)(B)(C)%
    {AGC}{fillstyle=solid,fillcolor=blue}
\end{pspicture}
\end{LTXexample}




\subsection{New quadripole macro-components}
\subsubsection{Coupler}
\begin{description}
  \item[\Lkeyword{couplerstyle}:] \Lkeyval{hybrid} | \Lkeyval{directional}
  \item[\Lkeyword{inputarrow}:] \false | \true
  \item[\Lkeyword{quadripoleinput}:] \Lkeyval{left} | \Lkeyval{right}
\end{description}

\begin{LTXexample}[width=3.5cm,rframe={}]
\begin{pspicture}(3,2)
  \pnode(0,1.4){A} \pnode(0,0.6){B}
  \pnode(3,1.4){C} \pnode(3,0.6){D}
  \coupler[couplerstyle=hybrid,inputarrow=true](A)(B)(C)(D)%
    {Hyb. $180$\ensuremath{^\circ}}%
    {fillstyle=solid,fillcolor=yellow}
\end{pspicture}
\end{LTXexample}

\begin{LTXexample}[width=3.5cm,rframe={}]
\begin{pspicture}(3,2)
  \pnode(0,1.4){A} \pnode(0,0.6){B}
  \pnode(3,1.4){C} \pnode(3,0.6){D}
  \coupler[couplerstyle=directional](A)(B)(C)(D){10~dB}{}
\end{pspicture}
\end{LTXexample}

\begin{LTXexample}[width=3.5cm,rframe={}]
\begin{pspicture}(3,2)
  \pnode(0,1.4){A} \pnode(0,0.6){B}
  \pnode(3,1.4){C} \pnode(3,0.6){D}
  \coupler[couplerstyle=hybrid,inputarrow=true,%
    quadripoleinput=right](A)(B)(C)(D)%
    {Hyb. $180$\ensuremath{^\circ}}{}
\end{pspicture}
\end{LTXexample}

\begin{LTXexample}[width=3.5cm,rframe={}]
\begin{pspicture}(3,2)
  \pnode(0,1.4){A} \pnode(0,0.6){B}
  \pnode(3,1.4){C} \pnode(3,0.6){D}
  \coupler[couplerstyle=directional,quadripoleinput=right,%
  inputarrow=true](A)(B)(C)(D){10~dB}{}
\end{pspicture}
\end{LTXexample}


\subsection{Examples}

\subsubsection{Radar emission diagram}
\begin{pspicture}[labelangle=:U](8,5)
  \pnodes(1, 3){A}(1.8, 3){B}(3.5, 3){C}(5, 3){D}(7, 3){E}(7.1, 3){EE}(6, 2){F}(5, 1){G}%
    (7, 1){H}(7.1, 1){HH}(4, 1){I}(3.5, 1){J}
  \rput(1, 3.9){\SI{3.57}{\decibel}}
  \rput(1, 4.4){\SI{2.4}{\giga\hertz}}
  \oscillator[output=right,fillstyle=solid,fillcolor=WildStrawberry](A){}{}
  \attenuator[dipoleinput=left, labeloffset=0.9, inputarrow=true](B)(C){\SI{-3}{\decibel}}
  \amplifier[dipoleinput=left, labeloffset=0.9, inputarrow=true](C)(D){+\SI{14}{\decibel}}
  \splitter[tripolestyle=bottom,inputarrow=true, tripoleinput=left,fillstyle=solid,fillcolor=Magenta](D)(E)(F){}{}
  \psline[arrowinset=0]{->}(E)(EE)
  \mixer[tripolestyle=top,inputarrow=true, tripoleinput=right, labeloffset=0.9,fillstyle=solid,fillcolor=red](G)(H)(F){}{}
  \wire(H)(HH)
  \rput(7.8, 1.25){\SI{13}{\dbm}}
  \rput(7.8, 0.75){\SI{900}{\mega\hertz}}
  \attenuator[dipoleinput=left,labeloffset=-0.9,linecolor=blue](I)(G){\textcolor{blue}{\SI{-10}{\decibel}}}
  \psline[arrowinset=0]{->}(I)(J)
  \rput(6, 3.9){\SI{-4}{\decibel}}
  \rput(1.75, 1.25){\SI{3.3}{\mega\hertz} @\SI{-10}{\dbm}}
  \rput(1.75, 0.75){\SI{1.5}{\mega\hertz} @\SI{-10}{\dbm}}
\end{pspicture}

\subsubsection{Radiometer block diagram example}
From Chang, K., RF and Microwave Wireless Systems, Wiley InterScience, page 319, ISBN 0-471-35199-7

\noindent
\resizebox{\linewidth}{!}{%
\begin{pspicture}(1,1)(19,9)
  \pnode(2,8){A}
  \antenna{90}(A)
  \rput(4,8){\rnode{B}{\psframebox{\begin{tabular}{c}Ferrite\\Switch\end{tabular}}}}
  \ncline{A}{B}
  %%% Branche Calibration
  \pnode(4,6){C}
  \pnode(4,4){D}
  \pnode(5,5){E}
  \circulator[tripolestyle=isolator,tripoleconfig=right]{90}(C)(D)(E){Isolator}{}
  \ncline{B}{C}
  \pnode(3,3){F}
  \pnode(5,3){G}
  \resistor[unit=0.5,dipolestyle=zigzag,variable=true](F)(G){}
  \pnode(4,3){H}
  \ncline{D}{H}
  \rput[t](4,2.75){%
    \begin{tabular}{c}
        Hot and Cold\\
        loads for calibration
        \end{tabular}}
  %%% Branche reception
  \pnode(6,8){R1}
  \pnode(8,8){R2}
  \pnode(7,7){X1}
  \circulator[tripolestyle=isolator,tripoleconfig=right]{180}(R1)(R2)(X1){Isolator}{}
  \ncline{B}{R1}
  \pnode(10,8){R3}
  \pnode(9,7){X2}
  \mixer[inputarrow=true](R2)(R3)(X2){Mixer}{}
  \pnode(9,6){X3}
  \oscillator[output=top](X3){LO}{}
  \pnode(12,8){R4}
  \ncline{R3}{R4}
  \naput{0.5~GHZ}
  \pnode(14,8){R5}
  \filter(R4)(R5){BPF}%
  \pnode(16,8){R6}
  \amplifier[inputarrow=true](R5)(R6){IF~Amp}
  \pnode(18,8){R7}
  \detector[inputarrow=true](R6)(R7){Detector}
  \pnode(18,4){R8}
  \amplifier[inputarrow=true,labeloffset=-1](R7)(R8){Amp}
  \pscircle[fillstyle=solid,fillcolor=white](18,4){0.1}
  \rput[t](18,3.9){%
    \begin{tabular}{c}
        Output\\
        for processing
    \end{tabular}}
\end{pspicture}}

\begin{landscape}

\subsubsection{Ku-band Transceiver}
\resizebox{\linewidth}{!}{%
\psset{unit=1cm}
\begin{pspicture}(0,-3.5)(29.25,11)
  \rput[r](1.9,8){70/140MHz}
  \pnode(2,8){N1}
  \pnode(4,8){N2}
  \amplifier[fillstyle=solid,fillcolor=Thistle,inputarrow=true](N1)(N2){IF~Amp}

  \pnode(6,8){N3}
  \pnode(5,7){D1}
  \mixer(N2)(N3)(D1){}{}

  \pnode(5,5){D2}
  \amplifier[fillstyle=solid,fillcolor=CornflowerBlue,labeloffset=-1.5](D2)(D1){L-Band Buffers}
  \pnode(3,5){D3}
  \amplifier[fillstyle=solid,fillcolor=CornflowerBlue](D3)(D2){}
  \pnode(2,5){D4}
  \oscillator[output=right](D4){VCO}{fillstyle=solid,fillcolor=Orange}
  \psframe(1.25,3)(2.75,5.75)
  \rput[b](2,3.1){\large\textbf{L-band}}
  \pnode(5,3){D5}
  \amplifier[fillstyle=solid,fillcolor=CornflowerBlue](D2)(D5){}

  \pnode(2,2){R1}
  \pnode(4,2){R2}
  \amplifier[fillstyle=solid,fillcolor=Thistle](R2)(R1){IF~Amp}
  \rput[r](1.9,2){70/140MHz}
  \pnode(6,2){R3}
  \filter(R3)(R2){}
  \pnode(8,2){R4}
  \pnode(7,3){D6}
  \mixer[tripolestyle=top](R3)(R4)(D6){}{}
  \ncline{D5}{D6}

  \pnode(8,8){N4}
  \filter(N3)(N4){}
  \pnode(10,8){N5}
  \amplifier[fillstyle=solid,fillcolor=NavyBlue](N4)(N5){L-band Amp}
  \pnode(12,8){N6}
  \resistor[unit=0.5,dipolestyle=zigzag,variable=true,labeloffset=-0.8](N5)(N6){RF Atten}
  \pnode(11,8){U0}

  \pnode(14,8){N7}
  \amplifier[fillstyle=solid,fillcolor=NavyBlue](N6)(N7){}
  \pnode(16,8){N8}
  \pnode(15,7){D7}
  \mixer(N7)(N8)(D7){Mixer}{fillstyle=solid,fillcolor=BurntOrange}
  \pnode(18,8){N9}
  \filter(N8)(N9){}
  \pnode(20,8){N10}
  \amplifier[fillstyle=solid,fillcolor=NavyBlue,labeloffset=-0.8](N9)(N10){L-band Amp}
  \pnode(22,8){N11}
  \pnode(21,7){X1}
  \mixer(N10)(N11)(X1){Mixer}{fillstyle=solid,fillcolor=BurntOrange}
  \pnode(24,8){N12}
  \filter(N11)(N12){}
  \pnode(26,8){N13}
  \amplifier[fillstyle=solid,fillcolor=RubineRed,labeloffset=-0.8](N12)(N13){Ku-band Amp}

  \pnode(18,10){U2}
  \pnode(20,10){U3}
  \detector[fillstyle=solid,fillcolor=NavyBlue,dipoleinput=right](U2)(U3){Det L-Band}
  \pnode(11,10){U1}
  \ncline{U2}{U1}
  \ncline{U1}{U0}
  \pnode(24,10){U4}
  \pnode(26,10){U5}
  \detector[fillstyle=solid,fillcolor=RubineRed,dipoleinput=right](U4)(U5){Det Ku-Band}
  \ncline{U4}{U3}
  \ncline{U5}{N13}

  \pnode(15,5){D8}
  \amplifier[fillstyle=solid,fillcolor=CornflowerBlue,labeloffset=-1.5](D8)(D7){L-Band Buffers}
  \pnode(13,5){D9}
  \amplifier[fillstyle=solid,fillcolor=CornflowerBlue](D9)(D8){}
  \pnode(13,4){D10}
  \ncline{D9}{D10}
  \pnode(11,4){D11}
  \vco[fillstyle=solid,fillcolor=Orange](D11)(D10){VCO}
  \rput(10,4){\rnode{D12}{\psframebox{\textbf{PLL}}}}
  \ncline{D11}{D12}
  \pnode(10,6){D13}
  \ncline{D12}{D13}
  \pnode(11,6){D14}
  \ncline{D13}{D14}
  \pnode(13,6){D15}
  \freqmult[fillstyle=solid,fillcolor=Goldenrod,dipolestyle=divider](D14)(D15){Prescaler}
  \ncline{D15}{D9}
  \psframe(9.5,3.25)(13.1,7)
  \rput[tl](9.7,6.8){\large\textbf{L-Band}}

  \pnode(10,2){R5}
  \amplifier[fillstyle=solid,fillcolor=NavyBlue](R5)(R4){L-band Amp}
  \pnode(12,2){R6}
  \resistor[unit=0.5,dipolestyle=zigzag,variable=true,labeloffset=-0.8](R5)(R6){RF Atten}
  \pnode(14,2){R7}
  \amplifier[fillstyle=solid,fillcolor=NavyBlue](R7)(R6){L-band Amp}
  \pnode(16,2){R8}
  \filter(R8)(R7){}
  \pnode(18,2){R9}
  \pnode(17,3){D17}
  \mixer[tripolestyle=top](R8)(R9)(D17){Mixer}{fillstyle=solid,fillcolor=BurntOrange}

  \pnode(15,3){D16}
  \amplifier[fillstyle=solid,fillcolor=CornflowerBlue](D8)(D16){}
  \ncline{D16}{D17}

  \pnode(20,2){R10}
  \amplifier[fillstyle=solid,fillcolor=NavyBlue](R10)(R9){L-band Amp}
  \pnode(22,2){R11}
  \amplifier[fillstyle=solid,fillcolor=NavyBlue](R11)(R10){L-band Amp}
  \pnode(24,2){R12}
  \filter(R12)(R11){}
  \pnode(26,2){R13}
  \pnode(25,1){R15}
  \mixer(R12)(R13)(R15){Mixer}{fillstyle=solid,fillcolor=BurntOrange}
  \pnode(28,2){R14}
  \amplifier[fillstyle=solid,fillcolor=Purple](R14)(R13){Ku-band LNA}

  \pnode(25,-1){R16}
  \amplifier[fillstyle=solid,fillcolor=OliveGreen,labeloffset=-1.6](R16)(R15){Ku-band Buffers}
  \pnode(24,-1){R17}
  \ncline{R16}{R17}
  \pnode(24,-2){R18}
  \ncline{R17}{R18}
  \pnode(22,-2){R19}
  \vco[fillstyle=solid,fillcolor=Red](R18)(R19){Ku-band}
  \rput(21,-2){\rnode{R20}{\psframebox{\textbf{PLL}}}}
  \ncline{R19}{R20}
  \pnode(21,0){R21}
  \ncline{R20}{R21}
  \pnode(22,0){R22}
  \ncline{R21}{R22}
  \pnode(24,0){R23}
  \freqmult[fillstyle=solid,fillcolor=Goldenrod,dipolestyle=divider](R22)(R23){Prescaler}
  \ncline{R23}{R17}
  \psframe(18,-3)(28.5,3)
  \rput[br](28,-2.75){\large\textbf{LNB}}
  \rput[bl](18,3.1){%
    \begin{tabular}{l}
    \textbf{950-1540 MHz}\\
    \textbf{900-1700 MHz}
    \end{tabular}}
  \cnode(29,2){.1}{S2}
  \ncline{R14}{S2}

  \pnode(21,5.5){X2}
  \ncline{X1}{X2}
  \pnode(24,5.5){X3}
  \amplifier[fillstyle=solid,fillcolor=OliveGreen](X3)(X2){Ku-band Buffers}
  \pnode(24,6.25){X4}
  \ncline{X3}{X4}
  \pnode(26,6.25){X5}
  \freqmult[fillstyle=solid,fillcolor=Goldenrod,dipolestyle=divider,labeloffset=-0.7](X4)(X5){Prescaler}
  \pnode(27,6.25){X6}
  \ncline{X5}{X6}
  \rput(27,4.75){\rnode{X7}{\psframebox{\textbf{PLL}}}}
  \ncline{X6}{X7}
  \pnode(26,4.75){X8}
  \ncline{X7}{X8}
  \pnode(24,4.75){X9}
  \vco[fillstyle=solid,fillcolor=Red](X8)(X9){Ku-band}
  \ncline{X9}{X3}
  \psframe(23.75,3.25)(28.5,7)
  \rput[br](28,3.5){\large\textbf{Ku-band}}

  \pnode(28.5,8){N14}
  \amplifier[fillstyle=solid,fillcolor=RubineRed](N13)(N14){}
  \cnode(29,8){.1}{S1}
  \ncline{N14}{S1}
  \psframe(26.25,7.25)(28.5,10)
  \rput[t](27.375,9.75){\large \textbf{SSPA}}
 
  \rput[lt](2,0){\large%
    \begin{tabular}{l}
        \textbf{Tx/GHz: 13.75-14.00, 14.00-14.50}\\
        \textbf{Rx/GHz: 10.95-11.70, 11.20-11.70, 11.70-12.20, 12.25-12.75}
    \end{tabular}}
\end{pspicture}}

\subsubsection{Circuit to harvest Solar Energy}
\resizebox{\linewidth}{!}{%
\begin{pspicture}[labelangle=:U, showgrid=false](40,10)
  \pnodes(1.75, 1){A}(3, 1){B}(3, 3.5){C}(4, 3.5){D}(6, 3.5){E}(6, 4.5){F}(3, 5.5){G}(6, 6){H}%
    (2.5, 6.5){I}(2.5, 8.5){J}(0.5, 6.5){K}(0, 6.5){K1}(8.5, 6){L}(8.5, 5.5){L1}(8.5, 7){M}%
    (8.5, 9){N}(8.5, 3.5){O}(7.25, 3.5){P}(11, 6){Q}(11, 4){R}(11, 3.5){S}(13, 6){T}%
    (13, 5){U}(16, 5.5){V}(13, 3.5){W}(13, 1){X}(11.75, 1){X1}(17.5, 3.5){Y}(17.5, 5.5){Z}%
    (20, 5.5){AA}(20, 6.5){AA1}(20, 5){AA2}(17.5, 9){BB}(22.5, 5.5){CC}(20, 3){DD}(20, 8.5){EE}%
    (22.5, 3.5){FF}(22.5, 3){FF1}(18.75, 3){GG}(25, 5.5){HH}(25, 4.5){II}(28, 5){JJ}(25, 3){KK}%
    (25, 0.5){LL}(23.75, 0.5){MM}(27, 3){NN}(29.5, 3){OO}(29.5, 5){PP}(29.5, 8.5){QQ}(31.5, 5){RR}(32.5, 5){SS}
%
  \newground [groundstyle=triangle]{180}(A)  \rput(1.75, 2.2){+5}
  \wire(A)(B)
  \cell(B)(C){}
  \wire[linecolor=blue](C)(D)
  \potentiometer [dipolestyle =zigzag ,labelangle =:U, labeloffset=-0.6, linecolor=blue](D)(E)(F){\textcolor{blue}{$\SI{10}{\kilo\ohm}$}}
  \wire[arrows=-*, linecolor=blue](E)(F)  %$
  \wire(C)(G)
  \wire[arrows=-*](F)(H)
  \OA[OAperfect=false, OAinvert=false](I)(G)(H)
  \cell(I)(J){}
  \capacitor[labelangle =0, labeloffset=-0.8](K)(I){$\SI{1}{\micro\farad}$}
  \wire[arrows=-o](K)(K1)
  \psarc(0, 6.5){0.15}{90}{-90}
  \newground(0, 6.35)
  \newground [groundstyle=triangle]{180}(J)
  \resistor[arrows=-*, dipolestyle =zigzag ,labelangle =0, labeloffset=0.6](H)(L){$\SI{8.45}{\kilo\ohm}~ 1\%$}
  \wire(L)(M)
  \capacitor[labelangle =0, labeloffset=-0.8](M)(N){$\SI{1}{\nano\farad}$}
  \wire(L)(L1)
  \resistor[arrows=*-, dipolestyle =zigzag ,labelangle =0, labeloffset=-1.2](O)(L1){$\SI{102}{\kilo\ohm}~ 1\%$}
  \wire(O)(P)
  \newground [groundstyle=triangle]{180}(P)
  \resistor[arrows=-*, dipolestyle =zigzag ,labelangle =0, labeloffset=0.6](L)(Q){$\SI{7.15}{\kilo\ohm}~ 1\%$}
  \capacitor[labelangle =0, labeloffset=-0.8](R)(Q){$\SI{1}{\nano\farad}$}
  \wire(O)(S)\wire(S)(R)\wire(Q)(T)
  \OA[OAperfect=false, OAinvert=false, OApower = true](T)(U)(V)
  \rput(2.5, 9.7){+5}
  \rput(7.2, 4.7){+5}
  \newground [groundstyle=triangle]{180}(14.5, 6)
  \rput(14.5, 7.2){+12}
  \newground(14.5, 5)
  \wire[arrows = -*](U)(W)
  \cell(X)(W){}
  \wire(X)(X1)
  \newground [groundstyle=triangle]{180}(X1)
  \rput(11.75, 2.3){+5}
  \wire(W)(15, 3.5)
  \resistor[dipolestyle =zigzag](15, 3.5)(Y){$\SI{1}{\kilo\ohm}~ 1\%$}
  \wire[arrows = -*](Y)(Z)
  \wire(V)(Z)
  \resistor[dipolestyle =zigzag, arrows =-*](Z)(AA){$\SI{17.4}{\kilo\ohm}~ 1\%$}
  \wire(N)(BB)
  \wire(BB)(Z)
  \resistor[dipolestyle =zigzag, arrows =-*](AA)(CC){$\SI{4.12}{\kilo\ohm}~ 1\%$}
  \wire(AA)(AA2)
  \resistor[dipolestyle =zigzag ,labelangle =0, labeloffset=1.1, arrows =-*](AA2)(DD){$\SI{28}{\kilo\ohm}~ 1\%$}
  \wire(AA)(AA1)
  \capacitor[labelangle =0, labeloffset=-0.8](AA1)(EE){$\SI{1}{\nano\farad}$}
  \capacitor[labelangle =0, labeloffset=0.8](CC)(FF){$\SI{1}{\nano\farad}$}
  \wire(FF)(FF1)
  \wire(FF1)(GG)
  \newground [groundstyle=triangle]{180}(GG)
  \rput(18.75, 4.2){+5}
	\wire(CC)(HH)
	\OA[OAperfect=false, OAinvert=false](HH)(II)(JJ)
	\wire[arrows=-*](II)(KK)
	\cell(LL)(KK){}
	\wire(LL)(MM)
	\newground [groundstyle=triangle]{180}(MM)
	\rput(23.75, 1.7){+5}
	\wire(KK)(NN)
	\resistor[dipolestyle =zigzag](NN)(OO){$\SI{1}{\kilo\ohm}~ 1\%$}
	\wire(OO)(PP)
	\wire[arrows = -*](JJ)(PP)
	\wire(EE)(QQ)
	\wire(QQ)(PP)
	\resistor[dipolestyle =zigzag](PP)(RR){$\SI{47}{\kilo\ohm}$}
	\wire[arrows=-o](RR)(SS)
	%% OP AMP PINS
	\rput(3.2, 6.7){\texttt{10}}
	\rput(3.2, 5.7){\texttt{9}}
	\rput(5.7, 6.2){\texttt{8}}
	\rput(13.2, 6.2){\texttt{3}}
	\rput(13.2, 5.2){\texttt{2}}
	\rput(15.7, 5.7){\texttt{1}}
	\rput(14.7, 6.2){\texttt{4}}
	\rput(14.7, 4.8){\texttt{11}}
	\rput(25.2, 5.7){\texttt{5}}
	\rput(25.2, 4.7){\texttt{6}}
	\rput(27.7, 5.2){\texttt{7}}
\end{pspicture}}

\subsubsection{Amplificator for hearing aid}
\scalebox{0.7}{%
\begin{pspicture}[showgrid=false](-0.5, -0.5)(25,10)
	\pnodes(0, 1){A}(-0.1, 2){BC}(0, 2){B}(2, 1.5){C}(6, 1.5){D}(4, 4.5){E}(0, 7.5){F}%
	(6, 7.5){G}(0, 5){AA}(0, 6){BB}(2, 5.5){CC}(2.5, 5.5){DD}(2.5, 3.5){EE}(6, 5){H}%
	(7.5, 5){HH}(7.5, 3){HK}(9, 4){I}(9, 5){II}(11.5, 4.5){J}(9, 6.5){K}(11.5, 6.5){KL}%
	(13, 3.5){L}(15, 4){M}(19, 4){N}(19, 7){O}(13, 7){LL}(13, 8){JJ}(13, 9.5){PP}%
	(15, 7.5){MM}(15.5, 7.5){NN}(15.5, 6){OO}(19, 9.5){PQ}(17, 6.5){QQ}(19, 7){RR}%
	(20.5, 7){P}(20.5, 5.5){Q}(22, 7){R}(22, 6){S}(24.5, 6.5){T}(22, 8.5){U}(24.5, 8.5){V}%
	(25, 6.5){TT}
	%%%%%%%%%%%%%%%%%%%%%%%%%%%%%%%%%%%%%%%%%%%%%%%%%%%%%%
	% Première cellule	
	
	\wire[arrows=o-](BC)(B)
	\GM[GMinvert=false](B)(A)(C)
		\newground[connectingdot=false, groundstyle=triangle](A)
	\wire(C)(D)
	\wire(BB)(F)
		\wire(F)(G)
			\wire[arrows=-*](G)(H)
	\GM[GMinvert=false](BB)(AA)(CC)
		\newground[connectingdot=false, groundstyle=triangle](AA)
	\capacitor[arrows=*-, labeloffset=0.9](DD)(EE){$C_{A1}$}
		\newground[connectingdot=false, groundstyle=triangle](EE)

	\GM[GMinvert=true](CC)(E)(H)
		\newground[connectingdot=false, groundstyle=triangle](E)

	\wire(D)(H)
	
	\capacitor[arrows=*-, labeloffset=0.9](HH)(HK){$C_{B1}$}
		\newground[connectingdot=false, groundstyle=triangle](HK)
	
	\GM[GMinvert=true](H)(I)(J)
		\newground[connectingdot=false, groundstyle=triangle](I)
	\wire[arrows=*-](II)(K)
		\wire(K)(KL)
			\wire[arrows=-*](KL)(J)
	% Seconde cellule	
	
	\GM[GMinvert=false](J)(L)(M)
		\newground[connectingdot=false, groundstyle=triangle](L)
	\GM[GMinvert=false](J)(L)(M)
	\newground[connectingdot=false, groundstyle=triangle](L)
	\wire(M)(N)
		\wire(N)(O)
	
	\GM[GMinvert=false](JJ)(LL)(MM)
		\newground[connectingdot=false, groundstyle=triangle](LL)
	\capacitor[arrows=*-, labeloffset=0.9](NN)(OO){$C_{A2}$}
		\newground[connectingdot=false, groundstyle=triangle](OO)
	\wire(JJ)(PP)
		\wire(PP)(PQ)
			\wire[arrows=-*](PQ)(RR)
	\GM[GMinvert=false](MM)(QQ)(RR)
		\newground[connectingdot=false, groundstyle=triangle](QQ)
	
	\capacitor[arrows=*-, labeloffset=0.9](P)(Q){$C_{B2}$}
		\newground[connectingdot=false, groundstyle=triangle](Q)
		
	\GM[GMinvert=false](RR)(S)(T)
		\newground[connectingdot=false, groundstyle=triangle](S)
	\wire[arrows=*-](R)(U)
		\wire(U)(V)
			\wire[arrows=-*](V)(T)	
			\wire[arrows=-o](T)(TT)	
			
			
	
	%%%%%%%%%%%
	\rput[B](1, 0.1){$Gm_{5,1}$}
	\rput[B](1, 4.1){$Gm_{1,1}$}
	\rput[B](5, 3.6){$Gm_{2,1}$}
	\rput[B](10.25, 3.1){$Gm_{3,1}$}
    	\rput[B](14, 2.6){$Gm_{5,2}$}
    	\rput[B](14, 6.1){$Gm_{1,2}$}
    	\rput[B](18, 5.6){$Gm_{2,2}$}
    	\rput[B](23.25, 5.1){$Gm_{3,2}$}
	
	\uput{0.2}[180](BC){\textbf{$V_{IN}$}}
	\uput{0.2}[0](TT){\textbf{$V_{OUT}$}}

\end{pspicture}
}

\end{landscape}



\clearpage
\section{Flip Flops -- logical elements}

The syntax for all logical base circuits is
\begin{BDef}
\Lcs{logic}\OptArgs\OptArg*{\coord0}\Largb{label}
\end{BDef}

\noindent where the options and the origin are optional. If they are missing,
then the default options, described in the next section and the default
origin $(0,0)$ is used. The origin specifies the lower left corner
of the logical circuit.

\xLcs{logic}xLkeyword{logicType}
\begin{lstlisting}[style=code]
\logic{Demo}
\logic[logicType=and]{Demo}
\logic(0,0){Demo}
\logic[logicType=and](0,0){Demo}
\end{lstlisting}

The above four ,,different`` calls of the \Lcs{logic} macro give the
same output, because they are equivalent. 

\subsection{The Options}

\begin{longtable}{@{}>{\ttfamily}l l l@{}}
\textrm{\emph{name}} & \emph{type} & \emph{default}\\\hline
\endhead
\Lkeyword{logicShowNode} & boolean & \false \\
\Lkeyword{logicShowDot} & boolean & \false \\
\Lkeyword{logicNodestyle} & command & \emph{\textbackslash footnotesize} \\
\Lkeyword{logicSymbolstyle} & command & \emph{\textbackslash large} \\
\Lkeyword{logicSymbolpos} & value & \emph{0.5} \\
\Lkeyword{logicLabelstyle} & command & \emph{\textbackslash small} \\
\Lkeyword{logicType} & string & \Lkeyval{and} \\
\Lkeyword{logicChangeLR} & boolean & \false \\
\Lkeyword{logicWidth} & length & \emph{1.5} \\
\Lkeyword{logicHeight} & length & \emph{2.5} \\
\Lkeyword{logicWireLength} & length & \emph{0.5} \\
\Lkeyword{logicNInput} & number & \emph{2} \\
\Lkeyword{logicJInput} & number & \emph{2} \\
\Lkeyword{logicKInput} & number &\emph{2}
\end{longtable}

\subsection{Basic Logical Circuits}
At least the basic objects require a unique label name, otherwise it is
not sure, that all nodes will work well. The label may contain any
alphanumerical character and most of all symbols. But it is save
using only combinations of letters and digits. For example:
\begin{verbatim}
And0
a0
a123
12
NOT123a
\end{verbatim}

\verb|A_1| is not a good choice, the underscore may cause some problems.

\subsubsection{And}
\psset{subgriddiv=0,griddots=5,gridlabels=7pt}
\begin{LTXexample}[width=4.5cm](3,3)
  \begin{pspicture}(-1,0)(3,3)
  \logic{AND1}
  \end{pspicture}
\end{LTXexample}

\begin{LTXexample}[width=4.5cm](3,3)
  \begin{pspicture}(-0.5,0)(3,3)
  \logic[logicChangeLR]{AND2}
  \end{pspicture}
\end{LTXexample}

\xLkeyword{logicShowNode}\xLkeyset{logicType=and}\xLkeyword{logicChangeLR}\xLkeyword{logicNInput}
\xLkeyword{logicWidth}\xLkeyword{logicHeight}\xLkeyword{logicChangeLR}
\begin{LTXexample}[width=4.5cm](4,6)
  \begin{pspicture}(-0.5,0)(4,5)
  \logic[logicShowNode,%
     logicWidth=2,
     logicHeight=4,
     logicNInput=6,
     logicChangeLR](1,1){AND3}
  \end{pspicture}
\end{LTXexample}

\subsubsection{NotAnd}
\begin{LTXexample}[width=4.5cm](3,3)
  \begin{pspicture}(-0.5,0)(3,3)
  \logic[logicType=nand,
     logicShowNode]{NAND1}
  \end{pspicture}
\end{LTXexample}


\begin{LTXexample}[width=4.5cm](3,3)
  \begin{pspicture}(-0.5,0)(3,3)
  \logic[logicType=nand,
     logicChangeLR]{NAND2}
  \end{pspicture}
\end{LTXexample}

\xLkeyword{logicShowNode}\xLkeyset{logicType=nand}\xLkeyword{logicChangeLR}\xLkeyword{logicNInput}
\xLkeyword{logicWidth}\xLkeyword{logicHeight}\xLkeyword{logicChangeLR}
\begin{LTXexample}[width=4.5cm](4,6)
  \begin{pspicture}(4,5)
  \logic[logicType=nand,
     logicShowNode,
     logicWidth=2,
     logicHeight=4,
     logicNInput=6,
     logicChangeLR](1,1){NAND3}
  \end{pspicture}
\end{LTXexample}

\subsubsection{Or}
\begin{LTXexample}[width=4.5cm](3,3)
  \begin{pspicture}(-0.5,0)(3,3)
  \logic[logicType=or,
     logicShowNode]{OR1}
  \end{pspicture}
\end{LTXexample}


\begin{LTXexample}[width=4.5cm](3,3)
  \begin{pspicture}(-0.5,0)(3,3)
  \logic[logicType=or,
     logicChangeLR]{OR2}
  \end{pspicture}
\end{LTXexample}


\xLkeyword{logicShowNode}\xLkeyset{logicType=or}\xLkeyword{logicChangeLR}\xLkeyword{logicNInput}
\xLkeyword{logicWidth}\xLkeyword{logicHeight}\xLkeyword{logicChangeLR}
\begin{LTXexample}[width=4.5cm](4,6)
  \begin{pspicture}(4,5)
  \logic[logicType=or,
     logicShowNode,
     logicWidth=2,
     logicHeight=4,
     logicNInput=6,
     logicChangeLR](1,1){OR3}
  \end{pspicture}
\end{LTXexample}

\clearpage
\subsubsection{Not Or}

\begin{LTXexample}[width=4.5cm](3,3)
  \begin{pspicture}(-0.5,0)(3,3)
  \logic[logicType=nor,
     logicShowNode]{NOR1}
  \end{pspicture}
\end{LTXexample}


\begin{LTXexample}[width=4.5cm](3,3)
  \begin{pspicture}(-0.5,0)(3,3)
  \logic[logicType=nor,
     logicChangeLR]{NOR2}
  \end{pspicture}
\end{LTXexample}

\xLkeyword{logicShowNode}\xLkeyset{logicType=nor}\xLkeyword{logicChangeLR}\xLkeyword{logicNInput}
\xLkeyword{logicWidth}\xLkeyword{logicHeight}\xLkeyword{logicChangeLR}
\begin{LTXexample}[width=4.5cm](4,6)
  \begin{pspicture}(4,5)
  \logic[logicType=nor,
     logicShowNode,
     logicWidth=2,
     logicHeight=4,
     logicNInput=6,
     logicChangeLR](1,1){NOR3}
  \end{pspicture}
\end{LTXexample}


\subsubsection{Not}

\begin{LTXexample}[width=4.5cm](3,3)
  \begin{pspicture}(-0.5,0)(3,3)
  \logic[logicType=not,
     logicShowNode]{NOT1}
  \end{pspicture}
\end{LTXexample}


\begin{LTXexample}[width=4.5cm](3,3)
  \begin{pspicture}(-0.5,0)(3,3)
  \logic[logicType=not,
     logicChangeLR]{NOT2}
  \end{pspicture}
\end{LTXexample}

\xLkeyword{logicShowNode}\xLkeyset{logicType=not}\xLkeyword{logicChangeLR}
\xLkeyword{logicWidth}\xLkeyword{logicHeight}\xLkeyword{logicChangeLR}
\begin{LTXexample}[width=4.5cm](4,6)
  \begin{pspicture}(4,5)
  \logic[logicType=not,
     logicShowNode,
     logicWidth=2,
     logicHeight=4,
     logicChangeLR](1,1){NOT3}
  \end{pspicture}
\end{LTXexample}

\subsubsection{Exclusive OR}

\begin{LTXexample}[width=4.5cm](3,3)
  \begin{pspicture}(-0.5,0)(3,3)
  \logic[logicType=exor,
     logicShowNode]{ExOR1}
  \end{pspicture}
\end{LTXexample}


\begin{LTXexample}[width=4.5cm](3,3)
  \begin{pspicture}(-0.5,0)(3,3)
  \logic[logicType=exor,
     logicChangeLR]{ExOR2}
  \end{pspicture}
\end{LTXexample}

\xLkeyword{logicShowNode}\xLkeyset{logicType=exor}\xLkeyword{logicChangeLR}\xLkeyword{logicNInput}
\xLkeyword{logicWidth}\xLkeyword{logicHeight}\xLkeyword{logicChangeLR}
\begin{LTXexample}[width=4.5cm](4,6)
  \begin{pspicture}(4,5)
  \logic[logicType=exor,
     logicShowNode,
     logicNInput=6,
     logicWidth=2,
     logicHeight=4,
     logicChangeLR](1,1){ExOR3}
  \end{pspicture}
\end{LTXexample}


\clearpage
\subsubsection{Exclusive NOR}

\xLkeyset{logicType=exnor}\xLkeyword{logicShowNode}
\begin{LTXexample}[width=4.5cm](3,3)
  \begin{pspicture}(-0.5,0)(3,3)
  \logic[logicType=exnor,
     logicShowNode]{ExNOR1}
  \end{pspicture}
\end{LTXexample}


\xLkeyset{logicType=exor}\xLkeyword{logicChangeLR}
\begin{LTXexample}[width=4.5cm](3,3)
  \begin{pspicture}(-0.5,0)(3,3)
  \logic[logicType=exnor,
     logicChangeLR]{ExNOR2}
  \end{pspicture}
\end{LTXexample}

\xLkeyword{logicShowNode}\xLkeyset{logicType=exor}\xLkeyword{logicChangeLR}\xLkeyword{logicNInput}
\xLkeyword{logicWidth}\xLkeyword{logicHeight}\xLkeyword{logicChangeLR}
\begin{LTXexample}[width=4.5cm](4,6)
  \begin{pspicture}(4,5)
  \logic[logicType=exnor,
     logicShowNode,
     logicNInput=6,
     logicWidth=2,
     logicHeight=4,
     logicChangeLR](1,1){ExNOR3}
  \end{pspicture}
\end{LTXexample}


\subsection{RS Flip Flop}

\xLkeyword{logicShowNode}\xLkeyset{logicType=RS}
\begin{LTXexample}[width=4.5cm](3,4.5)
  \begin{pspicture}(-1,-1)(3,3)
  \logic[logicShowNode,
     logicType=RS]{RS1}
  \end{pspicture}
\end{LTXexample}


\xLkeyword{logicShowNode}\xLkeyset{logicType=RS}\xLkeyword{logicChangeLR}
\begin{LTXexample}[width=4.5cm](3,4.5)
  \begin{pspicture}(-1,-1)(3,3)
  \logic[logicShowNode,
     logicType=RS,
     logicChangeLR]{RS2}
  \end{pspicture}
\end{LTXexample}


\subsection{D Flip Flop}

\begin{LTXexample}[width=4.5cm](3,4.5)
  \begin{pspicture}(-1,-1)(3,3)
  \logic[logicShowNode,
     logicType=D]{D1}
  \end{pspicture}
\end{LTXexample}

\xLkeyword{logicShowNode}\xLkeyset{logicType=D}\xLkeyword{logicChangeLR}
\begin{LTXexample}[width=4.5cm](3,4.5)
  \begin{pspicture}(-1,-1)(3,3)
  \logic[logicShowNode=true,
     logicType=D,
     logicChangeLR]{D2}
  \end{pspicture}
\end{LTXexample}


\subsection{JK Flip Flop}
\xLkeyword{logicShowNode}\xLkeyset{logicType=JK}\xLkeyword{logicJInput}\xLkeyword{logicKInput}
\begin{LTXexample}[width=4.5cm](3,4.5)
\begin{pspicture}(-1,-1)(3,3)
  \logic[logicShowNode,
     logicType=JK,
     logicKInput=2,
     logicJInput=2]{JK1}
\end{pspicture}
\end{LTXexample}

\xLkeyword{logicShowNode}\xLkeyset{logicType=JK}\xLkeyword{logicJInput}\xLkeyword{logicKInput}\xLkeyword{logicChangeLR}
\begin{LTXexample}[width=4.5cm](3,4.5)
  \begin{pspicture}(-1,-1)(3,3)
  \logic[logicShowNode,logicType=JK,
     logicKInput=2, logicJInput=4,
     logicChangeLR]{JK2}
  \end{pspicture}
\end{LTXexample}

\subsection{Other Options}

\xLkeyword{logicShowDot}
\begin{LTXexample}[width=3.5cm](3,3)
  \begin{pspicture}(-0.5,0)(3,2.5)
  \logic[logicShowDot]{A0}
  \end{pspicture}
\end{LTXexample}

\xLkeyword{logicWireLength}
\begin{LTXexample}[width=4.5cm](4,3)
  \begin{pspicture}(-1,0)(3,2.5)
  \logic[logicWireLength=1,
     logicShowDot]{A1}
  \end{pspicture}
\end{LTXexample}

\bigskip
The unit of \Lkeyword{logicWireLength} is the same than the actual one for pstricks, set by
the \Lkeyword{unit} option.

\subsection{The Node Names}
Every logic circuit is defined with its name, which should be a unique one.
If we have the following NAND circuit, then \LPack{pst-circ} defines the nodes
\begin{lstlisting}[style=syntax]
NAND11, NAND12, NAND13, NAND14, NAND1Q
\end{lstlisting}

\noindent If there exists an inverted output, like for alle Flip Flops,
then the negated one gets the appendix \verb|neg| to the node name. For 
example:
\begin{lstlisting}[style=syntax]
NAND1Q, NAND1Qneg
\end{lstlisting}

\begin{LTXexample}[width=3cm](3,3.5)
  \begin{pspicture}(-0.5,0)(2.5,3)
  \logic[logicShowNode=true,%
      logicLabelstyle=\footnotesize,%
      logicType=nand,%
      logicNInput=4]{NAND1}
  \multido{\n=1+1}{4}{%
     \pscircle*[linecolor=red](NAND1\n){2pt}%
  }
  \pscircle*[linecolor=blue](NAND1Q){2pt}
  \end{pspicture}
\end{LTXexample}

\vspace{0.5cm}
Now it is possible to draw a line from the output to the input 

\begin{lstlisting}[style=syntax]
\ncbar[angleA=0,angleB=180]{<Node A>}{<Node B>}
\end{lstlisting}

It may be easier to print a grid since the drawing phase and then comment it out if
all is finished.

\bigskip
\begin{LTXexample}[width=3.5cm](3,3.5)
  \begin{pspicture}(-1,-1)(2.5,3)
  \logic[logicShowNode=true,%
      logicLabelstyle=\footnotesize,%
      logicType=nand,%
      logicWireLength=1,%
      logicNInput=4]{NAND1}
      \pnode(-0.5,0|NAND11){tempA}
      \pnode(2,0|NAND1Q){tempB}
  \end{pspicture}
  \ncbar[angleA=-90,angleB=0,arm=0.75,%
      arrows=*-*, dotsize=0.15]{tempA}{tempB}
\end{LTXexample}


AN empty argument to the \Lkeyword{logicSymbolstyle} and \Lkeyword{logicLabelstyle} will suppress the
output of the symbol and/or the label. The label, of course, is a mandatory argument because it is the prefix
of the node names.



\subsection{Examples}

\begin{LTXexample}[pos=t]
   \begin{pspicture}(-1,0)(5,5)
     \psset{logicType=nor, logicLabelstyle=\normalsize,%
          logicWidth=1, logicHeight=1.5, dotsize=0.15}
     \logic(1.5,0){nor1}
     \logic(1.5,3){nor2}
     \psline(nor2Q)(4,0|nor2Q)
     \uput[0](4,0|nor2Q){$Q$}
     \psline(nor1Q)(4,0|nor1Q)
     \uput[0](4,0|nor1Q){$\overline{Q}$}
     \psline{*-}(3.50,0|nor2Q)(3.5,2.5)(1.5,2.5)
         (0.5,1.75)(0.5,0|nor12)(nor12)
     \psline{*-}(3.50,0|nor1Q)(3.5,2)(1.5,2)
         (0.5,2.5)(0.5,0|nor21)(nor21)
     \psline(0,0|nor11)(nor11)\uput[180](0,0|nor11){R}
     \psline(0,0|nor22)(nor22)\uput[180](0,0|nor22){S}
   \end{pspicture}
\end{LTXexample}

\bigskip
\begin{LTXexample}[pos=t]
  \begin{pspicture}(-4,0)(5,7)
     \psset{logicWidth=1, logicHeight=2, dotsize=0.15}
     \logic[logicWireLength=0](-2,0){A0}
     \logic[logicWireLength=0](-2,5){A1}
     \ncbar[angleA=-180,angleB=-180,arm=0.5]{A11}{A02}
     \psline[dotsize=0.15]{-*}(-3.5,3.5)(-2.5,3.5)
     \uput[180](-3.5,3.5){$T$}
     \psline(-3.5,0.5)(A01)\uput[180](-3.5,0.5){$S$}
     \psline(-3.5,6.5)(A12)\uput[180](-3.5,6.5){$R$}
     \psset{logicType=nor, logicLabelstyle=\normalsize}
     \logic(1,0.5){nor1}
     \logic(1,4.5){nor2}
     \psline(nor2Q)(4,0|nor2Q)
     \uput[0](4,0|nor2Q){$Q$}
     \psline(nor1Q)(4,0|nor1Q)
     \uput[0](4,0|nor1Q){$\overline{Q}$}
     \psline{*-}(3,0|nor2Q)(3,4)(1,4)(0,3)(0,0|nor12)(nor12)
     \psline{*-}(3,0|nor1Q)(3,3)(1,3)(0,4)(0,0|nor21)(nor21)
     \psline(A0Q)(nor11)
     \psline(A1Q)(nor22)
  \end{pspicture}
\end{LTXexample}


\clearpage

\section{Logical circuits in american style}

\begin{longtable}{ll>{\ttfamily}l}
\toprule
		\emph{macro} & \emph{option} & \emph{defaults} \\\midrule
		\endfirsthead
		\midrule
		\emph{macro} & \emph{option} & \emph{defaults} \\\midrule
		\endhead
		\midrule
		\small continued on next page $\dots$ & & \\
		\endfoot
		\midrule
		\endlastfoot
		\Lcs{logicnot} & \Lkeyword{input} & true \\
		& \Lkeyword{invertinput} & false \\
		& \Lkeyword{invertoutput} & false \\
		& \Lkeyword{iec} & false \\
		& \Lkeyword{iecinvert} & false \\
		& \Lkeyword{bubblesize} & 0.2 \\
		& \multicolumn{2}{@{}l}{\quad possible values \texttt{0.05, 0.10, 0.15, 0.20}} \\
		\midrule
		\Lcs{logicand} & \Lkeyword{ninputs} & 2 \\
		& \nxLkeyword{input?} & true \\
		& \multicolumn{2}{l}{\quad where ? = a--d} \\
		& \nxLkeyword{invertinput?} & false \\
		& \multicolumn{2}{l}{\quad where ? = a--d} \\
		& \Lkeyword{invertoutput} & false \\
		& \Lkeyword{iec} & false \\
		& \Lkeyword{iecinvert} & false \\
		& \Lkeyword{bubblesize} & 0.2 \\
		& \multicolumn{2}{@{}l}{\quad possible values \texttt{0.05, 0.10, 0.15, 0.20}} \\
		\midrule
		\Lcs{logicor} & \Lkeyword{ninputs} & 2 \\
		& \nxLkeyword{input?} & true \\
		& \multicolumn{2}{l}{\quad where ? = 1--4} \\
		& \nxLkeyword{invertinput?} & false \\
		& \multicolumn{2}{l}{\quad where ? = a--d} \\
		& \Lkeyword{invertoutput} & false \\
		& \Lkeyword{iec} & false \\
		& \Lkeyword{iecinvert} & false \\
		& \Lkeyword{bubblesize} & 0.2 \\
		& \multicolumn{2}{@{}l}{\quad possible values \texttt{0.05, 0.10, 0.15, 0.20}} \\
		\midrule
		\Lcs{logicxor} & \Lkeyword{ninputs} & 2 \\
		& \nxLkeyword{input?} & true \\
		& \multicolumn{2}{l}{\quad where ? = 1--4} \\
		& \nxLkeyword{invertinput?} & false \\
		& \multicolumn{2}{l}{\quad where ? = a--d} \\
		& \Lkeyword{invertoutput} & false \\
		& \Lkeyword{iec} & false \\
		& \Lkeyword{iecinvert} & false \\
		& \Lkeyword{bubblesize} & 0.2 \\
		& \multicolumn{2}{@{}l}{\quad possible values \texttt{0.05, 0.10, 0.15, 0.20}} \\
		\midrule
		\Lcs{logicff} & \Lkeyword{inputa} & true \\
		& \Lkeyword{invertinputa} & false \\
		& \Lkeyword{inputalabel} & \\
		& \Lkeyword{inputb} & true \\
		& \Lkeyword{invertinputb} & false \\
		& \Lkeyword{inputblabel} & \\
		& \Lkeyword{enable} & false \\
		& \Lkeyword{invertenable} & false \\
		& \Lkeyword{clock} & false \\
		& \Lkeyword{invertclock} & false \\
		& \Lkeyword{set} & false \\
		& \Lkeyword{invertset} & false \\
		& \Lkeyword{reset} & false \\
		& \Lkeyword{invertreset} & false \\
		& \Lkeyword{bubblesize} & 0.2 \\
		& \multicolumn{2}{@{}l}{\quad possible values \texttt{0.05, 0.10, 0.15, 0.20}} \\
		\midrule
		\Lcs{logicic} & \Lkeyword{nicpins} & 8 \\
		& \multicolumn{2}{@{}l}{\quad possible values \texttt{8, 14, 16, 20, 32}} \\
		& \nxLkeyword{pin?} & true \\
		& \nxLkeyword{invertpin?} & false \\
		& \nxLkeyword{pin?label} & \\
		& \nxLkeyword{pin?number} & \\
		& \multicolumn{2}{@{}l}{\quad where \texttt{? = a--z,aa,ab,ac,ad,ae,af}} \\
		& \Lkeyword{bubblesize} & 0.2 \\
		& \multicolumn{2}{@{}l}{\quad possible values \texttt{0.05, 0.10, 0.15, 0.20}} \\
		\midrule
		\Lcs{xic} & \Lkeyword{plcaddress} & \\
		& \Lkeyword{plcsymbol} & \\
		\midrule
		\Lcs{xio} & \Lkeyword{plcaddress} & \\
		& \Lkeyword{plcsymbol} & \\
		\midrule
		\Lcs{ote} & \Lkeyword{plcaddress} & \\
		& \Lkeyword{plcsymbol} & \\
		& \Lkeyword{latch} & false \\
		& \Lkeyword{unlatch} & false \\
		\midrule
		\Lcs{osr} & \Lkeyword{plcaddress} & \\
		& \Lkeyword{plcsymbol} & \\
		\midrule
		\Lcs{res} & \Lkeyword{plcaddress} & \\
		& \Lkeyword{plcsymbol} & \\
		\midrule
		\Lcs{swpb} & \Lkeyword{contactclosed} & false \\
		\midrule
		\Lcs{swtog} & \Lkeyword{contactclosed} & false \\
		\midrule
		\Lcs{contact} & \Lkeyword{contactclosed} & false \\
		\end{longtable}



\subsection{Examples}

%
% NOT Example
%


\begin{LTXexample}[pos=t]
\begin{pspicture}(-1,-1)(8.5,3)
  \logicnot[invertoutput=true](0,0){IEEE}
  \logicnot[invertoutput=true,iec=true,iecinvert=true](4,0){IEC}
\end{pspicture}
\end{LTXexample}


%
% AND Example
%
\begin{LTXexample}[pos=t]
\begin{pspicture}(-1,-1)(9.5,3)
  \logicand[ninputs=2](0,0){IEEE}
  \logicand[ninputs=2,iec=true](5,0){IEC}
\end{pspicture}
\end{LTXexample}

%
% NAND Example
%
\begin{LTXexample}[pos=t]
\begin{pspicture}(-1,-1)(9.5,3)
  \logicand[ninputs=2,invertoutput=true](0,0){IEEE}	
  \logicand[ninputs=2,invertoutput=true,iec=true,iecinvert=true](5,0){IEC}
\end{pspicture}
\end{LTXexample}


%
% Negative-AND Example
%
\begin{LTXexample}[pos=l]
\begin{pspicture}(-1,-1)(5,3)
  \logicand[ninputs=2,invertinputa=true,
            invertinputb=true](0,0){Name}
\end{pspicture}
\end{LTXexample}

% OR Example
%
\begin{LTXexample}[pos=t]
\begin{pspicture}(-1,-1)(9.5,3)
  \logicor[ninputs=2](0,0){IEEE}
  \logicor[ninputs=2,iec=true](5,0){IEC}
\end{pspicture}
\end{LTXexample}

%
% NOR Example
%
\begin{LTXexample}[pos=t]
\begin{pspicture}(-1,-1)(9.5,3)
  \logicor[ninputs=2,invertoutput=true](0,0){IEEE}
  \logicor[ninputs=2,invertoutput=true,iec=true,iecinvert=true](5,0){IEC}
\end{pspicture}
\end{LTXexample}

%
% Negative-OR Example
%
\begin{LTXexample}[pos=l]
\begin{pspicture}(-1,-1)(5,3)
  \logicor[ninputs=2,invertinputa=true,
           invertinputb=true](0,0){Name}
\end{pspicture}
\end{LTXexample}

%
% Exclusive-OR Example
%
\begin{LTXexample}[pos=t]
\begin{pspicture}(-1,-1)(9.5,3)
  \logicxor[ninputs=2]{0}(0,0){IEEE}	
  \logicxor[ninputs=2,iec=true]{0}(5,0){IEC}	
\end{pspicture}
\end{LTXexample}

%
% Exclusive-NOR Example
%
\begin{LTXexample}[pos=t]
\begin{pspicture}(-1,-1)(9.5,3)
  \logicxor[ninputs=2,invertoutput=true]{0}(0,0){IEEE}	
  \logicxor[ninputs=2,invertoutput=true,iec=true,iecinvert=true]{0}(5,0){IEC}	
\end{pspicture}
\end{LTXexample}

\subsubsection{$S$-$R$ Flip-Flop with Clock}
%
% FF Example
%
\begin{LTXexample}[pos=l]
\begin{pspicture}(-1,-1)(5,4)
  \logicff[clock=true,inputalabel=$S$,inputblabel=$R$](0,0){Name}	
\end{pspicture}
\end{LTXexample}

%
% FF Example
%
\subsubsection{$\bar{S}$-$\bar{R}$ Flip-Flop with Enable} %$
\begin{LTXexample}[pos=l]
\begin{pspicture}(-1,-1)(5,4)
  \logicff[enable=true,inputalabel=$\bar{S}$,inputblabel=$\bar{R}$](0,0){Name}
\end{pspicture}
\end{LTXexample}

%
% FF Example
%
\subsubsection{$J$-$K$ Flip-Flop}
\begin{LTXexample}[pos=l]
\begin{pspicture}(-1,-1)(5,4)
  \logicff[inputalabel=$J$,inputblabel=$K$](0,0){Name}
\end{pspicture}
\end{LTXexample}

%
% FF Example
%
\subsubsection{$J$-$K$ Flip-Flop with Set and Reset}
\begin{LTXexample}[pos=l]
\begin{pspicture}(-1,-1)(5,4)
  \logicff[set=true,reset=true,invertreset=true,%
      inputalabel=$J$,inputblabel=$K$](0,0){Name}	
\end{pspicture}
\end{LTXexample}

%
% FF Example
%
\subsubsection{$D$ Flip-Flop}
\begin{LTXexample}[pos=l]
\begin{pspicture}(-1,-1)(5,4)
  \logicff[inputb=false,inputalabel=$D$](0,0){Name}
\end{pspicture}
\end{LTXexample}

%
% FF Example
%
\subsubsection{Full Adder}
\begin{LTXexample}[pos=l]
\begin{pspicture}(-1,-1)(5,4)
  \logicff[enable=true,invertoutputb=false,inputalabel=$A$,
     inputblabel=$C_{in}$,inputenlabel=$B$,outputalabel=$\Sigma$,
     outputblabel=$C_{out}$](0,0){Name}
\end{pspicture}
\end{LTXexample}

%
\subsubsection{7-Segment Display}
%
\begin{LTXexample}[pos=l]
\begin{pspicture}(6.5,5)
  \sevensegmentdisplay(0,0){Name}
\end{pspicture}
\end{LTXexample}

%
% 7-Segment Display Example
%
\begin{LTXexample}[pos=t]
\begin{pspicture}(-1,-2)(6.5,6)
  \sevensegmentdisplay[pinld=false,pinle=false,pinrc=false,pinlalabel=a,
    pinlblabel=f,pinlglabel=e,pinrglabel=d,pinrelabel=c,pinrdlabel=g,
    pinrblabel=b,pinralabel={$V_{CC}$},pinlanumber=1,pinlbnumber=2,
    pinlcnumber=3,pinlfnumber=6,pinlgnumber=7,pinrgnumber=8,pinrfnumber=9,
    pinrenumber=10,pinrdnumber=11,pinrbnumber=13,pinranumber=14](0,0){Name}
\end{pspicture}
\end{LTXexample}

%
% 7-Segment Display Example
%
\begin{LTXexample}[pos=t]
\begin{pspicture}(-1,-2)(6.5,6)
  \sevensegmentdisplay[segmentdisplay=5](0,0){Name}
\end{pspicture}
\end{LTXexample}

%
% 7-Segment Display Example
%
\begin{LTXexample}[pos=t]
\begin{pspicture}(-1,-2)(6.5,6)
  \sevensegmentdisplay[segmentdisplay=0,segmentcolor=red,segmentlabels=false,
    pinlalabel=la,pinlblabel=lb,pinlclabel=lc,pinldlabel=ld,pinlelabel=le,
    pinlflabel=lf,pinlglabel=lg,pinrglabel=rg,pinrflabel=rf,pinrelabel=re,
    pinrdlabel=rd,pinrclabel=rc,pinrblabel=rb,pinralabel=ra,pinlanumber=1,
    pinlbnumber=2,pinlcnumber=3,pinldnumber=4,pinlenumber=5,pinlfnumber=6,
    pinlgnumber=7,pinrgnumber=8,pinrfnumber=9,pinrenumber=10,pinrdnumber=11,
    pinrcnumber=12,pinrbnumber=13,pinranumber=14,pinta=true,pintalabel=ta,
    pintanumber=0,pintb=true,pintblabel=tb,pintbnumber=0,pintc=true,
    pintclabel=tc,pintcnumber=0,pintd=true,pintdlabel=td,pintdnumber=0,
    pinte=true,pintelabel=te,pintenumber=0,pinba=true,pinbalabel=ba,
    pinbanumber=0,pinbb=true,pinbblabel=bb,pinbbnumber=0,pinbc=true,
    pinbclabel=bc,pinbcnumber=0,pinbd=true,pinbdlabel=bd,pinbdnumber=0,
    pinbe=true,pinbelabel=be,pinbenumber=0](0,0){Name}
\end{pspicture}
\end{LTXexample}

%
% 7-Segment Display Example
%
\begin{LTXexample}[pos=t]
\begin{pspicture}(-1,-2)(6.5,6)
  \sevensegmentdisplay[segmentdisplay=10,pinla=false,pinlb=false,
    pinlc=false,pinld=false,pinle=false,pinlf=false,pinlg=false,pinrg=false,
    pinrf=false,pinre=false,pinrd=false,pinrc=false,pinrb=false,pinra=false,
    pinta=true,pintalabel=g,pintanumber=0,pintb=true,pintblabel=f,pintbnumber=0,
    pintc=true,pintclabel=$V_{cc}$,pintcnumber=0,pintd=true,pintdlabel=a,
    pintdnumber=0,pinte=true,pintelabel=b,pintenumber=0,pinba=true,pinbalabel=e,
    pinbanumber=0,pinbb=true,pinbblabel=d,pinbbnumber=0,pinbc=true,
    pinbclabel=$V_{cc}$,pinbcnumber=0,pinbd=true,pinbdlabel=c,pinbdnumber=0,
    pinbe=true,pinbelabel=dp,pinbenumber=0](0,0){Name}
\end{pspicture}
\end{LTXexample}

%
% IC Example 8-pins
%
\clearpage
\subsubsection{8-Pin DIP IC}
\begin{LTXexample}[pos=t]
\begin{pspicture}(-1,-2)(5,4)
  \logicic[nicpins=8,bubblesize=0.1,%
	pintl=true,pintllabel=tl,pintlnumber=1,%
	pintc=true,pintclabel=tc,pintcnumber=2,%
	pintr=true,pintrlabel=tr,pintrnumber=3,%
	invertpintl=true,invertpintc=true,invertpintr=true,%
	pinbl=true,pinbllabel=bl,pinblnumber=1,%
	pinbc=true,pinbclabel=bc,pinbcnumber=2,%
	pinbr=true,pinbrlabel=br,pinbrnumber=3,%
	invertpinbl=true,invertpinbc=true,invertpinbr=true,%
	pinalabel=a,pinblabel=b,pinclabel=c,pindlabel=d,%
	pinelabel=e,pinflabel=f,pinglabel=g,pinhlabel=h,%
	pinanumber=1,pinbnumber=2,pincnumber=3,pindnumber=4,%
	pinenumber=5,pinfnumber=6,pingnumber=7,pinhnumber=8](0,0){Name}
\end{pspicture}
\end{LTXexample}

%
% IC Example 8-pins all inverted
%
\begin{LTXexample}[pos=t]
\begin{pspicture}(-1,-2)(5,4)
  \logicic[nicpins=8,%
	pintl=true,pintllabel=tl,pintlnumber=1,%
	pintc=true,pintclabel=tc,pintcnumber=2,%
	pintr=true,pintrlabel=tr,pintrnumber=3,%
	invertpintl=true,invertpintc=true,invertpintr=true,%
	pinbl=true,pinbllabel=bl,pinblnumber=1,%
	pinbc=true,pinbclabel=bc,pinbcnumber=2,%
	pinbr=true,pinbrlabel=br,pinbrnumber=3,%
	invertpinbl=true,invertpinbc=true,invertpinbr=true,%
	pinalabel=a,pinblabel=b,pinclabel=c,pindlabel=d,%
	pinelabel=e,pinflabel=f,pinglabel=g,pinhlabel=h,%
	pinanumber=1,pinbnumber=2,pincnumber=3,pindnumber=4,%
	pinenumber=5,pinfnumber=6,pingnumber=7,pinhnumber=8,%
	invertpina=true,invertpinb=true,invertpinc=true,invertpind=true,%
	invertpine=true,invertpinf=true,invertping=true,invertpinh=true](0,0){Name}
\end{pspicture}
\end{LTXexample}

%
% IC Example 14-pin
%
\clearpage
\subsubsection{14-Pin DIP IC}
\begin{LTXexample}[pos=t]
\begin{pspicture}(-1,-2)(5,6)
  \logicic[nicpins=14,%
	pintl=true,pintllabel=tl,pintlnumber=1,%
	pintc=true,pintclabel=tc,pintcnumber=2,%
	pintr=true,pintrlabel=tr,pintrnumber=3,%
	invertpintl=true,invertpintc=true,invertpintr=true,%
	pinbl=true,pinbllabel=bl,pinblnumber=1,%
	pinbc=true,pinbclabel=bc,pinbcnumber=2,%
	pinbr=true,pinbrlabel=br,pinbrnumber=3,%
	invertpinbl=true,invertpinbc=true,invertpinbr=true,%
	pinalabel=a,pinblabel=b,pinclabel=c,pindlabel=d,%
	pinelabel=e,pinflabel=f,pinglabel=g,pinhlabel=h,%
	pinilabel=i,pinjlabel=j,pinklabel=k,pinllabel=l,%
	pinmlabel=m,pinnlabel=n,%
	pinanumber=1,pinbnumber=2,pincnumber=3,pindnumber=4,%
	pinenumber=5,pinfnumber=6,pingnumber=7,pinhnumber=8,
	pininumber=9,pinjnumber=10,pinknumber=11,pinlnumber=12,%
	pinmnumber=13,pinnnumber=14]%
	(0,0){Name}
\end{pspicture}
\end{LTXexample}

%
\clearpage
\subsubsection{14-Pin DIP IC all inverted}
% IC Example 14-pin all inverted
%
\begin{LTXexample}[pos=t]
\begin{pspicture}(-1,-2)(5,6)
  \logicic[nicpins=14,%
	pintl=true,pintllabel=tl,pintlnumber=1,%
	pintc=true,pintclabel=tc,pintcnumber=2,%
	pintr=true,pintrlabel=tr,pintrnumber=3,%
	invertpintl=true,invertpintc=true,invertpintr=true,%
	pinbl=true,pinbllabel=bl,pinblnumber=1,%
	pinbc=true,pinbclabel=bc,pinbcnumber=2,%
	pinbr=true,pinbrlabel=br,pinbrnumber=3,%
	invertpinbl=true,invertpinbc=true,invertpinbr=true,%
	pinalabel=a,pinblabel=b,pinclabel=c,pindlabel=d,%
	pinelabel=e,pinflabel=f,pinglabel=g,pinhlabel=h,%
	pinilabel=i,pinjlabel=j,pinklabel=k,pinllabel=l,%
	pinmlabel=m,pinnlabel=n,%
	pinanumber=1,pinbnumber=2,pincnumber=3,pindnumber=4,%
	pinenumber=5,pinfnumber=6,pingnumber=7,pinhnumber=8,
	pininumber=9,pinjnumber=10,pinknumber=11,pinlnumber=12,%
	pinmnumber=13,pinnnumber=14,
	invertpina=true,invertpinb=true,invertpinc=true,invertpind=true,%
	invertpine=true,invertpinf=true,invertping=true,invertpinh=true,%
	invertpini=true,invertpinj=true,invertpink=true,invertpinl=true,%
	invertpinm=true,invertpinn=true]%
	(0,0){Name}
\end{pspicture}
\end{LTXexample}

%
% IC Example 16-pin
%
\clearpage
\subsubsection{16-Pin DIP IC}
\begin{LTXexample}[pos=t]
\begin{pspicture}(-1,-2)(5,6)
  \logicic[nicpins=16,%
	pintl=true,pintllabel=tl,pintlnumber=1,%
	pintc=true,pintclabel=tc,pintcnumber=2,%
	pintr=true,pintrlabel=tr,pintrnumber=3,%
	invertpintl=true,invertpintc=true,invertpintr=true,%
	pinbl=true,pinbllabel=bl,pinblnumber=1,%
	pinbc=true,pinbclabel=bc,pinbcnumber=2,%
	pinbr=true,pinbrlabel=br,pinbrnumber=3,%
	invertpinbl=true,invertpinbc=true,invertpinbr=true,%
	pinalabel=a,pinblabel=b,pinclabel=c,pindlabel=d,%
	pinelabel=e,pinflabel=f,pinglabel=g,pinhlabel=h,%
	pinilabel=i,pinjlabel=j,pinklabel=k,pinllabel=l,%
	pinmlabel=m,pinnlabel=n,pinolabel=o,pinplabel=p,%
	pinanumber=1,pinbnumber=2,pincnumber=3,pindnumber=4,%
	pinenumber=5,pinfnumber=6,pingnumber=7,pinhnumber=8,
	pininumber=9,pinjnumber=10,pinknumber=11,pinlnumber=12,%
	pinmnumber=13,pinnnumber=14,pinonumber=15,pinpnumber=16]%
	(0,0){Name}
\end{pspicture}
\end{LTXexample}

% IC Example 16-pin all inverted
%
\clearpage
\subsubsection{16-Pin DIP IC all inverted}
\begin{LTXexample}[pos=t]
\begin{pspicture}(-1,-2)(5,6)
  \logicic[nicpins=16,%
	pintl=true,pintllabel=tl,pintlnumber=1,%
	pintc=true,pintclabel=tc,pintcnumber=2,%
	pintr=true,pintrlabel=tr,pintrnumber=3,%
	invertpintl=true,invertpintc=true,invertpintr=true,%
	pinbl=true,pinbllabel=bl,pinblnumber=1,%
	pinbc=true,pinbclabel=bc,pinbcnumber=2,%
	pinbr=true,pinbrlabel=br,pinbrnumber=3,%
	invertpinbl=true,invertpinbc=true,invertpinbr=true,%
	pinalabel=a,pinblabel=b,pinclabel=c,pindlabel=d,%
	pinelabel=e,pinflabel=f,pinglabel=g,pinhlabel=h,%
	pinilabel=i,pinjlabel=j,pinklabel=k,pinllabel=l,%
	pinmlabel=m,pinnlabel=n,pinolabel=o,pinplabel=p,%
	pinanumber=1,pinbnumber=2,pincnumber=3,pindnumber=4,%
	pinenumber=5,pinfnumber=6,pingnumber=7,pinhnumber=8,
	pininumber=9,pinjnumber=10,pinknumber=11,pinlnumber=12,%
	pinmnumber=13,pinnnumber=14,pinonumber=15,pinpnumber=16,
	invertpina=true,invertpinb=true,invertpinc=true,invertpind=true,%
	invertpine=true,invertpinf=true,invertping=true,invertpinh=true,%
	invertpini=true,invertpinj=true,invertpink=true,invertpinl=true,%
	invertpinm=true,invertpinn=true,invertpino=true,invertpinp=true]%
	(0,0){Name}
\end{pspicture}
\end{LTXexample}

%
% IC Example 20-pin
%
\clearpage
\subsubsection{20-Pin DIP IC}
\begin{LTXexample}[pos=t]
\begin{pspicture}(-1,-2)(5,7)
  \logicic[nicpins=20,%
	pintl=true,pintllabel=tl,pintlnumber=1,%
	pintc=true,pintclabel=tc,pintcnumber=2,%
	pintr=true,pintrlabel=tr,pintrnumber=3,%
	invertpintl=true,invertpintc=true,invertpintr=true,%
	pinbl=true,pinbllabel=bl,pinblnumber=1,%
	pinbc=true,pinbclabel=bc,pinbcnumber=2,%
	pinbr=true,pinbrlabel=br,pinbrnumber=3,%
	invertpinbl=true,invertpinbc=true,invertpinbr=true,%
	pinalabel=a,pinblabel=b,pinclabel=c,pindlabel=d,%
	pinelabel=e,pinflabel=f,pinglabel=g,pinhlabel=h,%
	pinilabel=i,pinjlabel=j,pinklabel=k,pinllabel=l,%
	pinmlabel=m,pinnlabel=n,pinolabel=o,pinplabel=p,%
	pinqlabel=q,pinrlabel=r,pinslabel=s,pintlabel=t,%
	pinanumber=1,pinbnumber=2,pincnumber=3,pindnumber=4,%
	pinenumber=5,pinfnumber=6,pingnumber=7,pinhnumber=8,
	pininumber=9,pinjnumber=10,pinknumber=11,pinlnumber=12,%
	pinmnumber=13,pinnnumber=14,pinonumber=15,pinpnumber=16,%
	pinqnumber=17,pinrnumber=18,pinsnumber=19,pintnumber=20]%
	(0,0){Name}
\end{pspicture}
\end{LTXexample}

%
% IC Example 20-pin all inverted
%
\clearpage
\subsubsection{20-Pin DIP IC all inverted}
\begin{LTXexample}[pos=t]
\begin{pspicture}(-1,-2)(5,7)
  \logicic[nicpins=20,%
	pintl=true,pintllabel=tl,pintlnumber=1,%
	pintc=true,pintclabel=tc,pintcnumber=2,%
	pintr=true,pintrlabel=tr,pintrnumber=3,%
	invertpintl=true,invertpintc=true,invertpintr=true,%
	pinbl=true,pinbllabel=bl,pinblnumber=1,%
	pinbc=true,pinbclabel=bc,pinbcnumber=2,%
	pinbr=true,pinbrlabel=br,pinbrnumber=3,%
	invertpinbl=true,invertpinbc=true,invertpinbr=true,%
	pinalabel=a,pinblabel=b,pinclabel=c,pindlabel=d,%
	pinelabel=e,pinflabel=f,pinglabel=g,pinhlabel=h,%
	pinilabel=i,pinjlabel=j,pinklabel=k,pinllabel=l,%
	pinmlabel=m,pinnlabel=n,pinolabel=o,pinplabel=p,%
	pinqlabel=q,pinrlabel=r,pinslabel=s,pintlabel=t,%
	pinanumber=1,pinbnumber=2,pincnumber=3,pindnumber=4,%
	pinenumber=5,pinfnumber=6,pingnumber=7,pinhnumber=8,
	pininumber=9,pinjnumber=10,pinknumber=11,pinlnumber=12,%
	pinmnumber=13,pinnnumber=14,pinonumber=15,pinpnumber=16,%
	pinqnumber=17,pinrnumber=18,pinsnumber=19,pintnumber=20,%
	invertpina=true,invertpinb=true,invertpinc=true,invertpind=true,%
	invertpine=true,invertpinf=true,invertping=true,invertpinh=true,%
	invertpini=true,invertpinj=true,invertpink=true,invertpinl=true,%
	invertpinm=true,invertpinn=true,invertpino=true,invertpinp=true,%
	invertpinq=true,invertpinr=true,invertpins=true,invertpint=true]%
	(0,0){Name}
\end{pspicture}
\end{LTXexample}

%
% IC Example 32-pin
%
\clearpage
\subsubsection{32-Pin DIP IC}
\begin{LTXexample}[pos=t]
\begin{pspicture}(-1,-2)(6,9.5)
\logicic[nicpins=32, pintl=true,pintllabel=tl,pintlnumber=1,
  pintc=true,pintclabel=tc,pintcnumber=2,pintr=true,pintrlabel=tr,pintrnumber=3,%
  invertpintl=true,invertpintc=true,invertpintr=true,
  pinbl=true,pinbllabel=bl,pinblnumber=1,pinbc=true,pinbclabel=bc,pinbcnumber=2,%
  pinbr=true,pinbrlabel=br,pinbrnumber=3,%
  invertpinbl=true,invertpinbc=true,invertpinbr=true,%
  pinalabel=a,pinblabel=b,pinclabel=c,pindlabel=d,%
  pinelabel=e,pinflabel=f,pinglabel=g,pinhlabel=h,%
  pinilabel=i,pinjlabel=j,pinklabel=k,pinllabel=l,%
  pinmlabel=m,pinnlabel=n,pinolabel=o,pinplabel=p,%
  pinqlabel=q,pinrlabel=r,pinslabel=s,pintlabel=t,%
  pinulabel=u,pinvlabel=v,pinwlabel=w,pinxlabel=x,%
  pinylabel=y,pinzlabel=z,pinaalabel=aa,pinablabel=ab,%
  pinaclabel=ac,pinadlabel=ad,pinaelabel=ae,pinaflabel=af,%
  pinanumber=1,pinbnumber=2,pincnumber=3,pindnumber=4,%
  pinenumber=5,pinfnumber=6,pingnumber=7,pinhnumber=8,
  pininumber=9,pinjnumber=10,pinknumber=11,pinlnumber=12,%
  pinmnumber=13,pinnnumber=14,pinonumber=15,pinpnumber=16,%
  pinqnumber=17,pinrnumber=18,pinsnumber=19,pintnumber=20,%
  pinunumber=21,pinvnumber=22,pinwnumber=23,pinxnumber=24,%
  pinynumber=25,pinznumber=26,pinaanumber=27,pinabnumber=28,%
  pinacnumber=29,pinadnumber=30,pinaenumber=31,pinafnumber=32](0,0){Name}
\end{pspicture}
\end{LTXexample}

%
% IC Example 32-pin all inverted
%
\clearpage
\subsubsection{32-Pin DIP IC all inverted}
\begin{center}
\begin{pspicture}(-1,-2)(6,9.5)
\logicic[nicpins=32,pintl=true,pintllabel=tl,pintlnumber=1,
  pintc=true,pintclabel=tc,pintcnumber=2,pintr=true,pintrlabel=tr,pintrnumber=3,
  invertpintl=true,invertpintc=true,invertpintr=true,%
  pinbl=true,pinbllabel=bl,pinblnumber=1,pinbc=true,pinbclabel=bc,pinbcnumber=2,%
  pinbr=true,pinbrlabel=br,pinbrnumber=3,
  invertpinbl=true,invertpinbc=true,invertpinbr=true,%
  pinalabel=a,pinblabel=b,pinclabel=c,pindlabel=d,%
  pinelabel=e,pinflabel=f,pinglabel=g,pinhlabel=h,%
  pinilabel=i,pinjlabel=j,pinklabel=k,pinllabel=l,%
  pinmlabel=m,pinnlabel=n,pinolabel=o,pinplabel=p,%
  pinqlabel=q,pinrlabel=r,pinslabel=s,pintlabel=t,%
  pinulabel=u,pinvlabel=v,pinwlabel=w,pinxlabel=x,%
  pinylabel=y,pinzlabel=z,pinaalabel=aa,pinablabel=ab,%
  pinaclabel=ac,pinadlabel=ad,pinaelabel=ae,pinaflabel=af,%
  pinanumber=1,pinbnumber=2,pincnumber=3,pindnumber=4,%
  pinenumber=5,pinfnumber=6,pingnumber=7,pinhnumber=8,
  pininumber=9,pinjnumber=10,pinknumber=11,pinlnumber=12,%
  pinmnumber=13,pinnnumber=14,pinonumber=15,pinpnumber=16,%
  pinqnumber=17,pinrnumber=18,pinsnumber=19,pintnumber=20,%
  pinunumber=21,pinvnumber=22,pinwnumber=23,pinxnumber=24,%
  pinynumber=25,pinznumber=26,pinaanumber=27,pinabnumber=28,%
  pinacnumber=29,pinadnumber=30,pinaenumber=31,pinafnumber=32,%
  invertpina=true,invertpinb=true,invertpinc=true,invertpind=true,%
  invertpine=true,invertpinf=true,invertping=true,invertpinh=true,%
  invertpini=true,invertpinj=true,invertpink=true,invertpinl=true,%
  invertpinm=true,invertpinn=true,invertpino=true,invertpinp=true,%
  invertpinq=true,invertpinr=true,invertpins=true,invertpint=true,%
  invertpinu=true,invertpinv=true,invertpinw=true,invertpinx=true,%
  invertpiny=true,invertpinz=true,invertpinaa=true,invertpinab=true,%
  invertpinac=true,invertpinad=true,invertpinae=true,invertpinaf=true](0,0){Name}
\end{pspicture}
\end{center}


\begin{lstlisting}
\begin{pspicture}(-1,-2)(6,9.5)
  \logicic[nicpins=32,%
	pintl=true,pintllabel=tl,pintlnumber=1,%
	pintc=true,pintclabel=tc,pintcnumber=2,%
	pintr=true,pintrlabel=tr,pintrnumber=3,%
	invertpintl=true,invertpintc=true,invertpintr=true,%
	pinbl=true,pinbllabel=bl,pinblnumber=1,%
	pinbc=true,pinbclabel=bc,pinbcnumber=2,%
	pinbr=true,pinbrlabel=br,pinbrnumber=3,%
	invertpinbl=true,invertpinbc=true,invertpinbr=true,%
	pinalabel=a,pinblabel=b,pinclabel=c,pindlabel=d,%
	pinelabel=e,pinflabel=f,pinglabel=g,pinhlabel=h,%
	pinilabel=i,pinjlabel=j,pinklabel=k,pinllabel=l,%
	pinmlabel=m,pinnlabel=n,pinolabel=o,pinplabel=p,%
	pinqlabel=q,pinrlabel=r,pinslabel=s,pintlabel=t,%
	pinulabel=u,pinvlabel=v,pinwlabel=w,pinxlabel=x,%
	pinylabel=y,pinzlabel=z,pinaalabel=aa,pinablabel=ab,%
	pinaclabel=ac,pinadlabel=ad,pinaelabel=ae,pinaflabel=af,%
	pinanumber=1,pinbnumber=2,pincnumber=3,pindnumber=4,%
	pinenumber=5,pinfnumber=6,pingnumber=7,pinhnumber=8,
	pininumber=9,pinjnumber=10,pinknumber=11,pinlnumber=12,%
	pinmnumber=13,pinnnumber=14,pinonumber=15,pinpnumber=16,%
	pinqnumber=17,pinrnumber=18,pinsnumber=19,pintnumber=20,%
	pinunumber=21,pinvnumber=22,pinwnumber=23,pinxnumber=24,%
	pinynumber=25,pinznumber=26,pinaanumber=27,pinabnumber=28,%
	pinacnumber=29,pinadnumber=30,pinaenumber=31,pinafnumber=32,%
	invertpina=true,invertpinb=true,invertpinc=true,invertpind=true,%
	invertpine=true,invertpinf=true,invertping=true,invertpinh=true,%
	invertpini=true,invertpinj=true,invertpink=true,invertpinl=true,%
	invertpinm=true,invertpinn=true,invertpino=true,invertpinp=true,%
	invertpinq=true,invertpinr=true,invertpins=true,invertpint=true,%
	invertpinu=true,invertpinv=true,invertpinw=true,invertpinx=true,%
	invertpiny=true,invertpinz=true,invertpinaa=true,invertpinab=true,%
	invertpinac=true,invertpinad=true,invertpinae=true,invertpinaf=true]%
	(0,0){Name}
\end{pspicture}
\end{lstlisting}

\clearpage
\section{Relay Ladder Logic}

%
% XIC
%
\subsubsection{XIC}
\begin{LTXexample}[pos=l]
\begin{pspicture}(-1,-1)(1,1)
  \xic[plcaddress=I:1/0,
       plcsymbol=Symbol](0,0)
\end{pspicture}
\end{LTXexample}

%
% XIO
%
\subsubsection{XI0}
\begin{LTXexample}[pos=l]
\begin{pspicture}(-1,-1)(1,1)
  \xio[plcaddress=I:1/0,
       plcsymbol=Symbol](0,0)
\end{pspicture}
\end{LTXexample}

%
% OTE
%
\subsubsection{OTE}
\begin{LTXexample}[pos=l]
\begin{pspicture}(-1,-1)(1,1)
  \ote[plcaddress=O:2/0,
       plcsymbol=Symbol](0,0)
\end{pspicture}
\end{LTXexample}

%
% OTL
%
\subsubsection{OTL}
\begin{LTXexample}[pos=l]
\begin{pspicture}(-1,-1)(1,1)
  \ote[latch=true,
       plcaddress=O:2/0,
       plcsymbol=Symbol](0,0)
\end{pspicture}
\end{LTXexample}

%
% OTU
%
\subsubsection{OTE}
\begin{LTXexample}[pos=l]
\begin{pspicture}(-1,-1)(1,1)
  \ote[unlatch=true,
       plcaddress=O:2/0,
       plcsymbol=Symbol](0,0)
\end{pspicture}
\end{LTXexample}

%
% OSR
%
\subsubsection{OSR}
\begin{LTXexample}[pos=l]
\begin{pspicture}(-1,-1)(1,1)
  \osr[plcaddress=O:2/0,
       plcsymbol=Symbol](0,0)
\end{pspicture}
\end{LTXexample}

%
% RES
%
\subsubsection{RES}
\begin{LTXexample}[pos=l]
\begin{pspicture}(-1,-1)(1,1)
  \res[plcaddress=O:2/0,
       plcsymbol=Symbol](0,0)
\end{pspicture}
\end{LTXexample}

%
% PB NO
%
\subsubsection{Switch PB NO}
\begin{LTXexample}[pos=l]
\begin{pspicture}(-1,-1)(1,1)
  \swpb(0,0)
\end{pspicture}
\end{LTXexample}

%
% PB NC
%
\subsubsection{Switch PB NC}
\begin{LTXexample}[pos=l]
\begin{pspicture}(-1,-1)(1,1)
  \swpb[contactclosed=true](0,0)
\end{pspicture}
\end{LTXexample}

%
% Switch Toggle NO
%
\subsubsection{Switch TOGGLE NO}
\begin{LTXexample}[pos=l]
\begin{pspicture}(-1,-1)(1,1)
  \swtog(0,0)
\end{pspicture}
\end{LTXexample}

%
% Switch Toggle NC
%
\subsubsection{Switch PB NC}
\begin{LTXexample}[pos=l]
\begin{pspicture}(-1,-1)(1,1)
  \swtog[contactclosed=true](0,0)
\end{pspicture}
\end{LTXexample}

%
% Contact NO
%
\subsubsection{Contact NO}
\begin{LTXexample}[pos=l]
\begin{pspicture}(-1,-1)(1,1)
  \contact(0,0)
\end{pspicture}
\end{LTXexample}

%
% Contact NC
%
\subsubsection{Contact NC}
\begin{LTXexample}[pos=l]
\begin{pspicture}(-1,-1)(1,1)
  \contact[contactclosed=true](0,0)
\end{pspicture}
\end{LTXexample}

%
% Motor Armature
%
\subsubsection{Motor Armature}
\begin{LTXexample}[pos=l]
\begin{pspicture}(-1,-1)(1,1)
  \armature(0,0)
\end{pspicture}
\end{LTXexample}


\clearpage

\subsection{Examples}

\begin{LTXexample}[pos=t]
\begin{pspicture}(0,0)(15,6)
  \pnode(0.5,0){A}  \pnode(0.5,2.75){B}  \pnode(0.5,5.5){C}
  \pnode(3,0){D}  \pnode(3,2.75){E}  \pnode(3,5.5){F}
  \pnode(4.75,0){G}  \pnode(4.75,5.50){H}
  \pnode(6.5,0){I}  \pnode(6.5,5.5){J}
  \vac(B)(E){$V$}
  \newdiode(B)(C){$D_1$}
  \newdiode[ison=false](E)(F){$D_2$}
  \newdiode[ison=false](A)(B){$D_3$}
  \newdiode(D)(E){$D_4$}
  \capacitor(G)(H){$C$}
  \newarmature[labelInside=1](I)(J){}
  \wire(C)(F)	\wire(A)(D)	\wire(D)(G)	\wire(I)(G)	\wire(F)(H)	\wire(H)(J)

  \pnode(9,0){K}  \pnode(9,2.75){L}  \pnode(9,5.5){M}
  \pnode(11.5,0){N}  \pnode(11.5,2.75){O}  
  \pnode(11.5,5.5){P}
  \pnode(13.25,0){Q}  \pnode(13.25,5.5){R}
  \pnode(15,0){S}  \pnode(15,5.5){T}
  \vac(L)(O){$V$}
  \newdiode[ison=false](L)(M){$D_1$}
  \newdiode(O)(P){$D_2$}
  \newdiode(K)(L){$D_3$}
  \newdiode[ison=false](N)(O){$D_4$}
  \newcapacitor(Q)(R){$C$}
  \newarmature[labelInside=1](S)(T){}
  \wire(M)(P)	\wire(K)(N)	\wire(N)(Q)	\wire(S)(Q)	\wire(P)(R)	\wire(R)(T)
\end{pspicture}
\end{LTXexample}

\begin{LTXexample}[pos=l]
\begin{pspicture}(-1,-1)(4,4)
  \vac[labeloffset=-0.7](0,0)(4,0){$\backslash$vac}
  \vac[labeloffset=1](0,0)(2,3.464){$\backslash$vac}
  \vac[labeloffset=1](2,3.464)(4,0){$\backslash$vac}
\end{pspicture}
\end{LTXexample}




\section{Adding new components}

Adding new components is not simple unless you need only a simple dipole. For dipoles a macro is provided that generates all helping macros for a new component so that you need to write only the actual drawing code.

If you want to add a new dipole component, you only need the following code:
\begin{lstlisting}[language=TeX]
  \newCircDipole{ComponentName}%
  \def\pst@draw@ComponentName{%
    % The PSTricks code for your component
    % The center of the component is at (0,0)
    \pnode(component_left_end,0){dipole@1}
    \pnode(component_right_end,0){dipole@2}}
\end{lstlisting}
This code can be placed in the core code or somewhere in the respective document in which case it must be surrounded by \lstinline[language=TeX]{\makeatletter...\makeatother}.

If your new dipole should also work with \Lcs{multidipole} then you have to make some changes in the \Lcs{multidipole} core code. In the definition
of \Lcs{pst@multidipole}, look for the last \Lcs{ifx} test
\begin{lstlisting}[language=TeX]
  % ...
  % Extract from \pst@multidipole
     \else\ifx\OpenDipol   #4\let\pscirc@next\pst@multidipole@OpenDipol% 27
     \else\ifx\OpenTripol  #4\let\pscirc@next\pst@multidipole@OpenTripol% 28
     \else                         % Put your modification here
     \else\let\pscirc@next\ignorespaces
     \fi\fi\fi
  % Extract form \pst@multidipole
  % ...
\end{lstlisting}
and add (marked with \verb+%%%+)
\begin{lstlisting}[language=TeX]
  % ...
  % Extract from \pst@multidipole
     \else\ifx\OpenDipol   #4\let\pscirc@next\pst@multidipole@OpenDipol% 27
     \else\ifx\OpenTripol  #4\let\pscirc@next\pst@multidipole@OpenTripol% 28
     \else\ifx\ComponentName#4\let\next\pst@multidipole@ComponentName%%%
     \else\let\pscirc@next\ignorespaces
     \fi\fi\fi
  % Extract form \pst@multidipole
  % ...
\end{lstlisting}
Do the same in \verb+\pst@multidipole@+
\begin{lstlisting}[language=TeX]
  % ...
  % Extract from \pst@multidipole@
     \else\ifx\OpenDipol#1\let\pscirc@next\pst@multidipole@OpenDipol% 27
     \else\ifx\OpenTripol#1\let\pscirc@next\pst@multidipole@OpenTripol% 28
     \else\ifx\ComponentName#1\let\next\pst@multidipole@ComponentName%%%
     \else\let\pscirc@next\ignorespaces\pst@multidipole@output
     \fi\fi\fi
  % Extract form \pst@multidipole@
  % ...
\end{lstlisting}
and that's it! All you have to do then is send your modified \LFile{pst-circ.tex} to 
me and  it will become part of the official release of \LPack{pst-circ}.

\bigskip
\begin{LTXexample}[width=3.5cm]
  \begin{pspicture}(3,2)
    \newCircDipole{delayline}
    \makeatletter
    \def\pst@draw@delayline{%
       \psset{linewidth=1.5\pslinewidth}%
       \psframe(-0.5,-0.3)(0.5,0.3)
       \psline[arrows=->](-0.2,-0.5)(0.2,0.5)
       \pnode(-0.5,0){dipole@1}
       \pnode(0.5,0){dipole@2}}%
    \makeatother
    \pnode(0,1){A}\pnode(3,1){B}
    \delayline(A)(B){delay}
  \end{pspicture}
\end{LTXexample}




\clearpage
\section{List of all optional arguments for \texttt{pst-circ}}\label{sec:para}
Note: the default for booleans is always false.

\xkvview{family=pst-circ,columns={key,type,default}}

\bgroup
\raggedright
\nocite{*}
%\bibliographystyle{plain}
\printbibliography
\egroup

\printindex

\end{document}

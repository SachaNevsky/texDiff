\documentclass[a4paper]{article}
\usepackage{pst-geo}% 
\usepackage[utf8]{inputenc}%
\usepackage[dvips,margin=1cm]{geometry}%
\pagestyle{empty}

\begin{document}

\iffalse
\psset{path=../data}
  \begin{pspicture}*(-4.5,-26)(11,-10)
    \psset{xunit=7.5,yunit=7.5,type=8,latitude0=48.85,longitude0=2.316667,level=50,all=true}
    \WorldMapII[maillage=false,linewidth=0.75\pslinewidth,limiteL=190,borders=true]
    \input capitales
    \pnodeMap(20,35){MerMed}
    \rput{15}(MerMed){\shortstack{MER\\MÉDITERANNÉE}}
    \pnodeMap(35,43){MerNoire}
    \rput{15}(MerNoire){\shortstack{MER\\NOIRE}}
    \pnodeMap(-15,50){OceanAtlan}
    \rput{80}(OceanAtlan){OCÉAN ATLANTIQUE}
    \pnodeMap(4,56){MerNoire}
    \rput(MerNoire){\shortstack{Mer\\du\\Nord}}
  \end{pspicture}
\fi


\begin{pspicture}*(-5,4)(10,22)
  \psset{unit=4,type=1,level=50,all=true,path=../data}
  \WorldMapII[maillage=false,linewidth=0.5pt,limiteL=190,borders=true,
    SeaFillColor=blue!60]
  \input{capitals.tex}
  \pnodeMap(20,35){MerMed}
  \rput(MerMed){\shortstack{MER\\MÉDITERANNÉE}}
  \pnodeMap(35,43){MerNoire}
  \rput(MerNoire){\shortstack{MER\\NOIRE}}
  \pnodeMap(-15,50){OceanAtlan}
  \rput{90}(OceanAtlan){O C É A N\hspace{2em} A T L A N T I Q U E}
  \pnodeMap(4,56){MerNoire}
  \rput(MerNoire){\shortstack{Mer\\du\\Nord}}
\end{pspicture}

\end{document}

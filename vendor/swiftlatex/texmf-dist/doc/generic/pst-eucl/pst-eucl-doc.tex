\documentclass[11pt,english,BCOR10mm,DIV12,bibliography=totoc,parskip=false,smallheadings,
    headexclude,footexclude,oneside,english]{pst-doc}
\usepackage{pst-eucl}
\let\pstEuclideFV\fileversion
\usepackage{multicol}
\usepackage{ntheorem}
\newtheorem{theorem}{Theorem}
\usepackage{pst-func,pst-plot,paralist}
\usepackage[mathscr]{eucal}
\def\eV{e.\kern-1pt{}V\kern-1pt}


\lstset{pos=l,wide=false,basicstyle=\footnotesize\ttfamily,explpreset={language=[PSTricks]{TeX}}}
%
\def\Argsans#1{$\langle$#1$\rangle$}
\def\DefaultVal#1{(by default #1)}

\usepackage{biblatex}
\addbibresource{\jobname.bib}


\title{\texttt{pst-euclide}}
\subtitle{A PSTricks package for drawing geometric pictures; v.\pstEuclideFV}
\author{Dominique Rodriguez\\Herbert Voß}
\docauthor{Herbert Voß}
\date{\today}
\begin{document}
\maketitle

\begin{abstract}
  The \LPack{pst-eucl} package allow the drawing of Euclidean
  geometric figures using \LaTeX\ macros for specifying mathematical
  constraints. It is thus possible to build point using common
  transformations or intersections. The use of coordinates is limited
  to points which controlled the figure.

  \vfill
  I would like to thanks the following persons for the help they gave
  me for development of this package:

  \begin{compactitem}
  \item Denis Girou pour ses critiques pertinentes et ses
    encouragement lors de la découverte de l'embryon initial et pour
    sa relecture du présent manuel;
  \item Michael Vulis for his fast testing of the documentation using
    V\TeX\ which leads to the correction of a bug in the \PS\ code;
  \item Manuel Luque and Olivier Reboux for their remarks and their examples.
  \item Alain Delplanque for its modification theorems on automatic
    placing of points name and the ability of giving a list of points in
    \Lcs{pstGeonode}.
  \end{compactitem}
\end{abstract}


\vfill
\noindent
Thanks to:
Manuel Luque;
Thomas Söll;
Liao Xiongfei.



\clearpage
\tableofcontents


\clearpage
\part{The package}
\section{Special specifications}

\subsection{\PST\ Options}

The package activates the \Lcs{SpecialCoor} mode. This mode extend the
coordinates specification. Furthermore the plotting type is set to
\Lkeyset{dimen=middle}, which indicates that the position of the
drawing is done according to the middle of the line. Please look at
the user manual for more information about these setting.

At last, the working axes are supposed to be (ortho)normed.

\subsection{Conventions}

For this manual, I used the geometric French conventions for naming
the points:

\begin{compactitem}
\item $O$ is a centre (circle, axes, symmetry, homothety, rotation);
\item $I$ defined the unity of the abscissa axe, or a midpoint;
\item $J$ defined the unity of the ordinate axe;
\item $A$, $B$, $C$, $D$ are points ;
\item $M'$ is the image of $M$ by a transformation ;
\end{compactitem}

At last, although these are nodes in \PST, I treat them
intentionally as points.

\section{Basic Objects}
\subsection{Points}
%\subsubsection{default axes}

%\defcom[Creates a list of points using the common axis. \protect\ParamList{\param{PointName},
%  \param{PointNameSep}, \param{PosAngle}, \param{PointSymbol}, \param{PtNameMath}}]
\begin{BDef}
\Lcs{pstGeonode}\OptArgs\coord1\Largb{$A_1$}\coord2\Largb{$A_1$}\ldots\cAny\Largb{$A_n$}
\end{BDef}
This command defines one or more geometrical points associated with a node in the default cartesian coordinate system. Each
point has a node name $A_i$ which defines the default label put on the
picture. This label is managed by default in mathematical mode, the boolean parameter
\Lkeyword{PtNameMath} (default \true) can modify this behavior and let manage the
label in normal mode.  It is placed at a distance of \Lkeyword{PointNameSep}
(default 1em) of the center of the node with a angle of
\Lkeyword{PosAngle} (default 0). It is possible to specify another label using the
parameter \Lkeyset{PointName=default}, and an empty label can be specified
by selecting the value \Lkeyval{none}, in that case the point will have no name on the
picture.

The point symbol is given by the parameter \Lkeyset{PointSymbol=*}.  The
symbol is the same as used by the macro \Lcs{psdot}.  This parameter can be set to
\texttt{none}, which means that the point will not be drawn on the picture.

Here are the possible values for this parameter:

\begin{multicols}{3}
  \begin{compactitem}\psset{dotscale=2}
  \item \Lkeyword{*}: \psdots(.5ex,.5ex)
  \item \Lkeyword{o}: \psdots[dotstyle=o](.5ex,.5ex)
  \item \Lkeyword{+}: \psdots[dotstyle=+](.5ex,.5ex)
  \item \Lkeyword{x}: \psdots[dotstyle=x](.5ex,.5ex)
  \item \Lkeyword{asterisk} : \psdots[dotstyle=asterisk](.5ex,.5ex)
  \item \Lkeyword{oplus}: \psdots[dotstyle=oplus](.5ex,.5ex)
  \item \Lkeyword{otimes}: \psdots[dotstyle=otimes](.5ex,.5ex)
  \item \Lkeyword{triangle}: \psdots[dotstyle=triangle](.5ex,.5ex)
  \item \Lkeyword{triangle*}: \psdots[dotstyle=triangle*](.5ex,.5ex)
  \item \Lkeyword{square}: \psdots[dotstyle=square](.5ex,.5ex)
  \item \Lkeyword{square*}: \psdots[dotstyle=square*](.5ex,.5ex)
  \item \Lkeyword{diamond}: \psdots[dotstyle=diamond](.5ex,.5ex)
  \item \Lkeyword{diamond*}: \psdots[dotstyle=diamond*](.5ex,.5ex)
  \item \Lkeyword{pentagon}: \psdots[dotstyle=pentagon](.5ex,.5ex)
  \item \Lkeyword{pentagon*}: \psdots[dotstyle=pentagon*](.5ex,.5ex)
  \item \Lkeyword{|}: \psdots[dotstyle=|](.5ex,.5ex)
  \end{compactitem}
\end{multicols}

Furthermore, these symbols can be controlled with some others \PST,
several of these are :

\begin{compactitem}
\item their scale with \Lkeyword{dotscale}, the value of whom is either two numbers
  defining the horizontal and vertical scale factor, or one single value being the
  same for both,
\item their angle with parameter \Lkeyword{dotangle}.
\end{compactitem}

Please consult the \PST\ documentation for further details.
The
parameters \Lkeyword{PosAngle}, \Lkeyword{PointSymbol}, \Lkeyword{PointName} and
\Lkeyword{PointNameSep} can be set to :

\begin{compactitem}
\item either a single value, the same for all points ;
\item or a list of values delimited by accolads \texttt{\{ ... \}} and
  separated with comma \textit{without any blanks}, allowing to differenciate the
  value for each point.
\end{compactitem}

In the later case, the list can have less values than point which means that the
last value is used for all the remaining points.
%
At least, the parameter setting \Lkeyword{CurveType=none} can be used to
draw a line between the points:

\begin{compactitem}
\item opened \verb$polyline$ ;
\item closed \verb$polygon$ ;
\item open and curved \verb$curve$.
\end{compactitem}

\begin{LTXexample}[width=5cm,pos=l]
\begin{pspicture}[showgrid=true](-2,-2)(3,3)
\pstGeonode{A}
\pstGeonode[PosAngle=-135, PointNameSep=1.3](0,3){B_1}
\pstGeonode[PointSymbol=pentagon, dotscale=2, fillstyle=solid,
            fillcolor=OliveGreen, PtNameMath=false,
            PointName=$B_2$, linecolor=red](-2,1){B2}
\pstGeonode[PosAngle={90,0,-90}, PointSymbol={*,o},
            linestyle=dashed, CurveType=polygon,
            PointNameSep={1em,2em,3mm}]
  (1,2){M_1}(2,1){M_2}(1,0){M_3}
\pstGeonode[PosAngle={50,100,90}, PointSymbol={*,x,*},
            PointNameSep=3mm, CurveType=curve,
            PointName={\alpha,\beta,\gamma,default}]
  (-2,0){alpha}(-1,-2){beta}(0,-1){gamma}(2,-1.5){T}
\end{pspicture}
\end{LTXexample}

Obviously, the nodes appearing in the picture can be used as normal
\PST\ nodes. Thus, it is possible to reference a point from
\rnode{ici}{here}.
\nccurve[arrowscale=2]{->}{ici}{B_1}

After v1.65, we add macros \Lcs{pstAbscissa} and \Lcs{pstOrdinate} to
get the abscissa and ordinate of the specified node, so it is possible
to define a new node from an already constructed node with them.

\begin{BDef}
\Lcs{pstAbscissa}\Largb{A}\\
\Lcs{pstOrdinate}\Largb{A}
\end{BDef}

Note that the value of abscissa or ordinate are transformed to the \texttt{User coordinate},
and then put into the stack of \PS, so they can be used to do some compound arithmetic
without concerned the \texttt{xunit} and \texttt{yunit} in the \PST{} \texttt{SpecialCoor}
function. You need the other third package to do float arithmetic operation,
like \Lcs{pscalculate} \footnote{Provided by package \texttt{pst-calculate},
sometimes it results the numbers more than 9 fraction digits,
which are not supported good by \PST\space with '! number too big' issue.} to generate the numerical values,
or the expandable command \Lcs{fpeval}\footnote{Provided by package \texttt{xfp},
it can truncate the fraction part digits using the \texttt{trunc} function perfectly,
e.g. \texttt{\textbackslash{}fpeval\{trunc(18/7,3)\}}.} to get a purely numerical result.

The macro \Lcs{pstMoveNode} use them to move node $A$ by abscissa increment $dx$
and ordinate increment $dy$ to get the target node $B$.

\begin{BDef}
\Lcs{pstMoveNode}\OptArgs($dx$,\kern 1pt$dy$)\Largb{$A$}\Largb{$B$}
\end{BDef}

for example:

\begin{LTXexample}[width=6cm,pos=l]
\begin{pspicture}[showgrid=true](0,0)(4,4)
\def\ra{3.0}\def\rb{4.0}
\pstGeonode[PosAngle=-90](1.0,1.5){A}
\pstGeonode[PosAngle=90](! \pstAbscissa{A} 1 add \pstOrdinate{A} 2 add){B}
\pstLineAB[linecolor=blue]{A}{B}
\pstMoveNode[PosAngle=-90,PointSymbol=asterisk](3,2){A}{C}
\pstLineAB[linecolor=red]{A}{C}
\pstMoveNode[PosAngle=-90,PointSymbol=diamond](\pscalculate{sqrt(\ra*\ra+\rb*\rb)/2},\pscalculate{\ra*\rb/(2*(\ra+\rb))}){A}{D}
\pstLineAB[linecolor=cyan]{A}{D}
\pstMoveNode[PosAngle=-90](\pstAbscissa{B} 3 div,\pstOrdinate{B} neg 3 div){D}{E}
\pstLineAB[linecolor=green]{A}{E}
\end{pspicture}
\end{LTXexample}

%\subsubsection{User defined axes}

\Lcs{pstOIJGeonode} creates a list of points in the landmark $(O;I;J)$. Possible
parameters are \Lkeyword{PointName}, \Lkeyword{PointNameSep}, \Lkeyword{PosAngle},
\Lkeyword{PointSymbol}, and \Lkeyword{PtNameMath}.

\begin{BDef}
\Lcs{pstOIJGeonode}\OptArgs\coord1\Largb{$A_1$}\Largb{$O$}\Largb{$I$}\Largb{$J$}
    \coord2\Largb{$A_2$}\ldots\cAny\Largb{$A_n$}
\end{BDef}

\begin{LTXexample}[width=5.6cm,pos=l]
\psset{unit=.7}
\begin{pspicture*}[showgrid=true](-4,-4)(4,4)
  \pstGeonode[PosAngle={-135,-90,180}]{O}(1,0.5){I}(0.5,2){J}
  \pstLineAB[nodesep=10]{O}{I}
  \pstLineAB[nodesep=10]{O}{J}
  \multips(-5,-2.5)(1,0.5){11}{\psline(0,-.15)(0,.15)}
  \multips(-2,-8)(0.5,2){9}{\psline(-.15,0)(.15,0)}
  \psset{linestyle=dotted}%
  \multips(-5,-2.5)(1,0.5){11}{\psline(-10,-40)(10,40)}
  \multips(-2,-8)(0.5,2){9}{\psline(-10,-5)(10,5)}
  \psset{PointSymbol=x, linestyle=solid}
  \pstOIJGeonode[PosAngle={-90,0}, CurveType=curve,
    linecolor=red] (3,1){A}{O}{I}{J}(-2,1){B}(-1,-1.5){C}(2,-1){D}
\end{pspicture*}
\end{LTXexample}


\subsection{Segment mark}

A segment can be drawn using the \Lcs{ncline} command. However,
for marking a segment there is the following command:

\begin{BDef}
\Lcs{pstSegmentMark}\OptArgs\Largb{A}\Largb{B}
\end{BDef}


The symbol drawn on the segment is given by the parameter
\Lkeyword{SegmentSymbol}. Its value can be any valid command which can be
used in math mode. Its default value is \Lkeyval{MarkHashh},
which produced two slashes on the segment. The segment is drawn.

Several commands are predefined for marking the segment:

\begingroup
\psset{PointSymbol=none,PointName=none,unit=.8}
  \newcommand\Seg[1]{%
    \Lkeyval{#1} \begin{pspicture}[shift=*](1.75,1)
                 \pstGeonode(0.3,.5){A}(1.7,.5){B}\pstSegmentMark[SegmentSymbol=#1]{A}{B}
               \end{pspicture}}%
\begin{multicols}{3}
  \begin{compactitem}
  \item \Seg{pstslash}
  \item \Seg{pstslashh}
  \item \Seg{pstslashhh}
  \item \Seg{MarkHash}
  \item \Seg{MarkHashh}
  \item \Seg{MarkHashhh}
  \item \Seg{MarkCros}
  \item \Seg{MarkCross}
  \item \Seg{MarkArrow}
  \item \Seg{MarkArroww}
  \item \Seg{MarkArrowww}
  \end{compactitem}
\end{multicols}
\endgroup

The three commands of the family \Lkeyval{MarkHash} draw a line whose inclination is
controled by the parameter \Lkeyword{MarkAngle} (default is 45). Their width and colour
depends of the width and color of the line when the drawing is done, as shown is the
next example.

\begin{LTXexample}[width=5cm,pos=l]
\begin{pspicture}[showgrid=true](-2,-2)(2,2)
\rput{18}{%
  \pstGeonode[PosAngle={0,90,180,-90}](2,0){A}(2;72){B}
    (2;144){C}(2;216){D}(2;288){E}}
\pstSegmentMark[SegmentSymbol=none]{A}{B}
\pstSegmentMark[linecolor=green]{B}{C}
\psset{linewidth=2\pslinewidth}
\pstSegmentMark[linewidth=2\pslinewidth]{C}{D}
\pstSegmentMark[MarkAngle=90]{D}{E}
\pstSegmentMark{E}{A}
\end{pspicture}
\end{LTXexample}

The length and the separation of multiple hases can be set by \Lkeyword{MarkHashLength} and \Lkeyword{MarkHashSep}.

\subsection{Segment labels}

According to the manual of \PST, you can use the macros \Lcs{naput}, \Lcs{ncput} and \Lcs{nbput}
to put the label \textit{above}, \textit{cover}, \textit{below} the segment. The macro \Lcs{pstLabelAB}
just use them to draw a ruler bar and put the label on the ruler bar.

\begin{BDef}
\Lcs{pstLabelAB}\OptArgs\Largb{A}\Largb{B}\Largb{label}
\end{BDef}

You can use the parameters of \Lcs{ncline} to control the ruler bar,
such as \Lkeyword{linestyle}, \Lkeyword{linecolor}, \Lkeyword{linewidth},
\Lkeyword{arrows}, \Lkeyword{nodesep} etc; and use the parameters of \Lcs{ncput}
to control the label position, such as \Lkeyword{nrot}, \Lkeyword{npos} etc;
there is another parameter \Lkeyword{offset} to control the separation between
the rule bar and the segment.

It does not display the ruler bar as default, and you need to setup \Lkeyword{linestyle}
to display it.

\begin{LTXexample}[width=6cm,pos=l]
\begin{pspicture}[showgrid=true](-1,-1)(4,4)
\psset{dotscale=0.5}\psset{PointSymbol=*}\footnotesize
\pstGeonode[PosAngle=-90](0.5,1.5){A}
\pstGeonode[PosAngle=-90](2.5,1.5){B}\pstLineAB{A}{B}
\pstLabelAB{A}{B}{$\sqrt{a^2+b^2}$}
\pstGeonode[PosAngle=0](0,0.5){C}
\pstGeonode[PosAngle=0](0,3.5){D}\pstLineAB{C}{D}
\pstLabelAB[linestyle=dashed]{C}{D}{$\sqrt{a^2+b^2}$}
\pstGeonode[PosAngle=190](-1,-1){E}
\pstGeonode[PosAngle=10](3,0){F}\pstLineAB{E}{F}
\pstLabelAB[linestyle=dashed,arrows=|-|,offset=10pt,linecolor=blue!50]{E}{F}{$\sqrt{a^2+b^2}$}
\pstLabelAB[linestyle=dashed,arrows=|<->|,offset=10pt,nrot=:D]{F}{E}{$\sqrt{a^2+b^2}$}
\pstGeonode[PosAngle=100](0,4){G}
\pstGeonode[PosAngle=-50](4,2){H}\pstLineAB{G}{H}
\pstLabelAB[linestyle=solid,linecolor=red!50,arrows=|-|,offset=15pt,nrot=:U,npos=0.7]{G}{H}{\textcolor{red!50}{$\dfrac{a}{b}$}}
\end{pspicture}
\end{LTXexample}

%$

\subsection{Triangles}

The more classical figure, it has its own macro \Lcs{pstTriangle} for a quick definition:

\begin{BDef}
\Lcs{pstTriangle}\OptArgs\coord1\Largb{A}\coord2\Largb{B}\coord3\Largb{C}
\end{BDef}


\begin{sloppypar}
Valid optional arguments are \Lkeyword{PointName},
  \Lkeyword{PointNameSep}, %\Lkeyword{PosAngle},
  \Lkeyword{PointSymbol}, \Lkeyword{PointNameA},
  \Lkeyword{PosAngleA}, \Lkeyword{PointSymbolA}, \Lkeyword{PointNameB},
  \Lkeyword{PosAngleB}, \Lkeyword{PointSymbolB}, \Lkeyword{PointNameC},
  \Lkeyword{PosAngleC}, and \Lkeyword{PointSymbolC}.
%  $(x_A;y_A)$\Arg{$A$}$(x_B;y_B)$\Arg{$B$}$(x_C;y_C)$\Arg{$C$}}
%
In order to accurately put the name of the points, there are three parameters
\Lkeyword{PosAngleA}, \Lkeyword{PosAngleB} and \Lkeyword{PosAngleC}, which are associated
respectively to the nodes \Argsans{$A$}, \Argsans{$B$} and \Argsans{$C$}. Obviously
they have the same meaning as the parameter \Lkeyword{PosAngle}. If one or more of such
parameters is omitted, the value of \Lkeyword{PosAngle} is taken.  If no angle
is specified, points name are placed on the bissector line.
\end{sloppypar}

In the same way there are parameters for controlling the symbol used
for each points: \Lkeyword{PointSymbolA}, \Lkeyword{PointSymbolB} and
\Lkeyword{PointSymbolC}. They are equivalent to the parameter
\Lkeyword{PointSymbol}. The management of the default value followed the
same rule.

The macros \Lcs{pstTriangleSSS}, \Lcs{pstTriangleSAS}, \Lcs{pstTriangleAAS} and
\Lcs{pstTriangleASA} are used to draw the triangle according the specified sides
or angles.

\begin{BDef}
\Lcs{pstTriangleSSS}\OptArgs\Largr{pos}\Largb{A}\Largr{a,b,c}\Largb{B}\Largb{C}\\
\Lcs{pstTriangleSAS}\OptArgs\Largr{pos}\Largb{A}\Largr{b,$\angle{A}$,c}\Largb{B}\Largb{C}\\
\Lcs{pstTriangleAAS}\OptArgs\Largr{pos}\Largb{A}\Largr{$\angle{C}$,$\angle{A}$,c}\Largb{B}\Largb{C}\\
\Lcs{pstTriangleASA}\OptArgs\Largr{pos}\Largb{A}\Largr{$\angle{A}$,c,$\angle{B}$}\Largb{B}\Largb{C}
\end{BDef}

%$

- Macro \Lcs{pstTriangleSSS} create a triangle $ABC$ with given $A(x_1,y_1)$, and the three sides $a,b,c$,
it output $B(x_2,y_2)$ and $C(x_3,y_3)$.\\
- Macro \Lcs{pstTriangleSAS} create a triangle $ABC$ with given $A(x_1,y_1)$, the angle of $\angle{A}$, and the other two sides $b,c$,
it output $B(x_2,y_2)$ and $C(x_3,y_3)$.\\
- Macro \Lcs{pstTriangleAAS} create a triangle $ABC$ with given $A(x_1,y_1)$, the angle of $\angle{C}$, the angle of $\angle{A}$, and the side of $AB=c$,
it output $B(x_2,y_2)$ and $C(x_3,y_3)$.\\
- Macro \Lcs{pstTriangleASA} create a triangle $ABC$ with given $A(x_1,y_1)$, the angle of $\angle{A}$, the angle of $\angle{B}$, and the side of $AB=c$,
it output $B(x_2,y_2)$ and $C(x_3,y_3)$.

\begin{LTXexample}[width=6cm,pos=l]
\begin{pspicture}[showgrid=true](0,0)(4,4)
\psset{dotscale=0.5}\psset{PointSymbol=*}\footnotesize
\pstGeonode[PosAngle=-90](0,1){A}
\pstTriangleSSS[linecolor=red!60,PosAngle={-90,90}]{A}(3,4,5){B}{C}
\pstTriangleSSS[linecolor=blue!60,PosAngle={-90,90}]{A}(2,4.5,3.8){D}{E}
\pstTriangleSAS[linecolor=green!60,PosAngle={-90,90}]{A}(3,40,2.8){F}{G}
\pstTriangleAAS[linecolor=black!60,PosAngle={-90,90}]{A}(40,50,1.8){H}{I}
\pstTriangleASA[linecolor=purple!60,PosAngle={-90,90}]{A}(70,1.0,60){J}{K}
\end{pspicture}
\end{LTXexample}

The optional parameter \texttt{pos} setup the position of the first node $A$, it should be 'L' for left, 'R' for right, 'U' for up and 'D' for down.
If you don't input this parameter, the default value is 'L'. The following example explains how to draw an isoceles triangle with the given isoceles sides
and the vertex angle.

\begin{LTXexample}[width=6cm,pos=l]
\begin{pspicture}[showgrid=true](0,0)(4,4)
\psset{dotscale=0.5}\psset{PointSymbol=*}\footnotesize
\pstGeonode[PosAngle=205](2,2){A}
\pstTriangleSAS[linecolor=red!60,PosAngle={-90,90}]{A}(2,40,2){B}{C}
\pstTriangleSAS[linecolor=blue!60,PosAngle={-90,-90}](U){A}(2,40,2){B}{C}
\pstTriangleSAS[linecolor=purple!60,PosAngle={90,90}](D){A}(2,40,2){B}{C}
\pstTriangleSAS[linecolor=green!60,PosAngle={90,-90}](R){A}(2,40,2){B}{C}
\end{pspicture}
\end{LTXexample}

The macros \Lcs{pstTriangleIC} and \Lcs{pstTriangleOC} are used to draw the inner circle
and outer circle of triangle $ABC$.

\begin{BDef}
\Lcs{pstTriangleIC}\OptArgs\Largb{A}\Largb{B}\Largb{C}\OptArg{I}\OptArg{H}\\
\Lcs{pstTriangleOC}\OptArgs\Largb{A}\Largb{B}\Largb{C}\OptArg{O}
\end{BDef}

\begin{LTXexample}[width=5cm,pos=l]
\begin{pspicture}[showgrid](-2,-2)(2,2)
\pstTriangle[PointSymbol=square,PointSymbolC=o,
  linecolor=blue,linewidth=1.5\pslinewidth]
  (1.5,-1){A}(0,1){B}(-1,-.5){C}
\pstTriangleIC[linecolor=cyan]{A}{B}{C}
\pstTriangleOC[linecolor=red]{A}{B}{C}
\end{pspicture}
\end{LTXexample}

The center of the inner circle is called \verb|IC_O| as default and the outer circle \verb|OC_O| as default,
but you can change the node names by the optional parameters \OptArg{I} \OptArg{H} and \OptArg{O}.
The optional node name $H$ is a node on the inner circle,
so you can operate the inner circle by center $I$ and node $H$ later.

The inner center $I$, node $H$ and outer circle center $O$ are not printed out as default,
but you can setup \Lkeyword{PointSymbol} and \Lkeyword{PointName} to display them.
For example:

\begin{lstlisting}
\pstTriangleIC[PosAngle={-90,160},PointName={I,none},PointSymbol={*,none}]{A}{B}{C}[I][D]
\pstTriangleIC[PosAngle=-90,PointName=I,PointSymbol=*]{A}{B}{C}[I]
\pstTriangleOC[PosAngle=90,PointSymbol=*,PointName=X]{A}{B}{C}[X]
\end{lstlisting}

The macros \Lcs{pstTriangleGC}, \Lcs{pstTriangleHC} and \Lcs{pstTriangleEC} are used to draw the barycenter $G$, the orthocentre $H$ and the escenter $E$ of the triangle $ABC$.

\begin{BDef}
\Lcs{pstTriangleGC}\OptArgs\Largb{A}\Largb{B}\Largb{C}\Largb{G}\OptArg{$M_1$}\OptArg{$M_2$}\\
\Lcs{pstTriangleHC}\OptArgs\Largb{A}\Largb{B}\Largb{C}\Largb{H}\OptArg{$H_1$}\OptArg{$H_2$}\\
\Lcs{pstTriangleEC}\OptArgs\Largb{A}\Largb{B}\Largb{C}\Largb{E}\OptArg{$T_1$}
\end{BDef}

You can use the options of node like as \verb|PointName=...|, \verb|PosAngle=...|, \verb|PointSymbol=...| to control the output nodes $G,H,E$. But if you give the optional output parameters $M_1,M_2$, or $H_1,H_2$ or $T_1$, then you should pass the option value in list like as \verb|PointName={...}|, \verb|PosAngle={...}|, \verb|PointSymbol={...}|.
For example,

\begin{LTXexample}[width=6cm,pos=l]
\begin{pspicture}[showgrid=true](-3,-3)(3,2)
\psset{dotscale=0.5}\psset{PointSymbol=*}\footnotesize
\pstGeonode[PosAngle=90](0,1){A}
\pstGeonode[PosAngle=-90](-1,-0.6){B}
\pstGeonode[PosAngle=-90](1.5,-0.6){C}
\pstTriangleGC[PointSymbol={*,none,*},PosAngle={150,-80,30}]{A}{B}{C}{G}[M_1][M_2]
\pstTriangleHC[PointSymbol={*,*,none},PosAngle={-30,-100,30}]{A}{B}{C}{H}[H_1][H_2]
\pstTriangleEC[PointSymbol={*,none},PosAngle={90,30}]{A}{B}{C}{E_1}[T_1]
\pstTriangleEC[PointSymbol=*,PosAngle=0]{B}{C}{A}{E_2}
\pstTriangleEC[PointSymbol=*,PosAngle=180]{C}{A}{B}{E_3}
\pstLineAB{A}{B}\pstLineAB{B}{C}\pstLineAB{C}{A}
\pstCircleOA[linestyle=dashed,linecolor=gray!40]{E_1}{T_1}[30][150]
\pstLineAB[linestyle=dashed,linecolor=blue!40]{A}{M_1}
\pstLineAB[linestyle=dashed,linecolor=blue!40]{B}{M_2}
\pstLineAB[linestyle=dashed,linecolor=red!40]{A}{H_1}
\pstLineAB[linestyle=dashed,linecolor=red!40]{B}{H_2}
\end{pspicture}
\end{LTXexample}

\subsection{Angles}

Each angle is defined with three points. The vertex is the second
point. Their order is important because it is assumed that the angle is
specified in the direct order. The first command is the marking of a
right angle:


\begin{BDef}
\Lcs{pstRightAngle}\OptArgs\Largb{A}\Largb{B}\Largb{C}
\end{BDef}


\begin{sloppypar}
Valid optional arguments are \Lkeyword{RightAngleType}, \Lkeyword{RightAngleSize}, 
        \Lkeyword{RightAngleSize}, and \Lkeyword{RightAngleDotDistance}
\end{sloppypar}

The symbol is controlled by the parameter \Lkeyword{RightAngleType}
\nxLkeyval{default}. Its possible values are:

\begin{compactitem}
\item \Lkeyval{*} : standard symbol ;
\item \Lkeyval{german} : german symbol (given by U. Dirr) ;
\item \Lkeyval{suisseromand} : swiss romand symbol (given P. Schnewlin).
\end{compactitem}

\begin{sloppypar}
The only parameters controlling this command, excepting the ones which
controlled the line, is \Lkeyword{RightAngleSize} which defines the size
of the symbol \DefaultVal{0.28 unit} and \Lkeyword{RightAngleDotDistance}. For a 
right angle style \Lkeyval{german} or \Lkeyval{swissromand} the distance of the dot
is preset to 0.5 (\Lkeyval{german}) or 0.45 (\Lkeyval{swissromand}), relative to the radius.
It can be controlled by the optional argument \Lkeyword{RightAngleDotDistance} which is
preset to 1. A greater value moves the dot away from the reference point.
\end{sloppypar}




For other angles, there is the command:

\begin{BDef}
\Lcs{pstMarkAngle}\OptArgs\Largb{A}\Largb{B}\Largb{C}
\end{BDef}


\begin{sloppypar}
Valid optional arguments are \Lkeyword{MarkAngleRadius}, \Lkeyword{LabelAngleOffset},
  \Lkeyword{MarkAngleType} and
  \Lkeyword{Mark}
%
The \Lkeyword{label} can be any valid \TeX\ box, it is put at \Lkeyword{LabelSep}
\DefaultVal{1 unit} of the node in the direction of the bisector of the angle
modified by \Lkeyword{LabelAngleOffset}\DefaultVal{0} and positioned using
\Lkeyword{LabelRefPt} \DefaultVal{c}. Furthermore the arc used for marking has a radius
of \Lkeyword{MarkAngleRadius} \DefaultVal{.4~unit}.  At least, it is possible to place
an arrow using the parameter \Lkeyword{arrows}.Finally, it is possible to mark
the angle by specifying a \TeX{} command as argument of parameter \Lkeyword{Mark}.
\end{sloppypar}

\begin{LTXexample}[width=5cm,pos=l]
\begin{pspicture}[showgrid](-2,-2)(2,2)
\psset{PointSymbol=none}
\pstTriangle(2;15){A}(2;85){B}(2;195){C}
\psset{PointName=none}
\pstTriangle[PointNameA=default](2;-130){B'}(2;15){A'}(2;195){C'}
\pstTriangle[PointNameA=default](2;-55){B''}(2;15){A''}(2;195){C''}
\pstRightAngle[linecolor=red,fillstyle=solid,fillcolor=blue]{C}{B}{A}
\pstRightAngle[linecolor=blue, RightAngleType=suisseromand]{A}{B'}{C}
\pstRightAngle[linecolor=magenta, RightAngleType=german]{A}{B''}{C}
\psset{arcsep=\pslinewidth}
\pstMarkAngle[linecolor=cyan, Mark=MarkHash]{A}{C}{B}{$\theta$}
\pstMarkAngle[linecolor=red, arrows=->,fillcolor=red!30,
  fillstyle=solid]{B}{A}{C}{$\gamma$}
\end{pspicture}
\end{LTXexample}


\begin{LTXexample}[width=\linewidth,pos=t]
\begin{pspicture}(-0.5,-0.5)(9,3)
\psset{PointSymbol=none,PointNameMathSize=\scriptstyle,PointNameSep=6pt,
	 RightAngleSize=0.15,PosAngle={135,225,-45,45}}
\pstGeonode(1,2){A}(1,1){B}(2,1){C}(2,2){D}%
\pstRightAngle[fillstyle=solid,fillcolor=blue!40]{C}{B}{A}
\pstRightAngle{D}{C}{B} \pstRightAngle{A}{D}{C}
\pstRightAngle{B}{A}{D} \pspolygon(A)(B)(C)(D)
\psset{RightAngleType=suisseromand}
\pstGeonode(3,2){A}(3,1){B}(4,1){C}(4,2){D}%
\pstRightAngle[fillstyle=solid,fillcolor=blue!40]{C}{B}{A}
\pstRightAngle{D}{C}{B} \pstRightAngle{A}{D}{C}
\pstRightAngle{B}{A}{D} \pspolygon(A)(B)(C)(D)
\psset{RightAngleType=german}
\pstGeonode(5,2){A}(5,1){B}(6,1){C}(6,2){D}%
\pstRightAngle[fillstyle=solid,fillcolor=blue!40]{C}{B}{A}
\pstRightAngle{D}{C}{B} \pstRightAngle{A}{D}{C}
\pstRightAngle{B}{A}{D} \pspolygon(A)(B)(C)(D)
\end{pspicture}
\end{LTXexample}

\begin{LTXexample}[width=\linewidth,pos=t]
\begin{pspicture}[showgrid=false](-1.0,-1.0)(4,4)
\pstGeonode[PosAngle=-90](0.0,0.0){A}
\pstGeonode[PosAngle=-90](3.0,0.0){B}
\pstGeonode[PosAngle=90](1.8,2.5){C}
\pstGeonode[PosAngle=-90](1.5,0.0){D}
\pstMarkAngle[LabelSep=.6,MarkAngleRadius=.4,MarkAngleType=double]{A}{C}{B}{$\gamma$}
\pstMarkAngle[LabelSep=.6,MarkAngleRadius=.4,MarkAngleType=default]{C}{B}{A}{$\beta$}
\pstMarkAngle[LabelSep=.6,MarkAngleRadius=.4,MarkAngleType=double,fillcolor=red!30,fillstyle=solid]{B}{A}{C}{$\alpha$}
\pstMarkAngle[LabelSep=.6,MarkAngleRadius=.4,MarkAngleType=triple,fillcolor=red!30,fillstyle=solid]{B}{D}{C}{$\theta$}
\pstMarkAngle[LabelSep=.6,MarkAngleRadius=.4,MarkAngleType=triple]{C}{D}{A}{$\zeta$}
\pstLineAB{A}{B}
\pstLineAB{B}{C}
\pstLineAB{C}{A}
\pstLineAB{C}{D}
\end{pspicture}
\end{LTXexample}


\subsection{Lines, half-lines and segments}

The classical line  $(\overline{AB})$!

\begin{BDef}
\Lcs{pstLineAB}\OptArgs\Largb{A}\Largb{B}
\end{BDef}

In order to control its length\footnote{which is the comble for a
line!}, the two parameters \Lkeyword{nodesepA} et \Lkeyword{nodesepB}
specify the abscissa of the extremity of the drawing part of the line.
A negative abscissa specify an outside point, while a positive
abscissa specify an internal point. If these parameters have to be
equal, \Lkeyword{nodesep} can be used instead. The default value of these
parameters is equal to 0.

\begin{LTXexample}[width=5cm,pos=l]
\begin{pspicture}[showgrid](-2,-2)(2,2)
\pstGeonode(1,1){A}(-1,-1){B}
\pstLineAB[nodesepA=-.4,nodesepB=-1,
           linecolor=green]{A}{B}
\pstLineAB[nodesep=.4,linecolor=red]{A}{B}
\end{pspicture}
\end{LTXexample}

\vspace{10pt}

The macro \Lcs{pstLine} draws a new line with two nodes, or two coordinates
or one node and one coordinate. This macro is similar with \Lcs{pstLineAB},
but more compatible.

\begin{BDef}
\Lcs{pstLine}\OptArgs\Largb{A}\Largb{B}\\
\Lcs{pstLine}\OptArgs\Largb{A}\cAny\\
\Lcs{pstLine}\OptArgs\cAny\Largb{B}\\
\Lcs{pstLine}\OptArgs\cAny\cAny
\end{BDef}

\vspace{10pt}

The macros \Lcs{pstLineAA} and \Lcs{pstLineAS} draw a new line with one node,
the slope \texttt{angle} between the line and the horizontal axis, or the
slope \texttt{gradient} of the line, and create a new node $B$ on the line.

\begin{BDef}
\Lcs{pstLineAA}\OptArgs\Largb{A}\Largb{angle}\Largb{B}\\
\Lcs{pstLineAA}\OptArgs\cAny\Largb{angle}\Largb{B}\\
\Lcs{pstLineAS}\OptArgs\Largb{A}\Largb{gradient}\Largb{B}\\
\Lcs{pstLineAS}\OptArgs\cAny\Largb{gradient}\Largb{B}
\end{BDef}

Here are some examples:

\begin{LTXexample}[width=6cm,pos=l]
\begin{pspicture}[showgrid=true](0,0)(4,4)
\pstGeonode[PosAngle=90](1.5,1.5){A}
% draw a line with angle atan(2/1), about 63.43 degree.
\pstLineAA[linecolor=red,nodesep=-0.5,PosAngle=90]{A}{2 1 atan}{B}
\pstLineAA[linecolor=yellow,nodesep=-0.5,PosAngle=-120]{A}{-45}{C}
\pstLineAS[linecolor=green,nodesep=-0.5,PosAngle=30]{A}{-0.5}{D}
% draw a line with gradient (cos50/sin50).
\pstLineAS[linecolor=cyan,nodesep=-0.5]{A}{50 cos 50 sin div}{E}
\end{pspicture}
\end{LTXexample}

\vspace{10pt}

The macros \Lcs{pstLineCoef} is used to draw a line $ax+by+c=0$ with the given coefficents $a,b,c$,
and create two new node $A,B$ on the line.

\begin{BDef}
\Lcs{pstLineCoef}\OptArgs\Largb{a,b,c}\Largb{A}\Largb{B}
\end{BDef}

Here are some examples:

\begin{LTXexample}[width=6cm,pos=l]
\begin{pspicture}[showgrid=true](-2,-2)(2,2)
\pstLineCoef[linecolor=red!60, PosAngle={210,0}]{3,-2,1}{A}{B}
\pstLineCoef[linecolor=blue!60, PosAngle={180,0}]{4,3,2}{C}{D}
\pstLineCoef[linecolor=green!60, PosAngle={90,90}]{0,3,-3}{E}{F}
\pstLineCoef[linecolor=purple!60, PosAngle={180,180}]{4,0,4}{G}{H}
\end{pspicture}
\end{LTXexample}

\vspace{10pt}

The macro \Lcs{pstLineAbsNode} creates a new node $C$ whose abscissa
is the given value $x_1$ on the line $AB$. The macro \Lcs{pstLineOrdNode} creates a new node $C$ whose ordinate is the given value $y_1$ on the line $AB$.
You can input $x_1$ or $y_1$ as any number(e.g, 2.0),
or use \Lcs{pscalculate} or \Lcs{fpeval} to get a purely numerical result,
or use \Lcs{pstAbscissa} and \Lcs{pstOrdinate} to get the abscissa and ordinate of any other node.

\begin{BDef}
\Lcs{pstLineAbsNode}\OptArgs\Largb{A}\Largb{B}\Largb{$x_1$}\Largb{C}\\
\Lcs{pstLineOrdNode}\OptArgs\Largb{A}\Largb{B}\Largb{$y_1$}\Largb{C}
\end{BDef}

For example,

\begin{LTXexample}[width=6cm,pos=l]
\begin{pspicture}[showgrid=true](0,0)(4,4)
\pstGeonode[PosAngle=-40](0.8,0.5){A}
\pstGeonode[PosAngle=-40](1.2,1.0){B}
\pstLineAB[linecolor=red,nodesep=-0.5]{A}{B}
\pstLineAbsNode[PosAngle=-40,PointSymbol=o]{A}{B}{2.5}{C}
\pstLineOrdNode[PosAngle=-40,PointSymbol=o]{A}{B}{3.0}{D}
\pstLineAB[linecolor=blue,nodesep=-0.5]{C}{D}
\end{pspicture}
\end{LTXexample}

\vspace{10pt}

The macro \Lcs{pstProportionNode} creates the nodes $C$ and $C'$ on segment $AB$ which are satisfied $|AC|:|BC|=\lambda,\;(\lambda>0)$.
The node $C$ is inside the segment $AB$ and the node $C'$ is outside the segment $AB$, we have
\begin{equation*}
\left\{
\begin{array}{l}
x_{C}=\dfrac{x_{A}+\lambda{}x_{B}}{1+\lambda}\\
y_{C}=\dfrac{y_{A}+\lambda{}y_{B}}{1+\lambda}
\end{array}
\right.
\quad\text{and}\quad
\left\{
\begin{array}{l}
x_{C'}=\dfrac{x_{A}-\lambda{}x_{B}}{1-\lambda}\\
y_{C'}=\dfrac{y_{A}-\lambda{}y_{B}}{1-\lambda}
\end{array}
\right.
\end{equation*}

\begin{BDef}
\Lcs{pstProportionNode}\OptArgs\Largb{A}\Largb{B}\Largb{$\lambda$}\Largb{C}\Largb{C'}
\end{BDef}

You can use \Lcs{pstDistDiv} to get the ratio of two segments to $\lambda$,
we will introduce \Lcs{pstDistDiv} later.

\begin{LTXexample}[width=6cm,pos=l]
\begin{pspicture}[showgrid=true](-1,-1)(4,4)
\pstGeonode[PosAngle=-40,PointSymbol=|](0.5,1.5){A}
\pstGeonode[PosAngle=-40,PointSymbol=|](3.0,3.0){B}
\pstGeonode[PosAngle=90,linecolor=purple!60,CurveType=polyline](3,0){X}(4,0){Y}
\pstGeonode[PosAngle=90,linecolor=brown!60,CurveType=polyline](1.5,-1){X'}(4,-1){Y'}
\pstLineAB[linecolor=red,nodesep=-2.5]{A}{B}
\psset{PosAngle=-40,PointSymbol=*,dotscale=1.5}
\pstProportionNode[linecolor=yellow]{A}{B}{3.0}{C}{C'}
\pstProportionNode[linecolor=blue]{A}{B}{1.0}{D}{D'}
\pstProportionNode[linecolor=green]{A}{B}{0.2}{E}{E'}
\pstProportionNode[linecolor=brown]{A}{B}{\pstDistDiv{X}{Y}{X'}{Y'}}{F}{F'}
\end{pspicture}
\end{LTXexample}

One application of \Lcs{pstProportionNode} is used to find the bisector and out bisector of a given angle.
So we define the macro \Lcs{pstBisectorAOB} to do this work, it is more friendly than the macros
\Lcs{pstBissectBAC} and \Lcs{pstOutBissectBAC}, as it put the new node $T_1$ and $T_2$ on line $AB$, not arc $AB$.

\begin{BDef}
\Lcs{pstBisectorAOB}\OptArgs\Largb{A}\Largb{O}\Largb{B}\Largb{$T_1$}\Largb{$T_2$}
\end{BDef}

\begin{LTXexample}[width=6cm,pos=l]
\begin{pspicture}[showgrid=true](0,-1)(5,3)
\psset{unit=0.6cm}\footnotesize\psset{PointSymbol=*}
\pstGeonode[PosAngle=-90](0,0){A}
\pstGeonode[PosAngle=90](3,3){C}
\pstGeonode[PosAngle=-90](2,0){B}
\pstBisectorAOB[PosAngle={-90,-90}]{A}{C}{B}{T_1}{T_2}
\pstLineAB[linecolor=black!50]{A}{C}
\pstLineAB[linecolor=black!50]{B}{C}
\pstLineAB[linecolor=black!50]{A}{B}
\pstLineAB[linestyle=dashed,linecolor=gray]{C}{T_1}
\pstLineAB[linestyle=dashed,linecolor=gray]{C}{T_2}
\pstLineAB[linestyle=dashed,linecolor=gray]{B}{T_2}
\end{pspicture}
\end{LTXexample}

\vspace{10pt}

The four collinear points $A,B,C,D$ are called \texttt{Harmonic Conjugation Points} if their cross ratio is $-1$,
that is
$$(AB,CD)=\dfrac{AC}{BC}:\dfrac{AD}{BD}=-1$$
If given three collinear points $A,B,C$, how can we get the fourth harmonic point?
The following macro \Lcs{pstFourthHarmonicNode} is used to get the fourth harmonic point.
It create a new node $X$ on the same line, but when $A,B,C$ are not collinear, we put it at origin.

\begin{BDef}
\Lcs{pstFourthHarmonicNode}\OptArgs\Largb{A}\Largb{B}\Largb{C}\Largb{X}
\end{BDef}

\begin{LTXexample}[width=6cm,pos=l]
\begin{pspicture}[showgrid=true](-2,-2)(3,3)\footnotesize
\psset{unit=0.6cm}\psset{dotscale=0.5}\psset{PointSymbol=*}
\pstGeonode[PosAngle=-90](-2,-2){A}(5,-2){I}(-1,-2){J}
\pstLineAA[linestyle=none,PointName=none,PointSymbol=none]{A}{38}{A'}
\pstLineAbsNode[PosAngle=-80]{A}{A'}{0}{B}
\pstLineAbsNode[PosAngle=20]{A}{A'}{2.5}{C}
\pstFourthHarmonicNode[PosAngle=180,PointNameSep=0.2]{A}{B}{C}{X}
% check if A,N,M are also collinear.
\pstInterLL[PosAngle=90]{J}{X}{I}{C}{P}
\pstLineAB[linestyle=dashed]{J}{P}
\pstLineAB[linestyle=dashed]{I}{P}
\pstInterLL[PosAngle=20]{J}{B}{I}{P}{M}
\pstInterLL[PosAngle=140]{I}{B}{J}{P}{N}
\pstLineAB[linestyle=dashed]{J}{M}
\pstLineAB[linestyle=dashed]{I}{N}
\pstLineAB[linestyle=dashed]{A}{M}
\pstLineAB[linestyle=dashed]{A}{I}
\pstLineAB{A}{C}
\end{pspicture}
\end{LTXexample}

\vspace{10pt}

If you want to draw a node like \textsf{'Given $EF$, please find node $C$ on $AB$ such that $AC=EF$'},
you can use the macro \Lcs{pstLocateAB} to do this, it can seek the node $C$ from $A$ to $B$ with the
specified length $L$, which can be got from \Lcs{pstDist}, \Lcs{pstDistConst}, \Lcs{pstDistAdd}, \Lcs{pstDistSub},
etc.

\begin{BDef}
\Lcs{pstLocateAB}\OptArgs\Largb{A}\Largb{B}\Largb{$L$}\Largb{C}
\end{BDef}

Note that seek from $B$ will get the node $C$ in the reverse order, for example,

\begin{LTXexample}[width=5cm,pos=l]
\begin{pspicture}[showgrid=true](-3,-3)(3,3)\footnotesize
\psset{unit=0.5cm}\psset{dotscale=0.5}\psset{PointSymbol=*}
\pstGeonode[PosAngle=90,CurveType=polyline](-2,0){A}(-1,0){B}
\pstGeonode[PosAngle=90,CurveType=polyline](-2,1){A'}(0,1){B'}
\pstLocateAB[PosAngle=90]{A}{B}{\pstDist{A'}{B'}}{C}
\pstLocateAB[PosAngle=90]{B}{A}{\pstDist{A'}{B'}}{C'}
\psset{linestyle=dashed,SegmentSymbol=MarkHashh,MarkAngle=90}
\pstSegmentMark{A'}{B'}\pstSegmentMark{B}{C'}\pstSegmentMark{A}{C}
\pstGeonode[PosAngle=90,CurveType=polyline](-3,-2){D}(3,-4){E}
\pstGeonode[PosAngle=90,CurveType=polyline](-4,2){D'}(-3,4){E'}
\pstLocateAB[PosAngle=90]{D}{E}{\pstDist{D'}{E'}}{F}
\pstLocateAB[PosAngle=90]{E}{D}{\pstDist{D'}{E'}}{F'}
\psset{linestyle=dashed,SegmentSymbol=MarkHashhh,MarkAngle=90}
\pstSegmentMark{D'}{E'}\pstSegmentMark{E}{F'}\pstSegmentMark{D}{F}
\pstGeonode[PosAngle=0,CurveType=polyline](2,0){I}(2,1){J}
\pstGeonode[PosAngle=0,CurveType=polyline](3,2){I'}(3,4){J'}
\pstLocateAB[PosAngle=0]{I}{J}{\pstDist{I'}{J'}}{K}
\pstLocateAB[PosAngle=0]{J}{I}{\pstDist{I'}{J'}}{K'}
\psset{linestyle=dashed,SegmentSymbol=MarkHashhh,MarkAngle=45}
\pstSegmentMark{I'}{J'}\pstSegmentMark{J}{K'}\pstSegmentMark{I}{K}
\end{pspicture}
\end{LTXexample}

\vspace{10pt}

If you want to draw a node like \textsf{'Given $EF$, please extend $AB$ to $C$ such that $BC=EF$'},
you can use the macro \Lcs{pstExtendAB} to do this, it can extend $AB$ from $B$ to one node with the
specified length $L$, which can be got from \Lcs{pstDist}, \Lcs{pstDistConst}, \Lcs{pstDistAdd}, \Lcs{pstDistSub},
etc.

\begin{BDef}
\Lcs{pstExtendAB}\OptArgs\Largb{A}\Largb{B}\Largb{$L$}\Largb{C}
\end{BDef}

Note that extend $BA$ to $C$ will get the node $C$ in the reverse order, for example,

\begin{LTXexample}[width=5cm,pos=l]
\begin{pspicture}[showgrid=true](-3,-3)(3,3)\footnotesize
\psset{unit=0.5cm}\psset{dotscale=0.5}\psset{PointSymbol=*}
\pstGeonode[PosAngle=90,CurveType=polyline](-2,0){A}(-1,0){B}
\pstGeonode[PosAngle=90,CurveType=polyline](-2,1){A'}(0,1){B'}
\pstExtendAB[PosAngle=90]{A}{B}{\pstDist{A'}{B'}}{C}
\pstExtendAB[PosAngle=90]{B}{A}{\pstDist{A'}{B'}}{C'}
\psset{linestyle=dashed,SegmentSymbol=MarkHashh,MarkAngle=90}
\pstSegmentMark{A'}{B'}\pstSegmentMark{B}{C}\pstSegmentMark{A}{C'}
\pstGeonode[PosAngle=90,CurveType=polyline](-2,-2){D}(0,-3){E}
\pstGeonode[PosAngle=90,CurveType=polyline](-4,2){D'}(-3,4){E'}
\pstExtendAB[PosAngle=90]{D}{E}{\pstDist{D'}{E'}}{F}
\pstExtendAB[PosAngle=90]{E}{D}{\pstDist{D'}{E'}}{F'}
\psset{linestyle=dashed,SegmentSymbol=MarkHashhh,MarkAngle=90}
\pstSegmentMark{D'}{E'}\pstSegmentMark{E}{F}\pstSegmentMark{D}{F'}
\pstGeonode[PosAngle=0,CurveType=polyline](2,0){I}(2,1){J}
\pstGeonode[PosAngle=0,CurveType=polyline](3,2){I'}(3,4){J'}
\pstExtendAB[PosAngle=0]{I}{J}{\pstDist{I'}{J'}}{K}
\pstExtendAB[PosAngle=0]{J}{I}{\pstDist{I'}{J'}}{K'}
\psset{linestyle=dashed,SegmentSymbol=MarkHashhh,MarkAngle=45}
\pstSegmentMark{I'}{J'}\pstSegmentMark{J}{K}\pstSegmentMark{I}{K'}
\end{pspicture}
\end{LTXexample}

You can find the node $C$ on segment $AB$ satisfied $|AC|$:$|AB|$=\Lkeyword{DistCoef}
using \Lcs{pstTranslation}, but it can't do the same thing like \Lcs{pstLocateAB} and \Lcs{pstExtendAB}
when the given segment $EF$ is not parallel with $AB$, it will be introduced in the later sections.

\vspace{10pt}

If you want to find the inversion point $C'$ of $C$ to the inversion center $O$ with inversion raduis $R$,
that is, the point $C'$ is satisfied the inversion transform equation
$$|OC|\times|OC'|=R^2$$
you can use the macro \Lcs{pstInversion} to do this work.
In fact, we use the macro \Lcs{pstLocateAB} to implement this macro
by passing the value $\dfrac{R^2}{|OC|}$ to parameter length.

\begin{BDef}
\Lcs{pstInversion}\OptArgs\Largb{O}\Largb{A}\Largb{C}\Largb{C'}
\end{BDef}

It is possible to omit the parameter $A$ and then to specify the inversion radius or
the inversion diameter using the parameters \Lkeyword{Radius} and \Lkeyword{Diameter},
which will be introduced in the next section.

It is clear that the inversion mapping of a line is a circle, and the inversion mapping
of a point on the inversion circle is itself.

\begin{LTXexample}[width=6cm,pos=l]
\begin{pspicture}[showgrid=true](-1,-2)(4,3)
\psset{dotscale=0.5}\psset{PointSymbol=*}\footnotesize
\def\ra{1.5}
\pstGeonode[PosAngle=180](1,1){O}
\pstCircleOA[linecolor=red!50,Radius=\pstDistVal{\ra}]{O}{}
\pstCircleRotNode[PosAngle=180,RotAngle=180,Radius=\pstDistVal{\ra}]{O}{}{A}
\pstInversion[PosAngle=0,Radius=\pstDistVal{\ra}]{O}{}{A}{A'}
\pstGeonode[PosAngle=0](3,3){C}
\pstInversion[PosAngle=100,Radius=\pstDistVal{\ra}]{O}{}{C}{C'}
\pstLineAB{O}{C}\pstLineAB{O}{C'}
\pstGeonode[PosAngle=0](3,1.5){D}
\pstInversion[PosAngle=90,Radius=\pstDistVal{\ra}]{O}{}{D}{D'}
\pstLineAB{O}{D}\pstLineAB{O}{D'}
\pstGeonode[PosAngle=0](3,0){E}
\pstInversion[PosAngle=-90,Radius=\pstDistVal{\ra}]{O}{}{E}{E'}
\pstLineAB{O}{E}\pstLineAB{O}{E'}
\pstGeonode[PosAngle=0](3,-2){F}
\pstInversion[PosAngle=-120]{O}{A}{F}{F'}
\pstLineAB{O}{F}\pstLineAB{O}{F'}
\pstLineAB[linecolor=black!50]{C}{F}
\pstCircleABC[linestyle=dashed,linecolor=blue!40,PosAngle=0]{C'}{D'}{E'}{O'}
\end{pspicture}
\end{LTXexample}

\vspace{10pt}

If you want to find the node $C$ from $A$ to $B$, such that $C$ is the golden section of the
given segments $AB$, that is,
$$|AC|^2=|AB|\times|BC|\quad\text{or}\quad{}AC:AB=BC:AC\quad\text{or}\quad{}AC=\dfrac{\sqrt{5}-1}{2}AB$$
you can use the macro \Lcs{pstGoldenMean} to do this work.

\begin{BDef}
\Lcs{pstGoldenMean}\OptArgs\Largb{A}\Largb{B}\Largb{C}
\end{BDef}

In fact, we use the macro \Lcs{pstLocateAB} to implement this macro
by passing the value $\dfrac{\sqrt{5}-1}{2}|AB|$ to parameter length.

\begin{LTXexample}[width=6cm,pos=l]
\begin{pspicture}[showgrid=true](0,1)(4,4)
\psset{dotscale=0.5}\psset{PointSymbol=*}\footnotesize
\pstGeonode[PosAngle=-90,CurveType=polyline](0,1){A}(4,2){B}
\pstGoldenMean[PosAngle=-90,PointSymbol=o]{A}{B}{C}
% geometrical method to draw the golden point
\pstMiddleAB[PosAngle=-90]{A}{B}{M}
\pstRotation[RotAngle=-90,PosAngle=90]{B}{M}[N]
\pstLineAB[linestyle=dashed,linecolor=gray!60]{A}{N}
\pstLineAB[linestyle=dashed,linecolor=gray!60]{B}{N}
\pstInterLC[PointNameA=,PosAngleB=90]{N}{A}{N}{B}{B0}{E}
\pstCircleOA[linecolor=green!60,linestyle=dashed]{N}{B}[190][300]
\pstCircleOA[linecolor=purple!60,linestyle=dashed]{A}{E}[0][60]
\end{pspicture}
\end{LTXexample}

\vspace{10pt}

If you want to find the node $C$ from $A$ to $B$, such that $AC$ is the geometric mean of two
given segments $DE$ of $FG$, that is,
$$|AC|^2=|DE|\times|FG|$$
you can use the macro \Lcs{pstGeometricMean} to do this work.
It also can be used to draw a circle when given two points on the circle,
and a line tangents to the circle.

\begin{BDef}
\Lcs{pstGeometricMean}\OptArgs\Largb{A}\Largb{B}\Largb{$L_1$}\Largb{$L_2$}\Largb{C}
\end{BDef}

In fact, we use the macro \Lcs{pstLocateAB} to implement this macro
by passing the value $\sqrt{L_1\times{}L_2}$ to parameter length.
The length $L_1$ and $L_2$ can be got from \Lcs{pstDist}, \Lcs{pstDistConst},
\Lcs{pstDistAdd}, \Lcs{pstDistSub}, etc.

\begin{LTXexample}[width=6cm,pos=l]
\begin{pspicture}[showgrid=true](0,-3)(4,3)
\psset{dotscale=0.5}\psset{PointSymbol=*}\footnotesize
\pstGeonode[PosAngle=90](0,1){A}
\pstGeonode[PosAngle=0](3.2,2){C}(3.2,1){D}(3.2,-2){E}
\pstGeometricMean[PosAngle=90]{C}{A}{\pstDistAB{C}{D}}{\pstDistAB{D}{E}}{B}
\pstCircleABC[linecolor=gray!60]{B}{D}{E}{O}
\pstLineAB[linecolor=red!40]{C}{D}
\pstLineAB[linecolor=blue!40]{D}{E}
\psset{linestyle=dashed}
\pstLineAB[linecolor=purple!80]{C}{A}
\pstLineAB{D}{B}\pstLineAB{E}{B}
\end{pspicture}
\end{LTXexample}

\vspace{10pt}

If you want to find the node $C$ from $A$ to $B$, such that $AC$ is the harmonic mean of two
given segments $DE$ of $FG$, that is,
$$\dfrac{1}{|AC|}=\dfrac{1}{2}(\dfrac{1}{|DE|}+\dfrac{1}{|FG|})$$
you can use the macro \Lcs{pstHarmonicMean} to do this work.

\begin{BDef}
\Lcs{pstHarmonicMean}\OptArgs\Largb{A}\Largb{B}\Largb{$L_1$}\Largb{$L_2$}\Largb{C}
\end{BDef}

In fact, we use the macro \Lcs{pstLocateAB} to implement this macro
by passing the value $\dfrac{2L_1L_2}{L_1+L_2}$ to parameter length.
The length $L_1$ and $L_2$ can be got from \Lcs{pstDist}, \Lcs{pstDistConst},
\Lcs{pstDistAdd}, \Lcs{pstDistSub}, etc.

\begin{LTXexample}[width=6cm,pos=l]
\begin{pspicture}[showgrid=true](0,-3)(4,3)
\psset{dotscale=0.5}\psset{PointSymbol=*}\footnotesize
\pstGeonode[PosAngle=90](1,2){A}
\pstGeonode[PosAngle=-90](0,-2){C}(2.5,-2){D}(4,-2){E}
\pstHarmonicMean[PosAngle=60]{D}{A}{\pstDistAB{C}{D}}{\pstDistAB{D}{E}}{B}
\pstLineAB[linecolor=red!40]{C}{D}
\pstLineAB[linecolor=blue!40]{D}{E}
\pstLineAB[linecolor=purple!80]{A}{D}
\pstLineAB[linestyle=dashed]{C}{B}
\pstLineAB[linestyle=dashed]{E}{B}
\pstLineAB{C}{A}\pstLineAB{E}{A}
\end{pspicture}
\end{LTXexample}

\subsection{Distance}
Like as coordinates, the distance works at the PostScript level,
that is, it should be used where the code is interpreted by PostScript engine,
but not \TeX\ engine. There were three macros to operate the distance before v1.66:

\begin{BDef}
\Lcs{pstDistAB}\Largb{A}\Largb{B}\\
\Lcs{pstDistVal}\Largb{l}\\
\Lcs{pstDistCalc}\Largb{expr}
\end{BDef}

The first specifies a distance between two points. The second macro can be used to
specify an explicit numerical value $l$, which is in \texttt{User coordinate}.
The third one uses the \Lcs{pscalculate} to calculate
the result of the input expression, which is in \texttt{User coordinate} too.
The parameter \Lkeyword{DistCoef} can be used to specify
a coefficient to reduce or enlarge the result distance.
This parameter will come into effect if it is specified before these macros.

After v1.66, We provide three macros which disable the effect of parameter \Lkeyword{DistCoef}
one to one as following:

\begin{BDef}
\Lcs{pstDist}\Largb{A}\Largb{B}\\
\Lcs{pstDistConst}\Largb{l}\\
\Lcs{pstDistExpr}\Largb{expr}
\end{BDef}

We provide the macro \Lcs{pstDistCoef} to reduce or enlarge a given distance explicitly,
for example: \verb|\pstDistCoef{\pstDist{A}{B}}|, or use macro \Lcs{pstDistMul} to multiply
the input coefficient.

\vspace{10pt}\noindent{}{\large{\textbf{Note}}}:
The series of macros \verb|\pstDist*| get the length result in the \texttt{Screen coordinate},
so you need to convert the length to the \texttt{User coordinate} by macro \Lcs{pstUserDist},
when use them where need the user coordinate numbers, e.g,

\begin{lstlisting}
\pnode(! 1 \pstUserDist{\pstDistAdd{A}{B}{C}{D}}){A}
\pstMoveNode(0,\pstUserDist{\pstDistAdd{A}{B}{C}{D}}){A}{E}
\end{lstlisting}

You can convert the distance in \texttt{User coordinate} to \texttt{Screen coordinate} by
macro \Lcs{pstScreenDist}, it is just another name of \Lcs{pstDistConst}. As we said before,
macros \Lcs{pstAbscissa} and \Lcs{pstOrdinate} give the coordinate of one node in
\texttt{User coordinate}, so if you want to draw a circle using them, you should type:\\[8pt]
\verb|\pstCircleOA[Radius=\pstDistConst{\pstAbscissa{A}}]{A}{}|

\vspace{10pt}

It is possible to use the raw PostScript command to make more complex arithmetic operations.
In order to hide the lower level Postscipt language, we add more macros for distance
addition and subtraction, such as \Lcs{pstDistAdd}[Val/Coef] and \Lcs{pstDistSub}[Val/Coef], etc.
These macros can be used to calculate the Radius or Diameter to define a circle.

The macros \Lcs{pstDistAdd} and \Lcs{pstDistSub} are used to get the addition and subtraction
of the given segments $AB$ and $CD$. The macro \Lcs{pstDistDiv} is used to
get the length ratio of the given segments $AB$ and $CD$, you can pass the ratio to
macro \Lcs{pstProportionNode}, or setup the ratio to parameter \Lkeyword{DistCoef} in
macro \Lcs{pstTranslation}, or pass the ratio to any \verb|\pstDist|* macros which need
a $\lambda$ parameter.

\begin{BDef}
\Lcs{pstDistMul}\Largb{A}\Largb{B}\Largb{$\lambda$}\\
\Lcs{pstDistAdd}\Largb{A}\Largb{B}\Largb{C}\Largb{D}\\
\Lcs{pstDistAddVal}\Largb{A}\Largb{B}\Largb{$\lambda$}\Largb{$L$}\\
\Lcs{pstDistAddCoef}\Largb{A}\Largb{B}\Largb{$\lambda_1$}\Largb{C}\Largb{D}\Largb{$\lambda_2$}\\
\Lcs{pstDistSub}\Largb{A}\Largb{B}\Largb{C}\Largb{D}\\
\Lcs{pstDistSubVal}\Largb{A}\Largb{B}\Largb{$\lambda$}\Largb{$L$}\\
\Lcs{pstDistSubCoef}\Largb{A}\Largb{B}\Largb{$\lambda_1$}\Largb{C}\Largb{D}\Largb{$\lambda_2$}\\
\Lcs{pstDistDiv}\Largb{A}\Largb{B}\Largb{C}\Largb{D}
\end{BDef}

In these macros, the length $L$ is a numerical value in the \texttt{Screen Coordinate},
so it is possible to pass the result of any macros like \verb|\pstDist| to it.
$\lambda$ is a numerical value to multiply, and most important is that the parameter
\Lkeyword{DistCoef} doesn't take effect any more.
It is better to describe in formula:
\\
- macro \Lcs{pstDistAB} get the screen length of $\text{DistCoef}*|AB|$\\
- macro \Lcs{pstDistVal} get the screen length of $\text{DistCoef}*l$\\
- macro \Lcs{pstDistCalc} get the screen length of $\text{DistCoef}*\text{expr}$\\
- macro \Lcs{pstDistCoef} get the screen length of $\text{DistCoef}*\text{<arg>}$\\
- macro \Lcs{pstDist} get the screen length of $|AB|$\\
- macro \Lcs{pstDistConst} get the screen length of $l$\\
- macro \Lcs{pstDistExpr} get the screen length of $\text{expr}$\\
- macro \Lcs{pstDistMul} get the screen length of $\lambda{}|AB|$\\
- macro \Lcs{pstDistAdd} get the screen length of $|AB|+|CD|$\\
- macro \Lcs{pstDistAddVal} get the screen length of $\lambda{}|AB|+L$\\
- macro \Lcs{pstDistAddCoef} get the screen length of $\lambda_1{}|AB|+\lambda_2{}|CD|$\\
- macro \Lcs{pstDistSub} get the screen length of $abs(|AB|-|CD|)$\\
- macro \Lcs{pstDistSubVal} get the screen length of $abs(\lambda{}|AB|-L)$\\
- macro \Lcs{pstDistSubCoef} get the screen length of $abs(\lambda_1{}|AB|-\lambda_2{}|CD|)$\\
- macro \Lcs{pstDistDiv} get the the ratio of length $|AB|:|CD|$

For example, the following one draw a circle with radius length $2|AB|+3|CD|+4|EF|$,
it shows how to operate more than two distances.
\begin{lstlisting}
\pstCircleOA[Radius=\pstDistAddVal{A}{B}{2.0}{\pstDistAddCoef{C}{D}{3.0}{E}{F}{4.0}}]{A}{}
\end{lstlisting}

Another example is for \Lcs{pstDistMul}, the old code like as
\begin{lstlisting}
\pstCircleOA[DistCoef=1 3 div,Radius=\pstDistAB{A}{B}]{O}{}
\pstCircleOA[DistCoef=1 3 div,Radius=\pstDistAB{A}{B}]{A}{B}{O}{}{I}{J}
\pstInterCC[DistCoef=1 3 div,RadiusA=\pstDistAB{A}{B},DistCoef=none,RadiusA=\pstDistAB{C}{D}]{O1}{}{O2}{}{I}{J}
\end{lstlisting}
could be simplified to
\begin{lstlisting}
\pstCircleOA[Radius=\pstDistMul{A}{B}{1 3 div}]{O}{}
\pstInterLC[Radius=\pstDistMul{A}{B}{1 3 div}]{A}{B}{O}{}{I}{J}
\pstInterCC[RadiusA=\pstDistMul{A}{B}{1 3 div},RadiusA=\pstDistAB{C}{D}]{O1}{}{O2}{}{I}{J}
\end{lstlisting}

\vspace{10pt}\noindent{}{\Large{\textbf{Important}}}!
We recommend that you should use the distance macros which disable the parameter \Lkeyword{DistCoef}
instead of \verb|\pstDistAB|, \verb|\pstDistVal| or \verb|\pstDistCalc|,
when you need to pass their result into \Lcs{pstDistAddVal} or \Lcs{pstDistSubVal},
as it will give you the error result sometimes.
For example, the following code \\[8pt]
\verb|\pstDistAddVal{A}{B}{2.0}{\pstDistAB{C}{D}}|\\[8pt]
is expected to get the length of $2|AB|+|CD|$. If current \Lkeyword{DistCoef} is $\lambda$,
then it will give the error result as $2|AB|+\lambda|CD|$.
The right way is \\[8pt]
\verb|\pstDistAddVal{A}{B}{2.0}{\pstDist{C}{D}}|

\vspace{10pt}
At last, we provide a macro named \Lcs{pstDistABC} to get the distance from $C$ to line $AB$.

\begin{BDef}
\Lcs{pstDistABC}\Largb{A}\Largb{B}\Largb{C}
\end{BDef}

\begin{LTXexample}[width=6cm,pos=l]
\begin{pspicture}[showgrid=true](-1,-1)(3,3)
\psset{dotscale=0.5}\psset{PointSymbol=*}\footnotesize
\pstGeonode[PosAngle=-90](0,0){O}
\pstGeonode[PosAngle=-90](2,0){A}
\pstGeonode[PosAngle=90](1,1.5){B}
\pstCircleOA[linecolor=red,Radius=\pstDistABC{O}{A}{B}]{B}{}
\pstCircleOA[linecolor=blue,Radius=\pstDistABC{B}{O}{A}]{A}{}
\pstCircleOA[linecolor=green,Radius=\pstDistABC{A}{B}{O}]{O}{}
\pstLineAB[linecolor=red]{O}{A}
\pstLineAB[linecolor=blue]{B}{O}
\pstLineAB[linecolor=green]{A}{B}
\end{pspicture}
\end{LTXexample}

\subsection{Circles}

A circle can be defined either with its center and a point of its
circumference, or with two diameterly opposed points. There are two
commands:

\begin{BDef}
\Lcs{pstCircleOA}\OptArgs\Largb{O}\Largb{A}\OptArg{angleA}\OptArg{angleB}\\
\Lcs{pstCircleAB}\OptArgs\Largb{A}\Largb{B}\OptArg{angleA}\OptArg{angleB}
\end{BDef}

\Lcs{pstCircleOA} draws the circle of center $O$ crossing $A$ from \Lkeyword{angleA} to \Lkeyword{angleB},
going counter clockwise. Possible options are \Lkeyword{Radius} and \Lkeyword{Diameter}.

\Lcs{pstCircleAB} draws the circle of diameter $AB$ with the same options.

For the first macro, it is possible to omit the second point and then
to specify a radius or a diameter using the parameters \Lkeyword{Radius}
\footnote{The package \texttt{pst-fractal} also defines an optional key
named \texttt{Radius}, if you need to use this package with \texttt{pst-eucl},
you need to setup the key \texttt{Radius} as following:
\texttt{\textbackslash{}psset[pst-eucl]\{Radius=\textbackslash{}pstDistVal\{3\}\}}.}
and \Lkeyword{Diameter}. The values of these parameters can be specified
with one of the \verb|\pstDist|* series macros.

We will see later how to draw the circle crossing three points.
With this package, it becomes possible to draw:
\begin{compactitem}
\item {\color{red} the circle of center $A$ crossing $B$;}
\item {\color{green} the circle of center $A$ whose radius is $AC$;}
\item {\color{blue} the circle of center $A$ whose radius is $BC$;}
\item {\color{Sepia} the circle of center $B$ whose radius is $AC$;}
\item {\color{Aquamarine} the circle of center $B$ of diameter $AC$;}
\item {\color{RoyalBlue} the circle whose diameter is $BC$.}
\end{compactitem}

\begin{LTXexample}[width=6cm,pos=l]
\begin{pspicture}[showgrid](-3,-3)(3,3)\footnotesize
\psset{unit=0.65cm}\psset{dotscale=0.5}\psset{PointSymbol=*}
\pstGeonode[PosAngle={0,-135,90},PointSymbol={*,*,square}](1,0){A}(-2,-1){B}(0,1){C}
\pstCircleOA[linecolor=red]{A}{B}
\pstCircleOA[linecolor=green, Radius=\pstDistMul{A}{C}{2 3 div}]{A}{}
\pstCircleOA[linecolor=blue, Radius=\pstDistAB{B}{C}]{A}{}[45][270]
\pstCircleOA[linecolor=Sepia, Radius=\pstDistAB{A}{C}]{B}{}
\pstCircleOA[linecolor=Aquamarine, Diameter=\pstDistAB{A}{C}]{B}{}[80][320]
\pstCircleAB[linecolor=RoyalBlue]{B}{C}
\end{pspicture}
\end{LTXexample}

\vspace{10pt}

The following example show how to use the more complex distance macros,
and the parameter to fill the circle.

\begin{LTXexample}[width=6cm,pos=l]
\begin{pspicture}[showgrid=true](-3,-3)(3,3)\footnotesize
\psset{unit=0.65cm}\psset{dotscale=0.5}\psset{PointSymbol=*}
\pstGeonode[PosAngle=90,CurveType=polyline](0,0){A}(1,0){B}
\pstGeonode[PosAngle=90,CurveType=polyline](0,1){A'}(2,1){B'}
\pstCircleOA[linecolor=gray,Radius=\pstDistAdd{A}{B}{A'}{B'}]{A}{} % R=|AB|+|A'B'|
\pstCircleOA[linecolor=red,Radius=\pstDistAddVal{A}{B}{1.0}{\pstDistConst{0.5}}]{A}{} % R=|AB|+0.5
\pstCircleOA[linecolor=blue,Radius=\pstDistAddCoef{A}{B}{0.5}{A'}{B'}{1.5}]{A}{} % R=0.5|AB|+1.5|A'B'|
\pstCircleOA[linecolor=green,Radius=\pstDistSub{A}{B}{A'}{B'}]{B'}{} % R=|AB|-|A'B'|
\pstCircleOA[linecolor=brown,Radius=\pstDistSubCoef{A}{B}{1.8}{A'}{B'}{0.5}]{A}{} % R=1.8|AB|-0.5|A'B'|
\pnode(-1.5,-2){D}
\pstCircleOA[linecolor=pink,fillstyle=solid,fillcolor=pink!40,Radius=\pstDistMul{A}{B}{0.8}]{D}{} % R=0.8|AB|
\psdot(D)\uput{0.2}[-45](D){$D$}
\pstCircleOA[linecolor=purple,Radius=\pstDistConst{\pstAbscissa{D}} abs]{D}{} % R=|D.x|
\end{pspicture}
\end{LTXexample}

The last row set the absolute value of the abscissa of node $D$ to \Lkeyword{Radius},
and then draw a circle at center $D$. Note that it does not work before v1.67,
as the \Lcs{pstCircleOA} and \Lcs{pstCircleAB} were implemented with a \Lcs{rput} command,
which will set the center $D$'s coordinate to origin, it causes that the Radius was set to zero
and none circle will be draw out, so we remove the \Lcs{rput} code in v1.67,
and everything works well now.

\subsection{Circle arcs}

\begin{BDef}
\Lcs{pstArcOAB}\OptArgs\Largb{O}\Largb{A}\Largb{B}\\
\Lcs{pstArcnOAB}\OptArgs\Largb{O}\Largb{A}\Largb{B}
\end{BDef}

These two macros draw circle arcs, $O$ is the center, the radius
defined by $OA$, the beginning angle given by $A$ and the final angle
by $B$. Finally, the first macro draws the arc in the direct way,
whereas the second in the indirect way. It is not necessary that the
two points are at the same distance of $O$.

\begin{LTXexample}[width=5cm,pos=l]
\begin{pspicture}[showgrid](-2,-2)(2,2)
\pstGeonode[PosAngle={180,0}](1.5;24){A}(1.8;-31){B}
\pstGeonode{O}
\psset{arrows=->,arrowscale=2}
\pstArcOAB[linecolor=red,linewidth=1pt]{O}{A}{B}
\pstArcOAB[linecolor=blue,linewidth=1pt]{O}{B}{A}
\pstArcnOAB[linecolor=green]{O}{A}{B}
\pstArcnOAB[linecolor=magenta]{O}{B}{A}
\end{pspicture}
\end{LTXexample}

\subsection{Circle nodes}

Do you want to draw a point on the circle? A point can be positioned on a circle
using its rotation angle by macro \Lcs{pstCircleNode} or \Lcs{pstCircleRotNode}.
The first \Lcs{pstCircleNode} requires an explicit parameter angle $\theta$ to calculate the point;
but the second \Lcs{pstCircleRotNode} requires an implicit parameter \Lkeyword{RotAngle} to calculate the point,
If you not set \Lkeyword{RotAngle}, the default value is $60^\circ$.

The circle is defined by center $O$ and point $A$ on the circle or \Lkeyword{Radius} or \Lkeyword{Diameter} in parameter.

\begin{BDef}
\Lcs{pstCircleNode}\OptArgs\Largb{O}\Largb{A}\Largb{$\theta$}\Largb{X}\\
\Lcs{pstCircleRotNode}\OptArgs\Largb{O}\Largb{A}\Largb{X}
\end{BDef}

\begin{LTXexample}[width=6cm,pos=l]
\begin{pspicture}[showgrid=true](-1,-1)(4,4)
\psset{dotscale=0.5}\psset{PointSymbol=*}\footnotesize
\psset{Radius=\pstDistVal{2.0}}
\pstGeonode[PosAngle=0](1.5,1.5){O}
\pstCircleOA[linecolor=red]{O}{}
\pstCircleRotNode[PosAngle=0,RotAngle=0]{O}{}{A}
\pstCircleRotNode[PosAngle=60]{O}{}{B} % default 60 degree
\pstCircleRotNode[PosAngle=90,RotAngle=90]{O}{}{C}
\pstCircleRotNode[PosAngle=150,RotAngle=\pscalculate{3*360/7}]{O}{}{D}
\pstCircleRotNode[PosAngle=180,RotAngle=180]{O}{}{E}
\pstCircleRotNode[PosAngle=230,RotAngle=230]{O}{}{F}
\pstCircleRotNode[PosAngle=270,RotAngle=270]{O}{}{G}
\pstCircleNode[PosAngle=-45]{O}{}{-45}{H}
\end{pspicture}
\end{LTXexample}

Sometimes we need to draw a chord with the given length from the start node,
it is not possible to get the end node via the already defined macros,
so we provide the macro \Lcs{pstCircleChordNode} to do this work.
This macro find the node $X$ on the circle such that the length of chord $AX$ is the given value $L$,
which can be got from \Lcs{pstDist}, \Lcs{pstDistConst}, \Lcs{pstDistAdd}, \Lcs{pstDistSub}, etc.

\begin{BDef}
\Lcs{pstCircleChordNode}\OptArgs\Largb{O}\Largb{A}\Largb{$L$}\Largb{X}
\end{BDef}

The circle is just defined by center $O$ and point $A$ in this macro,
so you can't omit the parameter $A$.

The direction to find node $X$ is anti-clockwise by default.
The parameter \Lkeyword{CurvAbsNeg}\DefaultVal{false} can change this behavior.

At last, the chord length $L$ chouldn't large than the diameter of the circle,
else we will put the node $X$ at origin.

\begin{LTXexample}[width=6cm,pos=l]
\begin{pspicture}[showgrid=true](-1,-1)(3,3)
\psset{dotscale=0.5}\psset{PointSymbol=*}\footnotesize
\pstGeonode[PosAngle={180,0}](1,1){O}(2.5,1){A}
\pstCircleOA[linecolor=red]{O}{A}
\pstCircleChordNode[PosAngle=60]{O}{A}{\pstDistConst{1}}{B}
\pstCircleChordNode[PosAngle=90]{O}{A}{\pstDistConst{2}}{C}
\pstCircleChordNode[PosAngle=-30,CurvAbsNeg=true]{O}{A}{\pstDistConst{1}}{B'}
\pstCircleChordNode[PosAngle=-90,CurvAbsNeg=true]{O}{A}{\pstDistConst{2}}{C'}
\pstLineAB{O}{A}\pstLineAB{O}{B}\pstLineAB{O}{C}
\pstLineAB{O}{B'}\pstLineAB{O}{C'}
\pstLineAB{A}{B}\pstLineAB{A}{C}
\pstLineAB{A}{B'}\pstLineAB{A}{C'}
\end{pspicture}
\end{LTXexample}

\vspace{10pt}

A point can be positioned on a circle using its absolute abscissa or ordinate too.
You can input $x_1$ or $y_1$ as any number(e.g, 2.0), or use \Lcs{pscalculate} or \Lcs{fpeval} to generate the value,
or use \Lcs{pstAbscissa} and \Lcs{pstOrdinate} to get the abscissa and ordinate of any other node.

\begin{BDef}
\Lcs{pstCircleAbsNode}\OptArgs\Largb{O}\Largb{A}\Largb{$x_1$}\Largb{C}\Largb{D}\\
\Lcs{pstCircleOrdNode}\OptArgs\Largb{O}\Largb{A}\Largb{$y_1$}\Largb{C}\Largb{D}
\end{BDef}

for example,
\begin{LTXexample}[width=6cm,pos=l]
\begin{pspicture}[showgrid=true](-1,-1)(4,4)
\pstGeonode[PosAngle=60](1.5,1.5){O}
\pstGeonode[PosAngle=-30](2.5,0){A}
\pstCircleOA[linecolor=red]{O}{A}
\pstCircleAbsNode[PosAngle={-60,60},PointSymbol=*]{O}{A}{1.0}{C}{D}
\pstCircleOrdNode[PosAngle={150,30},PointSymbol=*]{O}{A}{1.0}{E}{F}
\pstLineAB[linestyle=dashed,linecolor=gray!40,nodesep=-0.5]{C}{D}
\pstLineAB[linestyle=dashed,linecolor=gray!40,nodesep=-0.5]{E}{F}
\end{pspicture}
\end{LTXexample}

\vspace{10pt}

A point can be positioned on a circle using its curved abscissa, that is,
the arc length from a given node.

\begin{BDef}
\Lcs{pstCurvAbsNode}\OptArgs\Largb{O}\Largb{A}\Largb{B}\Largb{Abs}
\end{BDef}

\begin{sloppypar}
Possible optional arguments are \Lkeyword{PointSymbol}, \Lkeyword{PosAngle},
  \Lkeyword{PointName}, \Lkeyword{PointNameSep}, \Lkeyword{PtNameMath}, and \Lkeyword{CurvAbsNeg}.
%
The point \Argsans{$B$} is positioned on the circle of center
\Argsans{$O$} crossing \Argsans{$A$}, with the curved abscissa
\Argsans{Abs}. The origin is \Argsans{$A$} and the direction is
anti-clockwise by default. The parameter \Lkeyword{CurvAbsNeg}
\DefaultVal{false} can change this behavior.
\end{sloppypar}

If the parameter \Lkeyword{PosAngle} is not specified, the point label is put
automatically in oirder to be alined with the circle center and the point.

\begin{LTXexample}[width=5cm,pos=l]
\begin{pspicture}[showgrid](-2.5,-2.5)(2.5,2.5)
\pstGeonode{O}(2,0){A}
\pstCircleOA{O}{A}
\pstCurvAbsNode{O}{A}{M_1}{\pstDistVal{5}}
\pstCurvAbsNode[CurvAbsNeg=true]%
  {O}{A}{M_2}{\pstDistAB{A}{M_1}}
\end{pspicture}
\end{LTXexample}



\subsection{Circle tangent}

The macro \Lcs{pstCircleTangentLine} is used to draw a tangent line $AT$ from a point $A$ on the circle,
and the macro \Lcs{pstCircleTangentNode} is used to draw the tangent points $T_1$ and $T_2$ from a point $P$ out of the circle.

\begin{BDef}
\Lcs{pstCircleTangentLine}\OptArgs\Largb{O}\Largb{A}\Largb{T}\\
\Lcs{pstCircleTangentNode}\OptArgs\Largb{O}\Largb{A}\Largb{P}\Largb{T1}\Largb{T2}
\end{BDef}

\begin{LTXexample}[width=6cm,pos=l]
\begin{pspicture}[showgrid=true](-2,-1)(4,3)
\psset{dotscale=0.5}\psset{PointSymbol=*}\footnotesize
\psset{nodesep=-0.8}
\pstGeonode[PosAngle={90,120,-30}](1,1){O}(-1,0){T}(3,0){S}
\pstCircleOA[Radius=\pstDistVal{1.5},linecolor=red]{O}{}
\pstCircleRotNode[Radius=\pstDistVal{1.5},PosAngle=-30,RotAngle=-30]{O}{}{A}
\pstCircleTangentLine[PosAngle=-10,PointName=A_1]{O}{A}{A1}
\pstCircleRotNode[Radius=\pstDistVal{1.5},PosAngle=90,RotAngle=90]{O}{}{B}
\pstCircleTangentLine[PosAngle=90,PointName=B_1]{O}{B}{B1}
\pstCircleTangentNode[Radius=\pstDistVal{1.5},PosAngle={150,90},PointName={T_1,T_2}]{O}{}{T}{T1}{T2}
\pstCircleTangentNode[PosAngle={80,200},PointName={S_1,S_2}]{O}{A}{S}{S1}{S2}
\end{pspicture}
\end{LTXexample}

The macro \Lcs{pstCircleExternalCommonTangent} is used to find the external common tangent lines of two circle $A(O_1)$ and $B(O_2)$,
and the macro \Lcs{pstCircleInternalCommonTangent} is used to find the internal common tangent lines of two circle $A(O_1)$ and $B(O_2)$.
They both create four tangent point nodes $T_1,T_2,T_3,T_4$, where $T_1,T_2$ lie on circle $A(O_1)$, and $T_3,T_4$ lie on circle $B(O_2)$.

\begin{BDef}
\Lcs{pstCircleExternalCommonTangent}\OptArgs\Largb{$O_1$}\Largb{A}\Largb{$O_2$}\Largb{B}\Largb{$T_1$}\Largb{$T_2$}\Largb{$T_3$}\Largb{$T_4$}\\
\Lcs{pstCircleInternalCommonTangent}\OptArgs\Largb{$O_1$}\Largb{A}\Largb{$O_2$}\Largb{B}\Largb{$T_1$}\Largb{$T_2$}\Largb{$T_3$}\Largb{$T_4$}
\end{BDef}

You can use \Lkeyword{RadiusA} and \Lkeyword{RadiusB} to define the two circles like as following:
\begin{LTXexample}[width=6cm,pos=l]
\begin{pspicture}[showgrid=true](-2,-2)(3,3)
\psset{dotscale=0.5}\psset{PointSymbol=*}\footnotesize
\pstGeonode[PosAngle=-90](-1,0){O1}
\pstGeonode[PosAngle=-60](1.5,1.5){O2}
\pstCircleOA[Radius=\pstDistVal{2},linecolor=red]{O1}{}
\pstCircleOA[Radius=\pstDistVal{1},linecolor=blue]{O2}{}
\pstCircleExternalCommonTangent[RadiusA=\pstDistVal{2},RadiusB=\pstDistVal{1},PosAngle={90,-60,90,-60}]{O1}{}{O2}{}{P}{Q}{R}{S}
\pstLine[nodesep=-1]{P}{R}
\pstLine[nodesep=-1]{Q}{S}
\pstCircleInternalCommonTangent[RadiusA=\pstDistVal{2},RadiusB=\pstDistVal{1},PosAngle={120,60,120,60}]{O1}{}{O2}{}{H}{I}{J}{K}
\pstLine[nodesep=-1]{H}{J}
\pstLine[nodesep=-1]{I}{K}
\end{pspicture}
\end{LTXexample}

You also can use \Lkeyword{DiameterA} and \Lkeyword{DiameterB} to define the two circles like as following:
\begin{LTXexample}[width=6cm,pos=l]
\begin{pspicture}[showgrid=true](-2,-2)(3,3)
\psset{dotscale=0.5}\psset{PointSymbol=*}\footnotesize
\pstGeonode[PosAngle=-90](-1,0){O1}
\pstGeonode[PosAngle=-60](1.5,1.5){O2}
\pstCircleOA[Diameter=\pstDistVal{3},linecolor=red]{O1}{}
\pstCircleOA[Diameter=\pstDistVal{2},linecolor=blue]{O2}{}
\pstCircleExternalCommonTangent[DiameterA=\pstDistVal{3},DiameterB=\pstDistVal{2},PosAngle={100,-60,90,-60}]{O1}{}{O2}{}{P}{Q}{R}{S}
\pstLine[nodesep=-1]{P}{R}
\pstLine[nodesep=-1]{Q}{S}
\pstCircleInternalCommonTangent[DiameterA=\pstDistVal{3},DiameterB=\pstDistVal{2},PosAngle={80,-60,-90,140}]{O1}{}{O2}{}{H}{I}{J}{K}
\pstLine[nodesep=-1]{H}{J}
\pstLine[nodesep=-1]{I}{K}
\end{pspicture}
\end{LTXexample}

\subsection{Circle radical axis}
If you want to draw the \texttt{Radical Axis} of two given circles, read the following sentenses.
For given $\odot{O_1}$ with radius $r_1$ and $\odot{O_2}$ with radius $r_2$, and the center
$O_1(x_1,y_1)$, $O_2(x_2,y_2)$, then any point $P(x,y)$ on the \texttt{Radical Axis} is satisfied:
$$(x-x_1)^2+(y-y_1)^2-r_1^2=(x-x_2)^2+(y-y_2)^2-r_2^2$$
It can be simplified to a equation of a line:
$$2(x_2-x_1)x+2(y_2-y_1)y=(x_2^2+y_2^2-r_2^2)-(x_1^2+y_1^2-r_1^2)$$
It is clear that the circles with same center have no radical axis,
and the radical axis is perpendicular to the line of centers.

We provide the macro \Lcs{pstCircleRadicalAxis} to draw the \texttt{Radical Axis} of two given circles.
It can handler every position relations of circles such as separation, intersection and inclusion.

\begin{BDef}
\Lcs{pstCircleRadicalAxis}\OptArgs\Largb{$O_1$}\Largb{A}\Largb{$O_2$}\Largb{B}\Largb{C}\Largb{D}
\end{BDef}

Both parameter $A$ and $B$ can be omitted and then to specify the each radius or
diameter using the parameters \Lkeyword{RadiusA}, \Lkeyword{DiameterA}, and \Lkeyword{RadiusB}, \Lkeyword{DiameterB}.
This macro create two new nodes $C$ and $D$ on the radical axis, you can find them in following examples.

When they are intersected, we can see the radical axis is the intersected chord line.

\begin{LTXexample}[width=6cm,pos=l]
\begin{pspicture}[showgrid=true](-1,-1)(4,4)
\psset{dotscale=0.5}\psset{PointSymbol=*}\footnotesize
\def\ra{1.2}\def\rb{2.0}
\pstGeonode[PosAngle=0](0,1){O1}(1.5,1.5){O2}
\pstCircleOA[linecolor=red!50,Radius=\pstDistVal{\ra}]{O1}{}
\pstCircleOA[linecolor=blue!50,Radius=\pstDistVal{\rb}]{O2}{}
\pstCircleRadicalAxis[PosAngle={0,0},RadiusA=\pstDistVal{\ra},RadiusB=\pstDistVal{\rb},nodesep=-1,linecolor=brown]{O1}{}{O2}{}{A}{B}
\end{pspicture}
\end{LTXexample}

When they are tangent, we can see the radical axis is the common tangent line.

\begin{LTXexample}[width=6cm,pos=l]
\begin{pspicture}[showgrid=true](-1,-1)(3,4)
\psset{dotscale=0.5}\psset{PointSymbol=*}\footnotesize
\def\ra{1.2}\def\rb{2.0}
\pstGeonode[PosAngle=-90,PointName={O_1,O_2}](1,1){O1}(1,1.8){O2}
\pstCircleOA[linecolor=red!50,Radius=\pstDistVal{\ra}]{O1}{}
\pstCircleOA[linecolor=blue!50,Radius=\pstDistVal{\rb}]{O2}{}
\pstCircleRadicalAxis[nodesep=-2,PosAngle={-90,-90},RadiusA=\pstDistVal{\ra},RadiusB=\pstDistVal{\rb}]{O1}{}{O2}{}{A}{B}
\pstLineAB[linecolor=red,nodesep=-3]{A}{B}
\end{pspicture}
\end{LTXexample}

When one of them contains the other, the radical axis is out of the circles.

\begin{LTXexample}[width=6cm,pos=l]
\begin{pspicture}[showgrid=true](-1,-2)(4,3)
\psset{dotscale=0.5}\psset{PointSymbol=*}\footnotesize
\def\ra{1.2}\def\rb{2.0}
\pstGeonode[PosAngle=0](1.2,1){O1}(1.5,1.5){O2}
\pstCircleOA[linecolor=red!50,Radius=\pstDistVal{\ra}]{O1}{}
\pstCircleOA[linecolor=blue!50,Radius=\pstDistVal{\rb}]{O2}{}
\pstCircleRadicalAxis[PosAngle={-90,90},RadiusA=\pstDistVal{\ra},RadiusB=\pstDistVal{\rb},nodesepA=-1,nodesepB=-3,linecolor=brown]{O1}{}{O2}{}{A}{B}
\end{pspicture}
\end{LTXexample}

When they are separated, the radical axis is between of the circles.

\begin{LTXexample}[width=6cm,pos=l]
\begin{pspicture}[showgrid=true](-1,-2)(4,3)
\psset{dotscale=0.5}\psset{PointSymbol=*}\footnotesize
\def\ra{1.2}\def\rb{2.0}
\pstGeonode[PosAngle=0](-1,0){O1}(2.5,1.5){O2}
\pstCircleOA[linecolor=red!50,Radius=\pstDistVal{\ra}]{O1}{}
\pstCircleOA[linecolor=blue!50,Radius=\pstDistVal{\rb}]{O2}{}
\pstCircleRadicalAxis[PosAngle={-90,90},RadiusA=\pstDistVal{\ra},RadiusB=\pstDistVal{\rb},nodesep=-1,linecolor=brown]{O1}{}{O2}{}{A}{B}
\psset{linestyle=dashed,linecolor=gray!40}
\pstCircleTangentNode[Radius=\pstDistVal{\ra},PosAngle={90,200}]{O1}{}{A}{P}{Q}
\pstCircleTangentNode[Radius=\pstDistVal{\rb},PosAngle={10,100}]{O2}{}{A}{X}{Y}
\pstCircleOA{A}{P}
\pstCircleTangentNode[Radius=\pstDistVal{\ra},PosAngle={210,200}]{O1}{}{B}{I}{J}
\pstCircleTangentNode[Radius=\pstDistVal{\rb},PosAngle={10,-10}]{O2}{}{B}{R}{S}
\pstCircleOA{B}{I}
\end{pspicture}
\end{LTXexample}

\subsection{Generic curve}

It is possible to generate a set of points using a loop, and to give
them a generic name defined by a radical and a number. The following
command can draw a interpolated curve crossing all such kind of
points.

\begin{BDef}
\Lcs{pstGenericCurve}\OptArgs\Largb{Radical}\Largb{$n_1$}\Largb{$n_2$}
\end{BDef}

\begin{sloppypar}
Possible optional arguments are \Lkeyword{GenCurvFirst}, \Lkeyword{GenCurvInc}, and
  \Lkeyword{GenCurvLast}
The curve is drawn on the points whose name is defined using the
radical \Argsans{Radical} followed by a number from \Argsans{$n_1$} to
\Argsans{$n_2$}. In order to manage side effect, the parameters
\Lkeyword{GenCurvFirst} et \Lkeyword{GenCurvLast} can be used to specified
special first or last point. The parameter \Lkeyword{GenCurvInc} can be
used to modify the increment from a point to the next one
\DefaultVal{1}.
\end{sloppypar}


\begin{LTXexample}[width=5cm,pos=l]
\begin{pspicture}[showgrid](-2.5,-2.5)(2.5,1)
\psset{unit=.00625}
\pstGeonode{A}
\multido{\n=20+20}{18}{%
 \pstGeonode[PointName=M_{\n}](\n;\n){M_\n}}
\pstGenericCurve[GenCurvFirst=A,GenCurvInc=20,
 linecolor=blue,linewidth=.5\pslinewidth]{M_}{20}{360}
 \end{pspicture}
\end{LTXexample}

\clearpage

\section{Conics}
\subsection{Standard Ellipse}
The Standard Ellipse $E$ with coordinate translation is defined by center $O(x_0,y_0)$,
the half of the major axis $max(abs(a),abs(b))$, the half of the minor axis $min(abs(a),abs(b))$,
the equation as following:
\begin{equation}\label{FunctionOfStandardEllipse}
\dfrac{(x-x_0)^2}{a^2}+\dfrac{(y-y_0)^2}{b^2}=1
\end{equation}
Sometimes we use the parametric function of the Standard Ellipse with coordinate translation:
\begin{equation}\label{ParametricFunctionOfEllipse}
\left\{\begin{array}{l}
x=a\cos\alpha+x_0\\
y=b\sin\alpha+y_0
\end{array}\right.
\end{equation}

The Macro \Lcs{pstEllipse} is used to draw a Standard Ellipse with center $O$ from
\Lkeyword{angleA} to \Lkeyword{angleB}, going counter clockwise.
It combines the function like \Lcs{psellipse} and \Lcs{psellipticarc} in \PST.
If \Lkeyword{angleA} and \Lkeyword{angleB} are not specified,
the macro will draw the whole ellipse.

\begin{BDef}
\Lcs{pstEllipse}\OptArgs\Largr{O}\Largr{$a,\,b$}\OptArg{angleA}\OptArg{angleB}
\end{BDef}

\begin{LTXexample}[width=6cm,pos=l]
\begin{pspicture}[showgrid=true](0,0)(4,4)
\psset{dotscale=0.5}\psset{PointSymbol=*}\footnotesize
\def\ra{2.4}\def\rb{0.8}
\pstGeonode[PosAngle=-90,PointNameSep=0.2](2,2){O}
%\psellipse[linecolor=red!60](O)(\ra,\rb)
\pstEllipse[linecolor=red!60](O)(\ra,\rb)[0][120]
\pstEllipse[linecolor=green!60,linestyle=dashed,arrows=->,arrowscale=1.2](O)(\ra,\rb)[120][200]
\pstEllipse[linecolor=blue!60](O)(\ra,\rb)[200][300]
\pstEllipse[linecolor=purple!60,linestyle=dashed,arrows=->,arrowscale=1.2](O)(\ra,\rb)[300][360]
\pstEllipse[linecolor=cyan!60](O)(\rb,\ra)
\end{pspicture}
\end{LTXexample}

Like as the coordinates, the parameters $a,b$ can be got by the raw PostScript commands too,
where you can use the macros \Lcs{pstDist}*, for example,

\begin{LTXexample}[width=6cm,pos=l]
\begin{pspicture}[showgrid=true](-3,-3)(3,3)
\psset{dotscale=0.5}\psset{PointSymbol=*}\footnotesize
\pstGeonode[PosAngle=-90,PointSymbol=*](0,0){O}(0,-2.5){R}
\pstCircleOA[linecolor=blue!60]{O}{R}
\pstGeonode[PosAngle=0,PointName=O_1,PointSymbol=*](0,0.8){O1}
\pstGeonode[PosAngle=0,PointName=O_2,PointSymbol=*](0,1.6){O2}
\pstCircleOrdNode[PointName={P,Q},PosAngle={180,0}]{O}{R}{\pstOrdinate{O1}}{P}{Q}
\pstCircleOrdNode[PointName={P',Q'},PosAngle={180,0}]{O}{R}{\pstOrdinate{O2}}{P'}{Q'}
\pstEllipse[linecolor=green!60,linestyle=dashed](O1)(! \pstUserDist{\pstDist{O1}{P}} 0.7)[0][180]
\pstEllipse[linecolor=green!60](O1)(! \pstUserDist{\pstDist{O1}{P}} 0.7)[180][360]
\pstEllipse[linecolor=red!60,linestyle=dashed](O2)(! \pstUserDist{\pstDist{O2}{P'}} 0.5)[0][180]
\pstEllipse[linecolor=red!60](O2)(! \pstUserDist{\pstDist{O2}{P'}} 0.5)[180][360]
\end{pspicture}
\end{LTXexample}

Now you can draw some points on this Ellipse using macro \Lcs{pstEllipseNode} or \Lcs{pstEllipseRotNode}.
The macro \Lcs{pstEllipseNode} requires an explicit parameter $t$ as $\alpha$ in equation (\ref{ParametricFunctionOfEllipse})
to calculate the point; but the macro \Lcs{pstEllipseRotNode} requires an implicit parameter \Lkeyword{RotAngle}
as $\alpha$ in equation (\ref{ParametricFunctionOfEllipse}) to calculate the point.

\begin{BDef}
\Lcs{pstEllipseNode}\OptArgs\Largr{O}\Largr{$a,b$}\Largb{$t$}\Largb{P}\\
\Lcs{pstEllipseRotNode}\OptArgs\Largr{O}\Largr{$a,b$}\Largb{P}
\end{BDef}

The following is the example, note that the \Lkeyword{RotAngle} is not $\angle{HOX}$ in geometrical,
but $\angle{HOA}$ or $\angle{HOB}$.
\begin{LTXexample}[width=6cm,pos=l]
\begin{pspicture}[showgrid=true](0,0)(4,4)
\psset{dotscale=0.5}\psset{PointSymbol=*}\footnotesize
\def\ra{2.4}\def\rb{0.8}\def\rot{56}
\pstGeonode[PosAngle=-90,PointNameSep=0.2](2,2){O}
%\psellipse[linecolor=red!60](O)(\ra,\rb)
\pstEllipse[linecolor=red!60](O)(\ra,\rb)
\pstEllipseNode[PosAngle=180](O)(\ra,\rb){180}{P}
\pstEllipseRotNode[PosAngle=0,RotAngle=0](O)(\ra,\rb){Q}
\pstEllipseRotNode[PosAngle=90,RotAngle=90](O)(\ra,\rb){M}
\pstEllipseRotNode[PosAngle=-90,RotAngle=-90](O)(\ra,\rb){N}
\pstCircleOA[linecolor=blue!60,Radius=\pstDistVal{\ra}]{O}{}
\pstCircleRotNode[PosAngle=\rot,RotAngle=\rot,Radius=\pstDistVal{\ra}]{O}{}{A}
\pstCircleOA[linecolor=green!60,Radius=\pstDistVal{\rb}]{O}{}
\pstCircleRotNode[PosAngle=180,RotAngle=\rot,Radius=\pstDistVal{\rb}]{O}{}{B}
\pstEllipseRotNode[PosAngle=30,RotAngle=\rot](O)(\ra,\rb){X}
\pstProjection[PosAngle=-90]{P}{Q}{A}[H]
\pstLineAB[linestyle=dashed]{A}{O}
\pstLineAB[linestyle=dashed]{A}{H}
\pstLineAB[linestyle=dashed]{B}{X}
\pstLineAB[linestyle=dashed]{O}{H}
\pstMarkAngle[LabelSep=.6,MarkAngleRadius=.3,MarkAngleType=double,fillcolor=red!30,fillstyle=solid]{H}{O}{A}{$\rot^\circ$}
\end{pspicture}
\end{LTXexample}

The macros \Lcs{pstEllipseAbsNode} and \Lcs{pstEllipseOrdNode} are used to get the two nodes $A$ and $B$
whose abscissas or ordinates are the given value $x_1$ or $y_1$ on the Standard Ellipse $E$.

If there is no such point satisfied this condition, then the nodes $A$ and $B$ will be put at the origin.

\begin{BDef}
\Lcs{pstEllipseAbsNode}\OptArgs\Largr{O}\Largr{$a,b$}\Largb{$x_1$}\Largb{A}\Largb{B}\\
\Lcs{pstEllipseOrdNode}\OptArgs\Largr{O}\Largr{$a,b$}\Largb{$y_1$}\Largb{A}\Largb{B}
\end{BDef}

\begin{LTXexample}[width=6cm,pos=l]
\begin{pspicture}[showgrid=true](0,0)(4,4)
\psset{dotscale=0.5}\psset{PointSymbol=*}\footnotesize
\def\ra{2.0}\def\rb{-1.2}
\pstGeonode[PosAngle=-50,PointNameSep=0.2](2,2){O}
\pstEllipse[linecolor=red!40](O)(\ra,\rb)
\pstEllipse[linecolor=blue!40](O)(\rb,\ra)
\pstEllipseAbsNode[PosAngle={120,200}](O)(\ra,\rb){2.5}{A}{B}
\pstEllipseAbsNode(O)(\ra,\rb){6}{X}{Y} % not exist
\pstEllipseOrdNode(O)(\ra,\rb){2.5}{A'}{B'}
\pstEllipseOrdNode(O)(\ra,\rb){6}{X'}{Y'} % not exist
\end{pspicture}
\end{LTXexample}

Here we find the focus node of Standard Ellipse! Please use macro \Lcs{pstEllipseFocusNode} to do this work.

\begin{BDef}
\Lcs{pstEllipseFocusNode}\OptArgs\Largr{O}\Largr{$a,b$}\Largb{A}\Largb{B}
\end{BDef}

For example:

\begin{LTXexample}[width=6cm,pos=l]
\begin{pspicture}[showgrid=true](0,0)(4,4)
\psset{dotscale=0.5}\psset{PointSymbol=*}\footnotesize
\def\ra{2.0}\def\rb{-1.2}
\pstGeonode[PosAngle=-50,PointNameSep=0.2](2,2){O}
\pstEllipse[linecolor=red!40](O)(\ra,\rb)
\pstEllipse[linecolor=blue!40](O)(\rb,\ra)
\pstEllipseFocusNode(O)(\ra,\rb){L}{R}
\pstEllipseFocusNode(O)(\rb,\ra){D}{U}
\end{pspicture}
\end{LTXexample}

The macro \Lcs{pstEllipseDirectrixLine} is used to draw the two directrix lines of Standard Ellipse,
and create two new nodes on each of them. The nodes $L_x$, $L_y$ are on the left/down directrix line,
and $R_x$, $R_y$ are on the right/up directrix line. They are lie on the tangent line of the vertex
on the other axis.

\begin{BDef}
\Lcs{pstEllipseDirectrixLine}\OptArgs\Largr{O}\Largr{$a,b$}\Largb{$L_x$}\Largb{$L_y$}\Largb{$R_x$}\Largb{$R_y$}
\end{BDef}

For example:

\begin{LTXexample}[width=6cm,pos=l]
\begin{pspicture}[showgrid=true](0,0)(4,4)
\psset{dotscale=0.5}\psset{PointSymbol=*}\footnotesize
\def\ra{2.0}\def\rb{-1.2}
\pstGeonode[PosAngle=-50,PointNameSep=0.2](2,2){O}
\pstEllipse[linecolor=red!40](O)(\ra,\rb)
\pstEllipse[linecolor=blue!40](O)(\rb,\ra)
\pstEllipseDirectrixLine[PointName={L_x,L_y,R_x,R_y},PosAngle={210,210,-30,-30},nodesep=-1,linecolor=red!40](O)(\ra,\rb){Lx}{Ly}{Rx}{Ry}
\pstEllipseDirectrixLine[PointName={D_x,D_y,U_x,U_y},PosAngle={-30,-30,30,30},nodesep=-1,linecolor=blue!40](O)(\rb,\ra){Dx}{Dy}{Ux}{Uy}
\pstLine[nodesep=-0.5,linecolor=black!40,linestyle=dashed]{Lx}{Rx}
\pstLine[nodesep=-0.5,linecolor=black!40,linestyle=dashed]{Ly}{Ry}
\pstLine[nodesep=-0.5,linecolor=black!40,linestyle=dashed]{Dx}{Ux}
\pstLine[nodesep=-0.5,linecolor=black!40,linestyle=dashed]{Dy}{Uy}
\end{pspicture}
\end{LTXexample}

Sometimes we need to find the intersection of Ellipse and line,
the Macro \Lcs{pstEllipseLineInter} can do this work, and it can handle any type of line,
i.e, horizontal, vertical or others lines. It get the two intersection $C$ and $D$ of the
Standard Ellipse $E$ and the given line $AB$. When there is none intersection,
$C$ and $D$ are both put at the origin; When there is only on intersection, it will be saved
at node $C$, and $D$ will be put at the origin.

\begin{BDef}
\Lcs{pstEllipseLineInter}\OptArgs\Largr{O}\Largr{$a,b$}\Largb{$A$}\Largb{$B$}\Largb{$C$}\Largb{$D$}
\end{BDef}

Here is examples:

\begin{LTXexample}[width=6cm,pos=l]
\begin{pspicture}[showgrid=true](0,0)(4,4)
\psset{dotscale=0.5}\psset{PointSymbol=*}\footnotesize
\def\ra{2.0}\def\rb{-1.2}
\pstGeonode[PosAngle=-50,PointNameSep=0.2](2,2){O}
\pstEllipse[linecolor=red!40](O)(\ra,\rb)
\pstEllipse[linecolor=blue!40](O)(\rb,\ra)
\pstLine[nodesep=-0.5,linecolor=black!40,linestyle=dashed]{0,1}{3,4}
\pstEllipseLineInter[PosAngle={-90,90}](O)(\ra,\rb){0,1}{3,4}{C}{D}
\pstEllipseLineInter[PosAngle={-90,90}](O)(\rb,\ra){0,1}{3,4}{C'}{D'}
\pstLine[nodesep=-0.5,linecolor=black!40,linestyle=dashed]{1.5,0}{1.5,4}
\pstEllipseLineInter[PosAngle={40,60}](O)(\ra,\rb){1.5,0}{1.5,4}{E}{F}
\pstEllipseLineInter[PosAngle={40,130}](O)(\rb,\ra){1.5,1}{1.5,4}{E'}{F'}
\pstLine[nodesep=-0.5,linecolor=black!40,linestyle=dashed]{4,2.5}{0,2.5}
\pstEllipseLineInter[PosAngle={130,50}](O)(\ra,\rb){4,2.5}{0,2.5}{G}{H}
\pstEllipseLineInter[PosAngle={130,50}](O)(\rb,\ra){4,2.5}{0,2.5}{G'}{H'}
\end{pspicture}
\end{LTXexample}

The macro \Lcs{pstEllipsePolarNode} is use to draw the tangent line of a point $A$ or $B$
on the Standard Ellipse. It draws the every tangent line through the point $A$ and $B$ on
the Standard Ellipse $E$ and get the insection node $T$ of the two tangent lines.
We call $T$ as the polar point of chord $AB$ as normal.

\begin{BDef}
\Lcs{pstEllipsePolarNode}\OptArgs\Largr{O}\Largr{$a,b$}\Largb{$A$}\Largb{$B$}\Largb{$T$}
\end{BDef}

We use the following theorem to find the node $T$:
\begin{theorem}\label{EllipsePolarPointTheorem}
Give chord $AB$ on the ellipse, we draw any other two chords $PQ$ and $RS$, $AB$ and $PQ$ intersect at $I$,
$AQ$ and $BP$ intersect at $X$, $AP$ and $BQ$ intersect at $Y$, we call $XY$ is the polar line of point $I$.
Also $AB$ and $RS$ intersect at $J$, $AR$ and $BS$ intersect at $M$, $AS$ and $BR$ intersect at $N$,
we call $MN$ is the polar line of point $J$. Then the intersection $T$ of $XY$ and $MN$ is the polar point of chord $AB$,
i.e. $TA$ is the tangent line through $A$ and $TB$ is the tangent line through $B$.
\end{theorem}

\begin{LTXexample}[width=6cm,pos=l]
\begin{pspicture}[showgrid=true](0,0)(4,4)
\psset{dotscale=0.5}\psset{PointSymbol=*}\footnotesize
\def\rb{2.0}\def\ra{-1.2}
\pstGeonode[PosAngle=-50,PointNameSep=0.2](2,2){O}
\pstEllipse[linecolor=red!40](O)(\ra,\rb)
\pstLine[nodesep=-0.8,linecolor=black!40,linestyle=dashed]{1,2}{2.5,3.5}
\pstEllipseLineInter[PosAngle={-100,90}](O)(\ra,\rb){1,2}{2.5,3.5}{A}{B}
\pstEllipsePolarNode[PosAngle=120](O)(\ra,\rb){A}{B}{T}
% Here are the auxiliary lines to explain Theorem 1.
\pstEllipseRotNode[PosAngle=0,RotAngle=5](O)(\ra,\rb){P}
\pstEllipseRotNode[PosAngle=-10,RotAngle=-61](O)(\ra,\rb){Q}
\pstEllipseRotNode[PosAngle=-100,RotAngle=-92](O)(\ra,\rb){R}
\pstEllipseRotNode[PosAngle=0,RotAngle=-30](O)(\ra,\rb){S}
\pstInterLL[PosAngle=-90]{A}{Q}{B}{P}{X}
\pstInterLL[PosAngle=-10]{A}{P}{B}{Q}{Y}
\pstInterLL[PosAngle=-90]{A}{R}{B}{S}{M}
\pstInterLL[PosAngle=190]{A}{S}{B}{R}{N}
\psset{linestyle=dashed,linecolor=gray!40}
\pstLine{A}{Q}\pstLine{B}{P}\pstLine{A}{P}\pstLine{B}{Q}
\pstLine{A}{R}\pstLine{B}{S}\pstLine{A}{S}\pstLine{B}{R}
\pstLine{Q}{X}\pstLine{Q}{Y}\pstLine{P}{X}\pstLine{P}{Y}
\pstLine{R}{M}\pstLine{S}{M}\pstLine{T}{Y}\pstLine{T}{N}
\pstLine[linestyle=dashed,linecolor=red!40]{X}{Y}
\pstLine[linestyle=dashed,linecolor=red!40]{M}{N}
\end{pspicture}
\end{LTXexample}

The macro \Lcs{pstEllipseTangentNode} is use to draw the tangent line of a point $T$
out of the Standard Ellipse $E$. It draw the two tangent lines through the point $T$
to the Standard Ellipse $E$ and get the node $A$ and $B$ on the Ellipse.

\begin{BDef}
\Lcs{pstEllipseTangentNode}\OptArgs\Largr{O}\Largr{$a,b$}\Largb{$T$}\Largb{$A$}\Largb{$B$}
\end{BDef}

We use the following theorem to find the tangent node of the given $T$:
\begin{theorem}\label{EllipseTangentPointTheorem}
Give point $T$ outside of the ellipse, we draw any other two chords $TPQ$ and $TRS$,
let $PS$ and $QR$ intersect at $I$, $PR$ and $QS$ intersect at $X$, $XI$ and Ellipse intersect at $A$ and $B$,
then $TA$ is the tangent line through $A$ and $TB$ is the tangent line through $B$.
\end{theorem}

\begin{LTXexample}[width=6cm,pos=l]
\begin{pspicture}[showgrid=true](0,0)(4,4)
\psset{dotscale=0.5}\psset{PointSymbol=*}\footnotesize
\def\ra{2.0}\def\rb{-1.2}
\pstGeonode[PosAngle=-50,PointNameSep=0.2](2,2){O}
\pstEllipse[linecolor=red!40](O)(\ra,\rb)
\pstGeonode[PosAngle=-50,PointNameSep=0.2](-1,-1){T}
\pstEllipseTangentNode[PosAngle=120](O)(\ra,\rb){T}{A}{B}
% Here are the auxiliary lines to explain Theorem 2.
\pstEllipseRotNode[PointName=none,RotAngle=71](O)(\ra,\rb){P0}
\pstEllipseRotNode[PointName=none,RotAngle=31](O)(\ra,\rb){R0}
\pstEllipseLineInter[PosAngle=0](O)(\ra,\rb){T}{P0}{P}{Q}
\pstEllipseLineInter[PosAngle=0](O)(\ra,\rb){T}{R0}{R}{S}
\pstInterLL[PosAngle=0]{P}{S}{Q}{R}{I}
\pstInterLL[PosAngle=0]{P}{R}{Q}{S}{X}
\psset{linestyle=dashed,linecolor=gray!40}
\pstLine{T}{P}\pstLine{P}{Q}\pstLine{T}{R}\pstLine{R}{S}
\pstLine{P}{S}\pstLine{Q}{R}\pstLine{P}{R}\pstLine{Q}{S}
\end{pspicture}
\end{LTXexample}

\subsection{General Ellipse}
Now we will introduce some macros for the General Ellipse as same as the Standard Ellipse.
The General Ellipse $E$ with coordinate translation and rotation is defined by center $O(x_0,y_0)$,
the half of the major axis $max(abs(a),abs(b))$, the half of the minor axis $min(abs(a),abs(b))$,
and the rotation angle $\theta$ of the major axis.

The equation can be got from the parametric function of the ellipse equation (\ref{ParametricFunctionOfEllipse}),
using the rotation transform formula:
\begin{equation}\label{RotationTransformFormula}
\left\{\begin{array}{l}
x'=x\cos\theta-y\sin\theta\\
y'=x\sin\theta+y\cos\theta
\end{array}\right.
\end{equation}
then we have
\begin{equation}
\left\{\begin{array}{l}
x'=(a\cos\alpha+x_0)\cos\theta-(b\sin\alpha+y_0)\sin\theta=a\cos\alpha\cos\theta-b\sin\alpha\sin\theta+x_0'\\
y'=(a\cos\alpha+x_0)\sin\theta+(b\sin\alpha+y_0)\cos\theta=a\cos\alpha\sin\theta+b\sin\alpha\cos\theta+y_0'
\end{array}\right.
\end{equation}
where the $x_0'$ and $y_0'$ are the coordinate of the given center $O$ after rotation.
So we get the parametric function of the General Ellipse with coordinate translation and rotation as following:
\begin{equation}\label{ParametricFunctionOfGeneralEllipse}
\left\{\begin{array}{l}
x=a\cos\alpha\cos\theta-b\sin\alpha\sin\theta+x_0\\
y=a\cos\alpha\sin\theta+b\sin\alpha\cos\theta+y_0
\end{array}\right.
\end{equation}

The Macro \Lcs{pstGeneralEllipse} is used to draw a General Ellipse with center $O$ from
\Lkeyword{angleA} to \Lkeyword{angleB}, going counter clockwise.
If \Lkeyword{angleA} and \Lkeyword{angleB} are not specified,
the macro will draw the whole ellipse.
If you not input rotation angle $\theta$, the default value is $0^\circ$,
at this time, the result of this macro is same as \Lcs{pstEllipse}.
That is, \Lcs{pstGeneralEllipse} is more complex than \Lcs{pstEllipse}!

\begin{BDef}
\Lcs{pstGeneralEllipse}\OptArgs\Largr{O}\Largr{$a,\,b$}\OptArg{$\theta$}\OptArg{angleA}\OptArg{angleB}
\end{BDef}

\begin{LTXexample}[width=6cm,pos=l]
\begin{pspicture}[showgrid=true](0,0)(4,4)
\psset{dotscale=0.5}\psset{PointSymbol=*}\footnotesize
\def\ra{2.4}\def\rb{-1.5}
\pstGeonode[PosAngle=-90,PointNameSep=0.2](2,2){O}
\pstGeneralEllipse[linecolor=red!40](O)(\ra,\rb)[0]
\pstGeneralEllipse[linecolor=gray!10](O)(\ra,\rb)[10]
\pstGeneralEllipse[linecolor=gray!20](O)(\ra,\rb)[20]
\pstGeneralEllipse[linecolor=gray!30](O)(\ra,\rb)[30]
\pstGeneralEllipse[linecolor=gray!40](O)(\ra,\rb)[40]
\pstGeneralEllipse[linecolor=magenta!40](O)(\ra,\rb)[50]
\end{pspicture}
\end{LTXexample}

\vspace{10pt}

The Macro \Lcs{pstGeneralEllipseFle} is used to define a General Ellipse with Focus $F$, directrix line $l$,
and the eccentricity $e$, where $0\le{}e<1$. It just calculate the center $O$, major radius $a$, minor radius $b$ and the rotation
angle $\theta$ of the major axis, then you can pass them into macro \Lcs{pstGeneralEllipse} to draw this ellipse.

\begin{BDef}
\Lcs{pstGeneralEllipseFle}\OptArgs\Largb{F}\Largb{A}\Largb{B}\Largb{$e$}\Largb{O}\Largb{Rab}\Largb{$\theta$}
\end{BDef}

The output parameter \texttt{O} is a node name to store the center point, its label and symbol can
be controlled by the options for \PST\ node, such as \Lkeyword{PosAngle}.
The output parameter \texttt{Rab} is a PostScript key to store the pair of major radius and minor radius,
it just use \PST\ node coordinate to store a pair of value, but not a geometrical point.
The output parameter \texttt{$\theta$} is also a PostScript key to store the rotation angle of major axis,
when you pass it to \Lcs{pstGeneralEllipse}, PostScript will lookup the value of this key in current dictionary.

\begin{LTXexample}[width=6cm,pos=l]
\begin{pspicture}[showgrid=true](-2,-2)(2,2)
\psset{unit=1.0cm}\psset{dotscale=0.5}\footnotesize
\psset{CodeFig=true,CodeFigColor=gray!50}\psset{PointSymbol=*}
\pstGeonode[PosAngle=30](1,-1){F_1}
\pstGeonode[PosAngle=30](-1,1){F_2}
\pstGeonode[PosAngle=-60](-2,-1){A}
\pstGeonode[PosAngle=-60](2,1){B}
\pstGeneralEllipseFle[PosAngle=30]{F_1}{A}{B}{0.5}{O_1}{R_1}{MajorRotAngle1}
\pstGeneralEllipse[linecolor=red!60](O_1)(R_1)[MajorRotAngle1]
\pstGeneralEllipseFle[PosAngle=30]{F_2}{A}{B}{0.5}{O_2}{R_2}{MajorRotAngle2}
\pstGeneralEllipse[linecolor=blue!60](O_2)(R_2)[MajorRotAngle2]
\pstGeneralEllipseFle[PosAngle=20]{F_1}{A}{B}{0.6}{O_3}{R_3}{MajorRotAngle3}
\pstGeneralEllipse[linecolor=green!60](O_3)(R_3)[MajorRotAngle3]
\end{pspicture}
\end{LTXexample}

\vspace{10pt}

The Macro \Lcs{pstGeneralEllipseCoef} is used to define a General Ellipse by the quadratic curve equation $ax^2+bxy+cy^2+dx+ey+f=0$,
it just calculate the center $O$, major radius $a$, minor radius $b$ and the rotation angle $\theta$ of the major axis,
then you can pass them into macro \Lcs{pstGeneralEllipse} to draw this ellipse.
The package \texttt{pst-func} provides macro \Lcs{psplotImp} to draw an implicit defined functions too,
but it can't tell you the geometrical elements like as center or radii, and it will take more time to
calculate the function value point by point.

\begin{BDef}
\Lcs{pstGeneralEllipseCoef}\OptArgs\Largb{a,b,c,d,e,f}\Largb{O}\Largb{Rab}\Largb{$\theta$}
\end{BDef}

The output parameter \texttt{O}, the output parameter \texttt{Rab} and the output parameter \texttt{$\theta$}
are same with \Lcs{pstGeneralEllipseFle}. They are set to zero if the coeffients are invalid to construct an ellipse.

\begin{LTXexample}[width=6cm,pos=l]
\begin{pspicture}[showgrid=true](-4,-3)(2,3)
\psset{unit=1.0cm}\footnotesize\psset{dotscale=0.5}
\psset{CodeFig=true}\psset{PointSymbol=*}
%2x^2-2xy+3y^2+6x+5y+8=0
\pstGeneralEllipseCoef[PosAngle=-100,CodeFigColor=red!50]{2,-2,3,6,5,8}{O_1}{R_1}{MajorRotAngle1}
\pstGeneralEllipse[linecolor=red!60](O_1)(R_1)[MajorRotAngle1]
%3x^2-2xy+2y^2-3x+6y+3=0
\pstGeneralEllipseCoef[PosAngle=-80,CodeFigColor=purple!50]{3,-2,2,-3,6,3}{O_2}{R_2}{MajorRotAngle2}
\pstGeneralEllipse[linecolor=purple!60](O_2)(R_2)[MajorRotAngle2]
%x^2-xy+y^2+x-3y+1=0
\pstGeneralEllipseCoef[PosAngle=-90,CodeFigColor=green!50]{1,-1,1,1,-3,1}{O_3}{R_3}{MajorRotAngle3}
\pstGeneralEllipse[linecolor=green!60](O_3)(R_3)[MajorRotAngle3]
%2x^2+4xy+3y^2+8x+6y+8=0
\pstGeneralEllipseCoef[PosAngle=-80,CodeFigColor=blue!50]{2,4,3,8,6,8}{O_4}{R_4}{MajorRotAngle4}
\pstGeneralEllipse[linecolor=blue!60](O_4)(R_4)[MajorRotAngle4]
\end{pspicture}
\end{LTXexample}

You can verify the output figures with \Lcs{psplotImp} as following:
\begin{LTXexample}[width=6cm,pos=l]
\begin{pspicture}[showgrid=true](-4,-3)(2,3)
\psset{unit=1.0cm}\footnotesize\psset{dotscale=0.5}
%2x^2-2xy+3y^2+6x+5y+8=0
\psplotImp[linecolor=red!60](-4,-3)(2,3){ 2 x dup mul mul -2 x mul y mul add 3 y dup mul mul add 6 x mul add 5 y mul add 8 add }
%3x^2-2xy+2y^2-3x+6y+3=0
\psplotImp[linecolor=purple!60](-4,-3)(2,3){ 3 x dup mul mul -2 x mul y mul add 2 y dup mul mul add -3 x mul add 6 y mul add 3 add }
%x^2-xy+y^2+x-3y+1=0
\psplotImp[linecolor=green!60](-4,-3)(2,3){ 1 x dup mul mul -1 x mul y mul add 1 y dup mul mul add 1 x mul add -3 y mul add 1 add }
%2x^2+4xy+3y^2+8x+6y+8=0
\psplotImp[linecolor=blue!60](-5,-3)(2,3){ 2 x dup mul mul 4 x mul y mul add 3 y dup mul mul add 8 x mul add 6 y mul add 8 add }
\end{pspicture}
\end{LTXexample}

\vspace{10pt}

The Macro \Lcs{pstGeneralEllipseABCDE} is used to define a General Ellipse by the given five points $A,B,C,D,E$,
it just calculate the center $O$, major radius $a$, minor radius $b$ and the rotation angle $\theta$ of the major axis,
then you can pass them into macro \Lcs{pstGeneralEllipse} to draw this ellipse.

\begin{BDef}
\Lcs{pstGeneralEllipseABCDE}\OptArgs\Largb{A}\Largb{B}\Largb{C}\Largb{D}\Largb{E}\Largb{O}\Largb{Rab}\Largb{$\theta$}
\end{BDef}

The output parameter \texttt{O}, the output parameter \texttt{Rab} and the output parameter \texttt{$\theta$}
are same with \Lcs{pstGeneralEllipseFle}. They are set to zero if the points are invalid to construct an ellipse.

\begin{LTXexample}[width=6cm,pos=l]
\begin{pspicture}[showgrid=true](-1,-2)(5,4)
\psset{unit=1.0cm}\footnotesize\psset{PointSymbol=*}
\psset{CodeFig=true,CodeFigColor=gray!50}
\pstGeonode[PosAngle=180](0,0){A}
\pstGeonode[PosAngle=-90](2,-1){B}
\pstGeonode[PosAngle=90](3,3){C}
\pstGeonode[PosAngle=-90](4,0){D}
\pstGeonode[PosAngle=0](5,2){E}
\pstGeneralEllipseABCDE[PosAngle=0]{A}{B}{C}{D}{E}{O}{R}{MajorRotAngle}
\pstGeneralEllipse[linecolor=red!60](O)(R)[MajorRotAngle]
\end{pspicture}
\end{LTXexample}

\vspace{10pt}

We can location the points on the General Ellipse using the macros
\Lcs{pstGeneralEllipseNode}, \Lcs{pstGeneralEllipseRotNode}, \Lcs{pstGeneralEllipseAbsNode}
and \Lcs{pstGeneralEllipseOrdNode} as following.

\begin{BDef}
\Lcs{pstGeneralEllipseNode}\OptArgs\Largr{O}\Largr{$a,\,b$}\OptArg{$\theta$}\Largb{$t$}\Largb{A}\\
\Lcs{pstGeneralEllipseRotNode}\OptArgs\Largr{O}\Largr{$a,\,b$}\OptArg{$\theta$}\Largb{A}\\
\Lcs{pstGeneralEllipseAbsNode}\OptArgs\Largr{O}\Largr{$a,\,b$}\OptArg{$\theta$}\Largb{$x_1$}\Largb{A}\Largb{B}\\
\Lcs{pstGeneralEllipseOrdNode}\OptArgs\Largr{O}\Largr{$a,\,b$}\OptArg{$\theta$}\Largb{$y_1$}\Largb{A}\Largb{B}
\end{BDef}

Some examples all together:
\begin{LTXexample}[width=6cm,pos=l]
\begin{pspicture}[showgrid=true](0,0)(4,4)
\psset{dotscale=0.5}\psset{PointSymbol=*}\footnotesize
\def\ra{2.4}\def\rb{-1.5}
\pstGeonode[PosAngle=-90,PointNameSep=0.2](2,2){O}
\pstGeneralEllipse[linecolor=magenta!40](O)(\ra,\rb)[50]
\pstGeneralEllipseNode[PosAngle=30](O)(\ra,\rb)[50]{30}{A}
\pstGeneralEllipseRotNode[PosAngle=120,RotAngle=120](O)(\ra,\rb)[50]{B}
\pstGeneralEllipseRotNode[PosAngle=0,RotAngle=0](O)(\ra,\rb)[50]{C}
\pstGeneralEllipseRotNode[PosAngle=0,RotAngle=90](O)(\ra,\rb)[50]{D}
\pstGeneralEllipseRotNode[PosAngle=-90,RotAngle=180](O)(\ra,\rb)[50]{E}
\pstGeneralEllipseRotNode[PosAngle=90,RotAngle=-90](O)(\ra,\rb)[50]{F}
\pstGeneralEllipseAbsNode[PosAngle={60,240}](O)(\ra,\rb)[50]{2}{I}{J}
\pstGeneralEllipseOrdNode[PosAngle={-40,210}](O)(\ra,\rb)[50]{1}{M}{N}
\pstLineAB[nodesep=-1,linecolor=blue!40]{C}{E}
\pstLineAB[nodesep=-1,linecolor=blue!40]{D}{F}
\end{pspicture}
\end{LTXexample}

Using macro \Lcs{pstGeneralEllipseFocusNode} to find the two focus nodes, and macro \\
\Lcs{pstGeneralEllipseDirectrixLine} to get the two directrix lines.

\begin{BDef}
\Lcs{pstGeneralEllipseFocusNode}\OptArgs\Largr{O}\Largr{$a,\,b$}\OptArg{$\theta$}\Largb{$t$}\Largb{$F_1$}\Largb{$F_2$}\\
\Lcs{pstGeneralEllipseDirectrixLine}\OptArgs\Largr{O}\Largr{$a,\,b$}\OptArg{$\theta$}\Largb{$L_x$}\Largb{$L_y$}\Largb{$R_x$}\Largb{$R_y$}
\end{BDef}

for example,
\begin{LTXexample}[width=6cm,pos=l]
\begin{pspicture}[showgrid=true](0,0)(4,4)
\psset{dotscale=0.5}\psset{PointSymbol=*}\footnotesize
\def\ra{2.4}\def\rb{-1.5}
\pstGeonode[PosAngle=-90,PointNameSep=0.2](2,2){O}
\pstGeneralEllipse[linecolor=magenta!40](O)(\ra,\rb)[50]
\pstGeneralEllipseFocusNode[PosAngle={-40,-40}](O)(\ra,\rb)[50]{L}{R}
\pstGeneralEllipseDirectrixLine[PointName={L_x,L_y,R_x,R_y},nodesep=-1,linecolor=magenta](O)(\ra,\rb)[50]{Lx}{Ly}{Rx}{Ry}
\pstLine[nodesep=-1,linecolor=red!40]{L}{R}
\pstLine[nodesep=-1,linecolor=red!40,linestyle=dashed]{Lx}{Rx}
\pstLine[nodesep=-1,linecolor=red!40,linestyle=dashed]{Ly}{Ry}
\end{pspicture}
\end{LTXexample}

Using \Lcs{pstGeneralEllipseLineInter} to get the two intersections $C$ and $D$ of the General Ellipse $E$ and the given line $AB$!

\begin{BDef}
\Lcs{pstGeneralEllipseLineInter}\OptArgs\Largr{O}\Largr{$a,\,b$}\OptArg{$\theta$}\Largb{A}\Largb{B}\Largb{C}\Largb{D}
\end{BDef}

\begin{LTXexample}[width=6cm,pos=l]
\begin{pspicture}[showgrid=true](0,0)(4,4)
\psset{dotscale=0.5}\psset{PointSymbol=*}\footnotesize
\def\ra{1.5}\def\rb{-2.4}
\pstGeonode[PosAngle=-90,PointNameSep=0.2](2,2){O}
\pstGeneralEllipse[linecolor=blue!40](O)(\ra,\rb)[50]
\pstLine[nodesep=-0.5,linecolor=black!40,linestyle=dashed]{0,1}{1.5,4}
\pstGeneralEllipseLineInter[PosAngle={-90,90}](O)(\ra,\rb)[50]{0,1}{1.5,4}{A}{B}
\pstLine[nodesep=-0.5,linecolor=black!40,linestyle=dashed]{0,3}{3,3}
\pstGeneralEllipseLineInter[PosAngle={-90,240}](O)(\ra,\rb)[50]{0,3}{3,3}{C}{D}
\pstLine[nodesep=-0.5,linecolor=black!40,linestyle=dashed]{1,0}{1,4}
\pstGeneralEllipseLineInter[PosAngle={30,10}](O)(\ra,\rb)[50]{1,1}{1,4}{E}{F}
\end{pspicture}
\end{LTXexample}

Using \Lcs{pstGeneralEllipsePolarNode} to find the polar point $T$ of chord $AB$,
please refer to Theorem \ref{EllipsePolarPointTheorem}.

\begin{BDef}
\Lcs{pstGeneralEllipsePolarNode}\OptArgs\Largr{O}\Largr{$a,\,b$}\OptArg{$\theta$}\Largb{A}\Largb{B}\Largb{T}
\end{BDef}

\begin{LTXexample}[width=6cm,pos=l]
\begin{pspicture}[showgrid=true](0,0)(4,4)
\psset{dotscale=0.5}\psset{PointSymbol=*}\footnotesize
\def\ra{1.5}\def\rb{-2.4}
\pstGeonode[PosAngle=-90,PointNameSep=0.2](2,2){O}
\pstGeneralEllipse[linecolor=blue!40](O)(\ra,\rb)[50]
\pstLine[nodesep=-0.5,linecolor=black!40,linestyle=dashed]{0,1}{1.5,4}
\pstGeneralEllipseLineInter[PosAngle={-90,90}](O)(\ra,\rb)[50]{0,1}{1.5,4}{A}{B}
\pstGeneralEllipsePolarNode[PosAngle=90](O)(\ra,\rb)[50]{A}{B}{T}
\end{pspicture}
\end{LTXexample}

Using \Lcs{pstGeneralEllipseTangentNode} to find the tangent point $A$ and $B$ of outside point $T$,
please refer to Theorem \ref{EllipseTangentPointTheorem}.

\begin{BDef}
\Lcs{pstGeneralEllipseTangentNode}\OptArgs\Largr{O}\Largr{$a,\,b$}\OptArg{$\theta$}\Largb{T}\Largb{A}\Largb{B}
\end{BDef}

\begin{LTXexample}[width=6cm,pos=l]
\begin{pspicture}[showgrid=true](0,0)(4,4)
\psset{dotscale=0.5}\psset{PointSymbol=*}\footnotesize
\def\ra{1.5}\def\rb{-2.4}
\pstGeonode[PosAngle=-90,PointNameSep=0.2](2,2){O}
\pstGeneralEllipse[linecolor=blue!40](O)(\ra,\rb)[50]
\pstGeonode[PosAngle=-90,PointNameSep=0.2](-1,-1){P}
\pstGeneralEllipseTangentNode[PosAngle=90](O)(\ra,\rb)[50]{P}{X}{Y}
\end{pspicture}
\end{LTXexample}

\subsection{Standard Parabola}
The Standard Parabola $P$ with coordinate translation is defined by vertex $O(x_0,y_0)$,
the half of the focus chord axis $abs(p)$.
Note that the sign of $p$ indicates the direction of the parabola.

The equation can be written as:
\begin{equation}\label{FunctionOfStandardParabola}
(x-x_0)^2=2p(y-y_0)
\end{equation}
and the parametric function can be written as:
\begin{equation}\label{ParametricFunctionOfStandardParabola}
\left\{\begin{array}{l}
x=t+x_0\\
y=\dfrac{t^2}{2p}+y_0
\end{array}\right.
\end{equation}

The macro \Lcs{pstParabola} is used to draw a Parabola from $x_1$ to $x_2$ with Vertex $O$,
the half of the focus chord axis $abs(p)$.

\begin{BDef}
    \Lcs{pstParabola}\OptArgs\Largr{O}\Largb{$p$}\Largb{$x_1$}\Largb{$x_2$}
\end{BDef}

The macro \Lcs{pstParabolaNode} is used to draw a node whose parameter is the given value $t$ on parabola,
please refer to equation (\ref{ParametricFunctionOfStandardParabola}).
The macro \Lcs{pstParabolaAbsNode} is used to draw a node whose abscissa is the given value $x_1$ on parabola.
The macro \Lcs{pstParabolaOrdNode} is used to draw a node whose ordinate is the given value $y_1$ on parabola.
Note that \Lcs{pstParabolaOrdNode} will create two nodes $A$ and $B$ at most time.

\begin{BDef}
\Lcs{pstParabolaNode}\OptArgs\Largr{O}\Largb{$p$}\Largb{$t$}\Largb{A}\\
\Lcs{pstParabolaAbsNode}\OptArgs\Largr{O}\Largb{$p$}\Largb{$x_1$}\Largb{A}\\
\Lcs{pstParabolaOrdNode}\OptArgs\Largr{O}\Largb{$p$}\Largb{$y_1$}\Largb{A}\Largb{B}
\end{BDef}

\begin{LTXexample}[width=6cm,pos=l]
\begin{pspicture}[showgrid=true](0,-1)(4,3)
\psset{dotscale=0.5}\psset{PointSymbol=*}\footnotesize
\def\p{0.4}
\pstGeonode[PosAngle=-130,PointNameSep=0.2](2,0){O}
\pstParabola[linecolor=red!40](O){\p}{-1.5}{1.5}
\pstParabolaNode[PosAngle=-90](O){\p}{1.5}{A}
\pstParabolaAbsNode[PosAngle=-90,PointName=X_1](O){\p}{1.5}{X1}
\pstParabolaOrdNode[PosAngle=40,PointName={Y_1,Y_2}](O){\p}{1.5}{Y1}{Y2}
\end{pspicture}
\end{LTXexample}

The macro \Lcs{pstParabolaFocusNode} is used to find the focus of the parabola,
and the macro \Lcs{pstParabolaDirectrixLine} is used to find the directrix line of the parabola.

\begin{BDef}
\Lcs{pstParabolaFocusNode}\OptArgs\Largr{O}\Largb{$p$}\Largb{F}\\
\Lcs{pstParabolaDirectrixLine}\OptArgs\Largr{O}\Largb{$p$}\Largb{$L_x$}\Largb{$L_y$}
\end{BDef}

\begin{LTXexample}[width=6cm,pos=l]
\begin{pspicture}[showgrid=true](0,-1)(4,3)
\psset{dotscale=0.5}\psset{PointSymbol=*}\footnotesize
\def\p{0.4}
\pstGeonode[PosAngle=-130,PointNameSep=0.2](2,0){O}
\pstParabola[linecolor=red!40](O){\p}{-1.5}{1.5}
\pstParabolaFocusNode[linecolor=red!40,PosAngle=50](O){\p}{F}
\pstParabolaDirectrixLine[linecolor=red!40,nodesepA=-1.8,nodesepB=-1,PosAngle={-50,-50}](O){\p}{A}{B}
\pstLine[linecolor=red!40,nodesepA=-0.8,nodesepB=-2.5]{A}{F}
\end{pspicture}
\end{LTXexample}

The macro \Lcs{pstParabolaLineInter} is used to find the intersections $C$ and $D$ of the parabola and the given line $AB$.

\begin{BDef}
\Lcs{pstParabolaLineInter}\OptArgs\Largr{O}\Largb{$p$}\Largb{A}\Largb{B}\Largb{C}\Largb{D}
\end{BDef}

\begin{LTXexample}[width=6cm,pos=l]
\begin{pspicture}[showgrid=true](0,-1)(4,3)
\psset{dotscale=0.5}\psset{PointSymbol=*}\footnotesize
\def\p{0.4}
\pstGeonode[PosAngle=-90,PointNameSep=0.2](2,0){O}
\pstParabola[linecolor=red!40](O){\p}{-1.5}{1.5}
\pstLine[linecolor=gray!40,nodesepA=-0.8,nodesepB=-0.8]{0,2}{4,1}
\pstParabolaLineInter[linecolor=gray!40,PosAngle={120,210}](O){\p}{0,2}{4,1}{P}{Q}
\pstLine[linecolor=purple!40,nodesepA=-0.8,nodesepB=-0.8]{2.5,0}{2.5,3}
\pstParabolaLineInter[linecolor=purple!40,PosAngle={0,210}](O){\p}{2.5,0}{2.5,3}{U}{V}
\pstLine[linecolor=green!40,nodesepA=-2.5,nodesepB=-1.6]{1.5,2.5}{0.5,2.5}
\pstParabolaLineInter[linecolor=green!40,PosAngle={210,210}](O){\p}{1.5,2.5}{0.5,2.5}{M}{N}
\end{pspicture}
\end{LTXexample}

The macro \Lcs{pstParabolaPolarNode} is used to find the polar point $T$ of chord $AB$ on Parabola $P$.

\begin{BDef}
\Lcs{pstParabolaPolarNode}\OptArgs\Largr{O}\Largb{$p$}\Largb{A}\Largb{B}\Largb{T}\\
\Lcs{pstParabolaPolarNode}\OptArgs\Largr{O}\Largb{$p$}\Largr{F}\Largb{A}\Largb{B}\Largb{T}\\
\Lcs{pstParabolaPolarNode}\OptArgs\Largr{O}\Largb{$p$}\Largr{F}\OptArg{$L_x$}\OptArg{$L_y$}\Largb{A}\Largb{B}\Largb{T}
\end{BDef}

We use the following theorem to find the polar point $T$ of chord $AB$:
\begin{theorem}\label{ParabolaPolarPointTheorem}
Give any chord $AB$ on parabola, drawing two focal chord $AFC$ and $BFD$, where $F$ is the focus of parabola,
then drawing $FX$ which is perpendicular to $AFC$ at point $F$, and intersect with the directrix line at $X$;
also drawing $FY$ which is perpendicular to $BFD$ at point $F$, and intersect with the directrix line at $Y$.
Then the intersection $T$ of $AX$ and $BY$ is the polar point of chord $AB$.
\end{theorem}

If you don't know the focus $F$, or the directrix line, we will find them automated, otherwise you can pass them to this macro.

\begin{LTXexample}[width=6cm,pos=l]
\begin{pspicture}[showgrid=true](-1,-2)(4,4)
\psset{dotscale=0.5}\psset{PointSymbol=*}\footnotesize
\def\p{0.4}
\pstGeonode[PosAngle=-130,PointNameSep=0.2](2,0){O}
\pstParabola[linecolor=red!40](O){\p}{-1.5}{1.5}
\pstLine[linecolor=gray!40,nodesepA=-0.8,nodesepB=-0.8]{0,2}{4,1}
\pstParabolaLineInter[linecolor=gray!40,PosAngle={120,210}](O){\p}{0,2}{4,1}{P}{Q}
% if you don't know focus F or directrix line
\pstParabolaPolarNode[linecolor=purple!40,PosAngle=-90](O){\p}{P}{Q}{T}
\end{pspicture}
\end{LTXexample}

\begin{LTXexample}[width=6cm,pos=l]
\begin{pspicture}[showgrid=true](-1,-2)(4,4)
\psset{dotscale=0.5}\psset{PointSymbol=*}\footnotesize
\def\p{0.4}
\pstGeonode[PosAngle=-130,PointNameSep=0.2](2,0){O}
\pstParabola[linecolor=red!40](O){\p}{-1.5}{1.5}
\pstParabolaFocusNode[linecolor=red!40](O){\p}{F}
\pstLine[linecolor=gray!40,nodesepA=-0.8,nodesepB=-0.8]{0,2}{4,1}
\pstParabolaLineInter[linecolor=gray!40,PosAngle={120,210}](O){\p}{0,2}{4,1}{P}{Q}
% if you know focus F, but don't know directrix line
\pstParabolaPolarNode[linecolor=purple!40,PosAngle=-90](O){\p}(F){P}{Q}{T}
\end{pspicture}
\end{LTXexample}

\vspace{1cm}

\begin{LTXexample}[width=6cm,pos=l]
\begin{pspicture}[showgrid=true](-1,-2)(4,4)
\psset{dotscale=0.5}\psset{PointSymbol=*}\footnotesize
\def\p{0.4}
\pstGeonode[PosAngle=-130,PointNameSep=0.2](2,0){O}
\pstParabola[linecolor=red!40](O){\p}{-1.5}{1.5}
\pstParabolaFocusNode[linecolor=red!40](O){\p}{F}
\pstParabolaDirectrixLine[linecolor=red!40,nodesepA=-2.8,nodesepB=-2,PosAngle={-50,-50}](O){\p}{A}{B}
\pstLineAB[linecolor=red!40,nodesepA=-0.8,nodesepB=-2.5]{A}{F}
\pstLine[linecolor=gray!40,nodesepA=-0.8,nodesepB=-0.8]{0,2}{4,1}
\pstParabolaLineInter[linecolor=gray!40,PosAngle={120,210}](O){\p}{0,2}{4,1}{P}{Q}
% if you know focus F and also directrix line
\pstParabolaPolarNode[linecolor=purple!40,PosAngle=-90](O){\p}(F)[A][B]{P}{Q}{T}
\end{pspicture}
\end{LTXexample}

\vspace{10pt}

The macro \Lcs{pstParabolaTangentNode} is used to find the two nodes $A$ and $B$ on the Parabola through the point $T$.

\begin{BDef}
\Lcs{pstParabolaTangentNode}\OptArgs\Largr{O}\Largb{$p$}\Largb{T}\Largb{A}\Largb{B}
\end{BDef}

We use the following theorem to find the tangent node $A$ and $B$ of outside point $T$:
\begin{theorem}\label{ParabolaTangentPointTheorem}
Give point $T$ outside of the parabola, we draw any other two chords $TPQ$ and $TRS$,
$PS$ and $QR$ intersect at $I$, $PR$ and $QS$ intersect at $X$, $XI$ and Parabola intersect at $A$ and $B$,
then $TA$ is the tangent line through $A$ and $TB$ is the tangent line through $B$.
\end{theorem}

\begin{LTXexample}[width=6cm,pos=l]
\begin{pspicture}[showgrid=true](0,-2)(4,4)
\psset{dotscale=0.5}\psset{PointSymbol=*}\footnotesize
\def\p{0.4}
\pstGeonode[PosAngle=-130,PointNameSep=0.2](2,0){O}
\pstParabola[linecolor=red!40](O){\p}{-1.5}{1.5}
\pstGeonode[PosAngle=-90](1.5,-1){T}
\pstParabolaTangentNode[linecolor=red!50,PosAngle={80,140},PointName={T_1,T_2}](O){\p}{T}{T1}{T2}
\pstGeonode[PosAngle=-90](2,-1){P}
\pstParabolaTangentNode[linecolor=red!50,PosAngle={80,140},PointName={P_1,P_2}](O){\p}{P}{P1}{P2}
\pstGeonode[PosAngle=-90](2.3,-1){X}
\pstParabolaTangentNode[linecolor=red!50,PosAngle={80,140},PointName={X_1,X_2}](O){\p}{X}{X1}{X2}
\end{pspicture}
\end{LTXexample}

\subsection{Standard Inversion Parabola}
The Inversion Parabola $P$ with coordinate translation is defined by vertex $O(x_0,y_0)$,
the half of the focus chord axis $abs(p)$.
Note that the sign of $p$ indicates the direction of the parabola.
The equation can be written as:
\begin{equation}\label{StandardInversionParabola}
(y-y_0)^2=2p(x-x_0)
\end{equation}
and the parametric function can be written as:
\begin{equation}\label{ParametricFunctionOfStandardInversionParabola}
\left\{\begin{array}{l}
x=\dfrac{t^2}{2p}+x_0\\
y=t+y_0
\end{array}\right.
\end{equation}

The macro \Lcs{pstIParabola} is used to draw a Standard Inversion Parabola from $y_1$ to $y_2$ with Vertex $O$,
the half of the focus chord axis $abs(p)$.

\begin{BDef}
\Lcs{pstIParabola}\OptArgs\Largr{O}\Largb{$p$}\Largb{$y_1$}\Largb{$y_2$}
\end{BDef}

The macro \Lcs{pstIParabolaNode} is used to draw a node whose parameter is the given value $t$ on parabola,
please refer to equation (\ref{ParametricFunctionOfStandardInversionParabola}).
The macro \Lcs{pstIParabolaAbsNode} is used to draw a node whose abscissa is the given value $x_1$ on parabola.
The macro \Lcs{pstIParabolaOrdNode} is used to draw a node whose ordinate is the given value $y_1$ on parabola.
Note that \Lcs{pstIParabolaAbsNode} will create two nodes $A$ and $B$ at most time.

\begin{BDef}
\Lcs{pstIParabolaNode}\OptArgs\Largr{O}\Largb{$p$}\Largb{$t$}\Largb{A}\\
\Lcs{pstIParabolaAbsNode}\OptArgs\Largr{O}\Largb{$p$}\Largb{$x_1$}\Largb{A}\Largb{B}\\
\Lcs{pstIParabolaOrdNode}\OptArgs\Largr{O}\Largb{$p$}\Largb{$y_1$}\Largb{A}
\end{BDef}

\begin{LTXexample}[width=6cm,pos=l]
\begin{pspicture}[showgrid=true](-2,-2)(3,2)
\psset{dotscale=0.5}\psset{PointSymbol=*}\footnotesize
\def\p{0.4}
\pstGeonode[PosAngle=0,PointNameSep=0.2](2,0){O}
\pstIParabola[linecolor=blue!40](O){-\p}{-1.5}{1.5}
\pstIParabolaNode[PosAngle=90](O){-\p}{1}{A}
\pstIParabolaAbsNode[PosAngle=90,PointName={X_2,X_3},PosAngle={-90,90}](O){-\p}{1.5}{X2}{X3}
\pstIParabolaOrdNode[PosAngle=-90,PointName=Y_3](O){-\p}{-1}{Y3}
\end{pspicture}
\end{LTXexample}

The macro \Lcs{pstIParabolaFocusNode} is used to find the focus of the parabola,
and the macro \Lcs{pstIParabolaDirectrixLine} is used to find the directrix line of the parabola.

\begin{BDef}
\Lcs{pstIParabolaFocusNode}\OptArgs\Largr{O}\Largb{$p$}\Largb{F}\\
\Lcs{pstIParabolaDirectrixLine}\OptArgs\Largr{O}\Largb{$p$}\Largb{$L_x$}\Largb{$L_y$}
\end{BDef}

\begin{LTXexample}[width=6cm,pos=l]
\begin{pspicture}[showgrid=true](-2,-2)(3,2)
\psset{dotscale=0.5}\psset{PointSymbol=*}\footnotesize
\def\p{0.4}
\pstGeonode[PosAngle=-30,PointNameSep=0.2](2,0){O}
\pstIParabola[linecolor=blue!40](O){-\p}{-1.5}{1.5}
\pstIParabolaFocusNode[linecolor=blue!40,PosAngle=120](O){-\p}{F}
\pstIParabolaDirectrixLine[linecolor=blue!40,nodesepA=-2,nodesepB=-1,PosAngle={50,20}](O){-\p}{C}{D}
\pstLine[linecolor=blue!40,nodesepA=-0.8,nodesepB=-2.5]{C}{F}
\end{pspicture}
\end{LTXexample}

The macro \Lcs{pstIParabolaLineInter} is used to find the intersections $C$ and $D$ of the parabola and the given line $AB$.

\begin{BDef}
\Lcs{pstIParabolaLineInter}\OptArgs\Largr{O}\Largb{$p$}\Largb{A}\Largb{B}\Largb{C}\Largb{D}
\end{BDef}

\begin{LTXexample}[width=6cm,pos=l]
\begin{pspicture}[showgrid=true](-2,-2)(3,2)
\psset{dotscale=0.5}\psset{PointSymbol=*}\footnotesize
\def\p{0.4}
\pstGeonode[PosAngle=0,PointNameSep=0.2](2,0){O}
\pstIParabola[linecolor=blue!40](O){-\p}{-1.5}{1.5}
\pstLine[linecolor=gray!40]{0,2}{1,-2}
\pstIParabolaLineInter[linecolor=gray!40,PosAngle={70,-90}](O){-\p}{1,-2}{0,2}{P}{Q}
\pstLine[linecolor=purple!40]{1.2,-1.5}{1.2,1.5}
\pstIParabolaLineInter[linecolor=purple!40,PosAngle={-40,210}](O){-\p}{1.2,-1.5}{1.2,1.5}{U}{V}
\pstLine[linecolor=green!40]{-1,0.5}{2.5,0.5}
\pstIParabolaLineInter[linecolor=green!40,PosAngle={70,-90}](O){-\p}{-1,0.5}{2.5,0.5}{M}{N}
\end{pspicture}
\end{LTXexample}

The macro \Lcs{pstIParabolaPolarNode} is used to find the polar point $T$ of chord $AB$ on Parabola $P$.

\begin{BDef}
\Lcs{pstIParabolaPolarNode}\OptArgs\Largr{O}\Largb{$p$}\Largb{A}\Largb{B}\Largb{T}\\
\Lcs{pstIParabolaPolarNode}\OptArgs\Largr{O}\Largb{$p$}\Largr{F}\Largb{A}\Largb{B}\Largb{T}\\
\Lcs{pstIParabolaPolarNode}\OptArgs\Largr{O}\Largb{$p$}\Largr{F}\OptArg{$L_x$}\OptArg{$L_y$}\Largb{A}\Largb{B}\Largb{T}
\end{BDef}

We also use the theorem \ref{ParabolaPolarPointTheorem} to find the polar point $T$ of chord $AB$.
If you don't know the focus $F$, or the directrix line, we will find them automated, otherwise you can pass them to this macro.

\begin{LTXexample}[width=6cm,pos=l]
\begin{pspicture}[showgrid=true](1,-2)(5,2)
\psset{dotscale=0.5}\psset{PointSymbol=*}\footnotesize
\def\p{0.4}
\pstGeonode[PosAngle=-130,PointNameSep=0.2](2,0){O}
\pstIParabola[linecolor=red!40](O){\p}{-1.5}{1.5}
\pstLine[linecolor=gray!40,nodesepA=-0.5]{2,1}{4,-2}
\pstIParabolaLineInter[PosAngle={80,-100}](O){\p}{2,1}{4,-2}{P}{Q}
% if you don't know focus F or directrix line
\pstIParabolaPolarNode[linecolor=purple!40,PosAngle=-90](O){\p}{P}{Q}{T}
\end{pspicture}
\end{LTXexample}

\begin{LTXexample}[width=6cm,pos=l]
\begin{pspicture}[showgrid=true](1,-2)(5,2)
\psset{dotscale=0.5}\psset{PointSymbol=*}\footnotesize
\def\p{0.4}
\pstGeonode[PosAngle=-130,PointNameSep=0.2](2,0){O}
\pstIParabola[linecolor=red!40](O){\p}{-1.5}{1.5}
\pstIParabolaFocusNode[linecolor=red!40](O){\p}{F}
\pstLine[linecolor=gray!40,nodesepA=-0.5]{2,1}{4,-2}
\pstIParabolaLineInter[PosAngle={80,-100}](O){\p}{2,1}{4,-2}{P}{Q}
% if you know focus F, but don't known directrix line
\pstIParabolaPolarNode[linecolor=purple!40,PosAngle=-90](O){\p}(F){P}{Q}{T}
\end{pspicture}
\end{LTXexample}

\vspace{1cm}

\begin{LTXexample}[width=6cm,pos=l]
\begin{pspicture}[showgrid=true](1,-2)(5,2)
\psset{dotscale=0.5}\psset{PointSymbol=*}\footnotesize
\def\p{0.4}
\pstGeonode[PosAngle=-130,PointNameSep=0.2](2,0){O}
\pstIParabola[linecolor=red!40](O){\p}{-1.5}{1.5}
\pstIParabolaFocusNode[linecolor=red!40](O){\p}{F}
\pstIParabolaDirectrixLine[linecolor=red!40,nodesepA=-2,nodesepB=-1,PosAngle={180,180}](O){\p}{A}{B}
\pstLine[linecolor=gray!40,nodesepA=-0.5]{2,1}{4,-2}
\pstIParabolaLineInter[PosAngle={80,-100}](O){\p}{2,1}{4,-2}{P}{Q}
% if you know focus F and also directrix line
\pstIParabolaPolarNode[linecolor=purple!40,PosAngle=-90](O){\p}(F)[A][B]{P}{Q}{T}
\end{pspicture}
\end{LTXexample}

\vspace{10pt}

The macro \Lcs{pstIParabolaTangentNode} is used to find the two nodes $A$ and $B$ on the Parabola through the point $T$.

\begin{BDef}
\Lcs{pstIParabolaTangentNode}\OptArgs\Largr{O}\Largb{$p$}\Largb{T}\Largb{A}\Largb{B}
\end{BDef}

We also use the theorem \ref{ParabolaTangentPointTheorem} to find the tangent node $A$ and $B$ of outside point $T$!

\begin{LTXexample}[width=6cm,pos=l]
\begin{pspicture}[showgrid=true](1,0)(5,4)
\psset{dotscale=0.5}\psset{PointSymbol=*}\footnotesize
\def\p{0.4}
\pstGeonode[PosAngle=-45,PointNameSep=0.2](4,2){O}
\pstIParabola[linecolor=blue!40](O){-\p}{-1.5}{1.5}
\pstGeonode[PosAngle=0](5,1.5){T}
\pstIParabolaTangentNode[linecolor=red!50,PosAngle={80,-100},PointName={T_1,T_2}](O){-\p}{T}{T1}{T2}
\pstGeonode[PosAngle=0](5,2.5){P}
\pstIParabolaTangentNode[linecolor=red!50,PosAngle={80,90},PointName={P_1,P_2}](O){-\p}{P}{P1}{P2}
\pstGeonode[PosAngle=0](5,2){X}
\pstIParabolaTangentNode[linecolor=red!50,PosAngle={80,-100},PointName={X_1,X_2}](O){-\p}{X}{X1}{X2}
\end{pspicture}
\end{LTXexample}

\subsection{General Parabola}
The General Parabola $P$ with coordinate translation and rotation is defined by vertex $O(x_0,y_0)$,
the half of the focus chord axis $abs(p)$, the sign of $p$ indicates the direction of the parabola,
and the rotation angle $\theta$ of the symmetrical axis.
The symmetrical axis is perpendicular to x-axis when $\theta=0^\circ$, and perpendicular to y-axis when $\theta=90^\circ$.

The equation can be got from the parametric function of the parabola equation (\ref{ParametricFunctionOfStandardParabola}),
using the rotation transform formula (\ref{RotationTransformFormula}), then we have
\begin{equation}
\left\{\begin{array}{l}
x'=(t+x_0)\cos\theta-(\dfrac{t^2}{2p}+y_0)\sin\theta=x_0'+t\cos\theta-t^2\dfrac{\sin\theta}{2p}\\
y'=(t+x_0)\sin\theta+(\dfrac{t^2}{2p}+y_0)\cos\theta=y_0'+t\sin\theta+t^2\dfrac{\cos\theta}{2p}
\end{array}\right.
\end{equation}
where the $x_0'$ and $y_0'$ are the coordinate of the given vertex O after rotation.
So we get the parametric function of the General Parabola with coordinate translation and rotation as following:
\begin{equation}\label{ParametricFunctionOfGeneralParabola}
\left\{\begin{array}{l}
x=x_0+t\cos\theta-t^2\dfrac{\sin\theta}{2p}\\
y=y_0+t\sin\theta+t^2\dfrac{\cos\theta}{2p}
\end{array}\right.
\end{equation}

The macro \Lcs{pstGeneralParabola} is used to draw a General Parabola from $x_1$ to $x_2$ with Vertex $O$,
the half of the focus chord axis $abs(p)$.

\begin{BDef}
\Lcs{pstGeneralParabola}\OptArgs\Largr{O}\Largb{$p$}\OptArg{$\theta$}\Largb{$x_1$}\Largb{$x_2$}
\end{BDef}

The macro \Lcs{pstGeneralParabolaNode} is used to draw a node whose parameter is the given value $t$ on parabola,
please refer to equation (\ref{ParametricFunctionOfGeneralParabola}).
The macro \Lcs{pstGeneralParabolaAbsNode} is used to draw a node whose abscissa is the given value $x_1$ on parabola.
The macro \Lcs{pstGeneralParabolaOrdNode} is used to draw a node whose ordinate is the given value $y_1$ on parabola.

Note that \Lcs{pstGeneralParabolaAbsNode} and \Lcs{pstGeneralParabolaOrdNode} both create two nodes $A$ and $B$
at most time.

\begin{BDef}
\Lcs{pstGeneralParabolaNode}\OptArgs\Largr{O}\Largb{$p$}\OptArg{$\theta$}\Largb{$t$}\Largb{A}\\
\Lcs{pstGeneralParabolaAbsNode}\OptArgs\Largr{O}\Largb{$p$}\OptArg{$\theta$}\Largb{$x_1$}\Largb{A}\Largb{B}\\
\Lcs{pstGeneralParabolaOrdNode}\OptArgs\Largr{O}\Largb{$p$}\OptArg{$\theta$}\Largb{$y_1$}\Largb{A}\Largb{B}
\end{BDef}

\begin{LTXexample}[width=6cm,pos=l]
\begin{pspicture}[showgrid=true](-1,-1)(4,4)
\psset{dotscale=0.5}\psset{PointSymbol=*}\footnotesize
\def\p{0.4}
\pstGeonode[PosAngle=-40,PointNameSep=0.2](2,0){O}
\pstGeneralParabola[linecolor=red!10](O){\p}[0]{-1.5}{1.5}
\pstGeneralParabola[linecolor=red!15](O){\p}[10]{-1.5}{1.5}
\pstGeneralParabola[linecolor=red!25](O){\p}[30]{-1.5}{1.5}
\pstGeneralParabola[linecolor=red!40](O){\p}[50]{-1.5}{1.5}
\pstGeneralParabola[linecolor=red!60](O){\p}[90]{-1.5}{1.5}
\pstGeneralParabolaNode[PosAngle=0,linecolor=blue!60](O){\p}[30]{1.0}{A}
\pstGeneralParabolaAbsNode[PosAngle={0,0},linecolor=blue!60](O){\p}[30]{1.0}{D}{E}
\pstGeneralParabolaAbsNode[PosAngle={0,0},linecolor=blue!60](O){\p}[50]{1.0}{F}{G}
\pstGeneralParabolaAbsNode[PosAngle={0,0},linecolor=blue!60](O){\p}[90]{1.0}{H}{I}
\pstGeneralParabolaOrdNode[PosAngle={90,0},linecolor=purple!60](O){\p}[30]{0.5}{U}{V}
\pstGeneralParabolaOrdNode[PosAngle={90,90},linecolor=purple!60](O){\p}[50]{0.5}{M}{N}
\pstGeneralParabolaOrdNode[PosAngle={90,-90},linecolor=purple!60](O){\p}[90]{0.5}{S}{T}
\end{pspicture}
\end{LTXexample}

\vspace{10pt}

The Macro \Lcs{pstGeneralParabolaFl} is used to define a General Parabola with Focus $F$, and the directrix line $l$. It just calculate the vertex $O$, half focal chord $p$, and the rotation
angle $\theta$ of the symmetrical axis, then you can pass them into macro \Lcs{pstGeneralParabola} to draw this parabola.

\begin{BDef}
\Lcs{pstGeneralParabolaFl}\OptArgs\Largb{F}\Largb{A}\Largb{B}\Largb{O}\Largb{p}\Largb{$\theta$}
\end{BDef}

The output parameter \texttt{O} is a node name to store the vertex point, its label and symbol can
be controlled by the options for \PST\ node, such as \Lkeyword{PosAngle}.
The output parameter \texttt{p} is a PostScript key to store the value of half focal chord.
The output parameter \texttt{$\theta$} is also a PostScript key to store the rotation angle of symmetrical axis,
when you pass it to \Lcs{pstGeneralParabola}, PostScript will lookup the value of this key in current dictionary.

\begin{LTXexample}[width=6cm,pos=l]
\begin{pspicture}[showgrid=true](-2,-2)(2,2)
\psset{unit=1.0cm}\psset{dotscale=0.5}\footnotesize
\psset{CodeFig=true,CodeFigColor=gray!50}\psset{PointSymbol=*}
\pstGeonode[PosAngle=-60](1,1){F_1}
\pstGeonode[PosAngle=-60](-1,-1){F_2}
\pstGeonode[PosAngle=60](1,-1){B}
\pstGeonode[PosAngle=60](-1,1){A}
\pstGeneralParabolaFl[PosAngle=60]{F_1}{A}{B}{O_1}{semifocalchordp1}{SymAxisRotAngle1}
\pstGeneralParabola[linecolor=red!60](O_1){semifocalchordp1}[SymAxisRotAngle1]{-2}{2}
\pstGeneralParabolaFl[PosAngle=60]{F_2}{A}{B}{O_2}{semifocalchordp2}{SymAxisRotAngle2}
\pstGeneralParabola[linecolor=blue!60](O_2){semifocalchordp2}[SymAxisRotAngle2]{-2}{2}
\pstGeneralParabolaFl[PosAngle=60]{B}{F_1}{F2}{O_3}{semifocalchordp3}{SymAxisRotAngle3}
\pstGeneralParabola[linecolor=green!60](O_3){semifocalchordp3}[SymAxisRotAngle3]{-2}{2}
\pstGeneralParabolaFl[PosAngle=60]{A}{F_1}{F2}{O_4}{semifocalchordp4}{SymAxisRotAngle4}
\pstGeneralParabola[linecolor=green!60](O_4){semifocalchordp4}[SymAxisRotAngle4]{-2}{2}
\end{pspicture}
\end{LTXexample}

\vspace{10pt}

The Macro \Lcs{pstGeneralParabolaCoef} is used to define a General Parabola by the quadratic curve equation $ax^2+bxy+cy^2+dx+ey+f=0$,
it just calculate the vertex $O$, half focal chord $p$ and the rotation angle $\theta$ of the symmetrical axis,
then you can pass them into macro \Lcs{pstGeneralParabola} to draw this parabola.
The package \texttt{pst-func} provides macro \Lcs{psplotImp} to draw an implicit defined functions too,
but it can't tell you the geometrical elements like as center or radii, and it will take more time to
calculate the function value point by point.

\begin{BDef}
\Lcs{pstGeneralParabolaCoef}\OptArgs\Largb{a,b,c,d,e,f}\Largb{O}\Largb{p}\Largb{$\theta$}
\end{BDef}

The output parameter \texttt{O}, \texttt{p} and \texttt{$\theta$}are same with \Lcs{pstGeneralParabolaFl}.
They are set to zero if the coeffients are invalid to construct a parabola.
If you pass the zero $p$ into macro \Lcs{pstGeneralParabola}, it will abort with the exception of dividing by zero.

In the following example, we use \Lcs{psplotImp} to draw the same parabolas, just to check the results
given by macros \Lcs{pstGeneralParabolaCoef} are correct.

\begin{LTXexample}[width=6cm,pos=l]
\begin{pspicture}[showgrid=true](-1,-1)(4,4)
\psset{unit=0.40cm}\footnotesize\psset{dotscale=0.5}
\psset{CodeFig=true}\psset{PointSymbol=*}
%x^2-2xy+y^2-8x+16=0
\psplotImp[linecolor=green!30](-10,-10)(10,10){ 1 x dup mul mul -2 x mul y mul add 1 y dup mul mul add -8 x mul add 0 y mul add 16 add }
\pstGeneralParabolaCoef[PosAngle=0,CodeFigColor=red!50]{1,-2,1,-8,0,16}{O_1}{P1}{SymAxisRotAngle1}
\pstGeneralParabola[linecolor=red!60](O_1){P1}[SymAxisRotAngle1]{-3}{3}
%x^2+2xy+y^2+2x-2y-5=0
\psplotImp[linecolor=green!30](-10,-10)(10,10){ 1 x dup mul mul 2 x mul y mul add 1 y dup mul mul add 2 x mul add -2 y mul add -5 add }
\pstGeneralParabolaCoef[PosAngle=-90,CodeFigColor=black!60]{1,2,1,2,-2,-5}{O_2}{P2}{SymAxisRotAngle2}
\pstGeneralParabola[linecolor=black!60](O_2){P2}[SymAxisRotAngle2]{-3}{3}
\end{pspicture}
\end{LTXexample}

\vspace{10pt}

The Macro \Lcs{pstGeneralParabolaABCDE} is used to define a General Parabola by the given five points $A,B,C,D,E$,
it just calculate the vertex $O$, half focal chord $p$ and the rotation angle $\theta$ of the symmetrical axis,
then you can pass them into macro \Lcs{pstGeneralParabola} to draw this parabola.

\begin{BDef}
\Lcs{pstGeneralParabolaABCDE}\OptArgs\Largb{A}\Largb{B}\Largb{C}\Largb{D}\Largb{E}\Largb{O}\Largb{p}\Largb{$\theta$}
\end{BDef}

The output parameter \texttt{O}, \texttt{p} and \texttt{$\theta$} are same with \Lcs{pstGeneralParabolaFl}.
They are set to zero if the points are invalid to construct a parabola.
If you pass the zero $p$ into macro \Lcs{pstGeneralParabola}, it will abort with the exception of dividing by zero.

Note the algorithm may fit a hyperbola quadratic curve from the given five points,
in order to get the right parabola curve, you must input the point coordinates very precisely.
In the following example, if you input point $A$ as $(3,1.732)$, it will fail as no such parabola
can fit these five points.

\begin{LTXexample}[width=6cm,pos=l]
\begin{pspicture}[showgrid=true](0,-2)(4,2)
\psset{unit=0.5cm}\footnotesize\psset{PointSymbol=*}
\psset{CodeFig=true,CodeFigColor=gray!50}
% five points from y^2-2x+3=0
\pstGeonode[PosAngle=90](3,1.73205){A}
\pstGeonode[PosAngle=90](2,1){B}
\pstGeonode[PosAngle=-90](2,-1){C}
\pstGeonode[PosAngle=90](6,3){D}
\pstGeonode[PosAngle=-90](6,-3){E}
\pstGeneralParabolaABCDE[PosAngle=235]{A}{B}{C}{D}{E}{O}{P}{SymAxisRotAngle}
\pstGeneralParabola[linecolor=red!60](O){P}[SymAxisRotAngle]{-3}{3}
\end{pspicture}
\end{LTXexample}

\vspace{10pt}

The macro \Lcs{pstGeneralParabolaFocusNode} is used to find the focus of the parabola,
and the macro \Lcs{pstGeneralParabolaDirectrixLine} is used to find the directrix line of the parabola.

\begin{BDef}
\Lcs{pstGeneralParabolaFocusNode}\OptArgs\Largr{O}\Largb{$p$}\OptArg{$\theta$}\Largb{F}\\
\Lcs{pstGeneralParabolaDirectrixLine}\OptArgs\Largr{O}\Largb{$p$}\OptArg{$\theta$}\Largb{$L_x$}\Largb{$L_y$}
\end{BDef}

\begin{LTXexample}[width=6cm,pos=l]
\begin{pspicture}[showgrid=true](0,-1)(4,3)
\psset{dotscale=0.5}\psset{PointSymbol=*}\footnotesize
\def\p{0.4}
\pstGeonode[PosAngle=-90,PointNameSep=0.2](2,0){O}
\pstGeneralParabola[linecolor=red!40](O){\p}[50]{-1.5}{1.5}
\pstGeneralParabolaFocusNode[linecolor=red!40,PosAngle=90](O){\p}[50]{F}
\pstLineAB[linestyle=dashed,linecolor=black!25,nodesepA=-0.5,nodesepB=-2.5]{O}{F}
\pstGeneralParabolaDirectrixLine[linecolor=red!40,nodesepA=-2,nodesepB=-1,PosAngle={-60,-60},PointName={L_1,L_2}](O){\p}[50]{L1}{L2}
\pstGeneralParabolaNode[linecolor=red!60](O){\p}[50]{1.0}{A}
\end{pspicture}
\end{LTXexample}

The macro \Lcs{pstGeneralParabolaLineInter} is used to find the intersections $C$ and $D$ of the parabola and the given line $AB$.

\begin{BDef}
\Lcs{pstGeneralParabolaLineInter}\OptArgs\Largr{O}\Largb{$p$}\OptArg{$\theta$}\Largb{A}\Largb{B}\Largb{C}\Largb{D}
\end{BDef}

When General Parabola becomes a Standard Parabola, the intersections with any kind of lines:
\begin{LTXexample}[width=6cm,pos=l]
\begin{pspicture}[showgrid=true](0,-1)(4,3)
\psset{dotscale=0.5}\psset{PointSymbol=*}\footnotesize
\def\p{0.4}
\pstGeonode[PosAngle=-90,PointNameSep=0.2](2,0){O}
\pstGeneralParabola[linecolor=red!40](O){\p}[0]{-1.5}{1.5}
\pstGeneralParabolaFocusNode[linecolor=red!40,PosAngle=50](O){\p}[0]{F}
\pstLineAB[linestyle=dashed,linecolor=black!25,nodesepA=-0.2,nodesepB=-2.5]{O}{F}
\pstLine[linestyle=dashed,linecolor=gray!40,nodesep=-0.8]{1,0}{1,2}
\pstGeneralParabolaLineInter[linecolor=red!40,PosAngle={40,-90}](O){\p}[0]{1,0}{1,2}{A}{B}
\pstLine[linestyle=dashed,linecolor=gray!40,nodesep=0]{0.5,0.5}{3.5,1}
\pstGeneralParabolaLineInter[linecolor=red!40,PosAngle={-110,-60}](O){\p}[0]{0.5,0.5}{3.5,1}{C}{D}
\end{pspicture}
\end{LTXexample}

Here is the intersections of a real General Parabola with any kind of lines:
\begin{LTXexample}[width=6cm,pos=l]
\begin{pspicture}[showgrid=true](-1,-1)(3,3)
\psset{dotscale=0.5}\psset{PointSymbol=*}\footnotesize
\def\p{0.4}
\pstGeonode[PosAngle=-60,PointNameSep=0.2](2,0){O}
\pstGeneralParabola[linecolor=red!40](O){\p}[50]{-1.5}{1.5}
\pstGeneralParabolaFocusNode[linecolor=red!40,PosAngle=80](O){\p}[50]{F}
\pstLineAB[linestyle=dashed,linecolor=black!25,nodesepA=-0.2,nodesepB=-2.5]{O}{F}
\pstLine[linestyle=dashed,linecolor=gray!40,nodesep=-0.8]{1,-1}{1,3}
\pstGeneralParabolaLineInter[linecolor=red!40,PosAngle={-150,40}](O){\p}[50]{1,-1}{1,3}{A}{B}
\pstLine[linestyle=dashed,linecolor=gray!40,nodesep=0.0]{-1,0}{3,2}
\pstGeneralParabolaLineInter[linecolor=red!40,PosAngle={90,70}](O){\p}[50]{-1,0}{3,2}{C}{D}
% a line with gradient k=-\cos50/\sin50 parallel to OF
\pstLineAS[linestyle=dashed,linecolor=gray!40,nodesep=-0.8,PointName=none,PointSymbol=none](0,1){50 cos 50 sin div neg}{X}
\pstGeneralParabolaLineInter[linecolor=red!40,PosAngle={-90,-90}](O){\p}[50]{0,1}{X}{E}{G}
\end{pspicture}
\end{LTXexample}

When General Parabola becomes a Standard Inversion Parabola, the intersections with any kind of lines:
\begin{LTXexample}[width=6cm,pos=l]
\begin{pspicture}[showgrid=true](-1,-2)(3,2)
\psset{dotscale=0.5}\psset{PointSymbol=*}\footnotesize
\def\p{0.4}
\pstGeonode[PosAngle=0,PointNameSep=0.2](2,0){O}
\pstGeneralParabola[linecolor=red!40](O){\p}[90]{-1.5}{1.5}
\pstGeneralParabolaFocusNode[linecolor=red!40,PosAngle=120](O){\p}[90]{F}
\pstLineAB[linestyle=dashed,linecolor=black!25,nodesepA=-0.2,nodesepB=-2.5]{O}{F}
\pstLine[linestyle=dashed,linecolor=gray!40,nodesep=-0.8]{1,-1}{1,2}
\pstGeneralParabolaLineInter[linecolor=red!40,PosAngle={-60,60}](O){\p}[90]{1,-1}{1,2}{A}{B}
\pstLine[linestyle=dashed,linecolor=gray!40,nodesep=-0.8]{0,-1}{2,1}
\pstGeneralParabolaLineInter[linecolor=red!40,PosAngle={-90,5}](O){\p}[90]{0,-1}{2,1}{C}{D}
\end{pspicture}
\end{LTXexample}

The macro \Lcs{pstGeneralParabolaPolarNode} is used to find the polar point $T$ of chord $AB$ on Parabola $P$.

\begin{BDef}
\Lcs{pstGeneralParabolaPolarNode}\OptArgs\Largr{O}\Largb{$p$}\OptArg{$\theta$}\Largb{A}\Largb{B}\Largb{T}\\
\Lcs{pstGeneralParabolaPolarNode}\OptArgs\Largr{O}\Largb{$p$}\OptArg{$\theta$}\Largr{F}\Largb{A}\Largb{B}\Largb{T}\\
\Lcs{pstGeneralParabolaPolarNode}\OptArgs\Largr{O}\Largb{$p$}\OptArg{$\theta$}\Largr{F}\OptArg{$L_x$}\OptArg{$L_y$}\Largb{A}\Largb{B}\Largb{T}
\end{BDef}

We also use the theorem \ref{ParabolaPolarPointTheorem} to find the polar point $T$ of chord $AB$.
If you don't know the focus $F$, or the directrix line, we will find them automated, otherwise you can pass them to this macro.

\begin{LTXexample}[width=6cm,pos=l]
\begin{pspicture}[showgrid=true](-1,-2)(3,2)
\psset{dotscale=0.5}\psset{PointSymbol=*}\footnotesize
\def\p{0.4}
\pstGeonode[PosAngle=-60,PointNameSep=0.2](2,0){O}
\pstGeneralParabola[linecolor=red!40](O){\p}[80]{-1.5}{1.5}
\pstGeneralParabolaFocusNode[linecolor=red!40,PosAngle=200](O){\p}[80]{F}
\pstGeneralParabolaDirectrixLine[linecolor=red!40,nodesepA=-2,nodesepB=-1,PosAngle={0,0},PointName={L_x,L_y}](O){\p}[80]{Lx}{Ly}
\pstLine[linestyle=dashed,linecolor=black!25,nodesepA=-0.2,nodesepB=-2.5]{O}{F}
\pstLine[linestyle=dashed,linecolor=gray!40,nodesep=-0.4]{0.5,-1.2}{2,1}
\pstGeneralParabolaLineInter[linecolor=red!40,PosAngle={-60,90}](O){\p}[80]{0.5,-1.2}{2,1}{A}{B}
%\pstGeneralParabolaPolarNode[linecolor=red!40,PosAngle=-90](O){\p}[80]{A}{B}{T}
%\pstGeneralParabolaPolarNode[linecolor=red!40,PosAngle=-90](O){\p}[80](F){A}{B}{T}
\pstGeneralParabolaPolarNode[linecolor=red!40,PosAngle=-90](O){\p}[80](F)[Lx][Ly]{A}{B}{T}
\end{pspicture}
\end{LTXexample}

The macro \Lcs{pstGeneralParabolaTangentNode} is used to find the two nodes $A$ and $B$ on the Parabola through the point $T$.

\begin{BDef}
\Lcs{pstGeneralParabolaTangentNode}\OptArgs\Largr{O}\Largb{$p$}\OptArg{$\theta$}\Largb{T}\Largb{A}\Largb{B}
\end{BDef}

We also use the theorem \ref{ParabolaTangentPointTheorem} to find the tangent node $A$ and $B$ of outside point $T$.

\begin{LTXexample}[width=6cm,pos=l]
\begin{pspicture}[showgrid=true](-1,-2)(3,2)
\psset{dotscale=0.5}\psset{PointSymbol=*}\footnotesize
\def\p{0.4}
\pstGeonode[PosAngle=0,PointNameSep=0.2](2,0){O}
\pstGeneralParabola[linecolor=red!40](O){\p}[80]{-1.5}{1.5}
\pstGeonode[PosAngle=0](2.5,-0.5){R}(2.5,-0.2){T}(2.5,0.6){S}
\pstGeneralParabolaTangentNode[linecolor=red!40,PosAngle={-90,220},PointName={R_1,R_2}](O){\p}[80]{R}{R1}{R2}
\pstGeneralParabolaTangentNode[linecolor=red!40,PosAngle={-90,170},PointName={T_1,T_2}](O){\p}[80]{T}{T1}{T2}
\pstGeneralParabolaTangentNode[linecolor=red!40,PosAngle={-90,180},PointName={S_1,S_2}](O){\p}[80]{S}{S1}{S2}
\end{pspicture}
\end{LTXexample}

\subsection{General Inversion Parabola}
The General Inversion Parabola $P$ with coordinate translation and rotation is defined by vertex $O$,
the half of the focus chord axis $abs(p)$, the sign of $p$ indicates the direction of the parabola,
and the rotation angle $\theta$ of the symmetrical axis.

The equation can be got from the parametric function of the inversion parabola (\ref{ParametricFunctionOfStandardInversionParabola}),
using the rotation transform formula (\ref{RotationTransformFormula}), then we have
\begin{equation}
\left\{\begin{array}{l}
x'=(\dfrac{t^2}{2p}+x_0)\cos\theta-(t+y_0)\sin\theta=x_0'-t\sin\theta+t^2\dfrac{\cos\theta}{2p}\\
y'=(\dfrac{t^2}{2p}+x_0)\sin\theta+(t+y_0)\cos\theta=y_0'+t\cos\theta+t^2\dfrac{\sin\theta}{2p}
\end{array}\right.
\end{equation}
where the $x_0'$ and $y_0'$ are the coordinate of the given vertex O after rotation.
So we get the parametric function of the General Inversion Parabola with coordinate translation and rotation as following:
\begin{equation}\label{ParametricFunctionOfGeneralInversionParabola}
\left\{\begin{array}{l}
x=x_0-t\sin\theta+t^2\dfrac{\cos\theta}{2p}\\
y=y_0+t\cos\theta+t^2\dfrac{\sin\theta}{2p}
\end{array}\right.
\end{equation}

The macro \Lcs{pstGeneralIParabola} is used to draw a Standard Inversion Parabola from $y_1$ to $y_2$ with Vertex $O$,
the half of the focus chord axis $abs(p)$.

\begin{BDef}
\Lcs{pstGeneralIParabola}\OptArgs\Largr{O}\Largb{$p$}\OptArg{$\theta$}\Largb{$y_1$}\Largb{$y_2$}
\end{BDef}

The macro \Lcs{pstGeneralIParabolaNode} is used to draw a node whose parameter is the given value $t$ on parabola,
please refer to equation (\ref{ParametricFunctionOfGeneralInversionParabola}).
The macro \Lcs{pstGeneralIParabolaAbsNode} is used to draw a node whose abscissa is the given value $x_1$ on parabola.
The macro \Lcs{pstGeneralIParabolaOrdNode} is used to draw a node whose ordinate is the given value $y_1$ on parabola.

Note that \Lcs{pstGeneralIParabolaAbsNode} and \Lcs{pstGeneralIParabolaOrdNode} will create two nodes $A$ and $B$ at most time.

\begin{BDef}
\Lcs{pstGeneralIParabolaNode}\OptArgs\Largr{O}\Largb{$p$}\OptArg{$\theta$}\Largb{$t$}\Largb{A}\\
\Lcs{pstGeneralIParabolaAbsNode}\OptArgs\Largr{O}\Largb{$p$}\OptArg{$\theta$}\Largb{$x_1$}\Largb{A}\Largb{B}\\
\Lcs{pstGeneralIParabolaOrdNode}\OptArgs\Largr{O}\Largb{$p$}\OptArg{$\theta$}\Largb{$y_1$}\Largb{A}\Largb{B}
\end{BDef}

\begin{LTXexample}[width=6cm,pos=l]
\begin{pspicture}[showgrid=true](-1,0)(3,5)
\psset{dotscale=0.5}\psset{PointSymbol=*}\footnotesize
\def\p{0.4}
\pstGeonode[PosAngle=210,PointNameSep=0.2](0,2){O}
\pstGeneralIParabola[linecolor=blue!10](O){\p}[0]{-1.5}{1.5}
\pstGeneralIParabola[linecolor=blue!15](O){\p}[10]{-1.5}{1.5}
\pstGeneralIParabola[linecolor=blue!25](O){\p}[30]{-1.5}{1.5}
\pstGeneralIParabola[linecolor=blue!30](O){\p}[40]{-1.5}{1.5}
\pstGeneralIParabola[linecolor=blue!40](O){\p}[50]{-1.5}{1.5}
\pstGeneralIParabola[linecolor=blue!60](O){\p}[90]{-1.5}{1.5}
\pstGeneralIParabolaNode[linecolor=red!60,PosAngle=90](O){\p}[30]{1.0}{A}
\pstGeneralIParabolaNode[linecolor=red!60,PosAngle=170](O){\p}[50]{1.0}{B}
\pstGeneralIParabolaAbsNode[linecolor=red!40,PosAngle={-45,90}](O){\p}[50]{1.0}{C}{D}
\pstGeneralIParabolaAbsNode[linecolor=red!60,PosAngle={0,-90}](O){\p}[90]{1.0}{E}{F}
\pstGeneralIParabolaOrdNode[linecolor=red!60,PosAngle={90,150}](O){\p}[50]{2.5}{G}{H}
\pstGeneralIParabolaOrdNode[linecolor=blue!60,PosAngle={180,-90}](O){\p}[90]{2.5}{J}{K}
\end{pspicture}
\end{LTXexample}

The macro \Lcs{pstGeneralIParabolaFocusNode} is used to find the focus of the parabola,
and the macro \Lcs{pstGeneralIParabolaDirectrixLine} is used to find the directrix line of the parabola.

\begin{BDef}
\Lcs{pstGeneralIParabolaFocusNode}\OptArgs\Largr{O}\Largb{$p$}\OptArg{$\theta$}\Largb{F}\\
\Lcs{pstGeneralIParabolaDirectrixLine}\OptArgs\Largr{O}\Largb{$p$}\OptArg{$\theta$}\Largb{$L_x$}\Largb{$L_y$}
\end{BDef}

\begin{LTXexample}[width=6cm,pos=l]
\begin{pspicture}[showgrid=true](-2,0)(2,4)
\psset{dotscale=0.5}\psset{PointSymbol=*}\footnotesize
\psset{PointName=none,nodesepA=-2,nodesepB=-1}
\pstGeonode(0,2){O}\def\p{0.8}
\psset{linecolor=blue!60}
\pstGeneralIParabola(O){\p}[0]{-1.5}{1.5}
\pstGeneralIParabolaFocusNode(O){\p}[0]{A}
\pstGeneralIParabolaDirectrixLine(O){\p}[0]{A1}{A2}
\psset{linecolor=red!60}
\pstGeneralIParabola(O){\p}[45]{-1.5}{1.5}
\pstGeneralIParabolaFocusNode(O){\p}[45]{B}
\pstGeneralIParabolaDirectrixLine(O){\p}[45]{B1}{B2}
\psset{linecolor=green!60}
\pstGeneralIParabola(O){\p}[90]{-1.5}{1.5}
\pstGeneralIParabolaFocusNode(O){\p}[90]{C}
\pstGeneralIParabolaDirectrixLine(O){\p}[90]{C1}{C2}
\psset{linecolor=cyan!60}
\pstGeneralIParabola(O){\p}[135]{-1.5}{1.5}
\pstGeneralIParabolaFocusNode(O){\p}[135]{D}
\pstGeneralIParabolaDirectrixLine(O){\p}[135]{D1}{D2}
\psset{linecolor=purple!60}
\pstGeneralIParabola(O){\p}[180]{-1.5}{1.5}
\pstGeneralIParabolaFocusNode(O){\p}[180]{E}
\pstGeneralIParabolaDirectrixLine(O){\p}[180]{E1}{E2}
\psset{linecolor=yellow!60}
\pstGeneralIParabola(O){\p}[225]{-1.5}{1.5}
\pstGeneralIParabolaFocusNode(O){\p}[225]{F}
\pstGeneralIParabolaDirectrixLine(O){\p}[225]{F1}{F2}
\psset{linecolor=black!60}
\pstGeneralIParabola(O){\p}[270]{-1.5}{1.5}
\pstGeneralIParabolaFocusNode(O){\p}[270]{G}
\pstGeneralIParabolaDirectrixLine(O){\p}[270]{G1}{G2}
\psset{linecolor=brown!60}
\pstGeneralIParabola(O){\p}[315]{-1.5}{1.5}
\pstGeneralIParabolaFocusNode(O){\p}[315]{H}
\pstGeneralIParabolaDirectrixLine(O){\p}[315]{H1}{H2}
\end{pspicture}
\end{LTXexample}

The macro \Lcs{pstGeneralIParabolaLineInter} is used to find the intersections $C$ and $D$ of the parabola and the given line $AB$.

\begin{BDef}
\Lcs{pstGeneralIParabolaLineInter}\OptArgs\Largr{O}\Largb{$p$}\OptArg{$\theta$}\Largb{A}\Largb{B}\Largb{C}\Largb{D}
\end{BDef}

When $\theta=0$, the intersections with any kind of lines:

\begin{LTXexample}[width=6cm,pos=l]
\begin{pspicture}[showgrid=true](1,-2)(5,2)
\psset{dotscale=0.5}\psset{PointSymbol=*}\footnotesize
\def\p{0.4}
\pstGeonode[PosAngle=180,PointNameSep=0.2](2,0){O}
\pstGeneralIParabola[linecolor=red!40](O){\p}[0]{-1.5}{1.5}
\pstLine[linestyle=dashed,linecolor=gray!40]{3,-2}{3,2}
\pstGeneralIParabolaLineInter[linecolor=red!40,PosAngle={40,150}](O){\p}[0]{3,-2}{3,2}{A}{B}
\pstLine[linestyle=dashed,linecolor=gray!40]{2,-2}{4,2}
\pstGeneralIParabolaLineInter[linecolor=red!40,PosAngle={100,210}](O){\p}[0]{2,-2}{4,2}{C}{D}
\pstLine[linestyle=dashed,linecolor=gray!40]{1.5,0.5}{4.5,0.5}
\pstGeneralIParabolaLineInter[linecolor=red!40,PosAngle={120,-90}](O){\p}[0]{1.5,0.5}{4.5,0.5}{E}{F}
\end{pspicture}
\end{LTXexample}

When $\theta=50$, the intersections with any kind of lines:

\begin{LTXexample}[width=6cm,pos=l]
\begin{pspicture}[showgrid=true](1,-1)(5,4)
\psset{dotscale=0.5}\psset{PointSymbol=*}\footnotesize
\def\p{0.4}
\pstGeonode[PosAngle=-70,PointNameSep=0.2](2,0){O}
\pstGeneralIParabola[linecolor=red!40](O){\p}[50]{-1.5}{1.5}
\pstGeneralIParabolaFocusNode[linecolor=red!40,PosAngle=80](O){\p}[50]{F}
\pstLineAB[linestyle=dashed,linecolor=black!25,nodesepA=-0.2,nodesepB=-2.5]{O}{F}
\pstLine[linestyle=dashed,linecolor=gray!40,nodesep=-0.8]{3,-1}{3,3}
\pstGeneralIParabolaLineInter[linecolor=red!40,PosAngle={-60,40}](O){\p}[50]{3,-1}{3,3}{A}{B}
\pstLine[linestyle=dashed,linecolor=gray!40,nodesep=0.0]{2,3}{4,0}
\pstGeneralIParabolaLineInter[linecolor=red!40,PosAngle={-10,170}](O){\p}[50]{2,3}{4,0}{C}{D}
% a line with gradient k=\tan50 parallel to OF
\pstLineAS[linestyle=dashed,linecolor=gray!40,nodesep=-0.8,PointName=none,PointSymbol=none](2,1){50 tan}{X}
\pstGeneralIParabolaLineInter[linecolor=red!40,PosAngle={180,-90}](O){\p}[50]{2,1}{X}{E}{G}
\end{pspicture}
\end{LTXexample}

When $\theta=90$, the intersections with any kind of lines:

\begin{LTXexample}[width=6cm,pos=l]
\begin{pspicture}[showgrid=true](0,-1)(4,4)
\psset{dotscale=0.5}\psset{PointSymbol=*}\footnotesize
\def\p{0.4}
\pstGeonode[PosAngle=-90,PointNameSep=0.2](2,0){O}
\pstGeneralIParabola[linecolor=red!40](O){\p}[90]{-1.5}{1.5}
\pstLine[linestyle=dashed,linecolor=gray!40,nodesep=-0.5]{1,0}{1,2}
\pstGeneralIParabolaLineInter[linecolor=red!40,PosAngle={180,-90}](O){\p}[90]{1,0}{1,2}{A}{B}
\pstLine[linestyle=dashed,linecolor=gray!40,nodesep=-0.5]{1,0}{3,1}
\pstGeneralIParabolaLineInter[linecolor=red!40,PosAngle={-60,-90}](O){\p}[90]{1,0}{3,1}{C}{D}
\pstLine[linestyle=dashed,linecolor=gray!40,nodesep=-0.5]{0.8,2}{3,2}
\pstGeneralIParabolaLineInter[linecolor=red!40,PosAngle={120,60}](O){\p}[90]{0.8,2}{3,2}{E}{G}
\end{pspicture}
\end{LTXexample}

The macro \Lcs{pstGeneralIParabolaPolarNode} is used to find the polar point $T$ of chord $AB$ on Parabola $P$.

\begin{BDef}
\Lcs{pstGeneralIParabolaPolarNode}\OptArgs\Largr{O}\Largb{$p$}\OptArg{$\theta$}\Largb{A}\Largb{B}\Largb{T}\\
\Lcs{pstGeneralIParabolaPolarNode}\OptArgs\Largr{O}\Largb{$p$}\OptArg{$\theta$}\Largr{F}\Largb{A}\Largb{B}\Largb{T}\\
\Lcs{pstGeneralIParabolaPolarNode}\OptArgs\Largr{O}\Largb{$p$}\OptArg{$\theta$}\Largr{F}\OptArg{$L_x$}\OptArg{$L_y$}\Largb{A}\Largb{B}\Largb{T}
\end{BDef}

\begin{LTXexample}[width=6cm,pos=l]
\begin{pspicture}[showgrid=true](1,-1)(5,4)
\psset{dotscale=0.5}\psset{PointSymbol=*}\footnotesize
\def\p{0.4}
\pstGeonode[PosAngle=240,PointNameSep=0.4](2,0){O}
\pstGeneralIParabola[linecolor=red!40](O){\p}[50]{-1.5}{1.5}
\pstGeneralIParabolaFocusNode[linecolor=red!40,PosAngle=80](O){\p}[50]{F}
\pstLine[linestyle=dashed,linecolor=gray!40,nodesep=0.0]{2,3}{4,0}
\pstGeneralIParabolaLineInter[linecolor=red!40,PosAngle={-10,170}](O){\p}[50]{2,3}{4,0}{A}{B}
\pstGeneralIParabolaPolarNode[linecolor=red!40,PosAngle=-90](O){\p}[50](F){A}{B}{T}
\end{pspicture}
\end{LTXexample}

We also use the theorem \ref{ParabolaPolarPointTheorem} to find the polar point $T$ of chord $AB$.
If you don't know the focus $F$, or the directrix line, we will find them automated, otherwise you can pass them to this macro.

The macro \Lcs{pstGeneralIParabolaTangentNode} is used to find the two nodes $A$ and $B$ on the Parabola through the point $T$.

\begin{BDef}
\Lcs{pstGeneralIParabolaTangentNode}\OptArgs\Largr{O}\Largb{$p$}\OptArg{$\theta$}\Largb{T}\Largb{A}\Largb{B}
\end{BDef}

\begin{LTXexample}[width=6cm,pos=l]
\begin{pspicture}[showgrid=true](1,-1)(5,4)
\psset{dotscale=0.5}\psset{PointSymbol=*}\footnotesize
\def\p{0.4}
\pstGeonode[PosAngle=-90,PointNameSep=0.2](2,0){O}
\pstGeneralIParabola[linecolor=red!40](O){\p}[60]{-1.5}{1.5}
\pstGeonode[PosAngle=-90](1,-1){R}(2,-1){T}(2.5,-1){S}
\pstGeneralIParabolaTangentNode[linecolor=red!40,PosAngle={90,220},PointName={R_1,R_2}](O){\p}[60]{R}{R1}{R2}
\pstGeneralIParabolaTangentNode[linecolor=red!40,PosAngle={160,60},PointName={T_1,T_2}](O){\p}[60]{T}{T1}{T2}
\pstGeneralIParabolaTangentNode[linecolor=red!40,PosAngle={-60,40},PointName={S_1,S_2}](O){\p}[60]{S}{S1}{S2}
\end{pspicture}
\end{LTXexample}

\subsection{Standard Hyperbola}
The Standard Hyperbola $H$ with coordinate translation is defined by center $O$,
the half of the real axis $a$, the half of the imaginary axis $b$.
The equation can be written as:
\begin{equation}\label{FunctionOfStandardHyperbola}
\dfrac{(x-x_0)^2}{a^2}-\dfrac{(y-y_0)^2}{b^2}=1
\end{equation}
and the parametric function can be written as:
\begin{equation}\label{ParametricFunctionOfStandardHyperbola}
\left\{\begin{array}{l}
x=a\sec\alpha+x_0\\
y=b\tan\alpha+y_0
\end{array}\right.
\end{equation}

The macro \Lcs{pstHyperbola} is used to draw a Standard Hyperbola with Center $O$,
the half of the real axis $a$, the half of the imaginary axis $b$.
The parameter \texttt{angleX} is used to truncate the width of the figure,
it should be setup from 0 to 90.

\begin{BDef}
\Lcs{pstHyperbola}\OptArgs\Largr{O}\Largr{$a,\,b$}\OptArg{angleX}
\end{BDef}

The macro \Lcs{pstHyperbolaNode} is used to draw a node whose parameter is the given value $t$ on Hyperbola,
please refer to equation (\ref{ParametricFunctionOfStandardHyperbola}).
The macro \Lcs{pstHyperbolaAbsNode} is used to draw the nodes whose abscissa are the given value $x_1$ on Hyperbola.
The macro \Lcs{pstHyperbolaOrdNode} is used to draw the nodes whose ordinate are the given value $y_1$ on Hyperbola.

Note that \Lcs{pstHyperbolaAbsNode} and \Lcs{pstHyperbolaOrdNode} will create two nodes $A$ and $B$ at most time.

\begin{BDef}
\Lcs{pstHyperbolaNode}\OptArgs\Largr{O}\Largr{$a,\,b$}\Largb{$t$}\Largb{A}\\
\Lcs{pstHyperbolaAbsNode}\OptArgs\Largr{O}\Largr{$a,\,b$}\Largb{$x_1$}\Largb{A}\Largb{B}\\
\Lcs{pstHyperbolaOrdNode}\OptArgs\Largr{O}\Largr{$a,\,b$}\Largb{$y_1$}\Largb{A}\Largb{B}
\end{BDef}

\begin{LTXexample}[width=6cm,pos=l]
\begin{pspicture}[showgrid=true](-2,-2)(4,4)
\psset{dotscale=0.5}\psset{PointSymbol=*}\footnotesize
\def\a{0.5}\def\b{0.3}
\pstGeonode[PosAngle=-90,PointNameSep=0.2](1,1){O}
\pstHyperbola[linecolor=blue!40](O)(\a,\b)[80]
\pstHyperbolaNode[linecolor=blue!40,PosAngle=90](O)(\a,\b){80}{A}
\pstHyperbolaAbsNode[linecolor=blue!40,PointName={X_1,X_2},PosAngle=0](O)(\a,\b){0}{X1}{X2}
\pstHyperbolaOrdNode[linecolor=blue!40,PointName={Y_1,Y_2},PosAngle=-90](O)(\a,\b){0}{Y1}{Y2}
\pstHyperbola[linecolor=red!40](O)(\b,\a)[78]
\pstHyperbolaNode[linecolor=red!40](O)(\b,\a){-75}{B}
\pstHyperbolaAbsNode[linecolor=red!40,PointName={X_3,X_4},PosAngle=0](O)(\b,\a){0}{X3}{X4}
\pstHyperbolaOrdNode[linecolor=red!40,PointName={Y_3,Y_4},PosAngle=-90](O)(\b,\a){0}{Y3}{Y4}
\end{pspicture}
\end{LTXexample}

The macro \Lcs{pstHyperbolaFocusNode} is used to find the focus nodes of the Hyperbola,
and the macro \Lcs{pstHyperbolaDirectrixLine} is used to find the directrix lines of the Hyperbola.

\begin{BDef}
\Lcs{pstHyperbolaFocusNode}\OptArgs\Largr{O}\Largr{$a,\,b$}\Largb{$F_1$}\Largb{$F_2$}\\
\Lcs{pstHyperbolaDirectrixLine}\OptArgs\Largr{O}\Largr{$a,\,b$}\Largb{$L_x$}\Largb{$L_y$}\Largb{$R_x$}\Largb{$R_y$}
\end{BDef}

Note that you can use \Lcs{pstLineAS} to draw the asymptote line of the hyperbola by passing the slope gradient $k=\pm\dfrac{b}{a}$;
or you can use the macro \Lcs{pstHyperbolaAsymptoteLine} to get them, this macro only create one node on each asymptote line,
as the other one is the center of the hyperbola.

\begin{BDef}
\Lcs{pstHyperbolaAsymptoteLine}\OptArgs\Largr{O}\Largr{$a,\,b$}\Largb{$L_1$}\Largb{$L_2$}
\end{BDef}

\begin{LTXexample}[width=6cm,pos=l]
\begin{pspicture}[showgrid=true](-2,-2)(4,4)
\psset{dotscale=0.5}\psset{PointSymbol=*}\footnotesize
\def\a{0.5}\def\b{0.3}
\pstGeonode[PosAngle=-90,PointNameSep=0.2](1,1){O}
\pstHyperbola[linecolor=blue!40](O)(\a,\b)[80]
\pstHyperbolaNode[linecolor=blue!40](O)(\a,\b){80}{A}
\pstLineAS[PointName=S_1,PosAngle=90,nodesepA=-3,nodesepB=-1.5,linecolor=blue!20]{O}{\b\space \a\space div}{S1}
\pstLineAS[PointName=S_2,PosAngle=-90,nodesepA=-3,nodesepB=-1.5,linecolor=blue!20]{O}{\b\space \a\space div neg}{S2}
\pstHyperbolaFocusNode[linecolor=blue!40,PointName={F_1,F_2},PosAngle={180,0}](O)(\a,\b){F1}{F2}
\pstHyperbolaDirectrixLine[linecolor=blue!40,nodesepA=-2,nodesepB=-1,PointName={1,2,3,4},PosAngle=90,PointNameSep=0.2](O)(\a,\b){Lx}{Ly}{Rx}{Ry}
\pstHyperbola[linecolor=red!40](O)(\b,\a)[78]
\pstHyperbolaFocusNode[linecolor=red!40,PointName={H_1,H_2},PosAngle={180,0}](O)(\b,\a){H1}{H2}
\pstHyperbolaDirectrixLine[linecolor=red!40,nodesepA=-2,nodesepB=-1,PointName={5,6,7,8},PosAngle=90,PointNameSep=0.2](O)(\b,\a){Mx}{My}{Nx}{Ny}
\pstHyperbolaAsymptoteLine[linecolor=red!40,nodesepA=-2,nodesepB=-1,PointName={T_1,T_2},PosAngle=90,PointNameSep=0.2](O)(\b,\a){T1}{T2}
\end{pspicture}
\end{LTXexample}

The macro \Lcs{pstHyperbolaLineInter} is used to find the intersections $C$ and $D$ of the hyperbola and the given line $AB$.

\begin{BDef}
\Lcs{pstHyperbolaLineInter}\OptArgs\Largr{O}\Largr{$a,\,b$}\Largb{$A$}\Largb{$B$}\Largb{$C$}\Largb{$D$}
\end{BDef}

In the following example, the Line $CX$ and $CY$ are parallel to the asymptote of the hyperbola.

\begin{LTXexample}[width=6cm,pos=l]
\begin{pspicture}[showgrid=true](-2,-1)(4,3)
\psset{dotscale=0.5}\psset{PointSymbol=*}\footnotesize
\def\a{0.5}\def\b{0.3}\psset{PointNameSep=0.3}
\pstGeonode[PosAngle={-90,90},PointNameSep=0.2](1,1){O}(1,1.5){C}
\pstHyperbola[linecolor=blue!40](O)(\a,\b)[80]
\pstLine[linestyle=dashed,linecolor=gray!40]{2,-1}{2,3}
\pstHyperbolaLineInter[linecolor=blue!40,PosAngle={210,-40}](O)(\a,\b){2,-1}{2,3}{I}{J}
\pstLineAS[linestyle=dashed,linecolor=gray!60,nodesep=-2,PosAngle=150]{1,1.5}{\b\space \a\space div}{X}
\pstLineAS[linestyle=dashed,linecolor=gray!60,nodesep=-2,PosAngle=-10]{1,1.5}{\b\space \a\space div neg}{Y}
\pstLineAS[linestyle=dashed,linecolor=gray!60,nodesep=-2,PosAngle=150]{1,1.5}{0.2}{Z}
\pstHyperbolaLineInter[linecolor=blue!40,PosAngle={-10,-90}](O)(\a,\b){1,1.5}{X}{P}{Q}
\pstHyperbolaLineInter[linecolor=blue!40,PosAngle={90,-30}](O)(\a,\b){1,1.5}{Y}{M}{N}
\pstHyperbolaLineInter[linecolor=blue!40,PosAngle={90,90}](O)(\a,\b){1,1.5}{Z}{D}{E}
\end{pspicture}
\end{LTXexample}

The macro \Lcs{pstHyperbolaPolarNode} is used to find the polar point $T$ of chord $AB$ on the hyperbola.

\begin{BDef}
\Lcs{pstHyperbolaPolarNode}\OptArgs\Largr{O}\Largr{$a,\,b$}\Largb{$A$}\Largb{$B$}\Largb{$T$}
\end{BDef}

We use the following theorem to find the polar point $T$ of chord $AB$:
\begin{theorem}\label{HyperbolaPolarPointTheorem}
Let $P$, $Q$ are vertex points of the hyperbola, for any chord $AB$ of hyperbola, suppose $PA$ and $BQ$ intersect at $E$,
$PB$ and $AQ$ intersect at $F$, then the middle point $T$ of $EF$ is the polar point of chord $AB$.
\end{theorem}

\begin{LTXexample}[width=6cm,pos=l]
\begin{pspicture}[showgrid=true](-2,-1)(4,3)
\psset{dotscale=0.5}\psset{PointSymbol=*}\footnotesize
\def\a{0.5}\def\b{0.3}\psset{PointNameSep=0.3}
\pstGeonode[PosAngle=90,PointNameSep=0.2](1,1){O}
\pstHyperbola[linecolor=blue!40](O)(\a,\b)[80]
\pstHyperbolaNode[linecolor=blue!40,PosAngle=80](O)(\a,\b){50}{A}
\pstHyperbolaNode[linecolor=blue!40,PosAngle=-100](O)(\a,\b){-70}{B}
\pstHyperbolaPolarNode[linecolor=red!40,PosAngle=-100](O)(\a,\b){A}{B}{T}
\pstLine[linestyle=dashed,linecolor=gray!40,nodesep=-1]{A}{B}
\end{pspicture}
\end{LTXexample}

The macro \Lcs{pstHyperbolaTangentNode} is used to find the tangent point $A$ and $B$ of point $T$ outside of the hyperbola.

\begin{BDef}
\Lcs{pstHyperbolaTangentNode}\OptArgs\Largr{O}\Largr{$a,\,b$}\Largb{$T$}\Largb{$A$}\Largb{$B$}
\end{BDef}

We use the following theorem to find the tangent points $A$ and $B$ of $T$:
\begin{theorem}\label{HyperbolaTangentPointTheorem}
Let $T$ is a point out of the hyperbola, for any two chords $TPQ$ and $TRS$ of the hyperbola, suppose $PR$ and $QS$ intersect at $X$,
$RQ$ and $PS$ intersect at $Y$, then the intersection points $A$ and $B$ of $XY$ and the hyperbola are the tangent points from $T$.
\end{theorem}

\begin{LTXexample}[width=6cm,pos=l]
\begin{pspicture}[showgrid=true](-2,-1)(4,3)
\psset{dotscale=0.5}\psset{PointSymbol=*}\footnotesize
\def\a{0.5}\def\b{0.3}\psset{PointNameSep=0.3}
\pstGeonode[PosAngle=90,PointNameSep=0.2](1,1){O}
\pstHyperbola[linecolor=blue!40](O)(\a,\b)[80]
\pstGeonode[PosAngle=-90](1.2,0.8){T}
\pstHyperbolaTangentNode[linecolor=red!40,PosAngle={90,90},nodesep=-0.5](O)(\a,\b){T}{A}{B}
\end{pspicture}
\end{LTXexample}

\subsection{Standard Inversion Hyperbola}
The Standard Inversion Hyperbola $H$ with coordinate translation is defined by center $O$,
the half of the real axis $a$, the half of the imaginary axis $b$.
The equation can be written as:
\begin{equation}\label{FunctionOfStandardInversionHyperbola}
\dfrac{(y-y_0)^2}{a^2}-\dfrac{(x-x_0)^2}{b^2}=1
\end{equation}
and the parametric function can be written as:
\begin{equation}\label{ParametricFunctionOfStandardInversionHyperbola}
\left\{\begin{array}{l}
x=b\tan\alpha+x_0\\
y=a\sec\alpha+y_0
\end{array}\right.
\end{equation}

The macro \Lcs{pstIHyperbola} is used to draw a Standard Inversion Hyperbola with Center $O$,
the half of the real axis $a$, the half of the imaginary axis $b$.
The parameter \texttt{angleY} is used to truncate the height of the figure,
it should be setup from 0 to 90.

\begin{BDef}
\Lcs{pstIHyperbola}\OptArgs\Largr{O}\Largr{$a,\,b$}\OptArg{angleY}
\end{BDef}

The macro \Lcs{pstIHyperbolaNode} is used to draw a node whose parameter is the given value $t$ on Inversion Hyperbola,
please refer to equation (\ref{ParametricFunctionOfStandardInversionHyperbola}).
The macro \Lcs{pstIHyperbolaAbsNode} is used to draw the nodes whose abscissa are the given value $x_1$ on Inversion Hyperbola.
The macro \Lcs{pstIHyperbolaOrdNode} is used to draw the nodes whose ordinate are the given value $y_1$ on Inversion Hyperbola.

Note that \Lcs{pstIHyperbolaAbsNode} and \Lcs{pstIHyperbolaOrdNode} will create two nodes $A$ and $B$ at most time.

\begin{BDef}
\Lcs{pstIHyperbolaNode}\OptArgs\Largr{O}\Largr{$a,\,b$}\Largb{$t$}\Largb{A}\\
\Lcs{pstIHyperbolaAbsNode}\OptArgs\Largr{O}\Largr{$a,\,b$}\Largb{$x_1$}\Largb{A}\Largb{B}\\
\Lcs{pstIHyperbolaOrdNode}\OptArgs\Largr{O}\Largr{$a,\,b$}\Largb{$y_1$}\Largb{A}\Largb{B}
\end{BDef}

\begin{LTXexample}[width=6cm,pos=l]
\begin{pspicture}[showgrid=true](-2,-2)(4,4)
\psset{dotscale=0.5}\psset{PointSymbol=*}\footnotesize
\def\a{0.5}\def\b{0.3}
\pstGeonode[PosAngle=-90,PointNameSep=0.2](1,1){O}
\pstIHyperbola[linecolor=blue!40](O)(\a,\b)[80]
\pstIHyperbolaNode[linecolor=blue!40](O)(\a,\b){75}{A}
\pstIHyperbolaAbsNode[linecolor=blue!40,PointName={Y_1,Y_2},PosAngle=0](O)(\a,\b){0}{Y1}{Y2}
\pstIHyperbolaOrdNode[linecolor=red!40,PointName={X_1,X_2},PosAngle=-90](O)(\a,\b){0}{X1}{X2}
\pstIHyperbola[linecolor=red!40](O)(\b,\a)[78]
\pstIHyperbolaNode[linecolor=red!40](O)(\b,\a){-75}{B}
\pstIHyperbolaAbsNode[linecolor=red!40,PointName={Y_3,Y_4},PosAngle=0](O)(\b,\a){0}{Y3}{Y4}
\pstIHyperbolaOrdNode[linecolor=red!40,PointName={X_3,X_4},PosAngle=-90](O)(\b,\a){0}{X3}{X4}
\end{pspicture}
\end{LTXexample}

The macro \Lcs{pstIHyperbolaFocusNode} is used to find the focus nodes of the Inversion Hyperbola,
and the macro \Lcs{pstIHyperbolaDirectrixLine} is used to find the directrix lines of the Inversion Hyperbola.

\begin{BDef}
\Lcs{pstIHyperbolaFocusNode}\OptArgs\Largr{O}\Largr{$a,\,b$}\Largb{$F_1$}\Largb{$F_2$}\\
\Lcs{pstIHyperbolaDirectrixLine}\OptArgs\Largr{O}\Largr{$a,\,b$}\Largb{$L_x$}\Largb{$L_y$}\Largb{$R_x$}\Largb{$R_y$}
\end{BDef}

Note that you can use \Lcs{pstLineAS} to draw the asymptote line of the hyperbola by passing the slope gradient $k=\pm\dfrac{a}{b}$;
or you can use the macro \Lcs{pstIHyperbolaAsymptoteLine} to get them, this macro only create one node on each asymptote line,
as the other one is the center of the hyperbola.

\begin{BDef}
\Lcs{pstHyperbolaAsymptoteLine}\OptArgs\Largr{O}\Largr{$a,\,b$}\Largb{$L_1$}\Largb{$L_2$}
\end{BDef}

\begin{LTXexample}[width=6cm,pos=l]
\begin{pspicture}[showgrid=true](-2,-2)(4,4)
\psset{dotscale=0.5}\psset{PointSymbol=*}\footnotesize
\def\a{0.5}\def\b{0.3}
\pstGeonode[PosAngle=180,PointNameSep=0.2](1,1){O}
\pstIHyperbola[linecolor=blue!40](O)(\a,\b)[80]
\pstIHyperbolaFocusNode[linecolor=blue!40,PointName={F_1,F_2},PosAngle={-90,90}](O)(\a,\b){F1}{F2}
\pstIHyperbolaDirectrixLine[linecolor=blue!40,nodesepA=-2,nodesepB=-1,PointName={1,2,3,4},PosAngle=180,PointNameSep=0.2](O)(\a,\b){Lx}{Ly}{Rx}{Ry}
\pstLineAS[PointName=S_1,PosAngle=90,nodesepA=-3,nodesepB=-1.5,linecolor=blue!20]{O}{\a\space \b\space div}{S1}
\pstLineAS[PointName=S_2,PosAngle=-90,nodesepA=-3,nodesepB=-1.5,linecolor=blue!20]{O}{\a\space \b\space div neg}{S2}
\pstIHyperbola[linecolor=red!40](O)(\b,\a)[78]
\pstIHyperbolaFocusNode[linecolor=red!40,PointName={H_1,H_2},PosAngle={-90,90}](O)(\b,\a){H1}{H2}
\pstIHyperbolaDirectrixLine[linecolor=red!40,nodesepA=-2,nodesepB=-1,PointName={5,6,7,8},PosAngle=0,PointNameSep=0.2](O)(\b,\a){Mx}{My}{Nx}{Ny}
\pstIHyperbolaAsymptoteLine[linecolor=red!40,nodesepA=-2,nodesepB=-1,PointName={T_1,T_2},PosAngle=90,PointNameSep=0.2](O)(\b,\a){T1}{T2}
\end{pspicture}
\end{LTXexample}

The macro \Lcs{pstIHyperbolaLineInter} is used to find the intersections $C$ and $D$ of the hyperbola and the given line $AB$.

\begin{BDef}
\Lcs{pstIHyperbolaLineInter}\OptArgs\Largr{O}\Largr{$a,\,b$}\Largb{$A$}\Largb{$B$}\Largb{$C$}\Largb{$D$}
\end{BDef}

In the following example, the Line $CX$ and $CY$ are parallel to the asymptote of the hyperbola.

\begin{LTXexample}[width=6cm,pos=l]
\begin{pspicture}[showgrid=true](-2,-2)(4,4)
\psset{dotscale=0.5}\psset{PointSymbol=*}\footnotesize
\def\a{0.5}\def\b{0.3}\psset{PointNameSep=0.35}
\pstGeonode[PosAngle={0,180}](1,1){O}(0,1){C}
\pstIHyperbola[linecolor=blue!40](O)(\a,\b)[82]
\pstLine[linestyle=dashed,linecolor=gray!40]{2,-2}{2,4}
\pstIHyperbolaLineInter[linecolor=blue!40,PosAngle={0,-30}](O)(\a,\b){2,-2}{2,4}{I}{J}
\pstLineAS[linestyle=dashed,linecolor=gray!60,nodesep=-2,PosAngle=150]{0,1}{\a\space \b\space div}{X}
\pstLineAS[linestyle=dashed,linecolor=gray!60,nodesep=-2,PosAngle=150]{0,1}{\a\space \b\space div neg}{Y}
\pstLineAS[linestyle=dashed,linecolor=gray!60,nodesepA=-4,PosAngle=210]{0,1}{-3.5}{Z}
\pstIHyperbolaLineInter[linecolor=blue!40,PosAngle={180,-100}](O)(\a,\b){0,1}{X}{P}{Q}
\pstIHyperbolaLineInter[linecolor=blue!40,PosAngle={0,180}](O)(\a,\b){0,1}{Y}{M}{N}
\pstIHyperbolaLineInter[linecolor=blue!40,PosAngle={190,-100}](O)(\a,\b){0,1}{Z}{D}{E}
\end{pspicture}
\end{LTXexample}

The macro \Lcs{pstIHyperbolaPolarNode} is used to find the polar point $T$ of chord $AB$ on the hyperbola.

\begin{BDef}
\Lcs{pstIHyperbolaPolarNode}\OptArgs\Largr{O}\Largr{$a,\,b$}\Largb{$A$}\Largb{$B$}\Largb{$T$}
\end{BDef}

We also use the theorem \ref{HyperbolaPolarPointTheorem} to find the polar point $T$ of chord $AB$:

\begin{LTXexample}[width=6cm,pos=l]
\begin{pspicture}[showgrid=true](-1,-1)(3,3)
\psset{dotscale=0.5}\psset{PointSymbol=*}\footnotesize
\def\a{0.5}\def\b{0.3}\psset{PointNameSep=0.3}
\pstGeonode[PosAngle=0,PointNameSep=0.2](1,1){O}
\pstIHyperbola[linecolor=blue!40](O)(\a,\b)[76]
\pstIHyperbolaNode[linecolor=blue!40,PosAngle=80](O)(\a,\b){50}{A}
\pstIHyperbolaNode[linecolor=blue!40,PosAngle=-100](O)(\a,\b){-70}{B}
\pstIHyperbolaPolarNode[linecolor=red!40,PosAngle=180](O)(\a,\b){A}{B}{T}
\pstLine[linestyle=dashed,linecolor=gray!40,nodesep=-1]{A}{B}
\end{pspicture}
\end{LTXexample}

The macro \Lcs{pstIHyperbolaTangentNode} is used to find the tangent point $A$ and $B$ of point $T$ outside of the hyperbola.

\begin{BDef}
\Lcs{pstIHyperbolaTangentNode}\OptArgs\Largr{O}\Largr{$a,\,b$}\Largb{$T$}\Largb{$A$}\Largb{$B$}
\end{BDef}

We also use the theorem \ref{HyperbolaTangentPointTheorem} to find the tangent points $A$ and $B$ of $T$.

\begin{LTXexample}[width=6cm,pos=l]
\begin{pspicture}[showgrid=true](-2,-1)(4,3)
\psset{dotscale=0.5}\psset{PointSymbol=*}\footnotesize
\def\a{0.5}\def\b{0.3}\psset{PointNameSep=0.3}
\pstGeonode[PosAngle=180](1,1){O}
\pstIHyperbola[linecolor=blue!40](O)(\a,\b)[78]
\pstGeonode[PosAngle=0](1.2,0.8){T}
\pstIHyperbolaTangentNode[linecolor=red!40,PosAngle={80,-90},nodesep=-0.5](O)(\a,\b){T}{A}{B}
\end{pspicture}
\end{LTXexample}

\subsection{General Hyperbola}
The General Hyperbola $H$ with coordinate translation and rotation is defined by center $O$,
the half of the real axis $a$, the half of the imaginary axis $b$,
and the rotation angle $\theta$ of the principal axis.
The equation can be got from the parametric function of the Standard Hyperbola equation (\ref{ParametricFunctionOfStandardHyperbola}),
using the rotation transform formula (\ref{RotationTransformFormula}), then we have
\begin{equation}
\left\{\begin{array}{l}
x'=(a\sec\alpha+x_0)\cos\theta-(b\tan\alpha+y_0)\sin\theta=x_0'+a\sec\alpha\cos\theta-b\tan\alpha\sin\theta\\
y'=(a\sec\alpha+x_0)\sin\theta+(b\tan\alpha+y_0)\cos\theta=y_0'+a\sec\alpha\sin\theta+b\tan\alpha\cos\theta
\end{array}\right.
\end{equation}
where the $x_0'$ and $y_0'$ are the coordinate of the given center $O$ after rotation.
So we get the parametric function of the General Hyperbola with coordinate translation and rotation as following:
\begin{equation}\label{ParametricFunctionOfGeneralHyperbola}
\left\{\begin{array}{l}
x=x_0+a\sec\alpha\cos\theta-b\tan\alpha\sin\theta\\
y=y_0+a\sec\alpha\sin\theta+b\tan\alpha\cos\theta
\end{array}\right.
\end{equation}

The macro \Lcs{pstGeneralHyperbola} is used to draw a General Hyperbola with Center $O$,
the half of the real axis $a$, the half of the imaginary axis $b$,
and the rotation angle $\theta$ of the symmetrical axis.
The parameter \texttt{angleX} is used to truncate the width of the figure,
it should be setup from 0 to 90.

\begin{BDef}
\Lcs{pstGeneralHyperbola}\OptArgs\Largr{O}\Largr{$a,\,b$}\OptArg{$\theta$}\OptArg{angleX}
\end{BDef}

\begin{LTXexample}[width=6cm,pos=l]
\begin{pspicture}[showgrid=true](-2,-1)(4,3)
\psset{dotscale=0.5}\psset{PointSymbol=*}\footnotesize
\def\a{0.5}\def\b{0.3}\psset{PointNameSep=0.3}
\pstGeonode[PosAngle=-90](1,1){O}
\pstGeneralHyperbola[linecolor=red!20](O)(\a,\b)[0][80]
\pstGeneralHyperbolaNode[linecolor=red!80,PosAngle=5](O)(\a,\b)[0]{0}{A}
\pstGeneralHyperbola[linecolor=blue!40](O)(\a,\b)[40][80]
\pstGeneralHyperbolaNode[linecolor=blue!40,PosAngle=10](O)(\a,\b)[40]{40}{B}
\pstGeneralHyperbola[linecolor=green!60](O)(\a,\b)[90][80]
\pstGeneralHyperbolaNode[linecolor=green!60,PosAngle=-90](O)(\a,\b)[90]{200}{C}
\pstGeneralHyperbola[linecolor=purple!80](O)(\a,\b)[150][80]
\pstGeneralHyperbolaNode[linecolor=purple!80,PosAngle=150](O)(\a,\b)[150]{50}{D}
\end{pspicture}
\end{LTXexample}

\vspace{10pt}

The Macro \Lcs{pstGeneralHyperbolaFle} is used to define a General Hyperbola with Focus $F$, directrix line $l$,
and the eccentricity $e$, where $e>1$. It just calculate the center $O$, real radius $a$, imaginary radius $b$ and the rotation
angle $\theta$ of the real axis, then you can pass them into macro \Lcs{pstGeneralHyperbola} to draw this hyperbola.

\begin{BDef}
\Lcs{pstGeneralHyperbolaFle}\OptArgs\Largb{F}\Largb{A}\Largb{B}\Largb{$e$}\Largb{O}\Largb{Rab}\Largb{$\theta$}
\end{BDef}

The output parameter \texttt{O} is a node name to store the center point, its label and symbol can
be controlled by the options for \PST\ node, such as \Lkeyword{PosAngle}.
The output parameter \texttt{Rab} is a PostScript key to store the pair of real radius and imaginary radius,
it just use \PST\ node coordinate to store a pair of value, but not a geometrical point.
The output parameter \texttt{$\theta$} is also a PostScript key to store the rotation angle of real axis,
when you pass it to \Lcs{pstGeneralHyperbola}, PostScript will lookup the value of this key in current dictionary.

\begin{LTXexample}[width=6cm,pos=l]
\begin{pspicture}[showgrid=true](-2,-2)(2,2)
\psset{unit=1.0cm}\psset{dotscale=0.5}\footnotesize
\psset{CodeFig=true,CodeFigColor=gray!50}\psset{PointSymbol=*}
\pstGeonode[PosAngle=-60](1,1){F_1}
\pstGeonode[PosAngle=-60](-1,-1){F_2}
\pstGeonode[PosAngle=-60](1,-1){B}
\pstGeonode[PosAngle=-60](-1,1){A}
\pstGeneralHyperbolaFle[PosAngle=-60]{F_1}{A}{B}{2.4}{O_1}{R_1}{RealAxisRotAngle1}
\pstGeneralHyperbola[linecolor=red!60](O_1)(R_1)[RealAxisRotAngle1][65]
\pstGeneralHyperbolaFle[PosAngle=-60]{F_2}{A}{B}{2.4}{O_2}{R_2}{RealAxisRotAngle2}
\pstGeneralHyperbola[linecolor=blue!60](O_2)(R_2)[RealAxisRotAngle2][65]
\end{pspicture}
\end{LTXexample}

\vspace{10pt}

The Macro \Lcs{pstGeneralHyperbolaCoef} is used to define a General Hyperbola by the quadratic curve equation $ax^2+bxy+cy^2+dx+ey+f=0$,
it just calculate the center $O$, real radius $a$ and imaginary radius $b$ and the rotation angle $\theta$ of the real axis,
then you can pass them into macro \Lcs{pstGeneralHyperbola} to draw this hyperbola.
The package \texttt{pst-func} provides macro \Lcs{psplotImp} to draw an implicit defined functions too,
but it can't tell you the geometrical elements like as center or radii, and it will take more time to
calculate the function value point by point.

\begin{BDef}
\Lcs{pstGeneralHyperbolaCoef}\OptArgs\Largb{a,b,c,d,e,f}\Largb{O}\Largb{Rab}\Largb{$\theta$}
\end{BDef}

The output parameter \texttt{O}, the output parameter \texttt{Rab} and the output parameter \texttt{$\theta$}~
are same with \Lcs{pstGeneralHyperbolaFle}. They are set to zero if the coeffients are invalid to construct a hyperbola.

In the following example, we use \Lcs{psplotImp} to draw the same hyperbolas, just to check the results
given by macros \Lcs{pstGeneralHyperbolaCoef} are correct.

\begin{LTXexample}[width=6cm,pos=l]
\begin{pspicture}[showgrid=true](-4,-4)(4,4)
\psset{unit=0.40cm}\footnotesize\psset{dotscale=0.5}
\psset{CodeFig=true}\psset{PointSymbol=*}
%x^2+3xy-2y^2+10x-5y+6=0
\psplotImp[linecolor=green!30](-10,-10)(10,10){ 1 x dup mul mul 3 x mul y mul add -2 y dup mul mul add 10 x mul add -5 y mul add 6 add }
\pstGeneralHyperbolaCoef[PosAngle=-30,CodeFigColor=red]{1,3,-2,10,-5,6}{O_1}{R_1}{RealAxisRotAngle1}
\pstGeneralHyperbola[linecolor=red](O_1)(R_1)[RealAxisRotAngle1][60]
%x^2-3xy+y^2+10x-10y+21=0
\psplotImp[linecolor=blue!30](-10,-10)(10,10){ 1 x dup mul mul -3 x mul y mul add 1 y dup mul mul add 10 x mul add -10 y mul add 21 add }
\pstGeneralHyperbolaCoef[PosAngle=-10,CodeFigColor=black]{1,-3,1,10,-10,21}{O_2}{R_2}{RealAxisRotAngle2}
\pstGeneralHyperbola[linecolor=black](O_2)(R_2)[RealAxisRotAngle2][60]
\end{pspicture}
\end{LTXexample}

\vspace{10pt}

The Macro \Lcs{pstGeneralHyperbolaABCDE} is used to define a General Hyperbola by the given five points $A,B,C,D,E$,
it just calculate the center $O$, real radius $a$ and imaginary radius $b$ and the rotation angle $\theta$ of the real axis,
then you can pass them into macro \Lcs{pstGeneralHyperbola} to draw this hyperbola.

\begin{BDef}
\Lcs{pstGeneralHyperbolaABCDE}\OptArgs\Largb{A}\Largb{B}\Largb{C}\Largb{D}\Largb{E}\Largb{O}\Largb{Rab}\Largb{$\theta$}
\end{BDef}

The output parameter \texttt{O}, the output parameter \texttt{Rab} and the output parameter \texttt{$\theta$}~
are same with \Lcs{pstGeneralHyperbolaFle}. They are set to zero if the points are invalid to construct a hyperbola.

\begin{LTXexample}[width=6cm,pos=l]
\begin{pspicture}[showgrid=true](-1,-3)(4,2)
\psset{unit=0.5cm}\footnotesize\psset{PointSymbol=*}
\psset{CodeFig=true,CodeFigColor=gray!50}
\pstGeonode[PosAngle=180](0,0){A}
\pstGeonode[PosAngle=-90](2,-1){B}
\pstGeonode[PosAngle=-90](3,-3){C}
\pstGeonode[PosAngle=-90](4,0){D}
\pstGeonode[PosAngle=0](5,2){E}
\pstGeneralHyperbolaABCDE[PosAngle=0]{A}{B}{C}{D}{E}{O}{R}{RealAxisRotAngle}
\pstGeneralHyperbola[linecolor=red!60](O)(R)[RealAxisRotAngle][80]
\end{pspicture}
\end{LTXexample}

\vspace{10pt}

The macro \Lcs{pstGeneralHyperbolaNode} is used to draw a node whose parameter is the given value $t$ on General Hyperbola,
please refer to equation (\ref{ParametricFunctionOfGeneralHyperbola}).
The macro \Lcs{pstGeneralHyperbolaAbsNode} is used to draw the nodes whose abscissa are the given value $x_1$ on General Hyperbola.
The macro \Lcs{pstGeneralHyperbolaOrdNode} is used to draw the nodes whose ordinate are the given value $y_1$ on General Hyperbola.

Note that \Lcs{pstGeneralHyperbolaAbsNode} and \Lcs{pstGeneralHyperbolaOrdNode} will create two nodes $A$ and $B$ at most time.

\begin{BDef}
\Lcs{pstGeneralHyperbolaNode}\OptArgs\Largr{O}\Largr{$a,\,b$}\OptArg{$\theta$}\Largb{$t$}\Largb{A}\\
\Lcs{pstGeneralHyperbolaAbsNode}\OptArgs\Largr{O}\Largr{$a,\,b$}\OptArg{$\theta$}\Largb{$x_1$}\Largb{A}\Largb{B}\\
\Lcs{pstGeneralHyperbolaOrdNode}\OptArgs\Largr{O}\Largr{$a,\,b$}\OptArg{$\theta$}\Largb{$y_1$}\Largb{A}\Largb{B}
\end{BDef}

\begin{LTXexample}[width=6cm,pos=l]
\begin{pspicture}[showgrid=true](-2,-1)(4,3)
\psset{dotscale=0.5}\psset{PointSymbol=*}\footnotesize
\def\a{0.5}\def\b{0.3}\psset{PointNameSep=0.3}
\pstGeonode[PosAngle=-90](1,1){O}
\pstGeneralHyperbola[linecolor=purple!80](O)(\a,\b)[150][80]
\pstGeneralHyperbolaAbsNode[linecolor=purple!80,PosAngle={200,90}](O)(\a,\b)[150]{2}{P}{Q}
\pstGeneralHyperbolaAbsNode[linecolor=purple!80,PosAngle={-90,200}](O)(\a,\b)[150]{0}{X}{Y}
\pstGeneralHyperbolaAbsNode[linecolor=purple!80,PosAngle={40,-40}](O)(\a,\b)[150]{0.59378}{M}{N}
\pstLine[linestyle=dashed,linecolor=gray!40]{0.59378,-1}{0.59378,3}
\pstGeneralHyperbolaOrdNode[linecolor=purple!80,PosAngle={200,90}](O)(\a,\b)[150]{2}{G}{H}
\pstGeneralHyperbolaOrdNode[linecolor=purple!80,PosAngle={-90,200}](O)(\a,\b)[150]{0}{I}{J}
\pstGeneralHyperbolaOrdNode[linecolor=purple!80,PosAngle={90,-90}](O)(\a,\b)[150]{1}{K}{L}
\pstLine[linestyle=dashed,linecolor=gray!80,nodesep=-1.5]{K}{L}
\end{pspicture}
\end{LTXexample}

The macro \Lcs{pstGeneralHyperbolaFocusNode} is used to find the focus nodes of the General Hyperbola,
the macro \Lcs{pstGeneralHyperbolaVertexNode} is used to find the vertex  nodes of the General Hyperbola,
and the macro \Lcs{pstGeneralHyperbolaDirectrixLine} is used to find the directrix lines of the General Hyperbola.

\begin{BDef}
\Lcs{pstGeneralHyperbolaFocusNode}\OptArgs\Largr{O}\Largr{$a,\,b$}\OptArg{$\theta$}\Largb{$F_1$}\Largb{$F_2$}\\
\Lcs{pstGeneralHyperbolaVertexNode}\OptArgs\Largr{O}\Largr{$a,\,b$}\OptArg{$\theta$}\Largb{$V_1$}\Largb{$V_2$}\\
\Lcs{pstGeneralHyperbolaDirectrixLine}\OptArgs\Largr{O}\Largr{$a,\,b$}\OptArg{$\theta$}\Largb{$L_x$}\Largb{$L_y$}\Largb{$R_x$}\Largb{$R_y$}
\end{BDef}

Note that you can use the macro \Lcs{pstGeneralHyperbolaAsymptoteLine} to get the asymptote lines, this macro only create one node on each asymptote line,
as the other one is the center of the hyperbola.

\begin{BDef}
\Lcs{pstGeneralHyperbolaAsymptoteLine}\OptArgs\Largr{O}\Largr{$a,\,b$}\OptArg{$\theta$}\Largb{$L_1$}\Largb{$L_2$}
\end{BDef}

\begin{LTXexample}[width=6cm,pos=l]
\begin{pspicture}[showgrid=true](-2,-2)(4,4)
\psset{dotscale=0.5}\psset{PointSymbol=*}\footnotesize
\def\a{0.5}\def\b{0.3}
\pstGeonode[PosAngle=180,PointNameSep=0.2](1,1){O}
\pstGeneralHyperbola[linecolor=red!40](O)(\a,\b)[0][80]
\pstGeneralHyperbolaFocusNode[linecolor=red!40,PointName={X_1,X_2},PosAngle={180,0}](O)(\a,\b)[0]{X1}{X2}
\pstGeneralHyperbolaDirectrixLine[linecolor=red!40,nodesepA=-2,nodesepB=-1,PointName=none](O)(\a,\b)[0]{Lx}{Ly}{Rx}{Ry}
\pstGeneralHyperbolaAsymptoteLine[linecolor=red!40,nodesepA=-2,nodesepB=-1,PointName=none](O)(\a,\b)[0]{L1}{L2}
\pstGeneralHyperbola[linecolor=blue!40](O)(\a,\b)[40][80]
\pstGeneralHyperbolaFocusNode[linecolor=blue!40,PointName={F_1,F_2},PosAngle={220,40}](O)(\a,\b)[40]{F1}{F2}
\pstGeneralHyperbolaDirectrixLine[linecolor=blue!40,nodesepA=-2,nodesepB=-1,PointName=none](O)(\a,\b)[40]{Dx}{Dy}{Ux}{Uy}
\pstGeneralHyperbolaAsymptoteLine[linecolor=blue!40,nodesepA=-2,nodesepB=-1,PointName=none](O)(\a,\b)[40]{S1}{S2}
\pstGeneralHyperbola[linecolor=brown!40](O)(\a,\b)[90][80]
\pstGeneralHyperbolaFocusNode[linecolor=brown!40,PointName={Y_1,Y_2},PosAngle={-90,90}](O)(\a,\b)[90]{Y1}{Y2}
\pstGeneralHyperbolaDirectrixLine[linecolor=brown!40,nodesepA=-2,nodesepB=-1,PointName=none](O)(\a,\b)[90]{Tx}{Ty}{Sx}{Sy}
\pstGeneralHyperbolaAsymptoteLine[linecolor=brown!40,nodesepA=-2,nodesepB=-1,PointName=none](O)(\a,\b)[90]{T1}{T2}
\end{pspicture}
\end{LTXexample}

The macro \Lcs{pstGeneralHyperbolaLineInter} is used to find the intersections $C$ and $D$ of the general hyperbola and the given line $AB$.

\begin{BDef}
\Lcs{pstGeneralHyperbolaLineInter}\OptArgs\Largr{O}\Largr{$a,\,b$}\OptArg{$\theta$}\Largb{$A$}\Largb{$B$}\Largb{$C$}\Largb{$D$}
\end{BDef}

In the following example, the lines $YY'$ and $ZZ'$ are parallel to the asymptote of the hyperbola,
so there are only one intersection $M$ and $P$ for each line, and the second node $N$ and $Q$ are put at the origin.

\begin{LTXexample}[width=6cm,pos=l]
\begin{pspicture}[showgrid=true](-2,-1)(4,3)
\psset{dotscale=0.5}\psset{PointSymbol=*}\footnotesize
\def\a{0.5}\def\b{0.3}\psset{PointNameSep=0.3}
\pstGeonode[PosAngle=-90](1,1){O}
\pstGeneralHyperbola[linecolor=blue!40](O)(\a,\b)[30][80]
\pstLine[linestyle=dashed,linecolor=gray!40]{0.5,-1}{0.5,3}
\pstGeneralHyperbolaLineInter[linecolor=blue!40,PosAngle={-30,210}](O)(\a,\b)[30]{0.5,-1}{0.5,3}{A}{B}
\pstLine[linestyle=dashed,linecolor=gray!40]{-2,0}{3,3}
\pstGeneralHyperbolaLineInter[linecolor=blue!40,PosAngle={130,-90}](O)(\a,\b)[30]{-2,0}{3,3}{C}{D}
\pstGeonode[PosAngle={0,100}](2,0){Y}(1,1.8){Z}
\pstLineAA[nodesepA=-3,nodesepB=-2,linecolor=gray!40,PointName=none,PointSymbol=none]{O}{\b\space \a\space div 1 atan 30 add}{U}
\pstLineAA[nodesepA=-3,nodesepB=-2,linecolor=red!40,PointName=none,PointSymbol=none]{O}{\b\space \a\space div neg 1 atan 30 add}{V}
\pstLineAA[nodesepA=-3,nodesepB=-2,linecolor=gray!40,PosAngle=-30]{Y}{\b\space \a\space div 1 atan 30 add}{Y'}
\pstLineAA[nodesepA=-3,nodesepB=-2,linecolor=red!40,PosAngle=80]{Z}{\b\space \a\space div neg 1 atan 30 add}{Z'}
\pstGeneralHyperbolaLineInter[linecolor=blue!40,PosAngle={-50,-90}](O)(\a,\b)[30]{Z}{Z'}{M}{N}
\pstGeneralHyperbolaLineInter[linecolor=blue!40,PosAngle={30,210}](O)(\a,\b)[30]{Y}{Y'}{P}{Q}
\end{pspicture}
\end{LTXexample}

The macro \Lcs{pstGeneralHyperbolaPolarNode} is used to find the polar point $T$ of chord $AB$ on the general hyperbola.

\begin{BDef}
\Lcs{pstGeneralHyperbolaPolarNode}\OptArgs\Largr{O}\Largr{$a,\,b$}\OptArg{$\theta$}\Largb{$A$}\Largb{$B$}\Largb{$T$}
\end{BDef}

We also use the theorem \ref{HyperbolaPolarPointTheorem} to find the polar point $T$ of chord $AB$:

\begin{LTXexample}[width=6cm,pos=l]
\begin{pspicture}[showgrid=true](-1,-1)(3,3)
\psset{dotscale=0.5}\psset{PointSymbol=*}\footnotesize
\def\a{0.5}\def\b{0.3}\psset{PointNameSep=0.3}
\pstGeonode[PosAngle=120,PointNameSep=0.2](1,1){O}
\pstGeneralHyperbola[linecolor=blue!40](O)(\a,\b)[40][80]
\pstGeneralHyperbolaNode[linecolor=blue!40,PosAngle=110](O)(\a,\b)[40]{50}{A}
\pstGeneralHyperbolaNode[linecolor=blue!40,PosAngle=-100](O)(\a,\b)[40]{-70}{B}
\pstGeneralHyperbolaPolarNode[linecolor=red!40,PosAngle=-90](O)(\a,\b)[40]{A}{B}{T}
\pstLine[linestyle=dashed,linecolor=gray!40,nodesep=-1]{A}{B}
\end{pspicture}
\end{LTXexample}

The macro \Lcs{pstGeneralHyperbolaTangentNode} is used to find the tangent point $A$ and $B$ of point $T$ outside of the general hyperbola.

\begin{BDef}
\Lcs{pstGeneralHyperbolaTangentNode}\OptArgs\Largr{O}\Largr{$a,\,b$}\OptArg{$\theta$}\Largb{$T$}\Largb{$A$}\Largb{$B$}
\end{BDef}

We also use the theorem \ref{HyperbolaTangentPointTheorem} to find the tangent points $A$ and $B$ of $T$.

\begin{LTXexample}[width=6cm,pos=l]
\begin{pspicture}[showgrid=true](-2,-1)(4,3)
\psset{dotscale=0.5}\psset{PointSymbol=*}\footnotesize
\def\a{0.5}\def\b{0.3}\psset{PointNameSep=0.3}
\pstGeonode[PosAngle=120](1,1){O}
\pstGeneralHyperbola[linecolor=blue!40](O)(\a,\b)[40][80]
\pstGeonode[PosAngle=-40](1.2,0.8){T}
\pstGeneralHyperbolaTangentNode[linecolor=red!40,PosAngle={140,-90},nodesep=-0.5](O)(\a,\b)[40]{T}{A}{B}
\end{pspicture}
\end{LTXexample}

\subsection{General Inversion Hyperbola}
The General Inversion Hyperbola $H$ with coordinate translation and rotation is defined by center $O$,
the half of the real axis $a$, the half of the imaginary axis $b$,
and the rotation angle $\theta$ of the principal axis.
The equation can be got from the parametric function of the Standard Inversion Hyperbola equation (\ref{ParametricFunctionOfStandardInversionHyperbola}),
using the rotation transform formula (\ref{RotationTransformFormula}), then we have
\begin{equation}
\left\{\begin{array}{l}
x'=(b\tan\alpha+x_0)\cos\theta-(a\sec\alpha+y_0)\sin\theta=x_0'+b\tan\alpha\cos\theta-a\sec\alpha\sin\theta\\
y'=(b\tan\alpha+x_0)\sin\theta+(a\sec\alpha+y_0)\cos\theta=y_0'+b\tan\alpha\sin\theta+a\sec\alpha\cos\theta
\end{array}\right.
\end{equation}
where the $x_0'$ and $y_0'$ are the coordinate of the given center $O$ after rotation.
So we get the parametric function of the General Inversion Hyperbola with coordinate translation and rotation as following:
\begin{equation}\label{ParametricFunctionOfGeneralInversionHyperbola}
\left\{\begin{array}{l}
x=x_0+b\tan\alpha\cos\theta-a\sec\alpha\sin\theta\\
y=y_0+b\tan\alpha\sin\theta+a\sec\alpha\cos\theta
\end{array}\right.
\end{equation}

The macro \Lcs{pstGeneralIHyperbola} is used to draw a General Inversion Hyperbola with Center $O$,
the half of the real axis $a$, the half of the imaginary axis $b$,
and the rotation angle $\theta$ of the symmetrical axis.
The parameter \texttt{angleY} is used to truncate the height of the figure,
it should be setup from 0 to 90.

\begin{BDef}
\Lcs{pstGeneralIHyperbola}\OptArgs\Largr{O}\Largr{$a,\,b$}\OptArg{$\theta$}\OptArg{angleY}
\end{BDef}

\begin{LTXexample}[width=6cm,pos=l]
\begin{pspicture}[showgrid=true](-2,-1)(4,3)
\psset{dotscale=0.5}\psset{PointSymbol=*}\footnotesize
\def\a{0.5}\def\b{0.3}\psset{PointNameSep=0.3}
\pstGeonode[PosAngle=-90](1,1){O}
\pstGeneralIHyperbola[linecolor=red!20](O)(\a,\b)[0][80]
\pstGeneralIHyperbolaNode[linecolor=red!80,PosAngle=-90](O)(\a,\b)[0]{0}{A}
\pstGeneralIHyperbola[linecolor=blue!40](O)(\a,\b)[40][80]
\pstGeneralIHyperbolaNode[linecolor=blue!40,PosAngle=190](O)(\a,\b)[40]{40}{B}
\pstGeneralIHyperbola[linecolor=green!60](O)(\a,\b)[90][80]
\pstGeneralIHyperbolaNode[linecolor=green!60,PosAngle=0](O)(\a,\b)[90]{200}{C}
\pstGeneralIHyperbola[linecolor=purple!80](O)(\a,\b)[150][80]
\pstGeneralIHyperbolaNode[linecolor=purple!80,PosAngle=-90](O)(\a,\b)[150]{50}{D}
\end{pspicture}
\end{LTXexample}

The macro \Lcs{pstGeneralIHyperbolaNode} is used to draw a node whose parameter is the given value $t$ on General Inversion Hyperbola,
please refer to equation (\ref{ParametricFunctionOfGeneralInversionHyperbola}).

The macro \Lcs{pstGeneralIHyperbolaAbsNode} is used to draw the nodes whose abscissa are the given value $x_1$ on General Inversion Hyperbola.
The macro \Lcs{pstGeneralIHyperbolaOrdNode} is used to draw the nodes whose ordinate are the given value $y_1$ on General Inversion Hyperbola.

Note that \Lcs{pstGeneralIHyperbolaAbsNode} and \Lcs{pstGeneralIHyperbolaOrdNode} will create two nodes $A$ and $B$ at most time.

\begin{BDef}
\Lcs{pstGeneralIHyperbolaNode}\OptArgs\Largr{O}\Largr{$a,\,b$}\OptArg{$\theta$}\Largb{$t$}\Largb{A}\\
\Lcs{pstGeneralIHyperbolaAbsNode}\OptArgs\Largr{O}\Largr{$a,\,b$}\OptArg{$\theta$}\Largb{$x_1$}\Largb{A}\Largb{B}\\
\Lcs{pstGeneralIHyperbolaOrdNode}\OptArgs\Largr{O}\Largr{$a,\,b$}\OptArg{$\theta$}\Largb{$y_1$}\Largb{A}\Largb{B}
\end{BDef}

\begin{LTXexample}[width=6cm,pos=l]
\begin{pspicture}[showgrid=true](-2,-1)(4,3)
\psset{dotscale=0.5}\psset{PointSymbol=*}\footnotesize
\def\a{0.5}\def\b{0.3}\psset{PointNameSep=0.3}
\pstGeonode[PosAngle=180](1,1){O}
\pstGeneralIHyperbola[linecolor=purple!80](O)(\a,\b)[150][80]
\pstGeneralIHyperbolaAbsNode[linecolor=purple!80,PosAngle={200,90}](O)(\a,\b)[150]{2}{P}{Q}
\pstGeneralIHyperbolaAbsNode[linecolor=purple!80,PosAngle={90,200}](O)(\a,\b)[150]{0}{X}{Y}
\pstGeneralIHyperbolaAbsNode[linecolor=purple!80,PosAngle={40,-40}](O)(\a,\b)[150]{1}{M}{N}
\pstLine[linestyle=dashed,linecolor=gray!40,nodesep=-1.5]{M}{N}
\pstGeneralIHyperbolaOrdNode[linecolor=purple!80,PosAngle={180,90}](O)(\a,\b)[150]{2}{G}{H}
\pstGeneralIHyperbolaOrdNode[linecolor=purple!80,PosAngle={90,240}](O)(\a,\b)[150]{0}{I}{J}
\pstGeneralIHyperbolaOrdNode[linecolor=purple!80,PosAngle={-100,-60}](O)(\a,\b)[150]{1.4063}{K}{L}
\pstLine[linestyle=dashed,linecolor=gray!80,nodesep=-1]{I}{J}
\pstLine[linestyle=dashed,linecolor=gray!80,nodesep=-1]{-1,1.4063}{3,1.4063}
\end{pspicture}
\end{LTXexample}

The macro \Lcs{pstGeneralIHyperbolaFocusNode} is used to find the focus nodes of the General Inversion Hyperbola,
the macro \Lcs{pstGeneralIHyperbolaVertexNode} is used to find the vertex nodes of the General Inversion Hyperbola,
and the macro \Lcs{pstGeneralIHyperbolaDirectrixLine} is used to find the directrix lines of the General Inversion Hyperbola.

\begin{BDef}
\Lcs{pstGeneralIHyperbolaFocusNode}\OptArgs\Largr{O}\Largr{$a,\,b$}\OptArg{$\theta$}\Largb{$F_1$}\Largb{$F_2$}\\
\Lcs{pstGeneralIHyperbolaVertexNode}\OptArgs\Largr{O}\Largr{$a,\,b$}\OptArg{$\theta$}\Largb{$V_1$}\Largb{$V_2$}\\
\Lcs{pstGeneralIHyperbolaDirectrixLine}\OptArgs\Largr{O}\Largr{$a,\,b$}\OptArg{$\theta$}\Largb{$L_x$}\Largb{$L_y$}\Largb{$R_x$}\Largb{$R_y$}
\end{BDef}

Note that you can use the macro \Lcs{pstGeneralIHyperbolaAsymptoteLine} to get the asymptote lines, this macro only create one node on each asymptote line,
as the other one is the center of the hyperbola.

\begin{BDef}
\Lcs{pstGeneralHyperbolaAsymptoteLine}\OptArgs\Largr{O}\Largr{$a,\,b$}\OptArg{$\theta$}\Largb{$L_1$}\Largb{$L_2$}
\end{BDef}

\begin{LTXexample}[width=6cm,pos=l]
\begin{pspicture}[showgrid=true](-2,-2)(4,4)
\psset{dotscale=0.5}\psset{PointSymbol=*}\footnotesize
\def\a{0.5}\def\b{0.3}
\pstGeonode[PosAngle=180,PointNameSep=0.2](1,1){O}
\pstGeneralIHyperbola[linecolor=red!40](O)(\a,\b)[0][80]
\pstGeneralIHyperbolaFocusNode[linecolor=red!40,PointName={X_1,X_2},PosAngle={90,-90}](O)(\a,\b)[0]{X1}{X2}
\pstGeneralIHyperbolaDirectrixLine[linecolor=red!40,nodesepA=-2,nodesepB=-1,PointName=none](O)(\a,\b)[0]{Lx}{Ly}{Rx}{Ry}
\pstGeneralIHyperbolaAsymptoteLine[linecolor=red!40,nodesepA=-2,nodesepB=-1,PointName=none](O)(\a,\b)[0]{T1}{T2}
\pstGeneralIHyperbola[linecolor=blue!40](O)(\a,\b)[40][80]
\pstGeneralIHyperbolaFocusNode[linecolor=blue!40,PointName={F_1,F_2},PosAngle={130,-40}](O)(\a,\b)[40]{F1}{F2}
\pstGeneralIHyperbolaDirectrixLine[linecolor=blue!40,nodesepA=-2,nodesepB=-1,PointName=none](O)(\a,\b)[40]{Dx}{Dy}{Ux}{Uy}
\pstGeneralIHyperbolaAsymptoteLine[linecolor=blue!40,nodesepA=-2,nodesepB=-1,PointName=none](O)(\a,\b)[40]{S1}{S2}
\pstGeneralIHyperbola[linecolor=brown!40](O)(\a,\b)[90][80]
\pstGeneralIHyperbolaFocusNode[linecolor=brown!40,PointName={Y_1,Y_2},PosAngle={180,0}](O)(\a,\b)[90]{Y1}{Y2}
\pstGeneralIHyperbolaDirectrixLine[linecolor=brown!40,nodesepA=-2,nodesepB=-1,PointName=none](O)(\a,\b)[90]{Tx}{Ty}{Sx}{Sy}
\pstGeneralIHyperbolaAsymptoteLine[linecolor=brown!40,nodesepA=-2,nodesepB=-1,PointName=none](O)(\a,\b)[90]{R1}{R2}
\end{pspicture}
\end{LTXexample}


The macro \Lcs{pstGeneralIHyperbolaLineInter} is used to find the intersections $C$ and $D$ of the general inversion hyperbola and the given line $AB$.

\begin{BDef}
\Lcs{pstGeneralIHyperbolaLineInter}\OptArgs\Largr{O}\Largr{$a,\,b$}\OptArg{$\theta$}\Largb{$A$}\Largb{$B$}\Largb{$C$}\Largb{$D$}
\end{BDef}

In the following example, the lines $YY'$ and $ZZ'$ are parallel to the asymptote of the hyperbola,
so there are only one intersection $M$ and $P$ for each line, and the second node $N$ and $Q$ are put at the origin.

\begin{LTXexample}[width=6cm,pos=l]
\begin{pspicture}[showgrid=true](-2,-1)(4,3)
\psset{dotscale=0.5}\psset{PointSymbol=*}\footnotesize
\def\a{0.5}\def\b{0.3}\psset{PointNameSep=0.3}
\pstGeonode[PosAngle=225](1,1){O}
\pstGeneralIHyperbola[linecolor=blue!40](O)(\a,\b)[30][80]
\pstLine[linestyle=dashed,linecolor=gray!40]{-1,-1}{-1,3}
\pstGeneralIHyperbolaLineInter[linecolor=blue!40,PosAngle={-30,210}](O)(\a,\b)[30]{-1,-1}{-1,3}{A}{B}
\pstLine[linestyle=dashed,linecolor=gray!40]{-2,1}{3,3}
\pstGeneralIHyperbolaLineInter[linecolor=blue!40,PosAngle={130,-90}](O)(\a,\b)[30]{-2,1}{3,3}{C}{D}
\pstGeonode[PosAngle={-20,100}](2,0){Y}(1.8,2){Z}
\pstLineAA[nodesepA=-3,nodesepB=-2,linecolor=gray!40,PointName=none,PointSymbol=none]{O}{\a\space \b\space div 1 atan 30 add}{U}
\pstLineAA[nodesepA=-3,nodesepB=-2,linecolor=red!40,PointName=none,PointSymbol=none]{O}{\a\space \b\space div neg 1 atan 30 add}{V}
\pstLineAA[nodesepA=-3,nodesepB=-2,linecolor=gray!40,PosAngle=10]{Y}{\a\space \b\space div 1 atan 30 add}{Y'}
\pstLineAA[nodesepA=-3,nodesepB=-2,linecolor=red!40,PosAngle=80]{Z}{\a\space \b\space div neg 1 atan 30 add}{Z'}
\pstGeneralIHyperbolaLineInter[linecolor=blue!40,PosAngle={30,-90}](O)(\a,\b)[30]{Z}{Z'}{M}{N}
\pstGeneralIHyperbolaLineInter[linecolor=blue!40,PosAngle={30,210}](O)(\a,\b)[30]{Y}{Y'}{P}{Q}
\end{pspicture}
\end{LTXexample}

The macro \Lcs{pstGeneralIHyperbolaPolarNode} is used to find the polar point $T$ of chord $AB$ on the general inversion hyperbola.

\begin{BDef}
\Lcs{pstGeneralIHyperbolaPolarNode}\OptArgs\Largr{O}\Largr{$a,\,b$}\OptArg{$\theta$}\Largb{$A$}\Largb{$B$}\Largb{$T$}
\end{BDef}

We also use the theorem \ref{HyperbolaPolarPointTheorem} to find the polar point $T$ of chord $AB$:

\begin{LTXexample}[width=6cm,pos=l]
\begin{pspicture}[showgrid=true](-1,-1)(3,3)
\psset{dotscale=0.5}\psset{PointSymbol=*}\footnotesize
\def\a{0.5}\def\b{0.3}\psset{PointNameSep=0.3}
\pstGeonode[PosAngle=40,PointNameSep=0.2](1,1){O}
\pstGeneralIHyperbola[linecolor=blue!40](O)(\a,\b)[40][80]
\pstGeneralIHyperbolaNode[linecolor=blue!40,PosAngle=40](O)(\a,\b)[40]{50}{A}
\pstGeneralIHyperbolaNode[linecolor=blue!40,PosAngle=-100](O)(\a,\b)[40]{-70}{B}
\pstGeneralIHyperbolaPolarNode[linecolor=red!40,PosAngle=-90](O)(\a,\b)[40]{A}{B}{T}
\pstLine[linestyle=dashed,linecolor=gray!40,nodesep=-1]{A}{B}
\end{pspicture}
\end{LTXexample}

The macro \Lcs{pstGeneralIHyperbolaTangentNode} is used to find the tangent point $A$ and $B$ of point $T$ outside of the general inversion  hyperbola.

\begin{BDef}
\Lcs{pstGeneralIHyperbolaTangentNode}\OptArgs\Largr{O}\Largr{$a,\,b$}\OptArg{$\theta$}\Largb{$T$}\Largb{$A$}\Largb{$B$}
\end{BDef}

We also use the theorem \ref{HyperbolaTangentPointTheorem} to find the tangent points $A$ and $B$ of $T$.

\begin{LTXexample}[width=6cm,pos=l]
\begin{pspicture}[showgrid=true](-2,-1)(4,3)
\psset{dotscale=0.5}\psset{PointSymbol=*}\footnotesize
\def\a{0.5}\def\b{0.3}\psset{PointNameSep=0.3}
\pstGeonode[PosAngle=20](1,1){O}
\pstGeneralIHyperbola[linecolor=blue!40](O)(\a,\b)[40][80]
\pstGeonode[PosAngle=-40](0,1){T}
\pstGeneralIHyperbolaTangentNode[linecolor=red!40,PosAngle={-80,120},nodesep=-0.5](O)(\a,\b)[40]{T}{A}{B}
\end{pspicture}
\end{LTXexample}

\section{Geometric Transformations}

The geometric transformations are the ideal tools to construct geometric figures. All
the classical transformations are available with the following macros  which
share the same syntaxic scheme end two parameters.

The common syntax put at the end two point lists whose second is optional or with a
cardinal at least equal. These two lists contain the antecedent points and their
respective images. In the case no image is given for some points the a  default name
is build appending a\verb$'$ to the antecedent name.

The first shared parameter is \Lkeyword{CodeFig} which draws the specific
constructions lines. Its default value is \Lkeyword{false}, and a
\Lkeyword{true} value activates this optional drawing.
The drawing is done using the line style \Lkeyword{CodeFigStyle}
\DefaultVal{dashed}, with the color \Lkeyword{CodeFigColor}
\DefaultVal{cyan}.

Their second shared parameter is \Lkeyword{CurveType} which controls the drawing of a
line crossing all images, and thus allow a quick description of a transformed figure.

\subsection{Central symmetry}

\begin{BDef}
\Lcs{pstSymO}\OptArgs\Largb{$O$}\Largb{$M_1, M_2, \cdots, M_n$}\OptArg{$M'_1, M'_2, \cdots, M'_p$}
\end{BDef}

\begin{sloppypar}
Possible optional arguments are
  \Lkeyword{PointSymbol}, \Lkeyword{PosAngle},
  \Lkeyword{PointName}, \Lkeyword{PointNameSep}, \Lkeyword{PtNameMath},
  \Lkeyword{CodeFig}, \Lkeyword{CodeFigColor}, and \Lkeyword{CodeFigStyle}.
Draw the symmetric point in relation to point $O$. The classical
parameter of point creation are usable here, and also for all the
following functions.
\end{sloppypar}

\begin{LTXexample}[width=5cm,pos=l]
\begin{pspicture}[showgrid](-2,-2)(2,2)
\psset{CodeFig=true}
\pstGeonode[PosAngle={20,90,0}]{O}(-.6,1.5){A}(1.6,-.5){B}
\pstSymO[CodeFigColor=blue,
  PosAngle={-90,180}]{O}{A, B}[C, D]
\pstLineAB{A}{B}\pstLineAB{C}{D}
\pstLineAB{A}{D}\pstLineAB{C}{B}
\end{pspicture}
\end{LTXexample}

\subsection{Orthogonal (or axial) symmetry}

\begin{BDef}
\Lcs{pstOrtSym}\OptArgs\Largb{$A$}\Largb{$B$}\Largb{$M_1, M_2, \cdots, M_n$}\OptArg{$M'_1, M'_2, \cdots, M'_p$}
\end{BDef}

\begin{sloppypar}
Possible optional arguments are
\Lkeyword{PointSymbol}, \Lkeyword{PosAngle},
  \Lkeyword{PointName}, \Lkeyword{PointNameSep}, \Lkeyword{PtNameMath},
  \Lkeyword{CodeFig}, \Lkeyword{CodeFigColor}, and \Lkeyword{CodeFigStyle}.
%
Draws the symmetric point in relation to line $(AB)$.
\end{sloppypar}

\begin{LTXexample}[width=5cm,pos=l]
\psset{unit=0.6}
\begin{pspicture}[showgrid](0,-2)(8,7)
\pstTriangle(1,3){B}(5,5){C}(4,1){A}
\pstOrtSym{A}{B}{C}[D]
\psset{CodeFig=true}
\pstOrtSym[dash=2mm 2mm,CodeFigColor=red]%
  {C}{B}{A}
\pstOrtSym[SegmentSymbol=pstslash,
  linestyle=dotted,dotsep=3mm,CodeFigColor=blue]%
  {C}{A}{B}
\end{pspicture}
\end{LTXexample}


\subsection{Rotation}


\begin{BDef}
\Lcs{pstRotation}\OptArgs\Largb{$O$}\Largb{$M_1, M_2, \cdots, M_n$}\OptArg{$M'_1, M'_2, \cdots, M'_p$}\\
\Lcs{pstAngleAOB}\Largb{$A$}\Largb{$O$}\Largb{$B$}
\end{BDef}

\begin{sloppypar}
Possible optional arguments are
  \Lkeyword{PointSymbol}, \Lkeyword{PosAngle},
  \Lkeyword{PointName}, \Lkeyword{PointNameSep}, \Lkeyword{PtNameMath}, and \Lkeyword{RotAngle}
for \Lcs{pstRotation} and \Lkeyword{AngleCoef}, \Lkeyword{RotAngle} for \Lcs{pstAngleABC}.
%
Draw the image of $M_i$ by the rotation of center $O$ and angle given by
the parameter \Lkeyword{RotAngle}. This later can be an angle specified
by three points. In such a case, the following function must be used:
\end{sloppypar}



Never forget to use the rotation for drawing a square or an equilateral
triangle. The parameter \Lkeyword{CodeFig} puts a bow with an arrow between the
point and its image, and if \Lkeyword{TransformLabel} \DefaultVal{none}
contain some text, it is put on the corresponding angle in mathematical mode.

\begin{LTXexample}[width=5cm,pos=l]
\begin{pspicture}[showgrid](-2,-2)(2,2)
\psset{arrowscale=2}
\pstGeonode[PosAngle=-135](-1.5,-.2){A}%
  (.5,.2){B}(0,-2){D}
\pstRotation[PosAngle=90,RotAngle=60,
  CodeFig,CodeFigColor=blue,
  TransformLabel=\frac{\pi}{3}]{A}{B}[C]
\pstRotation[AngleCoef=.5,
  RotAngle=\pstAngleAOB{B}{A}{C},
  CodeFigColor=red, CodeFig,
  TransformLabel=\frac{1}{2}\widehat{BAC}]{A}{D}[E]
\end{pspicture}
\end{LTXexample}


 \subsection{Translation}

\begin{BDef}
\Lcs{pstTranslation}\OptArgs\Largb{$A$}\Largb{$B$}\Largb{$M_1, M_2, \cdots, M_n$}\OptArg{$M'_1, M'_2, \cdots, M'_p$}
\end{BDef}

\begin{sloppypar}
Possible optional arguments are
\Lkeyword{PointSymbol}, \Lkeyword{PosAngle},
  \Lkeyword{PointName}, \Lkeyword{PointNameSep}, \Lkeyword{PtNameMath}, and \Lkeyword{DistCoef}
%
Draws the translated $M'_i$ of $M_i$ using the vector $\vec{AB}$. Useful for drawing a
parallel line.
\end{sloppypar}

The parameter \Lkeyword{DistCoef} can be used as a multiplicand
coefficient to modify the translation vector. The parameter \Lkeyword{CodeFig}
draws the translation vector le vecteur de translation between the
point and its image, labeled in its middle defaultly with the vector name or by the
text specified with \Lkeyword{TransformLabel} \DefaultVal{none}.

\begin{LTXexample}[width=5cm,pos=l]
\begin{pspicture}[showgrid](-2,-2)(2,2)
\psset{linecolor=green,nodesep=-1,
  PosAngle=90,arrowscale=2}
\pstGeonode(-1.5,-1.2){A}(.5,-.8){B}(.5,1){C}(-1,0){D}(-2,-2){E}
\pstTranslation{B}{A}{C}
\psset{CodeFig,TransformLabel=default}
\pstTranslation{A}{B}{D}
\pstTranslation[DistCoef=1.5]{A}{B}{E}
\pstLineAB{A}{B}\pstLineAB{C}{C'}
\end{pspicture}
\end{LTXexample}



\subsection{Homothetie}


\begin{BDef}
\Lcs{pstHomO}\OptArgs\Largb{$O$}\Largb{$M_1, M_2, \cdots, M_n$}\OptArg{$M'_1, M'_2, \cdots, M'_p$}
\end{BDef}

\begin{sloppypar}
Possible optional arguments are
\Lkeyword{HomCoef},
  \Lkeyword{PointSymbol}, \Lkeyword{PosAngle},
  \Lkeyword{PointName}, \Lkeyword{PointNameSep}, \Lkeyword{PtNameMath}, and \Lkeyword{HomCoef}.
%
Draws $M'_i$ the image of $M_i$ by the homotethy of center $O$ and
coefficient specified with the parameter \Lkeyword{HomCoef}.
\end{sloppypar}

\begin{LTXexample}[width=5cm,pos=l]
\begin{pspicture}[showgrid](-2,-2)(2,2)
\pstGeonode[PosAngle={0,-45}](.5,1){O}%
       (-1.5,-1.2){A}(.5,-.8){B}
\pstHomO[HomCoef=.62,PosAngle=-45]{O}{A,B}[C,D]
\psset{linecolor=green,nodesep=-1}
\pstLineAB{A}{O}\pstLineAB{B}{O}
\psset{linecolor=red,nodesep=-.5}
\pstLineAB{A}{B}\pstLineAB{C}{D}
\end{pspicture}
\end{LTXexample}


\subsection{Orthogonal projection}


\begin{BDef}
\Lcs{pstProjection}\OptArgs\Largb{$A$}\Largb{$B$}\Largb{$M_1, M_2, \cdots, M_n$}\OptArg{$M'_1, M'_2, \cdots, M'_p$}
\end{BDef}

\begin{sloppypar}
Possible optional arguments are
  \Lkeyword{PointSymbol}, \Lkeyword{PosAngle},
  \Lkeyword{PointName}, \Lkeyword{PointNameSep}, \Lkeyword{PtNameMath},
  \Lkeyword{CodeFig}, \Lkeyword{CodeFigColor}, and\Lkeyword{CodeFigStyle}
%
Projects orthogonally the point $M_i$ on the line $(AB)$. Useful for the altitude of a
triangle. The name is aligned with the point and the projected point as
shown in the exemple.
\end{sloppypar}

\begin{LTXexample}[width=5cm,pos=l]
\begin{pspicture}[showgrid](-3,-2)(2,2)
\psset{PointSymbol=none,CodeFig,CodeFigColor=red}
\pstTriangle(1,1){A}(-2,1){C}(-1,-1){B}
\pstProjection{A}{B}{C}[I]
\pstProjection{A}{C}{B}[J]
\pstProjection{C}{B}{A}[K]
\end{pspicture}
\end{LTXexample}

\section{Special object}

  %%%%%%%%%%%%%%%%%%%%%%%%%%%%%%%%%%%%%%%%%%%%%%%%%%%%%%%%%%%%%%%%%%%%
  \subsection{Midpoint}



\begin{BDef}
\Lcs{pstMiddleAB}\OptArgs\Largb{$A$}\Largb{$B$}\Largb{$I$}
\end{BDef}

\begin{sloppypar}
\Lkeyword{PointSymbol}, \Lkeyword{PosAngle},
  \Lkeyword{PointName}, \Lkeyword{PointNameSep}, \Lkeyword{PtNameMath}, \Lkeyword{SegmentSymbol},
  \Lkeyword{CodeFig}, \Lkeyword{CodeFigColor}, and \Lkeyword{CodeFigStyle}
%
Draw the midpoint $I$ of segment $[AB]$. By default, the point name is
automatically put below the segment.
\end{sloppypar}

\begin{LTXexample}[width=5cm,pos=l]
\begin{pspicture}[showgrid](-3,-2)(2,2)
\pstTriangle[PointSymbol=none]%
  (1,1){A}(-1,-1){B}(-2,1){C}
\pstMiddleAB{A}{B}{C'}
\pstMiddleAB{C}{A}{B'}
\pstMiddleAB{B}{C}{A'}
\end{pspicture}
\end{LTXexample}


  \subsection{Triangle center of gravity}


\begin{BDef}
\Lcs{pstCGravABC}\OptArgs\Largb{$A$}\Largb{$B$}\Largb{$C$}\Largb{$G$}
\end{BDef}

\begin{sloppypar}
Possible optional arguments are
\Lkeyword{PointName}, \Lkeyword{PointNameSep}, \Lkeyword{PosAngle},
        \Lkeyword{PointSymbol}, and \Lkeyword{PtNameMath}
%
Draw the $ABC$ triangle centre of gravity $G$.
\end{sloppypar}

\begin{LTXexample}[width=5cm,pos=l]
\begin{pspicture}[showgrid](-3,-2)(2,2)
\pstTriangle[PointSymbol=none]%
  (1,1){A}(-1,-1){B}(-2,1){C}
\pstCGravABC{A}{B}{C}{G}
\end{pspicture}
\end{LTXexample}


  \subsection{Centre of the circumcircle of a triangle}



\begin{BDef}
\Lcs{pstCircleABC}\OptArgs\Largb{$A$}\Largb{$B$}\Largb{$C$}\Largb{$O$}
\end{BDef}

\begin{sloppypar}
Possible optional arguments are
\Lkeyword{PointName}, \Lkeyword{PointNameSep}, \Lkeyword{PosAngle},
        \Lkeyword{PointSymbol}, \Lkeyword{PtNameMath}, \Lkeyword{DrawCirABC}, \Lkeyword{CodeFig},
        \Lkeyword{CodeFigColor}, \Lkeyword{CodeFigStyle}, \Lkeyword{SegmentSymbolA},
        \Lkeyword{SegmentSymbolB}, and \Lkeyword{SegmentSymbolC}.
%
Draws the circle crossing three points (the circum circle) and put its center $O$.
The effective drawing is controlled by the boolean parameter \Lkeyword{DrawCirABC}
\DefaultVal{true}. Moreover the intermediate constructs (mediator lines) can
be drawn by setting the boolean parameter \Lkeyword{CodeFig}. In that case the middle
points are marked on the segemnts using three different marks given by the parameters
\Lkeyword{SegmentSymbolA}, \Lkeyword{SegmentSymbolB} et \Lkeyword{SegmentSymbolC}.
\end{sloppypar}

\begin{LTXexample}[width=6cm,pos=l]
\begin{pspicture}[showgrid](6,6)
\pstTriangle[PointSymbol=none]%
  (4,1){A}(1,3){B}(5,5){C}
\pstCircleABC[CodeFig,CodeFigColor=blue,
  linecolor=red,PointSymbol=none]{A}{B}{C}{O}
\end{pspicture}
\end{LTXexample}


  \subsection{Perpendicular bisector of a segment}

\begin{BDef}
\Lcs{pstMediatorAB}\OptArgs\Largb{$A$}\Largb{$B$}\Largb{$I$}\Largb{$M$}
\end{BDef}

\begin{sloppypar}
Possible optional arguments are
\Lkeyword{PointName}, \Lkeyword{PointNameSep}, \Lkeyword{PosAngle},
        \Lkeyword{PointSymbol}, \Lkeyword{PtNameMath}, \Lkeyword{CodeFig},
        \Lkeyword{CodeFigColor}, \Lkeyword{CodeFigStyle}, and \Lkeyword{SegmentSymbol}.
%
The perpendicular bisector of a segment is a line perpendicular to
this segment in its midpoint. The segment is $[AB]$, the midpoint $I$,
and $M$ is a point belonging to the perpendicular bisector line. It is
build by a rotation of $B$ of 90 degrees around $I$. This mean
that the order of $A$ and $B$ is important, it controls the position
of $M$. The command creates the two points $M$ end $I$. The
construction is controlled by the following parameters:
\end{sloppypar}

\begin{compactitem}
\item \Lkeyword{CodeFig}, \Lkeyword{CodeFigColor} and \Lkeyword{SegmentSymbol}
  for marking the right angle ;
\item \Lkeyword{PointSymbol} et \Lkeyword{PointName} for controlling the
  drawing of the two points, each of them can be specified
  separately with the parameters \Lkeyword{...A} and \Lkeyword{...B} ;
\item parameters controlling the line drawing.
\end{compactitem}


\begin{LTXexample}[width=6cm,pos=l]
\begin{pspicture}[showgrid](6,6)
\pstTriangle[PointSymbol=none](3.5,1){A}(1,4){B}(5,4.2){C}
\psset{linecolor=red,CodeFigColor=red,nodesep=-1}
\pstMediatorAB[PointSymbolA=none]{A}{B}{I}{M_I}
\psset{PointSymbol=none,PointNameB=none}
\pstMediatorAB[CodeFig=true]{A}{C}{J}{M_J}
\pstMediatorAB[PosAngleA=45,linecolor=blue]
        {C}{B}{K}{M_K}
\end{pspicture}
\end{LTXexample}



  \subsection{Bisectors of angles}



\begin{BDef}
\Lcs{pstBissectBAC}\OptArgs\Largb{$B$}\Largb{$A$}\Largb{$C$}\Largb{$N$}\\
\Lcs{pstOutBissectBAC}\OptArgs\Largb{$B$}\Largb{$A$}\Largb{$C$}\Largb{$N$}
\end{BDef}

\begin{sloppypar}
Possible optional arguments are
\Lkeyword{PointSymbol}, \Lkeyword{PosAngle},
  \Lkeyword{PointName}, \Lkeyword{PointNameSep}, and \Lkeyword{PtNameMath}.
%
There are two bisectors for a given geometric angle: the inside one and
the outside one; this is why there is two commands. The angle is
specified by three points specified in the trigonometric direction
(anti-clockwise). The result of the commands is the specific line and
a point belonging to this line. This point is built by a rotation of
point $B$.
\end{sloppypar}


\begin{LTXexample}[width=6cm,pos=l]
\begin{pspicture}[showgrid](6,6)
\psset{CurveType=polyline,linecolor=red}
\pstGeonode[PosAngle={180,-75,45}]%
  (1,4){B}(4,1){A}(5,4){C}
\pstBissectBAC[linecolor=blue]{C}{A}{B}{A'}
\pstOutBissectBAC[linecolor=green,PosAngle=180]%
  {C}{A}{B}{A''}
\end{pspicture}
\end{LTXexample}


\section{Intersections}

Points can be defined by intersections. Six intersection types  are
managed:

\begin{compactitem}
\item line-line;
\item line-circle;
\item circle-circle;
\item function-function;
\item function-line;
\item function-circle.
\end{compactitem}

An intersection can not exist: case of parallel lines. In such a case,
the point(s) are positioned at the origin. In fact, the user has to
manage the existence of these points.

  \subsection{Line-Line}



\begin{BDef}
\Lcs{pstInterLL}\OptArgs\Largb{$A$}\Largb{$B$}\Largb{$C$}\Largb{$D$}\Largb{$M$}
\end{BDef}

\begin{sloppypar}
Possible optional arguments are
\Lkeyword{PointSymbol}, \Lkeyword{PosAngle},
  \Lkeyword{PointName}, \Lkeyword{PointNameSep}, and \Lkeyword{PtNameMath}.
%
Draw the intersection point between lines $(AB)$ and $(CD)$.
\end{sloppypar}

\begin{description}
\item[basique]


\begin{LTXexample}[width=6cm,pos=l]
\begin{pspicture}[showgrid](-1,-2)(4,3)
\pstGeonode(0,-1){A}(3,2){B}(3,0){C}(1,2){D}
\pstInterLL[PointSymbol=square]{A}{B}{C}{D}{E}
\psset{linecolor=blue, nodesep=-1}
\pstLineAB{A}{B}\pstLineAB{C}{D}
\end{pspicture}
\end{LTXexample}


\item[Horthocentre]

\begin{LTXexample}[width=6cm,pos=l]
\begin{pspicture}[showgrid](-2,-2)(3,3)
\psset{CodeFig,PointSymbol=none}
\pstTriangle[PosAngleA=180](-1,0){A}(3,-1){B}(3,2){C}
\pstProjection[PosAngle=-90]{B}{A}{C}
\pstProjection{B}{C}{A}
\pstProjection[PosAngle=90]{A}{C}{B}
\pstInterLL[PosAngle=135,PointSymbol=square]{A}{A'}{B}{B'}{H}
\end{pspicture}
\end{LTXexample}

\end{description}

  \subsection{Circle--Line}

\begin{BDef}
\Lcs{pstInterLC}\OptArgs\Largb{$A$}\Largb{$B$}\Largb{$O$}\Largb{$C$}\Largb{$M_1$}\Largb{$M_2$}
\end{BDef}

\begin{sloppypar}
Possible optional arguments are
\Lkeyword{PointSymbol}, \Lkeyword{PosAngle},
        \Lkeyword{PointName}, \Lkeyword{PointNameSep}, \Lkeyword{PtNameMath},
        \Lkeyword{PointSymbolA}, \Lkeyword{PosAngleA}, \Lkeyword{PointNameA},
        \Lkeyword{PointSymbolB}, \Lkeyword{PosAngleB}, \Lkeyword{PointNameB},
        \Lkeyword{Radius}, and \Lkeyword{Diameter}.
%
Draw the one or two intersection point(s) between the line  $(AB)$ and
the circle of centre $O$ and with radius $OC$.
\end{sloppypar}

The circle is specified with its center and either a point of its
circumference or with a radius specified with parameter \Lkeyword{radius}
or its diameter specified with parameter \Lkeyword{Diameter}. These two
parameters can be specified by macros \Lcs{pstDist},\Lcs{pstDistMul},\Lcs{pstDistAdd},
\Lcs{pstDistSub} etc.

The position of the wo points is such that the vectors $\vec{AB}$ abd
$\vec{M_1M_2}$ are in the same direction. Thus, if the points
definig the line are switch, then the resulting points will be also
switched. If the intersection is void, then the points are positionned
at the center of the circle.


\begin{LTXexample}[width=6cm,pos=l]
\psset{unit=0.8}
\begin{pspicture}[showgrid](-3,-2)(4,4)
\pstGeonode[PosAngle={-135,80,0}](-1,0){B}(3,-1){C}(-.9,.5){O}(0,2){A}
\pstGeonode(-2,3){I}
\pstCircleOA[linecolor=red]{O}{A}
\pstInterLC[PosAngle=-80]{C}{B}{O}{A}{D}{E}
\pstInterLC[PosAngleB=60, Radius=\pstDistAB{O}{D}]{I}{C}{O}{}{F}{G}
\pstInterLC[PosAngleB=180,Diameter=\pstDistMul{O}{D}{1.3}]
  {I}{B}{O}{}{H}{J}
\pstCircleOA[linecolor=red,Diameter=\pstDistMul{O}{D}{1.3}]{O}{}
\psset{nodesep=-1}
\pstLineAB[linecolor=green]{E}{C}
\pstLineAB[linecolor=cyan]{I}{C}
\pstLineAB[linecolor=magenta]{J}{I}
\end{pspicture}
\end{LTXexample}



\subsection{Circle--Circle}

\begin{BDef}
\Lcs{pstInterCC}\OptArgs\Largb{$O_1$}\Largb{$B$}\Largb{$O_2$}\Largb{$C$}\Largb{$M_1$}\Largb{$M_2$}
\end{BDef}


This function is similar to the last one. The boolean parameters
\Lkeyword{CodeFigA} et \Lkeyword{CodeFigB} allow the drawing of the arcs
at the intersection. In order to get a coherence \Lkeyword{CodeFig} allow
the drawing of both arcs. The boolean parameters \Lkeyword{CodeFigAarc} and
\Lkeyword{CodeFigBarc} specified the direction of these optional arcs:
trigonometric (by default) or clockwise. Here is a first example.

\begin{LTXexample}[width=5cm,pos=l]
\begin{pspicture}[showgrid](0,-1)(4,3)
\psset{dash=2mm 2mm}
\rput{10}{%
  \pstGeonode[PosAngle={0,-90,-90,90}]
     (1,-1){O}(2,1){A}(2,0.1){B}(2.5,1){C}}
\pstCircleOA[linecolor=red]{C}{B}
\pstInterCC[PosAngleA=135, CodeFigA=true, CodeFigAarc=false,
  CodeFigColor=magenta]{O}{A}{C}{B}{D}{E}
\pstInterCC[PosAngleA=170, CodeFigA=true,
   CodeFigAarc=false,
   CodeFigColor=green]{B}{E}{C}{B}{F}{G}
\end{pspicture}
\end{LTXexample}


And a more complete one, which includes the special circle
specification using radius and diameter. For such specifications it
exists the parameters \Lkeyword{RadiusA}, \Lkeyword{RadiusB},
\Lkeyword{DiameterA} and \Lkeyword{DiameterB}.

The macro \Lcs{pstInterCC} will not display the intersections as default, if you want to display
the label or symbol of the intersections, you must setup the parameters \Lkeyword{PosAngleA}
and \Lkeyword{PosAngleB} to change the default behavior.

\begin{LTXexample}
\begin{pspicture}[showgrid](-3,-4)(7,3)
\pstGeonode[PointName={\Omega,O}](3,-1){Omega}(1,-1){O}
\pstGeonode[PointSymbol=square, PosAngle={-90,90}](0,3){A}(2,2){B}
\psset{PointSymbol=o}
\pstCircleOA[linecolor=red, Radius=\pstDistMul{A}{B}{1 3 10 div add}]{O}{}
\pstCircleOA[linecolor=Orange, Diameter=\pstDistAB{A}{B}]{O}{}
\pstCircleOA[linecolor=Violet, Radius=\pstDistAB{A}{B}]{Omega}{}
\pstCircleOA[linecolor=Purple, Diameter=\pstDistAB{A}{B}]{Omega}{}
\pstInterCC[RadiusA=\pstDistMul{A}{B}{1 3 10 div add},
            RadiusB=\pstDistAB{A}{B}]{O}{}{Omega}{}{D}{E}
\pstInterCC[DiameterA=\pstDistAB{A}{B}, RadiusB=\pstDistAB{A}{B},
            PosAngleA=90,PosAngleB=-90]{O}{}{Omega}{}{F}{G}
\pstInterCC[RadiusA=\pstDistMul{A}{B}{1 3 10 div add}, DiameterB=\pstDistAB{A}{B},
            PosAngleA=90,PosAngleB=-90]{O}{}{Omega}{}{H}{I}
\pstInterCC[DiameterA=\pstDistAB{A}{B}, DiameterB=\pstDistAB{A}{B},
            PosAngleA=90,PosAngleB=-90]{O}{}{Omega}{}{J}{K}
\end{pspicture}
\end{LTXexample}

  \subsection{Function--function}


\begin{BDef}
\Lcs{pstInterFF}\OptArgs\Largb{$f$}\Largb{$g$}\Largb{$x_0$}\Largb{$M$}
\end{BDef}

This function put a point at the intersection between two curves
defined by a function. $x_0$ is an intersection approximated value of
the abscissa. It is obviously possible to ise this function several
time if more than one intersection is present. Each function is
describerd in \PS in the same way as the description used by
the \Lcs{psplot} macro of \PST. A constant function can be
specified, and then seaching function root is possible.

The Newton algorithm is used for the research, and the intersection
may not to be found. In such a case the point is positionned at the
origin. On the other hand, the research can be trapped (in a local
extremum near zero).

\begin{LTXexample}[width=5cm,pos=l]
\begin{pspicture}[showgrid](-3,-1)(2,4)
\psaxes{->}(0,0)(-2,0)(2,4)
\psset{linewidth=1.5pt,algebraic}
\psplot[linecolor=gray]{-2}{2}{x^2}
\psplot{-2}{2}{2-x/2}
\psset{PointSymbol=o}
\pstInterFF{2-x/2}{x^2}{1}{M_1}
\pstInterFF{2-x/2}{x^2}{-2}{M_0}
\end{pspicture}
\end{LTXexample}

\subsection{Function--line}

\begin{BDef}
\Lcs{pstInterFL}\OptArgs\Largb{$f$}\Largb{$A$}\Largb{$B$}\Largb{$x_0$}\Largb{$M$}
\end{BDef}

Puts a point at the intersection between the function $f$ and the line
$(AB)$.

\begin{LTXexample}[width=6cm,pos=l]
\begin{pspicture}[showgrid](-3,-1.5)(3,4)
\def\F{x^3/3 - x + 2/3 }
\psaxes{->}(0,0)(-3,-1)(3,4)
\psplot[linewidth=1.5pt,algebraic]{-2.5}{2.5}{\F}
\psset{PointSymbol=*}
\pstGeonode[PosAngle={-45,0}](0,-.2){N}(2.5,1){M}
\pstLineAB[nodesepA=-3cm]{N}{M}
\psset{PointSymbol=o,algebraic}
\pstInterFL{\F}{N}{M}{2}{A}
\pstInterFL[PosAngle=90]{\F}{N}{M}{0}{A'}
\pstInterFL{\F}{N}{M}{-2}{A''}
\end{pspicture}
\end{LTXexample}


\vspace{1cm}
\subsection{Function--Circle}

\begin{BDef}
\Lcs{pstInterFC}\OptArgs\Largb{$f$}\Largb{$O$}\Largb{$A$}\Largb{$x_0$}\Largb{$M$}
\end{BDef}

Puts a point at the intersection between the function $f$ and the circle
of centre $O$ and radius $OA$.

\begin{LTXexample}[width=6cm,pos=l]
\begin{pspicture}[showgrid](-3,-4)(3,4)
\def\F{2*cos(x)}
\psset{algebraic}
\pstGeonode(0.3,-1){O}(2,.5){M}
\ncline[linecolor=blue, arrowscale=2]{->}{O}{M}
\psaxes{->}(0,0)(-3,-3)(3,4)
\psplot[linewidth=1.5pt]{-3.14}{3.14}{\F}
\pstCircleOA[PointSymbol=*]{O}{M}
\psset{PointSymbol=o}
\pstInterFC{\F}{O}{M}{1}{N0}
\pstInterFC{\F}{O}{M}{-1}{N1}
\pstInterFC{\F}{O}{M}{-2}{N2}
\pstInterFC{\F}{O}{M}{2}{N3}
\end{pspicture}
\end{LTXexample}



\section{Helper Macros}

\begin{BDef}
\Lcs{psGetDistanceAB}\OptArgs\coord1\coord2\Largb{<name>}\\
\Lcs{psGetAngleABC}\OptArgs\coord1\coord2\coord3\Largb{<symbol>}
\end{BDef}


Calculates and prints the values. This is only possible on PostScript level!


\begin{pspicture}[showgrid](-1,-1)(8,7)
\def\sideC{6}  \def\sideA{7}  \def\sideB{8}
\psset{unit=0.8cm,PointSymbol=none,linejoin=1,linewidth=0.4pt,PtNameMath=false,labelsep=0.07,MarkAngleRadius=1.1,decimals=1}
\pstGeonode[PosAngle={-90,-90}](0,0){A}(\sideC;10){B}
\pstInterCC[RadiusA=\pstDistVal{\sideB},RadiusB=\pstDistVal{\sideA},PosAngleA=90,PointNameA=C]{A}{}{B}{}{C}{C-}
\pstInterCC[RadiusA=\pstDistAB{A}{B},RadiusB=\pstDistAB{B}{C}]{C}{}{A}{}{D-}{D}
\psset{PointName=none}
\pstInterLC[Radius=\pstDistAB{A}{C}]{C}{D}{C}{}{A'-}{A'}
\pstInterCC[RadiusA=\pstDistAB{A}{B},RadiusB=\pstDistAB{B}{C}]{A'}{}{C}{}{B'}{B'-}
\pstInterLL[PosAngle=-90,PointName=default]{B'}{C}{A}{B}{E}
\pspolygon(A)(B)(C)
\pspolygon[fillstyle=solid,fillcolor=magenta,opacity=0.1](C)(E)(B)
%
\psGetAngleABC[ArcColor=blue,AngleValue=true,LabelSep=0.4,arrows=->,decimals=0,PSfont=Palatino-Roman](B)(A)(C){}
\psGetAngleABC[AngleValue=true,ArcColor=red,arrows=->,WedgeOpacity=0.6,WedgeColor=yellow!30,LabelSep=0.4](C)(B)(A){$\beta$}
\psGetAngleABC[LabelSep=0.4,AngleValue=true,WedgeColor=green,xShift=-6,yShift=-10](A)(C)(B){$\gamma$}
\psGetAngleABC[LabelSep=0.4,AngleValue=true,AngleArc=false,WedgeColor=green,arrows=->,xShift=-15,yShift=0](C)(E)(B){\color{blue}$\gamma$}
\psGetAngleABC[AngleValue=true,MarkAngleRadius=1.0,LabelSep=0.5,ShowWedge=false,xShift=-5,yShift=7,arrows=->](E)(B)(C){}
%
\pcline[linestyle=none](A)(B)\nbput{\sideC}
\pcline[linestyle=none](C)(B)\naput{\sideA}
\psGetDistanceAB[xShift=-8,yShift=4](B)(E){MW}
\psGetDistanceAB[fontscale=15,xShift=4,decimals=0](A)(C){MAC}
\psGetDistanceAB[xShift=-17,decimals=2](E)(C){MEC}
\end{pspicture}

\begin{lstlisting}
\begin{pspicture}(-1,0)(11,8)
\psgrid[gridlabels=0pt,subgriddiv=2,gridwidth=0.4pt,subgridwidth=0.2pt,gridcolor=black!60,subgridcolor=black!40]
\def\sideC{6}  \def\sideA{7}  \def\sideB{8}
\psset{PointSymbol=none,linejoin=1,linewidth=0.4pt,PtNameMath=false,labelsep=0.07,MarkAngleRadius=1.1,decimals=1,comma}
\pstGeonode[PosAngle={-90,-90}](0,0){A}(\sideC;10){B}
\pstInterCC[RadiusA=\pstDistVal{\sideB},RadiusB=\pstDistVal{\sideA},PosAngle=90,PointNameA=C]{A}{}{B}{}{C}{C-}
\pstInterCC[RadiusA=\pstDistAB{A}{B},RadiusB=\pstDistAB{B}{C}]{C}{}{A}{}{D-}{D}
\pstInterLC[Radius=\pstDistAB{A}{C}]{C}{D}{C}{}{A'-}{A'}
\pstInterCC[RadiusA=\pstDistAB{A}{B},RadiusB=\pstDistAB{B}{C}]{A'}{}{C}{}{B'}{B'-}
\pstInterLL[PosAngle=90,PointName=default]{B'}{C}{A}{B}{E}
\pspolygon(A)(B)(C)
\pspolygon[fillstyle=solid,fillcolor=magenta,opacity=0.1](C)(E)(B)
%
\psGetAngleABC[ArcColor=blue,AngleValue=true,LabelSep=0.8,arrows=->,decimals=0,PSfont=Palatino-Roman](B)(A)(C){}
\psGetAngleABC[AngleValue=true,ArcColor=red,arrows=->,WedgeOpacity=0.6,WedgeColor=yellow!30,LabelSep=0.5](C)(B)(A){$\beta$}
\psGetAngleABC[LabelSep=0.8,WedgeColor=green,xShift=-6,yShift=-10](A)(C)(B){$\gamma$}
\psGetAngleABC[LabelSep=0.8,AngleArc=false,WedgeColor=green,arrows=->,xShift=-15,yShift=0](C)(E)(B){\color{blue}$\gamma$}
\psGetAngleABC[AngleValue=true,MarkAngleRadius=1.0,LabelSep=0.5,ShowWedge=false,xShift=-5,yShift=7,arrows=->](E)(B)(C){}
%
\pcline[linestyle=none](A)(B)\nbput{\sideC}
\pcline[linestyle=none](C)(B)\naput{\sideA}
\psGetDistanceAB[xShift=-8,yShift=4](B)(E){MW}
\psGetDistanceAB[fontscale=15,xShift=4,decimals=0](A)(C){MAC}
\psGetDistanceAB[xShift=-17,decimals=2](E)(C){MEC}
\end{pspicture}
\end{lstlisting}

\clearpage



\addtocontents{toc}{\protect\newpage}

\part{Examples gallery}
\appendix
\section{Basic geometry}

\subsection{Drawing of the bissector}

\begin{LTXexample}[width=5cm,pos=l]
\begin{pspicture}[showgrid](-1,-1)(4.4,5)
\psset{PointSymbol=none,PointName=none}
\pstGeonode[PosAngle={180,130,-90},PointSymbol={*,none},
  PointName=default](2,0){B}(0,1){O}(1,4){A}
\pstLineAB[nodesepB=-1,linecolor=red]{O}{A}
\pstLineAB[nodesepB=-1,linecolor=red]{O}{B}
\pstInterLC[PosAngleB=-45]{O}{B}{O}{A}{G}{C}
\psset{arcsepA=-1, arcsepB=-1}
\pstArcOAB[linecolor=green,linestyle=dashed]{O}{C}{A}
\pstInterCC[PosAngleA=100]{A}{O}{C}{O}{O'}{OO}
\pstArcOAB[linecolor=blue,linestyle=dashed]{A}{O'}{O'}
\pstArcOAB[linecolor=blue,linestyle=dashed]{C}{O'}{O'}
\pstLineAB[nodesepB=-1,linecolor=cyan]{O}{O'}
\psset{arcsep=1pt,linecolor=magenta,Mark=MarkHash}
\pstMarkAngle{C}{O}{O'}{}
\pstMarkAngle[MarkAngleRadius=.5]{O'}{O}{A}{}
\end{pspicture}
\end{LTXexample}


\newpage

\subsection{Transformation de polygones et courbes}

Here is an example of the use of \Lkeyword{CurveType} with transformation.

\begin{LTXexample}
\begin{pspicture}(-5,-5)(10,5)
\pstGeonode{O}
\rput(-3,0){\pstGeonode[CurveType=polygon](1,0){A}(1;51.43){B}(1;102.86){C}
  (1;154.29){D}(1;205.71){E}(1;257.14){F}(1;308.57){G}}
\rput(-4,-1){\pstGeonode[CurveType=curve](1,3){M}(4,5){N}(6,2){P}(8,5){Q}}
\pstRotation[linecolor=green, RotAngle=100, CurveType=polygon]{O}{A, B, C, D, E, F, G}
\pstHomO[linecolor=red, HomCoef=.3, CurveType=curve]{O}{M,N,P,Q}
\pstTranslation[linecolor=blue, CurveType=polygon]{C}{O}{A', B', C', D', E', F', G'}
\pstSymO[linecolor=yellow, CurveType=curve]{O}{M',N',P',Q'}
\pstOrtSym[linecolor=magenta, CurveType=polygon]{Q}{F''}
  {A', B', C', D', E', F', G'}[A''', B''', C''', D''', E''', F''', G''']
\end{pspicture}
\end{LTXexample}

\newpage


\subsection{Triangle lines}


\begin{LTXexample}
\psset{unit=2}
\begin{pspicture}(-3,-2)(3,3)
\psset{PointSymbol=none}
\pstTriangle[PointSymbol=none](-2,-1){A}(1,2){B}(2,0){C}
{ \psset{linestyle=none, PointNameB=none}
  \pstMediatorAB{A}{B}{K}{KP}
  \pstMediatorAB[PosAngleA=-40]{C}{A}{J}{JP}
  \pstMediatorAB[PosAngleA=75]{B}{C}{I}{IP}
}% fin
\pstInterLL[PointSymbol=square, PosAngle=-170]{I}{IP}{J}{JP}{O}
{% encapsulation de modif parametres
  \psset{nodesep=-.8, linecolor=green}
  \pstLineAB{O}{I}\pstLineAB{O}{J}\pstLineAB{O}{K}
}% fin
\pstCircleOA[linecolor=red]{O}{A}
% pour que le symbol de O soit sur et non sous les droites
\psdot[dotstyle=square](O)
% les hauteurs et l'orthocentre
\pstProjection{B}{A}{C}
\pstProjection{B}{C}{A}
\pstProjection{A}{C}{B}
\psset{linecolor=blue}\ncline{A}{A'}\ncline{C}{C'}\ncline{B}{B'}
\pstInterLL[PointSymbol=square]{A}{A'}{B}{B'}{H}
% les medianes et le centre de gravite
\psset{linecolor=magenta}\ncline{A}{I}\ncline{C}{K}\ncline{B}{J}
\pstCGravABC[PointSymbol=square, PosAngle=95]{A}{B}{C}{G}
\end{pspicture}
\end{LTXexample}


\newpage
\subsection{Euler circle}


\begin{LTXexample}
\psset{unit=2}
\begin{pspicture}(-3,-1.5)(3,2.5)
\psset{PointSymbol=none}
\pstTriangle(-2,-1){A}(1,2){B}(2,-1){C}
{% encapsulation de modif parametres
  \psset{linestyle=none, PointSymbolB=none, PointNameB=none}
  \pstMediatorAB{A}{B}{K}{KP}
  \pstMediatorAB{C}{A}{J}{JP}
  \pstMediatorAB{B}{C}{I}{IP}
}% fin
\pstInterLL[PointSymbol=square, PosAngle=-170]{I}{IP}{J}{JP}{O}
{% encapsulation de modif parametres
  \psset{nodesep=-.8, linecolor=green}
  \pstLineAB{O}{I}\pstLineAB{O}{J}\pstLineAB{O}{K}
}% fin
\psdot[dotstyle=square](O)
\pstProjection{B}{A}{C}
\pstProjection{B}{C}{A}
\pstProjection{A}{C}{B}
\psset{linecolor=blue}\ncline{A}{A'}\ncline{C}{C'}\ncline{B}{B'}
\pstInterLL[PointSymbol=square]{A}{A'}{B}{B'}{H}
% le cercle d'Euler (centre au milieu de [OH])
\pstMiddleAB[PointSymbol=o, PointName=\omega]{O}{H}{omega}
\pstCircleOA[linecolor=Orange, linestyle=dashed, dash=5mm 1mm]{omega}{B'}
\psset{PointName=none}
% il passe par le milieu des segments joignant l'orthocentre et les sommets
\pstMiddleAB{H}{A}{AH}\pstMiddleAB{H}{B}{BH}\pstMiddleAB{H}{C}{CH}
\pstSegmentMark{H}{AH}\pstSegmentMark{AH}{A}
\psset{SegmentSymbol=wedge}\pstSegmentMark{H}{BH}\pstSegmentMark{BH}{B}
\psset{SegmentSymbol=cup}\pstSegmentMark{H}{CH}\pstSegmentMark{CH}{C}
\end{pspicture}
\end{LTXexample}

\newpage
\subsection{Orthocenter and hyperbola}

The orthocenter of a triangle whose points are on the branches of the
hyperbola ${\mathscr H} : y=a/x$ belong to this hyperbola.

\begin{LTXexample}
\psset{unit=0.7}
\begin{pspicture}(-11,-5)(11,7)
\psset{linecolor=blue, linewidth=2\pslinewidth}
\psplot[yMaxValue=6,plotpoints=500]{-10}{-.1}{1 x div}
\psplot[yMaxValue=6,plotpoints=500]{.1}{10}{1 x div}
\psset{%PointSymbol=none,
linewidth=.5\pslinewidth}
\pstTriangle[linecolor=magenta, PosAngleB=-85, PosAngleC=-90](.2,5){A}(1,1){B}(10,.1){C}
\psset{linecolor=magenta,CodeFig=true, CodeFigColor=red}
\pstProjection{B}{A}{C}
\ncline[nodesepA=-1,linestyle=dashed,linecolor=magenta]{C'}{B}
\pstProjection{B}{C}{A}
\ncline[nodesepA=-1,linestyle=dashed,linecolor=magenta]{A'}{B}
\pstProjection{A}{C}{B}
\pstInterLL[PosAngle=135,PointSymbol=square]{A}{A'}{B}{B'}{H}
\psset{linecolor=green, nodesep=-1}
\pstLineAB{A}{H}\pstLineAB{B'}{H}\pstLineAB{C}{H}
\psdot[dotstyle=square](H)
\end{pspicture}
\end{LTXexample}


\resetEUCLvalues


\newpage
\subsection{17 sides regular polygon}

Striking picture created by K. F. Gauss.
he also prooved that it is possible to build the regular polygons which
have $2^{2^p}+1$ sides, the following one has 257 sides!


\begin{pspicture}(-5.5,-5.5)(5.5,6)
  \psset{CodeFig, RightAngleSize=.14, CodeFigColor=red,
     CodeFigB=true, linestyle=dashed, dash=2mm 2mm}
  \pstGeonode[PosAngle={-90,0}]{O}(5;0){P_1}
  \pstCircleOA{O}{P_1}
  \pstSymO[PointSymbol=none, PointName=none, CodeFig=false]{O}{P_1}[PP_1]
  \ncline[linestyle=solid]{PP_1}{P_1}
  \pstRotation[RotAngle=90, PosAngle=90]{O}{P_1}[B]
  \pstRightAngle[linestyle=solid]{B}{O}{PP_1}\ncline[linestyle=solid]{O}{B}
  \pstHomO[HomCoef=.25]{O}{B}[J]  \ncline{J}{P_1}
  \pstBissectBAC[PointSymbol=none, PointName=none]{O}{J}{P_1}{PE1}
  \pstBissectBAC[PointSymbol=none, PointName=none]{O}{J}{PE1}{PE2}
  \pstInterLL[PosAngle=-90]{O}{P_1}{J}{PE2}{E}
  \pstRotation[PosAngle=-90, RotAngle=-45, PointSymbol=none, PointName=none]{J}{E}[PF1]
  \pstInterLL[PosAngle=-90]{O}{P_1}{J}{PF1}{F}
  \pstMiddleAB[PointSymbol=none, PointName=none]{F}{P_1}{MFP1}  \pstCircleOA{MFP1}{P_1}
  \pstInterLC[%PointSymbolA=none, PointNameA=none
  ]{O}{B}{MFP1}{P_1}{H}{K}
  \pstCircleOA{E}{K}    \pstInterLC{O}{P_1}{E}{K}{N_6}{N_4}
  \pstRotation[RotAngle=90,PointSymbol=none, PointName=none]{N_6}{E}[PP_6]
  \pstInterLC[PosAngleA=90,PosAngleB=-90, PointNameB=P_{13}]{PP_6}{N_6}{O}{P_1}{P_6}{P_13}
  \pstSegmentMark[SegmentSymbol=wedge]{N_6}{P_6}
  \pstSegmentMark[SegmentSymbol=wedge]{P_13}{N_6}
  \pstRotation[RotAngle=90,PointSymbol=none, PointName=none]{N_4}{E}[PP_4]
  \pstInterLC[PosAngleA=90,PosAngleB=-90,PointNameB=P_{15}]{N_4}{PP_4}{O}{P_1}{P_4}{P_15}
  \pstSegmentMark[SegmentSymbol=cup]{N_4}{P_4}
  \pstSegmentMark[SegmentSymbol=cup]{P_15}{N_4}
  \pstRightAngle[linestyle=solid]{P_1}{N_6}{P_6}
  \pstRightAngle[linestyle=solid]{P_1}{N_4}{P_4}
  \pstBissectBAC[PosAngle=90, linestyle=none]{P_4}{O}{P_6}{P_5}
  \pstInterCC[PosAngleB=90, PointSymbolA=none, PointNameA=none]{O}{P_1}{P_4}{P_5}{H}{P_3}
  \pstInterCC[PosAngleB=90, PointSymbolA=none, PointNameA=none]{O}{P_1}{P_3}{P_4}{H}{P_2}
  \pstInterCC[PosAngleA=90, PointSymbolB=none, PointNameB=none]{O}{P_1}{P_6}{P_5}{P_7}{H}
  \pstInterCC[PosAngleA=100, PointSymbolB=none, PointNameB=none]{O}{P_1}{P_7}{P_6}{P_8}{H}
  \pstInterCC[PosAngleA=135, PointSymbolB=none, PointNameB=none]{O}{P_1}{P_8}{P_7}{P_9}{H}
  \pstOrtSym[PosAngle={-90,-90,-90,-100,-135},PointName={P_{17},P_{16},P_{14},P_{12},P_{11},P_{10}}]
             {O}{P_1}{P_2,P_3,P_5,P_7,P_8,P_9}[P_17,P_16,P_14,P_12,P_11,P_10]
  \pspolygon[linecolor=green, linestyle=solid, linewidth=2\pslinewidth]
    (P_1)(P_2)(P_3)(P_4)(P_5)(P_6)(P_7)(P_8)(P_9)(P_10)(P_11)(P_12)(P_13)(P_14)(P_15)(P_16)(P_17)
\end{pspicture}


\begin{lstlisting}
\begin{pspicture}(-5.5,-5.5)(5.5,6)
  \psset{CodeFig, RightAngleSize=.14, CodeFigColor=red,
     CodeFigB=true, linestyle=dashed, dash=2mm 2mm}
  \pstGeonode[PosAngle={-90,0}]{O}(5;0){P_1}
  \pstCircleOA{O}{P_1}
  \pstSymO[PointSymbol=none, PointName=none, CodeFig=false]{O}{P_1}[PP_1]
  \ncline[linestyle=solid]{PP_1}{P_1}
  \pstRotation[RotAngle=90, PosAngle=90]{O}{P_1}[B]
  \pstRightAngle[linestyle=solid]{B}{O}{PP_1}\ncline[linestyle=solid]{O}{B}
  \pstHomO[HomCoef=.25]{O}{B}[J]  \ncline{J}{P_1}
  \pstBissectBAC[PointSymbol=none, PointName=none]{O}{J}{P_1}{PE1}
  \pstBissectBAC[PointSymbol=none, PointName=none]{O}{J}{PE1}{PE2}
  \pstInterLL[PosAngle=-90]{O}{P_1}{J}{PE2}{E}
  \pstRotation[PosAngle=-90, RotAngle=-45, PointSymbol=none, PointName=none]{J}{E}[PF1]
  \pstInterLL[PosAngle=-90]{O}{P_1}{J}{PF1}{F}
  \pstMiddleAB[PointSymbol=none, PointName=none]{F}{P_1}{MFP1}  \pstCircleOA{MFP1}{P_1}
  \pstInterLC[PointSymbolA=none, PointNameA=none]{O}{B}{MFP1}{P_1}{H}{K}
  \pstCircleOA{E}{K}    \pstInterLC{O}{P_1}{E}{K}{N_6}{N_4}
  \pstRotation[RotAngle=90,PointSymbol=none, PointName=none]{N_6}{E}[PP_6]
  \pstInterLC[PosAngleA=90,PosAngleB=-90, PointNameB=P_{13}]{PP_6}{N_6}{O}{P_1}{P_6}{P_13}
  \pstSegmentMark[SegmentSymbol=wedge]{N_6}{P_6}
  \pstSegmentMark[SegmentSymbol=wedge]{P_13}{N_6}
  \pstRotation[RotAngle=90,PointSymbol=none, PointName=none]{N_4}{E}[PP_4]
  \pstInterLC[PosAngleA=90,PosAngleB=-90,PointNameB=P_{15}]{N_4}{PP_4}{O}{P_1}{P_4}{P_15}
  \pstSegmentMark[SegmentSymbol=cup]{N_4}{P_4}
  \pstSegmentMark[SegmentSymbol=cup]{P_15}{N_4}
  \pstRightAngle[linestyle=solid]{P_1}{N_6}{P_6}
  \pstRightAngle[linestyle=solid]{P_1}{N_4}{P_4}
  \pstBissectBAC[PosAngle=90, linestyle=none]{P_4}{O}{P_6}{P_5}
  \pstInterCC[PosAngleB=90, PointSymbolA=none, PointNameA=none]{O}{P_1}{P_4}{P_5}{H}{P_3}
  \pstInterCC[PosAngleB=90, PointSymbolA=none, PointNameA=none]{O}{P_1}{P_3}{P_4}{H}{P_2}
  \pstInterCC[PosAngleA=90, PointSymbolB=none, PointNameB=none]{O}{P_1}{P_6}{P_5}{P_7}{H}
  \pstInterCC[PosAngleA=100, PointSymbolB=none, PointNameB=none]{O}{P_1}{P_7}{P_6}{P_8}{H}
  \pstInterCC[PosAngleA=135, PointSymbolB=none, PointNameB=none]{O}{P_1}{P_8}{P_7}{P_9}{H}
  \pstOrtSym[PosAngle={-90,-90,-90,-100,-135},PointName={P_{17},P_{16},P_{14},P_{12},P_{11},P_{10}}]
             {O}{P_1}{P_2,P_3,P_5,P_7,P_8,P_9}[P_17,P_16,P_14,P_12,P_11,P_10]
  \pspolygon[linecolor=green, linestyle=solid, linewidth=2\pslinewidth]
    (P_1)(P_2)(P_3)(P_4)(P_5)(P_6)(P_7)(P_8)(P_9)(P_10)(P_11)(P_12)(P_13)(P_14)(P_15)(P_16)(P_17)
\end{pspicture}
\end{lstlisting}


\newpage
\subsection{Circles \& tangents}

The drawing of the circle tangents which crosses a given point.

\begin{LTXexample}
\begin{pspicture}(15,10)
\pstGeonode(5, 5){O}(14,2){M}
\pstCircleOA[Radius=\pstDistVal{4}]{O}{}
\pstMiddleAB[PointSymbol=none, PointName=none]{O}{M}{O'}
\pstInterCC[RadiusA=\pstDistVal{4}, DiameterB=\pstDistAB{O}{M},
            CodeFigB=true, CodeFigColor=magenta, PosAngleB=45]{O}{}{O'}{}{A}{B}
\psset{linecolor=red, linewidth=1.3\pslinewidth, nodesep=-2}
\pstLineAB{M}{A}\pstLineAB{M}{B}
\end{pspicture}
\end{LTXexample}


\begin{LTXexample}
\begin{pspicture}(-2,0)(13,9)
\pstGeonode(9,3){O}(3,6){O'}\psset{PointSymbol=none, PointName=none}
\pstCircleOA[Radius=\pstDistVal{3}]{O}{}\pstCircleOA[Radius=\pstDistVal{1}]{O'}{}
\pstInterLC[Radius=\pstDistVal{3}]{O}{O'}{O}{}{M}{toto}
\pstInterLC[Radius=\pstDistVal{1}]{O}{O'}{O'}{}{M'}{toto}
\pstRotation[RotAngle=30]{O}{M}[N]
\pstRotation[RotAngle=30]{O'}{M'}[N']
\pstInterLL[PointSymbol=*, PointName=\Omega]{O}{O'}{N}{N'}{Omega}
\pstMiddleAB{O}{Omega}{I}   \pstInterCC{I}{O}{O}{M}{A}{B}
\psset{nodesepA=-1, nodesepB=-3, linecolor=blue, linewidth=1.3\pslinewidth}
\pstLineAB[nodesep=-2]{A}{Omega}\pstLineAB[nodesep=-2]{B}{Omega}
\pstRotation[RotAngle=-150]{O'}{M'}[N'']
\pstInterLL[PointSymbol=*, PointName=\Omega']{O}{O'}{N}{N''}{Omega'}
\pstMiddleAB{O}{Omega'}{J}
\pstInterCC{J}{O}{O}{M}{A'}{B'}
\psset{nodesepA=-1, nodesepB=-3, linecolor=red}
\pstLineAB{A'}{Omega'}\pstLineAB{B'}{Omega'}
\end{pspicture}
\end{LTXexample}


\newpage
\subsection{Fermat's point}

Drawing of Manuel Luque.

\begin{LTXexample}
\begin{pspicture}(-7,-6)(5,5)
\psset{PointSymbol=none, PointName=none}
\pstTriangle[PosAngleA=-160,PosAngleB=90,PosAngleC=-25](-3,-2){B}(0,3){A}(2,-1){C}%
\psset{RotAngle=-60}
\pstRotation[PosAngle=-90]{B}{C}[A']
\pstRotation{C}{A}[B']
\pstRotation[PosAngle=160]{A}{B}[C']
\pstLineAB{A}{B'}
\pstLineAB{C}{B'}
\pstLineAB{B}{A'}
\pstLineAB{C}{A'}
\pstLineAB{B}{C'}
\pstLineAB{A}{C'}
\pstCircleABC[linecolor=red]{A}{B}{C'}{O_1}
\pstCircleABC[linecolor=blue]{A}{C}{B'}{O_2}
\pstCircleABC[linecolor=Aquamarine]{A'}{C}{B}{O_3}
\pstInterCC[PointSymbolA=none]{O_1}{A}{O_2}{A}{E}{F}
\end{pspicture}
\end{LTXexample}

\newpage
\subsection{Escribed and inscribed circles of a triangle}

%% cercles inscrit et exinscrits d'un triangle


\begin{pspicture}(-6,-5)(11,15)
\psset{PointSymbol=none}
\pstTriangle[linewidth=2\pslinewidth,linecolor=red](4,1){A}(0,3){B}(5,5){C}
\psset{linecolor=blue}
\pstBissectBAC[PointSymbol=none,PointName=none]{C}{A}{B}{AB}
\pstBissectBAC[PointSymbol=none,PointName=none]{A}{B}{C}{BB}
\pstBissectBAC[PointSymbol=none,PointName=none]{B}{C}{A}{CB}
\pstInterLL{A}{AB}{B}{BB}{I}
\psset{linecolor=magenta, linestyle=dashed}  \pstProjection{A}{B}{I}[I_C]
\pstLineAB{I}{I_C}\pstRightAngle[linestyle=solid]{A}{I_C}{I}
\pstProjection{A}{C}{I}[I_B]
\pstLineAB{I}{I_B}\pstRightAngle[linestyle=solid]{C}{I_B}{I}
\pstProjection[PosAngle=80]{C}{B}{I}[I_A]
\pstLineAB{I}{IA}\pstRightAngle[linestyle=solid]{B}{I_A}{I}
\pstCircleOA[linecolor=yellow, linestyle=solid]{I}{I_A}
\psset{linecolor=magenta, linestyle=none}
\pstOutBissectBAC[PointSymbol=none,PointName=none]{C}{A}{B}{AOB}
\pstOutBissectBAC[PointSymbol=none,PointName=none]{A}{B}{C}{BOB}
\pstOutBissectBAC[PointSymbol=none,PointName=none]{B}{C}{A}{COB}
\pstInterLL[PosAngle=-90]{A}{AOB}{B}{BOB}{I_1} \pstInterLL{A}{AOB}{C}{COB}{I_2}
\pstInterLL[PosAngle=90]{C}{COB}{B}{BOB}{I_3}  \psset{linecolor=magenta, linestyle=dashed}
\pstProjection[PointName=I_{1C}]{A}{B}{I_1}[I1C]
\pstLineAB{I_1}{I1C}\pstRightAngle[linestyle=solid]{I_1}{I1C}{A}
\pstProjection[PointName=I_{1B}]{A}{C}{I_1}[I1B]
\pstLineAB{I_1}{I1B}\pstRightAngle[linestyle=solid]{A}{I1B}{I_1}
\pstProjection[PointName=I_{1A}]{C}{B}{I_1}[I1A]
\pstLineAB{I_1}{I1A}\pstRightAngle[linestyle=solid]{I_1}{I1A}{C}
\pstProjection[PointName=I_{2B}]{A}{C}{I_2}[I2B]
\pstLineAB{I_2}{I2B}\pstRightAngle[linestyle=solid]{A}{I2B}{I_2}
\pstProjection[PointName=I_{2C}]{A}{B}{I_2}[I2C]
\pstLineAB{I_2}{I2C}\pstRightAngle[linestyle=solid]{I_2}{I2C}{A}
\pstProjection[PointName=I_{2A}]{B}{C}{I_2}[I2A]
\pstLineAB{I_2}{I2A}\pstRightAngle[linestyle=solid]{C}{I2A}{I_2}
\pstProjection[PointName=I_{3A}]{C}{B}{I_3}[I3A]
\pstLineAB{I_3}{I3A}\pstRightAngle[linestyle=solid]{C}{I3A}{I_3}
\pstProjection[PointName=I_{3C}]{A}{B}{I_3}[I3C]
\pstLineAB{I_3}{I3C}\pstRightAngle[linestyle=solid]{A}{I3C}{I_3}
\pstProjection[PointName=I_{3B}]{C}{A}{I_3}[I3B]
\pstLineAB{I_3}{I3B}\pstRightAngle[linestyle=solid]{I_3}{I3B}{A}
\psset{linecolor=black!40, linestyle=dashed}
\pstCircleOA{I_1}{I1C} \pstCircleOA{I_2}{I2B} \pstCircleOA{I_3}{I3A}
\psset{linecolor=red, linestyle=solid, nodesepA=-1, nodesepB=-1}
\pstLineAB{I1B}{I3B}\pstLineAB{I1A}{I2A}\pstLineAB{I2C}{I3C}
\end{pspicture}


\begin{lstlisting}
\begin{pspicture}(-6,-5)(11,15)
\psset{PointSymbol=none}
\pstTriangle[linewidth=2\pslinewidth,linecolor=red](4,1){A}(0,3){B}(5,5){C}
\psset{linecolor=blue}
\pstBissectBAC[PointSymbol=none,PointName=none]{C}{A}{B}{AB}
\pstBissectBAC[PointSymbol=none,PointName=none]{A}{B}{C}{BB}
\pstBissectBAC[PointSymbol=none,PointName=none]{B}{C}{A}{CB}
\pstInterLL{A}{AB}{B}{BB}{I}
\psset{linecolor=magenta, linestyle=dashed}
\pstProjection{A}{B}{I}[I_C]
\pstLineAB{I}{I_C}\pstRightAngle[linestyle=solid]{A}{I_C}{I}
\pstProjection{A}{C}{I}[I_B]
\pstLineAB{I}{I_B}\pstRightAngle[linestyle=solid]{C}{I_B}{I}
\pstProjection[PosAngle=80]{C}{B}{I}[I_A]
\pstLineAB{I}{IA}\pstRightAngle[linestyle=solid]{B}{I_A}{I}
\pstCircleOA[linecolor=yellow, linestyle=solid]{I}{I_A}
\psset{linecolor=magenta, linestyle=none}
\pstOutBissectBAC[PointSymbol=none,PointName=none]{C}{A}{B}{AOB}
\pstOutBissectBAC[PointSymbol=none,PointName=none]{A}{B}{C}{BOB}
\pstOutBissectBAC[PointSymbol=none,PointName=none]{B}{C}{A}{COB}
\pstInterLL[PosAngle=-90]{A}{AOB}{B}{BOB}{I_1}
\pstInterLL{A}{AOB}{C}{COB}{I_2}
\pstInterLL[PosAngle=90]{C}{COB}{B}{BOB}{I_3}
\psset{linecolor=magenta, linestyle=dashed}
\pstProjection[PointName=I_{1C}]{A}{B}{I_1}[I1C]
\pstLineAB{I_1}{I1C}\pstRightAngle[linestyle=solid]{I_1}{I1C}{A}
\pstProjection[PointName=I_{1B}]{A}{C}{I_1}[I1B]
\pstLineAB{I_1}{I1B}\pstRightAngle[linestyle=solid]{A}{I1B}{I_1}
\pstProjection[PointName=I_{1A}]{C}{B}{I_1}[I1A]
\pstLineAB{I_1}{I1A}\pstRightAngle[linestyle=solid]{I_1}{I1A}{C}
\pstProjection[PointName=I_{2B}]{A}{C}{I_2}[I2B]
\pstLineAB{I_2}{I2B}\pstRightAngle[linestyle=solid]{A}{I2B}{I_2}
\pstProjection[PointName=I_{2C}]{A}{B}{I_2}[I2C]
\pstLineAB{I_2}{I2C}\pstRightAngle[linestyle=solid]{I_2}{I2C}{A}
\pstProjection[PointName=I_{2A}]{B}{C}{I_2}[I2A]
\pstLineAB{I_2}{I2A}\pstRightAngle[linestyle=solid]{C}{I2A}{I_2}
\pstProjection[PointName=I_{3A}]{C}{B}{I_3}[I3A]
\pstLineAB{I_3}{I3A}\pstRightAngle[linestyle=solid]{C}{I3A}{I_3}
\pstProjection[PointName=I_{3C}]{A}{B}{I_3}[I3C]
\pstLineAB{I_3}{I3C}\pstRightAngle[linestyle=solid]{A}{I3C}{I_3}
\pstProjection[PointName=I_{3B}]{C}{A}{I_3}[I3B]
\pstLineAB{I_3}{I3B}\pstRightAngle[linestyle=solid]{I_3}{I3B}{A}
\psset{linecolor=yellow, linestyle=solid}
\pstCircleOA{I_1}{I1C} \pstCircleOA{I_2}{I2B} \pstCircleOA{I_3}{I3A}
\psset{linecolor=red, linestyle=solid, nodesepA=-1, nodesepB=-1}
\pstLineAB{I1B}{I3B}\pstLineAB{I1A}{I2A}\pstLineAB{I2C}{I3C}
\end{pspicture}
\end{lstlisting}



\newpage
\section{Some locus points}

\subsection{Parabola}

The parabola is the set of points which are at the same distance
between a point and a line.


\begin{LTXexample}
\def\NbPt{11}
\begin{pspicture}(-0.5,0)(11,10)
\psset{linewidth=1.2\pslinewidth}\renewcommand{\NbPt}{11}
\pstGeonode[PosAngle={0,-90}](5,4){O}(1,2){A}(9,1.5){B}
\newcommand\Parabole[1][100]{%
  \pstLineAB[nodesep=-.9, linecolor=green]{A}{B}
  \psset{RotAngle=90, PointSymbol=none, PointName=none}
  \multido{\n=1+1}{\NbPt}{%
    \pstHomO[HomCoef=\n\space \NbPt\space 1 add div]{A}{B}[M\n]
    \pstMediatorAB[linestyle=none]{M\n}{O}{M\n_I}{M\n_IP}
    \pstRotation{M\n}{A}[M\n_P]
    \pstInterLL[PointSymbol=square, PointName=none]{M\n_I}{M\n_IP}{M\n}{M\n_P}{P_\n}
    \ifnum\n=#1 \bgroup
      \pstRightAngle{A}{M\n}{M\n_P}
      \psset{linewidth=.5\pslinewidth, nodesep=-1, linecolor=blue}
      \pstLineAB{M\n_I}{P_\n}\pstLineAB{M\n}{P_\n}
      \pstRightAngle{P_\n}{M\n_I}{M\n}
      \psset{linecolor=red}\pstSegmentMark{M\n}{M\n_I}\pstSegmentMark{M\n_I}{O}
      \egroup \fi}}
\Parabole[2]\pstGenericCurve[linecolor=magenta]{P_}{1}{\NbPt}
\pstGeonode[PointSymbol=*, PosAngle=-90](10,3.5){B}
\Parabole\pstGenericCurve[linecolor=magenta,linestyle=dashed]{P_}{1}{\NbPt}
\end{pspicture}
\end{LTXexample}

\newpage
\subsection{Hyperbola}

The hyperbola is the set of points whose difference between their
distance of two points (the focus) is constant.

\iffalse
\begin{verbatim}
%% QQ RAPPELS : a=\Sommet, c=\PosFoyer,
%% b^2=c^2-a^2, e=c/a
%% pour une hyperbole -> e>1, donc c>a,
%% ici on choisi a=\sqrt{2}, c=2, e=\sqrt{2}
%% M est sur H <=> |MF-MF'|=2a
\end{verbatim}
\fi

\begin{LTXexample}
\begin{pspicture}[showgrid](-4,-4)(4,4)
\newcommand\Sommet{1.4142135623730951 } \newcounter{i} \setcounter{i}{1}
\newcommand\PosFoyer{2 } \newcommand\HypAngle{0}
\newcounter{CoefDiv}\setcounter{CoefDiv}{20}
\newcounter{Inc}\setcounter{Inc}{2} \newcounter{n}\setcounter{n}{2}
\newcommand\Ri{ \PosFoyer \Sommet sub \arabic{i}\space\arabic{CoefDiv}\space div add }
\newcommand\Rii{\Ri \Sommet 2 mul add }
\pstGeonode[PosAngle=90]{O}(\PosFoyer;\HypAngle){F}
\pstSymO[PosAngle=180]{O}{F}\pstLineAB{F}{F'}  \pstCircleOA{O}{F}
\pstGeonode[PosAngle=-135](\Sommet;\HypAngle){S}
\pstGeonode[PosAngle=-45](-\Sommet;\HypAngle){S'}
\pstRotation[RotAngle=90, PointSymbol=none]{S}{O}[B]
\pstInterLC[PosAngleA=90, PosAngleB=-90]{S}{B}{O}{F}{A_1}{A_2}
\pstLineAB[nodesepA=-3,nodesepB=-5]{A_1}{O}\pstLineAB[nodesepA=-3,nodesepB=-5]{A_2}{O}
\pstMarkAngle[LabelSep=.8,MarkAngleRadius=.7,arrows=->,LabelSep=1.1]{F}{O}{A_1}{$\Psi$}
\ncline[linecolor=red]{A_1}{A_2}   \pstRightAngle[RightAngleSize=.15]{A_1}{S}{O}
\psset{PointName=none}
\whiledo{\value{n}<8}{%
  \psset{RadiusA=\pstDistVal{\Ri},RadiusB=\pstDistVal{\Rii},PointSymbol=none}
  \pstInterCC{F}{}{F'}{}{M\arabic{n}}{P\arabic{n}}
  \pstInterCC{F'}{}{F}{}{M'\arabic{n}}{P'\arabic{n}}
  \stepcounter{n}\addtocounter{i}{\value{Inc}}
  \addtocounter{Inc}{\value{Inc}}}%% fin de whiledo
\psset{linecolor=blue}
\pstGenericCurve[GenCurvFirst=S]{M}{2}{7}
\pstGenericCurve[GenCurvFirst=S]{P}{2}{7}
\pstGenericCurve[GenCurvFirst=S']{M'}{2}{7}
\pstGenericCurve[GenCurvFirst=S']{P'}{2}{7}
\end{pspicture}
\end{LTXexample}



 \subsection{Cycloid}

The wheel rolls from $M$ to $A$. The circle points are on a
cycloid.


\begin{LTXexample}
\begin{pspicture}[showgrid](-2,-1)(13,3)
\providecommand\NbPt{11}
\psset{linewidth=1.2\pslinewidth}
\pstGeonode[PointSymbol={*,none}, PointName={default,none}, PosAngle=180]{M}(0,1){O}
\pstGeonode(12.5663706144,0){A}
\pstTranslation[PointSymbol=none, PointName=none]{M}{A}{O}[B]
\multido{\nA=1+1}{\NbPt}{%
  \pstHomO[HomCoef=\nA\space\NbPt\space 1 add div,PointSymbol=none,PointName=none]{O}{B}[O\nA]
  \pstProjection[PointSymbol=none, PointName=none]{M}{A}{O\nA}[P\nA]
  \pstCurvAbsNode[PointSymbol=square, PointName=none,CurvAbsNeg=true]%
    {O\nA}{P\nA}{M\nA}{\pstDistAB{O}{O\nA}}
  \ifnum\nA=2 \bgroup
    \pstCircleOA{O\nA}{M\nA}
    \psset{linecolor=magenta, linewidth=1.5\pslinewidth}
    \pstArcnOAB{O\nA}{P\nA}{M\nA}
    \ncline{O\nA}{M\nA}\ncline{P\nA}{M}
    \egroup \fi
  }% fin du multido
\psset{linecolor=blue, linewidth=1.5\pslinewidth}
\pstGenericCurve[GenCurvFirst=M]{M}{1}{6} \pstGenericCurve[GenCurvLast=A]{M}{6}{\NbPt}
\end{pspicture}
\end{LTXexample}

\newpage
\subsection{Hypocycloids (Astroid and Deltoid)}

A wheel rolls inside a circle, and depending of the radius ratio, it
is an astroid, a deltoid and in the general case hypo-cycloids.



\begin{LTXexample}
\newcommand\HypoCyclo[4][100]{%
  \def\R{#2}\def\petitR{#3}\def\NbPt{#4}
  \def\Anglen{\n\space 360 \NbPt\space 1 add div mul}
  \psset{PointSymbol=none,PointName=none}
  \pstGeonode[PointSymbol={*,none},PointName={default,none}, PosAngle=0]{O}(\R;0){P}
  \pstCircleOA{O}{P}
  \pstHomO[HomCoef=\petitR\space\R\space div]{P}{O}[M]
  \multido{\n=1+1}{\NbPt}{%
    \pstRotation[RotAngle=\Anglen]{O}{M}[M\n]
    \rput(M\n){\pstGeonode(\petitR;0){Q}}
    \pstRotation[RotAngle=\Anglen]{M\n}{Q}[N]
    \pstRotation[RotAngle=\n\space -360 \NbPt\space 1 add div
    mul \R\space\petitR\space div mul,PointSymbol=*,PointName=none]{M\n}{N}[N\n]
    \ifnum\n=#1
      \pstCircleOA{M\n}{N\n}\ncline{M\n}{N\n}%
      {\psset{linecolor=red, linewidth=2\pslinewidth}
      \pstArcOAB{M\n}{N\n}{N}\pstArcOAB{O}{P}{N}}
    \fi}}%fin multido-newcommand
\begin{pspicture}[showgrid](-3.5,-3.4)(3.5,4)
\HypoCyclo[3]{3}{1}{17}
\psset{linecolor=blue,linewidth=1.5\pslinewidth}
\pstGenericCurve[GenCurvFirst=P]{N}{1}{6}
\pstGenericCurve{N}{6}{12}
\pstGenericCurve[GenCurvLast=P]{N}{12}{17}
\end{pspicture}
\end{LTXexample}



\begin{LTXexample}
\newcommand\HypoCyclo[4][100]{%
  \def\R{#2}\def\petitR{#3}\def\NbPt{#4}
  \def\Anglen{\n\space 360 \NbPt\space 1 add div mul}
  \psset{PointSymbol=none,PointName=none}
  \pstGeonode[PointSymbol={*,none},PointName={default,none}, PosAngle=0]{O}(\R;0){P}
  \pstCircleOA{O}{P}
  \pstHomO[HomCoef=\petitR\space\R\space div]{P}{O}[M]
  \multido{\n=1+1}{\NbPt}{%
    \pstRotation[RotAngle=\Anglen]{O}{M}[M\n]
    \rput(M\n){\pstGeonode(\petitR;0){Q}}
    \pstRotation[RotAngle=\Anglen]{M\n}{Q}[N]
    \pstRotation[RotAngle=\n\space -360 \NbPt\space 1 add div
    mul \R\space\petitR\space div mul, PointSymbol=*, PointName=none]{M\n}{N}[N\n]
    \ifnum\n=#1
      \pstCircleOA{M\n}{N\n}\ncline{M\n}{N\n}%
      {\psset{linecolor=red, linewidth=2\pslinewidth}
      \pstArcOAB{M\n}{N\n}{N}\pstArcOAB{O}{P}{N}}
    \fi}}%fin multido-newcommand
\begin{pspicture}(-4.5,-4)(4.5,4.5)
\HypoCyclo[4]{4}{1}{27}
\psset{linecolor=blue, linewidth=1.5\pslinewidth}
\pstGenericCurve[GenCurvFirst=P]{N}{1}{7}
\pstGenericCurve{N}{7}{14}\pstGenericCurve{N}{14}{21}
\pstGenericCurve[GenCurvLast=P]{N}{21}{27}
\end{pspicture}
\end{LTXexample}


\newpage
 \section{Lines and circles envelope}

\subsection{Conics}

Let's consider a circle and a point $A$ not on the circle. The
set of all the mediator lines of segments defined by $A$ and the
circle points, create two conics depending of the position of $A$:

\begin{compactitem}
\item inside the circle: a hyperbola;
\item outside the circle: an ellipse.
\end{compactitem}

(figure of O. Reboux).

\begin{LTXexample}
\begin{pspicture}(-6,-6)(6,6)
\psset{linewidth=0.4\pslinewidth,PointSymbol=none, PointName=none}
\pstGeonode[PosAngle=-90, PointSymbol={none,*,none}, PointName={none,default,none}]
  {O}(4;132){A}(5,0){O'}
\pstCircleOA{O}{O'}
\multido{\n=5+5}{72}{%
  \pstGeonode(5;\n){M_\n}
  \pstMediatorAB[nodesep=-15,linecolor=magenta]
    {A}{M_\n}{I}{J}}% fin multido
\end{pspicture}
\end{LTXexample}

\newpage
\subsection{Cardioid}

The cardioid is defined by the circles centered on a circle and
crossing a given point.

\begin{LTXexample}
\begin{pspicture}(-6,-6)(3,5)
\psset{linewidth=0.4\pslinewidth,PointSymbol=x,nodesep=0,linecolor=magenta}
\pstGeonode[PointName=none]{O}(2,0){O'}
\pstCircleOA[linecolor=black]{O}{O'}
\multido{\n=5+5}{72}{%
  \pstGeonode[PointSymbol=none, PointName=none](2;\n){M_\n}
  \pstCircleOA{M_\n}{O'}}
  \end{pspicture}
\end{LTXexample}


\newpage
  \section{Homotethy and fractals}

\begin{LTXexample}[width=6cm.pos=l]
\begin{pspicture}(-2.8,-3)(2.8,3)
\pstGeonode[PosAngle={0,90}](2,2){A_0}(-2,2){B_0}%
\psset{RotAngle=90}
\pstRotation[PosAngle=270]{A_0}{B_0}[D_0]
\pstRotation[PosAngle=180]{D_0}{A_0}[C_0]
\pspolygon(A_0)(B_0)(C_0)(D_0)%
\psset{PointSymbol=none, PointName=none, HomCoef=.2}
\multido{\n=1+1,\i=0+1}{20}{%
  \pstHomO[PosAngle=0]{B_\i}{A_\i}[A_\n]
  \pstHomO[PosAngle=90]{C_\i}{B_\i}[B_\n]
  \pstHomO[PosAngle=180]{D_\i}{C_\i}[C_\n]
  \pstHomO[PosAngle=270]{A_\i}{D_\i}[D_\n]
  \pspolygon(A_\n)(B_\n)(C_\n)(D_\n)}% fin multido
\end{pspicture}
\end{LTXexample}

\newpage
\section{hyperbolic geometry: a triangle and its altitudes}

\begin{LTXexample}
\begin{pspicture}(-5,-5)(5,5)
\psclip{\pscircle(0,0){4}}
  \pstGeonode(1, 2){M}\pstGeonode(-2,2){N}\pstGeonode(0,-2){P}
  \psset{DrawCirABC=false, PointSymbol=none, PointName=none}%
  \pstGeonode(0,0){O}\pstGeonode(4,0){A}\pstCircleOA{O}{A}
  \pstHomO[HomCoef=\pstDistAB{O}{A} 2 mul \pstDistAB{O}{M} sub
    \pstDistAB{O}{M} div]{O}{M}[M']%
  \pstHomO[HomCoef=\pstDistAB{O}{A} 2 mul \pstDistAB{O}{P} sub
    \pstDistAB{O}{P} div]{O}{P}[P']%
  \pstHomO[HomCoef=\pstDistAB{O}{A} 2 mul \pstDistAB{O}{N} sub
    \pstDistAB{O}{N} div]{O}{N}[N']%
  \psset{linecolor=green, linewidth=1.5pt}%
  \pstCircleABC{M}{N}{M'}{OmegaMN}\pstArcOAB{OmegaMN}{N}{M}
  \pstCircleABC{M}{P}{M'}{OmegaMP}\pstArcOAB{OmegaMP}{M}{P}
  \pstCircleABC{N}{P}{P'}{OmegaNP}\pstArcOAB{OmegaNP}{P}{N}
  \psset{linecolor=blue}
  \pstHomO[HomCoef=\pstDistAB{OmegaNP}{N} 2 mul \pstDistAB{OmegaNP}{M} sub %% M
    \pstDistAB{OmegaNP}{M} div]{OmegaNP}{M}[MH']
  \pstCircleABC{M}{M'}{MH'}{OmegaMH}\pstArcOAB{OmegaMH}{MH'}{M} %% N
  \pstHomO[HomCoef=\pstDistAB{OmegaMP}{M} 2 mul \pstDistAB{OmegaMP}{N} sub
    \pstDistAB{OmegaMP}{N} div]{OmegaMP}{N}[NH']
  \pstCircleABC{N}{N'}{NH'}{OmegaNH}\pstArcOAB{OmegaNH}{N}{NH'} %% P
  \pstHomO[HomCoef=\pstDistAB{OmegaMN}{M} 2 mul \pstDistAB{OmegaMN}{P} sub
    \pstDistAB{OmegaMN}{P} div]{OmegaMN}{P}[PH']
  \pstCircleABC{P}{P'}{PH'}{OmegaPH}\pstArcOAB{OmegaPH}{P}{PH'}
\endpsclip
\end{pspicture}
\end{LTXexample}




\clearpage
\section{List of all optional arguments for \texttt{pst-eucl}}

\xkvview{family=pst-eucl,columns={key,type,default}}

\nocite{*}
\bgroup
\RaggedRight
%\bibliographystyle{plain}
\printbibliography
\egroup

\printindex


\end{document}



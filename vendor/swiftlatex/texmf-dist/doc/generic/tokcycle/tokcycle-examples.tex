\documentclass[10pt]{article}
\usepackage[margin=1.3in]{geometry}
\usepackage[T1]{fontenc}
\usepackage{tokcycle,xcolor,manfnt}
\colorlet{bred}{red!90!blue}
\usepackage{tikz}
% MACROS: REDEFINE TO ALLOW \cytoks TO ESCAPE SCOPE OF exampleA
\aftertokcycle{\global\cytoks\expandafter{\the\cytoks}}
% ENVIRONMENTS: REDEFINE TO ALLOW \cytoks TO ESCAPE SCOPE OF exampleA
\makeatletter\renewcommand\tcenvscope{\global\cytoks}\makeatother

\def\ddbend{\setbox0=\hbox{\dbend\dbend}\leavevmode\llap{%
  \raisebox{10pt}{\box0~}}}

\usepackage[skins,listings]{tcolorbox}

\newtcblisting{exampleA}[2][]{%
  colframe=red!70!yellow!50!black,
  colback=red!75!yellow!5!white,
  coltitle=red!50!yellow!3!white,
  bicolor,colbacklower=white,
  top=0mm,bottom=0mm,left=0mm,right=0mm,
  arc=1mm,boxrule=1pt,
  fonttitle=\sffamily\bfseries,
  sidebyside,
  title=#2,#1}


\newtcblisting{exampleB}[2][]{%
  colframe=red!50!yellow!50!black,
  colback=red!50!yellow!5!white,%white,
  coltitle=red!50!yellow!3!white,
  bicolor,colbacklower=white,%red!50!yellow!5!white,
  fonttitle=\sffamily\bfseries,
  sidebyside,text and listing,
  top=0mm,bottom=0mm,left=0mm,right=0mm,
  arc=1mm,boxrule=1pt,
  title=#2,#1}

\newtcblisting{exampleC}[2][]{%
  colframe=cyan!70!yellow!50!black,
  colback=cyan!30!yellow!5!white,%white,
  coltitle=cyan!50!yellow!3!white,
  fonttitle=\sffamily\bfseries,
  listing only,
  top=0mm,bottom=0mm,left=0mm,right=0mm,
  arc=1mm,boxrule=1pt,
  title=#2,#1}

\def\altdecytoks{\par\medskip\begingroup\noindent\macname{cytoks}
  \ \textit{alt}detokenization:\\\raggedright\footnotesize\ttfamily
  \expandafter\altdetokenize\expandafter{\the\cytoks}\par\endgroup}
\newcommand\TokCycle{\textsf{tokcycle}}

\parindent0pt
\parskip0pt
\begin{document}
{\centering\LARGE \TokCycle{} Package Examples\par\smallskip
\large\today\par}

\tableofcontents

%%%%%%%%%%%%%% SUPPORT MACRO

\newcommand\macname[1]{\texttt{\char92#1}} 

\newcommand\underdot[1]{\ooalign{#1\cr\hfil{\raisebox{-5pt}{.}}\hfil}}

\newcounter{colorindex}

\newcommand\restorecolor{\setcounter{colorindex}{100}}

\newcommand\reducecolor[1]{%
  \color{red!\thecolorindex!cyan}%
  \addtocounter{colorindex}{-#1}%
  \ifnum\thecolorindex<1\relax\setcounter{colorindex}{1}\fi
}

\newif\ifmacro
\newcommand\altdetokenize[1]{\begingroup\stripgroupingtrue\macrofalse
  \tokcycle
    {\ifmacro\def\tmp{##1}\ifcat\tmp A\else\unskip\allowbreak\fi\macrofalse\fi
     \detokenize{##1}}
    {\ifmacro\unskip\macrofalse\fi\{\processtoks{##1}\ifmacro\unskip\fi\}\allowbreak}
    {\tctestifx{\\##1}{\\}{\ifmacro\unskip\allowbreak\fi
     \allowbreak\detokenize{##1}\macrotrue}}
    { \hspace{0pt plus 3em minus .3ex}}
    {#1}%
  \unskip
\endgroup}

\newcommand\plusl[1]{\char\numexpr`#1+1\relax}

\newcommand\vowelcap[1]{%
  \ifx a#1A\else
  \ifx e#1E\else
  \ifx i#1I\else
  \ifx o#1O\else
  \ifx u#1U\else
  #1\fi\fi\fi\fi\fi
}

\newcommand\spaceouttext[2]{%
  \tokcycle
    {\addcytoks{##1\nobreak\hspace{#1}}}%
    {\processtoks{##1}}
    {\addcytoks{##1}}%
    {\addcytoks{##1\hspace{#1}}}
    {#2}%
  \the\cytoks\unskip}

\newcommand\growdim[2]{%
\tokcycle{\addcytoks{##1}}
         {\addcytoks{#1\dimexpr##1}}
         {\addcytoks{##1}}
         {\addcytoks{##1}}{%
    #2}%
\the\cytoks}

\newcommand\nextcap[1]{%
       \edef\tmp{#1}%
       \tctestifx{-#1}{\def\capnext{T}}{}%
       \tctestifcon{\if T\capnext}%
         {\tctestifcon{\ifcat\tmp A}%
           {\uppercase{#1}\def\capnext{F}}%
           {#1}}%
         {#1}%
}

\newcommand\TitleCase[1]{%
  \def\capnext{T}
  \tokcycle
    {\addcytoks{\nextcap{##1}}}
    {\processtoks{##1}}
    {\addcytoks{##1}}
    {\addcytoks{##1\def\capnext{T}}}
    {#1}%
  \the\cytoks
}

\tokcycleenvironment\spaceBgone
    {\addcytoks{##1}}
    {\processtoks{##1}}
    {\addcytoks{##1}}
    {\addcytoks{\hspace{.2pt plus .2pt minus .8pt}}}%


\tokcycleenvironment\remaptext
    {\addcytoks[x]{\tcremap{##1}}}
    {\processtoks{##1}}
    {\addcytoks{##1}}
    {\addcytoks{##1}}
\newcommand*\tcmapto[2]{\expandafter\def\csname tcmapto#1\endcsname{#2}}
\newcommand*\tcremap[1]{\ifcsname tcmapto#1\endcsname
                 \csname tcmapto#1\endcsname\else#1\fi}
\tcmapto am   \tcmapto bf   \tcmapto cz   \tcmapto de   \tcmapto ey
\tcmapto fl   \tcmapto gx   \tcmapto hb   \tcmapto ic   \tcmapto jn
\tcmapto ki   \tcmapto lr   \tcmapto mh   \tcmapto nt   \tcmapto ok
\tcmapto ps   \tcmapto qa   \tcmapto ro   \tcmapto sq   \tcmapto tw
\tcmapto uj   \tcmapto vp   \tcmapto wd   \tcmapto xg   \tcmapto yu
\tcmapto zv

\newcommand\findinstring[2]{\begingroup%
  \stripgroupingtrue
  \setcounter{runcount}{0}%
  \tokcycle
    {\nextctltok{##1}}
    {\nextctltok{\opengroup}\processtoks{##1}\nextctltok{\closegroup}}
    {\nextctltok{##1}}
    {\nextctltok{\tcspace}}
    {#1}%
  \edef\numlet{\theruncount}%
  \expandafter\def\expandafter\searchword\expandafter{\the\cytoks}%
%
  \aftertokcycle{\matchfound}%
  \setcounter{runcount}{0}%
  \def\matchfound{F}%
  \tokcycle
    {\nextcmptok{##1}}
    {\nextcmptok{\opengroup}\processtoks{##1}\nextcmptok{\closegroup}}
    {\nextcmptok{##1}}
    {\nextcmptok{\tcspace}}
    {#2}%
\endgroup}
\newcounter{runcount}
\makeatletter
\newcommand\rotcytoks[1]{\cytoks\expandafter\expandafter\expandafter{%
  \expandafter\tc@gobble\the\cytoks#1}}
\makeatother
\newcommand\testmatch[1]{\ifx#1\searchword\gdef\matchfound{T}\fi}%
\newcommand\rotoradd[2]{\stepcounter{runcount}%
  \ifnum\theruncount>\numlet\relax#1\else#2\fi
  \expandafter\def\expandafter\tmp\expandafter{\the\cytoks}}
\newcommand\nextcmptok[1]{\rotoradd{\rotcytoks{#1}}{\addcytoks{#1}}\testmatch{\tmp}}
\newcommand\nextctltok[1]{\stepcounter{runcount}\addcytoks{#1}}


%%%%%%%%%%%%%%%%%%%%%%%%%%%%%%%%%%%%%%%%%%%%%%%%%%%%%%%%%%%%
\section{Examples, examples, and more examples}

Often, the best way to learn a new tool is to see examples of
  it being used.
Here, a number of examples are gathered that span the spectrum 
  of \TokCycle{} usage.

\subsection{Application basics}

\subsubsection{Using the CGMS directives}

Apply different directives to Characters (under-dot), Groups
  (visible braces), Macros (boxed, detokenized), and 
  Spaces (visible space).

\begin{exampleC}[]{The \macname{underdot} macro}
\newcommand\underdot[1]{\ooalign{#1\cr\hfil{\raisebox{-5pt}{.}}\hfil}}
\end{exampleC}


\begingroup
\begin{exampleA}[lefthand width=7.8cm]{}
\tokcycle{\addcytoks{\underdot{#1}}}
         {\addcytoks{\{}\processtoks{#1}
          \addcytoks{\}}}
         {\addcytoks{\fbox{\detokenize{#1}}}}
         {\addcytoks{\textvisiblespace}}
{This \textit{is \textbf{a}} test.}
\the\cytoks
\end{exampleA}

\endgroup

\subsubsection{Escaping text}

Text between two successive escape characters is bypassed
  by \TokCycle{} and instead echoed to the output register.
Default escape character is $|$.  One can change it with
  \macname{settcEscapechar} macro.

\begingroup

\begin{exampleC}[]{The unexpandable \macname{plusl} macro}
\newcommand\plusl[1]{\char\numexpr`#1+1\relax}
\end{exampleC}

\begin{exampleA}[lefthand width=5cm]{Escaping text in the input stream}
\tokcycle
{\addcytoks{\plusl{#1}}}
{\processtoks{#1}}
{\addcytoks{#1}}
{\addcytoks{#1}}
{This \fbox{code is a test 
  |(I can also escape text)|}
  of |\rule{1em}{.5em}|
  {\bfseries mine}.}
\the\cytoks
\end{exampleA}

\endgroup

\subsubsection{Unexpandable, unexpanded, and expanded Character 
  directives}

\begingroup
This section concerns the issue of whether the characters of the input
  stream are transformed before or after being placed in the 
  output token register (\macname{cytoks}).\medskip

\noindent\begin{minipage}[t]{\textwidth}
Transform characters (+1 ASCII) via unexpandable macro:\medskip

\begin{exampleA}[lefthand width=8.3cm]
  {Unexpandable Character directive}
\tokcycle
{\addcytoks{\plusl{#1}}}
{\processtoks{#1}}
{\addcytoks{#1}}
{\addcytoks{#1}}{%
  This \textit{code \textup{is} a test} of mine.}
\the\cytoks
\end{exampleA}

\altdecytoks

\par\noindent\hrulefill\medskip
\end{minipage}
\endgroup

\noindent\begin{minipage}[t]{\textwidth}
Capitalize vowels (but don't expand the character directive)

\begingroup

\begin{exampleC}[]{The expandable\macname{vowelcap} macro}
\newcommand\vowelcap[1]{%
  \ifx a#1A\else
  \ifx e#1E\else
  \ifx i#1I\else
  \ifx o#1O\else
  \ifx u#1U\else
  #1\fi\fi\fi\fi\fi
}
\end{exampleC}

\begin{exampleA}[lefthand width=8.3cm]
  {Not expanded Character directive}
\tokcycle
{\addcytoks{\vowelcap{#1}}}
{\processtoks{#1}}
{\addcytoks{#1}}
{\addcytoks{#1}}{%
  This \textit{code \textup{is} a test} of mine.}
\the\cytoks
\end{exampleA}

\altdecytoks 

\par\noindent\hrulefill\medskip
\endgroup
\end{minipage}

\begingroup
Capitalize vowels (expanding the character directive)

\begin{exampleA}[lefthand width=8.3cm]
  {Expanded Character directive}
\tokcycle
{\addcytoks[x]{\vowelcap{#1}}}
{\processtoks{#1}}
{\addcytoks{#1}}
{\addcytoks{#1}}{%
  This \textit{code \textup{is} a test} of mine.}
\the\cytoks
\end{exampleA}

\altdecytoks

\endgroup

\subsubsection{Unexpanded vs. pre-expanded input stream}

\begingroup
\Characterdirective{\addcytoks[x]{\vowelcap{#1}}}

\noindent\begin{minipage}[t]{\textwidth}
\begin{exampleA}[lefthand width=5.5cm]{Normal token cycle (input stream
  not pre-expanded)}
\tokcycle
{\addcytoks[x]{\vowelcap{#1}}}
{\processtoks{#1}}
{\addcytoks{#1}}
{\addcytoks{#1}}%
{This \fbox{code 
  is a test \today} of 
  {\bfseries mine}.}
\the\cytoks
\end{exampleA}

\altdecytoks

\noindent\hrulefill\strut
\end{minipage}

Note that, when pre-expanding the input stream, one must 
  \macname{noexpand} the macros that are \textit{not} 
  to be pre-expanded.

\begin{exampleA}[lefthand width=5.5cm]{Pre-\macname{expanded} token cycle
  input stream}
\expandedtokcyclexpress
{This \noexpand\fbox{code 
  is a test \today} of 
  {\noexpand\bfseries mine}.}
\the\cytoks
\end{exampleA}

\altdecytoks

\endgroup

\subsection{Grouping}

Differentiating explicit groups, e.g., \{...\}, from
  implicit groups, e.g. \macname{bgroup}...\macname{egroup},
  is done automatically by \TokCycle.
The user has options on how \TokCycle{} should treat these tokens.
The desired options are to be set prior to the \TokCycle{} invocation.


\subsubsection{Treatment options for implicit groups}

\begingroup

The macro
\macname{stripimplicitgroupingcase} can take three possible
  integer arguments: 0 (default) to automatically place unaltered
  implicit group tokens in the output register; 1 to strip
  implicit group tokens from the output; or $-$1 to instead pass the
  implicit group tokens to the Character directive (as implicit
  tokens) for separate processing (typically, when detokenization
  is desired).

\begin{exampleA}[lefthand width=6cm]{Using 
  \macname{stripimplicitgroupingcase} to affect treatment of 
  implicit grouping tokens}
\resettokcycle
\Characterdirective{\addcytoks[x]{%
  \vowelcap{#1}}}
\def\z{Announcement: 
  {\bfseries\bgroup\itshape 
  Today \egroup it is} \today, 
  a Wednesday}
\expandafter\tokencyclexpress\z
\endtokencyclexpress\medskip

\detokenize\expandafter{\the\cytoks}
\bigskip

\stripimplicitgroupingcase{1}
\expandafter\tokencyclexpress\z
\endtokencyclexpress\medskip

\detokenize\expandafter{\the\cytoks}
\end{exampleA}

%\tokcyclexpress{Announcement: {\bfseries\bgroup\itshape 
%  Today \egroup it is} \today, a Wednesday.}\the\cytoks

%\detokenize\expandafter{\the\cytoks}

\endgroup

\subsubsection{Treatment options for explicit groups}

\begingroup

For explicit group tokens, e.g., \{ \}, there are only two 
  options to be had.
These are embodied in the if-condition \macname{ifstripgrouping} 
  (default \macname{stripgroupingfalse}).
Regardless of which condition is set, the tokens within the explicit
  group are still passed to the Group directive for processing.


Permutations of the following code are used in the subsequent examples.
Group stripping, brought about by \macname{stripgroupingtrue},
  involves removing the grouping braces from around the group.
The choice of \macname{processtoks} vs.\@ \macname{addcytoks}
  affects whether the tokens inside the group are recommitted
  to \TokCycle{} for processing, or are merely sent to the output
  register in their original unprocessed form.

Note that, in these examples, underdots and visible spaces
  will only appear on characters and spaces that have been directed 
  to the Character and Space directives, respectively.
Without \macname{processtoks}, that will not occur to tokens
  \textit{inside} of groups.

\begin{exampleB}[listing only,lefthand width = \dimexpr\textwidth-19pt]
{Code permutations on group stripping and inner-group token processing}
\stripgroupingfalse  OR  \stripgroupingtrue
\tokcycle{\addcytoks{\underdot{#1}}}
         {\processtoks{#1}}  OR  {\addcytoks{#1}}}
         {\addcytoks{#1}}
         {\addcytoks{\textvisiblespace}}
{This \fbox{is a \fbox{token}} test.}
\the\cytoks
\end{exampleB}


\begin{exampleB}[text only,  colback=white]%
{\macname{stripgroupingfalse \string\processtoks} }
\stripgroupingfalse% RETAIN GROUPING (PACKAGE DEFAULT)
\tokcycle{\addcytoks{\underdot{#1}}}
         {\processtoks{#1}}% GROUP TOKENS ARE PROCESSED
         {\addcytoks{#1}}
         {\addcytoks{\textvisiblespace}}{%
This \fbox{is a \fbox{token}} test.}
\the\cytoks
\end{exampleB} 

\begin{exampleB}[text only,  colback=white]%
{\macname{stripgroupingfalse \string\addcytoks} }
\tokcycle{\addcytoks{\underdot{#1}}}
         {\addcytoks{#1}}% GROUP TOKENS OUTPUT DIRECTLY
         {\addcytoks{#1}}
         {\addcytoks{\textvisiblespace}}{%
This \fbox{is a \fbox{token}} test.}
\the\cytoks
\end{exampleB}

\begin{exampleB}[text only,  colback=white]%
{\macname{stripgroupingtrue \string\processtoks} }
\stripgroupingtrue% DANGEROUS AS IT REMOVES GROUPING FROM ARGUMENTS
\tokcycle{\addcytoks{{\underdot{#1}}}}% EXTRA GROUPING TO COMPENSATE
         {\processtoks{#1}}% GROUP TOKENS PROCESSED
         {\addcytoks{#1}}
         {\addcytoks{\textvisiblespace}}{%
This \fbox{is a \fbox{token}} test.}
\the\cytoks
\end{exampleB}

\begin{exampleB}[text only,  colback=white]%
{\macname{stripgroupingtrue \string\addcytoks} }
\stripgroupingtrue% DANGEROUS AS IT REMOVES GROUPING FROM ARGUMENTS
\tokcycle{\addcytoks{\underdot{#1}}}
         {\addcytoks{#1}}% GROUP TOKENS OUTPUT DIRECTLY
         {\addcytoks{#1}}
         {\addcytoks{\textvisiblespace}}{%
This \fbox{is a \fbox{token}} test.}
\the\cytoks
\end{exampleB} 

Note that the content of groups can be altogether eliminated if 
  \textit{neither} \macname{processtoks\{\#1\}} nor 
  \macname{addcytoks\{\#1\}} are used in the Group directive.



\endgroup

\subsubsection{Group nesting}

\begingroup

\begin{exampleC}[]{The \macname{reducecolor} and \macname{restorecolor}
  macros}
\newcounter{colorindex}
\newcommand\restorecolor{\setcounter{colorindex}{100}}
\newcommand\reducecolor[1]{%
  \color{red!\thecolorindex!cyan}%
  \addtocounter{colorindex}{-#1}%
  \ifnum\thecolorindex<1\relax\setcounter{colorindex}{1}\fi}
\end{exampleC}

\begin{exampleA}[lefthand width=6cm]{Group nesting is no impediment to
  tokcycle}
\restorecolor
\tokcycle
  {\addcytoks{(#1)}}
  {\addcytoks{\reducecolor{11}}%
   \addcytoks{[}\processtoks{#1}%
   \addcytoks{]}}
  {\addcytoks{#1}}
  {}{%
  {1{{3{{5{{7{{9{1{}0}}8}}6}}4}}2}}}
\the\cytoks
\end{exampleA}

\endgroup


\subsection{Direct use of \TokCycle}

\TokCycle{} (in regular or \texttt{xpress} form) may be invoked 
  directly from the document, without
  being first encapsulated within a macro or environment.

\subsubsection{Modifying counters as part of the Character
  directive}

\begingroup

\begin{exampleA}[lefthand width=8.5cm]{Using a period token (.) to
  reset a changing color}
\restorecolor
\tokencycle
  {\addcytoks{\bgroup\reducecolor{3}#1\egroup}%
   \ifx.#1\addcytoks{\restorecolor}\fi}
  {\processtoks{#1}}
  {\addcytoks{#1}}
  {\addcytoks{#1}}%
This right \textit{here is a sentence in italic}.  
And \textbf{here we have another sentence in bold}.

{\scshape Now in a new paragraph, the sentence 
is long.} Now, it is short.
\endtokencycle
\end{exampleA}

\endgroup

\subsection{Macro encapsulation of \TokCycle}

\subsubsection{Spacing out text}

\begingroup

\begin{exampleC}[]{The \macname{spaceouttext} macro}
\newcommand\spaceouttext[2]{%
  \tokcycle
    {\addcytoks{##1\nobreak\hspace{#1}}}%
    {\processtoks{##1}}
    {\addcytoks{##1}}%
    {\addcytoks{##1\hspace{#1}}}
    {#2}%
  \the\cytoks\unskip}
\end{exampleC}

\begin{exampleA}[lefthand width=6cm]{\macname{spaceouttext} demo}
\spaceouttext{3pt plus 3pt}{This 
  \textit{text \textbf{is} 
  very} spaced out}. Back 
  to regular text.

\spaceouttext{1.5pt}{This 
  \textit{text \textbf{is} 
  somewhat} spaced out}. 
  Back to regular text.
\end{exampleA}

\endgroup


\subsubsection{Alternate presentation of detokenized content}

\begingroup
This macro attempts to give a more natural presentation of 
  \macname{detokenize}'d material.
It is \textbf{not} to be confused as a replacement for
  \macname{detokenize}.
In certain applications, it may offer a more pleasingly formatted
  typesetting of detokenized material.

It is an unusual application of \TokCycle{} in that it does not
  actually use the \macname{cytoks} token register to collect
  its output.
This is only possible because all macros in the input stream
  are detokenized, rather than executed.

\medskip

\begin{exampleC}[]{The \macname{altdetokenize }\ macro}
\newif\ifmacro
\newcommand\altdetokenize[1]{\begingroup\stripgroupingtrue\macrofalse
  \tokcycle
    {\ifmacro\def\tmp{##1}\ifcat\tmp A\else\unskip\allowbreak\fi\macrofalse\fi
     \detokenize{##1}}
    {\ifmacro\unskip\macrofalse\fi\{\processtoks{##1}\ifmacro\unskip\fi\}\allowbreak}
    {\tctestifx{\\##1}{\\}{\ifmacro\unskip\allowbreak\fi
     \allowbreak\detokenize{##1}\macrotrue}}
    { \hspace{0pt plus 3em minus .3ex}}
    {#1}%
  \unskip
\endgroup}
\end{exampleC}

\let\mac\relax
\begin{exampleA}[lefthand width=6.3cm]
  {\macname{altdetokenize} demo}
\string\altdetokenize: \\
\texttt{\altdetokenize{a\mac a \mac2 {\mac}\mac{a\mac\mac}\mac}}!

\string\detokenize: \\ 
   \texttt{\detokenize{a\mac a \mac2 {\mac}\mac{a\mac\mac}\mac}}!
\end{exampleA}
\endgroup


\subsubsection{Capitalize all words, including compound and 
  parenthetical words}

\begingroup

\begin{exampleC}[]{The \macname{Titlecase} and \texttt{\string\nextcap} macros}
\newcommand\TitleCase[1]{%
  \def\capnext{T}
  \tokcycle
    {\addcytoks{\nextcap{##1}}}
    {\processtoks{##1}}
    {\addcytoks{##1}}
    {\addcytoks{##1\def\capnext{T}}}
    {#1}%
  \the\cytoks
}
\newcommand\nextcap[1]{%
       \edef\tmp{#1}%
       \tctestifx{-#1}{\def\capnext{T}}{}%
       \tctestifcon{\if T\capnext}%
         {\tctestifcon{\ifcat\tmp A}%
           {\uppercase{#1}\def\capnext{F}}%
           {#1}}%
         {#1}%
}
\end{exampleC}

\begin{exampleA}[]{A demo of \macname{Titlecase} showing raw (escaped)
  input and processed output}
\TitleCase{%
|here, {\bfseries\today{}, is [my]}
  really-big-test  
  (\textit{capitalizing} words).|

 here, {\bfseries\today{}, is [my]} 
  really-big-test 
  (\textit{capitalizing} words).}
\end{exampleA}

\endgroup

\subsubsection{Scaling rule dimensions}
\begingroup

\ddbend This example only applies if one can guarantee that
  the input stream will contain only text and rules...

\begin{exampleC}[]{The \macname{growdim} macro}
\newcommand\growdim[2]{%
\tokcycle{\addcytoks{##1}}
         {\addcytoks{#1\dimexpr##1}}
         {\addcytoks{##1}}
         {\addcytoks{##1}}{%
    #2}%
\the\cytoks}
\end{exampleC}

\begin{exampleA}[]{Using \TokCycle{} to change \macname{rule} dimensions}
\growdim{2}{This rule is exactly 4pt: 
  \rule|{4pt}{4pt}| , whereas this
  rule is 2x bigger than 4pt: 
  \rule{4pt}{4pt} .}\par
\growdim{4}{This rule is exactly 5pt: 
  \rule|{5pt}{5pt}| , whereas this
  rule is 4x bigger than 5pt: 
  \rule{5pt}{5pt} .}
\end{exampleA}

\endgroup

\subsubsection{String search, including non-regex material}

\begingroup

\begin{exampleC}[]{The \macname{findinstring} macro for string searches}
\newcommand\findinstring[2]{\begingroup%
  \stripgroupingtrue
  \setcounter{runcount}{0}%
  \tokcycle
    {\nextctltok{##1}}
    {\nextctltok{\opengroup}\processtoks{##1}\nextctltok{\closegroup}}
    {\nextctltok{##1}}
    {\nextctltok{\tcspace}}
    {#1}%
  \edef\numlet{\theruncount}%
  \expandafter\def\expandafter\searchword\expandafter{\the\cytoks}%
%
  \aftertokcycle{\matchfound}%
  \setcounter{runcount}{0}%
  \def\matchfound{F}%
  \tokcycle
    {\nextcmptok{##1}}
    {\nextcmptok{\opengroup}\processtoks{##1}\nextcmptok{\closegroup}}
    {\nextcmptok{##1}}
    {\nextcmptok{\tcspace}}
    {#2}%
\endgroup}
\newcounter{runcount}
\makeatletter
\newcommand\rotcytoks[1]{\cytoks\expandafter\expandafter\expandafter{%
  \expandafter\tc@gobble\the\cytoks#1}}
\makeatother
\newcommand\testmatch[1]{\ifx#1\searchword\gdef\matchfound{T}\fi}%
\newcommand\rotoradd[2]{\stepcounter{runcount}%
  \ifnum\theruncount>\numlet\relax#1\else#2\fi
  \expandafter\def\expandafter\tmp\expandafter{\the\cytoks}}
\newcommand\nextcmptok[1]{\rotoradd{\rotcytoks{#1}}{\addcytoks{#1}}\testmatch{\tmp}}
\newcommand\nextctltok[1]{\stepcounter{runcount}\addcytoks{#1}}
\end{exampleC}

\begin{exampleA}[lefthand width=11.5cm]{Demo of the 
  \macname{findinstring} macro}
1. \findinstring{this}{A test of the times}
   \findinstring{the} {A test of the times}\par
2. \findinstring{This is}{Here, This is a test}
   \findinstring{Thisis} {Here, This is a test}\par
3. \findinstring{the}        {This is the\bfseries{} test}
   \findinstring{he\bfseries}{This is the\bfseries{} test}\par
4. \findinstring{a{bc}} {gf{vf{a{b c}g}gh}hn}
   \findinstring{a{b c}}{gf{vf{a{b c}g}gh}hn}\par
5. \findinstring{a\notmymac{b c}}{gf{vf{a\mymac{b c}g}gh}hn}
   \findinstring{a\mymac{b c}}   {gf{vf{a\mymac{b c}g}gh}hn}\par
6. \findinstring{\textit{Italic}}{this is an \textit{italic} test}
   \findinstring{\textit{italic}}{this is an \textit{italic} test}
\end{exampleA}

\endgroup


\subsection{\TokCycle-based environments}

The \macname{tokcycleenvironment} macro allows users to define
  their own \TokCycle{} environments.  
Here are some examples.

\subsubsection{``Removing'' spaces, but still breakable/hyphenatable}

\begingroup

\begin{exampleC}[]{The \macname{spaceBgone} environment}
\tokcycleenvironment\spaceBgone
    {\addcytoks{##1}}
    {\processtoks{##1}}
    {\addcytoks{##1}}
    {\addcytoks{\hspace{.2pt plus .2pt minus .8pt}}}%
\end{exampleC}

\begin{exampleA}[lefthand width=6.0cm]{}
\spaceBgone
  Here we have a \textit{test} of 
  whether the spaces are removed.  
  We are choosing to use the 
  tokencycle environment.

  We are also testing the use of 
  paragraph breaks in the 
  environment.
\endspaceBgone
\end{exampleA}

\endgroup

\subsubsection{Remapping text}

\begin{minipage}[t]{\textwidth}
\begingroup

\begin{exampleC}[]{The \macname{remaptext} environment with
  supporting macros}
\tokcycleenvironment\remaptext
    {\addcytoks[x]{\tcremap{##1}}}
    {\processtoks{##1}}
    {\addcytoks{##1}}
    {\addcytoks{##1}}
\newcommand*\tcmapto[2]{\expandafter\def\csname tcmapto#1\endcsname{#2}}
\newcommand*\tcremap[1]{\ifcsname tcmapto#1\endcsname
                 \csname tcmapto#1\endcsname\else#1\fi}
\tcmapto am   \tcmapto bf   \tcmapto cz   \tcmapto de   \tcmapto ey
\tcmapto fl   \tcmapto gx   \tcmapto hb   \tcmapto ic   \tcmapto jn
\tcmapto ki   \tcmapto lr   \tcmapto mh   \tcmapto nt   \tcmapto ok
\tcmapto ps   \tcmapto qa   \tcmapto ro   \tcmapto sq   \tcmapto tw
\tcmapto uj   \tcmapto vp   \tcmapto wd   \tcmapto xg   \tcmapto yu
\tcmapto zv
\end{exampleC}

\begin{exampleA}[lefthand width=8.5cm]{Demo of \macname{remaptext}}
\remaptext
What can't we \textit{accomplish} if we try?

Let us be of good spirit and put our minds to it!
\endremaptext
\end{exampleA}

Because \macname{tcremap} is expandable, the original text is totally absent 
  from the processed output:

\altdecytoks

\endgroup
\end{minipage}

\subsection{Advanced topics: implicit tokens and catcode changes}

\subsubsection{Trap Active Characters (catcode 13)}

\begin{minipage}[t]{\textwidth}
\begingroup
Active characters in the \TokCycle{} input stream are processed in
  their original form.
Their active substitutions arising from \macname{def}s 
  only occur \textit{afterwards}, when
  the \TokCycle{} output is typeset.
They may be identified with the \macname{ifactivetok} test.
If \macname{let} to a character, they may be identified in the Character
  directive;
If \macname{let} to a control sequence or defined via \macname{def},
  they may be identified in the Macro directive.
\let\svt T
\Characterdirective{\tctestifcon\ifactivetok
  {\addcytoks{\fbox{#1-chr}}}{\addcytoks{#1}}}
\Macrodirective{\tctestifcon\ifactivetok
  {\addcytoks{\fbox{#1-mac}}}{\addcytoks{#1}}}

\begin{exampleA}[lefthand width=7.8cm]{Processing active characters}
\resettokcycle
\tokencyclexpress 
This is a test!!\endtokencyclexpress

\catcode`!=\active
\def !{?}
\tokencyclexpress 
This is a test!!\endtokencyclexpress

\Characterdirective{\tctestifcon\ifactivetok
  {\addcytoks{\fbox{#1-chr}}}{\addcytoks{#1}}}
\Macrodirective{\tctestifcon\ifactivetok
  {\addcytoks{\fbox{#1-mac}}}{\addcytoks{#1}}}
\tokencyclexpress 
This is a test!!\endtokencyclexpress

\catcode`T=\active
\let T+
\tokencyclexpress 
This is a test!!\endtokencyclexpress

\detokenize\expandafter{\the\cytoks}
\end{exampleA}

%%%

\ddbend If the input stream is pre-\textit{expanded}, any active 
  substitutions that are expandable (i.e., those involving \macname{def} 
  as well as those \macname{let} to something expandable) 
  are made before reaching \TokCycle{} processing.  
\svt hey are, thus, no longer detected as active, unless 
  \macname{noexpand} is applied before the pre-expansion.
In this example, the \texttt{!} that is not \macname{noexpand}\kern.3pted
  is converted to a \texttt{?} prior to reaching \TokCycle{}
  processing (and thus, not detected as \macname{active}):

\catcode`!=\active
\def !{?}
\catcode`T=\active
\let T+
\begin{exampleA}[lefthand width=7.8cm]{Expanded input stream acts 
  upon active \macname{def}ed characters unless \macname{noexpand} is applied}
\expandedtokcyclexpress{This is a test!\noexpand!}
\the\cytoks\par
\detokenize\expandafter{\the\cytoks}
\end{exampleA}

However, pre-tokenization does not suffer this behavior:
\begin{exampleA}[lefthand width=7.8cm]{Pre-tokenized input stream
  does not affect active characters}
\def\tmp{This is a test!!}
\expandafter\tokcyclexpress\expandafter{\tmp}
\the\cytoks\par
\detokenize\expandafter{\the\cytoks}
\end{exampleA}

%%%
\endgroup
\end{minipage}

\begingroup
\ddbend One aspect of \TeX{} to remember is that catcodes
  are assigned at tokenization; however, for active characters,
  the substitution assignment is evaluated only upon execution.
So, if a cat-13 token is placed into a \macname{def}, it will
  remain active even if the catcode of that character code is
  later changed.
But if the cat-13 active definition is changed prior to the 
  execution of the \macname{def}'ed token, the revised
  token assignment will apply.

The following example demonstrates this concept, while showing,
  without changing the input in any way, that \TokCycle{} 
  can properly digest active and implicit grouping (cat-1,2) characters:

\begin{exampleA}[lefthand width=8cm]{Active and implicit grouping
  tokens digestible by \TokCycle}
\catcode`Y=13
\catcode`Z=13
\let Y{
\let Z}
\let\Y{
\let\Z}
\def\tmp{\textit YabcZ de\Y\itshape f\Zg}%

\def Y{\bgroup[NEW]}% APPLIES AT EXECUTION
\catcode`Y=11% DOES NOT AFFECT Y IN \tmp

\expandafter\tokcyclexpress\expandafter{\tmp}
\the\cytoks

\detokenize\expandafter{\the\cytoks}
\end{exampleA}
\endgroup

\subsubsection{Trap Catcode 6 (explicit \& implicit) tokens}

\begingroup

\let\myhash#

\medskip

Typically, cat-6 tokens (like \texttt{\#}) are used to designate the
  following digit (1-9) as a parameter.
Since they are unlikely to be used in that capacity inside a 
  \TokCycle{} input stream, the package behavior is to convert them
  into something cat-12 and set the if-condition \macname{catSIXtrue}.
In this manner, \macname{ifcatSIX} can be used inside the
  Character directive to convert cat-6 tokens into something of
  the user's choosing.

As to this cat-12 conversion, explicit cat-6 characters are converted
  into the same character with cat-12.  
On the other hand, implicit cat-6 control sequences (e.g., \verb|\let\myhash#|) 
  are converted into a fixed-name macro, \macname{implicitsixtok}, 
  whose cat-12 substitution text is a \macname{string} of the original
  implicit-macro name.

\resettokcycle
\Characterdirective{\ifcatSIX
  \addcytoks{\fbox{#1}}
  \else\addcytoks{#1}\fi}

%\tokencyclexpress This# isQ \textit{a Q# test\myhash}!\endtokencyclexpress

\begin{exampleA}[lefthand width=6.5cm]{Treatment of cat-6 tokens}
\resettokcycle
\Characterdirective{\ifcatSIX
  \addcytoks{\fbox{#1}}
  \else\addcytoks{#1}\fi}
\let\myhash#
\tokcyclexpress{This# isQ 
  \textit{a Q# test\myhash}!}
\the\cytoks\bigskip\par
\detokenize\expandafter{\the\cytoks}
\end{exampleA}

%\tokcyclexpress{This# isQ \textit{a Q# t\,e\, s\,t\myhash}!}\the\cytoks{}
%with 1-character macros

%%%
%\expandedtokcyclexpress{This# isQ \noexpand\textit{a Q# test\myhash}!}\the\cytoks
%{} PRE-EXPANDED
%%%


%\tokencyclexpress This# isQ \textit{a Q# test}!\endtokencyclexpress

%\Characterdirective{\ifcatSIX\addcytoks{\fbox{#1}}\else\addcytoks{#1}\fi}

\begin{exampleA}[lefthand width=6.5cm]{Multiple explicit cat-6 tokens are not 
  a problem}
\catcode`Q=6
\tokcyclexpress{This# isQ 
  \textit{a Q# test\myhash}!}
\the\cytoks
\end{exampleA}

\catcode`Q=11
\hrulefill

\ddbend For what is, perhaps, a rare situation, one can even process input 
  streams that contain cat-6 macro parameters.
A package macro, \macname{whennotprocessingparameter\#1\{<}%
  \textit{directive when not a parameter}\verb|>}|, can be
  used inside of the Character directive to intercept parameters.
In this example, a macro is defined and then executed, subject to
  token replacements brought about by the expandable Character directive.

\Characterdirective{%
  \whennotprocessingparameter#1{\addcytoks[x]{\vowelcap{#1}}}}
\aftertokcycle{\global\cytoks\expandafter{\the\cytoks}}
\def\Characterdirective#1{}
\begin{exampleA}[lefthand width=7cm]{Preserving parameters (e.g.
  \texttt{\#1, \#2}) in the \TokCycle{} input stream}
\Characterdirective{%
  \whennotprocessingparameter#1{%
    \addcytoks[x]{\vowelcap{#1}}}}
\tokcyclexpress{%
  \def\zQ#1#2{[one:#1](two:#2)}
  This is a \zQ big test.

  \renewcommand\zQ[2]{\ifx t#1[#1]\fi(#2)}
  This is a \zQ test.}
\the\cytoks
\end{exampleA}

\altdecytoks

\endgroup

\subsubsection{Trap implicit tokens in general}

\begingroup

Implicit control sequences (assigned via \macname{let} to characters) 
  were already mentioned in the context of cat-6.
However, implicit control sequences can be of any valid catcode (except
  for cat-0, which we instead call macros or primitives).
The condition \macname{ifimplicittok} is used to flag such
  tokens for special processing, as well as active tokens that
  are \macname{let} to anything unexpandable.

In the next example, implicit, cat-6 and implicit-cat-6 tokens may
  all be differentiated, shown here with a multiplicity of 
  \macname{fbox}es.

\fboxsep=1.5pt

\begin{exampleA}[lefthand width=8cm]{Implicit\ensuremath{{}={}}%
  single box\char44 \ cat-6\ensuremath{{}={}}double box\char44 \ %
  implicit-cat-6\ensuremath{{}={}}triple box}
\catcode`Q=\active \let QN
\let\littlet=t
\let\littlel=l
\let\svhash#
\Characterdirective{\ifimplicittok
  \ifcatSIX\addcytoks{\fbox{\fbox{\fbox{#1}}}}%
  \else\addcytoks{\fbox{#1}}\fi\else\ifcatSIX
  \addcytoks{\fbox{\fbox{#1}}}\else
  \addcytoks{#1}\fi\fi}

\tokencyclexpress We wi\littlel\littlel# 
  \textit{ make a \littlet est #} \littlet

This \textit{is a \textbf{big}} \littlet est.

Qext pa#agraph ending with implicit cat six 
  \svhash.\endtokencyclexpress
\end{exampleA}

In the following example, we use both control sequences and active characters
 in \macname{def} and \macname{let} capacities, to demonstrate how 
  \TokCycle{} digests things.
Implicit tokens (tokens \macname{let} to characters) are shown in a box,
  with both the token name and the implicit value (note that tokens 
  \macname{let} to macros and primitives are not considered implicit).
Active tokens processed through the character directive are followed with
  a \dag, whereas those processed through the macro directive are followed
  with a \ddag.


\begin{exampleA}[lefthand width=7.5cm]{Non-active vs{.} active \macname{def} 
  \& \macname{let}}
\Characterdirective{\ifimplicittok
  \addcytoks{\fbox{\detokenize{#1}:#1}}%
  \else\addcytoks{#1}\fi\ifactivetok
  \addcytoks{\rlap{\dag}}\fi\addcytoks{~\,}}
\Macrodirective{\ifimplicittok
  \addcytoks{\fbox{\detokenize{#1}}}%
  \else\addcytoks{#1}\fi\ifactivetok
  \addcytoks{\rlap{\ddag}}\fi
  \ifx\par#1\else\addcytoks{~\,}\fi}

\def\A{a}
\let\B i
\let\C\today
\let\D\relax
\def\E{\relax}
\catcode`V=13 \def V{a}
\catcode`W=13 \let Ww
\catcode`X=13 \let X\today
\catcode`Y=13 \let Y\relax
\catcode`Z=13 \def Z{\relax}
\tokcyclexpress{\A\B\C\D\E ab\par VWXYZab}
\the\cytoks
\end{exampleA}


\ddbend If the input stream is subject to pre-expansion, one
  will require \macname{noexpand} for macros where no pre-expansion
  is desired.\medskip

\ddbend If the input stream is provided pre-tokenized via 
  \macname{def}, \TeX{} convention requires cat-6 tokens to appear 
  in the input stream as duplicate, e.g. \texttt{\#\#}.


\subsubsection{Changing grouping tokens (catcodes 1,2)}

\begingroup

Changing grouping tokens (catcodes 1,2) may require something more,
  if the output stream is to be detokenized.  
In the following examples, pay attention to the
  detokenized grouping around the argument to \macname{fbox}.

As we will see, the issues raised here only affect the situation
  when detokenization of the output stream is required.
\medskip

\begin{exampleA}[]{\TokCycle{} defaults grouping tokens to braces:}
\tokencycle
 {\addcytoks{(#1)}}
 {\processtoks{#1}}
 {\addcytoks{#1}}
 {\addcytoks{ }}
This \fbox{is a} test.
\endtokencycle\medskip

\detokenize\expandafter{\the\cytoks}
\end{exampleA}

One can make brackets cat-1,2, redefining bgroup/egroup to [ ].
However, while one can now use brackets in input stream, braces 
  will still appear in the detokenized \TokCycle{} output stream:

\begin{exampleA}[lefthand width=6.5cm]
{\TokCycle{} will not automatically change its grouping tokens}
\catcode`\[=1
\catcode`\]=2
\let\bgroup[
\let\egroup]
\tokencycle
 {\addcytoks{(#1)}}
 {\processtoks{#1}}
 {\addcytoks{#1}}
 {\addcytoks{ }}
This \fbox[is a] test.
\endtokencycle\medskip

\detokenize\expandafter{\the\cytoks}
\end{exampleA}

If it is necessary to reflect revised grouping tokens in the 
  output stream, the \macname{settcgrouping} macro is 
  to be used.

\begin{exampleA}[lefthand width=6.5cm]
{Redefine \TokCycle{} grouping tokens as angle
  brackets using \macname{settcGrouping}}
\catcode`\<=1
\catcode`\>=2
\catcode`\{=12
\catcode`\}=12
\let\bgroup<
\let\egroup>
\settcGrouping<<#1>>
\tokencycle
 <\addcytoks<(#1)>>
 <\processtoks<#1>>
 <\addcytoks<#1>>
 <\addcytoks< >>
This \fbox<is a> test.
\endtokencycle\medskip

\detokenize\expandafter<\the\cytoks>
\end{exampleA}

Angle brackets are now seen in the above detokenization.
Until subsequently changed, cat-1,2 angle brackets now appear in 
  detokenized \TokCycle{} groups, even if other cat-1,2 tokens were
  used in the input stream.
Bottom line: 
\begin{itemize}
\item adding, deleting, or changing catcode 1,2 explicit 
  grouping tokens, e.g., \{\}, (in conjunction with their 
  associated implicit  \macname{bgroup}\macname{egroup}) 
  tokens will not affect \TokCycle's ability to digest proper 
  grouping of the input stream, regardless of which tokens 
  are catcode 1,2 at the moment.

\item The grouping tokens used in \TokCycle's output 
  default to \verb|{}| braces (with cat-1,2), but can be 
  changed deliberately  using \macname{settcGrouping}.

\item The package, currently, has no way to reproduce in the
   output stream the actual grouping tokens that occur in the 
  input stream, but one should ask, for the particular
  application, if it really matters, as long as the the 
  proper catcodes-1,2 are preserved?

\end{itemize}

\endgroup

\subsubsection{Catcode 10 space tokens}

\begingroup

Here we demonstrate that \TokCycle{} can handle arbitrary redesignation
  of tokens to cat-10, as well as implicit space tokens.

\ddbend While it should seem natural, we note that implicit
  space tokens are directed to the Space directive rather than
  the Character directive.
However, \macname{ifimplicittok} may still be used to 
  differentiate an explicit space from an implicit one.
\medskip

Note in the following examples that cat-10 tokens do \textit{not}
  get under-dots.
The next three examples all use the same input, but with different
catcode settings for the space and the underscore.

\fboxsep=1pt

\begin{exampleA}[lefthand width=10cm]{space cat-10{,} underscore cat-12}
\catcode`\_=12 %
\catcode`\ =10 %

\tokencycle{\addcytoks{\underdot{#1}}}%
{\processtoks{#1}}%
{\addcytoks{#1}}%
{\addcytoks{#1}}%
\fbox{a_c d} b_g\itshape f\upshape\endtokencycle
\end{exampleA}
%
\begin{exampleA}[lefthand width=10cm]{space cat-10{,} underscore cat-10}
\catcode`\_=10 %
\catcode`\ =10 %

\tokencycle{\addcytoks{\underdot{#1}}}%
{\processtoks{#1}}%
{\addcytoks{#1}}%
{\addcytoks{#1}}%
\fbox{a_c d} b_g\itshape f\upshape\endtokencycle
\end{exampleA}

%

\begin{exampleA}[lefthand width=10cm]{space cat-12{,} underscore cat-10}
\catcode`\_=10 %
\catcode`\ =12 %

\tokencycle{\addcytoks{\underdot{#1}}}%
{\processtoks{#1}}%
{\addcytoks{#1}}%
{\addcytoks{#1}}%
\fbox{a_c d} b_g\itshape f\upshape\endtokencycle
\end{exampleA}

%%%
\resettokcycle
\Characterdirective{\addcytoks{\underdot{#1}}}

\begin{exampleA}[lefthand width=10cm]{Implicit spaces work{,} too}
\resettokcycle
\Characterdirective{\addcytoks{\underdot{#1}}}
\def\:{\let\mysptoken= } \: %
\catcode`\_=10 %
\catcode`\ =12 %

\tokencyclexpress
\fbox{a\mysptoken{}c d} b_g\itshape f\upshape
\endtokencyclexpress
\end{exampleA}

\endgroup


\subsubsection{Changes to catcode 0}

\begingroup

\begin{exampleA}[]{Cat-0 changes are not a hindrance to \TokCycle}
\let\littlet=t
\catcode`\! 0 !catcode`!\ 12
!Characterdirective{!ifimplicittok
  !addcytoks{!fbox{#1}}!else!ifcatSIX
  !addcytoks{!fbox{!fbox{#1}}}
  !else!addcytoks{#1}!fi!fi}
!tokencyclexpress Here, {!scshape!bgroup 
  on !today!itshape{} we are !egroup 
  !littlet es!littlet ing} cat-0 
  changes{!bgroup}!egroup
!endtokencyclexpress!medskip

!detokenize!expandafter{!the!cytoks}
\end{exampleA}

\endgroup


\end{document}


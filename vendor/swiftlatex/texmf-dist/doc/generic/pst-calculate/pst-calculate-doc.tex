%% $Id: pst-calculate-doc.tex 919 2019-01-24 20:23:36Z herbert $
%
\listfiles
\documentclass[fontsize=11pt,DIV=14,parskip=half-]{scrartcl}
\usepackage[T1]{fontenc}

\usepackage{graphicx}
\usepackage{multido}
\usepackage{libertine} 
\usepackage[scaled=0.88]{beramono}
\usepackage{pst-calculate}  % support opentype spark fonts
\makeatletter
\let\Version\pstcalculate@version
\makeatother

\usepackage{showexpl,xltabular,booktabs,xurl,xcolor,ragged2e}
\lstset{%
    language=[LaTeX]TeX,%
    float=hbp,%
    basicstyle=\ttfamily\small, %
    keywordstyle=\bfseries, %
    columns=flexible, %
    tabsize=4, %
%    frame=single, %
    extendedchars=true, %
    showspaces=false, %
    showstringspaces=false, %
%    numbers=left, 
%    numbersep=0.8em,
%    numberstyle=\tiny, %
    breaklines=true, %
    breakautoindent=true, 
    captionpos=b,
    xleftmargin=1em
}

\usepackage[style=dtk]{biblatex}
\addbibresource{\jobname.bib}

\usepackage{dtk-logos} % for Wikipedia W

\pagestyle{headings}

\usepackage[colorlinks,hyperfootnotes=false]{hyperref}
% define \code for url-like breaking of typewriter fragments.
\ifx\nolinkurl\undefined \let\code\url \else \let\code\nolinkurl \fi

% Define \cs to prepend a backslash, and be unbreakable:
\DeclareRobustCommand\cs[1]{\mbox{\texttt{\char`\\#1}}}


\title{Support for floating point operations on \LaTeX-Level \\--\\ v.~\Version}
\author{Herbert Voß}
\begin{document}
\maketitle
\tableofcontents

\section{Introduction}

The upcoming \LaTeX3 can already be used. It is more or less stable and
macros will change only if really needed.


\section{Package options}
The package knows two optional arguments which, of course, have a corresponding
name in package \texttt{siunitx}. One can also use that one.

\begin{tabularx}{\linewidth}{@{}llX@{}}\toprule
\emph{name} & \emph{\texttt{siunitx}} & \emph{description}\\\midrule
\texttt{useComma} & \texttt{output-decimal-marker=\{,\}} & Output always a comma instead of the default dot.\\
\texttt{roundDigit}& \texttt{round-mode=places,round-precision=<value>} & round the given digit number.\\
\bottomrule
\end{tabularx}

\section{Using the macros}

\begin{LTXexample}[pos=t]
\psCalculate{3.14126*314^2}\\  % Uses \num from siunitx
\pscalculate{3.14126*314^2}    % doesn't use \num
\end{LTXexample}

Without using any additional argument all available digits are printed.

\section{Optional arguments}

All optional arguments of package \texttt{siunitx} can be used:

\begin{LTXexample}[pos=t]
\psCalculate[group-digits=false]{3.14126*314^2}\\ 
\psCalculate[output-decimal-marker={,}]{3.14126*314^2/sin(3)}\\
\psCalculate[exponent-product=\cdot,scientific-notation=true]{3.14126*314^2/sin(3)}\\
\psCalculate[scientific-notation=engineering]{3.14126*314^2/sin(3)}\\
\psCalculate[fixed-exponent=2,scientific-notation=fixed]{3.14126*314^2/sin(3)}\\
\psCalculate[round-precision=3]{3.14126*314^2/sin(3)}\\
\psCalculate[round-mode=places,round-precision=3]{3.14126*314^2/sin(3)}
\end{LTXexample}

For more optional argument to format the output have a look at the documentation of \texttt{siunitx}.


\RaggedRight
\nocite{*}
\printbibliography


\end{document}

y
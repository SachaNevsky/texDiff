\documentclass[preprint,pra,aps,secnumarabic,eqsecnum]{revtex4}

%\documentclass[titlepage]{article}
%\usepackage[sort&compress,numbers,comma]{natbib}
 
\begin{document}

\bibliographystyle{ijqc}
 
\title{Xample application of ijqc.bst}
 
\author{Richard J. Mathar}
\homepage{http://www.strw.leidenuniv.nl/~mathar}
\preprint{Int.\ J. Quant.\ Chem.\ 9x (x) (200x) xxx--xx}
\affiliation{
Sterrewacht, Leiden University, 2300RA Leiden, The Netherlands}

\date{\today}
 
\begin{abstract}
References created by the Bib\TeX\ style {\tt ijqc} are shown that
give an idea of the rendering of the most common types of entries.
The style is inclined towards the standards of Wiley's Int.\ J.\ Quant.\ Chem.

The BibTeX database that provides the input to what is shown here comes
with the file {\tt mybib.bib}; some
of the entries are erroneous (missing titles, page numbers\ldots)
and lead to complains from the {\tt bibtex} call.

key words: BibTeX; IJQC;
\end{abstract}
\maketitle
 
\section{Summary of the Style}
The characteristics of the {\tt ijqc} Bib\TeX\ file are
\begin{itemize}
\item
The order of appearance follows the mentioning in the text.
\item
Author lists are given with the last name first, separating
authors by a semicolon. The initials and the last name are separated
by a comma.
\item
Journal titles are stripped from dots that might indicate abbreviations.
The volume numbers are printed, but not the issue numbers.
\item
The bibliography entries are terminated by a full stop.
\item
Years are followed by the journal volume and page number. The final page
number is suppressed.
\end{itemize}

\section{article}

Articles with one author are \cite{Adler1962,BoettgerPRB62,LaiJOSAA8}.

Articles with two authors list are \cite{ZeissMP30,Abate,Arakane,Archer,EngelbergJOSAA21},
with more \cite{Adler,AlfaroJCompApplMath99}.

Articles with a long author list are \cite{GlindemannASS286,Baldiniarxiv04,JacquinetJQSRT62}. All authors are listed.

Articles with a {\tt note} entry are \cite{AlbrechtTheoChemAcc107}.

Articles with {\tt annote} or {\tt eprint} entries
are \cite{EIS,Iliev}. If an {\tt article} has both, a {\tt journal}
and an {\tt eprint} entry, only the journal is produced.

Articles without titles are \cite{Biersack82,Klopper} which does not matter
since titles are omitted anyway.
Articles with an {\tt url} are \cite{Carleer,Fliege},
but this could also be done in the {\tt journal}: \cite{MatharDia}.

Articles with an {\tt mrclass} or {\tt mrnumber} are \cite{Cody,Nitsche,Hasegawa1983}.
These entries do not effect the output.

The {\tt pages} may contain letters: \cite{Golek}, and the numbers
be combinations \cite{Helgaker}.

The MACRO list is short and does for example not know
{\tt pasp}: \cite{BergerPACS116}.

\section{book}

Standard entries are \cite{Wolfl,Taff,Bradley1972,Callaway,Green1985,Ince1956}.
If they have both {\tt author} and {\tt editor} they look like \cite{Nussbaumer,Hurwitz},
with only one {\tt editor} like \cite{AS}, and with two editors like \cite{Hog}.
If the {\tt mrnnumber} is present: \cite{GR,Rivlin,Byrd}.

Demonstration of {\tt isbn} entry: \cite{Messiah,Karttunen}.

With a {\tt series} entry we get \cite{Evarestov,ASILXXIV,Martensen},
with an {\tt edition} \cite{Press,Byrd,MO},
and with an {\tt annote} \cite{Born,Byrd,Green1985}.

Books with {\tt annote} entries are \cite{Derr,Dymond},
and with a {\tt volume} we have \cite{Hahn,Knopp,Martensen}.

\section{inbook}

The standard {\tt inbook} is \cite{Moran-76},
with an {\tt editor} and a {\tt volume} it looks like \cite{Fugate}.
If the title is missing we get \cite{Yarkony,Landolt}.
Adding a {\tt series} is \cite{HerringLN},

With {\tt chapter} and {\tt pages} : \cite{Browder}.

\section{inproceedings}

If {\tt inproceedings} include an {\tt annote}:
\cite{BergerSPIE4838,CassaingSPIE4006,KoehlerSPIE4006,KoehlerSPIE5491}

With a {\tt number}: \cite{PorroASP194}.
With a {\tt volume}: \cite{AkesonSPIE4006,MassonASP59}.

A sample case with a single editor is \cite{AlbrechtSPIE5491}
with two editors is \cite{Avila},
or without is \cite{AnkerstSIGMOD99,SPIE400605,CubySPIE3355,EsterKDD96}.

With a {\tt month} we have \cite{ChangSPIE3350,GorhamSPIE3350,DerieLC2002}
and without \cite{Cotton-95}.

\section{incollection}

With {\tt address} \cite{SawadaLNCS2517}
or without \cite{Cotton-97}.

{\tt author}, {\tt editor} and {\tt publisher} present
results in \cite{CreathProgOpt26,StrohbeinProgOpt9}, with
two {\tt editor}s we have \cite{MatharAQC45}.

If a {\tt series} is added we have \cite{TangoProgOpt17,RoddierProgOpt19}.

\section{theses}

Plain minimalistic theses are \cite{MatharPhd,Tatulli04,darcio99,Heidmann}.

Example with a {\tt month} entry: \cite{Levthesis}.
Example with a {\tt journal}, {\tt volume} and {\tt pages}
as if it were an article: \cite{MasuiMet34}.
Example with a {\tt pages} entry: \cite{MeisnerPhd,Peetz}.
Example with a {\tt annote} entry: \cite{buscher88,Chagnon}.

\section{unpublished}

Depending on whether {\tt howpublished} and {\tt title}
are present: \cite{Blaha,Hase,GaiWN041210}.

\section{techreport}

A {\tt techreport} having a {\tt url} looks like
\cite{Bos,Chance1,RuegerIAG1999,Sarazin},
if there is also a {\tt note} like \cite{McCarthyIERS32}.

A {\tt type} yields \cite{LordNASA1992}.

If the {\tt month} has been given: \cite{Sageatbd}

\section{manual}

A manual with {\tt address} and {\tt year} is \cite{PartrMemo}.

A disguised article with {\tt journal}, {\tt number} and {\tt pages}
is \cite{binFITS}.

A {\tt month} in the entry would be swallowed: \cite{vlt157530001}

Handling of {\tt url} and {\tt note}: \cite{Globalview}.

With authors, {\tt isbn} and {\tt url}: \cite{LAPACK}.

%\begin{acknowledgments}
%The work was supported in the middle.
%\end{acknowledgments}
% 
%\appendix
%
%\section{This is an appendix title}
 
\bibliography{mybib}
\end{document}

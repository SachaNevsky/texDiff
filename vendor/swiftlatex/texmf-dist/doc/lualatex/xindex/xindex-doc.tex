%% $Id: xindex-doc.tex 635 2017-10-24 18:22:59Z herbert $
%
\listfiles\setcounter{errorcontextlines}{100}
\documentclass[paper=a4,fontsize=11pt,DIV=13,parskip=half-,
               captions=tableabove,twoside=on]{scrartcl}
\usepackage{fontspec}
%\usepackage{libertinus-otf}
\setmainfont{AccanthisADFStdNo3}[
  UprightFont   =*-Regular,
  BoldFont      =*-Bold,
  ItalicFont    =*-Italic,
  BoldItalicFont=*-BoldItalic,
  RawFeature    = -rlig,
]
\setsansfont{GilliusADF}[
  UprightFont   =*-Regular,
  BoldFont      =*-Bold,
  ItalicFont    =*-Italic,
  BoldItalicFont=*-BoldItalic,
  RawFeature    = -rlig,
]
\setmonofont{DejaVu Sans Mono}[Scale=MatchLowercase,FakeStretch=0.9]
%\setmonofont{Anonymous Pro}[Scale=MatchLowercase,FakeStretch=0.9]

\newfontfamily\Japanese[Scale=MatchUppercase]{SourceHanSans}
\newfontfamily\CODE{CODE2000}

\usepackage[english]{babel}
\usepackage{scrlayer-scrpage}
\automark[section]{section}
\automark*[subsection]{}
\pagestyle{scrheadings}

{\catcode`\%=12 
 \gdef\perCent{%}
 \gdef\DoubleperCent
}


\usepackage{xcolor,fancyvrb,varioref}
\usepackage{xltabular}
\usepackage{dtk-logos} % for Wikipedia W
\usepackage{listings}
\usepackage{dtk-extern} % for Wikipedia W
\lstset{basicstyle=\ttfamily\small,language=[LaTeX]TeX}
\usepackage{ragged2e}
\usepackage{makeidx}\makeindex
\usepackage{hvindex}
\def\Lfile#1{\texttt{#1}\index{#1 file@\texttt{#1} file}}
\def\Lext#1{\texttt{.#1}\index{#1 file extension@\texttt{.#1} file extension}}
\def\Lcs#1{\texttt{\textbackslash#1}\index{#1@\texttt{\textbackslash#1}}}
\def\Lenv#1{\texttt{#1}\index{#1 environment@\texttt{#1} environment}}
\def\Lpack#1{\texttt{#1}\index{#1 package@\texttt{#1} package}}
\def\Lprog#1{\texttt{#1}\index{#1 program@\texttt{#1} program}}
\def\Loption#1{\texttt{#1}\index{#1@\texttt{#1} package option}}

%\usepackage[bibstyle=dtk]{biblatex}
%\addbibresource{\jobname.bib}

\raggedbottom
\usepackage{url}
\usepackage[colorlinks,linktocpage]{hyperref}

\makeatletter% from: doc.sty
\newcommand*\GetFileInfo[1]{%
  \def\filename{#1}%
  \def\@tempb##1 ##2 ##3\relax##4\relax{%
    \def\filedate{##1}%
    \def\fileversion{##2}%
    \def\fileinfo{##3}}%
  \edef\@tempa{\csname ver@#1\endcsname}%
  \expandafter\@tempb\@tempa\relax? ? \relax\relax}
\makeatother

%\GetFileInfo{xltabular.sty}
\usepackage{xspace}

\newcommand\XI{\texttt{xindex}\xspace}

\def\setVersion#1{\setVVersion#1!!}
\def\setVVersion#1=#2!!{\def\xIndexVersion{#2}} 

\setVersion{version = 0.20}%  can be autimatically changed by perl

\setkeys{dtk}{cleanup=true,force=false}


\title{Program and package \texttt{xindex} \\--\\ \normalsize \xIndexVersion\ (\today)}
\author{\shortstack{Herbert Voß\\\small \href{mailto:hvoss@tug.org}{hvoss@tug.org}}}
\begin{document}
\maketitle
\tableofcontents

\vfill
Thanks to: \\
Mark Baudoin;
Heiko Oberdiek

\clearpage

\section{Introduction}
The Lua program
\XI is a  \Index{unicode} aware program for creating an index file from an \Lext{idx} source
file. It is completely compatible to the current \Lprog{makeindex} program, but can handle
\Index{UTF-8}, 16, 32, and 64. The \LaTeX\ package \Lpack{xindex} is a package which provides
a \LaTeX\ command which writes additional text into the index file. This text (comments and/or macros)
will be accepted by the program \XI.

The general structure of a \Index{data element} in the Lua table is:

\begin{verbatim}
data = { Entry = <text>,  -- like the input line without command \indexentry
         pages = {
                    { number  = <roman/arabic number or text>,
                      special = <macro> },  -- the part after | in the input
                    [...]
                    { number  = <roman/arabic number or text>,
                      special = <macro> }
                  },
          sortChar = <unicode codepoint>, -- of the first character of Entry
          Macro    = <TeX macro>  -- only useful with LaTeX package \Lpack{xindex}
        }
\end{verbatim}

After reading the input file the table \texttt{pages} has only one element for the number\index{page number}
and the so-called special command. When the pages are compressed the table will collect all pages which
refer to the same \Index{entry name}.
 

\iffalse
EntryList[2] = { 
      Entry   = "Johann",   -- the index item  foo@bar 
      pages   = {           -- the page number(s) and the part after |
        { number  = 111,
          special = '\\emph'  },
        { number  = 11,
          special = "\\textit"}
      },  
      sortChar= 80, -- Initial for later output
      Macro   = ""
}

\fi



\subsection{Syntax}

The \Index{syntax} is \verb|xindex [...] <file>| where \verb|[...]|  are optional \Index{argument}s, either in short or
long form which, of course, can be mixed:

\begin{verbatim}
xindex
    [-q,--quiet ]
    [-h,--help ]
    [-v ]                                 verbose
    [-c,--config ]                        default is cfg
    [-e,--escapechar ]                    default is " 
    [-a, --nocasesensitive ]              default is false
    [-n,--noheadings ]                    default is false
    [-o,--output ]                        default is <input>.ind
    [-l,--language ]                      default is en 
    [-p,--prefix]                         default L
    <input file> 
\end{verbatim}



The \Index{language} has to be chosen as an international abbreviation in lower- or uppercase letters, see 
\url{https://en.wikipedia.org/wiki/ISO_3166-2}


\subsection{How it works}
\XI creates by default an \Index{output} file \texttt{<input>.ind} which can be read
by the \LaTeX\ document with the default command \Lcs{printindex}. One can use another
output filename, which makes only sense if one doesn't use the \Lcs{printindex} command
for typesetting the index.
The default \Index{sorting}
is given by the configuration file, which defines replacements for \Index{accented characters},
like \texttt{ö}$\rightarrow$\texttt{o}. 

\subsection{The \texttt{.idx} file}
There are four characters which must be escaped if used in the command \Lcs{index}:
\verb=!=, \verb=@=, \verb="=, or \verb=|=. These charactzers have a special meaning for 
the index.\index{"!}\index{"@}\index{"|}
The default \Index{escape character} is the double quote \verb|"|:

\iffalse
To put a !\index{"!}, @\index{"@}, or |\index{"|} character in an index entry, quote it by preceding the
character with a quote ". More precisely, any character is said to be quoted if it
follows an unquoted " that is not part of a \" command. A quoted !, @, or
| character is treated like an ordinary character rather than having its usual
meaning. The " preceding a quoted character is deleted before the entries
are alphabetized.
\fi


\begin{externalDocument}[
%  grfOptions={width=\dimexpr\linewidth-2\fboxsep-2\fboxrule},
  pages={1,2},
  frame,
  compiler=lualatex,
  crop,
  xindex,
  force=false,
  runs=2,code,docType=latex,
  showFilename,
  align=\centering,
  lstOptions={columns=fixed}]{xindex}
\documentclass{scrartcl}
\makeatletter
\def\theindex{%    only for demonstration
  \section*{\indexname}
  \parskip\z@ \@plus .3\p@\relax \parindent\z@
  \let\item\@idxitem}
\makeatother

%StartVisiblePreamble
\usepackage{makeidx}\makeindex
%StopVisiblePreamble
\pagestyle{empty}
\begin{document}
\section{Escaping characters}
\begin{itemize}
\item Exclamation mark ! \index{exclaim ("!)}
\item Vertical bar| \index{Vertical bar ("|)}
\item Doublequote \verb|"| \index{""}
\item Double doublequote \verb|""| \index{""""}
\item At character @ \index{At ("@)}
\end{itemize}
run \texttt{xindex <file.idx>}\index{<file.idx>}\index{123}
\newpage
\printindex
\end{document}
\end{externalDocument}


For the german language the double quote is an active character and it makes live easier
if one chooses another character.
The escape character can be changed easily by the optional argument \verb|-e "<char>"| or 
\verb|--escapechar "<char">|. The following example shows how
it works for the \Index{escape character} > (greater). Internally the escape sequences are
now defined as:

\begin{verbatim}
escape_chars = { -- by default " is the escape char
  {esc_char..'"', '//escapedquote//', '\\"{}' },
  {esc_char..'@', '//escapedat//',    '@'    },
  {esc_char..'|', '//escapedvert//',  "|"    },
  {esc_char..'!', '//scapedexcl//',  '!'    }
}
\end{verbatim}

which is, of course, not of interest for the user. With the beginning the escaped chars
are converted into the internal strings and later back to the origin meaning.


\begin{externalDocument}[
%  grfOptions={width=\dimexpr\linewidth-2\fboxsep-2\fboxrule},
  pages={1,2},
  frame,
  compiler=lualatex,
  crop,
  xindex,
  xindexOptions=-e ">",
  force=false,
  runs=2,code,docType=latex,
  showFilename,
  align=\centering,
  lstOptions={columns=fixed}]{xindex}
\documentclass{scrartcl}
\makeatletter
\def\theindex{%    only for demonstration
  \section*{\indexname}
  \parskip\z@ \@plus .3\p@\relax \parindent\z@
  \let\item\@idxitem}
\makeatother
%StartVisiblePreamble
\usepackage{makeidx}\makeindex
%StopVisiblePreamble
\pagestyle{empty}
\begin{document}
\section{Escaping characters with >}
\begin{itemize}
\item Exclamation mark ! \index{exclaim (>!)}
\item Vertical bar| \index{Vertical bar (>|)}
\item Doublequote \verb|"| \index{>"}
\item Double doublequote \verb|""| \index{>">"}
\item At character @ \index{At (>@)}
\end{itemize}
Run \texttt{xindex} with \texttt{xindex -e ">"}\index{<file.idx>}\index{123}
\newpage
\printindex
\end{document}
\end{externalDocument}


\section{Language}\label{language}

The \Index{language} is only important for the first two headers in the output of the index data.
They are by default \emph{Symbols} followed by \emph{Numbers}. In a new version of \Lprog{xindex}
it will be customizable.
The predefined language is »en« and currently the following languages are possible:

{\Japanese
\begin{Verbatim}[fontfamily=helvetica,commandchars=\\<>]
indexheader = { 
  de = {"Symbole", "Zahlen"},
  en = {"Symbols", "Numbers"},
  fr = {"Symboles","Chiffre"},
\Japanese  jp = {"シンボル","番号"},
}
\end{Verbatim}
}

The following example was run with \verb|xindex -l de <file>.idx|:

\begin{externalDocument}[
%  grfOptions={width=\dimexpr\linewidth-2\fboxsep-2\fboxrule},
  pages={1,2},
  frame,
  compiler=lualatex,
  crop,
  xindex,
  xindexOptions=-l de -e ">",
  force=false,
  runs=2,code,docType=latex,
  showFilename,
  align=\centering,
  lstOptions={columns=fixed}]{xindex}
\documentclass{scrartcl}
\makeatletter
\def\theindex{%    only for demonstration
  \section*{\indexname}
  \parskip\z@ \@plus .3\p@\relax \parindent\z@
  \let\item\@idxitem}
\makeatother
%StartVisiblePreamble
\usepackage{makeidx}\makeindex
%StopVisiblePreamble
\pagestyle{empty}
\begin{document}
\section{Escaping characters with >}
\begin{itemize}
\item Exclamation mark ! \index{exclaim (>!)}
\item Vertical bar| \index{Vertical bar (>|)}
\item Doublequote \verb|"| \index{>"}
\item Double doublequote \verb|""| \index{>">"}
\item At character @ \index{At (>@)}
\end{itemize}
Run \texttt{xindex} with \texttt{xindex -l de -e ">"}\index{123}
\newpage
\printindex
\end{document}
\end{externalDocument}


\section{Sorting}

The default sorting is unicode aware and uses a translation table for accented characters:

\begin{verbatim}
alphabet_lower = { --   for sorting
    { ' ' },  -- only for internal tests
    { 'a', 'á', 'à', 'ä'},
    { 'b' },
    { 'c' },
    { 'd' },
    { 'e', 'é', 'è', 'ë' },
    { 'f' },
    { 'g' },
    { 'h' },
    { 'i', 'í', 'ì', 'ï' },
    { 'j' },
    { 'k' },
    { 'l' },
    { 'm' },
    { 'n', 'ñ' },
    { 'o', 'ó', 'ò', 'ö' },
    { 'p' },
    { 'q' },
    { 'r' },
    { 's' },
    { 't' },
    { 'u', 'ú', 'ù', 'ü' },
    { 'v' },
    { 'w' },
    { 'x' },
    { 'y' },
    { 'z' }
}
\end{verbatim}

There is also a table for the uppercase letters. If it should be edited or extended then copy first the
base configuration file \Lfile{xindex-cfg.lua} and modify that new file. It can be used by \XI
with the optional argument \texttt{-c newfile} if it is named as \Lfile{xindex-newfile.lua}. 
For german there already exists a configuration
file \Lfile{xindex-DIN2.lua} which uses the so-called »Telefonbuchsortierung« which converts 
the umlauts like ö$\rightarrow$oe:

\begin{verbatim}
alphabet_upper = { -- for sorting
    { ' ' },
    { 'A', 'Á', 'À', 'Ä'},
    { 'B' },
    { 'C' },
    { 'D' },
    { 'E', 'È', 'È', 'ë' },
    { 'F' },
    { 'G' },
    { 'H' },
    { 'I', 'Í', 'Ì', 'ï' },
    { 'J' },
    { 'K' },
    { 'L' },
    { 'M' },
    { 'N', 'Ñ' },
    { 'O', 'Ó', 'Ò', 'Ö' },
    { 'P' },
    { 'Q' },
    { 'R' },
    { 'S' },
    { 'T' },
    { 'U', 'Ú', 'Ù', 'Ü' },
    { 'V' },
    { 'W' },
    { 'X' },
    { 'Y' },
    { 'Z' }
}
\end{verbatim}



\begin{externalDocument}[
  grfOptions={scale=0.9},%width=\dimexpr\linewidth-2\fboxsep-2\fboxrule},
  mpwidth=0.4\linewidth,
  pages={2},
  frame=false,
  compiler=lualatex,
  crop,
  xindex,
  xindexOptions=-l de,
  force=false,
  runs=2,code,docType=latex,
  showFilename,
  align=\centering,
  lstOptions={columns=fixed}]{xindex}
\documentclass[paper=a5]{scrartcl}
\usepackage[ngerman]{babel}
%StartVisiblePreamble
\usepackage{makeidx}\makeindex
\newcommand\Index[1]{\index{#1}#1}
%StopVisiblePreamble
\usepackage{multicol}
\makeatletter
\def\theindex{%    only for demonstration
  \newpage
  \section*{\indexname}
  \begin{multicols}{2}
  \parskip\z@ \@plus .3\p@\relax \parindent\z@
  \let\item\@idxitem}
\def\endtheindex{\end{multicols}}
\makeatother
\pagestyle{empty}
\begin{document}
Sorted with \verb|-l DE|

\Index{Österreich} \Index{Öresund} 
\Index{Ostern} \Index{Ober} 
\Index{Oberin} \Index{Österreich} 
\Index{Öresund} \Index{Ostern} 
\Index{Ober} \Index{Oberin} 
\Index{Obstler} \Index{Öl} 
\Index{ölen} \Index{Ödem} 
\Index{Oligarch} \Index{Oder} 
\Index{oder} \index{Fluss!Oder} 
\index{Oder|seealso{Fluss}} 
\Index{Göbel} \Index{Goethe} 
\Index{Göthe} \Index{Götz} 
\Index{Goldmann}

\printindex
\end{document}
\end{externalDocument}



The same sorted with the german DIN variant 2 with \verb|--config DIN2|, which
is part of the \TeX\ distribution. In this case a letter Ö is converted to Oe before sorting
the word beginninjg with the letter Ö:

\noindent
\begin{externalDocument}[
  grfOptions={scale=0.9},%width=\dimexpr\linewidth-2\fboxsep-2\fboxrule},
  mpwidth=0.4\linewidth,
  pages={2},
  frame=false,compiler=lualatex,
  crop,
  xindex,
  xindexOptions=-c DIN2 -l DE,
  force=false,
  runs=2,code,docType=latex,
  showFilename,
  align=\centering,
  lstOptions={columns=fixed}]{xindex}
\documentclass[paper=a5]{scrartcl}
\usepackage[ngerman]{babel}
%StartVisiblePreamble
\usepackage{makeidx}\makeindex
\newcommand\Index[1]{\index{#1}#1}
%StopVisiblePreamble
\usepackage{multicol}
\makeatletter
\def\theindex{%    only for demonstration
  \newpage
  \section*{\indexname}
  \begin{multicols}{2}
  \parskip\z@ \@plus .3\p@\relax \parindent\z@
  \let\item\@idxitem}
\def\endtheindex{\end{multicols}}
\makeatother
\pagestyle{empty}
\begin{document}
Sorted with 
\verb|--config DIN2 -l DE|

\Index{Österreich} \Index{Öresund} 
\Index{Ostern} \Index{Ober} 
\Index{Oberin} \Index{Österreich} 
\Index{Öresund} \Index{Ostern} 
\Index{Ober} \Index{Oberin} 
\Index{Obstler} \Index{Öl} 
\Index{ölen} \Index{Ödem} 
\Index{Oligarch} \Index{Oder} 
\Index{oder} \index{Fluss!Oder} 
\index{Oder|seealso{Fluss}} 
\Index{Göbel} \Index{Goethe} 
\Index{Göthe} \Index{Götz} 
\Index{Goldmann}

\printindex
\end{document}
\end{externalDocument}



The following runs with \verb|xindex -l jp <file>|:

%  {begin\string{document\string}}%       #2
%  {end\string{document\string}}%         #3
%  {\perCent StartVisiblePreamble}% #4
%  {\perCent StopVisiblePreamble}%  #5

\noindent
\begin{minipage}[t]{0.45\linewidth}
\vspace{0pt}
\setsansfont{SourceHanSans}\sffamily
\edef\FancyVerbStartString{\perCent StartVisiblePreamble}
\edef\FancyVerbStopString{\perCent StopVisiblePreamble}
\colorbox{black!20}{\parbox{\linewidth}{%
\VerbatimInput[fontfamily=helvetica]{Examples/xindex-6.tex}
}}

\edef\FancyVerbStartString{\string\begin\string{document\string}}
\edef\FancyVerbStopString{\string\end\string{document\string}}
\colorbox{black!10}{\parbox{\linewidth}{%
\VerbatimInput[fontfamily=helvetica]{Examples/xindex-6.tex}
}}
\end{minipage}\hfill
\begin{minipage}[t]{0.54\linewidth}
\vspace{0pt}
\begin{externalDocument}[
%  grfOptions={width=\dimexpr\linewidth-2\fboxsep-2\fboxrule},
%  mpwidth=0.5\linewidth,
  pages={2},
  frame=false,
  compiler=lualatex,
  crop,
  xindex,
  xindexOptions=-l jp,
  force=false,
  runs=2,
  code=false,
  docType=latex,
  showFilename,
  align=\centering,
%  usefancyvrb,
  lstOptions={fontfamily=helvetica}]{xindex}
\documentclass[a5paper]{article}
%StartVisiblePreamble
\usepackage{fontspec}
\setmainfont{SourceHanSans}
\usepackage[japanese]{babel}
\addto\captionsjapanese{%
  \def\indexname{指数}}
\usepackage{hvindex}% for \Index
\usepackage{makeidx}\makeindex
%StopVisiblePreamble
\usepackage{multicol}
\makeatletter
\def\theindex{%    only for demonstration
  \section*{\indexname}  
  \begin{multicols}{2}
  \parskip\z@ \@plus .3\p@\relax \parindent\z@
  \let\item\@idxitem}
\def\endtheindex{\end{multicols}}
\makeatother
\pagestyle{empty}
\begin{document}
\Index{車} \Index{車道} 
日本\index{日本|fbox} 
\Index{病院} \Index{コンピュータ} 
\Index{プリンタ} \Index{印刷} 
\Index{スイミングプール} \Index{天王} 
\Index{広島} \Index{ドイツ} 
\Index{日本} \Index{ワープロ} 
\Index{foo} und \Index{bar}
\Index{//} \Index{4711}
\newpage\printindex
\end{document}
\end{externalDocument}
\end{minipage}



%\clearpage

\section{Compressing pagenumbers}\label{Compressing}
By default page sequences of an entry are compressed to 
\begin{description}
\item[8f] page 8 and 9
\item[8ff] page 8, 9, and 10
\item[8-12] page 8, 9, \ldots, 12
\end{description}

The so-called folio abbreviation is language dependent and defined in the
file \Lfile{xindex-cfg-common.cfg}:

{\setsansfont[Scale=MatchUppercase]{SourceHanSans}
\small
\begin{Verbatim}[fontfamily=helvetica]
folium = { 
  de = {"f", "ff"},
  en = {"f", "ff"},
  fr = {"\,sq","\,sqq"},
  jp = {"シンボル","番号"},
}
\end{Verbatim}
}

\begin{externalDocument}[
%  grfOptions={width=\dimexpr\linewidth-2\fboxsep-2\fboxrule},
  mpwidth=0.55\linewidth,
  pages={6},
  frame=false,
  compiler=lualatex,
  crop,
  xindex,
  xindexOptions=-l fr,
  force=false,
  runs=2,code,docType=latex,
  showFilename,
  align=\centering,
  lstOptions={columns=fixed}]{xindex}
\documentclass{scrartcl}
\makeatletter
\def\theindex{%    only for demonstration
  \section*{\indexname}
  \parskip\z@ \@plus .3\p@\relax \parindent\z@
  \let\item\@idxitem}
\makeatother
%StartVisiblePreamble
\usepackage{makeidx}\makeindex
%StopVisiblePreamble
\pagestyle{empty}
\begin{document}
Sorted with \verb|-l fr|

foobar\index{foobar|(}
foo\index{foo}\index{bar}\index{baz}\newpage
foo\index{foo}\index{bar}\index{baz}\newpage
foo\index{bar}\index{baz}\newpage
foo\index{baz}\newpage
foo\index{foo}foobar\index{foobar|)}
\newpage
\printindex
\end{document}
\end{externalDocument}



\section{\texttt{hyperref}}

Using the package \Lpack{hyperref} is no problem:

\enlargethispage{2.5\normalbaselineskip}

\begin{externalDocument}[
%  grfOptions={width=\dimexpr\linewidth-2\fboxsep-2\fboxrule},
  mpwidth=0.55\linewidth,
  pages={3},
  frame=false,
  compiler=lualatex,
  crop,
  xindex,
  xindexOptions=-l de,
  force,
  runs=2,code,docType=latex,
  showFilename,
  align=\centering,
  lstOptions={columns=fixed}]{xindex}
\documentclass[paper=a6]{scrartcl}
\usepackage{multicol}
\makeatletter
\def\theindex{%    only for demonstration
  \section*{\indexname}
  \begin{multicols}{2}
  \parskip\z@ \@plus .3\p@\relax \parindent\z@
  \let\item\@idxitem}
\def\endtheindex{\end{multicols}}
\makeatother
%StartVisiblePreamble
\usepackage{makeidx}\makeindex
\usepackage{hvindex}% for \Index
\usepackage[colorlinks]{hyperref}
%StopVisiblePreamble
\pagestyle{empty}
\begin{document}
Sorted with \verb|-l DE|
\Index{Österreich} \Index{Öresund} 
\Index{Ostern} \Index{Ober} \Index{Oberin} 
\Index{Österreich} \index{Öresund|textbf} 
\Index{Ostern} \Index{Ober} \Index{Oberin} 
\Index{Obstler} \Index{Öl} \Index{ölen} 
\Index{Ödem} \Index{Oligarch} \Index{Oder} 
\Index{oder} \index{Fluss!Oder|textit} 
\Index{Oder|seealso{Fluss}} \Index{Göbel} 
\Index{Goethe} \Index{Göthe} \Index{Götz} 
\newpage\Index{Goldmann} \Index{Goethe}
\newpage \printindex
\end{document}
\end{externalDocument}


\section{Page argument}
Every page can be combined with an additional macro, like \verb=\index{foo|fbox}=, the page number
will be set into a framebox. If we have on the same page the two commands:

\begin{verbatim}
foo\index{foo} and foo\index{foo|bar}
\end{verbatim}

then we have two \emph{different} index entries which will not be compressed to one entry.
In the follwoing example we have four different entries for \emph{foo} which is the reason
that we do not get an output like \texttt{foo, 1--4}. Only the first two entries are
of the same type, so we get \texttt{1f} in the output.

\begin{externalDocument}[
%  grfOptions={width=\dimexpr\linewidth-2\fboxsep-2\fboxrule},
  mpwidth=0.55\linewidth,
  pages={5},
  frame=false,
  compiler=lualatex,
  crop,
  xindex,
  xindexOptions=-l de,
  force=false,
  runs=2,code,docType=latex,
  showFilename,
  align=\centering,
  lstOptions={columns=fixed}]{xindex}
\documentclass{scrartcl}
\makeatletter
\def\theindex{%    only for demonstration
  \section*{\indexname}
  \parskip\z@ \@plus .3\p@\relax \parindent\z@
  \let\item\@idxitem}
\makeatother
%StartVisiblePreamble
\usepackage{makeidx}\makeindex
%StopVisiblePreamble
\pagestyle{empty}
\begin{document}
Ein foo\index{foo} \newpage und \index{foo} 
ein foo\index{foo|textit} \newpage 
und foo\index{foo|textbf} \newpage 
und foo\index{foo|fbox}

\newpage
\printindex
\end{document}
\end{externalDocument}






\section{The config file}

The main config file is \Lfile{xindex-cfg.lua} is used by default and loading it by the 
optional parameter -c makes no sense. A new config file must have the prefix \texttt{xindex-}
and the file extension \Lext{lua}, for example: \Lfile{xindex-HAdW-eKO.lua} which can be used
with \verb|--config HAdW-eKO|. The file must be saved in the documents directory or in
one which is known to \Lprog{kpsewhich}, for example\footnote{The directory \texttt{xindex} must be created
before saving the file.} \verb|$TEXMFLOCAL/tex/lualatex/xindex/| %$
Don not forgot to update the filename database.

A new config file must declare at least the variables which are part of the
default config file: the translation tables and

\begin{verbatim}
itemPageDelimiter = ","     -- Hello, 14
compressPages     = true    
  -- something like 12--15, instaead of 12,13,14,15. the |( ... |) syntax is still valid
fCompress	  = true    -- 3f -> page 3, 4 and 3ff -> page 3, 4, 5
minCompress       = 3       -- 14--17 or 
rangeSymbol       = "--"    
numericPage       = true    -- for non numerical pagenumbers, like "VI-17"
sublabels         = {"", "-\\-", "--\\-", "---\\-"} 
   -- for the sub(sub(sub-items, first one is empty
pageNoPrefixDel   = ""     -- a delimiter for page numbers like "VI-17"  -- not used !!!
indexOpening      = ""     -- commands/text after \begin{theindex}
\end{verbatim}

The new config file can define own functions for compressing the pagelist
for a given entry and for the formatting of the output. They must be called 
\verb|specialCompressPageList| and  \verb|specialGetPageList|.
%needs
%the unmodified list of pages for one entry. 

For example:

\begin{verbatim}
function specialCompressPageList(pages)
  if (pages[1]["number"] == "") then pages[1]["number"] = " " end
  if (#pages <= 1) then 
    pages[1]["number"] = pages[1]["number"]:gsub('-',':~')-- replace "-" with ":~"
    return pages 
  end  -- only one pageno
  local sortPages = {}
  local roman 
  local volume
  local page
  local i
  for i=1,#pages do
     roman = string.gsub(pages[i]["number"],'%U*','') -- only uppercase to catch VII/1-123f and VII/3-123ff (folium pages)
     if romanToNumber(roman) then
       roman = string.format("%05d",tonumber(romanToNumber(roman))) -- only roman part VII
     else
       roman = ""
     end
     volume = string.gsub(pages[i]["number"],'%a*','') -- only the number /2 123 or /2-123
     if volume then volume = volume:gsub('-%d*','') end -- delete - char to get /2
     page = string.gsub(pages[i]["number"],'.*-','')
     page = string.format("%5s",page)
     sortPages[#sortPages+1] = {
       origin = pages[i],
       sort = roman..volume.." "..page }  -- no minus between Roman/Volume and first page
   end
  table.sort(sortPages, function(a,b) return a["sort"] < b["sort"] end )

[...]

    return pages
  end
end
\end{verbatim}

is a special function which can handle \Index{page number}s like VII-17, VIII/2/1-186.
Internally exists a function \verb|compressPageList| which is used if no
\verb|specialCompressPageList| is defined.

\begin{externalDocument}[
  grfOptions={scale=0.9},% width=\dimexpr\linewidth-2\fboxsep-2\fboxrule},
%  mpwidth=0.25\linewidth,
  pages={2},
  frame=false,
  compiler=lualatex,
  crop,
  xindex,
  xindexOptions=-c HAdW-eKO,
  force,
  runs=2,code,docType=latex,
  showFilename,
  align=\centering,
  lstOptions={columns=fixed}]{xindex}
\RequirePackage{filecontents}
\begin{filecontents*}{\jobname.idx}
\indexentry{Aachen, Johannes von}{VII/1-215}
\indexentry{Aarones}{VII/2/1-1003}
\indexentry{Aarones}{VII/2/1-1012}
\indexentry{Abrahamson}{VII/2/1-864}
\indexentry{Abrahamson}{VII/2/1-991}
\indexentry{Abrahamson}{VII/2/1-1048}
\indexentry{Abrahamson}{VII/2/1-1067}
\indexentry{Abrahamson}{VII/2/1-1156}
\indexentry{Adamson}{VII/2/1-1223}
\indexentry{Adamson}{IX/1-1228}
\indexentry{Karl!der Große, Kaiser}{VI/2-987}
\indexentry{Karl!der Große, Kaiser}{VI/2-989}
\indexentry{Karl!der Große, Kaiser}{VI/2-1028}
\indexentry{Karl!IV., Kaiser}{VI/1-12}
\indexentry{Karl!V., Kaiser}{VI/1-84}
\indexentry{Karl!V., Kaiser}{VI/1-284}
\indexentry{Karl!V., Kaiser}{VI/1-654}
\indexentry{Karl!V., Kaiser}{VI/2-708}
\indexentry{Karl!V., Kaiser}{VI/2-1014}
\indexentry{Karl!V., Kaiser}{VI/2-1043}
\indexentry{Karl!V., Kaiser}{VI/2-1131}
\indexentry{Karl!V., Kaiser}{VI/2-1210}
\indexentry{Braunschweig-Wolfenbüttel!Karl Viktor von, Herzog}{VI/1-83}
\indexentry{Schleswig-Holstein!Rudolf von, Herzog}{VII/2/1-758}
\indexentry{Schleswig-Holstein!Rudolf von, Herzog}{VII/2/1-759}
\indexentry{Schleswig-Holstein!Rudolf von, Herzog}{VII/2/1-760}
\indexentry{Schleswig-Holstein!Rudolf von, Herzog}{VII/2/1-761}
\indexentry{Schleswig-Holstein!Rudolf von, Herzog}{VII/2/1-765}
\indexentry{Adrian!Hauster}{VII/1-514}
\indexentry{Adrian!Hauster}{XI/1-515}
\indexentry{Alting!Mensa}{VII/1-426}
\indexentry{Alting!Mensa}{VII/1-434}
\indexentry{Alting!Mensa}{VII/1-453}
\indexentry{Alting!Mensa}{VII/1-455}
\indexentry{Alting!Mensa}{VII/1-466}
\indexentry{Alting!Mensa}{VII/1-467}
\indexentry{Bremen!Heinz von, Erzbischof|see{Sachsen-Lauenburg}}{}
\indexentry{Osnabrück!Heinz  von, Bischof|see{Sachsen-Lauenburg}}{}
\indexentry{Zwingl, Haldrich}{IX-479}
\indexentry{Zwingl, Haldrich}{IX-692}
\indexentry{Julian!Apostata, römischer Kaiser}{VII/2/1-904}
\indexentry{Justinian I., byzantinischer Kaiser}{VII/1-326}
\indexentry{Justinian I., byzantinischer Kaiser}{VII/1-734}
\indexentry{Justinian I., byzantinischer Kaiser}{VII/2/1-1011}
\indexentry{Karl!V., Kaiser}{VII/1-34}
\indexentry{Karl!II., Kaiser}{VII/1-147}
\indexentry{Karl!III., Kaiser}{VII/1-149}
\indexentry{Karl!VI., Kaiser}{VII/1-296}
\indexentry{Karl!IV., Kaiser}{VII/1-34}
\indexentry{Karl!IV., Kaiser}{VII/1-147}
\indexentry{Karl!X., Kaiser}{VII/1-149}
\indexentry{Karl!IX., Kaiser}{VII/1-296}
\end{filecontents*}
\documentclass[a4paper]{article}
\usepackage[margin=1cm]{geometry}
\usepackage{xcolor}
\usepackage{url}
\usepackage{multicol}
\makeatletter
\def\theindex{%    only for demonstration
  \newpage
  \section*{Personenverzeichnis}
  \begin{multicols}{2}
  \parskip\z@ \@plus .3\p@\relax \parindent\z@ \raggedright
  \let\item\@idxitem}
\def\endtheindex{\end{multicols}}
\makeatother
%StartVisiblePreamble
\usepackage{makeidx}
%StopVisiblePreamble
\pagestyle{empty}
\begin{document}
\mbox{}\printindex
\end{document}
\end{externalDocument}






The \Index{config file} \Lfile{xindex-dtk.lua} defines a special page output:

\begin{verbatim}
function specialGetPageList(v,hyperpage)  -- Entry table, boolean
  local Pages = {}
[..]
      if (Pages[1]["special"] == nil) or (Pages[1]["number"] == nil) then return ""  end 
      if #Pages == 1 then
        return "\\relax"..Pages[1]["number"].."\\@nil"
      else        
        pageNo = "\\relax"..Pages[1]["number"] 
        for i=2,#Pages do
          if Pages[i]["number"] then
            pageNo = pageNo..", "..Pages[i]["number"].."\\@nil"
            Pages[i] = {}
          end
        end
[..]
end
\end{verbatim}


The following example runs \verb|xindex -c dtk -l de -n <input>|

\begin{externalDocument}[
  grfOptions={scale=0.9},% width=\dimexpr\linewidth-2\fboxsep-2\fboxrule},
  mpwidth=0.25\linewidth,
  pages={1},
  frame=false,
  compiler=lualatex,
  crop,
  xindex,
  xindexOptions=-c dtk -n -l de,
  force,
  runs=2,code,docType=latex,
  showFilename,
  align=\centering,
  lstOptions={columns=fixed}]{xindex}
\RequirePackage{filecontents}
\begin{filecontents*}{\jobname.idx}
\indexentry{VoßHerbert@Herbert Voß!Wasgenstraße 121\protect \\10127 Potsdam\protect \\\Email {herbert"@xyz.de}}{3}
\indexentry{SeversMartin@Martin Severs!siehe Seite~\protect \pageref  {president}}{4}
\indexentry{VoßHerbert@Herbert Voß!Wasgensteig 12\protect \\10127 Potsdam\protect \\\Email {herbert"@xyz.de}}{5}
\indexentry{ZiegendatenMichael@Michael Ziegendaten!Lokostr. 19 \protect \\ 20713 Kalln \protect \\\Email {ziegendaten"@mail.com}}{9}
\indexentry{BährendtsenElke@Elke Baehrendtsen!\Email {dori"@xyz.de}}{14}
\indexentry{JacekJonasson Jared@Jonasson Jared Jazek!\Email {mail"@jones.net}}{20}
\indexentry{KoomerMartin@Martin Koomer!Freiherr-von-Stein-Weg~16\protect \\ 15525~Erdingen-Neckar\protect \\ \Email {kooma"@xyz.info}}{24}
\indexentry{KoomerMartin@Martin Koomer!Freiherr-von-Stein-Weg~16\protect \\ 15525~Erdingen-Neckar\protect \\ \Email {kooma"@xyz.info}}{31}
\indexentry{SchusterEike@Eike Schuster!Haussteig~15\protect \\ 36396~Stuttens\protect \\ \Email {elke.schuster"@kabelxyz.de}}{40}
\indexentry{FanntHorst@Horst Fannt!Friedrichallee 74\protect \\13233 Neu-Isenburg\protect \\\Email {juergen.fannt"@gmxnet.de}}{48}
\end{filecontents*}
\documentclass{article}
\usepackage{url}
\DeclareUrlCommand\Email{%
  \def\UrlLeft{}%
  \def\UrlRight{}%
  \def\UrlLinkPrefix{mailto:}%
  \def\UrlType{email}%
}
\usepackage{multicol}
\makeatletter
\def\DTK@scan@item#1\subitem#2\relax#3\@nil{%
  \def\DTK@tempa{#1}\def\DTK@tempb{#2}\def\DTK@tempc{#3}%
}
\def\theindex{%    only for demonstration
  \columnseprule=\z@ \columnsep=10\p@
  \begin{multicols}{2}[\noindent\textbf{\large Autorenliste}]%
    \makeatletter
    \def\indexspace{}%
    \parindent\z@
    \setlength{\parskip}{\z@ \@plus .3\p@}%
    \setlength{\parfillskip}{\z@ \@plus 1fil}%
    \raggedright
    \def\item##1\@nil{\DTK@scan@item##1\@nil
      \par\parbox{\columnwidth}{%
        \textbf{\DTK@tempa}\hfill[\DTK@tempc]\par\DTK@tempb
      }%
      \par\bigskip
    }%
}
\def\endtheindex{\end{multicols}}
\makeatother
%StartVisiblePreamble
\usepackage{makeidx}
%StopVisiblePreamble
\pagestyle{empty}
\begin{document}
\mbox{}\ref{president}
\printindex
\end{document}
\end{externalDocument}


\subsection{Sublabels}
There are three predefined sublabels for \Lcs{subitems}. The program itself can handle more, there is
no limit for \Lprog{xindex}.

\begin{externalDocument}[
%  grfOptions={width=\dimexpr\linewidth-2\fboxsep-2\fboxrule},
  mpwidth=0.55\linewidth,
  pages={2},
  frame=false,
  compiler=lualatex,
  crop,
  xindex,
%  xindexOptions=,
  force=false,
  runs=2,code,docType=latex,
  showFilename,
  align=\centering,
  lstOptions={columns=fixed}]{xindex}
\documentclass{article}
\makeatletter
\def\theindex{%    only for demonstration
  \section*{\indexname}
  \pagestyle{empty}%
  \parskip\z@ \@plus .3\p@\relax \parindent\z@
  \let\item\@idxitem}
\makeatother
\pagestyle{empty}%
%StartVisiblePreamble
\makeatletter
\g@addto@macro{\theindex}{%
  \def\subsubsubitem{\@idxitem\hspace*{35\p@}}
  \def\subsubsubsubitem{\@idxitem\hspace*{40\p@}}
}
\makeatother
\usepackage{makeidx}\makeindex
%StopVisiblePreamble
\begin{document}
foo\index{foo} bar\index{foo!bar}
baz\index{foo!bar!baz} foobar%
\index{foo!bar!baz!foobar} Kuba 
\index{foo!bar!baz!foobar!Kuba}
\newpage \printindex
\end{document}
\end{externalDocument}








\section{Including \LaTeX\ commands into the \Lext{idx} file}
The command \Lcs{addtocontents} doesn't work for the index file. With the \LaTeX\ package
\Lpack{xindex} (same name as the Lua program \Lprog{xindex}) defines a macro \Lcs{writeidx}
which writes its argument into the \Lext{idx} file. This can be usefull to insert a 
\Index{pagebreak}/""\Index{columnbreak}
before a new letter in the output of the index file:


\begin{verbatim}
\documentclass{article}
\usepackage{makeidx}
\makeindex
\usepackage{xindex}
\begin{document}

\index{foo}foo and
\writeidx{\clearpage}
\index{bar}bar

\printindex
\end{document} 
\end{verbatim}


Such commands are then taken into account by the program \Lprog{xindex}. With the often used program \Lprog{makeindex}
such commands are ignored. In the following example we put an horizontal line after the first entry:

\begin{externalDocument}[
%  grfOptions={width=\dimexpr\linewidth-2\fboxsep-2\fboxrule},
  mpwidth=0.55\linewidth,
  pages={2},
  frame=false,
  compiler=lualatex --shell-escape, 
  crop,
  xindex,
  xindexOptions=-l de,
  force,
  runs=2,code,docType=latex,
  showFilename,
  align=\centering,
  lstOptions={columns=fixed}]{xindex}
\documentclass{scrartcl}
\usepackage{libertinus-otf}
%StartVisiblePreamble
\usepackage{xindex}
\makeindex
%StopVisiblePreamble
\pagestyle{empty}
\makeatletter
\def\theindex{%    only for demonstration
  \newpage
  \section*{\indexname}
  \parskip\z@ \@plus .3\p@\relax \parindent\z@
  \let\item\@idxitem}
\makeatother
\begin{document}
\index{foo}foo and
\writeidx{\item\protect\hrulefill}
\index{bar}bar
\index{gex}gex
\printindex
\end{document}
\end{externalDocument}


\section{Headings}
By default the output uses the english headings: \textit{Symbols}, \textit{Numbers}, and \textit{A} \ldots
There are three predefined languages \texttt{en}, \texttt{de}, and \texttt{fr}. The definition is in the file
\Lfile{xindex-cfg-common.lua} (see also section \vref{language}).
%
It can easily be extended for other \Index{language}s. Sometimes the headers are not needed, for example in a name
list. With the optional argument \verb=-n= or \verb=--noheadings= the created \Lext{ind} file has only
the vertical space between different first letters:

\begin{externalDocument}[
%  grfOptions={width=\dimexpr\linewidth-2\fboxsep-2\fboxrule},
  mpwidth=0.55\linewidth,
  pages={5},
  frame=false,
  compiler=lualatex,
  crop,
  xindex,
  xindexOptions=-n,
  force,
  runs=2,code,docType=latex,
  showFilename,
  align=\centering,
  lstOptions={columns=fixed}]{xindex}
\documentclass{scrartcl}
\makeatletter
\def\theindex{%    only for demonstration
  \section*{\indexname}
  \parskip\z@ \@plus .3\p@\relax \parindent\z@
  \let\item\@idxitem}
\makeatother
%StartVisiblePreamble
\usepackage{makeidx}\makeindex
%StopVisiblePreamble
\pagestyle{empty}
\begin{document}
Ein foo\index{foo}\index{bar|(}
 \newpage und \index{foo} 
ein foo\index{foo|textit} \newpage 
und foo\index{foo|textbf} \newpage 
und foo\index{foo|fbox}
\index{bar|)}
\newpage
\verb|xindex -n <file>|
\printindex
\end{document}
\end{externalDocument}


The headings are printed by default as \Lcs{textbf}. This can be changed in the config file 
by setting the
variable \texttt{idxnewletter}, for example: \verb|idxnewletter = "\\textit"|. If you need some
more code here then define an own  macro for it, which can be seen in the following example. It has
an own config file \Lfile{xindex-header.lua} which has the line

\verb|idxnewletter = "\\idxnewletter"|

In the documents preamble there is the definition:

\verb|\newcommand\idxnewletter[1]{\textbf{\textit{#1}}}|

\begin{externalDocument}[
%  grfOptions={width=\dimexpr\linewidth-2\fboxsep-2\fboxrule},
  mpwidth=0.55\linewidth,
  pages={2},
  frame=false,
  compiler=lualatex,
  crop,
  xindex,
  xindexOptions=-c header,
  force,
  runs=2,code,docType=latex,
  showFilename,
  align=\centering,
  lstOptions={columns=fixed}]{xindex}
\documentclass{scrartcl}
\usepackage{filecontents}
\begin{filecontents*}{xindex-header.lua}
-----------------------------------------------------------------------
--         FILE:  xindex-header.lua
--  DESCRIPTION:  configuration file for xindex.lua
-- REQUIREMENTS:  
--       AUTHOR:  Herbert Voß
--      LICENSE:  LPPL1.3
-----------------------------------------------------------------------

if not modules then modules = { } end modules ['xindex-header'] = {
      version = 0.19,
      comment = "main configuration to xindex.lua",
       author = "Herbert Voss",
    copyright = "Herbert Voss",
      license = "LPPL 1.3"
}

itemPageDelimiter = ","     -- Hello, 14
compressPages     = true    -- something like 12--15, instaead of 12,13,14,15. the |( ... |) syntax is still valid
fCompress	  = true    -- 3f -> page 3, 4 and 3ff -> page 3, 4, 5
minCompress       = 3       -- 14--17 or 
rangeSymbol       = "--"
numericPage       = true    -- for non numerical pagenumbers, like "VI-17"
sublabels         = {"", "-\\,", "--\\,", "---\\,"} -- for the (sub(sub(sub-items  first one is for item
pageNoPrefixDel   = ""     -- a delimiter for page numbers like "VI-17"  -- not used !!!
indexOpening      = ""     -- commands after \begin{theindex}
idxnewletter      = "\\idxnewletter"

--[[
    Each character's position in this array-like table determines its 'priority'.
    Several characters in the same slot have the same 'priority'.
]]
alphabet_lower = { --   for sorting
    { ' ' },  -- only for internal tests
    { 'a', 'á', 'à', 'ä', 'â', 'å', 'æ', },
    { 'b' },
    { 'c', 'ç' },
    { 'd' },
    { 'e', 'é', 'è', 'ë', 'ê' },
    { 'f' },
    { 'g' },
    { 'h' },
    { 'i', 'í', 'ì', 'î', 'ï' },
    { 'j' },
    { 'k' },
    { 'l' },
    { 'm' },
    { 'n', 'ñ' },
    { 'o', 'ó', 'ò', 'ö', 'ô', 'ø', 'œ', 'ø'},
    { 'p' },
    { 'q' },
    { 'r' },
    { 's', 'š', 'ß' },
    { 't' },
    { 'u', 'ú', 'ù', 'ü' , 'û'},
    { 'v' },
    { 'w' },
    { 'x' },
    { 'y', 'ý', 'ÿ' },
    { 'z', 'ž' }
}
alphabet_upper = { -- for sorting
    { ' ' },
    { 'A', 'Á', 'À', 'Ä', 'Å', 'Æ'},
    { 'B' },
    { 'C', 'Ç' },
    { 'D' },
    { 'E', 'È', 'É', 'Ë' },
    { 'F' },
    { 'G' },
    { 'H' },
    { 'I', 'Í', 'Ì', 'Ï' },
    { 'J' },
    { 'K' },
    { 'L' },
    { 'M' },
    { 'N', 'Ñ' },
    { 'O', 'Ó', 'Ò', 'Ö', 'Ø','Œ', 'Ø' },
    { 'P' },
    { 'Q' },
    { 'R' },
    { 'S', 'Š' },
    { 'T' },
    { 'U', 'Ú', 'Ù', 'Ü' },
    { 'V' },
    { 'W' },
    { 'X' },
    { 'Y', 'Ý', 'Ÿ' },
    { 'Z', 'Ž' }
}
\end{filecontents*}
\makeatletter
\def\theindex{%    only for demonstration
  \section*{\indexname}
  \parskip\z@ \@plus .3\p@\relax \parindent\z@
  \let\item\@idxitem}
\makeatother
%StartVisiblePreamble
\usepackage{makeidx}\makeindex
\newcommand\idxnewletter[1]{\textbf{\textit{#1}}}
%StopVisiblePreamble
\pagestyle{empty}
\begin{document}
\section{Escaping characters}
\begin{itemize}
\item Exclamation mark ! \index{exclaim ("!)}
\item Vertical bar| \index{Vertical bar ("|)}
\item Doublequote \verb|"| \index{""}
\item Double doublequote \verb|""| \index{""""}
\item At character @ \index{At ("@)}
\end{itemize}
run \verb|xindex -c header <file.idx>|
\index{<file.idx>@\texttt{<file.idx>}}
\index{123}
\newpage
\printindex
\end{document}
\end{externalDocument}



\section{Case sensitive index entries}
By default \textsf{foo} and \textsf{Foo} are two different entries and will handled differently
by \Lprog{xindex}: \textsf{Foo} will be as an own entry \emph{before} \textsf{foo}. Let's see 
a more complex example. In the index the entry \verb|xindex-DIN2.lua| is the first one of
the \verb|xindex-???| series because uppercase letters are sorted before lowercase letters.

\begin{externalDocument}[
%  grfOptions={width=\dimexpr\linewidth-2\fboxsep-2\fboxrule},
  mpwidth=0.55\linewidth,
  pages={2},
  frame=false,
  compiler=lualatex --shell-escape, 
  crop,
  xindex,
%  xindexOptions=,
  force,
  runs=2,code,docType=latex,
  showFilename,
  align=\centering,
  lstOptions={columns=fixed}]{xindex}
\documentclass{scrartcl}
\usepackage{libertinus-otf}
\makeatletter
\def\theindex{%    only for demonstration
  \section*{\indexname}
  \parskip\z@ \@plus .3\p@\relax \parindent\z@
  \let\item\@idxitem}
\makeatother
\usepackage{filecontents}
\begin{filecontents*}{\jobname.idx}
\indexentry{xindex package@\texttt  {xindex} package|hyperpage}{2}
\indexentry{xindex program@\texttt  {xindex} program|hyperpage}{4}
\indexentry{xindex-cfg.lua file@\texttt  {xindex-cfg.lua} file|hyperpage}{6}
\indexentry{xindex-newfile.lua file@\texttt  {xindex-newfile.lua} file|hyperpage}{6}
\indexentry{xindex-DIN2.lua file@\texttt  {xindex-DIN2.lua} file|hyperpage}{6}
\indexentry{xindex-cfg-common.cfg file@\texttt  {xindex-cfg-common.cfg} file|hyperpage}{9}
\indexentry{xindex-cfg.lua file@\texttt  {xindex-cfg.lua} file|hyperpage}{10}
\indexentry{xindex-HAdW-eKO.lua file@\texttt  {xindex-HAdW-eKO.lua} file|hyperpage}{10}
\indexentry{xindex-dtk.lua file@\texttt  {xindex-dtk.lua} file|hyperpage}{12}
\indexentry{xindex program@\texttt  {xindex} program|hyperpage}{13}
\indexentry{xindex program@\texttt  {xindex} program|hyperpage}{14}
\indexentry{xindex-cfg-common.lua file@\texttt  {xindex-cfg-common.lua} file|hyperpage}{14}
\indexentry{xindex package@\texttt  {xindex} package|hyperpage}{15}
\end{filecontents*}
%StartVisiblePreamble
\usepackage{makeidx}
\usepackage{hyperref}
%StopVisiblePreamble
\pagestyle{empty}
\begin{document}
foo\newpage
\printindex
\end{document}
\end{externalDocument}


The same example sorted with the \verb|-a| or \verb|--nocasesensitive| has another output: now
\verb|xindex-cfg-common.lua| is the first one of the \verb|xindex-???| series.


\begin{externalDocument}[
%  grfOptions={width=\dimexpr\linewidth-2\fboxsep-2\fboxrule},
  mpwidth=0.55\linewidth,
  pages={2},
  frame=false,
  compiler=lualatex --shell-escape, 
  crop,
  xindex,
  xindexOptions=-a,
  force,
  runs=2,code,docType=latex,
  showFilename,
  align=\centering,
  lstOptions={columns=fixed}]{xindex}
\documentclass{scrartcl}
\usepackage{libertinus-otf}
\makeatletter
\def\theindex{%    only for demonstration
  \section*{\indexname}
  \parskip\z@ \@plus .3\p@\relax \parindent\z@
  \let\item\@idxitem}
\makeatother
\usepackage{filecontents}
\begin{filecontents*}{\jobname.idx}
\indexentry{xindex package@\texttt  {xindex} package|hyperpage}{2}
\indexentry{xindex program@\texttt  {xindex} program|hyperpage}{4}
\indexentry{xindex-cfg.lua file@\texttt  {xindex-cfg.lua} file|hyperpage}{6}
\indexentry{xindex-newfile.lua file@\texttt  {xindex-newfile.lua} file|hyperpage}{6}
\indexentry{xindex-DIN2.lua file@\texttt  {xindex-DIN2.lua} file|hyperpage}{6}
\indexentry{xindex-cfg-common.cfg file@\texttt  {xindex-cfg-common.cfg} file|hyperpage}{9}
\indexentry{xindex-cfg.lua file@\texttt  {xindex-cfg.lua} file|hyperpage}{10}
\indexentry{xindex-HAdW-eKO.lua file@\texttt  {xindex-HAdW-eKO.lua} file|hyperpage}{10}
\indexentry{xindex-dtk.lua file@\texttt  {xindex-dtk.lua} file|hyperpage}{12}
\indexentry{xindex program@\texttt  {xindex} program|hyperpage}{13}
\indexentry{xindex program@\texttt  {xindex} program|hyperpage}{14}
\indexentry{xindex-cfg-common.lua file@\texttt  {xindex-cfg-common.lua} file|hyperpage}{14}
\indexentry{xindex package@\texttt  {xindex} package|hyperpage}{15}
\end{filecontents*}
%StartVisiblePreamble
\usepackage{makeidx}
\usepackage{hyperref}
%StopVisiblePreamble
\pagestyle{empty}
\begin{document}
foo\newpage
\printindex
\end{document}
\end{externalDocument}




\section{Automatic index creation}
With package \Lpack{xindex} one can define several different index files, e.\,g.
an \Index{index of names}. With the optional argument \Loption{imakeidx} the package itself loads  \Lpack{imakeidx}
and adds the program \Lprog{xindex} as the default program to  \Lpack{imakeidx}. 

\begin{externalDocument}[
%  grfOptions={width=\dimexpr\linewidth-2\fboxsep-2\fboxrule},
  mpwidth=0.55\linewidth,
  pages={6},
  frame=false,
  compiler=lualatex --shell-escape, 
  crop,
  xindex,
  xindexOptions=-l de,
  force,
  runs=2,code,docType=latex,
  showFilename,
  align=\centering,
  lstOptions={columns=fixed}]{xindex}
\documentclass{scrartcl}
\usepackage{libertinus-otf}
\makeatletter
\let\ps@plain\ps@empty
\makeatother
%StartVisiblePreamble
\usepackage[imakeidx]{xindex}
\makeindex[name=persons,title=Index of names,
   columns=1,options=--noheadings]
\def\ThanhVN{Hàn Thê\protect\llap{%
  \raise 0.5ex\hbox{\'{}}}}
%StopVisiblePreamble
\pagestyle{empty}\renewcommand\thepage{}
\begin{document}
foo\index[persons]{Niepraschk,~ Rolf}
foo\index[persons]{Lamport,~ Leslie}
foo\index[persons]{Knuth,~ Donald}
foo\index[persons]{Knuth,~ Donald}
\newpage
foo\index[persons]{Lamport,~ Leslie}
foo\index[persons]{Thành,~ \ThanhVN}
foo\index[persons]{Kew,~ Jonathan}
foo\index[persons]{Kohm,~ Markus}
foo\index[persons]{Preining,~ Norbert}
\newpage
foo\index[persons]{Schenk,~ Christian}
foo\index[persons]{Feuerstack,~ Thomas}
foo\index[persons]{Tobin,~ Geoffrey}
foo\index[persons]{Wilson,~ Peter}
\newpage
foo\index[persons]{Kohm,~ Markus}
foo\index[persons]{Theiling,~ Henrik}
foo\index[persons]{Pégourié-Gonnard,~ Manuel}
foo\index[persons]{Roux,~ Élie}
\newpage
foo\index[persons]{Mittelbach,~ Frank}
foo\index[persons]{Fairbairns,~ Robin}
foo\index[persons]{Lemberg,~ Werner}
foo\index[persons]{Volovich,~ Vladimir}

\printindex[persons]
\end{document}
\end{externalDocument}


You have to run \LaTeX\  with the \verb|--shell-escape|\index{Shell escape} option to run \Lprog{xindex}
from within the \LaTeX\ document.

\section{Demerits}
\begin{itemize}
\item For more than 5000 entries in the \Lext{idx} file the internal Lua function
for \Index{sorting} may take some time.
\item The \Lext{idx} file is not checked for \LaTeX\  errors\index{LaTeX\ errors@\LaTeX\  errors} 
in the argument of \Lcs{indexentry}.
\end{itemize}





%\nocite{*}
%\printbibliography


\printindex

\end{document}



folium = { 
  de = {"f", "ff"},
  en = {"f", "ff"},
  fr = {"sq","sqq"},
}

%%%%%%%%%%%%%%%%%%%%%%%%%%%%%%%%%%%%%%%%%% LICENSE %%%%%%%%%%%%%%%%%%%%%%%%%%%%%%%%%%%%%%%%%
% Copyright 2018 Terrence P. Murphy and Rosalie D'Amico.
% This file may be distributed and/or modified under the conditions of the LaTeX Project 
% Public License (“LPPL”), either version 1.3c of this license or (at your option) any 
% later version. The latest version of this license is at:
%     http://www.latex-project.org/lppl.txt.
% This file is author-maintained and  is one of the files that, collectively, make up the 
% MAKECOOKBOOK bundle available at https://ctan.org/pkg/makecookbook
% For additional information, please see the associated README file.
%
% The MAKECOOKBOOK bundle includes an example cookbook with seven recipes. Those recipes are 
% courtesy of Rosalie D'Amico. You are, of course, welcome to try them!  They are included in 
% the bundle to provide real-world examples of using LaTex code to enter recipes. We only ask 
% that you consider those recipes as for you personal use and not (without attribution) for 
% further food-related publication (further publication OK in a LaTex context).
%%%%%%%%%%%%%%%%%%%%%%%%%%%%%%%%%%%%%%%%%%%%%%%%%%%%%%%%%%%%%%%%%%%%%%%%%%%%%%%%%%%%%%%%%%%%
\chapter{Another Chapter Name}
\ChapterIntro[16]{
	\lipsum[1-3]
}

\SideBySide[LeftCaption={left caption}, RightCaption={right caption}]{image-a}{image-b}
\FinishChapterIntro{}

\newpage
\RecipeNameAndYield{Name=H -- Chicken Scarpariello}
\RecipeStory{\lettrine{S}{carpariello} is \textit{shoemaker} in Italian.  \textit{Shoemaker’s} chicken may refer to Neapolitan shoemakers making delicious food in the little time they had at the end of the day.}

\begin{IngredientsAndSteps}
	
	\ListIngredientsAndSteps
	{
		\fr1/2 \Pd Italian sausage, links or bulk
		
		6 boneless, skinless chicken thighs 
		
		1 large yellow onion diced small
		
		\fr1/2 large red bell pepper diced small
		
		6 cloves finely minced garlic
		
		1 cup dry white wine
		
		1 cup chicken broth
		
		\fr1/2 cup diced pickled Peppadew peppers* 
		
		\fr1/4 cup white wine vinegar
		
		3 sprigs fresh rosemary
		
		Italian parsley for serving
	}
	{  	
		Brown sausage in skillet with a little oil for 6-8 minutes (they will not be fully cooked).  If using links, slice before cooking.  Transfer to a plate.  
		
		Salt and pepper the chicken and dredge in flour.  Cook in same skillet until nice and brown and remove from skillet (it will not be fully cooked).  Transfer to plate with sausage.  
		
		Cook onions, bell pepper and garlic in same skillet for 8 to 10 minutes, adding a bit more oil if necessary.  
		
		Add wine and cook about 5 minutes until slightly reduced.  
		
		Add broth, peppers, vinegar, and rosemary and bring to a boil.  
		
		Add chicken and sausages to skillet and cook until chicken is cooked through, about 10 to 15 minutes.  
		
		Garnish with chopped parsley and serve with steamed rice.
	}
	
\end{IngredientsAndSteps}
\Attribution{Recipe courtesy of Rosalie D'Amico -- Adapted from Bon Appetit}

\begin{Tip}
	{* Peppadew is the brand name for the pickled grape-size red pepper known as Juanita. They are available on Amazon or can be found bulk in “Olive Bars” in many grocery stores.}
\end{Tip}
\newpage
\RecipeNameAndYield{Name=I -- Pumpkin Pancakes, Yield=Yield: 6 Pancakes}

\RecipeStory{\lettrine{W}{e tell a story} here. \lipsum[66]}

\begin{IngredientsAndSteps}
	
	\ListIngredientsAndSteps
	{
	\IngredientsHeading {Dry Ingredients}%
	
	1\fr1/4 cups all-purpose flour
	
	2 \tsp[s] baking powder
	
	\fr1/2 \tsp cinnamon
	
	\fr1/2 \tsp ginger
	
	\fr1/2 \tsp nutmeg
	
	Pinch of cloves and allspice
	
	\fr1/4 \tsp salt
	
	\InsertHiddenLines{2}  % also try the multicol \columnbreak command		
	
	\IngredientsHeading [2]{Wet Ingredients}%
	
	\fr1/2 cup canned solid pack pumpkin
	
	2 \Tbl[s] brown sugar or Maple Syrup 
	
	1 large egg
	
	2 \Tbl[s] oil
	
	1 cup milk	
	}
	{  
	Mix dry ingredients in bowl. In another bowl, whisk pumpkin and remaining ingredients together until well mixed. Add to dry ingredients and fold together.  Do not overmix.
	}
	
\end{IngredientsAndSteps}
\Attribution{Recipe courtesy of Rosalie D'Amico}
% \iffalse meta comment
% This file is a part of the ANTOMEGA project version \fileversion
% -----------------------------------------------------
% 
% It may be distributed under the terms of the LaTeX Project Public
% License, as described in lppl.txt in the base LaTeX distribution.
% Either version 1.0 or, at your option, any later version.
% Copyright (C) 2001 -- 2005 by Alexej Kryukov
% Please report errors to: A.M. Kryukov <basileia@yandex.ru>
%
% \fi
%
% \iffalse
%
%<*dtx>
\ProvidesFile{antomega.dtx}
%</dtx>
%
%<antomega>\NeedsTeXFormat{LaTeX2e}
%<antomega>\ProvidesPackage{antomega}
%<antomega>\RequirePackage{keyval,ifthen,calc}
%
%<driver>\ProvidesFile{antomega.dtx}
%
%<*driver>
\documentclass{ltxdoc}
\GetFileInfo{antomega.dtx}
\def\fileversion{0.8}
\def\filedate{7 May 2005}
\def\docdate{7 May 2005}

\newcommand*\file[1]{\texttt{#1}}
\title{Typesetting multilingual documents with ANTOMEGA
        \thanks{This file
        has version number \fileversion, last
        revised on \filedate.}}
\author{Alexej Kryukov}

\begin{document}
   \maketitle
   \DocInput{antomega.dtx}
   \DocInput{antenc.dtx}
\end{document}
%</driver>
%
% \fi
%
% %%%%%%%%%%%%%%%%%%%%%%%%%%%%%%%%%%%%%%%%%%%%%%%%%%%%%%%%%%%%%%%%%%%%
%
% \changes{v0.2}{2002/10/12}{First public release}
% \changes{v0.5}{2002/11/14}{Added new interface for loading translation 
%       processes}
% \changes{v0.5}{2002/11/14}{The documentation is totally rewritten using
%       the doc package}
% \changes{v0.5}{2002/11/14}{Many bugs fixed}
% \changes{v0.6}{2003/03/18}{Added new translation processes in order
%       to make antomega more independent from the original `omega'
%       package}
% \changes{v0.6}{2003/03/18}{Added new language switching commands,
%       compatible with the Babel package. Old commands are still
%       supported for backwards compatibility purposes.}
% \changes{v0.61}{2003/03/27}{Fixed a bug in uni1f00.def}
% \changes{v0.7}{2003/08/30}{Keyval syntax is now used for package
%       options. Old options are still supported for backwards 
%       compatibility.}
% \changes{v0.7}{2003/08/30}{Now it is possible to specify `input=utf-8'
%       in options of the whole package. This option will be inherited
%       by commands used to load specific languages.}
% \changes{v0.7}{2003/08/30}{Added support files for new languages: German
%        (thanks to Olaf Dietrich <olaf.dietrich@ikra.med.uni-muenchen.de>),
%        Polish (thanks to Mariusz Wodzicki <wodzicki@Math.Berkeley.EDU>)
%        and Latvian (thanks to Dmitry Ivanov <dimss@solutions.lv>).}
% \changes{v0.7}{2003/08/30}{Added special version of the file `hyphen.cfg',
%        allowing to load character codes for specific Unicode ranges
%        into Lambda format.}
% \changes{v0.7}{2003/08/30}{Removed options `localtoc' and `nolocaltoc'.
%        These options were used to control if language markup should
%        be included into the arguments of `contentsline'. This is not
%        needed now, because language markup commands are written
%        into *.toc files separately, exactly as in Babel.}
% \changes{v0.7}{2003/08/30}{Various minor improvements and bugfixes}
% \changes{v0.8}{2004/04/30}{Bugfix: `local@hyphenmins' produced an 
%        extra space in the output.}
% \changes{v0.8}{2004/04/30}{Bugfix: An OCP was assigned to a wrong slot 
%        in German and Polish language support files -- now fixed.}
% \changes{v0.8}{2005/01/08}{Bugfix: fixed a bug which caused a language name 
%        to be reproduced in the output while using that name as a name of a 
%        language switching environment.}
% \changes{v0.8}{2005/01/08}{Added an (experimental) support file for Latin
%        language.}
% \changes{v0.8}{2005/01/09}{Bugfix: the `specials' key didn't work as
%        expected.}
% \changes{v0.8}{2005/04/15}{Bugfix: the correct language environment was not 
%        restored after exiting from `otherlanguage'.}
% \changes{v0.8}{2005/04/17}{Language switching commands (i.~e.
%        `seletlanguage' and `otherlanguage' now cause locale dependent 
%        captions to be displayed in the current language. So the concept
%        of having a special set of extras, applicable only for the 
%        `background' document language (inherited from older omega.sty) 
%        is abandoned.}
% \changes{v0.8}{2005/04/17}{ANTOMEGA now prevents some special characters,
%        produced by text commands, from placing back to the OCP stack. To
%        implement this feature, I had to add a set of encoding definition
%        files, designed specially for ANTOMEGA. In particular this means 
%        ANTOMEGA is no longer dependent from ut1enc.def.}
% \changes{v0.8}{2005/04/17}{uppercase.otp (now called uppercase-dflt.otp) 
%        is incorporated into ANTOMEGA. OCP-based conversion to uppercase
%        is now used by default, since Omega incorrectly performs the
%        standard (uccode/lccode based) conversion in the UTF-8 mode.}
% \changes{v0.8}{2005/05/03}{Added an experimental support for Scientific
%        Word/Scientific Workplace.}
% \changes{v0.8}{2005/05/07}{Workaround for the `accent' command: now all
%        standard commands used to type accents are redefined to produce
%        a character followed by a Unicode combining accent. Later this
%        sequence may be processed by OCP and turned back to an accented
%        character. So arguments of such commands may be processed by OCP
%        and correctly converted from user's input encoding to Unicode.}
%
% \MakeShortVerb{\|}
%
% \begin{abstract}
% Antomega is a language support package for Lambda, based on the
% original \file{omega.sty} file. However, it provides some additional 
% functionality.
% \end{abstract}
%
% \section{Introduction}
% Moving from \LaTeX\ to $\Omega$ is always difficult for an average
% user, since the $\Omega$ distribution doesn't include any language support
% package which could be used as a Babel replacement. The \file{omega.sty}
% file, version of 1999/06/01, was released by the $\Omega$ developers as
% a first attempt to make something like `Omega-Babel', but, unfortunately, 
% this work was not finished. Moreover, more recent versions of
% \file{omega.sty} are suitable only for testing some right-to-left languages, 
% but not for regular work. So I prepared my own package, based on the original
% \file{omega.sty}, which fixes some bugs and provides some additional
% functionality.
%
% \section{Installation instructions}
%
% First, download and install the $\Omega$ binaries or ensure that your \TeX\ 
% installation already includes them. Unpack the archive file with ANTOMEGA
% and move all files to the appropriate directories (for example,
% everything in \file{omega/lambda/} to |$$texmf/omega/lambda|, everything
% in \file{omega/ocp/} to |$$texmf/omega/ocp|, and so on. If you already
% have a file named \file{language.dat} in |$$texmf/omega/lambda/base|,
% replace it with the provided file (called language.dat.sample) in case you 
% want to get correct hyphenation for Russian and/or Greek.
%
% Note that ANTOMEGA still needs some files from the original $\Omega$ 
% distribution. The most important file is \file{ut1omlgc.fd}. 
% Unfortunately, this file was not included into some recent 
% $\Omega$ distributions. I can neither include
% it into my package as is (this might cause name clashes) nor rename
% it (since I can't rename the default font and the default encoding
% vector used in $\Omega$). So in case you haven't this file already
% installed you have to install it separately. Either take it from 
% an older \TeX\ distribution or from the $\Omega$ CVS tree.
%
% There are also some additional translation processes (useful mainly
% for typesetting polytonic (classical) Greek), which you may want to
% take from old $\Omega$.
%
% Of course, after installing new files you have to update the \TeX\
% file names database. On teTeX or fpTeX systems this is performed 
% with |texconfig rehash| or |mktexlsr| commands. On MikTeX you
% can do the same via a menu item. Refer yourself to a special 
% section (\ref{sw}) of this manual, in case you are interested in 
% installing and configuring ANTOMEGA under Scientific Word/Scientific 
% Workplace.
%
% \subsection{Deprecated files}
%
% If you are upgrading from an older version of ANTOMEGA, you can
% safely delete the following deprecated files:
%
% \begin{itemize}
%
% \item In |$$TEXMF/omega/unidata|: 
%
% \begin{itemize}
%
% \item \file{uni0300.def} (now called \file{uni0370.def};
% 
% \item \file{grahyph.tex} and \file{grmhyph.tex} (replaced with
% \file{ograhyph.tex} and \file{ogrmhyph.tex} correspondingly according
% to the naming convention proposed by Dimitrios Filippou);
% 
% \end{itemize}
%
% \item In |$$TEXMF/omega/hyphen|: \file{greek2uni.tex} and
% \file{greek2omega.tex} (no longer needed);
%
% \item In |$$TEXMF/omega/otp/antomega|: \file{latin2punct.otp} and
% \file{cyr2punct.otp} (replaced with files \file{tex2punct.otp},
% \file{babel2punct.otp} and \file{babel2ru.otp});
%
% \item In |$$TEXMF/omega/ocp/antomega|: \file{latin2punct.ocp} and
% \file{cyr2punct.ocp} (replaced with files \file{tex2punct.ocp},
% \file{babel2punct.ocp} and \file{babel2ru.ocp}).
%
% \end{itemize}
%
% \subsection{Updating Lambda format}
%
% With older versions of ANTOMEGA you could do without updating Lambda
% format (at least if you work only with Latin-based languages), but now 
% this operation is strongly recommended for the following
% reasons. As you probably know, the core of the \LaTeX\ system consists of
% \file{latex.ltx} and some other files which should be loaded into format.
% Since there are no special versions of those files for $\Omega$,
% the same files are used also for Lambda format. 
%
% Of course, files designed for \LaTeX\ not always work well for 
% $\Omega$. For example, in order to get correct hyphenation for a text
% in a specific language, we have to set |\catcode| (which should be equal 
% to 11 or 12) and |\lccode| for all characters used in that language.
% For correct conversion to uppercase we need also |\uccode| values.
% In standard \LaTeX\ these values are defined for all 256 characters 
% of the encoding table, but with $\Omega$ we need similar definitions for 
% a wider range of Unicode characters, and it would be nice to have these
% definitions loaded into format. Some people achieve this effect putting
% tables of such codes together with their hyphenation patterns.
% 
% However, this approach causes some problems. First of all, it is very
% inconvenient to set |\catcode|, |\lccode| and |\uccode| with primitive
% commands for each letter from a large amount of characters. Antomega defines
% some commands (namely |\makeletter|, |\makeucletter|, |\makelcletter|
% and |\makesameletter|), allowing to simplify this process. It also
% provides definition tables for some Unicode ranges, written using
% these commands. However, in order to load these tables into the format
% the commands they use should be already known to \file{iniomega}.
%
% There is another problem, even more important. The file responsible for
% loading hyphenation patterns is called \file{hyphen.cfg}. This file
% is part of the \file{babel} package, but some language-specific formats
% like cslatex, platex etc. include their own versions of \file{hyphen.cfg}. 
% Note that those versions are mainly incompatible with ANTOMEGA. The
% official $\Omega$ distribution also included its own simplified version
% of \file{hyphen.cfg}, but this file was removed from the most recent versions.
% That's why original babel's version of the file is also often loaded
% into Lambda format.
%
% \emph{This} \file{hyphen.cfg} is mainly compatible with $\Omega$,
% although it defines a lot of specific commands not needed if we are not
% planning to use Babel itself. However, in last versions it has the following 
% feature: hyphenation patterns are loaded inside a \TeX\ group. This 
% means that hyphenation rules itself will be saved, but character codes
% loaded together with them will be forgotten immediately after
% processing hyphenation rules. That's correct, because character codes
% required by specific hyphenation patterns may not match codes normally used 
% by \LaTeX. For example, Russian hyphenation patterns usually have the
% koi8 encoding, which is not directly used by \LaTeX. So we need a specific
% table of codes in order to reencode these patterns into an internal 
% font encoding supported by \TeX. Once reencoding is performed, these
% character codes are no longer needed.
%
% However, this means that hyphenation patterns is an incorrect place
% to store character codes for Unicode symbols, because it is very
% possible that our settings will take no effect. And even if they are
% saved (in case we have old Omega's \file{hyphen.cfg} installed) the
% result may be rather unexpected, because $\Omega$ has no way to
% determine which codes are necessary only for processing specific
% hyphenation patterns, and which should really be stored for further
% use.
%
% That's why ANTOMEGA now includes its own version of \file{hyphen.cfg}
% and special file \file{antomega.cfg} which contains references to
% tables of character codes for all supported Unicode ranges. This
% version of \file{hyphen.cfg} first defines commands |\makeletter|,
% |\makelcletter|, |\makeucletter| and |\makesameletter|. After that
% it loads \file{antomega.cfg}. So if you created a custom table of 
% character codes for your script, you may load it via 
% \file{antomega.cfg} instead of including it into your hyphenation patterns. 
% Of course you can also prevent codes for a specific Unicode range
% from loading into your format. For example, if you never use polytonic
% Greek, comment out the following line in \file{antomega.cfg}:
%
% \begin{verbatim}
% \input{uni1f00.def}
% \end{verbatim}
%
% Only after loading character codes into format, hyphenation patterns
% are processed. As well as in Babel, this procession is done inside a
% group, and so all character codes defined here should be used only
% for converting your patterns into another encoding.
%
% Note that, while old Omega's \file{hyphen.cfg} resides in
% |$TEXMF/omega/lambda/config|, ANTOMEGA installes its version of the file 
% into |$TEXMF/omega/lambda/antomega|. This prevents name clashes, but,
% in case you have old \file{hyphen.cfg} installed, you have to remove
% it manually to ensure that the correct version will be found and loaded
% by \file{iniomega}. Only after that you may rebuild the \file{lambda}
% format file. On teTeX or fpTeX systems you have to run
%
% \begin{verbatim}
% fmtutil --byfmt lambda
% \end{verbatim}
%
% See section~\ref{sw} for information on doing that with True\TeX{}/
% Scientific Word systems.
% 
% It is easy to test if you have the correct \file{hyphen.cfg} version loaded 
% into your format: just take a look at any |.log| file produced by \file{lambda}. 
% In case everything is correct it should contain the following text at the
% beginning: ``Antomega and hyphenation patters for\ldots loaded''.
% Now you should be able to typeset you documents with ANTOMEGA.
%
% \section{Loading ANTOMEGA}
% One of the main advantages of \file{omega.sty} was using different commands
% for setting the main language of the document and for loading additional
% languages. ANTOMEGA preserves this feature, using the same |\background|
% and |\load| commands. So if you want to prepare an English
% document including some Greek text, you can do it by the same way as
% with \file{omega.sty}, for example:
%
% \begin{verbatim}
% \usepackage{antomega}
% \background{english}
% \load{greek}
% \end{verbatim}
%
% However, \file{omega.sty} needs two different files for each language: 
% first of them (with the *.bgd extension) is used by the |\background|
% command, and second (with the *.lay extension) by the |\load| command.
% Of course, these two files usually have very similar code. ANTOMEGA
% fixes this problem: both |\background| and |\load| commands load the same
% language definition file with the .ldf extension, but process it in a 
% different way.
%
% \section{Typesetting in different languages}
%
% \file{omega.sty} supported only a limited set of languages, which included
% |usenglish|, |french| and |greek|. ANTOMEGA supports these languages too,
% but separate support for |usenglish| is no longer available. Instead you
% can load |english| with options |dialect=british| or |dialect=american|,
% for example:
%
% \begin{verbatim}
% \background[dialect=american]{english}
% \end{verbatim}
%
% I added support files for Russian, and later also for German, Polish
% and Latvian.
%
% Generally speaking, it is not 
% difficult to provide support for a new language, since language definition 
% files are quite independent from the core package, and so you can write
% a file with definitions for your language without changing anything
% in \file{antomega.sty}, using the existing .ldf files as an example.
%
% In the original \file{omega} package we could use for switching to
% another language either an environment with the same
% name as a name of your language, or
% (for small pieces of text) the |\local<$language>| macro, there
% |<$language>| is your language name. These commands had to be defined in
% the language definition file. For example, \file{usenglish.bgd}
% defined the |usenglish| environment and the |\localusenglish| command.
%
% These commands are still supported in ANTOMEGA. However, beginning
% from the version 0.6 ANTOMEGA provides new language switching
% commands, compatible with the Babel package. So you can use the
% |\selectlanguage| and |\foreignlanguage| macros and the |otherlanguage|
% environment exactly as you did with Babel. This means that your old
% documents may be transferred to $\Omega$ with minimal changes.
%
% \section{Loading languages with options}
%
% As well as the original \file{omega.sty} file, ANTOMEGA requires the \file{keyval} 
% package. So all commands used for loading languages may be executed with
% different parameters, which may take different values. Each language has
% its own set of such parameters. However, some options are suitable for all
% supported languages. The most important of them are |input| and |output|
% parameters, which replace the \file{inputenc} and \file{fontenc} packages,
% used in standard \LaTeX{}.
%
% Beginning from ANTOMEGA v.~0.7 the same \file{keyval} syntax is also
% supported for options of the \file{antomega} package itself.
%
% \subsection{The ``input'' parameter}
% 
% Of course, this parameter is language-specific. However, there are two
% values, which are always supported: |utf-8| and |ucs-2|. The later really
% means ``no conversion'', since ucs-2 is the native format for $\Omega$.
% Since these two encodings are suitable for most languages, they may be 
% specified in options of the ANTOMEGA package, for example:
%
% \begin{verbatim}
% \usepackage[input=utf-8]{antomega}
% \end{verbatim}
%
% The |\load| and |\background| commands also support this option, 
% but they may additionally accept other values for it, depending
% from the language you want to load. For example, if you want to type an
% English document with some international symbols encoded in iso-8859-1, 
% you may put the following line into your \LaTeX{} preamble:
%
% \begin{verbatim}
% \background[input=iso-8859-1]{english}
% \end{verbatim}
%
% \subsection{The ``output'' parameter}
%
% For this parameter you can use one of the following values: |unicode|,
% |omega| and |tex|. |unicode| is used by default. Note that the |omlgc|
% font, distributed with $\Omega$, is not fully compatible with Unicode.
% For example, it has a specific encoding for Latin ligatures and
% general punctuation. So you have to set |output=omega| if you want
% to use this font, and |output=unicode| if you have another font, more
% strictly conforming the Unicode standard. You my want to set also
% |output=tex| if you prefer using 8-bit fonts in standard \TeX{} encodings
% (T1 for Western languages, T2A for Russian, LGR for Greek).
%
% For example, if you want to typeset your English text with the standard 
% EC fonts, but haven't any corresponding font for Greek, you may use the
% following preamble:
%
% \begin{verbatim}
% \documentclass{article}
% \usepackage{antomega}
% \background[output=tex]{english}
% \load[output=omega]{greek}
% \end{verbatim}
%
% \subsection{The ``shorthands'' parameter}
%
% Since ANTOMEGA tries to completely reproduce the functionality of
% the Babel package, it also supports combinations with  the |"| character 
% (\emph{shorthands}) which have a special meaning in Babel. Note
% that ANTOMEGA always uses translation processes to emulate Babel's
% behavior, and so it never really makes |"| an active character. The
% set of supported shorthands differs from language to language, but
% there is a minimal set available by default. For example, you may 
% use |"<| and |">| for guillemots. You may turn off support for Babel
% shorthands by setting `shorthands=off', and, of course, you may explicitly
% enable it (`shortahnds=on'). Currently this parameter is supported for
% all languages except Greek.
%
% \section{Translation processes}
%
% Since the last $\Omega$ versions are suitable only for testing
% purposes, they don't include many useful files, originally
% provided by J.~Plaice ang Y.~Haralambous. Particularly some $\Omega$
% translation processes were removed, and some are incorrect (e. g.
% don't correspond to the |omlgc| font). That's why \file{antomega}
% provides its own set of |ocp| and |otp| files, which makes it 
% rather independent from $\Omega$'s texmf part.
%
% For conversion to different encodings I added some new .otp and .ocp
% files, which (I hope) work correctly. Beginning from v.~0.6 
% ANTOMEGA includes improved translation processes for conversion from 
% some standard iso-8859 and windows-125*  codepages to Unicode.
% However, some original |.ocp| files are still necessary for \file{antomega}
% to work. There also some rarely used (but still supported in \file{antomega})
% files, not present neither in \file{antomega} nor in the most recent $\Omega$ 
% distributions.
%
% \section{Selecting fonts with ANTOMEGA}
%
% Of course, it is not enough to set an input encoding for your language.
% You will need also a correct font matching your encoding. With
% ANTOMEGA you can select a font separately for each script you use. For
% example, it defines new commands |\westernrm|, |\westernsf| and |\westerntt|.
% So, if you want to use Computer Modern for English but prefer to
% keep standard omlgc for Greek, simply put the following
% line in your preamble:
%
% \begin{verbatim}
% \renewcommand{\westernrm}{cmr}
% \end{verbatim}
%
% In ANTOMEGA's language support files some similar commands for
% other scripts are defined. For example, \file{omega-russian.ldf}
% introduces |\russianrm|, and \file{omega-latvian.ldf}~--- 
% |\balticrm|. Note that there is no need to define specific commands
% for Central European languages, because not only Unicode, but even
% T1 covers both Western and Central European character sets.
%
% Of course, you can write your own special packages to make selecting 
% a new font a bit more easy. You can even use standard font selecting 
% packages, but, in this case, you must load them \emph{after} ANTOMEGA
% itself and \emph{before} any language-specific commands. For example:
%
% \begin{verbatim}
% \usepackage{antomega}
% \usepackage{palatino}
% \background{english}
% \end{verbatim}
%
% \iffalse
%<*kernel>
% \fi
%
% \section{Using ANTOMEGA with MacKichan software products}
% 
% \label{sw}
% Generally speaking, adapting MacKichan software products (like 
% Scientific Word or Scientific Workplace) for non-Latin languages
% represents a not so trivial task. The problem is, that their shell
% (very powerful by itself) knows nothing about various input encodings,
% supported by \LaTeX, and so it can't take advantage of multilingual
% capabilities, provided by Babel. Instead, it represents any national
% characters, typed by user, in the form |\U{<hexadecimal Unicode index>}|.
% This representation is hardly legible for standard \LaTeX, because it
% is Unicode-based. That's why $\Omega$ is traditionally used for preparing
% multilingual documents with Scientific Word or Scientific Workplace.
%
% So, adapting ANTOMEGA (which is already distributed together with
% these software packages) to MacKichan shell is a very natural solution
% for those who want to get their text correctly typeset according to
% typographic rules used in their native languages. Preparing your Scientific
% products to use ANTOMEGA may be divided into two main stages: 
% configuring your True\TeX{} installation (which lies in the background
% of Scientific Word/Scientific Workplace) and configuring the graphical
% front-end itself.
%
% \subsection{Configuring True\TeX}
%
% First, ensure you have ANTOMEGA v.~0.8 or above. If necessary,
% download the latest version from CTAN and copy the contents of 
% all directories available in the downloaded package into the
% corresponding directories found in |TCITeX\Omega\|.
%
% Second, locate the file
% |TCITeX\Omega\Lambda\base\languages.dat.sample| and rename it
% to \file{languages.dat}. Edit this file to enable hyphenation
% patterns for your language.
%
% Now you have to rebuild the Lambda format. This operation is
% mandatory, because all default format files supplied with
% Scientific products are built without multilingual extensions
% provided by Babel, so that using ANTOMEGA with your default
% \file{lambda.oft} just will cause an error. This is also probably
% the hardest part, because True\TeX{}, unlike Mik\TeX{} or fp\TeX{},
% provides no special tools allowing to call \file{iniomega} with
% desired parameters. However, to simplify this task, you can use the
% following batch script (call it, say, \file{runinilambda.bat}):
%
% \begin{verbatim}
% echo off
% setlocal
% set TEXMF=C:\sw50\TCITeX
% set TEXINPUTS=.;%TEXMF%/{omega,tex}//
% %TEXMF%\web2c\iniomega %TEXMF%\omega\lambda\config\lambda.ini
% endlocal
% \end{verbatim}
%
% Note that you can run this script from any location, but the
% resulting \file{lambda.oft} file must be placed into your
% |TCITeX\web2c\| directory.
%
% \subsection{Configuring the Scientific Word / Scientific Workplace shell}
%
% When you have done with updating Lambda format, you can start Scientific 
% Word and create a new document (or load an existing one). To make its
% processing with Lambda possible, you should do the following:
%
% \begin{itemize}
%
% \item From the main menu bar, select `Typeset'---`Expert Settings'. 
% The dialog box with several tabs will appear, where you should select 
% the `DVI Format Settings' page, and then the `TeX Live Lambda' entry 
% from the drop-down box.
% 
% \item Go to the `Typeset'---`Options and packages' menu.
% Ensure that \file{sw2unicode}, \file{swtimes} and \file{fontenc} are
% \textbf{not} in the list of loaded packages. These packages are not
% needed, since their functuionality is completely incorporated by
% ANTOMEGA.
%
% \item From the same dialog box, click the `Go native' button. A dialog
% box with a multi-line input field will appear. Add the following line
% to that input field:
%
% \begin{verbatim}
% [input=sw,ffi=ligatures]{antomega}
% \end{verbatim}
%
% Note that the |ffi=ligatures| option is mandatory, since Times New
% Roman (the only Unicode font which is supported in the Scientific
% Word/Scientific Workplace distribution by default) hasn't some latin 
% ligatures at the places ANTOMEGA expects to find them, so that enabling the
% corresponding translation process will result in missing glyphs in the
% output. Unfortunately, the default \file{ofm} files for Times New Roman also
% contain no information about ligature substitution, so that 
% |ffi=ligatures| practically means `no ligatures at all' in this case.
%
% \item Go to `Typeset'---`Preamble' and input any number of |\background|
% and |\load| commands for the languages you are planning to use.
%
% \end{itemize}
%
% Now you should be able to compile your document with Lambda.
%
% \section{The ANTOMEGA code}
%
% \subsection{Handling identification codes for our files}
%
% Unlike in Babel, no tests if |\ProvidesFile| is already defined. 
% Since \LaTeXe\ was released in 1994, and the $\Omega$ project
% started also in 1994, probably nobody will use $\Omega$ with the 2.09
% format.
%
% \DescribeMacro{\ProvidesFile}
% We save the original definition of |\ProvidesFile| in
% |\ant@tempa| and restore it after we have stored the version of
% the file in |\toks8|.
%
%    \begin{macrocode}
\let\ant@tempa\ProvidesFile
\def\ProvidesFile#1[#2 #3 #4]{%
   \toks8{Antomega <#3> and hyphenation patterns for }%
   \ant@tempa#1[#2 #3 #4]%
   \let\ProvidesFile\ant@tempa}
%    \end{macrocode}
%
% \DescribeMacro{\ProvidesLanguage}
% As an alternative for |\ProvidesFile| we define
% |\ProvidesLanguage| here to be used in the language definition
% files.
%
%    \begin{macrocode}
\def\ProvidesLanguage#1{%
   \begingroup
      \catcode`\ 10 %
      \@makeother\/%
      \@ifnextchar[%]
         {\@provideslanguage{#1}}{\@provideslanguage{#1}[]}}
\def\@provideslanguage#1[#2]{%
   \wlog{Language: #1 #2}%
      \expandafter\xdef\csname ver@#1.ldf\endcsname{#2}%
      \endgroup}
\ProvidesFile{hyphen.cfg}
   [2005/05/07 v0.8
   Taken from Babel language switching mechanism
   and modified for Antomega]
%    \end{macrocode}
%
% \iffalse
%</kernel>
%<*antomega>
% \fi
%
% \subsection{Handling $\Omega$CP files}
%
% \DescribeMacro{\LoadOCPByName}
% The macro |\LoadOCPByName| takes two arguments: an OCP file name
% (without extension) and an $\Omega$ command which will be used for
% loading this file. If the referenced .ocp file doesn't exist in user's
% system, \file{id.ocp} will be used instead. So it is possible to proceed
% with document processing, even if some .ocp files were not found.
%
%    \begin{macrocode}
\def\LoadOCPByName#1#2{\IfFileExists{#2.ocp}{\ocp#1=#2}{
   \PackageWarning{antomega}{#2.ocp not found. 
      Identity will be used instead.}{}
   \ocp#1=id}}
%    \end{macrocode}
%
% Now we load some commonly used translation processes, using the
% macro |\LoadOCPByName|.
%
%    \begin{macrocode}
\ocp\IdOCP=id
\LoadOCPByName{\BasicIsoUni}{uniutf2uni}
\LoadOCPByName{\BasicWinUni}{uniutf2uni}
\LoadOCPByName{\BasicUtfUni}{uniutf2uni}
\LoadOCPByName{\BasicTexUni}{tex2punct}
\LoadOCPByName{\BasicBabelUni}{babel2punct}
\LoadOCPByName{\BasicAccentsUni}{uni2accents}
\LoadOCPByName{\UniToOmega}{uni2omega}
\LoadOCPByName{\Oldstyle}{oldstyle}
\LoadOCPByName{\LatinUniToTex}{uni2t1}
%    \end{macrocode}
%
% \file{uni2lig.ocp} is used for setting up Latin ligatures. However,
% the standard \TeX\ ligature mechanism should be a better choice, since
% using OCP for combinations like `fi' or `fl' may break hyphenation. So I
% provided a special |ffi| option which may be set either to `ocp' or to 
% `ligatures'. Setting it to `ligatures' simply prevents the translation 
% process from loading. Note that \file{uni2lig.ocp} is designed for pure 
% Unicode fonts and it is never used, if |output| is set to `omega'. In this 
% case you can't turn off processing ligatures via OCP, since the \file{omlgc}
% font doesn't contain any ligatures at all.
%
%    \begin{macrocode}
\def\opt@ocp{ocp}
\def\opt@ligatures{ligatures}
\define@key{antomega}{ffi}[ocp]{%
   \def\@tmpa{#1}
   \ifx\@tmpa\opt@ocp%
      \LoadOCPByName{\LatinUniToLig}{uni2lig}
   \else\ifx\@tmpa\opt@ligatures%
      \typeout{Antomega package option: ffi=ligatures}
      \LoadOCPByName{\LatinUniToLig}{id}
   \fi\fi
}
%    \end{macrocode}
%
% The following option is deprecated and preserved for backwards compatibility 
% only. Use `ffi=ligatures' instead.
%
%    \begin{macrocode}
\DeclareOption{noffi}{\setkeys{antomega}{ffi=ligatures}}
%    \end{macrocode}
%
% By default we use OCP for Latin ligatures in order to prevent MikTeX crashes.
%
%    \begin{macrocode}
\setkeys{antomega}{ffi=ocp}
%    \end{macrocode}
%
% Now we can define some standard OCP lists, useful generally
% for languages with Latin-based scripts.
%
% \DescribeMacro{\BasicTexOCP} This OCP list loads the default translation
% process for standard TeX ligatures and punctuation characters.
%
%    \begin{macrocode}
\ocplist\BasicTexOCP=
   \addbeforeocplist 1750 \BasicTexUni
\nullocplist
%    \end{macrocode}
%
% \DescribeMacro{\BasicBabelOCP} This OCP list loads the default translation
% process for Babel-like shorthands. If you don't like them, set
% `shorthands=off' for your language.
%
%    \begin{macrocode}
\ocplist\BasicBabelOCP=
   \addbeforeocplist 2000 \BasicBabelUni
\nullocplist
%    \end{macrocode}
%
% \DescribeMacro{\BasicAccentsOCP} This OCP list converts Unicode 
% combining accents to \TeX-styled |\accent| commands.
%
%    \begin{macrocode}
\ocplist\BasicAccentsOCP=
   \addbeforeocplist 2250 \BasicAccentsUni
\nullocplist
%    \end{macrocode}
%
% \DescribeMacro{\BasicInputUcsOCP} This is ANTOMEGA's default OCP
% list. It doesn't translate text to any other character set
% (so, sctually, does nothing).
%
%    \begin{macrocode}
\ocplist\BasicInputUcsOCP=
   \addbeforeocplist 500 \IdOCP
\nullocplist
%    \end{macrocode}
%
% \DescribeMacro{\BasicInputUtfOCP} This OCP list should be
% used for utf-8 encoded texts.
%
%    \begin{macrocode}
\ocplist\BasicInputUtfOCP=
   \addbeforeocplist 500 \BasicUtfUni
\nullocplist
%    \end{macrocode}
%
% \DescribeMacro{\BasicInputIsoOCP} This OCP list is intended for
% 8-bit texts using Latin ISO-8859-1 codepage, but note that
% it doesn't perform any real conversion, since ISO-8859-1 character
% codes are the same as in Unicode, and $\Omega$ automatically
% distinguishes 8-bit and 2-byte input.
%
%    \begin{macrocode}
\ocplist\BasicInputIsoOCP=
   \addbeforeocplist 500 \BasicIsoUni
\nullocplist
%    \end{macrocode}
%
% \DescribeMacro{\BasicInputWinOCP} This OCP list is intended for
% 8-bit texts using Latin windows-1252 codepage.
%
%    \begin{macrocode}
\ocplist\BasicInputWinOCP=
   \addbeforeocplist 500 \BasicWinUni
\nullocplist
%    \end{macrocode}
%
% The following OCP lists are used to convert a text to an
% $\Omega$ output.
%
% \DescribeMacro{\LatinUniOutOCP} A conversion to a Unicode
% font. The only operation which may be performed here is setting
% up the Latin ligatures.
%
%    \begin{macrocode}
\ocplist\LatinUniOutOCP=
   \addbeforeocplist 3500 \LatinUniToLig
\nullocplist
%    \end{macrocode}
%
% \DescribeMacro{\LatinOmegaOutOCP} A conversion to the default
% omlgc font. Its encoding differs from Unicode, and so a
% special conversion routine is required.
%
%    \begin{macrocode}
\ocplist\LatinOmegaOutOCP=
   \addbeforeocplist 3500 \UniToOmega
\nullocplist
%    \end{macrocode}
%
% \DescribeMacro{\LatinTexOutOCP} A conversion from Unicode
% to the T1 encoding.
%
%    \begin{macrocode}
\ocplist\LatinTexOutOCP=
   \addbeforeocplist 3500 \LatinUniToTex
\nullocplist
%    \end{macrocode}
%
% \DescribeMacro{\OldstyleOCP} This OCP list converts ASCII
% digits to their oldstyle equivalents. Note that it is not 
% compatible with the |omlgc| font.
%
%    \begin{macrocode}
\ocplist\OldstyleOCP=
   \addbeforeocplist 4000 \Oldstyle
\nullocplist
%    \end{macrocode}
%
% The following key allows to set input encoding globally for the whole
% document instead of setting it separately for each language. Of course,
% from all the standard codepages only `utf-8' makes a sense in this context. 
% `ucs-2' is also supported,  but this encoding doesn't require any 
% translation processes, because $\Omega$ uses it by default anyway.
%
% Beginning from ANTOMEGA~v.~0.8 you can also select `sw' to match a
% specific Unicode character representation, used in files generated
% by MacKichan software products.
%
%    \begin{macrocode}
\let\BasicInputOCP\BasicInputUcsOCP
   \define@key{antomega}{input}[ucs-2]{
      \def\@tmpa{#1}%
      \ifx\@tmpa\opt@utf%
         \let\BasicInputOCP\BasicInputUtfOCP%
         \typeout{Antomega package option: input=utf-8}
      \else\ifx\@tmpa\opt@sw%
         \def\U##1{/QQ[##1]}%
         \def\rmdefault{swtimes}%
         \let\westernrm\rmdefault%
         \LoadOCPByName{\BasicSWordUni}{sw2uni}%
         \ocplist\BasicInputSWordOCP=
            \addbeforeocplist 500 \BasicSWordUni
         \nullocplist
         \let\BasicInputOCP\BasicInputSWordOCP%
         \typeout{Antomega package option: input=sw}
      \else%
         \let\BasicInputOCP\BasicInputUcsOCP%
         \typeout{Antomega package option: input=ucs-2}
      \fi}
%    \end{macrocode}
%
% \DescribeMacro{\UppercaseOCP}.
% ANTOMEGA includes a special translation process, \file{uppercase-dflt.ocp}, 
% based on the \file{uppercase.ocp} file, available in older $\Omega$ 
% distribution, which may be used for lowercase to uppercase conversion. 
% Although standard conversion rules, based on |\lccode| and |\uccode| 
% settings, usually produce a better result, I have to use the OCP-based 
% conversion by default, because $\Omega$ incorrectly processes some character 
% codes in its UTF-8 mode. You can disable this feature by setting 
% |uppercase=standard|.
%
%    \begin{macrocode}
\LoadOCPByName{\Uppercase}{uppercase-dflt}
\ocplist\UppercaseOCP=
   \addbeforeocplist 3000 \Uppercase
\nullocplist
\def\MakeUppercase#1{{\pushocplist\UppercaseOCP#1}}
\def\opt@standard{standard}
\define@key{antomega}{uppercase}[ocp]{
   \def\@tmpa{#1}
   \ifx\@tmpa\opt@standard
      \let\MakeUppercase\uppercase
      \typeout{Antomega package option: use character codes}
      \typeout{for conversion to Uppercase}
   \fi}
%    \end{macrocode}
%
% \DescribeMacro{\oldstylenums}
% This command supposes that our text font contains old style
% numerals and that they are mapped to their places in the 
% Unicode Private Use area as defined in AGL. Don't use it
% with the |omlgc| font.
%
%    \begin{macrocode}
\def\oldstylenums#1{{\pushocplist\OldstyleOCP#1}}
%    \end{macrocode}
%
% \subsection{Encoding-independent commands for printing special characters}
%
% \DescribeMacro{\noocpchar}
% After expanding a command, Omega puts the result back into the OCP stack.
% If the result of expansion contains some characters, which have a 
% special meaning in the \TeX{} system, they will be processed again,
% instead of typing into the output. So it is impossible e.~g. to
% use the |\%| command in order to obtain the percent sign, because the
% character produced by that command is recognized anyway as a comment mark
% after processing it via OCP. The same problem affects character codes 
% less than 0x20 (used in most 8-bit encodings), since these characters
% usually have no |\catcode| assigned, and so are treated as invalid
% in the input.
%
% Previously some hacks were used to prevent this effect. For example, 
% in the \file{omlgc} some ASCII characters are reproduced once again 
% in the 0x80--0x0F range, not used in Unicode, so that e.~g. the |\%| 
% command could actually refer to a slot different from the `real' percent
% sign. For 8-bit fonts we had to assign code 12 to some characters
% in order to make them `valid'. Older ANTOMEGA versions had a special
% option, |specials|, used to control such situations.
%
% However, this option is deprecated (and removed) now. Instead, we 
% just define a special command which puts |\clearocplists| before 
% |\char| in order to prevent the result from placing into the OCP
% stack\ldots
%
%    \begin{macrocode}
\def\noocpchar#1{{\clearocplists\char#1}}
%    \end{macrocode}
%
% \ldots{}and then apply it to the most commonly used special characters.
%
%    \begin{macrocode}
\def\#{\noocpchar{"23}}
\def\%{\noocpchar{"25}}
\def\&{\noocpchar{"26}}
%    \end{macrocode}
%
% \subsection{Omega-specific commands to handle \TeX{} font encodings}
%
% \DescribeMacro{\uniencoding}
% It is necessary to declare a special encoding for Omega-specific
% 2-byte fonts. $\Omega$ developers called it |UT1|.
%
%    \begin{macrocode}
\def\uniencoding{UT1}
%    \end{macrocode}
%
% \DescribeMacro{\ant@load@encoding}
% The concept of current font encoding doesn't really matter for ANTOMEGA.
% Although some text commands traditionally used in \LaTeX\ are allowed
% in the input, they just should be always mapped to the same Unicode
% codepoints, and it is a task of translation processes to translate
% them to the current font encoding. 
%
% So the only reason why we have to declare font encodings at all is that
% Omega needs to know the current encoding in order to select an
% appropriate font. That's why \file{antomega} no longer loads any encoding
% definition files. Instead, whatever encoding is requested,
% \file{antomega} always declares it itself, and then loads the same
% list of mappings between text commands and Unicode codepoints.
%
%    \begin{macrocode}
\def\ant@load@encoding#1{%
   \edef\ant@encodingfile{%
      \lowercase{\def\noexpand\ant@encodingfile{#1enc-antomega.def}}}%
   \ant@encodingfile
   \InputIfFileExists{\ant@encodingfile}{}{%
      \DeclareFontEncoding{#1}{}{}
      \PackageWarning{antomega}{The \ant@encodingfile\ file was not found.
         The #1 encoding was defined by antomega.}{}
      }
   \let\ant@encodingfile\@undefined
}
\ant@load@encoding{\uniencoding}
\def\encodingdefault{\uniencoding}
%    \end{macrocode}
%
% Since T1 is loaded into the Lambda format, we have to redefine it now.
%
%    \begin{macrocode}
\ant@load@encoding{T1}
%    \end{macrocode}
%
% \subsection{Font issues}
%
% The omlgc font is not perfect, but it is included into all standard
% \TeX\ distributions. So, it will be used by default.
%
%    \begin{macrocode}
\def\rmdefault{omlgc}
%    \end{macrocode}
%
% \file{Antomega} stores its default font names in the |\westernrm|,
% |\westernsf| and |\westerntt| variables, since |\rmdefault|, |\sfdefault|
% and |\ttdefault| will be redefined each time we switch to a new
% language.
%
%    \begin{macrocode}
\ifx\westernrm\@undefined\let\westernrm=\rmdefault\fi
\ifx\westernsf\@undefined\let\westernsf=\sfdefault\fi
\ifx\westerntt\@undefined\let\westerntt=\ttdefault\fi
%    \end{macrocode}
%
% Generally speaking, with $\Omega$ we should use translation processes 
% rather than active characters. So I made |textasciitilde| an 
% `other symbol'.
%
%    \begin{macrocode}
\catcode`\~=12
%    \end{macrocode}
%
% \iffalse
%</antomega>
%<*kernel>
% \fi
%
% \subsection{Language-specific commands which should be loaded
% into the Lambda format}
%
% Again, no tests for |\language| and |\newlanguage|, because $\Omega$
% should be always compatible with \TeX\ version~3.0.
%
%    \begin{macrocode}
\countdef\last@language=19
%    \end{macrocode}
%
% \DescribeMacro{\addlanguage}
% To add languages to \TeX's memory plain \TeX\ version~3.0
% supplies |\newlanguage|. However, a new macro is defined here,
% because the original |\newlanguage| was defined to be |\outer|.
%
%    \begin{macrocode}
\def\addlanguage{\alloc@9\language\chardef\@cclvi}
%    \end{macrocode}
%
% \DescribeMacro{\adddialect}
% The macro |\adddialect| can be used to add the name of a dialect
% or variant language, for which an already defined hyphenation
% table can be used.
%
%    \begin{macrocode}
\def\adddialect#1#2{%
   \global\chardef#1#2\relax
   \wlog{\string#1 = a dialect from \string\language#2}}
%    \end{macrocode}
%
% \DescribeMacro{\iflanguage}
% Users might want to test (in a private package for instance)
% which language is currently active. For this we provide a test
% macro, |\iflanguage|, that has three arguments.  It checks
% whether the first argument is a known language. If so, it
% compares the first argument with the value of |\language|. Then,
% depending on the result of the comparison, it executes either the
% second or the third argument.
%
%    \begin{macrocode}
\def\iflanguage#1{%
   \expandafter\ifx\csname l@#1\endcsname\relax
      \PackageWarning{antomega}{#1 is not a known language.}%
   \else
      \ifnum\csname l@#1\endcsname=\language
         \expandafter\@firstoftwo
      \else
         \expandafter\@secondoftwo
      \fi%
   \fi}
%    \end{macrocode}
%
% \subsection{Handling character codes}
%
% \iffalse
%</kernel>
%<*kernel|antomega>
% \fi
%
% We can't get correct hyphenation for our 2-byte characters without 
% setting |\catcode|, |\lccode| and |\uccode| for each of them.
% The following commands simplify making such definitions.
%
% \DescribeMacro{\makeletter} This command takes two arguments, the first
% being an uppercase character and the second a corresponding lowercase
% character, and sets |\lccode| and |\uccode| for both characters.
%
%    \begin{macrocode}
\ifx\makeletter\@undefined
   \def\makeletter#1#2{%
      \ifnum\catcode#2=11\else\catcode#2=12 \fi
      \ifnum\catcode#1=11\else\catcode#1=12 \fi
      \uccode#1=#1 \uccode#2=#1%
      \lccode#1=#2 \lccode#2=#2}
\fi
%    \end{macrocode}
%
% \DescribeMacro{\makelcletter} This command takes two arguments, the first
% being an uppercase character and the second a corresponding lowercase
% character, and sets |\lccode| and |\uccode| for the lowercase character.
%
%    \begin{macrocode}
\ifx\makelcletter\@undefined
   \def\makelcletter#1#2{%
      \ifnum\catcode#2=11\else\catcode#2=12 \fi
      \uccode#2=#1%
      \lccode#2=#2}
\fi
%    \end{macrocode}
%
% \DescribeMacro{\makeucletter} This command takes two arguments, the first
% being an uppercase character and the second a corresponding lowercase
% character, and sets |\lccode| and |\uccode| for the uppercase character.
%
%    \begin{macrocode}
\ifx\makeucletter\@undefined
   \def\makeucletter#1#2{%
      \ifnum\catcode#1=11\else\catcode#1=12 \fi
      \uccode#1=#1%
      \lccode#1=#2}
\fi
%    \end{macrocode}
%
% \DescribeMacro{\makesameletter} This command takes two arguments, both
% of them being uppercase or lowercase characters,
% and sets |\lccode| and |\uccode| for character 1 equal to character 2.
%
%    \begin{macrocode}
\ifx\makesameletter\@undefined
   \def\makesameletter#1#2{%
      \ifnum\catcode#1=11\else\catcode#1=12 \fi
      \uccode#1=\uccode#2%
      \lccode#1=\lccode#2}
\fi
%    \end{macrocode}
%
% \iffalse
%</kernel|antomega>
%<*config>
% \fi
%
% The following code should be written into \file{antomega.cfg}. You may 
% edit that file depending from which Unicode ranges you really need.
%
%    \begin{macrocode}
\input{uni0100.def} % Latin Extended-A
\input{uni0370.def} % Greek Basic
\input{uni0400.def} % Cyrillic
\input{uni1f00.def} % Greek Extended
%    \end{macrocode}
%
% \iffalse
%</config>
%<*kernel>
% \fi
%
% In \file{hyphen.cfg} first we test if \file{antomega.cfg} exists, and 
% then load it.
%
%    \begin{macrocode}
\openin1 = antomega.cfg
\ifeof1
   \message{I couldn't find the file antomega.cfg.\space
           Codes for Unicode characters will not be loaded.}
\else
   %%
%% This file generates file antomega.sty from file antomega.dtx by means
%% of the LaTeX program docstrip. This program eliminates almost all
%% commented lines thus speeding up the loading of the code. Documentation
%% may be obtained by running LaTeX on the source file antomega.dtx.
%%
%%
%% ----------------------------------------
%%
%%  This system is distributed in the hope that it will be useful,
%%  but WITHOUT ANY WARRANTY; without even the implied warranty of
%%  MERCHANTABILITY or FITNESS FOR A PARTICULAR PURPOSE.
%%
%% IMPORTANT NOTICE:
%%
%% Copyright 2002 Alexej Kryukov
%% All rights reserved.
%%
%% Permission is granted to distribute verbatim copies of this file
%% together with antomega.dtx.
%%
%% No other permissions to copy or distribute this file in any form
%% are granted and in particular no permission to modify its contents.
%%
%% --------------- start of docstrip commands ------------------
%%
\def\batchfile{antomega.ins}
\def\fileversion{0.8}
\def\filedate{7 May 2005}
\input docstrip.tex
\preamble

This file is a part of the ANTOMEGA project version \fileversion
-----------------------------------------------------

It may be distributed under the terms of the LaTeX Project Public
License, as described in lppl.txt in the base LaTeX distribution.
Either version 1.0 or, at your option, any later version.
Copyright (C) 2001 -- 2002 by Alexej Kryukov
Please report errors to: A.M. Kryukov <basileia@yandex.ru>

\endpreamble

\keepsilent

\Msg{*** Generating the extension package antomega.sty ***}

\generateFile{antomega.sty}{t}{\from{antomega.dtx}{antomega}}

\ifToplevel{
\Msg{***********************************************************}
\Msg{*}
\Msg{* Now move antomega.sty where Omega can find it }
\Msg{*}
\Msg{* Happy TeXing!}
\Msg{***********************************************************}
}

\Msg{*** Generating the files which should be loaded into Lambda format ***}

\generateFile{hyphen.cfg}{t}{\from{antomega.dtx}{kernel,patterns}}
\generateFile{antomega.cfg}{t}{\from{antomega.dtx}{config}}

\ifToplevel{
\Msg{***********************************************************}
\Msg{*}
\Msg{* Now move hyphen.cfg and antomega.cfg where iniomega can }
\Msg{* find them }
\Msg{*}
\Msg{* Happy TeXing!}
\Msg{***********************************************************}
}

\endinput

\fi
\closein1
%    \end{macrocode}
%
% \iffalse
%</kernel>
%<*antomega>
% \fi
%
% \subsection{Warnings and error messages}
%
% \DescribeMacro{\ant@nocodes}
% This command is used to show a warning message if $\Omega$ can't find 
% a file with lccodes/uccodes for the specified Unicode range.
%
%    \begin{macrocode}
\providecommand*{\ant@nocodes}[3]{%
   \PackageWarningNoLine{antomega}%
      {No file was found with symbol codes\MessageBreak
         for the #2 range #3.\MessageBreak
         You may proceed, but your #1 texts\MessageBreak
         probably will not be correctly hyphenated.}}
%    \end{macrocode}
%
% \DescribeMacro{\ant@nopatterns}
% This macro is based on Babel's |\@nopatterns| command.
% However I removed test if |\PackageWarningNoLine| is defined,
% because probably nobody will try to build \LaTeX~2.09 based
% format for $\Omega$.
%
%    \begin{macrocode}
\providecommand*{\ant@nopatterns}[1]{%
   \PackageWarningNoLine{antomega}%
      {No hyphenation patterns were loaded for\MessageBreak
         the language `#1'\MessageBreak
         I will use the patterns loaded for \string\language=0
         instead}}
%    \end{macrocode}
%
% \DescribeMacro{\ant@nolang}
% This macro defines the error message which will be displayed
% if the requested language definition file was not found.
%
%    \begin{macrocode}
\providecommand*{\ant@nolang}[1]{%
   \PackageWarningNoLine{antomega}%
      {Couldn't find file omega-#1.ldf!!}}
%    \end{macrocode}
%
% \subsection{Different corrections for standard \LaTeX\ commands}
%
% With $\Omega$ we usually have to control all commands which print
% some strings (for example, to headers/footers), so that they always 
% apply correct translation processes and correct font to the text they 
% produce. However, modifying these commands may be inconvenient if we 
% have to use some packages which also try to redefine them. If you
% want to prevent \file{antomega} from modifying these commands, you
% may control its behavior by setting the |localmarks| option either 
% to `on' or to `off'.
%
% \DescribeMacro{\local@marks}
% This command is executed every time we are switching to a new
% language. It applies all rules specific for this language to
% the text, which is written to headers/footers.
%
%    \begin{macrocode}
\def\opt@enabled{on}
\def\opt@disabled{off}
\def\opt@tex{tex}
\def\opt@omega{omega}
\def\opt@unicode{unicode}
\def\opt@utf{utf-8}
\def\opt@ucs{ucs-2}
\def\opt@sw{sw}
\define@key{antomega}{localmarks}[on]{%
   \def\@tmpa{#1}
   \ifx\@tmpa\opt@enabled
      \def\local@marks##1{%
         \def\markboth####1####2{%
            \begingroup%
               \let\label\relax \let\index\relax \let\glossary\relax%
               \unrestored@protected@xdef\@themark%
               {{\foreignlanguage{##1}{####1}}{\foreignlanguage{##1}{####2}}}%
               \@temptokena \expandafter{\@themark}%
               \mark{\the\@temptokena}%
            \endgroup%
            \if@nobreak\ifvmode\nobreak\fi\fi}%
            \def\markright####1{%
               \begingroup%
                  \let\label\relax \let\index\relax \let\glossary\relax%
                  \expandafter\@markright\@themark{\foreignlanguage{##1}{####1}}%
                  \@temptokena \expandafter{\@themark}%
                  \mark{\the\@temptokena}%
               \endgroup%
               \if@nobreak\ifvmode\nobreak\fi\fi}%
            \def\@markright####1####2####3{\@temptokena{####1}%
               \unrestored@protected@xdef\@themark{{\the\@temptokena}%
               {{####3}}}}}
   \else\ifx\@tmpa\opt@disabled
      \def\local@marks#1{}
      \typeout{Antomega package option: localmarks=off}
   \fi\fi
}
%    \end{macrocode}
%
% The following option is preserved for backwards compatibility only.
% Use `localmarks=off' instead.
%
%    \begin{macrocode}
\DeclareOption{nolocalmarks}{\setkeys{antomega}{localmarks=off}}
%    \end{macrocode}
%
% By default string conversion in headers and footers is enabled.
%
%    \begin{macrocode}
\setkeys{antomega}{localmarks=on}
%    \end{macrocode}
%
% \DescribeMacro{\oaddto}
% This command was taken from the Babel package and renamed in
% order to avoid conflicts. It is useful for modifying
% some language-specific commands, predefined in *.lfd files.
%
%    \begin{macrocode}
\def\oaddto#1#2{%
   \ifx#1\@undefined
      \def#1{#2}%
   \else
      \ifx#1\relax
         \def#1{#2}%
      \else
         {\toks@\expandafter{#1#2}%
           \xdef#1{\the\toks@}}%
      \fi
   \fi
}
%    \end{macrocode}
%
% \subsection{Loading languages}
%
% Standard commands for loading languages
% (the core of the antomega package).
%
% \DescribeMacro{\background} This command requires one arguments which
% must be a language name and loads it as the first language for 
% our document.
%
% The optional argument is a set of parameters and their values
% for the given language.
%
%    \begin{macrocode}
\newcommand{\background}[2][]{%
   \IfFileExists{omega-#2.ldf}%
   {\input{omega-#2.ldf}%
      \AtBeginDocument{\selectlanguage[#1]{#2}}%
   \newenvironment{#2}[1][]{\begin{otherlanguage}[####1]{#2}}%
      {\end{otherlanguage}}%
   \expandafter\newcommand\csname local#2\endcsname[2][]{%
      \foreignlanguage[####1]{#2}{####2}}}%
   {\ant@nolang{#2}}%
}
%    \end{macrocode}
%
% \DescribeMacro{\load} This command takes one argument which
% must be a language name and loads it in addition to the first
% language.
%
% The optional argument is a set of parameters and their values
% for the given language.
%
% Both |\background| and |\load| commands are used to define 
% a |\local<$language>| command and a |<$language>| environment.
% Commands and environments with these names were standard way
% to switch languages in the original \file{omega} package, as well as
% in \file{antomega} until the version 0.6. Now they are defined
% in terms of standard babel-like commands.
%
%    \begin{macrocode}
\newcommand{\load}[2][]{\IfFileExists{omega-#2.ldf}
   {\input{omega-#2.ldf}\setkeys{#2}{#1}%
   \newenvironment{#2}[1][]{\begin{otherlanguage}[####1]{#2}}%
      {\end{otherlanguage}}
   \expandafter\newcommand\csname local#2\endcsname[2][]{%
      \foreignlanguage[####1]{#2}{####2}}}
   {\ant@nolang{#2}}}
%    \end{macrocode}
%
% \subsection{Default values for language-specific settings}
%
% First we define some standard values for the punctuation
% commands, used by \file{lat2punct.otp}. The command names
% are self-explanative.
%
%    \begin{macrocode}
\def\common@punctuation{%
   \def\InitialThinSpace{\nobreak\hskip.2em\ignorespaces}%
   \def\ExplicitHyphen{\nobreak\-\nobreak\hskip\z@skip}%
   \def\AllowHyphenation{\hskip\z@skip}%
   \def\DisableLigature{\textormath{\nobreak\discretionary{-}{}%
      {\kern.03em}\allowhyphens}{}}%
   \def\CompoundWordMarkWithBreakpoint{\nobreak-\hskip\z@skip}%
   \def\CompoundWordMarkNoBreakpoint{\textormath{\leavevmode\hbox{-}}{-}}%
   \def\LeftDoubleQuotationMark{^^^^201c}%
   \def\RightDoubleQuotationMark{^^^^201d}%
   \def\LeftPointingDoubleAngleQuotationMark{^^^^00ab}%
   \def\RightPointingDoubleAngleQuotationMark{^^^^00bb}%
   \def\GermanLeftDoubleQuotationMark{^^^^201e}%
   \def\GermanRightDoubleQuotationMark{^^^^201c}%
   \def\QuestionMark{?}%
   \def\ExclamationMark{!}%
   \def\InvertedQuestionMark{^^^^00bf}%
   \def\InvertedExclamationMark{^^^^00a1}%
   \def\Semicolon{;}%
   \def\Colon{:}%
   \def\NonBreakingSpace{\leavevmode\nobreak\ }}
%    \end{macrocode}
%
% \DescribeMacro{\common@font}
% The |\common@font| macro will be used at the beginning of the
% document and also each time we should return to the default
% fonts (e. g. before switching to another language).
%
%    \begin{macrocode}
\def\common@font{\normalfont\fontfamily{\westernrm}%
   \fontencoding{\uniencoding}\selectfont%
   \let\rmdefault=\westernrm\let\sfdefault=\westernsf%
   \let\ttdefault=\westerntt\let\encodingdefault=\uniencoding}
%    \end{macrocode}
%
% \DescribeMacro{\common@language}
% This macro is used for enabling default hyphenation patterns.
%
%    \begin{macrocode}
\def\common@language{%
   \protect\language=0%
   \lefthyphenmin=2\righthyphenmin=3}
%    \end{macrocode}
%
% \DescribeMacro{\noextrascurrent}
% \DescribeMacro{\originalOmega}
% The |\originalOmega| macro is used to switch all settings,
% which could be modified by the language switching commands,
% to their default values.
%
%    \begin{macrocode}
\def\noextrascurrent#1{\@ifundefined{noextras@#1}{}%
   {\csname noextras@#1\endcsname}}
\def\originalOmega{\@ifundefined{languagename}{}%
   {\noextrascurrent{\languagename}}%
   \common@language%
   \common@punctuation%
   \common@font%
   \clearocplists%
   }
\AtBeginDocument{\originalOmega}
%    \end{macrocode}
%
% \subsection{Language switching commands}
%
% In case we have Babel's \file{hyphen.cfg} loaded into format,
% |\foreignlanguage| is already defined, and so we have to unset 
% it first.
%
%    \begin{macrocode}
\@ifundefined{foreignlanguage}{}%
   {\let\foreignlanguage\@undefined}
%    \end{macrocode}
%
% \DescribeMacro{\foreignlanguage}
% This macro works exactly as Babel's |\foreignlanguage| command,
% but it takes 3 arguments. The first (optional) argument allows
% to set any options, defined in the support file for the
% given language. The second argument is languages's name itself,
% and the third~--- the piece of text, which should be typeset in
% this language.
%
%    \begin{macrocode}
\newcommand{\foreignlanguage}[3][]{%
   \@ifundefined{inlineextras@#2}{\ant@nolang{#2}}{%
     {\def\languagename{#2}%
      \setkeys{#2}{#1}%
      \csname inlineextras@#2\endcsname#3}%
}}
%    \end{macrocode}
%
% |\selectlanguage| have to be redefined too.
%
%    \begin{macrocode}
\@ifundefined{selectlanguage}{}%
   {\let\selectlanguage\@undefined}
%    \end{macrocode}
%
% \DescribeMacro{\selectlanguage}
% This macro works exactly as Babel's |\selectlanguage| command,
% but it takes 2 arguments. The second argument is languages's name 
% itself, and the first (optional) allows to set any options, defined 
% in the support file for the given language. 
%
%    \begin{macrocode}
\newcommand{\selectlanguage}[2][]{%
   \@ifundefined{blockextras@#2}{\ant@nolang{#2}}{%
      \def\ant@pop@language{%
         \ant@set@language{\languagename}%
         \let\emp@langname\undefined}%
      \aftergroup\ant@pop@language%
      \setkeys{#2}{#1}%
      \ant@set@language{#2}%
}}
\newcommand{\ant@set@language}[1]{%
   \select@language{#1}%
   \if@filesw%
      \protected@write\@auxout{}{\protect\select@language{#1}}%
      \addtocontents{toc}{\protect\select@language{#1}}%
      \addtocontents{lof}{\protect\select@language{#1}}%
      \addtocontents{lot}{\protect\select@language{#1}}%
   \fi%
}
\@ifundefined{select@language}{}%
   {\let\select@language\@undefined}
\newcommand{\select@language}[1]{%
   \originalOmega%
   \edef\languagename{#1}%
   \csname blockextras@#1\endcsname%
}
\let\ant@pop@language\relax
%    \end{macrocode}
%
% We have to redefine the |otherlanguage| environment as well.
%
%    \begin{macrocode}
\@ifundefined{otherlanguage}{}%
   {\let\otherlanguage\@undefined}
\@ifundefined{endotherlanguage}{}%
   {\let\endotherlanguage\@undefined}
%    \end{macrocode}
%
% \DescribeEnv{otherlanguage}
% This environment works exactly as Babel's |otherlanguage| environment.
% The only difference is that is has an optional argument allowing
% to set any options, defined in the .ldf file.
%
%    \begin{macrocode}
\newenvironment{otherlanguage}[2][]{%
   \selectlanguage[#1]{#2}%
   }{}
%    \end{macrocode}
%
% \iffalse
%</antomega>
%<*patterns>
% \fi
%
% \subsection{Handling hyphenation rules}
%
% \DescribeEnv{hyphenrules}
% The environment \texttt{hyphenrules} can be used to select
% \emph{just} the hyphenation rules. This environment does
% \emph{not} change |\languagename| and when the hyphenation rules
% specified were not loaded it has no effect.
%
%    \begin{macrocode}
\def\hyphenrules#1{%
   \expandafter\ifx\csname l@#1\endcsname\@undefined
      \@nolanerr{#1}%
   \else
      \language=\csname l@#1\endcsname\relax
   \fi
}
\def\endhyphenrules{}
%    \end{macrocode}
%
% \DescribeMacro{\set@hyphenmins}
% This macro sets the values of |\lefthyphenmin| and
% |\righthyphenmin|. It expects two values as its argument.
%
%    \begin{macrocode}
\def\set@hyphenmins#1#2{\lefthyphenmin#1\righthyphenmin#2}
%    \end{macrocode}
%
% \iffalse
%</patterns>
%<*antomega>
% \fi
%
% \DescribeMacro{\local@hyphenmins}
% This macro takes 3 arguments: a language name and default
% |\lefthyphenmin| and |\righthyphenmin| values for that language.
% First it tests if |\<language>hyphenmins| is already defined (i.~e.
% some |\lefthyphenmin| and |\righthyphenmin| values were specified in
% the hyphenation patterns loaded into format), and either executes this
% command, or sets both variables to the provided default values.
%
%    \begin{macrocode}
\newcommand{\local@hyphenmins}[3]{%
   \@ifundefined{#1hyphenmins}%
      {\lefthyphenmin=#2\righthyphenmin=#3}%
      {\csname #1hyphenmins\endcsname}%
}
%    \end{macrocode}
%
% \iffalse
%</antomega>
%<*patterns>
% \fi
%
% \subsection{Loading hyphenation rules into format}
%
% \DescribeMacro{\process@line}
% Each line in the file \file{language.dat} is processed by
% |\process@line| after it is read. The first thing this macro does
% is to check wether the line starts with \texttt{=}.
% When the first token of a line is an \texttt{=}, the macro
% |\process@synonym| is called; otherwise the macro
% |\process@language| will continue.
%
%    \begin{macrocode}
\def\process@line#1#2 #3/{%
   \ifx=#1
      \process@synonym#2 /
   \else
      \process@language#1#2 #3/%
   \fi
}
%    \end{macrocode}
%
% \DescribeMacro{\process@synonym}
% This macro takes care of the lines which start with an
% \texttt{=}. It needs an empty token register to begin with.
%
%    \begin{macrocode}
\toks@{}
\def\process@synonym#1 /{%
   \ifnum\last@language=\m@ne
%    \end{macrocode}
%
% When no languages have been loaded yet the name following the
% \texttt{=} will be a synonym for hyphenation register 0.
%
%    \begin{macrocode}
      \expandafter\chardef\csname l@#1\endcsname0\relax
      \wlog{\string\l@#1=\string\language0}
%    \end{macrocode}
%
% As no hyphenation patterns are read in yet, we can not yet set
% the hyphenmin paramaters. Therefor a commands to do so is stored
% in a token register and executed when the first pattern file has
% been processed.
%
%    \begin{macrocode}
      \toks@\expandafter{\the\toks@
         \expandafter\let\csname #1hyphenmins\expandafter\endcsname
         \csname\languagename hyphenmins\endcsname}%
   \else
%    \end{macrocode}
%
% Otherwise the name will be a synonym for the language loaded last.
%
%    \begin{macrocode}
      \expandafter\chardef\csname l@#1\endcsname\last@language
      \wlog{\string\l@#1=\string\language\the\last@language}
%    \end{macrocode}
%
%    We also need to copy the hyphenmin paramaters for the synonym.
%
%    \begin{macrocode}
      \expandafter\let\csname #1hyphenmins\expandafter\endcsname
      \csname\languagename hyphenmins\endcsname
   \fi
  }
%    \end{macrocode}
%
% \DescribeMacro{\process@language}
% The macro |\process@language| is used to process a non-empty line
% from the `configuration file'. It has three arguments, each
% delimited by white space. The third argument is optional,
% therefore a |/| character is expected to delimit the last
% argument.  The first argument is the `name' of a language, the
% second is the name of the file that contains the patterns. The
% optional third argument is the name of a file containing
% hyphenation exceptions.
%
% The first thing to do is call |\addlanguage| to allocate a
% pattern register and to make that register `active'.
%
%    \begin{macrocode}
\def\process@language#1 #2 #3/{%
   \expandafter\addlanguage\csname l@#1\endcsname
   \expandafter\language\csname l@#1\endcsname
   \def\languagename{#1}%
%    \end{macrocode}
%
% Then the `name' of the language that will be loaded now is
% added to the token register |\toks8|. and finally
% the pattern file is read.
%
%    \begin{macrocode}
   \global\toks8\expandafter{\the\toks8#1, }%
%    \end{macrocode}
%
% For some hyphenation patterns it is needed to load them with a
% specific font encoding selected. This can be specified in the
% file \file{language.dat} by adding for instance `\texttt{:T1}' to
% the name of the language. The macro |\ant@get@enc| extracts the
% font encoding from the language name and stores it in
% |\ant@hyph@enc|.
%
%    \begin{macrocode}
   \begingroup
      \ant@get@enc#1:\@@@
      \ifx\ant@hyph@enc\@empty
      \else
         \fontencoding{\ant@hyph@enc}\selectfont
      \fi
%    \end{macrocode}
%
% Some pattern files contain assignments to |\lefthyphenmin| and
% |\righthyphenmin|. \TeX\ does not keep track of these
% assignments. Therefor we try to detect such assignments and
% store them in the |\<langvar>hyphenmins| macro. When no
% assignments were made we provide a default setting.
%
%    \begin{macrocode}
      \lefthyphenmin\m@ne
%    \end{macrocode}
%
% Some pattern files contain changes to the |\lccode| en |\uccode|
% arrays. Such changes should remain local to the language;
% therefor we process the pattern file in a group; the |\patterns|
% command acts globally so it's effect will be remembered.
%
%    \begin{macrocode}
      \input #2\relax
%    \end{macrocode}
%
% Now we globally store the settings of |\lefthyphenmin| and
% |\righthyphenmin| and close the group.
%
%    \begin{macrocode}
      \ifnum\lefthyphenmin=\m@ne
      \else
         \expandafter\xdef\csname #1hyphenmins\endcsname{%
            \set@hyphenmins{\the\lefthyphenmin}{\the\righthyphenmin}}%
      \fi
   \endgroup
%    \end{macrocode}
%
% If the counter |\language| is still equal to zero we set the
% hyphenmin parameters to the values for the language loaded on
% pattern register 0.
%
%    \begin{macrocode}
   \ifnum\the\language=\z@
      \expandafter\ifx\csname #1hyphenmins\endcsname\relax
         \set@hyphenmins\tw@\thr@@\relax
      \else
         \expandafter\expandafter\expandafter\set@hyphenmins
            \csname #1hyphenmins\endcsname
      \fi
%    \end{macrocode}
%
% Now execute the contents of token register zero as it may
% contain commands which set the hyphenmin parameters for synonyms
% that were defined before the first pattern file is read in.
%
%    \begin{macrocode}
      \the\toks@
   \fi
%    \end{macrocode}
%
% Empty the token register after use.
%
%    \begin{macrocode}
   \toks@{}%
%    \end{macrocode}
%
% When the hyphenation patterns have been processed we need to see
% if a file with hyphenation exceptions needs to be read. This is
% the case when the third argument is not empty and when it does
% not contain a space token.
%
%    \begin{macrocode}
   \def\ant@tempa{#3}%
   \ifx\ant@tempa\@empty
   \else
      \ifx\ant@tempa\space
      \else
         \input #3\relax
      \fi
   \fi
}
%    \end{macrocode}
%
% \DescribeMacro{\ant@get@enc}
% \DescribeMacro{\ant@hyph@enc}
% The macro |\ant@get@enc| extracts the font encoding from the
% language name and stores it in |\ant@hyph@enc|. It uses delimited
% arguments to acheive this.
%
%    \begin{macrocode}
\def\ant@get@enc#1:#2\@@@{%
%    \end{macrocode}
%
% First store both arguments in temporary macros,
%
%    \begin{macrocode}
  \def\ant@tempa{#1}%
  \def\ant@tempb{#2}%
%    \end{macrocode}
%
% then, if the second argument was empty, no font encoding was
% specified and we're done.
%
%    \begin{macrocode}
  \ifx\ant@tempb\@empty
    \let\ant@hyph@enc\@empty
  \else
%    \end{macrocode}
%
% But if the second argument was \emph{not} empty it will now have
% a superfluous colon attached to it which we need to remove. This
% done by feeding it to |\ant@get@enc|. The string that we are
% after will then be in the first argument and be stored in
% |\ant@tempa|.
%
%    \begin{macrocode}
    \ant@get@enc#2\@@@
    \edef\ant@hyph@enc{\ant@tempa}%
  \fi}
%    \end{macrocode}
%
% The configuration file can now be opened for reading.
%
%    \begin{macrocode}
\openin1 = language.dat
%    \end{macrocode}
%
% See if the file exists, if not, use the default hyphenation file
% \file{hyphen.tex}. The user will be informed about this.
%
%    \begin{macrocode}
\ifeof1
  \message{I couldn't find the file language.dat,\space
           I will try the file hyphen.tex}
  \input hyphen.tex\relax
\else
%    \end{macrocode}
%
% Pattern registers are allocated using count register
% |\last@language|. Its initial value is~0. The definition of the
% macro |\newlanguage| is such that it first increments the count
% register and then defines the language. In order to have the
% first patterns loaded in pattern register number~0 we initialize
% |\last@language| with the value~$-1$.
%
%    \begin{macrocode}
  \last@language\m@ne
%    \end{macrocode}
%
% We now read lines from the file untill the end is found
%
%    \begin{macrocode}
  \loop
%    \end{macrocode}
%
% While reading from the input it is useful to switch off
% recognition of the end-of-line character. This saves us stripping
% off spaces from the contents of the controlsequence.
%
%    \begin{macrocode}
    \endlinechar\m@ne
    \read1 to \ant@line
    \endlinechar`\^^M
%    \end{macrocode}
%
% Empty lines are skipped.
%
%    \begin{macrocode}
    \ifx\ant@line\@empty
    \else
%    \end{macrocode}
%
% Now we add a space and a |/| character to the end of
% |\ant@line|. This is needed to be able to recognize the third,
% optional, argument of |\process@language| later on.
%
%    \begin{macrocode}
      \edef\ant@line{\ant@line\space/}%
      \expandafter\process@line\ant@line
    \fi
%    \end{macrocode}
%
% Check for the end of the file.  To avoid a new \texttt{if}
% control sequence we create the necessary |\iftrue| or |\iffalse|
% with the help of |\csname|.  But there is one complication with
% this approach: when skipping the \texttt{loop...repeat} \TeX\ has
% to read |\if|/|\fi| pairs.  So we have to insert a `dummy'
% |\iftrue|.
%
%    \begin{macrocode}
    \iftrue \csname fi\endcsname
    \csname if\ifeof1 false\else true\fi\endcsname
  \repeat
%    \end{macrocode}
%
% Reactivate the default patterns,
%
%    \begin{macrocode}
  \language=0
\fi
%    \end{macrocode}
%
% and close the configuration file.
%
%    \begin{macrocode}
\closein1
%    \end{macrocode}
%
% Also remove some macros from memory
%
%    \begin{macrocode}
\let\process@language\@undefined
\let\process@synonym\@undefined
\let\process@line\@undefined
\let\ant@tempa\@undefined
\let\ant@tempb\@undefined
\let\ant@eq@\@undefined
\let\ant@line\@undefined
\let\ant@get@enc\@undefined
\ifx\addto@hook\@undefined
\else
  \expandafter\addto@hook\expandafter\everyjob\expandafter{%
    \expandafter\typeout\expandafter{\the\toks8 loaded.}}
\fi
%    \end{macrocode}
%
% \iffalse
%</patterns>
%<*antomega>
% \fi
%
% \subsection{Processing options}
%
%    \begin{macrocode}
\DeclareOption*{%
   \edef\@temp{\noexpand\setkeys{antomega}{\CurrentOption}}%
   \@temp%
}
\ProcessOptions
%    \end{macrocode}
%
% \iffalse
%</antomega>
% \fi
%
% \Finale
%
\endinput

% \iffalse meta-comment
%
% Extracted from classpack.xml
% classpack.dtx is copyright © 2012-2013 by Peter Flynn <peter@silmaril.ie>
%
% This work may be distributed and/or modified under the
% conditions of the LaTeX Project Public License, either
% version 1.3 of this license or (at your option) any later
% version. The latest version of this license is in:
%
%     http://www.latex-project.org/lppl.txt
%
% and version 1.3 or later is part of all distributions of
% LaTeX version 2005/12/01 or later.
%
% This work has the LPPL maintenance status `maintained'.
% 
% The current maintainer of this work is Peter Flynn <peter@silmaril.ie>
%
% This work consists of the files classpack.dtx and classpack.ins,
% the derived file classpack.sty, and any ancillary files listed
% in the MANIFEST.
%
% \fi
% \iffalse
%<package>\NeedsTeXFormat{LaTeX2e}[2011/06/27]
%<package>\ProvidesPackage{classpack}[2013/05/28 v0.77
%<package> XML mastering for LaTeX classes and packages]
%<*driver>
\PassOptionsToPackage{svgnames}{xcolor}
\documentclass[12pt]{ltxdoc}
%%
%% Packages added automatically
%%
\let\SavedShow\show
\usepackage[utf8x]{inputenc}% was detected.
\AtBeginDocument{\let\show\SavedShow}
\DeclareUnicodeCharacter{9251}{\textvisiblespace}
\usepackage[T1]{fontenc}% was detected.
\usepackage{lmodern}% was detected.
\usepackage{mflogo}% was detected.
\usepackage{array}% tgroup was detected.
\usepackage{calc}% was detected.
\usepackage[inline]{enumitem}% orderedlist/@spacing='compact' was detected.
\setlist[description]{style=unboxed}
\usepackage{fancybox}% guibutton was detected.
\usepackage{fancyvrb}% programlisting was detected.
\usepackage{fmtcount}% xref/@linkend='varlistentry' was detected.
\usepackage[textwidth=150mm,textheight=8.5in]{geometry}% was detected.
\usepackage{listings}% programlisting was detected.
\lstdefinelanguage[DocBook]{XML} {morekeywords={abstract,address,affiliation,annotation,arg,author, book,chapter,classname,cmdsynopsis,command,constraintdef,contrib, copyright,cover,date,email,emphasis,envar,filename,firstname, footnote,guibutton,guilabel,guimenu,guimenuitem,guisubmenu, holder,info,itemizedlist,listitem,literal,member,option, orderedlist,orgdiv,orgname,package,para,parameter,part, personname,phrase,procedure,productname,programlisting,quote, refsection,remark,constructorsynopsis,methodparan,modifier, funcparams,olink,bibliography,biblioentry,biblioset,subtitle, artpagenums,volumenum,issuenum, releaseinfo,replaceable,revdescription,revhistory,revision, sect1,sect2,sect3,sect4,seg,seglistitem,segmentedlist,segtitle, simplelist,step,surname,systemitem,tag,term,title,uri,userinput, variablelist,varlistentry,wordasword,xref,year,xml:id,xlink:href},}
\lstdefinelanguage[TEI]{XML} {morekeywords={TEI.2,teiHeader,div2,head,p,num,val,xml:id}, }[keywords,comments,strings]
\lstloadlanguages{[LaTeX]TeX,[DocBook]XML,XSLT,bash,R}
\lstset{basicstyle=\small\color{Black}\ttfamily, literate= {Ö}{{\"O}}1 {Ä}{{\"A}}1 {Ü}{{\"U}}1 {ß}{{\ss}}2 {ü}{{\"u}}1 {á}{{\'a}}1 {é}{{\'e}}1 {í}{{\'i}}1 {ó}{{\'o}}1 {ú}{{\'u}}1 {ä}{{\"a}}1 {ö}{{\"o}}1, keywordstyle=\color{DarkGreen}\bfseries, commentstyle=\color{Gray}\upshape, stringstyle=\color{DarkRed}\upshape, emphstyle=\color{MediumBlue}\itshape, showstringspaces=false, columns=fullflexible, keepspaces=true}
\usepackage{makeidx}% was detected.
\makeindex
\usepackage{nicefrac}% was detected.
\def\textonehalf{\ensuremath{\nicefrac12}}
\usepackage{parskip}% was detected.
\usepackage{url}% uri was detected.
\usepackage{varioref}% xref was detected.
\vrefwarning
\labelformat{chapter}{Chapter~#1}
\makeatletter
\labelformat{chapter}{\@chapapp~#1}
\makeatother
\labelformat{section}{section~#1}
\labelformat{subsection}{section~#1}
\labelformat{subsubsection}{section~#1}
\labelformat{paragraph}{section~#1}
\labelformat{figure}{Figure~#1}
\labelformat{table}{Table~#1}
\labelformat{item}{item~#1}
\renewcommand{\reftextcurrent}{elsewhere on this page}
\def\reftextafter{on the \reftextvario{next}{following} page}
\usepackage[svgnames]{xcolor}% was detected.
\makeatletter
\@ifundefined{T}{\newcommand{\T}[2]{{\fontencoding{T1}\selectfont#2}}}{}
\makeatother
\usepackage{dox}% was detected.
\doxitem{Attribute}{attribute}{attributes}
\doxitem{AttributeValue}{attributevalue}{attribute values}
\doxitem{Class}{class}{classes}
\doxitem{Colour}{colour}{colours}
\doxitem{Counter}{counter}{counters}
\doxitem{DTD}{dtd}{DTDs/Schemas}
\doxitem{Element}{element}{element types}
\doxitem{Entity}{entity}{entities}
\doxitem{Error}{error}{errors}
\doxitem{Function}{function}{functions}
\doxitem[macrolike]{Length}{length}{lengths}
\doxitem{Mode}{mode}{mode}
\doxitem{Option}{option}{options}
\doxitem{Package}{package}{packages}
\doxitem{Template}{template}{templates}
\newcommand{\LabelFont}[2][\relax]{\strut {\fontencoding\encodingdefault \fontfamily{lmtt}\fontseries{lc}\selectfont#1#2}\space}
\makeatletter
\def\PrintDescribeAttribute#1{\LabelFont{@#1}}
\makeatother
\def\PrintDescribeAttributeValue#1{\LabelFont[\itshape]{#1}}
\def\PrintDescribeClass#1{\LabelFont{#1}}
\def\PrintDescribeColour#1{\LabelFont[\color{#1}]{#1}}
\def\PrintDescribeCounter#1{\LabelFont{#1}}
\def\PrintDescribeDTD#1{\LabelFont{#1}}
\def\PrintDescribeElement#1{\LabelFont{<#1>}}
\def\PrintDescribeEntity#1{\LabelFont{\&#1;}}
\def\PrintDescribeError#1{\LabelFont[\color{Red}!]{#1}}
\def\PrintDescribeFunction#1{\LabelFont{#1}}
\def\PrintDescribeLength#1{\LabelFont{#1}}
\def\PrintDescribeMode#1{\LabelFont[\sffamily]{\textlangle#1\textrangle}}
\def\PrintDescribeOption#1{\LabelFont[\itshape]{#1}}
\def\PrintDescribePackage#1{\LabelFont{#1}}
\def\PrintDescribeTemplate#1{\LabelFont[\fontfamily{lmss}\fontseries{sbc}\selectfont]{#1}}
\usepackage{apacite}
%%
%% Packages added by author
%%
\usepackage{dejavu}
\usepackage{textgreek}
\usepackage{classpack}
%%
%% Settings for docstrip and latexdoc 
%%
\EnableCrossrefs
\CodelineIndex
\RecordChanges
\newlength{\revmarg}
\setlength{\revmarg}{1in}
\newcommand{\vstrut}{\vrule height1.2em depth.6667ex width0pt}
\newcommand{\prestrut}{\vrule height1em width0pt}
\newcommand{\poststrut}{\vrule depth.5ex width0pt}
\begin{document}\raggedright\hyphenation{ele-ment attri-bute-value}
  \DocInput{classpack.dtx}
\end{document}
%</driver>
% \fi
%
% \CheckSum{45}
%
% \CharacterTable
%  {Upper-case    \A\B\C\D\E\F\G\H\I\J\K\L\M\N\O\P\Q\R\S\T\U\V\W\X\Y\Z
%   Lower-case    \a\b\c\d\e\f\g\h\i\j\k\l\m\n\o\p\q\r\s\t\u\v\w\x\y\z
%   Digits        \0\1\2\3\4\5\6\7\8\9
%   Exclamation   \!     Double quote  \"     Hash (number) \#
%   Dollar        \$     Percent       \%     Ampersand     \&
%   Acute accent  \'     Left paren    \(     Right paren   \)
%   Asterisk      \*     Plus          \+     Comma         \,
%   Minus         \-     Point         \.     Solidus       \/
%   Colon         \:     Semicolon     \;     Less than     \<
%   Equals        \=     Greater than  \>     Question mark \?
%   Commercial at \@     Left bracket  \[     Backslash     \\
%   Right bracket \]     Circumflex    \^     Underscore    \_
%   Grave accent  \`     Left brace    \{     Vertical bar  \|
%   Right brace   \}     Tilde         \~}
% 
% \changes{v0.77}{2013/05/28}{Removed unwanted definitions: classorpackage.}
% \changes{v0.76}{2013/05/05}{Modified documentation: Started working on Makefile.}
% \changes{v0.75}{2013/03/14}{Added secondary files: Secondary output files possible; reversed usage of role attribute on keywords;.}
% \changes{v0.74}{2013/02/21}{Added experimental autopackage: This implements automated package inclusion based on the markup used by the author..}
% \changes{v0.73}{2012/06/19}{Added readme.xml and db2plaintext.xsl: This implements dynamic README generation..}
% \changes{v0.72}{2012/02/11}{Wrote internal documentation: Created the classpack.xml template as an example..}
% \changes{v0.71}{2012/02/08}{First time this was used to document itself: The title element and subtitle element are now subsumed beneath the generated title in the output..}
%
% \GetFileInfo{classpack.dtx}
%
% \DoNotIndex{\@,\@@par,\@beginparpenalty,\@empty}
% \DoNotIndex{\@flushglue,\@gobble,\@input}
% \DoNotIndex{\@makefnmark,\@makeother,\@maketitle}
% \DoNotIndex{\@namedef,\@ne,\@spaces,\@tempa}
% \DoNotIndex{\@tempb,\@tempswafalse,\@tempswatrue}
% \DoNotIndex{\@thanks,\@thefnmark,\@topnum}
% \DoNotIndex{\@@,\@elt,\@forloop,\@fortmp,\@gtempa,\@totalleftmargin}
% \DoNotIndex{\",\/,\@ifundefined,\@nil,\@verbatim,\@vobeyspaces}
% \DoNotIndex{\|,\~,\ ,\active,\advance,\aftergroup,\begingroup,\bgroup}
% \DoNotIndex{\mathcal,\csname,\def,\documentstyle,\dospecials,\edef}
% \DoNotIndex{\egroup}
% \DoNotIndex{\else,\endcsname,\endgroup,\endinput,\endtrivlist}
% \DoNotIndex{\expandafter,\fi,\fnsymbol,\futurelet,\gdef,\global}
% \DoNotIndex{\hbox,\hss,\if,\if@inlabel,\if@tempswa,\if@twocolumn}
% \DoNotIndex{\ifcase}
% \DoNotIndex{\ifcat,\iffalse,\ifx,\ignorespaces,\index,\input,\item}
% \DoNotIndex{\jobname,\kern,\leavevmode,\leftskip,\let,\llap,\lower}
% \DoNotIndex{\m@ne,\next,\newpage,\nobreak,\noexpand,\nonfrenchspacing}
% \DoNotIndex{\obeylines,\or,\protect,\raggedleft,\rightskip,\rm,\sc}
% \DoNotIndex{\setbox,\setcounter,\small,\space,\string,\strut}
% \DoNotIndex{\strutbox}
% \DoNotIndex{\thefootnote,\thispagestyle,\topmargin,\trivlist,\tt}
% \DoNotIndex{\twocolumn,\typeout,\vss,\vtop,\xdef,\z@}
% \DoNotIndex{\,,\@bsphack,\@esphack,\@noligs,\@vobeyspaces,\@xverbatim}
% \DoNotIndex{\`,\catcode,\end,\escapechar,\frenchspacing,\glossary}
% \DoNotIndex{\hangindent,\hfil,\hfill,\hskip,\hspace,\ht,\it,\langle}
% \DoNotIndex{\leaders,\long,\makelabel,\marginpar,\markboth,\mathcode}
% \DoNotIndex{\mathsurround,\mbox,\newcount,\newdimen,\newskip}
% \DoNotIndex{\nopagebreak}
% \DoNotIndex{\parfillskip,\parindent,\parskip,\penalty,\raise,\rangle}
% \DoNotIndex{\section,\setlength,\TeX,\topsep,\underline,\unskip,\verb}
% \DoNotIndex{\vskip,\vspace,\widetilde,\\,\%,\@date,\@defpar}
% \DoNotIndex{\[,\{,\},\]}
% \DoNotIndex{\count@,\ifnum,\loop,\today,\uppercase,\uccode}
% \DoNotIndex{\baselineskip,\begin,\tw@}
% \DoNotIndex{\a,\b,\c,\d,\e,\f,\g,\h,\i,\j,\k,\l,\m,\n,\o,\p,\q}
% \DoNotIndex{\r,\s,\t,\u,\v,\w,\x,\y,\z,\A,\B,\C,\D,\E,\F,\G,\H}
% \DoNotIndex{\I,\J,\K,\L,\M,\N,\O,\P,\Q,\R,\S,\T,\U,\V,\W,\X,\Y,\Z}
% \DoNotIndex{\1,\2,\3,\4,\5,\6,\7,\8,\9,\0}
% \DoNotIndex{\!,\#,\$,\&,\',\(,\),\+,\.,\:,\;,\<,\=,\>,\?,\_}
% \DoNotIndex{\discretionary,\immediate,\makeatletter,\makeatother}
% \DoNotIndex{\meaning,\newenvironment,\par,\relax,\renewenvironment}
% \DoNotIndex{\repeat,\scriptsize,\selectfont,\the,\undefined}
% \DoNotIndex{\arabic,\do,\makeindex,\null,\number,\show,\write,\@ehc}
% \DoNotIndex{\@author,\@ehc,\@ifstar,\@sanitize,\@title,\everypar}
% \DoNotIndex{\if@minipage,\if@restonecol,\ifeof,\ifmmode}
% \DoNotIndex{\lccode,\newtoks,\onecolumn,\openin,\p@,\SelfDocumenting}
% \DoNotIndex{\settowidth,\@resetonecoltrue,\@resetonecolfalse,\bf}
% \DoNotIndex{\clearpage,\closein,\lowercase,\@inlabelfalse}
% \DoNotIndex{\selectfont,\mathcode,\newmathalphabet,\rmdefault}
% \DoNotIndex{\bfdefault,\DeclareRobustCommand}
% \DoNotIndex{\varname}
% \DoNotIndex{\section}
% \DoNotIndex{\subsection}
% \DoNotIndex{\raggedright}
% \DoNotIndex{\sffamily}
% \DoNotIndex{\usepackage}
% \DoNotIndex{\title}
% \DoNotIndex{\author}
% \DoNotIndex{\abstractname}
% \DoNotIndex{\changes}
% \DoNotIndex{\RequirePackage}
% \DoNotIndex{\LoadPackage}
% \DoNotIndex{\makeindex}
% \DoNotIndex{\makeatletter}
% \DoNotIndex{\makeatother}
% \DoNotIndex{\def}
% \DoNotIndex{\newwrite}
% \DoNotIndex{\classorpackage}
% \DoNotIndex{\@@doxdescribe}
% \DoNotIndex{\build}
% \DoNotIndex{\label}
% \DoNotIndex{\small}
% \DoNotIndex{\ttfamily}
% \DoNotIndex{\cite}
% \DoNotIndex{\citefield}
% \DoNotIndex{\url}
% \DoNotIndex{\ref}
% \DoNotIndex{\parskip}
% \DoNotIndex{\java}
% \DoNotIndex{\./build}
% \setcounter{secnumdepth}{5}
% \setcounter{tocdepth}{5}
% \setcounter{IndexColumns}{2}
% \setlength{\columnsep}{3pc}
%
% \def\fileversion{0.77}
% \def\filedate{2013/05/28}
% \title{The  \textsf{classpack} \LaTeXe\ package\thanks{%
% This document corresponds to \textsf{classpack}
% \textit{v.}\ \fileversion, dated \filedate.}\enspace\thanks{%
% Development has been supported by UCC Electronic Publishing Unit.}
% \\[1em]\Large 
% XML mastering for \LaTeX{} classes and packages}
% \author{Peter Flynn\\\normalsize Silmaril Consultants\\[-.25ex]\normalsize Textual Therapy Division\\\normalsize(\url{peter@silmaril.ie})}
% \maketitle
% \renewcommand{\abstractname}{Summary}\thispagestyle{empty}
% \begin{abstract}
% \parskip=0.5\baselineskip
% \advance\parskip by 0pt plus 2pt
% \parindent=0pt% \noindent
% \LaTeX{} document classes and packages are normally
% created, maintained, and distributed in
% {\ttfamily{}.dtx} format using the
% \textsf{ltxdoc} class, which provides facilities for
% modular or fragmentary coding combined with interleaved
% documentation. However, the accurate construction of these
% files is technically challenging.\par
% \emph{ClassPack} allows a developer to
% create an XML document containing user documentation and
% annotated code, based on the
% \emph{DocBook} vocabulary (with some minor
% abuses). An XSLT2 script then generates the
% {\ttfamily{}.dtx} and {\ttfamily{}.ins} files,
% ensuring that all the relevant pieces are emitted in the
% correct order and in the correct syntax.\par
% This is experimental software, and is incomplete. It has
% been successfully used in-house by the author since 2008 for
% several institutional and commercial packages and classes.
% There are some known deficiencies which remain to be
% corrected, and some legacy code (originally included for one
% specific package) which needs to be removed to an external
% file.\par
% A paper describing the system has been accepted for the
% Balisage markup conference 2013 in Montréal.\par
% \end{abstract}
% \clearpage
% \tableofcontents
% \clearpage
% \section{Introduction}
% \LaTeX{} document classes (templates) and packages (styles)
% are traditionally distributed as pairs of
% {\ttfamily{}.dtx} and{\ttfamily{}.ins} files,
% written to the specifications and recommendations of the
% \textsf{doc}, \textsf{ltxdoc}, and
% \textsf{clsguide} packages.\par
% \begin{itemize}
% \item The {\ttfamily{}.dtx} (Doc\TeX{}) file is a 
%     literate-programming document, containing modular code and
%     annotations interleaved in such a way that each fragment
%     of code and its explanation are adjacent;
% \item The {\ttfamily{}.ins} file is an installer:
%     when run through \LaTeX{}, it extracts the code from the
%     {\ttfamily{}.dtx} file into the relevant class
%     ({\ttfamily{}.cls}), package
%     ({\ttfamily{}.sty}), and other files;
% \item Running \LaTeX{} on the {\ttfamily{}.dtx} file
%     itself extracts and typesets the documentation.
% \end{itemize}
% The construction of a {\ttfamily{}.dtx} document
% is quite complex, with a special set of tags and conventions
% to allow documentation to be separately identifiable to code.
% The file format relies on the documentation being shielded by
% a leading percent-space (\verb|%␣|) armour
% on each line to prevent it being interpreted as part of the
% code; and the environment tags surrounding the code itself
% must be shielded by four such spaces
% (\verb|%␣␣␣␣|). Apart from the
% documentation, the treatment of the code and control
% statements resembles more a data specification (which in some
% ways it is) than a conventional text document.\par
% XML, particularly in its traditional
% `document' mode, as distinct from its
% use as a data exchange format, offers many similar features to
% \LaTeX{} (for example, the named identification of document
% components), but with a rigid and invariable syntax that can
% be checked programmatically by any validating XML processor.
% By contrast, a \LaTeX{} document (and more specifically, a
% {\ttfamily{}.dtx} document) can only be proved by
% running it through \LaTeX{} itself: there is no equivalent to
% the `pre-flight' type of standalone parsing or
% validating available with XML.\par
% The \emph{DocBook} vocabulary of XML
% is designed for technical documentation in computing. It
% provides markup both for textual documentation
% \emph{and} for data-like structures that occur
% in computer documentation, making it a viable candidate for
% describing a literate-programming type of document such as
% Doc\TeX{}.\par
% The \emph{ClassPack} system is an
% experiment in using \emph{DocBook} XML as
% the storage format for \LaTeX{} class and package source code,
% using the XLST2 language to transform the XML into pairs of
% {\ttfamily{}.dtx} and {\ttfamily{}.ins} files.
% There are a number of advantages to this approach:\par
% \begin{itemize}
% \item XML's syntax and document construction is extremely
%     robust, and the design of the language means that an XML
%     file can be machine-checked for errors of syntax and
%     construction;
% \item XML markup is traditionally self-descriptive, with
%     element types being named according to what they are
%     intended to contain. For example, a variable name can be
%     marked up as
%     \verb|<varname>foo</varname>|.\par
% While it is perfectly possible to create a
%     {\ttfamily{}\textbackslash{}varname} control sequence (macro) to do
%     the same in \LaTeX{}, it is rarely done. Instead, authors
%     have typically preferred to use visual formatting like
%     \verb|\textit{foo}| for italics or
%     \verb|\verb+foo+| for monospace type.
%     This method means the variable reference is not
%     immediately identifiable as containing a variable
%     name~--- it could be anything;
% \item Given suitably-descriptive markup, sharing document
%     fragments between applications can be done
%     programmatically, so a fragment implementing a \LaTeX{}
%     feature (with its associated documentation) can be re-used
%     in other class and package applications at the XML level
%     (eg with XInclude, or as an external entity) without the
%     need for manual cutting and pasting;
% \item \label{xmltools}There is a very wide range of software (editors and
%     processors, both free and non-free) available to handle
%     XML documents, including a lot of useful tools for
%     document management and information extraction.
% \end{itemize}
% Everything comes at a cost. The drawbacks of using XML for
% this include:\par
% \begin{itemize}
% \item It's another language to learn. Despite being so
%     widespread, it's not yet a common skill. In particular,
%     programmers dislike XML because it's a markup language,
%     not a programming language, and the syntax is different
%     from that of programming languages;
% \item Although there is plenty of software for editing XML,
%     it is not well-developed for text documents in synchronous
%     typographic form (often called
%     `WYSIWYG'); even the best or most
%     expensive editors are designed for XML experts, not for
%     the average user.
% \end{itemize}
% \subsection{Contents}
% The \emph{ClassPack} framework is
%   enabled by two principal software components:\par
% \begin{enumerate}
% \item An XML vocabulary (the
%       \emph{DocBook}~{\small DTD}\index{DTD} 
%       or Schema) used for naming the component parts of
%       documentation and code, and specifying where they belong
%       and how they fit together;
% \item An XSLT2 script to implement the logic of
%       combination and separation needed to create the
%       {\ttfamily{}.dtx} and {\ttfamily{}.ins}
%       files.
% \end{enumerate}
% The XML vocabulary used is
%   \emph{DocBook}, which is in widespread
%   use for the documentation of computer systems, and is
%   well-supported on all platforms. The current system uses
%   version 5. Although it is highly modular and easily adapted
%   for many purposes, only a few minor changes have been made
%   for its use here, but a number of element types have been
%   put to uses not envisaged by the developers.\par
% This habit of using (some say, abusing) XML markup for
%   different purposes is very common, and often deprecated
%   because it is usually undocumented. This document explains
%   what has been [ab]used and for what reasons. Once this
%   system settles down, a more formal expression of the
%   vocabulary can be made from the RNG source, removing the
%   parts that are not required, and making it simpler to edit
%   with.\par
% The adaptation of \emph{DocBook} in
%   this version subsists mainly in the addition of some
%   attributes and entity declarations to allow the conversion
%   to \LaTeX{} of special characters and other features that
%   would otherwise involve extensive re-parsing of the
%   character data content (text). The changes also implement a
%   few modifications to the way the \LaTeX{} code is output by
%   the XSLT2 program for typographical purposes.\par
% As an example, the entity defining the em rule character
%   \texttt{\&mdash;} is declared as a \LaTeX{}
%   tie (non-breaking space) followed by an em rule followed by
%   a normal space (\verb|~---␣|), so that it
%   can be used between words~--- like this~--- without the
%   need to worry about special spacing in the XML, or the
%   inadvertent breaking of a line before the rule:\par
% \iffalse
%<*ignore>
% \fi
\begin{lstlisting}[language=XML]
can be used between words&mdash;like this&mdash;without the
\end{lstlisting}
% \iffalse
%</ignore>
% \fi
% If a
%   document style requires the use of unspaced dashes, all that
%   needs changing is the entity declaration, not the whole
%   document.\par
% The current driver in Document Type
%     Definition~(\textsc{dtd})\index{Document Type Definition|see{DTD}}\index{DTD|textbf} format is listed in \vref{savedtd}, and is distributed as file
%   {\ttfamily{}doctexbook.dtd} so that it can be
%   referenced as such in a document without having to copy and
%   paste the declarations into every document.\par
% \subsection{Invocation}\label{invocation}
% To create or edit a \emph{ClassPack}
%   XML document you must have the
%   \emph{ClassPack} DTD
%   \emph{and} the DocBook DTD installed and known
%   to your XML editor. The DTD customisation file,
%   {\ttfamily{}docbooktex.dtd} is distributed with this
%   package; an RNG schema version will be available in the
%   future.\par
% With the DTD, the standard procedure is to specify it in
%   the first line of your XML document. Your class or package
%   documents must therefore start with a Document Type
%   Declaration. This can specify the Formal
%     Public Identifier~(\textsc{fpi})\index{Formal Public Identifier|see{FPI}}\index{FPI|textbf} and the filename of the
%   DTD:\par
% \iffalse
%<*ignore>
% \fi
\begin{lstlisting}[language=XML]
<?xml version="1.0"?>
<!DOCTYPE book PUBLIC 
  "+//Silmaril//DTD DocBook 5.0 for DocTeX//EN" 
  "doctexbook.dtd">
\end{lstlisting}
% \iffalse
%</ignore>
% \fi
% or
%   you can omit the {\small FPI}\index{FPI} and specify the DTD as a \texttt{SYSTEM} keyword instead:\par
% \iffalse
%<*ignore>
% \fi
\begin{lstlisting}[language=XML]
<?xml version="1.0"?>
<!DOCTYPE book SYSTEM "doctexbook.dtd">
\end{lstlisting}
% \iffalse
%</ignore>
% \fi
% The DTD file {\ttfamily{}doctexbook.dtd} can be
%   anywhere on your system: in these examples it is assumed to
%   be in the same directory as your document. If you store it
%   elsewhere, just give the full filepath, for example:\par
% \iffalse
%<*ignore>
% \fi
\begin{lstlisting}[language=XML]
<?xml version="1.0"?>
<!DOCTYPE book SYSTEM "/usr/local/lib/xml/dtds/doctexbook.dtd">
\end{lstlisting}
% \iffalse
%</ignore>
% \fi
% When you open a document starting like this, a
%   conformant XML editor will look for
%   {\ttfamily{}doctexbook.dtd} and read it, so that it
%   then knows the names of all the element types that you can
%   use, and how they fit together into a document.\par
% \subsection{Document structure}
% Every XML document must have one outermost enclosing
%   element (the `root' element) which
%   holds everything else. The root element type used in a
%   ClassPack document is \texttt{book}, as described in
%   detail in \vref{setup}.\par
% Within the \texttt{book} element, the
%   \emph{metadata} (information about the
%   document) is held in an \texttt{info} element, the
%   \emph{user documentation} in a \texttt{part}
%   element with the ID of \texttt{doc}, and the
%   \emph{annotated code} in a second
%   \texttt{part} element, with the ID of
%   \texttt{code}.\par
% \iffalse
%<*ignore>
% \fi
\begin{lstlisting}[language={[DocBook]XML}]
<?xml version="1.0"?>
<!DOCTYPE book SYSTEM "doctexbook.dtd">
<book>
  <info>...</info>
  <part xml:id="doc">...</part>
  <part xml:id="code">...</part>
  <part xml:id="files">...</part>
</book>
\end{lstlisting}
% \iffalse
%</ignore>
% \fi
% An optional third \texttt{part} with the ID of
%   \texttt{files} can be used to include ancillary files that
%   are to be recreated as-is, without separate annotation or
%   display, such as sample data or example documents (see \vref{ancfiles}).\par
% Ancillary files that require annotating must go in the
%   \texttt{code} part as described in
%   \vref{secfiles}.\par
% \subsubsection{User documentation}
% The documentation for the end user is descriptive text
%     which explains how to use the package or class. The main
%     division within a ClassPack \texttt{part} is the
%     \texttt{chapter}, which can contain \texttt{sect1}
%     sections, which can contain \texttt{sect2} subsections,
%     and so on, as described in
%     \vref{hierarchy}. User documentation must go in
%     a \texttt{chapter} and its sub-elements: it must not be
%     directly in the \texttt{part}.\par
% (Because the \textsf{latexdoc} package is
%     based on the \textsf{article} package, the
%     ClassPack \texttt{chapter}\thinspace{}s become
%     \textsf{latexdoc}~{\ttfamily{}\textbackslash{}section}s; 
%     the ClassPack \texttt{sect1}\thinspace{}s become
%     \textsf{latexdoc}~{\ttfamily{}\textbackslash{}subsection}s, 
%     and so on.)\par
% \paragraph{Structure}
% Your document can use most of the conventional
%       structural features of any DocBook document: paragraphs,
%       lists, figures, tables, and examples of code (both
%       illustrative and for extraction) as described in \vref{pool}. Most of the specialist constructs of
%       DocBook are not implemented in ClassPack except for a
%       few used in the \texttt{info} section for setup
%       purposes. You should refer to the list of markup in
%       \vref{pool} for exactly which element types
%       do what.\par
% \paragraph{Inline markup}
% Within the text you can use much of the conventional
%       semantic markup provided in
%       \emph{DocBook} as described in \vref{flow}.\par
% \subsubsection{Annotated code}
% In the \texttt{code} Part, you annotate the inner
%     workings of your package or class. The use of the
%     \texttt{chapter} and \texttt{sect1} divisional
%     structure is also mandated here: everything within the
%     chapters is regarded as the package or class.\par
% Your code can be explained in fragments, either as
%     alternating \texttt{para} and \texttt{programlisting}
%     elements, or as \texttt{annotation} elements, each one
%     describing a single macro or environment or other object,
%     and containing \texttt{para} and
%     \texttt{programlisting} (this makes it indexable). See
%     the examples in \vref{pool}.\par
% \subsubsection{Ancillary files}\label{ancfiles}
% There are two types of extractable ancillary
%   file:\par
% \begin{itemize}
% \item Ancillary files to be annotated and extracted
% along with the class or package file must each go in
% their own \texttt{appendix} element immediately
% after the last of the \texttt{chapter} element of
% the package or class in the \texttt{code}
% Part
% \item Ancillary files which just need to be extracted
% whole, and do not have any separate documentation must
% go in the \texttt{files} Part as described in \vref{ancfiles}.
% \end{itemize}
% \subsection{Tooling up}
% You need the following tools:\par
% \begin{itemize}
% \item an XML editor: I use
%       \emph{Emacs} with
%       \emph{psgml-mode} and
%       \emph{xxml-mode}, but any competent
%       XML editor will do
% \item an XSLT2 processor: I use
%       \emph{Saxon} (this also means I
%       installed \emph{Java})
% \item a full installation of \LaTeX{}
% \item a PDF reader
% \item XML tools: I find the
%       \emph{LTxml2} toolkit from Edinburgh
%       University invaluable for ad-hoc querying of
%       documents.
% \end{itemize}
% \clearpage
% \section{Setting up a ClassPack document}
% Using \emph{ClassPack} to create and
% maintain \LaTeX{} classes and packages requires the following
% initial steps. These are only done once, at the start of a new
% class or package. A few items need periodic updating, such as
% the version number, when it changes; the revision history; and
% the list of packages, as and when needed.\par
% \begin{enumerate}
% \item setting up the configuration (see \vref{setup})
%   to specify the name, date, version, type, audience, status,
%   and other key metadata;
% \item setting up the documentary metadata (see \vref{metadata}) such as the title, author, contact
%     details, abstract, and the initial entry in the revision
%     history;
% \item setting up the list of packages (see \vref{constraints}) required for 
%      \begin{enumerate*}[label=\emph{\alph*})]
% \item the documentation (see \vref{docpax})
% and 
% \item the class or package itself (see \vref{classtypax})
% \end{enumerate*}, plus any additional initialization
%     commands needed for the documentation.
% \end{enumerate}
% \emph{Note particularly that the list of packages
%   required by the class or package itself is
%   \emph{not} stored inline to the code of the
%   class or package. It is stored separately for reasons that
%   are explained in detail in \vref{constraints}.}\par
% \subsection{Configuration and setup}\label{setup}
% The \texttt{book} element is the outermost container
%   for the document. It is used to carry the configuration
%   information in attributes, for example:\par
% \iffalse
%<*ignore>
% \fi
\begin{lstlisting}[language={[DocBook]XML}]
<book xml:id="classpack" arch="class" version="0" revision="71"
  status="beta" conformance="LaTeX2e" condition="2011-06-27" 
  os="all" audience="lppl" security="0" vendor="Silmaril" 
  xml:base="tex/latex" xlink:role="xxx" userlevel="cls"
  annotations="\raggedright" remap="a4paper,12pt">
\end{lstlisting}
% \iffalse
%</ignore>
% \fi
% \begin{description}[style=unboxed]
% \item[\texttt{xml:id}\thinspace:]\label{rootid}This \textsc{must} be the
% name of the class or package. It will be used as the
% \LaTeX{} filename (with {\ttfamily{}.cls} or
% {\ttfamily{}.sty} and
% {\ttfamily{}.ins} added automatically), so it
% \textsc{should} be all lowercase
% and \textsc{must} consist of
% letters, digits, and hyphens only, starting with a
% letter.\par
% The XML rules for IDs require this restriction,
% which currently makes it impossible to use ClassPack
% to maintain a class or packages whose name begins with
% a digit.
% \item[\texttt{arch}\thinspace:]\label{arch}The `architecture' of the
% document, which defines the type of file you are going
% to produce; this must be either \texttt{"class"} or \texttt{"package"}, corresponding
% exactly with the value of the \texttt{userlevel}
% attribute below.
% \item[\texttt{userlevel}\thinspace:]\label{userlevel}The file type of the class or package document;
% this must be either {\ttfamily{}cls} or
% {\ttfamily{}sty}, corresponding
% exactly with the value of the \texttt{arch}
% attribute above.
% \item[\texttt{version}\thinspace:]The major version of the class or package.
% Conventionally, development (α) or pre-release
% (β) versions of software start at version
% zero.
% \item[\texttt{revision}\thinspace:]\label{revno}The sub-version or release of your class or
% package. This is combined with the major version
% number, separated by a dot, to produce the complete
% version number.\par
% The most recent revision history entry gets tested
% against this when the file is processed, and an error
% message is displayed if the version numbers do not
% match, as a warning that you have updated one without
% updating the other, and may therefore have forgotten
% to document a change (see the \texttt{revision}
% element in
% `\textbf{\texttt{revhistory}}', the last item in the list in \vref{revhist}).
% \item[\texttt{status}\thinspace:]The development status of the class or package, eg
% \texttt{"alpha"}, \texttt{"beta"}, \texttt{"candidate"}, \texttt{"draft"}, \texttt{"final"}, etc.
% \item[\texttt{conformance}\thinspace:]The \TeX{} format required to process the
% {\ttfamily{}.dtx} document. Only the value
% \texttt{"LaTeX2e"} is supported at
% the moment.
% \item[\texttt{condition}\thinspace:]The version of the format identified in \texttt{conformance}, expressed as an
% ISO date (\texttt{"yyyy-mm-dd"}).\par
% Note that this is \emph{not} in the
% \LaTeX{} format
% ({\ttfamily{}yyyy/mm/dd}).
% \item[\texttt{os}\thinspace:]The operating system[s] for which the class or
% package is relevant. Currently, only the value \texttt{"all"} is supported.
% \item[\texttt{audience}\thinspace:]The licence under which the class or package is
% made available. For normal publicly-available \LaTeX{}
% classes or packages which will be uploaded to CTAN,
% use the value \texttt{"lppl"}
% (\LaTeX{} Project Public Licence).\par
% A copy of the LPPL is distributed with ClassPack
% in a file called {\ttfamily{}lppl.xml}, which
% must be copied or soft-linked (aliased) to each
% directory in which you process ClassPack
% documents.\par
% Classes or packages for private or commercial use
% will probably need to use another value, but it is
% used as a filename, so a {\ttfamily{}.xml}
% file with that name must exist in the directory in
% which your ClassPack document is processed. It must
% contain a
% \emph{DocBook}~\texttt{chapter} 
% element containing the text of the licence, which will
% be included at the end of the documentation.
% \item[\texttt{security}\thinspace:]The checksum value emitted by
% \textsf{ltxdoc} when it processes your class
% or package to format your documentation. See the
% documentation for the \textsf{ltxdoc} package for details.\par
% Setting this to zero avoids the \LaTeX{} error
% message during early development, when every edit
% would change the checksum. As with the version values,
% you must update this value, if non-zero, to match the
% one that \textsf{ltxdoc} reports.
% \item[\texttt{vendor}\thinspace:]Your name, or the name of the organisation
% responsible for the work on this class or package.
% \item[\texttt{remap}\thinspace:]Any options to pass to the
% \textsf{ltxdoc} package, such as
% \textbf{\texttt{a4paper}}, \textbf{\texttt{12pt}},
% etc.\par
% This, and the following \texttt{annotations}
% attribute, make it easier to change the global
% formatting of the documentation.
% \item[\texttt{annotations}\thinspace:]Any \LaTeX{} commands required for global
% application at the start of the documentation that
% cannot easily be included anywhere else (eg
% {\ttfamily{}\textbackslash{}raggedright},
% {\ttfamily{}\textbackslash{}sffamily}, etc).
% \item[\texttt{xml:base}\thinspace:]The name of the \emph{subdirectory}
% where the resulting {\ttfamily{}.cls} or
% {\ttfamily{}.sty} file should be installed in
% a TDS-compliant \TeX{} installation, relative to the
% {\ttfamily{}texmf/tex/latex} directory of the
% tree. For most packages, this means the name of the
% directory you want created by the installation
% {\ttfamily{}.tds.zip} file, in which the
% {\ttfamily{}.cls} or {\ttfamily{}.sty}
% file will be put.
% \item[\texttt{xlink:role}\thinspace:]\label{selfspec}\textsc{Optional}. If this
% is set to a value, then the package being written will
% be included in the setup for the documentation
% (perhaps so that it can be used in examples), with the
% value of this attribute being used as the optional
% argument to the generated
% {\ttfamily{}\textbackslash{}usepackage} command.\par
% If the package is required with no options, use
% this attribute but set it to null (\texttt{""}). The date constraint is
% added automatically, using the current package date as
% defined by the most recent entry in the
% \texttt{revhistory}.\par
% Note that this only works for packages, not
% classes. Classes cannot not be documented using
% themselves.
% \end{description}
% It is essential to get these values correct, otherwise
%   subsequent processing will produce unexpected
%   results, or no results at all.\par
% \subsection{Metadata}\label{metadata}
% The titling, specification of packages (separately for
%   the documentation and for the class or package itself),
%   revision history, abstract, copyright, and availability are
%   all kept at the top of the document in an element called
%   \texttt{info}, immediately after the \texttt{book} start-tag:\par
% \iffalse
%<*ignore>
% \fi
\begin{lstlisting}[language={[DocBook]XML}]
<info>
  <cover>...</cover>
  <!-- THE METADATA STARTS HERE -->
  <title>XML mastering for &amp;LaTeX; document classes...
  <author>...
  <copyright>...
  <releaseinfo>http://latex.silmaril.ie/software</releaseinfo>
  <annotation>...
  <abstract>...
  <revhistory>...
</info>
\end{lstlisting}
% \iffalse
%</ignore>
% \fi
% Of these, the \texttt{cover} element is the most
%   complex, as it is [ab]used to hold all the details of
%   packages and settings required for the class or package
%   \emph{and} for the production of the
%   documentation. It therefore gets the whole of
%   \vref{constraints} to itself.\par
% The other metadata element types are more obvious, and
%   largely use the \emph{DocBook} markup as
%   intended.\par
% \begin{description}[style=unboxed]
% \item[\texttt{title}\thinspace:]This contains the title of the class or package in
% natural language. This is \emph{not} the
% content of the {\ttfamily{}\textbackslash{}title} command in
% \LaTeX{} terms, in the {\ttfamily{}.dtx} file,
% which is automatically preset to the phrase
% `The \textsf{{\itshape name}}
%   \LaTeXe{} document class' (or
% `package'; where
% {\itshape name} is the name of your
% class or package as specified in the \texttt{xml:id} attribute of the
% \texttt{book} root element).\par
% The title you give in
% \emph{this}~\texttt{title}
% element is the explanatory subtitle which appears
% \emph{under} the automatically-generated
% one on the first page of the typeset
% documentation.
% \item[\texttt{author}\thinspace:]The \texttt{author} element provides identity
% markup for the author[s], using a
% \texttt{personname} element for each author,
% containing subelements for firstname, surname, and
% other forms of naming; optionally followed by an
% \texttt{affiliation} element, where you can identify
% employer or other status; address; email; and URI, as
% in the following example. The name, affiliation, and
% email address are used in the
% {\ttfamily{}\textbackslash{}author} command of the
% {\ttfamily{}.dtx} file.\par
% \iffalse
%<*ignore>
% \fi
\begin{lstlisting}[language={[DocBook]XML}]
<author role="maintainer">
  <personname>
    <firstname>Peter</firstname> 
    <surname>Flynn</surname>
  </personname>
  <affiliation>
    <orgname>Silmaril Consultants</orgname>
    <orgdiv>Textual Therapy Division</orgdiv>
  </affiliation>
  <address>Cork, Ireland</address>
  <email>peter@silmaril.ie</email>
  <uri>http://blogs.silmaril.ie/peter</uri>
  <contrib role="sponsor">UCC</contrib>
</author>
\end{lstlisting}
% \iffalse
%</ignore>
% \fi
% For multiple authors, you must enclose multiple
% \texttt{author} elements (one per author) in an
% outer \texttt{authorgroup} container element. One of
% the authors must be identified as the maintainer of
% the package or class, by adding the \texttt{role}
% attribute with the value
% {\ttfamily{}maintainer}.\par
% A \texttt{contrib} element with a \texttt{role} attribute set to \texttt{"sponsor"} can be added in an
% \texttt{author} block, to give the
% name of an organisation sponsoring the development of
% the class or package.
% \item[\texttt{copyright}\thinspace:]This element lets you identify the \texttt{year}
% and the name of the \texttt{holder} (you, your
% employer, or some other entity).
% \iffalse
%<*ignore>
% \fi
\begin{lstlisting}[language={[DocBook]XML}]
<copyright>
  <year>2012</year>
  <holder>Silmaril Consultants</holder>
</copyright>
\end{lstlisting}
% \iffalse
%</ignore>
% \fi
% \item[\texttt{releaseinfo}\thinspace:]In the absence of other ways of identifying where
% to find your class or package (assuming it will
% eventually find its way onto CTAN), this element can
% be used to hold the URI of a location where it can be
% downloaded, such as your personal or business web
% site.
% \item[\texttt{annotation}\thinspace:]This element is used for a warning or notice you
% want placed in the Preamble of the
% {\ttfamily{}.ins} file, which is used for
% extracting your class or package from the
% {\ttfamily{}.dtx} file, where it will be seen
% by users installing the software.
% \item[\texttt{abstract}\thinspace:]The Abstract is formatted on the front page of
%       your documentation. Like any abstract, it should
%       summarise what the class or package does, and who might
%       want to use it.\par
% The \texttt{abstract} element may start with an
% optional \texttt{title} element, which (if present)
% will be used to change the value of the
% {\ttfamily{}\textbackslash{}abstractname} in the
% {\ttfamily{}.dtx} file
% (`Summary' is a common
% choice).\par
% The rest of the abstract is just paragraphs; note
% that lists, block quotations, figures, tables, etc are
% not allowed in an Abstract.
% \item[\texttt{revhistory}\thinspace:]\label{revhist}This holds the top-level information about each
% major and minor revision, outlining the main changes.
% The version number of the most recent revision, as
% identified by the latest value of the \texttt{conformance} attribute of the
% \texttt{date} element, must match the version number
% composed from the major version and the revision in
% the \texttt{book} root element (see `\textbf{\texttt{revision}}', the \ordinalstringnum{5} item in the list in \vref{revno}).\par
% \iffalse
%<*ignore>
% \fi
\begin{lstlisting}[language={[DocBook]XML}]
<revision version="0.72">
  <date conformance="2012-02-11"/>
  <revdescription>
    <itemizedlist>
      <title>Wrote internal documentation</title>
      <listitem>
        <para>Created the classpack.xml template 
                 as an example.</para>
      </listitem>
    </itemizedlist>
  </revdescription>
</revision>
\end{lstlisting}
% \iffalse
%</ignore>
% \fi
% Comments on individual changes to the code should
% be documented \emph{at the code
%   location}, using the \texttt{remark}
% element (see \vref{maintenance}), eg\par
% \iffalse
%<*ignore>
% \fi
\begin{lstlisting}[language={[DocBook]XML}]
<remark version="0.70" revision="2010-05-29">Added 
    timestamp</remark>
\end{lstlisting}
% \iffalse
%</ignore>
% \fi
% These \texttt{remark} elements get collated as
% {\ttfamily{}\textbackslash{}changes} commands, and gathered
% together in the changelog by \textsf{ltxdoc}
% during processing.
% \end{description}
% \subsection{Packages and related commands}\label{constraints}
% As mentioned above, the \texttt{cover} element is used
%   to provide a place where packages and other \LaTeX{}
%   preliminaries can be specified. Using this structure means
%   each entry is separately editable, and the same structure is
%   used both for packages for the documentation
%   \emph{and} packages for the class or package
%   itself.\par
% It would of course have been possible just to allow a
%   slab of \LaTeX{} code at these points, but that would have made
%   commenting and documentation harder, and would also have
%   made it more difficult to perform an XML
%   element-copy--element-paste or an XInclude when using
%   one package or class's settings as the basis for another.\par
% The most important reason is that specifying package
%   lists as separately-identifiable blocks makes it possible to
%   automate the invocation of frequently-used packages,
%   parameters, options, or settings which you may store
%   separately for re-use (see \vref{autospec}) by
%   adding your own modifications that you like to have included
%   whenever you use a particular package.\par
% In particular, the
%   \emph{autopackage} feature added to
%   ClassPack in v0.74 means that most packages needed for
%   documentation are now detected automatically on the basis of
%   features you use in your documentation, making it
%   unnecessary to specify them by hand. For example, if you use
%   compact lists, ClassPack will detect this and add the
%   \textsf{enumitem} package for you.\par
% The XSLT2 program also uses this markup in order to
%   modify the behaviour of the \LaTeX{} code at several points
%   (such as fixing the broken abstract formatting when the
%   \textsf{parskip} package is used).\par
% There are at least two, possibly four, sections in the
%   \texttt{cover} element where the packages, commands, and
%   other data can be defined:\par
% \iffalse
%<*ignore>
% \fi
\begin{lstlisting}[language={[DocBook]XML}]
<constraintdef xml:id="docpackages">
 ...packages for the user documentation are defined here...
</constraintdef>

<constraintdef xml:id="startdoc">
 ...special commands for the user documentation go here...
</constraintdef>

<constraintdef xml:id="clspackages" linkend="options">
 ...packages needed for the class or package are defined here...
</constraintdef>

<constraintdef xml:id="manifest">
 ...files to add to the MANIFEST/zip file are listed here...
</constraintdef>
\end{lstlisting}
% \iffalse
%</ignore>
% \fi
% These exact \texttt{xml:id} values
%   are mandatory when the relevant \texttt{constraintdef}
%   elements are used, as they are referenced from elsewhere in
%   the document by the XSLT2 program. The only variation is
%   that when writing a package ({\ttfamily{}.sty}
%   file), the \texttt{"clspackages"} must read
%   \texttt{"stypackages"} instead. The first
%   three letters are used to match the three-letter filetype
%   used as the value of the \texttt{arch}
%   attribute that you specified on the \texttt{book} root
%   element (see `\textbf{\texttt{arch}}', the \ordinalstringnum{2} item in the list in \vref{arch}).\par
% Each \texttt{constraintdef} element can
%   hold one or (in some circumstances, more) of the following
%   element types:\par
% \begin{itemize}
% \item \texttt{segmentedlist} (in a
%       \texttt{docpackages}, \texttt{clspackages}, or
%       \texttt{stypackages} type of \texttt{constraintdef}\emph{only}), a list structure used to
%       specify the packages required: see \vref{packspec};
% \item \label{cmdsynopsis}\texttt{cmdsynopsis}, used for defining
%       \emph{user documentation} setup commands
%       to be placed \emph{in} the Preamble. This
%       is only meaningful in the \texttt{"docpackages"} type of
%       \texttt{constraintdef}: see \vref{cmdspec};
% \item \texttt{procedure}, used for holding blocks of
%       \emph{user documentation} setup commands
%       to be placed \emph{after} the Preamble
%       (that is, at the start of the document body, after the
%       \verb|\begin{document}| command). This is
%       only meaningful in the \texttt{"docpackages"} type of
%       \texttt{constraintdef}.\par
% Note that this is distinct from commands to be
%       placed \emph{in} the Preamble, which are
%       held in a more structured manner in the
%       \texttt{cmdsynopsis} element deacribed above: see
%       \vref{addcmds};
% \item \texttt{simplelist}, a list whose
%       \texttt{member} elements are used to name additional
%       files to be included in the distribution zip file
%       ({\ttfamily{}MANIFEST}). This is only meaningful
%       in a \texttt{manifest} type of
%       \texttt{constraintdef}: see
%       \vref{manifestfiles}.
% \end{itemize}
% \subsubsection{Specifying packages}\label{packspec}
% There are three parts to using
%     \texttt{constraintdef} for this:\par
% \begin{enumerate}[noitemsep]
% \item specifying packages for your user documentation (see
% \vref{docpax});
% \item specifying packages for the class or package you
% are writing (see \vref{classtypax});
% \item automating the inclusion of extra settings to be
% used whenever you specify a particular package (see
% \vref{autospec}).
% \end{enumerate}
% For the first two, the \texttt{segmentedlist}
%     element is used. This contains a sequence of
%     \texttt{seglistitem}\thinspace{}s, one per package, each containing a
%     \texttt{seg} element holding the package name. An
%     optional \texttt{segtitle} element may start the list,
%     and if present, is used as a comment (for the
%     documentation packages) or a subheading (class or package
%     packages).\par
% \iffalse
%<*ignore>
% \fi
\begin{lstlisting}[language={[DocBook]XML}]
<segmentedlist>
  <segtitle>Packages required for documentation</segtitle>
  <seglistitem role="Use the Charter typeface for documentation."> 
    <seg version="2005-04-12">charter</seg>
  </seglistitem>
  <seglistitem role="Use Helvetica as the sans-serif, but scale it
     to fit">
    <seg role="scaled=0.8333">helvet</seg>
  </seglistitem>
    ...
</segmentedlist>
\end{lstlisting}
% \iffalse
%</ignore>
% \fi
% Each \texttt{seglistitem} provides for
%     a documentary comment about why this package is required,
%     using the \texttt{role}
%     attribute. In the case of the packages for your own class
%     or package, this comment is reproduced in the
%     documentation of the code.\par
% The package itself is specified as the content of the
%     \texttt{seg} element in each item.\par
% Any options for the package being loaded must be
%     supplied in the \texttt{role} attribute
%     of the \texttt{seg} element. If the
%     package must conform to a specific version, the date must
%     be provided (in ISO format) in the \texttt{version} attribute.\par
% \paragraph{Declaring packages needed for your documentation}\label{docpax}
% Packages required for your documentation must be
%       included in the type of list described above, in the
%       \texttt{constraintdef} element that has the \texttt{xml:id} value of \texttt{"docpackages"}. Note that some
%       packages are automatically when certain types of
%       formatting are implied: see \vref{autopackage} for details.\par
% The relevant {\ttfamily{}\textbackslash{}usepackage} commands
%       get included in the {\ttfamily{}.dtx} file right
%       after the \verb|\begin{document}|
%       command.\par
% If you also want the package you are maintaining to
%       be included in the documentation (perhaps so you can use
%       it for examples), remember to set the \texttt{xlink:role} attribute on the
%       \texttt{book} root element as described in `\textbf{\texttt{xlink:role}}', the last item in the list in \vref{selfspec}.\par
% See \vref{autospec} for details of how to
%       specify package command settings that you want included
%       by default every time you specify a particular
%       package.\par
% \paragraph{Declaring packages needed by the class or package
%       itself}\label{classtypax}
% All the packages required for the class or package
%       being written must be included in the type of list
%       described above, in the \texttt{constraintdef} element
%       that has the \texttt{xml:id} value
%       of \texttt{"clspackages"} (for classes)
%       or \texttt{"stypackages"} (for
%       packages).\par
% \subparagraph{Specifying where in the {\ttfamily{}.dtx}
% file to output them\thinspace:}
% Because your class or package design may include
% preliminary commands needed before packages are
% included, the relevant {\ttfamily{}\textbackslash{}RequirePackage}
% commands must be added to the {\ttfamily{}.dtx}
% file \emph{in a location that you must specify
%   yourself}. This is done by giving the \texttt{linkend} attribute on the
% enclosing \texttt{constraintdef} element the value
% of an \texttt{xml:id} which you
% have assigned to a chapter or section in your
% annotated code.\par
% \emph{There is no default: you must specify
%   this link yourself, otherwise the list of required
%   packages will not be output.}\par
% The reason is that you may need to write some of
% your class or package code (option declarations, for
% example, or a {\ttfamily{}\textbackslash{}LoadPackage} command),
% \emph{before} the specified packages are
% loaded.\par
% \begin{itemize}
% \item If the chapter or section you have specified
%     has content (text) in it, the
%     {\ttfamily{}\textbackslash{}RequirePackage} commands are
%     output as the content of a new chapter or section
%     immediately preceding or following it, as
%     specified by the value of the \texttt{role}
%     attribute (`before' or
%     `after').
% \item If the chapter or section you have specified
%     is (deliberately) empty, the
%     {\ttfamily{}\textbackslash{}RequirePackage} commands are
%     output as the content of that chapter or
%     section.
% \end{itemize}
% As an example, let us say you specify the
% \texttt{constraintdef} with\par
% \iffalse
%<*ignore>
% \fi
\begin{lstlisting}[language={[DocBook]XML}]
<constraintdef xml:id="clspackages" linkend="options" role="after">...
\end{lstlisting}
% \iffalse
%</ignore>
% \fi
% You must then have a chapter or section in your
% documented code with the \texttt{xml:id} value of \texttt{"options"}. If
% it is empty (no character data content) like this:\par
% \iffalse
%<*ignore>
% \fi
\begin{lstlisting}[language={[DocBook]XML}]
<sect1 xml:id="options">
  <title/>
  <para/>
</sect1>
\end{lstlisting}
% \iffalse
%</ignore>
% \fi
% then the list of {\ttfamily{}\textbackslash{}RequirePackage} commands will
% be output in its place.\par
% If, on the other hand, the specified section
% has text and code of its own:\par
% \iffalse
%<*ignore>
% \fi
\begin{lstlisting}[language={[DocBook]XML}]
<sect1 xml:id="options">
    <title>Options</title>
    <para>text...</para>
    <programlisting>
\some{code}
    </programlisting>
</sect1>
\end{lstlisting}
% \iffalse
%</ignore>
% \fi
% then the list of {\ttfamily{}\textbackslash{}RequirePackage}
% commands will be output immediately after it, as a new
% section at the same level.\par
% In both cases, the \texttt{segtitle} of the
% \texttt{segmentedlist} will be used as the title of
% the section.\par
% \subsubsection{Automated settings for declared packages}\label{autospec}
% There are several reasons for automating package setup:\par
% \begin{itemize}
% \item Many \LaTeX{} authors and designers have
% `favourite' settings that they
% like to use every time they specify a particular
% package.
% \item Some options have now become the \emph{de
%   facto} convention for their package,
% (for example, the \textbf{\texttt{T1}} option on the
% \textsf{fontenc} package).
% \item There are commands that need to be used whenever a
% particular package is invoked (for example, the
% \textsf{makeidx} package means you need to
% add the {\ttfamily{}\textbackslash{}makeindex} command to the
% Preamble).
% \item Some packages are only needed in the documentation
% if a particular formatting feature is used (for
% example compact list spacing requires the
% \textsf{enumitem} package). This avoids you
% having to remember to include a specific package when
% you use such a feature; and to remove it if you cease
% to use the feature.
% \end{itemize}
% To help automate these, an ancillary (lookup) file
%     called {\ttfamily{}prepost.xml} is used, which
%     is a \emph{DocBook} document with a
%     \texttt{refsection} root element type containing two
%     \texttt{procedure} elements, shown below.\par
% The {\ttfamily{}prepost.xml} file must be in
%     the directory specified by your setting of the
%     {\ttfamily{}repo} runtime parameter.\par
% \iffalse
%<*ignore>
% \fi
\begin{lstlisting}[language={[DocBook]XML}]
<refsection>
  <title>Commands to use before and after packages</title>
  <procedure xml:id="prepackage">
    ...steps...
  </procedure>
  <procedure xml:id="postpackage">
    ...steps...
  </procedure>
</refsection>
\end{lstlisting}
% \iffalse
%</ignore>
% \fi
% The \texttt{"prepackage"} procedure
%     is for material which needs to go
%     \emph{before} a package is invoked. The \texttt{"postpackage"} procedure is for
%     material which needs to go \emph{after} a
%     package is invoked.\par
% \paragraph{Identifying each package}
% Within these procedures, each package is identified in
%       a \texttt{step} element.\par
% \begin{itemize}[noitemsep]
% \item the \texttt{remap} attribute
%   holds the package name;
% \item the \texttt{condition}
%   attribute holds the type[s] of output it is intended
%   to be effective for, \texttt{"doc"}, \texttt{"cls"}, or \texttt{"sty"} (space-separated if
%   more than one);
% \item the \texttt{role} attribute
%   holds any default options (comma-separated);
% \item an optional \texttt{para} element holds a
%   textual description of the package and its use. This
%   is only meaningful for packages marked in the \texttt{condition} attribute for
%   use in the class or package. If present, this gets
%   output to the code documentation.
% \end{itemize}
% \iffalse
%<*ignore>
% \fi
\begin{lstlisting}[language={[DocBook]XML}]
<step role="utf8x" remap="inputenc" condition="cls sty">
  <para>UTF-8 is the default character set, to allow for use of
        any character in any writing system. Some characters 
        are not	specified for all fonts, so may have to be 
        specified manually.</para>
</step>
\end{lstlisting}
% \iffalse
%</ignore>
% \fi
% In this example, specifying
%       \textsf{inputenc} in the document, in a
%       \texttt{seg} element as described in \vref{packspec}, results in the package being
%       added with {\ttfamily{}\textbackslash{}RequirePackage} to the
%       class or package code.\par
% \paragraph{Automating inclusion}\label{autopackage}
% Each \texttt{step} may contain one
%       or more \texttt{constructorsynopsis} elements which
%       specify the condition[s] under which the package will
%       automatically be included \emph{without it needing
% to be specified} in a \texttt{seg} element
%       as described in \vref{packspec}\par
% \begin{itemize}[noitemsep]
% \item the \texttt{condition}
%   attribute holds the name of an element type which,
%   if present in the documentation, will cause the
%   package to be included automatically;
% \item one or more \texttt{methodparam} subelements
%   can be used to specify attribute conditions on the
%   element type named in the \texttt{condition}
%   attribute:
% \begin{itemize}[noitemsep]
% \item the \texttt{parameter} element specifies
%       the name of an attribute. If no
%       \texttt{modifier} element is present, the
%       specified attribute is simply tested for
%       presence (Boolean test)
% \item a
%       \texttt{modifier} element is used to
%       specify a value for which the attribute is tested
% \end{itemize}
% \end{itemize}
% \iffalse
%<*ignore>
% \fi
\begin{lstlisting}[language={[DocBook]XML}]
<step remap="dcolumn" condition="doc">
  <constructorsynopsis condition="colspec">
    <methodparam>
      <parameter>align</parameter>
      <modifier>char</modifier>
    </methodparam>
  </constructorsynopsis>
</step>
\end{lstlisting}
% \iffalse
%</ignore>
% \fi
% In the example above, a \texttt{colspec} element
%       anywhere in the document with an \texttt{align} attribute equal to \texttt{"char"} will cause the
%       \textsf{dcolumn} package to be included
%       automatically (the package handles decimal-column
%       alignment).\par
% A special case involves the use of the \texttt{funcparams} subelement instead of
%       the \texttt{parameter} attribute, to
%       specify that an IDREF attribute must be checked for the
%       \emph{type} of element it refers to.\par
% \iffalse
%<*ignore>
% \fi
\begin{lstlisting}[language={[DocBook]XML}]
<step condition="doc" remap="fmtcount">
  <constructorsynopsis condition="xref">
    <methodparam>
      <funcparams>linkend</funcparams>
      <modifier>varlistentry</modifier>
    </methodparam>
    <methodparam>
      <funcparams>linkend</funcparams>
      <modifier>listitem</modifier>
    </methodparam>
  </constructorsynopsis>
</step>
\end{lstlisting}
% \iffalse
%</ignore>
% \fi
% In this example, the \textsf{fmtcount}
%       package will be included if there is an \texttt{xref}
%       element anywhere in the documentation with a \texttt{linkend} attribute which points
%       at a \texttt{varlistentry} or \texttt{listitem}
%       element~--- that is, the \texttt{xml:id} attribute whose value
%       matches the \texttt{linkend} value
%       is on such an element type (the
%       \textsf{fmtcount} package enables ordinal
%       counting, needed when making a reference to an item in a
%       list that is not numbered).\par
% \paragraph{Adding extra code before or after a package}
% Each \texttt{step} may also contain
%       one or more \texttt{constraintdef} elements containing
%       \texttt{cmdsynopsis} elements containing
%       \texttt{command} elements to hold \LaTeX{} code to be
%       inserted, in exactly the same format as shown in \vref{constraints}.\par
% \iffalse
%<*ignore>
% \fi
\begin{lstlisting}[language={[DocBook]XML}]
<step remap="apacite" condition="doc">
  <constraintdef>
    <cmdsynopsis>
      <command>\AtBeginDocument{\edef\ApaciteRestoreAtCode%
    {\catcode`@=\the\catcode`@\relax}}</command>
    </cmdsynopsis>
  </constraintdef>
</step>
\end{lstlisting}
% \iffalse
%</ignore>
% \fi
% If this step is given in the \texttt{"prepackage"} section, the code is
%       inserted \emph{before} the package is
%       included; if the step is given in the \texttt{"postpackage"} section, the code
%       is inserted \emph{after} the package is
%       included.\par
% A package listed in this file can be given default
%     options by specifying them in the \texttt{role} attribute of the
%     \texttt{step} element. It is then unnecessary to specify
%     them additionally in the main document (although it won't
%     matter, as they are checked for duplication).\par
% \subsubsection{Defining commands required for documentation setup}\label{cmdspec}
% Documenting classes or packages will often require
%     additional commands to be defined in order to set up special
%     counters or lengths, establish conditions, or create new
%     macros to be used in your documentation.\par
% \emph{This is quite different from needing to
%       issue standard \LaTeX{} commands in order to
%       set existing standard \LaTeX{} values, such as }\verb|\setlength{\parskip}{5mm}|.
%     \emph{That kind of adjustment is dealt with in \vref{addcmds}.}\par
% Commands to be defined for the documentation to work
%     must go in \texttt{cmysynopsis} elements in the
%     \texttt{constraintdef} element that has the \texttt{xml:id} value of \texttt{"docpackages"},
%     \emph{after} the end of the
%     \texttt{segmentedlist} of packages.\par
% The commands are specified with the command name
%     (control sequence) or environment name
%     \emph{separately} from the definition: this
%     allows a structure to be imposed which enables the
%     identification and re-use of these specifications.\par
% In giving the commands (control sequences) that you
%     define, you specify just the names, with no backslash; and
%     the definitions you give must not have the outermost set
%     of curly braces. Both the backslash and the curly braces
%     are added by the XSLT2 program when writing the
%   {\ttfamily{}.dtx} file.\par
% Commands containing an `at'
%     sign ({\ttfamily{}@}) in their name or
%     definition are automatically enclosed in
%     {\ttfamily{}\textbackslash{}makeatletter} and
%     {\ttfamily{}\textbackslash{}makeatother} commands.\par
% There are several attributes which specify what you
%     are defining, so that the data is output to the
%     {\ttfamily{}.dtx} file correctly:\par
% \begin{description}[style=unboxed]
% \item[Simple commands\thinspace:]A simple new \LaTeX{} command with a textual
%   expansion is defined with the \texttt{command}
%   element holding the name of the command being
%   defined, and an \texttt{arg} element holding the
%   definition.\par
% \iffalse
%<*ignore>
% \fi
\begin{lstlisting}[language={[DocBook]XML}]
<cmdsynopsis>
  <command>LyX</command>
  <arg>L\kern-.1667em\lower.25em\hbox{Y}\kern-.125emX</arg>
</cmdsynopsis>
\end{lstlisting}
% \iffalse
%</ignore>
% \fi
% This creates the definition:
% \iffalse
%<*ignore>
% \fi
\begin{lstlisting}[language={[LaTeX]TeX}]
\newcommand{\LyX}{L\kern-.1667em\lower.25em\hbox{Y}\kern-.125emX}
\end{lstlisting}
% \iffalse
%</ignore>
% \fi
% \item[Renewed commands\thinspace:]If the command is a renewal of an existing
%   command, use a \texttt{role}
%   attribute of \texttt{"renew"} on
%   the \texttt{command} element.\par
% \iffalse
%<*ignore>
% \fi
\begin{lstlisting}[language={[DocBook]XML}]
<cmdsynopsis>
  <command role="renew">vstrut</command>
  <arg>\vrule height1.2em depth.6667ex width0pt</arg>
</cmdsynopsis>
\end{lstlisting}
% \iffalse
%</ignore>
% \fi
% This creates the definition:
% \iffalse
%<*ignore>
% \fi
\begin{lstlisting}[language={[LaTeX]TeX}]
\renewcommand{\vstrut}{\vrule height1.2em depth.6667ex width0pt}
\end{lstlisting}
% \iffalse
%</ignore>
% \fi
% \item[Plain \TeX{} commands\thinspace:]A {\ttfamily{}\textbackslash{}def} command in Plain \TeX{}
%   syntax can be specified with the attribute \texttt{xml:lang} set to \texttt{"TeX"} on the
%   \texttt{command} element.\par
% \iffalse
%<*ignore>
% \fi
\begin{lstlisting}[language={[DocBook]XML}]
<cmdsynopsis xml:lang="TeX">
  <command>hline</command>
  <arg>\noalign{\ifnum0=`}\fi
       \@ifnextchar[{\@@hline}{\@@hline[\arrayrulewidth]}</arg>
</cmdsynopsis>
\end{lstlisting}
% \iffalse
%</ignore>
% \fi
% This creates the definition:
% \iffalse
%<*ignore>
% \fi
\begin{lstlisting}[language={[LaTeX]TeX}]
\def\hline{\noalign{\ifnum0=`}\fi
    \@ifnextchar[{\@@hline}{\@@hline[\arrayrulewidth]}
\end{lstlisting}
% \iffalse
%</ignore>
% \fi
% \item[Commands with arguments\thinspace:]If a defined command needs arguments, specify
%   the number of arguments in the \texttt{wordsize} attribute of the
%   \texttt{arg} element:\par
% \iffalse
%<*ignore>
% \fi
\begin{lstlisting}[language={[DocBook]XML}]
<cmdsynopsis>
  <command>componentbox</command>
  <arg wordsize="2">\begin{tabular}[m]{@{}|c|@{}}\hline
      \cellcolor{#1}\hbox to1em{\hss%\vrule height1em width0pt
        \raisebox{.2ex}{\ttfamily\tiny#2}\hss}\\\hline
      \end{tabular}</arg>
</cmdsynopsis>
\end{lstlisting}
% \iffalse
%</ignore>
% \fi
% This creates the definition:\par
% \iffalse
%<*ignore>
% \fi
\begin{lstlisting}[language={[LaTeX]TeX}]
\newcommand{\componentbox}[2]{\begin{tabular}[m]{@{}|c|@{}}\hline
    \cellcolor{#1}\hbox to1em{\hss%\vrule height1em width0pt
        \raisebox{.2ex}{\ttfamily\tiny#2}\hss}\\\hline
    \end{tabular}
\end{lstlisting}
% \iffalse
%</ignore>
% \fi
% If a default first argument is required, the
%   value must be provided in the \texttt{condition} attribute of the
%   \texttt{command} element.\par
% There is no provision in this version of the
% software for the specification of the extended argument array
%   provided by the \textsf{xargs} package.
% \item[Counters, lengths, and \texttt{\textbackslash{}newwrite}s\thinspace:]Counters, lengths, and
%   {\ttfamily{}\textbackslash{}newwrite} commands can be defined
%   by using the \texttt{remap}
%   attribute set to the value \texttt{"counter"}, \texttt{"length"} or \texttt{"newwrite"} as appropriate. in
%   this case, no \texttt{arg} element is
%   required unless a counter or length is to be
%   assigned a default value.\par
% \iffalse
%<*ignore>
% \fi
\begin{lstlisting}[language={[DocBook]XML}]
<cmdsynopsis>
  <command remap="newwrite">fnotes</command></cmdsynopsis>
\end{lstlisting}
% \iffalse
%</ignore>
% \fi
% This creates the definition:\par
% \iffalse
%<*ignore>
% \fi
\begin{lstlisting}[language={[LaTeX]TeX}]
\newwrite\fnotes
\end{lstlisting}
% \iffalse
%</ignore>
% \fi
% However, if the length or counter needs an
%   initial value, give it in an \texttt{arg} element.\par
% \iffalse
%<*ignore>
% \fi
\begin{lstlisting}[language={[DocBook]XML}]
<cmdsynopsis>
  <command remap="length">revmarg</command>
  <arg>3cm</arg>
</cmdsynopsis>
<cmdsynopsis>
  <command remap="counter">cards</command>
  <arg>42</arg>
</cmdsynopsis>
\end{lstlisting}
% \iffalse
%</ignore>
% \fi
% This creates the definition:
% \iffalse
%<*ignore>
% \fi
\begin{lstlisting}[language={[LaTeX]TeX}]
\newlength{\revmarg}\setlength{\revmarg}{3cm}
\end{lstlisting}
% \iffalse
%</ignore>
% \fi
% \item[References to attributes\thinspace:]One specialist use is predefined: assigning the
% type of document (class or package) to a command:\par
% \iffalse
%<*ignore>
% \fi
\begin{lstlisting}[language={[DocBook]XML}]
<cmdsynopsis>
  <command>classorpackage</command>
  <arg remap="arch"/>
</cmdsynopsis>
\end{lstlisting}
% \iffalse
%</ignore>
% \fi
% This creates the definition:\par
% \iffalse
%<*ignore>
% \fi
\begin{lstlisting}[language={[LaTeX]TeX}]
\newcommand{\classorpackage}{...}
\end{lstlisting}
% \iffalse
%</ignore>
% \fi
% where [\dots{}] is the type of the current
%   document. The \texttt{arch} element in this case
%   has no content, but uses the \texttt{remap} attribute to specify
%   the name of an attribute (here, \texttt{arch}) on the
%   \texttt{book} root element. This results in the
%   command {\ttfamily{}\textbackslash{}classorpackage} being set
%   equal to \texttt{"class"} or \texttt{"package"}; this value is used
%   to provide a value for the entity \texttt{\&doctype;}. This entity can
%   then be used in shared XML documentation to refer to
%   the current document by type, knowing that it will
%   be correctly translated to the type of document when
%   the {\ttfamily{}.dtx} file is
%   created.
% \item[Environments\thinspace:]The same principle applies to environments as to
%   commands, but there are two arguments: one for the
%   `before' and one for the
%   `after':\par
% \iffalse
%<*ignore>
% \fi
\begin{lstlisting}[language={[DocBook]XML}]
<cmdsynopsis>
  <command remap="environment">panel</command>
  <arg wordsize="1" condition="\relax">\begin{Sbox}%
        \begin{minipage}{3in}%
        \if#1\relax\else\subsubsection*{#1}\fi</arg>
  <arg>\end{minipage}\end{Sbox}%
        \begin{center}\fbox{\theSbox}\end{center}%</arg>
</cmdsynopsis>
\end{lstlisting}
% \iffalse
%</ignore>
% \fi
% This creates the definition:\par
% \iffalse
%<*ignore>
% \fi
\begin{lstlisting}[language={[LaTeX]TeX}]
\newenvironment{panel}{%
    \begin{Sbox}%
        \begin{minipage}{3in}%
        \if#1\relax\else\subsubsection*{#1}\fi
}{%
    \end{minipage}\end{Sbox}%
        \begin{center}\fbox{\theSbox}\end{center}%
}
\end{lstlisting}
% \iffalse
%</ignore>
% \fi
% The controls for the number of arguments and any
%   default argument must go on the first \texttt{arg}
%   element. The same rule about setting the \texttt{role} attribute to \texttt{"renew"} applies as for
%   generating commands.
% \end{description}
% \subsubsection{Additional setup commands}\label{addcmds}
% Quite separately from the business of defining new
%     commands (or redefining existing ones) dealt with in \vref{cmdspec}, there is also usually a need to issue
%     commands that establish or reset a value needed for the
%   documentation.\par
% Commands that you want implemented every time you use
%     a particular package should go in your
%     {\ttfamily{}prepost.xml} file, as described in
%   \vref{autospec}. This section is for commands or
%     settings that only refer to the documentation for the
%     current class or package being written.\par
% These commands go in the third of the types of
%     \texttt{constraintdef} element, with the \texttt{xml:id} value of \texttt{"startdoc"} because they are output
%     at the start of the documentation
%     (\emph{after} the
%     \verb|\begin{document}|).\par
% They follow exactly the same syntax as those in the
%     {\ttfamily{}prepost.xml} file:\par
% \iffalse
%<*ignore>
% \fi
\begin{lstlisting}[language={[DocBook]XML}]
<constraintdef xml:id="startdoc">
  <procedure>
    <step>
      <cmdsynopsis>
        <command>\setcounter{tocdepth}{5}</command>
        <command>\setcounter{secnumdepth}{5}</command>
        <command>\def\@@doxdescribe#1#2{\endgroup
   \ifdox@noprint\else\marginpar{\raggedleft
   \textcolor{DarkRed}{\@nameuse{PrintDescribe#1}{#2}}}\fi
   \ifdox@noindex\else\@nameuse{Special#1Index}{#2}\fi
   \endgroup\@esphack\ignorespaces}</command>
      </cmdsynopsis>
    </step>
  </procedure>
</constraintdef>
\end{lstlisting}
% \iffalse
%</ignore>
% \fi
% The {\ttfamily{}\textbackslash{}@@doxdescribe} command is an
%     oddity here: it appears not to work if placed in the
%     {\ttfamily{}prepost.xml} document, where it gets
%     issued in the Preamble; instead it goes here, where it
%     gets issued after the
%     \verb|\begin{document}|.\par
% \subsubsection{The Manifest}\label{manifestfiles}
% The fourth and last variant of the
%     \texttt{constraintdef} element (with the \texttt{xml:id} attribute of \texttt{"manifest"}) is very simple. It is a
%     list of the names of any separate files that you want
%     included in the Zip file that the {\ttfamily{}build} command produces. This
%     means anything other than the {\ttfamily{}.dtx},
%     {\ttfamily{}.ins}, and {\ttfamily{}.pdf}
%     files that get included automatically.\par
% The content of this element is a
%     \texttt{simplelist}, containing \texttt{member}
%     elements, one per file:\par
% \iffalse
%<*ignore>
% \fi
\begin{lstlisting}[language={[DocBook]XML}]
<constraintdef xml:id="manifest">
  <simplelist>
    <member>doctexbook.dtd</member>
    <member>db2dtx.xsl</member>
    <member>db2bibtex.xsl</member>
    <member>prepost.xml</member>
    <member>lppl.xml</member>
    <member>decommentbbl.awk</member>
  </simplelist>
</constraintdef>
\end{lstlisting}
% \iffalse
%</ignore>
% \fi
% \subsubsection{The README file}\label{readme}
% One additional output file is produced automatically
%     by the XSLT2 program: the plaintext README file which
%     accompanies all classes and packages, with a brief
%     description of usage and installation, for the benefit of
%     people who cannot or will not read the PDF
%     documentation.\par
% This is generated automatically from the file
%     {\ttfamily{}readme.xml}, which is a DocBook5
%     \texttt{chapter} document with some changes to the
%     character entities to accommodate plain text. Note that
%     this document does \emph{not} use the
%     {\ttfamily{}doctexbook.dtd} used for your normal
%     class or package {\small XML} document. A sample
%     is included in the \textsf{classpack}
%     distribution.\par
% The {\ttfamily{}readme.xml} file  uses the
%     \texttt{olink} element type to act as a placeholder for
%     transcluded atomic information from the main document
%     (pending implementation of the proposed DocBook
%     Transclusions method). This element must have a
%     \texttt{targetptr} and a \texttt{type} attribute
%     specifying (respectively) the name of the element type and
%     the relevant attribute in the main document. For example,
%     to include the name of the class or package, we
%     use:\par
% \iffalse
%<*ignore>
% \fi
\begin{lstlisting}[language={[LaTeX]TeX}]
<olink targetptr="book" type="xml:id"/>
\end{lstlisting}
% \iffalse
%</ignore>
% \fi
% Two other element types have also had
%     \texttt{targetptr} and \texttt{type} attributes
%     added: \texttt{sect1} and \texttt{anchor}. These are
%     used to specify the inclusion of whole sections or
%     fragments of the main document, such as the Abstract or
%     the Copyright.\par
% The text is reformatted in plain text, omitting all
%     markup. Only a few element types have been implemented for
%     this in this version: see the ancillary XSLT2 routines in
%     {\ttfamily{}db2plaintext.xsl} for details.\par
% The resulting {\ttfamily{}README} file
%     includes the Abstract from your class or package
%     {\small XML} document as the first section. The
%     {\ttfamily{}db2plaintext.xsl} program uses a
%     template called \DescribeTemplate{normtext}\texttt{normtext} to
%     reformat text. This handles the conversion of entities
%     which occur in your Abstract (those declared in
%     {\ttfamily{}doctexbook.dtd}): by default, \texttt{\&TeX;}, \texttt{\&LaTeX;}, and \texttt{\&thinsp;} are catered for, but if
%     you use any others, you must modify the code in this
%     template to deal with them, using a nested
%     \DescribeFunction{replace()}\texttt{replace()} function.\par
% \clearpage
% \section{Using \emph{ClassPack}}\label{content}
% The body of your docmentation is held in a \texttt{part}
% element with the \texttt{xml:id} attribute
% set to \texttt{"doc"}.\par
% The tag-set of \emph{DocBook} is very
% large, and only a part of it is needed for this purpose,
% although support for additional elements is easily added in
% the XSLT2 program.\par
% The following sections describe the elements that are
% currently supported, for the hierarchical structure (chapters,
% sections, subsections, etc); the block-level structure
% (tables, figures, lists, etc; what
% \emph{DocBook} refers to as the
% `pool') and for the inline markup (element types
% in mixed content, used mostly in paragraph-like
% situations).\par
% \subsection{Hierarchical markup}\label{hierarchy}
% The outline top-level structure is described in \vref{invocation} and \vref{setup}.\par
% The \texttt{part}\thinspace{}s do not have any title or direct
%   textual content themselves: they just act as containers to
%   keep the user documentation separate from the documented
%   code and any other files that may be stored and
%   extracted.\par
% Within a \texttt{part}, the major subdivision is the
%   \texttt{chapter}, which translates to a
%   {\ttfamily{}\textbackslash{}section} in the {\ttfamily{}.dtx}
%   file. You should use the \texttt{chapter} element as your
%   major structural division. Each part (user documentation,
%   documented code, and additional files) must have at least
%   one chapter.\par
% Within chapters, the sections, subsections, and lower
%   structural divisions are identified with \texttt{sect1},
%   \texttt{sect2}, and so on. You can use as many or as few
%   of these as you feel you need to organise your writing. In
%   the documented code, it is a good idea to modularise the
%   class or package, so that you can describe each part of it
%   in a logical and consistent manner.\par
% The nested arrangement of chapters containing sections
%   conaining subsections should already be familiar to \LaTeX{}
%   users, although \LaTeX{} itself only uses headings as
%   separators, and has no physical
%   `containment' of the hierarchical
%   divisions of a document in the way that it does for the
%   block-level structures (environments).\par
% Each of these hierarchical divisions must have a
%   \texttt{title} element (see `\textbf{\texttt{title}}', the first item in the list in \vref{title}),
%   and can also have an \texttt{xml:id}
%   attribute to act as a target (like a \LaTeX{}
%   {\ttfamily{}\textbackslash{}label}) for cross-referencing (see `\textbf{\texttt{xref}}', the last item in the list in \vref{xref}).\par
% \iffalse
%<*ignore>
% \fi
\begin{lstlisting}[language={[DocBook]XML}]
<part xml:id="doc">...
  <chapter xml:id="ui">
    <title>User interface</title>
    ...
    <sect1>
      <title>Font selection</title>
      ...
    </sect1>
    <sect1 xml:id="margins">
      <title>Margins and spacing</title>
      ...
    </sect1>
  </chapter>
  ...
</part>
\end{lstlisting}
% \iffalse
%</ignore>
% \fi
% \subsection{Structural markup (block-level elements)}\label{pool}
% Within chapters, sections, subsections, etc, you can
%   have any arrangement or mixture of paragraphs, tables,
%   lists, figures, quotations, code samples, and other
%   conventional structures that will be familiar to you from
%   \LaTeX{} environments.\par
% The only requirement is that each hierarchical division
%   must start with a title, and must contain at least one other
%   structural component. Supported element types are:\par
% \begin{description}[style=unboxed]
% \item[\texttt{title}\thinspace:]\label{title}a title, used in all chapters, [sub]sections,
% figures, and tables (where it equates to a caption);
% and also optionally in lists, sidebars, and other
% block-level element types.
% \item[\texttt{para}\thinspace:]for normal paragraphs.
% \item[\texttt{itemizedlist}\thinspace:]for bulleted lists, containing
% \texttt{listitem}\thinspace{}s which contain paragraphs.
% \item[\texttt{orderedlist}\thinspace:]for numbered lists; the structure is identical to
% an \texttt{itemizedlist}.
% \item[\texttt{variablelist}\thinspace:]for description lists like this one (the
% \texttt{term} element in each
% \texttt{varlistentry} holds the reference term; the
% descriptive part is in the same \texttt{listitem}
% structure as for itemized and bulleted lists.
% \item[\texttt{simplelist}\thinspace:]for plain unnumbered, unbulleted lists; each item
% goes in a \texttt{member} element.
% \item[\texttt{programlisting}\thinspace:]for listings of code: use without attributes in
% the Code section. In the Documentation section, the
% basic style is in {\ttfamily{}\textbackslash{}small} type,
% black, {\ttfamily{}\textbackslash{}ttfamily}, and the following
% attributes control the appearance:
% \par\medskip{\sffamily\small
% \begingroup
% \centering
% \begin{tabular}{@{}%
% 	>{\ttfamily\prestrut\arraybackslash}l<{\poststrut\arraybackslash}%
% 	>{\raggedright{}\prestrut\arraybackslash}p{9cm}<{\poststrut\arraybackslash}%
% 	@{}}
% \hline
% \vstrut
% wordsize&either a size command or a
% pointsize/baseline like
% {\ttfamily{}8/9}\\
% language&{\ttfamily{}LaTeX} (default),
% {\ttfamily{}DocBook},
% {\ttfamily{}bash}, or another
% language supported by the
% \textsf{listings} package\\
% arch&{\ttfamily{}framed} will box
% the listing\\
% remap&\LaTeX{} styling commands for tokens to
% emphasise\\
% annotations&comma-separated list of tokens to
% emphasise\\[2pt]\hline
% \end{tabular}
% \par\endgroup
% }
% \item[\texttt{figure}\thinspace:]for Figures, containing a \texttt{caption} and a
% \texttt{media} element.
% \item[\texttt{table}\thinspace:]for Tables; the structure is explained in detail
% in \vref{table}.
% \item[\texttt{sidebar}\thinspace:]for sidebars.
% \item[\texttt{warning}\thinspace:]for warnings.
% \end{description}
% \subsubsection{Ancillary files documented inline}\label{secfiles}
% These are files which you want extracted at
%     installation time, which you describe
%     \emph{and} show in the user documentation
%     (the \texttt{doc} part).\par
% \emph{Files which you want extracted which are
%       \emph{not} documented or shown in the user
%       documentation must go in the \texttt{files} part (see
%       \vref{ancfiles}).}\par
% \subsubsection{Bibliography}\label{bibs}
% If bibliographic citations and reference is required,
%     the references themselves must be stored in a
%     \texttt{bibliography} element immediately after the last
%     \texttt{chapter} element in \emph{either}
%     the `doc' part
%     \emph{or} the `code'
%     part. This must contain a \texttt{biblioentry} element
%     for each entry you wish to cite (for how to cite, see
%     `\textbf{\texttt{biblioref}}', the first item in the list in \vref{bibrefs}).\par
% \iffalse
%<*ignore>
% \fi
\begin{lstlisting}[language={[DocBook]XML}]
<bibliography xreflabel="apacite" label="apacite">
  <biblioentry xml:id="tb97" xreflabel="article">
    <biblioset>
      <author>
        <personname>
          <surname>Flynn</surname>
          <firstname>Peter</firstname>
        </personname>
      </author> 
      <title>Typographers' Inn</title>
      <subtitle>Where have all the flowers gone?</subtitle>
    </biblioset>
    <artpagenums>21-22</artpagenums>
    <title>TUGboat</title>
    <volumenum>31</volumenum>
    <issuenum>1</issuenum>
    <date YYYY-MM-DD="2010"/>
  </biblioentry>
</bibliography>
\end{lstlisting}
% \iffalse
%</ignore>
% \fi
% The \texttt{bibliography} element
%     \textsc{must} have a \texttt{label} attribute giving the name
%     of the \BibTeX{} style file to use (without the
%     {\ttfamily{}.bst} filetype). The
%     \textsf{apacite} style is recommended.\par
% If the specified style requires a \LaTeX{} style
%     package for formatting (often called by the same name, eg
%     \textsf{apacite}, \textsf{natbib},
%     \textsf{chicago}, etc), this
%     \textsc{must} be given in an \texttt{xreflabel} attribute (without the
%     {\ttfamily{}.sty} filetype).\par
% There \textsc{may} also be an
%     \texttt{xlink:href} attribute giving
%     the name of the \BibTeX{} file (without the
%     {\ttfamily{}.bib} filetype) to which the
%     references should be written: the default is the name of
%     the class or package itself (as defined in `\textbf{\texttt{xml:id}}', the first item in the list in \vref{rootid}).\par
% Each \texttt{biblioentry} element
%     \textsc{must} have both an \texttt{xml:id} attribute by which it can
%     be cited with the \texttt{biblioref} element; and a \texttt{type} attribute classifying it
%     with one of the standard \BibTeX{} document types (article,
%     book, incollection, etc)\par
% \subsection{Inline markup (elements in mixed content)}\label{flow}
% \par
% \begin{description}[style=unboxed]
% \item[\texttt{biblioref}\thinspace:]\label{bibrefs}a citation (bibliographic reference) to an item in
% the Bibliography; the \texttt{linkend} attribute must be
% the value of the \texttt{xml:id}
% of a \texttt{biblioentry} element (see \vref{bibs}); this value is passed to a
% {\ttfamily{}\textbackslash{}cite} command.
% \item[\texttt{citetitle}\thinspace:]the title of a document being mentioned; usually
% formatted in italics or quotes; may be empty, with a \texttt{linkend} attribute pointing to
% an entry in the Bibliography (as for
% \texttt{biblioref}), in which case the title is
% automatically extracted and formatted, or passed to a
% {\ttfamily{}\textbackslash{}citefield} command.
% \item[\texttt{code}\thinspace:]a fragment of computer or data code, formatted in
%       monospace type.
% \item[\texttt{emphasis}\thinspace:]emphasis according to style, usually italics.
% \item[\texttt{exceptionname}\thinspace:]used for the keywords of RFC2119 in formal admonishments.
% \item[\texttt{filename}\thinspace:]name of a file, a full filepath, or just a part of
% the name (eg a filetype).
% \item[\texttt{firstterm}\thinspace:]the defining instance of a specialist term; this
%       may or may not actually be the first occurrence.
% \item[\texttt{footnote}\thinspace:]a footnote; contains a paragraph.
% \item[\texttt{foreignphrase}\thinspace:]for foreign-language expressions; identify the
% language with the \texttt{xml:lang} attribute if the
% phrase is long enough to need the
% \textsf{babel} package.
% \item[\texttt{guibutton}\thinspace:]represents a GUI {\fboxsep2pt\ovalbox{\sffamily Button}}.
% \item[\texttt{guilabel}\thinspace:]represents a GUI
%     {\fboxsep2pt\fbox{\sffamily Label}}.
% \item[\texttt{guimenu}\thinspace:]represents a GUI
%     \textsf{\bfseries Menu}.
% \item[\texttt{guimenuitem}\thinspace:]represents a GUI
%     \textsf{\itshape Menu item}.
% \item[\texttt{guisubmenu}\thinspace:]represents a GUI
%     \textsf{Sub-menu item}.
% \item[\texttt{literal}\thinspace:]marks a {\ttfamily{}literal string} on which
%       no interpretation is to be performed (markup characters
%       like backslashes and curly braces will
% not be escaped; the \texttt{xml:lang} attribute can be set to
% the name of the language (eg \texttt{"TeX"}, \texttt{"LaTeX"}, etc.
% \item[\texttt{phrase}\thinspace:]marks a phrase used as-is (ie not a quote from
%       anyone in particular) by enclosing it in quotation marks.
% \item[\texttt{productname}\thinspace:]marks a product or program name, eg
%       \emph{Emacs}.
% \item[\texttt{quote}\thinspace:]marks a quote from someone by putting it in
%       quotation marks.
% \item[\texttt{replaceable}\thinspace:]identifies text, commands, or keywords to be
% typed, for which the user must substitute a meaningful
% value, eg {\itshape password} (in
% italics).
% \item[\texttt{systemitem}\thinspace:]identifies generic computer-related strings such
%       as system commands, hostnames, Regular Expressions, etc
%       which need to be printed in monospace to eliminate any
%       confusion over 1/l/I, 0/1/I, etc.
% \item[\texttt{type}\thinspace:]marks a span for which special typographical
% treatment is needed. The \texttt{role} attribute must be set
% to `font' and the \texttt{remap} attribute must be set
% to the NFSS2e three-character
% \textsf{fontname}.
% \item[\texttt{uri}\thinspace:]marks a URI (formats it with the
%       {\ttfamily{}\textbackslash{}url} macro).
% \item[\texttt{wordasword}\thinspace:]marks a word that is being used as itself (usually
%       for purposes of clarification), so it goes in quotation
%       marks.
% \item[\texttt{xref}\thinspace:]\label{xref}a cross-reference using the \texttt{linkend} attribute to point
% at some other part of the document, which must have
% the matching \texttt{xml:id}
% value; the mechanism is identical to \LaTeX{}'s
% {\ttfamily{}\textbackslash{}label}\dots{}{\ttfamily{}\textbackslash{}ref}.
% \end{description}
% In addition, there are six special-purpose element types
%   used for functional documentation, that create the special
%   \textsf{dox} package commands for adding \LaTeX{}
%   (and XML) terms to the index, and highlighting them in the
%   left-hand margin:\par
% \begin{description}[style=unboxed]
% \item[\texttt{classname}\thinspace:]a \LaTeX{} document class name like
% \textsf{article}
% \item[\texttt{command}\thinspace:]a \LaTeX{} or other computer command, such as
% {\ttfamily{}\textbackslash{}parskip}; the backslash is added
% automatically for the default case of \LaTeX{}; other
% languages require the \texttt{xml:lang} attribute giving
% the name of the language
% \item[\texttt{envar}\thinspace:]a \LaTeX{} environment name like
% \texttt{enumerate}
% \item[\texttt{option}\thinspace:]a \LaTeX{} option to a class, package, or command,
% like \textbf{\texttt{a4paper}}
% \item[\texttt{package}\thinspace:]a \LaTeX{} package name like
% \textsf{fancybox}
% \item[\texttt{tag}\thinspace:]an XML element, attribute, attribute value, or
% entity name: the type is specified in the \texttt{class} attribute and is one
% of the predetermined list provided automatically by
% \emph{DocBook} (so your XML editor
% will guide you)
% \end{description}
% \subsection{Producing your class or package}\label{production}
% The XSLT2 program generates a number of output files,
%   principally the {\ttfamily{}.dtx} and
%   {\ttfamily{}.ins} files which are the package or
%   class itself. A third output is a {\ttfamily{}build}
%   file, which is a \emph{bash} shell
%   script customised for the production of the class or package
%   you are writing. A fourth is the
%   {\ttfamily{}MANIFEST} file, used for zipping
%   everything up for distribution,\par
% You should therefore keep each class or package
%   development in a separate directory, otherwise the
%   {\ttfamily{}build} file generated by one will
%   overwrite that generated by others.\par
% \iffalse
%<*ignore>
% \fi
\begin{lstlisting}[language=bash]
#! /bin/bash
#
# Bourne shell script to build the class file and documentation
# Note the following line is wrapped here to fit the width
java -jar /usr/local/saxon/saxon9he.jar \
     -o:classpack.dtx classpack.xml \
     /home/peter/texmf/dev/db2dtx.xsl \
     processor=/usr/local/saxon/saxon9he.jar \
     appdir=/home/peter/texmf/dev \
     cpdir=/home/peter/texmf/dev
yes|latex classpack.ins
pdflatex classpack.dtx
bibtex classpack
awk -f /home/peter/texmf/dev/decommentbbl.awk classpack.bbl >classpack.bdc
mv classpack.bdc classpack.bbl
pdflatex classpack.dtx
makeindex -s gind.ist -o classpack.ind classpack.idx
makeindex -s gglo.ist -o classpack.gls classpack.glo
pdflatex classpack.dtx
echo Copying files into dev tree...
mkdir -p doc/latex/classpack
mkdir -p source/latex/classpack
mkdir -p tex/latex/classpack
cp README MANIFEST classpack.pdf doc/latex/classpack
cp classpack.dtx classpack.ins source/latex/classpack
cp classpack.cls tex/latex/classpack
cp db2bibtex.xsl source/latex/classpack
cp db2dtx.xsl source/latex/classpack
cp db2plaintext.xsl source/latex/classpack
cp decommentbbl.awk source/latex/classpack
cp doctexbook.dtd source/latex/classpack
cp lppl.xml source/latex/classpack
cp prepost.xml source/latex/classpack
cp readme.xml source/latex/classpack
echo Zipping up files from dev tree...
zip -r --exclude=*.svn* --exclude=*.DS_Store* \
    classpack-0.73.tds.zip doc/latex/classpack \
    source/latex/classpack tex/latex/classpack
echo Installing working copy...
unzip -o -d ~/texmf classpack-0.73.tds.zip
\end{lstlisting}
% \iffalse
%</ignore>
% \fi
% Because this file will not exist the very first time you
%   process a new class or package, you will need to type that
%   first ({\ttfamily{}java}) command by
%   hand. The arguments are:\par
% \begin{enumerate}
% \item \marg{jar} the location of
%       your copy of the Saxon XSLT2 processor, a
%       {\ttfamily{}.jar} file;
% \item \textbf{\texttt{-o:}} the name of the
%       {\ttfamily{}.dtx} file you are producing;
% \item the name of the XML file you are processing;
% \item the full path to the DB2DTX program;
% \item the location of your copy of the Saxon XSLT2
%       processor (again) as the
%       {\ttfamily{}processor} parameter;
% \item the directory you use for this class or package as
%       the {\ttfamily{}appdir};
% \item the directory where your copy of the XSLT2 program
%       is stored as the
%       {\ttfamily{}cpdir} parameter (along with the DTD,
%       {\ttfamily{}prepost.xml},
%       {\ttfamily{}readme.xml},
%       {\ttfamily{}db2plaintext.xsl}, and
%       {\ttfamily{}lppl.xml} files) .
% \end{enumerate}
% For subsequent runs, you just type {\ttfamily{}./build} and the values and
%   parameters will be re-used automatically. If you ever need
%   to run the XSLT2 process by itself, use the command
%   \verb+grep java build | bash+\par
% The remainder of the {\ttfamily{}build} tests
%   the extraction of the class or package, and compiles the
%   full documentation in the standard sequence, including any
%   bibliography, index, or glossary.\par
% The use of the {\ttfamily{}decommentbbl.awk}
%   script on the \BibTeX{} output is to defeat the use of
%   terminal percent comment characters, which upset the
%   \textsf{ltxdoc} package because of the special use
%   of that character there.\par
% The final stage is to create a Zip file of
%   the class or package, which is placed in the current working
%     directory and then unzipped into your personal \TeX{} tree.\par
% \subsection{Maintaining your class or package}\label{maintenance}
% The things to control each time you make an update are:\par
% \begin{enumerate}
% \item on the \texttt{book} root element, update the \texttt{version} and \texttt{revision} attributes
% \item add a new \texttt{revision} element in the
%       Revision History, setting the \texttt{version} attribute to the
%       compound of the version and revision specified above,
%       and setting the \texttt{date} subelement's \texttt{conformance} attribute to
%       today's date in ISO format
% \item after processing the document once, set the
%       \texttt{book} root element's \texttt{security} attribute to the
%       checksum displayed by \LaTeX{}
% \item process the document again (run the
%       {\ttfamily{}build} script again)~--- the
%       security checksum should now match
% \end{enumerate}
% \clearpage
% \section{The \emph{db2dtx} program}\label{db2dtx}
% While the core of your class or package is the
% \emph{DocBook} XML document, the core of
% the \emph{classpack} system is the program
% that turns your XML into {\ttfamily{}.dtx} and
% {\ttfamily{}.ins} files for distribution as combined
% code and documentation.\par
% The \emph{db2dtx} program is written
% in XSLT2, a declarative language for processing XML. It
% consists of a set of templates, each of which matches a
% pattern in the XML document, usually an element type, or an
% element type in a particular position or with a particular
% attribute value or subelement.\par
% For example, there is a template which matches the
%       \texttt{biblioref} element whenever it occurs. This puts three
% things into the output:\par
% \begin{enumerate}
% \item the command \verb|\cite{| with its
%     opening curly brace;
% \item the value of the link to the bibliographic
%     entry;
% \item the closing \verb|}| curly
%     brace.
% \end{enumerate}
% \iffalse
%<*ignore>
% \fi
\begin{lstlisting}[language=XSLT]
<xsl:template match="db:biblioref">
  <xsl:text>\cite{</xsl:text>
  <xsl:value-of select="@linkend"/>
  <xsl:text>}</xsl:text>
</xsl:template>
\end{lstlisting}
% \iffalse
%</ignore>
% \fi
% The advantage of using a declarative language is that you
% don't need to know when and where each element will occur:
% XSLT2 will find them as they come up in processing, and apply
% the template when it happens. It's basically a case of
% `when you see one of \emph{these}, do
%   \emph{this}'.\par
% The next few sections of this document describe each part
% of the program and how it produces your class or package
% files.\par
% \subsection{XML Declaration and Namespace declarations}
% The program starts in the usual way with the XML
%   Declaration and the \texttt{xsl:stylesheet} start-tag with
% the Namespace declarations.\par
% \iffalse
%<*ignore>
% \fi
\begin{lstlisting}[language=XSLT]
<?xml version="1.0" encoding="UTF-8"?>
<xsl:stylesheet xmlns:xsl="http://www.w3.org/1999/XSL/Transform"
                xmlns:db="http://docbook.org/ns/docbook"
                xmlns:xlink="http://www.w3.org/1999/xlink"
                version="2.0">
\end{lstlisting}
% \iffalse
%</ignore>
% \fi
% Note that this is an XSLT2 program and requires an XSLT2
% processor.\par
% \iffalse
%<*ignore>
% \fi
\begin{lstlisting}[language=XSLT]
  <!-- db2dtx.xsl
       XSL script to transform DocBook5 documentation and code of a
       LaTeX package or class file into a DocTeX (.dtx and .ins)
       distribution.
       Full processing command chain is output to file 'build'
       Note this requires an XSLT2 processor (eg Saxon9 or above)
  -->
\end{lstlisting}
% \iffalse
%</ignore>
% \fi
% We identify the version of the program, output methods,
%   parameters, and the single \texttt{xsl:include} file: the
%   {\ttfamily{}db2bibtex.xsl} module for handling
%   bibliographic formatting.\par
% \iffalse
%<*ignore>
% \fi
\begin{lstlisting}[language=XSLT]
  <xsl:variable name="thisversion">
    <xsl:text>14.7</xsl:text>
  </xsl:variable>

  <xsl:output method="text"/>
  <xsl:output method="text" name="textFormat"/>

  <xsl:include href="db2bibtex.xsl"/>
  <xsl:include href="db2plaintext.xsl"/>
\end{lstlisting}
% \iffalse
%</ignore>
% \fi
% This is incomplete: the remainder of the program is not
%   yet included here.\par
% \StopEventually{\label{endcode}
%   \clearpage
%   \newgeometry{left=3cm}
%   \addcontentsline{toc}{section}{Change History}
%   \label{}
%   \PrintChanges
%   \clearpage
%   \label{codeindex}
%   \addcontentsline{toc}{section}{Index}
%   \PrintIndex}
% \setlength{\revmarg}{1in}
% \addtolength{\revmarg}{\widthof{\MacroFont{IndexColumns}}}
% \newgeometry{left=\revmarg}
% \iffalse
%<*package>
% \fi
% \clearpage
% \section{Service commands}
% \iffalse
%% 
%% SERVICE COMMANDS
% \fi
% As ClassPack itself is not a document class or package
% \emph{per se}, there is no operating
% code.\par
% \iffalse
%% 
%% As ClassPack itself is not a document class or package per se, there
%% is no operating code.
% \fi
% However, there are some ancillary commands commonly used
% in documentation which should be expected by authors of
% classes and packages using ClassPack.\par
% \iffalse
%% 
%% However, there are some ancillary commands commonly used in
%% documentation which should be expected by authors of classes and
%% packages using ClassPack.
% \fi
% This section therefore implements
% {\ttfamily{}classpack.sty}, which gets invoked
% automatically via its entry in
% {\ttfamily{}prepost.xml}.\par
% \iffalse
%% 
%% This section therefore implements classpack.sty, which gets invoked
%% automatically via its entry in prepost.xml.
% \fi
% \begin{counter}{IndexColumns}
% The \textsf{doctex} package uses a default
% three-column index, which is too narrow for most purposes. We
% therefore make the index in two columns, and space them
% slightly farther apart.\par
% \iffalse
%% 
%% The doctex package uses a default three-column index, which is too
%% narrow for most purposes. We therefore make the index in two columns,
%% and space them slightly farther apart.
% \fi
%    \begin{macrocode}
\setcounter{IndexColumns}{2}
\setlength{\columnsep}{3pc}
%    \end{macrocode}
% \end{counter}
% \subsection{\TeX{} and other logos}
% \iffalse
%% 
%% 1  TeX and other logos
% \fi
% \TeX{} and \LaTeX{} are defined in the \LaTeX{} kernel, but
%   most of the others are not. The following definitions are
%   taken from the \textsf{ltugboat} package, used for
%   typesetting the TUGboat journal.\par
% \iffalse
%% 
%% TeX and LaTeX are defined in the LaTeX kernel, but most of the others
%% are not. The following definitions are taken from the ltugboat
%% package, used for typesetting the TUGboat journal.
% \fi
% \begin{macro}{\ConTeXt}
% \ConTeXt{} is a typography and typesetting system meant
%   to provide users easy and consistent access to advanced
%     typographical control \cite{wp-context}.\par
% \iffalse
%% 
%% \ConTeXt{} is a typography and typesetting system meant to provide
%% users easy and consistent access to advanced typographical control (,
%% ).
% \fi
%    \begin{macrocode}
\def\ConTeXt{C\kern-.0333emon\-\kern-.0667em\TeX\kern-.0333emt}
%    \end{macrocode}
% \end{macro}
% \iffalse
%</package>
% \fi
% \clearpage
% \raggedright
% \bibliography{classpack}
% \bibliographystyle{apacite}
% \begin{VerbatimOut}{classpack.bib}
%<*ignore>
@article{wp-context,
author 	 = {Anon},
shortauthor 	 = {Anon},
title 	 = {{\ConTeXt{}}},
journal 	 = {{Wikipedia}},
url 	 = {http://en.wikipedia.org/wiki/ConTeXt},
lastchecked 	 = {27 March 2013}
}
%</ignore>
% \end{VerbatimOut}
% \appendix
% \iffalse
%<*savedtd>
% \fi
% \clearpage
% \section{The XML vocabulary}\label{savedtd}
% \iffalse
%% 
%% The XML vocabulary
% \fi
% There are currently no changes to the
% \emph{DocBook} element structure.\par
% \iffalse
%% 
%% There are currently no changes to the DocBook element structure.
% \fi
% \begin{dtd}{book}
% The DTD is a driver implementing a number of entity
%   declarations to ease the transformation to \LaTeX{}.\par
% \iffalse
%% 
%% The DTD is a driver implementing a number of entity declarations to
%% ease the transformation to LaTeX.
% \fi
%    \begin{macrocode}
<!ENTITY % db5dtd SYSTEM "/dtds/docbook/docbook-5.0/dtd/docbook.dtd">
<!ATTLIST date YYYY-MM-DD CDATA #IMPLIED>
<!ATTLIST blockquote units CDATA #IMPLIED 
                     begin CDATA #IMPLIED 
                     end CDATA #IMPLIED>
<!ATTLIST quote units CDATA #IMPLIED 
                begin CDATA #IMPLIED 
                end CDATA #IMPLIED>
<!ELEMENT html:form EMPTY>
<!ENTITY ampers "\&#38;#38;">
<!ENTITY BiBTeX "\BibTeX{}">
<!ENTITY BibTeX "\BibTeX{}">
<!ENTITY BIBTeX "\BibTeX{}">
<!ENTITY ConTeXt "\ConTeXt{}">
<!ENTITY LaTeX "\LaTeX{}">
<!ENTITY LaTeX2e "\LaTeXe{}">
<!ENTITY XeTeX "\XeTeX{}">
<!ENTITY LyX "\LyX{}">
<!ENTITY METAFONT "\MF{}">
<!ENTITY METAPOST "\MP{}">
<!ENTITY TeX "\TeX{}">
<!ENTITY bsol "{\texttt{\textbackslash}}">
<!ENTITY date "\filedate{}">
<!ENTITY degree "\textdegree{}">
<!ENTITY doctype "\classorpackage{}">
<!ENTITY filler "\hfil{}">
<!ENTITY frac12 "\nicefrac12">
<!ENTITY frac13 "\nicefrac13">
<!ENTITY frac23 "\nicefrac23">
<!ENTITY hellip "\dots{}">
<!ENTITY mdash "~--- ">
<!ENTITY mldr "\dotfill{}">
<!ENTITY nbsp "~">
<!ENTITY ndash "--">
<!ENTITY percnt "\&#x0025;">
<!ENTITY square "\raisebox{-1pt}{\Square}">
<!ENTITY thinsp "\thinspace{}">
<!ENTITY times "×">
<!ENTITY specialUuml '{\normalfont\"{\fontfamily{cdr}\selectfont U}}'>
<!ENTITY verbar "\menusep{}">
<!ENTITY version "\fileversion{}">
<!-- call the main DTD --> %db5dtd;
%    \end{macrocode}
% \end{dtd}
% \iffalse
%</savedtd>
% \fi
% \iffalse
%<*lxp>
% \fi
% \clearpage
% \section{Reusable XML}\label{lxp}
% \iffalse
%% 
%% Reusable XML
% \fi
% In the last item in the list in \vref{xmltools}, I said that one of the
% benefits of using XML for software generation and
% documentation was the re-usability of the data. Here are a
% couple of simple examples.\par
% \iffalse
%% 
%% In , I said that one of the benefits of using XML for software
%% generation and documentation was the re-usability of the data. Here
%% are a couple of simple examples.
% \fi
%    \begin{macrocode}
$ lxprintf -e productname "%s\n" . classpack.xml |\
  sort | uniq -c | sort -k 1nr
%    \end{macrocode}
% Checking that all element types have been described!\par
% \iffalse
%% 
%% Checking that all element types have been described!
% \fi
% \iffalse
%</lxp>
% \fi
% \newgeometry{left=3cm}
% \clearpage
% \section{The \LaTeX{} Project Public License}\label{LPPL:LPPL}
% \begin{quotation}\small\noindent
% Everyone is allowed to distribute verbatim copies of this
%       license document, but modification of it is not allowed.
% \end{quotation}
% \subsection{Preamble}\label{LPPL:Preamble}
% The \LaTeX{} Project Public License ({\small LPPL})
%       is the primary license under which the \LaTeX{} kernel and the
%       base \LaTeX{} packages are distributed.\par
% You may use this license for any work of which you hold the
%       copyright and which you wish to distribute.  This license may be
%       particularly suitable if your work is \TeX{}-related (such as a
%       \LaTeX{} package), but it is written in such a way that you can
%       use it even if your work is unrelated to \TeX{}.\par
% The section \emph{Whether and How to Distribute Works under This
%       License}, below, gives instructions, examples, and
%       recommendations for authors who are considering distributing
%       their works under this license.\par
% This license gives conditions under which a work may be
%       distributed and modified, as well as conditions under which
%       modified versions of that work may be distributed.\par
% We, the \LaTeX{3} Project, believe that the conditions below
%       give you the freedom to make and distribute modified versions of
%       your work that conform with whatever technical specifications
%       you wish while maintaining the availability, integrity, and
%       reliability of that work.  If you do not see how to achieve your
%       goal while meeting these conditions, then read the document
%       {\ttfamily{}cfgguide.tex} and {\ttfamily{}modguide.tex} in the base \LaTeX{}
%       distribution for suggestions.\par
% \subsection{Definitions}\label{LPPL:Definitions}
% In this license document the following terms are used:\par
% \begin{description}[style=unboxed]
% \item[Work\thinspace:]Any work being distributed under this License.
% \item[Derived Work\thinspace:]Any work that under any applicable law is derived from
%     the Work.
% \item[Modification\thinspace:]Any procedure that produces a Derived Work under any
%     applicable law~--- for example, the production of a file
%     containing an original file associated with the Work or a
%     significant portion of such a file, either verbatim or
%     with modifications and/or translated into another
%     language.
% \item[Modify\thinspace:]To apply any procedure that produces a Derived Work
%     under any applicable law.
% \item[Distribution\thinspace:]Making copies of the Work available from one person to
%     another, in whole or in part.  Distribution includes (but
%     is not limited to) making any electronic components of the
%     Work accessible by file transfer protocols such as
%     {\small FTP} or {\small HTTP} or by
%     shared file systems such as Sun's Network File System
%     ({\small NFS}).
% \item[Compiled Work\thinspace:]A version of the Work that has been processed into a
%     form where it is directly usable on a computer system.
%     This processing may include using installation facilities
%     provided by the Work, transformations of the Work, copying
%     of components of the Work, or other activities.  Note that
%     modification of any installation facilities provided by
%     the Work constitutes modification of the Work.
% \item[Current Maintainer\thinspace:]A person or persons nominated as such within the Work.
%     If there is no such explicit nomination then it is the
%     `Copyright Holder' under any applicable
%     law.
% \item[Base Interpreter\thinspace:]A program or process that is normally needed for
%     running or interpreting a part or the whole of the
%     Work.\par
% A Base Interpreter may depend on external components
%     but these are not considered part of the Base Interpreter
%     provided that each external component clearly identifies
%     itself whenever it is used interactively.  Unless
%     explicitly specified when applying the license to the
%     Work, the only applicable Base Interpreter is a
%     `\LaTeX{}-Format' or in the case of files
%     belonging to the `\LaTeX{}-format' a program
%     implementing the `\TeX{} language'.
% \end{description}
% \subsection{Conditions on Distribution and Modification}\label{LPPL:Conditions}
% \begin{enumerate}
% \item Activities other than distribution and/or modification
%   of the Work are not covered by this license; they are
%   outside its scope. In particular, the act of running the
%   Work is not restricted and no requirements are made
%   concerning any offers of support for the Work.
% \item \label{LPPL:item:distribute}You may distribute a complete, unmodified copy of the
%   Work as you received it.  Distribution of only part of the
%   Work is considered modification of the Work, and no right to
%   distribute such a Derived Work may be assumed under the
%   terms of this clause.
% \item You may distribute a Compiled Work that has been
%   generated from a complete, unmodified copy of the Work as
%   distributed under Clause~item~\ref{LPPL:item:distribute} above above, as
%   long as that Compiled Work is distributed in such a way that
%   the recipients may install the Compiled Work on their system
%   exactly as it would have been installed if they generated a
%   Compiled Work directly from the Work.
% \item \label{LPPL:item:currmaint}If you are the Current Maintainer of the Work, you may,
%   without restriction, modify the Work, thus creating a
%   Derived Work.  You may also distribute the Derived Work
%   without restriction, including Compiled Works generated from
%   the Derived Work.  Derived Works distributed in this manner
%   by the Current Maintainer are considered to be updated
%   versions of the Work.
% \item If you are not the Current Maintainer of the Work, you
%   may modify your copy of the Work, thus creating a Derived
%   Work based on the Work, and compile this Derived Work, thus
%   creating a Compiled Work based on the Derived Work.
% \item \label{LPPL:item:conditions}If you are not the Current Maintainer of the Work, you
%   may distribute a Derived Work provided the following
%   conditions are met for every component of the Work unless
%   that component clearly states in the copyright notice that
%   it is exempt from that condition.  Only the Current
%   Maintainer is allowed to add such statements of exemption to
%   a component of the Work.
% \begin{enumerate}
% \item If a component of this Derived Work can be a direct
%       replacement for a component of the Work when that
%       component is used with the Base Interpreter, then,
%       wherever this component of the Work identifies itself to
%       the user when used interactively with that Base
%       Interpreter, the replacement component of this Derived
%       Work clearly and unambiguously identifies itself as a
%       modified version of this component to the user when used
%       interactively with that Base Interpreter.
% \item Every component of the Derived Work contains
%       prominent notices detailing the nature of the changes to
%       that component, or a prominent reference to another file
%       that is distributed as part of the Derived Work and that
%       contains a complete and accurate log of the
%       changes.
% \item No information in the Derived Work implies that any
%       persons, including (but not limited to) the authors of
%       the original version of the Work, provide any support,
%       including (but not limited to) the reporting and
%       handling of errors, to recipients of the Derived Work
%       unless those persons have stated explicitly that they do
%       provide such support for the Derived Work.
% \item You distribute at least one of the following with
%       the Derived Work:
% \begin{enumerate}
% \item A complete, unmodified copy of the Work; if your
%   distribution of a modified component is made by
%   offering access to copy the modified component from
%   a designated place, then offering equivalent access
%   to copy the Work from the same or some similar place
%   meets this condition, even though third parties are
%   not compelled to copy the Work along with the
%   modified component;
% \item Information that is sufficient to obtain a
%   complete, unmodified copy of the Work.
% \end{enumerate}
% \end{enumerate}
% \item If you are not the Current Maintainer of the Work, you
%   may distribute a Compiled Work generated from a Derived
%   Work, as long as the Derived Work is distributed to all
%   recipients of the Compiled Work, and as long as the
%   conditions of Clause~item~\ref{LPPL:item:conditions} above, above, are met
%   with regard to the Derived Work.
% \item The conditions above are not intended to prohibit, and
%   hence do not apply to, the modification, by any method, of
%   any component so that it becomes identical to an updated
%   version of that component of the Work as it is distributed
%   by the Current Maintainer under Clause~item~\ref{LPPL:item:currmaint} above, above.
% \item Distribution of the Work or any Derived Work in an
%   alternative format, where the Work or that Derived Work (in
%   whole or in part) is then produced by applying some process
%   to that format, does not relax or nullify any sections of
%   this license as they pertain to the results of applying that
%   process.
% \item % \begin{enumerate}
% \item A Derived Work may be distributed under a different
%       license provided that license itself honors the
%       conditions listed in Clause~item~\ref{LPPL:item:conditions} above above, in
%       regard to the Work, though it does not have to honor the
%       rest of the conditions in this license.
% \item If a Derived Work is distributed under a different
%       license, that Derived Work must provide sufficient
%       documentation as part of itself to allow each recipient
%       of that Derived Work to honor the restrictions in
%       Clause~item~\ref{LPPL:item:conditions} above above, concerning
%       changes from the Work.
% \end{enumerate}
% \item This license places no restrictions on works that are
%   unrelated to the Work, nor does this license place any
%   restrictions on aggregating such works with the Work by any
%   means.
% \item Nothing in this license is intended to, or may be used
%   to, prevent complete compliance by all parties with all
%   applicable laws.
% \end{enumerate}
% \subsection{No Warranty}\label{LPPL:Warranty}
% There is no warranty for the Work.  Except when otherwise
%       stated in writing, the Copyright Holder provides the Work
%       `as is', without warranty of any kind, either
%       expressed or implied, including, but not limited to, the implied
%       warranties of merchantability and fitness for a particular
%       purpose.  The entire risk as to the quality and performance of
%       the Work is with you.  Should the Work prove defective, you
%       assume the cost of all necessary servicing, repair, or
%       correction.\par
% In no event unless required by applicable law or agreed to
%       in writing will The Copyright Holder, or any author named in the
%       components of the Work, or any other party who may distribute
%       and/or modify the Work as permitted above, be liable to you for
%       damages, including any general, special, incidental or
%       consequential damages arising out of any use of the Work or out
%       of inability to use the Work (including, but not limited to,
%       loss of data, data being rendered inaccurate, or losses
%       sustained by anyone as a result of any failure of the Work to
%       operate with any other programs), even if the Copyright Holder
%       or said author or said other party has been advised of the
%       possibility of such damages.\par
% \subsection{Maintenance of The Work}\label{LPPL:Maintenance}
% The Work has the status `author-maintained'
%       if the Copyright Holder explicitly and prominently states near
%       the primary copyright notice in the Work that the Work can only
%       be maintained by the Copyright Holder or simply that it is
%       `author-maintained'.\par
% The Work has the status `maintained' if there
%       is a Current Maintainer who has indicated in the Work that they
%       are willing to receive error reports for the Work (for example,
%       by supplying a valid e-mail address). It is not required for the
%       Current Maintainer to acknowledge or act upon these error
%       reports.\par
% The Work changes from status `maintained' to
%       `unmaintained' if there is no Current Maintainer,
%       or the person stated to be Current Maintainer of the work cannot
%       be reached through the indicated means of communication for a
%       period of six months, and there are no other significant signs
%       of active maintenance.\par
% You can become the Current Maintainer of the Work by
%       agreement with any existing Current Maintainer to take over this
%       role.\par
% If the Work is unmaintained, you can become the Current
%       Maintainer of the Work through the following steps:\par
% \begin{enumerate}
% \item Make a reasonable attempt to trace the Current
%   Maintainer (and the Copyright Holder, if the two differ)
%   through the means of an Internet or similar search.
% \item If this search is successful, then enquire whether the
%   Work is still maintained.
% \begin{enumerate}
% \item If it is being maintained, then ask the Current
%       Maintainer to update their communication data within one
%       month.
% \item \label{LPPL:item:intention}If the search is unsuccessful or no action to resume
%       active maintenance is taken by the Current Maintainer,
%       then announce within the pertinent community your
%       intention to take over maintenance.  (If the Work is a
%       \LaTeX{} work, this could be done, for example, by
%       posting to \url{news:comp.text.tex}.)
% \end{enumerate}
% \item % \begin{enumerate}
% \item If the Current Maintainer is reachable and agrees to
%       pass maintenance of the Work to you, then this takes
%       effect immediately upon announcement.
% \item \label{LPPL:item:announce}If the Current Maintainer is not reachable and the
%       Copyright Holder agrees that maintenance of the Work be
%       passed to you, then this takes effect immediately upon
%       announcement.
% \end{enumerate}
% \item \label{LPPL:item:change}If you make an `intention announcement'
%   as described in~item~\ref{LPPL:item:intention} above above and after three
%   months your intention is challenged neither by the Current
%   Maintainer nor by the Copyright Holder nor by other people,
%   then you may arrange for the Work to be changed so as to
%   name you as the (new) Current Maintainer.
% \item If the previously unreachable Current Maintainer becomes
%   reachable once more within three months of a change
%   completed under the terms of~item~\ref{LPPL:item:announce} above
%   or~item~\ref{LPPL:item:change} above, then that
%   Current
%   Maintainer must become or remain the Current Maintainer upon
%   request provided they then update their communication data
%   within one month.
% \end{enumerate}
% A change in the Current Maintainer does not, of itself,
%       alter the fact that the Work is distributed under the
%       {\small LPPL} license.\par
% If you become the Current Maintainer of the Work, you should
%       immediately provide, within the Work, a prominent and
%       unambiguous statement of your status as Current Maintainer.  You
%       should also announce your new status to the same pertinent
%       community as in~item~\ref{LPPL:item:intention} above
%       above.\par
% \subsection{Whether and How to Distribute Works under This
%       License}\label{LPPL:Distribute}
% This section contains important instructions, examples, and
%       recommendations for authors who are considering distributing
%       their works under this license.  These authors are addressed as
%       `you' in this section.\par
% \subsubsection{Choosing This License or Another License}\label{LPPL:Choosing}
% If for any part of your work you want or need to use
% \emph{distribution} conditions that differ
% significantly from those in this license, then do not refer to
% this license anywhere in your work but, instead, distribute
% your work under a different license. You may use the text of
% this license as a model for your own license, but your license
% should not refer to the {\small LPPL} or otherwise
% give the impression that your work is distributed under the
% {\small LPPL}.\par
% The document {\ttfamily{}modguide.tex} in the base \LaTeX{}
% distribution explains the motivation behind the conditions of
% this license.  It explains, for example, why distributing
% \LaTeX{} under the {\small GNU} General Public
% License ({\small GPL}) was considered inappropriate.
% Even if your work is unrelated to \LaTeX{}, the discussion in
% {\ttfamily{}modguide.tex} may still be
% relevant, and authors intending to distribute their works
% under any license are encouraged to read it.\par
% \subsubsection{A Recommendation on Modification Without
% Distribution}\label{LPPL:WithoutDistribution}
% It is wise never to modify a component of the Work, even
% for your own personal use, without also meeting the above
% conditions for distributing the modified component.  While you
% might intend that such modifications will never be
% distributed, often this will happen by accident~--- you may
% forget that you have modified that component; or it may not
% occur to you when allowing others to access the modified
% version that you are thus distributing it and violating the
% conditions of this license in ways that could have legal
% implications and, worse, cause problems for the community. It
% is therefore usually in your best interest to keep your copy
% of the Work identical with the public one.  Many works provide
% ways to control the behavior of that work without altering any
% of its licensed components.\par
% \subsubsection{How to Use This License}\label{LPPL:HowTo}
% To use this license, place in each of the components of
% your work both an explicit copyright notice including your
% name and the year the work was authored and/or last
% substantially modified.  Include also a statement that the
% distribution and/or modification of that component is
% constrained by the conditions in this license.\par
% Here is an example of such a notice and statement:\par
% \iffalse
%<*ignore>
% \fi
\begin{lstlisting}[language={[LaTeX]TeX}]
%%% pig.dtx
%%% Copyright 2005 M. Y. Name
%%
%% This work may be distributed and/or modified under the
%% conditions of the LaTeX Project Public License, either version 1.3
%% of this license or (at your option) any later version.
%% The latest version of this license is in
%%   http://www.latex-project.org/lppl.txt
%% and version 1.3 or later is part of all distributions of LaTeX
%% version 2005/12/01 or later.
%%
%% This work has the LPPL maintenance status `maintained'.
%% 
%% The Current Maintainer of this work is M. Y. Name.
%%
%% This work consists of the files pig.dtx and pig.ins
%% and the derived file pig.sty.
\end{lstlisting}
% \iffalse
%</ignore>
% \fi
% Given such a notice and statement in a file, the
% conditions given in this license document would apply, with
% the `Work' referring to the three files
% {\ttfamily{}pig.dtx}, {\ttfamily{}pig.ins}, and {\ttfamily{}pig.sty} (the last being generated
% from {\ttfamily{}pig.dtx} using {\ttfamily{}pig.ins}), the `Base
%   Interpreter' referring to any
% `\LaTeX{}-Format', and both `Copyright
%   Holder' and `Current Maintainer'
% referring to the person
% M.~Y.~Name\index{!}.\par
% If you do not want the Maintenance section of
% {\small LPPL} to apply to your Work, change
% `maintained' above into
% `author-maintained'. However, we recommend that
% you use `maintained' as the Maintenance
% section was added in order to ensure that your Work remains
% useful to the community even when you can no longer maintain
% and support it yourself.\par
% \subsubsection{Derived Works That Are Not Replacements}\label{LPPL:NotReplacements}
% Several clauses of the {\small LPPL} specify
% means to provide reliability and stability for the user
% community. They therefore concern themselves with the case
% that a Derived Work is intended to be used as a (compatible or
% incompatible) replacement of the original Work. If this is not
% the case (e.g., if a few lines of code are reused for a
% completely different task), then clauses 6b and 6d shall not
% apply.\par
% \subsubsection{Important Recommendations}\label{LPPL:Recommendations}
% \paragraph{Defining What Constitutes the Work}
% The {\small LPPL} requires that distributions
%   of the Work contain all the files of the Work.  It is
%   therefore important that you provide a way for the licensee
%   to determine which files constitute the Work.  This could,
%   for example, be achieved by explicitly listing all the files
%   of the Work near the copyright notice of each file or by
%   using a line such as:\par
% \iffalse
%<*ignore>
% \fi
\begin{lstlisting}[language={[LaTeX]TeX}]
%% This work consists of all files listed in manifest.txt.
\end{lstlisting}
% \iffalse
%</ignore>
% \fi
% in that place.  In the absence of an unequivocal list it
%   might be impossible for the licensee to determine what is
%   considered by you to comprise the Work and, in such a case,
%   the licensee would be entitled to make reasonable
%   conjectures as to which files comprise the Work.\par
% \iffalse
%<*testscript>
# This is a test script
echo Hello, World!
      
%</testscript>
% \fi
% \Finale


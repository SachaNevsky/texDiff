% \iffalse meta-comment, etc.
%%
%% Package `pst-fill.dtx'
%%
%% Denis Girou (CNRS/IDRIS - France) <Denis.Girou@idris.fr>
%% Herbert Voss <voss@pstricks.de>
%%
%% This program can be redistributed and/or modified under the terms
%% of the LaTeX Project Public License Distributed from CTAN archives
%% in directory macros/latex/base/lppl.txt.
%%
%% DESCRIPTION:
%%   `pst-fill' is a PSTricks package for filling and tiling areas 
%%
% \fi
% \changes{v1.01}{2007/03/10}{bugfix for incomplete ifx (hv)}
% \changes{v1.00}{2006/11/06}{use pst-xkey for extend keys (hv)}
% \changes{v0.99}{2004/08/17}{merge the VTeX and TeX versions (patch 4) (hv)}
% \changes{v0.98}{2004/06/22}{delete the Pst@Debug option and use the
%   the one from pstricks to prevent a clash with pst-gr3d (hv)}
% \changes{v0.97}{2001/10/09}{make it work with VTeX (mv)}
% \changes{v0.94}{1997/04/08}{With a \PstTiling macro defined (or "tiling" optional parameter
%   on \textbackslash usepackage[tiling]{pst-fill}), this file run exactly as
%   the original boxfill.tex file from Timothy, version 0.94,
%   except a correction in \textbackslash pst@ManualFillCycle to avoid a division by 0.
%   It's the default.}
% \changes{v0.93}{1997/04/07}{With a \textbackslash PstTiling macro defined (or "tiling" optional parameter
%   on \textbackslash usepackage[tiling]{pst-fill}) there are several add-ons
%   and changes to do `tiling' rather than `filling' in "automatic" mode :
%     - we fix the position of the beginning of tiling,
%     - we allow normally the framing of the area as expected, using
%       the line.... parameters
%     - we define move parameters fillmovex, fillmovey and fillmove,
%     - we define fillcyclex as previous fillcycle parameter, and add the
%       fillcycley and fillcycle (both fillcyclex and fillcycley) ones
%     - we can extend the tiling area using fillloopaddx, fillloopaddy and
%       fillloopadd parameters,
%     - we can debug and see the whole tiling area without clipping using
%       PstDebug parameter,
%     - for names consistancy, we can use fillangle in place of boxfillangle
%       and fillsize in place of boxfillsize,
%     - default value for fillsep is 0 and for fillsize is auto.}
%
% \DoNotIndex{\!,\",\#,\$,\%,\&,\',\(,\+,\*,\,,\-,\.,\/,\:,\;,\<,\=,\>,\?}
% \DoNotIndex{\@,\@B,\@K,\@cTq,\@f,\@fPl,\@ifnextchar,\@nameuse,\@oVk}
% \DoNotIndex{\[,\\,\],\^,\_,\ }
% \DoNotIndex{\^,\\^,\\\^,$\^$,$\\^$,$\\^$}
% \DoNotIndex{\0,\2,\4,\5,\6,\7,\8,}
% \DoNotIndex{\A,\a}
% \DoNotIndex{\B,\b,\Bc,\begin,\Bq,\Bqc}
% \DoNotIndex{\C,\c,\catcode,\cJA,\CodelineIndex,\csname}
% \DoNotIndex{\D,\def,\define@key,\Df,\divide,\DocInput,\documentclass,\pst@addfams}
% \DoNotIndex{\eCN,\edef,\else,\eHd,\eMcj,\EnableCrossrefs,\end,\endcsname}
% \DoNotIndex{\endCenterExample,\endExample,\endinput,\endpsclip}
% \DoNotIndex{\PrintIndex,\PrintChanges,\ProvidesFile}
% \DoNotIndex{\endpspicture,\endSideBySideExample,\Example}
% \DoNotIndex{\F,\f,\FdUrr,\fi,\filedate,\fileversion,\FV@Environment}
% \DoNotIndex{\FV@UseKeyValues,\FV@XRightMargin,\FVB@Example,\fvset}
% \DoNotIndex{\G,\g,\GetFileInfo,\gr,\GradientLoaded,\gsFKrbK@o,\gsj,\gsOX}
% \DoNotIndex{\hbadness,\hfuzz,\HLEmphasize,\HLMacro,\HLMacro@i}
% \DoNotIndex{\HLReverse,\HLReverse@i,\hqcu,\HqY}
% \DoNotIndex{\I,\i,\ifx,\input,\Ir,\IU}
% \DoNotIndex{\j,\jl,\JT,\JVodH}
% \DoNotIndex{\K,\k,\kfSlL}
% \DoNotIndex{\L,\let}
% \DoNotIndex{\message,\mHNa,\mIU}
% \DoNotIndex{\N,\nB,\newcmykcolor,\newdimen,\newif,\nW}
% \DoNotIndex{\O,\oCDJDo,\ocQhVI,\OnlyDescription,\oRKJ}
% \DoNotIndex{\P,\p,\ProvidesPackage,\psframe,\pslinewidth,\psset}
% \DoNotIndex{\PstAtCode,\PSTricksLoaded}
% \DoNotIndex{\q,\Qr,\qssRXq,\qu,\qXjFQp,\qYL}
% \DoNotIndex{\R,\r,\RecordChanges,\relax,\RlaYI,\rN,\Rp,\rp,\RPDXNn,\rput}
% \DoNotIndex{\S,\scalebox,\SgY,\SideBySide@Example,\SideBySideExample}
% \DoNotIndex{\SgY,\sk,\Sp,\space,\sZb}
% \DoNotIndex{\T,\the,\tw@}
% \DoNotIndex{\u,\UiSWGEf@,\uJi,\usepackage,\uVQdMM,\UYj}
% \DoNotIndex{\VerbatimEnvironment,\VerbatimInput,\VrC@}
% \DoNotIndex{\WhZ,\WjKCYb,\WNs}
% \DoNotIndex{\XkN,\XW}
% \DoNotIndex{\Z,\ZCM,\Ze}
% \DoNotIndex{\addtocounter,\advance,\alph,\arabic,\AtBeginDocument,\AtEndDocument}
% \DoNotIndex{\AtEndOfPackage,\begingroup,\bfseries,\bgroup,\box,\csname}
% \DoNotIndex{\else,\endcsname,\endgroup,\endinput,\expandafter,\fi}
% \DoNotIndex{\TeX,\z@,\p@,\@one,\xdef,\thr@@,\string,\sixt@@n,\reset,\or,\multiply,\repeat,\RequirePackage}
% \DoNotIndex{\@cclvi,\@ne,\@ehpa,\@nil,\copy,\dp,\global,\hbox,\hss,\ht,\ifodd,\ifdim,\ifcase,\kern}
% \DoNotIndex{\chardef,\loop,\leavevmode,\ifnum,\lower}
% \setcounter{IndexColumns}{2}
%
% ^^A To extend the height used for the text
%
% ^^A  Aligned labels in a description environment
%\newenvironment{Description}[1]{%
%\begin{list}{nothing}{\setlength{\leftmargin}{#1}
%\setlength{\labelwidth}{\leftmargin}\setlength{\labelsep}{1mm}}}
%{\end{list}}
%
% ^^A For macro names
%\DeclareRobustCommand\cs[1]{\texttt{\char`\\#1}}
%
%
% ^^A From ltugboat.cls
% ^^A For references
%\makeatletter
%\newcommand\acro[1]{\textsc{#1}\@}
%\def\CTAN{\acro{CTAN}}
%\let\texttub\textsl              % ^^A redefined in other situations
%\def\TUB{\texttub{TUGboat}}
%\def\TUG{\TeX\ \UG}
%\def\tug{\acro{TUG}}
%\def\UG{Users Group}
% ^^A For the bibliography 
%\let\@internalcite\cite
%\def\cite{\def\@citeseppen{-1000}%
%    \def\@cite##1##2{(##1\if@tempswa , ##2\fi)}%
%    \def\citeauthoryear##1##2##3{##1, ##3}\@internalcite}
%\def\etal{et\,al.\@}
%\newcommand\CTANdirectory[1]{\expandafter\urldef
%  \csname CTAN@#1\endcsname\path}
%\newcommand\CTANfile[1]{\expandafter\urldef
%  \csname CTAN@#1\endcsname\path}
%\newcommand\CTANref[1]{\expandafter\@setref\csname CTAN@#1\endcsname
%  \relax{#1}}
%\makeatother
% ^^A Define CTAN addresses 
%\CTANdirectory{mpattern}{graphics/metapost/macros/mpattern}
%\CTANdirectory{pstricks}{graphics/pstricks}
%\CTANdirectory{pst-fill.sty}{graphics/pstricks/latex/pst-fill.sty}
%\CTANdirectory{pst-fill}{graphics/pstricks/generic/pst-fill.tex}
%\CTANdirectory{Roegel}{graphics/metapost/contrib/macros/truchet}
%\CTANdirectory{xypic}{macros/generic/diagrams/xypic}
%
% ^^A Personal macros (D.G.)
% ^^A ----------------------
%
% ^^A Some colors used
%\definecolor{LemonChiffon}{rgb}{1.,0.98,0.8}
%\definecolor{LightBlue}   {rgb}{0.8,0.85,0.95}
%\definecolor{PaleGreen}   {rgb}{0.88,1,0.88}
%\definecolor{PeachPuff}   {rgb}{1.0,0.85,0.73}
%
% ^^A To define a unique string for TeX and LaTeX
%\newcommand{\AllTeX}{%
%{\rm(L\kern-.36em\raise.3ex\hbox{\sc a}\kern-.15em)%
%T\kern-.1667em\lower.7ex\hbox{E}\kern-.125emX}}
%
% ^^A Bibliography style
%\bibliographystyle{ltugbib}
%
% ^^A Name macros
%\newcommand{\FillPackage}{\textsf{`pst-fill'}}
%\newcommand{\XYpic}{%
%\leavevmode\hbox{\kern-.1em X\kern-.3em\lower.4ex\hbox{Y\kern-.15em}-pic}}
%
%\makeatletter
%
% ^^A Example environments
% ^^A (do not use in them the four JXYZ characters, that we will use
% ^^A as escape characters!)
%
% ^^A Default PSTricks parameters
%  \psset{dimen=middle}
%
% ^^A Translation in PSTricks from the one drawn by Emmanuel Chailloux and
% ^^A Guy Cousineau for the MLgraph system
% ^^A (see /ftp.ens.fr:/pub/unix/lang/MLgraph/version-2.1/MLgraph-refman.ps.gz)
% ^^A The kangaroo itself is reproduce from an original picture from Raoul Raba
% \newcommand{\DimX}{2.47}
% \newcommand{\DimY}{4.8}
% \newcommand{\DimXDivTwo}{1.235}
%
% \newcommand{\KangarooItself}[1]{%
% ^^A Body
% \pspolygon[fillstyle=solid,fillcolor=#1]%
%  (52.5,68)(55,72.5)(55.8,76.5)(56.8,79.8)(58.2,83)(60,85.8)(61.5,86.5)
% (64,87)(66,87.5)(67.8,87.3)(70,87)(71.5,87.3)(73,88)(74.7,88.5)
% (76,90.3)(77,91.5)(72.8,93.8)(69,96)(64.5,99)(59.4,103)(56.2,106.3)
% (53,110.5)(49.5,115.5)(47.2,119.9)(45.7,126)(43.2,123)(41.5,121)(37.5,125)
% (37,122.5)(36.8,120)(37,117)(37.6,113.5)(38.6,110)(40,106.3)(42,102.3)
%  (43.5,99.5)(45,97)(46.2,94)(46.8,91.7)(47.2,88)(47,83.5)(46.3,80.8)
%  (45.3,78.5)(42.5,76.5)(39.5,75.8)(36,75.9)(33,75.9)(29,76.2)(26,77)
%  (22.3,77.5)(18,78.4)(12.8,79.3)(8.6,80)(5.5,80.3)(3,80.5)(0,80)
%  (-5.2,78.5)(-9,76.3)(-11.2,74.8)(-13,72.5)(-16.5,68)(-16.5,68)(-19.5,62.5)
%  (-22,58)(-25.5,53)(-29,48.5)(-32.5,45)(-36,42)(-39,39.5)(-44,37)
%  (-49,35)(-51,34)(-53.5,34.5)(-55.5,36)(-56.5,38)(-56.5,40.5)(-55,41.5)
%  (-53.5,41)(-51.5,41)(-50.5,43)(-50.5,44.5)(-51,47)(-51.5,47.2)(-56.5,47)
%  (-58.5,46.5)(-60,44.7)(-62,42.3)(-63,39.5)(-63.5,36.3)(-63.5,33)(-63.1,29.5)
%  (-61.5,26)(-58,23.6)(-54,22.2)(-50.7,22)(-47.5,22)(-44.5,22.3)(-41,23.5)
%  (-36.8,25.8)(-33,28)(-28.5,31)(-23.4,35)(-20.2,38.3)(-17,42.5)(-13.5,47.5)
%  (-11.2,51.9)(-9.7,58)(-7.2,55)(-5.5,53)(-1.5,57)(-1,54.5)(-0.8,52)
%  (-1,49)(-1.6,45.5)(-2.6,42)(-4,38.3)(-6,34.3)(-7.5,31.5)(-9,29)
%  (-10.2,26)(-10.8,23.7)(-11.2,20)(-11,15.5)(-10.3,12.8)(-9.3,10.5)(-6.5,8.5)
%  (-3.5,7.8)(0,7.9)(3,7.9)(7,8.2)(10,9)(13.7,9.5)(18,10.4)
%  (23.2,11.3)(27.4,12)(30.5,12.3)(33,12.5)(36,12)(41.2,10.5)(45,8.3)
%  (47.2,6.8)(49,4.5)(52.5,0)(50,4.5)(49.2,8.5)(48.2,11.8)(46.8,15)
%  (45,17.8)(43.5,18.5)(41,19)(39,19.5)(37.2,19.3)(35,19)(33.5,19.3)
%  (32,20)(30.3,20.5)(29,22.3)(28,23.5)(28,23.5)(24.5,22.3)(21.5,22)
%  (18.3,22)(15,22.2)(11,23.6)(7.5,26)(5.9,29.5)(5.5,33)(5.5,36.3)
%  (6,39.5)(7,42.3)(9,44.7)(10.5,46.5)(12.5,47)(17.5,47.2)(18,47)
%  (18.5,44.5)(18.5,43)(17.5,41)(15.5,41)(14,41.5)(12.5,40.5)(12.5,38)
%  (13.5,36)(15.5,34.5)(18,34)(20,35)(25,37)(30,39.5)(33,42)
%  (36.5,45)(40,48.5)(43.5,53)(47,58)(49.5,62.5)(52.5,68)
% ^^A Eye
% \pscircle*[linecolor=white](58.2,98.3){2\psxunit}
% \pscircle*(58.2,97.3){\psxunit}
% ^^A Mouth
% \psline(71.5,88)(70,89.3)(68.5,90.3)(67,91.9)
% ^^A Tear
% \psline(42,121)(45,118)(47,115.3)(48.5,112.7)(50,110)(51.8,106.5)
%       (52.5,103.7)(53,100.5)
% \pspolygon(41.2,115.8)(43.2,114.7)(45,112.5)(47,109.8)(48,107)(49.5,104.2)%
%       (50.5,101.6)(51,98.5)(47.7,100.6)(46,102.2)(44.8,104)(43.5,106)
%       (42.5,108)(41.7,110.5)(41,113.2)}
%
% \newcommand{\Kangaroo}[1]{%
%   \begin{pspicture}(\DimX,\DimY)
%   \psset{unit=0.035278}
%   \KangarooItself{#1}
%   \end{pspicture}}
%
% \newcommand{\KangarooPstChart}[1]{{%
%   \psset{xunit=0.006784,yunit=0.00735,linewidth=0.01}
%   \begin{pspicture}(-65.5,0)(82,126)
%     \KangarooItself{#1}
%   \end{pspicture}}}
%
%
% ^^A For the possible index and changes log
% \setlength{\columnseprule}{0.6pt}
%
% ^^A Beginning of the documentation itself
%\title{\texttt{pst-fill}\\A PSTricks package for filling and tiling areas}
%\author{Timothy Van Zandt\thanks{\protect\url{tvz@econ.insead.fr}. (documentation by
% Denis Girou (\protect\url{Denis.Girou@idris.fr}) and Herbert Vo\ss (\protect\url{hvoss@tug.org}).}}
%
%\date{\shortstack{\today --- Version 1.00\\
%                  {\small Documentation revised \today}}}
% \maketitle
% \tableofcontents
%
%\begin{abstract}
%  \FillPackage{} is a PSTricks \cite{vanZandt93},\cite{Girou94},\cite{vanZandtGirou94}, 
%\cite{Hoenig97},\cite{LGC97} package to draw easily
%  various kinds of filling and tiling of areas. It is also a good example of
%  the great power and flexibility of PSTricks, as in fact it is a very short
%  program (it body is around 200~lines long) but nevertheless really powerful.
%
%  \hspace{5mm} It was written in 1994 by Timothy \textsc{van Zandt} but
%  publicly available only in PSTricks 97 and without any documentation.
%  We describe here the version \emph{97 patch 2} of December 12, 1997, which
%  is the original one modified by myself to manage \emph{tilings} in the
%  so-called \emph{automatic} mode. This article would like to serve both of
%  reference manual and of user's guide.
%
%This package is available on \CTAN{} in the
%  \texttt{graphics/pstricks} directory (files \texttt{latex/pst-fill.sty} and
%  \texttt{generic/pst-fill.tex}).
%\end{abstract}
%
%\section{Introduction}
%
%  Here we will refer as \emph{filling} as the operation which consist to fill
%a defined area by a pattern (or a composition of patterns). We will refer as
%\emph{tiling} as the operation which consist to do the same thing, but with
%the control of the starting point, which is here the upper left corner.
%The pattern is positioned relatively to this point. This make an essential
%difference between the two modes, as without control of the starting point we
%can't draw \emph{tilings} (sometimes  called \emph{tesselations}) as used in
%many fields of Art and Science%
%\footnote{For an extensive presentation of tilings, in their history and usage
%in many fields, see the reference book \cite{GS87}.
%
%  In the \TeX{} world, few work was done on tilings. You can look at the
%\emph{tile} extension of the \XYpic{} package \cite{XYpic}, at the articles of
%Kees \textsc{van der Laan} \cite[paragraph 7]{LAAN96} (the tiling was in
%fact directly done in PostScript) and \cite{LAAN97}, at the \MP{} program
%(available on \CTANref{Roegel}) by Denis \textsc{Roegel} for the
%\textsc{Truchet} contest in 1995 \cite{EsperetGirou98} and at the \MP{}
%package \cite{Bolek98} to draw patterns, which have a strong connection with
%tilings.}.
%
%  Nevertheless, as tilings are a wide and difficult field in mathematics, this
%package is limited to simple ones, mainly \emph{monohedral} tilings with one
%prototile (which can be composite, see section \ref{sec:KindTiles}). With some
%experience and wiliness we can do more and obtained easily rather
%sophisticated results, but obviously hyperbolic tilings like the famous
%\textsc{Escher} ones or aperiodic tilings like the \textsc{Penrose} ones are
%not in the capabilities of this package. For more complex needs, we must used
%low level and more painfull technics, with the basic \cs{multido}
%and \cs{multirput} macros.
%
%\section{Package history and description of it two different modes}
%
%  As already said, this package was written in 1994 by Timothy \textsc{van
%Zandt}. Two modes were defined, called respectively \emph{manual} and
%\emph{automatic}. For both, the pattern is generated on contiguous positions in
%a rather large area which include the region to fill, later cut to the
%required dimensions by clipping mechanism. In the first mode, the pattern is
%explicitely inserted in the PostScript file each time. In the second one, the
%result is the same but with an unique explicit insertion of the pattern and a
%repetition done by PostScript. Nevertheless, in this method, the control of
%the starting point was loosed, so it allowed only to \emph{fill} a region and
%not to \emph{tile} it.
%
%  See the difference between the two modes, \emph{tiling}:
% {\psset{unit=0.5cm}%
% \psboxfill{\begin{pspicture}(1,1)\psframe[dimen=middle](1,1)\end{pspicture}}
% \begin{pspicture}(3,3.3)
%   \psframe[fillstyle=boxfill](3,3)
% \end{pspicture}}
% and \emph{filling}:
%{%
% \makeatletter
%\pst@def{BoxFill}<
%  gsave
%    gsave \tx@STV CM grestore dtransform CM idtransform
%    abs /h ED abs /w ED
%    pathbbox
%    h div round 2 add cvi /y2 ED
%    w div round 2 add cvi /x2 ED
%    h div round 2 sub cvi /y1 ED
%    w div round 2 sub cvi /x1 ED
%    /y2 y2 y1 sub def
%    /x2 x2 x1 sub def
%    CP
%    y1 h mul sub neg /y1 ED
%    x1 w mul sub neg /x1 ED
%    clip
%    y2 {
%      /x x1 def
%      x2 {
%        save CP x y1 T moveto Box restore
%        /x x w add def
%      } repeat
%      /y1 y1 h add def
%    } repeat
% currentpoint currentfont grestore setfont moveto>
% \makeatother
%
% \psset{unit=0.5}
% \psboxfill{\begin{pspicture}(1,1)\psframe[dimen=middle](1,1)\end{pspicture}}
% \begin{pspicture}(3,3.3)
%   \psframe[fillstyle=boxfill](3,3)
% \end{pspicture}
% or
% \begin{pspicture}(3,3.3)
%   \psframe[fillstyle=boxfill](3,3)
% \end{pspicture}
%}
%as we can see that initial position is arbitrary and dependent of
%the current point.
%
%
% It's clear that usage of filling is very restrictive comparing to tiling,
%as desired effects required very often the possibility to control the starting 
%point. So, this mode was of limited interest, but unfortunately the
%\emph{manual} one has the very big disadvantage to require very huge amounts
%of ressources, mainly in disk space and consequently in printing time.
%A small tiling can require sometimes several megabytes in \emph{manual} mode!
%So, it was very often not really usable in practice.
%
%It is why I modified the code, to allow tilings in \emph{automatic} mode,
%controlling in this mode too the starting point. And most of the time, that is
%to say if some special options are not used, the tiling is done exactly in the
%region described, which make it faster. So there is no more reason to use the
%\emph{manual} mode, apart very special cases where \emph{automatic} one cannot
%work, as explained later -- currently, I know only one case.
%
%  To load this modified \emph{automatic} mode, with \LaTeX{} use
%simply:\newline 
%\verb+\usepackage[tiling]{pst-fill}+\newline
%and in plain \TeX{} after:\newline
%\verb+% \iffalse meta-comment, etc.
%%
%% Package `pst-fill.dtx'
%%
%% Denis Girou (CNRS/IDRIS - France) <Denis.Girou@idris.fr>
%% Herbert Voss <voss@pstricks.de>
%%
%% This program can be redistributed and/or modified under the terms
%% of the LaTeX Project Public License Distributed from CTAN archives
%% in directory macros/latex/base/lppl.txt.
%%
%% DESCRIPTION:
%%   `pst-fill' is a PSTricks package for filling and tiling areas 
%%
% \fi
% \changes{v1.01}{2007/03/10}{bugfix for incomplete ifx (hv)}
% \changes{v1.00}{2006/11/06}{use pst-xkey for extend keys (hv)}
% \changes{v0.99}{2004/08/17}{merge the VTeX and TeX versions (patch 4) (hv)}
% \changes{v0.98}{2004/06/22}{delete the Pst@Debug option and use the
%   the one from pstricks to prevent a clash with pst-gr3d (hv)}
% \changes{v0.97}{2001/10/09}{make it work with VTeX (mv)}
% \changes{v0.94}{1997/04/08}{With a \PstTiling macro defined (or "tiling" optional parameter
%   on \textbackslash usepackage[tiling]{pst-fill}), this file run exactly as
%   the original boxfill.tex file from Timothy, version 0.94,
%   except a correction in \textbackslash pst@ManualFillCycle to avoid a division by 0.
%   It's the default.}
% \changes{v0.93}{1997/04/07}{With a \textbackslash PstTiling macro defined (or "tiling" optional parameter
%   on \textbackslash usepackage[tiling]{pst-fill}) there are several add-ons
%   and changes to do `tiling' rather than `filling' in "automatic" mode :
%     - we fix the position of the beginning of tiling,
%     - we allow normally the framing of the area as expected, using
%       the line.... parameters
%     - we define move parameters fillmovex, fillmovey and fillmove,
%     - we define fillcyclex as previous fillcycle parameter, and add the
%       fillcycley and fillcycle (both fillcyclex and fillcycley) ones
%     - we can extend the tiling area using fillloopaddx, fillloopaddy and
%       fillloopadd parameters,
%     - we can debug and see the whole tiling area without clipping using
%       PstDebug parameter,
%     - for names consistancy, we can use fillangle in place of boxfillangle
%       and fillsize in place of boxfillsize,
%     - default value for fillsep is 0 and for fillsize is auto.}
%
% \DoNotIndex{\!,\",\#,\$,\%,\&,\',\(,\+,\*,\,,\-,\.,\/,\:,\;,\<,\=,\>,\?}
% \DoNotIndex{\@,\@B,\@K,\@cTq,\@f,\@fPl,\@ifnextchar,\@nameuse,\@oVk}
% \DoNotIndex{\[,\\,\],\^,\_,\ }
% \DoNotIndex{\^,\\^,\\\^,$\^$,$\\^$,$\\^$}
% \DoNotIndex{\0,\2,\4,\5,\6,\7,\8,}
% \DoNotIndex{\A,\a}
% \DoNotIndex{\B,\b,\Bc,\begin,\Bq,\Bqc}
% \DoNotIndex{\C,\c,\catcode,\cJA,\CodelineIndex,\csname}
% \DoNotIndex{\D,\def,\define@key,\Df,\divide,\DocInput,\documentclass,\pst@addfams}
% \DoNotIndex{\eCN,\edef,\else,\eHd,\eMcj,\EnableCrossrefs,\end,\endcsname}
% \DoNotIndex{\endCenterExample,\endExample,\endinput,\endpsclip}
% \DoNotIndex{\PrintIndex,\PrintChanges,\ProvidesFile}
% \DoNotIndex{\endpspicture,\endSideBySideExample,\Example}
% \DoNotIndex{\F,\f,\FdUrr,\fi,\filedate,\fileversion,\FV@Environment}
% \DoNotIndex{\FV@UseKeyValues,\FV@XRightMargin,\FVB@Example,\fvset}
% \DoNotIndex{\G,\g,\GetFileInfo,\gr,\GradientLoaded,\gsFKrbK@o,\gsj,\gsOX}
% \DoNotIndex{\hbadness,\hfuzz,\HLEmphasize,\HLMacro,\HLMacro@i}
% \DoNotIndex{\HLReverse,\HLReverse@i,\hqcu,\HqY}
% \DoNotIndex{\I,\i,\ifx,\input,\Ir,\IU}
% \DoNotIndex{\j,\jl,\JT,\JVodH}
% \DoNotIndex{\K,\k,\kfSlL}
% \DoNotIndex{\L,\let}
% \DoNotIndex{\message,\mHNa,\mIU}
% \DoNotIndex{\N,\nB,\newcmykcolor,\newdimen,\newif,\nW}
% \DoNotIndex{\O,\oCDJDo,\ocQhVI,\OnlyDescription,\oRKJ}
% \DoNotIndex{\P,\p,\ProvidesPackage,\psframe,\pslinewidth,\psset}
% \DoNotIndex{\PstAtCode,\PSTricksLoaded}
% \DoNotIndex{\q,\Qr,\qssRXq,\qu,\qXjFQp,\qYL}
% \DoNotIndex{\R,\r,\RecordChanges,\relax,\RlaYI,\rN,\Rp,\rp,\RPDXNn,\rput}
% \DoNotIndex{\S,\scalebox,\SgY,\SideBySide@Example,\SideBySideExample}
% \DoNotIndex{\SgY,\sk,\Sp,\space,\sZb}
% \DoNotIndex{\T,\the,\tw@}
% \DoNotIndex{\u,\UiSWGEf@,\uJi,\usepackage,\uVQdMM,\UYj}
% \DoNotIndex{\VerbatimEnvironment,\VerbatimInput,\VrC@}
% \DoNotIndex{\WhZ,\WjKCYb,\WNs}
% \DoNotIndex{\XkN,\XW}
% \DoNotIndex{\Z,\ZCM,\Ze}
% \DoNotIndex{\addtocounter,\advance,\alph,\arabic,\AtBeginDocument,\AtEndDocument}
% \DoNotIndex{\AtEndOfPackage,\begingroup,\bfseries,\bgroup,\box,\csname}
% \DoNotIndex{\else,\endcsname,\endgroup,\endinput,\expandafter,\fi}
% \DoNotIndex{\TeX,\z@,\p@,\@one,\xdef,\thr@@,\string,\sixt@@n,\reset,\or,\multiply,\repeat,\RequirePackage}
% \DoNotIndex{\@cclvi,\@ne,\@ehpa,\@nil,\copy,\dp,\global,\hbox,\hss,\ht,\ifodd,\ifdim,\ifcase,\kern}
% \DoNotIndex{\chardef,\loop,\leavevmode,\ifnum,\lower}
% \setcounter{IndexColumns}{2}
%
% ^^A To extend the height used for the text
%
% ^^A  Aligned labels in a description environment
%\newenvironment{Description}[1]{%
%\begin{list}{nothing}{\setlength{\leftmargin}{#1}
%\setlength{\labelwidth}{\leftmargin}\setlength{\labelsep}{1mm}}}
%{\end{list}}
%
% ^^A For macro names
%\DeclareRobustCommand\cs[1]{\texttt{\char`\\#1}}
%
%
% ^^A From ltugboat.cls
% ^^A For references
%\makeatletter
%\newcommand\acro[1]{\textsc{#1}\@}
%\def\CTAN{\acro{CTAN}}
%\let\texttub\textsl              % ^^A redefined in other situations
%\def\TUB{\texttub{TUGboat}}
%\def\TUG{\TeX\ \UG}
%\def\tug{\acro{TUG}}
%\def\UG{Users Group}
% ^^A For the bibliography 
%\let\@internalcite\cite
%\def\cite{\def\@citeseppen{-1000}%
%    \def\@cite##1##2{(##1\if@tempswa , ##2\fi)}%
%    \def\citeauthoryear##1##2##3{##1, ##3}\@internalcite}
%\def\etal{et\,al.\@}
%\newcommand\CTANdirectory[1]{\expandafter\urldef
%  \csname CTAN@#1\endcsname\path}
%\newcommand\CTANfile[1]{\expandafter\urldef
%  \csname CTAN@#1\endcsname\path}
%\newcommand\CTANref[1]{\expandafter\@setref\csname CTAN@#1\endcsname
%  \relax{#1}}
%\makeatother
% ^^A Define CTAN addresses 
%\CTANdirectory{mpattern}{graphics/metapost/macros/mpattern}
%\CTANdirectory{pstricks}{graphics/pstricks}
%\CTANdirectory{pst-fill.sty}{graphics/pstricks/latex/pst-fill.sty}
%\CTANdirectory{pst-fill}{graphics/pstricks/generic/pst-fill.tex}
%\CTANdirectory{Roegel}{graphics/metapost/contrib/macros/truchet}
%\CTANdirectory{xypic}{macros/generic/diagrams/xypic}
%
% ^^A Personal macros (D.G.)
% ^^A ----------------------
%
% ^^A Some colors used
%\definecolor{LemonChiffon}{rgb}{1.,0.98,0.8}
%\definecolor{LightBlue}   {rgb}{0.8,0.85,0.95}
%\definecolor{PaleGreen}   {rgb}{0.88,1,0.88}
%\definecolor{PeachPuff}   {rgb}{1.0,0.85,0.73}
%
% ^^A To define a unique string for TeX and LaTeX
%\newcommand{\AllTeX}{%
%{\rm(L\kern-.36em\raise.3ex\hbox{\sc a}\kern-.15em)%
%T\kern-.1667em\lower.7ex\hbox{E}\kern-.125emX}}
%
% ^^A Bibliography style
%\bibliographystyle{ltugbib}
%
% ^^A Name macros
%\newcommand{\FillPackage}{\textsf{`pst-fill'}}
%\newcommand{\XYpic}{%
%\leavevmode\hbox{\kern-.1em X\kern-.3em\lower.4ex\hbox{Y\kern-.15em}-pic}}
%
%\makeatletter
%
% ^^A Example environments
% ^^A (do not use in them the four JXYZ characters, that we will use
% ^^A as escape characters!)
%
% ^^A Default PSTricks parameters
%  \psset{dimen=middle}
%
% ^^A Translation in PSTricks from the one drawn by Emmanuel Chailloux and
% ^^A Guy Cousineau for the MLgraph system
% ^^A (see /ftp.ens.fr:/pub/unix/lang/MLgraph/version-2.1/MLgraph-refman.ps.gz)
% ^^A The kangaroo itself is reproduce from an original picture from Raoul Raba
% \newcommand{\DimX}{2.47}
% \newcommand{\DimY}{4.8}
% \newcommand{\DimXDivTwo}{1.235}
%
% \newcommand{\KangarooItself}[1]{%
% ^^A Body
% \pspolygon[fillstyle=solid,fillcolor=#1]%
%  (52.5,68)(55,72.5)(55.8,76.5)(56.8,79.8)(58.2,83)(60,85.8)(61.5,86.5)
% (64,87)(66,87.5)(67.8,87.3)(70,87)(71.5,87.3)(73,88)(74.7,88.5)
% (76,90.3)(77,91.5)(72.8,93.8)(69,96)(64.5,99)(59.4,103)(56.2,106.3)
% (53,110.5)(49.5,115.5)(47.2,119.9)(45.7,126)(43.2,123)(41.5,121)(37.5,125)
% (37,122.5)(36.8,120)(37,117)(37.6,113.5)(38.6,110)(40,106.3)(42,102.3)
%  (43.5,99.5)(45,97)(46.2,94)(46.8,91.7)(47.2,88)(47,83.5)(46.3,80.8)
%  (45.3,78.5)(42.5,76.5)(39.5,75.8)(36,75.9)(33,75.9)(29,76.2)(26,77)
%  (22.3,77.5)(18,78.4)(12.8,79.3)(8.6,80)(5.5,80.3)(3,80.5)(0,80)
%  (-5.2,78.5)(-9,76.3)(-11.2,74.8)(-13,72.5)(-16.5,68)(-16.5,68)(-19.5,62.5)
%  (-22,58)(-25.5,53)(-29,48.5)(-32.5,45)(-36,42)(-39,39.5)(-44,37)
%  (-49,35)(-51,34)(-53.5,34.5)(-55.5,36)(-56.5,38)(-56.5,40.5)(-55,41.5)
%  (-53.5,41)(-51.5,41)(-50.5,43)(-50.5,44.5)(-51,47)(-51.5,47.2)(-56.5,47)
%  (-58.5,46.5)(-60,44.7)(-62,42.3)(-63,39.5)(-63.5,36.3)(-63.5,33)(-63.1,29.5)
%  (-61.5,26)(-58,23.6)(-54,22.2)(-50.7,22)(-47.5,22)(-44.5,22.3)(-41,23.5)
%  (-36.8,25.8)(-33,28)(-28.5,31)(-23.4,35)(-20.2,38.3)(-17,42.5)(-13.5,47.5)
%  (-11.2,51.9)(-9.7,58)(-7.2,55)(-5.5,53)(-1.5,57)(-1,54.5)(-0.8,52)
%  (-1,49)(-1.6,45.5)(-2.6,42)(-4,38.3)(-6,34.3)(-7.5,31.5)(-9,29)
%  (-10.2,26)(-10.8,23.7)(-11.2,20)(-11,15.5)(-10.3,12.8)(-9.3,10.5)(-6.5,8.5)
%  (-3.5,7.8)(0,7.9)(3,7.9)(7,8.2)(10,9)(13.7,9.5)(18,10.4)
%  (23.2,11.3)(27.4,12)(30.5,12.3)(33,12.5)(36,12)(41.2,10.5)(45,8.3)
%  (47.2,6.8)(49,4.5)(52.5,0)(50,4.5)(49.2,8.5)(48.2,11.8)(46.8,15)
%  (45,17.8)(43.5,18.5)(41,19)(39,19.5)(37.2,19.3)(35,19)(33.5,19.3)
%  (32,20)(30.3,20.5)(29,22.3)(28,23.5)(28,23.5)(24.5,22.3)(21.5,22)
%  (18.3,22)(15,22.2)(11,23.6)(7.5,26)(5.9,29.5)(5.5,33)(5.5,36.3)
%  (6,39.5)(7,42.3)(9,44.7)(10.5,46.5)(12.5,47)(17.5,47.2)(18,47)
%  (18.5,44.5)(18.5,43)(17.5,41)(15.5,41)(14,41.5)(12.5,40.5)(12.5,38)
%  (13.5,36)(15.5,34.5)(18,34)(20,35)(25,37)(30,39.5)(33,42)
%  (36.5,45)(40,48.5)(43.5,53)(47,58)(49.5,62.5)(52.5,68)
% ^^A Eye
% \pscircle*[linecolor=white](58.2,98.3){2\psxunit}
% \pscircle*(58.2,97.3){\psxunit}
% ^^A Mouth
% \psline(71.5,88)(70,89.3)(68.5,90.3)(67,91.9)
% ^^A Tear
% \psline(42,121)(45,118)(47,115.3)(48.5,112.7)(50,110)(51.8,106.5)
%       (52.5,103.7)(53,100.5)
% \pspolygon(41.2,115.8)(43.2,114.7)(45,112.5)(47,109.8)(48,107)(49.5,104.2)%
%       (50.5,101.6)(51,98.5)(47.7,100.6)(46,102.2)(44.8,104)(43.5,106)
%       (42.5,108)(41.7,110.5)(41,113.2)}
%
% \newcommand{\Kangaroo}[1]{%
%   \begin{pspicture}(\DimX,\DimY)
%   \psset{unit=0.035278}
%   \KangarooItself{#1}
%   \end{pspicture}}
%
% \newcommand{\KangarooPstChart}[1]{{%
%   \psset{xunit=0.006784,yunit=0.00735,linewidth=0.01}
%   \begin{pspicture}(-65.5,0)(82,126)
%     \KangarooItself{#1}
%   \end{pspicture}}}
%
%
% ^^A For the possible index and changes log
% \setlength{\columnseprule}{0.6pt}
%
% ^^A Beginning of the documentation itself
%\title{\texttt{pst-fill}\\A PSTricks package for filling and tiling areas}
%\author{Timothy Van Zandt\thanks{\protect\url{tvz@econ.insead.fr}. (documentation by
% Denis Girou (\protect\url{Denis.Girou@idris.fr}) and Herbert Vo\ss (\protect\url{hvoss@tug.org}).}}
%
%\date{\shortstack{\today --- Version 1.00\\
%                  {\small Documentation revised \today}}}
% \maketitle
% \tableofcontents
%
%\begin{abstract}
%  \FillPackage{} is a PSTricks \cite{vanZandt93},\cite{Girou94},\cite{vanZandtGirou94}, 
%\cite{Hoenig97},\cite{LGC97} package to draw easily
%  various kinds of filling and tiling of areas. It is also a good example of
%  the great power and flexibility of PSTricks, as in fact it is a very short
%  program (it body is around 200~lines long) but nevertheless really powerful.
%
%  \hspace{5mm} It was written in 1994 by Timothy \textsc{van Zandt} but
%  publicly available only in PSTricks 97 and without any documentation.
%  We describe here the version \emph{97 patch 2} of December 12, 1997, which
%  is the original one modified by myself to manage \emph{tilings} in the
%  so-called \emph{automatic} mode. This article would like to serve both of
%  reference manual and of user's guide.
%
%This package is available on \CTAN{} in the
%  \texttt{graphics/pstricks} directory (files \texttt{latex/pst-fill.sty} and
%  \texttt{generic/pst-fill.tex}).
%\end{abstract}
%
%\section{Introduction}
%
%  Here we will refer as \emph{filling} as the operation which consist to fill
%a defined area by a pattern (or a composition of patterns). We will refer as
%\emph{tiling} as the operation which consist to do the same thing, but with
%the control of the starting point, which is here the upper left corner.
%The pattern is positioned relatively to this point. This make an essential
%difference between the two modes, as without control of the starting point we
%can't draw \emph{tilings} (sometimes  called \emph{tesselations}) as used in
%many fields of Art and Science%
%\footnote{For an extensive presentation of tilings, in their history and usage
%in many fields, see the reference book \cite{GS87}.
%
%  In the \TeX{} world, few work was done on tilings. You can look at the
%\emph{tile} extension of the \XYpic{} package \cite{XYpic}, at the articles of
%Kees \textsc{van der Laan} \cite[paragraph 7]{LAAN96} (the tiling was in
%fact directly done in PostScript) and \cite{LAAN97}, at the \MP{} program
%(available on \CTANref{Roegel}) by Denis \textsc{Roegel} for the
%\textsc{Truchet} contest in 1995 \cite{EsperetGirou98} and at the \MP{}
%package \cite{Bolek98} to draw patterns, which have a strong connection with
%tilings.}.
%
%  Nevertheless, as tilings are a wide and difficult field in mathematics, this
%package is limited to simple ones, mainly \emph{monohedral} tilings with one
%prototile (which can be composite, see section \ref{sec:KindTiles}). With some
%experience and wiliness we can do more and obtained easily rather
%sophisticated results, but obviously hyperbolic tilings like the famous
%\textsc{Escher} ones or aperiodic tilings like the \textsc{Penrose} ones are
%not in the capabilities of this package. For more complex needs, we must used
%low level and more painfull technics, with the basic \cs{multido}
%and \cs{multirput} macros.
%
%\section{Package history and description of it two different modes}
%
%  As already said, this package was written in 1994 by Timothy \textsc{van
%Zandt}. Two modes were defined, called respectively \emph{manual} and
%\emph{automatic}. For both, the pattern is generated on contiguous positions in
%a rather large area which include the region to fill, later cut to the
%required dimensions by clipping mechanism. In the first mode, the pattern is
%explicitely inserted in the PostScript file each time. In the second one, the
%result is the same but with an unique explicit insertion of the pattern and a
%repetition done by PostScript. Nevertheless, in this method, the control of
%the starting point was loosed, so it allowed only to \emph{fill} a region and
%not to \emph{tile} it.
%
%  See the difference between the two modes, \emph{tiling}:
% {\psset{unit=0.5cm}%
% \psboxfill{\begin{pspicture}(1,1)\psframe[dimen=middle](1,1)\end{pspicture}}
% \begin{pspicture}(3,3.3)
%   \psframe[fillstyle=boxfill](3,3)
% \end{pspicture}}
% and \emph{filling}:
%{%
% \makeatletter
%\pst@def{BoxFill}<
%  gsave
%    gsave \tx@STV CM grestore dtransform CM idtransform
%    abs /h ED abs /w ED
%    pathbbox
%    h div round 2 add cvi /y2 ED
%    w div round 2 add cvi /x2 ED
%    h div round 2 sub cvi /y1 ED
%    w div round 2 sub cvi /x1 ED
%    /y2 y2 y1 sub def
%    /x2 x2 x1 sub def
%    CP
%    y1 h mul sub neg /y1 ED
%    x1 w mul sub neg /x1 ED
%    clip
%    y2 {
%      /x x1 def
%      x2 {
%        save CP x y1 T moveto Box restore
%        /x x w add def
%      } repeat
%      /y1 y1 h add def
%    } repeat
% currentpoint currentfont grestore setfont moveto>
% \makeatother
%
% \psset{unit=0.5}
% \psboxfill{\begin{pspicture}(1,1)\psframe[dimen=middle](1,1)\end{pspicture}}
% \begin{pspicture}(3,3.3)
%   \psframe[fillstyle=boxfill](3,3)
% \end{pspicture}
% or
% \begin{pspicture}(3,3.3)
%   \psframe[fillstyle=boxfill](3,3)
% \end{pspicture}
%}
%as we can see that initial position is arbitrary and dependent of
%the current point.
%
%
% It's clear that usage of filling is very restrictive comparing to tiling,
%as desired effects required very often the possibility to control the starting 
%point. So, this mode was of limited interest, but unfortunately the
%\emph{manual} one has the very big disadvantage to require very huge amounts
%of ressources, mainly in disk space and consequently in printing time.
%A small tiling can require sometimes several megabytes in \emph{manual} mode!
%So, it was very often not really usable in practice.
%
%It is why I modified the code, to allow tilings in \emph{automatic} mode,
%controlling in this mode too the starting point. And most of the time, that is
%to say if some special options are not used, the tiling is done exactly in the
%region described, which make it faster. So there is no more reason to use the
%\emph{manual} mode, apart very special cases where \emph{automatic} one cannot
%work, as explained later -- currently, I know only one case.
%
%  To load this modified \emph{automatic} mode, with \LaTeX{} use
%simply:\newline 
%\verb+\usepackage[tiling]{pst-fill}+\newline
%and in plain \TeX{} after:\newline
%\verb+% \iffalse meta-comment, etc.
%%
%% Package `pst-fill.dtx'
%%
%% Denis Girou (CNRS/IDRIS - France) <Denis.Girou@idris.fr>
%% Herbert Voss <voss@pstricks.de>
%%
%% This program can be redistributed and/or modified under the terms
%% of the LaTeX Project Public License Distributed from CTAN archives
%% in directory macros/latex/base/lppl.txt.
%%
%% DESCRIPTION:
%%   `pst-fill' is a PSTricks package for filling and tiling areas 
%%
% \fi
% \changes{v1.01}{2007/03/10}{bugfix for incomplete ifx (hv)}
% \changes{v1.00}{2006/11/06}{use pst-xkey for extend keys (hv)}
% \changes{v0.99}{2004/08/17}{merge the VTeX and TeX versions (patch 4) (hv)}
% \changes{v0.98}{2004/06/22}{delete the Pst@Debug option and use the
%   the one from pstricks to prevent a clash with pst-gr3d (hv)}
% \changes{v0.97}{2001/10/09}{make it work with VTeX (mv)}
% \changes{v0.94}{1997/04/08}{With a \PstTiling macro defined (or "tiling" optional parameter
%   on \textbackslash usepackage[tiling]{pst-fill}), this file run exactly as
%   the original boxfill.tex file from Timothy, version 0.94,
%   except a correction in \textbackslash pst@ManualFillCycle to avoid a division by 0.
%   It's the default.}
% \changes{v0.93}{1997/04/07}{With a \textbackslash PstTiling macro defined (or "tiling" optional parameter
%   on \textbackslash usepackage[tiling]{pst-fill}) there are several add-ons
%   and changes to do `tiling' rather than `filling' in "automatic" mode :
%     - we fix the position of the beginning of tiling,
%     - we allow normally the framing of the area as expected, using
%       the line.... parameters
%     - we define move parameters fillmovex, fillmovey and fillmove,
%     - we define fillcyclex as previous fillcycle parameter, and add the
%       fillcycley and fillcycle (both fillcyclex and fillcycley) ones
%     - we can extend the tiling area using fillloopaddx, fillloopaddy and
%       fillloopadd parameters,
%     - we can debug and see the whole tiling area without clipping using
%       PstDebug parameter,
%     - for names consistancy, we can use fillangle in place of boxfillangle
%       and fillsize in place of boxfillsize,
%     - default value for fillsep is 0 and for fillsize is auto.}
%
% \DoNotIndex{\!,\",\#,\$,\%,\&,\',\(,\+,\*,\,,\-,\.,\/,\:,\;,\<,\=,\>,\?}
% \DoNotIndex{\@,\@B,\@K,\@cTq,\@f,\@fPl,\@ifnextchar,\@nameuse,\@oVk}
% \DoNotIndex{\[,\\,\],\^,\_,\ }
% \DoNotIndex{\^,\\^,\\\^,$\^$,$\\^$,$\\^$}
% \DoNotIndex{\0,\2,\4,\5,\6,\7,\8,}
% \DoNotIndex{\A,\a}
% \DoNotIndex{\B,\b,\Bc,\begin,\Bq,\Bqc}
% \DoNotIndex{\C,\c,\catcode,\cJA,\CodelineIndex,\csname}
% \DoNotIndex{\D,\def,\define@key,\Df,\divide,\DocInput,\documentclass,\pst@addfams}
% \DoNotIndex{\eCN,\edef,\else,\eHd,\eMcj,\EnableCrossrefs,\end,\endcsname}
% \DoNotIndex{\endCenterExample,\endExample,\endinput,\endpsclip}
% \DoNotIndex{\PrintIndex,\PrintChanges,\ProvidesFile}
% \DoNotIndex{\endpspicture,\endSideBySideExample,\Example}
% \DoNotIndex{\F,\f,\FdUrr,\fi,\filedate,\fileversion,\FV@Environment}
% \DoNotIndex{\FV@UseKeyValues,\FV@XRightMargin,\FVB@Example,\fvset}
% \DoNotIndex{\G,\g,\GetFileInfo,\gr,\GradientLoaded,\gsFKrbK@o,\gsj,\gsOX}
% \DoNotIndex{\hbadness,\hfuzz,\HLEmphasize,\HLMacro,\HLMacro@i}
% \DoNotIndex{\HLReverse,\HLReverse@i,\hqcu,\HqY}
% \DoNotIndex{\I,\i,\ifx,\input,\Ir,\IU}
% \DoNotIndex{\j,\jl,\JT,\JVodH}
% \DoNotIndex{\K,\k,\kfSlL}
% \DoNotIndex{\L,\let}
% \DoNotIndex{\message,\mHNa,\mIU}
% \DoNotIndex{\N,\nB,\newcmykcolor,\newdimen,\newif,\nW}
% \DoNotIndex{\O,\oCDJDo,\ocQhVI,\OnlyDescription,\oRKJ}
% \DoNotIndex{\P,\p,\ProvidesPackage,\psframe,\pslinewidth,\psset}
% \DoNotIndex{\PstAtCode,\PSTricksLoaded}
% \DoNotIndex{\q,\Qr,\qssRXq,\qu,\qXjFQp,\qYL}
% \DoNotIndex{\R,\r,\RecordChanges,\relax,\RlaYI,\rN,\Rp,\rp,\RPDXNn,\rput}
% \DoNotIndex{\S,\scalebox,\SgY,\SideBySide@Example,\SideBySideExample}
% \DoNotIndex{\SgY,\sk,\Sp,\space,\sZb}
% \DoNotIndex{\T,\the,\tw@}
% \DoNotIndex{\u,\UiSWGEf@,\uJi,\usepackage,\uVQdMM,\UYj}
% \DoNotIndex{\VerbatimEnvironment,\VerbatimInput,\VrC@}
% \DoNotIndex{\WhZ,\WjKCYb,\WNs}
% \DoNotIndex{\XkN,\XW}
% \DoNotIndex{\Z,\ZCM,\Ze}
% \DoNotIndex{\addtocounter,\advance,\alph,\arabic,\AtBeginDocument,\AtEndDocument}
% \DoNotIndex{\AtEndOfPackage,\begingroup,\bfseries,\bgroup,\box,\csname}
% \DoNotIndex{\else,\endcsname,\endgroup,\endinput,\expandafter,\fi}
% \DoNotIndex{\TeX,\z@,\p@,\@one,\xdef,\thr@@,\string,\sixt@@n,\reset,\or,\multiply,\repeat,\RequirePackage}
% \DoNotIndex{\@cclvi,\@ne,\@ehpa,\@nil,\copy,\dp,\global,\hbox,\hss,\ht,\ifodd,\ifdim,\ifcase,\kern}
% \DoNotIndex{\chardef,\loop,\leavevmode,\ifnum,\lower}
% \setcounter{IndexColumns}{2}
%
% ^^A To extend the height used for the text
%
% ^^A  Aligned labels in a description environment
%\newenvironment{Description}[1]{%
%\begin{list}{nothing}{\setlength{\leftmargin}{#1}
%\setlength{\labelwidth}{\leftmargin}\setlength{\labelsep}{1mm}}}
%{\end{list}}
%
% ^^A For macro names
%\DeclareRobustCommand\cs[1]{\texttt{\char`\\#1}}
%
%
% ^^A From ltugboat.cls
% ^^A For references
%\makeatletter
%\newcommand\acro[1]{\textsc{#1}\@}
%\def\CTAN{\acro{CTAN}}
%\let\texttub\textsl              % ^^A redefined in other situations
%\def\TUB{\texttub{TUGboat}}
%\def\TUG{\TeX\ \UG}
%\def\tug{\acro{TUG}}
%\def\UG{Users Group}
% ^^A For the bibliography 
%\let\@internalcite\cite
%\def\cite{\def\@citeseppen{-1000}%
%    \def\@cite##1##2{(##1\if@tempswa , ##2\fi)}%
%    \def\citeauthoryear##1##2##3{##1, ##3}\@internalcite}
%\def\etal{et\,al.\@}
%\newcommand\CTANdirectory[1]{\expandafter\urldef
%  \csname CTAN@#1\endcsname\path}
%\newcommand\CTANfile[1]{\expandafter\urldef
%  \csname CTAN@#1\endcsname\path}
%\newcommand\CTANref[1]{\expandafter\@setref\csname CTAN@#1\endcsname
%  \relax{#1}}
%\makeatother
% ^^A Define CTAN addresses 
%\CTANdirectory{mpattern}{graphics/metapost/macros/mpattern}
%\CTANdirectory{pstricks}{graphics/pstricks}
%\CTANdirectory{pst-fill.sty}{graphics/pstricks/latex/pst-fill.sty}
%\CTANdirectory{pst-fill}{graphics/pstricks/generic/pst-fill.tex}
%\CTANdirectory{Roegel}{graphics/metapost/contrib/macros/truchet}
%\CTANdirectory{xypic}{macros/generic/diagrams/xypic}
%
% ^^A Personal macros (D.G.)
% ^^A ----------------------
%
% ^^A Some colors used
%\definecolor{LemonChiffon}{rgb}{1.,0.98,0.8}
%\definecolor{LightBlue}   {rgb}{0.8,0.85,0.95}
%\definecolor{PaleGreen}   {rgb}{0.88,1,0.88}
%\definecolor{PeachPuff}   {rgb}{1.0,0.85,0.73}
%
% ^^A To define a unique string for TeX and LaTeX
%\newcommand{\AllTeX}{%
%{\rm(L\kern-.36em\raise.3ex\hbox{\sc a}\kern-.15em)%
%T\kern-.1667em\lower.7ex\hbox{E}\kern-.125emX}}
%
% ^^A Bibliography style
%\bibliographystyle{ltugbib}
%
% ^^A Name macros
%\newcommand{\FillPackage}{\textsf{`pst-fill'}}
%\newcommand{\XYpic}{%
%\leavevmode\hbox{\kern-.1em X\kern-.3em\lower.4ex\hbox{Y\kern-.15em}-pic}}
%
%\makeatletter
%
% ^^A Example environments
% ^^A (do not use in them the four JXYZ characters, that we will use
% ^^A as escape characters!)
%
% ^^A Default PSTricks parameters
%  \psset{dimen=middle}
%
% ^^A Translation in PSTricks from the one drawn by Emmanuel Chailloux and
% ^^A Guy Cousineau for the MLgraph system
% ^^A (see /ftp.ens.fr:/pub/unix/lang/MLgraph/version-2.1/MLgraph-refman.ps.gz)
% ^^A The kangaroo itself is reproduce from an original picture from Raoul Raba
% \newcommand{\DimX}{2.47}
% \newcommand{\DimY}{4.8}
% \newcommand{\DimXDivTwo}{1.235}
%
% \newcommand{\KangarooItself}[1]{%
% ^^A Body
% \pspolygon[fillstyle=solid,fillcolor=#1]%
%  (52.5,68)(55,72.5)(55.8,76.5)(56.8,79.8)(58.2,83)(60,85.8)(61.5,86.5)
% (64,87)(66,87.5)(67.8,87.3)(70,87)(71.5,87.3)(73,88)(74.7,88.5)
% (76,90.3)(77,91.5)(72.8,93.8)(69,96)(64.5,99)(59.4,103)(56.2,106.3)
% (53,110.5)(49.5,115.5)(47.2,119.9)(45.7,126)(43.2,123)(41.5,121)(37.5,125)
% (37,122.5)(36.8,120)(37,117)(37.6,113.5)(38.6,110)(40,106.3)(42,102.3)
%  (43.5,99.5)(45,97)(46.2,94)(46.8,91.7)(47.2,88)(47,83.5)(46.3,80.8)
%  (45.3,78.5)(42.5,76.5)(39.5,75.8)(36,75.9)(33,75.9)(29,76.2)(26,77)
%  (22.3,77.5)(18,78.4)(12.8,79.3)(8.6,80)(5.5,80.3)(3,80.5)(0,80)
%  (-5.2,78.5)(-9,76.3)(-11.2,74.8)(-13,72.5)(-16.5,68)(-16.5,68)(-19.5,62.5)
%  (-22,58)(-25.5,53)(-29,48.5)(-32.5,45)(-36,42)(-39,39.5)(-44,37)
%  (-49,35)(-51,34)(-53.5,34.5)(-55.5,36)(-56.5,38)(-56.5,40.5)(-55,41.5)
%  (-53.5,41)(-51.5,41)(-50.5,43)(-50.5,44.5)(-51,47)(-51.5,47.2)(-56.5,47)
%  (-58.5,46.5)(-60,44.7)(-62,42.3)(-63,39.5)(-63.5,36.3)(-63.5,33)(-63.1,29.5)
%  (-61.5,26)(-58,23.6)(-54,22.2)(-50.7,22)(-47.5,22)(-44.5,22.3)(-41,23.5)
%  (-36.8,25.8)(-33,28)(-28.5,31)(-23.4,35)(-20.2,38.3)(-17,42.5)(-13.5,47.5)
%  (-11.2,51.9)(-9.7,58)(-7.2,55)(-5.5,53)(-1.5,57)(-1,54.5)(-0.8,52)
%  (-1,49)(-1.6,45.5)(-2.6,42)(-4,38.3)(-6,34.3)(-7.5,31.5)(-9,29)
%  (-10.2,26)(-10.8,23.7)(-11.2,20)(-11,15.5)(-10.3,12.8)(-9.3,10.5)(-6.5,8.5)
%  (-3.5,7.8)(0,7.9)(3,7.9)(7,8.2)(10,9)(13.7,9.5)(18,10.4)
%  (23.2,11.3)(27.4,12)(30.5,12.3)(33,12.5)(36,12)(41.2,10.5)(45,8.3)
%  (47.2,6.8)(49,4.5)(52.5,0)(50,4.5)(49.2,8.5)(48.2,11.8)(46.8,15)
%  (45,17.8)(43.5,18.5)(41,19)(39,19.5)(37.2,19.3)(35,19)(33.5,19.3)
%  (32,20)(30.3,20.5)(29,22.3)(28,23.5)(28,23.5)(24.5,22.3)(21.5,22)
%  (18.3,22)(15,22.2)(11,23.6)(7.5,26)(5.9,29.5)(5.5,33)(5.5,36.3)
%  (6,39.5)(7,42.3)(9,44.7)(10.5,46.5)(12.5,47)(17.5,47.2)(18,47)
%  (18.5,44.5)(18.5,43)(17.5,41)(15.5,41)(14,41.5)(12.5,40.5)(12.5,38)
%  (13.5,36)(15.5,34.5)(18,34)(20,35)(25,37)(30,39.5)(33,42)
%  (36.5,45)(40,48.5)(43.5,53)(47,58)(49.5,62.5)(52.5,68)
% ^^A Eye
% \pscircle*[linecolor=white](58.2,98.3){2\psxunit}
% \pscircle*(58.2,97.3){\psxunit}
% ^^A Mouth
% \psline(71.5,88)(70,89.3)(68.5,90.3)(67,91.9)
% ^^A Tear
% \psline(42,121)(45,118)(47,115.3)(48.5,112.7)(50,110)(51.8,106.5)
%       (52.5,103.7)(53,100.5)
% \pspolygon(41.2,115.8)(43.2,114.7)(45,112.5)(47,109.8)(48,107)(49.5,104.2)%
%       (50.5,101.6)(51,98.5)(47.7,100.6)(46,102.2)(44.8,104)(43.5,106)
%       (42.5,108)(41.7,110.5)(41,113.2)}
%
% \newcommand{\Kangaroo}[1]{%
%   \begin{pspicture}(\DimX,\DimY)
%   \psset{unit=0.035278}
%   \KangarooItself{#1}
%   \end{pspicture}}
%
% \newcommand{\KangarooPstChart}[1]{{%
%   \psset{xunit=0.006784,yunit=0.00735,linewidth=0.01}
%   \begin{pspicture}(-65.5,0)(82,126)
%     \KangarooItself{#1}
%   \end{pspicture}}}
%
%
% ^^A For the possible index and changes log
% \setlength{\columnseprule}{0.6pt}
%
% ^^A Beginning of the documentation itself
%\title{\texttt{pst-fill}\\A PSTricks package for filling and tiling areas}
%\author{Timothy Van Zandt\thanks{\protect\url{tvz@econ.insead.fr}. (documentation by
% Denis Girou (\protect\url{Denis.Girou@idris.fr}) and Herbert Vo\ss (\protect\url{hvoss@tug.org}).}}
%
%\date{\shortstack{\today --- Version 1.00\\
%                  {\small Documentation revised \today}}}
% \maketitle
% \tableofcontents
%
%\begin{abstract}
%  \FillPackage{} is a PSTricks \cite{vanZandt93},\cite{Girou94},\cite{vanZandtGirou94}, 
%\cite{Hoenig97},\cite{LGC97} package to draw easily
%  various kinds of filling and tiling of areas. It is also a good example of
%  the great power and flexibility of PSTricks, as in fact it is a very short
%  program (it body is around 200~lines long) but nevertheless really powerful.
%
%  \hspace{5mm} It was written in 1994 by Timothy \textsc{van Zandt} but
%  publicly available only in PSTricks 97 and without any documentation.
%  We describe here the version \emph{97 patch 2} of December 12, 1997, which
%  is the original one modified by myself to manage \emph{tilings} in the
%  so-called \emph{automatic} mode. This article would like to serve both of
%  reference manual and of user's guide.
%
%This package is available on \CTAN{} in the
%  \texttt{graphics/pstricks} directory (files \texttt{latex/pst-fill.sty} and
%  \texttt{generic/pst-fill.tex}).
%\end{abstract}
%
%\section{Introduction}
%
%  Here we will refer as \emph{filling} as the operation which consist to fill
%a defined area by a pattern (or a composition of patterns). We will refer as
%\emph{tiling} as the operation which consist to do the same thing, but with
%the control of the starting point, which is here the upper left corner.
%The pattern is positioned relatively to this point. This make an essential
%difference between the two modes, as without control of the starting point we
%can't draw \emph{tilings} (sometimes  called \emph{tesselations}) as used in
%many fields of Art and Science%
%\footnote{For an extensive presentation of tilings, in their history and usage
%in many fields, see the reference book \cite{GS87}.
%
%  In the \TeX{} world, few work was done on tilings. You can look at the
%\emph{tile} extension of the \XYpic{} package \cite{XYpic}, at the articles of
%Kees \textsc{van der Laan} \cite[paragraph 7]{LAAN96} (the tiling was in
%fact directly done in PostScript) and \cite{LAAN97}, at the \MP{} program
%(available on \CTANref{Roegel}) by Denis \textsc{Roegel} for the
%\textsc{Truchet} contest in 1995 \cite{EsperetGirou98} and at the \MP{}
%package \cite{Bolek98} to draw patterns, which have a strong connection with
%tilings.}.
%
%  Nevertheless, as tilings are a wide and difficult field in mathematics, this
%package is limited to simple ones, mainly \emph{monohedral} tilings with one
%prototile (which can be composite, see section \ref{sec:KindTiles}). With some
%experience and wiliness we can do more and obtained easily rather
%sophisticated results, but obviously hyperbolic tilings like the famous
%\textsc{Escher} ones or aperiodic tilings like the \textsc{Penrose} ones are
%not in the capabilities of this package. For more complex needs, we must used
%low level and more painfull technics, with the basic \cs{multido}
%and \cs{multirput} macros.
%
%\section{Package history and description of it two different modes}
%
%  As already said, this package was written in 1994 by Timothy \textsc{van
%Zandt}. Two modes were defined, called respectively \emph{manual} and
%\emph{automatic}. For both, the pattern is generated on contiguous positions in
%a rather large area which include the region to fill, later cut to the
%required dimensions by clipping mechanism. In the first mode, the pattern is
%explicitely inserted in the PostScript file each time. In the second one, the
%result is the same but with an unique explicit insertion of the pattern and a
%repetition done by PostScript. Nevertheless, in this method, the control of
%the starting point was loosed, so it allowed only to \emph{fill} a region and
%not to \emph{tile} it.
%
%  See the difference between the two modes, \emph{tiling}:
% {\psset{unit=0.5cm}%
% \psboxfill{\begin{pspicture}(1,1)\psframe[dimen=middle](1,1)\end{pspicture}}
% \begin{pspicture}(3,3.3)
%   \psframe[fillstyle=boxfill](3,3)
% \end{pspicture}}
% and \emph{filling}:
%{%
% \makeatletter
%\pst@def{BoxFill}<
%  gsave
%    gsave \tx@STV CM grestore dtransform CM idtransform
%    abs /h ED abs /w ED
%    pathbbox
%    h div round 2 add cvi /y2 ED
%    w div round 2 add cvi /x2 ED
%    h div round 2 sub cvi /y1 ED
%    w div round 2 sub cvi /x1 ED
%    /y2 y2 y1 sub def
%    /x2 x2 x1 sub def
%    CP
%    y1 h mul sub neg /y1 ED
%    x1 w mul sub neg /x1 ED
%    clip
%    y2 {
%      /x x1 def
%      x2 {
%        save CP x y1 T moveto Box restore
%        /x x w add def
%      } repeat
%      /y1 y1 h add def
%    } repeat
% currentpoint currentfont grestore setfont moveto>
% \makeatother
%
% \psset{unit=0.5}
% \psboxfill{\begin{pspicture}(1,1)\psframe[dimen=middle](1,1)\end{pspicture}}
% \begin{pspicture}(3,3.3)
%   \psframe[fillstyle=boxfill](3,3)
% \end{pspicture}
% or
% \begin{pspicture}(3,3.3)
%   \psframe[fillstyle=boxfill](3,3)
% \end{pspicture}
%}
%as we can see that initial position is arbitrary and dependent of
%the current point.
%
%
% It's clear that usage of filling is very restrictive comparing to tiling,
%as desired effects required very often the possibility to control the starting 
%point. So, this mode was of limited interest, but unfortunately the
%\emph{manual} one has the very big disadvantage to require very huge amounts
%of ressources, mainly in disk space and consequently in printing time.
%A small tiling can require sometimes several megabytes in \emph{manual} mode!
%So, it was very often not really usable in practice.
%
%It is why I modified the code, to allow tilings in \emph{automatic} mode,
%controlling in this mode too the starting point. And most of the time, that is
%to say if some special options are not used, the tiling is done exactly in the
%region described, which make it faster. So there is no more reason to use the
%\emph{manual} mode, apart very special cases where \emph{automatic} one cannot
%work, as explained later -- currently, I know only one case.
%
%  To load this modified \emph{automatic} mode, with \LaTeX{} use
%simply:\newline 
%\verb+\usepackage[tiling]{pst-fill}+\newline
%and in plain \TeX{} after:\newline
%\verb+% \iffalse meta-comment, etc.
%%
%% Package `pst-fill.dtx'
%%
%% Denis Girou (CNRS/IDRIS - France) <Denis.Girou@idris.fr>
%% Herbert Voss <voss@pstricks.de>
%%
%% This program can be redistributed and/or modified under the terms
%% of the LaTeX Project Public License Distributed from CTAN archives
%% in directory macros/latex/base/lppl.txt.
%%
%% DESCRIPTION:
%%   `pst-fill' is a PSTricks package for filling and tiling areas 
%%
% \fi
% \changes{v1.01}{2007/03/10}{bugfix for incomplete ifx (hv)}
% \changes{v1.00}{2006/11/06}{use pst-xkey for extend keys (hv)}
% \changes{v0.99}{2004/08/17}{merge the VTeX and TeX versions (patch 4) (hv)}
% \changes{v0.98}{2004/06/22}{delete the Pst@Debug option and use the
%   the one from pstricks to prevent a clash with pst-gr3d (hv)}
% \changes{v0.97}{2001/10/09}{make it work with VTeX (mv)}
% \changes{v0.94}{1997/04/08}{With a \PstTiling macro defined (or "tiling" optional parameter
%   on \textbackslash usepackage[tiling]{pst-fill}), this file run exactly as
%   the original boxfill.tex file from Timothy, version 0.94,
%   except a correction in \textbackslash pst@ManualFillCycle to avoid a division by 0.
%   It's the default.}
% \changes{v0.93}{1997/04/07}{With a \textbackslash PstTiling macro defined (or "tiling" optional parameter
%   on \textbackslash usepackage[tiling]{pst-fill}) there are several add-ons
%   and changes to do `tiling' rather than `filling' in "automatic" mode :
%     - we fix the position of the beginning of tiling,
%     - we allow normally the framing of the area as expected, using
%       the line.... parameters
%     - we define move parameters fillmovex, fillmovey and fillmove,
%     - we define fillcyclex as previous fillcycle parameter, and add the
%       fillcycley and fillcycle (both fillcyclex and fillcycley) ones
%     - we can extend the tiling area using fillloopaddx, fillloopaddy and
%       fillloopadd parameters,
%     - we can debug and see the whole tiling area without clipping using
%       PstDebug parameter,
%     - for names consistancy, we can use fillangle in place of boxfillangle
%       and fillsize in place of boxfillsize,
%     - default value for fillsep is 0 and for fillsize is auto.}
%
% \DoNotIndex{\!,\",\#,\$,\%,\&,\',\(,\+,\*,\,,\-,\.,\/,\:,\;,\<,\=,\>,\?}
% \DoNotIndex{\@,\@B,\@K,\@cTq,\@f,\@fPl,\@ifnextchar,\@nameuse,\@oVk}
% \DoNotIndex{\[,\\,\],\^,\_,\ }
% \DoNotIndex{\^,\\^,\\\^,$\^$,$\\^$,$\\^$}
% \DoNotIndex{\0,\2,\4,\5,\6,\7,\8,}
% \DoNotIndex{\A,\a}
% \DoNotIndex{\B,\b,\Bc,\begin,\Bq,\Bqc}
% \DoNotIndex{\C,\c,\catcode,\cJA,\CodelineIndex,\csname}
% \DoNotIndex{\D,\def,\define@key,\Df,\divide,\DocInput,\documentclass,\pst@addfams}
% \DoNotIndex{\eCN,\edef,\else,\eHd,\eMcj,\EnableCrossrefs,\end,\endcsname}
% \DoNotIndex{\endCenterExample,\endExample,\endinput,\endpsclip}
% \DoNotIndex{\PrintIndex,\PrintChanges,\ProvidesFile}
% \DoNotIndex{\endpspicture,\endSideBySideExample,\Example}
% \DoNotIndex{\F,\f,\FdUrr,\fi,\filedate,\fileversion,\FV@Environment}
% \DoNotIndex{\FV@UseKeyValues,\FV@XRightMargin,\FVB@Example,\fvset}
% \DoNotIndex{\G,\g,\GetFileInfo,\gr,\GradientLoaded,\gsFKrbK@o,\gsj,\gsOX}
% \DoNotIndex{\hbadness,\hfuzz,\HLEmphasize,\HLMacro,\HLMacro@i}
% \DoNotIndex{\HLReverse,\HLReverse@i,\hqcu,\HqY}
% \DoNotIndex{\I,\i,\ifx,\input,\Ir,\IU}
% \DoNotIndex{\j,\jl,\JT,\JVodH}
% \DoNotIndex{\K,\k,\kfSlL}
% \DoNotIndex{\L,\let}
% \DoNotIndex{\message,\mHNa,\mIU}
% \DoNotIndex{\N,\nB,\newcmykcolor,\newdimen,\newif,\nW}
% \DoNotIndex{\O,\oCDJDo,\ocQhVI,\OnlyDescription,\oRKJ}
% \DoNotIndex{\P,\p,\ProvidesPackage,\psframe,\pslinewidth,\psset}
% \DoNotIndex{\PstAtCode,\PSTricksLoaded}
% \DoNotIndex{\q,\Qr,\qssRXq,\qu,\qXjFQp,\qYL}
% \DoNotIndex{\R,\r,\RecordChanges,\relax,\RlaYI,\rN,\Rp,\rp,\RPDXNn,\rput}
% \DoNotIndex{\S,\scalebox,\SgY,\SideBySide@Example,\SideBySideExample}
% \DoNotIndex{\SgY,\sk,\Sp,\space,\sZb}
% \DoNotIndex{\T,\the,\tw@}
% \DoNotIndex{\u,\UiSWGEf@,\uJi,\usepackage,\uVQdMM,\UYj}
% \DoNotIndex{\VerbatimEnvironment,\VerbatimInput,\VrC@}
% \DoNotIndex{\WhZ,\WjKCYb,\WNs}
% \DoNotIndex{\XkN,\XW}
% \DoNotIndex{\Z,\ZCM,\Ze}
% \DoNotIndex{\addtocounter,\advance,\alph,\arabic,\AtBeginDocument,\AtEndDocument}
% \DoNotIndex{\AtEndOfPackage,\begingroup,\bfseries,\bgroup,\box,\csname}
% \DoNotIndex{\else,\endcsname,\endgroup,\endinput,\expandafter,\fi}
% \DoNotIndex{\TeX,\z@,\p@,\@one,\xdef,\thr@@,\string,\sixt@@n,\reset,\or,\multiply,\repeat,\RequirePackage}
% \DoNotIndex{\@cclvi,\@ne,\@ehpa,\@nil,\copy,\dp,\global,\hbox,\hss,\ht,\ifodd,\ifdim,\ifcase,\kern}
% \DoNotIndex{\chardef,\loop,\leavevmode,\ifnum,\lower}
% \setcounter{IndexColumns}{2}
%
% ^^A To extend the height used for the text
%
% ^^A  Aligned labels in a description environment
%\newenvironment{Description}[1]{%
%\begin{list}{nothing}{\setlength{\leftmargin}{#1}
%\setlength{\labelwidth}{\leftmargin}\setlength{\labelsep}{1mm}}}
%{\end{list}}
%
% ^^A For macro names
%\DeclareRobustCommand\cs[1]{\texttt{\char`\\#1}}
%
%
% ^^A From ltugboat.cls
% ^^A For references
%\makeatletter
%\newcommand\acro[1]{\textsc{#1}\@}
%\def\CTAN{\acro{CTAN}}
%\let\texttub\textsl              % ^^A redefined in other situations
%\def\TUB{\texttub{TUGboat}}
%\def\TUG{\TeX\ \UG}
%\def\tug{\acro{TUG}}
%\def\UG{Users Group}
% ^^A For the bibliography 
%\let\@internalcite\cite
%\def\cite{\def\@citeseppen{-1000}%
%    \def\@cite##1##2{(##1\if@tempswa , ##2\fi)}%
%    \def\citeauthoryear##1##2##3{##1, ##3}\@internalcite}
%\def\etal{et\,al.\@}
%\newcommand\CTANdirectory[1]{\expandafter\urldef
%  \csname CTAN@#1\endcsname\path}
%\newcommand\CTANfile[1]{\expandafter\urldef
%  \csname CTAN@#1\endcsname\path}
%\newcommand\CTANref[1]{\expandafter\@setref\csname CTAN@#1\endcsname
%  \relax{#1}}
%\makeatother
% ^^A Define CTAN addresses 
%\CTANdirectory{mpattern}{graphics/metapost/macros/mpattern}
%\CTANdirectory{pstricks}{graphics/pstricks}
%\CTANdirectory{pst-fill.sty}{graphics/pstricks/latex/pst-fill.sty}
%\CTANdirectory{pst-fill}{graphics/pstricks/generic/pst-fill.tex}
%\CTANdirectory{Roegel}{graphics/metapost/contrib/macros/truchet}
%\CTANdirectory{xypic}{macros/generic/diagrams/xypic}
%
% ^^A Personal macros (D.G.)
% ^^A ----------------------
%
% ^^A Some colors used
%\definecolor{LemonChiffon}{rgb}{1.,0.98,0.8}
%\definecolor{LightBlue}   {rgb}{0.8,0.85,0.95}
%\definecolor{PaleGreen}   {rgb}{0.88,1,0.88}
%\definecolor{PeachPuff}   {rgb}{1.0,0.85,0.73}
%
% ^^A To define a unique string for TeX and LaTeX
%\newcommand{\AllTeX}{%
%{\rm(L\kern-.36em\raise.3ex\hbox{\sc a}\kern-.15em)%
%T\kern-.1667em\lower.7ex\hbox{E}\kern-.125emX}}
%
% ^^A Bibliography style
%\bibliographystyle{ltugbib}
%
% ^^A Name macros
%\newcommand{\FillPackage}{\textsf{`pst-fill'}}
%\newcommand{\XYpic}{%
%\leavevmode\hbox{\kern-.1em X\kern-.3em\lower.4ex\hbox{Y\kern-.15em}-pic}}
%
%\makeatletter
%
% ^^A Example environments
% ^^A (do not use in them the four JXYZ characters, that we will use
% ^^A as escape characters!)
%
% ^^A Default PSTricks parameters
%  \psset{dimen=middle}
%
% ^^A Translation in PSTricks from the one drawn by Emmanuel Chailloux and
% ^^A Guy Cousineau for the MLgraph system
% ^^A (see /ftp.ens.fr:/pub/unix/lang/MLgraph/version-2.1/MLgraph-refman.ps.gz)
% ^^A The kangaroo itself is reproduce from an original picture from Raoul Raba
% \newcommand{\DimX}{2.47}
% \newcommand{\DimY}{4.8}
% \newcommand{\DimXDivTwo}{1.235}
%
% \newcommand{\KangarooItself}[1]{%
% ^^A Body
% \pspolygon[fillstyle=solid,fillcolor=#1]%
%  (52.5,68)(55,72.5)(55.8,76.5)(56.8,79.8)(58.2,83)(60,85.8)(61.5,86.5)
% (64,87)(66,87.5)(67.8,87.3)(70,87)(71.5,87.3)(73,88)(74.7,88.5)
% (76,90.3)(77,91.5)(72.8,93.8)(69,96)(64.5,99)(59.4,103)(56.2,106.3)
% (53,110.5)(49.5,115.5)(47.2,119.9)(45.7,126)(43.2,123)(41.5,121)(37.5,125)
% (37,122.5)(36.8,120)(37,117)(37.6,113.5)(38.6,110)(40,106.3)(42,102.3)
%  (43.5,99.5)(45,97)(46.2,94)(46.8,91.7)(47.2,88)(47,83.5)(46.3,80.8)
%  (45.3,78.5)(42.5,76.5)(39.5,75.8)(36,75.9)(33,75.9)(29,76.2)(26,77)
%  (22.3,77.5)(18,78.4)(12.8,79.3)(8.6,80)(5.5,80.3)(3,80.5)(0,80)
%  (-5.2,78.5)(-9,76.3)(-11.2,74.8)(-13,72.5)(-16.5,68)(-16.5,68)(-19.5,62.5)
%  (-22,58)(-25.5,53)(-29,48.5)(-32.5,45)(-36,42)(-39,39.5)(-44,37)
%  (-49,35)(-51,34)(-53.5,34.5)(-55.5,36)(-56.5,38)(-56.5,40.5)(-55,41.5)
%  (-53.5,41)(-51.5,41)(-50.5,43)(-50.5,44.5)(-51,47)(-51.5,47.2)(-56.5,47)
%  (-58.5,46.5)(-60,44.7)(-62,42.3)(-63,39.5)(-63.5,36.3)(-63.5,33)(-63.1,29.5)
%  (-61.5,26)(-58,23.6)(-54,22.2)(-50.7,22)(-47.5,22)(-44.5,22.3)(-41,23.5)
%  (-36.8,25.8)(-33,28)(-28.5,31)(-23.4,35)(-20.2,38.3)(-17,42.5)(-13.5,47.5)
%  (-11.2,51.9)(-9.7,58)(-7.2,55)(-5.5,53)(-1.5,57)(-1,54.5)(-0.8,52)
%  (-1,49)(-1.6,45.5)(-2.6,42)(-4,38.3)(-6,34.3)(-7.5,31.5)(-9,29)
%  (-10.2,26)(-10.8,23.7)(-11.2,20)(-11,15.5)(-10.3,12.8)(-9.3,10.5)(-6.5,8.5)
%  (-3.5,7.8)(0,7.9)(3,7.9)(7,8.2)(10,9)(13.7,9.5)(18,10.4)
%  (23.2,11.3)(27.4,12)(30.5,12.3)(33,12.5)(36,12)(41.2,10.5)(45,8.3)
%  (47.2,6.8)(49,4.5)(52.5,0)(50,4.5)(49.2,8.5)(48.2,11.8)(46.8,15)
%  (45,17.8)(43.5,18.5)(41,19)(39,19.5)(37.2,19.3)(35,19)(33.5,19.3)
%  (32,20)(30.3,20.5)(29,22.3)(28,23.5)(28,23.5)(24.5,22.3)(21.5,22)
%  (18.3,22)(15,22.2)(11,23.6)(7.5,26)(5.9,29.5)(5.5,33)(5.5,36.3)
%  (6,39.5)(7,42.3)(9,44.7)(10.5,46.5)(12.5,47)(17.5,47.2)(18,47)
%  (18.5,44.5)(18.5,43)(17.5,41)(15.5,41)(14,41.5)(12.5,40.5)(12.5,38)
%  (13.5,36)(15.5,34.5)(18,34)(20,35)(25,37)(30,39.5)(33,42)
%  (36.5,45)(40,48.5)(43.5,53)(47,58)(49.5,62.5)(52.5,68)
% ^^A Eye
% \pscircle*[linecolor=white](58.2,98.3){2\psxunit}
% \pscircle*(58.2,97.3){\psxunit}
% ^^A Mouth
% \psline(71.5,88)(70,89.3)(68.5,90.3)(67,91.9)
% ^^A Tear
% \psline(42,121)(45,118)(47,115.3)(48.5,112.7)(50,110)(51.8,106.5)
%       (52.5,103.7)(53,100.5)
% \pspolygon(41.2,115.8)(43.2,114.7)(45,112.5)(47,109.8)(48,107)(49.5,104.2)%
%       (50.5,101.6)(51,98.5)(47.7,100.6)(46,102.2)(44.8,104)(43.5,106)
%       (42.5,108)(41.7,110.5)(41,113.2)}
%
% \newcommand{\Kangaroo}[1]{%
%   \begin{pspicture}(\DimX,\DimY)
%   \psset{unit=0.035278}
%   \KangarooItself{#1}
%   \end{pspicture}}
%
% \newcommand{\KangarooPstChart}[1]{{%
%   \psset{xunit=0.006784,yunit=0.00735,linewidth=0.01}
%   \begin{pspicture}(-65.5,0)(82,126)
%     \KangarooItself{#1}
%   \end{pspicture}}}
%
%
% ^^A For the possible index and changes log
% \setlength{\columnseprule}{0.6pt}
%
% ^^A Beginning of the documentation itself
%\title{\texttt{pst-fill}\\A PSTricks package for filling and tiling areas}
%\author{Timothy Van Zandt\thanks{\protect\url{tvz@econ.insead.fr}. (documentation by
% Denis Girou (\protect\url{Denis.Girou@idris.fr}) and Herbert Vo\ss (\protect\url{hvoss@tug.org}).}}
%
%\date{\shortstack{\today --- Version 1.00\\
%                  {\small Documentation revised \today}}}
% \maketitle
% \tableofcontents
%
%\begin{abstract}
%  \FillPackage{} is a PSTricks \cite{vanZandt93},\cite{Girou94},\cite{vanZandtGirou94}, 
%\cite{Hoenig97},\cite{LGC97} package to draw easily
%  various kinds of filling and tiling of areas. It is also a good example of
%  the great power and flexibility of PSTricks, as in fact it is a very short
%  program (it body is around 200~lines long) but nevertheless really powerful.
%
%  \hspace{5mm} It was written in 1994 by Timothy \textsc{van Zandt} but
%  publicly available only in PSTricks 97 and without any documentation.
%  We describe here the version \emph{97 patch 2} of December 12, 1997, which
%  is the original one modified by myself to manage \emph{tilings} in the
%  so-called \emph{automatic} mode. This article would like to serve both of
%  reference manual and of user's guide.
%
%This package is available on \CTAN{} in the
%  \texttt{graphics/pstricks} directory (files \texttt{latex/pst-fill.sty} and
%  \texttt{generic/pst-fill.tex}).
%\end{abstract}
%
%\section{Introduction}
%
%  Here we will refer as \emph{filling} as the operation which consist to fill
%a defined area by a pattern (or a composition of patterns). We will refer as
%\emph{tiling} as the operation which consist to do the same thing, but with
%the control of the starting point, which is here the upper left corner.
%The pattern is positioned relatively to this point. This make an essential
%difference between the two modes, as without control of the starting point we
%can't draw \emph{tilings} (sometimes  called \emph{tesselations}) as used in
%many fields of Art and Science%
%\footnote{For an extensive presentation of tilings, in their history and usage
%in many fields, see the reference book \cite{GS87}.
%
%  In the \TeX{} world, few work was done on tilings. You can look at the
%\emph{tile} extension of the \XYpic{} package \cite{XYpic}, at the articles of
%Kees \textsc{van der Laan} \cite[paragraph 7]{LAAN96} (the tiling was in
%fact directly done in PostScript) and \cite{LAAN97}, at the \MP{} program
%(available on \CTANref{Roegel}) by Denis \textsc{Roegel} for the
%\textsc{Truchet} contest in 1995 \cite{EsperetGirou98} and at the \MP{}
%package \cite{Bolek98} to draw patterns, which have a strong connection with
%tilings.}.
%
%  Nevertheless, as tilings are a wide and difficult field in mathematics, this
%package is limited to simple ones, mainly \emph{monohedral} tilings with one
%prototile (which can be composite, see section \ref{sec:KindTiles}). With some
%experience and wiliness we can do more and obtained easily rather
%sophisticated results, but obviously hyperbolic tilings like the famous
%\textsc{Escher} ones or aperiodic tilings like the \textsc{Penrose} ones are
%not in the capabilities of this package. For more complex needs, we must used
%low level and more painfull technics, with the basic \cs{multido}
%and \cs{multirput} macros.
%
%\section{Package history and description of it two different modes}
%
%  As already said, this package was written in 1994 by Timothy \textsc{van
%Zandt}. Two modes were defined, called respectively \emph{manual} and
%\emph{automatic}. For both, the pattern is generated on contiguous positions in
%a rather large area which include the region to fill, later cut to the
%required dimensions by clipping mechanism. In the first mode, the pattern is
%explicitely inserted in the PostScript file each time. In the second one, the
%result is the same but with an unique explicit insertion of the pattern and a
%repetition done by PostScript. Nevertheless, in this method, the control of
%the starting point was loosed, so it allowed only to \emph{fill} a region and
%not to \emph{tile} it.
%
%  See the difference between the two modes, \emph{tiling}:
% {\psset{unit=0.5cm}%
% \psboxfill{\begin{pspicture}(1,1)\psframe[dimen=middle](1,1)\end{pspicture}}
% \begin{pspicture}(3,3.3)
%   \psframe[fillstyle=boxfill](3,3)
% \end{pspicture}}
% and \emph{filling}:
%{%
% \makeatletter
%\pst@def{BoxFill}<
%  gsave
%    gsave \tx@STV CM grestore dtransform CM idtransform
%    abs /h ED abs /w ED
%    pathbbox
%    h div round 2 add cvi /y2 ED
%    w div round 2 add cvi /x2 ED
%    h div round 2 sub cvi /y1 ED
%    w div round 2 sub cvi /x1 ED
%    /y2 y2 y1 sub def
%    /x2 x2 x1 sub def
%    CP
%    y1 h mul sub neg /y1 ED
%    x1 w mul sub neg /x1 ED
%    clip
%    y2 {
%      /x x1 def
%      x2 {
%        save CP x y1 T moveto Box restore
%        /x x w add def
%      } repeat
%      /y1 y1 h add def
%    } repeat
% currentpoint currentfont grestore setfont moveto>
% \makeatother
%
% \psset{unit=0.5}
% \psboxfill{\begin{pspicture}(1,1)\psframe[dimen=middle](1,1)\end{pspicture}}
% \begin{pspicture}(3,3.3)
%   \psframe[fillstyle=boxfill](3,3)
% \end{pspicture}
% or
% \begin{pspicture}(3,3.3)
%   \psframe[fillstyle=boxfill](3,3)
% \end{pspicture}
%}
%as we can see that initial position is arbitrary and dependent of
%the current point.
%
%
% It's clear that usage of filling is very restrictive comparing to tiling,
%as desired effects required very often the possibility to control the starting 
%point. So, this mode was of limited interest, but unfortunately the
%\emph{manual} one has the very big disadvantage to require very huge amounts
%of ressources, mainly in disk space and consequently in printing time.
%A small tiling can require sometimes several megabytes in \emph{manual} mode!
%So, it was very often not really usable in practice.
%
%It is why I modified the code, to allow tilings in \emph{automatic} mode,
%controlling in this mode too the starting point. And most of the time, that is
%to say if some special options are not used, the tiling is done exactly in the
%region described, which make it faster. So there is no more reason to use the
%\emph{manual} mode, apart very special cases where \emph{automatic} one cannot
%work, as explained later -- currently, I know only one case.
%
%  To load this modified \emph{automatic} mode, with \LaTeX{} use
%simply:\newline 
%\verb+\usepackage[tiling]{pst-fill}+\newline
%and in plain \TeX{} after:\newline
%\verb+\input{pst-fill}+\newline
%add the following definition:\newline
%\verb+\def\PstTiling{true}+
%
%  To obtain the original behaviour, just don't use the \emph{tiling} optional
%keyword at loading.
%
%  Take care than in \emph{tiling} mode, I introduce also some other changes.
%First I define aliases on some parameter names for consistancy (all specific
%parameters will begin by the \texttt{fill} prefix in this case) and I change
%some default values, which were not well adapted for tilings (\texttt{fillsep}
%is set to 0 and as explained \texttt{fillsize} set to \texttt{auto}). I rename 
%\texttt{fillcycle} to \texttt{fillcyclex}. I also restore normal way so that
%the frame of the area is drawn and all line (\texttt{linestyle},
%\texttt{linecolor}, \texttt{doubleline}, etc.) parameters are now active (but
%there are not in non \emph{tiling} mode). And I also introduce new parameters
%to control the tilings (see below).
%
%  \textbf{In all the following examples, we will consider only the
% \emph{tiling} mode.}
%
%  To do a tiling, we have just to define the pattern with the
% \verb+\psboxfill+ macro and to use the new \texttt{fillstyle}
% \verb+boxfill+.
%
%  Note that tilings are drawn from left to right and top to bottom, which can
%have an importance in some circonstances.
%
%  PostScript programmers can be also interested to know that, even in the
%\emph{automatic} mode, the iterations of the pattern are managed directly by
%the PostScript code of the package which used only PostScript Level 1
%operators. The special ones introduced in Level 2 for drawing of patterns
%\cite[section 4.9]{PostScript95} are not used.
%
%  And first, for conveniance, we define a simple \cs{Tiling} macro, which
%will simplify our examples:
%
%\begin{verbatim}
%  \newcommand{\Tiling}[2][]{%
%    \edef\Temp{#1}%
%    \begin{pspicture}#2
%      \ifx\Temp\empty
%        \psframe[fillstyle=boxfill]#2
%      \else
%        \psframe[fillstyle=boxfill,#1]#2
%      \fi
%    \end{pspicture}}
%\end{verbatim}
%
%
%\newcommand{\Tiling}[2][]{%
%  \edef\Temp{#1}%
%  \begin{pspicture}#2
%    \ifx\Temp\empty
%      \psframe[fillstyle=boxfill]#2
%    \else
%      \psframe[fillstyle=boxfill,#1]#2
%    \fi
% \end{pspicture}}
%
%\subsection{Parameters}
%
%  There are \textbf{14} specific parameters available to change the way the
% filling/tiling is defined, and one debugging option.
%
% \begin{Description}{2cm}
%  \item [fillangle (real)\hfill :] the value of the rotation
%  applied to the patterns (\emph{Default:~0}).
% \end{Description}
%
%
%   In this case, we must force the tiling area to be notably larger than the
% area to cover, to be sure that the defined area will be covered after rotation.
% \lstset{gobble=2}
% \begin{LTXexample}
% \newcommand{\Square}{%
%   \begin{pspicture}(1,1)
%     \psframe[dimen=middle](1,1)
%   \end{pspicture}}
% \psset{unit=0.5}
% \psboxfill{\Square}
% \Tiling[fillangle=45]{(3,3)}\quad
% \Tiling[fillangle=-60]{(3,3)}
% \end{LTXexample}
% 
% \newcommand{\Square}{\begin{pspicture}(1,1)\psframe[dimen=middle](1,1)\end{pspicture}}
% 
% \begin{Description}{2cm}
%   \setcounter{footnote}{1}
%   \item[\texttt{fillsepx} (real$\|$dim) :] value of the horizontal
%   separation between consecutive patterns (\emph{Default:~0 for
%   tilings\footnotemark, 2pt otherwise}).  \footnotetext{This option was added
%   by me, is not part of the original package and is available only if the
%   \texttt{tiling} keyword is used when loading the package.}
%   \setcounter{footnote}{1}
%   \item [\texttt{fillsepy} (real$\|$dim)\hfill :] value of the vertical
%   separation between consecutive patterns (\emph{Default:~0 for
%   ti\-lings\footnotemark, 2pt otherwise}).
%   \setcounter{footnote}{1}
%   \item [\texttt{fillsep} (real$\|$dim)\hfill :] value of horizontal and
%   vertical separations between consecutive patterns (\emph{Default:~0 for
%   tilings\footnotemark, 2pt otherwise}).
% \end{Description}
% 
%   These values can be negative, which allow the tiles to overlap.
% 
% \begin{LTXexample}
% \psset{unit=0.5}
% \psboxfill{\Square}
% \Tiling[fillsepx=2mm]{(3,3)} 
% \Tiling[fillsepy=1mm]{(3,3)}\\
% \Tiling[fillsep=0.5]{(3,3)} 
% \Tiling[fillsep=-0.5]{(3,3)}
% \end{LTXexample}
% 
% \begin{Description}{2cm}
%   \item [\texttt{fillcyclex}\footnotemark\ (integer)\hfill :] Shift
%   coefficient applied to each row (\emph{Default:~0}).
%   \footnotetext{It was \texttt{fillcycle} in the original version.}
%   \setcounter{footnote}{1}
%   \item [\texttt{fillcycley}\footnotemark\ (integer)\hfill :] Same thing for
%   columns (\emph{Default:~0}).
%   \setcounter{footnote}{1}
%   \item [\texttt{fillcycle}\footnotemark\ (integer)\hfill :] Allow to fix
%   both \texttt{fillcyclex} and \texttt{fillcycley} directly to the same value
%   (\emph{Default:~0}).
% \end{Description}
% 
%   For instance, if \texttt{fillcyclex} is 2, the second row of patterns will
% be horizontally shifted by a factor of $\frac{1}{2}=0.5$, and by a factor of
% 0.333 if \texttt{fillcyclex} is 3, etc.). These values can be negative.
% 
% \begin{LTXexample}[width=0.35\linewidth]
% \psset{unit=0.5}
% \psboxfill{\Square}
% \newcommand{\TilingA}[1]{\Tiling[fillcyclex=#1]{(3,3)}}
% \TilingA{0} \TilingA{1}\\
% \TilingA{2} \TilingA{3}\\[3mm]
% \TilingA{4} \TilingA{5}\\
% \TilingA{6} \TilingA{-3}\\[3mm]
% \Tiling[fillcycley=2]{(3,3)}
% \Tiling[fillcycley=3]{(3,3)}\\
% \Tiling[fillcycley=-3]{(3,3)}
% \Tiling[fillcycle=2]{(3,3)}
% \end{LTXexample}
% 
% \begin{Description}{2cm}
%   \setcounter{footnote}{1}
%   \item [\texttt{fillmovex}\footnotemark\ (real$\|$dim)\hfill :] value of the
%   horizontal moves between consecutive patterns (\emph{Default:~0}).
%   \setcounter{footnote}{1}
%   \item [\texttt{fillmovey}\footnotemark\ (real$\|$dim)\hfill :] value of the
%   vertical moves between consecutive patterns (\emph{Default:~0}).
%   \setcounter{footnote}{1}
%   \item [\texttt{fillmove}\footnotemark\ (real$\|$dim)\hfill :] value of
%   horizontal and vertical moves between consecutive patterns
%   (\emph{Default:~0}).
% \end{Description}
% 
%   These parameters allow the patterns to overlap and to draw some special
% kinds of tilings. They are implemented only for the \emph{automatic} and
% \emph{tiling} modes and their values can be negative.
% 
%   In some cases, the effect of these parameters will be the same that with the 
% \texttt{fillcycle?} ones, but you can see that it is not true for some other
% values.
% 
% \begin{LTXexample}
% \psset{unit=0.5}
% \psboxfill{\Square}
% \Tiling[fillmovex=0.5]{(3,3)} 
% \Tiling[fillmovey=0.5]{(3,3)}\\
% \Tiling[fillmove=0.5]{(3,3)}
% \Tiling[fillmove=-0.5]{(3,3)}
% \end{LTXexample}
% 
% \begin{Description}{2cm}
%   \item [\texttt{fillsize}
%   (auto$\|$\{(real$\|$dim,real$\|$dim)(real$\|$dim,real$\|$dim)\}) :] The
%   choice of \emph{automatic} mode or the size of the area in \emph{manual}
%   mode. If first pair values are not given, (0,0) is used. (\emph{Default:
%   auto when \emph{tiling} mode is used, {(-15cm,-15cm)(15cm,15cm)}
%   otherwise}).
% \end{Description}
% 
%   As explained in the introduction, the \emph{manual} mode can require very
% huge amount of computer ressources. So, it usage is to discourage in front off
% the \emph{automatic} mode. It seems only useful in special circonstances, in
% fact when the \emph{automatic} mode failed, which is known only in one case,
% for some kinds of EPS files, as the ones produce by dump of portions of
% screens (see \ref{sec:GraphicFiles}).
% 
% \begin{Description}{2cm}
%   \setcounter{footnote}{1}
%   \item [\texttt{fillloopaddx}\footnotemark\ (integer)\hfill :] number of
%   times the pattern is added on left and right positions (\emph{Default:~0}).
%   \setcounter{footnote}{1}
%   \item [\texttt{fillloopaddy}\footnotemark\ (integer)\hfill :] number of
%   times the pattern is added on top and bottom positions (\emph{Default:~0}).
%   \setcounter{footnote}{1}
%   \item [\texttt{fillloopadd}\footnotemark\ (integer)\hfill :] number of
%   times the pattern is added on left, right, top and bottom positions
%   (\emph{Default:~0}).
% \end{Description}
% 
%   These parameters are only useful in special circonstances, as for complex
% patterns when the size of the rectangular box used to tile the area doesn't 
% correspond to the pattern itself (see an example in Figure~\ref{fig:Sheeps})
% and also sometimes when the size of the pattern is not a divisor of the size
% of the area to fill and that the number of loop repeats is not properly
% computed, which can occur.
% 
%   They are implemented only for the \emph{tiling} mode.
% 
% \begin{Description}{2cm}
%   \setcounter{footnote}{1}
%   \item [\texttt{PstDebug}\footnotemark\ (integer, 0 or 1)\hfill :] to
%   require to see the exact tiling done, without clipping (\emph{Default:~0}).
% \end{Description}
% 
%   It's mainly useful for debugging or to understand better how the tilings
% are done. It is implemented only for the \emph{tiling} mode.
% 
% \begin{LTXexample}
% \psset{unit=0.3,PstDebug=1}
% \psboxfill{\Square}
% \psset{linewidth=1mm}
% \Tiling{(2,2)}\\[5mm]
% \Tiling[fillcyclex=2]{(2,2)}\\[1cm]
% \Tiling[fillmove=0.5]{(2,2)}
% \end{LTXexample}
% 
% \vspace{3cm}
% \section{Examples}
% 
%   In fact this unique \cs{psboxfill} macro allow a lot a variations and
% different usages. We will try here to demonstrate this.
% 
% \subsection{Kind of tiles}
% \label{sec:KindTiles}
% 
%   Of course, we can access to all the power of PSTricks macros to define the
% \emph{tiles} (\emph{patterns}) used. So, we can define complicated ones.
% 
%   Here we give four other Archimedian tilings (those built with only some
% regular polygons) among the twelve existing, first discovered completely by
% Johanes \textsc{Kepler} at the beginning of 17th century \cite{GS87}, the two
% other \emph{regular} ones with the tiling by squares, formed by a unique
% regular polygon, and two other formed by two different regular polygons.
% 
% \begin{LTXexample}[pos=t]
%   \newcommand{\Triangle}{%
%     \begin{pspicture}(1,1)
%       \pstriangle[dimen=middle](0.5,0)(1,1)
%     \end{pspicture}}
%   \newcommand{\Hexagon}{
% ^^A sin(60)=0.866
%     \begin{pspicture}(0.866,0.75)
%       \SpecialCoor
% ^^A  Hexagon  
%       \pspolygon[dimen=middle]%
%         (0.5;30)(0.5;90)(0.5;150)(0.5;210)(0.5;270)(0.5;330)
%     \end{pspicture}}
% 
%   \psset{unit=0.5}
%   \psboxfill{\Triangle}
%   \Tiling{(4,4)}\hfill
% ^^A The two other regular tilings
%   \Tiling[fillcyclex=2]{(4,4)}\hfill
%   \psboxfill{\Hexagon}
%   \Tiling[fillcyclex=2,fillloopaddy=1]{(5,5)}
% \end{LTXexample}
% 
% \begin{LTXexample}[pos=t]
%   \newcommand{\ArchimedianA}{%
%      ^^A Archimedian tiling 3^2.4.3.4
%     \psset{dimen=middle}
%      ^^A sin(60)=0.866
%     \begin{pspicture}(1.866,1.866)
%       \psframe(1,1)
%       \psline(1,0)(1.866,0.5)(1,1)(0.5,1.866)(0,1)(-0.866,0.5)
%       \psline(0,0)(0.5,-0.866)
%     \end{pspicture}}
%   \newcommand{\ArchimedianB}{%
%      ^^A Archimedian tiling 4.8^2
%     \psset{dimen=middle,unit=1.5}
%      ^^A sin(22.5)=0.3827 ; cos(22.5)=0.9239
%     \begin{pspicture}(1.3066,0.6533)
%       \SpecialCoor
%      ^^A Octogon
%       \pspolygon(0.5;22.5)(0.5;67.5)(0.5;112.5)(0.5;157.5)
%                 (0.5;202.5)(0.5;247.5)(0.5;292.5)(0.5;337.5)
%     \end{pspicture}}
% 
%   \psset{unit=0.5}
%   \psboxfill{\ArchimedianA}
%   \Tiling[fillmove=0.5]{(7,7)}\hfill
%   \psboxfill{\ArchimedianB}
%   \Tiling[fillcyclex=2,fillloopaddy=1]{(7,7)}
% \end{LTXexample}
% 
%   \setcounter{footnote}{3}
%   We can of course tile an area arbitrarily defined. And with the
% \texttt{addfillstyle} parameter\footnote{Introduced in PSTricks 97.}, we can
% easily mix the \texttt{boxfill} style with another one.
% 
% \begin{LTXexample}[width=6cm]
%   \psset{unit=0.5,dimen=middle}
%   \psboxfill{%
%     \begin{pspicture}(1,1)
%       \psframe(1,1)
%       \pscircle(0.5,0.5){0.25}
%     \end{pspicture}}
%   \begin{pspicture}(4,6)
%     \pspolygon[fillstyle=boxfill,fillsep=0.25](0,1)(1,4)(4,6)(4,0)(2,1)
%   \end{pspicture}\hspace{1em}
%   \begin{pspicture}(4,4)
%%     \pscircle[linestyle=none,fillstyle=solid,fillcolor=yellow,fillsep=0.5,
%%               addfillstyle=boxfill](2,2){2}
%   \end{pspicture}
% \end{LTXexample}
%
%   Various effects can be obtained, sometimes complicated ones very easily, as
% in this example reproduced from one shown by Slavik \textsc{Jablan} in the
% field of \emph{OpTiles}, inspired by the \emph{Op-art}:
% 
% \begin{LTXexample}[pos=t]
% \newcommand{\ProtoTile}{%
%  \begin{pspicture}(1,1)%%% 1/12=0.08333
%   \psset{linestyle=none,linewidth=0,
%     hatchwidth=0.08333\psunit,hatchsep=0.08333\psunit}
%   \psframe[fillstyle=solid,fillcolor=black,addfillstyle=hlines,hatchcolor=white](1,1)
%   \pswedge[fillstyle=solid,fillcolor=white,addfillstyle=hlines]{1}{0}{90}
%  \end{pspicture}}
% \newcommand{\BasicTile}{%
%  \begin{pspicture}(2,1)
%    \rput[lb](0,0){\ProtoTile}\rput[lb](1,0){\psrotateleft{\ProtoTile}}
%  \end{pspicture}}
% \ProtoTile\hfill\BasicTile\hfill
% \psboxfill{\BasicTile}
% \Tiling[fillcyclex=2]{(4,4)}
% \end{LTXexample}
% 
%   It is also directly possible to surimpose several different tilings. Here is
% the splendid visual proof of the \textsc{Pytha\-gore} theorem done by the arab
% mathematician \textsc{Annairizi} around the year 900, given by superposition
% of two tilings by squares of different sizes.
% 
% \begin{LTXexample}[pos=t]
% \psset{unit=1.5,dimen=middle}
% \begin{pspicture*}(3,3)
%   \psboxfill{\begin{pspicture}(1,1)
%     \psframe(1,1)\end{pspicture}}
%   \psframe[fillstyle=boxfill](3,3)
%   \psboxfill{\begin{pspicture}(1,1)
%     \rput{-37}{\psframe[linecolor=red](0.8,0.8)}
%   \end{pspicture}}
%   \psframe[fillstyle=boxfill](3,4)
%   \pspolygon[fillstyle=hlines,hatchangle=90](1,2)(1.64,1.53)(2,2)
% \end{pspicture*}
% \end{LTXexample}
% 
%   In a same way, it is possible to build tilings based on figurative patterns,
% in the style of the famous \textsc{Escher} ones. Following an example of
% Andr\'e \textsc{Deledicq} \cite{Deledicq97}, we first show a simple tiling of
% the \emph{p1} category (according to the international classification of the
% 17~symmetry groups of the plane first discovered by the russian
% crystalographer Jevgraf \textsc{Fedorov} at the end of the 19th century):
% 
% \begin{LTXexample}[pos=t]
%  \newcommand{\SheepHead}[1]{%
%    \begin{pspicture}(3,1.5)
%      \pscustom[liftpen=2,fillstyle=solid,fillcolor=#1]{%
%        \pscurve(0.5,-0.2)(0.6,0.5)(0.2,1.3)(0,1.5)(0,1.5)
%          (0.4,1.3)(0.8,1.5)(2.2,1.9)(3,1.5)(3,1.5)(3.2,1.3)
%          (3.6,0.5)(3.4,-0.3)(3,0)(2.2,0.4)(0.5,-0.2)}
%      \pscircle*(2.65,1.25){0.12\psunit} % Eye
%      \psccurve*(3.5,0.3)(3.35,0.45)(3.5,0.6)(3.6,0.4)% Muzzle
%     ^^A   % Mouth
%       \pscurve(3,0.35)(3.3,0.1)(3.6,0.05)
%     ^^A   % Ear
%       \pscurve(2.3,1.3)(2.1,1.5)(2.15,1.7)\pscurve(2.1,1.7)(2.35,1.6)(2.45,1.4)
%   \end{pspicture}}
%  \psboxfill{\psset{unit=0.5}\SheepHead{yellow}\SheepHead{cyan}}
%  \Tiling[fillcyclex=2,fillloopadd=1]{(10,5)}
% \end{LTXexample}
% \label{fig:Sheeps}
% 
%   Now a tiling of the \emph{pg} category (the code for the kangaroo itself is
% too long to be shown here, but has no difficulties ; the kangaroo is reproduce
% from an original picture from Raoul \textsc{Raba} and here is a translation in
% PSTricks from the one drawn by Emmanuel \textsc{Chailloux} and Guy
% \textsc{Cousineau} for their MLgraph system \cite{MLgraphTSI}):
% 
% \begin{LTXexample}[pos=t]
% \psboxfill{\psset{unit=0.4}
%   \Kangaroo{yellow}\Kangaroo{red}\Kangaroo{cyan}\Kangaroo{green}%
%   \psscalebox{-1 1}{%
%     \rput(1.235,4.8){\Kangaroo{green}\Kangaroo{cyan}\Kangaroo{red}\Kangaroo{yellow}}}}
%   \Tiling[fillloopadd=1]{(10,6)}
% \end{LTXexample}
% 
%   And here a \textsc{Wang} tiling \cite{Wang65}, \cite[chapter
% 11]{GS87}, based on very simple tiles of the form of a square and composed
% of four colored triangles. Such tilings are built with only a matching color
% constraint. Despite of it simplicity, it is an important kind of tilings, as
% \textsc{Wang} and others used them to study the special class of
% \emph{aperiodic} tilings, and also because it was shown that surprisingly this 
% tiling is similar to a \textsc{Turing} machine.
% 
% \begin{LTXexample}[pos=t]
%   \newcommand{\WangTile}[4]{%
%     \begin{pspicture}(1,1)
%       \pspolygon*[linecolor=#1](0,0)(0,1)(0.5,0.5)
%       \pspolygon*[linecolor=#2](0,1)(1,1)(0.5,0.5)
%       \pspolygon*[linecolor=#3](1,1)(1,0)(0.5,0.5)
%       \pspolygon*[linecolor=#4](1,0)(0,0)(0.5,0.5)
%     \end{pspicture}}
%   \newcommand{\WangTileA}{\WangTile{cyan}{yellow}{cyan}{cyan}}
%   \newcommand{\WangTileB}{\WangTile{yellow}{cyan}{cyan}{red}}
%   \newcommand{\WangTileC}{\WangTile{cyan}{red}{yellow}{yellow}}
%   \newcommand{\WangTiles}[1][]{%
%     \begin{pspicture}(3,3) \psset{ref=lb}
%       \rput(0,2){\WangTileB}  \rput(1,2){\WangTileA}%
%       \rput(2,2){\WangTileC}  \rput(0,1){\WangTileC}%
%       \rput(1,1){\WangTileB}  \rput(2,1){\WangTileA}
%       \rput(0,0){\WangTileA}  \rput(1,0){\WangTileC}%
%       \rput(2,0){\WangTileB}
%       #1
%     \end{pspicture}}
%   \WangTileA\hfill\WangTileB\hfill\WangTileC\hfill
%   \WangTiles[{\psgrid[subgriddiv=0,gridlabels=0](3,3)}]\hfill
%   \psset{unit=0.4} \psboxfill{\WangTiles} \Tiling{(12,12)}
% \end{LTXexample}
% 
% \subsection{External graphic files}
% \label{sec:GraphicFiles}
% 
%   We can also fill an arbitrary area with an external image. We have only, 
% as usual, to matter of the \emph{BoundingBox} definition if there is no one
% provided or if it is not the accurate one, as for the well known
% \texttt{tiger} picture part of the \texttt{ghostscript} distribution.
% 
% \begin{LTXexample}[pos=t]
%   \psboxfill{%% Strangely require x1=x2...
%     \begin{pspicture}(0,1)(0,4.1)
%       \includegraphics[bb=17 176 560 74,width=3cm]{tiger}
%     \end{pspicture}}
%   \Tiling{(6,6.2)}
% \end{LTXexample}
% 
%   Nevertheless, there are some special files for which the \emph{automatic}
% mode doesn't work, specially for some files obtained by a screen dump, as in
% the next example, where a picture was reduced before it conversion in the
% \emph{Encapsulated PostScript} format by a screen dump utility. In this case,
% usage of the \emph{manual} mode is the only alternative, at the price of the
% real multiple inclusion of the EPS file. We must take care to specify the
% correct \texttt{fillsize} parameter, because otherwise the default values are
% large and will load the file many times, perhaps just really using few
% occurrences as the other ones would be clipped...
% 
% \begin{LTXexample}[pos=t]
%   \psboxfill{\includegraphics{flowers}}
%   \begin{pspicture}(8,4)
%     \psellipse[fillstyle=boxfill,fillsize={(8,4)}](4,2)(4,2)
%   \end{pspicture}
% \end{LTXexample}
% 
% \subsection{Tiling of characters}
% 
%   We can also use the \cs{psboxfill} macro to fill the interior of characters
% for special effects like these ones:
% 
% \begin{LTXexample}[pos=t]
%   \DeclareFixedFont{\bigsf}{T1}{phv}{b}{n}{4.5cm}
%   \DeclareFixedFont{\smallrm}{T1}{ptm}{m}{n}{3mm}
%   \psboxfill{\smallrm Since 182 days...}
%   \begin{pspicture*}(8,4)
%     \centerline{%
%       \pscharpath[fillstyle=gradient,gradangle=-45,
%                   gradmidpoint=0.5,addfillstyle=boxfill,
%                   fillangle=45,fillsep=0.7mm]
%                  {\rput[b](0,0.1){\bigsf 2000}}}
%   \end{pspicture*}
% \end{LTXexample}
% 
% \begin{LTXexample}[pos=t]
%   \DeclareFixedFont{\mediumrm}{T1}{ptm}{m}{n}{2cm}
%   \psboxfill{%
%     \psset{unit=0.1,linewidth=0.2pt}
%     \Kangaroo{PeachPuff}\Kangaroo{PaleGreen}%
%       \Kangaroo{LightBlue}\Kangaroo{LemonChiffon}%
%     \psscalebox{-1 1}{%
%       \rput(1.235,4.8){%
%         \Kangaroo{LemonChiffon}\Kangaroo{LightBlue}%
%           \Kangaroo{PaleGreen}\Kangaroo{PeachPuff}}}}
% ^^A   % A kangaroo of kangaroos...
%   \begin{pspicture}(8,2)
%     \pscharpath[linestyle=none,fillstyle=boxfill,fillloopadd=1]
%                {\rput[b](4,0){\mediumrm Kangaroo}}
%   \end{pspicture}
% \end{LTXexample}
% 
% \subsection{Other kinds of usage}
% 
%   Other kinds of usage can be imagined. For instance, we can use tilings in a
% sort of degenerated way to draw some special lines made by a unique or
% multiple repeating patterns. But it can be only a special dashed line, as here
% with three different dashes:
% 
% \begin{LTXexample}[pos=t]
%   \newcommand{\Dashes}{%
%     \psset{dimen=middle}
%     \begin{pspicture}(0,-0.5\pslinewidth)(1,0.5\pslinewidth)
%       \rput(0,0){\psline(0.4,0)}%
%         \rput(0.5,0){\psline(0.2,0)}%
%         \rput(0.8,0){\psline(0.1,0)}
%     \end{pspicture}}
% 
%   \newcommand{\SpecialDashedLine}[3]{%
%     \psboxfill{#3}
%     \Tiling[linestyle=none]
%            {(#1,-0.5\pslinewidth)(#2,0.5\pslinewidth)}}
% 
%   \SpecialDashedLine{0}{7}{\Dashes}
% 
%   \psset{unit=0.5,linewidth=1mm,linecolor=red}
%   \SpecialDashedLine{0}{10}{\Dashes}
% \end{LTXexample}
% 
%   It allow also to use special patterns in business graphics, as in the
% following example generated by \texttt{PstChart}\footnote{A personal
% development to draw business charts with PSTricks, not distributed.}.
% 
% \vspace{3mm}
% \begin{figure}[!ht]
% \centering
% \psset{unit=0.75}
% ^^A % Generated by pstchart.sh version 0.21 (11/28/97)
% {\psset{dimen=middle}
% \psset{xunit=2,yunit=0.005}
% \begin{pspicture}(-0.6,-200)(6.6,2300)
% ^^A   % Title
%   \rput(3,2200){\shortstack{Fantaisist repartition of kangaroos\\
%                             in the world (in thousands)}}
% ^^A   % Frame background
%   \psframe[fillstyle=solid,fillcolor=LemonChiffon](0,0)(6,2000)
% ^^A   % Graduations
%   \multido{\n=0+500}{5}{\rput[r](-0.12,\n){\psscalebox{0.8}{\n}}}
% ^^A   % Minor ticks
%   \multips(0,100)(0,100){19}{\psline[unit=4.8pt](1,0)}
%   \multips(6,100)(0,100){19}{\psline[unit=4.8pt](-1,0)}
% ^^A   % Major ticks
%   \multips(0,500)(0,500){3}{\psline[unit=9.6pt](1,0)}
%   \multips(6,500)(0,500){3}{\psline[unit=9.6pt](-1,0)}
% ^^A   % Lines from major ticks marks
%   \multips(0,500)(0,500){3}{\psline[linestyle=dotted,linewidth=0.6pt](6,0)}
% ^^A   % Drawing for the data
%   \psboxfill{\psset{unit=0.78\psxunit}\KangarooPstChart{red}}
%   \psframe[linestyle=none,fillstyle=boxfill,fillloopaddy=1](0.61,0)(1.39,1800)
%   \psboxfill{\psset{unit=0.78\psxunit}\KangarooPstChart{yellow}}
%   \psframe[linestyle=none,fillstyle=boxfill,fillloopaddy=1](1.61,0)(2.39,800)
%   \psboxfill{\psset{unit=0.78\psxunit}\KangarooPstChart{cyan}}
%   \psframe[linestyle=none,fillstyle=boxfill,fillloopaddy=1](2.61,0)(3.39,550)
%   \psboxfill{\psset{unit=0.78\psxunit}\KangarooPstChart{magenta}}
%   \psframe[linestyle=none,fillstyle=boxfill,fillloopaddy=1](3.61,0)(4.39,500)
%   \psboxfill{\psset{unit=0.78\psxunit}\KangarooPstChart{green}}
%   \psframe[linestyle=none,fillstyle=boxfill,fillloopaddy=1](4.61,0)(5.39,200)
% ^^A   % Bottom labels
%   \uput{0.2}[270]{0}(1,0){\psscalebox{0.7}{Oceania}}
%   \uput{0.2}[270]{0}(2,0){\psscalebox{0.7}{Africa}}
%   \uput{0.2}[270]{0}(3,0){\psscalebox{0.7}{Asia}}
%   \uput{0.2}[270]{0}(4,0){\psscalebox{0.7}{America}}
%   \uput{0.2}[270]{0}(5,0){\psscalebox{0.7}{Europe}}
% ^^A   % Frame box around the chart
%   \psframe[linestyle=solid](0,0)(6,2000)
% \end{pspicture}}
%   \caption{Bar chart generated by PstChart, with bars filled by patterns}
%   \label{fig:PstChart}
% \end{figure}
% 
% \section{``Dynamic'' tilings}
% 
%   In some cases, tilings used non \emph{static} tiles, that is to say that the 
% \emph{prototile(s)}, even if unique, can have several forms, by instance
% specified by different colors or rotations, not fixed before generation or
% varying each time.
% 
% \subsection{Lewthwaite-Pickover-Truchet tiling}
% 
%   We give here for example the so-called \emph{Truchet} tiling, which much be
% in fact better called \emph{Lewthwaite-Pick\-over-Truchet (LPT)} tiling%
% \footnote{For description of the context, history and references about
% S\'ebastien \textsc{Truchet} and this tiling, see \cite{EsperetGirou98}.}.
% 
%   The unique prototile is only a square with two opposite circle arcs.
% This tile has obviously two positions, if we rotate it from 90 degrees (see
% the two tiles on the next figure). A \emph{LPT tiling} is a tiling with
% randomly oriented LPT tiles. We can see that even if it is very simple in it
% principle, it draw sophisticated curves with strange properties.
% 
%   Nevertheless, in the straightforward way \FillPackage{} does not work,
% because the \cs{psboxfill} macro store the content of the tile used in a
% \TeX{} box, which is static. So the calling to the random function is done
% only one time, which explain that only one rotation of the tile is used for
% all the tiling. It's only the one of the two rotations which could differ from
% one drawing to the next one...
% 
% ^^A % Truchet (Lewthwaite-Pickover-Truchet) tiling
% ^^A % --------------------------------------------
% 
% \begin{LTXexample}[pos=t]
% ^^A   % LPT prototile
%   \newcommand{\ProtoTileLPT}{%
%     \psset{dimen=middle}
%     \begin{pspicture}(1,1)
%       \psframe(1,1)
%       \psarc(0,0){0.5}{0}{90}
%       \psarc(1,1){0.5}{-180}{-90}
%     \end{pspicture}}
% 
% ^^A   % LPT tile
%   \newcount\Boolean
%   \newcommand{\BasicTileLPT}{%
% ^^A     % From random.tex by Donald Arseneau
%     \setrannum{\Boolean}{0}{1}%
%     \ifnum\Boolean=0
%       \ProtoTileLPT%
%     \else
%       \psrotateleft{\ProtoTileLPT}%
%     \fi}
% 
%   \ProtoTileLPT\hfill\psrotateleft{\ProtoTileLPT}\hfill
%   \psset{unit=0.5}
%   \psboxfill{\BasicTileLPT}
%   \Tiling{(5,5)}
% \end{LTXexample}
% 
%   But, for simple cases, there is a solution to this problem using a mixture
% of PSTricks and PostScript programming. Here the PSTricks
% construction \verb+\pscustom{\code{...}}+ allow to insert PostScript code
% inside the \LaTeX{} + PSTricks one.
% 
%   Programmation is less straightforward, but it has also the advantage to be
% notably faster, as all the tilings operations are done in PostScript, and
% mainly to not be limited by \TeX{} memory (the \TeX{} + PSTricks solution
% I wrote in 1995 for the colored problem was limited to small sizes for this
% reason). Just note also that \cs{pslbrace} and \cs{psrbrace} are two
% PSTricks macros to define and be able to insert the \verb+{+ and \verb+}+
% characters.
% 
% \begin{LTXexample}[pos=t]
% ^^A   % LPT prototile
%   \newcommand{\ProtoTileLPT}{%
%     \psset{dimen=middle}
%     \psframe(1,1)
%     \psarc(0,0){0.5}{0}{90}
%     \psarc(1,1){0.5}{-180}{-90}}
% 
% ^^A   % Counter to change the random seed
%   \newcount\InitCounter
% ^^A   % LPT tile
%   \newcommand{\BasicTileLPT}{%
%     \InitCounter=\the\time
%     \pscustom{\code{%
%       rand \the\InitCounter\space sub 2 mod 0 eq \pslbrace}}
%     \begin{pspicture}(1,1)
%       \ProtoTileLPT
%     \end{pspicture}%
%     \pscustom{\code{\psrbrace \pslbrace}}
%     \psrotateleft{\ProtoTileLPT}%
%     \pscustom{\code{\psrbrace ifelse}}}
% 
%   \psset{unit=0.4,linewidth=0.4pt}
%   \psboxfill{\BasicTileLPT}
%   \Tiling{(15,15)}
% \end{LTXexample}
% 
%   Using the very surprising fact (see \cite{EsperetGirou98}) that
% coloration of these tiles do not depend of their neighbors (even if it is
% difficult to believe as the opposite seems obvious!) but only of the parity of
% the value of row and column positions, we can directly program in the same way
% a colored version of the LPT tiling.
% 
% \setcounter{footnote}{1}
%   We have also introduce in the \FillPackage{} code for \emph{tiling} mode two
% new accessible Post\-Script variables, \texttt{row} and
% \texttt{column}\footnotemark, which can be useful in some circonstances, like
% this one.
% 
% \begin{LTXexample}[pos=t]
% ^^A   % LPT prototile
%   \newcommand{\ProtoTileLPT}[2]{%
%     \psset{dimen=middle,linestyle=none,fillstyle=solid}
%     \psframe[fillcolor=#1](1,1)
%     \psset{fillcolor=#2}
%     \pswedge(0,0){0.5}{0}{90} \pswedge(1,1){0.5}{-180}{-90}}
% ^^A   % Counter to change the random seed
%   \newcount\InitCounter
% ^^A   % LPT tile
%   \newcommand{\BasicTileLPT}[2]{%
%     \InitCounter=\the\time
%     \pscustom{\code{%
%       rand \the\InitCounter\space sub 2 mod 0 eq \pslbrace
%       row column add 2 mod 0 eq \pslbrace}}
%     \begin{pspicture}(1,1)\ProtoTileLPT{#1}{#2}\end{pspicture}%
%     \pscustom{\code{\psrbrace \pslbrace}}
%     \ProtoTileLPT{#2}{#1}%
%     \pscustom{\code{%
%       \psrbrace ifelse \psrbrace \pslbrace row column add 2 mod 0 eq \pslbrace}}
%     \psrotateleft{\ProtoTileLPT{#2}{#1}}\pscustom{\code{\psrbrace \pslbrace}}
%     \psrotateleft{\ProtoTileLPT{#1}{#2}}\pscustom{\code{\psrbrace ifelse \psrbrace ifelse}}}
%   \psboxfill{\BasicTileLPT{red}{yellow}}
%   \Tiling{(4,4)}\hfill
%   \psset{unit=0.4}\psboxfill{\BasicTileLPT{blue}{cyan}}
%   \Tiling{(15,15)}
% \end{LTXexample}
% 
%   Another classic example is to generate coordinates and numerotation for a
% grid. Of course, it is possible to do it directly in PSTricks using nested
% \cs{multido} commands. It would be clearly easy to program, but, nevertheless, 
% for users who have a little knowledge of PostScript programming, this offer
% an alternative which is useful for large cases, because on this way it will
% be notably faster and less computer ressources consuming.
% 
%   Remember here that the tiling is drawn from left to right, and top to
% bottom, and note that the PostScript variable \texttt{x2} give the total
% number of columns.
% 
% \begin{LTXexample}[pos=t]
% ^^A   % \Escape will be the \ character
%   {\catcode`\!=0\catcode`\\=11!gdef!Escape{\}}
%   \newcommand{\ProtoTile}{%
%     \Square\pscustom{%
%       \moveto(-0.9,0.75) % In PSTricks units
%       \code{ /Times-Italic findfont 8 scalefont setfont
%         (\Escape() show row 3 string cvs show (,) show 
%         column 3 string cvs show (\Escape)) show}
%       \moveto(-0.5,0.25) % In PSTricks units
%       \code{ /Times-Bold findfont 18 scalefont setfont
%         1 0 0 setrgbcolor % Red color
%         /center {dup stringwidth pop 2 div neg 0 rmoveto} def
%         row 1 sub x2 mul column add 3 string cvs center show}}}
%   \psboxfill{\ProtoTile}
%   \Tiling{(6,4)}
% \end{LTXexample}
% 
% \subsection{A complete example: the Poisson equation}
% 
%   To finish, we will show a complete real example, a drawing to explain the
% method used to solve the \textsc{Poisson} equation by a domain
% decomposition method, adapted to distributed memory computers. The
% objective is to show the communications required between processes and the
% position of the data to exchange. This code also show some useful and powerful
% technics for PSTricks programming (look specially at the way some higher level
% macros are defined, and how the same object is used to draw the four
% neighbors).
%
%\psset{unit=1cm}
%\newcommand{\Pattern}[1]{%
%   \begin{pspicture}(-0.25,-0.25)(0.25,0.25)\rput{*0}{\psdot[dotstyle=#1]}
%   \end{pspicture}}
%\newcommand{\West}{\Pattern{o}}   \newcommand{\South}{\Pattern{x}}
%\newcommand{\Central}{\Pattern{+}}\newcommand{\North}{\Pattern{square}}
%\newcommand{\East}{\Pattern{triangle}}
%\newcommand{\Cross}{%
%  \pspolygon[unit=0.5,linewidth=0.2,linecolor=red](0,0)(0,1)(1,1)(1,2)(2,2)(2,1)%
%              (3,1)(3,0)(2,0)(2,-1)(1,-1)(1,0)}
%\newcommand{\StylePosition}[1]{\LARGE\textcolor{red}{\textbf{#1}}}
%\newcommand{\SubDomain}[4]{%
%    \psboxfill{#4}\begin{psclip}{\psframe[linestyle=none]#1}%
%      \psframe[linestyle=#3](5,5)\psframe[fillstyle=boxfill]#2%
%    \end{psclip}}
%\newcommand{\SendArea}[1]{\psframe[fillstyle=solid,fillcolor=cyan]#1}
%\newcommand{\ReceiveData}[2]{%
%  \psboxfill{#2}\psframe[fillstyle=solid,fillcolor=yellow,addfillstyle=boxfill]#1}%
%\newcommand{\Neighbor}[2]{%
%    \begin{pspicture}(5,5)
%      \rput{*0}(2.5,2.5){\StylePosition{#1}}
%      \ReceiveData{(0.5,0)(4.5,0.5)}{\Central}\SendArea{(0.5,0.5)(4.5,1)}%
%      \SubDomain{(5,2)}{(0.5,0.5)(4.5,3)}{dashed}{#2}%
%      \pcarc[arcangle=45,arrows=->](0.5,-1.25)(0.5,0.25)%
%      \pcarc[arcangle=45,arrows=->,linestyle=dotted,dotsep=2pt](4.5,0.75)(4.5,-0.75)%
%    \end{pspicture}}%
%  \psset{dimen=middle,dotscale=2,fillloopadd=2}
%\begin{pspicture}(-5.7,-5.7)(5.7,5.7)
%  \rput(0,0){%
%      \begin{pspicture}(5,5)
%        \ReceiveData{(0,0.5)(0.5,4.5)}{\West} \ReceiveData{(4.5,0.5)(5,4.5)}{\East}
%        \ReceiveData{(0.5,4.5)(4.5,5)}{\North}\ReceiveData{(0.5,0)(4.5,0.5)}{\South}
%        \SendArea{(0.5,0.5)(1,4.5)}\SendArea{(4,0.5)(4.5,4.5)}
%        \SendArea{(0.5,0.5)(4.5,1)}\SendArea{(0.5,4)(4.5,4.5)}
%        \SubDomain{(5,5)}{(0.5,0.5)(4.5,4.5)}{solid}{\Central}
%        \psline(1,0.5)(1,4.5)\psline(4,0.5)(4,4.5)%
%        \rput(1.5,4){\Cross}\rput(2,2){\Cross}%
%      \end{pspicture}}%
%  \rput(0,5.5){\Neighbor{N}{\North}}\rput{-90}(5.5,0){\Neighbor{E}{\East}}%
%  \rput{90}(-5.5,0){\Neighbor{W}{\West}}\rput{180}(0,-5.5){\Neighbor{S}{\South}}%
%\end{pspicture}
%
% \begin{lstlisting}
%   \newcommand{\Pattern}[1]{%
%     \begin{pspicture}(-0.25,-0.25)(0.25,0.25)\rput{*0}{\psdot[dotstyle=#1]}
%     \end{pspicture}}
%   \newcommand{\West}{\Pattern{o}}   \newcommand{\South}{\Pattern{x}}
%   \newcommand{\Central}{\Pattern{+}}\newcommand{\North}{\Pattern{square}}
%   \newcommand{\East}{\Pattern{triangle}}
%   \newcommand{\Cross}{%
%     \pspolygon[unit=0.5,linewidth=0.2,linecolor=red](0,0)(0,1)(1,1)(1,2)(2,2)(2,1)
%               (3,1)(3,0)(2,0)(2,-1)(1,-1)(1,0)}
%   \newcommand{\StylePosition}[1]{\LARGE\textcolor{red}{\textbf{#1}}}
%   \newcommand{\SubDomain}[4]{%
%     \psboxfill{#4}
%     \begin{psclip}{\psframe[linestyle=none]#1}
%       \psframe[linestyle=#3](5,5)\psframe[fillstyle=boxfill]#2
%     \end{psclip}}
%   \newcommand{\SendArea}[1]{\psframe[fillstyle=solid,fillcolor=cyan]#1}
%   \newcommand{\ReceiveData}[2]{%
%     \psboxfill{#2}
%     \psframe[fillstyle=solid,fillcolor=yellow,addfillstyle=boxfill]#1}
%   \newcommand{\Neighbor}[2]{%
%     \begin{pspicture}(5,5)
%       \rput{*0}(2.5,2.5){\StylePosition{#1}}
%       \ReceiveData{(0.5,0)(4.5,0.5)}{\Central}\SendArea{(0.5,0.5)(4.5,1)}
%       \SubDomain{(5,2)}{(0.5,0.5)(4.5,3)}{dashed}{#2}%
% ^^A       % Receive and send arrows
%       \pcarc[arcangle=45,arrows=->](0.5,-1.25)(0.5,0.25)
%       \pcarc[arcangle=45,arrows=->,linestyle=dotted,dotsep=2pt](4.5,0.75)(4.5,-0.75)
%     \end{pspicture}}
%   \psset{dimen=middle,dotscale=2,fillloopadd=2}
%   \begin{pspicture}(-5.7,-5.7)(5.7,5.7)
% ^^A     % Central domain
%     \rput(0,0){%
%       \begin{pspicture}(5,5)
% ^^A         % Receive from West, East, North and South
%         \ReceiveData{(0,0.5)(0.5,4.5)}{\West} \ReceiveData{(4.5,0.5)(5,4.5)}{\East}
%         \ReceiveData{(0.5,4.5)(4.5,5)}{\North}\ReceiveData{(0.5,0)(4.5,0.5)}{\South}
% ^^A         % send area for West, East, North and South
%         \SendArea{(0.5,0.5)(1,4.5)} \SendArea{(4,0.5)(4.5,4.5)}
%         \SendArea{(0.5,0.5)(4.5,1)} \SendArea{(0.5,4)(4.5,4.5)}
% ^^A         % Central domain
%         \SubDomain{(5,5)}{(0.5,0.5)(4.5,4.5)}{solid}{\Central}
% ^^A         % Redraw overlapped linesY
%         \psline(1,0.5)(1,4.5)  \psline(4,0.5)(4,4.5)
% ^^A         % Two crossesY
%         \rput(1.5,4){\Cross}  \rput(2,2){\Cross}
%       \end{pspicture}}
% ^^A     % The four neighborsY
%     \rput(0,5.5){\Neighbor{N}{\North}}     \rput{-90}(5.5,0){\Neighbor{E}{\East}}
%     \rput{90}(-5.5,0){\Neighbor{W}{\West}} \rput{180}(0,-5.5){\Neighbor{S}{\South}}
%   \end{pspicture}
% \end{lstlisting}
%
%
%
% Bibliography
% \begin{thebibliography}{99}
% \bibitem{PostScript95} Adobe, Systems~Incorporated, \emph{PostScript Language
% Reference Manual}, Addison-Wesley, 2~edition, 1995.
%
% \bibitem{Bolek98} Piotr Bolek, \MP{} and patterns, \emph{\TUB}, Volume~19,
% Number~3, pages 276--283, September 1998, \CTANref{mpattern}.
%
% \bibitem{MLgraphTSI} Emmanuel Chailloux, Guy Cousineau and Asc\'ander
% Su\'arez, Programmation fonctionnelle de graphismes pour la production
% d'illustrations techniques, \emph{Technique et science informatique},
% Volume~15, Number~7, pages 977--1007, 1996 (in french).
%
% \bibitem{Deledicq97} Andr\'e Deledicq, \emph{Le monde des pavages}, ACL
% \'Editions, 1997 (in french).
%
% \bibitem{EsperetGirou98} Philippe Esperet and Denis Girou,
% Coloriage du pavage dit de Truchet, Cahiers GUTenberg, Number~31,
% pages 5--18, December~1998  (in french).
%
% \bibitem{Girou94} Denis Girou, Pr\'esentation de PSTricks, \emph{Cahiers
% GUTenberg}, Number~16, pages 21--70, February~1994 (in french).
%
% \bibitem{LGC97} Michel Goossens, Sebastian Rahtz and Frank Mittelbach,
% \emph{The \LaTeX{} Graphics Companion}, Addison-Wesley, 2005.
%
% \bibitem{GS87} Branko Gr\"unbaum and Geoffrey Shephard, \emph{Tilings and
% Patterns}, Freeman and Company, 1987.
%
% \bibitem{Hoenig97} Alan Hoenig, \emph{\TeX{} Unbound: \LaTeX{} \& \TeX{}
% Strategies, Fonts, Graphics, and More}, Oxford University Press, 1997.
%
% \bibitem{XYpic} Kristoffer~H. Rose and Ross Moore, \XYpic. Pattern and Tile
% extension, available from \CTAN, 1991-1998, \CTANref{xypic}.
%
% \bibitem{LAAN96} Kees van der Laan, Paradigms: Just a little bit of PostScript,
% \emph{MAPS}, Volume~17, pages 137--150, 1996.
%
% \bibitem{LAAN97} Kees van der Laan, Tiling in PostScript and \MF{} -- Escher's
% wink, \emph{MAPS}, Volume~19, Number~2, pages 39--67, 1997.
%
% \bibitem{vanZandt93} Timothy Van Zandt, PSTricks. PostScript macros for
% Generic \TeX, available from \CTAN, 1993, \CTANref{pstricks}.
%
% \bibitem{vanZandtGirou94} Timothy Van Zandt and Denis Girou, Inside PSTricks,
% \emph{\TUB}, Volume~15, Number~3, pages 239--246, September 1994.
%
%
% \bibitem{voss07} Herbert Vo\ss, PSTricks -- Graphics for \TeX\ and \LaTeX, DANTE/Lehmanns, 4th ed., 2007.
% \bibitem{Wang65} Hao Wang, Games, Logic and Computers, \emph{Scientific
% American}, pages 98--106, November 1965.
% \end{thebibliography}
%
%
% \StopEventually{}
%
% ^^A .................... End of the documentation part ....................
%
% \section{Driver file}
%
%   The next bit of code contains the documentation driver file for \TeX{},
% i.e., the file that will produce the documentation you are currently
% reading. It will be extracted from this file by the \texttt{docstrip}
% program.
%
%    \begin{macrocode}
%<*driver>
\documentclass{ltxdoc}
\GetFileInfo{pst-fill.dtx}
%
\usepackage[T1]{fontenc}
\usepackage{lmodern}               % For PDF
\usepackage{graphicx}              % `graphicx' LaTeX standard package
\usepackage{showexpl}
\usepackage{mflogo}                % For the MetaFont and MetaPost logos
\input{random.tex}                 % Random macros from Donald Arseneau
\usepackage{url}                   % URLs convenient typesetting
\usepackage{multido}               % General loop macro
\usepackage[dvipsnames]{pstricks}  % PSTricks with the `color' extension
\usepackage{pst-text}              % PSTricks package for character path
\usepackage{pst-grad}              % PSTricks package for gradient filling
\usepackage{pst-node}              % PSTricks package for nodes
\usepackage[tiling]{pst-fill}      % PSTricks package for filling/tiling
%
\AtBeginDocument{%
%  \OnlyDescription % comment out for implementation details
  \EnableCrossrefs
  \CodelineIndex
  \RecordChanges}
\AtEndDocument{%
  \PrintIndex
  \setcounter{IndexColumns}{1}
  \PrintChanges}
\hbadness=7000            % Over and under full box warnings
\hfuzz=3pt
\begin{document}
  \DocInput{pst-fill.dtx}
\end{document}
%</driver>
%    \end{macrocode}
%
% \section{\texttt{pst-fill} \LaTeX{} wrapper}
%
%    \begin{macrocode}
%<*latex-wrapper>
\RequirePackage{pstricks}
\ProvidesPackage{pst-fill}[2005/09/13 package wrapper for 
  pst-fill.tex (hv)]
\DeclareOption{tiling}{\def\PstTiling{true}}
\ProcessOptions\relax
\input{pst-fill.tex}
\ProvidesFile{pst-fill.tex}
  [\filedate\space v\fileversion\space `PST-fill' (tvz,dg)]
%</latex-wrapper>
%    \end{macrocode}
%
%
% \section{Pst-Fill Package{} code}
%
%    \begin{macrocode}
%<*pst-fill>
%    \end{macrocode}
%
% \subsection{Preamble}
%
%   Who we are.
%
%    \begin{macrocode}
\def\fileversion{1.01}
\def\filedate{2007/03/10}
\message{`PST-Fill' v\fileversion, \filedate\space (tvz,dg,hv)}
\csname PSTboxfillLoaded\endcsname
\let\PSTboxfillLoaded\endinput
%    \end{macrocode}
%
%   Require the main PSTricks package.
%
%    \begin{macrocode}
\ifx\PSTricksLoaded\endinput\else\input pstricks.tex\fi
%    \end{macrocode}
%
%   interface to the extended `\textsf{keyval}' package.
%
%    \begin{macrocode}
\ifx\PSTXKeyLoaded\endinput\else\input pst-xkey\fi
%
%    \end{macrocode}
%
%   Catcodes changes and defining the family name for xkeyval.
%
%    \begin{macrocode}
\edef\PstAtCode{\the\catcode`\@}\catcode`\@=11\relax

\pst@addfams{pst-fill}
%
%    \end{macrocode}
%
%
% \subsection{The size of the box}
% \begin{macro}{pst@@boxfillsize}
%    \begin{macrocode}
%
\def\pst@@boxfillsize#1(#2,#3)#4(#5,#6)#7(#8\@nil{%
  \begingroup
    \ifx\@empty#7\relax
      \pst@dima\z@
      \pst@dimb\z@
      \pssetxlength\pst@dimc{#2}%
      \pssetylength\pst@dimd{#3}%
    \else
      \pssetxlength\pst@dima{#2}%
      \pssetylength\pst@dimb{#3}%
      \pssetxlength\pst@dimc{#5}%
      \pssetylength\pst@dimd{#6}%
    \fi
    \xdef\pst@tempg{%
      \pst@dima=\number\pst@dima sp
      \pst@dimb=\number\pst@dimb sp
      \pst@dimc=\number\pst@dimc sp
      \pst@dimd=\number\pst@dimd sp }%
  \endgroup
  \let\psk@boxfillsize\pst@tempg}
%    \end{macrocode}
% \end{macro}
%

% \subsection{Definition of the parameters}
%
%    \begin{macrocode}
\define@key[psset]{pst-fill}{boxfillsize}{%
  \def\pst@tempg{#1}\def\pst@temph{auto}%
  \ifx\pst@tempg\pst@temph
    \let\psk@boxfillsize\relax
  \else
    \pst@@boxfillsize#1(\z@,\z@)\@empty(\z@,\z@)(\@nil
  \fi}
\psset{boxfillsize={(-15cm,-15cm)(15cm,15cm)}}
\define@key[psset]{pst-fill}{boxfillcolor}{\pst@getcolor{#1}\psboxfillcolor}
\psset{boxfillcolor=black}% hv
\define@key[psset]{pst-fill}{boxfillangle}{\pst@getangle{#1}\psk@boxfillangle}
\psset{boxfillangle=0}
\define@key[psset]{pst-fill}{fillsepx}{%
  \pst@getlength{#1}\psk@fillsepx}
\define@key[psset]{pst-fill}{fillsepy}{%
  \pst@getlength{#1}\psk@fillsepy}
\define@key[psset]{pst-fill}{fillsep}{%
  \pst@getlength{#1}\psk@fillsepx%
  \let\psk@fillsepy\psk@fillsepx}
\psset{fillsep=2pt}

\ifx\PstTiling\@undefined
  \define@key[psset]{pst-fill}{fillcycle}{\pst@getint{#1}\psk@fillcycle}
  \psset{fillcycle=0}
\else
  \define@key[psset]{pst-fill}{fillangle}{\pst@getangle{#1}\psk@boxfillangle}
  \define@key[psset]{pst-fill}{fillsize}{%
      \def\pst@tempg{#1}\def\pst@temph{auto}%
      \ifx\pst@tempg\pst@temph\let\psk@boxfillsize\relax
      \else\pst@@boxfillsize#1(\z@,\z@)\@empty(\z@,\z@)(\@nil\fi}
  \psset{fillsep=0,fillsize=auto}
  \define@key[psset]{pst-fill}{fillcyclex}{\pst@getint{#1}\psk@fillcyclex}
  \define@key[psset]{pst-fill}{fillcycley}{\pst@getint{#1}\psk@fillcycley}
  \define@key[psset]{pst-fill}{fillcycle}{%
    \pst@getint{#1}\psk@fillcyclex\let\psk@fillcycley\psk@fillcyclex}
  \psset{fillcycle=0}
  \define@key[psset]{pst-fill}{fillmovex}{\pst@getlength{#1}\psk@fillmovex}
  \define@key[psset]{pst-fill}{fillmovey}{\pst@getlength{#1}\psk@fillmovey}
  \define@key[psset]{pst-fill}{fillmove}{%
      \pst@getlength{#1}\psk@fillmovex\let\psk@fillmovey\psk@fillmovex}
  \psset{fillmove=0pt}
  \define@key[psset]{pst-fill}{fillloopaddx}{\pst@getint{#1}\psk@fillloopaddx}
  \define@key[psset]{pst-fill}{fillloopaddy}{\pst@getint{#1}\psk@fillloopaddy}
  \define@key[psset]{pst-fill}{fillloopadd}{%
    \pst@getint{#1}\psk@fillloopaddx\let\psk@fillloopaddy\psk@fillloopaddx}
  \psset{fillloopadd=0}
%    \end{macrocode}
%
%    \begin{macrocode}
% For debugging (to debug, set PstDebug=1)
% we now use the one from pstricks to prevent a clash with package
% pstricks                        2004-06-22
%%    \define@key[psset]{pst-fill}{PstDebug}{\pst@getint{#1}\psk@PstDebug}
    \psset{PstDebug=0}
\fi
% DG addition end
%    \end{macrocode}

% \subsection{Definition of the fill box}
% \begin{macro}{psboxfill}
%    \begin{macrocode}
\newbox\pst@fillbox
\def\psboxfill{\pst@killglue\pst@makebox\psboxfill@i}
\def\psboxfill@i{\setbox\pst@fillbox\box\pst@hbox\ignorespaces}
%    \end{macrocode}
% \end{macro}
% \subsection{The main macros}
%
% \begin{macro}{psfs@boxfill}
%    \begin{macrocode}
\def\psfs@boxfill{%
  \ifvoid\pst@fillbox
    \@pstrickserr{Fill box is empty. Use \string\psboxfill\space first.}\@ehpa
  \else
    \ifx\psk@boxfillsize\relax \pst@AutoBoxFill
    \else\pst@ManualBoxFill\fi
  \fi}
%    \end{macrocode}
% \end{macro}
%
% \begin{macro}{pst@ManualBoxFill}
%    \begin{macrocode}
\def\pst@ManualBoxFill{%
  \leavevmode
  \begingroup
    \pst@FlushCode
    \begin@psclip
    \pstVerb{clip}%
    \expandafter\pst@AddFillBox\psk@boxfillsize
    \end@psclip
  \endgroup}
%    \end{macrocode}
% \end{macro}
%
% \begin{macro}{pst@FlushCode}
%    \begin{macrocode}
\def\pst@FlushCode{%
  \pst@Verb{%
    /mtrxc CM def
    CP CP T
    \tx@STV
    \psk@origin
    \psk@swapaxes
    \pst@newpath
    \pst@code
    mtrxc setmatrix
    moveto
    0 setgray}%
  \gdef\pst@code{}}
%    \end{macrocode}
% \end{macro}
%
% \begin{macro}{pst@AddFillBox}
%    \begin{macrocode}
\def\pst@AddFillBox#1 #2 #3 #4 {%
  \begingroup
    \setbox\pst@fillbox=\vbox{%
      \hbox{\unhcopy\pst@fillbox\kern\psk@fillsepx\p@}%
      \vskip\psk@fillsepy\p@}%
    \psk@boxfillsize
    \pst@cnta=\pst@dimc
    \advance\pst@cnta-\pst@dima
    \divide\pst@cnta\wd\pst@fillbox
    \pst@cntb=\pst@dimd
    \advance\pst@cntb-\pst@dimb
    \pst@dimd=\ht\pst@fillbox
    \divide\pst@cntb\pst@dimd
    \def\pst@tempa{%
      \pst@tempg
      \copy\pst@fillbox
      \advance\pst@cntc\@ne
      \ifnum\pst@cntc<\pst@cntd\expandafter\pst@tempa\fi}%
    \let\pst@tempg\relax
    \pst@cntc-\tw@
    \pst@cntd\pst@cnta
    \setbox\pst@fillbox=\hbox to \z@{%
      \kern\pst@dima
      \kern-\wd\pst@fillbox
      \pst@tempa
      \hss}%
    \pst@cntd\pst@cntb
%% DG modification begin - Dec. 11, 1997 - Patch 2
    \ifx\PstTiling\@undefined
      \ifnum\psk@fillcycle=\z@\pst@ManualFillCycle\fi
    \else
      \ifnum\psk@fillcyclex=\z@\pst@ManualFillCycle\fi
    \fi
%% DG modification end
    \global\setbox\pst@boxg=\vbox to\z@{%
      \offinterlineskip
      \vss
      \pst@tempa
      \vskip\pst@dimb}%
  \endgroup
  \setbox\pst@fillbox\box\pst@boxg
  \pst@rotate\psk@boxfillangle\pst@fillbox
  \box\pst@fillbox}
%    \end{macrocode}
% \end{macro}
%
% \begin{macro}{pst@ManualFillCycle}
%    \begin{macrocode}
\def\pst@ManualFillCycle{%
  \ifx\PstTiling\@undefined
    \pst@cntg=\psk@fillcycle
  \else
    \pst@cntg=\psk@fillcyclex
  \fi
  \pst@dimg=\wd\pst@fillbox
  \ifnum\pst@cntg=\z@
  \else
  \divide\pst@dimg\pst@cntg
  \fi
  \ifnum\pst@cntg<\z@\pst@cntg=-\pst@cntg\fi
  \advance\pst@cntg\m@ne
  \pst@cnth=\pst@cntg
  \def\pst@tempg{%
    \ifnum\pst@cnth<\pst@cntg\advance\pst@cnth\@ne\else\pst@cnth\z@\fi
    \moveright\pst@cnth\pst@dimg}}
%    \end{macrocode}
% \end{macro}
%
%% Auto box fill:        !! Fix dictionary
%
% \subsection{The PostScript subroutines}
%
%    \begin{macrocode}
%% DG addition begin - Apr. 8, 1997 and Dec. 1997 - Patch 2
\ifx\PstTiling\@undefined
\pst@def{AutoFillCycle}<%
  /c ED
  /n 0 def
  /s {
    /x x w c div n mul add def
    /n n c abs 1 sub lt { n 1 add } { 0 } ifelse def
  } def>

\pst@def{BoxFill}<%
  gsave
    gsave \tx@STV CM grestore dtransform CM idtransform
    abs /h ED abs /w ED
    pathbbox
    h div round 2 add cvi /y2 ED
    w div round 2 add cvi /x2 ED
    h div round 2 sub cvi /y1 ED
    w div round 2 sub cvi /x1 ED
    /y2 y2 y1 sub def
    /x2 x2 x1 sub def
    CP
    y1 h mul sub neg /y1 ED
    x1 w mul sub neg /x1 ED
    clip
    y2 {
      /x x1 def
      s
      x2 {
        save CP x y1
%% patch 4   hv --------------
        \ifx\VTeXversion\undefined
        \else
%%============ mv: 09-10-01 ??? this is likely to be a right change
        neg
%%============
        \fi
%% end patch 4
T moveto Box restore
        /x x w add def
      } repeat
      /y1 y1 h add def
    } repeat
    % Next line not useful... To see that, suppress clipping (DG)
    CP x y1 T moveto Box
  currentpoint currentfont grestore setfont moveto>
\else
%% DG modification begin - Apr. 8, 1997 and Nov. / Dec. 1997 - Patch 2
\pst@def{AutoFillCycleX}<%
  /cX ED
  /nX 0 def
  /CycleX {
    /x x w cX div nX mul add def
    /nX nX cX abs 1 sub lt { nX 1 add } { 0 } ifelse def
  } def>
\pst@def{AutoFillCycleY}<%
  /cY ED
  /mY 0 def
  /nY 0 def
  /CycleY {
    /y1 y1 h cY div mY mul sub def
    nY cY abs 1 sub lt { /nY nY 1 add def /mY 1 def }
                       { /nY 0 def        /mY cY abs 1 sub neg def } ifelse
  } def>

\pst@def{BoxFill}<%
  gsave
    gsave \tx@STV CM grestore dtransform CM idtransform
    abs /h ED abs /w ED
    pathbbox
    h div round 2 add cvi /y2 ED
    w div round 2 add cvi /x2 ED
    h div round 2 sub cvi /y1 ED
    w div round 2 sub cvi /x1 ED
    /CoefLoopX 0 def
    /CoefLoopY 0 def
    /CoefMoveX 0 def
    /CoefMoveY 0 def
    \psk@boxfillangle\space 0 ne {/CoefLoopX 8 def /CoefLoopY 8 def} if
    \psk@fillcyclex\space 0 ne {/CoefLoopX CoefLoopX 1 add def} if
    \psk@fillcycley\space 0 ne {/CoefLoopY CoefLoopY 1 add def} if
    \psk@fillmovex\space 0 ne
      {/CoefLoopX CoefLoopX 2 add def
       \psk@fillmovex\space 0 gt {/CoefMoveX CoefLoopX def}
                           {/CoefMoveX CoefLoopX neg def} ifelse} if
    \psk@fillmovey\space 0 ne
      {/CoefLoopY CoefLoopY 2 add def
       \psk@fillmovey\space 0 gt {/CoefMoveY CoefLoopY def}
                           {/CoefMoveY CoefLoopY neg def} ifelse} if
    \psk@fillsepx\space 0 ne {/CoefLoopX CoefLoopX 1 add def} if
    \psk@fillsepy\space 0 ne {/CoefLoopY CoefLoopY 1 add def} if
    /CoefLoopX CoefLoopX \psk@fillloopaddx\space add def
    /CoefLoopY CoefLoopY \psk@fillloopaddy\space add def
    /x2 x2 x1 sub 4 sub CoefLoopX 2 mul add def
    /y2 y2 y1 sub 4 sub CoefLoopY 2 mul add def
%% We must fix the origin of tiling, as it must not vary according other stuff
%% in the page!
    w x1 CoefLoopX add CoefMoveX add mul
      h y1 y2 add 1 sub CoefLoopY sub CoefMoveY sub mul moveto
    CP
    y1 h mul sub neg /y1 ED
    x1 w mul sub neg /x1 ED
%%  hv 2004-06-22   to prevent clash with pst-gr3d
%%    \psk@PstDebug 0 eq {clip} if
    \Pst@Debug 0 eq {clip} if
%% end hv
    \psk@fillmovex\space \psk@fillmovey
    gsave \tx@STV CM grestore dtransform CM idtransform
    /hmove ED /wmove ED
    /row 0 def
   y2 {
       /row row 1 add def
       /column 0 def
       /x x1 def
       CycleX
       save
       x2 {
          /column column 1 add def
          CycleY
          save CP x y1
%% patch 4   hv --------------
          \ifx\VTeXversion\undefined
          \else
%%============ mv: 09-10-01 ??? this is likely to be a right change
          neg
%%============
          \fi
  T moveto Box restore
          /x x w add def
          0 hmove translate
          } repeat
       restore
       /y1 y1 h add def
       wmove 0 translate
       } repeat
  currentpoint currentfont grestore setfont moveto>
\fi
%    \end{macrocode}

%    \begin{macrocode}
\def\pst@AutoBoxFill{%
  \leavevmode
  \begingroup
    \pst@stroke
    \pst@FlushCode
    \pst@Verb{\psk@boxfillangle\space \tx@RotBegin}%
    \pstVerb{\pst@dict /Box \pslbrace end}%
    \ifx\PstTiling\@undefined
    \else
      \ifx\pst@tempa\@undefined % Undefined for instance for \pscharpath
      \else\ifx\pst@tempa\@empty\else
        \def\pst@temph{0}%
        \ifx\pst@tempa\pst@temph
        \else
          \pstVerb{/TR {pop pop currentpoint translate \pst@tempa\space translate } def}%
        \fi
      \fi\fi
    \fi
    \hbox to \z@{\vbox to\z@{\vss\copy\pst@fillbox\vskip-\dp\pst@fillbox}\hss}%
    \ifx\PstTiling\@undefined
      \pstVerb{%
        tx@Dict begin \psrbrace def
        \ifnum\psk@fillcycle=\z@
          /s {} def
        \else
          \psk@fillcycle \tx@AutoFillCycle
        \fi
        \pst@number{\wd\pst@fillbox}%
        \psk@fillsepx\space add
        \pst@number{\ht\pst@fillbox}%
        \pst@number{\dp\pst@fillbox}%
        \psk@fillsepy\space add add
        \tx@BoxFill
        end}%
      \else
      \pstVerb{%
        tx@Dict begin \psrbrace def
        \ifnum\psk@fillcyclex=\z@
          /CycleX {} def
        \else
          \psk@fillcyclex\space \tx@AutoFillCycleX
        \fi
        \ifnum\psk@fillcycley=\z@
          /CycleY {} def
        \else
          \psk@fillcycley\space \tx@AutoFillCycleY
        \fi
        \pst@number{\wd\pst@fillbox}%
        \psk@fillsepx\space add
        \pst@number{\ht\pst@fillbox}%
        \pst@number{\dp\pst@fillbox}%
        \psk@fillsepy\space add add
        \tx@BoxFill
        end}%
    \fi
    \pst@Verb{\tx@RotEnd}%
  \endgroup}
%    \end{macrocode}
% \subsection{Closing}
%
%   Catcodes restoration.
%
%    \begin{macrocode}
\catcode`\@=\PstAtCode\relax
%    \end{macrocode}
%
%    \begin{macrocode}
%</pst-fill>
%    \end{macrocode}
%
% \Finale
%
\endinput
%%
%% End of file `pst-fill.dtx'
+\newline
%add the following definition:\newline
%\verb+\def\PstTiling{true}+
%
%  To obtain the original behaviour, just don't use the \emph{tiling} optional
%keyword at loading.
%
%  Take care than in \emph{tiling} mode, I introduce also some other changes.
%First I define aliases on some parameter names for consistancy (all specific
%parameters will begin by the \texttt{fill} prefix in this case) and I change
%some default values, which were not well adapted for tilings (\texttt{fillsep}
%is set to 0 and as explained \texttt{fillsize} set to \texttt{auto}). I rename 
%\texttt{fillcycle} to \texttt{fillcyclex}. I also restore normal way so that
%the frame of the area is drawn and all line (\texttt{linestyle},
%\texttt{linecolor}, \texttt{doubleline}, etc.) parameters are now active (but
%there are not in non \emph{tiling} mode). And I also introduce new parameters
%to control the tilings (see below).
%
%  \textbf{In all the following examples, we will consider only the
% \emph{tiling} mode.}
%
%  To do a tiling, we have just to define the pattern with the
% \verb+\psboxfill+ macro and to use the new \texttt{fillstyle}
% \verb+boxfill+.
%
%  Note that tilings are drawn from left to right and top to bottom, which can
%have an importance in some circonstances.
%
%  PostScript programmers can be also interested to know that, even in the
%\emph{automatic} mode, the iterations of the pattern are managed directly by
%the PostScript code of the package which used only PostScript Level 1
%operators. The special ones introduced in Level 2 for drawing of patterns
%\cite[section 4.9]{PostScript95} are not used.
%
%  And first, for conveniance, we define a simple \cs{Tiling} macro, which
%will simplify our examples:
%
%\begin{verbatim}
%  \newcommand{\Tiling}[2][]{%
%    \edef\Temp{#1}%
%    \begin{pspicture}#2
%      \ifx\Temp\empty
%        \psframe[fillstyle=boxfill]#2
%      \else
%        \psframe[fillstyle=boxfill,#1]#2
%      \fi
%    \end{pspicture}}
%\end{verbatim}
%
%
%\newcommand{\Tiling}[2][]{%
%  \edef\Temp{#1}%
%  \begin{pspicture}#2
%    \ifx\Temp\empty
%      \psframe[fillstyle=boxfill]#2
%    \else
%      \psframe[fillstyle=boxfill,#1]#2
%    \fi
% \end{pspicture}}
%
%\subsection{Parameters}
%
%  There are \textbf{14} specific parameters available to change the way the
% filling/tiling is defined, and one debugging option.
%
% \begin{Description}{2cm}
%  \item [fillangle (real)\hfill :] the value of the rotation
%  applied to the patterns (\emph{Default:~0}).
% \end{Description}
%
%
%   In this case, we must force the tiling area to be notably larger than the
% area to cover, to be sure that the defined area will be covered after rotation.
% \lstset{gobble=2}
% \begin{LTXexample}
% \newcommand{\Square}{%
%   \begin{pspicture}(1,1)
%     \psframe[dimen=middle](1,1)
%   \end{pspicture}}
% \psset{unit=0.5}
% \psboxfill{\Square}
% \Tiling[fillangle=45]{(3,3)}\quad
% \Tiling[fillangle=-60]{(3,3)}
% \end{LTXexample}
% 
% \newcommand{\Square}{\begin{pspicture}(1,1)\psframe[dimen=middle](1,1)\end{pspicture}}
% 
% \begin{Description}{2cm}
%   \setcounter{footnote}{1}
%   \item[\texttt{fillsepx} (real$\|$dim) :] value of the horizontal
%   separation between consecutive patterns (\emph{Default:~0 for
%   tilings\footnotemark, 2pt otherwise}).  \footnotetext{This option was added
%   by me, is not part of the original package and is available only if the
%   \texttt{tiling} keyword is used when loading the package.}
%   \setcounter{footnote}{1}
%   \item [\texttt{fillsepy} (real$\|$dim)\hfill :] value of the vertical
%   separation between consecutive patterns (\emph{Default:~0 for
%   ti\-lings\footnotemark, 2pt otherwise}).
%   \setcounter{footnote}{1}
%   \item [\texttt{fillsep} (real$\|$dim)\hfill :] value of horizontal and
%   vertical separations between consecutive patterns (\emph{Default:~0 for
%   tilings\footnotemark, 2pt otherwise}).
% \end{Description}
% 
%   These values can be negative, which allow the tiles to overlap.
% 
% \begin{LTXexample}
% \psset{unit=0.5}
% \psboxfill{\Square}
% \Tiling[fillsepx=2mm]{(3,3)} 
% \Tiling[fillsepy=1mm]{(3,3)}\\
% \Tiling[fillsep=0.5]{(3,3)} 
% \Tiling[fillsep=-0.5]{(3,3)}
% \end{LTXexample}
% 
% \begin{Description}{2cm}
%   \item [\texttt{fillcyclex}\footnotemark\ (integer)\hfill :] Shift
%   coefficient applied to each row (\emph{Default:~0}).
%   \footnotetext{It was \texttt{fillcycle} in the original version.}
%   \setcounter{footnote}{1}
%   \item [\texttt{fillcycley}\footnotemark\ (integer)\hfill :] Same thing for
%   columns (\emph{Default:~0}).
%   \setcounter{footnote}{1}
%   \item [\texttt{fillcycle}\footnotemark\ (integer)\hfill :] Allow to fix
%   both \texttt{fillcyclex} and \texttt{fillcycley} directly to the same value
%   (\emph{Default:~0}).
% \end{Description}
% 
%   For instance, if \texttt{fillcyclex} is 2, the second row of patterns will
% be horizontally shifted by a factor of $\frac{1}{2}=0.5$, and by a factor of
% 0.333 if \texttt{fillcyclex} is 3, etc.). These values can be negative.
% 
% \begin{LTXexample}[width=0.35\linewidth]
% \psset{unit=0.5}
% \psboxfill{\Square}
% \newcommand{\TilingA}[1]{\Tiling[fillcyclex=#1]{(3,3)}}
% \TilingA{0} \TilingA{1}\\
% \TilingA{2} \TilingA{3}\\[3mm]
% \TilingA{4} \TilingA{5}\\
% \TilingA{6} \TilingA{-3}\\[3mm]
% \Tiling[fillcycley=2]{(3,3)}
% \Tiling[fillcycley=3]{(3,3)}\\
% \Tiling[fillcycley=-3]{(3,3)}
% \Tiling[fillcycle=2]{(3,3)}
% \end{LTXexample}
% 
% \begin{Description}{2cm}
%   \setcounter{footnote}{1}
%   \item [\texttt{fillmovex}\footnotemark\ (real$\|$dim)\hfill :] value of the
%   horizontal moves between consecutive patterns (\emph{Default:~0}).
%   \setcounter{footnote}{1}
%   \item [\texttt{fillmovey}\footnotemark\ (real$\|$dim)\hfill :] value of the
%   vertical moves between consecutive patterns (\emph{Default:~0}).
%   \setcounter{footnote}{1}
%   \item [\texttt{fillmove}\footnotemark\ (real$\|$dim)\hfill :] value of
%   horizontal and vertical moves between consecutive patterns
%   (\emph{Default:~0}).
% \end{Description}
% 
%   These parameters allow the patterns to overlap and to draw some special
% kinds of tilings. They are implemented only for the \emph{automatic} and
% \emph{tiling} modes and their values can be negative.
% 
%   In some cases, the effect of these parameters will be the same that with the 
% \texttt{fillcycle?} ones, but you can see that it is not true for some other
% values.
% 
% \begin{LTXexample}
% \psset{unit=0.5}
% \psboxfill{\Square}
% \Tiling[fillmovex=0.5]{(3,3)} 
% \Tiling[fillmovey=0.5]{(3,3)}\\
% \Tiling[fillmove=0.5]{(3,3)}
% \Tiling[fillmove=-0.5]{(3,3)}
% \end{LTXexample}
% 
% \begin{Description}{2cm}
%   \item [\texttt{fillsize}
%   (auto$\|$\{(real$\|$dim,real$\|$dim)(real$\|$dim,real$\|$dim)\}) :] The
%   choice of \emph{automatic} mode or the size of the area in \emph{manual}
%   mode. If first pair values are not given, (0,0) is used. (\emph{Default:
%   auto when \emph{tiling} mode is used, {(-15cm,-15cm)(15cm,15cm)}
%   otherwise}).
% \end{Description}
% 
%   As explained in the introduction, the \emph{manual} mode can require very
% huge amount of computer ressources. So, it usage is to discourage in front off
% the \emph{automatic} mode. It seems only useful in special circonstances, in
% fact when the \emph{automatic} mode failed, which is known only in one case,
% for some kinds of EPS files, as the ones produce by dump of portions of
% screens (see \ref{sec:GraphicFiles}).
% 
% \begin{Description}{2cm}
%   \setcounter{footnote}{1}
%   \item [\texttt{fillloopaddx}\footnotemark\ (integer)\hfill :] number of
%   times the pattern is added on left and right positions (\emph{Default:~0}).
%   \setcounter{footnote}{1}
%   \item [\texttt{fillloopaddy}\footnotemark\ (integer)\hfill :] number of
%   times the pattern is added on top and bottom positions (\emph{Default:~0}).
%   \setcounter{footnote}{1}
%   \item [\texttt{fillloopadd}\footnotemark\ (integer)\hfill :] number of
%   times the pattern is added on left, right, top and bottom positions
%   (\emph{Default:~0}).
% \end{Description}
% 
%   These parameters are only useful in special circonstances, as for complex
% patterns when the size of the rectangular box used to tile the area doesn't 
% correspond to the pattern itself (see an example in Figure~\ref{fig:Sheeps})
% and also sometimes when the size of the pattern is not a divisor of the size
% of the area to fill and that the number of loop repeats is not properly
% computed, which can occur.
% 
%   They are implemented only for the \emph{tiling} mode.
% 
% \begin{Description}{2cm}
%   \setcounter{footnote}{1}
%   \item [\texttt{PstDebug}\footnotemark\ (integer, 0 or 1)\hfill :] to
%   require to see the exact tiling done, without clipping (\emph{Default:~0}).
% \end{Description}
% 
%   It's mainly useful for debugging or to understand better how the tilings
% are done. It is implemented only for the \emph{tiling} mode.
% 
% \begin{LTXexample}
% \psset{unit=0.3,PstDebug=1}
% \psboxfill{\Square}
% \psset{linewidth=1mm}
% \Tiling{(2,2)}\\[5mm]
% \Tiling[fillcyclex=2]{(2,2)}\\[1cm]
% \Tiling[fillmove=0.5]{(2,2)}
% \end{LTXexample}
% 
% \vspace{3cm}
% \section{Examples}
% 
%   In fact this unique \cs{psboxfill} macro allow a lot a variations and
% different usages. We will try here to demonstrate this.
% 
% \subsection{Kind of tiles}
% \label{sec:KindTiles}
% 
%   Of course, we can access to all the power of PSTricks macros to define the
% \emph{tiles} (\emph{patterns}) used. So, we can define complicated ones.
% 
%   Here we give four other Archimedian tilings (those built with only some
% regular polygons) among the twelve existing, first discovered completely by
% Johanes \textsc{Kepler} at the beginning of 17th century \cite{GS87}, the two
% other \emph{regular} ones with the tiling by squares, formed by a unique
% regular polygon, and two other formed by two different regular polygons.
% 
% \begin{LTXexample}[pos=t]
%   \newcommand{\Triangle}{%
%     \begin{pspicture}(1,1)
%       \pstriangle[dimen=middle](0.5,0)(1,1)
%     \end{pspicture}}
%   \newcommand{\Hexagon}{
% ^^A sin(60)=0.866
%     \begin{pspicture}(0.866,0.75)
%       \SpecialCoor
% ^^A  Hexagon  
%       \pspolygon[dimen=middle]%
%         (0.5;30)(0.5;90)(0.5;150)(0.5;210)(0.5;270)(0.5;330)
%     \end{pspicture}}
% 
%   \psset{unit=0.5}
%   \psboxfill{\Triangle}
%   \Tiling{(4,4)}\hfill
% ^^A The two other regular tilings
%   \Tiling[fillcyclex=2]{(4,4)}\hfill
%   \psboxfill{\Hexagon}
%   \Tiling[fillcyclex=2,fillloopaddy=1]{(5,5)}
% \end{LTXexample}
% 
% \begin{LTXexample}[pos=t]
%   \newcommand{\ArchimedianA}{%
%      ^^A Archimedian tiling 3^2.4.3.4
%     \psset{dimen=middle}
%      ^^A sin(60)=0.866
%     \begin{pspicture}(1.866,1.866)
%       \psframe(1,1)
%       \psline(1,0)(1.866,0.5)(1,1)(0.5,1.866)(0,1)(-0.866,0.5)
%       \psline(0,0)(0.5,-0.866)
%     \end{pspicture}}
%   \newcommand{\ArchimedianB}{%
%      ^^A Archimedian tiling 4.8^2
%     \psset{dimen=middle,unit=1.5}
%      ^^A sin(22.5)=0.3827 ; cos(22.5)=0.9239
%     \begin{pspicture}(1.3066,0.6533)
%       \SpecialCoor
%      ^^A Octogon
%       \pspolygon(0.5;22.5)(0.5;67.5)(0.5;112.5)(0.5;157.5)
%                 (0.5;202.5)(0.5;247.5)(0.5;292.5)(0.5;337.5)
%     \end{pspicture}}
% 
%   \psset{unit=0.5}
%   \psboxfill{\ArchimedianA}
%   \Tiling[fillmove=0.5]{(7,7)}\hfill
%   \psboxfill{\ArchimedianB}
%   \Tiling[fillcyclex=2,fillloopaddy=1]{(7,7)}
% \end{LTXexample}
% 
%   \setcounter{footnote}{3}
%   We can of course tile an area arbitrarily defined. And with the
% \texttt{addfillstyle} parameter\footnote{Introduced in PSTricks 97.}, we can
% easily mix the \texttt{boxfill} style with another one.
% 
% \begin{LTXexample}[width=6cm]
%   \psset{unit=0.5,dimen=middle}
%   \psboxfill{%
%     \begin{pspicture}(1,1)
%       \psframe(1,1)
%       \pscircle(0.5,0.5){0.25}
%     \end{pspicture}}
%   \begin{pspicture}(4,6)
%     \pspolygon[fillstyle=boxfill,fillsep=0.25](0,1)(1,4)(4,6)(4,0)(2,1)
%   \end{pspicture}\hspace{1em}
%   \begin{pspicture}(4,4)
%%     \pscircle[linestyle=none,fillstyle=solid,fillcolor=yellow,fillsep=0.5,
%%               addfillstyle=boxfill](2,2){2}
%   \end{pspicture}
% \end{LTXexample}
%
%   Various effects can be obtained, sometimes complicated ones very easily, as
% in this example reproduced from one shown by Slavik \textsc{Jablan} in the
% field of \emph{OpTiles}, inspired by the \emph{Op-art}:
% 
% \begin{LTXexample}[pos=t]
% \newcommand{\ProtoTile}{%
%  \begin{pspicture}(1,1)%%% 1/12=0.08333
%   \psset{linestyle=none,linewidth=0,
%     hatchwidth=0.08333\psunit,hatchsep=0.08333\psunit}
%   \psframe[fillstyle=solid,fillcolor=black,addfillstyle=hlines,hatchcolor=white](1,1)
%   \pswedge[fillstyle=solid,fillcolor=white,addfillstyle=hlines]{1}{0}{90}
%  \end{pspicture}}
% \newcommand{\BasicTile}{%
%  \begin{pspicture}(2,1)
%    \rput[lb](0,0){\ProtoTile}\rput[lb](1,0){\psrotateleft{\ProtoTile}}
%  \end{pspicture}}
% \ProtoTile\hfill\BasicTile\hfill
% \psboxfill{\BasicTile}
% \Tiling[fillcyclex=2]{(4,4)}
% \end{LTXexample}
% 
%   It is also directly possible to surimpose several different tilings. Here is
% the splendid visual proof of the \textsc{Pytha\-gore} theorem done by the arab
% mathematician \textsc{Annairizi} around the year 900, given by superposition
% of two tilings by squares of different sizes.
% 
% \begin{LTXexample}[pos=t]
% \psset{unit=1.5,dimen=middle}
% \begin{pspicture*}(3,3)
%   \psboxfill{\begin{pspicture}(1,1)
%     \psframe(1,1)\end{pspicture}}
%   \psframe[fillstyle=boxfill](3,3)
%   \psboxfill{\begin{pspicture}(1,1)
%     \rput{-37}{\psframe[linecolor=red](0.8,0.8)}
%   \end{pspicture}}
%   \psframe[fillstyle=boxfill](3,4)
%   \pspolygon[fillstyle=hlines,hatchangle=90](1,2)(1.64,1.53)(2,2)
% \end{pspicture*}
% \end{LTXexample}
% 
%   In a same way, it is possible to build tilings based on figurative patterns,
% in the style of the famous \textsc{Escher} ones. Following an example of
% Andr\'e \textsc{Deledicq} \cite{Deledicq97}, we first show a simple tiling of
% the \emph{p1} category (according to the international classification of the
% 17~symmetry groups of the plane first discovered by the russian
% crystalographer Jevgraf \textsc{Fedorov} at the end of the 19th century):
% 
% \begin{LTXexample}[pos=t]
%  \newcommand{\SheepHead}[1]{%
%    \begin{pspicture}(3,1.5)
%      \pscustom[liftpen=2,fillstyle=solid,fillcolor=#1]{%
%        \pscurve(0.5,-0.2)(0.6,0.5)(0.2,1.3)(0,1.5)(0,1.5)
%          (0.4,1.3)(0.8,1.5)(2.2,1.9)(3,1.5)(3,1.5)(3.2,1.3)
%          (3.6,0.5)(3.4,-0.3)(3,0)(2.2,0.4)(0.5,-0.2)}
%      \pscircle*(2.65,1.25){0.12\psunit} % Eye
%      \psccurve*(3.5,0.3)(3.35,0.45)(3.5,0.6)(3.6,0.4)% Muzzle
%     ^^A   % Mouth
%       \pscurve(3,0.35)(3.3,0.1)(3.6,0.05)
%     ^^A   % Ear
%       \pscurve(2.3,1.3)(2.1,1.5)(2.15,1.7)\pscurve(2.1,1.7)(2.35,1.6)(2.45,1.4)
%   \end{pspicture}}
%  \psboxfill{\psset{unit=0.5}\SheepHead{yellow}\SheepHead{cyan}}
%  \Tiling[fillcyclex=2,fillloopadd=1]{(10,5)}
% \end{LTXexample}
% \label{fig:Sheeps}
% 
%   Now a tiling of the \emph{pg} category (the code for the kangaroo itself is
% too long to be shown here, but has no difficulties ; the kangaroo is reproduce
% from an original picture from Raoul \textsc{Raba} and here is a translation in
% PSTricks from the one drawn by Emmanuel \textsc{Chailloux} and Guy
% \textsc{Cousineau} for their MLgraph system \cite{MLgraphTSI}):
% 
% \begin{LTXexample}[pos=t]
% \psboxfill{\psset{unit=0.4}
%   \Kangaroo{yellow}\Kangaroo{red}\Kangaroo{cyan}\Kangaroo{green}%
%   \psscalebox{-1 1}{%
%     \rput(1.235,4.8){\Kangaroo{green}\Kangaroo{cyan}\Kangaroo{red}\Kangaroo{yellow}}}}
%   \Tiling[fillloopadd=1]{(10,6)}
% \end{LTXexample}
% 
%   And here a \textsc{Wang} tiling \cite{Wang65}, \cite[chapter
% 11]{GS87}, based on very simple tiles of the form of a square and composed
% of four colored triangles. Such tilings are built with only a matching color
% constraint. Despite of it simplicity, it is an important kind of tilings, as
% \textsc{Wang} and others used them to study the special class of
% \emph{aperiodic} tilings, and also because it was shown that surprisingly this 
% tiling is similar to a \textsc{Turing} machine.
% 
% \begin{LTXexample}[pos=t]
%   \newcommand{\WangTile}[4]{%
%     \begin{pspicture}(1,1)
%       \pspolygon*[linecolor=#1](0,0)(0,1)(0.5,0.5)
%       \pspolygon*[linecolor=#2](0,1)(1,1)(0.5,0.5)
%       \pspolygon*[linecolor=#3](1,1)(1,0)(0.5,0.5)
%       \pspolygon*[linecolor=#4](1,0)(0,0)(0.5,0.5)
%     \end{pspicture}}
%   \newcommand{\WangTileA}{\WangTile{cyan}{yellow}{cyan}{cyan}}
%   \newcommand{\WangTileB}{\WangTile{yellow}{cyan}{cyan}{red}}
%   \newcommand{\WangTileC}{\WangTile{cyan}{red}{yellow}{yellow}}
%   \newcommand{\WangTiles}[1][]{%
%     \begin{pspicture}(3,3) \psset{ref=lb}
%       \rput(0,2){\WangTileB}  \rput(1,2){\WangTileA}%
%       \rput(2,2){\WangTileC}  \rput(0,1){\WangTileC}%
%       \rput(1,1){\WangTileB}  \rput(2,1){\WangTileA}
%       \rput(0,0){\WangTileA}  \rput(1,0){\WangTileC}%
%       \rput(2,0){\WangTileB}
%       #1
%     \end{pspicture}}
%   \WangTileA\hfill\WangTileB\hfill\WangTileC\hfill
%   \WangTiles[{\psgrid[subgriddiv=0,gridlabels=0](3,3)}]\hfill
%   \psset{unit=0.4} \psboxfill{\WangTiles} \Tiling{(12,12)}
% \end{LTXexample}
% 
% \subsection{External graphic files}
% \label{sec:GraphicFiles}
% 
%   We can also fill an arbitrary area with an external image. We have only, 
% as usual, to matter of the \emph{BoundingBox} definition if there is no one
% provided or if it is not the accurate one, as for the well known
% \texttt{tiger} picture part of the \texttt{ghostscript} distribution.
% 
% \begin{LTXexample}[pos=t]
%   \psboxfill{%% Strangely require x1=x2...
%     \begin{pspicture}(0,1)(0,4.1)
%       \includegraphics[bb=17 176 560 74,width=3cm]{tiger}
%     \end{pspicture}}
%   \Tiling{(6,6.2)}
% \end{LTXexample}
% 
%   Nevertheless, there are some special files for which the \emph{automatic}
% mode doesn't work, specially for some files obtained by a screen dump, as in
% the next example, where a picture was reduced before it conversion in the
% \emph{Encapsulated PostScript} format by a screen dump utility. In this case,
% usage of the \emph{manual} mode is the only alternative, at the price of the
% real multiple inclusion of the EPS file. We must take care to specify the
% correct \texttt{fillsize} parameter, because otherwise the default values are
% large and will load the file many times, perhaps just really using few
% occurrences as the other ones would be clipped...
% 
% \begin{LTXexample}[pos=t]
%   \psboxfill{\includegraphics{flowers}}
%   \begin{pspicture}(8,4)
%     \psellipse[fillstyle=boxfill,fillsize={(8,4)}](4,2)(4,2)
%   \end{pspicture}
% \end{LTXexample}
% 
% \subsection{Tiling of characters}
% 
%   We can also use the \cs{psboxfill} macro to fill the interior of characters
% for special effects like these ones:
% 
% \begin{LTXexample}[pos=t]
%   \DeclareFixedFont{\bigsf}{T1}{phv}{b}{n}{4.5cm}
%   \DeclareFixedFont{\smallrm}{T1}{ptm}{m}{n}{3mm}
%   \psboxfill{\smallrm Since 182 days...}
%   \begin{pspicture*}(8,4)
%     \centerline{%
%       \pscharpath[fillstyle=gradient,gradangle=-45,
%                   gradmidpoint=0.5,addfillstyle=boxfill,
%                   fillangle=45,fillsep=0.7mm]
%                  {\rput[b](0,0.1){\bigsf 2000}}}
%   \end{pspicture*}
% \end{LTXexample}
% 
% \begin{LTXexample}[pos=t]
%   \DeclareFixedFont{\mediumrm}{T1}{ptm}{m}{n}{2cm}
%   \psboxfill{%
%     \psset{unit=0.1,linewidth=0.2pt}
%     \Kangaroo{PeachPuff}\Kangaroo{PaleGreen}%
%       \Kangaroo{LightBlue}\Kangaroo{LemonChiffon}%
%     \psscalebox{-1 1}{%
%       \rput(1.235,4.8){%
%         \Kangaroo{LemonChiffon}\Kangaroo{LightBlue}%
%           \Kangaroo{PaleGreen}\Kangaroo{PeachPuff}}}}
% ^^A   % A kangaroo of kangaroos...
%   \begin{pspicture}(8,2)
%     \pscharpath[linestyle=none,fillstyle=boxfill,fillloopadd=1]
%                {\rput[b](4,0){\mediumrm Kangaroo}}
%   \end{pspicture}
% \end{LTXexample}
% 
% \subsection{Other kinds of usage}
% 
%   Other kinds of usage can be imagined. For instance, we can use tilings in a
% sort of degenerated way to draw some special lines made by a unique or
% multiple repeating patterns. But it can be only a special dashed line, as here
% with three different dashes:
% 
% \begin{LTXexample}[pos=t]
%   \newcommand{\Dashes}{%
%     \psset{dimen=middle}
%     \begin{pspicture}(0,-0.5\pslinewidth)(1,0.5\pslinewidth)
%       \rput(0,0){\psline(0.4,0)}%
%         \rput(0.5,0){\psline(0.2,0)}%
%         \rput(0.8,0){\psline(0.1,0)}
%     \end{pspicture}}
% 
%   \newcommand{\SpecialDashedLine}[3]{%
%     \psboxfill{#3}
%     \Tiling[linestyle=none]
%            {(#1,-0.5\pslinewidth)(#2,0.5\pslinewidth)}}
% 
%   \SpecialDashedLine{0}{7}{\Dashes}
% 
%   \psset{unit=0.5,linewidth=1mm,linecolor=red}
%   \SpecialDashedLine{0}{10}{\Dashes}
% \end{LTXexample}
% 
%   It allow also to use special patterns in business graphics, as in the
% following example generated by \texttt{PstChart}\footnote{A personal
% development to draw business charts with PSTricks, not distributed.}.
% 
% \vspace{3mm}
% \begin{figure}[!ht]
% \centering
% \psset{unit=0.75}
% ^^A % Generated by pstchart.sh version 0.21 (11/28/97)
% {\psset{dimen=middle}
% \psset{xunit=2,yunit=0.005}
% \begin{pspicture}(-0.6,-200)(6.6,2300)
% ^^A   % Title
%   \rput(3,2200){\shortstack{Fantaisist repartition of kangaroos\\
%                             in the world (in thousands)}}
% ^^A   % Frame background
%   \psframe[fillstyle=solid,fillcolor=LemonChiffon](0,0)(6,2000)
% ^^A   % Graduations
%   \multido{\n=0+500}{5}{\rput[r](-0.12,\n){\psscalebox{0.8}{\n}}}
% ^^A   % Minor ticks
%   \multips(0,100)(0,100){19}{\psline[unit=4.8pt](1,0)}
%   \multips(6,100)(0,100){19}{\psline[unit=4.8pt](-1,0)}
% ^^A   % Major ticks
%   \multips(0,500)(0,500){3}{\psline[unit=9.6pt](1,0)}
%   \multips(6,500)(0,500){3}{\psline[unit=9.6pt](-1,0)}
% ^^A   % Lines from major ticks marks
%   \multips(0,500)(0,500){3}{\psline[linestyle=dotted,linewidth=0.6pt](6,0)}
% ^^A   % Drawing for the data
%   \psboxfill{\psset{unit=0.78\psxunit}\KangarooPstChart{red}}
%   \psframe[linestyle=none,fillstyle=boxfill,fillloopaddy=1](0.61,0)(1.39,1800)
%   \psboxfill{\psset{unit=0.78\psxunit}\KangarooPstChart{yellow}}
%   \psframe[linestyle=none,fillstyle=boxfill,fillloopaddy=1](1.61,0)(2.39,800)
%   \psboxfill{\psset{unit=0.78\psxunit}\KangarooPstChart{cyan}}
%   \psframe[linestyle=none,fillstyle=boxfill,fillloopaddy=1](2.61,0)(3.39,550)
%   \psboxfill{\psset{unit=0.78\psxunit}\KangarooPstChart{magenta}}
%   \psframe[linestyle=none,fillstyle=boxfill,fillloopaddy=1](3.61,0)(4.39,500)
%   \psboxfill{\psset{unit=0.78\psxunit}\KangarooPstChart{green}}
%   \psframe[linestyle=none,fillstyle=boxfill,fillloopaddy=1](4.61,0)(5.39,200)
% ^^A   % Bottom labels
%   \uput{0.2}[270]{0}(1,0){\psscalebox{0.7}{Oceania}}
%   \uput{0.2}[270]{0}(2,0){\psscalebox{0.7}{Africa}}
%   \uput{0.2}[270]{0}(3,0){\psscalebox{0.7}{Asia}}
%   \uput{0.2}[270]{0}(4,0){\psscalebox{0.7}{America}}
%   \uput{0.2}[270]{0}(5,0){\psscalebox{0.7}{Europe}}
% ^^A   % Frame box around the chart
%   \psframe[linestyle=solid](0,0)(6,2000)
% \end{pspicture}}
%   \caption{Bar chart generated by PstChart, with bars filled by patterns}
%   \label{fig:PstChart}
% \end{figure}
% 
% \section{``Dynamic'' tilings}
% 
%   In some cases, tilings used non \emph{static} tiles, that is to say that the 
% \emph{prototile(s)}, even if unique, can have several forms, by instance
% specified by different colors or rotations, not fixed before generation or
% varying each time.
% 
% \subsection{Lewthwaite-Pickover-Truchet tiling}
% 
%   We give here for example the so-called \emph{Truchet} tiling, which much be
% in fact better called \emph{Lewthwaite-Pick\-over-Truchet (LPT)} tiling%
% \footnote{For description of the context, history and references about
% S\'ebastien \textsc{Truchet} and this tiling, see \cite{EsperetGirou98}.}.
% 
%   The unique prototile is only a square with two opposite circle arcs.
% This tile has obviously two positions, if we rotate it from 90 degrees (see
% the two tiles on the next figure). A \emph{LPT tiling} is a tiling with
% randomly oriented LPT tiles. We can see that even if it is very simple in it
% principle, it draw sophisticated curves with strange properties.
% 
%   Nevertheless, in the straightforward way \FillPackage{} does not work,
% because the \cs{psboxfill} macro store the content of the tile used in a
% \TeX{} box, which is static. So the calling to the random function is done
% only one time, which explain that only one rotation of the tile is used for
% all the tiling. It's only the one of the two rotations which could differ from
% one drawing to the next one...
% 
% ^^A % Truchet (Lewthwaite-Pickover-Truchet) tiling
% ^^A % --------------------------------------------
% 
% \begin{LTXexample}[pos=t]
% ^^A   % LPT prototile
%   \newcommand{\ProtoTileLPT}{%
%     \psset{dimen=middle}
%     \begin{pspicture}(1,1)
%       \psframe(1,1)
%       \psarc(0,0){0.5}{0}{90}
%       \psarc(1,1){0.5}{-180}{-90}
%     \end{pspicture}}
% 
% ^^A   % LPT tile
%   \newcount\Boolean
%   \newcommand{\BasicTileLPT}{%
% ^^A     % From random.tex by Donald Arseneau
%     \setrannum{\Boolean}{0}{1}%
%     \ifnum\Boolean=0
%       \ProtoTileLPT%
%     \else
%       \psrotateleft{\ProtoTileLPT}%
%     \fi}
% 
%   \ProtoTileLPT\hfill\psrotateleft{\ProtoTileLPT}\hfill
%   \psset{unit=0.5}
%   \psboxfill{\BasicTileLPT}
%   \Tiling{(5,5)}
% \end{LTXexample}
% 
%   But, for simple cases, there is a solution to this problem using a mixture
% of PSTricks and PostScript programming. Here the PSTricks
% construction \verb+\pscustom{\code{...}}+ allow to insert PostScript code
% inside the \LaTeX{} + PSTricks one.
% 
%   Programmation is less straightforward, but it has also the advantage to be
% notably faster, as all the tilings operations are done in PostScript, and
% mainly to not be limited by \TeX{} memory (the \TeX{} + PSTricks solution
% I wrote in 1995 for the colored problem was limited to small sizes for this
% reason). Just note also that \cs{pslbrace} and \cs{psrbrace} are two
% PSTricks macros to define and be able to insert the \verb+{+ and \verb+}+
% characters.
% 
% \begin{LTXexample}[pos=t]
% ^^A   % LPT prototile
%   \newcommand{\ProtoTileLPT}{%
%     \psset{dimen=middle}
%     \psframe(1,1)
%     \psarc(0,0){0.5}{0}{90}
%     \psarc(1,1){0.5}{-180}{-90}}
% 
% ^^A   % Counter to change the random seed
%   \newcount\InitCounter
% ^^A   % LPT tile
%   \newcommand{\BasicTileLPT}{%
%     \InitCounter=\the\time
%     \pscustom{\code{%
%       rand \the\InitCounter\space sub 2 mod 0 eq \pslbrace}}
%     \begin{pspicture}(1,1)
%       \ProtoTileLPT
%     \end{pspicture}%
%     \pscustom{\code{\psrbrace \pslbrace}}
%     \psrotateleft{\ProtoTileLPT}%
%     \pscustom{\code{\psrbrace ifelse}}}
% 
%   \psset{unit=0.4,linewidth=0.4pt}
%   \psboxfill{\BasicTileLPT}
%   \Tiling{(15,15)}
% \end{LTXexample}
% 
%   Using the very surprising fact (see \cite{EsperetGirou98}) that
% coloration of these tiles do not depend of their neighbors (even if it is
% difficult to believe as the opposite seems obvious!) but only of the parity of
% the value of row and column positions, we can directly program in the same way
% a colored version of the LPT tiling.
% 
% \setcounter{footnote}{1}
%   We have also introduce in the \FillPackage{} code for \emph{tiling} mode two
% new accessible Post\-Script variables, \texttt{row} and
% \texttt{column}\footnotemark, which can be useful in some circonstances, like
% this one.
% 
% \begin{LTXexample}[pos=t]
% ^^A   % LPT prototile
%   \newcommand{\ProtoTileLPT}[2]{%
%     \psset{dimen=middle,linestyle=none,fillstyle=solid}
%     \psframe[fillcolor=#1](1,1)
%     \psset{fillcolor=#2}
%     \pswedge(0,0){0.5}{0}{90} \pswedge(1,1){0.5}{-180}{-90}}
% ^^A   % Counter to change the random seed
%   \newcount\InitCounter
% ^^A   % LPT tile
%   \newcommand{\BasicTileLPT}[2]{%
%     \InitCounter=\the\time
%     \pscustom{\code{%
%       rand \the\InitCounter\space sub 2 mod 0 eq \pslbrace
%       row column add 2 mod 0 eq \pslbrace}}
%     \begin{pspicture}(1,1)\ProtoTileLPT{#1}{#2}\end{pspicture}%
%     \pscustom{\code{\psrbrace \pslbrace}}
%     \ProtoTileLPT{#2}{#1}%
%     \pscustom{\code{%
%       \psrbrace ifelse \psrbrace \pslbrace row column add 2 mod 0 eq \pslbrace}}
%     \psrotateleft{\ProtoTileLPT{#2}{#1}}\pscustom{\code{\psrbrace \pslbrace}}
%     \psrotateleft{\ProtoTileLPT{#1}{#2}}\pscustom{\code{\psrbrace ifelse \psrbrace ifelse}}}
%   \psboxfill{\BasicTileLPT{red}{yellow}}
%   \Tiling{(4,4)}\hfill
%   \psset{unit=0.4}\psboxfill{\BasicTileLPT{blue}{cyan}}
%   \Tiling{(15,15)}
% \end{LTXexample}
% 
%   Another classic example is to generate coordinates and numerotation for a
% grid. Of course, it is possible to do it directly in PSTricks using nested
% \cs{multido} commands. It would be clearly easy to program, but, nevertheless, 
% for users who have a little knowledge of PostScript programming, this offer
% an alternative which is useful for large cases, because on this way it will
% be notably faster and less computer ressources consuming.
% 
%   Remember here that the tiling is drawn from left to right, and top to
% bottom, and note that the PostScript variable \texttt{x2} give the total
% number of columns.
% 
% \begin{LTXexample}[pos=t]
% ^^A   % \Escape will be the \ character
%   {\catcode`\!=0\catcode`\\=11!gdef!Escape{\}}
%   \newcommand{\ProtoTile}{%
%     \Square\pscustom{%
%       \moveto(-0.9,0.75) % In PSTricks units
%       \code{ /Times-Italic findfont 8 scalefont setfont
%         (\Escape() show row 3 string cvs show (,) show 
%         column 3 string cvs show (\Escape)) show}
%       \moveto(-0.5,0.25) % In PSTricks units
%       \code{ /Times-Bold findfont 18 scalefont setfont
%         1 0 0 setrgbcolor % Red color
%         /center {dup stringwidth pop 2 div neg 0 rmoveto} def
%         row 1 sub x2 mul column add 3 string cvs center show}}}
%   \psboxfill{\ProtoTile}
%   \Tiling{(6,4)}
% \end{LTXexample}
% 
% \subsection{A complete example: the Poisson equation}
% 
%   To finish, we will show a complete real example, a drawing to explain the
% method used to solve the \textsc{Poisson} equation by a domain
% decomposition method, adapted to distributed memory computers. The
% objective is to show the communications required between processes and the
% position of the data to exchange. This code also show some useful and powerful
% technics for PSTricks programming (look specially at the way some higher level
% macros are defined, and how the same object is used to draw the four
% neighbors).
%
%\psset{unit=1cm}
%\newcommand{\Pattern}[1]{%
%   \begin{pspicture}(-0.25,-0.25)(0.25,0.25)\rput{*0}{\psdot[dotstyle=#1]}
%   \end{pspicture}}
%\newcommand{\West}{\Pattern{o}}   \newcommand{\South}{\Pattern{x}}
%\newcommand{\Central}{\Pattern{+}}\newcommand{\North}{\Pattern{square}}
%\newcommand{\East}{\Pattern{triangle}}
%\newcommand{\Cross}{%
%  \pspolygon[unit=0.5,linewidth=0.2,linecolor=red](0,0)(0,1)(1,1)(1,2)(2,2)(2,1)%
%              (3,1)(3,0)(2,0)(2,-1)(1,-1)(1,0)}
%\newcommand{\StylePosition}[1]{\LARGE\textcolor{red}{\textbf{#1}}}
%\newcommand{\SubDomain}[4]{%
%    \psboxfill{#4}\begin{psclip}{\psframe[linestyle=none]#1}%
%      \psframe[linestyle=#3](5,5)\psframe[fillstyle=boxfill]#2%
%    \end{psclip}}
%\newcommand{\SendArea}[1]{\psframe[fillstyle=solid,fillcolor=cyan]#1}
%\newcommand{\ReceiveData}[2]{%
%  \psboxfill{#2}\psframe[fillstyle=solid,fillcolor=yellow,addfillstyle=boxfill]#1}%
%\newcommand{\Neighbor}[2]{%
%    \begin{pspicture}(5,5)
%      \rput{*0}(2.5,2.5){\StylePosition{#1}}
%      \ReceiveData{(0.5,0)(4.5,0.5)}{\Central}\SendArea{(0.5,0.5)(4.5,1)}%
%      \SubDomain{(5,2)}{(0.5,0.5)(4.5,3)}{dashed}{#2}%
%      \pcarc[arcangle=45,arrows=->](0.5,-1.25)(0.5,0.25)%
%      \pcarc[arcangle=45,arrows=->,linestyle=dotted,dotsep=2pt](4.5,0.75)(4.5,-0.75)%
%    \end{pspicture}}%
%  \psset{dimen=middle,dotscale=2,fillloopadd=2}
%\begin{pspicture}(-5.7,-5.7)(5.7,5.7)
%  \rput(0,0){%
%      \begin{pspicture}(5,5)
%        \ReceiveData{(0,0.5)(0.5,4.5)}{\West} \ReceiveData{(4.5,0.5)(5,4.5)}{\East}
%        \ReceiveData{(0.5,4.5)(4.5,5)}{\North}\ReceiveData{(0.5,0)(4.5,0.5)}{\South}
%        \SendArea{(0.5,0.5)(1,4.5)}\SendArea{(4,0.5)(4.5,4.5)}
%        \SendArea{(0.5,0.5)(4.5,1)}\SendArea{(0.5,4)(4.5,4.5)}
%        \SubDomain{(5,5)}{(0.5,0.5)(4.5,4.5)}{solid}{\Central}
%        \psline(1,0.5)(1,4.5)\psline(4,0.5)(4,4.5)%
%        \rput(1.5,4){\Cross}\rput(2,2){\Cross}%
%      \end{pspicture}}%
%  \rput(0,5.5){\Neighbor{N}{\North}}\rput{-90}(5.5,0){\Neighbor{E}{\East}}%
%  \rput{90}(-5.5,0){\Neighbor{W}{\West}}\rput{180}(0,-5.5){\Neighbor{S}{\South}}%
%\end{pspicture}
%
% \begin{lstlisting}
%   \newcommand{\Pattern}[1]{%
%     \begin{pspicture}(-0.25,-0.25)(0.25,0.25)\rput{*0}{\psdot[dotstyle=#1]}
%     \end{pspicture}}
%   \newcommand{\West}{\Pattern{o}}   \newcommand{\South}{\Pattern{x}}
%   \newcommand{\Central}{\Pattern{+}}\newcommand{\North}{\Pattern{square}}
%   \newcommand{\East}{\Pattern{triangle}}
%   \newcommand{\Cross}{%
%     \pspolygon[unit=0.5,linewidth=0.2,linecolor=red](0,0)(0,1)(1,1)(1,2)(2,2)(2,1)
%               (3,1)(3,0)(2,0)(2,-1)(1,-1)(1,0)}
%   \newcommand{\StylePosition}[1]{\LARGE\textcolor{red}{\textbf{#1}}}
%   \newcommand{\SubDomain}[4]{%
%     \psboxfill{#4}
%     \begin{psclip}{\psframe[linestyle=none]#1}
%       \psframe[linestyle=#3](5,5)\psframe[fillstyle=boxfill]#2
%     \end{psclip}}
%   \newcommand{\SendArea}[1]{\psframe[fillstyle=solid,fillcolor=cyan]#1}
%   \newcommand{\ReceiveData}[2]{%
%     \psboxfill{#2}
%     \psframe[fillstyle=solid,fillcolor=yellow,addfillstyle=boxfill]#1}
%   \newcommand{\Neighbor}[2]{%
%     \begin{pspicture}(5,5)
%       \rput{*0}(2.5,2.5){\StylePosition{#1}}
%       \ReceiveData{(0.5,0)(4.5,0.5)}{\Central}\SendArea{(0.5,0.5)(4.5,1)}
%       \SubDomain{(5,2)}{(0.5,0.5)(4.5,3)}{dashed}{#2}%
% ^^A       % Receive and send arrows
%       \pcarc[arcangle=45,arrows=->](0.5,-1.25)(0.5,0.25)
%       \pcarc[arcangle=45,arrows=->,linestyle=dotted,dotsep=2pt](4.5,0.75)(4.5,-0.75)
%     \end{pspicture}}
%   \psset{dimen=middle,dotscale=2,fillloopadd=2}
%   \begin{pspicture}(-5.7,-5.7)(5.7,5.7)
% ^^A     % Central domain
%     \rput(0,0){%
%       \begin{pspicture}(5,5)
% ^^A         % Receive from West, East, North and South
%         \ReceiveData{(0,0.5)(0.5,4.5)}{\West} \ReceiveData{(4.5,0.5)(5,4.5)}{\East}
%         \ReceiveData{(0.5,4.5)(4.5,5)}{\North}\ReceiveData{(0.5,0)(4.5,0.5)}{\South}
% ^^A         % send area for West, East, North and South
%         \SendArea{(0.5,0.5)(1,4.5)} \SendArea{(4,0.5)(4.5,4.5)}
%         \SendArea{(0.5,0.5)(4.5,1)} \SendArea{(0.5,4)(4.5,4.5)}
% ^^A         % Central domain
%         \SubDomain{(5,5)}{(0.5,0.5)(4.5,4.5)}{solid}{\Central}
% ^^A         % Redraw overlapped linesY
%         \psline(1,0.5)(1,4.5)  \psline(4,0.5)(4,4.5)
% ^^A         % Two crossesY
%         \rput(1.5,4){\Cross}  \rput(2,2){\Cross}
%       \end{pspicture}}
% ^^A     % The four neighborsY
%     \rput(0,5.5){\Neighbor{N}{\North}}     \rput{-90}(5.5,0){\Neighbor{E}{\East}}
%     \rput{90}(-5.5,0){\Neighbor{W}{\West}} \rput{180}(0,-5.5){\Neighbor{S}{\South}}
%   \end{pspicture}
% \end{lstlisting}
%
%
%
% Bibliography
% \begin{thebibliography}{99}
% \bibitem{PostScript95} Adobe, Systems~Incorporated, \emph{PostScript Language
% Reference Manual}, Addison-Wesley, 2~edition, 1995.
%
% \bibitem{Bolek98} Piotr Bolek, \MP{} and patterns, \emph{\TUB}, Volume~19,
% Number~3, pages 276--283, September 1998, \CTANref{mpattern}.
%
% \bibitem{MLgraphTSI} Emmanuel Chailloux, Guy Cousineau and Asc\'ander
% Su\'arez, Programmation fonctionnelle de graphismes pour la production
% d'illustrations techniques, \emph{Technique et science informatique},
% Volume~15, Number~7, pages 977--1007, 1996 (in french).
%
% \bibitem{Deledicq97} Andr\'e Deledicq, \emph{Le monde des pavages}, ACL
% \'Editions, 1997 (in french).
%
% \bibitem{EsperetGirou98} Philippe Esperet and Denis Girou,
% Coloriage du pavage dit de Truchet, Cahiers GUTenberg, Number~31,
% pages 5--18, December~1998  (in french).
%
% \bibitem{Girou94} Denis Girou, Pr\'esentation de PSTricks, \emph{Cahiers
% GUTenberg}, Number~16, pages 21--70, February~1994 (in french).
%
% \bibitem{LGC97} Michel Goossens, Sebastian Rahtz and Frank Mittelbach,
% \emph{The \LaTeX{} Graphics Companion}, Addison-Wesley, 2005.
%
% \bibitem{GS87} Branko Gr\"unbaum and Geoffrey Shephard, \emph{Tilings and
% Patterns}, Freeman and Company, 1987.
%
% \bibitem{Hoenig97} Alan Hoenig, \emph{\TeX{} Unbound: \LaTeX{} \& \TeX{}
% Strategies, Fonts, Graphics, and More}, Oxford University Press, 1997.
%
% \bibitem{XYpic} Kristoffer~H. Rose and Ross Moore, \XYpic. Pattern and Tile
% extension, available from \CTAN, 1991-1998, \CTANref{xypic}.
%
% \bibitem{LAAN96} Kees van der Laan, Paradigms: Just a little bit of PostScript,
% \emph{MAPS}, Volume~17, pages 137--150, 1996.
%
% \bibitem{LAAN97} Kees van der Laan, Tiling in PostScript and \MF{} -- Escher's
% wink, \emph{MAPS}, Volume~19, Number~2, pages 39--67, 1997.
%
% \bibitem{vanZandt93} Timothy Van Zandt, PSTricks. PostScript macros for
% Generic \TeX, available from \CTAN, 1993, \CTANref{pstricks}.
%
% \bibitem{vanZandtGirou94} Timothy Van Zandt and Denis Girou, Inside PSTricks,
% \emph{\TUB}, Volume~15, Number~3, pages 239--246, September 1994.
%
%
% \bibitem{voss07} Herbert Vo\ss, PSTricks -- Graphics for \TeX\ and \LaTeX, DANTE/Lehmanns, 4th ed., 2007.
% \bibitem{Wang65} Hao Wang, Games, Logic and Computers, \emph{Scientific
% American}, pages 98--106, November 1965.
% \end{thebibliography}
%
%
% \StopEventually{}
%
% ^^A .................... End of the documentation part ....................
%
% \section{Driver file}
%
%   The next bit of code contains the documentation driver file for \TeX{},
% i.e., the file that will produce the documentation you are currently
% reading. It will be extracted from this file by the \texttt{docstrip}
% program.
%
%    \begin{macrocode}
%<*driver>
\documentclass{ltxdoc}
\GetFileInfo{pst-fill.dtx}
%
\usepackage[T1]{fontenc}
\usepackage{lmodern}               % For PDF
\usepackage{graphicx}              % `graphicx' LaTeX standard package
\usepackage{showexpl}
\usepackage{mflogo}                % For the MetaFont and MetaPost logos
\input{random.tex}                 % Random macros from Donald Arseneau
\usepackage{url}                   % URLs convenient typesetting
\usepackage{multido}               % General loop macro
\usepackage[dvipsnames]{pstricks}  % PSTricks with the `color' extension
\usepackage{pst-text}              % PSTricks package for character path
\usepackage{pst-grad}              % PSTricks package for gradient filling
\usepackage{pst-node}              % PSTricks package for nodes
\usepackage[tiling]{pst-fill}      % PSTricks package for filling/tiling
%
\AtBeginDocument{%
%  \OnlyDescription % comment out for implementation details
  \EnableCrossrefs
  \CodelineIndex
  \RecordChanges}
\AtEndDocument{%
  \PrintIndex
  \setcounter{IndexColumns}{1}
  \PrintChanges}
\hbadness=7000            % Over and under full box warnings
\hfuzz=3pt
\begin{document}
  \DocInput{pst-fill.dtx}
\end{document}
%</driver>
%    \end{macrocode}
%
% \section{\texttt{pst-fill} \LaTeX{} wrapper}
%
%    \begin{macrocode}
%<*latex-wrapper>
\RequirePackage{pstricks}
\ProvidesPackage{pst-fill}[2005/09/13 package wrapper for 
  pst-fill.tex (hv)]
\DeclareOption{tiling}{\def\PstTiling{true}}
\ProcessOptions\relax
% \iffalse meta-comment, etc.
%%
%% Package `pst-fill.dtx'
%%
%% Denis Girou (CNRS/IDRIS - France) <Denis.Girou@idris.fr>
%% Herbert Voss <voss@pstricks.de>
%%
%% This program can be redistributed and/or modified under the terms
%% of the LaTeX Project Public License Distributed from CTAN archives
%% in directory macros/latex/base/lppl.txt.
%%
%% DESCRIPTION:
%%   `pst-fill' is a PSTricks package for filling and tiling areas 
%%
% \fi
% \changes{v1.01}{2007/03/10}{bugfix for incomplete ifx (hv)}
% \changes{v1.00}{2006/11/06}{use pst-xkey for extend keys (hv)}
% \changes{v0.99}{2004/08/17}{merge the VTeX and TeX versions (patch 4) (hv)}
% \changes{v0.98}{2004/06/22}{delete the Pst@Debug option and use the
%   the one from pstricks to prevent a clash with pst-gr3d (hv)}
% \changes{v0.97}{2001/10/09}{make it work with VTeX (mv)}
% \changes{v0.94}{1997/04/08}{With a \PstTiling macro defined (or "tiling" optional parameter
%   on \textbackslash usepackage[tiling]{pst-fill}), this file run exactly as
%   the original boxfill.tex file from Timothy, version 0.94,
%   except a correction in \textbackslash pst@ManualFillCycle to avoid a division by 0.
%   It's the default.}
% \changes{v0.93}{1997/04/07}{With a \textbackslash PstTiling macro defined (or "tiling" optional parameter
%   on \textbackslash usepackage[tiling]{pst-fill}) there are several add-ons
%   and changes to do `tiling' rather than `filling' in "automatic" mode :
%     - we fix the position of the beginning of tiling,
%     - we allow normally the framing of the area as expected, using
%       the line.... parameters
%     - we define move parameters fillmovex, fillmovey and fillmove,
%     - we define fillcyclex as previous fillcycle parameter, and add the
%       fillcycley and fillcycle (both fillcyclex and fillcycley) ones
%     - we can extend the tiling area using fillloopaddx, fillloopaddy and
%       fillloopadd parameters,
%     - we can debug and see the whole tiling area without clipping using
%       PstDebug parameter,
%     - for names consistancy, we can use fillangle in place of boxfillangle
%       and fillsize in place of boxfillsize,
%     - default value for fillsep is 0 and for fillsize is auto.}
%
% \DoNotIndex{\!,\",\#,\$,\%,\&,\',\(,\+,\*,\,,\-,\.,\/,\:,\;,\<,\=,\>,\?}
% \DoNotIndex{\@,\@B,\@K,\@cTq,\@f,\@fPl,\@ifnextchar,\@nameuse,\@oVk}
% \DoNotIndex{\[,\\,\],\^,\_,\ }
% \DoNotIndex{\^,\\^,\\\^,$\^$,$\\^$,$\\^$}
% \DoNotIndex{\0,\2,\4,\5,\6,\7,\8,}
% \DoNotIndex{\A,\a}
% \DoNotIndex{\B,\b,\Bc,\begin,\Bq,\Bqc}
% \DoNotIndex{\C,\c,\catcode,\cJA,\CodelineIndex,\csname}
% \DoNotIndex{\D,\def,\define@key,\Df,\divide,\DocInput,\documentclass,\pst@addfams}
% \DoNotIndex{\eCN,\edef,\else,\eHd,\eMcj,\EnableCrossrefs,\end,\endcsname}
% \DoNotIndex{\endCenterExample,\endExample,\endinput,\endpsclip}
% \DoNotIndex{\PrintIndex,\PrintChanges,\ProvidesFile}
% \DoNotIndex{\endpspicture,\endSideBySideExample,\Example}
% \DoNotIndex{\F,\f,\FdUrr,\fi,\filedate,\fileversion,\FV@Environment}
% \DoNotIndex{\FV@UseKeyValues,\FV@XRightMargin,\FVB@Example,\fvset}
% \DoNotIndex{\G,\g,\GetFileInfo,\gr,\GradientLoaded,\gsFKrbK@o,\gsj,\gsOX}
% \DoNotIndex{\hbadness,\hfuzz,\HLEmphasize,\HLMacro,\HLMacro@i}
% \DoNotIndex{\HLReverse,\HLReverse@i,\hqcu,\HqY}
% \DoNotIndex{\I,\i,\ifx,\input,\Ir,\IU}
% \DoNotIndex{\j,\jl,\JT,\JVodH}
% \DoNotIndex{\K,\k,\kfSlL}
% \DoNotIndex{\L,\let}
% \DoNotIndex{\message,\mHNa,\mIU}
% \DoNotIndex{\N,\nB,\newcmykcolor,\newdimen,\newif,\nW}
% \DoNotIndex{\O,\oCDJDo,\ocQhVI,\OnlyDescription,\oRKJ}
% \DoNotIndex{\P,\p,\ProvidesPackage,\psframe,\pslinewidth,\psset}
% \DoNotIndex{\PstAtCode,\PSTricksLoaded}
% \DoNotIndex{\q,\Qr,\qssRXq,\qu,\qXjFQp,\qYL}
% \DoNotIndex{\R,\r,\RecordChanges,\relax,\RlaYI,\rN,\Rp,\rp,\RPDXNn,\rput}
% \DoNotIndex{\S,\scalebox,\SgY,\SideBySide@Example,\SideBySideExample}
% \DoNotIndex{\SgY,\sk,\Sp,\space,\sZb}
% \DoNotIndex{\T,\the,\tw@}
% \DoNotIndex{\u,\UiSWGEf@,\uJi,\usepackage,\uVQdMM,\UYj}
% \DoNotIndex{\VerbatimEnvironment,\VerbatimInput,\VrC@}
% \DoNotIndex{\WhZ,\WjKCYb,\WNs}
% \DoNotIndex{\XkN,\XW}
% \DoNotIndex{\Z,\ZCM,\Ze}
% \DoNotIndex{\addtocounter,\advance,\alph,\arabic,\AtBeginDocument,\AtEndDocument}
% \DoNotIndex{\AtEndOfPackage,\begingroup,\bfseries,\bgroup,\box,\csname}
% \DoNotIndex{\else,\endcsname,\endgroup,\endinput,\expandafter,\fi}
% \DoNotIndex{\TeX,\z@,\p@,\@one,\xdef,\thr@@,\string,\sixt@@n,\reset,\or,\multiply,\repeat,\RequirePackage}
% \DoNotIndex{\@cclvi,\@ne,\@ehpa,\@nil,\copy,\dp,\global,\hbox,\hss,\ht,\ifodd,\ifdim,\ifcase,\kern}
% \DoNotIndex{\chardef,\loop,\leavevmode,\ifnum,\lower}
% \setcounter{IndexColumns}{2}
%
% ^^A To extend the height used for the text
%
% ^^A  Aligned labels in a description environment
%\newenvironment{Description}[1]{%
%\begin{list}{nothing}{\setlength{\leftmargin}{#1}
%\setlength{\labelwidth}{\leftmargin}\setlength{\labelsep}{1mm}}}
%{\end{list}}
%
% ^^A For macro names
%\DeclareRobustCommand\cs[1]{\texttt{\char`\\#1}}
%
%
% ^^A From ltugboat.cls
% ^^A For references
%\makeatletter
%\newcommand\acro[1]{\textsc{#1}\@}
%\def\CTAN{\acro{CTAN}}
%\let\texttub\textsl              % ^^A redefined in other situations
%\def\TUB{\texttub{TUGboat}}
%\def\TUG{\TeX\ \UG}
%\def\tug{\acro{TUG}}
%\def\UG{Users Group}
% ^^A For the bibliography 
%\let\@internalcite\cite
%\def\cite{\def\@citeseppen{-1000}%
%    \def\@cite##1##2{(##1\if@tempswa , ##2\fi)}%
%    \def\citeauthoryear##1##2##3{##1, ##3}\@internalcite}
%\def\etal{et\,al.\@}
%\newcommand\CTANdirectory[1]{\expandafter\urldef
%  \csname CTAN@#1\endcsname\path}
%\newcommand\CTANfile[1]{\expandafter\urldef
%  \csname CTAN@#1\endcsname\path}
%\newcommand\CTANref[1]{\expandafter\@setref\csname CTAN@#1\endcsname
%  \relax{#1}}
%\makeatother
% ^^A Define CTAN addresses 
%\CTANdirectory{mpattern}{graphics/metapost/macros/mpattern}
%\CTANdirectory{pstricks}{graphics/pstricks}
%\CTANdirectory{pst-fill.sty}{graphics/pstricks/latex/pst-fill.sty}
%\CTANdirectory{pst-fill}{graphics/pstricks/generic/pst-fill.tex}
%\CTANdirectory{Roegel}{graphics/metapost/contrib/macros/truchet}
%\CTANdirectory{xypic}{macros/generic/diagrams/xypic}
%
% ^^A Personal macros (D.G.)
% ^^A ----------------------
%
% ^^A Some colors used
%\definecolor{LemonChiffon}{rgb}{1.,0.98,0.8}
%\definecolor{LightBlue}   {rgb}{0.8,0.85,0.95}
%\definecolor{PaleGreen}   {rgb}{0.88,1,0.88}
%\definecolor{PeachPuff}   {rgb}{1.0,0.85,0.73}
%
% ^^A To define a unique string for TeX and LaTeX
%\newcommand{\AllTeX}{%
%{\rm(L\kern-.36em\raise.3ex\hbox{\sc a}\kern-.15em)%
%T\kern-.1667em\lower.7ex\hbox{E}\kern-.125emX}}
%
% ^^A Bibliography style
%\bibliographystyle{ltugbib}
%
% ^^A Name macros
%\newcommand{\FillPackage}{\textsf{`pst-fill'}}
%\newcommand{\XYpic}{%
%\leavevmode\hbox{\kern-.1em X\kern-.3em\lower.4ex\hbox{Y\kern-.15em}-pic}}
%
%\makeatletter
%
% ^^A Example environments
% ^^A (do not use in them the four JXYZ characters, that we will use
% ^^A as escape characters!)
%
% ^^A Default PSTricks parameters
%  \psset{dimen=middle}
%
% ^^A Translation in PSTricks from the one drawn by Emmanuel Chailloux and
% ^^A Guy Cousineau for the MLgraph system
% ^^A (see /ftp.ens.fr:/pub/unix/lang/MLgraph/version-2.1/MLgraph-refman.ps.gz)
% ^^A The kangaroo itself is reproduce from an original picture from Raoul Raba
% \newcommand{\DimX}{2.47}
% \newcommand{\DimY}{4.8}
% \newcommand{\DimXDivTwo}{1.235}
%
% \newcommand{\KangarooItself}[1]{%
% ^^A Body
% \pspolygon[fillstyle=solid,fillcolor=#1]%
%  (52.5,68)(55,72.5)(55.8,76.5)(56.8,79.8)(58.2,83)(60,85.8)(61.5,86.5)
% (64,87)(66,87.5)(67.8,87.3)(70,87)(71.5,87.3)(73,88)(74.7,88.5)
% (76,90.3)(77,91.5)(72.8,93.8)(69,96)(64.5,99)(59.4,103)(56.2,106.3)
% (53,110.5)(49.5,115.5)(47.2,119.9)(45.7,126)(43.2,123)(41.5,121)(37.5,125)
% (37,122.5)(36.8,120)(37,117)(37.6,113.5)(38.6,110)(40,106.3)(42,102.3)
%  (43.5,99.5)(45,97)(46.2,94)(46.8,91.7)(47.2,88)(47,83.5)(46.3,80.8)
%  (45.3,78.5)(42.5,76.5)(39.5,75.8)(36,75.9)(33,75.9)(29,76.2)(26,77)
%  (22.3,77.5)(18,78.4)(12.8,79.3)(8.6,80)(5.5,80.3)(3,80.5)(0,80)
%  (-5.2,78.5)(-9,76.3)(-11.2,74.8)(-13,72.5)(-16.5,68)(-16.5,68)(-19.5,62.5)
%  (-22,58)(-25.5,53)(-29,48.5)(-32.5,45)(-36,42)(-39,39.5)(-44,37)
%  (-49,35)(-51,34)(-53.5,34.5)(-55.5,36)(-56.5,38)(-56.5,40.5)(-55,41.5)
%  (-53.5,41)(-51.5,41)(-50.5,43)(-50.5,44.5)(-51,47)(-51.5,47.2)(-56.5,47)
%  (-58.5,46.5)(-60,44.7)(-62,42.3)(-63,39.5)(-63.5,36.3)(-63.5,33)(-63.1,29.5)
%  (-61.5,26)(-58,23.6)(-54,22.2)(-50.7,22)(-47.5,22)(-44.5,22.3)(-41,23.5)
%  (-36.8,25.8)(-33,28)(-28.5,31)(-23.4,35)(-20.2,38.3)(-17,42.5)(-13.5,47.5)
%  (-11.2,51.9)(-9.7,58)(-7.2,55)(-5.5,53)(-1.5,57)(-1,54.5)(-0.8,52)
%  (-1,49)(-1.6,45.5)(-2.6,42)(-4,38.3)(-6,34.3)(-7.5,31.5)(-9,29)
%  (-10.2,26)(-10.8,23.7)(-11.2,20)(-11,15.5)(-10.3,12.8)(-9.3,10.5)(-6.5,8.5)
%  (-3.5,7.8)(0,7.9)(3,7.9)(7,8.2)(10,9)(13.7,9.5)(18,10.4)
%  (23.2,11.3)(27.4,12)(30.5,12.3)(33,12.5)(36,12)(41.2,10.5)(45,8.3)
%  (47.2,6.8)(49,4.5)(52.5,0)(50,4.5)(49.2,8.5)(48.2,11.8)(46.8,15)
%  (45,17.8)(43.5,18.5)(41,19)(39,19.5)(37.2,19.3)(35,19)(33.5,19.3)
%  (32,20)(30.3,20.5)(29,22.3)(28,23.5)(28,23.5)(24.5,22.3)(21.5,22)
%  (18.3,22)(15,22.2)(11,23.6)(7.5,26)(5.9,29.5)(5.5,33)(5.5,36.3)
%  (6,39.5)(7,42.3)(9,44.7)(10.5,46.5)(12.5,47)(17.5,47.2)(18,47)
%  (18.5,44.5)(18.5,43)(17.5,41)(15.5,41)(14,41.5)(12.5,40.5)(12.5,38)
%  (13.5,36)(15.5,34.5)(18,34)(20,35)(25,37)(30,39.5)(33,42)
%  (36.5,45)(40,48.5)(43.5,53)(47,58)(49.5,62.5)(52.5,68)
% ^^A Eye
% \pscircle*[linecolor=white](58.2,98.3){2\psxunit}
% \pscircle*(58.2,97.3){\psxunit}
% ^^A Mouth
% \psline(71.5,88)(70,89.3)(68.5,90.3)(67,91.9)
% ^^A Tear
% \psline(42,121)(45,118)(47,115.3)(48.5,112.7)(50,110)(51.8,106.5)
%       (52.5,103.7)(53,100.5)
% \pspolygon(41.2,115.8)(43.2,114.7)(45,112.5)(47,109.8)(48,107)(49.5,104.2)%
%       (50.5,101.6)(51,98.5)(47.7,100.6)(46,102.2)(44.8,104)(43.5,106)
%       (42.5,108)(41.7,110.5)(41,113.2)}
%
% \newcommand{\Kangaroo}[1]{%
%   \begin{pspicture}(\DimX,\DimY)
%   \psset{unit=0.035278}
%   \KangarooItself{#1}
%   \end{pspicture}}
%
% \newcommand{\KangarooPstChart}[1]{{%
%   \psset{xunit=0.006784,yunit=0.00735,linewidth=0.01}
%   \begin{pspicture}(-65.5,0)(82,126)
%     \KangarooItself{#1}
%   \end{pspicture}}}
%
%
% ^^A For the possible index and changes log
% \setlength{\columnseprule}{0.6pt}
%
% ^^A Beginning of the documentation itself
%\title{\texttt{pst-fill}\\A PSTricks package for filling and tiling areas}
%\author{Timothy Van Zandt\thanks{\protect\url{tvz@econ.insead.fr}. (documentation by
% Denis Girou (\protect\url{Denis.Girou@idris.fr}) and Herbert Vo\ss (\protect\url{hvoss@tug.org}).}}
%
%\date{\shortstack{\today --- Version 1.00\\
%                  {\small Documentation revised \today}}}
% \maketitle
% \tableofcontents
%
%\begin{abstract}
%  \FillPackage{} is a PSTricks \cite{vanZandt93},\cite{Girou94},\cite{vanZandtGirou94}, 
%\cite{Hoenig97},\cite{LGC97} package to draw easily
%  various kinds of filling and tiling of areas. It is also a good example of
%  the great power and flexibility of PSTricks, as in fact it is a very short
%  program (it body is around 200~lines long) but nevertheless really powerful.
%
%  \hspace{5mm} It was written in 1994 by Timothy \textsc{van Zandt} but
%  publicly available only in PSTricks 97 and without any documentation.
%  We describe here the version \emph{97 patch 2} of December 12, 1997, which
%  is the original one modified by myself to manage \emph{tilings} in the
%  so-called \emph{automatic} mode. This article would like to serve both of
%  reference manual and of user's guide.
%
%This package is available on \CTAN{} in the
%  \texttt{graphics/pstricks} directory (files \texttt{latex/pst-fill.sty} and
%  \texttt{generic/pst-fill.tex}).
%\end{abstract}
%
%\section{Introduction}
%
%  Here we will refer as \emph{filling} as the operation which consist to fill
%a defined area by a pattern (or a composition of patterns). We will refer as
%\emph{tiling} as the operation which consist to do the same thing, but with
%the control of the starting point, which is here the upper left corner.
%The pattern is positioned relatively to this point. This make an essential
%difference between the two modes, as without control of the starting point we
%can't draw \emph{tilings} (sometimes  called \emph{tesselations}) as used in
%many fields of Art and Science%
%\footnote{For an extensive presentation of tilings, in their history and usage
%in many fields, see the reference book \cite{GS87}.
%
%  In the \TeX{} world, few work was done on tilings. You can look at the
%\emph{tile} extension of the \XYpic{} package \cite{XYpic}, at the articles of
%Kees \textsc{van der Laan} \cite[paragraph 7]{LAAN96} (the tiling was in
%fact directly done in PostScript) and \cite{LAAN97}, at the \MP{} program
%(available on \CTANref{Roegel}) by Denis \textsc{Roegel} for the
%\textsc{Truchet} contest in 1995 \cite{EsperetGirou98} and at the \MP{}
%package \cite{Bolek98} to draw patterns, which have a strong connection with
%tilings.}.
%
%  Nevertheless, as tilings are a wide and difficult field in mathematics, this
%package is limited to simple ones, mainly \emph{monohedral} tilings with one
%prototile (which can be composite, see section \ref{sec:KindTiles}). With some
%experience and wiliness we can do more and obtained easily rather
%sophisticated results, but obviously hyperbolic tilings like the famous
%\textsc{Escher} ones or aperiodic tilings like the \textsc{Penrose} ones are
%not in the capabilities of this package. For more complex needs, we must used
%low level and more painfull technics, with the basic \cs{multido}
%and \cs{multirput} macros.
%
%\section{Package history and description of it two different modes}
%
%  As already said, this package was written in 1994 by Timothy \textsc{van
%Zandt}. Two modes were defined, called respectively \emph{manual} and
%\emph{automatic}. For both, the pattern is generated on contiguous positions in
%a rather large area which include the region to fill, later cut to the
%required dimensions by clipping mechanism. In the first mode, the pattern is
%explicitely inserted in the PostScript file each time. In the second one, the
%result is the same but with an unique explicit insertion of the pattern and a
%repetition done by PostScript. Nevertheless, in this method, the control of
%the starting point was loosed, so it allowed only to \emph{fill} a region and
%not to \emph{tile} it.
%
%  See the difference between the two modes, \emph{tiling}:
% {\psset{unit=0.5cm}%
% \psboxfill{\begin{pspicture}(1,1)\psframe[dimen=middle](1,1)\end{pspicture}}
% \begin{pspicture}(3,3.3)
%   \psframe[fillstyle=boxfill](3,3)
% \end{pspicture}}
% and \emph{filling}:
%{%
% \makeatletter
%\pst@def{BoxFill}<
%  gsave
%    gsave \tx@STV CM grestore dtransform CM idtransform
%    abs /h ED abs /w ED
%    pathbbox
%    h div round 2 add cvi /y2 ED
%    w div round 2 add cvi /x2 ED
%    h div round 2 sub cvi /y1 ED
%    w div round 2 sub cvi /x1 ED
%    /y2 y2 y1 sub def
%    /x2 x2 x1 sub def
%    CP
%    y1 h mul sub neg /y1 ED
%    x1 w mul sub neg /x1 ED
%    clip
%    y2 {
%      /x x1 def
%      x2 {
%        save CP x y1 T moveto Box restore
%        /x x w add def
%      } repeat
%      /y1 y1 h add def
%    } repeat
% currentpoint currentfont grestore setfont moveto>
% \makeatother
%
% \psset{unit=0.5}
% \psboxfill{\begin{pspicture}(1,1)\psframe[dimen=middle](1,1)\end{pspicture}}
% \begin{pspicture}(3,3.3)
%   \psframe[fillstyle=boxfill](3,3)
% \end{pspicture}
% or
% \begin{pspicture}(3,3.3)
%   \psframe[fillstyle=boxfill](3,3)
% \end{pspicture}
%}
%as we can see that initial position is arbitrary and dependent of
%the current point.
%
%
% It's clear that usage of filling is very restrictive comparing to tiling,
%as desired effects required very often the possibility to control the starting 
%point. So, this mode was of limited interest, but unfortunately the
%\emph{manual} one has the very big disadvantage to require very huge amounts
%of ressources, mainly in disk space and consequently in printing time.
%A small tiling can require sometimes several megabytes in \emph{manual} mode!
%So, it was very often not really usable in practice.
%
%It is why I modified the code, to allow tilings in \emph{automatic} mode,
%controlling in this mode too the starting point. And most of the time, that is
%to say if some special options are not used, the tiling is done exactly in the
%region described, which make it faster. So there is no more reason to use the
%\emph{manual} mode, apart very special cases where \emph{automatic} one cannot
%work, as explained later -- currently, I know only one case.
%
%  To load this modified \emph{automatic} mode, with \LaTeX{} use
%simply:\newline 
%\verb+\usepackage[tiling]{pst-fill}+\newline
%and in plain \TeX{} after:\newline
%\verb+\input{pst-fill}+\newline
%add the following definition:\newline
%\verb+\def\PstTiling{true}+
%
%  To obtain the original behaviour, just don't use the \emph{tiling} optional
%keyword at loading.
%
%  Take care than in \emph{tiling} mode, I introduce also some other changes.
%First I define aliases on some parameter names for consistancy (all specific
%parameters will begin by the \texttt{fill} prefix in this case) and I change
%some default values, which were not well adapted for tilings (\texttt{fillsep}
%is set to 0 and as explained \texttt{fillsize} set to \texttt{auto}). I rename 
%\texttt{fillcycle} to \texttt{fillcyclex}. I also restore normal way so that
%the frame of the area is drawn and all line (\texttt{linestyle},
%\texttt{linecolor}, \texttt{doubleline}, etc.) parameters are now active (but
%there are not in non \emph{tiling} mode). And I also introduce new parameters
%to control the tilings (see below).
%
%  \textbf{In all the following examples, we will consider only the
% \emph{tiling} mode.}
%
%  To do a tiling, we have just to define the pattern with the
% \verb+\psboxfill+ macro and to use the new \texttt{fillstyle}
% \verb+boxfill+.
%
%  Note that tilings are drawn from left to right and top to bottom, which can
%have an importance in some circonstances.
%
%  PostScript programmers can be also interested to know that, even in the
%\emph{automatic} mode, the iterations of the pattern are managed directly by
%the PostScript code of the package which used only PostScript Level 1
%operators. The special ones introduced in Level 2 for drawing of patterns
%\cite[section 4.9]{PostScript95} are not used.
%
%  And first, for conveniance, we define a simple \cs{Tiling} macro, which
%will simplify our examples:
%
%\begin{verbatim}
%  \newcommand{\Tiling}[2][]{%
%    \edef\Temp{#1}%
%    \begin{pspicture}#2
%      \ifx\Temp\empty
%        \psframe[fillstyle=boxfill]#2
%      \else
%        \psframe[fillstyle=boxfill,#1]#2
%      \fi
%    \end{pspicture}}
%\end{verbatim}
%
%
%\newcommand{\Tiling}[2][]{%
%  \edef\Temp{#1}%
%  \begin{pspicture}#2
%    \ifx\Temp\empty
%      \psframe[fillstyle=boxfill]#2
%    \else
%      \psframe[fillstyle=boxfill,#1]#2
%    \fi
% \end{pspicture}}
%
%\subsection{Parameters}
%
%  There are \textbf{14} specific parameters available to change the way the
% filling/tiling is defined, and one debugging option.
%
% \begin{Description}{2cm}
%  \item [fillangle (real)\hfill :] the value of the rotation
%  applied to the patterns (\emph{Default:~0}).
% \end{Description}
%
%
%   In this case, we must force the tiling area to be notably larger than the
% area to cover, to be sure that the defined area will be covered after rotation.
% \lstset{gobble=2}
% \begin{LTXexample}
% \newcommand{\Square}{%
%   \begin{pspicture}(1,1)
%     \psframe[dimen=middle](1,1)
%   \end{pspicture}}
% \psset{unit=0.5}
% \psboxfill{\Square}
% \Tiling[fillangle=45]{(3,3)}\quad
% \Tiling[fillangle=-60]{(3,3)}
% \end{LTXexample}
% 
% \newcommand{\Square}{\begin{pspicture}(1,1)\psframe[dimen=middle](1,1)\end{pspicture}}
% 
% \begin{Description}{2cm}
%   \setcounter{footnote}{1}
%   \item[\texttt{fillsepx} (real$\|$dim) :] value of the horizontal
%   separation between consecutive patterns (\emph{Default:~0 for
%   tilings\footnotemark, 2pt otherwise}).  \footnotetext{This option was added
%   by me, is not part of the original package and is available only if the
%   \texttt{tiling} keyword is used when loading the package.}
%   \setcounter{footnote}{1}
%   \item [\texttt{fillsepy} (real$\|$dim)\hfill :] value of the vertical
%   separation between consecutive patterns (\emph{Default:~0 for
%   ti\-lings\footnotemark, 2pt otherwise}).
%   \setcounter{footnote}{1}
%   \item [\texttt{fillsep} (real$\|$dim)\hfill :] value of horizontal and
%   vertical separations between consecutive patterns (\emph{Default:~0 for
%   tilings\footnotemark, 2pt otherwise}).
% \end{Description}
% 
%   These values can be negative, which allow the tiles to overlap.
% 
% \begin{LTXexample}
% \psset{unit=0.5}
% \psboxfill{\Square}
% \Tiling[fillsepx=2mm]{(3,3)} 
% \Tiling[fillsepy=1mm]{(3,3)}\\
% \Tiling[fillsep=0.5]{(3,3)} 
% \Tiling[fillsep=-0.5]{(3,3)}
% \end{LTXexample}
% 
% \begin{Description}{2cm}
%   \item [\texttt{fillcyclex}\footnotemark\ (integer)\hfill :] Shift
%   coefficient applied to each row (\emph{Default:~0}).
%   \footnotetext{It was \texttt{fillcycle} in the original version.}
%   \setcounter{footnote}{1}
%   \item [\texttt{fillcycley}\footnotemark\ (integer)\hfill :] Same thing for
%   columns (\emph{Default:~0}).
%   \setcounter{footnote}{1}
%   \item [\texttt{fillcycle}\footnotemark\ (integer)\hfill :] Allow to fix
%   both \texttt{fillcyclex} and \texttt{fillcycley} directly to the same value
%   (\emph{Default:~0}).
% \end{Description}
% 
%   For instance, if \texttt{fillcyclex} is 2, the second row of patterns will
% be horizontally shifted by a factor of $\frac{1}{2}=0.5$, and by a factor of
% 0.333 if \texttt{fillcyclex} is 3, etc.). These values can be negative.
% 
% \begin{LTXexample}[width=0.35\linewidth]
% \psset{unit=0.5}
% \psboxfill{\Square}
% \newcommand{\TilingA}[1]{\Tiling[fillcyclex=#1]{(3,3)}}
% \TilingA{0} \TilingA{1}\\
% \TilingA{2} \TilingA{3}\\[3mm]
% \TilingA{4} \TilingA{5}\\
% \TilingA{6} \TilingA{-3}\\[3mm]
% \Tiling[fillcycley=2]{(3,3)}
% \Tiling[fillcycley=3]{(3,3)}\\
% \Tiling[fillcycley=-3]{(3,3)}
% \Tiling[fillcycle=2]{(3,3)}
% \end{LTXexample}
% 
% \begin{Description}{2cm}
%   \setcounter{footnote}{1}
%   \item [\texttt{fillmovex}\footnotemark\ (real$\|$dim)\hfill :] value of the
%   horizontal moves between consecutive patterns (\emph{Default:~0}).
%   \setcounter{footnote}{1}
%   \item [\texttt{fillmovey}\footnotemark\ (real$\|$dim)\hfill :] value of the
%   vertical moves between consecutive patterns (\emph{Default:~0}).
%   \setcounter{footnote}{1}
%   \item [\texttt{fillmove}\footnotemark\ (real$\|$dim)\hfill :] value of
%   horizontal and vertical moves between consecutive patterns
%   (\emph{Default:~0}).
% \end{Description}
% 
%   These parameters allow the patterns to overlap and to draw some special
% kinds of tilings. They are implemented only for the \emph{automatic} and
% \emph{tiling} modes and their values can be negative.
% 
%   In some cases, the effect of these parameters will be the same that with the 
% \texttt{fillcycle?} ones, but you can see that it is not true for some other
% values.
% 
% \begin{LTXexample}
% \psset{unit=0.5}
% \psboxfill{\Square}
% \Tiling[fillmovex=0.5]{(3,3)} 
% \Tiling[fillmovey=0.5]{(3,3)}\\
% \Tiling[fillmove=0.5]{(3,3)}
% \Tiling[fillmove=-0.5]{(3,3)}
% \end{LTXexample}
% 
% \begin{Description}{2cm}
%   \item [\texttt{fillsize}
%   (auto$\|$\{(real$\|$dim,real$\|$dim)(real$\|$dim,real$\|$dim)\}) :] The
%   choice of \emph{automatic} mode or the size of the area in \emph{manual}
%   mode. If first pair values are not given, (0,0) is used. (\emph{Default:
%   auto when \emph{tiling} mode is used, {(-15cm,-15cm)(15cm,15cm)}
%   otherwise}).
% \end{Description}
% 
%   As explained in the introduction, the \emph{manual} mode can require very
% huge amount of computer ressources. So, it usage is to discourage in front off
% the \emph{automatic} mode. It seems only useful in special circonstances, in
% fact when the \emph{automatic} mode failed, which is known only in one case,
% for some kinds of EPS files, as the ones produce by dump of portions of
% screens (see \ref{sec:GraphicFiles}).
% 
% \begin{Description}{2cm}
%   \setcounter{footnote}{1}
%   \item [\texttt{fillloopaddx}\footnotemark\ (integer)\hfill :] number of
%   times the pattern is added on left and right positions (\emph{Default:~0}).
%   \setcounter{footnote}{1}
%   \item [\texttt{fillloopaddy}\footnotemark\ (integer)\hfill :] number of
%   times the pattern is added on top and bottom positions (\emph{Default:~0}).
%   \setcounter{footnote}{1}
%   \item [\texttt{fillloopadd}\footnotemark\ (integer)\hfill :] number of
%   times the pattern is added on left, right, top and bottom positions
%   (\emph{Default:~0}).
% \end{Description}
% 
%   These parameters are only useful in special circonstances, as for complex
% patterns when the size of the rectangular box used to tile the area doesn't 
% correspond to the pattern itself (see an example in Figure~\ref{fig:Sheeps})
% and also sometimes when the size of the pattern is not a divisor of the size
% of the area to fill and that the number of loop repeats is not properly
% computed, which can occur.
% 
%   They are implemented only for the \emph{tiling} mode.
% 
% \begin{Description}{2cm}
%   \setcounter{footnote}{1}
%   \item [\texttt{PstDebug}\footnotemark\ (integer, 0 or 1)\hfill :] to
%   require to see the exact tiling done, without clipping (\emph{Default:~0}).
% \end{Description}
% 
%   It's mainly useful for debugging or to understand better how the tilings
% are done. It is implemented only for the \emph{tiling} mode.
% 
% \begin{LTXexample}
% \psset{unit=0.3,PstDebug=1}
% \psboxfill{\Square}
% \psset{linewidth=1mm}
% \Tiling{(2,2)}\\[5mm]
% \Tiling[fillcyclex=2]{(2,2)}\\[1cm]
% \Tiling[fillmove=0.5]{(2,2)}
% \end{LTXexample}
% 
% \vspace{3cm}
% \section{Examples}
% 
%   In fact this unique \cs{psboxfill} macro allow a lot a variations and
% different usages. We will try here to demonstrate this.
% 
% \subsection{Kind of tiles}
% \label{sec:KindTiles}
% 
%   Of course, we can access to all the power of PSTricks macros to define the
% \emph{tiles} (\emph{patterns}) used. So, we can define complicated ones.
% 
%   Here we give four other Archimedian tilings (those built with only some
% regular polygons) among the twelve existing, first discovered completely by
% Johanes \textsc{Kepler} at the beginning of 17th century \cite{GS87}, the two
% other \emph{regular} ones with the tiling by squares, formed by a unique
% regular polygon, and two other formed by two different regular polygons.
% 
% \begin{LTXexample}[pos=t]
%   \newcommand{\Triangle}{%
%     \begin{pspicture}(1,1)
%       \pstriangle[dimen=middle](0.5,0)(1,1)
%     \end{pspicture}}
%   \newcommand{\Hexagon}{
% ^^A sin(60)=0.866
%     \begin{pspicture}(0.866,0.75)
%       \SpecialCoor
% ^^A  Hexagon  
%       \pspolygon[dimen=middle]%
%         (0.5;30)(0.5;90)(0.5;150)(0.5;210)(0.5;270)(0.5;330)
%     \end{pspicture}}
% 
%   \psset{unit=0.5}
%   \psboxfill{\Triangle}
%   \Tiling{(4,4)}\hfill
% ^^A The two other regular tilings
%   \Tiling[fillcyclex=2]{(4,4)}\hfill
%   \psboxfill{\Hexagon}
%   \Tiling[fillcyclex=2,fillloopaddy=1]{(5,5)}
% \end{LTXexample}
% 
% \begin{LTXexample}[pos=t]
%   \newcommand{\ArchimedianA}{%
%      ^^A Archimedian tiling 3^2.4.3.4
%     \psset{dimen=middle}
%      ^^A sin(60)=0.866
%     \begin{pspicture}(1.866,1.866)
%       \psframe(1,1)
%       \psline(1,0)(1.866,0.5)(1,1)(0.5,1.866)(0,1)(-0.866,0.5)
%       \psline(0,0)(0.5,-0.866)
%     \end{pspicture}}
%   \newcommand{\ArchimedianB}{%
%      ^^A Archimedian tiling 4.8^2
%     \psset{dimen=middle,unit=1.5}
%      ^^A sin(22.5)=0.3827 ; cos(22.5)=0.9239
%     \begin{pspicture}(1.3066,0.6533)
%       \SpecialCoor
%      ^^A Octogon
%       \pspolygon(0.5;22.5)(0.5;67.5)(0.5;112.5)(0.5;157.5)
%                 (0.5;202.5)(0.5;247.5)(0.5;292.5)(0.5;337.5)
%     \end{pspicture}}
% 
%   \psset{unit=0.5}
%   \psboxfill{\ArchimedianA}
%   \Tiling[fillmove=0.5]{(7,7)}\hfill
%   \psboxfill{\ArchimedianB}
%   \Tiling[fillcyclex=2,fillloopaddy=1]{(7,7)}
% \end{LTXexample}
% 
%   \setcounter{footnote}{3}
%   We can of course tile an area arbitrarily defined. And with the
% \texttt{addfillstyle} parameter\footnote{Introduced in PSTricks 97.}, we can
% easily mix the \texttt{boxfill} style with another one.
% 
% \begin{LTXexample}[width=6cm]
%   \psset{unit=0.5,dimen=middle}
%   \psboxfill{%
%     \begin{pspicture}(1,1)
%       \psframe(1,1)
%       \pscircle(0.5,0.5){0.25}
%     \end{pspicture}}
%   \begin{pspicture}(4,6)
%     \pspolygon[fillstyle=boxfill,fillsep=0.25](0,1)(1,4)(4,6)(4,0)(2,1)
%   \end{pspicture}\hspace{1em}
%   \begin{pspicture}(4,4)
%%     \pscircle[linestyle=none,fillstyle=solid,fillcolor=yellow,fillsep=0.5,
%%               addfillstyle=boxfill](2,2){2}
%   \end{pspicture}
% \end{LTXexample}
%
%   Various effects can be obtained, sometimes complicated ones very easily, as
% in this example reproduced from one shown by Slavik \textsc{Jablan} in the
% field of \emph{OpTiles}, inspired by the \emph{Op-art}:
% 
% \begin{LTXexample}[pos=t]
% \newcommand{\ProtoTile}{%
%  \begin{pspicture}(1,1)%%% 1/12=0.08333
%   \psset{linestyle=none,linewidth=0,
%     hatchwidth=0.08333\psunit,hatchsep=0.08333\psunit}
%   \psframe[fillstyle=solid,fillcolor=black,addfillstyle=hlines,hatchcolor=white](1,1)
%   \pswedge[fillstyle=solid,fillcolor=white,addfillstyle=hlines]{1}{0}{90}
%  \end{pspicture}}
% \newcommand{\BasicTile}{%
%  \begin{pspicture}(2,1)
%    \rput[lb](0,0){\ProtoTile}\rput[lb](1,0){\psrotateleft{\ProtoTile}}
%  \end{pspicture}}
% \ProtoTile\hfill\BasicTile\hfill
% \psboxfill{\BasicTile}
% \Tiling[fillcyclex=2]{(4,4)}
% \end{LTXexample}
% 
%   It is also directly possible to surimpose several different tilings. Here is
% the splendid visual proof of the \textsc{Pytha\-gore} theorem done by the arab
% mathematician \textsc{Annairizi} around the year 900, given by superposition
% of two tilings by squares of different sizes.
% 
% \begin{LTXexample}[pos=t]
% \psset{unit=1.5,dimen=middle}
% \begin{pspicture*}(3,3)
%   \psboxfill{\begin{pspicture}(1,1)
%     \psframe(1,1)\end{pspicture}}
%   \psframe[fillstyle=boxfill](3,3)
%   \psboxfill{\begin{pspicture}(1,1)
%     \rput{-37}{\psframe[linecolor=red](0.8,0.8)}
%   \end{pspicture}}
%   \psframe[fillstyle=boxfill](3,4)
%   \pspolygon[fillstyle=hlines,hatchangle=90](1,2)(1.64,1.53)(2,2)
% \end{pspicture*}
% \end{LTXexample}
% 
%   In a same way, it is possible to build tilings based on figurative patterns,
% in the style of the famous \textsc{Escher} ones. Following an example of
% Andr\'e \textsc{Deledicq} \cite{Deledicq97}, we first show a simple tiling of
% the \emph{p1} category (according to the international classification of the
% 17~symmetry groups of the plane first discovered by the russian
% crystalographer Jevgraf \textsc{Fedorov} at the end of the 19th century):
% 
% \begin{LTXexample}[pos=t]
%  \newcommand{\SheepHead}[1]{%
%    \begin{pspicture}(3,1.5)
%      \pscustom[liftpen=2,fillstyle=solid,fillcolor=#1]{%
%        \pscurve(0.5,-0.2)(0.6,0.5)(0.2,1.3)(0,1.5)(0,1.5)
%          (0.4,1.3)(0.8,1.5)(2.2,1.9)(3,1.5)(3,1.5)(3.2,1.3)
%          (3.6,0.5)(3.4,-0.3)(3,0)(2.2,0.4)(0.5,-0.2)}
%      \pscircle*(2.65,1.25){0.12\psunit} % Eye
%      \psccurve*(3.5,0.3)(3.35,0.45)(3.5,0.6)(3.6,0.4)% Muzzle
%     ^^A   % Mouth
%       \pscurve(3,0.35)(3.3,0.1)(3.6,0.05)
%     ^^A   % Ear
%       \pscurve(2.3,1.3)(2.1,1.5)(2.15,1.7)\pscurve(2.1,1.7)(2.35,1.6)(2.45,1.4)
%   \end{pspicture}}
%  \psboxfill{\psset{unit=0.5}\SheepHead{yellow}\SheepHead{cyan}}
%  \Tiling[fillcyclex=2,fillloopadd=1]{(10,5)}
% \end{LTXexample}
% \label{fig:Sheeps}
% 
%   Now a tiling of the \emph{pg} category (the code for the kangaroo itself is
% too long to be shown here, but has no difficulties ; the kangaroo is reproduce
% from an original picture from Raoul \textsc{Raba} and here is a translation in
% PSTricks from the one drawn by Emmanuel \textsc{Chailloux} and Guy
% \textsc{Cousineau} for their MLgraph system \cite{MLgraphTSI}):
% 
% \begin{LTXexample}[pos=t]
% \psboxfill{\psset{unit=0.4}
%   \Kangaroo{yellow}\Kangaroo{red}\Kangaroo{cyan}\Kangaroo{green}%
%   \psscalebox{-1 1}{%
%     \rput(1.235,4.8){\Kangaroo{green}\Kangaroo{cyan}\Kangaroo{red}\Kangaroo{yellow}}}}
%   \Tiling[fillloopadd=1]{(10,6)}
% \end{LTXexample}
% 
%   And here a \textsc{Wang} tiling \cite{Wang65}, \cite[chapter
% 11]{GS87}, based on very simple tiles of the form of a square and composed
% of four colored triangles. Such tilings are built with only a matching color
% constraint. Despite of it simplicity, it is an important kind of tilings, as
% \textsc{Wang} and others used them to study the special class of
% \emph{aperiodic} tilings, and also because it was shown that surprisingly this 
% tiling is similar to a \textsc{Turing} machine.
% 
% \begin{LTXexample}[pos=t]
%   \newcommand{\WangTile}[4]{%
%     \begin{pspicture}(1,1)
%       \pspolygon*[linecolor=#1](0,0)(0,1)(0.5,0.5)
%       \pspolygon*[linecolor=#2](0,1)(1,1)(0.5,0.5)
%       \pspolygon*[linecolor=#3](1,1)(1,0)(0.5,0.5)
%       \pspolygon*[linecolor=#4](1,0)(0,0)(0.5,0.5)
%     \end{pspicture}}
%   \newcommand{\WangTileA}{\WangTile{cyan}{yellow}{cyan}{cyan}}
%   \newcommand{\WangTileB}{\WangTile{yellow}{cyan}{cyan}{red}}
%   \newcommand{\WangTileC}{\WangTile{cyan}{red}{yellow}{yellow}}
%   \newcommand{\WangTiles}[1][]{%
%     \begin{pspicture}(3,3) \psset{ref=lb}
%       \rput(0,2){\WangTileB}  \rput(1,2){\WangTileA}%
%       \rput(2,2){\WangTileC}  \rput(0,1){\WangTileC}%
%       \rput(1,1){\WangTileB}  \rput(2,1){\WangTileA}
%       \rput(0,0){\WangTileA}  \rput(1,0){\WangTileC}%
%       \rput(2,0){\WangTileB}
%       #1
%     \end{pspicture}}
%   \WangTileA\hfill\WangTileB\hfill\WangTileC\hfill
%   \WangTiles[{\psgrid[subgriddiv=0,gridlabels=0](3,3)}]\hfill
%   \psset{unit=0.4} \psboxfill{\WangTiles} \Tiling{(12,12)}
% \end{LTXexample}
% 
% \subsection{External graphic files}
% \label{sec:GraphicFiles}
% 
%   We can also fill an arbitrary area with an external image. We have only, 
% as usual, to matter of the \emph{BoundingBox} definition if there is no one
% provided or if it is not the accurate one, as for the well known
% \texttt{tiger} picture part of the \texttt{ghostscript} distribution.
% 
% \begin{LTXexample}[pos=t]
%   \psboxfill{%% Strangely require x1=x2...
%     \begin{pspicture}(0,1)(0,4.1)
%       \includegraphics[bb=17 176 560 74,width=3cm]{tiger}
%     \end{pspicture}}
%   \Tiling{(6,6.2)}
% \end{LTXexample}
% 
%   Nevertheless, there are some special files for which the \emph{automatic}
% mode doesn't work, specially for some files obtained by a screen dump, as in
% the next example, where a picture was reduced before it conversion in the
% \emph{Encapsulated PostScript} format by a screen dump utility. In this case,
% usage of the \emph{manual} mode is the only alternative, at the price of the
% real multiple inclusion of the EPS file. We must take care to specify the
% correct \texttt{fillsize} parameter, because otherwise the default values are
% large and will load the file many times, perhaps just really using few
% occurrences as the other ones would be clipped...
% 
% \begin{LTXexample}[pos=t]
%   \psboxfill{\includegraphics{flowers}}
%   \begin{pspicture}(8,4)
%     \psellipse[fillstyle=boxfill,fillsize={(8,4)}](4,2)(4,2)
%   \end{pspicture}
% \end{LTXexample}
% 
% \subsection{Tiling of characters}
% 
%   We can also use the \cs{psboxfill} macro to fill the interior of characters
% for special effects like these ones:
% 
% \begin{LTXexample}[pos=t]
%   \DeclareFixedFont{\bigsf}{T1}{phv}{b}{n}{4.5cm}
%   \DeclareFixedFont{\smallrm}{T1}{ptm}{m}{n}{3mm}
%   \psboxfill{\smallrm Since 182 days...}
%   \begin{pspicture*}(8,4)
%     \centerline{%
%       \pscharpath[fillstyle=gradient,gradangle=-45,
%                   gradmidpoint=0.5,addfillstyle=boxfill,
%                   fillangle=45,fillsep=0.7mm]
%                  {\rput[b](0,0.1){\bigsf 2000}}}
%   \end{pspicture*}
% \end{LTXexample}
% 
% \begin{LTXexample}[pos=t]
%   \DeclareFixedFont{\mediumrm}{T1}{ptm}{m}{n}{2cm}
%   \psboxfill{%
%     \psset{unit=0.1,linewidth=0.2pt}
%     \Kangaroo{PeachPuff}\Kangaroo{PaleGreen}%
%       \Kangaroo{LightBlue}\Kangaroo{LemonChiffon}%
%     \psscalebox{-1 1}{%
%       \rput(1.235,4.8){%
%         \Kangaroo{LemonChiffon}\Kangaroo{LightBlue}%
%           \Kangaroo{PaleGreen}\Kangaroo{PeachPuff}}}}
% ^^A   % A kangaroo of kangaroos...
%   \begin{pspicture}(8,2)
%     \pscharpath[linestyle=none,fillstyle=boxfill,fillloopadd=1]
%                {\rput[b](4,0){\mediumrm Kangaroo}}
%   \end{pspicture}
% \end{LTXexample}
% 
% \subsection{Other kinds of usage}
% 
%   Other kinds of usage can be imagined. For instance, we can use tilings in a
% sort of degenerated way to draw some special lines made by a unique or
% multiple repeating patterns. But it can be only a special dashed line, as here
% with three different dashes:
% 
% \begin{LTXexample}[pos=t]
%   \newcommand{\Dashes}{%
%     \psset{dimen=middle}
%     \begin{pspicture}(0,-0.5\pslinewidth)(1,0.5\pslinewidth)
%       \rput(0,0){\psline(0.4,0)}%
%         \rput(0.5,0){\psline(0.2,0)}%
%         \rput(0.8,0){\psline(0.1,0)}
%     \end{pspicture}}
% 
%   \newcommand{\SpecialDashedLine}[3]{%
%     \psboxfill{#3}
%     \Tiling[linestyle=none]
%            {(#1,-0.5\pslinewidth)(#2,0.5\pslinewidth)}}
% 
%   \SpecialDashedLine{0}{7}{\Dashes}
% 
%   \psset{unit=0.5,linewidth=1mm,linecolor=red}
%   \SpecialDashedLine{0}{10}{\Dashes}
% \end{LTXexample}
% 
%   It allow also to use special patterns in business graphics, as in the
% following example generated by \texttt{PstChart}\footnote{A personal
% development to draw business charts with PSTricks, not distributed.}.
% 
% \vspace{3mm}
% \begin{figure}[!ht]
% \centering
% \psset{unit=0.75}
% ^^A % Generated by pstchart.sh version 0.21 (11/28/97)
% {\psset{dimen=middle}
% \psset{xunit=2,yunit=0.005}
% \begin{pspicture}(-0.6,-200)(6.6,2300)
% ^^A   % Title
%   \rput(3,2200){\shortstack{Fantaisist repartition of kangaroos\\
%                             in the world (in thousands)}}
% ^^A   % Frame background
%   \psframe[fillstyle=solid,fillcolor=LemonChiffon](0,0)(6,2000)
% ^^A   % Graduations
%   \multido{\n=0+500}{5}{\rput[r](-0.12,\n){\psscalebox{0.8}{\n}}}
% ^^A   % Minor ticks
%   \multips(0,100)(0,100){19}{\psline[unit=4.8pt](1,0)}
%   \multips(6,100)(0,100){19}{\psline[unit=4.8pt](-1,0)}
% ^^A   % Major ticks
%   \multips(0,500)(0,500){3}{\psline[unit=9.6pt](1,0)}
%   \multips(6,500)(0,500){3}{\psline[unit=9.6pt](-1,0)}
% ^^A   % Lines from major ticks marks
%   \multips(0,500)(0,500){3}{\psline[linestyle=dotted,linewidth=0.6pt](6,0)}
% ^^A   % Drawing for the data
%   \psboxfill{\psset{unit=0.78\psxunit}\KangarooPstChart{red}}
%   \psframe[linestyle=none,fillstyle=boxfill,fillloopaddy=1](0.61,0)(1.39,1800)
%   \psboxfill{\psset{unit=0.78\psxunit}\KangarooPstChart{yellow}}
%   \psframe[linestyle=none,fillstyle=boxfill,fillloopaddy=1](1.61,0)(2.39,800)
%   \psboxfill{\psset{unit=0.78\psxunit}\KangarooPstChart{cyan}}
%   \psframe[linestyle=none,fillstyle=boxfill,fillloopaddy=1](2.61,0)(3.39,550)
%   \psboxfill{\psset{unit=0.78\psxunit}\KangarooPstChart{magenta}}
%   \psframe[linestyle=none,fillstyle=boxfill,fillloopaddy=1](3.61,0)(4.39,500)
%   \psboxfill{\psset{unit=0.78\psxunit}\KangarooPstChart{green}}
%   \psframe[linestyle=none,fillstyle=boxfill,fillloopaddy=1](4.61,0)(5.39,200)
% ^^A   % Bottom labels
%   \uput{0.2}[270]{0}(1,0){\psscalebox{0.7}{Oceania}}
%   \uput{0.2}[270]{0}(2,0){\psscalebox{0.7}{Africa}}
%   \uput{0.2}[270]{0}(3,0){\psscalebox{0.7}{Asia}}
%   \uput{0.2}[270]{0}(4,0){\psscalebox{0.7}{America}}
%   \uput{0.2}[270]{0}(5,0){\psscalebox{0.7}{Europe}}
% ^^A   % Frame box around the chart
%   \psframe[linestyle=solid](0,0)(6,2000)
% \end{pspicture}}
%   \caption{Bar chart generated by PstChart, with bars filled by patterns}
%   \label{fig:PstChart}
% \end{figure}
% 
% \section{``Dynamic'' tilings}
% 
%   In some cases, tilings used non \emph{static} tiles, that is to say that the 
% \emph{prototile(s)}, even if unique, can have several forms, by instance
% specified by different colors or rotations, not fixed before generation or
% varying each time.
% 
% \subsection{Lewthwaite-Pickover-Truchet tiling}
% 
%   We give here for example the so-called \emph{Truchet} tiling, which much be
% in fact better called \emph{Lewthwaite-Pick\-over-Truchet (LPT)} tiling%
% \footnote{For description of the context, history and references about
% S\'ebastien \textsc{Truchet} and this tiling, see \cite{EsperetGirou98}.}.
% 
%   The unique prototile is only a square with two opposite circle arcs.
% This tile has obviously two positions, if we rotate it from 90 degrees (see
% the two tiles on the next figure). A \emph{LPT tiling} is a tiling with
% randomly oriented LPT tiles. We can see that even if it is very simple in it
% principle, it draw sophisticated curves with strange properties.
% 
%   Nevertheless, in the straightforward way \FillPackage{} does not work,
% because the \cs{psboxfill} macro store the content of the tile used in a
% \TeX{} box, which is static. So the calling to the random function is done
% only one time, which explain that only one rotation of the tile is used for
% all the tiling. It's only the one of the two rotations which could differ from
% one drawing to the next one...
% 
% ^^A % Truchet (Lewthwaite-Pickover-Truchet) tiling
% ^^A % --------------------------------------------
% 
% \begin{LTXexample}[pos=t]
% ^^A   % LPT prototile
%   \newcommand{\ProtoTileLPT}{%
%     \psset{dimen=middle}
%     \begin{pspicture}(1,1)
%       \psframe(1,1)
%       \psarc(0,0){0.5}{0}{90}
%       \psarc(1,1){0.5}{-180}{-90}
%     \end{pspicture}}
% 
% ^^A   % LPT tile
%   \newcount\Boolean
%   \newcommand{\BasicTileLPT}{%
% ^^A     % From random.tex by Donald Arseneau
%     \setrannum{\Boolean}{0}{1}%
%     \ifnum\Boolean=0
%       \ProtoTileLPT%
%     \else
%       \psrotateleft{\ProtoTileLPT}%
%     \fi}
% 
%   \ProtoTileLPT\hfill\psrotateleft{\ProtoTileLPT}\hfill
%   \psset{unit=0.5}
%   \psboxfill{\BasicTileLPT}
%   \Tiling{(5,5)}
% \end{LTXexample}
% 
%   But, for simple cases, there is a solution to this problem using a mixture
% of PSTricks and PostScript programming. Here the PSTricks
% construction \verb+\pscustom{\code{...}}+ allow to insert PostScript code
% inside the \LaTeX{} + PSTricks one.
% 
%   Programmation is less straightforward, but it has also the advantage to be
% notably faster, as all the tilings operations are done in PostScript, and
% mainly to not be limited by \TeX{} memory (the \TeX{} + PSTricks solution
% I wrote in 1995 for the colored problem was limited to small sizes for this
% reason). Just note also that \cs{pslbrace} and \cs{psrbrace} are two
% PSTricks macros to define and be able to insert the \verb+{+ and \verb+}+
% characters.
% 
% \begin{LTXexample}[pos=t]
% ^^A   % LPT prototile
%   \newcommand{\ProtoTileLPT}{%
%     \psset{dimen=middle}
%     \psframe(1,1)
%     \psarc(0,0){0.5}{0}{90}
%     \psarc(1,1){0.5}{-180}{-90}}
% 
% ^^A   % Counter to change the random seed
%   \newcount\InitCounter
% ^^A   % LPT tile
%   \newcommand{\BasicTileLPT}{%
%     \InitCounter=\the\time
%     \pscustom{\code{%
%       rand \the\InitCounter\space sub 2 mod 0 eq \pslbrace}}
%     \begin{pspicture}(1,1)
%       \ProtoTileLPT
%     \end{pspicture}%
%     \pscustom{\code{\psrbrace \pslbrace}}
%     \psrotateleft{\ProtoTileLPT}%
%     \pscustom{\code{\psrbrace ifelse}}}
% 
%   \psset{unit=0.4,linewidth=0.4pt}
%   \psboxfill{\BasicTileLPT}
%   \Tiling{(15,15)}
% \end{LTXexample}
% 
%   Using the very surprising fact (see \cite{EsperetGirou98}) that
% coloration of these tiles do not depend of their neighbors (even if it is
% difficult to believe as the opposite seems obvious!) but only of the parity of
% the value of row and column positions, we can directly program in the same way
% a colored version of the LPT tiling.
% 
% \setcounter{footnote}{1}
%   We have also introduce in the \FillPackage{} code for \emph{tiling} mode two
% new accessible Post\-Script variables, \texttt{row} and
% \texttt{column}\footnotemark, which can be useful in some circonstances, like
% this one.
% 
% \begin{LTXexample}[pos=t]
% ^^A   % LPT prototile
%   \newcommand{\ProtoTileLPT}[2]{%
%     \psset{dimen=middle,linestyle=none,fillstyle=solid}
%     \psframe[fillcolor=#1](1,1)
%     \psset{fillcolor=#2}
%     \pswedge(0,0){0.5}{0}{90} \pswedge(1,1){0.5}{-180}{-90}}
% ^^A   % Counter to change the random seed
%   \newcount\InitCounter
% ^^A   % LPT tile
%   \newcommand{\BasicTileLPT}[2]{%
%     \InitCounter=\the\time
%     \pscustom{\code{%
%       rand \the\InitCounter\space sub 2 mod 0 eq \pslbrace
%       row column add 2 mod 0 eq \pslbrace}}
%     \begin{pspicture}(1,1)\ProtoTileLPT{#1}{#2}\end{pspicture}%
%     \pscustom{\code{\psrbrace \pslbrace}}
%     \ProtoTileLPT{#2}{#1}%
%     \pscustom{\code{%
%       \psrbrace ifelse \psrbrace \pslbrace row column add 2 mod 0 eq \pslbrace}}
%     \psrotateleft{\ProtoTileLPT{#2}{#1}}\pscustom{\code{\psrbrace \pslbrace}}
%     \psrotateleft{\ProtoTileLPT{#1}{#2}}\pscustom{\code{\psrbrace ifelse \psrbrace ifelse}}}
%   \psboxfill{\BasicTileLPT{red}{yellow}}
%   \Tiling{(4,4)}\hfill
%   \psset{unit=0.4}\psboxfill{\BasicTileLPT{blue}{cyan}}
%   \Tiling{(15,15)}
% \end{LTXexample}
% 
%   Another classic example is to generate coordinates and numerotation for a
% grid. Of course, it is possible to do it directly in PSTricks using nested
% \cs{multido} commands. It would be clearly easy to program, but, nevertheless, 
% for users who have a little knowledge of PostScript programming, this offer
% an alternative which is useful for large cases, because on this way it will
% be notably faster and less computer ressources consuming.
% 
%   Remember here that the tiling is drawn from left to right, and top to
% bottom, and note that the PostScript variable \texttt{x2} give the total
% number of columns.
% 
% \begin{LTXexample}[pos=t]
% ^^A   % \Escape will be the \ character
%   {\catcode`\!=0\catcode`\\=11!gdef!Escape{\}}
%   \newcommand{\ProtoTile}{%
%     \Square\pscustom{%
%       \moveto(-0.9,0.75) % In PSTricks units
%       \code{ /Times-Italic findfont 8 scalefont setfont
%         (\Escape() show row 3 string cvs show (,) show 
%         column 3 string cvs show (\Escape)) show}
%       \moveto(-0.5,0.25) % In PSTricks units
%       \code{ /Times-Bold findfont 18 scalefont setfont
%         1 0 0 setrgbcolor % Red color
%         /center {dup stringwidth pop 2 div neg 0 rmoveto} def
%         row 1 sub x2 mul column add 3 string cvs center show}}}
%   \psboxfill{\ProtoTile}
%   \Tiling{(6,4)}
% \end{LTXexample}
% 
% \subsection{A complete example: the Poisson equation}
% 
%   To finish, we will show a complete real example, a drawing to explain the
% method used to solve the \textsc{Poisson} equation by a domain
% decomposition method, adapted to distributed memory computers. The
% objective is to show the communications required between processes and the
% position of the data to exchange. This code also show some useful and powerful
% technics for PSTricks programming (look specially at the way some higher level
% macros are defined, and how the same object is used to draw the four
% neighbors).
%
%\psset{unit=1cm}
%\newcommand{\Pattern}[1]{%
%   \begin{pspicture}(-0.25,-0.25)(0.25,0.25)\rput{*0}{\psdot[dotstyle=#1]}
%   \end{pspicture}}
%\newcommand{\West}{\Pattern{o}}   \newcommand{\South}{\Pattern{x}}
%\newcommand{\Central}{\Pattern{+}}\newcommand{\North}{\Pattern{square}}
%\newcommand{\East}{\Pattern{triangle}}
%\newcommand{\Cross}{%
%  \pspolygon[unit=0.5,linewidth=0.2,linecolor=red](0,0)(0,1)(1,1)(1,2)(2,2)(2,1)%
%              (3,1)(3,0)(2,0)(2,-1)(1,-1)(1,0)}
%\newcommand{\StylePosition}[1]{\LARGE\textcolor{red}{\textbf{#1}}}
%\newcommand{\SubDomain}[4]{%
%    \psboxfill{#4}\begin{psclip}{\psframe[linestyle=none]#1}%
%      \psframe[linestyle=#3](5,5)\psframe[fillstyle=boxfill]#2%
%    \end{psclip}}
%\newcommand{\SendArea}[1]{\psframe[fillstyle=solid,fillcolor=cyan]#1}
%\newcommand{\ReceiveData}[2]{%
%  \psboxfill{#2}\psframe[fillstyle=solid,fillcolor=yellow,addfillstyle=boxfill]#1}%
%\newcommand{\Neighbor}[2]{%
%    \begin{pspicture}(5,5)
%      \rput{*0}(2.5,2.5){\StylePosition{#1}}
%      \ReceiveData{(0.5,0)(4.5,0.5)}{\Central}\SendArea{(0.5,0.5)(4.5,1)}%
%      \SubDomain{(5,2)}{(0.5,0.5)(4.5,3)}{dashed}{#2}%
%      \pcarc[arcangle=45,arrows=->](0.5,-1.25)(0.5,0.25)%
%      \pcarc[arcangle=45,arrows=->,linestyle=dotted,dotsep=2pt](4.5,0.75)(4.5,-0.75)%
%    \end{pspicture}}%
%  \psset{dimen=middle,dotscale=2,fillloopadd=2}
%\begin{pspicture}(-5.7,-5.7)(5.7,5.7)
%  \rput(0,0){%
%      \begin{pspicture}(5,5)
%        \ReceiveData{(0,0.5)(0.5,4.5)}{\West} \ReceiveData{(4.5,0.5)(5,4.5)}{\East}
%        \ReceiveData{(0.5,4.5)(4.5,5)}{\North}\ReceiveData{(0.5,0)(4.5,0.5)}{\South}
%        \SendArea{(0.5,0.5)(1,4.5)}\SendArea{(4,0.5)(4.5,4.5)}
%        \SendArea{(0.5,0.5)(4.5,1)}\SendArea{(0.5,4)(4.5,4.5)}
%        \SubDomain{(5,5)}{(0.5,0.5)(4.5,4.5)}{solid}{\Central}
%        \psline(1,0.5)(1,4.5)\psline(4,0.5)(4,4.5)%
%        \rput(1.5,4){\Cross}\rput(2,2){\Cross}%
%      \end{pspicture}}%
%  \rput(0,5.5){\Neighbor{N}{\North}}\rput{-90}(5.5,0){\Neighbor{E}{\East}}%
%  \rput{90}(-5.5,0){\Neighbor{W}{\West}}\rput{180}(0,-5.5){\Neighbor{S}{\South}}%
%\end{pspicture}
%
% \begin{lstlisting}
%   \newcommand{\Pattern}[1]{%
%     \begin{pspicture}(-0.25,-0.25)(0.25,0.25)\rput{*0}{\psdot[dotstyle=#1]}
%     \end{pspicture}}
%   \newcommand{\West}{\Pattern{o}}   \newcommand{\South}{\Pattern{x}}
%   \newcommand{\Central}{\Pattern{+}}\newcommand{\North}{\Pattern{square}}
%   \newcommand{\East}{\Pattern{triangle}}
%   \newcommand{\Cross}{%
%     \pspolygon[unit=0.5,linewidth=0.2,linecolor=red](0,0)(0,1)(1,1)(1,2)(2,2)(2,1)
%               (3,1)(3,0)(2,0)(2,-1)(1,-1)(1,0)}
%   \newcommand{\StylePosition}[1]{\LARGE\textcolor{red}{\textbf{#1}}}
%   \newcommand{\SubDomain}[4]{%
%     \psboxfill{#4}
%     \begin{psclip}{\psframe[linestyle=none]#1}
%       \psframe[linestyle=#3](5,5)\psframe[fillstyle=boxfill]#2
%     \end{psclip}}
%   \newcommand{\SendArea}[1]{\psframe[fillstyle=solid,fillcolor=cyan]#1}
%   \newcommand{\ReceiveData}[2]{%
%     \psboxfill{#2}
%     \psframe[fillstyle=solid,fillcolor=yellow,addfillstyle=boxfill]#1}
%   \newcommand{\Neighbor}[2]{%
%     \begin{pspicture}(5,5)
%       \rput{*0}(2.5,2.5){\StylePosition{#1}}
%       \ReceiveData{(0.5,0)(4.5,0.5)}{\Central}\SendArea{(0.5,0.5)(4.5,1)}
%       \SubDomain{(5,2)}{(0.5,0.5)(4.5,3)}{dashed}{#2}%
% ^^A       % Receive and send arrows
%       \pcarc[arcangle=45,arrows=->](0.5,-1.25)(0.5,0.25)
%       \pcarc[arcangle=45,arrows=->,linestyle=dotted,dotsep=2pt](4.5,0.75)(4.5,-0.75)
%     \end{pspicture}}
%   \psset{dimen=middle,dotscale=2,fillloopadd=2}
%   \begin{pspicture}(-5.7,-5.7)(5.7,5.7)
% ^^A     % Central domain
%     \rput(0,0){%
%       \begin{pspicture}(5,5)
% ^^A         % Receive from West, East, North and South
%         \ReceiveData{(0,0.5)(0.5,4.5)}{\West} \ReceiveData{(4.5,0.5)(5,4.5)}{\East}
%         \ReceiveData{(0.5,4.5)(4.5,5)}{\North}\ReceiveData{(0.5,0)(4.5,0.5)}{\South}
% ^^A         % send area for West, East, North and South
%         \SendArea{(0.5,0.5)(1,4.5)} \SendArea{(4,0.5)(4.5,4.5)}
%         \SendArea{(0.5,0.5)(4.5,1)} \SendArea{(0.5,4)(4.5,4.5)}
% ^^A         % Central domain
%         \SubDomain{(5,5)}{(0.5,0.5)(4.5,4.5)}{solid}{\Central}
% ^^A         % Redraw overlapped linesY
%         \psline(1,0.5)(1,4.5)  \psline(4,0.5)(4,4.5)
% ^^A         % Two crossesY
%         \rput(1.5,4){\Cross}  \rput(2,2){\Cross}
%       \end{pspicture}}
% ^^A     % The four neighborsY
%     \rput(0,5.5){\Neighbor{N}{\North}}     \rput{-90}(5.5,0){\Neighbor{E}{\East}}
%     \rput{90}(-5.5,0){\Neighbor{W}{\West}} \rput{180}(0,-5.5){\Neighbor{S}{\South}}
%   \end{pspicture}
% \end{lstlisting}
%
%
%
% Bibliography
% \begin{thebibliography}{99}
% \bibitem{PostScript95} Adobe, Systems~Incorporated, \emph{PostScript Language
% Reference Manual}, Addison-Wesley, 2~edition, 1995.
%
% \bibitem{Bolek98} Piotr Bolek, \MP{} and patterns, \emph{\TUB}, Volume~19,
% Number~3, pages 276--283, September 1998, \CTANref{mpattern}.
%
% \bibitem{MLgraphTSI} Emmanuel Chailloux, Guy Cousineau and Asc\'ander
% Su\'arez, Programmation fonctionnelle de graphismes pour la production
% d'illustrations techniques, \emph{Technique et science informatique},
% Volume~15, Number~7, pages 977--1007, 1996 (in french).
%
% \bibitem{Deledicq97} Andr\'e Deledicq, \emph{Le monde des pavages}, ACL
% \'Editions, 1997 (in french).
%
% \bibitem{EsperetGirou98} Philippe Esperet and Denis Girou,
% Coloriage du pavage dit de Truchet, Cahiers GUTenberg, Number~31,
% pages 5--18, December~1998  (in french).
%
% \bibitem{Girou94} Denis Girou, Pr\'esentation de PSTricks, \emph{Cahiers
% GUTenberg}, Number~16, pages 21--70, February~1994 (in french).
%
% \bibitem{LGC97} Michel Goossens, Sebastian Rahtz and Frank Mittelbach,
% \emph{The \LaTeX{} Graphics Companion}, Addison-Wesley, 2005.
%
% \bibitem{GS87} Branko Gr\"unbaum and Geoffrey Shephard, \emph{Tilings and
% Patterns}, Freeman and Company, 1987.
%
% \bibitem{Hoenig97} Alan Hoenig, \emph{\TeX{} Unbound: \LaTeX{} \& \TeX{}
% Strategies, Fonts, Graphics, and More}, Oxford University Press, 1997.
%
% \bibitem{XYpic} Kristoffer~H. Rose and Ross Moore, \XYpic. Pattern and Tile
% extension, available from \CTAN, 1991-1998, \CTANref{xypic}.
%
% \bibitem{LAAN96} Kees van der Laan, Paradigms: Just a little bit of PostScript,
% \emph{MAPS}, Volume~17, pages 137--150, 1996.
%
% \bibitem{LAAN97} Kees van der Laan, Tiling in PostScript and \MF{} -- Escher's
% wink, \emph{MAPS}, Volume~19, Number~2, pages 39--67, 1997.
%
% \bibitem{vanZandt93} Timothy Van Zandt, PSTricks. PostScript macros for
% Generic \TeX, available from \CTAN, 1993, \CTANref{pstricks}.
%
% \bibitem{vanZandtGirou94} Timothy Van Zandt and Denis Girou, Inside PSTricks,
% \emph{\TUB}, Volume~15, Number~3, pages 239--246, September 1994.
%
%
% \bibitem{voss07} Herbert Vo\ss, PSTricks -- Graphics for \TeX\ and \LaTeX, DANTE/Lehmanns, 4th ed., 2007.
% \bibitem{Wang65} Hao Wang, Games, Logic and Computers, \emph{Scientific
% American}, pages 98--106, November 1965.
% \end{thebibliography}
%
%
% \StopEventually{}
%
% ^^A .................... End of the documentation part ....................
%
% \section{Driver file}
%
%   The next bit of code contains the documentation driver file for \TeX{},
% i.e., the file that will produce the documentation you are currently
% reading. It will be extracted from this file by the \texttt{docstrip}
% program.
%
%    \begin{macrocode}
%<*driver>
\documentclass{ltxdoc}
\GetFileInfo{pst-fill.dtx}
%
\usepackage[T1]{fontenc}
\usepackage{lmodern}               % For PDF
\usepackage{graphicx}              % `graphicx' LaTeX standard package
\usepackage{showexpl}
\usepackage{mflogo}                % For the MetaFont and MetaPost logos
\input{random.tex}                 % Random macros from Donald Arseneau
\usepackage{url}                   % URLs convenient typesetting
\usepackage{multido}               % General loop macro
\usepackage[dvipsnames]{pstricks}  % PSTricks with the `color' extension
\usepackage{pst-text}              % PSTricks package for character path
\usepackage{pst-grad}              % PSTricks package for gradient filling
\usepackage{pst-node}              % PSTricks package for nodes
\usepackage[tiling]{pst-fill}      % PSTricks package for filling/tiling
%
\AtBeginDocument{%
%  \OnlyDescription % comment out for implementation details
  \EnableCrossrefs
  \CodelineIndex
  \RecordChanges}
\AtEndDocument{%
  \PrintIndex
  \setcounter{IndexColumns}{1}
  \PrintChanges}
\hbadness=7000            % Over and under full box warnings
\hfuzz=3pt
\begin{document}
  \DocInput{pst-fill.dtx}
\end{document}
%</driver>
%    \end{macrocode}
%
% \section{\texttt{pst-fill} \LaTeX{} wrapper}
%
%    \begin{macrocode}
%<*latex-wrapper>
\RequirePackage{pstricks}
\ProvidesPackage{pst-fill}[2005/09/13 package wrapper for 
  pst-fill.tex (hv)]
\DeclareOption{tiling}{\def\PstTiling{true}}
\ProcessOptions\relax
\input{pst-fill.tex}
\ProvidesFile{pst-fill.tex}
  [\filedate\space v\fileversion\space `PST-fill' (tvz,dg)]
%</latex-wrapper>
%    \end{macrocode}
%
%
% \section{Pst-Fill Package{} code}
%
%    \begin{macrocode}
%<*pst-fill>
%    \end{macrocode}
%
% \subsection{Preamble}
%
%   Who we are.
%
%    \begin{macrocode}
\def\fileversion{1.01}
\def\filedate{2007/03/10}
\message{`PST-Fill' v\fileversion, \filedate\space (tvz,dg,hv)}
\csname PSTboxfillLoaded\endcsname
\let\PSTboxfillLoaded\endinput
%    \end{macrocode}
%
%   Require the main PSTricks package.
%
%    \begin{macrocode}
\ifx\PSTricksLoaded\endinput\else\input pstricks.tex\fi
%    \end{macrocode}
%
%   interface to the extended `\textsf{keyval}' package.
%
%    \begin{macrocode}
\ifx\PSTXKeyLoaded\endinput\else\input pst-xkey\fi
%
%    \end{macrocode}
%
%   Catcodes changes and defining the family name for xkeyval.
%
%    \begin{macrocode}
\edef\PstAtCode{\the\catcode`\@}\catcode`\@=11\relax

\pst@addfams{pst-fill}
%
%    \end{macrocode}
%
%
% \subsection{The size of the box}
% \begin{macro}{pst@@boxfillsize}
%    \begin{macrocode}
%
\def\pst@@boxfillsize#1(#2,#3)#4(#5,#6)#7(#8\@nil{%
  \begingroup
    \ifx\@empty#7\relax
      \pst@dima\z@
      \pst@dimb\z@
      \pssetxlength\pst@dimc{#2}%
      \pssetylength\pst@dimd{#3}%
    \else
      \pssetxlength\pst@dima{#2}%
      \pssetylength\pst@dimb{#3}%
      \pssetxlength\pst@dimc{#5}%
      \pssetylength\pst@dimd{#6}%
    \fi
    \xdef\pst@tempg{%
      \pst@dima=\number\pst@dima sp
      \pst@dimb=\number\pst@dimb sp
      \pst@dimc=\number\pst@dimc sp
      \pst@dimd=\number\pst@dimd sp }%
  \endgroup
  \let\psk@boxfillsize\pst@tempg}
%    \end{macrocode}
% \end{macro}
%

% \subsection{Definition of the parameters}
%
%    \begin{macrocode}
\define@key[psset]{pst-fill}{boxfillsize}{%
  \def\pst@tempg{#1}\def\pst@temph{auto}%
  \ifx\pst@tempg\pst@temph
    \let\psk@boxfillsize\relax
  \else
    \pst@@boxfillsize#1(\z@,\z@)\@empty(\z@,\z@)(\@nil
  \fi}
\psset{boxfillsize={(-15cm,-15cm)(15cm,15cm)}}
\define@key[psset]{pst-fill}{boxfillcolor}{\pst@getcolor{#1}\psboxfillcolor}
\psset{boxfillcolor=black}% hv
\define@key[psset]{pst-fill}{boxfillangle}{\pst@getangle{#1}\psk@boxfillangle}
\psset{boxfillangle=0}
\define@key[psset]{pst-fill}{fillsepx}{%
  \pst@getlength{#1}\psk@fillsepx}
\define@key[psset]{pst-fill}{fillsepy}{%
  \pst@getlength{#1}\psk@fillsepy}
\define@key[psset]{pst-fill}{fillsep}{%
  \pst@getlength{#1}\psk@fillsepx%
  \let\psk@fillsepy\psk@fillsepx}
\psset{fillsep=2pt}

\ifx\PstTiling\@undefined
  \define@key[psset]{pst-fill}{fillcycle}{\pst@getint{#1}\psk@fillcycle}
  \psset{fillcycle=0}
\else
  \define@key[psset]{pst-fill}{fillangle}{\pst@getangle{#1}\psk@boxfillangle}
  \define@key[psset]{pst-fill}{fillsize}{%
      \def\pst@tempg{#1}\def\pst@temph{auto}%
      \ifx\pst@tempg\pst@temph\let\psk@boxfillsize\relax
      \else\pst@@boxfillsize#1(\z@,\z@)\@empty(\z@,\z@)(\@nil\fi}
  \psset{fillsep=0,fillsize=auto}
  \define@key[psset]{pst-fill}{fillcyclex}{\pst@getint{#1}\psk@fillcyclex}
  \define@key[psset]{pst-fill}{fillcycley}{\pst@getint{#1}\psk@fillcycley}
  \define@key[psset]{pst-fill}{fillcycle}{%
    \pst@getint{#1}\psk@fillcyclex\let\psk@fillcycley\psk@fillcyclex}
  \psset{fillcycle=0}
  \define@key[psset]{pst-fill}{fillmovex}{\pst@getlength{#1}\psk@fillmovex}
  \define@key[psset]{pst-fill}{fillmovey}{\pst@getlength{#1}\psk@fillmovey}
  \define@key[psset]{pst-fill}{fillmove}{%
      \pst@getlength{#1}\psk@fillmovex\let\psk@fillmovey\psk@fillmovex}
  \psset{fillmove=0pt}
  \define@key[psset]{pst-fill}{fillloopaddx}{\pst@getint{#1}\psk@fillloopaddx}
  \define@key[psset]{pst-fill}{fillloopaddy}{\pst@getint{#1}\psk@fillloopaddy}
  \define@key[psset]{pst-fill}{fillloopadd}{%
    \pst@getint{#1}\psk@fillloopaddx\let\psk@fillloopaddy\psk@fillloopaddx}
  \psset{fillloopadd=0}
%    \end{macrocode}
%
%    \begin{macrocode}
% For debugging (to debug, set PstDebug=1)
% we now use the one from pstricks to prevent a clash with package
% pstricks                        2004-06-22
%%    \define@key[psset]{pst-fill}{PstDebug}{\pst@getint{#1}\psk@PstDebug}
    \psset{PstDebug=0}
\fi
% DG addition end
%    \end{macrocode}

% \subsection{Definition of the fill box}
% \begin{macro}{psboxfill}
%    \begin{macrocode}
\newbox\pst@fillbox
\def\psboxfill{\pst@killglue\pst@makebox\psboxfill@i}
\def\psboxfill@i{\setbox\pst@fillbox\box\pst@hbox\ignorespaces}
%    \end{macrocode}
% \end{macro}
% \subsection{The main macros}
%
% \begin{macro}{psfs@boxfill}
%    \begin{macrocode}
\def\psfs@boxfill{%
  \ifvoid\pst@fillbox
    \@pstrickserr{Fill box is empty. Use \string\psboxfill\space first.}\@ehpa
  \else
    \ifx\psk@boxfillsize\relax \pst@AutoBoxFill
    \else\pst@ManualBoxFill\fi
  \fi}
%    \end{macrocode}
% \end{macro}
%
% \begin{macro}{pst@ManualBoxFill}
%    \begin{macrocode}
\def\pst@ManualBoxFill{%
  \leavevmode
  \begingroup
    \pst@FlushCode
    \begin@psclip
    \pstVerb{clip}%
    \expandafter\pst@AddFillBox\psk@boxfillsize
    \end@psclip
  \endgroup}
%    \end{macrocode}
% \end{macro}
%
% \begin{macro}{pst@FlushCode}
%    \begin{macrocode}
\def\pst@FlushCode{%
  \pst@Verb{%
    /mtrxc CM def
    CP CP T
    \tx@STV
    \psk@origin
    \psk@swapaxes
    \pst@newpath
    \pst@code
    mtrxc setmatrix
    moveto
    0 setgray}%
  \gdef\pst@code{}}
%    \end{macrocode}
% \end{macro}
%
% \begin{macro}{pst@AddFillBox}
%    \begin{macrocode}
\def\pst@AddFillBox#1 #2 #3 #4 {%
  \begingroup
    \setbox\pst@fillbox=\vbox{%
      \hbox{\unhcopy\pst@fillbox\kern\psk@fillsepx\p@}%
      \vskip\psk@fillsepy\p@}%
    \psk@boxfillsize
    \pst@cnta=\pst@dimc
    \advance\pst@cnta-\pst@dima
    \divide\pst@cnta\wd\pst@fillbox
    \pst@cntb=\pst@dimd
    \advance\pst@cntb-\pst@dimb
    \pst@dimd=\ht\pst@fillbox
    \divide\pst@cntb\pst@dimd
    \def\pst@tempa{%
      \pst@tempg
      \copy\pst@fillbox
      \advance\pst@cntc\@ne
      \ifnum\pst@cntc<\pst@cntd\expandafter\pst@tempa\fi}%
    \let\pst@tempg\relax
    \pst@cntc-\tw@
    \pst@cntd\pst@cnta
    \setbox\pst@fillbox=\hbox to \z@{%
      \kern\pst@dima
      \kern-\wd\pst@fillbox
      \pst@tempa
      \hss}%
    \pst@cntd\pst@cntb
%% DG modification begin - Dec. 11, 1997 - Patch 2
    \ifx\PstTiling\@undefined
      \ifnum\psk@fillcycle=\z@\pst@ManualFillCycle\fi
    \else
      \ifnum\psk@fillcyclex=\z@\pst@ManualFillCycle\fi
    \fi
%% DG modification end
    \global\setbox\pst@boxg=\vbox to\z@{%
      \offinterlineskip
      \vss
      \pst@tempa
      \vskip\pst@dimb}%
  \endgroup
  \setbox\pst@fillbox\box\pst@boxg
  \pst@rotate\psk@boxfillangle\pst@fillbox
  \box\pst@fillbox}
%    \end{macrocode}
% \end{macro}
%
% \begin{macro}{pst@ManualFillCycle}
%    \begin{macrocode}
\def\pst@ManualFillCycle{%
  \ifx\PstTiling\@undefined
    \pst@cntg=\psk@fillcycle
  \else
    \pst@cntg=\psk@fillcyclex
  \fi
  \pst@dimg=\wd\pst@fillbox
  \ifnum\pst@cntg=\z@
  \else
  \divide\pst@dimg\pst@cntg
  \fi
  \ifnum\pst@cntg<\z@\pst@cntg=-\pst@cntg\fi
  \advance\pst@cntg\m@ne
  \pst@cnth=\pst@cntg
  \def\pst@tempg{%
    \ifnum\pst@cnth<\pst@cntg\advance\pst@cnth\@ne\else\pst@cnth\z@\fi
    \moveright\pst@cnth\pst@dimg}}
%    \end{macrocode}
% \end{macro}
%
%% Auto box fill:        !! Fix dictionary
%
% \subsection{The PostScript subroutines}
%
%    \begin{macrocode}
%% DG addition begin - Apr. 8, 1997 and Dec. 1997 - Patch 2
\ifx\PstTiling\@undefined
\pst@def{AutoFillCycle}<%
  /c ED
  /n 0 def
  /s {
    /x x w c div n mul add def
    /n n c abs 1 sub lt { n 1 add } { 0 } ifelse def
  } def>

\pst@def{BoxFill}<%
  gsave
    gsave \tx@STV CM grestore dtransform CM idtransform
    abs /h ED abs /w ED
    pathbbox
    h div round 2 add cvi /y2 ED
    w div round 2 add cvi /x2 ED
    h div round 2 sub cvi /y1 ED
    w div round 2 sub cvi /x1 ED
    /y2 y2 y1 sub def
    /x2 x2 x1 sub def
    CP
    y1 h mul sub neg /y1 ED
    x1 w mul sub neg /x1 ED
    clip
    y2 {
      /x x1 def
      s
      x2 {
        save CP x y1
%% patch 4   hv --------------
        \ifx\VTeXversion\undefined
        \else
%%============ mv: 09-10-01 ??? this is likely to be a right change
        neg
%%============
        \fi
%% end patch 4
T moveto Box restore
        /x x w add def
      } repeat
      /y1 y1 h add def
    } repeat
    % Next line not useful... To see that, suppress clipping (DG)
    CP x y1 T moveto Box
  currentpoint currentfont grestore setfont moveto>
\else
%% DG modification begin - Apr. 8, 1997 and Nov. / Dec. 1997 - Patch 2
\pst@def{AutoFillCycleX}<%
  /cX ED
  /nX 0 def
  /CycleX {
    /x x w cX div nX mul add def
    /nX nX cX abs 1 sub lt { nX 1 add } { 0 } ifelse def
  } def>
\pst@def{AutoFillCycleY}<%
  /cY ED
  /mY 0 def
  /nY 0 def
  /CycleY {
    /y1 y1 h cY div mY mul sub def
    nY cY abs 1 sub lt { /nY nY 1 add def /mY 1 def }
                       { /nY 0 def        /mY cY abs 1 sub neg def } ifelse
  } def>

\pst@def{BoxFill}<%
  gsave
    gsave \tx@STV CM grestore dtransform CM idtransform
    abs /h ED abs /w ED
    pathbbox
    h div round 2 add cvi /y2 ED
    w div round 2 add cvi /x2 ED
    h div round 2 sub cvi /y1 ED
    w div round 2 sub cvi /x1 ED
    /CoefLoopX 0 def
    /CoefLoopY 0 def
    /CoefMoveX 0 def
    /CoefMoveY 0 def
    \psk@boxfillangle\space 0 ne {/CoefLoopX 8 def /CoefLoopY 8 def} if
    \psk@fillcyclex\space 0 ne {/CoefLoopX CoefLoopX 1 add def} if
    \psk@fillcycley\space 0 ne {/CoefLoopY CoefLoopY 1 add def} if
    \psk@fillmovex\space 0 ne
      {/CoefLoopX CoefLoopX 2 add def
       \psk@fillmovex\space 0 gt {/CoefMoveX CoefLoopX def}
                           {/CoefMoveX CoefLoopX neg def} ifelse} if
    \psk@fillmovey\space 0 ne
      {/CoefLoopY CoefLoopY 2 add def
       \psk@fillmovey\space 0 gt {/CoefMoveY CoefLoopY def}
                           {/CoefMoveY CoefLoopY neg def} ifelse} if
    \psk@fillsepx\space 0 ne {/CoefLoopX CoefLoopX 1 add def} if
    \psk@fillsepy\space 0 ne {/CoefLoopY CoefLoopY 1 add def} if
    /CoefLoopX CoefLoopX \psk@fillloopaddx\space add def
    /CoefLoopY CoefLoopY \psk@fillloopaddy\space add def
    /x2 x2 x1 sub 4 sub CoefLoopX 2 mul add def
    /y2 y2 y1 sub 4 sub CoefLoopY 2 mul add def
%% We must fix the origin of tiling, as it must not vary according other stuff
%% in the page!
    w x1 CoefLoopX add CoefMoveX add mul
      h y1 y2 add 1 sub CoefLoopY sub CoefMoveY sub mul moveto
    CP
    y1 h mul sub neg /y1 ED
    x1 w mul sub neg /x1 ED
%%  hv 2004-06-22   to prevent clash with pst-gr3d
%%    \psk@PstDebug 0 eq {clip} if
    \Pst@Debug 0 eq {clip} if
%% end hv
    \psk@fillmovex\space \psk@fillmovey
    gsave \tx@STV CM grestore dtransform CM idtransform
    /hmove ED /wmove ED
    /row 0 def
   y2 {
       /row row 1 add def
       /column 0 def
       /x x1 def
       CycleX
       save
       x2 {
          /column column 1 add def
          CycleY
          save CP x y1
%% patch 4   hv --------------
          \ifx\VTeXversion\undefined
          \else
%%============ mv: 09-10-01 ??? this is likely to be a right change
          neg
%%============
          \fi
  T moveto Box restore
          /x x w add def
          0 hmove translate
          } repeat
       restore
       /y1 y1 h add def
       wmove 0 translate
       } repeat
  currentpoint currentfont grestore setfont moveto>
\fi
%    \end{macrocode}

%    \begin{macrocode}
\def\pst@AutoBoxFill{%
  \leavevmode
  \begingroup
    \pst@stroke
    \pst@FlushCode
    \pst@Verb{\psk@boxfillangle\space \tx@RotBegin}%
    \pstVerb{\pst@dict /Box \pslbrace end}%
    \ifx\PstTiling\@undefined
    \else
      \ifx\pst@tempa\@undefined % Undefined for instance for \pscharpath
      \else\ifx\pst@tempa\@empty\else
        \def\pst@temph{0}%
        \ifx\pst@tempa\pst@temph
        \else
          \pstVerb{/TR {pop pop currentpoint translate \pst@tempa\space translate } def}%
        \fi
      \fi\fi
    \fi
    \hbox to \z@{\vbox to\z@{\vss\copy\pst@fillbox\vskip-\dp\pst@fillbox}\hss}%
    \ifx\PstTiling\@undefined
      \pstVerb{%
        tx@Dict begin \psrbrace def
        \ifnum\psk@fillcycle=\z@
          /s {} def
        \else
          \psk@fillcycle \tx@AutoFillCycle
        \fi
        \pst@number{\wd\pst@fillbox}%
        \psk@fillsepx\space add
        \pst@number{\ht\pst@fillbox}%
        \pst@number{\dp\pst@fillbox}%
        \psk@fillsepy\space add add
        \tx@BoxFill
        end}%
      \else
      \pstVerb{%
        tx@Dict begin \psrbrace def
        \ifnum\psk@fillcyclex=\z@
          /CycleX {} def
        \else
          \psk@fillcyclex\space \tx@AutoFillCycleX
        \fi
        \ifnum\psk@fillcycley=\z@
          /CycleY {} def
        \else
          \psk@fillcycley\space \tx@AutoFillCycleY
        \fi
        \pst@number{\wd\pst@fillbox}%
        \psk@fillsepx\space add
        \pst@number{\ht\pst@fillbox}%
        \pst@number{\dp\pst@fillbox}%
        \psk@fillsepy\space add add
        \tx@BoxFill
        end}%
    \fi
    \pst@Verb{\tx@RotEnd}%
  \endgroup}
%    \end{macrocode}
% \subsection{Closing}
%
%   Catcodes restoration.
%
%    \begin{macrocode}
\catcode`\@=\PstAtCode\relax
%    \end{macrocode}
%
%    \begin{macrocode}
%</pst-fill>
%    \end{macrocode}
%
% \Finale
%
\endinput
%%
%% End of file `pst-fill.dtx'

\ProvidesFile{pst-fill.tex}
  [\filedate\space v\fileversion\space `PST-fill' (tvz,dg)]
%</latex-wrapper>
%    \end{macrocode}
%
%
% \section{Pst-Fill Package{} code}
%
%    \begin{macrocode}
%<*pst-fill>
%    \end{macrocode}
%
% \subsection{Preamble}
%
%   Who we are.
%
%    \begin{macrocode}
\def\fileversion{1.01}
\def\filedate{2007/03/10}
\message{`PST-Fill' v\fileversion, \filedate\space (tvz,dg,hv)}
\csname PSTboxfillLoaded\endcsname
\let\PSTboxfillLoaded\endinput
%    \end{macrocode}
%
%   Require the main PSTricks package.
%
%    \begin{macrocode}
\ifx\PSTricksLoaded\endinput\else\input pstricks.tex\fi
%    \end{macrocode}
%
%   interface to the extended `\textsf{keyval}' package.
%
%    \begin{macrocode}
\ifx\PSTXKeyLoaded\endinput\else\input pst-xkey\fi
%
%    \end{macrocode}
%
%   Catcodes changes and defining the family name for xkeyval.
%
%    \begin{macrocode}
\edef\PstAtCode{\the\catcode`\@}\catcode`\@=11\relax

\pst@addfams{pst-fill}
%
%    \end{macrocode}
%
%
% \subsection{The size of the box}
% \begin{macro}{pst@@boxfillsize}
%    \begin{macrocode}
%
\def\pst@@boxfillsize#1(#2,#3)#4(#5,#6)#7(#8\@nil{%
  \begingroup
    \ifx\@empty#7\relax
      \pst@dima\z@
      \pst@dimb\z@
      \pssetxlength\pst@dimc{#2}%
      \pssetylength\pst@dimd{#3}%
    \else
      \pssetxlength\pst@dima{#2}%
      \pssetylength\pst@dimb{#3}%
      \pssetxlength\pst@dimc{#5}%
      \pssetylength\pst@dimd{#6}%
    \fi
    \xdef\pst@tempg{%
      \pst@dima=\number\pst@dima sp
      \pst@dimb=\number\pst@dimb sp
      \pst@dimc=\number\pst@dimc sp
      \pst@dimd=\number\pst@dimd sp }%
  \endgroup
  \let\psk@boxfillsize\pst@tempg}
%    \end{macrocode}
% \end{macro}
%

% \subsection{Definition of the parameters}
%
%    \begin{macrocode}
\define@key[psset]{pst-fill}{boxfillsize}{%
  \def\pst@tempg{#1}\def\pst@temph{auto}%
  \ifx\pst@tempg\pst@temph
    \let\psk@boxfillsize\relax
  \else
    \pst@@boxfillsize#1(\z@,\z@)\@empty(\z@,\z@)(\@nil
  \fi}
\psset{boxfillsize={(-15cm,-15cm)(15cm,15cm)}}
\define@key[psset]{pst-fill}{boxfillcolor}{\pst@getcolor{#1}\psboxfillcolor}
\psset{boxfillcolor=black}% hv
\define@key[psset]{pst-fill}{boxfillangle}{\pst@getangle{#1}\psk@boxfillangle}
\psset{boxfillangle=0}
\define@key[psset]{pst-fill}{fillsepx}{%
  \pst@getlength{#1}\psk@fillsepx}
\define@key[psset]{pst-fill}{fillsepy}{%
  \pst@getlength{#1}\psk@fillsepy}
\define@key[psset]{pst-fill}{fillsep}{%
  \pst@getlength{#1}\psk@fillsepx%
  \let\psk@fillsepy\psk@fillsepx}
\psset{fillsep=2pt}

\ifx\PstTiling\@undefined
  \define@key[psset]{pst-fill}{fillcycle}{\pst@getint{#1}\psk@fillcycle}
  \psset{fillcycle=0}
\else
  \define@key[psset]{pst-fill}{fillangle}{\pst@getangle{#1}\psk@boxfillangle}
  \define@key[psset]{pst-fill}{fillsize}{%
      \def\pst@tempg{#1}\def\pst@temph{auto}%
      \ifx\pst@tempg\pst@temph\let\psk@boxfillsize\relax
      \else\pst@@boxfillsize#1(\z@,\z@)\@empty(\z@,\z@)(\@nil\fi}
  \psset{fillsep=0,fillsize=auto}
  \define@key[psset]{pst-fill}{fillcyclex}{\pst@getint{#1}\psk@fillcyclex}
  \define@key[psset]{pst-fill}{fillcycley}{\pst@getint{#1}\psk@fillcycley}
  \define@key[psset]{pst-fill}{fillcycle}{%
    \pst@getint{#1}\psk@fillcyclex\let\psk@fillcycley\psk@fillcyclex}
  \psset{fillcycle=0}
  \define@key[psset]{pst-fill}{fillmovex}{\pst@getlength{#1}\psk@fillmovex}
  \define@key[psset]{pst-fill}{fillmovey}{\pst@getlength{#1}\psk@fillmovey}
  \define@key[psset]{pst-fill}{fillmove}{%
      \pst@getlength{#1}\psk@fillmovex\let\psk@fillmovey\psk@fillmovex}
  \psset{fillmove=0pt}
  \define@key[psset]{pst-fill}{fillloopaddx}{\pst@getint{#1}\psk@fillloopaddx}
  \define@key[psset]{pst-fill}{fillloopaddy}{\pst@getint{#1}\psk@fillloopaddy}
  \define@key[psset]{pst-fill}{fillloopadd}{%
    \pst@getint{#1}\psk@fillloopaddx\let\psk@fillloopaddy\psk@fillloopaddx}
  \psset{fillloopadd=0}
%    \end{macrocode}
%
%    \begin{macrocode}
% For debugging (to debug, set PstDebug=1)
% we now use the one from pstricks to prevent a clash with package
% pstricks                        2004-06-22
%%    \define@key[psset]{pst-fill}{PstDebug}{\pst@getint{#1}\psk@PstDebug}
    \psset{PstDebug=0}
\fi
% DG addition end
%    \end{macrocode}

% \subsection{Definition of the fill box}
% \begin{macro}{psboxfill}
%    \begin{macrocode}
\newbox\pst@fillbox
\def\psboxfill{\pst@killglue\pst@makebox\psboxfill@i}
\def\psboxfill@i{\setbox\pst@fillbox\box\pst@hbox\ignorespaces}
%    \end{macrocode}
% \end{macro}
% \subsection{The main macros}
%
% \begin{macro}{psfs@boxfill}
%    \begin{macrocode}
\def\psfs@boxfill{%
  \ifvoid\pst@fillbox
    \@pstrickserr{Fill box is empty. Use \string\psboxfill\space first.}\@ehpa
  \else
    \ifx\psk@boxfillsize\relax \pst@AutoBoxFill
    \else\pst@ManualBoxFill\fi
  \fi}
%    \end{macrocode}
% \end{macro}
%
% \begin{macro}{pst@ManualBoxFill}
%    \begin{macrocode}
\def\pst@ManualBoxFill{%
  \leavevmode
  \begingroup
    \pst@FlushCode
    \begin@psclip
    \pstVerb{clip}%
    \expandafter\pst@AddFillBox\psk@boxfillsize
    \end@psclip
  \endgroup}
%    \end{macrocode}
% \end{macro}
%
% \begin{macro}{pst@FlushCode}
%    \begin{macrocode}
\def\pst@FlushCode{%
  \pst@Verb{%
    /mtrxc CM def
    CP CP T
    \tx@STV
    \psk@origin
    \psk@swapaxes
    \pst@newpath
    \pst@code
    mtrxc setmatrix
    moveto
    0 setgray}%
  \gdef\pst@code{}}
%    \end{macrocode}
% \end{macro}
%
% \begin{macro}{pst@AddFillBox}
%    \begin{macrocode}
\def\pst@AddFillBox#1 #2 #3 #4 {%
  \begingroup
    \setbox\pst@fillbox=\vbox{%
      \hbox{\unhcopy\pst@fillbox\kern\psk@fillsepx\p@}%
      \vskip\psk@fillsepy\p@}%
    \psk@boxfillsize
    \pst@cnta=\pst@dimc
    \advance\pst@cnta-\pst@dima
    \divide\pst@cnta\wd\pst@fillbox
    \pst@cntb=\pst@dimd
    \advance\pst@cntb-\pst@dimb
    \pst@dimd=\ht\pst@fillbox
    \divide\pst@cntb\pst@dimd
    \def\pst@tempa{%
      \pst@tempg
      \copy\pst@fillbox
      \advance\pst@cntc\@ne
      \ifnum\pst@cntc<\pst@cntd\expandafter\pst@tempa\fi}%
    \let\pst@tempg\relax
    \pst@cntc-\tw@
    \pst@cntd\pst@cnta
    \setbox\pst@fillbox=\hbox to \z@{%
      \kern\pst@dima
      \kern-\wd\pst@fillbox
      \pst@tempa
      \hss}%
    \pst@cntd\pst@cntb
%% DG modification begin - Dec. 11, 1997 - Patch 2
    \ifx\PstTiling\@undefined
      \ifnum\psk@fillcycle=\z@\pst@ManualFillCycle\fi
    \else
      \ifnum\psk@fillcyclex=\z@\pst@ManualFillCycle\fi
    \fi
%% DG modification end
    \global\setbox\pst@boxg=\vbox to\z@{%
      \offinterlineskip
      \vss
      \pst@tempa
      \vskip\pst@dimb}%
  \endgroup
  \setbox\pst@fillbox\box\pst@boxg
  \pst@rotate\psk@boxfillangle\pst@fillbox
  \box\pst@fillbox}
%    \end{macrocode}
% \end{macro}
%
% \begin{macro}{pst@ManualFillCycle}
%    \begin{macrocode}
\def\pst@ManualFillCycle{%
  \ifx\PstTiling\@undefined
    \pst@cntg=\psk@fillcycle
  \else
    \pst@cntg=\psk@fillcyclex
  \fi
  \pst@dimg=\wd\pst@fillbox
  \ifnum\pst@cntg=\z@
  \else
  \divide\pst@dimg\pst@cntg
  \fi
  \ifnum\pst@cntg<\z@\pst@cntg=-\pst@cntg\fi
  \advance\pst@cntg\m@ne
  \pst@cnth=\pst@cntg
  \def\pst@tempg{%
    \ifnum\pst@cnth<\pst@cntg\advance\pst@cnth\@ne\else\pst@cnth\z@\fi
    \moveright\pst@cnth\pst@dimg}}
%    \end{macrocode}
% \end{macro}
%
%% Auto box fill:        !! Fix dictionary
%
% \subsection{The PostScript subroutines}
%
%    \begin{macrocode}
%% DG addition begin - Apr. 8, 1997 and Dec. 1997 - Patch 2
\ifx\PstTiling\@undefined
\pst@def{AutoFillCycle}<%
  /c ED
  /n 0 def
  /s {
    /x x w c div n mul add def
    /n n c abs 1 sub lt { n 1 add } { 0 } ifelse def
  } def>

\pst@def{BoxFill}<%
  gsave
    gsave \tx@STV CM grestore dtransform CM idtransform
    abs /h ED abs /w ED
    pathbbox
    h div round 2 add cvi /y2 ED
    w div round 2 add cvi /x2 ED
    h div round 2 sub cvi /y1 ED
    w div round 2 sub cvi /x1 ED
    /y2 y2 y1 sub def
    /x2 x2 x1 sub def
    CP
    y1 h mul sub neg /y1 ED
    x1 w mul sub neg /x1 ED
    clip
    y2 {
      /x x1 def
      s
      x2 {
        save CP x y1
%% patch 4   hv --------------
        \ifx\VTeXversion\undefined
        \else
%%============ mv: 09-10-01 ??? this is likely to be a right change
        neg
%%============
        \fi
%% end patch 4
T moveto Box restore
        /x x w add def
      } repeat
      /y1 y1 h add def
    } repeat
    % Next line not useful... To see that, suppress clipping (DG)
    CP x y1 T moveto Box
  currentpoint currentfont grestore setfont moveto>
\else
%% DG modification begin - Apr. 8, 1997 and Nov. / Dec. 1997 - Patch 2
\pst@def{AutoFillCycleX}<%
  /cX ED
  /nX 0 def
  /CycleX {
    /x x w cX div nX mul add def
    /nX nX cX abs 1 sub lt { nX 1 add } { 0 } ifelse def
  } def>
\pst@def{AutoFillCycleY}<%
  /cY ED
  /mY 0 def
  /nY 0 def
  /CycleY {
    /y1 y1 h cY div mY mul sub def
    nY cY abs 1 sub lt { /nY nY 1 add def /mY 1 def }
                       { /nY 0 def        /mY cY abs 1 sub neg def } ifelse
  } def>

\pst@def{BoxFill}<%
  gsave
    gsave \tx@STV CM grestore dtransform CM idtransform
    abs /h ED abs /w ED
    pathbbox
    h div round 2 add cvi /y2 ED
    w div round 2 add cvi /x2 ED
    h div round 2 sub cvi /y1 ED
    w div round 2 sub cvi /x1 ED
    /CoefLoopX 0 def
    /CoefLoopY 0 def
    /CoefMoveX 0 def
    /CoefMoveY 0 def
    \psk@boxfillangle\space 0 ne {/CoefLoopX 8 def /CoefLoopY 8 def} if
    \psk@fillcyclex\space 0 ne {/CoefLoopX CoefLoopX 1 add def} if
    \psk@fillcycley\space 0 ne {/CoefLoopY CoefLoopY 1 add def} if
    \psk@fillmovex\space 0 ne
      {/CoefLoopX CoefLoopX 2 add def
       \psk@fillmovex\space 0 gt {/CoefMoveX CoefLoopX def}
                           {/CoefMoveX CoefLoopX neg def} ifelse} if
    \psk@fillmovey\space 0 ne
      {/CoefLoopY CoefLoopY 2 add def
       \psk@fillmovey\space 0 gt {/CoefMoveY CoefLoopY def}
                           {/CoefMoveY CoefLoopY neg def} ifelse} if
    \psk@fillsepx\space 0 ne {/CoefLoopX CoefLoopX 1 add def} if
    \psk@fillsepy\space 0 ne {/CoefLoopY CoefLoopY 1 add def} if
    /CoefLoopX CoefLoopX \psk@fillloopaddx\space add def
    /CoefLoopY CoefLoopY \psk@fillloopaddy\space add def
    /x2 x2 x1 sub 4 sub CoefLoopX 2 mul add def
    /y2 y2 y1 sub 4 sub CoefLoopY 2 mul add def
%% We must fix the origin of tiling, as it must not vary according other stuff
%% in the page!
    w x1 CoefLoopX add CoefMoveX add mul
      h y1 y2 add 1 sub CoefLoopY sub CoefMoveY sub mul moveto
    CP
    y1 h mul sub neg /y1 ED
    x1 w mul sub neg /x1 ED
%%  hv 2004-06-22   to prevent clash with pst-gr3d
%%    \psk@PstDebug 0 eq {clip} if
    \Pst@Debug 0 eq {clip} if
%% end hv
    \psk@fillmovex\space \psk@fillmovey
    gsave \tx@STV CM grestore dtransform CM idtransform
    /hmove ED /wmove ED
    /row 0 def
   y2 {
       /row row 1 add def
       /column 0 def
       /x x1 def
       CycleX
       save
       x2 {
          /column column 1 add def
          CycleY
          save CP x y1
%% patch 4   hv --------------
          \ifx\VTeXversion\undefined
          \else
%%============ mv: 09-10-01 ??? this is likely to be a right change
          neg
%%============
          \fi
  T moveto Box restore
          /x x w add def
          0 hmove translate
          } repeat
       restore
       /y1 y1 h add def
       wmove 0 translate
       } repeat
  currentpoint currentfont grestore setfont moveto>
\fi
%    \end{macrocode}

%    \begin{macrocode}
\def\pst@AutoBoxFill{%
  \leavevmode
  \begingroup
    \pst@stroke
    \pst@FlushCode
    \pst@Verb{\psk@boxfillangle\space \tx@RotBegin}%
    \pstVerb{\pst@dict /Box \pslbrace end}%
    \ifx\PstTiling\@undefined
    \else
      \ifx\pst@tempa\@undefined % Undefined for instance for \pscharpath
      \else\ifx\pst@tempa\@empty\else
        \def\pst@temph{0}%
        \ifx\pst@tempa\pst@temph
        \else
          \pstVerb{/TR {pop pop currentpoint translate \pst@tempa\space translate } def}%
        \fi
      \fi\fi
    \fi
    \hbox to \z@{\vbox to\z@{\vss\copy\pst@fillbox\vskip-\dp\pst@fillbox}\hss}%
    \ifx\PstTiling\@undefined
      \pstVerb{%
        tx@Dict begin \psrbrace def
        \ifnum\psk@fillcycle=\z@
          /s {} def
        \else
          \psk@fillcycle \tx@AutoFillCycle
        \fi
        \pst@number{\wd\pst@fillbox}%
        \psk@fillsepx\space add
        \pst@number{\ht\pst@fillbox}%
        \pst@number{\dp\pst@fillbox}%
        \psk@fillsepy\space add add
        \tx@BoxFill
        end}%
      \else
      \pstVerb{%
        tx@Dict begin \psrbrace def
        \ifnum\psk@fillcyclex=\z@
          /CycleX {} def
        \else
          \psk@fillcyclex\space \tx@AutoFillCycleX
        \fi
        \ifnum\psk@fillcycley=\z@
          /CycleY {} def
        \else
          \psk@fillcycley\space \tx@AutoFillCycleY
        \fi
        \pst@number{\wd\pst@fillbox}%
        \psk@fillsepx\space add
        \pst@number{\ht\pst@fillbox}%
        \pst@number{\dp\pst@fillbox}%
        \psk@fillsepy\space add add
        \tx@BoxFill
        end}%
    \fi
    \pst@Verb{\tx@RotEnd}%
  \endgroup}
%    \end{macrocode}
% \subsection{Closing}
%
%   Catcodes restoration.
%
%    \begin{macrocode}
\catcode`\@=\PstAtCode\relax
%    \end{macrocode}
%
%    \begin{macrocode}
%</pst-fill>
%    \end{macrocode}
%
% \Finale
%
\endinput
%%
%% End of file `pst-fill.dtx'
+\newline
%add the following definition:\newline
%\verb+\def\PstTiling{true}+
%
%  To obtain the original behaviour, just don't use the \emph{tiling} optional
%keyword at loading.
%
%  Take care than in \emph{tiling} mode, I introduce also some other changes.
%First I define aliases on some parameter names for consistancy (all specific
%parameters will begin by the \texttt{fill} prefix in this case) and I change
%some default values, which were not well adapted for tilings (\texttt{fillsep}
%is set to 0 and as explained \texttt{fillsize} set to \texttt{auto}). I rename 
%\texttt{fillcycle} to \texttt{fillcyclex}. I also restore normal way so that
%the frame of the area is drawn and all line (\texttt{linestyle},
%\texttt{linecolor}, \texttt{doubleline}, etc.) parameters are now active (but
%there are not in non \emph{tiling} mode). And I also introduce new parameters
%to control the tilings (see below).
%
%  \textbf{In all the following examples, we will consider only the
% \emph{tiling} mode.}
%
%  To do a tiling, we have just to define the pattern with the
% \verb+\psboxfill+ macro and to use the new \texttt{fillstyle}
% \verb+boxfill+.
%
%  Note that tilings are drawn from left to right and top to bottom, which can
%have an importance in some circonstances.
%
%  PostScript programmers can be also interested to know that, even in the
%\emph{automatic} mode, the iterations of the pattern are managed directly by
%the PostScript code of the package which used only PostScript Level 1
%operators. The special ones introduced in Level 2 for drawing of patterns
%\cite[section 4.9]{PostScript95} are not used.
%
%  And first, for conveniance, we define a simple \cs{Tiling} macro, which
%will simplify our examples:
%
%\begin{verbatim}
%  \newcommand{\Tiling}[2][]{%
%    \edef\Temp{#1}%
%    \begin{pspicture}#2
%      \ifx\Temp\empty
%        \psframe[fillstyle=boxfill]#2
%      \else
%        \psframe[fillstyle=boxfill,#1]#2
%      \fi
%    \end{pspicture}}
%\end{verbatim}
%
%
%\newcommand{\Tiling}[2][]{%
%  \edef\Temp{#1}%
%  \begin{pspicture}#2
%    \ifx\Temp\empty
%      \psframe[fillstyle=boxfill]#2
%    \else
%      \psframe[fillstyle=boxfill,#1]#2
%    \fi
% \end{pspicture}}
%
%\subsection{Parameters}
%
%  There are \textbf{14} specific parameters available to change the way the
% filling/tiling is defined, and one debugging option.
%
% \begin{Description}{2cm}
%  \item [fillangle (real)\hfill :] the value of the rotation
%  applied to the patterns (\emph{Default:~0}).
% \end{Description}
%
%
%   In this case, we must force the tiling area to be notably larger than the
% area to cover, to be sure that the defined area will be covered after rotation.
% \lstset{gobble=2}
% \begin{LTXexample}
% \newcommand{\Square}{%
%   \begin{pspicture}(1,1)
%     \psframe[dimen=middle](1,1)
%   \end{pspicture}}
% \psset{unit=0.5}
% \psboxfill{\Square}
% \Tiling[fillangle=45]{(3,3)}\quad
% \Tiling[fillangle=-60]{(3,3)}
% \end{LTXexample}
% 
% \newcommand{\Square}{\begin{pspicture}(1,1)\psframe[dimen=middle](1,1)\end{pspicture}}
% 
% \begin{Description}{2cm}
%   \setcounter{footnote}{1}
%   \item[\texttt{fillsepx} (real$\|$dim) :] value of the horizontal
%   separation between consecutive patterns (\emph{Default:~0 for
%   tilings\footnotemark, 2pt otherwise}).  \footnotetext{This option was added
%   by me, is not part of the original package and is available only if the
%   \texttt{tiling} keyword is used when loading the package.}
%   \setcounter{footnote}{1}
%   \item [\texttt{fillsepy} (real$\|$dim)\hfill :] value of the vertical
%   separation between consecutive patterns (\emph{Default:~0 for
%   ti\-lings\footnotemark, 2pt otherwise}).
%   \setcounter{footnote}{1}
%   \item [\texttt{fillsep} (real$\|$dim)\hfill :] value of horizontal and
%   vertical separations between consecutive patterns (\emph{Default:~0 for
%   tilings\footnotemark, 2pt otherwise}).
% \end{Description}
% 
%   These values can be negative, which allow the tiles to overlap.
% 
% \begin{LTXexample}
% \psset{unit=0.5}
% \psboxfill{\Square}
% \Tiling[fillsepx=2mm]{(3,3)} 
% \Tiling[fillsepy=1mm]{(3,3)}\\
% \Tiling[fillsep=0.5]{(3,3)} 
% \Tiling[fillsep=-0.5]{(3,3)}
% \end{LTXexample}
% 
% \begin{Description}{2cm}
%   \item [\texttt{fillcyclex}\footnotemark\ (integer)\hfill :] Shift
%   coefficient applied to each row (\emph{Default:~0}).
%   \footnotetext{It was \texttt{fillcycle} in the original version.}
%   \setcounter{footnote}{1}
%   \item [\texttt{fillcycley}\footnotemark\ (integer)\hfill :] Same thing for
%   columns (\emph{Default:~0}).
%   \setcounter{footnote}{1}
%   \item [\texttt{fillcycle}\footnotemark\ (integer)\hfill :] Allow to fix
%   both \texttt{fillcyclex} and \texttt{fillcycley} directly to the same value
%   (\emph{Default:~0}).
% \end{Description}
% 
%   For instance, if \texttt{fillcyclex} is 2, the second row of patterns will
% be horizontally shifted by a factor of $\frac{1}{2}=0.5$, and by a factor of
% 0.333 if \texttt{fillcyclex} is 3, etc.). These values can be negative.
% 
% \begin{LTXexample}[width=0.35\linewidth]
% \psset{unit=0.5}
% \psboxfill{\Square}
% \newcommand{\TilingA}[1]{\Tiling[fillcyclex=#1]{(3,3)}}
% \TilingA{0} \TilingA{1}\\
% \TilingA{2} \TilingA{3}\\[3mm]
% \TilingA{4} \TilingA{5}\\
% \TilingA{6} \TilingA{-3}\\[3mm]
% \Tiling[fillcycley=2]{(3,3)}
% \Tiling[fillcycley=3]{(3,3)}\\
% \Tiling[fillcycley=-3]{(3,3)}
% \Tiling[fillcycle=2]{(3,3)}
% \end{LTXexample}
% 
% \begin{Description}{2cm}
%   \setcounter{footnote}{1}
%   \item [\texttt{fillmovex}\footnotemark\ (real$\|$dim)\hfill :] value of the
%   horizontal moves between consecutive patterns (\emph{Default:~0}).
%   \setcounter{footnote}{1}
%   \item [\texttt{fillmovey}\footnotemark\ (real$\|$dim)\hfill :] value of the
%   vertical moves between consecutive patterns (\emph{Default:~0}).
%   \setcounter{footnote}{1}
%   \item [\texttt{fillmove}\footnotemark\ (real$\|$dim)\hfill :] value of
%   horizontal and vertical moves between consecutive patterns
%   (\emph{Default:~0}).
% \end{Description}
% 
%   These parameters allow the patterns to overlap and to draw some special
% kinds of tilings. They are implemented only for the \emph{automatic} and
% \emph{tiling} modes and their values can be negative.
% 
%   In some cases, the effect of these parameters will be the same that with the 
% \texttt{fillcycle?} ones, but you can see that it is not true for some other
% values.
% 
% \begin{LTXexample}
% \psset{unit=0.5}
% \psboxfill{\Square}
% \Tiling[fillmovex=0.5]{(3,3)} 
% \Tiling[fillmovey=0.5]{(3,3)}\\
% \Tiling[fillmove=0.5]{(3,3)}
% \Tiling[fillmove=-0.5]{(3,3)}
% \end{LTXexample}
% 
% \begin{Description}{2cm}
%   \item [\texttt{fillsize}
%   (auto$\|$\{(real$\|$dim,real$\|$dim)(real$\|$dim,real$\|$dim)\}) :] The
%   choice of \emph{automatic} mode or the size of the area in \emph{manual}
%   mode. If first pair values are not given, (0,0) is used. (\emph{Default:
%   auto when \emph{tiling} mode is used, {(-15cm,-15cm)(15cm,15cm)}
%   otherwise}).
% \end{Description}
% 
%   As explained in the introduction, the \emph{manual} mode can require very
% huge amount of computer ressources. So, it usage is to discourage in front off
% the \emph{automatic} mode. It seems only useful in special circonstances, in
% fact when the \emph{automatic} mode failed, which is known only in one case,
% for some kinds of EPS files, as the ones produce by dump of portions of
% screens (see \ref{sec:GraphicFiles}).
% 
% \begin{Description}{2cm}
%   \setcounter{footnote}{1}
%   \item [\texttt{fillloopaddx}\footnotemark\ (integer)\hfill :] number of
%   times the pattern is added on left and right positions (\emph{Default:~0}).
%   \setcounter{footnote}{1}
%   \item [\texttt{fillloopaddy}\footnotemark\ (integer)\hfill :] number of
%   times the pattern is added on top and bottom positions (\emph{Default:~0}).
%   \setcounter{footnote}{1}
%   \item [\texttt{fillloopadd}\footnotemark\ (integer)\hfill :] number of
%   times the pattern is added on left, right, top and bottom positions
%   (\emph{Default:~0}).
% \end{Description}
% 
%   These parameters are only useful in special circonstances, as for complex
% patterns when the size of the rectangular box used to tile the area doesn't 
% correspond to the pattern itself (see an example in Figure~\ref{fig:Sheeps})
% and also sometimes when the size of the pattern is not a divisor of the size
% of the area to fill and that the number of loop repeats is not properly
% computed, which can occur.
% 
%   They are implemented only for the \emph{tiling} mode.
% 
% \begin{Description}{2cm}
%   \setcounter{footnote}{1}
%   \item [\texttt{PstDebug}\footnotemark\ (integer, 0 or 1)\hfill :] to
%   require to see the exact tiling done, without clipping (\emph{Default:~0}).
% \end{Description}
% 
%   It's mainly useful for debugging or to understand better how the tilings
% are done. It is implemented only for the \emph{tiling} mode.
% 
% \begin{LTXexample}
% \psset{unit=0.3,PstDebug=1}
% \psboxfill{\Square}
% \psset{linewidth=1mm}
% \Tiling{(2,2)}\\[5mm]
% \Tiling[fillcyclex=2]{(2,2)}\\[1cm]
% \Tiling[fillmove=0.5]{(2,2)}
% \end{LTXexample}
% 
% \vspace{3cm}
% \section{Examples}
% 
%   In fact this unique \cs{psboxfill} macro allow a lot a variations and
% different usages. We will try here to demonstrate this.
% 
% \subsection{Kind of tiles}
% \label{sec:KindTiles}
% 
%   Of course, we can access to all the power of PSTricks macros to define the
% \emph{tiles} (\emph{patterns}) used. So, we can define complicated ones.
% 
%   Here we give four other Archimedian tilings (those built with only some
% regular polygons) among the twelve existing, first discovered completely by
% Johanes \textsc{Kepler} at the beginning of 17th century \cite{GS87}, the two
% other \emph{regular} ones with the tiling by squares, formed by a unique
% regular polygon, and two other formed by two different regular polygons.
% 
% \begin{LTXexample}[pos=t]
%   \newcommand{\Triangle}{%
%     \begin{pspicture}(1,1)
%       \pstriangle[dimen=middle](0.5,0)(1,1)
%     \end{pspicture}}
%   \newcommand{\Hexagon}{
% ^^A sin(60)=0.866
%     \begin{pspicture}(0.866,0.75)
%       \SpecialCoor
% ^^A  Hexagon  
%       \pspolygon[dimen=middle]%
%         (0.5;30)(0.5;90)(0.5;150)(0.5;210)(0.5;270)(0.5;330)
%     \end{pspicture}}
% 
%   \psset{unit=0.5}
%   \psboxfill{\Triangle}
%   \Tiling{(4,4)}\hfill
% ^^A The two other regular tilings
%   \Tiling[fillcyclex=2]{(4,4)}\hfill
%   \psboxfill{\Hexagon}
%   \Tiling[fillcyclex=2,fillloopaddy=1]{(5,5)}
% \end{LTXexample}
% 
% \begin{LTXexample}[pos=t]
%   \newcommand{\ArchimedianA}{%
%      ^^A Archimedian tiling 3^2.4.3.4
%     \psset{dimen=middle}
%      ^^A sin(60)=0.866
%     \begin{pspicture}(1.866,1.866)
%       \psframe(1,1)
%       \psline(1,0)(1.866,0.5)(1,1)(0.5,1.866)(0,1)(-0.866,0.5)
%       \psline(0,0)(0.5,-0.866)
%     \end{pspicture}}
%   \newcommand{\ArchimedianB}{%
%      ^^A Archimedian tiling 4.8^2
%     \psset{dimen=middle,unit=1.5}
%      ^^A sin(22.5)=0.3827 ; cos(22.5)=0.9239
%     \begin{pspicture}(1.3066,0.6533)
%       \SpecialCoor
%      ^^A Octogon
%       \pspolygon(0.5;22.5)(0.5;67.5)(0.5;112.5)(0.5;157.5)
%                 (0.5;202.5)(0.5;247.5)(0.5;292.5)(0.5;337.5)
%     \end{pspicture}}
% 
%   \psset{unit=0.5}
%   \psboxfill{\ArchimedianA}
%   \Tiling[fillmove=0.5]{(7,7)}\hfill
%   \psboxfill{\ArchimedianB}
%   \Tiling[fillcyclex=2,fillloopaddy=1]{(7,7)}
% \end{LTXexample}
% 
%   \setcounter{footnote}{3}
%   We can of course tile an area arbitrarily defined. And with the
% \texttt{addfillstyle} parameter\footnote{Introduced in PSTricks 97.}, we can
% easily mix the \texttt{boxfill} style with another one.
% 
% \begin{LTXexample}[width=6cm]
%   \psset{unit=0.5,dimen=middle}
%   \psboxfill{%
%     \begin{pspicture}(1,1)
%       \psframe(1,1)
%       \pscircle(0.5,0.5){0.25}
%     \end{pspicture}}
%   \begin{pspicture}(4,6)
%     \pspolygon[fillstyle=boxfill,fillsep=0.25](0,1)(1,4)(4,6)(4,0)(2,1)
%   \end{pspicture}\hspace{1em}
%   \begin{pspicture}(4,4)
%%     \pscircle[linestyle=none,fillstyle=solid,fillcolor=yellow,fillsep=0.5,
%%               addfillstyle=boxfill](2,2){2}
%   \end{pspicture}
% \end{LTXexample}
%
%   Various effects can be obtained, sometimes complicated ones very easily, as
% in this example reproduced from one shown by Slavik \textsc{Jablan} in the
% field of \emph{OpTiles}, inspired by the \emph{Op-art}:
% 
% \begin{LTXexample}[pos=t]
% \newcommand{\ProtoTile}{%
%  \begin{pspicture}(1,1)%%% 1/12=0.08333
%   \psset{linestyle=none,linewidth=0,
%     hatchwidth=0.08333\psunit,hatchsep=0.08333\psunit}
%   \psframe[fillstyle=solid,fillcolor=black,addfillstyle=hlines,hatchcolor=white](1,1)
%   \pswedge[fillstyle=solid,fillcolor=white,addfillstyle=hlines]{1}{0}{90}
%  \end{pspicture}}
% \newcommand{\BasicTile}{%
%  \begin{pspicture}(2,1)
%    \rput[lb](0,0){\ProtoTile}\rput[lb](1,0){\psrotateleft{\ProtoTile}}
%  \end{pspicture}}
% \ProtoTile\hfill\BasicTile\hfill
% \psboxfill{\BasicTile}
% \Tiling[fillcyclex=2]{(4,4)}
% \end{LTXexample}
% 
%   It is also directly possible to surimpose several different tilings. Here is
% the splendid visual proof of the \textsc{Pytha\-gore} theorem done by the arab
% mathematician \textsc{Annairizi} around the year 900, given by superposition
% of two tilings by squares of different sizes.
% 
% \begin{LTXexample}[pos=t]
% \psset{unit=1.5,dimen=middle}
% \begin{pspicture*}(3,3)
%   \psboxfill{\begin{pspicture}(1,1)
%     \psframe(1,1)\end{pspicture}}
%   \psframe[fillstyle=boxfill](3,3)
%   \psboxfill{\begin{pspicture}(1,1)
%     \rput{-37}{\psframe[linecolor=red](0.8,0.8)}
%   \end{pspicture}}
%   \psframe[fillstyle=boxfill](3,4)
%   \pspolygon[fillstyle=hlines,hatchangle=90](1,2)(1.64,1.53)(2,2)
% \end{pspicture*}
% \end{LTXexample}
% 
%   In a same way, it is possible to build tilings based on figurative patterns,
% in the style of the famous \textsc{Escher} ones. Following an example of
% Andr\'e \textsc{Deledicq} \cite{Deledicq97}, we first show a simple tiling of
% the \emph{p1} category (according to the international classification of the
% 17~symmetry groups of the plane first discovered by the russian
% crystalographer Jevgraf \textsc{Fedorov} at the end of the 19th century):
% 
% \begin{LTXexample}[pos=t]
%  \newcommand{\SheepHead}[1]{%
%    \begin{pspicture}(3,1.5)
%      \pscustom[liftpen=2,fillstyle=solid,fillcolor=#1]{%
%        \pscurve(0.5,-0.2)(0.6,0.5)(0.2,1.3)(0,1.5)(0,1.5)
%          (0.4,1.3)(0.8,1.5)(2.2,1.9)(3,1.5)(3,1.5)(3.2,1.3)
%          (3.6,0.5)(3.4,-0.3)(3,0)(2.2,0.4)(0.5,-0.2)}
%      \pscircle*(2.65,1.25){0.12\psunit} % Eye
%      \psccurve*(3.5,0.3)(3.35,0.45)(3.5,0.6)(3.6,0.4)% Muzzle
%     ^^A   % Mouth
%       \pscurve(3,0.35)(3.3,0.1)(3.6,0.05)
%     ^^A   % Ear
%       \pscurve(2.3,1.3)(2.1,1.5)(2.15,1.7)\pscurve(2.1,1.7)(2.35,1.6)(2.45,1.4)
%   \end{pspicture}}
%  \psboxfill{\psset{unit=0.5}\SheepHead{yellow}\SheepHead{cyan}}
%  \Tiling[fillcyclex=2,fillloopadd=1]{(10,5)}
% \end{LTXexample}
% \label{fig:Sheeps}
% 
%   Now a tiling of the \emph{pg} category (the code for the kangaroo itself is
% too long to be shown here, but has no difficulties ; the kangaroo is reproduce
% from an original picture from Raoul \textsc{Raba} and here is a translation in
% PSTricks from the one drawn by Emmanuel \textsc{Chailloux} and Guy
% \textsc{Cousineau} for their MLgraph system \cite{MLgraphTSI}):
% 
% \begin{LTXexample}[pos=t]
% \psboxfill{\psset{unit=0.4}
%   \Kangaroo{yellow}\Kangaroo{red}\Kangaroo{cyan}\Kangaroo{green}%
%   \psscalebox{-1 1}{%
%     \rput(1.235,4.8){\Kangaroo{green}\Kangaroo{cyan}\Kangaroo{red}\Kangaroo{yellow}}}}
%   \Tiling[fillloopadd=1]{(10,6)}
% \end{LTXexample}
% 
%   And here a \textsc{Wang} tiling \cite{Wang65}, \cite[chapter
% 11]{GS87}, based on very simple tiles of the form of a square and composed
% of four colored triangles. Such tilings are built with only a matching color
% constraint. Despite of it simplicity, it is an important kind of tilings, as
% \textsc{Wang} and others used them to study the special class of
% \emph{aperiodic} tilings, and also because it was shown that surprisingly this 
% tiling is similar to a \textsc{Turing} machine.
% 
% \begin{LTXexample}[pos=t]
%   \newcommand{\WangTile}[4]{%
%     \begin{pspicture}(1,1)
%       \pspolygon*[linecolor=#1](0,0)(0,1)(0.5,0.5)
%       \pspolygon*[linecolor=#2](0,1)(1,1)(0.5,0.5)
%       \pspolygon*[linecolor=#3](1,1)(1,0)(0.5,0.5)
%       \pspolygon*[linecolor=#4](1,0)(0,0)(0.5,0.5)
%     \end{pspicture}}
%   \newcommand{\WangTileA}{\WangTile{cyan}{yellow}{cyan}{cyan}}
%   \newcommand{\WangTileB}{\WangTile{yellow}{cyan}{cyan}{red}}
%   \newcommand{\WangTileC}{\WangTile{cyan}{red}{yellow}{yellow}}
%   \newcommand{\WangTiles}[1][]{%
%     \begin{pspicture}(3,3) \psset{ref=lb}
%       \rput(0,2){\WangTileB}  \rput(1,2){\WangTileA}%
%       \rput(2,2){\WangTileC}  \rput(0,1){\WangTileC}%
%       \rput(1,1){\WangTileB}  \rput(2,1){\WangTileA}
%       \rput(0,0){\WangTileA}  \rput(1,0){\WangTileC}%
%       \rput(2,0){\WangTileB}
%       #1
%     \end{pspicture}}
%   \WangTileA\hfill\WangTileB\hfill\WangTileC\hfill
%   \WangTiles[{\psgrid[subgriddiv=0,gridlabels=0](3,3)}]\hfill
%   \psset{unit=0.4} \psboxfill{\WangTiles} \Tiling{(12,12)}
% \end{LTXexample}
% 
% \subsection{External graphic files}
% \label{sec:GraphicFiles}
% 
%   We can also fill an arbitrary area with an external image. We have only, 
% as usual, to matter of the \emph{BoundingBox} definition if there is no one
% provided or if it is not the accurate one, as for the well known
% \texttt{tiger} picture part of the \texttt{ghostscript} distribution.
% 
% \begin{LTXexample}[pos=t]
%   \psboxfill{%% Strangely require x1=x2...
%     \begin{pspicture}(0,1)(0,4.1)
%       \includegraphics[bb=17 176 560 74,width=3cm]{tiger}
%     \end{pspicture}}
%   \Tiling{(6,6.2)}
% \end{LTXexample}
% 
%   Nevertheless, there are some special files for which the \emph{automatic}
% mode doesn't work, specially for some files obtained by a screen dump, as in
% the next example, where a picture was reduced before it conversion in the
% \emph{Encapsulated PostScript} format by a screen dump utility. In this case,
% usage of the \emph{manual} mode is the only alternative, at the price of the
% real multiple inclusion of the EPS file. We must take care to specify the
% correct \texttt{fillsize} parameter, because otherwise the default values are
% large and will load the file many times, perhaps just really using few
% occurrences as the other ones would be clipped...
% 
% \begin{LTXexample}[pos=t]
%   \psboxfill{\includegraphics{flowers}}
%   \begin{pspicture}(8,4)
%     \psellipse[fillstyle=boxfill,fillsize={(8,4)}](4,2)(4,2)
%   \end{pspicture}
% \end{LTXexample}
% 
% \subsection{Tiling of characters}
% 
%   We can also use the \cs{psboxfill} macro to fill the interior of characters
% for special effects like these ones:
% 
% \begin{LTXexample}[pos=t]
%   \DeclareFixedFont{\bigsf}{T1}{phv}{b}{n}{4.5cm}
%   \DeclareFixedFont{\smallrm}{T1}{ptm}{m}{n}{3mm}
%   \psboxfill{\smallrm Since 182 days...}
%   \begin{pspicture*}(8,4)
%     \centerline{%
%       \pscharpath[fillstyle=gradient,gradangle=-45,
%                   gradmidpoint=0.5,addfillstyle=boxfill,
%                   fillangle=45,fillsep=0.7mm]
%                  {\rput[b](0,0.1){\bigsf 2000}}}
%   \end{pspicture*}
% \end{LTXexample}
% 
% \begin{LTXexample}[pos=t]
%   \DeclareFixedFont{\mediumrm}{T1}{ptm}{m}{n}{2cm}
%   \psboxfill{%
%     \psset{unit=0.1,linewidth=0.2pt}
%     \Kangaroo{PeachPuff}\Kangaroo{PaleGreen}%
%       \Kangaroo{LightBlue}\Kangaroo{LemonChiffon}%
%     \psscalebox{-1 1}{%
%       \rput(1.235,4.8){%
%         \Kangaroo{LemonChiffon}\Kangaroo{LightBlue}%
%           \Kangaroo{PaleGreen}\Kangaroo{PeachPuff}}}}
% ^^A   % A kangaroo of kangaroos...
%   \begin{pspicture}(8,2)
%     \pscharpath[linestyle=none,fillstyle=boxfill,fillloopadd=1]
%                {\rput[b](4,0){\mediumrm Kangaroo}}
%   \end{pspicture}
% \end{LTXexample}
% 
% \subsection{Other kinds of usage}
% 
%   Other kinds of usage can be imagined. For instance, we can use tilings in a
% sort of degenerated way to draw some special lines made by a unique or
% multiple repeating patterns. But it can be only a special dashed line, as here
% with three different dashes:
% 
% \begin{LTXexample}[pos=t]
%   \newcommand{\Dashes}{%
%     \psset{dimen=middle}
%     \begin{pspicture}(0,-0.5\pslinewidth)(1,0.5\pslinewidth)
%       \rput(0,0){\psline(0.4,0)}%
%         \rput(0.5,0){\psline(0.2,0)}%
%         \rput(0.8,0){\psline(0.1,0)}
%     \end{pspicture}}
% 
%   \newcommand{\SpecialDashedLine}[3]{%
%     \psboxfill{#3}
%     \Tiling[linestyle=none]
%            {(#1,-0.5\pslinewidth)(#2,0.5\pslinewidth)}}
% 
%   \SpecialDashedLine{0}{7}{\Dashes}
% 
%   \psset{unit=0.5,linewidth=1mm,linecolor=red}
%   \SpecialDashedLine{0}{10}{\Dashes}
% \end{LTXexample}
% 
%   It allow also to use special patterns in business graphics, as in the
% following example generated by \texttt{PstChart}\footnote{A personal
% development to draw business charts with PSTricks, not distributed.}.
% 
% \vspace{3mm}
% \begin{figure}[!ht]
% \centering
% \psset{unit=0.75}
% ^^A % Generated by pstchart.sh version 0.21 (11/28/97)
% {\psset{dimen=middle}
% \psset{xunit=2,yunit=0.005}
% \begin{pspicture}(-0.6,-200)(6.6,2300)
% ^^A   % Title
%   \rput(3,2200){\shortstack{Fantaisist repartition of kangaroos\\
%                             in the world (in thousands)}}
% ^^A   % Frame background
%   \psframe[fillstyle=solid,fillcolor=LemonChiffon](0,0)(6,2000)
% ^^A   % Graduations
%   \multido{\n=0+500}{5}{\rput[r](-0.12,\n){\psscalebox{0.8}{\n}}}
% ^^A   % Minor ticks
%   \multips(0,100)(0,100){19}{\psline[unit=4.8pt](1,0)}
%   \multips(6,100)(0,100){19}{\psline[unit=4.8pt](-1,0)}
% ^^A   % Major ticks
%   \multips(0,500)(0,500){3}{\psline[unit=9.6pt](1,0)}
%   \multips(6,500)(0,500){3}{\psline[unit=9.6pt](-1,0)}
% ^^A   % Lines from major ticks marks
%   \multips(0,500)(0,500){3}{\psline[linestyle=dotted,linewidth=0.6pt](6,0)}
% ^^A   % Drawing for the data
%   \psboxfill{\psset{unit=0.78\psxunit}\KangarooPstChart{red}}
%   \psframe[linestyle=none,fillstyle=boxfill,fillloopaddy=1](0.61,0)(1.39,1800)
%   \psboxfill{\psset{unit=0.78\psxunit}\KangarooPstChart{yellow}}
%   \psframe[linestyle=none,fillstyle=boxfill,fillloopaddy=1](1.61,0)(2.39,800)
%   \psboxfill{\psset{unit=0.78\psxunit}\KangarooPstChart{cyan}}
%   \psframe[linestyle=none,fillstyle=boxfill,fillloopaddy=1](2.61,0)(3.39,550)
%   \psboxfill{\psset{unit=0.78\psxunit}\KangarooPstChart{magenta}}
%   \psframe[linestyle=none,fillstyle=boxfill,fillloopaddy=1](3.61,0)(4.39,500)
%   \psboxfill{\psset{unit=0.78\psxunit}\KangarooPstChart{green}}
%   \psframe[linestyle=none,fillstyle=boxfill,fillloopaddy=1](4.61,0)(5.39,200)
% ^^A   % Bottom labels
%   \uput{0.2}[270]{0}(1,0){\psscalebox{0.7}{Oceania}}
%   \uput{0.2}[270]{0}(2,0){\psscalebox{0.7}{Africa}}
%   \uput{0.2}[270]{0}(3,0){\psscalebox{0.7}{Asia}}
%   \uput{0.2}[270]{0}(4,0){\psscalebox{0.7}{America}}
%   \uput{0.2}[270]{0}(5,0){\psscalebox{0.7}{Europe}}
% ^^A   % Frame box around the chart
%   \psframe[linestyle=solid](0,0)(6,2000)
% \end{pspicture}}
%   \caption{Bar chart generated by PstChart, with bars filled by patterns}
%   \label{fig:PstChart}
% \end{figure}
% 
% \section{``Dynamic'' tilings}
% 
%   In some cases, tilings used non \emph{static} tiles, that is to say that the 
% \emph{prototile(s)}, even if unique, can have several forms, by instance
% specified by different colors or rotations, not fixed before generation or
% varying each time.
% 
% \subsection{Lewthwaite-Pickover-Truchet tiling}
% 
%   We give here for example the so-called \emph{Truchet} tiling, which much be
% in fact better called \emph{Lewthwaite-Pick\-over-Truchet (LPT)} tiling%
% \footnote{For description of the context, history and references about
% S\'ebastien \textsc{Truchet} and this tiling, see \cite{EsperetGirou98}.}.
% 
%   The unique prototile is only a square with two opposite circle arcs.
% This tile has obviously two positions, if we rotate it from 90 degrees (see
% the two tiles on the next figure). A \emph{LPT tiling} is a tiling with
% randomly oriented LPT tiles. We can see that even if it is very simple in it
% principle, it draw sophisticated curves with strange properties.
% 
%   Nevertheless, in the straightforward way \FillPackage{} does not work,
% because the \cs{psboxfill} macro store the content of the tile used in a
% \TeX{} box, which is static. So the calling to the random function is done
% only one time, which explain that only one rotation of the tile is used for
% all the tiling. It's only the one of the two rotations which could differ from
% one drawing to the next one...
% 
% ^^A % Truchet (Lewthwaite-Pickover-Truchet) tiling
% ^^A % --------------------------------------------
% 
% \begin{LTXexample}[pos=t]
% ^^A   % LPT prototile
%   \newcommand{\ProtoTileLPT}{%
%     \psset{dimen=middle}
%     \begin{pspicture}(1,1)
%       \psframe(1,1)
%       \psarc(0,0){0.5}{0}{90}
%       \psarc(1,1){0.5}{-180}{-90}
%     \end{pspicture}}
% 
% ^^A   % LPT tile
%   \newcount\Boolean
%   \newcommand{\BasicTileLPT}{%
% ^^A     % From random.tex by Donald Arseneau
%     \setrannum{\Boolean}{0}{1}%
%     \ifnum\Boolean=0
%       \ProtoTileLPT%
%     \else
%       \psrotateleft{\ProtoTileLPT}%
%     \fi}
% 
%   \ProtoTileLPT\hfill\psrotateleft{\ProtoTileLPT}\hfill
%   \psset{unit=0.5}
%   \psboxfill{\BasicTileLPT}
%   \Tiling{(5,5)}
% \end{LTXexample}
% 
%   But, for simple cases, there is a solution to this problem using a mixture
% of PSTricks and PostScript programming. Here the PSTricks
% construction \verb+\pscustom{\code{...}}+ allow to insert PostScript code
% inside the \LaTeX{} + PSTricks one.
% 
%   Programmation is less straightforward, but it has also the advantage to be
% notably faster, as all the tilings operations are done in PostScript, and
% mainly to not be limited by \TeX{} memory (the \TeX{} + PSTricks solution
% I wrote in 1995 for the colored problem was limited to small sizes for this
% reason). Just note also that \cs{pslbrace} and \cs{psrbrace} are two
% PSTricks macros to define and be able to insert the \verb+{+ and \verb+}+
% characters.
% 
% \begin{LTXexample}[pos=t]
% ^^A   % LPT prototile
%   \newcommand{\ProtoTileLPT}{%
%     \psset{dimen=middle}
%     \psframe(1,1)
%     \psarc(0,0){0.5}{0}{90}
%     \psarc(1,1){0.5}{-180}{-90}}
% 
% ^^A   % Counter to change the random seed
%   \newcount\InitCounter
% ^^A   % LPT tile
%   \newcommand{\BasicTileLPT}{%
%     \InitCounter=\the\time
%     \pscustom{\code{%
%       rand \the\InitCounter\space sub 2 mod 0 eq \pslbrace}}
%     \begin{pspicture}(1,1)
%       \ProtoTileLPT
%     \end{pspicture}%
%     \pscustom{\code{\psrbrace \pslbrace}}
%     \psrotateleft{\ProtoTileLPT}%
%     \pscustom{\code{\psrbrace ifelse}}}
% 
%   \psset{unit=0.4,linewidth=0.4pt}
%   \psboxfill{\BasicTileLPT}
%   \Tiling{(15,15)}
% \end{LTXexample}
% 
%   Using the very surprising fact (see \cite{EsperetGirou98}) that
% coloration of these tiles do not depend of their neighbors (even if it is
% difficult to believe as the opposite seems obvious!) but only of the parity of
% the value of row and column positions, we can directly program in the same way
% a colored version of the LPT tiling.
% 
% \setcounter{footnote}{1}
%   We have also introduce in the \FillPackage{} code for \emph{tiling} mode two
% new accessible Post\-Script variables, \texttt{row} and
% \texttt{column}\footnotemark, which can be useful in some circonstances, like
% this one.
% 
% \begin{LTXexample}[pos=t]
% ^^A   % LPT prototile
%   \newcommand{\ProtoTileLPT}[2]{%
%     \psset{dimen=middle,linestyle=none,fillstyle=solid}
%     \psframe[fillcolor=#1](1,1)
%     \psset{fillcolor=#2}
%     \pswedge(0,0){0.5}{0}{90} \pswedge(1,1){0.5}{-180}{-90}}
% ^^A   % Counter to change the random seed
%   \newcount\InitCounter
% ^^A   % LPT tile
%   \newcommand{\BasicTileLPT}[2]{%
%     \InitCounter=\the\time
%     \pscustom{\code{%
%       rand \the\InitCounter\space sub 2 mod 0 eq \pslbrace
%       row column add 2 mod 0 eq \pslbrace}}
%     \begin{pspicture}(1,1)\ProtoTileLPT{#1}{#2}\end{pspicture}%
%     \pscustom{\code{\psrbrace \pslbrace}}
%     \ProtoTileLPT{#2}{#1}%
%     \pscustom{\code{%
%       \psrbrace ifelse \psrbrace \pslbrace row column add 2 mod 0 eq \pslbrace}}
%     \psrotateleft{\ProtoTileLPT{#2}{#1}}\pscustom{\code{\psrbrace \pslbrace}}
%     \psrotateleft{\ProtoTileLPT{#1}{#2}}\pscustom{\code{\psrbrace ifelse \psrbrace ifelse}}}
%   \psboxfill{\BasicTileLPT{red}{yellow}}
%   \Tiling{(4,4)}\hfill
%   \psset{unit=0.4}\psboxfill{\BasicTileLPT{blue}{cyan}}
%   \Tiling{(15,15)}
% \end{LTXexample}
% 
%   Another classic example is to generate coordinates and numerotation for a
% grid. Of course, it is possible to do it directly in PSTricks using nested
% \cs{multido} commands. It would be clearly easy to program, but, nevertheless, 
% for users who have a little knowledge of PostScript programming, this offer
% an alternative which is useful for large cases, because on this way it will
% be notably faster and less computer ressources consuming.
% 
%   Remember here that the tiling is drawn from left to right, and top to
% bottom, and note that the PostScript variable \texttt{x2} give the total
% number of columns.
% 
% \begin{LTXexample}[pos=t]
% ^^A   % \Escape will be the \ character
%   {\catcode`\!=0\catcode`\\=11!gdef!Escape{\}}
%   \newcommand{\ProtoTile}{%
%     \Square\pscustom{%
%       \moveto(-0.9,0.75) % In PSTricks units
%       \code{ /Times-Italic findfont 8 scalefont setfont
%         (\Escape() show row 3 string cvs show (,) show 
%         column 3 string cvs show (\Escape)) show}
%       \moveto(-0.5,0.25) % In PSTricks units
%       \code{ /Times-Bold findfont 18 scalefont setfont
%         1 0 0 setrgbcolor % Red color
%         /center {dup stringwidth pop 2 div neg 0 rmoveto} def
%         row 1 sub x2 mul column add 3 string cvs center show}}}
%   \psboxfill{\ProtoTile}
%   \Tiling{(6,4)}
% \end{LTXexample}
% 
% \subsection{A complete example: the Poisson equation}
% 
%   To finish, we will show a complete real example, a drawing to explain the
% method used to solve the \textsc{Poisson} equation by a domain
% decomposition method, adapted to distributed memory computers. The
% objective is to show the communications required between processes and the
% position of the data to exchange. This code also show some useful and powerful
% technics for PSTricks programming (look specially at the way some higher level
% macros are defined, and how the same object is used to draw the four
% neighbors).
%
%\psset{unit=1cm}
%\newcommand{\Pattern}[1]{%
%   \begin{pspicture}(-0.25,-0.25)(0.25,0.25)\rput{*0}{\psdot[dotstyle=#1]}
%   \end{pspicture}}
%\newcommand{\West}{\Pattern{o}}   \newcommand{\South}{\Pattern{x}}
%\newcommand{\Central}{\Pattern{+}}\newcommand{\North}{\Pattern{square}}
%\newcommand{\East}{\Pattern{triangle}}
%\newcommand{\Cross}{%
%  \pspolygon[unit=0.5,linewidth=0.2,linecolor=red](0,0)(0,1)(1,1)(1,2)(2,2)(2,1)%
%              (3,1)(3,0)(2,0)(2,-1)(1,-1)(1,0)}
%\newcommand{\StylePosition}[1]{\LARGE\textcolor{red}{\textbf{#1}}}
%\newcommand{\SubDomain}[4]{%
%    \psboxfill{#4}\begin{psclip}{\psframe[linestyle=none]#1}%
%      \psframe[linestyle=#3](5,5)\psframe[fillstyle=boxfill]#2%
%    \end{psclip}}
%\newcommand{\SendArea}[1]{\psframe[fillstyle=solid,fillcolor=cyan]#1}
%\newcommand{\ReceiveData}[2]{%
%  \psboxfill{#2}\psframe[fillstyle=solid,fillcolor=yellow,addfillstyle=boxfill]#1}%
%\newcommand{\Neighbor}[2]{%
%    \begin{pspicture}(5,5)
%      \rput{*0}(2.5,2.5){\StylePosition{#1}}
%      \ReceiveData{(0.5,0)(4.5,0.5)}{\Central}\SendArea{(0.5,0.5)(4.5,1)}%
%      \SubDomain{(5,2)}{(0.5,0.5)(4.5,3)}{dashed}{#2}%
%      \pcarc[arcangle=45,arrows=->](0.5,-1.25)(0.5,0.25)%
%      \pcarc[arcangle=45,arrows=->,linestyle=dotted,dotsep=2pt](4.5,0.75)(4.5,-0.75)%
%    \end{pspicture}}%
%  \psset{dimen=middle,dotscale=2,fillloopadd=2}
%\begin{pspicture}(-5.7,-5.7)(5.7,5.7)
%  \rput(0,0){%
%      \begin{pspicture}(5,5)
%        \ReceiveData{(0,0.5)(0.5,4.5)}{\West} \ReceiveData{(4.5,0.5)(5,4.5)}{\East}
%        \ReceiveData{(0.5,4.5)(4.5,5)}{\North}\ReceiveData{(0.5,0)(4.5,0.5)}{\South}
%        \SendArea{(0.5,0.5)(1,4.5)}\SendArea{(4,0.5)(4.5,4.5)}
%        \SendArea{(0.5,0.5)(4.5,1)}\SendArea{(0.5,4)(4.5,4.5)}
%        \SubDomain{(5,5)}{(0.5,0.5)(4.5,4.5)}{solid}{\Central}
%        \psline(1,0.5)(1,4.5)\psline(4,0.5)(4,4.5)%
%        \rput(1.5,4){\Cross}\rput(2,2){\Cross}%
%      \end{pspicture}}%
%  \rput(0,5.5){\Neighbor{N}{\North}}\rput{-90}(5.5,0){\Neighbor{E}{\East}}%
%  \rput{90}(-5.5,0){\Neighbor{W}{\West}}\rput{180}(0,-5.5){\Neighbor{S}{\South}}%
%\end{pspicture}
%
% \begin{lstlisting}
%   \newcommand{\Pattern}[1]{%
%     \begin{pspicture}(-0.25,-0.25)(0.25,0.25)\rput{*0}{\psdot[dotstyle=#1]}
%     \end{pspicture}}
%   \newcommand{\West}{\Pattern{o}}   \newcommand{\South}{\Pattern{x}}
%   \newcommand{\Central}{\Pattern{+}}\newcommand{\North}{\Pattern{square}}
%   \newcommand{\East}{\Pattern{triangle}}
%   \newcommand{\Cross}{%
%     \pspolygon[unit=0.5,linewidth=0.2,linecolor=red](0,0)(0,1)(1,1)(1,2)(2,2)(2,1)
%               (3,1)(3,0)(2,0)(2,-1)(1,-1)(1,0)}
%   \newcommand{\StylePosition}[1]{\LARGE\textcolor{red}{\textbf{#1}}}
%   \newcommand{\SubDomain}[4]{%
%     \psboxfill{#4}
%     \begin{psclip}{\psframe[linestyle=none]#1}
%       \psframe[linestyle=#3](5,5)\psframe[fillstyle=boxfill]#2
%     \end{psclip}}
%   \newcommand{\SendArea}[1]{\psframe[fillstyle=solid,fillcolor=cyan]#1}
%   \newcommand{\ReceiveData}[2]{%
%     \psboxfill{#2}
%     \psframe[fillstyle=solid,fillcolor=yellow,addfillstyle=boxfill]#1}
%   \newcommand{\Neighbor}[2]{%
%     \begin{pspicture}(5,5)
%       \rput{*0}(2.5,2.5){\StylePosition{#1}}
%       \ReceiveData{(0.5,0)(4.5,0.5)}{\Central}\SendArea{(0.5,0.5)(4.5,1)}
%       \SubDomain{(5,2)}{(0.5,0.5)(4.5,3)}{dashed}{#2}%
% ^^A       % Receive and send arrows
%       \pcarc[arcangle=45,arrows=->](0.5,-1.25)(0.5,0.25)
%       \pcarc[arcangle=45,arrows=->,linestyle=dotted,dotsep=2pt](4.5,0.75)(4.5,-0.75)
%     \end{pspicture}}
%   \psset{dimen=middle,dotscale=2,fillloopadd=2}
%   \begin{pspicture}(-5.7,-5.7)(5.7,5.7)
% ^^A     % Central domain
%     \rput(0,0){%
%       \begin{pspicture}(5,5)
% ^^A         % Receive from West, East, North and South
%         \ReceiveData{(0,0.5)(0.5,4.5)}{\West} \ReceiveData{(4.5,0.5)(5,4.5)}{\East}
%         \ReceiveData{(0.5,4.5)(4.5,5)}{\North}\ReceiveData{(0.5,0)(4.5,0.5)}{\South}
% ^^A         % send area for West, East, North and South
%         \SendArea{(0.5,0.5)(1,4.5)} \SendArea{(4,0.5)(4.5,4.5)}
%         \SendArea{(0.5,0.5)(4.5,1)} \SendArea{(0.5,4)(4.5,4.5)}
% ^^A         % Central domain
%         \SubDomain{(5,5)}{(0.5,0.5)(4.5,4.5)}{solid}{\Central}
% ^^A         % Redraw overlapped linesY
%         \psline(1,0.5)(1,4.5)  \psline(4,0.5)(4,4.5)
% ^^A         % Two crossesY
%         \rput(1.5,4){\Cross}  \rput(2,2){\Cross}
%       \end{pspicture}}
% ^^A     % The four neighborsY
%     \rput(0,5.5){\Neighbor{N}{\North}}     \rput{-90}(5.5,0){\Neighbor{E}{\East}}
%     \rput{90}(-5.5,0){\Neighbor{W}{\West}} \rput{180}(0,-5.5){\Neighbor{S}{\South}}
%   \end{pspicture}
% \end{lstlisting}
%
%
%
% Bibliography
% \begin{thebibliography}{99}
% \bibitem{PostScript95} Adobe, Systems~Incorporated, \emph{PostScript Language
% Reference Manual}, Addison-Wesley, 2~edition, 1995.
%
% \bibitem{Bolek98} Piotr Bolek, \MP{} and patterns, \emph{\TUB}, Volume~19,
% Number~3, pages 276--283, September 1998, \CTANref{mpattern}.
%
% \bibitem{MLgraphTSI} Emmanuel Chailloux, Guy Cousineau and Asc\'ander
% Su\'arez, Programmation fonctionnelle de graphismes pour la production
% d'illustrations techniques, \emph{Technique et science informatique},
% Volume~15, Number~7, pages 977--1007, 1996 (in french).
%
% \bibitem{Deledicq97} Andr\'e Deledicq, \emph{Le monde des pavages}, ACL
% \'Editions, 1997 (in french).
%
% \bibitem{EsperetGirou98} Philippe Esperet and Denis Girou,
% Coloriage du pavage dit de Truchet, Cahiers GUTenberg, Number~31,
% pages 5--18, December~1998  (in french).
%
% \bibitem{Girou94} Denis Girou, Pr\'esentation de PSTricks, \emph{Cahiers
% GUTenberg}, Number~16, pages 21--70, February~1994 (in french).
%
% \bibitem{LGC97} Michel Goossens, Sebastian Rahtz and Frank Mittelbach,
% \emph{The \LaTeX{} Graphics Companion}, Addison-Wesley, 2005.
%
% \bibitem{GS87} Branko Gr\"unbaum and Geoffrey Shephard, \emph{Tilings and
% Patterns}, Freeman and Company, 1987.
%
% \bibitem{Hoenig97} Alan Hoenig, \emph{\TeX{} Unbound: \LaTeX{} \& \TeX{}
% Strategies, Fonts, Graphics, and More}, Oxford University Press, 1997.
%
% \bibitem{XYpic} Kristoffer~H. Rose and Ross Moore, \XYpic. Pattern and Tile
% extension, available from \CTAN, 1991-1998, \CTANref{xypic}.
%
% \bibitem{LAAN96} Kees van der Laan, Paradigms: Just a little bit of PostScript,
% \emph{MAPS}, Volume~17, pages 137--150, 1996.
%
% \bibitem{LAAN97} Kees van der Laan, Tiling in PostScript and \MF{} -- Escher's
% wink, \emph{MAPS}, Volume~19, Number~2, pages 39--67, 1997.
%
% \bibitem{vanZandt93} Timothy Van Zandt, PSTricks. PostScript macros for
% Generic \TeX, available from \CTAN, 1993, \CTANref{pstricks}.
%
% \bibitem{vanZandtGirou94} Timothy Van Zandt and Denis Girou, Inside PSTricks,
% \emph{\TUB}, Volume~15, Number~3, pages 239--246, September 1994.
%
%
% \bibitem{voss07} Herbert Vo\ss, PSTricks -- Graphics for \TeX\ and \LaTeX, DANTE/Lehmanns, 4th ed., 2007.
% \bibitem{Wang65} Hao Wang, Games, Logic and Computers, \emph{Scientific
% American}, pages 98--106, November 1965.
% \end{thebibliography}
%
%
% \StopEventually{}
%
% ^^A .................... End of the documentation part ....................
%
% \section{Driver file}
%
%   The next bit of code contains the documentation driver file for \TeX{},
% i.e., the file that will produce the documentation you are currently
% reading. It will be extracted from this file by the \texttt{docstrip}
% program.
%
%    \begin{macrocode}
%<*driver>
\documentclass{ltxdoc}
\GetFileInfo{pst-fill.dtx}
%
\usepackage[T1]{fontenc}
\usepackage{lmodern}               % For PDF
\usepackage{graphicx}              % `graphicx' LaTeX standard package
\usepackage{showexpl}
\usepackage{mflogo}                % For the MetaFont and MetaPost logos
\input{random.tex}                 % Random macros from Donald Arseneau
\usepackage{url}                   % URLs convenient typesetting
\usepackage{multido}               % General loop macro
\usepackage[dvipsnames]{pstricks}  % PSTricks with the `color' extension
\usepackage{pst-text}              % PSTricks package for character path
\usepackage{pst-grad}              % PSTricks package for gradient filling
\usepackage{pst-node}              % PSTricks package for nodes
\usepackage[tiling]{pst-fill}      % PSTricks package for filling/tiling
%
\AtBeginDocument{%
%  \OnlyDescription % comment out for implementation details
  \EnableCrossrefs
  \CodelineIndex
  \RecordChanges}
\AtEndDocument{%
  \PrintIndex
  \setcounter{IndexColumns}{1}
  \PrintChanges}
\hbadness=7000            % Over and under full box warnings
\hfuzz=3pt
\begin{document}
  \DocInput{pst-fill.dtx}
\end{document}
%</driver>
%    \end{macrocode}
%
% \section{\texttt{pst-fill} \LaTeX{} wrapper}
%
%    \begin{macrocode}
%<*latex-wrapper>
\RequirePackage{pstricks}
\ProvidesPackage{pst-fill}[2005/09/13 package wrapper for 
  pst-fill.tex (hv)]
\DeclareOption{tiling}{\def\PstTiling{true}}
\ProcessOptions\relax
% \iffalse meta-comment, etc.
%%
%% Package `pst-fill.dtx'
%%
%% Denis Girou (CNRS/IDRIS - France) <Denis.Girou@idris.fr>
%% Herbert Voss <voss@pstricks.de>
%%
%% This program can be redistributed and/or modified under the terms
%% of the LaTeX Project Public License Distributed from CTAN archives
%% in directory macros/latex/base/lppl.txt.
%%
%% DESCRIPTION:
%%   `pst-fill' is a PSTricks package for filling and tiling areas 
%%
% \fi
% \changes{v1.01}{2007/03/10}{bugfix for incomplete ifx (hv)}
% \changes{v1.00}{2006/11/06}{use pst-xkey for extend keys (hv)}
% \changes{v0.99}{2004/08/17}{merge the VTeX and TeX versions (patch 4) (hv)}
% \changes{v0.98}{2004/06/22}{delete the Pst@Debug option and use the
%   the one from pstricks to prevent a clash with pst-gr3d (hv)}
% \changes{v0.97}{2001/10/09}{make it work with VTeX (mv)}
% \changes{v0.94}{1997/04/08}{With a \PstTiling macro defined (or "tiling" optional parameter
%   on \textbackslash usepackage[tiling]{pst-fill}), this file run exactly as
%   the original boxfill.tex file from Timothy, version 0.94,
%   except a correction in \textbackslash pst@ManualFillCycle to avoid a division by 0.
%   It's the default.}
% \changes{v0.93}{1997/04/07}{With a \textbackslash PstTiling macro defined (or "tiling" optional parameter
%   on \textbackslash usepackage[tiling]{pst-fill}) there are several add-ons
%   and changes to do `tiling' rather than `filling' in "automatic" mode :
%     - we fix the position of the beginning of tiling,
%     - we allow normally the framing of the area as expected, using
%       the line.... parameters
%     - we define move parameters fillmovex, fillmovey and fillmove,
%     - we define fillcyclex as previous fillcycle parameter, and add the
%       fillcycley and fillcycle (both fillcyclex and fillcycley) ones
%     - we can extend the tiling area using fillloopaddx, fillloopaddy and
%       fillloopadd parameters,
%     - we can debug and see the whole tiling area without clipping using
%       PstDebug parameter,
%     - for names consistancy, we can use fillangle in place of boxfillangle
%       and fillsize in place of boxfillsize,
%     - default value for fillsep is 0 and for fillsize is auto.}
%
% \DoNotIndex{\!,\",\#,\$,\%,\&,\',\(,\+,\*,\,,\-,\.,\/,\:,\;,\<,\=,\>,\?}
% \DoNotIndex{\@,\@B,\@K,\@cTq,\@f,\@fPl,\@ifnextchar,\@nameuse,\@oVk}
% \DoNotIndex{\[,\\,\],\^,\_,\ }
% \DoNotIndex{\^,\\^,\\\^,$\^$,$\\^$,$\\^$}
% \DoNotIndex{\0,\2,\4,\5,\6,\7,\8,}
% \DoNotIndex{\A,\a}
% \DoNotIndex{\B,\b,\Bc,\begin,\Bq,\Bqc}
% \DoNotIndex{\C,\c,\catcode,\cJA,\CodelineIndex,\csname}
% \DoNotIndex{\D,\def,\define@key,\Df,\divide,\DocInput,\documentclass,\pst@addfams}
% \DoNotIndex{\eCN,\edef,\else,\eHd,\eMcj,\EnableCrossrefs,\end,\endcsname}
% \DoNotIndex{\endCenterExample,\endExample,\endinput,\endpsclip}
% \DoNotIndex{\PrintIndex,\PrintChanges,\ProvidesFile}
% \DoNotIndex{\endpspicture,\endSideBySideExample,\Example}
% \DoNotIndex{\F,\f,\FdUrr,\fi,\filedate,\fileversion,\FV@Environment}
% \DoNotIndex{\FV@UseKeyValues,\FV@XRightMargin,\FVB@Example,\fvset}
% \DoNotIndex{\G,\g,\GetFileInfo,\gr,\GradientLoaded,\gsFKrbK@o,\gsj,\gsOX}
% \DoNotIndex{\hbadness,\hfuzz,\HLEmphasize,\HLMacro,\HLMacro@i}
% \DoNotIndex{\HLReverse,\HLReverse@i,\hqcu,\HqY}
% \DoNotIndex{\I,\i,\ifx,\input,\Ir,\IU}
% \DoNotIndex{\j,\jl,\JT,\JVodH}
% \DoNotIndex{\K,\k,\kfSlL}
% \DoNotIndex{\L,\let}
% \DoNotIndex{\message,\mHNa,\mIU}
% \DoNotIndex{\N,\nB,\newcmykcolor,\newdimen,\newif,\nW}
% \DoNotIndex{\O,\oCDJDo,\ocQhVI,\OnlyDescription,\oRKJ}
% \DoNotIndex{\P,\p,\ProvidesPackage,\psframe,\pslinewidth,\psset}
% \DoNotIndex{\PstAtCode,\PSTricksLoaded}
% \DoNotIndex{\q,\Qr,\qssRXq,\qu,\qXjFQp,\qYL}
% \DoNotIndex{\R,\r,\RecordChanges,\relax,\RlaYI,\rN,\Rp,\rp,\RPDXNn,\rput}
% \DoNotIndex{\S,\scalebox,\SgY,\SideBySide@Example,\SideBySideExample}
% \DoNotIndex{\SgY,\sk,\Sp,\space,\sZb}
% \DoNotIndex{\T,\the,\tw@}
% \DoNotIndex{\u,\UiSWGEf@,\uJi,\usepackage,\uVQdMM,\UYj}
% \DoNotIndex{\VerbatimEnvironment,\VerbatimInput,\VrC@}
% \DoNotIndex{\WhZ,\WjKCYb,\WNs}
% \DoNotIndex{\XkN,\XW}
% \DoNotIndex{\Z,\ZCM,\Ze}
% \DoNotIndex{\addtocounter,\advance,\alph,\arabic,\AtBeginDocument,\AtEndDocument}
% \DoNotIndex{\AtEndOfPackage,\begingroup,\bfseries,\bgroup,\box,\csname}
% \DoNotIndex{\else,\endcsname,\endgroup,\endinput,\expandafter,\fi}
% \DoNotIndex{\TeX,\z@,\p@,\@one,\xdef,\thr@@,\string,\sixt@@n,\reset,\or,\multiply,\repeat,\RequirePackage}
% \DoNotIndex{\@cclvi,\@ne,\@ehpa,\@nil,\copy,\dp,\global,\hbox,\hss,\ht,\ifodd,\ifdim,\ifcase,\kern}
% \DoNotIndex{\chardef,\loop,\leavevmode,\ifnum,\lower}
% \setcounter{IndexColumns}{2}
%
% ^^A To extend the height used for the text
%
% ^^A  Aligned labels in a description environment
%\newenvironment{Description}[1]{%
%\begin{list}{nothing}{\setlength{\leftmargin}{#1}
%\setlength{\labelwidth}{\leftmargin}\setlength{\labelsep}{1mm}}}
%{\end{list}}
%
% ^^A For macro names
%\DeclareRobustCommand\cs[1]{\texttt{\char`\\#1}}
%
%
% ^^A From ltugboat.cls
% ^^A For references
%\makeatletter
%\newcommand\acro[1]{\textsc{#1}\@}
%\def\CTAN{\acro{CTAN}}
%\let\texttub\textsl              % ^^A redefined in other situations
%\def\TUB{\texttub{TUGboat}}
%\def\TUG{\TeX\ \UG}
%\def\tug{\acro{TUG}}
%\def\UG{Users Group}
% ^^A For the bibliography 
%\let\@internalcite\cite
%\def\cite{\def\@citeseppen{-1000}%
%    \def\@cite##1##2{(##1\if@tempswa , ##2\fi)}%
%    \def\citeauthoryear##1##2##3{##1, ##3}\@internalcite}
%\def\etal{et\,al.\@}
%\newcommand\CTANdirectory[1]{\expandafter\urldef
%  \csname CTAN@#1\endcsname\path}
%\newcommand\CTANfile[1]{\expandafter\urldef
%  \csname CTAN@#1\endcsname\path}
%\newcommand\CTANref[1]{\expandafter\@setref\csname CTAN@#1\endcsname
%  \relax{#1}}
%\makeatother
% ^^A Define CTAN addresses 
%\CTANdirectory{mpattern}{graphics/metapost/macros/mpattern}
%\CTANdirectory{pstricks}{graphics/pstricks}
%\CTANdirectory{pst-fill.sty}{graphics/pstricks/latex/pst-fill.sty}
%\CTANdirectory{pst-fill}{graphics/pstricks/generic/pst-fill.tex}
%\CTANdirectory{Roegel}{graphics/metapost/contrib/macros/truchet}
%\CTANdirectory{xypic}{macros/generic/diagrams/xypic}
%
% ^^A Personal macros (D.G.)
% ^^A ----------------------
%
% ^^A Some colors used
%\definecolor{LemonChiffon}{rgb}{1.,0.98,0.8}
%\definecolor{LightBlue}   {rgb}{0.8,0.85,0.95}
%\definecolor{PaleGreen}   {rgb}{0.88,1,0.88}
%\definecolor{PeachPuff}   {rgb}{1.0,0.85,0.73}
%
% ^^A To define a unique string for TeX and LaTeX
%\newcommand{\AllTeX}{%
%{\rm(L\kern-.36em\raise.3ex\hbox{\sc a}\kern-.15em)%
%T\kern-.1667em\lower.7ex\hbox{E}\kern-.125emX}}
%
% ^^A Bibliography style
%\bibliographystyle{ltugbib}
%
% ^^A Name macros
%\newcommand{\FillPackage}{\textsf{`pst-fill'}}
%\newcommand{\XYpic}{%
%\leavevmode\hbox{\kern-.1em X\kern-.3em\lower.4ex\hbox{Y\kern-.15em}-pic}}
%
%\makeatletter
%
% ^^A Example environments
% ^^A (do not use in them the four JXYZ characters, that we will use
% ^^A as escape characters!)
%
% ^^A Default PSTricks parameters
%  \psset{dimen=middle}
%
% ^^A Translation in PSTricks from the one drawn by Emmanuel Chailloux and
% ^^A Guy Cousineau for the MLgraph system
% ^^A (see /ftp.ens.fr:/pub/unix/lang/MLgraph/version-2.1/MLgraph-refman.ps.gz)
% ^^A The kangaroo itself is reproduce from an original picture from Raoul Raba
% \newcommand{\DimX}{2.47}
% \newcommand{\DimY}{4.8}
% \newcommand{\DimXDivTwo}{1.235}
%
% \newcommand{\KangarooItself}[1]{%
% ^^A Body
% \pspolygon[fillstyle=solid,fillcolor=#1]%
%  (52.5,68)(55,72.5)(55.8,76.5)(56.8,79.8)(58.2,83)(60,85.8)(61.5,86.5)
% (64,87)(66,87.5)(67.8,87.3)(70,87)(71.5,87.3)(73,88)(74.7,88.5)
% (76,90.3)(77,91.5)(72.8,93.8)(69,96)(64.5,99)(59.4,103)(56.2,106.3)
% (53,110.5)(49.5,115.5)(47.2,119.9)(45.7,126)(43.2,123)(41.5,121)(37.5,125)
% (37,122.5)(36.8,120)(37,117)(37.6,113.5)(38.6,110)(40,106.3)(42,102.3)
%  (43.5,99.5)(45,97)(46.2,94)(46.8,91.7)(47.2,88)(47,83.5)(46.3,80.8)
%  (45.3,78.5)(42.5,76.5)(39.5,75.8)(36,75.9)(33,75.9)(29,76.2)(26,77)
%  (22.3,77.5)(18,78.4)(12.8,79.3)(8.6,80)(5.5,80.3)(3,80.5)(0,80)
%  (-5.2,78.5)(-9,76.3)(-11.2,74.8)(-13,72.5)(-16.5,68)(-16.5,68)(-19.5,62.5)
%  (-22,58)(-25.5,53)(-29,48.5)(-32.5,45)(-36,42)(-39,39.5)(-44,37)
%  (-49,35)(-51,34)(-53.5,34.5)(-55.5,36)(-56.5,38)(-56.5,40.5)(-55,41.5)
%  (-53.5,41)(-51.5,41)(-50.5,43)(-50.5,44.5)(-51,47)(-51.5,47.2)(-56.5,47)
%  (-58.5,46.5)(-60,44.7)(-62,42.3)(-63,39.5)(-63.5,36.3)(-63.5,33)(-63.1,29.5)
%  (-61.5,26)(-58,23.6)(-54,22.2)(-50.7,22)(-47.5,22)(-44.5,22.3)(-41,23.5)
%  (-36.8,25.8)(-33,28)(-28.5,31)(-23.4,35)(-20.2,38.3)(-17,42.5)(-13.5,47.5)
%  (-11.2,51.9)(-9.7,58)(-7.2,55)(-5.5,53)(-1.5,57)(-1,54.5)(-0.8,52)
%  (-1,49)(-1.6,45.5)(-2.6,42)(-4,38.3)(-6,34.3)(-7.5,31.5)(-9,29)
%  (-10.2,26)(-10.8,23.7)(-11.2,20)(-11,15.5)(-10.3,12.8)(-9.3,10.5)(-6.5,8.5)
%  (-3.5,7.8)(0,7.9)(3,7.9)(7,8.2)(10,9)(13.7,9.5)(18,10.4)
%  (23.2,11.3)(27.4,12)(30.5,12.3)(33,12.5)(36,12)(41.2,10.5)(45,8.3)
%  (47.2,6.8)(49,4.5)(52.5,0)(50,4.5)(49.2,8.5)(48.2,11.8)(46.8,15)
%  (45,17.8)(43.5,18.5)(41,19)(39,19.5)(37.2,19.3)(35,19)(33.5,19.3)
%  (32,20)(30.3,20.5)(29,22.3)(28,23.5)(28,23.5)(24.5,22.3)(21.5,22)
%  (18.3,22)(15,22.2)(11,23.6)(7.5,26)(5.9,29.5)(5.5,33)(5.5,36.3)
%  (6,39.5)(7,42.3)(9,44.7)(10.5,46.5)(12.5,47)(17.5,47.2)(18,47)
%  (18.5,44.5)(18.5,43)(17.5,41)(15.5,41)(14,41.5)(12.5,40.5)(12.5,38)
%  (13.5,36)(15.5,34.5)(18,34)(20,35)(25,37)(30,39.5)(33,42)
%  (36.5,45)(40,48.5)(43.5,53)(47,58)(49.5,62.5)(52.5,68)
% ^^A Eye
% \pscircle*[linecolor=white](58.2,98.3){2\psxunit}
% \pscircle*(58.2,97.3){\psxunit}
% ^^A Mouth
% \psline(71.5,88)(70,89.3)(68.5,90.3)(67,91.9)
% ^^A Tear
% \psline(42,121)(45,118)(47,115.3)(48.5,112.7)(50,110)(51.8,106.5)
%       (52.5,103.7)(53,100.5)
% \pspolygon(41.2,115.8)(43.2,114.7)(45,112.5)(47,109.8)(48,107)(49.5,104.2)%
%       (50.5,101.6)(51,98.5)(47.7,100.6)(46,102.2)(44.8,104)(43.5,106)
%       (42.5,108)(41.7,110.5)(41,113.2)}
%
% \newcommand{\Kangaroo}[1]{%
%   \begin{pspicture}(\DimX,\DimY)
%   \psset{unit=0.035278}
%   \KangarooItself{#1}
%   \end{pspicture}}
%
% \newcommand{\KangarooPstChart}[1]{{%
%   \psset{xunit=0.006784,yunit=0.00735,linewidth=0.01}
%   \begin{pspicture}(-65.5,0)(82,126)
%     \KangarooItself{#1}
%   \end{pspicture}}}
%
%
% ^^A For the possible index and changes log
% \setlength{\columnseprule}{0.6pt}
%
% ^^A Beginning of the documentation itself
%\title{\texttt{pst-fill}\\A PSTricks package for filling and tiling areas}
%\author{Timothy Van Zandt\thanks{\protect\url{tvz@econ.insead.fr}. (documentation by
% Denis Girou (\protect\url{Denis.Girou@idris.fr}) and Herbert Vo\ss (\protect\url{hvoss@tug.org}).}}
%
%\date{\shortstack{\today --- Version 1.00\\
%                  {\small Documentation revised \today}}}
% \maketitle
% \tableofcontents
%
%\begin{abstract}
%  \FillPackage{} is a PSTricks \cite{vanZandt93},\cite{Girou94},\cite{vanZandtGirou94}, 
%\cite{Hoenig97},\cite{LGC97} package to draw easily
%  various kinds of filling and tiling of areas. It is also a good example of
%  the great power and flexibility of PSTricks, as in fact it is a very short
%  program (it body is around 200~lines long) but nevertheless really powerful.
%
%  \hspace{5mm} It was written in 1994 by Timothy \textsc{van Zandt} but
%  publicly available only in PSTricks 97 and without any documentation.
%  We describe here the version \emph{97 patch 2} of December 12, 1997, which
%  is the original one modified by myself to manage \emph{tilings} in the
%  so-called \emph{automatic} mode. This article would like to serve both of
%  reference manual and of user's guide.
%
%This package is available on \CTAN{} in the
%  \texttt{graphics/pstricks} directory (files \texttt{latex/pst-fill.sty} and
%  \texttt{generic/pst-fill.tex}).
%\end{abstract}
%
%\section{Introduction}
%
%  Here we will refer as \emph{filling} as the operation which consist to fill
%a defined area by a pattern (or a composition of patterns). We will refer as
%\emph{tiling} as the operation which consist to do the same thing, but with
%the control of the starting point, which is here the upper left corner.
%The pattern is positioned relatively to this point. This make an essential
%difference between the two modes, as without control of the starting point we
%can't draw \emph{tilings} (sometimes  called \emph{tesselations}) as used in
%many fields of Art and Science%
%\footnote{For an extensive presentation of tilings, in their history and usage
%in many fields, see the reference book \cite{GS87}.
%
%  In the \TeX{} world, few work was done on tilings. You can look at the
%\emph{tile} extension of the \XYpic{} package \cite{XYpic}, at the articles of
%Kees \textsc{van der Laan} \cite[paragraph 7]{LAAN96} (the tiling was in
%fact directly done in PostScript) and \cite{LAAN97}, at the \MP{} program
%(available on \CTANref{Roegel}) by Denis \textsc{Roegel} for the
%\textsc{Truchet} contest in 1995 \cite{EsperetGirou98} and at the \MP{}
%package \cite{Bolek98} to draw patterns, which have a strong connection with
%tilings.}.
%
%  Nevertheless, as tilings are a wide and difficult field in mathematics, this
%package is limited to simple ones, mainly \emph{monohedral} tilings with one
%prototile (which can be composite, see section \ref{sec:KindTiles}). With some
%experience and wiliness we can do more and obtained easily rather
%sophisticated results, but obviously hyperbolic tilings like the famous
%\textsc{Escher} ones or aperiodic tilings like the \textsc{Penrose} ones are
%not in the capabilities of this package. For more complex needs, we must used
%low level and more painfull technics, with the basic \cs{multido}
%and \cs{multirput} macros.
%
%\section{Package history and description of it two different modes}
%
%  As already said, this package was written in 1994 by Timothy \textsc{van
%Zandt}. Two modes were defined, called respectively \emph{manual} and
%\emph{automatic}. For both, the pattern is generated on contiguous positions in
%a rather large area which include the region to fill, later cut to the
%required dimensions by clipping mechanism. In the first mode, the pattern is
%explicitely inserted in the PostScript file each time. In the second one, the
%result is the same but with an unique explicit insertion of the pattern and a
%repetition done by PostScript. Nevertheless, in this method, the control of
%the starting point was loosed, so it allowed only to \emph{fill} a region and
%not to \emph{tile} it.
%
%  See the difference between the two modes, \emph{tiling}:
% {\psset{unit=0.5cm}%
% \psboxfill{\begin{pspicture}(1,1)\psframe[dimen=middle](1,1)\end{pspicture}}
% \begin{pspicture}(3,3.3)
%   \psframe[fillstyle=boxfill](3,3)
% \end{pspicture}}
% and \emph{filling}:
%{%
% \makeatletter
%\pst@def{BoxFill}<
%  gsave
%    gsave \tx@STV CM grestore dtransform CM idtransform
%    abs /h ED abs /w ED
%    pathbbox
%    h div round 2 add cvi /y2 ED
%    w div round 2 add cvi /x2 ED
%    h div round 2 sub cvi /y1 ED
%    w div round 2 sub cvi /x1 ED
%    /y2 y2 y1 sub def
%    /x2 x2 x1 sub def
%    CP
%    y1 h mul sub neg /y1 ED
%    x1 w mul sub neg /x1 ED
%    clip
%    y2 {
%      /x x1 def
%      x2 {
%        save CP x y1 T moveto Box restore
%        /x x w add def
%      } repeat
%      /y1 y1 h add def
%    } repeat
% currentpoint currentfont grestore setfont moveto>
% \makeatother
%
% \psset{unit=0.5}
% \psboxfill{\begin{pspicture}(1,1)\psframe[dimen=middle](1,1)\end{pspicture}}
% \begin{pspicture}(3,3.3)
%   \psframe[fillstyle=boxfill](3,3)
% \end{pspicture}
% or
% \begin{pspicture}(3,3.3)
%   \psframe[fillstyle=boxfill](3,3)
% \end{pspicture}
%}
%as we can see that initial position is arbitrary and dependent of
%the current point.
%
%
% It's clear that usage of filling is very restrictive comparing to tiling,
%as desired effects required very often the possibility to control the starting 
%point. So, this mode was of limited interest, but unfortunately the
%\emph{manual} one has the very big disadvantage to require very huge amounts
%of ressources, mainly in disk space and consequently in printing time.
%A small tiling can require sometimes several megabytes in \emph{manual} mode!
%So, it was very often not really usable in practice.
%
%It is why I modified the code, to allow tilings in \emph{automatic} mode,
%controlling in this mode too the starting point. And most of the time, that is
%to say if some special options are not used, the tiling is done exactly in the
%region described, which make it faster. So there is no more reason to use the
%\emph{manual} mode, apart very special cases where \emph{automatic} one cannot
%work, as explained later -- currently, I know only one case.
%
%  To load this modified \emph{automatic} mode, with \LaTeX{} use
%simply:\newline 
%\verb+\usepackage[tiling]{pst-fill}+\newline
%and in plain \TeX{} after:\newline
%\verb+% \iffalse meta-comment, etc.
%%
%% Package `pst-fill.dtx'
%%
%% Denis Girou (CNRS/IDRIS - France) <Denis.Girou@idris.fr>
%% Herbert Voss <voss@pstricks.de>
%%
%% This program can be redistributed and/or modified under the terms
%% of the LaTeX Project Public License Distributed from CTAN archives
%% in directory macros/latex/base/lppl.txt.
%%
%% DESCRIPTION:
%%   `pst-fill' is a PSTricks package for filling and tiling areas 
%%
% \fi
% \changes{v1.01}{2007/03/10}{bugfix for incomplete ifx (hv)}
% \changes{v1.00}{2006/11/06}{use pst-xkey for extend keys (hv)}
% \changes{v0.99}{2004/08/17}{merge the VTeX and TeX versions (patch 4) (hv)}
% \changes{v0.98}{2004/06/22}{delete the Pst@Debug option and use the
%   the one from pstricks to prevent a clash with pst-gr3d (hv)}
% \changes{v0.97}{2001/10/09}{make it work with VTeX (mv)}
% \changes{v0.94}{1997/04/08}{With a \PstTiling macro defined (or "tiling" optional parameter
%   on \textbackslash usepackage[tiling]{pst-fill}), this file run exactly as
%   the original boxfill.tex file from Timothy, version 0.94,
%   except a correction in \textbackslash pst@ManualFillCycle to avoid a division by 0.
%   It's the default.}
% \changes{v0.93}{1997/04/07}{With a \textbackslash PstTiling macro defined (or "tiling" optional parameter
%   on \textbackslash usepackage[tiling]{pst-fill}) there are several add-ons
%   and changes to do `tiling' rather than `filling' in "automatic" mode :
%     - we fix the position of the beginning of tiling,
%     - we allow normally the framing of the area as expected, using
%       the line.... parameters
%     - we define move parameters fillmovex, fillmovey and fillmove,
%     - we define fillcyclex as previous fillcycle parameter, and add the
%       fillcycley and fillcycle (both fillcyclex and fillcycley) ones
%     - we can extend the tiling area using fillloopaddx, fillloopaddy and
%       fillloopadd parameters,
%     - we can debug and see the whole tiling area without clipping using
%       PstDebug parameter,
%     - for names consistancy, we can use fillangle in place of boxfillangle
%       and fillsize in place of boxfillsize,
%     - default value for fillsep is 0 and for fillsize is auto.}
%
% \DoNotIndex{\!,\",\#,\$,\%,\&,\',\(,\+,\*,\,,\-,\.,\/,\:,\;,\<,\=,\>,\?}
% \DoNotIndex{\@,\@B,\@K,\@cTq,\@f,\@fPl,\@ifnextchar,\@nameuse,\@oVk}
% \DoNotIndex{\[,\\,\],\^,\_,\ }
% \DoNotIndex{\^,\\^,\\\^,$\^$,$\\^$,$\\^$}
% \DoNotIndex{\0,\2,\4,\5,\6,\7,\8,}
% \DoNotIndex{\A,\a}
% \DoNotIndex{\B,\b,\Bc,\begin,\Bq,\Bqc}
% \DoNotIndex{\C,\c,\catcode,\cJA,\CodelineIndex,\csname}
% \DoNotIndex{\D,\def,\define@key,\Df,\divide,\DocInput,\documentclass,\pst@addfams}
% \DoNotIndex{\eCN,\edef,\else,\eHd,\eMcj,\EnableCrossrefs,\end,\endcsname}
% \DoNotIndex{\endCenterExample,\endExample,\endinput,\endpsclip}
% \DoNotIndex{\PrintIndex,\PrintChanges,\ProvidesFile}
% \DoNotIndex{\endpspicture,\endSideBySideExample,\Example}
% \DoNotIndex{\F,\f,\FdUrr,\fi,\filedate,\fileversion,\FV@Environment}
% \DoNotIndex{\FV@UseKeyValues,\FV@XRightMargin,\FVB@Example,\fvset}
% \DoNotIndex{\G,\g,\GetFileInfo,\gr,\GradientLoaded,\gsFKrbK@o,\gsj,\gsOX}
% \DoNotIndex{\hbadness,\hfuzz,\HLEmphasize,\HLMacro,\HLMacro@i}
% \DoNotIndex{\HLReverse,\HLReverse@i,\hqcu,\HqY}
% \DoNotIndex{\I,\i,\ifx,\input,\Ir,\IU}
% \DoNotIndex{\j,\jl,\JT,\JVodH}
% \DoNotIndex{\K,\k,\kfSlL}
% \DoNotIndex{\L,\let}
% \DoNotIndex{\message,\mHNa,\mIU}
% \DoNotIndex{\N,\nB,\newcmykcolor,\newdimen,\newif,\nW}
% \DoNotIndex{\O,\oCDJDo,\ocQhVI,\OnlyDescription,\oRKJ}
% \DoNotIndex{\P,\p,\ProvidesPackage,\psframe,\pslinewidth,\psset}
% \DoNotIndex{\PstAtCode,\PSTricksLoaded}
% \DoNotIndex{\q,\Qr,\qssRXq,\qu,\qXjFQp,\qYL}
% \DoNotIndex{\R,\r,\RecordChanges,\relax,\RlaYI,\rN,\Rp,\rp,\RPDXNn,\rput}
% \DoNotIndex{\S,\scalebox,\SgY,\SideBySide@Example,\SideBySideExample}
% \DoNotIndex{\SgY,\sk,\Sp,\space,\sZb}
% \DoNotIndex{\T,\the,\tw@}
% \DoNotIndex{\u,\UiSWGEf@,\uJi,\usepackage,\uVQdMM,\UYj}
% \DoNotIndex{\VerbatimEnvironment,\VerbatimInput,\VrC@}
% \DoNotIndex{\WhZ,\WjKCYb,\WNs}
% \DoNotIndex{\XkN,\XW}
% \DoNotIndex{\Z,\ZCM,\Ze}
% \DoNotIndex{\addtocounter,\advance,\alph,\arabic,\AtBeginDocument,\AtEndDocument}
% \DoNotIndex{\AtEndOfPackage,\begingroup,\bfseries,\bgroup,\box,\csname}
% \DoNotIndex{\else,\endcsname,\endgroup,\endinput,\expandafter,\fi}
% \DoNotIndex{\TeX,\z@,\p@,\@one,\xdef,\thr@@,\string,\sixt@@n,\reset,\or,\multiply,\repeat,\RequirePackage}
% \DoNotIndex{\@cclvi,\@ne,\@ehpa,\@nil,\copy,\dp,\global,\hbox,\hss,\ht,\ifodd,\ifdim,\ifcase,\kern}
% \DoNotIndex{\chardef,\loop,\leavevmode,\ifnum,\lower}
% \setcounter{IndexColumns}{2}
%
% ^^A To extend the height used for the text
%
% ^^A  Aligned labels in a description environment
%\newenvironment{Description}[1]{%
%\begin{list}{nothing}{\setlength{\leftmargin}{#1}
%\setlength{\labelwidth}{\leftmargin}\setlength{\labelsep}{1mm}}}
%{\end{list}}
%
% ^^A For macro names
%\DeclareRobustCommand\cs[1]{\texttt{\char`\\#1}}
%
%
% ^^A From ltugboat.cls
% ^^A For references
%\makeatletter
%\newcommand\acro[1]{\textsc{#1}\@}
%\def\CTAN{\acro{CTAN}}
%\let\texttub\textsl              % ^^A redefined in other situations
%\def\TUB{\texttub{TUGboat}}
%\def\TUG{\TeX\ \UG}
%\def\tug{\acro{TUG}}
%\def\UG{Users Group}
% ^^A For the bibliography 
%\let\@internalcite\cite
%\def\cite{\def\@citeseppen{-1000}%
%    \def\@cite##1##2{(##1\if@tempswa , ##2\fi)}%
%    \def\citeauthoryear##1##2##3{##1, ##3}\@internalcite}
%\def\etal{et\,al.\@}
%\newcommand\CTANdirectory[1]{\expandafter\urldef
%  \csname CTAN@#1\endcsname\path}
%\newcommand\CTANfile[1]{\expandafter\urldef
%  \csname CTAN@#1\endcsname\path}
%\newcommand\CTANref[1]{\expandafter\@setref\csname CTAN@#1\endcsname
%  \relax{#1}}
%\makeatother
% ^^A Define CTAN addresses 
%\CTANdirectory{mpattern}{graphics/metapost/macros/mpattern}
%\CTANdirectory{pstricks}{graphics/pstricks}
%\CTANdirectory{pst-fill.sty}{graphics/pstricks/latex/pst-fill.sty}
%\CTANdirectory{pst-fill}{graphics/pstricks/generic/pst-fill.tex}
%\CTANdirectory{Roegel}{graphics/metapost/contrib/macros/truchet}
%\CTANdirectory{xypic}{macros/generic/diagrams/xypic}
%
% ^^A Personal macros (D.G.)
% ^^A ----------------------
%
% ^^A Some colors used
%\definecolor{LemonChiffon}{rgb}{1.,0.98,0.8}
%\definecolor{LightBlue}   {rgb}{0.8,0.85,0.95}
%\definecolor{PaleGreen}   {rgb}{0.88,1,0.88}
%\definecolor{PeachPuff}   {rgb}{1.0,0.85,0.73}
%
% ^^A To define a unique string for TeX and LaTeX
%\newcommand{\AllTeX}{%
%{\rm(L\kern-.36em\raise.3ex\hbox{\sc a}\kern-.15em)%
%T\kern-.1667em\lower.7ex\hbox{E}\kern-.125emX}}
%
% ^^A Bibliography style
%\bibliographystyle{ltugbib}
%
% ^^A Name macros
%\newcommand{\FillPackage}{\textsf{`pst-fill'}}
%\newcommand{\XYpic}{%
%\leavevmode\hbox{\kern-.1em X\kern-.3em\lower.4ex\hbox{Y\kern-.15em}-pic}}
%
%\makeatletter
%
% ^^A Example environments
% ^^A (do not use in them the four JXYZ characters, that we will use
% ^^A as escape characters!)
%
% ^^A Default PSTricks parameters
%  \psset{dimen=middle}
%
% ^^A Translation in PSTricks from the one drawn by Emmanuel Chailloux and
% ^^A Guy Cousineau for the MLgraph system
% ^^A (see /ftp.ens.fr:/pub/unix/lang/MLgraph/version-2.1/MLgraph-refman.ps.gz)
% ^^A The kangaroo itself is reproduce from an original picture from Raoul Raba
% \newcommand{\DimX}{2.47}
% \newcommand{\DimY}{4.8}
% \newcommand{\DimXDivTwo}{1.235}
%
% \newcommand{\KangarooItself}[1]{%
% ^^A Body
% \pspolygon[fillstyle=solid,fillcolor=#1]%
%  (52.5,68)(55,72.5)(55.8,76.5)(56.8,79.8)(58.2,83)(60,85.8)(61.5,86.5)
% (64,87)(66,87.5)(67.8,87.3)(70,87)(71.5,87.3)(73,88)(74.7,88.5)
% (76,90.3)(77,91.5)(72.8,93.8)(69,96)(64.5,99)(59.4,103)(56.2,106.3)
% (53,110.5)(49.5,115.5)(47.2,119.9)(45.7,126)(43.2,123)(41.5,121)(37.5,125)
% (37,122.5)(36.8,120)(37,117)(37.6,113.5)(38.6,110)(40,106.3)(42,102.3)
%  (43.5,99.5)(45,97)(46.2,94)(46.8,91.7)(47.2,88)(47,83.5)(46.3,80.8)
%  (45.3,78.5)(42.5,76.5)(39.5,75.8)(36,75.9)(33,75.9)(29,76.2)(26,77)
%  (22.3,77.5)(18,78.4)(12.8,79.3)(8.6,80)(5.5,80.3)(3,80.5)(0,80)
%  (-5.2,78.5)(-9,76.3)(-11.2,74.8)(-13,72.5)(-16.5,68)(-16.5,68)(-19.5,62.5)
%  (-22,58)(-25.5,53)(-29,48.5)(-32.5,45)(-36,42)(-39,39.5)(-44,37)
%  (-49,35)(-51,34)(-53.5,34.5)(-55.5,36)(-56.5,38)(-56.5,40.5)(-55,41.5)
%  (-53.5,41)(-51.5,41)(-50.5,43)(-50.5,44.5)(-51,47)(-51.5,47.2)(-56.5,47)
%  (-58.5,46.5)(-60,44.7)(-62,42.3)(-63,39.5)(-63.5,36.3)(-63.5,33)(-63.1,29.5)
%  (-61.5,26)(-58,23.6)(-54,22.2)(-50.7,22)(-47.5,22)(-44.5,22.3)(-41,23.5)
%  (-36.8,25.8)(-33,28)(-28.5,31)(-23.4,35)(-20.2,38.3)(-17,42.5)(-13.5,47.5)
%  (-11.2,51.9)(-9.7,58)(-7.2,55)(-5.5,53)(-1.5,57)(-1,54.5)(-0.8,52)
%  (-1,49)(-1.6,45.5)(-2.6,42)(-4,38.3)(-6,34.3)(-7.5,31.5)(-9,29)
%  (-10.2,26)(-10.8,23.7)(-11.2,20)(-11,15.5)(-10.3,12.8)(-9.3,10.5)(-6.5,8.5)
%  (-3.5,7.8)(0,7.9)(3,7.9)(7,8.2)(10,9)(13.7,9.5)(18,10.4)
%  (23.2,11.3)(27.4,12)(30.5,12.3)(33,12.5)(36,12)(41.2,10.5)(45,8.3)
%  (47.2,6.8)(49,4.5)(52.5,0)(50,4.5)(49.2,8.5)(48.2,11.8)(46.8,15)
%  (45,17.8)(43.5,18.5)(41,19)(39,19.5)(37.2,19.3)(35,19)(33.5,19.3)
%  (32,20)(30.3,20.5)(29,22.3)(28,23.5)(28,23.5)(24.5,22.3)(21.5,22)
%  (18.3,22)(15,22.2)(11,23.6)(7.5,26)(5.9,29.5)(5.5,33)(5.5,36.3)
%  (6,39.5)(7,42.3)(9,44.7)(10.5,46.5)(12.5,47)(17.5,47.2)(18,47)
%  (18.5,44.5)(18.5,43)(17.5,41)(15.5,41)(14,41.5)(12.5,40.5)(12.5,38)
%  (13.5,36)(15.5,34.5)(18,34)(20,35)(25,37)(30,39.5)(33,42)
%  (36.5,45)(40,48.5)(43.5,53)(47,58)(49.5,62.5)(52.5,68)
% ^^A Eye
% \pscircle*[linecolor=white](58.2,98.3){2\psxunit}
% \pscircle*(58.2,97.3){\psxunit}
% ^^A Mouth
% \psline(71.5,88)(70,89.3)(68.5,90.3)(67,91.9)
% ^^A Tear
% \psline(42,121)(45,118)(47,115.3)(48.5,112.7)(50,110)(51.8,106.5)
%       (52.5,103.7)(53,100.5)
% \pspolygon(41.2,115.8)(43.2,114.7)(45,112.5)(47,109.8)(48,107)(49.5,104.2)%
%       (50.5,101.6)(51,98.5)(47.7,100.6)(46,102.2)(44.8,104)(43.5,106)
%       (42.5,108)(41.7,110.5)(41,113.2)}
%
% \newcommand{\Kangaroo}[1]{%
%   \begin{pspicture}(\DimX,\DimY)
%   \psset{unit=0.035278}
%   \KangarooItself{#1}
%   \end{pspicture}}
%
% \newcommand{\KangarooPstChart}[1]{{%
%   \psset{xunit=0.006784,yunit=0.00735,linewidth=0.01}
%   \begin{pspicture}(-65.5,0)(82,126)
%     \KangarooItself{#1}
%   \end{pspicture}}}
%
%
% ^^A For the possible index and changes log
% \setlength{\columnseprule}{0.6pt}
%
% ^^A Beginning of the documentation itself
%\title{\texttt{pst-fill}\\A PSTricks package for filling and tiling areas}
%\author{Timothy Van Zandt\thanks{\protect\url{tvz@econ.insead.fr}. (documentation by
% Denis Girou (\protect\url{Denis.Girou@idris.fr}) and Herbert Vo\ss (\protect\url{hvoss@tug.org}).}}
%
%\date{\shortstack{\today --- Version 1.00\\
%                  {\small Documentation revised \today}}}
% \maketitle
% \tableofcontents
%
%\begin{abstract}
%  \FillPackage{} is a PSTricks \cite{vanZandt93},\cite{Girou94},\cite{vanZandtGirou94}, 
%\cite{Hoenig97},\cite{LGC97} package to draw easily
%  various kinds of filling and tiling of areas. It is also a good example of
%  the great power and flexibility of PSTricks, as in fact it is a very short
%  program (it body is around 200~lines long) but nevertheless really powerful.
%
%  \hspace{5mm} It was written in 1994 by Timothy \textsc{van Zandt} but
%  publicly available only in PSTricks 97 and without any documentation.
%  We describe here the version \emph{97 patch 2} of December 12, 1997, which
%  is the original one modified by myself to manage \emph{tilings} in the
%  so-called \emph{automatic} mode. This article would like to serve both of
%  reference manual and of user's guide.
%
%This package is available on \CTAN{} in the
%  \texttt{graphics/pstricks} directory (files \texttt{latex/pst-fill.sty} and
%  \texttt{generic/pst-fill.tex}).
%\end{abstract}
%
%\section{Introduction}
%
%  Here we will refer as \emph{filling} as the operation which consist to fill
%a defined area by a pattern (or a composition of patterns). We will refer as
%\emph{tiling} as the operation which consist to do the same thing, but with
%the control of the starting point, which is here the upper left corner.
%The pattern is positioned relatively to this point. This make an essential
%difference between the two modes, as without control of the starting point we
%can't draw \emph{tilings} (sometimes  called \emph{tesselations}) as used in
%many fields of Art and Science%
%\footnote{For an extensive presentation of tilings, in their history and usage
%in many fields, see the reference book \cite{GS87}.
%
%  In the \TeX{} world, few work was done on tilings. You can look at the
%\emph{tile} extension of the \XYpic{} package \cite{XYpic}, at the articles of
%Kees \textsc{van der Laan} \cite[paragraph 7]{LAAN96} (the tiling was in
%fact directly done in PostScript) and \cite{LAAN97}, at the \MP{} program
%(available on \CTANref{Roegel}) by Denis \textsc{Roegel} for the
%\textsc{Truchet} contest in 1995 \cite{EsperetGirou98} and at the \MP{}
%package \cite{Bolek98} to draw patterns, which have a strong connection with
%tilings.}.
%
%  Nevertheless, as tilings are a wide and difficult field in mathematics, this
%package is limited to simple ones, mainly \emph{monohedral} tilings with one
%prototile (which can be composite, see section \ref{sec:KindTiles}). With some
%experience and wiliness we can do more and obtained easily rather
%sophisticated results, but obviously hyperbolic tilings like the famous
%\textsc{Escher} ones or aperiodic tilings like the \textsc{Penrose} ones are
%not in the capabilities of this package. For more complex needs, we must used
%low level and more painfull technics, with the basic \cs{multido}
%and \cs{multirput} macros.
%
%\section{Package history and description of it two different modes}
%
%  As already said, this package was written in 1994 by Timothy \textsc{van
%Zandt}. Two modes were defined, called respectively \emph{manual} and
%\emph{automatic}. For both, the pattern is generated on contiguous positions in
%a rather large area which include the region to fill, later cut to the
%required dimensions by clipping mechanism. In the first mode, the pattern is
%explicitely inserted in the PostScript file each time. In the second one, the
%result is the same but with an unique explicit insertion of the pattern and a
%repetition done by PostScript. Nevertheless, in this method, the control of
%the starting point was loosed, so it allowed only to \emph{fill} a region and
%not to \emph{tile} it.
%
%  See the difference between the two modes, \emph{tiling}:
% {\psset{unit=0.5cm}%
% \psboxfill{\begin{pspicture}(1,1)\psframe[dimen=middle](1,1)\end{pspicture}}
% \begin{pspicture}(3,3.3)
%   \psframe[fillstyle=boxfill](3,3)
% \end{pspicture}}
% and \emph{filling}:
%{%
% \makeatletter
%\pst@def{BoxFill}<
%  gsave
%    gsave \tx@STV CM grestore dtransform CM idtransform
%    abs /h ED abs /w ED
%    pathbbox
%    h div round 2 add cvi /y2 ED
%    w div round 2 add cvi /x2 ED
%    h div round 2 sub cvi /y1 ED
%    w div round 2 sub cvi /x1 ED
%    /y2 y2 y1 sub def
%    /x2 x2 x1 sub def
%    CP
%    y1 h mul sub neg /y1 ED
%    x1 w mul sub neg /x1 ED
%    clip
%    y2 {
%      /x x1 def
%      x2 {
%        save CP x y1 T moveto Box restore
%        /x x w add def
%      } repeat
%      /y1 y1 h add def
%    } repeat
% currentpoint currentfont grestore setfont moveto>
% \makeatother
%
% \psset{unit=0.5}
% \psboxfill{\begin{pspicture}(1,1)\psframe[dimen=middle](1,1)\end{pspicture}}
% \begin{pspicture}(3,3.3)
%   \psframe[fillstyle=boxfill](3,3)
% \end{pspicture}
% or
% \begin{pspicture}(3,3.3)
%   \psframe[fillstyle=boxfill](3,3)
% \end{pspicture}
%}
%as we can see that initial position is arbitrary and dependent of
%the current point.
%
%
% It's clear that usage of filling is very restrictive comparing to tiling,
%as desired effects required very often the possibility to control the starting 
%point. So, this mode was of limited interest, but unfortunately the
%\emph{manual} one has the very big disadvantage to require very huge amounts
%of ressources, mainly in disk space and consequently in printing time.
%A small tiling can require sometimes several megabytes in \emph{manual} mode!
%So, it was very often not really usable in practice.
%
%It is why I modified the code, to allow tilings in \emph{automatic} mode,
%controlling in this mode too the starting point. And most of the time, that is
%to say if some special options are not used, the tiling is done exactly in the
%region described, which make it faster. So there is no more reason to use the
%\emph{manual} mode, apart very special cases where \emph{automatic} one cannot
%work, as explained later -- currently, I know only one case.
%
%  To load this modified \emph{automatic} mode, with \LaTeX{} use
%simply:\newline 
%\verb+\usepackage[tiling]{pst-fill}+\newline
%and in plain \TeX{} after:\newline
%\verb+\input{pst-fill}+\newline
%add the following definition:\newline
%\verb+\def\PstTiling{true}+
%
%  To obtain the original behaviour, just don't use the \emph{tiling} optional
%keyword at loading.
%
%  Take care than in \emph{tiling} mode, I introduce also some other changes.
%First I define aliases on some parameter names for consistancy (all specific
%parameters will begin by the \texttt{fill} prefix in this case) and I change
%some default values, which were not well adapted for tilings (\texttt{fillsep}
%is set to 0 and as explained \texttt{fillsize} set to \texttt{auto}). I rename 
%\texttt{fillcycle} to \texttt{fillcyclex}. I also restore normal way so that
%the frame of the area is drawn and all line (\texttt{linestyle},
%\texttt{linecolor}, \texttt{doubleline}, etc.) parameters are now active (but
%there are not in non \emph{tiling} mode). And I also introduce new parameters
%to control the tilings (see below).
%
%  \textbf{In all the following examples, we will consider only the
% \emph{tiling} mode.}
%
%  To do a tiling, we have just to define the pattern with the
% \verb+\psboxfill+ macro and to use the new \texttt{fillstyle}
% \verb+boxfill+.
%
%  Note that tilings are drawn from left to right and top to bottom, which can
%have an importance in some circonstances.
%
%  PostScript programmers can be also interested to know that, even in the
%\emph{automatic} mode, the iterations of the pattern are managed directly by
%the PostScript code of the package which used only PostScript Level 1
%operators. The special ones introduced in Level 2 for drawing of patterns
%\cite[section 4.9]{PostScript95} are not used.
%
%  And first, for conveniance, we define a simple \cs{Tiling} macro, which
%will simplify our examples:
%
%\begin{verbatim}
%  \newcommand{\Tiling}[2][]{%
%    \edef\Temp{#1}%
%    \begin{pspicture}#2
%      \ifx\Temp\empty
%        \psframe[fillstyle=boxfill]#2
%      \else
%        \psframe[fillstyle=boxfill,#1]#2
%      \fi
%    \end{pspicture}}
%\end{verbatim}
%
%
%\newcommand{\Tiling}[2][]{%
%  \edef\Temp{#1}%
%  \begin{pspicture}#2
%    \ifx\Temp\empty
%      \psframe[fillstyle=boxfill]#2
%    \else
%      \psframe[fillstyle=boxfill,#1]#2
%    \fi
% \end{pspicture}}
%
%\subsection{Parameters}
%
%  There are \textbf{14} specific parameters available to change the way the
% filling/tiling is defined, and one debugging option.
%
% \begin{Description}{2cm}
%  \item [fillangle (real)\hfill :] the value of the rotation
%  applied to the patterns (\emph{Default:~0}).
% \end{Description}
%
%
%   In this case, we must force the tiling area to be notably larger than the
% area to cover, to be sure that the defined area will be covered after rotation.
% \lstset{gobble=2}
% \begin{LTXexample}
% \newcommand{\Square}{%
%   \begin{pspicture}(1,1)
%     \psframe[dimen=middle](1,1)
%   \end{pspicture}}
% \psset{unit=0.5}
% \psboxfill{\Square}
% \Tiling[fillangle=45]{(3,3)}\quad
% \Tiling[fillangle=-60]{(3,3)}
% \end{LTXexample}
% 
% \newcommand{\Square}{\begin{pspicture}(1,1)\psframe[dimen=middle](1,1)\end{pspicture}}
% 
% \begin{Description}{2cm}
%   \setcounter{footnote}{1}
%   \item[\texttt{fillsepx} (real$\|$dim) :] value of the horizontal
%   separation between consecutive patterns (\emph{Default:~0 for
%   tilings\footnotemark, 2pt otherwise}).  \footnotetext{This option was added
%   by me, is not part of the original package and is available only if the
%   \texttt{tiling} keyword is used when loading the package.}
%   \setcounter{footnote}{1}
%   \item [\texttt{fillsepy} (real$\|$dim)\hfill :] value of the vertical
%   separation between consecutive patterns (\emph{Default:~0 for
%   ti\-lings\footnotemark, 2pt otherwise}).
%   \setcounter{footnote}{1}
%   \item [\texttt{fillsep} (real$\|$dim)\hfill :] value of horizontal and
%   vertical separations between consecutive patterns (\emph{Default:~0 for
%   tilings\footnotemark, 2pt otherwise}).
% \end{Description}
% 
%   These values can be negative, which allow the tiles to overlap.
% 
% \begin{LTXexample}
% \psset{unit=0.5}
% \psboxfill{\Square}
% \Tiling[fillsepx=2mm]{(3,3)} 
% \Tiling[fillsepy=1mm]{(3,3)}\\
% \Tiling[fillsep=0.5]{(3,3)} 
% \Tiling[fillsep=-0.5]{(3,3)}
% \end{LTXexample}
% 
% \begin{Description}{2cm}
%   \item [\texttt{fillcyclex}\footnotemark\ (integer)\hfill :] Shift
%   coefficient applied to each row (\emph{Default:~0}).
%   \footnotetext{It was \texttt{fillcycle} in the original version.}
%   \setcounter{footnote}{1}
%   \item [\texttt{fillcycley}\footnotemark\ (integer)\hfill :] Same thing for
%   columns (\emph{Default:~0}).
%   \setcounter{footnote}{1}
%   \item [\texttt{fillcycle}\footnotemark\ (integer)\hfill :] Allow to fix
%   both \texttt{fillcyclex} and \texttt{fillcycley} directly to the same value
%   (\emph{Default:~0}).
% \end{Description}
% 
%   For instance, if \texttt{fillcyclex} is 2, the second row of patterns will
% be horizontally shifted by a factor of $\frac{1}{2}=0.5$, and by a factor of
% 0.333 if \texttt{fillcyclex} is 3, etc.). These values can be negative.
% 
% \begin{LTXexample}[width=0.35\linewidth]
% \psset{unit=0.5}
% \psboxfill{\Square}
% \newcommand{\TilingA}[1]{\Tiling[fillcyclex=#1]{(3,3)}}
% \TilingA{0} \TilingA{1}\\
% \TilingA{2} \TilingA{3}\\[3mm]
% \TilingA{4} \TilingA{5}\\
% \TilingA{6} \TilingA{-3}\\[3mm]
% \Tiling[fillcycley=2]{(3,3)}
% \Tiling[fillcycley=3]{(3,3)}\\
% \Tiling[fillcycley=-3]{(3,3)}
% \Tiling[fillcycle=2]{(3,3)}
% \end{LTXexample}
% 
% \begin{Description}{2cm}
%   \setcounter{footnote}{1}
%   \item [\texttt{fillmovex}\footnotemark\ (real$\|$dim)\hfill :] value of the
%   horizontal moves between consecutive patterns (\emph{Default:~0}).
%   \setcounter{footnote}{1}
%   \item [\texttt{fillmovey}\footnotemark\ (real$\|$dim)\hfill :] value of the
%   vertical moves between consecutive patterns (\emph{Default:~0}).
%   \setcounter{footnote}{1}
%   \item [\texttt{fillmove}\footnotemark\ (real$\|$dim)\hfill :] value of
%   horizontal and vertical moves between consecutive patterns
%   (\emph{Default:~0}).
% \end{Description}
% 
%   These parameters allow the patterns to overlap and to draw some special
% kinds of tilings. They are implemented only for the \emph{automatic} and
% \emph{tiling} modes and their values can be negative.
% 
%   In some cases, the effect of these parameters will be the same that with the 
% \texttt{fillcycle?} ones, but you can see that it is not true for some other
% values.
% 
% \begin{LTXexample}
% \psset{unit=0.5}
% \psboxfill{\Square}
% \Tiling[fillmovex=0.5]{(3,3)} 
% \Tiling[fillmovey=0.5]{(3,3)}\\
% \Tiling[fillmove=0.5]{(3,3)}
% \Tiling[fillmove=-0.5]{(3,3)}
% \end{LTXexample}
% 
% \begin{Description}{2cm}
%   \item [\texttt{fillsize}
%   (auto$\|$\{(real$\|$dim,real$\|$dim)(real$\|$dim,real$\|$dim)\}) :] The
%   choice of \emph{automatic} mode or the size of the area in \emph{manual}
%   mode. If first pair values are not given, (0,0) is used. (\emph{Default:
%   auto when \emph{tiling} mode is used, {(-15cm,-15cm)(15cm,15cm)}
%   otherwise}).
% \end{Description}
% 
%   As explained in the introduction, the \emph{manual} mode can require very
% huge amount of computer ressources. So, it usage is to discourage in front off
% the \emph{automatic} mode. It seems only useful in special circonstances, in
% fact when the \emph{automatic} mode failed, which is known only in one case,
% for some kinds of EPS files, as the ones produce by dump of portions of
% screens (see \ref{sec:GraphicFiles}).
% 
% \begin{Description}{2cm}
%   \setcounter{footnote}{1}
%   \item [\texttt{fillloopaddx}\footnotemark\ (integer)\hfill :] number of
%   times the pattern is added on left and right positions (\emph{Default:~0}).
%   \setcounter{footnote}{1}
%   \item [\texttt{fillloopaddy}\footnotemark\ (integer)\hfill :] number of
%   times the pattern is added on top and bottom positions (\emph{Default:~0}).
%   \setcounter{footnote}{1}
%   \item [\texttt{fillloopadd}\footnotemark\ (integer)\hfill :] number of
%   times the pattern is added on left, right, top and bottom positions
%   (\emph{Default:~0}).
% \end{Description}
% 
%   These parameters are only useful in special circonstances, as for complex
% patterns when the size of the rectangular box used to tile the area doesn't 
% correspond to the pattern itself (see an example in Figure~\ref{fig:Sheeps})
% and also sometimes when the size of the pattern is not a divisor of the size
% of the area to fill and that the number of loop repeats is not properly
% computed, which can occur.
% 
%   They are implemented only for the \emph{tiling} mode.
% 
% \begin{Description}{2cm}
%   \setcounter{footnote}{1}
%   \item [\texttt{PstDebug}\footnotemark\ (integer, 0 or 1)\hfill :] to
%   require to see the exact tiling done, without clipping (\emph{Default:~0}).
% \end{Description}
% 
%   It's mainly useful for debugging or to understand better how the tilings
% are done. It is implemented only for the \emph{tiling} mode.
% 
% \begin{LTXexample}
% \psset{unit=0.3,PstDebug=1}
% \psboxfill{\Square}
% \psset{linewidth=1mm}
% \Tiling{(2,2)}\\[5mm]
% \Tiling[fillcyclex=2]{(2,2)}\\[1cm]
% \Tiling[fillmove=0.5]{(2,2)}
% \end{LTXexample}
% 
% \vspace{3cm}
% \section{Examples}
% 
%   In fact this unique \cs{psboxfill} macro allow a lot a variations and
% different usages. We will try here to demonstrate this.
% 
% \subsection{Kind of tiles}
% \label{sec:KindTiles}
% 
%   Of course, we can access to all the power of PSTricks macros to define the
% \emph{tiles} (\emph{patterns}) used. So, we can define complicated ones.
% 
%   Here we give four other Archimedian tilings (those built with only some
% regular polygons) among the twelve existing, first discovered completely by
% Johanes \textsc{Kepler} at the beginning of 17th century \cite{GS87}, the two
% other \emph{regular} ones with the tiling by squares, formed by a unique
% regular polygon, and two other formed by two different regular polygons.
% 
% \begin{LTXexample}[pos=t]
%   \newcommand{\Triangle}{%
%     \begin{pspicture}(1,1)
%       \pstriangle[dimen=middle](0.5,0)(1,1)
%     \end{pspicture}}
%   \newcommand{\Hexagon}{
% ^^A sin(60)=0.866
%     \begin{pspicture}(0.866,0.75)
%       \SpecialCoor
% ^^A  Hexagon  
%       \pspolygon[dimen=middle]%
%         (0.5;30)(0.5;90)(0.5;150)(0.5;210)(0.5;270)(0.5;330)
%     \end{pspicture}}
% 
%   \psset{unit=0.5}
%   \psboxfill{\Triangle}
%   \Tiling{(4,4)}\hfill
% ^^A The two other regular tilings
%   \Tiling[fillcyclex=2]{(4,4)}\hfill
%   \psboxfill{\Hexagon}
%   \Tiling[fillcyclex=2,fillloopaddy=1]{(5,5)}
% \end{LTXexample}
% 
% \begin{LTXexample}[pos=t]
%   \newcommand{\ArchimedianA}{%
%      ^^A Archimedian tiling 3^2.4.3.4
%     \psset{dimen=middle}
%      ^^A sin(60)=0.866
%     \begin{pspicture}(1.866,1.866)
%       \psframe(1,1)
%       \psline(1,0)(1.866,0.5)(1,1)(0.5,1.866)(0,1)(-0.866,0.5)
%       \psline(0,0)(0.5,-0.866)
%     \end{pspicture}}
%   \newcommand{\ArchimedianB}{%
%      ^^A Archimedian tiling 4.8^2
%     \psset{dimen=middle,unit=1.5}
%      ^^A sin(22.5)=0.3827 ; cos(22.5)=0.9239
%     \begin{pspicture}(1.3066,0.6533)
%       \SpecialCoor
%      ^^A Octogon
%       \pspolygon(0.5;22.5)(0.5;67.5)(0.5;112.5)(0.5;157.5)
%                 (0.5;202.5)(0.5;247.5)(0.5;292.5)(0.5;337.5)
%     \end{pspicture}}
% 
%   \psset{unit=0.5}
%   \psboxfill{\ArchimedianA}
%   \Tiling[fillmove=0.5]{(7,7)}\hfill
%   \psboxfill{\ArchimedianB}
%   \Tiling[fillcyclex=2,fillloopaddy=1]{(7,7)}
% \end{LTXexample}
% 
%   \setcounter{footnote}{3}
%   We can of course tile an area arbitrarily defined. And with the
% \texttt{addfillstyle} parameter\footnote{Introduced in PSTricks 97.}, we can
% easily mix the \texttt{boxfill} style with another one.
% 
% \begin{LTXexample}[width=6cm]
%   \psset{unit=0.5,dimen=middle}
%   \psboxfill{%
%     \begin{pspicture}(1,1)
%       \psframe(1,1)
%       \pscircle(0.5,0.5){0.25}
%     \end{pspicture}}
%   \begin{pspicture}(4,6)
%     \pspolygon[fillstyle=boxfill,fillsep=0.25](0,1)(1,4)(4,6)(4,0)(2,1)
%   \end{pspicture}\hspace{1em}
%   \begin{pspicture}(4,4)
%%     \pscircle[linestyle=none,fillstyle=solid,fillcolor=yellow,fillsep=0.5,
%%               addfillstyle=boxfill](2,2){2}
%   \end{pspicture}
% \end{LTXexample}
%
%   Various effects can be obtained, sometimes complicated ones very easily, as
% in this example reproduced from one shown by Slavik \textsc{Jablan} in the
% field of \emph{OpTiles}, inspired by the \emph{Op-art}:
% 
% \begin{LTXexample}[pos=t]
% \newcommand{\ProtoTile}{%
%  \begin{pspicture}(1,1)%%% 1/12=0.08333
%   \psset{linestyle=none,linewidth=0,
%     hatchwidth=0.08333\psunit,hatchsep=0.08333\psunit}
%   \psframe[fillstyle=solid,fillcolor=black,addfillstyle=hlines,hatchcolor=white](1,1)
%   \pswedge[fillstyle=solid,fillcolor=white,addfillstyle=hlines]{1}{0}{90}
%  \end{pspicture}}
% \newcommand{\BasicTile}{%
%  \begin{pspicture}(2,1)
%    \rput[lb](0,0){\ProtoTile}\rput[lb](1,0){\psrotateleft{\ProtoTile}}
%  \end{pspicture}}
% \ProtoTile\hfill\BasicTile\hfill
% \psboxfill{\BasicTile}
% \Tiling[fillcyclex=2]{(4,4)}
% \end{LTXexample}
% 
%   It is also directly possible to surimpose several different tilings. Here is
% the splendid visual proof of the \textsc{Pytha\-gore} theorem done by the arab
% mathematician \textsc{Annairizi} around the year 900, given by superposition
% of two tilings by squares of different sizes.
% 
% \begin{LTXexample}[pos=t]
% \psset{unit=1.5,dimen=middle}
% \begin{pspicture*}(3,3)
%   \psboxfill{\begin{pspicture}(1,1)
%     \psframe(1,1)\end{pspicture}}
%   \psframe[fillstyle=boxfill](3,3)
%   \psboxfill{\begin{pspicture}(1,1)
%     \rput{-37}{\psframe[linecolor=red](0.8,0.8)}
%   \end{pspicture}}
%   \psframe[fillstyle=boxfill](3,4)
%   \pspolygon[fillstyle=hlines,hatchangle=90](1,2)(1.64,1.53)(2,2)
% \end{pspicture*}
% \end{LTXexample}
% 
%   In a same way, it is possible to build tilings based on figurative patterns,
% in the style of the famous \textsc{Escher} ones. Following an example of
% Andr\'e \textsc{Deledicq} \cite{Deledicq97}, we first show a simple tiling of
% the \emph{p1} category (according to the international classification of the
% 17~symmetry groups of the plane first discovered by the russian
% crystalographer Jevgraf \textsc{Fedorov} at the end of the 19th century):
% 
% \begin{LTXexample}[pos=t]
%  \newcommand{\SheepHead}[1]{%
%    \begin{pspicture}(3,1.5)
%      \pscustom[liftpen=2,fillstyle=solid,fillcolor=#1]{%
%        \pscurve(0.5,-0.2)(0.6,0.5)(0.2,1.3)(0,1.5)(0,1.5)
%          (0.4,1.3)(0.8,1.5)(2.2,1.9)(3,1.5)(3,1.5)(3.2,1.3)
%          (3.6,0.5)(3.4,-0.3)(3,0)(2.2,0.4)(0.5,-0.2)}
%      \pscircle*(2.65,1.25){0.12\psunit} % Eye
%      \psccurve*(3.5,0.3)(3.35,0.45)(3.5,0.6)(3.6,0.4)% Muzzle
%     ^^A   % Mouth
%       \pscurve(3,0.35)(3.3,0.1)(3.6,0.05)
%     ^^A   % Ear
%       \pscurve(2.3,1.3)(2.1,1.5)(2.15,1.7)\pscurve(2.1,1.7)(2.35,1.6)(2.45,1.4)
%   \end{pspicture}}
%  \psboxfill{\psset{unit=0.5}\SheepHead{yellow}\SheepHead{cyan}}
%  \Tiling[fillcyclex=2,fillloopadd=1]{(10,5)}
% \end{LTXexample}
% \label{fig:Sheeps}
% 
%   Now a tiling of the \emph{pg} category (the code for the kangaroo itself is
% too long to be shown here, but has no difficulties ; the kangaroo is reproduce
% from an original picture from Raoul \textsc{Raba} and here is a translation in
% PSTricks from the one drawn by Emmanuel \textsc{Chailloux} and Guy
% \textsc{Cousineau} for their MLgraph system \cite{MLgraphTSI}):
% 
% \begin{LTXexample}[pos=t]
% \psboxfill{\psset{unit=0.4}
%   \Kangaroo{yellow}\Kangaroo{red}\Kangaroo{cyan}\Kangaroo{green}%
%   \psscalebox{-1 1}{%
%     \rput(1.235,4.8){\Kangaroo{green}\Kangaroo{cyan}\Kangaroo{red}\Kangaroo{yellow}}}}
%   \Tiling[fillloopadd=1]{(10,6)}
% \end{LTXexample}
% 
%   And here a \textsc{Wang} tiling \cite{Wang65}, \cite[chapter
% 11]{GS87}, based on very simple tiles of the form of a square and composed
% of four colored triangles. Such tilings are built with only a matching color
% constraint. Despite of it simplicity, it is an important kind of tilings, as
% \textsc{Wang} and others used them to study the special class of
% \emph{aperiodic} tilings, and also because it was shown that surprisingly this 
% tiling is similar to a \textsc{Turing} machine.
% 
% \begin{LTXexample}[pos=t]
%   \newcommand{\WangTile}[4]{%
%     \begin{pspicture}(1,1)
%       \pspolygon*[linecolor=#1](0,0)(0,1)(0.5,0.5)
%       \pspolygon*[linecolor=#2](0,1)(1,1)(0.5,0.5)
%       \pspolygon*[linecolor=#3](1,1)(1,0)(0.5,0.5)
%       \pspolygon*[linecolor=#4](1,0)(0,0)(0.5,0.5)
%     \end{pspicture}}
%   \newcommand{\WangTileA}{\WangTile{cyan}{yellow}{cyan}{cyan}}
%   \newcommand{\WangTileB}{\WangTile{yellow}{cyan}{cyan}{red}}
%   \newcommand{\WangTileC}{\WangTile{cyan}{red}{yellow}{yellow}}
%   \newcommand{\WangTiles}[1][]{%
%     \begin{pspicture}(3,3) \psset{ref=lb}
%       \rput(0,2){\WangTileB}  \rput(1,2){\WangTileA}%
%       \rput(2,2){\WangTileC}  \rput(0,1){\WangTileC}%
%       \rput(1,1){\WangTileB}  \rput(2,1){\WangTileA}
%       \rput(0,0){\WangTileA}  \rput(1,0){\WangTileC}%
%       \rput(2,0){\WangTileB}
%       #1
%     \end{pspicture}}
%   \WangTileA\hfill\WangTileB\hfill\WangTileC\hfill
%   \WangTiles[{\psgrid[subgriddiv=0,gridlabels=0](3,3)}]\hfill
%   \psset{unit=0.4} \psboxfill{\WangTiles} \Tiling{(12,12)}
% \end{LTXexample}
% 
% \subsection{External graphic files}
% \label{sec:GraphicFiles}
% 
%   We can also fill an arbitrary area with an external image. We have only, 
% as usual, to matter of the \emph{BoundingBox} definition if there is no one
% provided or if it is not the accurate one, as for the well known
% \texttt{tiger} picture part of the \texttt{ghostscript} distribution.
% 
% \begin{LTXexample}[pos=t]
%   \psboxfill{%% Strangely require x1=x2...
%     \begin{pspicture}(0,1)(0,4.1)
%       \includegraphics[bb=17 176 560 74,width=3cm]{tiger}
%     \end{pspicture}}
%   \Tiling{(6,6.2)}
% \end{LTXexample}
% 
%   Nevertheless, there are some special files for which the \emph{automatic}
% mode doesn't work, specially for some files obtained by a screen dump, as in
% the next example, where a picture was reduced before it conversion in the
% \emph{Encapsulated PostScript} format by a screen dump utility. In this case,
% usage of the \emph{manual} mode is the only alternative, at the price of the
% real multiple inclusion of the EPS file. We must take care to specify the
% correct \texttt{fillsize} parameter, because otherwise the default values are
% large and will load the file many times, perhaps just really using few
% occurrences as the other ones would be clipped...
% 
% \begin{LTXexample}[pos=t]
%   \psboxfill{\includegraphics{flowers}}
%   \begin{pspicture}(8,4)
%     \psellipse[fillstyle=boxfill,fillsize={(8,4)}](4,2)(4,2)
%   \end{pspicture}
% \end{LTXexample}
% 
% \subsection{Tiling of characters}
% 
%   We can also use the \cs{psboxfill} macro to fill the interior of characters
% for special effects like these ones:
% 
% \begin{LTXexample}[pos=t]
%   \DeclareFixedFont{\bigsf}{T1}{phv}{b}{n}{4.5cm}
%   \DeclareFixedFont{\smallrm}{T1}{ptm}{m}{n}{3mm}
%   \psboxfill{\smallrm Since 182 days...}
%   \begin{pspicture*}(8,4)
%     \centerline{%
%       \pscharpath[fillstyle=gradient,gradangle=-45,
%                   gradmidpoint=0.5,addfillstyle=boxfill,
%                   fillangle=45,fillsep=0.7mm]
%                  {\rput[b](0,0.1){\bigsf 2000}}}
%   \end{pspicture*}
% \end{LTXexample}
% 
% \begin{LTXexample}[pos=t]
%   \DeclareFixedFont{\mediumrm}{T1}{ptm}{m}{n}{2cm}
%   \psboxfill{%
%     \psset{unit=0.1,linewidth=0.2pt}
%     \Kangaroo{PeachPuff}\Kangaroo{PaleGreen}%
%       \Kangaroo{LightBlue}\Kangaroo{LemonChiffon}%
%     \psscalebox{-1 1}{%
%       \rput(1.235,4.8){%
%         \Kangaroo{LemonChiffon}\Kangaroo{LightBlue}%
%           \Kangaroo{PaleGreen}\Kangaroo{PeachPuff}}}}
% ^^A   % A kangaroo of kangaroos...
%   \begin{pspicture}(8,2)
%     \pscharpath[linestyle=none,fillstyle=boxfill,fillloopadd=1]
%                {\rput[b](4,0){\mediumrm Kangaroo}}
%   \end{pspicture}
% \end{LTXexample}
% 
% \subsection{Other kinds of usage}
% 
%   Other kinds of usage can be imagined. For instance, we can use tilings in a
% sort of degenerated way to draw some special lines made by a unique or
% multiple repeating patterns. But it can be only a special dashed line, as here
% with three different dashes:
% 
% \begin{LTXexample}[pos=t]
%   \newcommand{\Dashes}{%
%     \psset{dimen=middle}
%     \begin{pspicture}(0,-0.5\pslinewidth)(1,0.5\pslinewidth)
%       \rput(0,0){\psline(0.4,0)}%
%         \rput(0.5,0){\psline(0.2,0)}%
%         \rput(0.8,0){\psline(0.1,0)}
%     \end{pspicture}}
% 
%   \newcommand{\SpecialDashedLine}[3]{%
%     \psboxfill{#3}
%     \Tiling[linestyle=none]
%            {(#1,-0.5\pslinewidth)(#2,0.5\pslinewidth)}}
% 
%   \SpecialDashedLine{0}{7}{\Dashes}
% 
%   \psset{unit=0.5,linewidth=1mm,linecolor=red}
%   \SpecialDashedLine{0}{10}{\Dashes}
% \end{LTXexample}
% 
%   It allow also to use special patterns in business graphics, as in the
% following example generated by \texttt{PstChart}\footnote{A personal
% development to draw business charts with PSTricks, not distributed.}.
% 
% \vspace{3mm}
% \begin{figure}[!ht]
% \centering
% \psset{unit=0.75}
% ^^A % Generated by pstchart.sh version 0.21 (11/28/97)
% {\psset{dimen=middle}
% \psset{xunit=2,yunit=0.005}
% \begin{pspicture}(-0.6,-200)(6.6,2300)
% ^^A   % Title
%   \rput(3,2200){\shortstack{Fantaisist repartition of kangaroos\\
%                             in the world (in thousands)}}
% ^^A   % Frame background
%   \psframe[fillstyle=solid,fillcolor=LemonChiffon](0,0)(6,2000)
% ^^A   % Graduations
%   \multido{\n=0+500}{5}{\rput[r](-0.12,\n){\psscalebox{0.8}{\n}}}
% ^^A   % Minor ticks
%   \multips(0,100)(0,100){19}{\psline[unit=4.8pt](1,0)}
%   \multips(6,100)(0,100){19}{\psline[unit=4.8pt](-1,0)}
% ^^A   % Major ticks
%   \multips(0,500)(0,500){3}{\psline[unit=9.6pt](1,0)}
%   \multips(6,500)(0,500){3}{\psline[unit=9.6pt](-1,0)}
% ^^A   % Lines from major ticks marks
%   \multips(0,500)(0,500){3}{\psline[linestyle=dotted,linewidth=0.6pt](6,0)}
% ^^A   % Drawing for the data
%   \psboxfill{\psset{unit=0.78\psxunit}\KangarooPstChart{red}}
%   \psframe[linestyle=none,fillstyle=boxfill,fillloopaddy=1](0.61,0)(1.39,1800)
%   \psboxfill{\psset{unit=0.78\psxunit}\KangarooPstChart{yellow}}
%   \psframe[linestyle=none,fillstyle=boxfill,fillloopaddy=1](1.61,0)(2.39,800)
%   \psboxfill{\psset{unit=0.78\psxunit}\KangarooPstChart{cyan}}
%   \psframe[linestyle=none,fillstyle=boxfill,fillloopaddy=1](2.61,0)(3.39,550)
%   \psboxfill{\psset{unit=0.78\psxunit}\KangarooPstChart{magenta}}
%   \psframe[linestyle=none,fillstyle=boxfill,fillloopaddy=1](3.61,0)(4.39,500)
%   \psboxfill{\psset{unit=0.78\psxunit}\KangarooPstChart{green}}
%   \psframe[linestyle=none,fillstyle=boxfill,fillloopaddy=1](4.61,0)(5.39,200)
% ^^A   % Bottom labels
%   \uput{0.2}[270]{0}(1,0){\psscalebox{0.7}{Oceania}}
%   \uput{0.2}[270]{0}(2,0){\psscalebox{0.7}{Africa}}
%   \uput{0.2}[270]{0}(3,0){\psscalebox{0.7}{Asia}}
%   \uput{0.2}[270]{0}(4,0){\psscalebox{0.7}{America}}
%   \uput{0.2}[270]{0}(5,0){\psscalebox{0.7}{Europe}}
% ^^A   % Frame box around the chart
%   \psframe[linestyle=solid](0,0)(6,2000)
% \end{pspicture}}
%   \caption{Bar chart generated by PstChart, with bars filled by patterns}
%   \label{fig:PstChart}
% \end{figure}
% 
% \section{``Dynamic'' tilings}
% 
%   In some cases, tilings used non \emph{static} tiles, that is to say that the 
% \emph{prototile(s)}, even if unique, can have several forms, by instance
% specified by different colors or rotations, not fixed before generation or
% varying each time.
% 
% \subsection{Lewthwaite-Pickover-Truchet tiling}
% 
%   We give here for example the so-called \emph{Truchet} tiling, which much be
% in fact better called \emph{Lewthwaite-Pick\-over-Truchet (LPT)} tiling%
% \footnote{For description of the context, history and references about
% S\'ebastien \textsc{Truchet} and this tiling, see \cite{EsperetGirou98}.}.
% 
%   The unique prototile is only a square with two opposite circle arcs.
% This tile has obviously two positions, if we rotate it from 90 degrees (see
% the two tiles on the next figure). A \emph{LPT tiling} is a tiling with
% randomly oriented LPT tiles. We can see that even if it is very simple in it
% principle, it draw sophisticated curves with strange properties.
% 
%   Nevertheless, in the straightforward way \FillPackage{} does not work,
% because the \cs{psboxfill} macro store the content of the tile used in a
% \TeX{} box, which is static. So the calling to the random function is done
% only one time, which explain that only one rotation of the tile is used for
% all the tiling. It's only the one of the two rotations which could differ from
% one drawing to the next one...
% 
% ^^A % Truchet (Lewthwaite-Pickover-Truchet) tiling
% ^^A % --------------------------------------------
% 
% \begin{LTXexample}[pos=t]
% ^^A   % LPT prototile
%   \newcommand{\ProtoTileLPT}{%
%     \psset{dimen=middle}
%     \begin{pspicture}(1,1)
%       \psframe(1,1)
%       \psarc(0,0){0.5}{0}{90}
%       \psarc(1,1){0.5}{-180}{-90}
%     \end{pspicture}}
% 
% ^^A   % LPT tile
%   \newcount\Boolean
%   \newcommand{\BasicTileLPT}{%
% ^^A     % From random.tex by Donald Arseneau
%     \setrannum{\Boolean}{0}{1}%
%     \ifnum\Boolean=0
%       \ProtoTileLPT%
%     \else
%       \psrotateleft{\ProtoTileLPT}%
%     \fi}
% 
%   \ProtoTileLPT\hfill\psrotateleft{\ProtoTileLPT}\hfill
%   \psset{unit=0.5}
%   \psboxfill{\BasicTileLPT}
%   \Tiling{(5,5)}
% \end{LTXexample}
% 
%   But, for simple cases, there is a solution to this problem using a mixture
% of PSTricks and PostScript programming. Here the PSTricks
% construction \verb+\pscustom{\code{...}}+ allow to insert PostScript code
% inside the \LaTeX{} + PSTricks one.
% 
%   Programmation is less straightforward, but it has also the advantage to be
% notably faster, as all the tilings operations are done in PostScript, and
% mainly to not be limited by \TeX{} memory (the \TeX{} + PSTricks solution
% I wrote in 1995 for the colored problem was limited to small sizes for this
% reason). Just note also that \cs{pslbrace} and \cs{psrbrace} are two
% PSTricks macros to define and be able to insert the \verb+{+ and \verb+}+
% characters.
% 
% \begin{LTXexample}[pos=t]
% ^^A   % LPT prototile
%   \newcommand{\ProtoTileLPT}{%
%     \psset{dimen=middle}
%     \psframe(1,1)
%     \psarc(0,0){0.5}{0}{90}
%     \psarc(1,1){0.5}{-180}{-90}}
% 
% ^^A   % Counter to change the random seed
%   \newcount\InitCounter
% ^^A   % LPT tile
%   \newcommand{\BasicTileLPT}{%
%     \InitCounter=\the\time
%     \pscustom{\code{%
%       rand \the\InitCounter\space sub 2 mod 0 eq \pslbrace}}
%     \begin{pspicture}(1,1)
%       \ProtoTileLPT
%     \end{pspicture}%
%     \pscustom{\code{\psrbrace \pslbrace}}
%     \psrotateleft{\ProtoTileLPT}%
%     \pscustom{\code{\psrbrace ifelse}}}
% 
%   \psset{unit=0.4,linewidth=0.4pt}
%   \psboxfill{\BasicTileLPT}
%   \Tiling{(15,15)}
% \end{LTXexample}
% 
%   Using the very surprising fact (see \cite{EsperetGirou98}) that
% coloration of these tiles do not depend of their neighbors (even if it is
% difficult to believe as the opposite seems obvious!) but only of the parity of
% the value of row and column positions, we can directly program in the same way
% a colored version of the LPT tiling.
% 
% \setcounter{footnote}{1}
%   We have also introduce in the \FillPackage{} code for \emph{tiling} mode two
% new accessible Post\-Script variables, \texttt{row} and
% \texttt{column}\footnotemark, which can be useful in some circonstances, like
% this one.
% 
% \begin{LTXexample}[pos=t]
% ^^A   % LPT prototile
%   \newcommand{\ProtoTileLPT}[2]{%
%     \psset{dimen=middle,linestyle=none,fillstyle=solid}
%     \psframe[fillcolor=#1](1,1)
%     \psset{fillcolor=#2}
%     \pswedge(0,0){0.5}{0}{90} \pswedge(1,1){0.5}{-180}{-90}}
% ^^A   % Counter to change the random seed
%   \newcount\InitCounter
% ^^A   % LPT tile
%   \newcommand{\BasicTileLPT}[2]{%
%     \InitCounter=\the\time
%     \pscustom{\code{%
%       rand \the\InitCounter\space sub 2 mod 0 eq \pslbrace
%       row column add 2 mod 0 eq \pslbrace}}
%     \begin{pspicture}(1,1)\ProtoTileLPT{#1}{#2}\end{pspicture}%
%     \pscustom{\code{\psrbrace \pslbrace}}
%     \ProtoTileLPT{#2}{#1}%
%     \pscustom{\code{%
%       \psrbrace ifelse \psrbrace \pslbrace row column add 2 mod 0 eq \pslbrace}}
%     \psrotateleft{\ProtoTileLPT{#2}{#1}}\pscustom{\code{\psrbrace \pslbrace}}
%     \psrotateleft{\ProtoTileLPT{#1}{#2}}\pscustom{\code{\psrbrace ifelse \psrbrace ifelse}}}
%   \psboxfill{\BasicTileLPT{red}{yellow}}
%   \Tiling{(4,4)}\hfill
%   \psset{unit=0.4}\psboxfill{\BasicTileLPT{blue}{cyan}}
%   \Tiling{(15,15)}
% \end{LTXexample}
% 
%   Another classic example is to generate coordinates and numerotation for a
% grid. Of course, it is possible to do it directly in PSTricks using nested
% \cs{multido} commands. It would be clearly easy to program, but, nevertheless, 
% for users who have a little knowledge of PostScript programming, this offer
% an alternative which is useful for large cases, because on this way it will
% be notably faster and less computer ressources consuming.
% 
%   Remember here that the tiling is drawn from left to right, and top to
% bottom, and note that the PostScript variable \texttt{x2} give the total
% number of columns.
% 
% \begin{LTXexample}[pos=t]
% ^^A   % \Escape will be the \ character
%   {\catcode`\!=0\catcode`\\=11!gdef!Escape{\}}
%   \newcommand{\ProtoTile}{%
%     \Square\pscustom{%
%       \moveto(-0.9,0.75) % In PSTricks units
%       \code{ /Times-Italic findfont 8 scalefont setfont
%         (\Escape() show row 3 string cvs show (,) show 
%         column 3 string cvs show (\Escape)) show}
%       \moveto(-0.5,0.25) % In PSTricks units
%       \code{ /Times-Bold findfont 18 scalefont setfont
%         1 0 0 setrgbcolor % Red color
%         /center {dup stringwidth pop 2 div neg 0 rmoveto} def
%         row 1 sub x2 mul column add 3 string cvs center show}}}
%   \psboxfill{\ProtoTile}
%   \Tiling{(6,4)}
% \end{LTXexample}
% 
% \subsection{A complete example: the Poisson equation}
% 
%   To finish, we will show a complete real example, a drawing to explain the
% method used to solve the \textsc{Poisson} equation by a domain
% decomposition method, adapted to distributed memory computers. The
% objective is to show the communications required between processes and the
% position of the data to exchange. This code also show some useful and powerful
% technics for PSTricks programming (look specially at the way some higher level
% macros are defined, and how the same object is used to draw the four
% neighbors).
%
%\psset{unit=1cm}
%\newcommand{\Pattern}[1]{%
%   \begin{pspicture}(-0.25,-0.25)(0.25,0.25)\rput{*0}{\psdot[dotstyle=#1]}
%   \end{pspicture}}
%\newcommand{\West}{\Pattern{o}}   \newcommand{\South}{\Pattern{x}}
%\newcommand{\Central}{\Pattern{+}}\newcommand{\North}{\Pattern{square}}
%\newcommand{\East}{\Pattern{triangle}}
%\newcommand{\Cross}{%
%  \pspolygon[unit=0.5,linewidth=0.2,linecolor=red](0,0)(0,1)(1,1)(1,2)(2,2)(2,1)%
%              (3,1)(3,0)(2,0)(2,-1)(1,-1)(1,0)}
%\newcommand{\StylePosition}[1]{\LARGE\textcolor{red}{\textbf{#1}}}
%\newcommand{\SubDomain}[4]{%
%    \psboxfill{#4}\begin{psclip}{\psframe[linestyle=none]#1}%
%      \psframe[linestyle=#3](5,5)\psframe[fillstyle=boxfill]#2%
%    \end{psclip}}
%\newcommand{\SendArea}[1]{\psframe[fillstyle=solid,fillcolor=cyan]#1}
%\newcommand{\ReceiveData}[2]{%
%  \psboxfill{#2}\psframe[fillstyle=solid,fillcolor=yellow,addfillstyle=boxfill]#1}%
%\newcommand{\Neighbor}[2]{%
%    \begin{pspicture}(5,5)
%      \rput{*0}(2.5,2.5){\StylePosition{#1}}
%      \ReceiveData{(0.5,0)(4.5,0.5)}{\Central}\SendArea{(0.5,0.5)(4.5,1)}%
%      \SubDomain{(5,2)}{(0.5,0.5)(4.5,3)}{dashed}{#2}%
%      \pcarc[arcangle=45,arrows=->](0.5,-1.25)(0.5,0.25)%
%      \pcarc[arcangle=45,arrows=->,linestyle=dotted,dotsep=2pt](4.5,0.75)(4.5,-0.75)%
%    \end{pspicture}}%
%  \psset{dimen=middle,dotscale=2,fillloopadd=2}
%\begin{pspicture}(-5.7,-5.7)(5.7,5.7)
%  \rput(0,0){%
%      \begin{pspicture}(5,5)
%        \ReceiveData{(0,0.5)(0.5,4.5)}{\West} \ReceiveData{(4.5,0.5)(5,4.5)}{\East}
%        \ReceiveData{(0.5,4.5)(4.5,5)}{\North}\ReceiveData{(0.5,0)(4.5,0.5)}{\South}
%        \SendArea{(0.5,0.5)(1,4.5)}\SendArea{(4,0.5)(4.5,4.5)}
%        \SendArea{(0.5,0.5)(4.5,1)}\SendArea{(0.5,4)(4.5,4.5)}
%        \SubDomain{(5,5)}{(0.5,0.5)(4.5,4.5)}{solid}{\Central}
%        \psline(1,0.5)(1,4.5)\psline(4,0.5)(4,4.5)%
%        \rput(1.5,4){\Cross}\rput(2,2){\Cross}%
%      \end{pspicture}}%
%  \rput(0,5.5){\Neighbor{N}{\North}}\rput{-90}(5.5,0){\Neighbor{E}{\East}}%
%  \rput{90}(-5.5,0){\Neighbor{W}{\West}}\rput{180}(0,-5.5){\Neighbor{S}{\South}}%
%\end{pspicture}
%
% \begin{lstlisting}
%   \newcommand{\Pattern}[1]{%
%     \begin{pspicture}(-0.25,-0.25)(0.25,0.25)\rput{*0}{\psdot[dotstyle=#1]}
%     \end{pspicture}}
%   \newcommand{\West}{\Pattern{o}}   \newcommand{\South}{\Pattern{x}}
%   \newcommand{\Central}{\Pattern{+}}\newcommand{\North}{\Pattern{square}}
%   \newcommand{\East}{\Pattern{triangle}}
%   \newcommand{\Cross}{%
%     \pspolygon[unit=0.5,linewidth=0.2,linecolor=red](0,0)(0,1)(1,1)(1,2)(2,2)(2,1)
%               (3,1)(3,0)(2,0)(2,-1)(1,-1)(1,0)}
%   \newcommand{\StylePosition}[1]{\LARGE\textcolor{red}{\textbf{#1}}}
%   \newcommand{\SubDomain}[4]{%
%     \psboxfill{#4}
%     \begin{psclip}{\psframe[linestyle=none]#1}
%       \psframe[linestyle=#3](5,5)\psframe[fillstyle=boxfill]#2
%     \end{psclip}}
%   \newcommand{\SendArea}[1]{\psframe[fillstyle=solid,fillcolor=cyan]#1}
%   \newcommand{\ReceiveData}[2]{%
%     \psboxfill{#2}
%     \psframe[fillstyle=solid,fillcolor=yellow,addfillstyle=boxfill]#1}
%   \newcommand{\Neighbor}[2]{%
%     \begin{pspicture}(5,5)
%       \rput{*0}(2.5,2.5){\StylePosition{#1}}
%       \ReceiveData{(0.5,0)(4.5,0.5)}{\Central}\SendArea{(0.5,0.5)(4.5,1)}
%       \SubDomain{(5,2)}{(0.5,0.5)(4.5,3)}{dashed}{#2}%
% ^^A       % Receive and send arrows
%       \pcarc[arcangle=45,arrows=->](0.5,-1.25)(0.5,0.25)
%       \pcarc[arcangle=45,arrows=->,linestyle=dotted,dotsep=2pt](4.5,0.75)(4.5,-0.75)
%     \end{pspicture}}
%   \psset{dimen=middle,dotscale=2,fillloopadd=2}
%   \begin{pspicture}(-5.7,-5.7)(5.7,5.7)
% ^^A     % Central domain
%     \rput(0,0){%
%       \begin{pspicture}(5,5)
% ^^A         % Receive from West, East, North and South
%         \ReceiveData{(0,0.5)(0.5,4.5)}{\West} \ReceiveData{(4.5,0.5)(5,4.5)}{\East}
%         \ReceiveData{(0.5,4.5)(4.5,5)}{\North}\ReceiveData{(0.5,0)(4.5,0.5)}{\South}
% ^^A         % send area for West, East, North and South
%         \SendArea{(0.5,0.5)(1,4.5)} \SendArea{(4,0.5)(4.5,4.5)}
%         \SendArea{(0.5,0.5)(4.5,1)} \SendArea{(0.5,4)(4.5,4.5)}
% ^^A         % Central domain
%         \SubDomain{(5,5)}{(0.5,0.5)(4.5,4.5)}{solid}{\Central}
% ^^A         % Redraw overlapped linesY
%         \psline(1,0.5)(1,4.5)  \psline(4,0.5)(4,4.5)
% ^^A         % Two crossesY
%         \rput(1.5,4){\Cross}  \rput(2,2){\Cross}
%       \end{pspicture}}
% ^^A     % The four neighborsY
%     \rput(0,5.5){\Neighbor{N}{\North}}     \rput{-90}(5.5,0){\Neighbor{E}{\East}}
%     \rput{90}(-5.5,0){\Neighbor{W}{\West}} \rput{180}(0,-5.5){\Neighbor{S}{\South}}
%   \end{pspicture}
% \end{lstlisting}
%
%
%
% Bibliography
% \begin{thebibliography}{99}
% \bibitem{PostScript95} Adobe, Systems~Incorporated, \emph{PostScript Language
% Reference Manual}, Addison-Wesley, 2~edition, 1995.
%
% \bibitem{Bolek98} Piotr Bolek, \MP{} and patterns, \emph{\TUB}, Volume~19,
% Number~3, pages 276--283, September 1998, \CTANref{mpattern}.
%
% \bibitem{MLgraphTSI} Emmanuel Chailloux, Guy Cousineau and Asc\'ander
% Su\'arez, Programmation fonctionnelle de graphismes pour la production
% d'illustrations techniques, \emph{Technique et science informatique},
% Volume~15, Number~7, pages 977--1007, 1996 (in french).
%
% \bibitem{Deledicq97} Andr\'e Deledicq, \emph{Le monde des pavages}, ACL
% \'Editions, 1997 (in french).
%
% \bibitem{EsperetGirou98} Philippe Esperet and Denis Girou,
% Coloriage du pavage dit de Truchet, Cahiers GUTenberg, Number~31,
% pages 5--18, December~1998  (in french).
%
% \bibitem{Girou94} Denis Girou, Pr\'esentation de PSTricks, \emph{Cahiers
% GUTenberg}, Number~16, pages 21--70, February~1994 (in french).
%
% \bibitem{LGC97} Michel Goossens, Sebastian Rahtz and Frank Mittelbach,
% \emph{The \LaTeX{} Graphics Companion}, Addison-Wesley, 2005.
%
% \bibitem{GS87} Branko Gr\"unbaum and Geoffrey Shephard, \emph{Tilings and
% Patterns}, Freeman and Company, 1987.
%
% \bibitem{Hoenig97} Alan Hoenig, \emph{\TeX{} Unbound: \LaTeX{} \& \TeX{}
% Strategies, Fonts, Graphics, and More}, Oxford University Press, 1997.
%
% \bibitem{XYpic} Kristoffer~H. Rose and Ross Moore, \XYpic. Pattern and Tile
% extension, available from \CTAN, 1991-1998, \CTANref{xypic}.
%
% \bibitem{LAAN96} Kees van der Laan, Paradigms: Just a little bit of PostScript,
% \emph{MAPS}, Volume~17, pages 137--150, 1996.
%
% \bibitem{LAAN97} Kees van der Laan, Tiling in PostScript and \MF{} -- Escher's
% wink, \emph{MAPS}, Volume~19, Number~2, pages 39--67, 1997.
%
% \bibitem{vanZandt93} Timothy Van Zandt, PSTricks. PostScript macros for
% Generic \TeX, available from \CTAN, 1993, \CTANref{pstricks}.
%
% \bibitem{vanZandtGirou94} Timothy Van Zandt and Denis Girou, Inside PSTricks,
% \emph{\TUB}, Volume~15, Number~3, pages 239--246, September 1994.
%
%
% \bibitem{voss07} Herbert Vo\ss, PSTricks -- Graphics for \TeX\ and \LaTeX, DANTE/Lehmanns, 4th ed., 2007.
% \bibitem{Wang65} Hao Wang, Games, Logic and Computers, \emph{Scientific
% American}, pages 98--106, November 1965.
% \end{thebibliography}
%
%
% \StopEventually{}
%
% ^^A .................... End of the documentation part ....................
%
% \section{Driver file}
%
%   The next bit of code contains the documentation driver file for \TeX{},
% i.e., the file that will produce the documentation you are currently
% reading. It will be extracted from this file by the \texttt{docstrip}
% program.
%
%    \begin{macrocode}
%<*driver>
\documentclass{ltxdoc}
\GetFileInfo{pst-fill.dtx}
%
\usepackage[T1]{fontenc}
\usepackage{lmodern}               % For PDF
\usepackage{graphicx}              % `graphicx' LaTeX standard package
\usepackage{showexpl}
\usepackage{mflogo}                % For the MetaFont and MetaPost logos
\input{random.tex}                 % Random macros from Donald Arseneau
\usepackage{url}                   % URLs convenient typesetting
\usepackage{multido}               % General loop macro
\usepackage[dvipsnames]{pstricks}  % PSTricks with the `color' extension
\usepackage{pst-text}              % PSTricks package for character path
\usepackage{pst-grad}              % PSTricks package for gradient filling
\usepackage{pst-node}              % PSTricks package for nodes
\usepackage[tiling]{pst-fill}      % PSTricks package for filling/tiling
%
\AtBeginDocument{%
%  \OnlyDescription % comment out for implementation details
  \EnableCrossrefs
  \CodelineIndex
  \RecordChanges}
\AtEndDocument{%
  \PrintIndex
  \setcounter{IndexColumns}{1}
  \PrintChanges}
\hbadness=7000            % Over and under full box warnings
\hfuzz=3pt
\begin{document}
  \DocInput{pst-fill.dtx}
\end{document}
%</driver>
%    \end{macrocode}
%
% \section{\texttt{pst-fill} \LaTeX{} wrapper}
%
%    \begin{macrocode}
%<*latex-wrapper>
\RequirePackage{pstricks}
\ProvidesPackage{pst-fill}[2005/09/13 package wrapper for 
  pst-fill.tex (hv)]
\DeclareOption{tiling}{\def\PstTiling{true}}
\ProcessOptions\relax
\input{pst-fill.tex}
\ProvidesFile{pst-fill.tex}
  [\filedate\space v\fileversion\space `PST-fill' (tvz,dg)]
%</latex-wrapper>
%    \end{macrocode}
%
%
% \section{Pst-Fill Package{} code}
%
%    \begin{macrocode}
%<*pst-fill>
%    \end{macrocode}
%
% \subsection{Preamble}
%
%   Who we are.
%
%    \begin{macrocode}
\def\fileversion{1.01}
\def\filedate{2007/03/10}
\message{`PST-Fill' v\fileversion, \filedate\space (tvz,dg,hv)}
\csname PSTboxfillLoaded\endcsname
\let\PSTboxfillLoaded\endinput
%    \end{macrocode}
%
%   Require the main PSTricks package.
%
%    \begin{macrocode}
\ifx\PSTricksLoaded\endinput\else\input pstricks.tex\fi
%    \end{macrocode}
%
%   interface to the extended `\textsf{keyval}' package.
%
%    \begin{macrocode}
\ifx\PSTXKeyLoaded\endinput\else\input pst-xkey\fi
%
%    \end{macrocode}
%
%   Catcodes changes and defining the family name for xkeyval.
%
%    \begin{macrocode}
\edef\PstAtCode{\the\catcode`\@}\catcode`\@=11\relax

\pst@addfams{pst-fill}
%
%    \end{macrocode}
%
%
% \subsection{The size of the box}
% \begin{macro}{pst@@boxfillsize}
%    \begin{macrocode}
%
\def\pst@@boxfillsize#1(#2,#3)#4(#5,#6)#7(#8\@nil{%
  \begingroup
    \ifx\@empty#7\relax
      \pst@dima\z@
      \pst@dimb\z@
      \pssetxlength\pst@dimc{#2}%
      \pssetylength\pst@dimd{#3}%
    \else
      \pssetxlength\pst@dima{#2}%
      \pssetylength\pst@dimb{#3}%
      \pssetxlength\pst@dimc{#5}%
      \pssetylength\pst@dimd{#6}%
    \fi
    \xdef\pst@tempg{%
      \pst@dima=\number\pst@dima sp
      \pst@dimb=\number\pst@dimb sp
      \pst@dimc=\number\pst@dimc sp
      \pst@dimd=\number\pst@dimd sp }%
  \endgroup
  \let\psk@boxfillsize\pst@tempg}
%    \end{macrocode}
% \end{macro}
%

% \subsection{Definition of the parameters}
%
%    \begin{macrocode}
\define@key[psset]{pst-fill}{boxfillsize}{%
  \def\pst@tempg{#1}\def\pst@temph{auto}%
  \ifx\pst@tempg\pst@temph
    \let\psk@boxfillsize\relax
  \else
    \pst@@boxfillsize#1(\z@,\z@)\@empty(\z@,\z@)(\@nil
  \fi}
\psset{boxfillsize={(-15cm,-15cm)(15cm,15cm)}}
\define@key[psset]{pst-fill}{boxfillcolor}{\pst@getcolor{#1}\psboxfillcolor}
\psset{boxfillcolor=black}% hv
\define@key[psset]{pst-fill}{boxfillangle}{\pst@getangle{#1}\psk@boxfillangle}
\psset{boxfillangle=0}
\define@key[psset]{pst-fill}{fillsepx}{%
  \pst@getlength{#1}\psk@fillsepx}
\define@key[psset]{pst-fill}{fillsepy}{%
  \pst@getlength{#1}\psk@fillsepy}
\define@key[psset]{pst-fill}{fillsep}{%
  \pst@getlength{#1}\psk@fillsepx%
  \let\psk@fillsepy\psk@fillsepx}
\psset{fillsep=2pt}

\ifx\PstTiling\@undefined
  \define@key[psset]{pst-fill}{fillcycle}{\pst@getint{#1}\psk@fillcycle}
  \psset{fillcycle=0}
\else
  \define@key[psset]{pst-fill}{fillangle}{\pst@getangle{#1}\psk@boxfillangle}
  \define@key[psset]{pst-fill}{fillsize}{%
      \def\pst@tempg{#1}\def\pst@temph{auto}%
      \ifx\pst@tempg\pst@temph\let\psk@boxfillsize\relax
      \else\pst@@boxfillsize#1(\z@,\z@)\@empty(\z@,\z@)(\@nil\fi}
  \psset{fillsep=0,fillsize=auto}
  \define@key[psset]{pst-fill}{fillcyclex}{\pst@getint{#1}\psk@fillcyclex}
  \define@key[psset]{pst-fill}{fillcycley}{\pst@getint{#1}\psk@fillcycley}
  \define@key[psset]{pst-fill}{fillcycle}{%
    \pst@getint{#1}\psk@fillcyclex\let\psk@fillcycley\psk@fillcyclex}
  \psset{fillcycle=0}
  \define@key[psset]{pst-fill}{fillmovex}{\pst@getlength{#1}\psk@fillmovex}
  \define@key[psset]{pst-fill}{fillmovey}{\pst@getlength{#1}\psk@fillmovey}
  \define@key[psset]{pst-fill}{fillmove}{%
      \pst@getlength{#1}\psk@fillmovex\let\psk@fillmovey\psk@fillmovex}
  \psset{fillmove=0pt}
  \define@key[psset]{pst-fill}{fillloopaddx}{\pst@getint{#1}\psk@fillloopaddx}
  \define@key[psset]{pst-fill}{fillloopaddy}{\pst@getint{#1}\psk@fillloopaddy}
  \define@key[psset]{pst-fill}{fillloopadd}{%
    \pst@getint{#1}\psk@fillloopaddx\let\psk@fillloopaddy\psk@fillloopaddx}
  \psset{fillloopadd=0}
%    \end{macrocode}
%
%    \begin{macrocode}
% For debugging (to debug, set PstDebug=1)
% we now use the one from pstricks to prevent a clash with package
% pstricks                        2004-06-22
%%    \define@key[psset]{pst-fill}{PstDebug}{\pst@getint{#1}\psk@PstDebug}
    \psset{PstDebug=0}
\fi
% DG addition end
%    \end{macrocode}

% \subsection{Definition of the fill box}
% \begin{macro}{psboxfill}
%    \begin{macrocode}
\newbox\pst@fillbox
\def\psboxfill{\pst@killglue\pst@makebox\psboxfill@i}
\def\psboxfill@i{\setbox\pst@fillbox\box\pst@hbox\ignorespaces}
%    \end{macrocode}
% \end{macro}
% \subsection{The main macros}
%
% \begin{macro}{psfs@boxfill}
%    \begin{macrocode}
\def\psfs@boxfill{%
  \ifvoid\pst@fillbox
    \@pstrickserr{Fill box is empty. Use \string\psboxfill\space first.}\@ehpa
  \else
    \ifx\psk@boxfillsize\relax \pst@AutoBoxFill
    \else\pst@ManualBoxFill\fi
  \fi}
%    \end{macrocode}
% \end{macro}
%
% \begin{macro}{pst@ManualBoxFill}
%    \begin{macrocode}
\def\pst@ManualBoxFill{%
  \leavevmode
  \begingroup
    \pst@FlushCode
    \begin@psclip
    \pstVerb{clip}%
    \expandafter\pst@AddFillBox\psk@boxfillsize
    \end@psclip
  \endgroup}
%    \end{macrocode}
% \end{macro}
%
% \begin{macro}{pst@FlushCode}
%    \begin{macrocode}
\def\pst@FlushCode{%
  \pst@Verb{%
    /mtrxc CM def
    CP CP T
    \tx@STV
    \psk@origin
    \psk@swapaxes
    \pst@newpath
    \pst@code
    mtrxc setmatrix
    moveto
    0 setgray}%
  \gdef\pst@code{}}
%    \end{macrocode}
% \end{macro}
%
% \begin{macro}{pst@AddFillBox}
%    \begin{macrocode}
\def\pst@AddFillBox#1 #2 #3 #4 {%
  \begingroup
    \setbox\pst@fillbox=\vbox{%
      \hbox{\unhcopy\pst@fillbox\kern\psk@fillsepx\p@}%
      \vskip\psk@fillsepy\p@}%
    \psk@boxfillsize
    \pst@cnta=\pst@dimc
    \advance\pst@cnta-\pst@dima
    \divide\pst@cnta\wd\pst@fillbox
    \pst@cntb=\pst@dimd
    \advance\pst@cntb-\pst@dimb
    \pst@dimd=\ht\pst@fillbox
    \divide\pst@cntb\pst@dimd
    \def\pst@tempa{%
      \pst@tempg
      \copy\pst@fillbox
      \advance\pst@cntc\@ne
      \ifnum\pst@cntc<\pst@cntd\expandafter\pst@tempa\fi}%
    \let\pst@tempg\relax
    \pst@cntc-\tw@
    \pst@cntd\pst@cnta
    \setbox\pst@fillbox=\hbox to \z@{%
      \kern\pst@dima
      \kern-\wd\pst@fillbox
      \pst@tempa
      \hss}%
    \pst@cntd\pst@cntb
%% DG modification begin - Dec. 11, 1997 - Patch 2
    \ifx\PstTiling\@undefined
      \ifnum\psk@fillcycle=\z@\pst@ManualFillCycle\fi
    \else
      \ifnum\psk@fillcyclex=\z@\pst@ManualFillCycle\fi
    \fi
%% DG modification end
    \global\setbox\pst@boxg=\vbox to\z@{%
      \offinterlineskip
      \vss
      \pst@tempa
      \vskip\pst@dimb}%
  \endgroup
  \setbox\pst@fillbox\box\pst@boxg
  \pst@rotate\psk@boxfillangle\pst@fillbox
  \box\pst@fillbox}
%    \end{macrocode}
% \end{macro}
%
% \begin{macro}{pst@ManualFillCycle}
%    \begin{macrocode}
\def\pst@ManualFillCycle{%
  \ifx\PstTiling\@undefined
    \pst@cntg=\psk@fillcycle
  \else
    \pst@cntg=\psk@fillcyclex
  \fi
  \pst@dimg=\wd\pst@fillbox
  \ifnum\pst@cntg=\z@
  \else
  \divide\pst@dimg\pst@cntg
  \fi
  \ifnum\pst@cntg<\z@\pst@cntg=-\pst@cntg\fi
  \advance\pst@cntg\m@ne
  \pst@cnth=\pst@cntg
  \def\pst@tempg{%
    \ifnum\pst@cnth<\pst@cntg\advance\pst@cnth\@ne\else\pst@cnth\z@\fi
    \moveright\pst@cnth\pst@dimg}}
%    \end{macrocode}
% \end{macro}
%
%% Auto box fill:        !! Fix dictionary
%
% \subsection{The PostScript subroutines}
%
%    \begin{macrocode}
%% DG addition begin - Apr. 8, 1997 and Dec. 1997 - Patch 2
\ifx\PstTiling\@undefined
\pst@def{AutoFillCycle}<%
  /c ED
  /n 0 def
  /s {
    /x x w c div n mul add def
    /n n c abs 1 sub lt { n 1 add } { 0 } ifelse def
  } def>

\pst@def{BoxFill}<%
  gsave
    gsave \tx@STV CM grestore dtransform CM idtransform
    abs /h ED abs /w ED
    pathbbox
    h div round 2 add cvi /y2 ED
    w div round 2 add cvi /x2 ED
    h div round 2 sub cvi /y1 ED
    w div round 2 sub cvi /x1 ED
    /y2 y2 y1 sub def
    /x2 x2 x1 sub def
    CP
    y1 h mul sub neg /y1 ED
    x1 w mul sub neg /x1 ED
    clip
    y2 {
      /x x1 def
      s
      x2 {
        save CP x y1
%% patch 4   hv --------------
        \ifx\VTeXversion\undefined
        \else
%%============ mv: 09-10-01 ??? this is likely to be a right change
        neg
%%============
        \fi
%% end patch 4
T moveto Box restore
        /x x w add def
      } repeat
      /y1 y1 h add def
    } repeat
    % Next line not useful... To see that, suppress clipping (DG)
    CP x y1 T moveto Box
  currentpoint currentfont grestore setfont moveto>
\else
%% DG modification begin - Apr. 8, 1997 and Nov. / Dec. 1997 - Patch 2
\pst@def{AutoFillCycleX}<%
  /cX ED
  /nX 0 def
  /CycleX {
    /x x w cX div nX mul add def
    /nX nX cX abs 1 sub lt { nX 1 add } { 0 } ifelse def
  } def>
\pst@def{AutoFillCycleY}<%
  /cY ED
  /mY 0 def
  /nY 0 def
  /CycleY {
    /y1 y1 h cY div mY mul sub def
    nY cY abs 1 sub lt { /nY nY 1 add def /mY 1 def }
                       { /nY 0 def        /mY cY abs 1 sub neg def } ifelse
  } def>

\pst@def{BoxFill}<%
  gsave
    gsave \tx@STV CM grestore dtransform CM idtransform
    abs /h ED abs /w ED
    pathbbox
    h div round 2 add cvi /y2 ED
    w div round 2 add cvi /x2 ED
    h div round 2 sub cvi /y1 ED
    w div round 2 sub cvi /x1 ED
    /CoefLoopX 0 def
    /CoefLoopY 0 def
    /CoefMoveX 0 def
    /CoefMoveY 0 def
    \psk@boxfillangle\space 0 ne {/CoefLoopX 8 def /CoefLoopY 8 def} if
    \psk@fillcyclex\space 0 ne {/CoefLoopX CoefLoopX 1 add def} if
    \psk@fillcycley\space 0 ne {/CoefLoopY CoefLoopY 1 add def} if
    \psk@fillmovex\space 0 ne
      {/CoefLoopX CoefLoopX 2 add def
       \psk@fillmovex\space 0 gt {/CoefMoveX CoefLoopX def}
                           {/CoefMoveX CoefLoopX neg def} ifelse} if
    \psk@fillmovey\space 0 ne
      {/CoefLoopY CoefLoopY 2 add def
       \psk@fillmovey\space 0 gt {/CoefMoveY CoefLoopY def}
                           {/CoefMoveY CoefLoopY neg def} ifelse} if
    \psk@fillsepx\space 0 ne {/CoefLoopX CoefLoopX 1 add def} if
    \psk@fillsepy\space 0 ne {/CoefLoopY CoefLoopY 1 add def} if
    /CoefLoopX CoefLoopX \psk@fillloopaddx\space add def
    /CoefLoopY CoefLoopY \psk@fillloopaddy\space add def
    /x2 x2 x1 sub 4 sub CoefLoopX 2 mul add def
    /y2 y2 y1 sub 4 sub CoefLoopY 2 mul add def
%% We must fix the origin of tiling, as it must not vary according other stuff
%% in the page!
    w x1 CoefLoopX add CoefMoveX add mul
      h y1 y2 add 1 sub CoefLoopY sub CoefMoveY sub mul moveto
    CP
    y1 h mul sub neg /y1 ED
    x1 w mul sub neg /x1 ED
%%  hv 2004-06-22   to prevent clash with pst-gr3d
%%    \psk@PstDebug 0 eq {clip} if
    \Pst@Debug 0 eq {clip} if
%% end hv
    \psk@fillmovex\space \psk@fillmovey
    gsave \tx@STV CM grestore dtransform CM idtransform
    /hmove ED /wmove ED
    /row 0 def
   y2 {
       /row row 1 add def
       /column 0 def
       /x x1 def
       CycleX
       save
       x2 {
          /column column 1 add def
          CycleY
          save CP x y1
%% patch 4   hv --------------
          \ifx\VTeXversion\undefined
          \else
%%============ mv: 09-10-01 ??? this is likely to be a right change
          neg
%%============
          \fi
  T moveto Box restore
          /x x w add def
          0 hmove translate
          } repeat
       restore
       /y1 y1 h add def
       wmove 0 translate
       } repeat
  currentpoint currentfont grestore setfont moveto>
\fi
%    \end{macrocode}

%    \begin{macrocode}
\def\pst@AutoBoxFill{%
  \leavevmode
  \begingroup
    \pst@stroke
    \pst@FlushCode
    \pst@Verb{\psk@boxfillangle\space \tx@RotBegin}%
    \pstVerb{\pst@dict /Box \pslbrace end}%
    \ifx\PstTiling\@undefined
    \else
      \ifx\pst@tempa\@undefined % Undefined for instance for \pscharpath
      \else\ifx\pst@tempa\@empty\else
        \def\pst@temph{0}%
        \ifx\pst@tempa\pst@temph
        \else
          \pstVerb{/TR {pop pop currentpoint translate \pst@tempa\space translate } def}%
        \fi
      \fi\fi
    \fi
    \hbox to \z@{\vbox to\z@{\vss\copy\pst@fillbox\vskip-\dp\pst@fillbox}\hss}%
    \ifx\PstTiling\@undefined
      \pstVerb{%
        tx@Dict begin \psrbrace def
        \ifnum\psk@fillcycle=\z@
          /s {} def
        \else
          \psk@fillcycle \tx@AutoFillCycle
        \fi
        \pst@number{\wd\pst@fillbox}%
        \psk@fillsepx\space add
        \pst@number{\ht\pst@fillbox}%
        \pst@number{\dp\pst@fillbox}%
        \psk@fillsepy\space add add
        \tx@BoxFill
        end}%
      \else
      \pstVerb{%
        tx@Dict begin \psrbrace def
        \ifnum\psk@fillcyclex=\z@
          /CycleX {} def
        \else
          \psk@fillcyclex\space \tx@AutoFillCycleX
        \fi
        \ifnum\psk@fillcycley=\z@
          /CycleY {} def
        \else
          \psk@fillcycley\space \tx@AutoFillCycleY
        \fi
        \pst@number{\wd\pst@fillbox}%
        \psk@fillsepx\space add
        \pst@number{\ht\pst@fillbox}%
        \pst@number{\dp\pst@fillbox}%
        \psk@fillsepy\space add add
        \tx@BoxFill
        end}%
    \fi
    \pst@Verb{\tx@RotEnd}%
  \endgroup}
%    \end{macrocode}
% \subsection{Closing}
%
%   Catcodes restoration.
%
%    \begin{macrocode}
\catcode`\@=\PstAtCode\relax
%    \end{macrocode}
%
%    \begin{macrocode}
%</pst-fill>
%    \end{macrocode}
%
% \Finale
%
\endinput
%%
%% End of file `pst-fill.dtx'
+\newline
%add the following definition:\newline
%\verb+\def\PstTiling{true}+
%
%  To obtain the original behaviour, just don't use the \emph{tiling} optional
%keyword at loading.
%
%  Take care than in \emph{tiling} mode, I introduce also some other changes.
%First I define aliases on some parameter names for consistancy (all specific
%parameters will begin by the \texttt{fill} prefix in this case) and I change
%some default values, which were not well adapted for tilings (\texttt{fillsep}
%is set to 0 and as explained \texttt{fillsize} set to \texttt{auto}). I rename 
%\texttt{fillcycle} to \texttt{fillcyclex}. I also restore normal way so that
%the frame of the area is drawn and all line (\texttt{linestyle},
%\texttt{linecolor}, \texttt{doubleline}, etc.) parameters are now active (but
%there are not in non \emph{tiling} mode). And I also introduce new parameters
%to control the tilings (see below).
%
%  \textbf{In all the following examples, we will consider only the
% \emph{tiling} mode.}
%
%  To do a tiling, we have just to define the pattern with the
% \verb+\psboxfill+ macro and to use the new \texttt{fillstyle}
% \verb+boxfill+.
%
%  Note that tilings are drawn from left to right and top to bottom, which can
%have an importance in some circonstances.
%
%  PostScript programmers can be also interested to know that, even in the
%\emph{automatic} mode, the iterations of the pattern are managed directly by
%the PostScript code of the package which used only PostScript Level 1
%operators. The special ones introduced in Level 2 for drawing of patterns
%\cite[section 4.9]{PostScript95} are not used.
%
%  And first, for conveniance, we define a simple \cs{Tiling} macro, which
%will simplify our examples:
%
%\begin{verbatim}
%  \newcommand{\Tiling}[2][]{%
%    \edef\Temp{#1}%
%    \begin{pspicture}#2
%      \ifx\Temp\empty
%        \psframe[fillstyle=boxfill]#2
%      \else
%        \psframe[fillstyle=boxfill,#1]#2
%      \fi
%    \end{pspicture}}
%\end{verbatim}
%
%
%\newcommand{\Tiling}[2][]{%
%  \edef\Temp{#1}%
%  \begin{pspicture}#2
%    \ifx\Temp\empty
%      \psframe[fillstyle=boxfill]#2
%    \else
%      \psframe[fillstyle=boxfill,#1]#2
%    \fi
% \end{pspicture}}
%
%\subsection{Parameters}
%
%  There are \textbf{14} specific parameters available to change the way the
% filling/tiling is defined, and one debugging option.
%
% \begin{Description}{2cm}
%  \item [fillangle (real)\hfill :] the value of the rotation
%  applied to the patterns (\emph{Default:~0}).
% \end{Description}
%
%
%   In this case, we must force the tiling area to be notably larger than the
% area to cover, to be sure that the defined area will be covered after rotation.
% \lstset{gobble=2}
% \begin{LTXexample}
% \newcommand{\Square}{%
%   \begin{pspicture}(1,1)
%     \psframe[dimen=middle](1,1)
%   \end{pspicture}}
% \psset{unit=0.5}
% \psboxfill{\Square}
% \Tiling[fillangle=45]{(3,3)}\quad
% \Tiling[fillangle=-60]{(3,3)}
% \end{LTXexample}
% 
% \newcommand{\Square}{\begin{pspicture}(1,1)\psframe[dimen=middle](1,1)\end{pspicture}}
% 
% \begin{Description}{2cm}
%   \setcounter{footnote}{1}
%   \item[\texttt{fillsepx} (real$\|$dim) :] value of the horizontal
%   separation between consecutive patterns (\emph{Default:~0 for
%   tilings\footnotemark, 2pt otherwise}).  \footnotetext{This option was added
%   by me, is not part of the original package and is available only if the
%   \texttt{tiling} keyword is used when loading the package.}
%   \setcounter{footnote}{1}
%   \item [\texttt{fillsepy} (real$\|$dim)\hfill :] value of the vertical
%   separation between consecutive patterns (\emph{Default:~0 for
%   ti\-lings\footnotemark, 2pt otherwise}).
%   \setcounter{footnote}{1}
%   \item [\texttt{fillsep} (real$\|$dim)\hfill :] value of horizontal and
%   vertical separations between consecutive patterns (\emph{Default:~0 for
%   tilings\footnotemark, 2pt otherwise}).
% \end{Description}
% 
%   These values can be negative, which allow the tiles to overlap.
% 
% \begin{LTXexample}
% \psset{unit=0.5}
% \psboxfill{\Square}
% \Tiling[fillsepx=2mm]{(3,3)} 
% \Tiling[fillsepy=1mm]{(3,3)}\\
% \Tiling[fillsep=0.5]{(3,3)} 
% \Tiling[fillsep=-0.5]{(3,3)}
% \end{LTXexample}
% 
% \begin{Description}{2cm}
%   \item [\texttt{fillcyclex}\footnotemark\ (integer)\hfill :] Shift
%   coefficient applied to each row (\emph{Default:~0}).
%   \footnotetext{It was \texttt{fillcycle} in the original version.}
%   \setcounter{footnote}{1}
%   \item [\texttt{fillcycley}\footnotemark\ (integer)\hfill :] Same thing for
%   columns (\emph{Default:~0}).
%   \setcounter{footnote}{1}
%   \item [\texttt{fillcycle}\footnotemark\ (integer)\hfill :] Allow to fix
%   both \texttt{fillcyclex} and \texttt{fillcycley} directly to the same value
%   (\emph{Default:~0}).
% \end{Description}
% 
%   For instance, if \texttt{fillcyclex} is 2, the second row of patterns will
% be horizontally shifted by a factor of $\frac{1}{2}=0.5$, and by a factor of
% 0.333 if \texttt{fillcyclex} is 3, etc.). These values can be negative.
% 
% \begin{LTXexample}[width=0.35\linewidth]
% \psset{unit=0.5}
% \psboxfill{\Square}
% \newcommand{\TilingA}[1]{\Tiling[fillcyclex=#1]{(3,3)}}
% \TilingA{0} \TilingA{1}\\
% \TilingA{2} \TilingA{3}\\[3mm]
% \TilingA{4} \TilingA{5}\\
% \TilingA{6} \TilingA{-3}\\[3mm]
% \Tiling[fillcycley=2]{(3,3)}
% \Tiling[fillcycley=3]{(3,3)}\\
% \Tiling[fillcycley=-3]{(3,3)}
% \Tiling[fillcycle=2]{(3,3)}
% \end{LTXexample}
% 
% \begin{Description}{2cm}
%   \setcounter{footnote}{1}
%   \item [\texttt{fillmovex}\footnotemark\ (real$\|$dim)\hfill :] value of the
%   horizontal moves between consecutive patterns (\emph{Default:~0}).
%   \setcounter{footnote}{1}
%   \item [\texttt{fillmovey}\footnotemark\ (real$\|$dim)\hfill :] value of the
%   vertical moves between consecutive patterns (\emph{Default:~0}).
%   \setcounter{footnote}{1}
%   \item [\texttt{fillmove}\footnotemark\ (real$\|$dim)\hfill :] value of
%   horizontal and vertical moves between consecutive patterns
%   (\emph{Default:~0}).
% \end{Description}
% 
%   These parameters allow the patterns to overlap and to draw some special
% kinds of tilings. They are implemented only for the \emph{automatic} and
% \emph{tiling} modes and their values can be negative.
% 
%   In some cases, the effect of these parameters will be the same that with the 
% \texttt{fillcycle?} ones, but you can see that it is not true for some other
% values.
% 
% \begin{LTXexample}
% \psset{unit=0.5}
% \psboxfill{\Square}
% \Tiling[fillmovex=0.5]{(3,3)} 
% \Tiling[fillmovey=0.5]{(3,3)}\\
% \Tiling[fillmove=0.5]{(3,3)}
% \Tiling[fillmove=-0.5]{(3,3)}
% \end{LTXexample}
% 
% \begin{Description}{2cm}
%   \item [\texttt{fillsize}
%   (auto$\|$\{(real$\|$dim,real$\|$dim)(real$\|$dim,real$\|$dim)\}) :] The
%   choice of \emph{automatic} mode or the size of the area in \emph{manual}
%   mode. If first pair values are not given, (0,0) is used. (\emph{Default:
%   auto when \emph{tiling} mode is used, {(-15cm,-15cm)(15cm,15cm)}
%   otherwise}).
% \end{Description}
% 
%   As explained in the introduction, the \emph{manual} mode can require very
% huge amount of computer ressources. So, it usage is to discourage in front off
% the \emph{automatic} mode. It seems only useful in special circonstances, in
% fact when the \emph{automatic} mode failed, which is known only in one case,
% for some kinds of EPS files, as the ones produce by dump of portions of
% screens (see \ref{sec:GraphicFiles}).
% 
% \begin{Description}{2cm}
%   \setcounter{footnote}{1}
%   \item [\texttt{fillloopaddx}\footnotemark\ (integer)\hfill :] number of
%   times the pattern is added on left and right positions (\emph{Default:~0}).
%   \setcounter{footnote}{1}
%   \item [\texttt{fillloopaddy}\footnotemark\ (integer)\hfill :] number of
%   times the pattern is added on top and bottom positions (\emph{Default:~0}).
%   \setcounter{footnote}{1}
%   \item [\texttt{fillloopadd}\footnotemark\ (integer)\hfill :] number of
%   times the pattern is added on left, right, top and bottom positions
%   (\emph{Default:~0}).
% \end{Description}
% 
%   These parameters are only useful in special circonstances, as for complex
% patterns when the size of the rectangular box used to tile the area doesn't 
% correspond to the pattern itself (see an example in Figure~\ref{fig:Sheeps})
% and also sometimes when the size of the pattern is not a divisor of the size
% of the area to fill and that the number of loop repeats is not properly
% computed, which can occur.
% 
%   They are implemented only for the \emph{tiling} mode.
% 
% \begin{Description}{2cm}
%   \setcounter{footnote}{1}
%   \item [\texttt{PstDebug}\footnotemark\ (integer, 0 or 1)\hfill :] to
%   require to see the exact tiling done, without clipping (\emph{Default:~0}).
% \end{Description}
% 
%   It's mainly useful for debugging or to understand better how the tilings
% are done. It is implemented only for the \emph{tiling} mode.
% 
% \begin{LTXexample}
% \psset{unit=0.3,PstDebug=1}
% \psboxfill{\Square}
% \psset{linewidth=1mm}
% \Tiling{(2,2)}\\[5mm]
% \Tiling[fillcyclex=2]{(2,2)}\\[1cm]
% \Tiling[fillmove=0.5]{(2,2)}
% \end{LTXexample}
% 
% \vspace{3cm}
% \section{Examples}
% 
%   In fact this unique \cs{psboxfill} macro allow a lot a variations and
% different usages. We will try here to demonstrate this.
% 
% \subsection{Kind of tiles}
% \label{sec:KindTiles}
% 
%   Of course, we can access to all the power of PSTricks macros to define the
% \emph{tiles} (\emph{patterns}) used. So, we can define complicated ones.
% 
%   Here we give four other Archimedian tilings (those built with only some
% regular polygons) among the twelve existing, first discovered completely by
% Johanes \textsc{Kepler} at the beginning of 17th century \cite{GS87}, the two
% other \emph{regular} ones with the tiling by squares, formed by a unique
% regular polygon, and two other formed by two different regular polygons.
% 
% \begin{LTXexample}[pos=t]
%   \newcommand{\Triangle}{%
%     \begin{pspicture}(1,1)
%       \pstriangle[dimen=middle](0.5,0)(1,1)
%     \end{pspicture}}
%   \newcommand{\Hexagon}{
% ^^A sin(60)=0.866
%     \begin{pspicture}(0.866,0.75)
%       \SpecialCoor
% ^^A  Hexagon  
%       \pspolygon[dimen=middle]%
%         (0.5;30)(0.5;90)(0.5;150)(0.5;210)(0.5;270)(0.5;330)
%     \end{pspicture}}
% 
%   \psset{unit=0.5}
%   \psboxfill{\Triangle}
%   \Tiling{(4,4)}\hfill
% ^^A The two other regular tilings
%   \Tiling[fillcyclex=2]{(4,4)}\hfill
%   \psboxfill{\Hexagon}
%   \Tiling[fillcyclex=2,fillloopaddy=1]{(5,5)}
% \end{LTXexample}
% 
% \begin{LTXexample}[pos=t]
%   \newcommand{\ArchimedianA}{%
%      ^^A Archimedian tiling 3^2.4.3.4
%     \psset{dimen=middle}
%      ^^A sin(60)=0.866
%     \begin{pspicture}(1.866,1.866)
%       \psframe(1,1)
%       \psline(1,0)(1.866,0.5)(1,1)(0.5,1.866)(0,1)(-0.866,0.5)
%       \psline(0,0)(0.5,-0.866)
%     \end{pspicture}}
%   \newcommand{\ArchimedianB}{%
%      ^^A Archimedian tiling 4.8^2
%     \psset{dimen=middle,unit=1.5}
%      ^^A sin(22.5)=0.3827 ; cos(22.5)=0.9239
%     \begin{pspicture}(1.3066,0.6533)
%       \SpecialCoor
%      ^^A Octogon
%       \pspolygon(0.5;22.5)(0.5;67.5)(0.5;112.5)(0.5;157.5)
%                 (0.5;202.5)(0.5;247.5)(0.5;292.5)(0.5;337.5)
%     \end{pspicture}}
% 
%   \psset{unit=0.5}
%   \psboxfill{\ArchimedianA}
%   \Tiling[fillmove=0.5]{(7,7)}\hfill
%   \psboxfill{\ArchimedianB}
%   \Tiling[fillcyclex=2,fillloopaddy=1]{(7,7)}
% \end{LTXexample}
% 
%   \setcounter{footnote}{3}
%   We can of course tile an area arbitrarily defined. And with the
% \texttt{addfillstyle} parameter\footnote{Introduced in PSTricks 97.}, we can
% easily mix the \texttt{boxfill} style with another one.
% 
% \begin{LTXexample}[width=6cm]
%   \psset{unit=0.5,dimen=middle}
%   \psboxfill{%
%     \begin{pspicture}(1,1)
%       \psframe(1,1)
%       \pscircle(0.5,0.5){0.25}
%     \end{pspicture}}
%   \begin{pspicture}(4,6)
%     \pspolygon[fillstyle=boxfill,fillsep=0.25](0,1)(1,4)(4,6)(4,0)(2,1)
%   \end{pspicture}\hspace{1em}
%   \begin{pspicture}(4,4)
%%     \pscircle[linestyle=none,fillstyle=solid,fillcolor=yellow,fillsep=0.5,
%%               addfillstyle=boxfill](2,2){2}
%   \end{pspicture}
% \end{LTXexample}
%
%   Various effects can be obtained, sometimes complicated ones very easily, as
% in this example reproduced from one shown by Slavik \textsc{Jablan} in the
% field of \emph{OpTiles}, inspired by the \emph{Op-art}:
% 
% \begin{LTXexample}[pos=t]
% \newcommand{\ProtoTile}{%
%  \begin{pspicture}(1,1)%%% 1/12=0.08333
%   \psset{linestyle=none,linewidth=0,
%     hatchwidth=0.08333\psunit,hatchsep=0.08333\psunit}
%   \psframe[fillstyle=solid,fillcolor=black,addfillstyle=hlines,hatchcolor=white](1,1)
%   \pswedge[fillstyle=solid,fillcolor=white,addfillstyle=hlines]{1}{0}{90}
%  \end{pspicture}}
% \newcommand{\BasicTile}{%
%  \begin{pspicture}(2,1)
%    \rput[lb](0,0){\ProtoTile}\rput[lb](1,0){\psrotateleft{\ProtoTile}}
%  \end{pspicture}}
% \ProtoTile\hfill\BasicTile\hfill
% \psboxfill{\BasicTile}
% \Tiling[fillcyclex=2]{(4,4)}
% \end{LTXexample}
% 
%   It is also directly possible to surimpose several different tilings. Here is
% the splendid visual proof of the \textsc{Pytha\-gore} theorem done by the arab
% mathematician \textsc{Annairizi} around the year 900, given by superposition
% of two tilings by squares of different sizes.
% 
% \begin{LTXexample}[pos=t]
% \psset{unit=1.5,dimen=middle}
% \begin{pspicture*}(3,3)
%   \psboxfill{\begin{pspicture}(1,1)
%     \psframe(1,1)\end{pspicture}}
%   \psframe[fillstyle=boxfill](3,3)
%   \psboxfill{\begin{pspicture}(1,1)
%     \rput{-37}{\psframe[linecolor=red](0.8,0.8)}
%   \end{pspicture}}
%   \psframe[fillstyle=boxfill](3,4)
%   \pspolygon[fillstyle=hlines,hatchangle=90](1,2)(1.64,1.53)(2,2)
% \end{pspicture*}
% \end{LTXexample}
% 
%   In a same way, it is possible to build tilings based on figurative patterns,
% in the style of the famous \textsc{Escher} ones. Following an example of
% Andr\'e \textsc{Deledicq} \cite{Deledicq97}, we first show a simple tiling of
% the \emph{p1} category (according to the international classification of the
% 17~symmetry groups of the plane first discovered by the russian
% crystalographer Jevgraf \textsc{Fedorov} at the end of the 19th century):
% 
% \begin{LTXexample}[pos=t]
%  \newcommand{\SheepHead}[1]{%
%    \begin{pspicture}(3,1.5)
%      \pscustom[liftpen=2,fillstyle=solid,fillcolor=#1]{%
%        \pscurve(0.5,-0.2)(0.6,0.5)(0.2,1.3)(0,1.5)(0,1.5)
%          (0.4,1.3)(0.8,1.5)(2.2,1.9)(3,1.5)(3,1.5)(3.2,1.3)
%          (3.6,0.5)(3.4,-0.3)(3,0)(2.2,0.4)(0.5,-0.2)}
%      \pscircle*(2.65,1.25){0.12\psunit} % Eye
%      \psccurve*(3.5,0.3)(3.35,0.45)(3.5,0.6)(3.6,0.4)% Muzzle
%     ^^A   % Mouth
%       \pscurve(3,0.35)(3.3,0.1)(3.6,0.05)
%     ^^A   % Ear
%       \pscurve(2.3,1.3)(2.1,1.5)(2.15,1.7)\pscurve(2.1,1.7)(2.35,1.6)(2.45,1.4)
%   \end{pspicture}}
%  \psboxfill{\psset{unit=0.5}\SheepHead{yellow}\SheepHead{cyan}}
%  \Tiling[fillcyclex=2,fillloopadd=1]{(10,5)}
% \end{LTXexample}
% \label{fig:Sheeps}
% 
%   Now a tiling of the \emph{pg} category (the code for the kangaroo itself is
% too long to be shown here, but has no difficulties ; the kangaroo is reproduce
% from an original picture from Raoul \textsc{Raba} and here is a translation in
% PSTricks from the one drawn by Emmanuel \textsc{Chailloux} and Guy
% \textsc{Cousineau} for their MLgraph system \cite{MLgraphTSI}):
% 
% \begin{LTXexample}[pos=t]
% \psboxfill{\psset{unit=0.4}
%   \Kangaroo{yellow}\Kangaroo{red}\Kangaroo{cyan}\Kangaroo{green}%
%   \psscalebox{-1 1}{%
%     \rput(1.235,4.8){\Kangaroo{green}\Kangaroo{cyan}\Kangaroo{red}\Kangaroo{yellow}}}}
%   \Tiling[fillloopadd=1]{(10,6)}
% \end{LTXexample}
% 
%   And here a \textsc{Wang} tiling \cite{Wang65}, \cite[chapter
% 11]{GS87}, based on very simple tiles of the form of a square and composed
% of four colored triangles. Such tilings are built with only a matching color
% constraint. Despite of it simplicity, it is an important kind of tilings, as
% \textsc{Wang} and others used them to study the special class of
% \emph{aperiodic} tilings, and also because it was shown that surprisingly this 
% tiling is similar to a \textsc{Turing} machine.
% 
% \begin{LTXexample}[pos=t]
%   \newcommand{\WangTile}[4]{%
%     \begin{pspicture}(1,1)
%       \pspolygon*[linecolor=#1](0,0)(0,1)(0.5,0.5)
%       \pspolygon*[linecolor=#2](0,1)(1,1)(0.5,0.5)
%       \pspolygon*[linecolor=#3](1,1)(1,0)(0.5,0.5)
%       \pspolygon*[linecolor=#4](1,0)(0,0)(0.5,0.5)
%     \end{pspicture}}
%   \newcommand{\WangTileA}{\WangTile{cyan}{yellow}{cyan}{cyan}}
%   \newcommand{\WangTileB}{\WangTile{yellow}{cyan}{cyan}{red}}
%   \newcommand{\WangTileC}{\WangTile{cyan}{red}{yellow}{yellow}}
%   \newcommand{\WangTiles}[1][]{%
%     \begin{pspicture}(3,3) \psset{ref=lb}
%       \rput(0,2){\WangTileB}  \rput(1,2){\WangTileA}%
%       \rput(2,2){\WangTileC}  \rput(0,1){\WangTileC}%
%       \rput(1,1){\WangTileB}  \rput(2,1){\WangTileA}
%       \rput(0,0){\WangTileA}  \rput(1,0){\WangTileC}%
%       \rput(2,0){\WangTileB}
%       #1
%     \end{pspicture}}
%   \WangTileA\hfill\WangTileB\hfill\WangTileC\hfill
%   \WangTiles[{\psgrid[subgriddiv=0,gridlabels=0](3,3)}]\hfill
%   \psset{unit=0.4} \psboxfill{\WangTiles} \Tiling{(12,12)}
% \end{LTXexample}
% 
% \subsection{External graphic files}
% \label{sec:GraphicFiles}
% 
%   We can also fill an arbitrary area with an external image. We have only, 
% as usual, to matter of the \emph{BoundingBox} definition if there is no one
% provided or if it is not the accurate one, as for the well known
% \texttt{tiger} picture part of the \texttt{ghostscript} distribution.
% 
% \begin{LTXexample}[pos=t]
%   \psboxfill{%% Strangely require x1=x2...
%     \begin{pspicture}(0,1)(0,4.1)
%       \includegraphics[bb=17 176 560 74,width=3cm]{tiger}
%     \end{pspicture}}
%   \Tiling{(6,6.2)}
% \end{LTXexample}
% 
%   Nevertheless, there are some special files for which the \emph{automatic}
% mode doesn't work, specially for some files obtained by a screen dump, as in
% the next example, where a picture was reduced before it conversion in the
% \emph{Encapsulated PostScript} format by a screen dump utility. In this case,
% usage of the \emph{manual} mode is the only alternative, at the price of the
% real multiple inclusion of the EPS file. We must take care to specify the
% correct \texttt{fillsize} parameter, because otherwise the default values are
% large and will load the file many times, perhaps just really using few
% occurrences as the other ones would be clipped...
% 
% \begin{LTXexample}[pos=t]
%   \psboxfill{\includegraphics{flowers}}
%   \begin{pspicture}(8,4)
%     \psellipse[fillstyle=boxfill,fillsize={(8,4)}](4,2)(4,2)
%   \end{pspicture}
% \end{LTXexample}
% 
% \subsection{Tiling of characters}
% 
%   We can also use the \cs{psboxfill} macro to fill the interior of characters
% for special effects like these ones:
% 
% \begin{LTXexample}[pos=t]
%   \DeclareFixedFont{\bigsf}{T1}{phv}{b}{n}{4.5cm}
%   \DeclareFixedFont{\smallrm}{T1}{ptm}{m}{n}{3mm}
%   \psboxfill{\smallrm Since 182 days...}
%   \begin{pspicture*}(8,4)
%     \centerline{%
%       \pscharpath[fillstyle=gradient,gradangle=-45,
%                   gradmidpoint=0.5,addfillstyle=boxfill,
%                   fillangle=45,fillsep=0.7mm]
%                  {\rput[b](0,0.1){\bigsf 2000}}}
%   \end{pspicture*}
% \end{LTXexample}
% 
% \begin{LTXexample}[pos=t]
%   \DeclareFixedFont{\mediumrm}{T1}{ptm}{m}{n}{2cm}
%   \psboxfill{%
%     \psset{unit=0.1,linewidth=0.2pt}
%     \Kangaroo{PeachPuff}\Kangaroo{PaleGreen}%
%       \Kangaroo{LightBlue}\Kangaroo{LemonChiffon}%
%     \psscalebox{-1 1}{%
%       \rput(1.235,4.8){%
%         \Kangaroo{LemonChiffon}\Kangaroo{LightBlue}%
%           \Kangaroo{PaleGreen}\Kangaroo{PeachPuff}}}}
% ^^A   % A kangaroo of kangaroos...
%   \begin{pspicture}(8,2)
%     \pscharpath[linestyle=none,fillstyle=boxfill,fillloopadd=1]
%                {\rput[b](4,0){\mediumrm Kangaroo}}
%   \end{pspicture}
% \end{LTXexample}
% 
% \subsection{Other kinds of usage}
% 
%   Other kinds of usage can be imagined. For instance, we can use tilings in a
% sort of degenerated way to draw some special lines made by a unique or
% multiple repeating patterns. But it can be only a special dashed line, as here
% with three different dashes:
% 
% \begin{LTXexample}[pos=t]
%   \newcommand{\Dashes}{%
%     \psset{dimen=middle}
%     \begin{pspicture}(0,-0.5\pslinewidth)(1,0.5\pslinewidth)
%       \rput(0,0){\psline(0.4,0)}%
%         \rput(0.5,0){\psline(0.2,0)}%
%         \rput(0.8,0){\psline(0.1,0)}
%     \end{pspicture}}
% 
%   \newcommand{\SpecialDashedLine}[3]{%
%     \psboxfill{#3}
%     \Tiling[linestyle=none]
%            {(#1,-0.5\pslinewidth)(#2,0.5\pslinewidth)}}
% 
%   \SpecialDashedLine{0}{7}{\Dashes}
% 
%   \psset{unit=0.5,linewidth=1mm,linecolor=red}
%   \SpecialDashedLine{0}{10}{\Dashes}
% \end{LTXexample}
% 
%   It allow also to use special patterns in business graphics, as in the
% following example generated by \texttt{PstChart}\footnote{A personal
% development to draw business charts with PSTricks, not distributed.}.
% 
% \vspace{3mm}
% \begin{figure}[!ht]
% \centering
% \psset{unit=0.75}
% ^^A % Generated by pstchart.sh version 0.21 (11/28/97)
% {\psset{dimen=middle}
% \psset{xunit=2,yunit=0.005}
% \begin{pspicture}(-0.6,-200)(6.6,2300)
% ^^A   % Title
%   \rput(3,2200){\shortstack{Fantaisist repartition of kangaroos\\
%                             in the world (in thousands)}}
% ^^A   % Frame background
%   \psframe[fillstyle=solid,fillcolor=LemonChiffon](0,0)(6,2000)
% ^^A   % Graduations
%   \multido{\n=0+500}{5}{\rput[r](-0.12,\n){\psscalebox{0.8}{\n}}}
% ^^A   % Minor ticks
%   \multips(0,100)(0,100){19}{\psline[unit=4.8pt](1,0)}
%   \multips(6,100)(0,100){19}{\psline[unit=4.8pt](-1,0)}
% ^^A   % Major ticks
%   \multips(0,500)(0,500){3}{\psline[unit=9.6pt](1,0)}
%   \multips(6,500)(0,500){3}{\psline[unit=9.6pt](-1,0)}
% ^^A   % Lines from major ticks marks
%   \multips(0,500)(0,500){3}{\psline[linestyle=dotted,linewidth=0.6pt](6,0)}
% ^^A   % Drawing for the data
%   \psboxfill{\psset{unit=0.78\psxunit}\KangarooPstChart{red}}
%   \psframe[linestyle=none,fillstyle=boxfill,fillloopaddy=1](0.61,0)(1.39,1800)
%   \psboxfill{\psset{unit=0.78\psxunit}\KangarooPstChart{yellow}}
%   \psframe[linestyle=none,fillstyle=boxfill,fillloopaddy=1](1.61,0)(2.39,800)
%   \psboxfill{\psset{unit=0.78\psxunit}\KangarooPstChart{cyan}}
%   \psframe[linestyle=none,fillstyle=boxfill,fillloopaddy=1](2.61,0)(3.39,550)
%   \psboxfill{\psset{unit=0.78\psxunit}\KangarooPstChart{magenta}}
%   \psframe[linestyle=none,fillstyle=boxfill,fillloopaddy=1](3.61,0)(4.39,500)
%   \psboxfill{\psset{unit=0.78\psxunit}\KangarooPstChart{green}}
%   \psframe[linestyle=none,fillstyle=boxfill,fillloopaddy=1](4.61,0)(5.39,200)
% ^^A   % Bottom labels
%   \uput{0.2}[270]{0}(1,0){\psscalebox{0.7}{Oceania}}
%   \uput{0.2}[270]{0}(2,0){\psscalebox{0.7}{Africa}}
%   \uput{0.2}[270]{0}(3,0){\psscalebox{0.7}{Asia}}
%   \uput{0.2}[270]{0}(4,0){\psscalebox{0.7}{America}}
%   \uput{0.2}[270]{0}(5,0){\psscalebox{0.7}{Europe}}
% ^^A   % Frame box around the chart
%   \psframe[linestyle=solid](0,0)(6,2000)
% \end{pspicture}}
%   \caption{Bar chart generated by PstChart, with bars filled by patterns}
%   \label{fig:PstChart}
% \end{figure}
% 
% \section{``Dynamic'' tilings}
% 
%   In some cases, tilings used non \emph{static} tiles, that is to say that the 
% \emph{prototile(s)}, even if unique, can have several forms, by instance
% specified by different colors or rotations, not fixed before generation or
% varying each time.
% 
% \subsection{Lewthwaite-Pickover-Truchet tiling}
% 
%   We give here for example the so-called \emph{Truchet} tiling, which much be
% in fact better called \emph{Lewthwaite-Pick\-over-Truchet (LPT)} tiling%
% \footnote{For description of the context, history and references about
% S\'ebastien \textsc{Truchet} and this tiling, see \cite{EsperetGirou98}.}.
% 
%   The unique prototile is only a square with two opposite circle arcs.
% This tile has obviously two positions, if we rotate it from 90 degrees (see
% the two tiles on the next figure). A \emph{LPT tiling} is a tiling with
% randomly oriented LPT tiles. We can see that even if it is very simple in it
% principle, it draw sophisticated curves with strange properties.
% 
%   Nevertheless, in the straightforward way \FillPackage{} does not work,
% because the \cs{psboxfill} macro store the content of the tile used in a
% \TeX{} box, which is static. So the calling to the random function is done
% only one time, which explain that only one rotation of the tile is used for
% all the tiling. It's only the one of the two rotations which could differ from
% one drawing to the next one...
% 
% ^^A % Truchet (Lewthwaite-Pickover-Truchet) tiling
% ^^A % --------------------------------------------
% 
% \begin{LTXexample}[pos=t]
% ^^A   % LPT prototile
%   \newcommand{\ProtoTileLPT}{%
%     \psset{dimen=middle}
%     \begin{pspicture}(1,1)
%       \psframe(1,1)
%       \psarc(0,0){0.5}{0}{90}
%       \psarc(1,1){0.5}{-180}{-90}
%     \end{pspicture}}
% 
% ^^A   % LPT tile
%   \newcount\Boolean
%   \newcommand{\BasicTileLPT}{%
% ^^A     % From random.tex by Donald Arseneau
%     \setrannum{\Boolean}{0}{1}%
%     \ifnum\Boolean=0
%       \ProtoTileLPT%
%     \else
%       \psrotateleft{\ProtoTileLPT}%
%     \fi}
% 
%   \ProtoTileLPT\hfill\psrotateleft{\ProtoTileLPT}\hfill
%   \psset{unit=0.5}
%   \psboxfill{\BasicTileLPT}
%   \Tiling{(5,5)}
% \end{LTXexample}
% 
%   But, for simple cases, there is a solution to this problem using a mixture
% of PSTricks and PostScript programming. Here the PSTricks
% construction \verb+\pscustom{\code{...}}+ allow to insert PostScript code
% inside the \LaTeX{} + PSTricks one.
% 
%   Programmation is less straightforward, but it has also the advantage to be
% notably faster, as all the tilings operations are done in PostScript, and
% mainly to not be limited by \TeX{} memory (the \TeX{} + PSTricks solution
% I wrote in 1995 for the colored problem was limited to small sizes for this
% reason). Just note also that \cs{pslbrace} and \cs{psrbrace} are two
% PSTricks macros to define and be able to insert the \verb+{+ and \verb+}+
% characters.
% 
% \begin{LTXexample}[pos=t]
% ^^A   % LPT prototile
%   \newcommand{\ProtoTileLPT}{%
%     \psset{dimen=middle}
%     \psframe(1,1)
%     \psarc(0,0){0.5}{0}{90}
%     \psarc(1,1){0.5}{-180}{-90}}
% 
% ^^A   % Counter to change the random seed
%   \newcount\InitCounter
% ^^A   % LPT tile
%   \newcommand{\BasicTileLPT}{%
%     \InitCounter=\the\time
%     \pscustom{\code{%
%       rand \the\InitCounter\space sub 2 mod 0 eq \pslbrace}}
%     \begin{pspicture}(1,1)
%       \ProtoTileLPT
%     \end{pspicture}%
%     \pscustom{\code{\psrbrace \pslbrace}}
%     \psrotateleft{\ProtoTileLPT}%
%     \pscustom{\code{\psrbrace ifelse}}}
% 
%   \psset{unit=0.4,linewidth=0.4pt}
%   \psboxfill{\BasicTileLPT}
%   \Tiling{(15,15)}
% \end{LTXexample}
% 
%   Using the very surprising fact (see \cite{EsperetGirou98}) that
% coloration of these tiles do not depend of their neighbors (even if it is
% difficult to believe as the opposite seems obvious!) but only of the parity of
% the value of row and column positions, we can directly program in the same way
% a colored version of the LPT tiling.
% 
% \setcounter{footnote}{1}
%   We have also introduce in the \FillPackage{} code for \emph{tiling} mode two
% new accessible Post\-Script variables, \texttt{row} and
% \texttt{column}\footnotemark, which can be useful in some circonstances, like
% this one.
% 
% \begin{LTXexample}[pos=t]
% ^^A   % LPT prototile
%   \newcommand{\ProtoTileLPT}[2]{%
%     \psset{dimen=middle,linestyle=none,fillstyle=solid}
%     \psframe[fillcolor=#1](1,1)
%     \psset{fillcolor=#2}
%     \pswedge(0,0){0.5}{0}{90} \pswedge(1,1){0.5}{-180}{-90}}
% ^^A   % Counter to change the random seed
%   \newcount\InitCounter
% ^^A   % LPT tile
%   \newcommand{\BasicTileLPT}[2]{%
%     \InitCounter=\the\time
%     \pscustom{\code{%
%       rand \the\InitCounter\space sub 2 mod 0 eq \pslbrace
%       row column add 2 mod 0 eq \pslbrace}}
%     \begin{pspicture}(1,1)\ProtoTileLPT{#1}{#2}\end{pspicture}%
%     \pscustom{\code{\psrbrace \pslbrace}}
%     \ProtoTileLPT{#2}{#1}%
%     \pscustom{\code{%
%       \psrbrace ifelse \psrbrace \pslbrace row column add 2 mod 0 eq \pslbrace}}
%     \psrotateleft{\ProtoTileLPT{#2}{#1}}\pscustom{\code{\psrbrace \pslbrace}}
%     \psrotateleft{\ProtoTileLPT{#1}{#2}}\pscustom{\code{\psrbrace ifelse \psrbrace ifelse}}}
%   \psboxfill{\BasicTileLPT{red}{yellow}}
%   \Tiling{(4,4)}\hfill
%   \psset{unit=0.4}\psboxfill{\BasicTileLPT{blue}{cyan}}
%   \Tiling{(15,15)}
% \end{LTXexample}
% 
%   Another classic example is to generate coordinates and numerotation for a
% grid. Of course, it is possible to do it directly in PSTricks using nested
% \cs{multido} commands. It would be clearly easy to program, but, nevertheless, 
% for users who have a little knowledge of PostScript programming, this offer
% an alternative which is useful for large cases, because on this way it will
% be notably faster and less computer ressources consuming.
% 
%   Remember here that the tiling is drawn from left to right, and top to
% bottom, and note that the PostScript variable \texttt{x2} give the total
% number of columns.
% 
% \begin{LTXexample}[pos=t]
% ^^A   % \Escape will be the \ character
%   {\catcode`\!=0\catcode`\\=11!gdef!Escape{\}}
%   \newcommand{\ProtoTile}{%
%     \Square\pscustom{%
%       \moveto(-0.9,0.75) % In PSTricks units
%       \code{ /Times-Italic findfont 8 scalefont setfont
%         (\Escape() show row 3 string cvs show (,) show 
%         column 3 string cvs show (\Escape)) show}
%       \moveto(-0.5,0.25) % In PSTricks units
%       \code{ /Times-Bold findfont 18 scalefont setfont
%         1 0 0 setrgbcolor % Red color
%         /center {dup stringwidth pop 2 div neg 0 rmoveto} def
%         row 1 sub x2 mul column add 3 string cvs center show}}}
%   \psboxfill{\ProtoTile}
%   \Tiling{(6,4)}
% \end{LTXexample}
% 
% \subsection{A complete example: the Poisson equation}
% 
%   To finish, we will show a complete real example, a drawing to explain the
% method used to solve the \textsc{Poisson} equation by a domain
% decomposition method, adapted to distributed memory computers. The
% objective is to show the communications required between processes and the
% position of the data to exchange. This code also show some useful and powerful
% technics for PSTricks programming (look specially at the way some higher level
% macros are defined, and how the same object is used to draw the four
% neighbors).
%
%\psset{unit=1cm}
%\newcommand{\Pattern}[1]{%
%   \begin{pspicture}(-0.25,-0.25)(0.25,0.25)\rput{*0}{\psdot[dotstyle=#1]}
%   \end{pspicture}}
%\newcommand{\West}{\Pattern{o}}   \newcommand{\South}{\Pattern{x}}
%\newcommand{\Central}{\Pattern{+}}\newcommand{\North}{\Pattern{square}}
%\newcommand{\East}{\Pattern{triangle}}
%\newcommand{\Cross}{%
%  \pspolygon[unit=0.5,linewidth=0.2,linecolor=red](0,0)(0,1)(1,1)(1,2)(2,2)(2,1)%
%              (3,1)(3,0)(2,0)(2,-1)(1,-1)(1,0)}
%\newcommand{\StylePosition}[1]{\LARGE\textcolor{red}{\textbf{#1}}}
%\newcommand{\SubDomain}[4]{%
%    \psboxfill{#4}\begin{psclip}{\psframe[linestyle=none]#1}%
%      \psframe[linestyle=#3](5,5)\psframe[fillstyle=boxfill]#2%
%    \end{psclip}}
%\newcommand{\SendArea}[1]{\psframe[fillstyle=solid,fillcolor=cyan]#1}
%\newcommand{\ReceiveData}[2]{%
%  \psboxfill{#2}\psframe[fillstyle=solid,fillcolor=yellow,addfillstyle=boxfill]#1}%
%\newcommand{\Neighbor}[2]{%
%    \begin{pspicture}(5,5)
%      \rput{*0}(2.5,2.5){\StylePosition{#1}}
%      \ReceiveData{(0.5,0)(4.5,0.5)}{\Central}\SendArea{(0.5,0.5)(4.5,1)}%
%      \SubDomain{(5,2)}{(0.5,0.5)(4.5,3)}{dashed}{#2}%
%      \pcarc[arcangle=45,arrows=->](0.5,-1.25)(0.5,0.25)%
%      \pcarc[arcangle=45,arrows=->,linestyle=dotted,dotsep=2pt](4.5,0.75)(4.5,-0.75)%
%    \end{pspicture}}%
%  \psset{dimen=middle,dotscale=2,fillloopadd=2}
%\begin{pspicture}(-5.7,-5.7)(5.7,5.7)
%  \rput(0,0){%
%      \begin{pspicture}(5,5)
%        \ReceiveData{(0,0.5)(0.5,4.5)}{\West} \ReceiveData{(4.5,0.5)(5,4.5)}{\East}
%        \ReceiveData{(0.5,4.5)(4.5,5)}{\North}\ReceiveData{(0.5,0)(4.5,0.5)}{\South}
%        \SendArea{(0.5,0.5)(1,4.5)}\SendArea{(4,0.5)(4.5,4.5)}
%        \SendArea{(0.5,0.5)(4.5,1)}\SendArea{(0.5,4)(4.5,4.5)}
%        \SubDomain{(5,5)}{(0.5,0.5)(4.5,4.5)}{solid}{\Central}
%        \psline(1,0.5)(1,4.5)\psline(4,0.5)(4,4.5)%
%        \rput(1.5,4){\Cross}\rput(2,2){\Cross}%
%      \end{pspicture}}%
%  \rput(0,5.5){\Neighbor{N}{\North}}\rput{-90}(5.5,0){\Neighbor{E}{\East}}%
%  \rput{90}(-5.5,0){\Neighbor{W}{\West}}\rput{180}(0,-5.5){\Neighbor{S}{\South}}%
%\end{pspicture}
%
% \begin{lstlisting}
%   \newcommand{\Pattern}[1]{%
%     \begin{pspicture}(-0.25,-0.25)(0.25,0.25)\rput{*0}{\psdot[dotstyle=#1]}
%     \end{pspicture}}
%   \newcommand{\West}{\Pattern{o}}   \newcommand{\South}{\Pattern{x}}
%   \newcommand{\Central}{\Pattern{+}}\newcommand{\North}{\Pattern{square}}
%   \newcommand{\East}{\Pattern{triangle}}
%   \newcommand{\Cross}{%
%     \pspolygon[unit=0.5,linewidth=0.2,linecolor=red](0,0)(0,1)(1,1)(1,2)(2,2)(2,1)
%               (3,1)(3,0)(2,0)(2,-1)(1,-1)(1,0)}
%   \newcommand{\StylePosition}[1]{\LARGE\textcolor{red}{\textbf{#1}}}
%   \newcommand{\SubDomain}[4]{%
%     \psboxfill{#4}
%     \begin{psclip}{\psframe[linestyle=none]#1}
%       \psframe[linestyle=#3](5,5)\psframe[fillstyle=boxfill]#2
%     \end{psclip}}
%   \newcommand{\SendArea}[1]{\psframe[fillstyle=solid,fillcolor=cyan]#1}
%   \newcommand{\ReceiveData}[2]{%
%     \psboxfill{#2}
%     \psframe[fillstyle=solid,fillcolor=yellow,addfillstyle=boxfill]#1}
%   \newcommand{\Neighbor}[2]{%
%     \begin{pspicture}(5,5)
%       \rput{*0}(2.5,2.5){\StylePosition{#1}}
%       \ReceiveData{(0.5,0)(4.5,0.5)}{\Central}\SendArea{(0.5,0.5)(4.5,1)}
%       \SubDomain{(5,2)}{(0.5,0.5)(4.5,3)}{dashed}{#2}%
% ^^A       % Receive and send arrows
%       \pcarc[arcangle=45,arrows=->](0.5,-1.25)(0.5,0.25)
%       \pcarc[arcangle=45,arrows=->,linestyle=dotted,dotsep=2pt](4.5,0.75)(4.5,-0.75)
%     \end{pspicture}}
%   \psset{dimen=middle,dotscale=2,fillloopadd=2}
%   \begin{pspicture}(-5.7,-5.7)(5.7,5.7)
% ^^A     % Central domain
%     \rput(0,0){%
%       \begin{pspicture}(5,5)
% ^^A         % Receive from West, East, North and South
%         \ReceiveData{(0,0.5)(0.5,4.5)}{\West} \ReceiveData{(4.5,0.5)(5,4.5)}{\East}
%         \ReceiveData{(0.5,4.5)(4.5,5)}{\North}\ReceiveData{(0.5,0)(4.5,0.5)}{\South}
% ^^A         % send area for West, East, North and South
%         \SendArea{(0.5,0.5)(1,4.5)} \SendArea{(4,0.5)(4.5,4.5)}
%         \SendArea{(0.5,0.5)(4.5,1)} \SendArea{(0.5,4)(4.5,4.5)}
% ^^A         % Central domain
%         \SubDomain{(5,5)}{(0.5,0.5)(4.5,4.5)}{solid}{\Central}
% ^^A         % Redraw overlapped linesY
%         \psline(1,0.5)(1,4.5)  \psline(4,0.5)(4,4.5)
% ^^A         % Two crossesY
%         \rput(1.5,4){\Cross}  \rput(2,2){\Cross}
%       \end{pspicture}}
% ^^A     % The four neighborsY
%     \rput(0,5.5){\Neighbor{N}{\North}}     \rput{-90}(5.5,0){\Neighbor{E}{\East}}
%     \rput{90}(-5.5,0){\Neighbor{W}{\West}} \rput{180}(0,-5.5){\Neighbor{S}{\South}}
%   \end{pspicture}
% \end{lstlisting}
%
%
%
% Bibliography
% \begin{thebibliography}{99}
% \bibitem{PostScript95} Adobe, Systems~Incorporated, \emph{PostScript Language
% Reference Manual}, Addison-Wesley, 2~edition, 1995.
%
% \bibitem{Bolek98} Piotr Bolek, \MP{} and patterns, \emph{\TUB}, Volume~19,
% Number~3, pages 276--283, September 1998, \CTANref{mpattern}.
%
% \bibitem{MLgraphTSI} Emmanuel Chailloux, Guy Cousineau and Asc\'ander
% Su\'arez, Programmation fonctionnelle de graphismes pour la production
% d'illustrations techniques, \emph{Technique et science informatique},
% Volume~15, Number~7, pages 977--1007, 1996 (in french).
%
% \bibitem{Deledicq97} Andr\'e Deledicq, \emph{Le monde des pavages}, ACL
% \'Editions, 1997 (in french).
%
% \bibitem{EsperetGirou98} Philippe Esperet and Denis Girou,
% Coloriage du pavage dit de Truchet, Cahiers GUTenberg, Number~31,
% pages 5--18, December~1998  (in french).
%
% \bibitem{Girou94} Denis Girou, Pr\'esentation de PSTricks, \emph{Cahiers
% GUTenberg}, Number~16, pages 21--70, February~1994 (in french).
%
% \bibitem{LGC97} Michel Goossens, Sebastian Rahtz and Frank Mittelbach,
% \emph{The \LaTeX{} Graphics Companion}, Addison-Wesley, 2005.
%
% \bibitem{GS87} Branko Gr\"unbaum and Geoffrey Shephard, \emph{Tilings and
% Patterns}, Freeman and Company, 1987.
%
% \bibitem{Hoenig97} Alan Hoenig, \emph{\TeX{} Unbound: \LaTeX{} \& \TeX{}
% Strategies, Fonts, Graphics, and More}, Oxford University Press, 1997.
%
% \bibitem{XYpic} Kristoffer~H. Rose and Ross Moore, \XYpic. Pattern and Tile
% extension, available from \CTAN, 1991-1998, \CTANref{xypic}.
%
% \bibitem{LAAN96} Kees van der Laan, Paradigms: Just a little bit of PostScript,
% \emph{MAPS}, Volume~17, pages 137--150, 1996.
%
% \bibitem{LAAN97} Kees van der Laan, Tiling in PostScript and \MF{} -- Escher's
% wink, \emph{MAPS}, Volume~19, Number~2, pages 39--67, 1997.
%
% \bibitem{vanZandt93} Timothy Van Zandt, PSTricks. PostScript macros for
% Generic \TeX, available from \CTAN, 1993, \CTANref{pstricks}.
%
% \bibitem{vanZandtGirou94} Timothy Van Zandt and Denis Girou, Inside PSTricks,
% \emph{\TUB}, Volume~15, Number~3, pages 239--246, September 1994.
%
%
% \bibitem{voss07} Herbert Vo\ss, PSTricks -- Graphics for \TeX\ and \LaTeX, DANTE/Lehmanns, 4th ed., 2007.
% \bibitem{Wang65} Hao Wang, Games, Logic and Computers, \emph{Scientific
% American}, pages 98--106, November 1965.
% \end{thebibliography}
%
%
% \StopEventually{}
%
% ^^A .................... End of the documentation part ....................
%
% \section{Driver file}
%
%   The next bit of code contains the documentation driver file for \TeX{},
% i.e., the file that will produce the documentation you are currently
% reading. It will be extracted from this file by the \texttt{docstrip}
% program.
%
%    \begin{macrocode}
%<*driver>
\documentclass{ltxdoc}
\GetFileInfo{pst-fill.dtx}
%
\usepackage[T1]{fontenc}
\usepackage{lmodern}               % For PDF
\usepackage{graphicx}              % `graphicx' LaTeX standard package
\usepackage{showexpl}
\usepackage{mflogo}                % For the MetaFont and MetaPost logos
\input{random.tex}                 % Random macros from Donald Arseneau
\usepackage{url}                   % URLs convenient typesetting
\usepackage{multido}               % General loop macro
\usepackage[dvipsnames]{pstricks}  % PSTricks with the `color' extension
\usepackage{pst-text}              % PSTricks package for character path
\usepackage{pst-grad}              % PSTricks package for gradient filling
\usepackage{pst-node}              % PSTricks package for nodes
\usepackage[tiling]{pst-fill}      % PSTricks package for filling/tiling
%
\AtBeginDocument{%
%  \OnlyDescription % comment out for implementation details
  \EnableCrossrefs
  \CodelineIndex
  \RecordChanges}
\AtEndDocument{%
  \PrintIndex
  \setcounter{IndexColumns}{1}
  \PrintChanges}
\hbadness=7000            % Over and under full box warnings
\hfuzz=3pt
\begin{document}
  \DocInput{pst-fill.dtx}
\end{document}
%</driver>
%    \end{macrocode}
%
% \section{\texttt{pst-fill} \LaTeX{} wrapper}
%
%    \begin{macrocode}
%<*latex-wrapper>
\RequirePackage{pstricks}
\ProvidesPackage{pst-fill}[2005/09/13 package wrapper for 
  pst-fill.tex (hv)]
\DeclareOption{tiling}{\def\PstTiling{true}}
\ProcessOptions\relax
% \iffalse meta-comment, etc.
%%
%% Package `pst-fill.dtx'
%%
%% Denis Girou (CNRS/IDRIS - France) <Denis.Girou@idris.fr>
%% Herbert Voss <voss@pstricks.de>
%%
%% This program can be redistributed and/or modified under the terms
%% of the LaTeX Project Public License Distributed from CTAN archives
%% in directory macros/latex/base/lppl.txt.
%%
%% DESCRIPTION:
%%   `pst-fill' is a PSTricks package for filling and tiling areas 
%%
% \fi
% \changes{v1.01}{2007/03/10}{bugfix for incomplete ifx (hv)}
% \changes{v1.00}{2006/11/06}{use pst-xkey for extend keys (hv)}
% \changes{v0.99}{2004/08/17}{merge the VTeX and TeX versions (patch 4) (hv)}
% \changes{v0.98}{2004/06/22}{delete the Pst@Debug option and use the
%   the one from pstricks to prevent a clash with pst-gr3d (hv)}
% \changes{v0.97}{2001/10/09}{make it work with VTeX (mv)}
% \changes{v0.94}{1997/04/08}{With a \PstTiling macro defined (or "tiling" optional parameter
%   on \textbackslash usepackage[tiling]{pst-fill}), this file run exactly as
%   the original boxfill.tex file from Timothy, version 0.94,
%   except a correction in \textbackslash pst@ManualFillCycle to avoid a division by 0.
%   It's the default.}
% \changes{v0.93}{1997/04/07}{With a \textbackslash PstTiling macro defined (or "tiling" optional parameter
%   on \textbackslash usepackage[tiling]{pst-fill}) there are several add-ons
%   and changes to do `tiling' rather than `filling' in "automatic" mode :
%     - we fix the position of the beginning of tiling,
%     - we allow normally the framing of the area as expected, using
%       the line.... parameters
%     - we define move parameters fillmovex, fillmovey and fillmove,
%     - we define fillcyclex as previous fillcycle parameter, and add the
%       fillcycley and fillcycle (both fillcyclex and fillcycley) ones
%     - we can extend the tiling area using fillloopaddx, fillloopaddy and
%       fillloopadd parameters,
%     - we can debug and see the whole tiling area without clipping using
%       PstDebug parameter,
%     - for names consistancy, we can use fillangle in place of boxfillangle
%       and fillsize in place of boxfillsize,
%     - default value for fillsep is 0 and for fillsize is auto.}
%
% \DoNotIndex{\!,\",\#,\$,\%,\&,\',\(,\+,\*,\,,\-,\.,\/,\:,\;,\<,\=,\>,\?}
% \DoNotIndex{\@,\@B,\@K,\@cTq,\@f,\@fPl,\@ifnextchar,\@nameuse,\@oVk}
% \DoNotIndex{\[,\\,\],\^,\_,\ }
% \DoNotIndex{\^,\\^,\\\^,$\^$,$\\^$,$\\^$}
% \DoNotIndex{\0,\2,\4,\5,\6,\7,\8,}
% \DoNotIndex{\A,\a}
% \DoNotIndex{\B,\b,\Bc,\begin,\Bq,\Bqc}
% \DoNotIndex{\C,\c,\catcode,\cJA,\CodelineIndex,\csname}
% \DoNotIndex{\D,\def,\define@key,\Df,\divide,\DocInput,\documentclass,\pst@addfams}
% \DoNotIndex{\eCN,\edef,\else,\eHd,\eMcj,\EnableCrossrefs,\end,\endcsname}
% \DoNotIndex{\endCenterExample,\endExample,\endinput,\endpsclip}
% \DoNotIndex{\PrintIndex,\PrintChanges,\ProvidesFile}
% \DoNotIndex{\endpspicture,\endSideBySideExample,\Example}
% \DoNotIndex{\F,\f,\FdUrr,\fi,\filedate,\fileversion,\FV@Environment}
% \DoNotIndex{\FV@UseKeyValues,\FV@XRightMargin,\FVB@Example,\fvset}
% \DoNotIndex{\G,\g,\GetFileInfo,\gr,\GradientLoaded,\gsFKrbK@o,\gsj,\gsOX}
% \DoNotIndex{\hbadness,\hfuzz,\HLEmphasize,\HLMacro,\HLMacro@i}
% \DoNotIndex{\HLReverse,\HLReverse@i,\hqcu,\HqY}
% \DoNotIndex{\I,\i,\ifx,\input,\Ir,\IU}
% \DoNotIndex{\j,\jl,\JT,\JVodH}
% \DoNotIndex{\K,\k,\kfSlL}
% \DoNotIndex{\L,\let}
% \DoNotIndex{\message,\mHNa,\mIU}
% \DoNotIndex{\N,\nB,\newcmykcolor,\newdimen,\newif,\nW}
% \DoNotIndex{\O,\oCDJDo,\ocQhVI,\OnlyDescription,\oRKJ}
% \DoNotIndex{\P,\p,\ProvidesPackage,\psframe,\pslinewidth,\psset}
% \DoNotIndex{\PstAtCode,\PSTricksLoaded}
% \DoNotIndex{\q,\Qr,\qssRXq,\qu,\qXjFQp,\qYL}
% \DoNotIndex{\R,\r,\RecordChanges,\relax,\RlaYI,\rN,\Rp,\rp,\RPDXNn,\rput}
% \DoNotIndex{\S,\scalebox,\SgY,\SideBySide@Example,\SideBySideExample}
% \DoNotIndex{\SgY,\sk,\Sp,\space,\sZb}
% \DoNotIndex{\T,\the,\tw@}
% \DoNotIndex{\u,\UiSWGEf@,\uJi,\usepackage,\uVQdMM,\UYj}
% \DoNotIndex{\VerbatimEnvironment,\VerbatimInput,\VrC@}
% \DoNotIndex{\WhZ,\WjKCYb,\WNs}
% \DoNotIndex{\XkN,\XW}
% \DoNotIndex{\Z,\ZCM,\Ze}
% \DoNotIndex{\addtocounter,\advance,\alph,\arabic,\AtBeginDocument,\AtEndDocument}
% \DoNotIndex{\AtEndOfPackage,\begingroup,\bfseries,\bgroup,\box,\csname}
% \DoNotIndex{\else,\endcsname,\endgroup,\endinput,\expandafter,\fi}
% \DoNotIndex{\TeX,\z@,\p@,\@one,\xdef,\thr@@,\string,\sixt@@n,\reset,\or,\multiply,\repeat,\RequirePackage}
% \DoNotIndex{\@cclvi,\@ne,\@ehpa,\@nil,\copy,\dp,\global,\hbox,\hss,\ht,\ifodd,\ifdim,\ifcase,\kern}
% \DoNotIndex{\chardef,\loop,\leavevmode,\ifnum,\lower}
% \setcounter{IndexColumns}{2}
%
% ^^A To extend the height used for the text
%
% ^^A  Aligned labels in a description environment
%\newenvironment{Description}[1]{%
%\begin{list}{nothing}{\setlength{\leftmargin}{#1}
%\setlength{\labelwidth}{\leftmargin}\setlength{\labelsep}{1mm}}}
%{\end{list}}
%
% ^^A For macro names
%\DeclareRobustCommand\cs[1]{\texttt{\char`\\#1}}
%
%
% ^^A From ltugboat.cls
% ^^A For references
%\makeatletter
%\newcommand\acro[1]{\textsc{#1}\@}
%\def\CTAN{\acro{CTAN}}
%\let\texttub\textsl              % ^^A redefined in other situations
%\def\TUB{\texttub{TUGboat}}
%\def\TUG{\TeX\ \UG}
%\def\tug{\acro{TUG}}
%\def\UG{Users Group}
% ^^A For the bibliography 
%\let\@internalcite\cite
%\def\cite{\def\@citeseppen{-1000}%
%    \def\@cite##1##2{(##1\if@tempswa , ##2\fi)}%
%    \def\citeauthoryear##1##2##3{##1, ##3}\@internalcite}
%\def\etal{et\,al.\@}
%\newcommand\CTANdirectory[1]{\expandafter\urldef
%  \csname CTAN@#1\endcsname\path}
%\newcommand\CTANfile[1]{\expandafter\urldef
%  \csname CTAN@#1\endcsname\path}
%\newcommand\CTANref[1]{\expandafter\@setref\csname CTAN@#1\endcsname
%  \relax{#1}}
%\makeatother
% ^^A Define CTAN addresses 
%\CTANdirectory{mpattern}{graphics/metapost/macros/mpattern}
%\CTANdirectory{pstricks}{graphics/pstricks}
%\CTANdirectory{pst-fill.sty}{graphics/pstricks/latex/pst-fill.sty}
%\CTANdirectory{pst-fill}{graphics/pstricks/generic/pst-fill.tex}
%\CTANdirectory{Roegel}{graphics/metapost/contrib/macros/truchet}
%\CTANdirectory{xypic}{macros/generic/diagrams/xypic}
%
% ^^A Personal macros (D.G.)
% ^^A ----------------------
%
% ^^A Some colors used
%\definecolor{LemonChiffon}{rgb}{1.,0.98,0.8}
%\definecolor{LightBlue}   {rgb}{0.8,0.85,0.95}
%\definecolor{PaleGreen}   {rgb}{0.88,1,0.88}
%\definecolor{PeachPuff}   {rgb}{1.0,0.85,0.73}
%
% ^^A To define a unique string for TeX and LaTeX
%\newcommand{\AllTeX}{%
%{\rm(L\kern-.36em\raise.3ex\hbox{\sc a}\kern-.15em)%
%T\kern-.1667em\lower.7ex\hbox{E}\kern-.125emX}}
%
% ^^A Bibliography style
%\bibliographystyle{ltugbib}
%
% ^^A Name macros
%\newcommand{\FillPackage}{\textsf{`pst-fill'}}
%\newcommand{\XYpic}{%
%\leavevmode\hbox{\kern-.1em X\kern-.3em\lower.4ex\hbox{Y\kern-.15em}-pic}}
%
%\makeatletter
%
% ^^A Example environments
% ^^A (do not use in them the four JXYZ characters, that we will use
% ^^A as escape characters!)
%
% ^^A Default PSTricks parameters
%  \psset{dimen=middle}
%
% ^^A Translation in PSTricks from the one drawn by Emmanuel Chailloux and
% ^^A Guy Cousineau for the MLgraph system
% ^^A (see /ftp.ens.fr:/pub/unix/lang/MLgraph/version-2.1/MLgraph-refman.ps.gz)
% ^^A The kangaroo itself is reproduce from an original picture from Raoul Raba
% \newcommand{\DimX}{2.47}
% \newcommand{\DimY}{4.8}
% \newcommand{\DimXDivTwo}{1.235}
%
% \newcommand{\KangarooItself}[1]{%
% ^^A Body
% \pspolygon[fillstyle=solid,fillcolor=#1]%
%  (52.5,68)(55,72.5)(55.8,76.5)(56.8,79.8)(58.2,83)(60,85.8)(61.5,86.5)
% (64,87)(66,87.5)(67.8,87.3)(70,87)(71.5,87.3)(73,88)(74.7,88.5)
% (76,90.3)(77,91.5)(72.8,93.8)(69,96)(64.5,99)(59.4,103)(56.2,106.3)
% (53,110.5)(49.5,115.5)(47.2,119.9)(45.7,126)(43.2,123)(41.5,121)(37.5,125)
% (37,122.5)(36.8,120)(37,117)(37.6,113.5)(38.6,110)(40,106.3)(42,102.3)
%  (43.5,99.5)(45,97)(46.2,94)(46.8,91.7)(47.2,88)(47,83.5)(46.3,80.8)
%  (45.3,78.5)(42.5,76.5)(39.5,75.8)(36,75.9)(33,75.9)(29,76.2)(26,77)
%  (22.3,77.5)(18,78.4)(12.8,79.3)(8.6,80)(5.5,80.3)(3,80.5)(0,80)
%  (-5.2,78.5)(-9,76.3)(-11.2,74.8)(-13,72.5)(-16.5,68)(-16.5,68)(-19.5,62.5)
%  (-22,58)(-25.5,53)(-29,48.5)(-32.5,45)(-36,42)(-39,39.5)(-44,37)
%  (-49,35)(-51,34)(-53.5,34.5)(-55.5,36)(-56.5,38)(-56.5,40.5)(-55,41.5)
%  (-53.5,41)(-51.5,41)(-50.5,43)(-50.5,44.5)(-51,47)(-51.5,47.2)(-56.5,47)
%  (-58.5,46.5)(-60,44.7)(-62,42.3)(-63,39.5)(-63.5,36.3)(-63.5,33)(-63.1,29.5)
%  (-61.5,26)(-58,23.6)(-54,22.2)(-50.7,22)(-47.5,22)(-44.5,22.3)(-41,23.5)
%  (-36.8,25.8)(-33,28)(-28.5,31)(-23.4,35)(-20.2,38.3)(-17,42.5)(-13.5,47.5)
%  (-11.2,51.9)(-9.7,58)(-7.2,55)(-5.5,53)(-1.5,57)(-1,54.5)(-0.8,52)
%  (-1,49)(-1.6,45.5)(-2.6,42)(-4,38.3)(-6,34.3)(-7.5,31.5)(-9,29)
%  (-10.2,26)(-10.8,23.7)(-11.2,20)(-11,15.5)(-10.3,12.8)(-9.3,10.5)(-6.5,8.5)
%  (-3.5,7.8)(0,7.9)(3,7.9)(7,8.2)(10,9)(13.7,9.5)(18,10.4)
%  (23.2,11.3)(27.4,12)(30.5,12.3)(33,12.5)(36,12)(41.2,10.5)(45,8.3)
%  (47.2,6.8)(49,4.5)(52.5,0)(50,4.5)(49.2,8.5)(48.2,11.8)(46.8,15)
%  (45,17.8)(43.5,18.5)(41,19)(39,19.5)(37.2,19.3)(35,19)(33.5,19.3)
%  (32,20)(30.3,20.5)(29,22.3)(28,23.5)(28,23.5)(24.5,22.3)(21.5,22)
%  (18.3,22)(15,22.2)(11,23.6)(7.5,26)(5.9,29.5)(5.5,33)(5.5,36.3)
%  (6,39.5)(7,42.3)(9,44.7)(10.5,46.5)(12.5,47)(17.5,47.2)(18,47)
%  (18.5,44.5)(18.5,43)(17.5,41)(15.5,41)(14,41.5)(12.5,40.5)(12.5,38)
%  (13.5,36)(15.5,34.5)(18,34)(20,35)(25,37)(30,39.5)(33,42)
%  (36.5,45)(40,48.5)(43.5,53)(47,58)(49.5,62.5)(52.5,68)
% ^^A Eye
% \pscircle*[linecolor=white](58.2,98.3){2\psxunit}
% \pscircle*(58.2,97.3){\psxunit}
% ^^A Mouth
% \psline(71.5,88)(70,89.3)(68.5,90.3)(67,91.9)
% ^^A Tear
% \psline(42,121)(45,118)(47,115.3)(48.5,112.7)(50,110)(51.8,106.5)
%       (52.5,103.7)(53,100.5)
% \pspolygon(41.2,115.8)(43.2,114.7)(45,112.5)(47,109.8)(48,107)(49.5,104.2)%
%       (50.5,101.6)(51,98.5)(47.7,100.6)(46,102.2)(44.8,104)(43.5,106)
%       (42.5,108)(41.7,110.5)(41,113.2)}
%
% \newcommand{\Kangaroo}[1]{%
%   \begin{pspicture}(\DimX,\DimY)
%   \psset{unit=0.035278}
%   \KangarooItself{#1}
%   \end{pspicture}}
%
% \newcommand{\KangarooPstChart}[1]{{%
%   \psset{xunit=0.006784,yunit=0.00735,linewidth=0.01}
%   \begin{pspicture}(-65.5,0)(82,126)
%     \KangarooItself{#1}
%   \end{pspicture}}}
%
%
% ^^A For the possible index and changes log
% \setlength{\columnseprule}{0.6pt}
%
% ^^A Beginning of the documentation itself
%\title{\texttt{pst-fill}\\A PSTricks package for filling and tiling areas}
%\author{Timothy Van Zandt\thanks{\protect\url{tvz@econ.insead.fr}. (documentation by
% Denis Girou (\protect\url{Denis.Girou@idris.fr}) and Herbert Vo\ss (\protect\url{hvoss@tug.org}).}}
%
%\date{\shortstack{\today --- Version 1.00\\
%                  {\small Documentation revised \today}}}
% \maketitle
% \tableofcontents
%
%\begin{abstract}
%  \FillPackage{} is a PSTricks \cite{vanZandt93},\cite{Girou94},\cite{vanZandtGirou94}, 
%\cite{Hoenig97},\cite{LGC97} package to draw easily
%  various kinds of filling and tiling of areas. It is also a good example of
%  the great power and flexibility of PSTricks, as in fact it is a very short
%  program (it body is around 200~lines long) but nevertheless really powerful.
%
%  \hspace{5mm} It was written in 1994 by Timothy \textsc{van Zandt} but
%  publicly available only in PSTricks 97 and without any documentation.
%  We describe here the version \emph{97 patch 2} of December 12, 1997, which
%  is the original one modified by myself to manage \emph{tilings} in the
%  so-called \emph{automatic} mode. This article would like to serve both of
%  reference manual and of user's guide.
%
%This package is available on \CTAN{} in the
%  \texttt{graphics/pstricks} directory (files \texttt{latex/pst-fill.sty} and
%  \texttt{generic/pst-fill.tex}).
%\end{abstract}
%
%\section{Introduction}
%
%  Here we will refer as \emph{filling} as the operation which consist to fill
%a defined area by a pattern (or a composition of patterns). We will refer as
%\emph{tiling} as the operation which consist to do the same thing, but with
%the control of the starting point, which is here the upper left corner.
%The pattern is positioned relatively to this point. This make an essential
%difference between the two modes, as without control of the starting point we
%can't draw \emph{tilings} (sometimes  called \emph{tesselations}) as used in
%many fields of Art and Science%
%\footnote{For an extensive presentation of tilings, in their history and usage
%in many fields, see the reference book \cite{GS87}.
%
%  In the \TeX{} world, few work was done on tilings. You can look at the
%\emph{tile} extension of the \XYpic{} package \cite{XYpic}, at the articles of
%Kees \textsc{van der Laan} \cite[paragraph 7]{LAAN96} (the tiling was in
%fact directly done in PostScript) and \cite{LAAN97}, at the \MP{} program
%(available on \CTANref{Roegel}) by Denis \textsc{Roegel} for the
%\textsc{Truchet} contest in 1995 \cite{EsperetGirou98} and at the \MP{}
%package \cite{Bolek98} to draw patterns, which have a strong connection with
%tilings.}.
%
%  Nevertheless, as tilings are a wide and difficult field in mathematics, this
%package is limited to simple ones, mainly \emph{monohedral} tilings with one
%prototile (which can be composite, see section \ref{sec:KindTiles}). With some
%experience and wiliness we can do more and obtained easily rather
%sophisticated results, but obviously hyperbolic tilings like the famous
%\textsc{Escher} ones or aperiodic tilings like the \textsc{Penrose} ones are
%not in the capabilities of this package. For more complex needs, we must used
%low level and more painfull technics, with the basic \cs{multido}
%and \cs{multirput} macros.
%
%\section{Package history and description of it two different modes}
%
%  As already said, this package was written in 1994 by Timothy \textsc{van
%Zandt}. Two modes were defined, called respectively \emph{manual} and
%\emph{automatic}. For both, the pattern is generated on contiguous positions in
%a rather large area which include the region to fill, later cut to the
%required dimensions by clipping mechanism. In the first mode, the pattern is
%explicitely inserted in the PostScript file each time. In the second one, the
%result is the same but with an unique explicit insertion of the pattern and a
%repetition done by PostScript. Nevertheless, in this method, the control of
%the starting point was loosed, so it allowed only to \emph{fill} a region and
%not to \emph{tile} it.
%
%  See the difference between the two modes, \emph{tiling}:
% {\psset{unit=0.5cm}%
% \psboxfill{\begin{pspicture}(1,1)\psframe[dimen=middle](1,1)\end{pspicture}}
% \begin{pspicture}(3,3.3)
%   \psframe[fillstyle=boxfill](3,3)
% \end{pspicture}}
% and \emph{filling}:
%{%
% \makeatletter
%\pst@def{BoxFill}<
%  gsave
%    gsave \tx@STV CM grestore dtransform CM idtransform
%    abs /h ED abs /w ED
%    pathbbox
%    h div round 2 add cvi /y2 ED
%    w div round 2 add cvi /x2 ED
%    h div round 2 sub cvi /y1 ED
%    w div round 2 sub cvi /x1 ED
%    /y2 y2 y1 sub def
%    /x2 x2 x1 sub def
%    CP
%    y1 h mul sub neg /y1 ED
%    x1 w mul sub neg /x1 ED
%    clip
%    y2 {
%      /x x1 def
%      x2 {
%        save CP x y1 T moveto Box restore
%        /x x w add def
%      } repeat
%      /y1 y1 h add def
%    } repeat
% currentpoint currentfont grestore setfont moveto>
% \makeatother
%
% \psset{unit=0.5}
% \psboxfill{\begin{pspicture}(1,1)\psframe[dimen=middle](1,1)\end{pspicture}}
% \begin{pspicture}(3,3.3)
%   \psframe[fillstyle=boxfill](3,3)
% \end{pspicture}
% or
% \begin{pspicture}(3,3.3)
%   \psframe[fillstyle=boxfill](3,3)
% \end{pspicture}
%}
%as we can see that initial position is arbitrary and dependent of
%the current point.
%
%
% It's clear that usage of filling is very restrictive comparing to tiling,
%as desired effects required very often the possibility to control the starting 
%point. So, this mode was of limited interest, but unfortunately the
%\emph{manual} one has the very big disadvantage to require very huge amounts
%of ressources, mainly in disk space and consequently in printing time.
%A small tiling can require sometimes several megabytes in \emph{manual} mode!
%So, it was very often not really usable in practice.
%
%It is why I modified the code, to allow tilings in \emph{automatic} mode,
%controlling in this mode too the starting point. And most of the time, that is
%to say if some special options are not used, the tiling is done exactly in the
%region described, which make it faster. So there is no more reason to use the
%\emph{manual} mode, apart very special cases where \emph{automatic} one cannot
%work, as explained later -- currently, I know only one case.
%
%  To load this modified \emph{automatic} mode, with \LaTeX{} use
%simply:\newline 
%\verb+\usepackage[tiling]{pst-fill}+\newline
%and in plain \TeX{} after:\newline
%\verb+\input{pst-fill}+\newline
%add the following definition:\newline
%\verb+\def\PstTiling{true}+
%
%  To obtain the original behaviour, just don't use the \emph{tiling} optional
%keyword at loading.
%
%  Take care than in \emph{tiling} mode, I introduce also some other changes.
%First I define aliases on some parameter names for consistancy (all specific
%parameters will begin by the \texttt{fill} prefix in this case) and I change
%some default values, which were not well adapted for tilings (\texttt{fillsep}
%is set to 0 and as explained \texttt{fillsize} set to \texttt{auto}). I rename 
%\texttt{fillcycle} to \texttt{fillcyclex}. I also restore normal way so that
%the frame of the area is drawn and all line (\texttt{linestyle},
%\texttt{linecolor}, \texttt{doubleline}, etc.) parameters are now active (but
%there are not in non \emph{tiling} mode). And I also introduce new parameters
%to control the tilings (see below).
%
%  \textbf{In all the following examples, we will consider only the
% \emph{tiling} mode.}
%
%  To do a tiling, we have just to define the pattern with the
% \verb+\psboxfill+ macro and to use the new \texttt{fillstyle}
% \verb+boxfill+.
%
%  Note that tilings are drawn from left to right and top to bottom, which can
%have an importance in some circonstances.
%
%  PostScript programmers can be also interested to know that, even in the
%\emph{automatic} mode, the iterations of the pattern are managed directly by
%the PostScript code of the package which used only PostScript Level 1
%operators. The special ones introduced in Level 2 for drawing of patterns
%\cite[section 4.9]{PostScript95} are not used.
%
%  And first, for conveniance, we define a simple \cs{Tiling} macro, which
%will simplify our examples:
%
%\begin{verbatim}
%  \newcommand{\Tiling}[2][]{%
%    \edef\Temp{#1}%
%    \begin{pspicture}#2
%      \ifx\Temp\empty
%        \psframe[fillstyle=boxfill]#2
%      \else
%        \psframe[fillstyle=boxfill,#1]#2
%      \fi
%    \end{pspicture}}
%\end{verbatim}
%
%
%\newcommand{\Tiling}[2][]{%
%  \edef\Temp{#1}%
%  \begin{pspicture}#2
%    \ifx\Temp\empty
%      \psframe[fillstyle=boxfill]#2
%    \else
%      \psframe[fillstyle=boxfill,#1]#2
%    \fi
% \end{pspicture}}
%
%\subsection{Parameters}
%
%  There are \textbf{14} specific parameters available to change the way the
% filling/tiling is defined, and one debugging option.
%
% \begin{Description}{2cm}
%  \item [fillangle (real)\hfill :] the value of the rotation
%  applied to the patterns (\emph{Default:~0}).
% \end{Description}
%
%
%   In this case, we must force the tiling area to be notably larger than the
% area to cover, to be sure that the defined area will be covered after rotation.
% \lstset{gobble=2}
% \begin{LTXexample}
% \newcommand{\Square}{%
%   \begin{pspicture}(1,1)
%     \psframe[dimen=middle](1,1)
%   \end{pspicture}}
% \psset{unit=0.5}
% \psboxfill{\Square}
% \Tiling[fillangle=45]{(3,3)}\quad
% \Tiling[fillangle=-60]{(3,3)}
% \end{LTXexample}
% 
% \newcommand{\Square}{\begin{pspicture}(1,1)\psframe[dimen=middle](1,1)\end{pspicture}}
% 
% \begin{Description}{2cm}
%   \setcounter{footnote}{1}
%   \item[\texttt{fillsepx} (real$\|$dim) :] value of the horizontal
%   separation between consecutive patterns (\emph{Default:~0 for
%   tilings\footnotemark, 2pt otherwise}).  \footnotetext{This option was added
%   by me, is not part of the original package and is available only if the
%   \texttt{tiling} keyword is used when loading the package.}
%   \setcounter{footnote}{1}
%   \item [\texttt{fillsepy} (real$\|$dim)\hfill :] value of the vertical
%   separation between consecutive patterns (\emph{Default:~0 for
%   ti\-lings\footnotemark, 2pt otherwise}).
%   \setcounter{footnote}{1}
%   \item [\texttt{fillsep} (real$\|$dim)\hfill :] value of horizontal and
%   vertical separations between consecutive patterns (\emph{Default:~0 for
%   tilings\footnotemark, 2pt otherwise}).
% \end{Description}
% 
%   These values can be negative, which allow the tiles to overlap.
% 
% \begin{LTXexample}
% \psset{unit=0.5}
% \psboxfill{\Square}
% \Tiling[fillsepx=2mm]{(3,3)} 
% \Tiling[fillsepy=1mm]{(3,3)}\\
% \Tiling[fillsep=0.5]{(3,3)} 
% \Tiling[fillsep=-0.5]{(3,3)}
% \end{LTXexample}
% 
% \begin{Description}{2cm}
%   \item [\texttt{fillcyclex}\footnotemark\ (integer)\hfill :] Shift
%   coefficient applied to each row (\emph{Default:~0}).
%   \footnotetext{It was \texttt{fillcycle} in the original version.}
%   \setcounter{footnote}{1}
%   \item [\texttt{fillcycley}\footnotemark\ (integer)\hfill :] Same thing for
%   columns (\emph{Default:~0}).
%   \setcounter{footnote}{1}
%   \item [\texttt{fillcycle}\footnotemark\ (integer)\hfill :] Allow to fix
%   both \texttt{fillcyclex} and \texttt{fillcycley} directly to the same value
%   (\emph{Default:~0}).
% \end{Description}
% 
%   For instance, if \texttt{fillcyclex} is 2, the second row of patterns will
% be horizontally shifted by a factor of $\frac{1}{2}=0.5$, and by a factor of
% 0.333 if \texttt{fillcyclex} is 3, etc.). These values can be negative.
% 
% \begin{LTXexample}[width=0.35\linewidth]
% \psset{unit=0.5}
% \psboxfill{\Square}
% \newcommand{\TilingA}[1]{\Tiling[fillcyclex=#1]{(3,3)}}
% \TilingA{0} \TilingA{1}\\
% \TilingA{2} \TilingA{3}\\[3mm]
% \TilingA{4} \TilingA{5}\\
% \TilingA{6} \TilingA{-3}\\[3mm]
% \Tiling[fillcycley=2]{(3,3)}
% \Tiling[fillcycley=3]{(3,3)}\\
% \Tiling[fillcycley=-3]{(3,3)}
% \Tiling[fillcycle=2]{(3,3)}
% \end{LTXexample}
% 
% \begin{Description}{2cm}
%   \setcounter{footnote}{1}
%   \item [\texttt{fillmovex}\footnotemark\ (real$\|$dim)\hfill :] value of the
%   horizontal moves between consecutive patterns (\emph{Default:~0}).
%   \setcounter{footnote}{1}
%   \item [\texttt{fillmovey}\footnotemark\ (real$\|$dim)\hfill :] value of the
%   vertical moves between consecutive patterns (\emph{Default:~0}).
%   \setcounter{footnote}{1}
%   \item [\texttt{fillmove}\footnotemark\ (real$\|$dim)\hfill :] value of
%   horizontal and vertical moves between consecutive patterns
%   (\emph{Default:~0}).
% \end{Description}
% 
%   These parameters allow the patterns to overlap and to draw some special
% kinds of tilings. They are implemented only for the \emph{automatic} and
% \emph{tiling} modes and their values can be negative.
% 
%   In some cases, the effect of these parameters will be the same that with the 
% \texttt{fillcycle?} ones, but you can see that it is not true for some other
% values.
% 
% \begin{LTXexample}
% \psset{unit=0.5}
% \psboxfill{\Square}
% \Tiling[fillmovex=0.5]{(3,3)} 
% \Tiling[fillmovey=0.5]{(3,3)}\\
% \Tiling[fillmove=0.5]{(3,3)}
% \Tiling[fillmove=-0.5]{(3,3)}
% \end{LTXexample}
% 
% \begin{Description}{2cm}
%   \item [\texttt{fillsize}
%   (auto$\|$\{(real$\|$dim,real$\|$dim)(real$\|$dim,real$\|$dim)\}) :] The
%   choice of \emph{automatic} mode or the size of the area in \emph{manual}
%   mode. If first pair values are not given, (0,0) is used. (\emph{Default:
%   auto when \emph{tiling} mode is used, {(-15cm,-15cm)(15cm,15cm)}
%   otherwise}).
% \end{Description}
% 
%   As explained in the introduction, the \emph{manual} mode can require very
% huge amount of computer ressources. So, it usage is to discourage in front off
% the \emph{automatic} mode. It seems only useful in special circonstances, in
% fact when the \emph{automatic} mode failed, which is known only in one case,
% for some kinds of EPS files, as the ones produce by dump of portions of
% screens (see \ref{sec:GraphicFiles}).
% 
% \begin{Description}{2cm}
%   \setcounter{footnote}{1}
%   \item [\texttt{fillloopaddx}\footnotemark\ (integer)\hfill :] number of
%   times the pattern is added on left and right positions (\emph{Default:~0}).
%   \setcounter{footnote}{1}
%   \item [\texttt{fillloopaddy}\footnotemark\ (integer)\hfill :] number of
%   times the pattern is added on top and bottom positions (\emph{Default:~0}).
%   \setcounter{footnote}{1}
%   \item [\texttt{fillloopadd}\footnotemark\ (integer)\hfill :] number of
%   times the pattern is added on left, right, top and bottom positions
%   (\emph{Default:~0}).
% \end{Description}
% 
%   These parameters are only useful in special circonstances, as for complex
% patterns when the size of the rectangular box used to tile the area doesn't 
% correspond to the pattern itself (see an example in Figure~\ref{fig:Sheeps})
% and also sometimes when the size of the pattern is not a divisor of the size
% of the area to fill and that the number of loop repeats is not properly
% computed, which can occur.
% 
%   They are implemented only for the \emph{tiling} mode.
% 
% \begin{Description}{2cm}
%   \setcounter{footnote}{1}
%   \item [\texttt{PstDebug}\footnotemark\ (integer, 0 or 1)\hfill :] to
%   require to see the exact tiling done, without clipping (\emph{Default:~0}).
% \end{Description}
% 
%   It's mainly useful for debugging or to understand better how the tilings
% are done. It is implemented only for the \emph{tiling} mode.
% 
% \begin{LTXexample}
% \psset{unit=0.3,PstDebug=1}
% \psboxfill{\Square}
% \psset{linewidth=1mm}
% \Tiling{(2,2)}\\[5mm]
% \Tiling[fillcyclex=2]{(2,2)}\\[1cm]
% \Tiling[fillmove=0.5]{(2,2)}
% \end{LTXexample}
% 
% \vspace{3cm}
% \section{Examples}
% 
%   In fact this unique \cs{psboxfill} macro allow a lot a variations and
% different usages. We will try here to demonstrate this.
% 
% \subsection{Kind of tiles}
% \label{sec:KindTiles}
% 
%   Of course, we can access to all the power of PSTricks macros to define the
% \emph{tiles} (\emph{patterns}) used. So, we can define complicated ones.
% 
%   Here we give four other Archimedian tilings (those built with only some
% regular polygons) among the twelve existing, first discovered completely by
% Johanes \textsc{Kepler} at the beginning of 17th century \cite{GS87}, the two
% other \emph{regular} ones with the tiling by squares, formed by a unique
% regular polygon, and two other formed by two different regular polygons.
% 
% \begin{LTXexample}[pos=t]
%   \newcommand{\Triangle}{%
%     \begin{pspicture}(1,1)
%       \pstriangle[dimen=middle](0.5,0)(1,1)
%     \end{pspicture}}
%   \newcommand{\Hexagon}{
% ^^A sin(60)=0.866
%     \begin{pspicture}(0.866,0.75)
%       \SpecialCoor
% ^^A  Hexagon  
%       \pspolygon[dimen=middle]%
%         (0.5;30)(0.5;90)(0.5;150)(0.5;210)(0.5;270)(0.5;330)
%     \end{pspicture}}
% 
%   \psset{unit=0.5}
%   \psboxfill{\Triangle}
%   \Tiling{(4,4)}\hfill
% ^^A The two other regular tilings
%   \Tiling[fillcyclex=2]{(4,4)}\hfill
%   \psboxfill{\Hexagon}
%   \Tiling[fillcyclex=2,fillloopaddy=1]{(5,5)}
% \end{LTXexample}
% 
% \begin{LTXexample}[pos=t]
%   \newcommand{\ArchimedianA}{%
%      ^^A Archimedian tiling 3^2.4.3.4
%     \psset{dimen=middle}
%      ^^A sin(60)=0.866
%     \begin{pspicture}(1.866,1.866)
%       \psframe(1,1)
%       \psline(1,0)(1.866,0.5)(1,1)(0.5,1.866)(0,1)(-0.866,0.5)
%       \psline(0,0)(0.5,-0.866)
%     \end{pspicture}}
%   \newcommand{\ArchimedianB}{%
%      ^^A Archimedian tiling 4.8^2
%     \psset{dimen=middle,unit=1.5}
%      ^^A sin(22.5)=0.3827 ; cos(22.5)=0.9239
%     \begin{pspicture}(1.3066,0.6533)
%       \SpecialCoor
%      ^^A Octogon
%       \pspolygon(0.5;22.5)(0.5;67.5)(0.5;112.5)(0.5;157.5)
%                 (0.5;202.5)(0.5;247.5)(0.5;292.5)(0.5;337.5)
%     \end{pspicture}}
% 
%   \psset{unit=0.5}
%   \psboxfill{\ArchimedianA}
%   \Tiling[fillmove=0.5]{(7,7)}\hfill
%   \psboxfill{\ArchimedianB}
%   \Tiling[fillcyclex=2,fillloopaddy=1]{(7,7)}
% \end{LTXexample}
% 
%   \setcounter{footnote}{3}
%   We can of course tile an area arbitrarily defined. And with the
% \texttt{addfillstyle} parameter\footnote{Introduced in PSTricks 97.}, we can
% easily mix the \texttt{boxfill} style with another one.
% 
% \begin{LTXexample}[width=6cm]
%   \psset{unit=0.5,dimen=middle}
%   \psboxfill{%
%     \begin{pspicture}(1,1)
%       \psframe(1,1)
%       \pscircle(0.5,0.5){0.25}
%     \end{pspicture}}
%   \begin{pspicture}(4,6)
%     \pspolygon[fillstyle=boxfill,fillsep=0.25](0,1)(1,4)(4,6)(4,0)(2,1)
%   \end{pspicture}\hspace{1em}
%   \begin{pspicture}(4,4)
%%     \pscircle[linestyle=none,fillstyle=solid,fillcolor=yellow,fillsep=0.5,
%%               addfillstyle=boxfill](2,2){2}
%   \end{pspicture}
% \end{LTXexample}
%
%   Various effects can be obtained, sometimes complicated ones very easily, as
% in this example reproduced from one shown by Slavik \textsc{Jablan} in the
% field of \emph{OpTiles}, inspired by the \emph{Op-art}:
% 
% \begin{LTXexample}[pos=t]
% \newcommand{\ProtoTile}{%
%  \begin{pspicture}(1,1)%%% 1/12=0.08333
%   \psset{linestyle=none,linewidth=0,
%     hatchwidth=0.08333\psunit,hatchsep=0.08333\psunit}
%   \psframe[fillstyle=solid,fillcolor=black,addfillstyle=hlines,hatchcolor=white](1,1)
%   \pswedge[fillstyle=solid,fillcolor=white,addfillstyle=hlines]{1}{0}{90}
%  \end{pspicture}}
% \newcommand{\BasicTile}{%
%  \begin{pspicture}(2,1)
%    \rput[lb](0,0){\ProtoTile}\rput[lb](1,0){\psrotateleft{\ProtoTile}}
%  \end{pspicture}}
% \ProtoTile\hfill\BasicTile\hfill
% \psboxfill{\BasicTile}
% \Tiling[fillcyclex=2]{(4,4)}
% \end{LTXexample}
% 
%   It is also directly possible to surimpose several different tilings. Here is
% the splendid visual proof of the \textsc{Pytha\-gore} theorem done by the arab
% mathematician \textsc{Annairizi} around the year 900, given by superposition
% of two tilings by squares of different sizes.
% 
% \begin{LTXexample}[pos=t]
% \psset{unit=1.5,dimen=middle}
% \begin{pspicture*}(3,3)
%   \psboxfill{\begin{pspicture}(1,1)
%     \psframe(1,1)\end{pspicture}}
%   \psframe[fillstyle=boxfill](3,3)
%   \psboxfill{\begin{pspicture}(1,1)
%     \rput{-37}{\psframe[linecolor=red](0.8,0.8)}
%   \end{pspicture}}
%   \psframe[fillstyle=boxfill](3,4)
%   \pspolygon[fillstyle=hlines,hatchangle=90](1,2)(1.64,1.53)(2,2)
% \end{pspicture*}
% \end{LTXexample}
% 
%   In a same way, it is possible to build tilings based on figurative patterns,
% in the style of the famous \textsc{Escher} ones. Following an example of
% Andr\'e \textsc{Deledicq} \cite{Deledicq97}, we first show a simple tiling of
% the \emph{p1} category (according to the international classification of the
% 17~symmetry groups of the plane first discovered by the russian
% crystalographer Jevgraf \textsc{Fedorov} at the end of the 19th century):
% 
% \begin{LTXexample}[pos=t]
%  \newcommand{\SheepHead}[1]{%
%    \begin{pspicture}(3,1.5)
%      \pscustom[liftpen=2,fillstyle=solid,fillcolor=#1]{%
%        \pscurve(0.5,-0.2)(0.6,0.5)(0.2,1.3)(0,1.5)(0,1.5)
%          (0.4,1.3)(0.8,1.5)(2.2,1.9)(3,1.5)(3,1.5)(3.2,1.3)
%          (3.6,0.5)(3.4,-0.3)(3,0)(2.2,0.4)(0.5,-0.2)}
%      \pscircle*(2.65,1.25){0.12\psunit} % Eye
%      \psccurve*(3.5,0.3)(3.35,0.45)(3.5,0.6)(3.6,0.4)% Muzzle
%     ^^A   % Mouth
%       \pscurve(3,0.35)(3.3,0.1)(3.6,0.05)
%     ^^A   % Ear
%       \pscurve(2.3,1.3)(2.1,1.5)(2.15,1.7)\pscurve(2.1,1.7)(2.35,1.6)(2.45,1.4)
%   \end{pspicture}}
%  \psboxfill{\psset{unit=0.5}\SheepHead{yellow}\SheepHead{cyan}}
%  \Tiling[fillcyclex=2,fillloopadd=1]{(10,5)}
% \end{LTXexample}
% \label{fig:Sheeps}
% 
%   Now a tiling of the \emph{pg} category (the code for the kangaroo itself is
% too long to be shown here, but has no difficulties ; the kangaroo is reproduce
% from an original picture from Raoul \textsc{Raba} and here is a translation in
% PSTricks from the one drawn by Emmanuel \textsc{Chailloux} and Guy
% \textsc{Cousineau} for their MLgraph system \cite{MLgraphTSI}):
% 
% \begin{LTXexample}[pos=t]
% \psboxfill{\psset{unit=0.4}
%   \Kangaroo{yellow}\Kangaroo{red}\Kangaroo{cyan}\Kangaroo{green}%
%   \psscalebox{-1 1}{%
%     \rput(1.235,4.8){\Kangaroo{green}\Kangaroo{cyan}\Kangaroo{red}\Kangaroo{yellow}}}}
%   \Tiling[fillloopadd=1]{(10,6)}
% \end{LTXexample}
% 
%   And here a \textsc{Wang} tiling \cite{Wang65}, \cite[chapter
% 11]{GS87}, based on very simple tiles of the form of a square and composed
% of four colored triangles. Such tilings are built with only a matching color
% constraint. Despite of it simplicity, it is an important kind of tilings, as
% \textsc{Wang} and others used them to study the special class of
% \emph{aperiodic} tilings, and also because it was shown that surprisingly this 
% tiling is similar to a \textsc{Turing} machine.
% 
% \begin{LTXexample}[pos=t]
%   \newcommand{\WangTile}[4]{%
%     \begin{pspicture}(1,1)
%       \pspolygon*[linecolor=#1](0,0)(0,1)(0.5,0.5)
%       \pspolygon*[linecolor=#2](0,1)(1,1)(0.5,0.5)
%       \pspolygon*[linecolor=#3](1,1)(1,0)(0.5,0.5)
%       \pspolygon*[linecolor=#4](1,0)(0,0)(0.5,0.5)
%     \end{pspicture}}
%   \newcommand{\WangTileA}{\WangTile{cyan}{yellow}{cyan}{cyan}}
%   \newcommand{\WangTileB}{\WangTile{yellow}{cyan}{cyan}{red}}
%   \newcommand{\WangTileC}{\WangTile{cyan}{red}{yellow}{yellow}}
%   \newcommand{\WangTiles}[1][]{%
%     \begin{pspicture}(3,3) \psset{ref=lb}
%       \rput(0,2){\WangTileB}  \rput(1,2){\WangTileA}%
%       \rput(2,2){\WangTileC}  \rput(0,1){\WangTileC}%
%       \rput(1,1){\WangTileB}  \rput(2,1){\WangTileA}
%       \rput(0,0){\WangTileA}  \rput(1,0){\WangTileC}%
%       \rput(2,0){\WangTileB}
%       #1
%     \end{pspicture}}
%   \WangTileA\hfill\WangTileB\hfill\WangTileC\hfill
%   \WangTiles[{\psgrid[subgriddiv=0,gridlabels=0](3,3)}]\hfill
%   \psset{unit=0.4} \psboxfill{\WangTiles} \Tiling{(12,12)}
% \end{LTXexample}
% 
% \subsection{External graphic files}
% \label{sec:GraphicFiles}
% 
%   We can also fill an arbitrary area with an external image. We have only, 
% as usual, to matter of the \emph{BoundingBox} definition if there is no one
% provided or if it is not the accurate one, as for the well known
% \texttt{tiger} picture part of the \texttt{ghostscript} distribution.
% 
% \begin{LTXexample}[pos=t]
%   \psboxfill{%% Strangely require x1=x2...
%     \begin{pspicture}(0,1)(0,4.1)
%       \includegraphics[bb=17 176 560 74,width=3cm]{tiger}
%     \end{pspicture}}
%   \Tiling{(6,6.2)}
% \end{LTXexample}
% 
%   Nevertheless, there are some special files for which the \emph{automatic}
% mode doesn't work, specially for some files obtained by a screen dump, as in
% the next example, where a picture was reduced before it conversion in the
% \emph{Encapsulated PostScript} format by a screen dump utility. In this case,
% usage of the \emph{manual} mode is the only alternative, at the price of the
% real multiple inclusion of the EPS file. We must take care to specify the
% correct \texttt{fillsize} parameter, because otherwise the default values are
% large and will load the file many times, perhaps just really using few
% occurrences as the other ones would be clipped...
% 
% \begin{LTXexample}[pos=t]
%   \psboxfill{\includegraphics{flowers}}
%   \begin{pspicture}(8,4)
%     \psellipse[fillstyle=boxfill,fillsize={(8,4)}](4,2)(4,2)
%   \end{pspicture}
% \end{LTXexample}
% 
% \subsection{Tiling of characters}
% 
%   We can also use the \cs{psboxfill} macro to fill the interior of characters
% for special effects like these ones:
% 
% \begin{LTXexample}[pos=t]
%   \DeclareFixedFont{\bigsf}{T1}{phv}{b}{n}{4.5cm}
%   \DeclareFixedFont{\smallrm}{T1}{ptm}{m}{n}{3mm}
%   \psboxfill{\smallrm Since 182 days...}
%   \begin{pspicture*}(8,4)
%     \centerline{%
%       \pscharpath[fillstyle=gradient,gradangle=-45,
%                   gradmidpoint=0.5,addfillstyle=boxfill,
%                   fillangle=45,fillsep=0.7mm]
%                  {\rput[b](0,0.1){\bigsf 2000}}}
%   \end{pspicture*}
% \end{LTXexample}
% 
% \begin{LTXexample}[pos=t]
%   \DeclareFixedFont{\mediumrm}{T1}{ptm}{m}{n}{2cm}
%   \psboxfill{%
%     \psset{unit=0.1,linewidth=0.2pt}
%     \Kangaroo{PeachPuff}\Kangaroo{PaleGreen}%
%       \Kangaroo{LightBlue}\Kangaroo{LemonChiffon}%
%     \psscalebox{-1 1}{%
%       \rput(1.235,4.8){%
%         \Kangaroo{LemonChiffon}\Kangaroo{LightBlue}%
%           \Kangaroo{PaleGreen}\Kangaroo{PeachPuff}}}}
% ^^A   % A kangaroo of kangaroos...
%   \begin{pspicture}(8,2)
%     \pscharpath[linestyle=none,fillstyle=boxfill,fillloopadd=1]
%                {\rput[b](4,0){\mediumrm Kangaroo}}
%   \end{pspicture}
% \end{LTXexample}
% 
% \subsection{Other kinds of usage}
% 
%   Other kinds of usage can be imagined. For instance, we can use tilings in a
% sort of degenerated way to draw some special lines made by a unique or
% multiple repeating patterns. But it can be only a special dashed line, as here
% with three different dashes:
% 
% \begin{LTXexample}[pos=t]
%   \newcommand{\Dashes}{%
%     \psset{dimen=middle}
%     \begin{pspicture}(0,-0.5\pslinewidth)(1,0.5\pslinewidth)
%       \rput(0,0){\psline(0.4,0)}%
%         \rput(0.5,0){\psline(0.2,0)}%
%         \rput(0.8,0){\psline(0.1,0)}
%     \end{pspicture}}
% 
%   \newcommand{\SpecialDashedLine}[3]{%
%     \psboxfill{#3}
%     \Tiling[linestyle=none]
%            {(#1,-0.5\pslinewidth)(#2,0.5\pslinewidth)}}
% 
%   \SpecialDashedLine{0}{7}{\Dashes}
% 
%   \psset{unit=0.5,linewidth=1mm,linecolor=red}
%   \SpecialDashedLine{0}{10}{\Dashes}
% \end{LTXexample}
% 
%   It allow also to use special patterns in business graphics, as in the
% following example generated by \texttt{PstChart}\footnote{A personal
% development to draw business charts with PSTricks, not distributed.}.
% 
% \vspace{3mm}
% \begin{figure}[!ht]
% \centering
% \psset{unit=0.75}
% ^^A % Generated by pstchart.sh version 0.21 (11/28/97)
% {\psset{dimen=middle}
% \psset{xunit=2,yunit=0.005}
% \begin{pspicture}(-0.6,-200)(6.6,2300)
% ^^A   % Title
%   \rput(3,2200){\shortstack{Fantaisist repartition of kangaroos\\
%                             in the world (in thousands)}}
% ^^A   % Frame background
%   \psframe[fillstyle=solid,fillcolor=LemonChiffon](0,0)(6,2000)
% ^^A   % Graduations
%   \multido{\n=0+500}{5}{\rput[r](-0.12,\n){\psscalebox{0.8}{\n}}}
% ^^A   % Minor ticks
%   \multips(0,100)(0,100){19}{\psline[unit=4.8pt](1,0)}
%   \multips(6,100)(0,100){19}{\psline[unit=4.8pt](-1,0)}
% ^^A   % Major ticks
%   \multips(0,500)(0,500){3}{\psline[unit=9.6pt](1,0)}
%   \multips(6,500)(0,500){3}{\psline[unit=9.6pt](-1,0)}
% ^^A   % Lines from major ticks marks
%   \multips(0,500)(0,500){3}{\psline[linestyle=dotted,linewidth=0.6pt](6,0)}
% ^^A   % Drawing for the data
%   \psboxfill{\psset{unit=0.78\psxunit}\KangarooPstChart{red}}
%   \psframe[linestyle=none,fillstyle=boxfill,fillloopaddy=1](0.61,0)(1.39,1800)
%   \psboxfill{\psset{unit=0.78\psxunit}\KangarooPstChart{yellow}}
%   \psframe[linestyle=none,fillstyle=boxfill,fillloopaddy=1](1.61,0)(2.39,800)
%   \psboxfill{\psset{unit=0.78\psxunit}\KangarooPstChart{cyan}}
%   \psframe[linestyle=none,fillstyle=boxfill,fillloopaddy=1](2.61,0)(3.39,550)
%   \psboxfill{\psset{unit=0.78\psxunit}\KangarooPstChart{magenta}}
%   \psframe[linestyle=none,fillstyle=boxfill,fillloopaddy=1](3.61,0)(4.39,500)
%   \psboxfill{\psset{unit=0.78\psxunit}\KangarooPstChart{green}}
%   \psframe[linestyle=none,fillstyle=boxfill,fillloopaddy=1](4.61,0)(5.39,200)
% ^^A   % Bottom labels
%   \uput{0.2}[270]{0}(1,0){\psscalebox{0.7}{Oceania}}
%   \uput{0.2}[270]{0}(2,0){\psscalebox{0.7}{Africa}}
%   \uput{0.2}[270]{0}(3,0){\psscalebox{0.7}{Asia}}
%   \uput{0.2}[270]{0}(4,0){\psscalebox{0.7}{America}}
%   \uput{0.2}[270]{0}(5,0){\psscalebox{0.7}{Europe}}
% ^^A   % Frame box around the chart
%   \psframe[linestyle=solid](0,0)(6,2000)
% \end{pspicture}}
%   \caption{Bar chart generated by PstChart, with bars filled by patterns}
%   \label{fig:PstChart}
% \end{figure}
% 
% \section{``Dynamic'' tilings}
% 
%   In some cases, tilings used non \emph{static} tiles, that is to say that the 
% \emph{prototile(s)}, even if unique, can have several forms, by instance
% specified by different colors or rotations, not fixed before generation or
% varying each time.
% 
% \subsection{Lewthwaite-Pickover-Truchet tiling}
% 
%   We give here for example the so-called \emph{Truchet} tiling, which much be
% in fact better called \emph{Lewthwaite-Pick\-over-Truchet (LPT)} tiling%
% \footnote{For description of the context, history and references about
% S\'ebastien \textsc{Truchet} and this tiling, see \cite{EsperetGirou98}.}.
% 
%   The unique prototile is only a square with two opposite circle arcs.
% This tile has obviously two positions, if we rotate it from 90 degrees (see
% the two tiles on the next figure). A \emph{LPT tiling} is a tiling with
% randomly oriented LPT tiles. We can see that even if it is very simple in it
% principle, it draw sophisticated curves with strange properties.
% 
%   Nevertheless, in the straightforward way \FillPackage{} does not work,
% because the \cs{psboxfill} macro store the content of the tile used in a
% \TeX{} box, which is static. So the calling to the random function is done
% only one time, which explain that only one rotation of the tile is used for
% all the tiling. It's only the one of the two rotations which could differ from
% one drawing to the next one...
% 
% ^^A % Truchet (Lewthwaite-Pickover-Truchet) tiling
% ^^A % --------------------------------------------
% 
% \begin{LTXexample}[pos=t]
% ^^A   % LPT prototile
%   \newcommand{\ProtoTileLPT}{%
%     \psset{dimen=middle}
%     \begin{pspicture}(1,1)
%       \psframe(1,1)
%       \psarc(0,0){0.5}{0}{90}
%       \psarc(1,1){0.5}{-180}{-90}
%     \end{pspicture}}
% 
% ^^A   % LPT tile
%   \newcount\Boolean
%   \newcommand{\BasicTileLPT}{%
% ^^A     % From random.tex by Donald Arseneau
%     \setrannum{\Boolean}{0}{1}%
%     \ifnum\Boolean=0
%       \ProtoTileLPT%
%     \else
%       \psrotateleft{\ProtoTileLPT}%
%     \fi}
% 
%   \ProtoTileLPT\hfill\psrotateleft{\ProtoTileLPT}\hfill
%   \psset{unit=0.5}
%   \psboxfill{\BasicTileLPT}
%   \Tiling{(5,5)}
% \end{LTXexample}
% 
%   But, for simple cases, there is a solution to this problem using a mixture
% of PSTricks and PostScript programming. Here the PSTricks
% construction \verb+\pscustom{\code{...}}+ allow to insert PostScript code
% inside the \LaTeX{} + PSTricks one.
% 
%   Programmation is less straightforward, but it has also the advantage to be
% notably faster, as all the tilings operations are done in PostScript, and
% mainly to not be limited by \TeX{} memory (the \TeX{} + PSTricks solution
% I wrote in 1995 for the colored problem was limited to small sizes for this
% reason). Just note also that \cs{pslbrace} and \cs{psrbrace} are two
% PSTricks macros to define and be able to insert the \verb+{+ and \verb+}+
% characters.
% 
% \begin{LTXexample}[pos=t]
% ^^A   % LPT prototile
%   \newcommand{\ProtoTileLPT}{%
%     \psset{dimen=middle}
%     \psframe(1,1)
%     \psarc(0,0){0.5}{0}{90}
%     \psarc(1,1){0.5}{-180}{-90}}
% 
% ^^A   % Counter to change the random seed
%   \newcount\InitCounter
% ^^A   % LPT tile
%   \newcommand{\BasicTileLPT}{%
%     \InitCounter=\the\time
%     \pscustom{\code{%
%       rand \the\InitCounter\space sub 2 mod 0 eq \pslbrace}}
%     \begin{pspicture}(1,1)
%       \ProtoTileLPT
%     \end{pspicture}%
%     \pscustom{\code{\psrbrace \pslbrace}}
%     \psrotateleft{\ProtoTileLPT}%
%     \pscustom{\code{\psrbrace ifelse}}}
% 
%   \psset{unit=0.4,linewidth=0.4pt}
%   \psboxfill{\BasicTileLPT}
%   \Tiling{(15,15)}
% \end{LTXexample}
% 
%   Using the very surprising fact (see \cite{EsperetGirou98}) that
% coloration of these tiles do not depend of their neighbors (even if it is
% difficult to believe as the opposite seems obvious!) but only of the parity of
% the value of row and column positions, we can directly program in the same way
% a colored version of the LPT tiling.
% 
% \setcounter{footnote}{1}
%   We have also introduce in the \FillPackage{} code for \emph{tiling} mode two
% new accessible Post\-Script variables, \texttt{row} and
% \texttt{column}\footnotemark, which can be useful in some circonstances, like
% this one.
% 
% \begin{LTXexample}[pos=t]
% ^^A   % LPT prototile
%   \newcommand{\ProtoTileLPT}[2]{%
%     \psset{dimen=middle,linestyle=none,fillstyle=solid}
%     \psframe[fillcolor=#1](1,1)
%     \psset{fillcolor=#2}
%     \pswedge(0,0){0.5}{0}{90} \pswedge(1,1){0.5}{-180}{-90}}
% ^^A   % Counter to change the random seed
%   \newcount\InitCounter
% ^^A   % LPT tile
%   \newcommand{\BasicTileLPT}[2]{%
%     \InitCounter=\the\time
%     \pscustom{\code{%
%       rand \the\InitCounter\space sub 2 mod 0 eq \pslbrace
%       row column add 2 mod 0 eq \pslbrace}}
%     \begin{pspicture}(1,1)\ProtoTileLPT{#1}{#2}\end{pspicture}%
%     \pscustom{\code{\psrbrace \pslbrace}}
%     \ProtoTileLPT{#2}{#1}%
%     \pscustom{\code{%
%       \psrbrace ifelse \psrbrace \pslbrace row column add 2 mod 0 eq \pslbrace}}
%     \psrotateleft{\ProtoTileLPT{#2}{#1}}\pscustom{\code{\psrbrace \pslbrace}}
%     \psrotateleft{\ProtoTileLPT{#1}{#2}}\pscustom{\code{\psrbrace ifelse \psrbrace ifelse}}}
%   \psboxfill{\BasicTileLPT{red}{yellow}}
%   \Tiling{(4,4)}\hfill
%   \psset{unit=0.4}\psboxfill{\BasicTileLPT{blue}{cyan}}
%   \Tiling{(15,15)}
% \end{LTXexample}
% 
%   Another classic example is to generate coordinates and numerotation for a
% grid. Of course, it is possible to do it directly in PSTricks using nested
% \cs{multido} commands. It would be clearly easy to program, but, nevertheless, 
% for users who have a little knowledge of PostScript programming, this offer
% an alternative which is useful for large cases, because on this way it will
% be notably faster and less computer ressources consuming.
% 
%   Remember here that the tiling is drawn from left to right, and top to
% bottom, and note that the PostScript variable \texttt{x2} give the total
% number of columns.
% 
% \begin{LTXexample}[pos=t]
% ^^A   % \Escape will be the \ character
%   {\catcode`\!=0\catcode`\\=11!gdef!Escape{\}}
%   \newcommand{\ProtoTile}{%
%     \Square\pscustom{%
%       \moveto(-0.9,0.75) % In PSTricks units
%       \code{ /Times-Italic findfont 8 scalefont setfont
%         (\Escape() show row 3 string cvs show (,) show 
%         column 3 string cvs show (\Escape)) show}
%       \moveto(-0.5,0.25) % In PSTricks units
%       \code{ /Times-Bold findfont 18 scalefont setfont
%         1 0 0 setrgbcolor % Red color
%         /center {dup stringwidth pop 2 div neg 0 rmoveto} def
%         row 1 sub x2 mul column add 3 string cvs center show}}}
%   \psboxfill{\ProtoTile}
%   \Tiling{(6,4)}
% \end{LTXexample}
% 
% \subsection{A complete example: the Poisson equation}
% 
%   To finish, we will show a complete real example, a drawing to explain the
% method used to solve the \textsc{Poisson} equation by a domain
% decomposition method, adapted to distributed memory computers. The
% objective is to show the communications required between processes and the
% position of the data to exchange. This code also show some useful and powerful
% technics for PSTricks programming (look specially at the way some higher level
% macros are defined, and how the same object is used to draw the four
% neighbors).
%
%\psset{unit=1cm}
%\newcommand{\Pattern}[1]{%
%   \begin{pspicture}(-0.25,-0.25)(0.25,0.25)\rput{*0}{\psdot[dotstyle=#1]}
%   \end{pspicture}}
%\newcommand{\West}{\Pattern{o}}   \newcommand{\South}{\Pattern{x}}
%\newcommand{\Central}{\Pattern{+}}\newcommand{\North}{\Pattern{square}}
%\newcommand{\East}{\Pattern{triangle}}
%\newcommand{\Cross}{%
%  \pspolygon[unit=0.5,linewidth=0.2,linecolor=red](0,0)(0,1)(1,1)(1,2)(2,2)(2,1)%
%              (3,1)(3,0)(2,0)(2,-1)(1,-1)(1,0)}
%\newcommand{\StylePosition}[1]{\LARGE\textcolor{red}{\textbf{#1}}}
%\newcommand{\SubDomain}[4]{%
%    \psboxfill{#4}\begin{psclip}{\psframe[linestyle=none]#1}%
%      \psframe[linestyle=#3](5,5)\psframe[fillstyle=boxfill]#2%
%    \end{psclip}}
%\newcommand{\SendArea}[1]{\psframe[fillstyle=solid,fillcolor=cyan]#1}
%\newcommand{\ReceiveData}[2]{%
%  \psboxfill{#2}\psframe[fillstyle=solid,fillcolor=yellow,addfillstyle=boxfill]#1}%
%\newcommand{\Neighbor}[2]{%
%    \begin{pspicture}(5,5)
%      \rput{*0}(2.5,2.5){\StylePosition{#1}}
%      \ReceiveData{(0.5,0)(4.5,0.5)}{\Central}\SendArea{(0.5,0.5)(4.5,1)}%
%      \SubDomain{(5,2)}{(0.5,0.5)(4.5,3)}{dashed}{#2}%
%      \pcarc[arcangle=45,arrows=->](0.5,-1.25)(0.5,0.25)%
%      \pcarc[arcangle=45,arrows=->,linestyle=dotted,dotsep=2pt](4.5,0.75)(4.5,-0.75)%
%    \end{pspicture}}%
%  \psset{dimen=middle,dotscale=2,fillloopadd=2}
%\begin{pspicture}(-5.7,-5.7)(5.7,5.7)
%  \rput(0,0){%
%      \begin{pspicture}(5,5)
%        \ReceiveData{(0,0.5)(0.5,4.5)}{\West} \ReceiveData{(4.5,0.5)(5,4.5)}{\East}
%        \ReceiveData{(0.5,4.5)(4.5,5)}{\North}\ReceiveData{(0.5,0)(4.5,0.5)}{\South}
%        \SendArea{(0.5,0.5)(1,4.5)}\SendArea{(4,0.5)(4.5,4.5)}
%        \SendArea{(0.5,0.5)(4.5,1)}\SendArea{(0.5,4)(4.5,4.5)}
%        \SubDomain{(5,5)}{(0.5,0.5)(4.5,4.5)}{solid}{\Central}
%        \psline(1,0.5)(1,4.5)\psline(4,0.5)(4,4.5)%
%        \rput(1.5,4){\Cross}\rput(2,2){\Cross}%
%      \end{pspicture}}%
%  \rput(0,5.5){\Neighbor{N}{\North}}\rput{-90}(5.5,0){\Neighbor{E}{\East}}%
%  \rput{90}(-5.5,0){\Neighbor{W}{\West}}\rput{180}(0,-5.5){\Neighbor{S}{\South}}%
%\end{pspicture}
%
% \begin{lstlisting}
%   \newcommand{\Pattern}[1]{%
%     \begin{pspicture}(-0.25,-0.25)(0.25,0.25)\rput{*0}{\psdot[dotstyle=#1]}
%     \end{pspicture}}
%   \newcommand{\West}{\Pattern{o}}   \newcommand{\South}{\Pattern{x}}
%   \newcommand{\Central}{\Pattern{+}}\newcommand{\North}{\Pattern{square}}
%   \newcommand{\East}{\Pattern{triangle}}
%   \newcommand{\Cross}{%
%     \pspolygon[unit=0.5,linewidth=0.2,linecolor=red](0,0)(0,1)(1,1)(1,2)(2,2)(2,1)
%               (3,1)(3,0)(2,0)(2,-1)(1,-1)(1,0)}
%   \newcommand{\StylePosition}[1]{\LARGE\textcolor{red}{\textbf{#1}}}
%   \newcommand{\SubDomain}[4]{%
%     \psboxfill{#4}
%     \begin{psclip}{\psframe[linestyle=none]#1}
%       \psframe[linestyle=#3](5,5)\psframe[fillstyle=boxfill]#2
%     \end{psclip}}
%   \newcommand{\SendArea}[1]{\psframe[fillstyle=solid,fillcolor=cyan]#1}
%   \newcommand{\ReceiveData}[2]{%
%     \psboxfill{#2}
%     \psframe[fillstyle=solid,fillcolor=yellow,addfillstyle=boxfill]#1}
%   \newcommand{\Neighbor}[2]{%
%     \begin{pspicture}(5,5)
%       \rput{*0}(2.5,2.5){\StylePosition{#1}}
%       \ReceiveData{(0.5,0)(4.5,0.5)}{\Central}\SendArea{(0.5,0.5)(4.5,1)}
%       \SubDomain{(5,2)}{(0.5,0.5)(4.5,3)}{dashed}{#2}%
% ^^A       % Receive and send arrows
%       \pcarc[arcangle=45,arrows=->](0.5,-1.25)(0.5,0.25)
%       \pcarc[arcangle=45,arrows=->,linestyle=dotted,dotsep=2pt](4.5,0.75)(4.5,-0.75)
%     \end{pspicture}}
%   \psset{dimen=middle,dotscale=2,fillloopadd=2}
%   \begin{pspicture}(-5.7,-5.7)(5.7,5.7)
% ^^A     % Central domain
%     \rput(0,0){%
%       \begin{pspicture}(5,5)
% ^^A         % Receive from West, East, North and South
%         \ReceiveData{(0,0.5)(0.5,4.5)}{\West} \ReceiveData{(4.5,0.5)(5,4.5)}{\East}
%         \ReceiveData{(0.5,4.5)(4.5,5)}{\North}\ReceiveData{(0.5,0)(4.5,0.5)}{\South}
% ^^A         % send area for West, East, North and South
%         \SendArea{(0.5,0.5)(1,4.5)} \SendArea{(4,0.5)(4.5,4.5)}
%         \SendArea{(0.5,0.5)(4.5,1)} \SendArea{(0.5,4)(4.5,4.5)}
% ^^A         % Central domain
%         \SubDomain{(5,5)}{(0.5,0.5)(4.5,4.5)}{solid}{\Central}
% ^^A         % Redraw overlapped linesY
%         \psline(1,0.5)(1,4.5)  \psline(4,0.5)(4,4.5)
% ^^A         % Two crossesY
%         \rput(1.5,4){\Cross}  \rput(2,2){\Cross}
%       \end{pspicture}}
% ^^A     % The four neighborsY
%     \rput(0,5.5){\Neighbor{N}{\North}}     \rput{-90}(5.5,0){\Neighbor{E}{\East}}
%     \rput{90}(-5.5,0){\Neighbor{W}{\West}} \rput{180}(0,-5.5){\Neighbor{S}{\South}}
%   \end{pspicture}
% \end{lstlisting}
%
%
%
% Bibliography
% \begin{thebibliography}{99}
% \bibitem{PostScript95} Adobe, Systems~Incorporated, \emph{PostScript Language
% Reference Manual}, Addison-Wesley, 2~edition, 1995.
%
% \bibitem{Bolek98} Piotr Bolek, \MP{} and patterns, \emph{\TUB}, Volume~19,
% Number~3, pages 276--283, September 1998, \CTANref{mpattern}.
%
% \bibitem{MLgraphTSI} Emmanuel Chailloux, Guy Cousineau and Asc\'ander
% Su\'arez, Programmation fonctionnelle de graphismes pour la production
% d'illustrations techniques, \emph{Technique et science informatique},
% Volume~15, Number~7, pages 977--1007, 1996 (in french).
%
% \bibitem{Deledicq97} Andr\'e Deledicq, \emph{Le monde des pavages}, ACL
% \'Editions, 1997 (in french).
%
% \bibitem{EsperetGirou98} Philippe Esperet and Denis Girou,
% Coloriage du pavage dit de Truchet, Cahiers GUTenberg, Number~31,
% pages 5--18, December~1998  (in french).
%
% \bibitem{Girou94} Denis Girou, Pr\'esentation de PSTricks, \emph{Cahiers
% GUTenberg}, Number~16, pages 21--70, February~1994 (in french).
%
% \bibitem{LGC97} Michel Goossens, Sebastian Rahtz and Frank Mittelbach,
% \emph{The \LaTeX{} Graphics Companion}, Addison-Wesley, 2005.
%
% \bibitem{GS87} Branko Gr\"unbaum and Geoffrey Shephard, \emph{Tilings and
% Patterns}, Freeman and Company, 1987.
%
% \bibitem{Hoenig97} Alan Hoenig, \emph{\TeX{} Unbound: \LaTeX{} \& \TeX{}
% Strategies, Fonts, Graphics, and More}, Oxford University Press, 1997.
%
% \bibitem{XYpic} Kristoffer~H. Rose and Ross Moore, \XYpic. Pattern and Tile
% extension, available from \CTAN, 1991-1998, \CTANref{xypic}.
%
% \bibitem{LAAN96} Kees van der Laan, Paradigms: Just a little bit of PostScript,
% \emph{MAPS}, Volume~17, pages 137--150, 1996.
%
% \bibitem{LAAN97} Kees van der Laan, Tiling in PostScript and \MF{} -- Escher's
% wink, \emph{MAPS}, Volume~19, Number~2, pages 39--67, 1997.
%
% \bibitem{vanZandt93} Timothy Van Zandt, PSTricks. PostScript macros for
% Generic \TeX, available from \CTAN, 1993, \CTANref{pstricks}.
%
% \bibitem{vanZandtGirou94} Timothy Van Zandt and Denis Girou, Inside PSTricks,
% \emph{\TUB}, Volume~15, Number~3, pages 239--246, September 1994.
%
%
% \bibitem{voss07} Herbert Vo\ss, PSTricks -- Graphics for \TeX\ and \LaTeX, DANTE/Lehmanns, 4th ed., 2007.
% \bibitem{Wang65} Hao Wang, Games, Logic and Computers, \emph{Scientific
% American}, pages 98--106, November 1965.
% \end{thebibliography}
%
%
% \StopEventually{}
%
% ^^A .................... End of the documentation part ....................
%
% \section{Driver file}
%
%   The next bit of code contains the documentation driver file for \TeX{},
% i.e., the file that will produce the documentation you are currently
% reading. It will be extracted from this file by the \texttt{docstrip}
% program.
%
%    \begin{macrocode}
%<*driver>
\documentclass{ltxdoc}
\GetFileInfo{pst-fill.dtx}
%
\usepackage[T1]{fontenc}
\usepackage{lmodern}               % For PDF
\usepackage{graphicx}              % `graphicx' LaTeX standard package
\usepackage{showexpl}
\usepackage{mflogo}                % For the MetaFont and MetaPost logos
\input{random.tex}                 % Random macros from Donald Arseneau
\usepackage{url}                   % URLs convenient typesetting
\usepackage{multido}               % General loop macro
\usepackage[dvipsnames]{pstricks}  % PSTricks with the `color' extension
\usepackage{pst-text}              % PSTricks package for character path
\usepackage{pst-grad}              % PSTricks package for gradient filling
\usepackage{pst-node}              % PSTricks package for nodes
\usepackage[tiling]{pst-fill}      % PSTricks package for filling/tiling
%
\AtBeginDocument{%
%  \OnlyDescription % comment out for implementation details
  \EnableCrossrefs
  \CodelineIndex
  \RecordChanges}
\AtEndDocument{%
  \PrintIndex
  \setcounter{IndexColumns}{1}
  \PrintChanges}
\hbadness=7000            % Over and under full box warnings
\hfuzz=3pt
\begin{document}
  \DocInput{pst-fill.dtx}
\end{document}
%</driver>
%    \end{macrocode}
%
% \section{\texttt{pst-fill} \LaTeX{} wrapper}
%
%    \begin{macrocode}
%<*latex-wrapper>
\RequirePackage{pstricks}
\ProvidesPackage{pst-fill}[2005/09/13 package wrapper for 
  pst-fill.tex (hv)]
\DeclareOption{tiling}{\def\PstTiling{true}}
\ProcessOptions\relax
\input{pst-fill.tex}
\ProvidesFile{pst-fill.tex}
  [\filedate\space v\fileversion\space `PST-fill' (tvz,dg)]
%</latex-wrapper>
%    \end{macrocode}
%
%
% \section{Pst-Fill Package{} code}
%
%    \begin{macrocode}
%<*pst-fill>
%    \end{macrocode}
%
% \subsection{Preamble}
%
%   Who we are.
%
%    \begin{macrocode}
\def\fileversion{1.01}
\def\filedate{2007/03/10}
\message{`PST-Fill' v\fileversion, \filedate\space (tvz,dg,hv)}
\csname PSTboxfillLoaded\endcsname
\let\PSTboxfillLoaded\endinput
%    \end{macrocode}
%
%   Require the main PSTricks package.
%
%    \begin{macrocode}
\ifx\PSTricksLoaded\endinput\else\input pstricks.tex\fi
%    \end{macrocode}
%
%   interface to the extended `\textsf{keyval}' package.
%
%    \begin{macrocode}
\ifx\PSTXKeyLoaded\endinput\else\input pst-xkey\fi
%
%    \end{macrocode}
%
%   Catcodes changes and defining the family name for xkeyval.
%
%    \begin{macrocode}
\edef\PstAtCode{\the\catcode`\@}\catcode`\@=11\relax

\pst@addfams{pst-fill}
%
%    \end{macrocode}
%
%
% \subsection{The size of the box}
% \begin{macro}{pst@@boxfillsize}
%    \begin{macrocode}
%
\def\pst@@boxfillsize#1(#2,#3)#4(#5,#6)#7(#8\@nil{%
  \begingroup
    \ifx\@empty#7\relax
      \pst@dima\z@
      \pst@dimb\z@
      \pssetxlength\pst@dimc{#2}%
      \pssetylength\pst@dimd{#3}%
    \else
      \pssetxlength\pst@dima{#2}%
      \pssetylength\pst@dimb{#3}%
      \pssetxlength\pst@dimc{#5}%
      \pssetylength\pst@dimd{#6}%
    \fi
    \xdef\pst@tempg{%
      \pst@dima=\number\pst@dima sp
      \pst@dimb=\number\pst@dimb sp
      \pst@dimc=\number\pst@dimc sp
      \pst@dimd=\number\pst@dimd sp }%
  \endgroup
  \let\psk@boxfillsize\pst@tempg}
%    \end{macrocode}
% \end{macro}
%

% \subsection{Definition of the parameters}
%
%    \begin{macrocode}
\define@key[psset]{pst-fill}{boxfillsize}{%
  \def\pst@tempg{#1}\def\pst@temph{auto}%
  \ifx\pst@tempg\pst@temph
    \let\psk@boxfillsize\relax
  \else
    \pst@@boxfillsize#1(\z@,\z@)\@empty(\z@,\z@)(\@nil
  \fi}
\psset{boxfillsize={(-15cm,-15cm)(15cm,15cm)}}
\define@key[psset]{pst-fill}{boxfillcolor}{\pst@getcolor{#1}\psboxfillcolor}
\psset{boxfillcolor=black}% hv
\define@key[psset]{pst-fill}{boxfillangle}{\pst@getangle{#1}\psk@boxfillangle}
\psset{boxfillangle=0}
\define@key[psset]{pst-fill}{fillsepx}{%
  \pst@getlength{#1}\psk@fillsepx}
\define@key[psset]{pst-fill}{fillsepy}{%
  \pst@getlength{#1}\psk@fillsepy}
\define@key[psset]{pst-fill}{fillsep}{%
  \pst@getlength{#1}\psk@fillsepx%
  \let\psk@fillsepy\psk@fillsepx}
\psset{fillsep=2pt}

\ifx\PstTiling\@undefined
  \define@key[psset]{pst-fill}{fillcycle}{\pst@getint{#1}\psk@fillcycle}
  \psset{fillcycle=0}
\else
  \define@key[psset]{pst-fill}{fillangle}{\pst@getangle{#1}\psk@boxfillangle}
  \define@key[psset]{pst-fill}{fillsize}{%
      \def\pst@tempg{#1}\def\pst@temph{auto}%
      \ifx\pst@tempg\pst@temph\let\psk@boxfillsize\relax
      \else\pst@@boxfillsize#1(\z@,\z@)\@empty(\z@,\z@)(\@nil\fi}
  \psset{fillsep=0,fillsize=auto}
  \define@key[psset]{pst-fill}{fillcyclex}{\pst@getint{#1}\psk@fillcyclex}
  \define@key[psset]{pst-fill}{fillcycley}{\pst@getint{#1}\psk@fillcycley}
  \define@key[psset]{pst-fill}{fillcycle}{%
    \pst@getint{#1}\psk@fillcyclex\let\psk@fillcycley\psk@fillcyclex}
  \psset{fillcycle=0}
  \define@key[psset]{pst-fill}{fillmovex}{\pst@getlength{#1}\psk@fillmovex}
  \define@key[psset]{pst-fill}{fillmovey}{\pst@getlength{#1}\psk@fillmovey}
  \define@key[psset]{pst-fill}{fillmove}{%
      \pst@getlength{#1}\psk@fillmovex\let\psk@fillmovey\psk@fillmovex}
  \psset{fillmove=0pt}
  \define@key[psset]{pst-fill}{fillloopaddx}{\pst@getint{#1}\psk@fillloopaddx}
  \define@key[psset]{pst-fill}{fillloopaddy}{\pst@getint{#1}\psk@fillloopaddy}
  \define@key[psset]{pst-fill}{fillloopadd}{%
    \pst@getint{#1}\psk@fillloopaddx\let\psk@fillloopaddy\psk@fillloopaddx}
  \psset{fillloopadd=0}
%    \end{macrocode}
%
%    \begin{macrocode}
% For debugging (to debug, set PstDebug=1)
% we now use the one from pstricks to prevent a clash with package
% pstricks                        2004-06-22
%%    \define@key[psset]{pst-fill}{PstDebug}{\pst@getint{#1}\psk@PstDebug}
    \psset{PstDebug=0}
\fi
% DG addition end
%    \end{macrocode}

% \subsection{Definition of the fill box}
% \begin{macro}{psboxfill}
%    \begin{macrocode}
\newbox\pst@fillbox
\def\psboxfill{\pst@killglue\pst@makebox\psboxfill@i}
\def\psboxfill@i{\setbox\pst@fillbox\box\pst@hbox\ignorespaces}
%    \end{macrocode}
% \end{macro}
% \subsection{The main macros}
%
% \begin{macro}{psfs@boxfill}
%    \begin{macrocode}
\def\psfs@boxfill{%
  \ifvoid\pst@fillbox
    \@pstrickserr{Fill box is empty. Use \string\psboxfill\space first.}\@ehpa
  \else
    \ifx\psk@boxfillsize\relax \pst@AutoBoxFill
    \else\pst@ManualBoxFill\fi
  \fi}
%    \end{macrocode}
% \end{macro}
%
% \begin{macro}{pst@ManualBoxFill}
%    \begin{macrocode}
\def\pst@ManualBoxFill{%
  \leavevmode
  \begingroup
    \pst@FlushCode
    \begin@psclip
    \pstVerb{clip}%
    \expandafter\pst@AddFillBox\psk@boxfillsize
    \end@psclip
  \endgroup}
%    \end{macrocode}
% \end{macro}
%
% \begin{macro}{pst@FlushCode}
%    \begin{macrocode}
\def\pst@FlushCode{%
  \pst@Verb{%
    /mtrxc CM def
    CP CP T
    \tx@STV
    \psk@origin
    \psk@swapaxes
    \pst@newpath
    \pst@code
    mtrxc setmatrix
    moveto
    0 setgray}%
  \gdef\pst@code{}}
%    \end{macrocode}
% \end{macro}
%
% \begin{macro}{pst@AddFillBox}
%    \begin{macrocode}
\def\pst@AddFillBox#1 #2 #3 #4 {%
  \begingroup
    \setbox\pst@fillbox=\vbox{%
      \hbox{\unhcopy\pst@fillbox\kern\psk@fillsepx\p@}%
      \vskip\psk@fillsepy\p@}%
    \psk@boxfillsize
    \pst@cnta=\pst@dimc
    \advance\pst@cnta-\pst@dima
    \divide\pst@cnta\wd\pst@fillbox
    \pst@cntb=\pst@dimd
    \advance\pst@cntb-\pst@dimb
    \pst@dimd=\ht\pst@fillbox
    \divide\pst@cntb\pst@dimd
    \def\pst@tempa{%
      \pst@tempg
      \copy\pst@fillbox
      \advance\pst@cntc\@ne
      \ifnum\pst@cntc<\pst@cntd\expandafter\pst@tempa\fi}%
    \let\pst@tempg\relax
    \pst@cntc-\tw@
    \pst@cntd\pst@cnta
    \setbox\pst@fillbox=\hbox to \z@{%
      \kern\pst@dima
      \kern-\wd\pst@fillbox
      \pst@tempa
      \hss}%
    \pst@cntd\pst@cntb
%% DG modification begin - Dec. 11, 1997 - Patch 2
    \ifx\PstTiling\@undefined
      \ifnum\psk@fillcycle=\z@\pst@ManualFillCycle\fi
    \else
      \ifnum\psk@fillcyclex=\z@\pst@ManualFillCycle\fi
    \fi
%% DG modification end
    \global\setbox\pst@boxg=\vbox to\z@{%
      \offinterlineskip
      \vss
      \pst@tempa
      \vskip\pst@dimb}%
  \endgroup
  \setbox\pst@fillbox\box\pst@boxg
  \pst@rotate\psk@boxfillangle\pst@fillbox
  \box\pst@fillbox}
%    \end{macrocode}
% \end{macro}
%
% \begin{macro}{pst@ManualFillCycle}
%    \begin{macrocode}
\def\pst@ManualFillCycle{%
  \ifx\PstTiling\@undefined
    \pst@cntg=\psk@fillcycle
  \else
    \pst@cntg=\psk@fillcyclex
  \fi
  \pst@dimg=\wd\pst@fillbox
  \ifnum\pst@cntg=\z@
  \else
  \divide\pst@dimg\pst@cntg
  \fi
  \ifnum\pst@cntg<\z@\pst@cntg=-\pst@cntg\fi
  \advance\pst@cntg\m@ne
  \pst@cnth=\pst@cntg
  \def\pst@tempg{%
    \ifnum\pst@cnth<\pst@cntg\advance\pst@cnth\@ne\else\pst@cnth\z@\fi
    \moveright\pst@cnth\pst@dimg}}
%    \end{macrocode}
% \end{macro}
%
%% Auto box fill:        !! Fix dictionary
%
% \subsection{The PostScript subroutines}
%
%    \begin{macrocode}
%% DG addition begin - Apr. 8, 1997 and Dec. 1997 - Patch 2
\ifx\PstTiling\@undefined
\pst@def{AutoFillCycle}<%
  /c ED
  /n 0 def
  /s {
    /x x w c div n mul add def
    /n n c abs 1 sub lt { n 1 add } { 0 } ifelse def
  } def>

\pst@def{BoxFill}<%
  gsave
    gsave \tx@STV CM grestore dtransform CM idtransform
    abs /h ED abs /w ED
    pathbbox
    h div round 2 add cvi /y2 ED
    w div round 2 add cvi /x2 ED
    h div round 2 sub cvi /y1 ED
    w div round 2 sub cvi /x1 ED
    /y2 y2 y1 sub def
    /x2 x2 x1 sub def
    CP
    y1 h mul sub neg /y1 ED
    x1 w mul sub neg /x1 ED
    clip
    y2 {
      /x x1 def
      s
      x2 {
        save CP x y1
%% patch 4   hv --------------
        \ifx\VTeXversion\undefined
        \else
%%============ mv: 09-10-01 ??? this is likely to be a right change
        neg
%%============
        \fi
%% end patch 4
T moveto Box restore
        /x x w add def
      } repeat
      /y1 y1 h add def
    } repeat
    % Next line not useful... To see that, suppress clipping (DG)
    CP x y1 T moveto Box
  currentpoint currentfont grestore setfont moveto>
\else
%% DG modification begin - Apr. 8, 1997 and Nov. / Dec. 1997 - Patch 2
\pst@def{AutoFillCycleX}<%
  /cX ED
  /nX 0 def
  /CycleX {
    /x x w cX div nX mul add def
    /nX nX cX abs 1 sub lt { nX 1 add } { 0 } ifelse def
  } def>
\pst@def{AutoFillCycleY}<%
  /cY ED
  /mY 0 def
  /nY 0 def
  /CycleY {
    /y1 y1 h cY div mY mul sub def
    nY cY abs 1 sub lt { /nY nY 1 add def /mY 1 def }
                       { /nY 0 def        /mY cY abs 1 sub neg def } ifelse
  } def>

\pst@def{BoxFill}<%
  gsave
    gsave \tx@STV CM grestore dtransform CM idtransform
    abs /h ED abs /w ED
    pathbbox
    h div round 2 add cvi /y2 ED
    w div round 2 add cvi /x2 ED
    h div round 2 sub cvi /y1 ED
    w div round 2 sub cvi /x1 ED
    /CoefLoopX 0 def
    /CoefLoopY 0 def
    /CoefMoveX 0 def
    /CoefMoveY 0 def
    \psk@boxfillangle\space 0 ne {/CoefLoopX 8 def /CoefLoopY 8 def} if
    \psk@fillcyclex\space 0 ne {/CoefLoopX CoefLoopX 1 add def} if
    \psk@fillcycley\space 0 ne {/CoefLoopY CoefLoopY 1 add def} if
    \psk@fillmovex\space 0 ne
      {/CoefLoopX CoefLoopX 2 add def
       \psk@fillmovex\space 0 gt {/CoefMoveX CoefLoopX def}
                           {/CoefMoveX CoefLoopX neg def} ifelse} if
    \psk@fillmovey\space 0 ne
      {/CoefLoopY CoefLoopY 2 add def
       \psk@fillmovey\space 0 gt {/CoefMoveY CoefLoopY def}
                           {/CoefMoveY CoefLoopY neg def} ifelse} if
    \psk@fillsepx\space 0 ne {/CoefLoopX CoefLoopX 1 add def} if
    \psk@fillsepy\space 0 ne {/CoefLoopY CoefLoopY 1 add def} if
    /CoefLoopX CoefLoopX \psk@fillloopaddx\space add def
    /CoefLoopY CoefLoopY \psk@fillloopaddy\space add def
    /x2 x2 x1 sub 4 sub CoefLoopX 2 mul add def
    /y2 y2 y1 sub 4 sub CoefLoopY 2 mul add def
%% We must fix the origin of tiling, as it must not vary according other stuff
%% in the page!
    w x1 CoefLoopX add CoefMoveX add mul
      h y1 y2 add 1 sub CoefLoopY sub CoefMoveY sub mul moveto
    CP
    y1 h mul sub neg /y1 ED
    x1 w mul sub neg /x1 ED
%%  hv 2004-06-22   to prevent clash with pst-gr3d
%%    \psk@PstDebug 0 eq {clip} if
    \Pst@Debug 0 eq {clip} if
%% end hv
    \psk@fillmovex\space \psk@fillmovey
    gsave \tx@STV CM grestore dtransform CM idtransform
    /hmove ED /wmove ED
    /row 0 def
   y2 {
       /row row 1 add def
       /column 0 def
       /x x1 def
       CycleX
       save
       x2 {
          /column column 1 add def
          CycleY
          save CP x y1
%% patch 4   hv --------------
          \ifx\VTeXversion\undefined
          \else
%%============ mv: 09-10-01 ??? this is likely to be a right change
          neg
%%============
          \fi
  T moveto Box restore
          /x x w add def
          0 hmove translate
          } repeat
       restore
       /y1 y1 h add def
       wmove 0 translate
       } repeat
  currentpoint currentfont grestore setfont moveto>
\fi
%    \end{macrocode}

%    \begin{macrocode}
\def\pst@AutoBoxFill{%
  \leavevmode
  \begingroup
    \pst@stroke
    \pst@FlushCode
    \pst@Verb{\psk@boxfillangle\space \tx@RotBegin}%
    \pstVerb{\pst@dict /Box \pslbrace end}%
    \ifx\PstTiling\@undefined
    \else
      \ifx\pst@tempa\@undefined % Undefined for instance for \pscharpath
      \else\ifx\pst@tempa\@empty\else
        \def\pst@temph{0}%
        \ifx\pst@tempa\pst@temph
        \else
          \pstVerb{/TR {pop pop currentpoint translate \pst@tempa\space translate } def}%
        \fi
      \fi\fi
    \fi
    \hbox to \z@{\vbox to\z@{\vss\copy\pst@fillbox\vskip-\dp\pst@fillbox}\hss}%
    \ifx\PstTiling\@undefined
      \pstVerb{%
        tx@Dict begin \psrbrace def
        \ifnum\psk@fillcycle=\z@
          /s {} def
        \else
          \psk@fillcycle \tx@AutoFillCycle
        \fi
        \pst@number{\wd\pst@fillbox}%
        \psk@fillsepx\space add
        \pst@number{\ht\pst@fillbox}%
        \pst@number{\dp\pst@fillbox}%
        \psk@fillsepy\space add add
        \tx@BoxFill
        end}%
      \else
      \pstVerb{%
        tx@Dict begin \psrbrace def
        \ifnum\psk@fillcyclex=\z@
          /CycleX {} def
        \else
          \psk@fillcyclex\space \tx@AutoFillCycleX
        \fi
        \ifnum\psk@fillcycley=\z@
          /CycleY {} def
        \else
          \psk@fillcycley\space \tx@AutoFillCycleY
        \fi
        \pst@number{\wd\pst@fillbox}%
        \psk@fillsepx\space add
        \pst@number{\ht\pst@fillbox}%
        \pst@number{\dp\pst@fillbox}%
        \psk@fillsepy\space add add
        \tx@BoxFill
        end}%
    \fi
    \pst@Verb{\tx@RotEnd}%
  \endgroup}
%    \end{macrocode}
% \subsection{Closing}
%
%   Catcodes restoration.
%
%    \begin{macrocode}
\catcode`\@=\PstAtCode\relax
%    \end{macrocode}
%
%    \begin{macrocode}
%</pst-fill>
%    \end{macrocode}
%
% \Finale
%
\endinput
%%
%% End of file `pst-fill.dtx'

\ProvidesFile{pst-fill.tex}
  [\filedate\space v\fileversion\space `PST-fill' (tvz,dg)]
%</latex-wrapper>
%    \end{macrocode}
%
%
% \section{Pst-Fill Package{} code}
%
%    \begin{macrocode}
%<*pst-fill>
%    \end{macrocode}
%
% \subsection{Preamble}
%
%   Who we are.
%
%    \begin{macrocode}
\def\fileversion{1.01}
\def\filedate{2007/03/10}
\message{`PST-Fill' v\fileversion, \filedate\space (tvz,dg,hv)}
\csname PSTboxfillLoaded\endcsname
\let\PSTboxfillLoaded\endinput
%    \end{macrocode}
%
%   Require the main PSTricks package.
%
%    \begin{macrocode}
\ifx\PSTricksLoaded\endinput\else\input pstricks.tex\fi
%    \end{macrocode}
%
%   interface to the extended `\textsf{keyval}' package.
%
%    \begin{macrocode}
\ifx\PSTXKeyLoaded\endinput\else\input pst-xkey\fi
%
%    \end{macrocode}
%
%   Catcodes changes and defining the family name for xkeyval.
%
%    \begin{macrocode}
\edef\PstAtCode{\the\catcode`\@}\catcode`\@=11\relax

\pst@addfams{pst-fill}
%
%    \end{macrocode}
%
%
% \subsection{The size of the box}
% \begin{macro}{pst@@boxfillsize}
%    \begin{macrocode}
%
\def\pst@@boxfillsize#1(#2,#3)#4(#5,#6)#7(#8\@nil{%
  \begingroup
    \ifx\@empty#7\relax
      \pst@dima\z@
      \pst@dimb\z@
      \pssetxlength\pst@dimc{#2}%
      \pssetylength\pst@dimd{#3}%
    \else
      \pssetxlength\pst@dima{#2}%
      \pssetylength\pst@dimb{#3}%
      \pssetxlength\pst@dimc{#5}%
      \pssetylength\pst@dimd{#6}%
    \fi
    \xdef\pst@tempg{%
      \pst@dima=\number\pst@dima sp
      \pst@dimb=\number\pst@dimb sp
      \pst@dimc=\number\pst@dimc sp
      \pst@dimd=\number\pst@dimd sp }%
  \endgroup
  \let\psk@boxfillsize\pst@tempg}
%    \end{macrocode}
% \end{macro}
%

% \subsection{Definition of the parameters}
%
%    \begin{macrocode}
\define@key[psset]{pst-fill}{boxfillsize}{%
  \def\pst@tempg{#1}\def\pst@temph{auto}%
  \ifx\pst@tempg\pst@temph
    \let\psk@boxfillsize\relax
  \else
    \pst@@boxfillsize#1(\z@,\z@)\@empty(\z@,\z@)(\@nil
  \fi}
\psset{boxfillsize={(-15cm,-15cm)(15cm,15cm)}}
\define@key[psset]{pst-fill}{boxfillcolor}{\pst@getcolor{#1}\psboxfillcolor}
\psset{boxfillcolor=black}% hv
\define@key[psset]{pst-fill}{boxfillangle}{\pst@getangle{#1}\psk@boxfillangle}
\psset{boxfillangle=0}
\define@key[psset]{pst-fill}{fillsepx}{%
  \pst@getlength{#1}\psk@fillsepx}
\define@key[psset]{pst-fill}{fillsepy}{%
  \pst@getlength{#1}\psk@fillsepy}
\define@key[psset]{pst-fill}{fillsep}{%
  \pst@getlength{#1}\psk@fillsepx%
  \let\psk@fillsepy\psk@fillsepx}
\psset{fillsep=2pt}

\ifx\PstTiling\@undefined
  \define@key[psset]{pst-fill}{fillcycle}{\pst@getint{#1}\psk@fillcycle}
  \psset{fillcycle=0}
\else
  \define@key[psset]{pst-fill}{fillangle}{\pst@getangle{#1}\psk@boxfillangle}
  \define@key[psset]{pst-fill}{fillsize}{%
      \def\pst@tempg{#1}\def\pst@temph{auto}%
      \ifx\pst@tempg\pst@temph\let\psk@boxfillsize\relax
      \else\pst@@boxfillsize#1(\z@,\z@)\@empty(\z@,\z@)(\@nil\fi}
  \psset{fillsep=0,fillsize=auto}
  \define@key[psset]{pst-fill}{fillcyclex}{\pst@getint{#1}\psk@fillcyclex}
  \define@key[psset]{pst-fill}{fillcycley}{\pst@getint{#1}\psk@fillcycley}
  \define@key[psset]{pst-fill}{fillcycle}{%
    \pst@getint{#1}\psk@fillcyclex\let\psk@fillcycley\psk@fillcyclex}
  \psset{fillcycle=0}
  \define@key[psset]{pst-fill}{fillmovex}{\pst@getlength{#1}\psk@fillmovex}
  \define@key[psset]{pst-fill}{fillmovey}{\pst@getlength{#1}\psk@fillmovey}
  \define@key[psset]{pst-fill}{fillmove}{%
      \pst@getlength{#1}\psk@fillmovex\let\psk@fillmovey\psk@fillmovex}
  \psset{fillmove=0pt}
  \define@key[psset]{pst-fill}{fillloopaddx}{\pst@getint{#1}\psk@fillloopaddx}
  \define@key[psset]{pst-fill}{fillloopaddy}{\pst@getint{#1}\psk@fillloopaddy}
  \define@key[psset]{pst-fill}{fillloopadd}{%
    \pst@getint{#1}\psk@fillloopaddx\let\psk@fillloopaddy\psk@fillloopaddx}
  \psset{fillloopadd=0}
%    \end{macrocode}
%
%    \begin{macrocode}
% For debugging (to debug, set PstDebug=1)
% we now use the one from pstricks to prevent a clash with package
% pstricks                        2004-06-22
%%    \define@key[psset]{pst-fill}{PstDebug}{\pst@getint{#1}\psk@PstDebug}
    \psset{PstDebug=0}
\fi
% DG addition end
%    \end{macrocode}

% \subsection{Definition of the fill box}
% \begin{macro}{psboxfill}
%    \begin{macrocode}
\newbox\pst@fillbox
\def\psboxfill{\pst@killglue\pst@makebox\psboxfill@i}
\def\psboxfill@i{\setbox\pst@fillbox\box\pst@hbox\ignorespaces}
%    \end{macrocode}
% \end{macro}
% \subsection{The main macros}
%
% \begin{macro}{psfs@boxfill}
%    \begin{macrocode}
\def\psfs@boxfill{%
  \ifvoid\pst@fillbox
    \@pstrickserr{Fill box is empty. Use \string\psboxfill\space first.}\@ehpa
  \else
    \ifx\psk@boxfillsize\relax \pst@AutoBoxFill
    \else\pst@ManualBoxFill\fi
  \fi}
%    \end{macrocode}
% \end{macro}
%
% \begin{macro}{pst@ManualBoxFill}
%    \begin{macrocode}
\def\pst@ManualBoxFill{%
  \leavevmode
  \begingroup
    \pst@FlushCode
    \begin@psclip
    \pstVerb{clip}%
    \expandafter\pst@AddFillBox\psk@boxfillsize
    \end@psclip
  \endgroup}
%    \end{macrocode}
% \end{macro}
%
% \begin{macro}{pst@FlushCode}
%    \begin{macrocode}
\def\pst@FlushCode{%
  \pst@Verb{%
    /mtrxc CM def
    CP CP T
    \tx@STV
    \psk@origin
    \psk@swapaxes
    \pst@newpath
    \pst@code
    mtrxc setmatrix
    moveto
    0 setgray}%
  \gdef\pst@code{}}
%    \end{macrocode}
% \end{macro}
%
% \begin{macro}{pst@AddFillBox}
%    \begin{macrocode}
\def\pst@AddFillBox#1 #2 #3 #4 {%
  \begingroup
    \setbox\pst@fillbox=\vbox{%
      \hbox{\unhcopy\pst@fillbox\kern\psk@fillsepx\p@}%
      \vskip\psk@fillsepy\p@}%
    \psk@boxfillsize
    \pst@cnta=\pst@dimc
    \advance\pst@cnta-\pst@dima
    \divide\pst@cnta\wd\pst@fillbox
    \pst@cntb=\pst@dimd
    \advance\pst@cntb-\pst@dimb
    \pst@dimd=\ht\pst@fillbox
    \divide\pst@cntb\pst@dimd
    \def\pst@tempa{%
      \pst@tempg
      \copy\pst@fillbox
      \advance\pst@cntc\@ne
      \ifnum\pst@cntc<\pst@cntd\expandafter\pst@tempa\fi}%
    \let\pst@tempg\relax
    \pst@cntc-\tw@
    \pst@cntd\pst@cnta
    \setbox\pst@fillbox=\hbox to \z@{%
      \kern\pst@dima
      \kern-\wd\pst@fillbox
      \pst@tempa
      \hss}%
    \pst@cntd\pst@cntb
%% DG modification begin - Dec. 11, 1997 - Patch 2
    \ifx\PstTiling\@undefined
      \ifnum\psk@fillcycle=\z@\pst@ManualFillCycle\fi
    \else
      \ifnum\psk@fillcyclex=\z@\pst@ManualFillCycle\fi
    \fi
%% DG modification end
    \global\setbox\pst@boxg=\vbox to\z@{%
      \offinterlineskip
      \vss
      \pst@tempa
      \vskip\pst@dimb}%
  \endgroup
  \setbox\pst@fillbox\box\pst@boxg
  \pst@rotate\psk@boxfillangle\pst@fillbox
  \box\pst@fillbox}
%    \end{macrocode}
% \end{macro}
%
% \begin{macro}{pst@ManualFillCycle}
%    \begin{macrocode}
\def\pst@ManualFillCycle{%
  \ifx\PstTiling\@undefined
    \pst@cntg=\psk@fillcycle
  \else
    \pst@cntg=\psk@fillcyclex
  \fi
  \pst@dimg=\wd\pst@fillbox
  \ifnum\pst@cntg=\z@
  \else
  \divide\pst@dimg\pst@cntg
  \fi
  \ifnum\pst@cntg<\z@\pst@cntg=-\pst@cntg\fi
  \advance\pst@cntg\m@ne
  \pst@cnth=\pst@cntg
  \def\pst@tempg{%
    \ifnum\pst@cnth<\pst@cntg\advance\pst@cnth\@ne\else\pst@cnth\z@\fi
    \moveright\pst@cnth\pst@dimg}}
%    \end{macrocode}
% \end{macro}
%
%% Auto box fill:        !! Fix dictionary
%
% \subsection{The PostScript subroutines}
%
%    \begin{macrocode}
%% DG addition begin - Apr. 8, 1997 and Dec. 1997 - Patch 2
\ifx\PstTiling\@undefined
\pst@def{AutoFillCycle}<%
  /c ED
  /n 0 def
  /s {
    /x x w c div n mul add def
    /n n c abs 1 sub lt { n 1 add } { 0 } ifelse def
  } def>

\pst@def{BoxFill}<%
  gsave
    gsave \tx@STV CM grestore dtransform CM idtransform
    abs /h ED abs /w ED
    pathbbox
    h div round 2 add cvi /y2 ED
    w div round 2 add cvi /x2 ED
    h div round 2 sub cvi /y1 ED
    w div round 2 sub cvi /x1 ED
    /y2 y2 y1 sub def
    /x2 x2 x1 sub def
    CP
    y1 h mul sub neg /y1 ED
    x1 w mul sub neg /x1 ED
    clip
    y2 {
      /x x1 def
      s
      x2 {
        save CP x y1
%% patch 4   hv --------------
        \ifx\VTeXversion\undefined
        \else
%%============ mv: 09-10-01 ??? this is likely to be a right change
        neg
%%============
        \fi
%% end patch 4
T moveto Box restore
        /x x w add def
      } repeat
      /y1 y1 h add def
    } repeat
    % Next line not useful... To see that, suppress clipping (DG)
    CP x y1 T moveto Box
  currentpoint currentfont grestore setfont moveto>
\else
%% DG modification begin - Apr. 8, 1997 and Nov. / Dec. 1997 - Patch 2
\pst@def{AutoFillCycleX}<%
  /cX ED
  /nX 0 def
  /CycleX {
    /x x w cX div nX mul add def
    /nX nX cX abs 1 sub lt { nX 1 add } { 0 } ifelse def
  } def>
\pst@def{AutoFillCycleY}<%
  /cY ED
  /mY 0 def
  /nY 0 def
  /CycleY {
    /y1 y1 h cY div mY mul sub def
    nY cY abs 1 sub lt { /nY nY 1 add def /mY 1 def }
                       { /nY 0 def        /mY cY abs 1 sub neg def } ifelse
  } def>

\pst@def{BoxFill}<%
  gsave
    gsave \tx@STV CM grestore dtransform CM idtransform
    abs /h ED abs /w ED
    pathbbox
    h div round 2 add cvi /y2 ED
    w div round 2 add cvi /x2 ED
    h div round 2 sub cvi /y1 ED
    w div round 2 sub cvi /x1 ED
    /CoefLoopX 0 def
    /CoefLoopY 0 def
    /CoefMoveX 0 def
    /CoefMoveY 0 def
    \psk@boxfillangle\space 0 ne {/CoefLoopX 8 def /CoefLoopY 8 def} if
    \psk@fillcyclex\space 0 ne {/CoefLoopX CoefLoopX 1 add def} if
    \psk@fillcycley\space 0 ne {/CoefLoopY CoefLoopY 1 add def} if
    \psk@fillmovex\space 0 ne
      {/CoefLoopX CoefLoopX 2 add def
       \psk@fillmovex\space 0 gt {/CoefMoveX CoefLoopX def}
                           {/CoefMoveX CoefLoopX neg def} ifelse} if
    \psk@fillmovey\space 0 ne
      {/CoefLoopY CoefLoopY 2 add def
       \psk@fillmovey\space 0 gt {/CoefMoveY CoefLoopY def}
                           {/CoefMoveY CoefLoopY neg def} ifelse} if
    \psk@fillsepx\space 0 ne {/CoefLoopX CoefLoopX 1 add def} if
    \psk@fillsepy\space 0 ne {/CoefLoopY CoefLoopY 1 add def} if
    /CoefLoopX CoefLoopX \psk@fillloopaddx\space add def
    /CoefLoopY CoefLoopY \psk@fillloopaddy\space add def
    /x2 x2 x1 sub 4 sub CoefLoopX 2 mul add def
    /y2 y2 y1 sub 4 sub CoefLoopY 2 mul add def
%% We must fix the origin of tiling, as it must not vary according other stuff
%% in the page!
    w x1 CoefLoopX add CoefMoveX add mul
      h y1 y2 add 1 sub CoefLoopY sub CoefMoveY sub mul moveto
    CP
    y1 h mul sub neg /y1 ED
    x1 w mul sub neg /x1 ED
%%  hv 2004-06-22   to prevent clash with pst-gr3d
%%    \psk@PstDebug 0 eq {clip} if
    \Pst@Debug 0 eq {clip} if
%% end hv
    \psk@fillmovex\space \psk@fillmovey
    gsave \tx@STV CM grestore dtransform CM idtransform
    /hmove ED /wmove ED
    /row 0 def
   y2 {
       /row row 1 add def
       /column 0 def
       /x x1 def
       CycleX
       save
       x2 {
          /column column 1 add def
          CycleY
          save CP x y1
%% patch 4   hv --------------
          \ifx\VTeXversion\undefined
          \else
%%============ mv: 09-10-01 ??? this is likely to be a right change
          neg
%%============
          \fi
  T moveto Box restore
          /x x w add def
          0 hmove translate
          } repeat
       restore
       /y1 y1 h add def
       wmove 0 translate
       } repeat
  currentpoint currentfont grestore setfont moveto>
\fi
%    \end{macrocode}

%    \begin{macrocode}
\def\pst@AutoBoxFill{%
  \leavevmode
  \begingroup
    \pst@stroke
    \pst@FlushCode
    \pst@Verb{\psk@boxfillangle\space \tx@RotBegin}%
    \pstVerb{\pst@dict /Box \pslbrace end}%
    \ifx\PstTiling\@undefined
    \else
      \ifx\pst@tempa\@undefined % Undefined for instance for \pscharpath
      \else\ifx\pst@tempa\@empty\else
        \def\pst@temph{0}%
        \ifx\pst@tempa\pst@temph
        \else
          \pstVerb{/TR {pop pop currentpoint translate \pst@tempa\space translate } def}%
        \fi
      \fi\fi
    \fi
    \hbox to \z@{\vbox to\z@{\vss\copy\pst@fillbox\vskip-\dp\pst@fillbox}\hss}%
    \ifx\PstTiling\@undefined
      \pstVerb{%
        tx@Dict begin \psrbrace def
        \ifnum\psk@fillcycle=\z@
          /s {} def
        \else
          \psk@fillcycle \tx@AutoFillCycle
        \fi
        \pst@number{\wd\pst@fillbox}%
        \psk@fillsepx\space add
        \pst@number{\ht\pst@fillbox}%
        \pst@number{\dp\pst@fillbox}%
        \psk@fillsepy\space add add
        \tx@BoxFill
        end}%
      \else
      \pstVerb{%
        tx@Dict begin \psrbrace def
        \ifnum\psk@fillcyclex=\z@
          /CycleX {} def
        \else
          \psk@fillcyclex\space \tx@AutoFillCycleX
        \fi
        \ifnum\psk@fillcycley=\z@
          /CycleY {} def
        \else
          \psk@fillcycley\space \tx@AutoFillCycleY
        \fi
        \pst@number{\wd\pst@fillbox}%
        \psk@fillsepx\space add
        \pst@number{\ht\pst@fillbox}%
        \pst@number{\dp\pst@fillbox}%
        \psk@fillsepy\space add add
        \tx@BoxFill
        end}%
    \fi
    \pst@Verb{\tx@RotEnd}%
  \endgroup}
%    \end{macrocode}
% \subsection{Closing}
%
%   Catcodes restoration.
%
%    \begin{macrocode}
\catcode`\@=\PstAtCode\relax
%    \end{macrocode}
%
%    \begin{macrocode}
%</pst-fill>
%    \end{macrocode}
%
% \Finale
%
\endinput
%%
%% End of file `pst-fill.dtx'

\ProvidesFile{pst-fill.tex}
  [\filedate\space v\fileversion\space `PST-fill' (tvz,dg)]
%</latex-wrapper>
%    \end{macrocode}
%
%
% \section{Pst-Fill Package{} code}
%
%    \begin{macrocode}
%<*pst-fill>
%    \end{macrocode}
%
% \subsection{Preamble}
%
%   Who we are.
%
%    \begin{macrocode}
\def\fileversion{1.01}
\def\filedate{2007/03/10}
\message{`PST-Fill' v\fileversion, \filedate\space (tvz,dg,hv)}
\csname PSTboxfillLoaded\endcsname
\let\PSTboxfillLoaded\endinput
%    \end{macrocode}
%
%   Require the main PSTricks package.
%
%    \begin{macrocode}
\ifx\PSTricksLoaded\endinput\else\input pstricks.tex\fi
%    \end{macrocode}
%
%   interface to the extended `\textsf{keyval}' package.
%
%    \begin{macrocode}
\ifx\PSTXKeyLoaded\endinput\else\input pst-xkey\fi
%
%    \end{macrocode}
%
%   Catcodes changes and defining the family name for xkeyval.
%
%    \begin{macrocode}
\edef\PstAtCode{\the\catcode`\@}\catcode`\@=11\relax

\pst@addfams{pst-fill}
%
%    \end{macrocode}
%
%
% \subsection{The size of the box}
% \begin{macro}{pst@@boxfillsize}
%    \begin{macrocode}
%
\def\pst@@boxfillsize#1(#2,#3)#4(#5,#6)#7(#8\@nil{%
  \begingroup
    \ifx\@empty#7\relax
      \pst@dima\z@
      \pst@dimb\z@
      \pssetxlength\pst@dimc{#2}%
      \pssetylength\pst@dimd{#3}%
    \else
      \pssetxlength\pst@dima{#2}%
      \pssetylength\pst@dimb{#3}%
      \pssetxlength\pst@dimc{#5}%
      \pssetylength\pst@dimd{#6}%
    \fi
    \xdef\pst@tempg{%
      \pst@dima=\number\pst@dima sp
      \pst@dimb=\number\pst@dimb sp
      \pst@dimc=\number\pst@dimc sp
      \pst@dimd=\number\pst@dimd sp }%
  \endgroup
  \let\psk@boxfillsize\pst@tempg}
%    \end{macrocode}
% \end{macro}
%

% \subsection{Definition of the parameters}
%
%    \begin{macrocode}
\define@key[psset]{pst-fill}{boxfillsize}{%
  \def\pst@tempg{#1}\def\pst@temph{auto}%
  \ifx\pst@tempg\pst@temph
    \let\psk@boxfillsize\relax
  \else
    \pst@@boxfillsize#1(\z@,\z@)\@empty(\z@,\z@)(\@nil
  \fi}
\psset{boxfillsize={(-15cm,-15cm)(15cm,15cm)}}
\define@key[psset]{pst-fill}{boxfillcolor}{\pst@getcolor{#1}\psboxfillcolor}
\psset{boxfillcolor=black}% hv
\define@key[psset]{pst-fill}{boxfillangle}{\pst@getangle{#1}\psk@boxfillangle}
\psset{boxfillangle=0}
\define@key[psset]{pst-fill}{fillsepx}{%
  \pst@getlength{#1}\psk@fillsepx}
\define@key[psset]{pst-fill}{fillsepy}{%
  \pst@getlength{#1}\psk@fillsepy}
\define@key[psset]{pst-fill}{fillsep}{%
  \pst@getlength{#1}\psk@fillsepx%
  \let\psk@fillsepy\psk@fillsepx}
\psset{fillsep=2pt}

\ifx\PstTiling\@undefined
  \define@key[psset]{pst-fill}{fillcycle}{\pst@getint{#1}\psk@fillcycle}
  \psset{fillcycle=0}
\else
  \define@key[psset]{pst-fill}{fillangle}{\pst@getangle{#1}\psk@boxfillangle}
  \define@key[psset]{pst-fill}{fillsize}{%
      \def\pst@tempg{#1}\def\pst@temph{auto}%
      \ifx\pst@tempg\pst@temph\let\psk@boxfillsize\relax
      \else\pst@@boxfillsize#1(\z@,\z@)\@empty(\z@,\z@)(\@nil\fi}
  \psset{fillsep=0,fillsize=auto}
  \define@key[psset]{pst-fill}{fillcyclex}{\pst@getint{#1}\psk@fillcyclex}
  \define@key[psset]{pst-fill}{fillcycley}{\pst@getint{#1}\psk@fillcycley}
  \define@key[psset]{pst-fill}{fillcycle}{%
    \pst@getint{#1}\psk@fillcyclex\let\psk@fillcycley\psk@fillcyclex}
  \psset{fillcycle=0}
  \define@key[psset]{pst-fill}{fillmovex}{\pst@getlength{#1}\psk@fillmovex}
  \define@key[psset]{pst-fill}{fillmovey}{\pst@getlength{#1}\psk@fillmovey}
  \define@key[psset]{pst-fill}{fillmove}{%
      \pst@getlength{#1}\psk@fillmovex\let\psk@fillmovey\psk@fillmovex}
  \psset{fillmove=0pt}
  \define@key[psset]{pst-fill}{fillloopaddx}{\pst@getint{#1}\psk@fillloopaddx}
  \define@key[psset]{pst-fill}{fillloopaddy}{\pst@getint{#1}\psk@fillloopaddy}
  \define@key[psset]{pst-fill}{fillloopadd}{%
    \pst@getint{#1}\psk@fillloopaddx\let\psk@fillloopaddy\psk@fillloopaddx}
  \psset{fillloopadd=0}
%    \end{macrocode}
%
%    \begin{macrocode}
% For debugging (to debug, set PstDebug=1)
% we now use the one from pstricks to prevent a clash with package
% pstricks                        2004-06-22
%%    \define@key[psset]{pst-fill}{PstDebug}{\pst@getint{#1}\psk@PstDebug}
    \psset{PstDebug=0}
\fi
% DG addition end
%    \end{macrocode}

% \subsection{Definition of the fill box}
% \begin{macro}{psboxfill}
%    \begin{macrocode}
\newbox\pst@fillbox
\def\psboxfill{\pst@killglue\pst@makebox\psboxfill@i}
\def\psboxfill@i{\setbox\pst@fillbox\box\pst@hbox\ignorespaces}
%    \end{macrocode}
% \end{macro}
% \subsection{The main macros}
%
% \begin{macro}{psfs@boxfill}
%    \begin{macrocode}
\def\psfs@boxfill{%
  \ifvoid\pst@fillbox
    \@pstrickserr{Fill box is empty. Use \string\psboxfill\space first.}\@ehpa
  \else
    \ifx\psk@boxfillsize\relax \pst@AutoBoxFill
    \else\pst@ManualBoxFill\fi
  \fi}
%    \end{macrocode}
% \end{macro}
%
% \begin{macro}{pst@ManualBoxFill}
%    \begin{macrocode}
\def\pst@ManualBoxFill{%
  \leavevmode
  \begingroup
    \pst@FlushCode
    \begin@psclip
    \pstVerb{clip}%
    \expandafter\pst@AddFillBox\psk@boxfillsize
    \end@psclip
  \endgroup}
%    \end{macrocode}
% \end{macro}
%
% \begin{macro}{pst@FlushCode}
%    \begin{macrocode}
\def\pst@FlushCode{%
  \pst@Verb{%
    /mtrxc CM def
    CP CP T
    \tx@STV
    \psk@origin
    \psk@swapaxes
    \pst@newpath
    \pst@code
    mtrxc setmatrix
    moveto
    0 setgray}%
  \gdef\pst@code{}}
%    \end{macrocode}
% \end{macro}
%
% \begin{macro}{pst@AddFillBox}
%    \begin{macrocode}
\def\pst@AddFillBox#1 #2 #3 #4 {%
  \begingroup
    \setbox\pst@fillbox=\vbox{%
      \hbox{\unhcopy\pst@fillbox\kern\psk@fillsepx\p@}%
      \vskip\psk@fillsepy\p@}%
    \psk@boxfillsize
    \pst@cnta=\pst@dimc
    \advance\pst@cnta-\pst@dima
    \divide\pst@cnta\wd\pst@fillbox
    \pst@cntb=\pst@dimd
    \advance\pst@cntb-\pst@dimb
    \pst@dimd=\ht\pst@fillbox
    \divide\pst@cntb\pst@dimd
    \def\pst@tempa{%
      \pst@tempg
      \copy\pst@fillbox
      \advance\pst@cntc\@ne
      \ifnum\pst@cntc<\pst@cntd\expandafter\pst@tempa\fi}%
    \let\pst@tempg\relax
    \pst@cntc-\tw@
    \pst@cntd\pst@cnta
    \setbox\pst@fillbox=\hbox to \z@{%
      \kern\pst@dima
      \kern-\wd\pst@fillbox
      \pst@tempa
      \hss}%
    \pst@cntd\pst@cntb
%% DG modification begin - Dec. 11, 1997 - Patch 2
    \ifx\PstTiling\@undefined
      \ifnum\psk@fillcycle=\z@\pst@ManualFillCycle\fi
    \else
      \ifnum\psk@fillcyclex=\z@\pst@ManualFillCycle\fi
    \fi
%% DG modification end
    \global\setbox\pst@boxg=\vbox to\z@{%
      \offinterlineskip
      \vss
      \pst@tempa
      \vskip\pst@dimb}%
  \endgroup
  \setbox\pst@fillbox\box\pst@boxg
  \pst@rotate\psk@boxfillangle\pst@fillbox
  \box\pst@fillbox}
%    \end{macrocode}
% \end{macro}
%
% \begin{macro}{pst@ManualFillCycle}
%    \begin{macrocode}
\def\pst@ManualFillCycle{%
  \ifx\PstTiling\@undefined
    \pst@cntg=\psk@fillcycle
  \else
    \pst@cntg=\psk@fillcyclex
  \fi
  \pst@dimg=\wd\pst@fillbox
  \ifnum\pst@cntg=\z@
  \else
  \divide\pst@dimg\pst@cntg
  \fi
  \ifnum\pst@cntg<\z@\pst@cntg=-\pst@cntg\fi
  \advance\pst@cntg\m@ne
  \pst@cnth=\pst@cntg
  \def\pst@tempg{%
    \ifnum\pst@cnth<\pst@cntg\advance\pst@cnth\@ne\else\pst@cnth\z@\fi
    \moveright\pst@cnth\pst@dimg}}
%    \end{macrocode}
% \end{macro}
%
%% Auto box fill:        !! Fix dictionary
%
% \subsection{The PostScript subroutines}
%
%    \begin{macrocode}
%% DG addition begin - Apr. 8, 1997 and Dec. 1997 - Patch 2
\ifx\PstTiling\@undefined
\pst@def{AutoFillCycle}<%
  /c ED
  /n 0 def
  /s {
    /x x w c div n mul add def
    /n n c abs 1 sub lt { n 1 add } { 0 } ifelse def
  } def>

\pst@def{BoxFill}<%
  gsave
    gsave \tx@STV CM grestore dtransform CM idtransform
    abs /h ED abs /w ED
    pathbbox
    h div round 2 add cvi /y2 ED
    w div round 2 add cvi /x2 ED
    h div round 2 sub cvi /y1 ED
    w div round 2 sub cvi /x1 ED
    /y2 y2 y1 sub def
    /x2 x2 x1 sub def
    CP
    y1 h mul sub neg /y1 ED
    x1 w mul sub neg /x1 ED
    clip
    y2 {
      /x x1 def
      s
      x2 {
        save CP x y1
%% patch 4   hv --------------
        \ifx\VTeXversion\undefined
        \else
%%============ mv: 09-10-01 ??? this is likely to be a right change
        neg
%%============
        \fi
%% end patch 4
T moveto Box restore
        /x x w add def
      } repeat
      /y1 y1 h add def
    } repeat
    % Next line not useful... To see that, suppress clipping (DG)
    CP x y1 T moveto Box
  currentpoint currentfont grestore setfont moveto>
\else
%% DG modification begin - Apr. 8, 1997 and Nov. / Dec. 1997 - Patch 2
\pst@def{AutoFillCycleX}<%
  /cX ED
  /nX 0 def
  /CycleX {
    /x x w cX div nX mul add def
    /nX nX cX abs 1 sub lt { nX 1 add } { 0 } ifelse def
  } def>
\pst@def{AutoFillCycleY}<%
  /cY ED
  /mY 0 def
  /nY 0 def
  /CycleY {
    /y1 y1 h cY div mY mul sub def
    nY cY abs 1 sub lt { /nY nY 1 add def /mY 1 def }
                       { /nY 0 def        /mY cY abs 1 sub neg def } ifelse
  } def>

\pst@def{BoxFill}<%
  gsave
    gsave \tx@STV CM grestore dtransform CM idtransform
    abs /h ED abs /w ED
    pathbbox
    h div round 2 add cvi /y2 ED
    w div round 2 add cvi /x2 ED
    h div round 2 sub cvi /y1 ED
    w div round 2 sub cvi /x1 ED
    /CoefLoopX 0 def
    /CoefLoopY 0 def
    /CoefMoveX 0 def
    /CoefMoveY 0 def
    \psk@boxfillangle\space 0 ne {/CoefLoopX 8 def /CoefLoopY 8 def} if
    \psk@fillcyclex\space 0 ne {/CoefLoopX CoefLoopX 1 add def} if
    \psk@fillcycley\space 0 ne {/CoefLoopY CoefLoopY 1 add def} if
    \psk@fillmovex\space 0 ne
      {/CoefLoopX CoefLoopX 2 add def
       \psk@fillmovex\space 0 gt {/CoefMoveX CoefLoopX def}
                           {/CoefMoveX CoefLoopX neg def} ifelse} if
    \psk@fillmovey\space 0 ne
      {/CoefLoopY CoefLoopY 2 add def
       \psk@fillmovey\space 0 gt {/CoefMoveY CoefLoopY def}
                           {/CoefMoveY CoefLoopY neg def} ifelse} if
    \psk@fillsepx\space 0 ne {/CoefLoopX CoefLoopX 1 add def} if
    \psk@fillsepy\space 0 ne {/CoefLoopY CoefLoopY 1 add def} if
    /CoefLoopX CoefLoopX \psk@fillloopaddx\space add def
    /CoefLoopY CoefLoopY \psk@fillloopaddy\space add def
    /x2 x2 x1 sub 4 sub CoefLoopX 2 mul add def
    /y2 y2 y1 sub 4 sub CoefLoopY 2 mul add def
%% We must fix the origin of tiling, as it must not vary according other stuff
%% in the page!
    w x1 CoefLoopX add CoefMoveX add mul
      h y1 y2 add 1 sub CoefLoopY sub CoefMoveY sub mul moveto
    CP
    y1 h mul sub neg /y1 ED
    x1 w mul sub neg /x1 ED
%%  hv 2004-06-22   to prevent clash with pst-gr3d
%%    \psk@PstDebug 0 eq {clip} if
    \Pst@Debug 0 eq {clip} if
%% end hv
    \psk@fillmovex\space \psk@fillmovey
    gsave \tx@STV CM grestore dtransform CM idtransform
    /hmove ED /wmove ED
    /row 0 def
   y2 {
       /row row 1 add def
       /column 0 def
       /x x1 def
       CycleX
       save
       x2 {
          /column column 1 add def
          CycleY
          save CP x y1
%% patch 4   hv --------------
          \ifx\VTeXversion\undefined
          \else
%%============ mv: 09-10-01 ??? this is likely to be a right change
          neg
%%============
          \fi
  T moveto Box restore
          /x x w add def
          0 hmove translate
          } repeat
       restore
       /y1 y1 h add def
       wmove 0 translate
       } repeat
  currentpoint currentfont grestore setfont moveto>
\fi
%    \end{macrocode}

%    \begin{macrocode}
\def\pst@AutoBoxFill{%
  \leavevmode
  \begingroup
    \pst@stroke
    \pst@FlushCode
    \pst@Verb{\psk@boxfillangle\space \tx@RotBegin}%
    \pstVerb{\pst@dict /Box \pslbrace end}%
    \ifx\PstTiling\@undefined
    \else
      \ifx\pst@tempa\@undefined % Undefined for instance for \pscharpath
      \else\ifx\pst@tempa\@empty\else
        \def\pst@temph{0}%
        \ifx\pst@tempa\pst@temph
        \else
          \pstVerb{/TR {pop pop currentpoint translate \pst@tempa\space translate } def}%
        \fi
      \fi\fi
    \fi
    \hbox to \z@{\vbox to\z@{\vss\copy\pst@fillbox\vskip-\dp\pst@fillbox}\hss}%
    \ifx\PstTiling\@undefined
      \pstVerb{%
        tx@Dict begin \psrbrace def
        \ifnum\psk@fillcycle=\z@
          /s {} def
        \else
          \psk@fillcycle \tx@AutoFillCycle
        \fi
        \pst@number{\wd\pst@fillbox}%
        \psk@fillsepx\space add
        \pst@number{\ht\pst@fillbox}%
        \pst@number{\dp\pst@fillbox}%
        \psk@fillsepy\space add add
        \tx@BoxFill
        end}%
      \else
      \pstVerb{%
        tx@Dict begin \psrbrace def
        \ifnum\psk@fillcyclex=\z@
          /CycleX {} def
        \else
          \psk@fillcyclex\space \tx@AutoFillCycleX
        \fi
        \ifnum\psk@fillcycley=\z@
          /CycleY {} def
        \else
          \psk@fillcycley\space \tx@AutoFillCycleY
        \fi
        \pst@number{\wd\pst@fillbox}%
        \psk@fillsepx\space add
        \pst@number{\ht\pst@fillbox}%
        \pst@number{\dp\pst@fillbox}%
        \psk@fillsepy\space add add
        \tx@BoxFill
        end}%
    \fi
    \pst@Verb{\tx@RotEnd}%
  \endgroup}
%    \end{macrocode}
% \subsection{Closing}
%
%   Catcodes restoration.
%
%    \begin{macrocode}
\catcode`\@=\PstAtCode\relax
%    \end{macrocode}
%
%    \begin{macrocode}
%</pst-fill>
%    \end{macrocode}
%
% \Finale
%
\endinput
%%
%% End of file `pst-fill.dtx'
+\newline
%add the following definition:\newline
%\verb+\def\PstTiling{true}+
%
%  To obtain the original behaviour, just don't use the \emph{tiling} optional
%keyword at loading.
%
%  Take care than in \emph{tiling} mode, I introduce also some other changes.
%First I define aliases on some parameter names for consistancy (all specific
%parameters will begin by the \texttt{fill} prefix in this case) and I change
%some default values, which were not well adapted for tilings (\texttt{fillsep}
%is set to 0 and as explained \texttt{fillsize} set to \texttt{auto}). I rename 
%\texttt{fillcycle} to \texttt{fillcyclex}. I also restore normal way so that
%the frame of the area is drawn and all line (\texttt{linestyle},
%\texttt{linecolor}, \texttt{doubleline}, etc.) parameters are now active (but
%there are not in non \emph{tiling} mode). And I also introduce new parameters
%to control the tilings (see below).
%
%  \textbf{In all the following examples, we will consider only the
% \emph{tiling} mode.}
%
%  To do a tiling, we have just to define the pattern with the
% \verb+\psboxfill+ macro and to use the new \texttt{fillstyle}
% \verb+boxfill+.
%
%  Note that tilings are drawn from left to right and top to bottom, which can
%have an importance in some circonstances.
%
%  PostScript programmers can be also interested to know that, even in the
%\emph{automatic} mode, the iterations of the pattern are managed directly by
%the PostScript code of the package which used only PostScript Level 1
%operators. The special ones introduced in Level 2 for drawing of patterns
%\cite[section 4.9]{PostScript95} are not used.
%
%  And first, for conveniance, we define a simple \cs{Tiling} macro, which
%will simplify our examples:
%
%\begin{verbatim}
%  \newcommand{\Tiling}[2][]{%
%    \edef\Temp{#1}%
%    \begin{pspicture}#2
%      \ifx\Temp\empty
%        \psframe[fillstyle=boxfill]#2
%      \else
%        \psframe[fillstyle=boxfill,#1]#2
%      \fi
%    \end{pspicture}}
%\end{verbatim}
%
%
%\newcommand{\Tiling}[2][]{%
%  \edef\Temp{#1}%
%  \begin{pspicture}#2
%    \ifx\Temp\empty
%      \psframe[fillstyle=boxfill]#2
%    \else
%      \psframe[fillstyle=boxfill,#1]#2
%    \fi
% \end{pspicture}}
%
%\subsection{Parameters}
%
%  There are \textbf{14} specific parameters available to change the way the
% filling/tiling is defined, and one debugging option.
%
% \begin{Description}{2cm}
%  \item [fillangle (real)\hfill :] the value of the rotation
%  applied to the patterns (\emph{Default:~0}).
% \end{Description}
%
%
%   In this case, we must force the tiling area to be notably larger than the
% area to cover, to be sure that the defined area will be covered after rotation.
% \lstset{gobble=2}
% \begin{LTXexample}
% \newcommand{\Square}{%
%   \begin{pspicture}(1,1)
%     \psframe[dimen=middle](1,1)
%   \end{pspicture}}
% \psset{unit=0.5}
% \psboxfill{\Square}
% \Tiling[fillangle=45]{(3,3)}\quad
% \Tiling[fillangle=-60]{(3,3)}
% \end{LTXexample}
% 
% \newcommand{\Square}{\begin{pspicture}(1,1)\psframe[dimen=middle](1,1)\end{pspicture}}
% 
% \begin{Description}{2cm}
%   \setcounter{footnote}{1}
%   \item[\texttt{fillsepx} (real$\|$dim) :] value of the horizontal
%   separation between consecutive patterns (\emph{Default:~0 for
%   tilings\footnotemark, 2pt otherwise}).  \footnotetext{This option was added
%   by me, is not part of the original package and is available only if the
%   \texttt{tiling} keyword is used when loading the package.}
%   \setcounter{footnote}{1}
%   \item [\texttt{fillsepy} (real$\|$dim)\hfill :] value of the vertical
%   separation between consecutive patterns (\emph{Default:~0 for
%   ti\-lings\footnotemark, 2pt otherwise}).
%   \setcounter{footnote}{1}
%   \item [\texttt{fillsep} (real$\|$dim)\hfill :] value of horizontal and
%   vertical separations between consecutive patterns (\emph{Default:~0 for
%   tilings\footnotemark, 2pt otherwise}).
% \end{Description}
% 
%   These values can be negative, which allow the tiles to overlap.
% 
% \begin{LTXexample}
% \psset{unit=0.5}
% \psboxfill{\Square}
% \Tiling[fillsepx=2mm]{(3,3)} 
% \Tiling[fillsepy=1mm]{(3,3)}\\
% \Tiling[fillsep=0.5]{(3,3)} 
% \Tiling[fillsep=-0.5]{(3,3)}
% \end{LTXexample}
% 
% \begin{Description}{2cm}
%   \item [\texttt{fillcyclex}\footnotemark\ (integer)\hfill :] Shift
%   coefficient applied to each row (\emph{Default:~0}).
%   \footnotetext{It was \texttt{fillcycle} in the original version.}
%   \setcounter{footnote}{1}
%   \item [\texttt{fillcycley}\footnotemark\ (integer)\hfill :] Same thing for
%   columns (\emph{Default:~0}).
%   \setcounter{footnote}{1}
%   \item [\texttt{fillcycle}\footnotemark\ (integer)\hfill :] Allow to fix
%   both \texttt{fillcyclex} and \texttt{fillcycley} directly to the same value
%   (\emph{Default:~0}).
% \end{Description}
% 
%   For instance, if \texttt{fillcyclex} is 2, the second row of patterns will
% be horizontally shifted by a factor of $\frac{1}{2}=0.5$, and by a factor of
% 0.333 if \texttt{fillcyclex} is 3, etc.). These values can be negative.
% 
% \begin{LTXexample}[width=0.35\linewidth]
% \psset{unit=0.5}
% \psboxfill{\Square}
% \newcommand{\TilingA}[1]{\Tiling[fillcyclex=#1]{(3,3)}}
% \TilingA{0} \TilingA{1}\\
% \TilingA{2} \TilingA{3}\\[3mm]
% \TilingA{4} \TilingA{5}\\
% \TilingA{6} \TilingA{-3}\\[3mm]
% \Tiling[fillcycley=2]{(3,3)}
% \Tiling[fillcycley=3]{(3,3)}\\
% \Tiling[fillcycley=-3]{(3,3)}
% \Tiling[fillcycle=2]{(3,3)}
% \end{LTXexample}
% 
% \begin{Description}{2cm}
%   \setcounter{footnote}{1}
%   \item [\texttt{fillmovex}\footnotemark\ (real$\|$dim)\hfill :] value of the
%   horizontal moves between consecutive patterns (\emph{Default:~0}).
%   \setcounter{footnote}{1}
%   \item [\texttt{fillmovey}\footnotemark\ (real$\|$dim)\hfill :] value of the
%   vertical moves between consecutive patterns (\emph{Default:~0}).
%   \setcounter{footnote}{1}
%   \item [\texttt{fillmove}\footnotemark\ (real$\|$dim)\hfill :] value of
%   horizontal and vertical moves between consecutive patterns
%   (\emph{Default:~0}).
% \end{Description}
% 
%   These parameters allow the patterns to overlap and to draw some special
% kinds of tilings. They are implemented only for the \emph{automatic} and
% \emph{tiling} modes and their values can be negative.
% 
%   In some cases, the effect of these parameters will be the same that with the 
% \texttt{fillcycle?} ones, but you can see that it is not true for some other
% values.
% 
% \begin{LTXexample}
% \psset{unit=0.5}
% \psboxfill{\Square}
% \Tiling[fillmovex=0.5]{(3,3)} 
% \Tiling[fillmovey=0.5]{(3,3)}\\
% \Tiling[fillmove=0.5]{(3,3)}
% \Tiling[fillmove=-0.5]{(3,3)}
% \end{LTXexample}
% 
% \begin{Description}{2cm}
%   \item [\texttt{fillsize}
%   (auto$\|$\{(real$\|$dim,real$\|$dim)(real$\|$dim,real$\|$dim)\}) :] The
%   choice of \emph{automatic} mode or the size of the area in \emph{manual}
%   mode. If first pair values are not given, (0,0) is used. (\emph{Default:
%   auto when \emph{tiling} mode is used, {(-15cm,-15cm)(15cm,15cm)}
%   otherwise}).
% \end{Description}
% 
%   As explained in the introduction, the \emph{manual} mode can require very
% huge amount of computer ressources. So, it usage is to discourage in front off
% the \emph{automatic} mode. It seems only useful in special circonstances, in
% fact when the \emph{automatic} mode failed, which is known only in one case,
% for some kinds of EPS files, as the ones produce by dump of portions of
% screens (see \ref{sec:GraphicFiles}).
% 
% \begin{Description}{2cm}
%   \setcounter{footnote}{1}
%   \item [\texttt{fillloopaddx}\footnotemark\ (integer)\hfill :] number of
%   times the pattern is added on left and right positions (\emph{Default:~0}).
%   \setcounter{footnote}{1}
%   \item [\texttt{fillloopaddy}\footnotemark\ (integer)\hfill :] number of
%   times the pattern is added on top and bottom positions (\emph{Default:~0}).
%   \setcounter{footnote}{1}
%   \item [\texttt{fillloopadd}\footnotemark\ (integer)\hfill :] number of
%   times the pattern is added on left, right, top and bottom positions
%   (\emph{Default:~0}).
% \end{Description}
% 
%   These parameters are only useful in special circonstances, as for complex
% patterns when the size of the rectangular box used to tile the area doesn't 
% correspond to the pattern itself (see an example in Figure~\ref{fig:Sheeps})
% and also sometimes when the size of the pattern is not a divisor of the size
% of the area to fill and that the number of loop repeats is not properly
% computed, which can occur.
% 
%   They are implemented only for the \emph{tiling} mode.
% 
% \begin{Description}{2cm}
%   \setcounter{footnote}{1}
%   \item [\texttt{PstDebug}\footnotemark\ (integer, 0 or 1)\hfill :] to
%   require to see the exact tiling done, without clipping (\emph{Default:~0}).
% \end{Description}
% 
%   It's mainly useful for debugging or to understand better how the tilings
% are done. It is implemented only for the \emph{tiling} mode.
% 
% \begin{LTXexample}
% \psset{unit=0.3,PstDebug=1}
% \psboxfill{\Square}
% \psset{linewidth=1mm}
% \Tiling{(2,2)}\\[5mm]
% \Tiling[fillcyclex=2]{(2,2)}\\[1cm]
% \Tiling[fillmove=0.5]{(2,2)}
% \end{LTXexample}
% 
% \vspace{3cm}
% \section{Examples}
% 
%   In fact this unique \cs{psboxfill} macro allow a lot a variations and
% different usages. We will try here to demonstrate this.
% 
% \subsection{Kind of tiles}
% \label{sec:KindTiles}
% 
%   Of course, we can access to all the power of PSTricks macros to define the
% \emph{tiles} (\emph{patterns}) used. So, we can define complicated ones.
% 
%   Here we give four other Archimedian tilings (those built with only some
% regular polygons) among the twelve existing, first discovered completely by
% Johanes \textsc{Kepler} at the beginning of 17th century \cite{GS87}, the two
% other \emph{regular} ones with the tiling by squares, formed by a unique
% regular polygon, and two other formed by two different regular polygons.
% 
% \begin{LTXexample}[pos=t]
%   \newcommand{\Triangle}{%
%     \begin{pspicture}(1,1)
%       \pstriangle[dimen=middle](0.5,0)(1,1)
%     \end{pspicture}}
%   \newcommand{\Hexagon}{
% ^^A sin(60)=0.866
%     \begin{pspicture}(0.866,0.75)
%       \SpecialCoor
% ^^A  Hexagon  
%       \pspolygon[dimen=middle]%
%         (0.5;30)(0.5;90)(0.5;150)(0.5;210)(0.5;270)(0.5;330)
%     \end{pspicture}}
% 
%   \psset{unit=0.5}
%   \psboxfill{\Triangle}
%   \Tiling{(4,4)}\hfill
% ^^A The two other regular tilings
%   \Tiling[fillcyclex=2]{(4,4)}\hfill
%   \psboxfill{\Hexagon}
%   \Tiling[fillcyclex=2,fillloopaddy=1]{(5,5)}
% \end{LTXexample}
% 
% \begin{LTXexample}[pos=t]
%   \newcommand{\ArchimedianA}{%
%      ^^A Archimedian tiling 3^2.4.3.4
%     \psset{dimen=middle}
%      ^^A sin(60)=0.866
%     \begin{pspicture}(1.866,1.866)
%       \psframe(1,1)
%       \psline(1,0)(1.866,0.5)(1,1)(0.5,1.866)(0,1)(-0.866,0.5)
%       \psline(0,0)(0.5,-0.866)
%     \end{pspicture}}
%   \newcommand{\ArchimedianB}{%
%      ^^A Archimedian tiling 4.8^2
%     \psset{dimen=middle,unit=1.5}
%      ^^A sin(22.5)=0.3827 ; cos(22.5)=0.9239
%     \begin{pspicture}(1.3066,0.6533)
%       \SpecialCoor
%      ^^A Octogon
%       \pspolygon(0.5;22.5)(0.5;67.5)(0.5;112.5)(0.5;157.5)
%                 (0.5;202.5)(0.5;247.5)(0.5;292.5)(0.5;337.5)
%     \end{pspicture}}
% 
%   \psset{unit=0.5}
%   \psboxfill{\ArchimedianA}
%   \Tiling[fillmove=0.5]{(7,7)}\hfill
%   \psboxfill{\ArchimedianB}
%   \Tiling[fillcyclex=2,fillloopaddy=1]{(7,7)}
% \end{LTXexample}
% 
%   \setcounter{footnote}{3}
%   We can of course tile an area arbitrarily defined. And with the
% \texttt{addfillstyle} parameter\footnote{Introduced in PSTricks 97.}, we can
% easily mix the \texttt{boxfill} style with another one.
% 
% \begin{LTXexample}[width=6cm]
%   \psset{unit=0.5,dimen=middle}
%   \psboxfill{%
%     \begin{pspicture}(1,1)
%       \psframe(1,1)
%       \pscircle(0.5,0.5){0.25}
%     \end{pspicture}}
%   \begin{pspicture}(4,6)
%     \pspolygon[fillstyle=boxfill,fillsep=0.25](0,1)(1,4)(4,6)(4,0)(2,1)
%   \end{pspicture}\hspace{1em}
%   \begin{pspicture}(4,4)
%%     \pscircle[linestyle=none,fillstyle=solid,fillcolor=yellow,fillsep=0.5,
%%               addfillstyle=boxfill](2,2){2}
%   \end{pspicture}
% \end{LTXexample}
%
%   Various effects can be obtained, sometimes complicated ones very easily, as
% in this example reproduced from one shown by Slavik \textsc{Jablan} in the
% field of \emph{OpTiles}, inspired by the \emph{Op-art}:
% 
% \begin{LTXexample}[pos=t]
% \newcommand{\ProtoTile}{%
%  \begin{pspicture}(1,1)%%% 1/12=0.08333
%   \psset{linestyle=none,linewidth=0,
%     hatchwidth=0.08333\psunit,hatchsep=0.08333\psunit}
%   \psframe[fillstyle=solid,fillcolor=black,addfillstyle=hlines,hatchcolor=white](1,1)
%   \pswedge[fillstyle=solid,fillcolor=white,addfillstyle=hlines]{1}{0}{90}
%  \end{pspicture}}
% \newcommand{\BasicTile}{%
%  \begin{pspicture}(2,1)
%    \rput[lb](0,0){\ProtoTile}\rput[lb](1,0){\psrotateleft{\ProtoTile}}
%  \end{pspicture}}
% \ProtoTile\hfill\BasicTile\hfill
% \psboxfill{\BasicTile}
% \Tiling[fillcyclex=2]{(4,4)}
% \end{LTXexample}
% 
%   It is also directly possible to surimpose several different tilings. Here is
% the splendid visual proof of the \textsc{Pytha\-gore} theorem done by the arab
% mathematician \textsc{Annairizi} around the year 900, given by superposition
% of two tilings by squares of different sizes.
% 
% \begin{LTXexample}[pos=t]
% \psset{unit=1.5,dimen=middle}
% \begin{pspicture*}(3,3)
%   \psboxfill{\begin{pspicture}(1,1)
%     \psframe(1,1)\end{pspicture}}
%   \psframe[fillstyle=boxfill](3,3)
%   \psboxfill{\begin{pspicture}(1,1)
%     \rput{-37}{\psframe[linecolor=red](0.8,0.8)}
%   \end{pspicture}}
%   \psframe[fillstyle=boxfill](3,4)
%   \pspolygon[fillstyle=hlines,hatchangle=90](1,2)(1.64,1.53)(2,2)
% \end{pspicture*}
% \end{LTXexample}
% 
%   In a same way, it is possible to build tilings based on figurative patterns,
% in the style of the famous \textsc{Escher} ones. Following an example of
% Andr\'e \textsc{Deledicq} \cite{Deledicq97}, we first show a simple tiling of
% the \emph{p1} category (according to the international classification of the
% 17~symmetry groups of the plane first discovered by the russian
% crystalographer Jevgraf \textsc{Fedorov} at the end of the 19th century):
% 
% \begin{LTXexample}[pos=t]
%  \newcommand{\SheepHead}[1]{%
%    \begin{pspicture}(3,1.5)
%      \pscustom[liftpen=2,fillstyle=solid,fillcolor=#1]{%
%        \pscurve(0.5,-0.2)(0.6,0.5)(0.2,1.3)(0,1.5)(0,1.5)
%          (0.4,1.3)(0.8,1.5)(2.2,1.9)(3,1.5)(3,1.5)(3.2,1.3)
%          (3.6,0.5)(3.4,-0.3)(3,0)(2.2,0.4)(0.5,-0.2)}
%      \pscircle*(2.65,1.25){0.12\psunit} % Eye
%      \psccurve*(3.5,0.3)(3.35,0.45)(3.5,0.6)(3.6,0.4)% Muzzle
%     ^^A   % Mouth
%       \pscurve(3,0.35)(3.3,0.1)(3.6,0.05)
%     ^^A   % Ear
%       \pscurve(2.3,1.3)(2.1,1.5)(2.15,1.7)\pscurve(2.1,1.7)(2.35,1.6)(2.45,1.4)
%   \end{pspicture}}
%  \psboxfill{\psset{unit=0.5}\SheepHead{yellow}\SheepHead{cyan}}
%  \Tiling[fillcyclex=2,fillloopadd=1]{(10,5)}
% \end{LTXexample}
% \label{fig:Sheeps}
% 
%   Now a tiling of the \emph{pg} category (the code for the kangaroo itself is
% too long to be shown here, but has no difficulties ; the kangaroo is reproduce
% from an original picture from Raoul \textsc{Raba} and here is a translation in
% PSTricks from the one drawn by Emmanuel \textsc{Chailloux} and Guy
% \textsc{Cousineau} for their MLgraph system \cite{MLgraphTSI}):
% 
% \begin{LTXexample}[pos=t]
% \psboxfill{\psset{unit=0.4}
%   \Kangaroo{yellow}\Kangaroo{red}\Kangaroo{cyan}\Kangaroo{green}%
%   \psscalebox{-1 1}{%
%     \rput(1.235,4.8){\Kangaroo{green}\Kangaroo{cyan}\Kangaroo{red}\Kangaroo{yellow}}}}
%   \Tiling[fillloopadd=1]{(10,6)}
% \end{LTXexample}
% 
%   And here a \textsc{Wang} tiling \cite{Wang65}, \cite[chapter
% 11]{GS87}, based on very simple tiles of the form of a square and composed
% of four colored triangles. Such tilings are built with only a matching color
% constraint. Despite of it simplicity, it is an important kind of tilings, as
% \textsc{Wang} and others used them to study the special class of
% \emph{aperiodic} tilings, and also because it was shown that surprisingly this 
% tiling is similar to a \textsc{Turing} machine.
% 
% \begin{LTXexample}[pos=t]
%   \newcommand{\WangTile}[4]{%
%     \begin{pspicture}(1,1)
%       \pspolygon*[linecolor=#1](0,0)(0,1)(0.5,0.5)
%       \pspolygon*[linecolor=#2](0,1)(1,1)(0.5,0.5)
%       \pspolygon*[linecolor=#3](1,1)(1,0)(0.5,0.5)
%       \pspolygon*[linecolor=#4](1,0)(0,0)(0.5,0.5)
%     \end{pspicture}}
%   \newcommand{\WangTileA}{\WangTile{cyan}{yellow}{cyan}{cyan}}
%   \newcommand{\WangTileB}{\WangTile{yellow}{cyan}{cyan}{red}}
%   \newcommand{\WangTileC}{\WangTile{cyan}{red}{yellow}{yellow}}
%   \newcommand{\WangTiles}[1][]{%
%     \begin{pspicture}(3,3) \psset{ref=lb}
%       \rput(0,2){\WangTileB}  \rput(1,2){\WangTileA}%
%       \rput(2,2){\WangTileC}  \rput(0,1){\WangTileC}%
%       \rput(1,1){\WangTileB}  \rput(2,1){\WangTileA}
%       \rput(0,0){\WangTileA}  \rput(1,0){\WangTileC}%
%       \rput(2,0){\WangTileB}
%       #1
%     \end{pspicture}}
%   \WangTileA\hfill\WangTileB\hfill\WangTileC\hfill
%   \WangTiles[{\psgrid[subgriddiv=0,gridlabels=0](3,3)}]\hfill
%   \psset{unit=0.4} \psboxfill{\WangTiles} \Tiling{(12,12)}
% \end{LTXexample}
% 
% \subsection{External graphic files}
% \label{sec:GraphicFiles}
% 
%   We can also fill an arbitrary area with an external image. We have only, 
% as usual, to matter of the \emph{BoundingBox} definition if there is no one
% provided or if it is not the accurate one, as for the well known
% \texttt{tiger} picture part of the \texttt{ghostscript} distribution.
% 
% \begin{LTXexample}[pos=t]
%   \psboxfill{%% Strangely require x1=x2...
%     \begin{pspicture}(0,1)(0,4.1)
%       \includegraphics[bb=17 176 560 74,width=3cm]{tiger}
%     \end{pspicture}}
%   \Tiling{(6,6.2)}
% \end{LTXexample}
% 
%   Nevertheless, there are some special files for which the \emph{automatic}
% mode doesn't work, specially for some files obtained by a screen dump, as in
% the next example, where a picture was reduced before it conversion in the
% \emph{Encapsulated PostScript} format by a screen dump utility. In this case,
% usage of the \emph{manual} mode is the only alternative, at the price of the
% real multiple inclusion of the EPS file. We must take care to specify the
% correct \texttt{fillsize} parameter, because otherwise the default values are
% large and will load the file many times, perhaps just really using few
% occurrences as the other ones would be clipped...
% 
% \begin{LTXexample}[pos=t]
%   \psboxfill{\includegraphics{flowers}}
%   \begin{pspicture}(8,4)
%     \psellipse[fillstyle=boxfill,fillsize={(8,4)}](4,2)(4,2)
%   \end{pspicture}
% \end{LTXexample}
% 
% \subsection{Tiling of characters}
% 
%   We can also use the \cs{psboxfill} macro to fill the interior of characters
% for special effects like these ones:
% 
% \begin{LTXexample}[pos=t]
%   \DeclareFixedFont{\bigsf}{T1}{phv}{b}{n}{4.5cm}
%   \DeclareFixedFont{\smallrm}{T1}{ptm}{m}{n}{3mm}
%   \psboxfill{\smallrm Since 182 days...}
%   \begin{pspicture*}(8,4)
%     \centerline{%
%       \pscharpath[fillstyle=gradient,gradangle=-45,
%                   gradmidpoint=0.5,addfillstyle=boxfill,
%                   fillangle=45,fillsep=0.7mm]
%                  {\rput[b](0,0.1){\bigsf 2000}}}
%   \end{pspicture*}
% \end{LTXexample}
% 
% \begin{LTXexample}[pos=t]
%   \DeclareFixedFont{\mediumrm}{T1}{ptm}{m}{n}{2cm}
%   \psboxfill{%
%     \psset{unit=0.1,linewidth=0.2pt}
%     \Kangaroo{PeachPuff}\Kangaroo{PaleGreen}%
%       \Kangaroo{LightBlue}\Kangaroo{LemonChiffon}%
%     \psscalebox{-1 1}{%
%       \rput(1.235,4.8){%
%         \Kangaroo{LemonChiffon}\Kangaroo{LightBlue}%
%           \Kangaroo{PaleGreen}\Kangaroo{PeachPuff}}}}
% ^^A   % A kangaroo of kangaroos...
%   \begin{pspicture}(8,2)
%     \pscharpath[linestyle=none,fillstyle=boxfill,fillloopadd=1]
%                {\rput[b](4,0){\mediumrm Kangaroo}}
%   \end{pspicture}
% \end{LTXexample}
% 
% \subsection{Other kinds of usage}
% 
%   Other kinds of usage can be imagined. For instance, we can use tilings in a
% sort of degenerated way to draw some special lines made by a unique or
% multiple repeating patterns. But it can be only a special dashed line, as here
% with three different dashes:
% 
% \begin{LTXexample}[pos=t]
%   \newcommand{\Dashes}{%
%     \psset{dimen=middle}
%     \begin{pspicture}(0,-0.5\pslinewidth)(1,0.5\pslinewidth)
%       \rput(0,0){\psline(0.4,0)}%
%         \rput(0.5,0){\psline(0.2,0)}%
%         \rput(0.8,0){\psline(0.1,0)}
%     \end{pspicture}}
% 
%   \newcommand{\SpecialDashedLine}[3]{%
%     \psboxfill{#3}
%     \Tiling[linestyle=none]
%            {(#1,-0.5\pslinewidth)(#2,0.5\pslinewidth)}}
% 
%   \SpecialDashedLine{0}{7}{\Dashes}
% 
%   \psset{unit=0.5,linewidth=1mm,linecolor=red}
%   \SpecialDashedLine{0}{10}{\Dashes}
% \end{LTXexample}
% 
%   It allow also to use special patterns in business graphics, as in the
% following example generated by \texttt{PstChart}\footnote{A personal
% development to draw business charts with PSTricks, not distributed.}.
% 
% \vspace{3mm}
% \begin{figure}[!ht]
% \centering
% \psset{unit=0.75}
% ^^A % Generated by pstchart.sh version 0.21 (11/28/97)
% {\psset{dimen=middle}
% \psset{xunit=2,yunit=0.005}
% \begin{pspicture}(-0.6,-200)(6.6,2300)
% ^^A   % Title
%   \rput(3,2200){\shortstack{Fantaisist repartition of kangaroos\\
%                             in the world (in thousands)}}
% ^^A   % Frame background
%   \psframe[fillstyle=solid,fillcolor=LemonChiffon](0,0)(6,2000)
% ^^A   % Graduations
%   \multido{\n=0+500}{5}{\rput[r](-0.12,\n){\psscalebox{0.8}{\n}}}
% ^^A   % Minor ticks
%   \multips(0,100)(0,100){19}{\psline[unit=4.8pt](1,0)}
%   \multips(6,100)(0,100){19}{\psline[unit=4.8pt](-1,0)}
% ^^A   % Major ticks
%   \multips(0,500)(0,500){3}{\psline[unit=9.6pt](1,0)}
%   \multips(6,500)(0,500){3}{\psline[unit=9.6pt](-1,0)}
% ^^A   % Lines from major ticks marks
%   \multips(0,500)(0,500){3}{\psline[linestyle=dotted,linewidth=0.6pt](6,0)}
% ^^A   % Drawing for the data
%   \psboxfill{\psset{unit=0.78\psxunit}\KangarooPstChart{red}}
%   \psframe[linestyle=none,fillstyle=boxfill,fillloopaddy=1](0.61,0)(1.39,1800)
%   \psboxfill{\psset{unit=0.78\psxunit}\KangarooPstChart{yellow}}
%   \psframe[linestyle=none,fillstyle=boxfill,fillloopaddy=1](1.61,0)(2.39,800)
%   \psboxfill{\psset{unit=0.78\psxunit}\KangarooPstChart{cyan}}
%   \psframe[linestyle=none,fillstyle=boxfill,fillloopaddy=1](2.61,0)(3.39,550)
%   \psboxfill{\psset{unit=0.78\psxunit}\KangarooPstChart{magenta}}
%   \psframe[linestyle=none,fillstyle=boxfill,fillloopaddy=1](3.61,0)(4.39,500)
%   \psboxfill{\psset{unit=0.78\psxunit}\KangarooPstChart{green}}
%   \psframe[linestyle=none,fillstyle=boxfill,fillloopaddy=1](4.61,0)(5.39,200)
% ^^A   % Bottom labels
%   \uput{0.2}[270]{0}(1,0){\psscalebox{0.7}{Oceania}}
%   \uput{0.2}[270]{0}(2,0){\psscalebox{0.7}{Africa}}
%   \uput{0.2}[270]{0}(3,0){\psscalebox{0.7}{Asia}}
%   \uput{0.2}[270]{0}(4,0){\psscalebox{0.7}{America}}
%   \uput{0.2}[270]{0}(5,0){\psscalebox{0.7}{Europe}}
% ^^A   % Frame box around the chart
%   \psframe[linestyle=solid](0,0)(6,2000)
% \end{pspicture}}
%   \caption{Bar chart generated by PstChart, with bars filled by patterns}
%   \label{fig:PstChart}
% \end{figure}
% 
% \section{``Dynamic'' tilings}
% 
%   In some cases, tilings used non \emph{static} tiles, that is to say that the 
% \emph{prototile(s)}, even if unique, can have several forms, by instance
% specified by different colors or rotations, not fixed before generation or
% varying each time.
% 
% \subsection{Lewthwaite-Pickover-Truchet tiling}
% 
%   We give here for example the so-called \emph{Truchet} tiling, which much be
% in fact better called \emph{Lewthwaite-Pick\-over-Truchet (LPT)} tiling%
% \footnote{For description of the context, history and references about
% S\'ebastien \textsc{Truchet} and this tiling, see \cite{EsperetGirou98}.}.
% 
%   The unique prototile is only a square with two opposite circle arcs.
% This tile has obviously two positions, if we rotate it from 90 degrees (see
% the two tiles on the next figure). A \emph{LPT tiling} is a tiling with
% randomly oriented LPT tiles. We can see that even if it is very simple in it
% principle, it draw sophisticated curves with strange properties.
% 
%   Nevertheless, in the straightforward way \FillPackage{} does not work,
% because the \cs{psboxfill} macro store the content of the tile used in a
% \TeX{} box, which is static. So the calling to the random function is done
% only one time, which explain that only one rotation of the tile is used for
% all the tiling. It's only the one of the two rotations which could differ from
% one drawing to the next one...
% 
% ^^A % Truchet (Lewthwaite-Pickover-Truchet) tiling
% ^^A % --------------------------------------------
% 
% \begin{LTXexample}[pos=t]
% ^^A   % LPT prototile
%   \newcommand{\ProtoTileLPT}{%
%     \psset{dimen=middle}
%     \begin{pspicture}(1,1)
%       \psframe(1,1)
%       \psarc(0,0){0.5}{0}{90}
%       \psarc(1,1){0.5}{-180}{-90}
%     \end{pspicture}}
% 
% ^^A   % LPT tile
%   \newcount\Boolean
%   \newcommand{\BasicTileLPT}{%
% ^^A     % From random.tex by Donald Arseneau
%     \setrannum{\Boolean}{0}{1}%
%     \ifnum\Boolean=0
%       \ProtoTileLPT%
%     \else
%       \psrotateleft{\ProtoTileLPT}%
%     \fi}
% 
%   \ProtoTileLPT\hfill\psrotateleft{\ProtoTileLPT}\hfill
%   \psset{unit=0.5}
%   \psboxfill{\BasicTileLPT}
%   \Tiling{(5,5)}
% \end{LTXexample}
% 
%   But, for simple cases, there is a solution to this problem using a mixture
% of PSTricks and PostScript programming. Here the PSTricks
% construction \verb+\pscustom{\code{...}}+ allow to insert PostScript code
% inside the \LaTeX{} + PSTricks one.
% 
%   Programmation is less straightforward, but it has also the advantage to be
% notably faster, as all the tilings operations are done in PostScript, and
% mainly to not be limited by \TeX{} memory (the \TeX{} + PSTricks solution
% I wrote in 1995 for the colored problem was limited to small sizes for this
% reason). Just note also that \cs{pslbrace} and \cs{psrbrace} are two
% PSTricks macros to define and be able to insert the \verb+{+ and \verb+}+
% characters.
% 
% \begin{LTXexample}[pos=t]
% ^^A   % LPT prototile
%   \newcommand{\ProtoTileLPT}{%
%     \psset{dimen=middle}
%     \psframe(1,1)
%     \psarc(0,0){0.5}{0}{90}
%     \psarc(1,1){0.5}{-180}{-90}}
% 
% ^^A   % Counter to change the random seed
%   \newcount\InitCounter
% ^^A   % LPT tile
%   \newcommand{\BasicTileLPT}{%
%     \InitCounter=\the\time
%     \pscustom{\code{%
%       rand \the\InitCounter\space sub 2 mod 0 eq \pslbrace}}
%     \begin{pspicture}(1,1)
%       \ProtoTileLPT
%     \end{pspicture}%
%     \pscustom{\code{\psrbrace \pslbrace}}
%     \psrotateleft{\ProtoTileLPT}%
%     \pscustom{\code{\psrbrace ifelse}}}
% 
%   \psset{unit=0.4,linewidth=0.4pt}
%   \psboxfill{\BasicTileLPT}
%   \Tiling{(15,15)}
% \end{LTXexample}
% 
%   Using the very surprising fact (see \cite{EsperetGirou98}) that
% coloration of these tiles do not depend of their neighbors (even if it is
% difficult to believe as the opposite seems obvious!) but only of the parity of
% the value of row and column positions, we can directly program in the same way
% a colored version of the LPT tiling.
% 
% \setcounter{footnote}{1}
%   We have also introduce in the \FillPackage{} code for \emph{tiling} mode two
% new accessible Post\-Script variables, \texttt{row} and
% \texttt{column}\footnotemark, which can be useful in some circonstances, like
% this one.
% 
% \begin{LTXexample}[pos=t]
% ^^A   % LPT prototile
%   \newcommand{\ProtoTileLPT}[2]{%
%     \psset{dimen=middle,linestyle=none,fillstyle=solid}
%     \psframe[fillcolor=#1](1,1)
%     \psset{fillcolor=#2}
%     \pswedge(0,0){0.5}{0}{90} \pswedge(1,1){0.5}{-180}{-90}}
% ^^A   % Counter to change the random seed
%   \newcount\InitCounter
% ^^A   % LPT tile
%   \newcommand{\BasicTileLPT}[2]{%
%     \InitCounter=\the\time
%     \pscustom{\code{%
%       rand \the\InitCounter\space sub 2 mod 0 eq \pslbrace
%       row column add 2 mod 0 eq \pslbrace}}
%     \begin{pspicture}(1,1)\ProtoTileLPT{#1}{#2}\end{pspicture}%
%     \pscustom{\code{\psrbrace \pslbrace}}
%     \ProtoTileLPT{#2}{#1}%
%     \pscustom{\code{%
%       \psrbrace ifelse \psrbrace \pslbrace row column add 2 mod 0 eq \pslbrace}}
%     \psrotateleft{\ProtoTileLPT{#2}{#1}}\pscustom{\code{\psrbrace \pslbrace}}
%     \psrotateleft{\ProtoTileLPT{#1}{#2}}\pscustom{\code{\psrbrace ifelse \psrbrace ifelse}}}
%   \psboxfill{\BasicTileLPT{red}{yellow}}
%   \Tiling{(4,4)}\hfill
%   \psset{unit=0.4}\psboxfill{\BasicTileLPT{blue}{cyan}}
%   \Tiling{(15,15)}
% \end{LTXexample}
% 
%   Another classic example is to generate coordinates and numerotation for a
% grid. Of course, it is possible to do it directly in PSTricks using nested
% \cs{multido} commands. It would be clearly easy to program, but, nevertheless, 
% for users who have a little knowledge of PostScript programming, this offer
% an alternative which is useful for large cases, because on this way it will
% be notably faster and less computer ressources consuming.
% 
%   Remember here that the tiling is drawn from left to right, and top to
% bottom, and note that the PostScript variable \texttt{x2} give the total
% number of columns.
% 
% \begin{LTXexample}[pos=t]
% ^^A   % \Escape will be the \ character
%   {\catcode`\!=0\catcode`\\=11!gdef!Escape{\}}
%   \newcommand{\ProtoTile}{%
%     \Square\pscustom{%
%       \moveto(-0.9,0.75) % In PSTricks units
%       \code{ /Times-Italic findfont 8 scalefont setfont
%         (\Escape() show row 3 string cvs show (,) show 
%         column 3 string cvs show (\Escape)) show}
%       \moveto(-0.5,0.25) % In PSTricks units
%       \code{ /Times-Bold findfont 18 scalefont setfont
%         1 0 0 setrgbcolor % Red color
%         /center {dup stringwidth pop 2 div neg 0 rmoveto} def
%         row 1 sub x2 mul column add 3 string cvs center show}}}
%   \psboxfill{\ProtoTile}
%   \Tiling{(6,4)}
% \end{LTXexample}
% 
% \subsection{A complete example: the Poisson equation}
% 
%   To finish, we will show a complete real example, a drawing to explain the
% method used to solve the \textsc{Poisson} equation by a domain
% decomposition method, adapted to distributed memory computers. The
% objective is to show the communications required between processes and the
% position of the data to exchange. This code also show some useful and powerful
% technics for PSTricks programming (look specially at the way some higher level
% macros are defined, and how the same object is used to draw the four
% neighbors).
%
%\psset{unit=1cm}
%\newcommand{\Pattern}[1]{%
%   \begin{pspicture}(-0.25,-0.25)(0.25,0.25)\rput{*0}{\psdot[dotstyle=#1]}
%   \end{pspicture}}
%\newcommand{\West}{\Pattern{o}}   \newcommand{\South}{\Pattern{x}}
%\newcommand{\Central}{\Pattern{+}}\newcommand{\North}{\Pattern{square}}
%\newcommand{\East}{\Pattern{triangle}}
%\newcommand{\Cross}{%
%  \pspolygon[unit=0.5,linewidth=0.2,linecolor=red](0,0)(0,1)(1,1)(1,2)(2,2)(2,1)%
%              (3,1)(3,0)(2,0)(2,-1)(1,-1)(1,0)}
%\newcommand{\StylePosition}[1]{\LARGE\textcolor{red}{\textbf{#1}}}
%\newcommand{\SubDomain}[4]{%
%    \psboxfill{#4}\begin{psclip}{\psframe[linestyle=none]#1}%
%      \psframe[linestyle=#3](5,5)\psframe[fillstyle=boxfill]#2%
%    \end{psclip}}
%\newcommand{\SendArea}[1]{\psframe[fillstyle=solid,fillcolor=cyan]#1}
%\newcommand{\ReceiveData}[2]{%
%  \psboxfill{#2}\psframe[fillstyle=solid,fillcolor=yellow,addfillstyle=boxfill]#1}%
%\newcommand{\Neighbor}[2]{%
%    \begin{pspicture}(5,5)
%      \rput{*0}(2.5,2.5){\StylePosition{#1}}
%      \ReceiveData{(0.5,0)(4.5,0.5)}{\Central}\SendArea{(0.5,0.5)(4.5,1)}%
%      \SubDomain{(5,2)}{(0.5,0.5)(4.5,3)}{dashed}{#2}%
%      \pcarc[arcangle=45,arrows=->](0.5,-1.25)(0.5,0.25)%
%      \pcarc[arcangle=45,arrows=->,linestyle=dotted,dotsep=2pt](4.5,0.75)(4.5,-0.75)%
%    \end{pspicture}}%
%  \psset{dimen=middle,dotscale=2,fillloopadd=2}
%\begin{pspicture}(-5.7,-5.7)(5.7,5.7)
%  \rput(0,0){%
%      \begin{pspicture}(5,5)
%        \ReceiveData{(0,0.5)(0.5,4.5)}{\West} \ReceiveData{(4.5,0.5)(5,4.5)}{\East}
%        \ReceiveData{(0.5,4.5)(4.5,5)}{\North}\ReceiveData{(0.5,0)(4.5,0.5)}{\South}
%        \SendArea{(0.5,0.5)(1,4.5)}\SendArea{(4,0.5)(4.5,4.5)}
%        \SendArea{(0.5,0.5)(4.5,1)}\SendArea{(0.5,4)(4.5,4.5)}
%        \SubDomain{(5,5)}{(0.5,0.5)(4.5,4.5)}{solid}{\Central}
%        \psline(1,0.5)(1,4.5)\psline(4,0.5)(4,4.5)%
%        \rput(1.5,4){\Cross}\rput(2,2){\Cross}%
%      \end{pspicture}}%
%  \rput(0,5.5){\Neighbor{N}{\North}}\rput{-90}(5.5,0){\Neighbor{E}{\East}}%
%  \rput{90}(-5.5,0){\Neighbor{W}{\West}}\rput{180}(0,-5.5){\Neighbor{S}{\South}}%
%\end{pspicture}
%
% \begin{lstlisting}
%   \newcommand{\Pattern}[1]{%
%     \begin{pspicture}(-0.25,-0.25)(0.25,0.25)\rput{*0}{\psdot[dotstyle=#1]}
%     \end{pspicture}}
%   \newcommand{\West}{\Pattern{o}}   \newcommand{\South}{\Pattern{x}}
%   \newcommand{\Central}{\Pattern{+}}\newcommand{\North}{\Pattern{square}}
%   \newcommand{\East}{\Pattern{triangle}}
%   \newcommand{\Cross}{%
%     \pspolygon[unit=0.5,linewidth=0.2,linecolor=red](0,0)(0,1)(1,1)(1,2)(2,2)(2,1)
%               (3,1)(3,0)(2,0)(2,-1)(1,-1)(1,0)}
%   \newcommand{\StylePosition}[1]{\LARGE\textcolor{red}{\textbf{#1}}}
%   \newcommand{\SubDomain}[4]{%
%     \psboxfill{#4}
%     \begin{psclip}{\psframe[linestyle=none]#1}
%       \psframe[linestyle=#3](5,5)\psframe[fillstyle=boxfill]#2
%     \end{psclip}}
%   \newcommand{\SendArea}[1]{\psframe[fillstyle=solid,fillcolor=cyan]#1}
%   \newcommand{\ReceiveData}[2]{%
%     \psboxfill{#2}
%     \psframe[fillstyle=solid,fillcolor=yellow,addfillstyle=boxfill]#1}
%   \newcommand{\Neighbor}[2]{%
%     \begin{pspicture}(5,5)
%       \rput{*0}(2.5,2.5){\StylePosition{#1}}
%       \ReceiveData{(0.5,0)(4.5,0.5)}{\Central}\SendArea{(0.5,0.5)(4.5,1)}
%       \SubDomain{(5,2)}{(0.5,0.5)(4.5,3)}{dashed}{#2}%
% ^^A       % Receive and send arrows
%       \pcarc[arcangle=45,arrows=->](0.5,-1.25)(0.5,0.25)
%       \pcarc[arcangle=45,arrows=->,linestyle=dotted,dotsep=2pt](4.5,0.75)(4.5,-0.75)
%     \end{pspicture}}
%   \psset{dimen=middle,dotscale=2,fillloopadd=2}
%   \begin{pspicture}(-5.7,-5.7)(5.7,5.7)
% ^^A     % Central domain
%     \rput(0,0){%
%       \begin{pspicture}(5,5)
% ^^A         % Receive from West, East, North and South
%         \ReceiveData{(0,0.5)(0.5,4.5)}{\West} \ReceiveData{(4.5,0.5)(5,4.5)}{\East}
%         \ReceiveData{(0.5,4.5)(4.5,5)}{\North}\ReceiveData{(0.5,0)(4.5,0.5)}{\South}
% ^^A         % send area for West, East, North and South
%         \SendArea{(0.5,0.5)(1,4.5)} \SendArea{(4,0.5)(4.5,4.5)}
%         \SendArea{(0.5,0.5)(4.5,1)} \SendArea{(0.5,4)(4.5,4.5)}
% ^^A         % Central domain
%         \SubDomain{(5,5)}{(0.5,0.5)(4.5,4.5)}{solid}{\Central}
% ^^A         % Redraw overlapped linesY
%         \psline(1,0.5)(1,4.5)  \psline(4,0.5)(4,4.5)
% ^^A         % Two crossesY
%         \rput(1.5,4){\Cross}  \rput(2,2){\Cross}
%       \end{pspicture}}
% ^^A     % The four neighborsY
%     \rput(0,5.5){\Neighbor{N}{\North}}     \rput{-90}(5.5,0){\Neighbor{E}{\East}}
%     \rput{90}(-5.5,0){\Neighbor{W}{\West}} \rput{180}(0,-5.5){\Neighbor{S}{\South}}
%   \end{pspicture}
% \end{lstlisting}
%
%
%
% Bibliography
% \begin{thebibliography}{99}
% \bibitem{PostScript95} Adobe, Systems~Incorporated, \emph{PostScript Language
% Reference Manual}, Addison-Wesley, 2~edition, 1995.
%
% \bibitem{Bolek98} Piotr Bolek, \MP{} and patterns, \emph{\TUB}, Volume~19,
% Number~3, pages 276--283, September 1998, \CTANref{mpattern}.
%
% \bibitem{MLgraphTSI} Emmanuel Chailloux, Guy Cousineau and Asc\'ander
% Su\'arez, Programmation fonctionnelle de graphismes pour la production
% d'illustrations techniques, \emph{Technique et science informatique},
% Volume~15, Number~7, pages 977--1007, 1996 (in french).
%
% \bibitem{Deledicq97} Andr\'e Deledicq, \emph{Le monde des pavages}, ACL
% \'Editions, 1997 (in french).
%
% \bibitem{EsperetGirou98} Philippe Esperet and Denis Girou,
% Coloriage du pavage dit de Truchet, Cahiers GUTenberg, Number~31,
% pages 5--18, December~1998  (in french).
%
% \bibitem{Girou94} Denis Girou, Pr\'esentation de PSTricks, \emph{Cahiers
% GUTenberg}, Number~16, pages 21--70, February~1994 (in french).
%
% \bibitem{LGC97} Michel Goossens, Sebastian Rahtz and Frank Mittelbach,
% \emph{The \LaTeX{} Graphics Companion}, Addison-Wesley, 2005.
%
% \bibitem{GS87} Branko Gr\"unbaum and Geoffrey Shephard, \emph{Tilings and
% Patterns}, Freeman and Company, 1987.
%
% \bibitem{Hoenig97} Alan Hoenig, \emph{\TeX{} Unbound: \LaTeX{} \& \TeX{}
% Strategies, Fonts, Graphics, and More}, Oxford University Press, 1997.
%
% \bibitem{XYpic} Kristoffer~H. Rose and Ross Moore, \XYpic. Pattern and Tile
% extension, available from \CTAN, 1991-1998, \CTANref{xypic}.
%
% \bibitem{LAAN96} Kees van der Laan, Paradigms: Just a little bit of PostScript,
% \emph{MAPS}, Volume~17, pages 137--150, 1996.
%
% \bibitem{LAAN97} Kees van der Laan, Tiling in PostScript and \MF{} -- Escher's
% wink, \emph{MAPS}, Volume~19, Number~2, pages 39--67, 1997.
%
% \bibitem{vanZandt93} Timothy Van Zandt, PSTricks. PostScript macros for
% Generic \TeX, available from \CTAN, 1993, \CTANref{pstricks}.
%
% \bibitem{vanZandtGirou94} Timothy Van Zandt and Denis Girou, Inside PSTricks,
% \emph{\TUB}, Volume~15, Number~3, pages 239--246, September 1994.
%
%
% \bibitem{voss07} Herbert Vo\ss, PSTricks -- Graphics for \TeX\ and \LaTeX, DANTE/Lehmanns, 4th ed., 2007.
% \bibitem{Wang65} Hao Wang, Games, Logic and Computers, \emph{Scientific
% American}, pages 98--106, November 1965.
% \end{thebibliography}
%
%
% \StopEventually{}
%
% ^^A .................... End of the documentation part ....................
%
% \section{Driver file}
%
%   The next bit of code contains the documentation driver file for \TeX{},
% i.e., the file that will produce the documentation you are currently
% reading. It will be extracted from this file by the \texttt{docstrip}
% program.
%
%    \begin{macrocode}
%<*driver>
\documentclass{ltxdoc}
\GetFileInfo{pst-fill.dtx}
%
\usepackage[T1]{fontenc}
\usepackage{lmodern}               % For PDF
\usepackage{graphicx}              % `graphicx' LaTeX standard package
\usepackage{showexpl}
\usepackage{mflogo}                % For the MetaFont and MetaPost logos
\input{random.tex}                 % Random macros from Donald Arseneau
\usepackage{url}                   % URLs convenient typesetting
\usepackage{multido}               % General loop macro
\usepackage[dvipsnames]{pstricks}  % PSTricks with the `color' extension
\usepackage{pst-text}              % PSTricks package for character path
\usepackage{pst-grad}              % PSTricks package for gradient filling
\usepackage{pst-node}              % PSTricks package for nodes
\usepackage[tiling]{pst-fill}      % PSTricks package for filling/tiling
%
\AtBeginDocument{%
%  \OnlyDescription % comment out for implementation details
  \EnableCrossrefs
  \CodelineIndex
  \RecordChanges}
\AtEndDocument{%
  \PrintIndex
  \setcounter{IndexColumns}{1}
  \PrintChanges}
\hbadness=7000            % Over and under full box warnings
\hfuzz=3pt
\begin{document}
  \DocInput{pst-fill.dtx}
\end{document}
%</driver>
%    \end{macrocode}
%
% \section{\texttt{pst-fill} \LaTeX{} wrapper}
%
%    \begin{macrocode}
%<*latex-wrapper>
\RequirePackage{pstricks}
\ProvidesPackage{pst-fill}[2005/09/13 package wrapper for 
  pst-fill.tex (hv)]
\DeclareOption{tiling}{\def\PstTiling{true}}
\ProcessOptions\relax
% \iffalse meta-comment, etc.
%%
%% Package `pst-fill.dtx'
%%
%% Denis Girou (CNRS/IDRIS - France) <Denis.Girou@idris.fr>
%% Herbert Voss <voss@pstricks.de>
%%
%% This program can be redistributed and/or modified under the terms
%% of the LaTeX Project Public License Distributed from CTAN archives
%% in directory macros/latex/base/lppl.txt.
%%
%% DESCRIPTION:
%%   `pst-fill' is a PSTricks package for filling and tiling areas 
%%
% \fi
% \changes{v1.01}{2007/03/10}{bugfix for incomplete ifx (hv)}
% \changes{v1.00}{2006/11/06}{use pst-xkey for extend keys (hv)}
% \changes{v0.99}{2004/08/17}{merge the VTeX and TeX versions (patch 4) (hv)}
% \changes{v0.98}{2004/06/22}{delete the Pst@Debug option and use the
%   the one from pstricks to prevent a clash with pst-gr3d (hv)}
% \changes{v0.97}{2001/10/09}{make it work with VTeX (mv)}
% \changes{v0.94}{1997/04/08}{With a \PstTiling macro defined (or "tiling" optional parameter
%   on \textbackslash usepackage[tiling]{pst-fill}), this file run exactly as
%   the original boxfill.tex file from Timothy, version 0.94,
%   except a correction in \textbackslash pst@ManualFillCycle to avoid a division by 0.
%   It's the default.}
% \changes{v0.93}{1997/04/07}{With a \textbackslash PstTiling macro defined (or "tiling" optional parameter
%   on \textbackslash usepackage[tiling]{pst-fill}) there are several add-ons
%   and changes to do `tiling' rather than `filling' in "automatic" mode :
%     - we fix the position of the beginning of tiling,
%     - we allow normally the framing of the area as expected, using
%       the line.... parameters
%     - we define move parameters fillmovex, fillmovey and fillmove,
%     - we define fillcyclex as previous fillcycle parameter, and add the
%       fillcycley and fillcycle (both fillcyclex and fillcycley) ones
%     - we can extend the tiling area using fillloopaddx, fillloopaddy and
%       fillloopadd parameters,
%     - we can debug and see the whole tiling area without clipping using
%       PstDebug parameter,
%     - for names consistancy, we can use fillangle in place of boxfillangle
%       and fillsize in place of boxfillsize,
%     - default value for fillsep is 0 and for fillsize is auto.}
%
% \DoNotIndex{\!,\",\#,\$,\%,\&,\',\(,\+,\*,\,,\-,\.,\/,\:,\;,\<,\=,\>,\?}
% \DoNotIndex{\@,\@B,\@K,\@cTq,\@f,\@fPl,\@ifnextchar,\@nameuse,\@oVk}
% \DoNotIndex{\[,\\,\],\^,\_,\ }
% \DoNotIndex{\^,\\^,\\\^,$\^$,$\\^$,$\\^$}
% \DoNotIndex{\0,\2,\4,\5,\6,\7,\8,}
% \DoNotIndex{\A,\a}
% \DoNotIndex{\B,\b,\Bc,\begin,\Bq,\Bqc}
% \DoNotIndex{\C,\c,\catcode,\cJA,\CodelineIndex,\csname}
% \DoNotIndex{\D,\def,\define@key,\Df,\divide,\DocInput,\documentclass,\pst@addfams}
% \DoNotIndex{\eCN,\edef,\else,\eHd,\eMcj,\EnableCrossrefs,\end,\endcsname}
% \DoNotIndex{\endCenterExample,\endExample,\endinput,\endpsclip}
% \DoNotIndex{\PrintIndex,\PrintChanges,\ProvidesFile}
% \DoNotIndex{\endpspicture,\endSideBySideExample,\Example}
% \DoNotIndex{\F,\f,\FdUrr,\fi,\filedate,\fileversion,\FV@Environment}
% \DoNotIndex{\FV@UseKeyValues,\FV@XRightMargin,\FVB@Example,\fvset}
% \DoNotIndex{\G,\g,\GetFileInfo,\gr,\GradientLoaded,\gsFKrbK@o,\gsj,\gsOX}
% \DoNotIndex{\hbadness,\hfuzz,\HLEmphasize,\HLMacro,\HLMacro@i}
% \DoNotIndex{\HLReverse,\HLReverse@i,\hqcu,\HqY}
% \DoNotIndex{\I,\i,\ifx,\input,\Ir,\IU}
% \DoNotIndex{\j,\jl,\JT,\JVodH}
% \DoNotIndex{\K,\k,\kfSlL}
% \DoNotIndex{\L,\let}
% \DoNotIndex{\message,\mHNa,\mIU}
% \DoNotIndex{\N,\nB,\newcmykcolor,\newdimen,\newif,\nW}
% \DoNotIndex{\O,\oCDJDo,\ocQhVI,\OnlyDescription,\oRKJ}
% \DoNotIndex{\P,\p,\ProvidesPackage,\psframe,\pslinewidth,\psset}
% \DoNotIndex{\PstAtCode,\PSTricksLoaded}
% \DoNotIndex{\q,\Qr,\qssRXq,\qu,\qXjFQp,\qYL}
% \DoNotIndex{\R,\r,\RecordChanges,\relax,\RlaYI,\rN,\Rp,\rp,\RPDXNn,\rput}
% \DoNotIndex{\S,\scalebox,\SgY,\SideBySide@Example,\SideBySideExample}
% \DoNotIndex{\SgY,\sk,\Sp,\space,\sZb}
% \DoNotIndex{\T,\the,\tw@}
% \DoNotIndex{\u,\UiSWGEf@,\uJi,\usepackage,\uVQdMM,\UYj}
% \DoNotIndex{\VerbatimEnvironment,\VerbatimInput,\VrC@}
% \DoNotIndex{\WhZ,\WjKCYb,\WNs}
% \DoNotIndex{\XkN,\XW}
% \DoNotIndex{\Z,\ZCM,\Ze}
% \DoNotIndex{\addtocounter,\advance,\alph,\arabic,\AtBeginDocument,\AtEndDocument}
% \DoNotIndex{\AtEndOfPackage,\begingroup,\bfseries,\bgroup,\box,\csname}
% \DoNotIndex{\else,\endcsname,\endgroup,\endinput,\expandafter,\fi}
% \DoNotIndex{\TeX,\z@,\p@,\@one,\xdef,\thr@@,\string,\sixt@@n,\reset,\or,\multiply,\repeat,\RequirePackage}
% \DoNotIndex{\@cclvi,\@ne,\@ehpa,\@nil,\copy,\dp,\global,\hbox,\hss,\ht,\ifodd,\ifdim,\ifcase,\kern}
% \DoNotIndex{\chardef,\loop,\leavevmode,\ifnum,\lower}
% \setcounter{IndexColumns}{2}
%
% ^^A To extend the height used for the text
%
% ^^A  Aligned labels in a description environment
%\newenvironment{Description}[1]{%
%\begin{list}{nothing}{\setlength{\leftmargin}{#1}
%\setlength{\labelwidth}{\leftmargin}\setlength{\labelsep}{1mm}}}
%{\end{list}}
%
% ^^A For macro names
%\DeclareRobustCommand\cs[1]{\texttt{\char`\\#1}}
%
%
% ^^A From ltugboat.cls
% ^^A For references
%\makeatletter
%\newcommand\acro[1]{\textsc{#1}\@}
%\def\CTAN{\acro{CTAN}}
%\let\texttub\textsl              % ^^A redefined in other situations
%\def\TUB{\texttub{TUGboat}}
%\def\TUG{\TeX\ \UG}
%\def\tug{\acro{TUG}}
%\def\UG{Users Group}
% ^^A For the bibliography 
%\let\@internalcite\cite
%\def\cite{\def\@citeseppen{-1000}%
%    \def\@cite##1##2{(##1\if@tempswa , ##2\fi)}%
%    \def\citeauthoryear##1##2##3{##1, ##3}\@internalcite}
%\def\etal{et\,al.\@}
%\newcommand\CTANdirectory[1]{\expandafter\urldef
%  \csname CTAN@#1\endcsname\path}
%\newcommand\CTANfile[1]{\expandafter\urldef
%  \csname CTAN@#1\endcsname\path}
%\newcommand\CTANref[1]{\expandafter\@setref\csname CTAN@#1\endcsname
%  \relax{#1}}
%\makeatother
% ^^A Define CTAN addresses 
%\CTANdirectory{mpattern}{graphics/metapost/macros/mpattern}
%\CTANdirectory{pstricks}{graphics/pstricks}
%\CTANdirectory{pst-fill.sty}{graphics/pstricks/latex/pst-fill.sty}
%\CTANdirectory{pst-fill}{graphics/pstricks/generic/pst-fill.tex}
%\CTANdirectory{Roegel}{graphics/metapost/contrib/macros/truchet}
%\CTANdirectory{xypic}{macros/generic/diagrams/xypic}
%
% ^^A Personal macros (D.G.)
% ^^A ----------------------
%
% ^^A Some colors used
%\definecolor{LemonChiffon}{rgb}{1.,0.98,0.8}
%\definecolor{LightBlue}   {rgb}{0.8,0.85,0.95}
%\definecolor{PaleGreen}   {rgb}{0.88,1,0.88}
%\definecolor{PeachPuff}   {rgb}{1.0,0.85,0.73}
%
% ^^A To define a unique string for TeX and LaTeX
%\newcommand{\AllTeX}{%
%{\rm(L\kern-.36em\raise.3ex\hbox{\sc a}\kern-.15em)%
%T\kern-.1667em\lower.7ex\hbox{E}\kern-.125emX}}
%
% ^^A Bibliography style
%\bibliographystyle{ltugbib}
%
% ^^A Name macros
%\newcommand{\FillPackage}{\textsf{`pst-fill'}}
%\newcommand{\XYpic}{%
%\leavevmode\hbox{\kern-.1em X\kern-.3em\lower.4ex\hbox{Y\kern-.15em}-pic}}
%
%\makeatletter
%
% ^^A Example environments
% ^^A (do not use in them the four JXYZ characters, that we will use
% ^^A as escape characters!)
%
% ^^A Default PSTricks parameters
%  \psset{dimen=middle}
%
% ^^A Translation in PSTricks from the one drawn by Emmanuel Chailloux and
% ^^A Guy Cousineau for the MLgraph system
% ^^A (see /ftp.ens.fr:/pub/unix/lang/MLgraph/version-2.1/MLgraph-refman.ps.gz)
% ^^A The kangaroo itself is reproduce from an original picture from Raoul Raba
% \newcommand{\DimX}{2.47}
% \newcommand{\DimY}{4.8}
% \newcommand{\DimXDivTwo}{1.235}
%
% \newcommand{\KangarooItself}[1]{%
% ^^A Body
% \pspolygon[fillstyle=solid,fillcolor=#1]%
%  (52.5,68)(55,72.5)(55.8,76.5)(56.8,79.8)(58.2,83)(60,85.8)(61.5,86.5)
% (64,87)(66,87.5)(67.8,87.3)(70,87)(71.5,87.3)(73,88)(74.7,88.5)
% (76,90.3)(77,91.5)(72.8,93.8)(69,96)(64.5,99)(59.4,103)(56.2,106.3)
% (53,110.5)(49.5,115.5)(47.2,119.9)(45.7,126)(43.2,123)(41.5,121)(37.5,125)
% (37,122.5)(36.8,120)(37,117)(37.6,113.5)(38.6,110)(40,106.3)(42,102.3)
%  (43.5,99.5)(45,97)(46.2,94)(46.8,91.7)(47.2,88)(47,83.5)(46.3,80.8)
%  (45.3,78.5)(42.5,76.5)(39.5,75.8)(36,75.9)(33,75.9)(29,76.2)(26,77)
%  (22.3,77.5)(18,78.4)(12.8,79.3)(8.6,80)(5.5,80.3)(3,80.5)(0,80)
%  (-5.2,78.5)(-9,76.3)(-11.2,74.8)(-13,72.5)(-16.5,68)(-16.5,68)(-19.5,62.5)
%  (-22,58)(-25.5,53)(-29,48.5)(-32.5,45)(-36,42)(-39,39.5)(-44,37)
%  (-49,35)(-51,34)(-53.5,34.5)(-55.5,36)(-56.5,38)(-56.5,40.5)(-55,41.5)
%  (-53.5,41)(-51.5,41)(-50.5,43)(-50.5,44.5)(-51,47)(-51.5,47.2)(-56.5,47)
%  (-58.5,46.5)(-60,44.7)(-62,42.3)(-63,39.5)(-63.5,36.3)(-63.5,33)(-63.1,29.5)
%  (-61.5,26)(-58,23.6)(-54,22.2)(-50.7,22)(-47.5,22)(-44.5,22.3)(-41,23.5)
%  (-36.8,25.8)(-33,28)(-28.5,31)(-23.4,35)(-20.2,38.3)(-17,42.5)(-13.5,47.5)
%  (-11.2,51.9)(-9.7,58)(-7.2,55)(-5.5,53)(-1.5,57)(-1,54.5)(-0.8,52)
%  (-1,49)(-1.6,45.5)(-2.6,42)(-4,38.3)(-6,34.3)(-7.5,31.5)(-9,29)
%  (-10.2,26)(-10.8,23.7)(-11.2,20)(-11,15.5)(-10.3,12.8)(-9.3,10.5)(-6.5,8.5)
%  (-3.5,7.8)(0,7.9)(3,7.9)(7,8.2)(10,9)(13.7,9.5)(18,10.4)
%  (23.2,11.3)(27.4,12)(30.5,12.3)(33,12.5)(36,12)(41.2,10.5)(45,8.3)
%  (47.2,6.8)(49,4.5)(52.5,0)(50,4.5)(49.2,8.5)(48.2,11.8)(46.8,15)
%  (45,17.8)(43.5,18.5)(41,19)(39,19.5)(37.2,19.3)(35,19)(33.5,19.3)
%  (32,20)(30.3,20.5)(29,22.3)(28,23.5)(28,23.5)(24.5,22.3)(21.5,22)
%  (18.3,22)(15,22.2)(11,23.6)(7.5,26)(5.9,29.5)(5.5,33)(5.5,36.3)
%  (6,39.5)(7,42.3)(9,44.7)(10.5,46.5)(12.5,47)(17.5,47.2)(18,47)
%  (18.5,44.5)(18.5,43)(17.5,41)(15.5,41)(14,41.5)(12.5,40.5)(12.5,38)
%  (13.5,36)(15.5,34.5)(18,34)(20,35)(25,37)(30,39.5)(33,42)
%  (36.5,45)(40,48.5)(43.5,53)(47,58)(49.5,62.5)(52.5,68)
% ^^A Eye
% \pscircle*[linecolor=white](58.2,98.3){2\psxunit}
% \pscircle*(58.2,97.3){\psxunit}
% ^^A Mouth
% \psline(71.5,88)(70,89.3)(68.5,90.3)(67,91.9)
% ^^A Tear
% \psline(42,121)(45,118)(47,115.3)(48.5,112.7)(50,110)(51.8,106.5)
%       (52.5,103.7)(53,100.5)
% \pspolygon(41.2,115.8)(43.2,114.7)(45,112.5)(47,109.8)(48,107)(49.5,104.2)%
%       (50.5,101.6)(51,98.5)(47.7,100.6)(46,102.2)(44.8,104)(43.5,106)
%       (42.5,108)(41.7,110.5)(41,113.2)}
%
% \newcommand{\Kangaroo}[1]{%
%   \begin{pspicture}(\DimX,\DimY)
%   \psset{unit=0.035278}
%   \KangarooItself{#1}
%   \end{pspicture}}
%
% \newcommand{\KangarooPstChart}[1]{{%
%   \psset{xunit=0.006784,yunit=0.00735,linewidth=0.01}
%   \begin{pspicture}(-65.5,0)(82,126)
%     \KangarooItself{#1}
%   \end{pspicture}}}
%
%
% ^^A For the possible index and changes log
% \setlength{\columnseprule}{0.6pt}
%
% ^^A Beginning of the documentation itself
%\title{\texttt{pst-fill}\\A PSTricks package for filling and tiling areas}
%\author{Timothy Van Zandt\thanks{\protect\url{tvz@econ.insead.fr}. (documentation by
% Denis Girou (\protect\url{Denis.Girou@idris.fr}) and Herbert Vo\ss (\protect\url{hvoss@tug.org}).}}
%
%\date{\shortstack{\today --- Version 1.00\\
%                  {\small Documentation revised \today}}}
% \maketitle
% \tableofcontents
%
%\begin{abstract}
%  \FillPackage{} is a PSTricks \cite{vanZandt93},\cite{Girou94},\cite{vanZandtGirou94}, 
%\cite{Hoenig97},\cite{LGC97} package to draw easily
%  various kinds of filling and tiling of areas. It is also a good example of
%  the great power and flexibility of PSTricks, as in fact it is a very short
%  program (it body is around 200~lines long) but nevertheless really powerful.
%
%  \hspace{5mm} It was written in 1994 by Timothy \textsc{van Zandt} but
%  publicly available only in PSTricks 97 and without any documentation.
%  We describe here the version \emph{97 patch 2} of December 12, 1997, which
%  is the original one modified by myself to manage \emph{tilings} in the
%  so-called \emph{automatic} mode. This article would like to serve both of
%  reference manual and of user's guide.
%
%This package is available on \CTAN{} in the
%  \texttt{graphics/pstricks} directory (files \texttt{latex/pst-fill.sty} and
%  \texttt{generic/pst-fill.tex}).
%\end{abstract}
%
%\section{Introduction}
%
%  Here we will refer as \emph{filling} as the operation which consist to fill
%a defined area by a pattern (or a composition of patterns). We will refer as
%\emph{tiling} as the operation which consist to do the same thing, but with
%the control of the starting point, which is here the upper left corner.
%The pattern is positioned relatively to this point. This make an essential
%difference between the two modes, as without control of the starting point we
%can't draw \emph{tilings} (sometimes  called \emph{tesselations}) as used in
%many fields of Art and Science%
%\footnote{For an extensive presentation of tilings, in their history and usage
%in many fields, see the reference book \cite{GS87}.
%
%  In the \TeX{} world, few work was done on tilings. You can look at the
%\emph{tile} extension of the \XYpic{} package \cite{XYpic}, at the articles of
%Kees \textsc{van der Laan} \cite[paragraph 7]{LAAN96} (the tiling was in
%fact directly done in PostScript) and \cite{LAAN97}, at the \MP{} program
%(available on \CTANref{Roegel}) by Denis \textsc{Roegel} for the
%\textsc{Truchet} contest in 1995 \cite{EsperetGirou98} and at the \MP{}
%package \cite{Bolek98} to draw patterns, which have a strong connection with
%tilings.}.
%
%  Nevertheless, as tilings are a wide and difficult field in mathematics, this
%package is limited to simple ones, mainly \emph{monohedral} tilings with one
%prototile (which can be composite, see section \ref{sec:KindTiles}). With some
%experience and wiliness we can do more and obtained easily rather
%sophisticated results, but obviously hyperbolic tilings like the famous
%\textsc{Escher} ones or aperiodic tilings like the \textsc{Penrose} ones are
%not in the capabilities of this package. For more complex needs, we must used
%low level and more painfull technics, with the basic \cs{multido}
%and \cs{multirput} macros.
%
%\section{Package history and description of it two different modes}
%
%  As already said, this package was written in 1994 by Timothy \textsc{van
%Zandt}. Two modes were defined, called respectively \emph{manual} and
%\emph{automatic}. For both, the pattern is generated on contiguous positions in
%a rather large area which include the region to fill, later cut to the
%required dimensions by clipping mechanism. In the first mode, the pattern is
%explicitely inserted in the PostScript file each time. In the second one, the
%result is the same but with an unique explicit insertion of the pattern and a
%repetition done by PostScript. Nevertheless, in this method, the control of
%the starting point was loosed, so it allowed only to \emph{fill} a region and
%not to \emph{tile} it.
%
%  See the difference between the two modes, \emph{tiling}:
% {\psset{unit=0.5cm}%
% \psboxfill{\begin{pspicture}(1,1)\psframe[dimen=middle](1,1)\end{pspicture}}
% \begin{pspicture}(3,3.3)
%   \psframe[fillstyle=boxfill](3,3)
% \end{pspicture}}
% and \emph{filling}:
%{%
% \makeatletter
%\pst@def{BoxFill}<
%  gsave
%    gsave \tx@STV CM grestore dtransform CM idtransform
%    abs /h ED abs /w ED
%    pathbbox
%    h div round 2 add cvi /y2 ED
%    w div round 2 add cvi /x2 ED
%    h div round 2 sub cvi /y1 ED
%    w div round 2 sub cvi /x1 ED
%    /y2 y2 y1 sub def
%    /x2 x2 x1 sub def
%    CP
%    y1 h mul sub neg /y1 ED
%    x1 w mul sub neg /x1 ED
%    clip
%    y2 {
%      /x x1 def
%      x2 {
%        save CP x y1 T moveto Box restore
%        /x x w add def
%      } repeat
%      /y1 y1 h add def
%    } repeat
% currentpoint currentfont grestore setfont moveto>
% \makeatother
%
% \psset{unit=0.5}
% \psboxfill{\begin{pspicture}(1,1)\psframe[dimen=middle](1,1)\end{pspicture}}
% \begin{pspicture}(3,3.3)
%   \psframe[fillstyle=boxfill](3,3)
% \end{pspicture}
% or
% \begin{pspicture}(3,3.3)
%   \psframe[fillstyle=boxfill](3,3)
% \end{pspicture}
%}
%as we can see that initial position is arbitrary and dependent of
%the current point.
%
%
% It's clear that usage of filling is very restrictive comparing to tiling,
%as desired effects required very often the possibility to control the starting 
%point. So, this mode was of limited interest, but unfortunately the
%\emph{manual} one has the very big disadvantage to require very huge amounts
%of ressources, mainly in disk space and consequently in printing time.
%A small tiling can require sometimes several megabytes in \emph{manual} mode!
%So, it was very often not really usable in practice.
%
%It is why I modified the code, to allow tilings in \emph{automatic} mode,
%controlling in this mode too the starting point. And most of the time, that is
%to say if some special options are not used, the tiling is done exactly in the
%region described, which make it faster. So there is no more reason to use the
%\emph{manual} mode, apart very special cases where \emph{automatic} one cannot
%work, as explained later -- currently, I know only one case.
%
%  To load this modified \emph{automatic} mode, with \LaTeX{} use
%simply:\newline 
%\verb+\usepackage[tiling]{pst-fill}+\newline
%and in plain \TeX{} after:\newline
%\verb+% \iffalse meta-comment, etc.
%%
%% Package `pst-fill.dtx'
%%
%% Denis Girou (CNRS/IDRIS - France) <Denis.Girou@idris.fr>
%% Herbert Voss <voss@pstricks.de>
%%
%% This program can be redistributed and/or modified under the terms
%% of the LaTeX Project Public License Distributed from CTAN archives
%% in directory macros/latex/base/lppl.txt.
%%
%% DESCRIPTION:
%%   `pst-fill' is a PSTricks package for filling and tiling areas 
%%
% \fi
% \changes{v1.01}{2007/03/10}{bugfix for incomplete ifx (hv)}
% \changes{v1.00}{2006/11/06}{use pst-xkey for extend keys (hv)}
% \changes{v0.99}{2004/08/17}{merge the VTeX and TeX versions (patch 4) (hv)}
% \changes{v0.98}{2004/06/22}{delete the Pst@Debug option and use the
%   the one from pstricks to prevent a clash with pst-gr3d (hv)}
% \changes{v0.97}{2001/10/09}{make it work with VTeX (mv)}
% \changes{v0.94}{1997/04/08}{With a \PstTiling macro defined (or "tiling" optional parameter
%   on \textbackslash usepackage[tiling]{pst-fill}), this file run exactly as
%   the original boxfill.tex file from Timothy, version 0.94,
%   except a correction in \textbackslash pst@ManualFillCycle to avoid a division by 0.
%   It's the default.}
% \changes{v0.93}{1997/04/07}{With a \textbackslash PstTiling macro defined (or "tiling" optional parameter
%   on \textbackslash usepackage[tiling]{pst-fill}) there are several add-ons
%   and changes to do `tiling' rather than `filling' in "automatic" mode :
%     - we fix the position of the beginning of tiling,
%     - we allow normally the framing of the area as expected, using
%       the line.... parameters
%     - we define move parameters fillmovex, fillmovey and fillmove,
%     - we define fillcyclex as previous fillcycle parameter, and add the
%       fillcycley and fillcycle (both fillcyclex and fillcycley) ones
%     - we can extend the tiling area using fillloopaddx, fillloopaddy and
%       fillloopadd parameters,
%     - we can debug and see the whole tiling area without clipping using
%       PstDebug parameter,
%     - for names consistancy, we can use fillangle in place of boxfillangle
%       and fillsize in place of boxfillsize,
%     - default value for fillsep is 0 and for fillsize is auto.}
%
% \DoNotIndex{\!,\",\#,\$,\%,\&,\',\(,\+,\*,\,,\-,\.,\/,\:,\;,\<,\=,\>,\?}
% \DoNotIndex{\@,\@B,\@K,\@cTq,\@f,\@fPl,\@ifnextchar,\@nameuse,\@oVk}
% \DoNotIndex{\[,\\,\],\^,\_,\ }
% \DoNotIndex{\^,\\^,\\\^,$\^$,$\\^$,$\\^$}
% \DoNotIndex{\0,\2,\4,\5,\6,\7,\8,}
% \DoNotIndex{\A,\a}
% \DoNotIndex{\B,\b,\Bc,\begin,\Bq,\Bqc}
% \DoNotIndex{\C,\c,\catcode,\cJA,\CodelineIndex,\csname}
% \DoNotIndex{\D,\def,\define@key,\Df,\divide,\DocInput,\documentclass,\pst@addfams}
% \DoNotIndex{\eCN,\edef,\else,\eHd,\eMcj,\EnableCrossrefs,\end,\endcsname}
% \DoNotIndex{\endCenterExample,\endExample,\endinput,\endpsclip}
% \DoNotIndex{\PrintIndex,\PrintChanges,\ProvidesFile}
% \DoNotIndex{\endpspicture,\endSideBySideExample,\Example}
% \DoNotIndex{\F,\f,\FdUrr,\fi,\filedate,\fileversion,\FV@Environment}
% \DoNotIndex{\FV@UseKeyValues,\FV@XRightMargin,\FVB@Example,\fvset}
% \DoNotIndex{\G,\g,\GetFileInfo,\gr,\GradientLoaded,\gsFKrbK@o,\gsj,\gsOX}
% \DoNotIndex{\hbadness,\hfuzz,\HLEmphasize,\HLMacro,\HLMacro@i}
% \DoNotIndex{\HLReverse,\HLReverse@i,\hqcu,\HqY}
% \DoNotIndex{\I,\i,\ifx,\input,\Ir,\IU}
% \DoNotIndex{\j,\jl,\JT,\JVodH}
% \DoNotIndex{\K,\k,\kfSlL}
% \DoNotIndex{\L,\let}
% \DoNotIndex{\message,\mHNa,\mIU}
% \DoNotIndex{\N,\nB,\newcmykcolor,\newdimen,\newif,\nW}
% \DoNotIndex{\O,\oCDJDo,\ocQhVI,\OnlyDescription,\oRKJ}
% \DoNotIndex{\P,\p,\ProvidesPackage,\psframe,\pslinewidth,\psset}
% \DoNotIndex{\PstAtCode,\PSTricksLoaded}
% \DoNotIndex{\q,\Qr,\qssRXq,\qu,\qXjFQp,\qYL}
% \DoNotIndex{\R,\r,\RecordChanges,\relax,\RlaYI,\rN,\Rp,\rp,\RPDXNn,\rput}
% \DoNotIndex{\S,\scalebox,\SgY,\SideBySide@Example,\SideBySideExample}
% \DoNotIndex{\SgY,\sk,\Sp,\space,\sZb}
% \DoNotIndex{\T,\the,\tw@}
% \DoNotIndex{\u,\UiSWGEf@,\uJi,\usepackage,\uVQdMM,\UYj}
% \DoNotIndex{\VerbatimEnvironment,\VerbatimInput,\VrC@}
% \DoNotIndex{\WhZ,\WjKCYb,\WNs}
% \DoNotIndex{\XkN,\XW}
% \DoNotIndex{\Z,\ZCM,\Ze}
% \DoNotIndex{\addtocounter,\advance,\alph,\arabic,\AtBeginDocument,\AtEndDocument}
% \DoNotIndex{\AtEndOfPackage,\begingroup,\bfseries,\bgroup,\box,\csname}
% \DoNotIndex{\else,\endcsname,\endgroup,\endinput,\expandafter,\fi}
% \DoNotIndex{\TeX,\z@,\p@,\@one,\xdef,\thr@@,\string,\sixt@@n,\reset,\or,\multiply,\repeat,\RequirePackage}
% \DoNotIndex{\@cclvi,\@ne,\@ehpa,\@nil,\copy,\dp,\global,\hbox,\hss,\ht,\ifodd,\ifdim,\ifcase,\kern}
% \DoNotIndex{\chardef,\loop,\leavevmode,\ifnum,\lower}
% \setcounter{IndexColumns}{2}
%
% ^^A To extend the height used for the text
%
% ^^A  Aligned labels in a description environment
%\newenvironment{Description}[1]{%
%\begin{list}{nothing}{\setlength{\leftmargin}{#1}
%\setlength{\labelwidth}{\leftmargin}\setlength{\labelsep}{1mm}}}
%{\end{list}}
%
% ^^A For macro names
%\DeclareRobustCommand\cs[1]{\texttt{\char`\\#1}}
%
%
% ^^A From ltugboat.cls
% ^^A For references
%\makeatletter
%\newcommand\acro[1]{\textsc{#1}\@}
%\def\CTAN{\acro{CTAN}}
%\let\texttub\textsl              % ^^A redefined in other situations
%\def\TUB{\texttub{TUGboat}}
%\def\TUG{\TeX\ \UG}
%\def\tug{\acro{TUG}}
%\def\UG{Users Group}
% ^^A For the bibliography 
%\let\@internalcite\cite
%\def\cite{\def\@citeseppen{-1000}%
%    \def\@cite##1##2{(##1\if@tempswa , ##2\fi)}%
%    \def\citeauthoryear##1##2##3{##1, ##3}\@internalcite}
%\def\etal{et\,al.\@}
%\newcommand\CTANdirectory[1]{\expandafter\urldef
%  \csname CTAN@#1\endcsname\path}
%\newcommand\CTANfile[1]{\expandafter\urldef
%  \csname CTAN@#1\endcsname\path}
%\newcommand\CTANref[1]{\expandafter\@setref\csname CTAN@#1\endcsname
%  \relax{#1}}
%\makeatother
% ^^A Define CTAN addresses 
%\CTANdirectory{mpattern}{graphics/metapost/macros/mpattern}
%\CTANdirectory{pstricks}{graphics/pstricks}
%\CTANdirectory{pst-fill.sty}{graphics/pstricks/latex/pst-fill.sty}
%\CTANdirectory{pst-fill}{graphics/pstricks/generic/pst-fill.tex}
%\CTANdirectory{Roegel}{graphics/metapost/contrib/macros/truchet}
%\CTANdirectory{xypic}{macros/generic/diagrams/xypic}
%
% ^^A Personal macros (D.G.)
% ^^A ----------------------
%
% ^^A Some colors used
%\definecolor{LemonChiffon}{rgb}{1.,0.98,0.8}
%\definecolor{LightBlue}   {rgb}{0.8,0.85,0.95}
%\definecolor{PaleGreen}   {rgb}{0.88,1,0.88}
%\definecolor{PeachPuff}   {rgb}{1.0,0.85,0.73}
%
% ^^A To define a unique string for TeX and LaTeX
%\newcommand{\AllTeX}{%
%{\rm(L\kern-.36em\raise.3ex\hbox{\sc a}\kern-.15em)%
%T\kern-.1667em\lower.7ex\hbox{E}\kern-.125emX}}
%
% ^^A Bibliography style
%\bibliographystyle{ltugbib}
%
% ^^A Name macros
%\newcommand{\FillPackage}{\textsf{`pst-fill'}}
%\newcommand{\XYpic}{%
%\leavevmode\hbox{\kern-.1em X\kern-.3em\lower.4ex\hbox{Y\kern-.15em}-pic}}
%
%\makeatletter
%
% ^^A Example environments
% ^^A (do not use in them the four JXYZ characters, that we will use
% ^^A as escape characters!)
%
% ^^A Default PSTricks parameters
%  \psset{dimen=middle}
%
% ^^A Translation in PSTricks from the one drawn by Emmanuel Chailloux and
% ^^A Guy Cousineau for the MLgraph system
% ^^A (see /ftp.ens.fr:/pub/unix/lang/MLgraph/version-2.1/MLgraph-refman.ps.gz)
% ^^A The kangaroo itself is reproduce from an original picture from Raoul Raba
% \newcommand{\DimX}{2.47}
% \newcommand{\DimY}{4.8}
% \newcommand{\DimXDivTwo}{1.235}
%
% \newcommand{\KangarooItself}[1]{%
% ^^A Body
% \pspolygon[fillstyle=solid,fillcolor=#1]%
%  (52.5,68)(55,72.5)(55.8,76.5)(56.8,79.8)(58.2,83)(60,85.8)(61.5,86.5)
% (64,87)(66,87.5)(67.8,87.3)(70,87)(71.5,87.3)(73,88)(74.7,88.5)
% (76,90.3)(77,91.5)(72.8,93.8)(69,96)(64.5,99)(59.4,103)(56.2,106.3)
% (53,110.5)(49.5,115.5)(47.2,119.9)(45.7,126)(43.2,123)(41.5,121)(37.5,125)
% (37,122.5)(36.8,120)(37,117)(37.6,113.5)(38.6,110)(40,106.3)(42,102.3)
%  (43.5,99.5)(45,97)(46.2,94)(46.8,91.7)(47.2,88)(47,83.5)(46.3,80.8)
%  (45.3,78.5)(42.5,76.5)(39.5,75.8)(36,75.9)(33,75.9)(29,76.2)(26,77)
%  (22.3,77.5)(18,78.4)(12.8,79.3)(8.6,80)(5.5,80.3)(3,80.5)(0,80)
%  (-5.2,78.5)(-9,76.3)(-11.2,74.8)(-13,72.5)(-16.5,68)(-16.5,68)(-19.5,62.5)
%  (-22,58)(-25.5,53)(-29,48.5)(-32.5,45)(-36,42)(-39,39.5)(-44,37)
%  (-49,35)(-51,34)(-53.5,34.5)(-55.5,36)(-56.5,38)(-56.5,40.5)(-55,41.5)
%  (-53.5,41)(-51.5,41)(-50.5,43)(-50.5,44.5)(-51,47)(-51.5,47.2)(-56.5,47)
%  (-58.5,46.5)(-60,44.7)(-62,42.3)(-63,39.5)(-63.5,36.3)(-63.5,33)(-63.1,29.5)
%  (-61.5,26)(-58,23.6)(-54,22.2)(-50.7,22)(-47.5,22)(-44.5,22.3)(-41,23.5)
%  (-36.8,25.8)(-33,28)(-28.5,31)(-23.4,35)(-20.2,38.3)(-17,42.5)(-13.5,47.5)
%  (-11.2,51.9)(-9.7,58)(-7.2,55)(-5.5,53)(-1.5,57)(-1,54.5)(-0.8,52)
%  (-1,49)(-1.6,45.5)(-2.6,42)(-4,38.3)(-6,34.3)(-7.5,31.5)(-9,29)
%  (-10.2,26)(-10.8,23.7)(-11.2,20)(-11,15.5)(-10.3,12.8)(-9.3,10.5)(-6.5,8.5)
%  (-3.5,7.8)(0,7.9)(3,7.9)(7,8.2)(10,9)(13.7,9.5)(18,10.4)
%  (23.2,11.3)(27.4,12)(30.5,12.3)(33,12.5)(36,12)(41.2,10.5)(45,8.3)
%  (47.2,6.8)(49,4.5)(52.5,0)(50,4.5)(49.2,8.5)(48.2,11.8)(46.8,15)
%  (45,17.8)(43.5,18.5)(41,19)(39,19.5)(37.2,19.3)(35,19)(33.5,19.3)
%  (32,20)(30.3,20.5)(29,22.3)(28,23.5)(28,23.5)(24.5,22.3)(21.5,22)
%  (18.3,22)(15,22.2)(11,23.6)(7.5,26)(5.9,29.5)(5.5,33)(5.5,36.3)
%  (6,39.5)(7,42.3)(9,44.7)(10.5,46.5)(12.5,47)(17.5,47.2)(18,47)
%  (18.5,44.5)(18.5,43)(17.5,41)(15.5,41)(14,41.5)(12.5,40.5)(12.5,38)
%  (13.5,36)(15.5,34.5)(18,34)(20,35)(25,37)(30,39.5)(33,42)
%  (36.5,45)(40,48.5)(43.5,53)(47,58)(49.5,62.5)(52.5,68)
% ^^A Eye
% \pscircle*[linecolor=white](58.2,98.3){2\psxunit}
% \pscircle*(58.2,97.3){\psxunit}
% ^^A Mouth
% \psline(71.5,88)(70,89.3)(68.5,90.3)(67,91.9)
% ^^A Tear
% \psline(42,121)(45,118)(47,115.3)(48.5,112.7)(50,110)(51.8,106.5)
%       (52.5,103.7)(53,100.5)
% \pspolygon(41.2,115.8)(43.2,114.7)(45,112.5)(47,109.8)(48,107)(49.5,104.2)%
%       (50.5,101.6)(51,98.5)(47.7,100.6)(46,102.2)(44.8,104)(43.5,106)
%       (42.5,108)(41.7,110.5)(41,113.2)}
%
% \newcommand{\Kangaroo}[1]{%
%   \begin{pspicture}(\DimX,\DimY)
%   \psset{unit=0.035278}
%   \KangarooItself{#1}
%   \end{pspicture}}
%
% \newcommand{\KangarooPstChart}[1]{{%
%   \psset{xunit=0.006784,yunit=0.00735,linewidth=0.01}
%   \begin{pspicture}(-65.5,0)(82,126)
%     \KangarooItself{#1}
%   \end{pspicture}}}
%
%
% ^^A For the possible index and changes log
% \setlength{\columnseprule}{0.6pt}
%
% ^^A Beginning of the documentation itself
%\title{\texttt{pst-fill}\\A PSTricks package for filling and tiling areas}
%\author{Timothy Van Zandt\thanks{\protect\url{tvz@econ.insead.fr}. (documentation by
% Denis Girou (\protect\url{Denis.Girou@idris.fr}) and Herbert Vo\ss (\protect\url{hvoss@tug.org}).}}
%
%\date{\shortstack{\today --- Version 1.00\\
%                  {\small Documentation revised \today}}}
% \maketitle
% \tableofcontents
%
%\begin{abstract}
%  \FillPackage{} is a PSTricks \cite{vanZandt93},\cite{Girou94},\cite{vanZandtGirou94}, 
%\cite{Hoenig97},\cite{LGC97} package to draw easily
%  various kinds of filling and tiling of areas. It is also a good example of
%  the great power and flexibility of PSTricks, as in fact it is a very short
%  program (it body is around 200~lines long) but nevertheless really powerful.
%
%  \hspace{5mm} It was written in 1994 by Timothy \textsc{van Zandt} but
%  publicly available only in PSTricks 97 and without any documentation.
%  We describe here the version \emph{97 patch 2} of December 12, 1997, which
%  is the original one modified by myself to manage \emph{tilings} in the
%  so-called \emph{automatic} mode. This article would like to serve both of
%  reference manual and of user's guide.
%
%This package is available on \CTAN{} in the
%  \texttt{graphics/pstricks} directory (files \texttt{latex/pst-fill.sty} and
%  \texttt{generic/pst-fill.tex}).
%\end{abstract}
%
%\section{Introduction}
%
%  Here we will refer as \emph{filling} as the operation which consist to fill
%a defined area by a pattern (or a composition of patterns). We will refer as
%\emph{tiling} as the operation which consist to do the same thing, but with
%the control of the starting point, which is here the upper left corner.
%The pattern is positioned relatively to this point. This make an essential
%difference between the two modes, as without control of the starting point we
%can't draw \emph{tilings} (sometimes  called \emph{tesselations}) as used in
%many fields of Art and Science%
%\footnote{For an extensive presentation of tilings, in their history and usage
%in many fields, see the reference book \cite{GS87}.
%
%  In the \TeX{} world, few work was done on tilings. You can look at the
%\emph{tile} extension of the \XYpic{} package \cite{XYpic}, at the articles of
%Kees \textsc{van der Laan} \cite[paragraph 7]{LAAN96} (the tiling was in
%fact directly done in PostScript) and \cite{LAAN97}, at the \MP{} program
%(available on \CTANref{Roegel}) by Denis \textsc{Roegel} for the
%\textsc{Truchet} contest in 1995 \cite{EsperetGirou98} and at the \MP{}
%package \cite{Bolek98} to draw patterns, which have a strong connection with
%tilings.}.
%
%  Nevertheless, as tilings are a wide and difficult field in mathematics, this
%package is limited to simple ones, mainly \emph{monohedral} tilings with one
%prototile (which can be composite, see section \ref{sec:KindTiles}). With some
%experience and wiliness we can do more and obtained easily rather
%sophisticated results, but obviously hyperbolic tilings like the famous
%\textsc{Escher} ones or aperiodic tilings like the \textsc{Penrose} ones are
%not in the capabilities of this package. For more complex needs, we must used
%low level and more painfull technics, with the basic \cs{multido}
%and \cs{multirput} macros.
%
%\section{Package history and description of it two different modes}
%
%  As already said, this package was written in 1994 by Timothy \textsc{van
%Zandt}. Two modes were defined, called respectively \emph{manual} and
%\emph{automatic}. For both, the pattern is generated on contiguous positions in
%a rather large area which include the region to fill, later cut to the
%required dimensions by clipping mechanism. In the first mode, the pattern is
%explicitely inserted in the PostScript file each time. In the second one, the
%result is the same but with an unique explicit insertion of the pattern and a
%repetition done by PostScript. Nevertheless, in this method, the control of
%the starting point was loosed, so it allowed only to \emph{fill} a region and
%not to \emph{tile} it.
%
%  See the difference between the two modes, \emph{tiling}:
% {\psset{unit=0.5cm}%
% \psboxfill{\begin{pspicture}(1,1)\psframe[dimen=middle](1,1)\end{pspicture}}
% \begin{pspicture}(3,3.3)
%   \psframe[fillstyle=boxfill](3,3)
% \end{pspicture}}
% and \emph{filling}:
%{%
% \makeatletter
%\pst@def{BoxFill}<
%  gsave
%    gsave \tx@STV CM grestore dtransform CM idtransform
%    abs /h ED abs /w ED
%    pathbbox
%    h div round 2 add cvi /y2 ED
%    w div round 2 add cvi /x2 ED
%    h div round 2 sub cvi /y1 ED
%    w div round 2 sub cvi /x1 ED
%    /y2 y2 y1 sub def
%    /x2 x2 x1 sub def
%    CP
%    y1 h mul sub neg /y1 ED
%    x1 w mul sub neg /x1 ED
%    clip
%    y2 {
%      /x x1 def
%      x2 {
%        save CP x y1 T moveto Box restore
%        /x x w add def
%      } repeat
%      /y1 y1 h add def
%    } repeat
% currentpoint currentfont grestore setfont moveto>
% \makeatother
%
% \psset{unit=0.5}
% \psboxfill{\begin{pspicture}(1,1)\psframe[dimen=middle](1,1)\end{pspicture}}
% \begin{pspicture}(3,3.3)
%   \psframe[fillstyle=boxfill](3,3)
% \end{pspicture}
% or
% \begin{pspicture}(3,3.3)
%   \psframe[fillstyle=boxfill](3,3)
% \end{pspicture}
%}
%as we can see that initial position is arbitrary and dependent of
%the current point.
%
%
% It's clear that usage of filling is very restrictive comparing to tiling,
%as desired effects required very often the possibility to control the starting 
%point. So, this mode was of limited interest, but unfortunately the
%\emph{manual} one has the very big disadvantage to require very huge amounts
%of ressources, mainly in disk space and consequently in printing time.
%A small tiling can require sometimes several megabytes in \emph{manual} mode!
%So, it was very often not really usable in practice.
%
%It is why I modified the code, to allow tilings in \emph{automatic} mode,
%controlling in this mode too the starting point. And most of the time, that is
%to say if some special options are not used, the tiling is done exactly in the
%region described, which make it faster. So there is no more reason to use the
%\emph{manual} mode, apart very special cases where \emph{automatic} one cannot
%work, as explained later -- currently, I know only one case.
%
%  To load this modified \emph{automatic} mode, with \LaTeX{} use
%simply:\newline 
%\verb+\usepackage[tiling]{pst-fill}+\newline
%and in plain \TeX{} after:\newline
%\verb+% \iffalse meta-comment, etc.
%%
%% Package `pst-fill.dtx'
%%
%% Denis Girou (CNRS/IDRIS - France) <Denis.Girou@idris.fr>
%% Herbert Voss <voss@pstricks.de>
%%
%% This program can be redistributed and/or modified under the terms
%% of the LaTeX Project Public License Distributed from CTAN archives
%% in directory macros/latex/base/lppl.txt.
%%
%% DESCRIPTION:
%%   `pst-fill' is a PSTricks package for filling and tiling areas 
%%
% \fi
% \changes{v1.01}{2007/03/10}{bugfix for incomplete ifx (hv)}
% \changes{v1.00}{2006/11/06}{use pst-xkey for extend keys (hv)}
% \changes{v0.99}{2004/08/17}{merge the VTeX and TeX versions (patch 4) (hv)}
% \changes{v0.98}{2004/06/22}{delete the Pst@Debug option and use the
%   the one from pstricks to prevent a clash with pst-gr3d (hv)}
% \changes{v0.97}{2001/10/09}{make it work with VTeX (mv)}
% \changes{v0.94}{1997/04/08}{With a \PstTiling macro defined (or "tiling" optional parameter
%   on \textbackslash usepackage[tiling]{pst-fill}), this file run exactly as
%   the original boxfill.tex file from Timothy, version 0.94,
%   except a correction in \textbackslash pst@ManualFillCycle to avoid a division by 0.
%   It's the default.}
% \changes{v0.93}{1997/04/07}{With a \textbackslash PstTiling macro defined (or "tiling" optional parameter
%   on \textbackslash usepackage[tiling]{pst-fill}) there are several add-ons
%   and changes to do `tiling' rather than `filling' in "automatic" mode :
%     - we fix the position of the beginning of tiling,
%     - we allow normally the framing of the area as expected, using
%       the line.... parameters
%     - we define move parameters fillmovex, fillmovey and fillmove,
%     - we define fillcyclex as previous fillcycle parameter, and add the
%       fillcycley and fillcycle (both fillcyclex and fillcycley) ones
%     - we can extend the tiling area using fillloopaddx, fillloopaddy and
%       fillloopadd parameters,
%     - we can debug and see the whole tiling area without clipping using
%       PstDebug parameter,
%     - for names consistancy, we can use fillangle in place of boxfillangle
%       and fillsize in place of boxfillsize,
%     - default value for fillsep is 0 and for fillsize is auto.}
%
% \DoNotIndex{\!,\",\#,\$,\%,\&,\',\(,\+,\*,\,,\-,\.,\/,\:,\;,\<,\=,\>,\?}
% \DoNotIndex{\@,\@B,\@K,\@cTq,\@f,\@fPl,\@ifnextchar,\@nameuse,\@oVk}
% \DoNotIndex{\[,\\,\],\^,\_,\ }
% \DoNotIndex{\^,\\^,\\\^,$\^$,$\\^$,$\\^$}
% \DoNotIndex{\0,\2,\4,\5,\6,\7,\8,}
% \DoNotIndex{\A,\a}
% \DoNotIndex{\B,\b,\Bc,\begin,\Bq,\Bqc}
% \DoNotIndex{\C,\c,\catcode,\cJA,\CodelineIndex,\csname}
% \DoNotIndex{\D,\def,\define@key,\Df,\divide,\DocInput,\documentclass,\pst@addfams}
% \DoNotIndex{\eCN,\edef,\else,\eHd,\eMcj,\EnableCrossrefs,\end,\endcsname}
% \DoNotIndex{\endCenterExample,\endExample,\endinput,\endpsclip}
% \DoNotIndex{\PrintIndex,\PrintChanges,\ProvidesFile}
% \DoNotIndex{\endpspicture,\endSideBySideExample,\Example}
% \DoNotIndex{\F,\f,\FdUrr,\fi,\filedate,\fileversion,\FV@Environment}
% \DoNotIndex{\FV@UseKeyValues,\FV@XRightMargin,\FVB@Example,\fvset}
% \DoNotIndex{\G,\g,\GetFileInfo,\gr,\GradientLoaded,\gsFKrbK@o,\gsj,\gsOX}
% \DoNotIndex{\hbadness,\hfuzz,\HLEmphasize,\HLMacro,\HLMacro@i}
% \DoNotIndex{\HLReverse,\HLReverse@i,\hqcu,\HqY}
% \DoNotIndex{\I,\i,\ifx,\input,\Ir,\IU}
% \DoNotIndex{\j,\jl,\JT,\JVodH}
% \DoNotIndex{\K,\k,\kfSlL}
% \DoNotIndex{\L,\let}
% \DoNotIndex{\message,\mHNa,\mIU}
% \DoNotIndex{\N,\nB,\newcmykcolor,\newdimen,\newif,\nW}
% \DoNotIndex{\O,\oCDJDo,\ocQhVI,\OnlyDescription,\oRKJ}
% \DoNotIndex{\P,\p,\ProvidesPackage,\psframe,\pslinewidth,\psset}
% \DoNotIndex{\PstAtCode,\PSTricksLoaded}
% \DoNotIndex{\q,\Qr,\qssRXq,\qu,\qXjFQp,\qYL}
% \DoNotIndex{\R,\r,\RecordChanges,\relax,\RlaYI,\rN,\Rp,\rp,\RPDXNn,\rput}
% \DoNotIndex{\S,\scalebox,\SgY,\SideBySide@Example,\SideBySideExample}
% \DoNotIndex{\SgY,\sk,\Sp,\space,\sZb}
% \DoNotIndex{\T,\the,\tw@}
% \DoNotIndex{\u,\UiSWGEf@,\uJi,\usepackage,\uVQdMM,\UYj}
% \DoNotIndex{\VerbatimEnvironment,\VerbatimInput,\VrC@}
% \DoNotIndex{\WhZ,\WjKCYb,\WNs}
% \DoNotIndex{\XkN,\XW}
% \DoNotIndex{\Z,\ZCM,\Ze}
% \DoNotIndex{\addtocounter,\advance,\alph,\arabic,\AtBeginDocument,\AtEndDocument}
% \DoNotIndex{\AtEndOfPackage,\begingroup,\bfseries,\bgroup,\box,\csname}
% \DoNotIndex{\else,\endcsname,\endgroup,\endinput,\expandafter,\fi}
% \DoNotIndex{\TeX,\z@,\p@,\@one,\xdef,\thr@@,\string,\sixt@@n,\reset,\or,\multiply,\repeat,\RequirePackage}
% \DoNotIndex{\@cclvi,\@ne,\@ehpa,\@nil,\copy,\dp,\global,\hbox,\hss,\ht,\ifodd,\ifdim,\ifcase,\kern}
% \DoNotIndex{\chardef,\loop,\leavevmode,\ifnum,\lower}
% \setcounter{IndexColumns}{2}
%
% ^^A To extend the height used for the text
%
% ^^A  Aligned labels in a description environment
%\newenvironment{Description}[1]{%
%\begin{list}{nothing}{\setlength{\leftmargin}{#1}
%\setlength{\labelwidth}{\leftmargin}\setlength{\labelsep}{1mm}}}
%{\end{list}}
%
% ^^A For macro names
%\DeclareRobustCommand\cs[1]{\texttt{\char`\\#1}}
%
%
% ^^A From ltugboat.cls
% ^^A For references
%\makeatletter
%\newcommand\acro[1]{\textsc{#1}\@}
%\def\CTAN{\acro{CTAN}}
%\let\texttub\textsl              % ^^A redefined in other situations
%\def\TUB{\texttub{TUGboat}}
%\def\TUG{\TeX\ \UG}
%\def\tug{\acro{TUG}}
%\def\UG{Users Group}
% ^^A For the bibliography 
%\let\@internalcite\cite
%\def\cite{\def\@citeseppen{-1000}%
%    \def\@cite##1##2{(##1\if@tempswa , ##2\fi)}%
%    \def\citeauthoryear##1##2##3{##1, ##3}\@internalcite}
%\def\etal{et\,al.\@}
%\newcommand\CTANdirectory[1]{\expandafter\urldef
%  \csname CTAN@#1\endcsname\path}
%\newcommand\CTANfile[1]{\expandafter\urldef
%  \csname CTAN@#1\endcsname\path}
%\newcommand\CTANref[1]{\expandafter\@setref\csname CTAN@#1\endcsname
%  \relax{#1}}
%\makeatother
% ^^A Define CTAN addresses 
%\CTANdirectory{mpattern}{graphics/metapost/macros/mpattern}
%\CTANdirectory{pstricks}{graphics/pstricks}
%\CTANdirectory{pst-fill.sty}{graphics/pstricks/latex/pst-fill.sty}
%\CTANdirectory{pst-fill}{graphics/pstricks/generic/pst-fill.tex}
%\CTANdirectory{Roegel}{graphics/metapost/contrib/macros/truchet}
%\CTANdirectory{xypic}{macros/generic/diagrams/xypic}
%
% ^^A Personal macros (D.G.)
% ^^A ----------------------
%
% ^^A Some colors used
%\definecolor{LemonChiffon}{rgb}{1.,0.98,0.8}
%\definecolor{LightBlue}   {rgb}{0.8,0.85,0.95}
%\definecolor{PaleGreen}   {rgb}{0.88,1,0.88}
%\definecolor{PeachPuff}   {rgb}{1.0,0.85,0.73}
%
% ^^A To define a unique string for TeX and LaTeX
%\newcommand{\AllTeX}{%
%{\rm(L\kern-.36em\raise.3ex\hbox{\sc a}\kern-.15em)%
%T\kern-.1667em\lower.7ex\hbox{E}\kern-.125emX}}
%
% ^^A Bibliography style
%\bibliographystyle{ltugbib}
%
% ^^A Name macros
%\newcommand{\FillPackage}{\textsf{`pst-fill'}}
%\newcommand{\XYpic}{%
%\leavevmode\hbox{\kern-.1em X\kern-.3em\lower.4ex\hbox{Y\kern-.15em}-pic}}
%
%\makeatletter
%
% ^^A Example environments
% ^^A (do not use in them the four JXYZ characters, that we will use
% ^^A as escape characters!)
%
% ^^A Default PSTricks parameters
%  \psset{dimen=middle}
%
% ^^A Translation in PSTricks from the one drawn by Emmanuel Chailloux and
% ^^A Guy Cousineau for the MLgraph system
% ^^A (see /ftp.ens.fr:/pub/unix/lang/MLgraph/version-2.1/MLgraph-refman.ps.gz)
% ^^A The kangaroo itself is reproduce from an original picture from Raoul Raba
% \newcommand{\DimX}{2.47}
% \newcommand{\DimY}{4.8}
% \newcommand{\DimXDivTwo}{1.235}
%
% \newcommand{\KangarooItself}[1]{%
% ^^A Body
% \pspolygon[fillstyle=solid,fillcolor=#1]%
%  (52.5,68)(55,72.5)(55.8,76.5)(56.8,79.8)(58.2,83)(60,85.8)(61.5,86.5)
% (64,87)(66,87.5)(67.8,87.3)(70,87)(71.5,87.3)(73,88)(74.7,88.5)
% (76,90.3)(77,91.5)(72.8,93.8)(69,96)(64.5,99)(59.4,103)(56.2,106.3)
% (53,110.5)(49.5,115.5)(47.2,119.9)(45.7,126)(43.2,123)(41.5,121)(37.5,125)
% (37,122.5)(36.8,120)(37,117)(37.6,113.5)(38.6,110)(40,106.3)(42,102.3)
%  (43.5,99.5)(45,97)(46.2,94)(46.8,91.7)(47.2,88)(47,83.5)(46.3,80.8)
%  (45.3,78.5)(42.5,76.5)(39.5,75.8)(36,75.9)(33,75.9)(29,76.2)(26,77)
%  (22.3,77.5)(18,78.4)(12.8,79.3)(8.6,80)(5.5,80.3)(3,80.5)(0,80)
%  (-5.2,78.5)(-9,76.3)(-11.2,74.8)(-13,72.5)(-16.5,68)(-16.5,68)(-19.5,62.5)
%  (-22,58)(-25.5,53)(-29,48.5)(-32.5,45)(-36,42)(-39,39.5)(-44,37)
%  (-49,35)(-51,34)(-53.5,34.5)(-55.5,36)(-56.5,38)(-56.5,40.5)(-55,41.5)
%  (-53.5,41)(-51.5,41)(-50.5,43)(-50.5,44.5)(-51,47)(-51.5,47.2)(-56.5,47)
%  (-58.5,46.5)(-60,44.7)(-62,42.3)(-63,39.5)(-63.5,36.3)(-63.5,33)(-63.1,29.5)
%  (-61.5,26)(-58,23.6)(-54,22.2)(-50.7,22)(-47.5,22)(-44.5,22.3)(-41,23.5)
%  (-36.8,25.8)(-33,28)(-28.5,31)(-23.4,35)(-20.2,38.3)(-17,42.5)(-13.5,47.5)
%  (-11.2,51.9)(-9.7,58)(-7.2,55)(-5.5,53)(-1.5,57)(-1,54.5)(-0.8,52)
%  (-1,49)(-1.6,45.5)(-2.6,42)(-4,38.3)(-6,34.3)(-7.5,31.5)(-9,29)
%  (-10.2,26)(-10.8,23.7)(-11.2,20)(-11,15.5)(-10.3,12.8)(-9.3,10.5)(-6.5,8.5)
%  (-3.5,7.8)(0,7.9)(3,7.9)(7,8.2)(10,9)(13.7,9.5)(18,10.4)
%  (23.2,11.3)(27.4,12)(30.5,12.3)(33,12.5)(36,12)(41.2,10.5)(45,8.3)
%  (47.2,6.8)(49,4.5)(52.5,0)(50,4.5)(49.2,8.5)(48.2,11.8)(46.8,15)
%  (45,17.8)(43.5,18.5)(41,19)(39,19.5)(37.2,19.3)(35,19)(33.5,19.3)
%  (32,20)(30.3,20.5)(29,22.3)(28,23.5)(28,23.5)(24.5,22.3)(21.5,22)
%  (18.3,22)(15,22.2)(11,23.6)(7.5,26)(5.9,29.5)(5.5,33)(5.5,36.3)
%  (6,39.5)(7,42.3)(9,44.7)(10.5,46.5)(12.5,47)(17.5,47.2)(18,47)
%  (18.5,44.5)(18.5,43)(17.5,41)(15.5,41)(14,41.5)(12.5,40.5)(12.5,38)
%  (13.5,36)(15.5,34.5)(18,34)(20,35)(25,37)(30,39.5)(33,42)
%  (36.5,45)(40,48.5)(43.5,53)(47,58)(49.5,62.5)(52.5,68)
% ^^A Eye
% \pscircle*[linecolor=white](58.2,98.3){2\psxunit}
% \pscircle*(58.2,97.3){\psxunit}
% ^^A Mouth
% \psline(71.5,88)(70,89.3)(68.5,90.3)(67,91.9)
% ^^A Tear
% \psline(42,121)(45,118)(47,115.3)(48.5,112.7)(50,110)(51.8,106.5)
%       (52.5,103.7)(53,100.5)
% \pspolygon(41.2,115.8)(43.2,114.7)(45,112.5)(47,109.8)(48,107)(49.5,104.2)%
%       (50.5,101.6)(51,98.5)(47.7,100.6)(46,102.2)(44.8,104)(43.5,106)
%       (42.5,108)(41.7,110.5)(41,113.2)}
%
% \newcommand{\Kangaroo}[1]{%
%   \begin{pspicture}(\DimX,\DimY)
%   \psset{unit=0.035278}
%   \KangarooItself{#1}
%   \end{pspicture}}
%
% \newcommand{\KangarooPstChart}[1]{{%
%   \psset{xunit=0.006784,yunit=0.00735,linewidth=0.01}
%   \begin{pspicture}(-65.5,0)(82,126)
%     \KangarooItself{#1}
%   \end{pspicture}}}
%
%
% ^^A For the possible index and changes log
% \setlength{\columnseprule}{0.6pt}
%
% ^^A Beginning of the documentation itself
%\title{\texttt{pst-fill}\\A PSTricks package for filling and tiling areas}
%\author{Timothy Van Zandt\thanks{\protect\url{tvz@econ.insead.fr}. (documentation by
% Denis Girou (\protect\url{Denis.Girou@idris.fr}) and Herbert Vo\ss (\protect\url{hvoss@tug.org}).}}
%
%\date{\shortstack{\today --- Version 1.00\\
%                  {\small Documentation revised \today}}}
% \maketitle
% \tableofcontents
%
%\begin{abstract}
%  \FillPackage{} is a PSTricks \cite{vanZandt93},\cite{Girou94},\cite{vanZandtGirou94}, 
%\cite{Hoenig97},\cite{LGC97} package to draw easily
%  various kinds of filling and tiling of areas. It is also a good example of
%  the great power and flexibility of PSTricks, as in fact it is a very short
%  program (it body is around 200~lines long) but nevertheless really powerful.
%
%  \hspace{5mm} It was written in 1994 by Timothy \textsc{van Zandt} but
%  publicly available only in PSTricks 97 and without any documentation.
%  We describe here the version \emph{97 patch 2} of December 12, 1997, which
%  is the original one modified by myself to manage \emph{tilings} in the
%  so-called \emph{automatic} mode. This article would like to serve both of
%  reference manual and of user's guide.
%
%This package is available on \CTAN{} in the
%  \texttt{graphics/pstricks} directory (files \texttt{latex/pst-fill.sty} and
%  \texttt{generic/pst-fill.tex}).
%\end{abstract}
%
%\section{Introduction}
%
%  Here we will refer as \emph{filling} as the operation which consist to fill
%a defined area by a pattern (or a composition of patterns). We will refer as
%\emph{tiling} as the operation which consist to do the same thing, but with
%the control of the starting point, which is here the upper left corner.
%The pattern is positioned relatively to this point. This make an essential
%difference between the two modes, as without control of the starting point we
%can't draw \emph{tilings} (sometimes  called \emph{tesselations}) as used in
%many fields of Art and Science%
%\footnote{For an extensive presentation of tilings, in their history and usage
%in many fields, see the reference book \cite{GS87}.
%
%  In the \TeX{} world, few work was done on tilings. You can look at the
%\emph{tile} extension of the \XYpic{} package \cite{XYpic}, at the articles of
%Kees \textsc{van der Laan} \cite[paragraph 7]{LAAN96} (the tiling was in
%fact directly done in PostScript) and \cite{LAAN97}, at the \MP{} program
%(available on \CTANref{Roegel}) by Denis \textsc{Roegel} for the
%\textsc{Truchet} contest in 1995 \cite{EsperetGirou98} and at the \MP{}
%package \cite{Bolek98} to draw patterns, which have a strong connection with
%tilings.}.
%
%  Nevertheless, as tilings are a wide and difficult field in mathematics, this
%package is limited to simple ones, mainly \emph{monohedral} tilings with one
%prototile (which can be composite, see section \ref{sec:KindTiles}). With some
%experience and wiliness we can do more and obtained easily rather
%sophisticated results, but obviously hyperbolic tilings like the famous
%\textsc{Escher} ones or aperiodic tilings like the \textsc{Penrose} ones are
%not in the capabilities of this package. For more complex needs, we must used
%low level and more painfull technics, with the basic \cs{multido}
%and \cs{multirput} macros.
%
%\section{Package history and description of it two different modes}
%
%  As already said, this package was written in 1994 by Timothy \textsc{van
%Zandt}. Two modes were defined, called respectively \emph{manual} and
%\emph{automatic}. For both, the pattern is generated on contiguous positions in
%a rather large area which include the region to fill, later cut to the
%required dimensions by clipping mechanism. In the first mode, the pattern is
%explicitely inserted in the PostScript file each time. In the second one, the
%result is the same but with an unique explicit insertion of the pattern and a
%repetition done by PostScript. Nevertheless, in this method, the control of
%the starting point was loosed, so it allowed only to \emph{fill} a region and
%not to \emph{tile} it.
%
%  See the difference between the two modes, \emph{tiling}:
% {\psset{unit=0.5cm}%
% \psboxfill{\begin{pspicture}(1,1)\psframe[dimen=middle](1,1)\end{pspicture}}
% \begin{pspicture}(3,3.3)
%   \psframe[fillstyle=boxfill](3,3)
% \end{pspicture}}
% and \emph{filling}:
%{%
% \makeatletter
%\pst@def{BoxFill}<
%  gsave
%    gsave \tx@STV CM grestore dtransform CM idtransform
%    abs /h ED abs /w ED
%    pathbbox
%    h div round 2 add cvi /y2 ED
%    w div round 2 add cvi /x2 ED
%    h div round 2 sub cvi /y1 ED
%    w div round 2 sub cvi /x1 ED
%    /y2 y2 y1 sub def
%    /x2 x2 x1 sub def
%    CP
%    y1 h mul sub neg /y1 ED
%    x1 w mul sub neg /x1 ED
%    clip
%    y2 {
%      /x x1 def
%      x2 {
%        save CP x y1 T moveto Box restore
%        /x x w add def
%      } repeat
%      /y1 y1 h add def
%    } repeat
% currentpoint currentfont grestore setfont moveto>
% \makeatother
%
% \psset{unit=0.5}
% \psboxfill{\begin{pspicture}(1,1)\psframe[dimen=middle](1,1)\end{pspicture}}
% \begin{pspicture}(3,3.3)
%   \psframe[fillstyle=boxfill](3,3)
% \end{pspicture}
% or
% \begin{pspicture}(3,3.3)
%   \psframe[fillstyle=boxfill](3,3)
% \end{pspicture}
%}
%as we can see that initial position is arbitrary and dependent of
%the current point.
%
%
% It's clear that usage of filling is very restrictive comparing to tiling,
%as desired effects required very often the possibility to control the starting 
%point. So, this mode was of limited interest, but unfortunately the
%\emph{manual} one has the very big disadvantage to require very huge amounts
%of ressources, mainly in disk space and consequently in printing time.
%A small tiling can require sometimes several megabytes in \emph{manual} mode!
%So, it was very often not really usable in practice.
%
%It is why I modified the code, to allow tilings in \emph{automatic} mode,
%controlling in this mode too the starting point. And most of the time, that is
%to say if some special options are not used, the tiling is done exactly in the
%region described, which make it faster. So there is no more reason to use the
%\emph{manual} mode, apart very special cases where \emph{automatic} one cannot
%work, as explained later -- currently, I know only one case.
%
%  To load this modified \emph{automatic} mode, with \LaTeX{} use
%simply:\newline 
%\verb+\usepackage[tiling]{pst-fill}+\newline
%and in plain \TeX{} after:\newline
%\verb+\input{pst-fill}+\newline
%add the following definition:\newline
%\verb+\def\PstTiling{true}+
%
%  To obtain the original behaviour, just don't use the \emph{tiling} optional
%keyword at loading.
%
%  Take care than in \emph{tiling} mode, I introduce also some other changes.
%First I define aliases on some parameter names for consistancy (all specific
%parameters will begin by the \texttt{fill} prefix in this case) and I change
%some default values, which were not well adapted for tilings (\texttt{fillsep}
%is set to 0 and as explained \texttt{fillsize} set to \texttt{auto}). I rename 
%\texttt{fillcycle} to \texttt{fillcyclex}. I also restore normal way so that
%the frame of the area is drawn and all line (\texttt{linestyle},
%\texttt{linecolor}, \texttt{doubleline}, etc.) parameters are now active (but
%there are not in non \emph{tiling} mode). And I also introduce new parameters
%to control the tilings (see below).
%
%  \textbf{In all the following examples, we will consider only the
% \emph{tiling} mode.}
%
%  To do a tiling, we have just to define the pattern with the
% \verb+\psboxfill+ macro and to use the new \texttt{fillstyle}
% \verb+boxfill+.
%
%  Note that tilings are drawn from left to right and top to bottom, which can
%have an importance in some circonstances.
%
%  PostScript programmers can be also interested to know that, even in the
%\emph{automatic} mode, the iterations of the pattern are managed directly by
%the PostScript code of the package which used only PostScript Level 1
%operators. The special ones introduced in Level 2 for drawing of patterns
%\cite[section 4.9]{PostScript95} are not used.
%
%  And first, for conveniance, we define a simple \cs{Tiling} macro, which
%will simplify our examples:
%
%\begin{verbatim}
%  \newcommand{\Tiling}[2][]{%
%    \edef\Temp{#1}%
%    \begin{pspicture}#2
%      \ifx\Temp\empty
%        \psframe[fillstyle=boxfill]#2
%      \else
%        \psframe[fillstyle=boxfill,#1]#2
%      \fi
%    \end{pspicture}}
%\end{verbatim}
%
%
%\newcommand{\Tiling}[2][]{%
%  \edef\Temp{#1}%
%  \begin{pspicture}#2
%    \ifx\Temp\empty
%      \psframe[fillstyle=boxfill]#2
%    \else
%      \psframe[fillstyle=boxfill,#1]#2
%    \fi
% \end{pspicture}}
%
%\subsection{Parameters}
%
%  There are \textbf{14} specific parameters available to change the way the
% filling/tiling is defined, and one debugging option.
%
% \begin{Description}{2cm}
%  \item [fillangle (real)\hfill :] the value of the rotation
%  applied to the patterns (\emph{Default:~0}).
% \end{Description}
%
%
%   In this case, we must force the tiling area to be notably larger than the
% area to cover, to be sure that the defined area will be covered after rotation.
% \lstset{gobble=2}
% \begin{LTXexample}
% \newcommand{\Square}{%
%   \begin{pspicture}(1,1)
%     \psframe[dimen=middle](1,1)
%   \end{pspicture}}
% \psset{unit=0.5}
% \psboxfill{\Square}
% \Tiling[fillangle=45]{(3,3)}\quad
% \Tiling[fillangle=-60]{(3,3)}
% \end{LTXexample}
% 
% \newcommand{\Square}{\begin{pspicture}(1,1)\psframe[dimen=middle](1,1)\end{pspicture}}
% 
% \begin{Description}{2cm}
%   \setcounter{footnote}{1}
%   \item[\texttt{fillsepx} (real$\|$dim) :] value of the horizontal
%   separation between consecutive patterns (\emph{Default:~0 for
%   tilings\footnotemark, 2pt otherwise}).  \footnotetext{This option was added
%   by me, is not part of the original package and is available only if the
%   \texttt{tiling} keyword is used when loading the package.}
%   \setcounter{footnote}{1}
%   \item [\texttt{fillsepy} (real$\|$dim)\hfill :] value of the vertical
%   separation between consecutive patterns (\emph{Default:~0 for
%   ti\-lings\footnotemark, 2pt otherwise}).
%   \setcounter{footnote}{1}
%   \item [\texttt{fillsep} (real$\|$dim)\hfill :] value of horizontal and
%   vertical separations between consecutive patterns (\emph{Default:~0 for
%   tilings\footnotemark, 2pt otherwise}).
% \end{Description}
% 
%   These values can be negative, which allow the tiles to overlap.
% 
% \begin{LTXexample}
% \psset{unit=0.5}
% \psboxfill{\Square}
% \Tiling[fillsepx=2mm]{(3,3)} 
% \Tiling[fillsepy=1mm]{(3,3)}\\
% \Tiling[fillsep=0.5]{(3,3)} 
% \Tiling[fillsep=-0.5]{(3,3)}
% \end{LTXexample}
% 
% \begin{Description}{2cm}
%   \item [\texttt{fillcyclex}\footnotemark\ (integer)\hfill :] Shift
%   coefficient applied to each row (\emph{Default:~0}).
%   \footnotetext{It was \texttt{fillcycle} in the original version.}
%   \setcounter{footnote}{1}
%   \item [\texttt{fillcycley}\footnotemark\ (integer)\hfill :] Same thing for
%   columns (\emph{Default:~0}).
%   \setcounter{footnote}{1}
%   \item [\texttt{fillcycle}\footnotemark\ (integer)\hfill :] Allow to fix
%   both \texttt{fillcyclex} and \texttt{fillcycley} directly to the same value
%   (\emph{Default:~0}).
% \end{Description}
% 
%   For instance, if \texttt{fillcyclex} is 2, the second row of patterns will
% be horizontally shifted by a factor of $\frac{1}{2}=0.5$, and by a factor of
% 0.333 if \texttt{fillcyclex} is 3, etc.). These values can be negative.
% 
% \begin{LTXexample}[width=0.35\linewidth]
% \psset{unit=0.5}
% \psboxfill{\Square}
% \newcommand{\TilingA}[1]{\Tiling[fillcyclex=#1]{(3,3)}}
% \TilingA{0} \TilingA{1}\\
% \TilingA{2} \TilingA{3}\\[3mm]
% \TilingA{4} \TilingA{5}\\
% \TilingA{6} \TilingA{-3}\\[3mm]
% \Tiling[fillcycley=2]{(3,3)}
% \Tiling[fillcycley=3]{(3,3)}\\
% \Tiling[fillcycley=-3]{(3,3)}
% \Tiling[fillcycle=2]{(3,3)}
% \end{LTXexample}
% 
% \begin{Description}{2cm}
%   \setcounter{footnote}{1}
%   \item [\texttt{fillmovex}\footnotemark\ (real$\|$dim)\hfill :] value of the
%   horizontal moves between consecutive patterns (\emph{Default:~0}).
%   \setcounter{footnote}{1}
%   \item [\texttt{fillmovey}\footnotemark\ (real$\|$dim)\hfill :] value of the
%   vertical moves between consecutive patterns (\emph{Default:~0}).
%   \setcounter{footnote}{1}
%   \item [\texttt{fillmove}\footnotemark\ (real$\|$dim)\hfill :] value of
%   horizontal and vertical moves between consecutive patterns
%   (\emph{Default:~0}).
% \end{Description}
% 
%   These parameters allow the patterns to overlap and to draw some special
% kinds of tilings. They are implemented only for the \emph{automatic} and
% \emph{tiling} modes and their values can be negative.
% 
%   In some cases, the effect of these parameters will be the same that with the 
% \texttt{fillcycle?} ones, but you can see that it is not true for some other
% values.
% 
% \begin{LTXexample}
% \psset{unit=0.5}
% \psboxfill{\Square}
% \Tiling[fillmovex=0.5]{(3,3)} 
% \Tiling[fillmovey=0.5]{(3,3)}\\
% \Tiling[fillmove=0.5]{(3,3)}
% \Tiling[fillmove=-0.5]{(3,3)}
% \end{LTXexample}
% 
% \begin{Description}{2cm}
%   \item [\texttt{fillsize}
%   (auto$\|$\{(real$\|$dim,real$\|$dim)(real$\|$dim,real$\|$dim)\}) :] The
%   choice of \emph{automatic} mode or the size of the area in \emph{manual}
%   mode. If first pair values are not given, (0,0) is used. (\emph{Default:
%   auto when \emph{tiling} mode is used, {(-15cm,-15cm)(15cm,15cm)}
%   otherwise}).
% \end{Description}
% 
%   As explained in the introduction, the \emph{manual} mode can require very
% huge amount of computer ressources. So, it usage is to discourage in front off
% the \emph{automatic} mode. It seems only useful in special circonstances, in
% fact when the \emph{automatic} mode failed, which is known only in one case,
% for some kinds of EPS files, as the ones produce by dump of portions of
% screens (see \ref{sec:GraphicFiles}).
% 
% \begin{Description}{2cm}
%   \setcounter{footnote}{1}
%   \item [\texttt{fillloopaddx}\footnotemark\ (integer)\hfill :] number of
%   times the pattern is added on left and right positions (\emph{Default:~0}).
%   \setcounter{footnote}{1}
%   \item [\texttt{fillloopaddy}\footnotemark\ (integer)\hfill :] number of
%   times the pattern is added on top and bottom positions (\emph{Default:~0}).
%   \setcounter{footnote}{1}
%   \item [\texttt{fillloopadd}\footnotemark\ (integer)\hfill :] number of
%   times the pattern is added on left, right, top and bottom positions
%   (\emph{Default:~0}).
% \end{Description}
% 
%   These parameters are only useful in special circonstances, as for complex
% patterns when the size of the rectangular box used to tile the area doesn't 
% correspond to the pattern itself (see an example in Figure~\ref{fig:Sheeps})
% and also sometimes when the size of the pattern is not a divisor of the size
% of the area to fill and that the number of loop repeats is not properly
% computed, which can occur.
% 
%   They are implemented only for the \emph{tiling} mode.
% 
% \begin{Description}{2cm}
%   \setcounter{footnote}{1}
%   \item [\texttt{PstDebug}\footnotemark\ (integer, 0 or 1)\hfill :] to
%   require to see the exact tiling done, without clipping (\emph{Default:~0}).
% \end{Description}
% 
%   It's mainly useful for debugging or to understand better how the tilings
% are done. It is implemented only for the \emph{tiling} mode.
% 
% \begin{LTXexample}
% \psset{unit=0.3,PstDebug=1}
% \psboxfill{\Square}
% \psset{linewidth=1mm}
% \Tiling{(2,2)}\\[5mm]
% \Tiling[fillcyclex=2]{(2,2)}\\[1cm]
% \Tiling[fillmove=0.5]{(2,2)}
% \end{LTXexample}
% 
% \vspace{3cm}
% \section{Examples}
% 
%   In fact this unique \cs{psboxfill} macro allow a lot a variations and
% different usages. We will try here to demonstrate this.
% 
% \subsection{Kind of tiles}
% \label{sec:KindTiles}
% 
%   Of course, we can access to all the power of PSTricks macros to define the
% \emph{tiles} (\emph{patterns}) used. So, we can define complicated ones.
% 
%   Here we give four other Archimedian tilings (those built with only some
% regular polygons) among the twelve existing, first discovered completely by
% Johanes \textsc{Kepler} at the beginning of 17th century \cite{GS87}, the two
% other \emph{regular} ones with the tiling by squares, formed by a unique
% regular polygon, and two other formed by two different regular polygons.
% 
% \begin{LTXexample}[pos=t]
%   \newcommand{\Triangle}{%
%     \begin{pspicture}(1,1)
%       \pstriangle[dimen=middle](0.5,0)(1,1)
%     \end{pspicture}}
%   \newcommand{\Hexagon}{
% ^^A sin(60)=0.866
%     \begin{pspicture}(0.866,0.75)
%       \SpecialCoor
% ^^A  Hexagon  
%       \pspolygon[dimen=middle]%
%         (0.5;30)(0.5;90)(0.5;150)(0.5;210)(0.5;270)(0.5;330)
%     \end{pspicture}}
% 
%   \psset{unit=0.5}
%   \psboxfill{\Triangle}
%   \Tiling{(4,4)}\hfill
% ^^A The two other regular tilings
%   \Tiling[fillcyclex=2]{(4,4)}\hfill
%   \psboxfill{\Hexagon}
%   \Tiling[fillcyclex=2,fillloopaddy=1]{(5,5)}
% \end{LTXexample}
% 
% \begin{LTXexample}[pos=t]
%   \newcommand{\ArchimedianA}{%
%      ^^A Archimedian tiling 3^2.4.3.4
%     \psset{dimen=middle}
%      ^^A sin(60)=0.866
%     \begin{pspicture}(1.866,1.866)
%       \psframe(1,1)
%       \psline(1,0)(1.866,0.5)(1,1)(0.5,1.866)(0,1)(-0.866,0.5)
%       \psline(0,0)(0.5,-0.866)
%     \end{pspicture}}
%   \newcommand{\ArchimedianB}{%
%      ^^A Archimedian tiling 4.8^2
%     \psset{dimen=middle,unit=1.5}
%      ^^A sin(22.5)=0.3827 ; cos(22.5)=0.9239
%     \begin{pspicture}(1.3066,0.6533)
%       \SpecialCoor
%      ^^A Octogon
%       \pspolygon(0.5;22.5)(0.5;67.5)(0.5;112.5)(0.5;157.5)
%                 (0.5;202.5)(0.5;247.5)(0.5;292.5)(0.5;337.5)
%     \end{pspicture}}
% 
%   \psset{unit=0.5}
%   \psboxfill{\ArchimedianA}
%   \Tiling[fillmove=0.5]{(7,7)}\hfill
%   \psboxfill{\ArchimedianB}
%   \Tiling[fillcyclex=2,fillloopaddy=1]{(7,7)}
% \end{LTXexample}
% 
%   \setcounter{footnote}{3}
%   We can of course tile an area arbitrarily defined. And with the
% \texttt{addfillstyle} parameter\footnote{Introduced in PSTricks 97.}, we can
% easily mix the \texttt{boxfill} style with another one.
% 
% \begin{LTXexample}[width=6cm]
%   \psset{unit=0.5,dimen=middle}
%   \psboxfill{%
%     \begin{pspicture}(1,1)
%       \psframe(1,1)
%       \pscircle(0.5,0.5){0.25}
%     \end{pspicture}}
%   \begin{pspicture}(4,6)
%     \pspolygon[fillstyle=boxfill,fillsep=0.25](0,1)(1,4)(4,6)(4,0)(2,1)
%   \end{pspicture}\hspace{1em}
%   \begin{pspicture}(4,4)
%%     \pscircle[linestyle=none,fillstyle=solid,fillcolor=yellow,fillsep=0.5,
%%               addfillstyle=boxfill](2,2){2}
%   \end{pspicture}
% \end{LTXexample}
%
%   Various effects can be obtained, sometimes complicated ones very easily, as
% in this example reproduced from one shown by Slavik \textsc{Jablan} in the
% field of \emph{OpTiles}, inspired by the \emph{Op-art}:
% 
% \begin{LTXexample}[pos=t]
% \newcommand{\ProtoTile}{%
%  \begin{pspicture}(1,1)%%% 1/12=0.08333
%   \psset{linestyle=none,linewidth=0,
%     hatchwidth=0.08333\psunit,hatchsep=0.08333\psunit}
%   \psframe[fillstyle=solid,fillcolor=black,addfillstyle=hlines,hatchcolor=white](1,1)
%   \pswedge[fillstyle=solid,fillcolor=white,addfillstyle=hlines]{1}{0}{90}
%  \end{pspicture}}
% \newcommand{\BasicTile}{%
%  \begin{pspicture}(2,1)
%    \rput[lb](0,0){\ProtoTile}\rput[lb](1,0){\psrotateleft{\ProtoTile}}
%  \end{pspicture}}
% \ProtoTile\hfill\BasicTile\hfill
% \psboxfill{\BasicTile}
% \Tiling[fillcyclex=2]{(4,4)}
% \end{LTXexample}
% 
%   It is also directly possible to surimpose several different tilings. Here is
% the splendid visual proof of the \textsc{Pytha\-gore} theorem done by the arab
% mathematician \textsc{Annairizi} around the year 900, given by superposition
% of two tilings by squares of different sizes.
% 
% \begin{LTXexample}[pos=t]
% \psset{unit=1.5,dimen=middle}
% \begin{pspicture*}(3,3)
%   \psboxfill{\begin{pspicture}(1,1)
%     \psframe(1,1)\end{pspicture}}
%   \psframe[fillstyle=boxfill](3,3)
%   \psboxfill{\begin{pspicture}(1,1)
%     \rput{-37}{\psframe[linecolor=red](0.8,0.8)}
%   \end{pspicture}}
%   \psframe[fillstyle=boxfill](3,4)
%   \pspolygon[fillstyle=hlines,hatchangle=90](1,2)(1.64,1.53)(2,2)
% \end{pspicture*}
% \end{LTXexample}
% 
%   In a same way, it is possible to build tilings based on figurative patterns,
% in the style of the famous \textsc{Escher} ones. Following an example of
% Andr\'e \textsc{Deledicq} \cite{Deledicq97}, we first show a simple tiling of
% the \emph{p1} category (according to the international classification of the
% 17~symmetry groups of the plane first discovered by the russian
% crystalographer Jevgraf \textsc{Fedorov} at the end of the 19th century):
% 
% \begin{LTXexample}[pos=t]
%  \newcommand{\SheepHead}[1]{%
%    \begin{pspicture}(3,1.5)
%      \pscustom[liftpen=2,fillstyle=solid,fillcolor=#1]{%
%        \pscurve(0.5,-0.2)(0.6,0.5)(0.2,1.3)(0,1.5)(0,1.5)
%          (0.4,1.3)(0.8,1.5)(2.2,1.9)(3,1.5)(3,1.5)(3.2,1.3)
%          (3.6,0.5)(3.4,-0.3)(3,0)(2.2,0.4)(0.5,-0.2)}
%      \pscircle*(2.65,1.25){0.12\psunit} % Eye
%      \psccurve*(3.5,0.3)(3.35,0.45)(3.5,0.6)(3.6,0.4)% Muzzle
%     ^^A   % Mouth
%       \pscurve(3,0.35)(3.3,0.1)(3.6,0.05)
%     ^^A   % Ear
%       \pscurve(2.3,1.3)(2.1,1.5)(2.15,1.7)\pscurve(2.1,1.7)(2.35,1.6)(2.45,1.4)
%   \end{pspicture}}
%  \psboxfill{\psset{unit=0.5}\SheepHead{yellow}\SheepHead{cyan}}
%  \Tiling[fillcyclex=2,fillloopadd=1]{(10,5)}
% \end{LTXexample}
% \label{fig:Sheeps}
% 
%   Now a tiling of the \emph{pg} category (the code for the kangaroo itself is
% too long to be shown here, but has no difficulties ; the kangaroo is reproduce
% from an original picture from Raoul \textsc{Raba} and here is a translation in
% PSTricks from the one drawn by Emmanuel \textsc{Chailloux} and Guy
% \textsc{Cousineau} for their MLgraph system \cite{MLgraphTSI}):
% 
% \begin{LTXexample}[pos=t]
% \psboxfill{\psset{unit=0.4}
%   \Kangaroo{yellow}\Kangaroo{red}\Kangaroo{cyan}\Kangaroo{green}%
%   \psscalebox{-1 1}{%
%     \rput(1.235,4.8){\Kangaroo{green}\Kangaroo{cyan}\Kangaroo{red}\Kangaroo{yellow}}}}
%   \Tiling[fillloopadd=1]{(10,6)}
% \end{LTXexample}
% 
%   And here a \textsc{Wang} tiling \cite{Wang65}, \cite[chapter
% 11]{GS87}, based on very simple tiles of the form of a square and composed
% of four colored triangles. Such tilings are built with only a matching color
% constraint. Despite of it simplicity, it is an important kind of tilings, as
% \textsc{Wang} and others used them to study the special class of
% \emph{aperiodic} tilings, and also because it was shown that surprisingly this 
% tiling is similar to a \textsc{Turing} machine.
% 
% \begin{LTXexample}[pos=t]
%   \newcommand{\WangTile}[4]{%
%     \begin{pspicture}(1,1)
%       \pspolygon*[linecolor=#1](0,0)(0,1)(0.5,0.5)
%       \pspolygon*[linecolor=#2](0,1)(1,1)(0.5,0.5)
%       \pspolygon*[linecolor=#3](1,1)(1,0)(0.5,0.5)
%       \pspolygon*[linecolor=#4](1,0)(0,0)(0.5,0.5)
%     \end{pspicture}}
%   \newcommand{\WangTileA}{\WangTile{cyan}{yellow}{cyan}{cyan}}
%   \newcommand{\WangTileB}{\WangTile{yellow}{cyan}{cyan}{red}}
%   \newcommand{\WangTileC}{\WangTile{cyan}{red}{yellow}{yellow}}
%   \newcommand{\WangTiles}[1][]{%
%     \begin{pspicture}(3,3) \psset{ref=lb}
%       \rput(0,2){\WangTileB}  \rput(1,2){\WangTileA}%
%       \rput(2,2){\WangTileC}  \rput(0,1){\WangTileC}%
%       \rput(1,1){\WangTileB}  \rput(2,1){\WangTileA}
%       \rput(0,0){\WangTileA}  \rput(1,0){\WangTileC}%
%       \rput(2,0){\WangTileB}
%       #1
%     \end{pspicture}}
%   \WangTileA\hfill\WangTileB\hfill\WangTileC\hfill
%   \WangTiles[{\psgrid[subgriddiv=0,gridlabels=0](3,3)}]\hfill
%   \psset{unit=0.4} \psboxfill{\WangTiles} \Tiling{(12,12)}
% \end{LTXexample}
% 
% \subsection{External graphic files}
% \label{sec:GraphicFiles}
% 
%   We can also fill an arbitrary area with an external image. We have only, 
% as usual, to matter of the \emph{BoundingBox} definition if there is no one
% provided or if it is not the accurate one, as for the well known
% \texttt{tiger} picture part of the \texttt{ghostscript} distribution.
% 
% \begin{LTXexample}[pos=t]
%   \psboxfill{%% Strangely require x1=x2...
%     \begin{pspicture}(0,1)(0,4.1)
%       \includegraphics[bb=17 176 560 74,width=3cm]{tiger}
%     \end{pspicture}}
%   \Tiling{(6,6.2)}
% \end{LTXexample}
% 
%   Nevertheless, there are some special files for which the \emph{automatic}
% mode doesn't work, specially for some files obtained by a screen dump, as in
% the next example, where a picture was reduced before it conversion in the
% \emph{Encapsulated PostScript} format by a screen dump utility. In this case,
% usage of the \emph{manual} mode is the only alternative, at the price of the
% real multiple inclusion of the EPS file. We must take care to specify the
% correct \texttt{fillsize} parameter, because otherwise the default values are
% large and will load the file many times, perhaps just really using few
% occurrences as the other ones would be clipped...
% 
% \begin{LTXexample}[pos=t]
%   \psboxfill{\includegraphics{flowers}}
%   \begin{pspicture}(8,4)
%     \psellipse[fillstyle=boxfill,fillsize={(8,4)}](4,2)(4,2)
%   \end{pspicture}
% \end{LTXexample}
% 
% \subsection{Tiling of characters}
% 
%   We can also use the \cs{psboxfill} macro to fill the interior of characters
% for special effects like these ones:
% 
% \begin{LTXexample}[pos=t]
%   \DeclareFixedFont{\bigsf}{T1}{phv}{b}{n}{4.5cm}
%   \DeclareFixedFont{\smallrm}{T1}{ptm}{m}{n}{3mm}
%   \psboxfill{\smallrm Since 182 days...}
%   \begin{pspicture*}(8,4)
%     \centerline{%
%       \pscharpath[fillstyle=gradient,gradangle=-45,
%                   gradmidpoint=0.5,addfillstyle=boxfill,
%                   fillangle=45,fillsep=0.7mm]
%                  {\rput[b](0,0.1){\bigsf 2000}}}
%   \end{pspicture*}
% \end{LTXexample}
% 
% \begin{LTXexample}[pos=t]
%   \DeclareFixedFont{\mediumrm}{T1}{ptm}{m}{n}{2cm}
%   \psboxfill{%
%     \psset{unit=0.1,linewidth=0.2pt}
%     \Kangaroo{PeachPuff}\Kangaroo{PaleGreen}%
%       \Kangaroo{LightBlue}\Kangaroo{LemonChiffon}%
%     \psscalebox{-1 1}{%
%       \rput(1.235,4.8){%
%         \Kangaroo{LemonChiffon}\Kangaroo{LightBlue}%
%           \Kangaroo{PaleGreen}\Kangaroo{PeachPuff}}}}
% ^^A   % A kangaroo of kangaroos...
%   \begin{pspicture}(8,2)
%     \pscharpath[linestyle=none,fillstyle=boxfill,fillloopadd=1]
%                {\rput[b](4,0){\mediumrm Kangaroo}}
%   \end{pspicture}
% \end{LTXexample}
% 
% \subsection{Other kinds of usage}
% 
%   Other kinds of usage can be imagined. For instance, we can use tilings in a
% sort of degenerated way to draw some special lines made by a unique or
% multiple repeating patterns. But it can be only a special dashed line, as here
% with three different dashes:
% 
% \begin{LTXexample}[pos=t]
%   \newcommand{\Dashes}{%
%     \psset{dimen=middle}
%     \begin{pspicture}(0,-0.5\pslinewidth)(1,0.5\pslinewidth)
%       \rput(0,0){\psline(0.4,0)}%
%         \rput(0.5,0){\psline(0.2,0)}%
%         \rput(0.8,0){\psline(0.1,0)}
%     \end{pspicture}}
% 
%   \newcommand{\SpecialDashedLine}[3]{%
%     \psboxfill{#3}
%     \Tiling[linestyle=none]
%            {(#1,-0.5\pslinewidth)(#2,0.5\pslinewidth)}}
% 
%   \SpecialDashedLine{0}{7}{\Dashes}
% 
%   \psset{unit=0.5,linewidth=1mm,linecolor=red}
%   \SpecialDashedLine{0}{10}{\Dashes}
% \end{LTXexample}
% 
%   It allow also to use special patterns in business graphics, as in the
% following example generated by \texttt{PstChart}\footnote{A personal
% development to draw business charts with PSTricks, not distributed.}.
% 
% \vspace{3mm}
% \begin{figure}[!ht]
% \centering
% \psset{unit=0.75}
% ^^A % Generated by pstchart.sh version 0.21 (11/28/97)
% {\psset{dimen=middle}
% \psset{xunit=2,yunit=0.005}
% \begin{pspicture}(-0.6,-200)(6.6,2300)
% ^^A   % Title
%   \rput(3,2200){\shortstack{Fantaisist repartition of kangaroos\\
%                             in the world (in thousands)}}
% ^^A   % Frame background
%   \psframe[fillstyle=solid,fillcolor=LemonChiffon](0,0)(6,2000)
% ^^A   % Graduations
%   \multido{\n=0+500}{5}{\rput[r](-0.12,\n){\psscalebox{0.8}{\n}}}
% ^^A   % Minor ticks
%   \multips(0,100)(0,100){19}{\psline[unit=4.8pt](1,0)}
%   \multips(6,100)(0,100){19}{\psline[unit=4.8pt](-1,0)}
% ^^A   % Major ticks
%   \multips(0,500)(0,500){3}{\psline[unit=9.6pt](1,0)}
%   \multips(6,500)(0,500){3}{\psline[unit=9.6pt](-1,0)}
% ^^A   % Lines from major ticks marks
%   \multips(0,500)(0,500){3}{\psline[linestyle=dotted,linewidth=0.6pt](6,0)}
% ^^A   % Drawing for the data
%   \psboxfill{\psset{unit=0.78\psxunit}\KangarooPstChart{red}}
%   \psframe[linestyle=none,fillstyle=boxfill,fillloopaddy=1](0.61,0)(1.39,1800)
%   \psboxfill{\psset{unit=0.78\psxunit}\KangarooPstChart{yellow}}
%   \psframe[linestyle=none,fillstyle=boxfill,fillloopaddy=1](1.61,0)(2.39,800)
%   \psboxfill{\psset{unit=0.78\psxunit}\KangarooPstChart{cyan}}
%   \psframe[linestyle=none,fillstyle=boxfill,fillloopaddy=1](2.61,0)(3.39,550)
%   \psboxfill{\psset{unit=0.78\psxunit}\KangarooPstChart{magenta}}
%   \psframe[linestyle=none,fillstyle=boxfill,fillloopaddy=1](3.61,0)(4.39,500)
%   \psboxfill{\psset{unit=0.78\psxunit}\KangarooPstChart{green}}
%   \psframe[linestyle=none,fillstyle=boxfill,fillloopaddy=1](4.61,0)(5.39,200)
% ^^A   % Bottom labels
%   \uput{0.2}[270]{0}(1,0){\psscalebox{0.7}{Oceania}}
%   \uput{0.2}[270]{0}(2,0){\psscalebox{0.7}{Africa}}
%   \uput{0.2}[270]{0}(3,0){\psscalebox{0.7}{Asia}}
%   \uput{0.2}[270]{0}(4,0){\psscalebox{0.7}{America}}
%   \uput{0.2}[270]{0}(5,0){\psscalebox{0.7}{Europe}}
% ^^A   % Frame box around the chart
%   \psframe[linestyle=solid](0,0)(6,2000)
% \end{pspicture}}
%   \caption{Bar chart generated by PstChart, with bars filled by patterns}
%   \label{fig:PstChart}
% \end{figure}
% 
% \section{``Dynamic'' tilings}
% 
%   In some cases, tilings used non \emph{static} tiles, that is to say that the 
% \emph{prototile(s)}, even if unique, can have several forms, by instance
% specified by different colors or rotations, not fixed before generation or
% varying each time.
% 
% \subsection{Lewthwaite-Pickover-Truchet tiling}
% 
%   We give here for example the so-called \emph{Truchet} tiling, which much be
% in fact better called \emph{Lewthwaite-Pick\-over-Truchet (LPT)} tiling%
% \footnote{For description of the context, history and references about
% S\'ebastien \textsc{Truchet} and this tiling, see \cite{EsperetGirou98}.}.
% 
%   The unique prototile is only a square with two opposite circle arcs.
% This tile has obviously two positions, if we rotate it from 90 degrees (see
% the two tiles on the next figure). A \emph{LPT tiling} is a tiling with
% randomly oriented LPT tiles. We can see that even if it is very simple in it
% principle, it draw sophisticated curves with strange properties.
% 
%   Nevertheless, in the straightforward way \FillPackage{} does not work,
% because the \cs{psboxfill} macro store the content of the tile used in a
% \TeX{} box, which is static. So the calling to the random function is done
% only one time, which explain that only one rotation of the tile is used for
% all the tiling. It's only the one of the two rotations which could differ from
% one drawing to the next one...
% 
% ^^A % Truchet (Lewthwaite-Pickover-Truchet) tiling
% ^^A % --------------------------------------------
% 
% \begin{LTXexample}[pos=t]
% ^^A   % LPT prototile
%   \newcommand{\ProtoTileLPT}{%
%     \psset{dimen=middle}
%     \begin{pspicture}(1,1)
%       \psframe(1,1)
%       \psarc(0,0){0.5}{0}{90}
%       \psarc(1,1){0.5}{-180}{-90}
%     \end{pspicture}}
% 
% ^^A   % LPT tile
%   \newcount\Boolean
%   \newcommand{\BasicTileLPT}{%
% ^^A     % From random.tex by Donald Arseneau
%     \setrannum{\Boolean}{0}{1}%
%     \ifnum\Boolean=0
%       \ProtoTileLPT%
%     \else
%       \psrotateleft{\ProtoTileLPT}%
%     \fi}
% 
%   \ProtoTileLPT\hfill\psrotateleft{\ProtoTileLPT}\hfill
%   \psset{unit=0.5}
%   \psboxfill{\BasicTileLPT}
%   \Tiling{(5,5)}
% \end{LTXexample}
% 
%   But, for simple cases, there is a solution to this problem using a mixture
% of PSTricks and PostScript programming. Here the PSTricks
% construction \verb+\pscustom{\code{...}}+ allow to insert PostScript code
% inside the \LaTeX{} + PSTricks one.
% 
%   Programmation is less straightforward, but it has also the advantage to be
% notably faster, as all the tilings operations are done in PostScript, and
% mainly to not be limited by \TeX{} memory (the \TeX{} + PSTricks solution
% I wrote in 1995 for the colored problem was limited to small sizes for this
% reason). Just note also that \cs{pslbrace} and \cs{psrbrace} are two
% PSTricks macros to define and be able to insert the \verb+{+ and \verb+}+
% characters.
% 
% \begin{LTXexample}[pos=t]
% ^^A   % LPT prototile
%   \newcommand{\ProtoTileLPT}{%
%     \psset{dimen=middle}
%     \psframe(1,1)
%     \psarc(0,0){0.5}{0}{90}
%     \psarc(1,1){0.5}{-180}{-90}}
% 
% ^^A   % Counter to change the random seed
%   \newcount\InitCounter
% ^^A   % LPT tile
%   \newcommand{\BasicTileLPT}{%
%     \InitCounter=\the\time
%     \pscustom{\code{%
%       rand \the\InitCounter\space sub 2 mod 0 eq \pslbrace}}
%     \begin{pspicture}(1,1)
%       \ProtoTileLPT
%     \end{pspicture}%
%     \pscustom{\code{\psrbrace \pslbrace}}
%     \psrotateleft{\ProtoTileLPT}%
%     \pscustom{\code{\psrbrace ifelse}}}
% 
%   \psset{unit=0.4,linewidth=0.4pt}
%   \psboxfill{\BasicTileLPT}
%   \Tiling{(15,15)}
% \end{LTXexample}
% 
%   Using the very surprising fact (see \cite{EsperetGirou98}) that
% coloration of these tiles do not depend of their neighbors (even if it is
% difficult to believe as the opposite seems obvious!) but only of the parity of
% the value of row and column positions, we can directly program in the same way
% a colored version of the LPT tiling.
% 
% \setcounter{footnote}{1}
%   We have also introduce in the \FillPackage{} code for \emph{tiling} mode two
% new accessible Post\-Script variables, \texttt{row} and
% \texttt{column}\footnotemark, which can be useful in some circonstances, like
% this one.
% 
% \begin{LTXexample}[pos=t]
% ^^A   % LPT prototile
%   \newcommand{\ProtoTileLPT}[2]{%
%     \psset{dimen=middle,linestyle=none,fillstyle=solid}
%     \psframe[fillcolor=#1](1,1)
%     \psset{fillcolor=#2}
%     \pswedge(0,0){0.5}{0}{90} \pswedge(1,1){0.5}{-180}{-90}}
% ^^A   % Counter to change the random seed
%   \newcount\InitCounter
% ^^A   % LPT tile
%   \newcommand{\BasicTileLPT}[2]{%
%     \InitCounter=\the\time
%     \pscustom{\code{%
%       rand \the\InitCounter\space sub 2 mod 0 eq \pslbrace
%       row column add 2 mod 0 eq \pslbrace}}
%     \begin{pspicture}(1,1)\ProtoTileLPT{#1}{#2}\end{pspicture}%
%     \pscustom{\code{\psrbrace \pslbrace}}
%     \ProtoTileLPT{#2}{#1}%
%     \pscustom{\code{%
%       \psrbrace ifelse \psrbrace \pslbrace row column add 2 mod 0 eq \pslbrace}}
%     \psrotateleft{\ProtoTileLPT{#2}{#1}}\pscustom{\code{\psrbrace \pslbrace}}
%     \psrotateleft{\ProtoTileLPT{#1}{#2}}\pscustom{\code{\psrbrace ifelse \psrbrace ifelse}}}
%   \psboxfill{\BasicTileLPT{red}{yellow}}
%   \Tiling{(4,4)}\hfill
%   \psset{unit=0.4}\psboxfill{\BasicTileLPT{blue}{cyan}}
%   \Tiling{(15,15)}
% \end{LTXexample}
% 
%   Another classic example is to generate coordinates and numerotation for a
% grid. Of course, it is possible to do it directly in PSTricks using nested
% \cs{multido} commands. It would be clearly easy to program, but, nevertheless, 
% for users who have a little knowledge of PostScript programming, this offer
% an alternative which is useful for large cases, because on this way it will
% be notably faster and less computer ressources consuming.
% 
%   Remember here that the tiling is drawn from left to right, and top to
% bottom, and note that the PostScript variable \texttt{x2} give the total
% number of columns.
% 
% \begin{LTXexample}[pos=t]
% ^^A   % \Escape will be the \ character
%   {\catcode`\!=0\catcode`\\=11!gdef!Escape{\}}
%   \newcommand{\ProtoTile}{%
%     \Square\pscustom{%
%       \moveto(-0.9,0.75) % In PSTricks units
%       \code{ /Times-Italic findfont 8 scalefont setfont
%         (\Escape() show row 3 string cvs show (,) show 
%         column 3 string cvs show (\Escape)) show}
%       \moveto(-0.5,0.25) % In PSTricks units
%       \code{ /Times-Bold findfont 18 scalefont setfont
%         1 0 0 setrgbcolor % Red color
%         /center {dup stringwidth pop 2 div neg 0 rmoveto} def
%         row 1 sub x2 mul column add 3 string cvs center show}}}
%   \psboxfill{\ProtoTile}
%   \Tiling{(6,4)}
% \end{LTXexample}
% 
% \subsection{A complete example: the Poisson equation}
% 
%   To finish, we will show a complete real example, a drawing to explain the
% method used to solve the \textsc{Poisson} equation by a domain
% decomposition method, adapted to distributed memory computers. The
% objective is to show the communications required between processes and the
% position of the data to exchange. This code also show some useful and powerful
% technics for PSTricks programming (look specially at the way some higher level
% macros are defined, and how the same object is used to draw the four
% neighbors).
%
%\psset{unit=1cm}
%\newcommand{\Pattern}[1]{%
%   \begin{pspicture}(-0.25,-0.25)(0.25,0.25)\rput{*0}{\psdot[dotstyle=#1]}
%   \end{pspicture}}
%\newcommand{\West}{\Pattern{o}}   \newcommand{\South}{\Pattern{x}}
%\newcommand{\Central}{\Pattern{+}}\newcommand{\North}{\Pattern{square}}
%\newcommand{\East}{\Pattern{triangle}}
%\newcommand{\Cross}{%
%  \pspolygon[unit=0.5,linewidth=0.2,linecolor=red](0,0)(0,1)(1,1)(1,2)(2,2)(2,1)%
%              (3,1)(3,0)(2,0)(2,-1)(1,-1)(1,0)}
%\newcommand{\StylePosition}[1]{\LARGE\textcolor{red}{\textbf{#1}}}
%\newcommand{\SubDomain}[4]{%
%    \psboxfill{#4}\begin{psclip}{\psframe[linestyle=none]#1}%
%      \psframe[linestyle=#3](5,5)\psframe[fillstyle=boxfill]#2%
%    \end{psclip}}
%\newcommand{\SendArea}[1]{\psframe[fillstyle=solid,fillcolor=cyan]#1}
%\newcommand{\ReceiveData}[2]{%
%  \psboxfill{#2}\psframe[fillstyle=solid,fillcolor=yellow,addfillstyle=boxfill]#1}%
%\newcommand{\Neighbor}[2]{%
%    \begin{pspicture}(5,5)
%      \rput{*0}(2.5,2.5){\StylePosition{#1}}
%      \ReceiveData{(0.5,0)(4.5,0.5)}{\Central}\SendArea{(0.5,0.5)(4.5,1)}%
%      \SubDomain{(5,2)}{(0.5,0.5)(4.5,3)}{dashed}{#2}%
%      \pcarc[arcangle=45,arrows=->](0.5,-1.25)(0.5,0.25)%
%      \pcarc[arcangle=45,arrows=->,linestyle=dotted,dotsep=2pt](4.5,0.75)(4.5,-0.75)%
%    \end{pspicture}}%
%  \psset{dimen=middle,dotscale=2,fillloopadd=2}
%\begin{pspicture}(-5.7,-5.7)(5.7,5.7)
%  \rput(0,0){%
%      \begin{pspicture}(5,5)
%        \ReceiveData{(0,0.5)(0.5,4.5)}{\West} \ReceiveData{(4.5,0.5)(5,4.5)}{\East}
%        \ReceiveData{(0.5,4.5)(4.5,5)}{\North}\ReceiveData{(0.5,0)(4.5,0.5)}{\South}
%        \SendArea{(0.5,0.5)(1,4.5)}\SendArea{(4,0.5)(4.5,4.5)}
%        \SendArea{(0.5,0.5)(4.5,1)}\SendArea{(0.5,4)(4.5,4.5)}
%        \SubDomain{(5,5)}{(0.5,0.5)(4.5,4.5)}{solid}{\Central}
%        \psline(1,0.5)(1,4.5)\psline(4,0.5)(4,4.5)%
%        \rput(1.5,4){\Cross}\rput(2,2){\Cross}%
%      \end{pspicture}}%
%  \rput(0,5.5){\Neighbor{N}{\North}}\rput{-90}(5.5,0){\Neighbor{E}{\East}}%
%  \rput{90}(-5.5,0){\Neighbor{W}{\West}}\rput{180}(0,-5.5){\Neighbor{S}{\South}}%
%\end{pspicture}
%
% \begin{lstlisting}
%   \newcommand{\Pattern}[1]{%
%     \begin{pspicture}(-0.25,-0.25)(0.25,0.25)\rput{*0}{\psdot[dotstyle=#1]}
%     \end{pspicture}}
%   \newcommand{\West}{\Pattern{o}}   \newcommand{\South}{\Pattern{x}}
%   \newcommand{\Central}{\Pattern{+}}\newcommand{\North}{\Pattern{square}}
%   \newcommand{\East}{\Pattern{triangle}}
%   \newcommand{\Cross}{%
%     \pspolygon[unit=0.5,linewidth=0.2,linecolor=red](0,0)(0,1)(1,1)(1,2)(2,2)(2,1)
%               (3,1)(3,0)(2,0)(2,-1)(1,-1)(1,0)}
%   \newcommand{\StylePosition}[1]{\LARGE\textcolor{red}{\textbf{#1}}}
%   \newcommand{\SubDomain}[4]{%
%     \psboxfill{#4}
%     \begin{psclip}{\psframe[linestyle=none]#1}
%       \psframe[linestyle=#3](5,5)\psframe[fillstyle=boxfill]#2
%     \end{psclip}}
%   \newcommand{\SendArea}[1]{\psframe[fillstyle=solid,fillcolor=cyan]#1}
%   \newcommand{\ReceiveData}[2]{%
%     \psboxfill{#2}
%     \psframe[fillstyle=solid,fillcolor=yellow,addfillstyle=boxfill]#1}
%   \newcommand{\Neighbor}[2]{%
%     \begin{pspicture}(5,5)
%       \rput{*0}(2.5,2.5){\StylePosition{#1}}
%       \ReceiveData{(0.5,0)(4.5,0.5)}{\Central}\SendArea{(0.5,0.5)(4.5,1)}
%       \SubDomain{(5,2)}{(0.5,0.5)(4.5,3)}{dashed}{#2}%
% ^^A       % Receive and send arrows
%       \pcarc[arcangle=45,arrows=->](0.5,-1.25)(0.5,0.25)
%       \pcarc[arcangle=45,arrows=->,linestyle=dotted,dotsep=2pt](4.5,0.75)(4.5,-0.75)
%     \end{pspicture}}
%   \psset{dimen=middle,dotscale=2,fillloopadd=2}
%   \begin{pspicture}(-5.7,-5.7)(5.7,5.7)
% ^^A     % Central domain
%     \rput(0,0){%
%       \begin{pspicture}(5,5)
% ^^A         % Receive from West, East, North and South
%         \ReceiveData{(0,0.5)(0.5,4.5)}{\West} \ReceiveData{(4.5,0.5)(5,4.5)}{\East}
%         \ReceiveData{(0.5,4.5)(4.5,5)}{\North}\ReceiveData{(0.5,0)(4.5,0.5)}{\South}
% ^^A         % send area for West, East, North and South
%         \SendArea{(0.5,0.5)(1,4.5)} \SendArea{(4,0.5)(4.5,4.5)}
%         \SendArea{(0.5,0.5)(4.5,1)} \SendArea{(0.5,4)(4.5,4.5)}
% ^^A         % Central domain
%         \SubDomain{(5,5)}{(0.5,0.5)(4.5,4.5)}{solid}{\Central}
% ^^A         % Redraw overlapped linesY
%         \psline(1,0.5)(1,4.5)  \psline(4,0.5)(4,4.5)
% ^^A         % Two crossesY
%         \rput(1.5,4){\Cross}  \rput(2,2){\Cross}
%       \end{pspicture}}
% ^^A     % The four neighborsY
%     \rput(0,5.5){\Neighbor{N}{\North}}     \rput{-90}(5.5,0){\Neighbor{E}{\East}}
%     \rput{90}(-5.5,0){\Neighbor{W}{\West}} \rput{180}(0,-5.5){\Neighbor{S}{\South}}
%   \end{pspicture}
% \end{lstlisting}
%
%
%
% Bibliography
% \begin{thebibliography}{99}
% \bibitem{PostScript95} Adobe, Systems~Incorporated, \emph{PostScript Language
% Reference Manual}, Addison-Wesley, 2~edition, 1995.
%
% \bibitem{Bolek98} Piotr Bolek, \MP{} and patterns, \emph{\TUB}, Volume~19,
% Number~3, pages 276--283, September 1998, \CTANref{mpattern}.
%
% \bibitem{MLgraphTSI} Emmanuel Chailloux, Guy Cousineau and Asc\'ander
% Su\'arez, Programmation fonctionnelle de graphismes pour la production
% d'illustrations techniques, \emph{Technique et science informatique},
% Volume~15, Number~7, pages 977--1007, 1996 (in french).
%
% \bibitem{Deledicq97} Andr\'e Deledicq, \emph{Le monde des pavages}, ACL
% \'Editions, 1997 (in french).
%
% \bibitem{EsperetGirou98} Philippe Esperet and Denis Girou,
% Coloriage du pavage dit de Truchet, Cahiers GUTenberg, Number~31,
% pages 5--18, December~1998  (in french).
%
% \bibitem{Girou94} Denis Girou, Pr\'esentation de PSTricks, \emph{Cahiers
% GUTenberg}, Number~16, pages 21--70, February~1994 (in french).
%
% \bibitem{LGC97} Michel Goossens, Sebastian Rahtz and Frank Mittelbach,
% \emph{The \LaTeX{} Graphics Companion}, Addison-Wesley, 2005.
%
% \bibitem{GS87} Branko Gr\"unbaum and Geoffrey Shephard, \emph{Tilings and
% Patterns}, Freeman and Company, 1987.
%
% \bibitem{Hoenig97} Alan Hoenig, \emph{\TeX{} Unbound: \LaTeX{} \& \TeX{}
% Strategies, Fonts, Graphics, and More}, Oxford University Press, 1997.
%
% \bibitem{XYpic} Kristoffer~H. Rose and Ross Moore, \XYpic. Pattern and Tile
% extension, available from \CTAN, 1991-1998, \CTANref{xypic}.
%
% \bibitem{LAAN96} Kees van der Laan, Paradigms: Just a little bit of PostScript,
% \emph{MAPS}, Volume~17, pages 137--150, 1996.
%
% \bibitem{LAAN97} Kees van der Laan, Tiling in PostScript and \MF{} -- Escher's
% wink, \emph{MAPS}, Volume~19, Number~2, pages 39--67, 1997.
%
% \bibitem{vanZandt93} Timothy Van Zandt, PSTricks. PostScript macros for
% Generic \TeX, available from \CTAN, 1993, \CTANref{pstricks}.
%
% \bibitem{vanZandtGirou94} Timothy Van Zandt and Denis Girou, Inside PSTricks,
% \emph{\TUB}, Volume~15, Number~3, pages 239--246, September 1994.
%
%
% \bibitem{voss07} Herbert Vo\ss, PSTricks -- Graphics for \TeX\ and \LaTeX, DANTE/Lehmanns, 4th ed., 2007.
% \bibitem{Wang65} Hao Wang, Games, Logic and Computers, \emph{Scientific
% American}, pages 98--106, November 1965.
% \end{thebibliography}
%
%
% \StopEventually{}
%
% ^^A .................... End of the documentation part ....................
%
% \section{Driver file}
%
%   The next bit of code contains the documentation driver file for \TeX{},
% i.e., the file that will produce the documentation you are currently
% reading. It will be extracted from this file by the \texttt{docstrip}
% program.
%
%    \begin{macrocode}
%<*driver>
\documentclass{ltxdoc}
\GetFileInfo{pst-fill.dtx}
%
\usepackage[T1]{fontenc}
\usepackage{lmodern}               % For PDF
\usepackage{graphicx}              % `graphicx' LaTeX standard package
\usepackage{showexpl}
\usepackage{mflogo}                % For the MetaFont and MetaPost logos
\input{random.tex}                 % Random macros from Donald Arseneau
\usepackage{url}                   % URLs convenient typesetting
\usepackage{multido}               % General loop macro
\usepackage[dvipsnames]{pstricks}  % PSTricks with the `color' extension
\usepackage{pst-text}              % PSTricks package for character path
\usepackage{pst-grad}              % PSTricks package for gradient filling
\usepackage{pst-node}              % PSTricks package for nodes
\usepackage[tiling]{pst-fill}      % PSTricks package for filling/tiling
%
\AtBeginDocument{%
%  \OnlyDescription % comment out for implementation details
  \EnableCrossrefs
  \CodelineIndex
  \RecordChanges}
\AtEndDocument{%
  \PrintIndex
  \setcounter{IndexColumns}{1}
  \PrintChanges}
\hbadness=7000            % Over and under full box warnings
\hfuzz=3pt
\begin{document}
  \DocInput{pst-fill.dtx}
\end{document}
%</driver>
%    \end{macrocode}
%
% \section{\texttt{pst-fill} \LaTeX{} wrapper}
%
%    \begin{macrocode}
%<*latex-wrapper>
\RequirePackage{pstricks}
\ProvidesPackage{pst-fill}[2005/09/13 package wrapper for 
  pst-fill.tex (hv)]
\DeclareOption{tiling}{\def\PstTiling{true}}
\ProcessOptions\relax
\input{pst-fill.tex}
\ProvidesFile{pst-fill.tex}
  [\filedate\space v\fileversion\space `PST-fill' (tvz,dg)]
%</latex-wrapper>
%    \end{macrocode}
%
%
% \section{Pst-Fill Package{} code}
%
%    \begin{macrocode}
%<*pst-fill>
%    \end{macrocode}
%
% \subsection{Preamble}
%
%   Who we are.
%
%    \begin{macrocode}
\def\fileversion{1.01}
\def\filedate{2007/03/10}
\message{`PST-Fill' v\fileversion, \filedate\space (tvz,dg,hv)}
\csname PSTboxfillLoaded\endcsname
\let\PSTboxfillLoaded\endinput
%    \end{macrocode}
%
%   Require the main PSTricks package.
%
%    \begin{macrocode}
\ifx\PSTricksLoaded\endinput\else\input pstricks.tex\fi
%    \end{macrocode}
%
%   interface to the extended `\textsf{keyval}' package.
%
%    \begin{macrocode}
\ifx\PSTXKeyLoaded\endinput\else\input pst-xkey\fi
%
%    \end{macrocode}
%
%   Catcodes changes and defining the family name for xkeyval.
%
%    \begin{macrocode}
\edef\PstAtCode{\the\catcode`\@}\catcode`\@=11\relax

\pst@addfams{pst-fill}
%
%    \end{macrocode}
%
%
% \subsection{The size of the box}
% \begin{macro}{pst@@boxfillsize}
%    \begin{macrocode}
%
\def\pst@@boxfillsize#1(#2,#3)#4(#5,#6)#7(#8\@nil{%
  \begingroup
    \ifx\@empty#7\relax
      \pst@dima\z@
      \pst@dimb\z@
      \pssetxlength\pst@dimc{#2}%
      \pssetylength\pst@dimd{#3}%
    \else
      \pssetxlength\pst@dima{#2}%
      \pssetylength\pst@dimb{#3}%
      \pssetxlength\pst@dimc{#5}%
      \pssetylength\pst@dimd{#6}%
    \fi
    \xdef\pst@tempg{%
      \pst@dima=\number\pst@dima sp
      \pst@dimb=\number\pst@dimb sp
      \pst@dimc=\number\pst@dimc sp
      \pst@dimd=\number\pst@dimd sp }%
  \endgroup
  \let\psk@boxfillsize\pst@tempg}
%    \end{macrocode}
% \end{macro}
%

% \subsection{Definition of the parameters}
%
%    \begin{macrocode}
\define@key[psset]{pst-fill}{boxfillsize}{%
  \def\pst@tempg{#1}\def\pst@temph{auto}%
  \ifx\pst@tempg\pst@temph
    \let\psk@boxfillsize\relax
  \else
    \pst@@boxfillsize#1(\z@,\z@)\@empty(\z@,\z@)(\@nil
  \fi}
\psset{boxfillsize={(-15cm,-15cm)(15cm,15cm)}}
\define@key[psset]{pst-fill}{boxfillcolor}{\pst@getcolor{#1}\psboxfillcolor}
\psset{boxfillcolor=black}% hv
\define@key[psset]{pst-fill}{boxfillangle}{\pst@getangle{#1}\psk@boxfillangle}
\psset{boxfillangle=0}
\define@key[psset]{pst-fill}{fillsepx}{%
  \pst@getlength{#1}\psk@fillsepx}
\define@key[psset]{pst-fill}{fillsepy}{%
  \pst@getlength{#1}\psk@fillsepy}
\define@key[psset]{pst-fill}{fillsep}{%
  \pst@getlength{#1}\psk@fillsepx%
  \let\psk@fillsepy\psk@fillsepx}
\psset{fillsep=2pt}

\ifx\PstTiling\@undefined
  \define@key[psset]{pst-fill}{fillcycle}{\pst@getint{#1}\psk@fillcycle}
  \psset{fillcycle=0}
\else
  \define@key[psset]{pst-fill}{fillangle}{\pst@getangle{#1}\psk@boxfillangle}
  \define@key[psset]{pst-fill}{fillsize}{%
      \def\pst@tempg{#1}\def\pst@temph{auto}%
      \ifx\pst@tempg\pst@temph\let\psk@boxfillsize\relax
      \else\pst@@boxfillsize#1(\z@,\z@)\@empty(\z@,\z@)(\@nil\fi}
  \psset{fillsep=0,fillsize=auto}
  \define@key[psset]{pst-fill}{fillcyclex}{\pst@getint{#1}\psk@fillcyclex}
  \define@key[psset]{pst-fill}{fillcycley}{\pst@getint{#1}\psk@fillcycley}
  \define@key[psset]{pst-fill}{fillcycle}{%
    \pst@getint{#1}\psk@fillcyclex\let\psk@fillcycley\psk@fillcyclex}
  \psset{fillcycle=0}
  \define@key[psset]{pst-fill}{fillmovex}{\pst@getlength{#1}\psk@fillmovex}
  \define@key[psset]{pst-fill}{fillmovey}{\pst@getlength{#1}\psk@fillmovey}
  \define@key[psset]{pst-fill}{fillmove}{%
      \pst@getlength{#1}\psk@fillmovex\let\psk@fillmovey\psk@fillmovex}
  \psset{fillmove=0pt}
  \define@key[psset]{pst-fill}{fillloopaddx}{\pst@getint{#1}\psk@fillloopaddx}
  \define@key[psset]{pst-fill}{fillloopaddy}{\pst@getint{#1}\psk@fillloopaddy}
  \define@key[psset]{pst-fill}{fillloopadd}{%
    \pst@getint{#1}\psk@fillloopaddx\let\psk@fillloopaddy\psk@fillloopaddx}
  \psset{fillloopadd=0}
%    \end{macrocode}
%
%    \begin{macrocode}
% For debugging (to debug, set PstDebug=1)
% we now use the one from pstricks to prevent a clash with package
% pstricks                        2004-06-22
%%    \define@key[psset]{pst-fill}{PstDebug}{\pst@getint{#1}\psk@PstDebug}
    \psset{PstDebug=0}
\fi
% DG addition end
%    \end{macrocode}

% \subsection{Definition of the fill box}
% \begin{macro}{psboxfill}
%    \begin{macrocode}
\newbox\pst@fillbox
\def\psboxfill{\pst@killglue\pst@makebox\psboxfill@i}
\def\psboxfill@i{\setbox\pst@fillbox\box\pst@hbox\ignorespaces}
%    \end{macrocode}
% \end{macro}
% \subsection{The main macros}
%
% \begin{macro}{psfs@boxfill}
%    \begin{macrocode}
\def\psfs@boxfill{%
  \ifvoid\pst@fillbox
    \@pstrickserr{Fill box is empty. Use \string\psboxfill\space first.}\@ehpa
  \else
    \ifx\psk@boxfillsize\relax \pst@AutoBoxFill
    \else\pst@ManualBoxFill\fi
  \fi}
%    \end{macrocode}
% \end{macro}
%
% \begin{macro}{pst@ManualBoxFill}
%    \begin{macrocode}
\def\pst@ManualBoxFill{%
  \leavevmode
  \begingroup
    \pst@FlushCode
    \begin@psclip
    \pstVerb{clip}%
    \expandafter\pst@AddFillBox\psk@boxfillsize
    \end@psclip
  \endgroup}
%    \end{macrocode}
% \end{macro}
%
% \begin{macro}{pst@FlushCode}
%    \begin{macrocode}
\def\pst@FlushCode{%
  \pst@Verb{%
    /mtrxc CM def
    CP CP T
    \tx@STV
    \psk@origin
    \psk@swapaxes
    \pst@newpath
    \pst@code
    mtrxc setmatrix
    moveto
    0 setgray}%
  \gdef\pst@code{}}
%    \end{macrocode}
% \end{macro}
%
% \begin{macro}{pst@AddFillBox}
%    \begin{macrocode}
\def\pst@AddFillBox#1 #2 #3 #4 {%
  \begingroup
    \setbox\pst@fillbox=\vbox{%
      \hbox{\unhcopy\pst@fillbox\kern\psk@fillsepx\p@}%
      \vskip\psk@fillsepy\p@}%
    \psk@boxfillsize
    \pst@cnta=\pst@dimc
    \advance\pst@cnta-\pst@dima
    \divide\pst@cnta\wd\pst@fillbox
    \pst@cntb=\pst@dimd
    \advance\pst@cntb-\pst@dimb
    \pst@dimd=\ht\pst@fillbox
    \divide\pst@cntb\pst@dimd
    \def\pst@tempa{%
      \pst@tempg
      \copy\pst@fillbox
      \advance\pst@cntc\@ne
      \ifnum\pst@cntc<\pst@cntd\expandafter\pst@tempa\fi}%
    \let\pst@tempg\relax
    \pst@cntc-\tw@
    \pst@cntd\pst@cnta
    \setbox\pst@fillbox=\hbox to \z@{%
      \kern\pst@dima
      \kern-\wd\pst@fillbox
      \pst@tempa
      \hss}%
    \pst@cntd\pst@cntb
%% DG modification begin - Dec. 11, 1997 - Patch 2
    \ifx\PstTiling\@undefined
      \ifnum\psk@fillcycle=\z@\pst@ManualFillCycle\fi
    \else
      \ifnum\psk@fillcyclex=\z@\pst@ManualFillCycle\fi
    \fi
%% DG modification end
    \global\setbox\pst@boxg=\vbox to\z@{%
      \offinterlineskip
      \vss
      \pst@tempa
      \vskip\pst@dimb}%
  \endgroup
  \setbox\pst@fillbox\box\pst@boxg
  \pst@rotate\psk@boxfillangle\pst@fillbox
  \box\pst@fillbox}
%    \end{macrocode}
% \end{macro}
%
% \begin{macro}{pst@ManualFillCycle}
%    \begin{macrocode}
\def\pst@ManualFillCycle{%
  \ifx\PstTiling\@undefined
    \pst@cntg=\psk@fillcycle
  \else
    \pst@cntg=\psk@fillcyclex
  \fi
  \pst@dimg=\wd\pst@fillbox
  \ifnum\pst@cntg=\z@
  \else
  \divide\pst@dimg\pst@cntg
  \fi
  \ifnum\pst@cntg<\z@\pst@cntg=-\pst@cntg\fi
  \advance\pst@cntg\m@ne
  \pst@cnth=\pst@cntg
  \def\pst@tempg{%
    \ifnum\pst@cnth<\pst@cntg\advance\pst@cnth\@ne\else\pst@cnth\z@\fi
    \moveright\pst@cnth\pst@dimg}}
%    \end{macrocode}
% \end{macro}
%
%% Auto box fill:        !! Fix dictionary
%
% \subsection{The PostScript subroutines}
%
%    \begin{macrocode}
%% DG addition begin - Apr. 8, 1997 and Dec. 1997 - Patch 2
\ifx\PstTiling\@undefined
\pst@def{AutoFillCycle}<%
  /c ED
  /n 0 def
  /s {
    /x x w c div n mul add def
    /n n c abs 1 sub lt { n 1 add } { 0 } ifelse def
  } def>

\pst@def{BoxFill}<%
  gsave
    gsave \tx@STV CM grestore dtransform CM idtransform
    abs /h ED abs /w ED
    pathbbox
    h div round 2 add cvi /y2 ED
    w div round 2 add cvi /x2 ED
    h div round 2 sub cvi /y1 ED
    w div round 2 sub cvi /x1 ED
    /y2 y2 y1 sub def
    /x2 x2 x1 sub def
    CP
    y1 h mul sub neg /y1 ED
    x1 w mul sub neg /x1 ED
    clip
    y2 {
      /x x1 def
      s
      x2 {
        save CP x y1
%% patch 4   hv --------------
        \ifx\VTeXversion\undefined
        \else
%%============ mv: 09-10-01 ??? this is likely to be a right change
        neg
%%============
        \fi
%% end patch 4
T moveto Box restore
        /x x w add def
      } repeat
      /y1 y1 h add def
    } repeat
    % Next line not useful... To see that, suppress clipping (DG)
    CP x y1 T moveto Box
  currentpoint currentfont grestore setfont moveto>
\else
%% DG modification begin - Apr. 8, 1997 and Nov. / Dec. 1997 - Patch 2
\pst@def{AutoFillCycleX}<%
  /cX ED
  /nX 0 def
  /CycleX {
    /x x w cX div nX mul add def
    /nX nX cX abs 1 sub lt { nX 1 add } { 0 } ifelse def
  } def>
\pst@def{AutoFillCycleY}<%
  /cY ED
  /mY 0 def
  /nY 0 def
  /CycleY {
    /y1 y1 h cY div mY mul sub def
    nY cY abs 1 sub lt { /nY nY 1 add def /mY 1 def }
                       { /nY 0 def        /mY cY abs 1 sub neg def } ifelse
  } def>

\pst@def{BoxFill}<%
  gsave
    gsave \tx@STV CM grestore dtransform CM idtransform
    abs /h ED abs /w ED
    pathbbox
    h div round 2 add cvi /y2 ED
    w div round 2 add cvi /x2 ED
    h div round 2 sub cvi /y1 ED
    w div round 2 sub cvi /x1 ED
    /CoefLoopX 0 def
    /CoefLoopY 0 def
    /CoefMoveX 0 def
    /CoefMoveY 0 def
    \psk@boxfillangle\space 0 ne {/CoefLoopX 8 def /CoefLoopY 8 def} if
    \psk@fillcyclex\space 0 ne {/CoefLoopX CoefLoopX 1 add def} if
    \psk@fillcycley\space 0 ne {/CoefLoopY CoefLoopY 1 add def} if
    \psk@fillmovex\space 0 ne
      {/CoefLoopX CoefLoopX 2 add def
       \psk@fillmovex\space 0 gt {/CoefMoveX CoefLoopX def}
                           {/CoefMoveX CoefLoopX neg def} ifelse} if
    \psk@fillmovey\space 0 ne
      {/CoefLoopY CoefLoopY 2 add def
       \psk@fillmovey\space 0 gt {/CoefMoveY CoefLoopY def}
                           {/CoefMoveY CoefLoopY neg def} ifelse} if
    \psk@fillsepx\space 0 ne {/CoefLoopX CoefLoopX 1 add def} if
    \psk@fillsepy\space 0 ne {/CoefLoopY CoefLoopY 1 add def} if
    /CoefLoopX CoefLoopX \psk@fillloopaddx\space add def
    /CoefLoopY CoefLoopY \psk@fillloopaddy\space add def
    /x2 x2 x1 sub 4 sub CoefLoopX 2 mul add def
    /y2 y2 y1 sub 4 sub CoefLoopY 2 mul add def
%% We must fix the origin of tiling, as it must not vary according other stuff
%% in the page!
    w x1 CoefLoopX add CoefMoveX add mul
      h y1 y2 add 1 sub CoefLoopY sub CoefMoveY sub mul moveto
    CP
    y1 h mul sub neg /y1 ED
    x1 w mul sub neg /x1 ED
%%  hv 2004-06-22   to prevent clash with pst-gr3d
%%    \psk@PstDebug 0 eq {clip} if
    \Pst@Debug 0 eq {clip} if
%% end hv
    \psk@fillmovex\space \psk@fillmovey
    gsave \tx@STV CM grestore dtransform CM idtransform
    /hmove ED /wmove ED
    /row 0 def
   y2 {
       /row row 1 add def
       /column 0 def
       /x x1 def
       CycleX
       save
       x2 {
          /column column 1 add def
          CycleY
          save CP x y1
%% patch 4   hv --------------
          \ifx\VTeXversion\undefined
          \else
%%============ mv: 09-10-01 ??? this is likely to be a right change
          neg
%%============
          \fi
  T moveto Box restore
          /x x w add def
          0 hmove translate
          } repeat
       restore
       /y1 y1 h add def
       wmove 0 translate
       } repeat
  currentpoint currentfont grestore setfont moveto>
\fi
%    \end{macrocode}

%    \begin{macrocode}
\def\pst@AutoBoxFill{%
  \leavevmode
  \begingroup
    \pst@stroke
    \pst@FlushCode
    \pst@Verb{\psk@boxfillangle\space \tx@RotBegin}%
    \pstVerb{\pst@dict /Box \pslbrace end}%
    \ifx\PstTiling\@undefined
    \else
      \ifx\pst@tempa\@undefined % Undefined for instance for \pscharpath
      \else\ifx\pst@tempa\@empty\else
        \def\pst@temph{0}%
        \ifx\pst@tempa\pst@temph
        \else
          \pstVerb{/TR {pop pop currentpoint translate \pst@tempa\space translate } def}%
        \fi
      \fi\fi
    \fi
    \hbox to \z@{\vbox to\z@{\vss\copy\pst@fillbox\vskip-\dp\pst@fillbox}\hss}%
    \ifx\PstTiling\@undefined
      \pstVerb{%
        tx@Dict begin \psrbrace def
        \ifnum\psk@fillcycle=\z@
          /s {} def
        \else
          \psk@fillcycle \tx@AutoFillCycle
        \fi
        \pst@number{\wd\pst@fillbox}%
        \psk@fillsepx\space add
        \pst@number{\ht\pst@fillbox}%
        \pst@number{\dp\pst@fillbox}%
        \psk@fillsepy\space add add
        \tx@BoxFill
        end}%
      \else
      \pstVerb{%
        tx@Dict begin \psrbrace def
        \ifnum\psk@fillcyclex=\z@
          /CycleX {} def
        \else
          \psk@fillcyclex\space \tx@AutoFillCycleX
        \fi
        \ifnum\psk@fillcycley=\z@
          /CycleY {} def
        \else
          \psk@fillcycley\space \tx@AutoFillCycleY
        \fi
        \pst@number{\wd\pst@fillbox}%
        \psk@fillsepx\space add
        \pst@number{\ht\pst@fillbox}%
        \pst@number{\dp\pst@fillbox}%
        \psk@fillsepy\space add add
        \tx@BoxFill
        end}%
    \fi
    \pst@Verb{\tx@RotEnd}%
  \endgroup}
%    \end{macrocode}
% \subsection{Closing}
%
%   Catcodes restoration.
%
%    \begin{macrocode}
\catcode`\@=\PstAtCode\relax
%    \end{macrocode}
%
%    \begin{macrocode}
%</pst-fill>
%    \end{macrocode}
%
% \Finale
%
\endinput
%%
%% End of file `pst-fill.dtx'
+\newline
%add the following definition:\newline
%\verb+\def\PstTiling{true}+
%
%  To obtain the original behaviour, just don't use the \emph{tiling} optional
%keyword at loading.
%
%  Take care than in \emph{tiling} mode, I introduce also some other changes.
%First I define aliases on some parameter names for consistancy (all specific
%parameters will begin by the \texttt{fill} prefix in this case) and I change
%some default values, which were not well adapted for tilings (\texttt{fillsep}
%is set to 0 and as explained \texttt{fillsize} set to \texttt{auto}). I rename 
%\texttt{fillcycle} to \texttt{fillcyclex}. I also restore normal way so that
%the frame of the area is drawn and all line (\texttt{linestyle},
%\texttt{linecolor}, \texttt{doubleline}, etc.) parameters are now active (but
%there are not in non \emph{tiling} mode). And I also introduce new parameters
%to control the tilings (see below).
%
%  \textbf{In all the following examples, we will consider only the
% \emph{tiling} mode.}
%
%  To do a tiling, we have just to define the pattern with the
% \verb+\psboxfill+ macro and to use the new \texttt{fillstyle}
% \verb+boxfill+.
%
%  Note that tilings are drawn from left to right and top to bottom, which can
%have an importance in some circonstances.
%
%  PostScript programmers can be also interested to know that, even in the
%\emph{automatic} mode, the iterations of the pattern are managed directly by
%the PostScript code of the package which used only PostScript Level 1
%operators. The special ones introduced in Level 2 for drawing of patterns
%\cite[section 4.9]{PostScript95} are not used.
%
%  And first, for conveniance, we define a simple \cs{Tiling} macro, which
%will simplify our examples:
%
%\begin{verbatim}
%  \newcommand{\Tiling}[2][]{%
%    \edef\Temp{#1}%
%    \begin{pspicture}#2
%      \ifx\Temp\empty
%        \psframe[fillstyle=boxfill]#2
%      \else
%        \psframe[fillstyle=boxfill,#1]#2
%      \fi
%    \end{pspicture}}
%\end{verbatim}
%
%
%\newcommand{\Tiling}[2][]{%
%  \edef\Temp{#1}%
%  \begin{pspicture}#2
%    \ifx\Temp\empty
%      \psframe[fillstyle=boxfill]#2
%    \else
%      \psframe[fillstyle=boxfill,#1]#2
%    \fi
% \end{pspicture}}
%
%\subsection{Parameters}
%
%  There are \textbf{14} specific parameters available to change the way the
% filling/tiling is defined, and one debugging option.
%
% \begin{Description}{2cm}
%  \item [fillangle (real)\hfill :] the value of the rotation
%  applied to the patterns (\emph{Default:~0}).
% \end{Description}
%
%
%   In this case, we must force the tiling area to be notably larger than the
% area to cover, to be sure that the defined area will be covered after rotation.
% \lstset{gobble=2}
% \begin{LTXexample}
% \newcommand{\Square}{%
%   \begin{pspicture}(1,1)
%     \psframe[dimen=middle](1,1)
%   \end{pspicture}}
% \psset{unit=0.5}
% \psboxfill{\Square}
% \Tiling[fillangle=45]{(3,3)}\quad
% \Tiling[fillangle=-60]{(3,3)}
% \end{LTXexample}
% 
% \newcommand{\Square}{\begin{pspicture}(1,1)\psframe[dimen=middle](1,1)\end{pspicture}}
% 
% \begin{Description}{2cm}
%   \setcounter{footnote}{1}
%   \item[\texttt{fillsepx} (real$\|$dim) :] value of the horizontal
%   separation between consecutive patterns (\emph{Default:~0 for
%   tilings\footnotemark, 2pt otherwise}).  \footnotetext{This option was added
%   by me, is not part of the original package and is available only if the
%   \texttt{tiling} keyword is used when loading the package.}
%   \setcounter{footnote}{1}
%   \item [\texttt{fillsepy} (real$\|$dim)\hfill :] value of the vertical
%   separation between consecutive patterns (\emph{Default:~0 for
%   ti\-lings\footnotemark, 2pt otherwise}).
%   \setcounter{footnote}{1}
%   \item [\texttt{fillsep} (real$\|$dim)\hfill :] value of horizontal and
%   vertical separations between consecutive patterns (\emph{Default:~0 for
%   tilings\footnotemark, 2pt otherwise}).
% \end{Description}
% 
%   These values can be negative, which allow the tiles to overlap.
% 
% \begin{LTXexample}
% \psset{unit=0.5}
% \psboxfill{\Square}
% \Tiling[fillsepx=2mm]{(3,3)} 
% \Tiling[fillsepy=1mm]{(3,3)}\\
% \Tiling[fillsep=0.5]{(3,3)} 
% \Tiling[fillsep=-0.5]{(3,3)}
% \end{LTXexample}
% 
% \begin{Description}{2cm}
%   \item [\texttt{fillcyclex}\footnotemark\ (integer)\hfill :] Shift
%   coefficient applied to each row (\emph{Default:~0}).
%   \footnotetext{It was \texttt{fillcycle} in the original version.}
%   \setcounter{footnote}{1}
%   \item [\texttt{fillcycley}\footnotemark\ (integer)\hfill :] Same thing for
%   columns (\emph{Default:~0}).
%   \setcounter{footnote}{1}
%   \item [\texttt{fillcycle}\footnotemark\ (integer)\hfill :] Allow to fix
%   both \texttt{fillcyclex} and \texttt{fillcycley} directly to the same value
%   (\emph{Default:~0}).
% \end{Description}
% 
%   For instance, if \texttt{fillcyclex} is 2, the second row of patterns will
% be horizontally shifted by a factor of $\frac{1}{2}=0.5$, and by a factor of
% 0.333 if \texttt{fillcyclex} is 3, etc.). These values can be negative.
% 
% \begin{LTXexample}[width=0.35\linewidth]
% \psset{unit=0.5}
% \psboxfill{\Square}
% \newcommand{\TilingA}[1]{\Tiling[fillcyclex=#1]{(3,3)}}
% \TilingA{0} \TilingA{1}\\
% \TilingA{2} \TilingA{3}\\[3mm]
% \TilingA{4} \TilingA{5}\\
% \TilingA{6} \TilingA{-3}\\[3mm]
% \Tiling[fillcycley=2]{(3,3)}
% \Tiling[fillcycley=3]{(3,3)}\\
% \Tiling[fillcycley=-3]{(3,3)}
% \Tiling[fillcycle=2]{(3,3)}
% \end{LTXexample}
% 
% \begin{Description}{2cm}
%   \setcounter{footnote}{1}
%   \item [\texttt{fillmovex}\footnotemark\ (real$\|$dim)\hfill :] value of the
%   horizontal moves between consecutive patterns (\emph{Default:~0}).
%   \setcounter{footnote}{1}
%   \item [\texttt{fillmovey}\footnotemark\ (real$\|$dim)\hfill :] value of the
%   vertical moves between consecutive patterns (\emph{Default:~0}).
%   \setcounter{footnote}{1}
%   \item [\texttt{fillmove}\footnotemark\ (real$\|$dim)\hfill :] value of
%   horizontal and vertical moves between consecutive patterns
%   (\emph{Default:~0}).
% \end{Description}
% 
%   These parameters allow the patterns to overlap and to draw some special
% kinds of tilings. They are implemented only for the \emph{automatic} and
% \emph{tiling} modes and their values can be negative.
% 
%   In some cases, the effect of these parameters will be the same that with the 
% \texttt{fillcycle?} ones, but you can see that it is not true for some other
% values.
% 
% \begin{LTXexample}
% \psset{unit=0.5}
% \psboxfill{\Square}
% \Tiling[fillmovex=0.5]{(3,3)} 
% \Tiling[fillmovey=0.5]{(3,3)}\\
% \Tiling[fillmove=0.5]{(3,3)}
% \Tiling[fillmove=-0.5]{(3,3)}
% \end{LTXexample}
% 
% \begin{Description}{2cm}
%   \item [\texttt{fillsize}
%   (auto$\|$\{(real$\|$dim,real$\|$dim)(real$\|$dim,real$\|$dim)\}) :] The
%   choice of \emph{automatic} mode or the size of the area in \emph{manual}
%   mode. If first pair values are not given, (0,0) is used. (\emph{Default:
%   auto when \emph{tiling} mode is used, {(-15cm,-15cm)(15cm,15cm)}
%   otherwise}).
% \end{Description}
% 
%   As explained in the introduction, the \emph{manual} mode can require very
% huge amount of computer ressources. So, it usage is to discourage in front off
% the \emph{automatic} mode. It seems only useful in special circonstances, in
% fact when the \emph{automatic} mode failed, which is known only in one case,
% for some kinds of EPS files, as the ones produce by dump of portions of
% screens (see \ref{sec:GraphicFiles}).
% 
% \begin{Description}{2cm}
%   \setcounter{footnote}{1}
%   \item [\texttt{fillloopaddx}\footnotemark\ (integer)\hfill :] number of
%   times the pattern is added on left and right positions (\emph{Default:~0}).
%   \setcounter{footnote}{1}
%   \item [\texttt{fillloopaddy}\footnotemark\ (integer)\hfill :] number of
%   times the pattern is added on top and bottom positions (\emph{Default:~0}).
%   \setcounter{footnote}{1}
%   \item [\texttt{fillloopadd}\footnotemark\ (integer)\hfill :] number of
%   times the pattern is added on left, right, top and bottom positions
%   (\emph{Default:~0}).
% \end{Description}
% 
%   These parameters are only useful in special circonstances, as for complex
% patterns when the size of the rectangular box used to tile the area doesn't 
% correspond to the pattern itself (see an example in Figure~\ref{fig:Sheeps})
% and also sometimes when the size of the pattern is not a divisor of the size
% of the area to fill and that the number of loop repeats is not properly
% computed, which can occur.
% 
%   They are implemented only for the \emph{tiling} mode.
% 
% \begin{Description}{2cm}
%   \setcounter{footnote}{1}
%   \item [\texttt{PstDebug}\footnotemark\ (integer, 0 or 1)\hfill :] to
%   require to see the exact tiling done, without clipping (\emph{Default:~0}).
% \end{Description}
% 
%   It's mainly useful for debugging or to understand better how the tilings
% are done. It is implemented only for the \emph{tiling} mode.
% 
% \begin{LTXexample}
% \psset{unit=0.3,PstDebug=1}
% \psboxfill{\Square}
% \psset{linewidth=1mm}
% \Tiling{(2,2)}\\[5mm]
% \Tiling[fillcyclex=2]{(2,2)}\\[1cm]
% \Tiling[fillmove=0.5]{(2,2)}
% \end{LTXexample}
% 
% \vspace{3cm}
% \section{Examples}
% 
%   In fact this unique \cs{psboxfill} macro allow a lot a variations and
% different usages. We will try here to demonstrate this.
% 
% \subsection{Kind of tiles}
% \label{sec:KindTiles}
% 
%   Of course, we can access to all the power of PSTricks macros to define the
% \emph{tiles} (\emph{patterns}) used. So, we can define complicated ones.
% 
%   Here we give four other Archimedian tilings (those built with only some
% regular polygons) among the twelve existing, first discovered completely by
% Johanes \textsc{Kepler} at the beginning of 17th century \cite{GS87}, the two
% other \emph{regular} ones with the tiling by squares, formed by a unique
% regular polygon, and two other formed by two different regular polygons.
% 
% \begin{LTXexample}[pos=t]
%   \newcommand{\Triangle}{%
%     \begin{pspicture}(1,1)
%       \pstriangle[dimen=middle](0.5,0)(1,1)
%     \end{pspicture}}
%   \newcommand{\Hexagon}{
% ^^A sin(60)=0.866
%     \begin{pspicture}(0.866,0.75)
%       \SpecialCoor
% ^^A  Hexagon  
%       \pspolygon[dimen=middle]%
%         (0.5;30)(0.5;90)(0.5;150)(0.5;210)(0.5;270)(0.5;330)
%     \end{pspicture}}
% 
%   \psset{unit=0.5}
%   \psboxfill{\Triangle}
%   \Tiling{(4,4)}\hfill
% ^^A The two other regular tilings
%   \Tiling[fillcyclex=2]{(4,4)}\hfill
%   \psboxfill{\Hexagon}
%   \Tiling[fillcyclex=2,fillloopaddy=1]{(5,5)}
% \end{LTXexample}
% 
% \begin{LTXexample}[pos=t]
%   \newcommand{\ArchimedianA}{%
%      ^^A Archimedian tiling 3^2.4.3.4
%     \psset{dimen=middle}
%      ^^A sin(60)=0.866
%     \begin{pspicture}(1.866,1.866)
%       \psframe(1,1)
%       \psline(1,0)(1.866,0.5)(1,1)(0.5,1.866)(0,1)(-0.866,0.5)
%       \psline(0,0)(0.5,-0.866)
%     \end{pspicture}}
%   \newcommand{\ArchimedianB}{%
%      ^^A Archimedian tiling 4.8^2
%     \psset{dimen=middle,unit=1.5}
%      ^^A sin(22.5)=0.3827 ; cos(22.5)=0.9239
%     \begin{pspicture}(1.3066,0.6533)
%       \SpecialCoor
%      ^^A Octogon
%       \pspolygon(0.5;22.5)(0.5;67.5)(0.5;112.5)(0.5;157.5)
%                 (0.5;202.5)(0.5;247.5)(0.5;292.5)(0.5;337.5)
%     \end{pspicture}}
% 
%   \psset{unit=0.5}
%   \psboxfill{\ArchimedianA}
%   \Tiling[fillmove=0.5]{(7,7)}\hfill
%   \psboxfill{\ArchimedianB}
%   \Tiling[fillcyclex=2,fillloopaddy=1]{(7,7)}
% \end{LTXexample}
% 
%   \setcounter{footnote}{3}
%   We can of course tile an area arbitrarily defined. And with the
% \texttt{addfillstyle} parameter\footnote{Introduced in PSTricks 97.}, we can
% easily mix the \texttt{boxfill} style with another one.
% 
% \begin{LTXexample}[width=6cm]
%   \psset{unit=0.5,dimen=middle}
%   \psboxfill{%
%     \begin{pspicture}(1,1)
%       \psframe(1,1)
%       \pscircle(0.5,0.5){0.25}
%     \end{pspicture}}
%   \begin{pspicture}(4,6)
%     \pspolygon[fillstyle=boxfill,fillsep=0.25](0,1)(1,4)(4,6)(4,0)(2,1)
%   \end{pspicture}\hspace{1em}
%   \begin{pspicture}(4,4)
%%     \pscircle[linestyle=none,fillstyle=solid,fillcolor=yellow,fillsep=0.5,
%%               addfillstyle=boxfill](2,2){2}
%   \end{pspicture}
% \end{LTXexample}
%
%   Various effects can be obtained, sometimes complicated ones very easily, as
% in this example reproduced from one shown by Slavik \textsc{Jablan} in the
% field of \emph{OpTiles}, inspired by the \emph{Op-art}:
% 
% \begin{LTXexample}[pos=t]
% \newcommand{\ProtoTile}{%
%  \begin{pspicture}(1,1)%%% 1/12=0.08333
%   \psset{linestyle=none,linewidth=0,
%     hatchwidth=0.08333\psunit,hatchsep=0.08333\psunit}
%   \psframe[fillstyle=solid,fillcolor=black,addfillstyle=hlines,hatchcolor=white](1,1)
%   \pswedge[fillstyle=solid,fillcolor=white,addfillstyle=hlines]{1}{0}{90}
%  \end{pspicture}}
% \newcommand{\BasicTile}{%
%  \begin{pspicture}(2,1)
%    \rput[lb](0,0){\ProtoTile}\rput[lb](1,0){\psrotateleft{\ProtoTile}}
%  \end{pspicture}}
% \ProtoTile\hfill\BasicTile\hfill
% \psboxfill{\BasicTile}
% \Tiling[fillcyclex=2]{(4,4)}
% \end{LTXexample}
% 
%   It is also directly possible to surimpose several different tilings. Here is
% the splendid visual proof of the \textsc{Pytha\-gore} theorem done by the arab
% mathematician \textsc{Annairizi} around the year 900, given by superposition
% of two tilings by squares of different sizes.
% 
% \begin{LTXexample}[pos=t]
% \psset{unit=1.5,dimen=middle}
% \begin{pspicture*}(3,3)
%   \psboxfill{\begin{pspicture}(1,1)
%     \psframe(1,1)\end{pspicture}}
%   \psframe[fillstyle=boxfill](3,3)
%   \psboxfill{\begin{pspicture}(1,1)
%     \rput{-37}{\psframe[linecolor=red](0.8,0.8)}
%   \end{pspicture}}
%   \psframe[fillstyle=boxfill](3,4)
%   \pspolygon[fillstyle=hlines,hatchangle=90](1,2)(1.64,1.53)(2,2)
% \end{pspicture*}
% \end{LTXexample}
% 
%   In a same way, it is possible to build tilings based on figurative patterns,
% in the style of the famous \textsc{Escher} ones. Following an example of
% Andr\'e \textsc{Deledicq} \cite{Deledicq97}, we first show a simple tiling of
% the \emph{p1} category (according to the international classification of the
% 17~symmetry groups of the plane first discovered by the russian
% crystalographer Jevgraf \textsc{Fedorov} at the end of the 19th century):
% 
% \begin{LTXexample}[pos=t]
%  \newcommand{\SheepHead}[1]{%
%    \begin{pspicture}(3,1.5)
%      \pscustom[liftpen=2,fillstyle=solid,fillcolor=#1]{%
%        \pscurve(0.5,-0.2)(0.6,0.5)(0.2,1.3)(0,1.5)(0,1.5)
%          (0.4,1.3)(0.8,1.5)(2.2,1.9)(3,1.5)(3,1.5)(3.2,1.3)
%          (3.6,0.5)(3.4,-0.3)(3,0)(2.2,0.4)(0.5,-0.2)}
%      \pscircle*(2.65,1.25){0.12\psunit} % Eye
%      \psccurve*(3.5,0.3)(3.35,0.45)(3.5,0.6)(3.6,0.4)% Muzzle
%     ^^A   % Mouth
%       \pscurve(3,0.35)(3.3,0.1)(3.6,0.05)
%     ^^A   % Ear
%       \pscurve(2.3,1.3)(2.1,1.5)(2.15,1.7)\pscurve(2.1,1.7)(2.35,1.6)(2.45,1.4)
%   \end{pspicture}}
%  \psboxfill{\psset{unit=0.5}\SheepHead{yellow}\SheepHead{cyan}}
%  \Tiling[fillcyclex=2,fillloopadd=1]{(10,5)}
% \end{LTXexample}
% \label{fig:Sheeps}
% 
%   Now a tiling of the \emph{pg} category (the code for the kangaroo itself is
% too long to be shown here, but has no difficulties ; the kangaroo is reproduce
% from an original picture from Raoul \textsc{Raba} and here is a translation in
% PSTricks from the one drawn by Emmanuel \textsc{Chailloux} and Guy
% \textsc{Cousineau} for their MLgraph system \cite{MLgraphTSI}):
% 
% \begin{LTXexample}[pos=t]
% \psboxfill{\psset{unit=0.4}
%   \Kangaroo{yellow}\Kangaroo{red}\Kangaroo{cyan}\Kangaroo{green}%
%   \psscalebox{-1 1}{%
%     \rput(1.235,4.8){\Kangaroo{green}\Kangaroo{cyan}\Kangaroo{red}\Kangaroo{yellow}}}}
%   \Tiling[fillloopadd=1]{(10,6)}
% \end{LTXexample}
% 
%   And here a \textsc{Wang} tiling \cite{Wang65}, \cite[chapter
% 11]{GS87}, based on very simple tiles of the form of a square and composed
% of four colored triangles. Such tilings are built with only a matching color
% constraint. Despite of it simplicity, it is an important kind of tilings, as
% \textsc{Wang} and others used them to study the special class of
% \emph{aperiodic} tilings, and also because it was shown that surprisingly this 
% tiling is similar to a \textsc{Turing} machine.
% 
% \begin{LTXexample}[pos=t]
%   \newcommand{\WangTile}[4]{%
%     \begin{pspicture}(1,1)
%       \pspolygon*[linecolor=#1](0,0)(0,1)(0.5,0.5)
%       \pspolygon*[linecolor=#2](0,1)(1,1)(0.5,0.5)
%       \pspolygon*[linecolor=#3](1,1)(1,0)(0.5,0.5)
%       \pspolygon*[linecolor=#4](1,0)(0,0)(0.5,0.5)
%     \end{pspicture}}
%   \newcommand{\WangTileA}{\WangTile{cyan}{yellow}{cyan}{cyan}}
%   \newcommand{\WangTileB}{\WangTile{yellow}{cyan}{cyan}{red}}
%   \newcommand{\WangTileC}{\WangTile{cyan}{red}{yellow}{yellow}}
%   \newcommand{\WangTiles}[1][]{%
%     \begin{pspicture}(3,3) \psset{ref=lb}
%       \rput(0,2){\WangTileB}  \rput(1,2){\WangTileA}%
%       \rput(2,2){\WangTileC}  \rput(0,1){\WangTileC}%
%       \rput(1,1){\WangTileB}  \rput(2,1){\WangTileA}
%       \rput(0,0){\WangTileA}  \rput(1,0){\WangTileC}%
%       \rput(2,0){\WangTileB}
%       #1
%     \end{pspicture}}
%   \WangTileA\hfill\WangTileB\hfill\WangTileC\hfill
%   \WangTiles[{\psgrid[subgriddiv=0,gridlabels=0](3,3)}]\hfill
%   \psset{unit=0.4} \psboxfill{\WangTiles} \Tiling{(12,12)}
% \end{LTXexample}
% 
% \subsection{External graphic files}
% \label{sec:GraphicFiles}
% 
%   We can also fill an arbitrary area with an external image. We have only, 
% as usual, to matter of the \emph{BoundingBox} definition if there is no one
% provided or if it is not the accurate one, as for the well known
% \texttt{tiger} picture part of the \texttt{ghostscript} distribution.
% 
% \begin{LTXexample}[pos=t]
%   \psboxfill{%% Strangely require x1=x2...
%     \begin{pspicture}(0,1)(0,4.1)
%       \includegraphics[bb=17 176 560 74,width=3cm]{tiger}
%     \end{pspicture}}
%   \Tiling{(6,6.2)}
% \end{LTXexample}
% 
%   Nevertheless, there are some special files for which the \emph{automatic}
% mode doesn't work, specially for some files obtained by a screen dump, as in
% the next example, where a picture was reduced before it conversion in the
% \emph{Encapsulated PostScript} format by a screen dump utility. In this case,
% usage of the \emph{manual} mode is the only alternative, at the price of the
% real multiple inclusion of the EPS file. We must take care to specify the
% correct \texttt{fillsize} parameter, because otherwise the default values are
% large and will load the file many times, perhaps just really using few
% occurrences as the other ones would be clipped...
% 
% \begin{LTXexample}[pos=t]
%   \psboxfill{\includegraphics{flowers}}
%   \begin{pspicture}(8,4)
%     \psellipse[fillstyle=boxfill,fillsize={(8,4)}](4,2)(4,2)
%   \end{pspicture}
% \end{LTXexample}
% 
% \subsection{Tiling of characters}
% 
%   We can also use the \cs{psboxfill} macro to fill the interior of characters
% for special effects like these ones:
% 
% \begin{LTXexample}[pos=t]
%   \DeclareFixedFont{\bigsf}{T1}{phv}{b}{n}{4.5cm}
%   \DeclareFixedFont{\smallrm}{T1}{ptm}{m}{n}{3mm}
%   \psboxfill{\smallrm Since 182 days...}
%   \begin{pspicture*}(8,4)
%     \centerline{%
%       \pscharpath[fillstyle=gradient,gradangle=-45,
%                   gradmidpoint=0.5,addfillstyle=boxfill,
%                   fillangle=45,fillsep=0.7mm]
%                  {\rput[b](0,0.1){\bigsf 2000}}}
%   \end{pspicture*}
% \end{LTXexample}
% 
% \begin{LTXexample}[pos=t]
%   \DeclareFixedFont{\mediumrm}{T1}{ptm}{m}{n}{2cm}
%   \psboxfill{%
%     \psset{unit=0.1,linewidth=0.2pt}
%     \Kangaroo{PeachPuff}\Kangaroo{PaleGreen}%
%       \Kangaroo{LightBlue}\Kangaroo{LemonChiffon}%
%     \psscalebox{-1 1}{%
%       \rput(1.235,4.8){%
%         \Kangaroo{LemonChiffon}\Kangaroo{LightBlue}%
%           \Kangaroo{PaleGreen}\Kangaroo{PeachPuff}}}}
% ^^A   % A kangaroo of kangaroos...
%   \begin{pspicture}(8,2)
%     \pscharpath[linestyle=none,fillstyle=boxfill,fillloopadd=1]
%                {\rput[b](4,0){\mediumrm Kangaroo}}
%   \end{pspicture}
% \end{LTXexample}
% 
% \subsection{Other kinds of usage}
% 
%   Other kinds of usage can be imagined. For instance, we can use tilings in a
% sort of degenerated way to draw some special lines made by a unique or
% multiple repeating patterns. But it can be only a special dashed line, as here
% with three different dashes:
% 
% \begin{LTXexample}[pos=t]
%   \newcommand{\Dashes}{%
%     \psset{dimen=middle}
%     \begin{pspicture}(0,-0.5\pslinewidth)(1,0.5\pslinewidth)
%       \rput(0,0){\psline(0.4,0)}%
%         \rput(0.5,0){\psline(0.2,0)}%
%         \rput(0.8,0){\psline(0.1,0)}
%     \end{pspicture}}
% 
%   \newcommand{\SpecialDashedLine}[3]{%
%     \psboxfill{#3}
%     \Tiling[linestyle=none]
%            {(#1,-0.5\pslinewidth)(#2,0.5\pslinewidth)}}
% 
%   \SpecialDashedLine{0}{7}{\Dashes}
% 
%   \psset{unit=0.5,linewidth=1mm,linecolor=red}
%   \SpecialDashedLine{0}{10}{\Dashes}
% \end{LTXexample}
% 
%   It allow also to use special patterns in business graphics, as in the
% following example generated by \texttt{PstChart}\footnote{A personal
% development to draw business charts with PSTricks, not distributed.}.
% 
% \vspace{3mm}
% \begin{figure}[!ht]
% \centering
% \psset{unit=0.75}
% ^^A % Generated by pstchart.sh version 0.21 (11/28/97)
% {\psset{dimen=middle}
% \psset{xunit=2,yunit=0.005}
% \begin{pspicture}(-0.6,-200)(6.6,2300)
% ^^A   % Title
%   \rput(3,2200){\shortstack{Fantaisist repartition of kangaroos\\
%                             in the world (in thousands)}}
% ^^A   % Frame background
%   \psframe[fillstyle=solid,fillcolor=LemonChiffon](0,0)(6,2000)
% ^^A   % Graduations
%   \multido{\n=0+500}{5}{\rput[r](-0.12,\n){\psscalebox{0.8}{\n}}}
% ^^A   % Minor ticks
%   \multips(0,100)(0,100){19}{\psline[unit=4.8pt](1,0)}
%   \multips(6,100)(0,100){19}{\psline[unit=4.8pt](-1,0)}
% ^^A   % Major ticks
%   \multips(0,500)(0,500){3}{\psline[unit=9.6pt](1,0)}
%   \multips(6,500)(0,500){3}{\psline[unit=9.6pt](-1,0)}
% ^^A   % Lines from major ticks marks
%   \multips(0,500)(0,500){3}{\psline[linestyle=dotted,linewidth=0.6pt](6,0)}
% ^^A   % Drawing for the data
%   \psboxfill{\psset{unit=0.78\psxunit}\KangarooPstChart{red}}
%   \psframe[linestyle=none,fillstyle=boxfill,fillloopaddy=1](0.61,0)(1.39,1800)
%   \psboxfill{\psset{unit=0.78\psxunit}\KangarooPstChart{yellow}}
%   \psframe[linestyle=none,fillstyle=boxfill,fillloopaddy=1](1.61,0)(2.39,800)
%   \psboxfill{\psset{unit=0.78\psxunit}\KangarooPstChart{cyan}}
%   \psframe[linestyle=none,fillstyle=boxfill,fillloopaddy=1](2.61,0)(3.39,550)
%   \psboxfill{\psset{unit=0.78\psxunit}\KangarooPstChart{magenta}}
%   \psframe[linestyle=none,fillstyle=boxfill,fillloopaddy=1](3.61,0)(4.39,500)
%   \psboxfill{\psset{unit=0.78\psxunit}\KangarooPstChart{green}}
%   \psframe[linestyle=none,fillstyle=boxfill,fillloopaddy=1](4.61,0)(5.39,200)
% ^^A   % Bottom labels
%   \uput{0.2}[270]{0}(1,0){\psscalebox{0.7}{Oceania}}
%   \uput{0.2}[270]{0}(2,0){\psscalebox{0.7}{Africa}}
%   \uput{0.2}[270]{0}(3,0){\psscalebox{0.7}{Asia}}
%   \uput{0.2}[270]{0}(4,0){\psscalebox{0.7}{America}}
%   \uput{0.2}[270]{0}(5,0){\psscalebox{0.7}{Europe}}
% ^^A   % Frame box around the chart
%   \psframe[linestyle=solid](0,0)(6,2000)
% \end{pspicture}}
%   \caption{Bar chart generated by PstChart, with bars filled by patterns}
%   \label{fig:PstChart}
% \end{figure}
% 
% \section{``Dynamic'' tilings}
% 
%   In some cases, tilings used non \emph{static} tiles, that is to say that the 
% \emph{prototile(s)}, even if unique, can have several forms, by instance
% specified by different colors or rotations, not fixed before generation or
% varying each time.
% 
% \subsection{Lewthwaite-Pickover-Truchet tiling}
% 
%   We give here for example the so-called \emph{Truchet} tiling, which much be
% in fact better called \emph{Lewthwaite-Pick\-over-Truchet (LPT)} tiling%
% \footnote{For description of the context, history and references about
% S\'ebastien \textsc{Truchet} and this tiling, see \cite{EsperetGirou98}.}.
% 
%   The unique prototile is only a square with two opposite circle arcs.
% This tile has obviously two positions, if we rotate it from 90 degrees (see
% the two tiles on the next figure). A \emph{LPT tiling} is a tiling with
% randomly oriented LPT tiles. We can see that even if it is very simple in it
% principle, it draw sophisticated curves with strange properties.
% 
%   Nevertheless, in the straightforward way \FillPackage{} does not work,
% because the \cs{psboxfill} macro store the content of the tile used in a
% \TeX{} box, which is static. So the calling to the random function is done
% only one time, which explain that only one rotation of the tile is used for
% all the tiling. It's only the one of the two rotations which could differ from
% one drawing to the next one...
% 
% ^^A % Truchet (Lewthwaite-Pickover-Truchet) tiling
% ^^A % --------------------------------------------
% 
% \begin{LTXexample}[pos=t]
% ^^A   % LPT prototile
%   \newcommand{\ProtoTileLPT}{%
%     \psset{dimen=middle}
%     \begin{pspicture}(1,1)
%       \psframe(1,1)
%       \psarc(0,0){0.5}{0}{90}
%       \psarc(1,1){0.5}{-180}{-90}
%     \end{pspicture}}
% 
% ^^A   % LPT tile
%   \newcount\Boolean
%   \newcommand{\BasicTileLPT}{%
% ^^A     % From random.tex by Donald Arseneau
%     \setrannum{\Boolean}{0}{1}%
%     \ifnum\Boolean=0
%       \ProtoTileLPT%
%     \else
%       \psrotateleft{\ProtoTileLPT}%
%     \fi}
% 
%   \ProtoTileLPT\hfill\psrotateleft{\ProtoTileLPT}\hfill
%   \psset{unit=0.5}
%   \psboxfill{\BasicTileLPT}
%   \Tiling{(5,5)}
% \end{LTXexample}
% 
%   But, for simple cases, there is a solution to this problem using a mixture
% of PSTricks and PostScript programming. Here the PSTricks
% construction \verb+\pscustom{\code{...}}+ allow to insert PostScript code
% inside the \LaTeX{} + PSTricks one.
% 
%   Programmation is less straightforward, but it has also the advantage to be
% notably faster, as all the tilings operations are done in PostScript, and
% mainly to not be limited by \TeX{} memory (the \TeX{} + PSTricks solution
% I wrote in 1995 for the colored problem was limited to small sizes for this
% reason). Just note also that \cs{pslbrace} and \cs{psrbrace} are two
% PSTricks macros to define and be able to insert the \verb+{+ and \verb+}+
% characters.
% 
% \begin{LTXexample}[pos=t]
% ^^A   % LPT prototile
%   \newcommand{\ProtoTileLPT}{%
%     \psset{dimen=middle}
%     \psframe(1,1)
%     \psarc(0,0){0.5}{0}{90}
%     \psarc(1,1){0.5}{-180}{-90}}
% 
% ^^A   % Counter to change the random seed
%   \newcount\InitCounter
% ^^A   % LPT tile
%   \newcommand{\BasicTileLPT}{%
%     \InitCounter=\the\time
%     \pscustom{\code{%
%       rand \the\InitCounter\space sub 2 mod 0 eq \pslbrace}}
%     \begin{pspicture}(1,1)
%       \ProtoTileLPT
%     \end{pspicture}%
%     \pscustom{\code{\psrbrace \pslbrace}}
%     \psrotateleft{\ProtoTileLPT}%
%     \pscustom{\code{\psrbrace ifelse}}}
% 
%   \psset{unit=0.4,linewidth=0.4pt}
%   \psboxfill{\BasicTileLPT}
%   \Tiling{(15,15)}
% \end{LTXexample}
% 
%   Using the very surprising fact (see \cite{EsperetGirou98}) that
% coloration of these tiles do not depend of their neighbors (even if it is
% difficult to believe as the opposite seems obvious!) but only of the parity of
% the value of row and column positions, we can directly program in the same way
% a colored version of the LPT tiling.
% 
% \setcounter{footnote}{1}
%   We have also introduce in the \FillPackage{} code for \emph{tiling} mode two
% new accessible Post\-Script variables, \texttt{row} and
% \texttt{column}\footnotemark, which can be useful in some circonstances, like
% this one.
% 
% \begin{LTXexample}[pos=t]
% ^^A   % LPT prototile
%   \newcommand{\ProtoTileLPT}[2]{%
%     \psset{dimen=middle,linestyle=none,fillstyle=solid}
%     \psframe[fillcolor=#1](1,1)
%     \psset{fillcolor=#2}
%     \pswedge(0,0){0.5}{0}{90} \pswedge(1,1){0.5}{-180}{-90}}
% ^^A   % Counter to change the random seed
%   \newcount\InitCounter
% ^^A   % LPT tile
%   \newcommand{\BasicTileLPT}[2]{%
%     \InitCounter=\the\time
%     \pscustom{\code{%
%       rand \the\InitCounter\space sub 2 mod 0 eq \pslbrace
%       row column add 2 mod 0 eq \pslbrace}}
%     \begin{pspicture}(1,1)\ProtoTileLPT{#1}{#2}\end{pspicture}%
%     \pscustom{\code{\psrbrace \pslbrace}}
%     \ProtoTileLPT{#2}{#1}%
%     \pscustom{\code{%
%       \psrbrace ifelse \psrbrace \pslbrace row column add 2 mod 0 eq \pslbrace}}
%     \psrotateleft{\ProtoTileLPT{#2}{#1}}\pscustom{\code{\psrbrace \pslbrace}}
%     \psrotateleft{\ProtoTileLPT{#1}{#2}}\pscustom{\code{\psrbrace ifelse \psrbrace ifelse}}}
%   \psboxfill{\BasicTileLPT{red}{yellow}}
%   \Tiling{(4,4)}\hfill
%   \psset{unit=0.4}\psboxfill{\BasicTileLPT{blue}{cyan}}
%   \Tiling{(15,15)}
% \end{LTXexample}
% 
%   Another classic example is to generate coordinates and numerotation for a
% grid. Of course, it is possible to do it directly in PSTricks using nested
% \cs{multido} commands. It would be clearly easy to program, but, nevertheless, 
% for users who have a little knowledge of PostScript programming, this offer
% an alternative which is useful for large cases, because on this way it will
% be notably faster and less computer ressources consuming.
% 
%   Remember here that the tiling is drawn from left to right, and top to
% bottom, and note that the PostScript variable \texttt{x2} give the total
% number of columns.
% 
% \begin{LTXexample}[pos=t]
% ^^A   % \Escape will be the \ character
%   {\catcode`\!=0\catcode`\\=11!gdef!Escape{\}}
%   \newcommand{\ProtoTile}{%
%     \Square\pscustom{%
%       \moveto(-0.9,0.75) % In PSTricks units
%       \code{ /Times-Italic findfont 8 scalefont setfont
%         (\Escape() show row 3 string cvs show (,) show 
%         column 3 string cvs show (\Escape)) show}
%       \moveto(-0.5,0.25) % In PSTricks units
%       \code{ /Times-Bold findfont 18 scalefont setfont
%         1 0 0 setrgbcolor % Red color
%         /center {dup stringwidth pop 2 div neg 0 rmoveto} def
%         row 1 sub x2 mul column add 3 string cvs center show}}}
%   \psboxfill{\ProtoTile}
%   \Tiling{(6,4)}
% \end{LTXexample}
% 
% \subsection{A complete example: the Poisson equation}
% 
%   To finish, we will show a complete real example, a drawing to explain the
% method used to solve the \textsc{Poisson} equation by a domain
% decomposition method, adapted to distributed memory computers. The
% objective is to show the communications required between processes and the
% position of the data to exchange. This code also show some useful and powerful
% technics for PSTricks programming (look specially at the way some higher level
% macros are defined, and how the same object is used to draw the four
% neighbors).
%
%\psset{unit=1cm}
%\newcommand{\Pattern}[1]{%
%   \begin{pspicture}(-0.25,-0.25)(0.25,0.25)\rput{*0}{\psdot[dotstyle=#1]}
%   \end{pspicture}}
%\newcommand{\West}{\Pattern{o}}   \newcommand{\South}{\Pattern{x}}
%\newcommand{\Central}{\Pattern{+}}\newcommand{\North}{\Pattern{square}}
%\newcommand{\East}{\Pattern{triangle}}
%\newcommand{\Cross}{%
%  \pspolygon[unit=0.5,linewidth=0.2,linecolor=red](0,0)(0,1)(1,1)(1,2)(2,2)(2,1)%
%              (3,1)(3,0)(2,0)(2,-1)(1,-1)(1,0)}
%\newcommand{\StylePosition}[1]{\LARGE\textcolor{red}{\textbf{#1}}}
%\newcommand{\SubDomain}[4]{%
%    \psboxfill{#4}\begin{psclip}{\psframe[linestyle=none]#1}%
%      \psframe[linestyle=#3](5,5)\psframe[fillstyle=boxfill]#2%
%    \end{psclip}}
%\newcommand{\SendArea}[1]{\psframe[fillstyle=solid,fillcolor=cyan]#1}
%\newcommand{\ReceiveData}[2]{%
%  \psboxfill{#2}\psframe[fillstyle=solid,fillcolor=yellow,addfillstyle=boxfill]#1}%
%\newcommand{\Neighbor}[2]{%
%    \begin{pspicture}(5,5)
%      \rput{*0}(2.5,2.5){\StylePosition{#1}}
%      \ReceiveData{(0.5,0)(4.5,0.5)}{\Central}\SendArea{(0.5,0.5)(4.5,1)}%
%      \SubDomain{(5,2)}{(0.5,0.5)(4.5,3)}{dashed}{#2}%
%      \pcarc[arcangle=45,arrows=->](0.5,-1.25)(0.5,0.25)%
%      \pcarc[arcangle=45,arrows=->,linestyle=dotted,dotsep=2pt](4.5,0.75)(4.5,-0.75)%
%    \end{pspicture}}%
%  \psset{dimen=middle,dotscale=2,fillloopadd=2}
%\begin{pspicture}(-5.7,-5.7)(5.7,5.7)
%  \rput(0,0){%
%      \begin{pspicture}(5,5)
%        \ReceiveData{(0,0.5)(0.5,4.5)}{\West} \ReceiveData{(4.5,0.5)(5,4.5)}{\East}
%        \ReceiveData{(0.5,4.5)(4.5,5)}{\North}\ReceiveData{(0.5,0)(4.5,0.5)}{\South}
%        \SendArea{(0.5,0.5)(1,4.5)}\SendArea{(4,0.5)(4.5,4.5)}
%        \SendArea{(0.5,0.5)(4.5,1)}\SendArea{(0.5,4)(4.5,4.5)}
%        \SubDomain{(5,5)}{(0.5,0.5)(4.5,4.5)}{solid}{\Central}
%        \psline(1,0.5)(1,4.5)\psline(4,0.5)(4,4.5)%
%        \rput(1.5,4){\Cross}\rput(2,2){\Cross}%
%      \end{pspicture}}%
%  \rput(0,5.5){\Neighbor{N}{\North}}\rput{-90}(5.5,0){\Neighbor{E}{\East}}%
%  \rput{90}(-5.5,0){\Neighbor{W}{\West}}\rput{180}(0,-5.5){\Neighbor{S}{\South}}%
%\end{pspicture}
%
% \begin{lstlisting}
%   \newcommand{\Pattern}[1]{%
%     \begin{pspicture}(-0.25,-0.25)(0.25,0.25)\rput{*0}{\psdot[dotstyle=#1]}
%     \end{pspicture}}
%   \newcommand{\West}{\Pattern{o}}   \newcommand{\South}{\Pattern{x}}
%   \newcommand{\Central}{\Pattern{+}}\newcommand{\North}{\Pattern{square}}
%   \newcommand{\East}{\Pattern{triangle}}
%   \newcommand{\Cross}{%
%     \pspolygon[unit=0.5,linewidth=0.2,linecolor=red](0,0)(0,1)(1,1)(1,2)(2,2)(2,1)
%               (3,1)(3,0)(2,0)(2,-1)(1,-1)(1,0)}
%   \newcommand{\StylePosition}[1]{\LARGE\textcolor{red}{\textbf{#1}}}
%   \newcommand{\SubDomain}[4]{%
%     \psboxfill{#4}
%     \begin{psclip}{\psframe[linestyle=none]#1}
%       \psframe[linestyle=#3](5,5)\psframe[fillstyle=boxfill]#2
%     \end{psclip}}
%   \newcommand{\SendArea}[1]{\psframe[fillstyle=solid,fillcolor=cyan]#1}
%   \newcommand{\ReceiveData}[2]{%
%     \psboxfill{#2}
%     \psframe[fillstyle=solid,fillcolor=yellow,addfillstyle=boxfill]#1}
%   \newcommand{\Neighbor}[2]{%
%     \begin{pspicture}(5,5)
%       \rput{*0}(2.5,2.5){\StylePosition{#1}}
%       \ReceiveData{(0.5,0)(4.5,0.5)}{\Central}\SendArea{(0.5,0.5)(4.5,1)}
%       \SubDomain{(5,2)}{(0.5,0.5)(4.5,3)}{dashed}{#2}%
% ^^A       % Receive and send arrows
%       \pcarc[arcangle=45,arrows=->](0.5,-1.25)(0.5,0.25)
%       \pcarc[arcangle=45,arrows=->,linestyle=dotted,dotsep=2pt](4.5,0.75)(4.5,-0.75)
%     \end{pspicture}}
%   \psset{dimen=middle,dotscale=2,fillloopadd=2}
%   \begin{pspicture}(-5.7,-5.7)(5.7,5.7)
% ^^A     % Central domain
%     \rput(0,0){%
%       \begin{pspicture}(5,5)
% ^^A         % Receive from West, East, North and South
%         \ReceiveData{(0,0.5)(0.5,4.5)}{\West} \ReceiveData{(4.5,0.5)(5,4.5)}{\East}
%         \ReceiveData{(0.5,4.5)(4.5,5)}{\North}\ReceiveData{(0.5,0)(4.5,0.5)}{\South}
% ^^A         % send area for West, East, North and South
%         \SendArea{(0.5,0.5)(1,4.5)} \SendArea{(4,0.5)(4.5,4.5)}
%         \SendArea{(0.5,0.5)(4.5,1)} \SendArea{(0.5,4)(4.5,4.5)}
% ^^A         % Central domain
%         \SubDomain{(5,5)}{(0.5,0.5)(4.5,4.5)}{solid}{\Central}
% ^^A         % Redraw overlapped linesY
%         \psline(1,0.5)(1,4.5)  \psline(4,0.5)(4,4.5)
% ^^A         % Two crossesY
%         \rput(1.5,4){\Cross}  \rput(2,2){\Cross}
%       \end{pspicture}}
% ^^A     % The four neighborsY
%     \rput(0,5.5){\Neighbor{N}{\North}}     \rput{-90}(5.5,0){\Neighbor{E}{\East}}
%     \rput{90}(-5.5,0){\Neighbor{W}{\West}} \rput{180}(0,-5.5){\Neighbor{S}{\South}}
%   \end{pspicture}
% \end{lstlisting}
%
%
%
% Bibliography
% \begin{thebibliography}{99}
% \bibitem{PostScript95} Adobe, Systems~Incorporated, \emph{PostScript Language
% Reference Manual}, Addison-Wesley, 2~edition, 1995.
%
% \bibitem{Bolek98} Piotr Bolek, \MP{} and patterns, \emph{\TUB}, Volume~19,
% Number~3, pages 276--283, September 1998, \CTANref{mpattern}.
%
% \bibitem{MLgraphTSI} Emmanuel Chailloux, Guy Cousineau and Asc\'ander
% Su\'arez, Programmation fonctionnelle de graphismes pour la production
% d'illustrations techniques, \emph{Technique et science informatique},
% Volume~15, Number~7, pages 977--1007, 1996 (in french).
%
% \bibitem{Deledicq97} Andr\'e Deledicq, \emph{Le monde des pavages}, ACL
% \'Editions, 1997 (in french).
%
% \bibitem{EsperetGirou98} Philippe Esperet and Denis Girou,
% Coloriage du pavage dit de Truchet, Cahiers GUTenberg, Number~31,
% pages 5--18, December~1998  (in french).
%
% \bibitem{Girou94} Denis Girou, Pr\'esentation de PSTricks, \emph{Cahiers
% GUTenberg}, Number~16, pages 21--70, February~1994 (in french).
%
% \bibitem{LGC97} Michel Goossens, Sebastian Rahtz and Frank Mittelbach,
% \emph{The \LaTeX{} Graphics Companion}, Addison-Wesley, 2005.
%
% \bibitem{GS87} Branko Gr\"unbaum and Geoffrey Shephard, \emph{Tilings and
% Patterns}, Freeman and Company, 1987.
%
% \bibitem{Hoenig97} Alan Hoenig, \emph{\TeX{} Unbound: \LaTeX{} \& \TeX{}
% Strategies, Fonts, Graphics, and More}, Oxford University Press, 1997.
%
% \bibitem{XYpic} Kristoffer~H. Rose and Ross Moore, \XYpic. Pattern and Tile
% extension, available from \CTAN, 1991-1998, \CTANref{xypic}.
%
% \bibitem{LAAN96} Kees van der Laan, Paradigms: Just a little bit of PostScript,
% \emph{MAPS}, Volume~17, pages 137--150, 1996.
%
% \bibitem{LAAN97} Kees van der Laan, Tiling in PostScript and \MF{} -- Escher's
% wink, \emph{MAPS}, Volume~19, Number~2, pages 39--67, 1997.
%
% \bibitem{vanZandt93} Timothy Van Zandt, PSTricks. PostScript macros for
% Generic \TeX, available from \CTAN, 1993, \CTANref{pstricks}.
%
% \bibitem{vanZandtGirou94} Timothy Van Zandt and Denis Girou, Inside PSTricks,
% \emph{\TUB}, Volume~15, Number~3, pages 239--246, September 1994.
%
%
% \bibitem{voss07} Herbert Vo\ss, PSTricks -- Graphics for \TeX\ and \LaTeX, DANTE/Lehmanns, 4th ed., 2007.
% \bibitem{Wang65} Hao Wang, Games, Logic and Computers, \emph{Scientific
% American}, pages 98--106, November 1965.
% \end{thebibliography}
%
%
% \StopEventually{}
%
% ^^A .................... End of the documentation part ....................
%
% \section{Driver file}
%
%   The next bit of code contains the documentation driver file for \TeX{},
% i.e., the file that will produce the documentation you are currently
% reading. It will be extracted from this file by the \texttt{docstrip}
% program.
%
%    \begin{macrocode}
%<*driver>
\documentclass{ltxdoc}
\GetFileInfo{pst-fill.dtx}
%
\usepackage[T1]{fontenc}
\usepackage{lmodern}               % For PDF
\usepackage{graphicx}              % `graphicx' LaTeX standard package
\usepackage{showexpl}
\usepackage{mflogo}                % For the MetaFont and MetaPost logos
\input{random.tex}                 % Random macros from Donald Arseneau
\usepackage{url}                   % URLs convenient typesetting
\usepackage{multido}               % General loop macro
\usepackage[dvipsnames]{pstricks}  % PSTricks with the `color' extension
\usepackage{pst-text}              % PSTricks package for character path
\usepackage{pst-grad}              % PSTricks package for gradient filling
\usepackage{pst-node}              % PSTricks package for nodes
\usepackage[tiling]{pst-fill}      % PSTricks package for filling/tiling
%
\AtBeginDocument{%
%  \OnlyDescription % comment out for implementation details
  \EnableCrossrefs
  \CodelineIndex
  \RecordChanges}
\AtEndDocument{%
  \PrintIndex
  \setcounter{IndexColumns}{1}
  \PrintChanges}
\hbadness=7000            % Over and under full box warnings
\hfuzz=3pt
\begin{document}
  \DocInput{pst-fill.dtx}
\end{document}
%</driver>
%    \end{macrocode}
%
% \section{\texttt{pst-fill} \LaTeX{} wrapper}
%
%    \begin{macrocode}
%<*latex-wrapper>
\RequirePackage{pstricks}
\ProvidesPackage{pst-fill}[2005/09/13 package wrapper for 
  pst-fill.tex (hv)]
\DeclareOption{tiling}{\def\PstTiling{true}}
\ProcessOptions\relax
% \iffalse meta-comment, etc.
%%
%% Package `pst-fill.dtx'
%%
%% Denis Girou (CNRS/IDRIS - France) <Denis.Girou@idris.fr>
%% Herbert Voss <voss@pstricks.de>
%%
%% This program can be redistributed and/or modified under the terms
%% of the LaTeX Project Public License Distributed from CTAN archives
%% in directory macros/latex/base/lppl.txt.
%%
%% DESCRIPTION:
%%   `pst-fill' is a PSTricks package for filling and tiling areas 
%%
% \fi
% \changes{v1.01}{2007/03/10}{bugfix for incomplete ifx (hv)}
% \changes{v1.00}{2006/11/06}{use pst-xkey for extend keys (hv)}
% \changes{v0.99}{2004/08/17}{merge the VTeX and TeX versions (patch 4) (hv)}
% \changes{v0.98}{2004/06/22}{delete the Pst@Debug option and use the
%   the one from pstricks to prevent a clash with pst-gr3d (hv)}
% \changes{v0.97}{2001/10/09}{make it work with VTeX (mv)}
% \changes{v0.94}{1997/04/08}{With a \PstTiling macro defined (or "tiling" optional parameter
%   on \textbackslash usepackage[tiling]{pst-fill}), this file run exactly as
%   the original boxfill.tex file from Timothy, version 0.94,
%   except a correction in \textbackslash pst@ManualFillCycle to avoid a division by 0.
%   It's the default.}
% \changes{v0.93}{1997/04/07}{With a \textbackslash PstTiling macro defined (or "tiling" optional parameter
%   on \textbackslash usepackage[tiling]{pst-fill}) there are several add-ons
%   and changes to do `tiling' rather than `filling' in "automatic" mode :
%     - we fix the position of the beginning of tiling,
%     - we allow normally the framing of the area as expected, using
%       the line.... parameters
%     - we define move parameters fillmovex, fillmovey and fillmove,
%     - we define fillcyclex as previous fillcycle parameter, and add the
%       fillcycley and fillcycle (both fillcyclex and fillcycley) ones
%     - we can extend the tiling area using fillloopaddx, fillloopaddy and
%       fillloopadd parameters,
%     - we can debug and see the whole tiling area without clipping using
%       PstDebug parameter,
%     - for names consistancy, we can use fillangle in place of boxfillangle
%       and fillsize in place of boxfillsize,
%     - default value for fillsep is 0 and for fillsize is auto.}
%
% \DoNotIndex{\!,\",\#,\$,\%,\&,\',\(,\+,\*,\,,\-,\.,\/,\:,\;,\<,\=,\>,\?}
% \DoNotIndex{\@,\@B,\@K,\@cTq,\@f,\@fPl,\@ifnextchar,\@nameuse,\@oVk}
% \DoNotIndex{\[,\\,\],\^,\_,\ }
% \DoNotIndex{\^,\\^,\\\^,$\^$,$\\^$,$\\^$}
% \DoNotIndex{\0,\2,\4,\5,\6,\7,\8,}
% \DoNotIndex{\A,\a}
% \DoNotIndex{\B,\b,\Bc,\begin,\Bq,\Bqc}
% \DoNotIndex{\C,\c,\catcode,\cJA,\CodelineIndex,\csname}
% \DoNotIndex{\D,\def,\define@key,\Df,\divide,\DocInput,\documentclass,\pst@addfams}
% \DoNotIndex{\eCN,\edef,\else,\eHd,\eMcj,\EnableCrossrefs,\end,\endcsname}
% \DoNotIndex{\endCenterExample,\endExample,\endinput,\endpsclip}
% \DoNotIndex{\PrintIndex,\PrintChanges,\ProvidesFile}
% \DoNotIndex{\endpspicture,\endSideBySideExample,\Example}
% \DoNotIndex{\F,\f,\FdUrr,\fi,\filedate,\fileversion,\FV@Environment}
% \DoNotIndex{\FV@UseKeyValues,\FV@XRightMargin,\FVB@Example,\fvset}
% \DoNotIndex{\G,\g,\GetFileInfo,\gr,\GradientLoaded,\gsFKrbK@o,\gsj,\gsOX}
% \DoNotIndex{\hbadness,\hfuzz,\HLEmphasize,\HLMacro,\HLMacro@i}
% \DoNotIndex{\HLReverse,\HLReverse@i,\hqcu,\HqY}
% \DoNotIndex{\I,\i,\ifx,\input,\Ir,\IU}
% \DoNotIndex{\j,\jl,\JT,\JVodH}
% \DoNotIndex{\K,\k,\kfSlL}
% \DoNotIndex{\L,\let}
% \DoNotIndex{\message,\mHNa,\mIU}
% \DoNotIndex{\N,\nB,\newcmykcolor,\newdimen,\newif,\nW}
% \DoNotIndex{\O,\oCDJDo,\ocQhVI,\OnlyDescription,\oRKJ}
% \DoNotIndex{\P,\p,\ProvidesPackage,\psframe,\pslinewidth,\psset}
% \DoNotIndex{\PstAtCode,\PSTricksLoaded}
% \DoNotIndex{\q,\Qr,\qssRXq,\qu,\qXjFQp,\qYL}
% \DoNotIndex{\R,\r,\RecordChanges,\relax,\RlaYI,\rN,\Rp,\rp,\RPDXNn,\rput}
% \DoNotIndex{\S,\scalebox,\SgY,\SideBySide@Example,\SideBySideExample}
% \DoNotIndex{\SgY,\sk,\Sp,\space,\sZb}
% \DoNotIndex{\T,\the,\tw@}
% \DoNotIndex{\u,\UiSWGEf@,\uJi,\usepackage,\uVQdMM,\UYj}
% \DoNotIndex{\VerbatimEnvironment,\VerbatimInput,\VrC@}
% \DoNotIndex{\WhZ,\WjKCYb,\WNs}
% \DoNotIndex{\XkN,\XW}
% \DoNotIndex{\Z,\ZCM,\Ze}
% \DoNotIndex{\addtocounter,\advance,\alph,\arabic,\AtBeginDocument,\AtEndDocument}
% \DoNotIndex{\AtEndOfPackage,\begingroup,\bfseries,\bgroup,\box,\csname}
% \DoNotIndex{\else,\endcsname,\endgroup,\endinput,\expandafter,\fi}
% \DoNotIndex{\TeX,\z@,\p@,\@one,\xdef,\thr@@,\string,\sixt@@n,\reset,\or,\multiply,\repeat,\RequirePackage}
% \DoNotIndex{\@cclvi,\@ne,\@ehpa,\@nil,\copy,\dp,\global,\hbox,\hss,\ht,\ifodd,\ifdim,\ifcase,\kern}
% \DoNotIndex{\chardef,\loop,\leavevmode,\ifnum,\lower}
% \setcounter{IndexColumns}{2}
%
% ^^A To extend the height used for the text
%
% ^^A  Aligned labels in a description environment
%\newenvironment{Description}[1]{%
%\begin{list}{nothing}{\setlength{\leftmargin}{#1}
%\setlength{\labelwidth}{\leftmargin}\setlength{\labelsep}{1mm}}}
%{\end{list}}
%
% ^^A For macro names
%\DeclareRobustCommand\cs[1]{\texttt{\char`\\#1}}
%
%
% ^^A From ltugboat.cls
% ^^A For references
%\makeatletter
%\newcommand\acro[1]{\textsc{#1}\@}
%\def\CTAN{\acro{CTAN}}
%\let\texttub\textsl              % ^^A redefined in other situations
%\def\TUB{\texttub{TUGboat}}
%\def\TUG{\TeX\ \UG}
%\def\tug{\acro{TUG}}
%\def\UG{Users Group}
% ^^A For the bibliography 
%\let\@internalcite\cite
%\def\cite{\def\@citeseppen{-1000}%
%    \def\@cite##1##2{(##1\if@tempswa , ##2\fi)}%
%    \def\citeauthoryear##1##2##3{##1, ##3}\@internalcite}
%\def\etal{et\,al.\@}
%\newcommand\CTANdirectory[1]{\expandafter\urldef
%  \csname CTAN@#1\endcsname\path}
%\newcommand\CTANfile[1]{\expandafter\urldef
%  \csname CTAN@#1\endcsname\path}
%\newcommand\CTANref[1]{\expandafter\@setref\csname CTAN@#1\endcsname
%  \relax{#1}}
%\makeatother
% ^^A Define CTAN addresses 
%\CTANdirectory{mpattern}{graphics/metapost/macros/mpattern}
%\CTANdirectory{pstricks}{graphics/pstricks}
%\CTANdirectory{pst-fill.sty}{graphics/pstricks/latex/pst-fill.sty}
%\CTANdirectory{pst-fill}{graphics/pstricks/generic/pst-fill.tex}
%\CTANdirectory{Roegel}{graphics/metapost/contrib/macros/truchet}
%\CTANdirectory{xypic}{macros/generic/diagrams/xypic}
%
% ^^A Personal macros (D.G.)
% ^^A ----------------------
%
% ^^A Some colors used
%\definecolor{LemonChiffon}{rgb}{1.,0.98,0.8}
%\definecolor{LightBlue}   {rgb}{0.8,0.85,0.95}
%\definecolor{PaleGreen}   {rgb}{0.88,1,0.88}
%\definecolor{PeachPuff}   {rgb}{1.0,0.85,0.73}
%
% ^^A To define a unique string for TeX and LaTeX
%\newcommand{\AllTeX}{%
%{\rm(L\kern-.36em\raise.3ex\hbox{\sc a}\kern-.15em)%
%T\kern-.1667em\lower.7ex\hbox{E}\kern-.125emX}}
%
% ^^A Bibliography style
%\bibliographystyle{ltugbib}
%
% ^^A Name macros
%\newcommand{\FillPackage}{\textsf{`pst-fill'}}
%\newcommand{\XYpic}{%
%\leavevmode\hbox{\kern-.1em X\kern-.3em\lower.4ex\hbox{Y\kern-.15em}-pic}}
%
%\makeatletter
%
% ^^A Example environments
% ^^A (do not use in them the four JXYZ characters, that we will use
% ^^A as escape characters!)
%
% ^^A Default PSTricks parameters
%  \psset{dimen=middle}
%
% ^^A Translation in PSTricks from the one drawn by Emmanuel Chailloux and
% ^^A Guy Cousineau for the MLgraph system
% ^^A (see /ftp.ens.fr:/pub/unix/lang/MLgraph/version-2.1/MLgraph-refman.ps.gz)
% ^^A The kangaroo itself is reproduce from an original picture from Raoul Raba
% \newcommand{\DimX}{2.47}
% \newcommand{\DimY}{4.8}
% \newcommand{\DimXDivTwo}{1.235}
%
% \newcommand{\KangarooItself}[1]{%
% ^^A Body
% \pspolygon[fillstyle=solid,fillcolor=#1]%
%  (52.5,68)(55,72.5)(55.8,76.5)(56.8,79.8)(58.2,83)(60,85.8)(61.5,86.5)
% (64,87)(66,87.5)(67.8,87.3)(70,87)(71.5,87.3)(73,88)(74.7,88.5)
% (76,90.3)(77,91.5)(72.8,93.8)(69,96)(64.5,99)(59.4,103)(56.2,106.3)
% (53,110.5)(49.5,115.5)(47.2,119.9)(45.7,126)(43.2,123)(41.5,121)(37.5,125)
% (37,122.5)(36.8,120)(37,117)(37.6,113.5)(38.6,110)(40,106.3)(42,102.3)
%  (43.5,99.5)(45,97)(46.2,94)(46.8,91.7)(47.2,88)(47,83.5)(46.3,80.8)
%  (45.3,78.5)(42.5,76.5)(39.5,75.8)(36,75.9)(33,75.9)(29,76.2)(26,77)
%  (22.3,77.5)(18,78.4)(12.8,79.3)(8.6,80)(5.5,80.3)(3,80.5)(0,80)
%  (-5.2,78.5)(-9,76.3)(-11.2,74.8)(-13,72.5)(-16.5,68)(-16.5,68)(-19.5,62.5)
%  (-22,58)(-25.5,53)(-29,48.5)(-32.5,45)(-36,42)(-39,39.5)(-44,37)
%  (-49,35)(-51,34)(-53.5,34.5)(-55.5,36)(-56.5,38)(-56.5,40.5)(-55,41.5)
%  (-53.5,41)(-51.5,41)(-50.5,43)(-50.5,44.5)(-51,47)(-51.5,47.2)(-56.5,47)
%  (-58.5,46.5)(-60,44.7)(-62,42.3)(-63,39.5)(-63.5,36.3)(-63.5,33)(-63.1,29.5)
%  (-61.5,26)(-58,23.6)(-54,22.2)(-50.7,22)(-47.5,22)(-44.5,22.3)(-41,23.5)
%  (-36.8,25.8)(-33,28)(-28.5,31)(-23.4,35)(-20.2,38.3)(-17,42.5)(-13.5,47.5)
%  (-11.2,51.9)(-9.7,58)(-7.2,55)(-5.5,53)(-1.5,57)(-1,54.5)(-0.8,52)
%  (-1,49)(-1.6,45.5)(-2.6,42)(-4,38.3)(-6,34.3)(-7.5,31.5)(-9,29)
%  (-10.2,26)(-10.8,23.7)(-11.2,20)(-11,15.5)(-10.3,12.8)(-9.3,10.5)(-6.5,8.5)
%  (-3.5,7.8)(0,7.9)(3,7.9)(7,8.2)(10,9)(13.7,9.5)(18,10.4)
%  (23.2,11.3)(27.4,12)(30.5,12.3)(33,12.5)(36,12)(41.2,10.5)(45,8.3)
%  (47.2,6.8)(49,4.5)(52.5,0)(50,4.5)(49.2,8.5)(48.2,11.8)(46.8,15)
%  (45,17.8)(43.5,18.5)(41,19)(39,19.5)(37.2,19.3)(35,19)(33.5,19.3)
%  (32,20)(30.3,20.5)(29,22.3)(28,23.5)(28,23.5)(24.5,22.3)(21.5,22)
%  (18.3,22)(15,22.2)(11,23.6)(7.5,26)(5.9,29.5)(5.5,33)(5.5,36.3)
%  (6,39.5)(7,42.3)(9,44.7)(10.5,46.5)(12.5,47)(17.5,47.2)(18,47)
%  (18.5,44.5)(18.5,43)(17.5,41)(15.5,41)(14,41.5)(12.5,40.5)(12.5,38)
%  (13.5,36)(15.5,34.5)(18,34)(20,35)(25,37)(30,39.5)(33,42)
%  (36.5,45)(40,48.5)(43.5,53)(47,58)(49.5,62.5)(52.5,68)
% ^^A Eye
% \pscircle*[linecolor=white](58.2,98.3){2\psxunit}
% \pscircle*(58.2,97.3){\psxunit}
% ^^A Mouth
% \psline(71.5,88)(70,89.3)(68.5,90.3)(67,91.9)
% ^^A Tear
% \psline(42,121)(45,118)(47,115.3)(48.5,112.7)(50,110)(51.8,106.5)
%       (52.5,103.7)(53,100.5)
% \pspolygon(41.2,115.8)(43.2,114.7)(45,112.5)(47,109.8)(48,107)(49.5,104.2)%
%       (50.5,101.6)(51,98.5)(47.7,100.6)(46,102.2)(44.8,104)(43.5,106)
%       (42.5,108)(41.7,110.5)(41,113.2)}
%
% \newcommand{\Kangaroo}[1]{%
%   \begin{pspicture}(\DimX,\DimY)
%   \psset{unit=0.035278}
%   \KangarooItself{#1}
%   \end{pspicture}}
%
% \newcommand{\KangarooPstChart}[1]{{%
%   \psset{xunit=0.006784,yunit=0.00735,linewidth=0.01}
%   \begin{pspicture}(-65.5,0)(82,126)
%     \KangarooItself{#1}
%   \end{pspicture}}}
%
%
% ^^A For the possible index and changes log
% \setlength{\columnseprule}{0.6pt}
%
% ^^A Beginning of the documentation itself
%\title{\texttt{pst-fill}\\A PSTricks package for filling and tiling areas}
%\author{Timothy Van Zandt\thanks{\protect\url{tvz@econ.insead.fr}. (documentation by
% Denis Girou (\protect\url{Denis.Girou@idris.fr}) and Herbert Vo\ss (\protect\url{hvoss@tug.org}).}}
%
%\date{\shortstack{\today --- Version 1.00\\
%                  {\small Documentation revised \today}}}
% \maketitle
% \tableofcontents
%
%\begin{abstract}
%  \FillPackage{} is a PSTricks \cite{vanZandt93},\cite{Girou94},\cite{vanZandtGirou94}, 
%\cite{Hoenig97},\cite{LGC97} package to draw easily
%  various kinds of filling and tiling of areas. It is also a good example of
%  the great power and flexibility of PSTricks, as in fact it is a very short
%  program (it body is around 200~lines long) but nevertheless really powerful.
%
%  \hspace{5mm} It was written in 1994 by Timothy \textsc{van Zandt} but
%  publicly available only in PSTricks 97 and without any documentation.
%  We describe here the version \emph{97 patch 2} of December 12, 1997, which
%  is the original one modified by myself to manage \emph{tilings} in the
%  so-called \emph{automatic} mode. This article would like to serve both of
%  reference manual and of user's guide.
%
%This package is available on \CTAN{} in the
%  \texttt{graphics/pstricks} directory (files \texttt{latex/pst-fill.sty} and
%  \texttt{generic/pst-fill.tex}).
%\end{abstract}
%
%\section{Introduction}
%
%  Here we will refer as \emph{filling} as the operation which consist to fill
%a defined area by a pattern (or a composition of patterns). We will refer as
%\emph{tiling} as the operation which consist to do the same thing, but with
%the control of the starting point, which is here the upper left corner.
%The pattern is positioned relatively to this point. This make an essential
%difference between the two modes, as without control of the starting point we
%can't draw \emph{tilings} (sometimes  called \emph{tesselations}) as used in
%many fields of Art and Science%
%\footnote{For an extensive presentation of tilings, in their history and usage
%in many fields, see the reference book \cite{GS87}.
%
%  In the \TeX{} world, few work was done on tilings. You can look at the
%\emph{tile} extension of the \XYpic{} package \cite{XYpic}, at the articles of
%Kees \textsc{van der Laan} \cite[paragraph 7]{LAAN96} (the tiling was in
%fact directly done in PostScript) and \cite{LAAN97}, at the \MP{} program
%(available on \CTANref{Roegel}) by Denis \textsc{Roegel} for the
%\textsc{Truchet} contest in 1995 \cite{EsperetGirou98} and at the \MP{}
%package \cite{Bolek98} to draw patterns, which have a strong connection with
%tilings.}.
%
%  Nevertheless, as tilings are a wide and difficult field in mathematics, this
%package is limited to simple ones, mainly \emph{monohedral} tilings with one
%prototile (which can be composite, see section \ref{sec:KindTiles}). With some
%experience and wiliness we can do more and obtained easily rather
%sophisticated results, but obviously hyperbolic tilings like the famous
%\textsc{Escher} ones or aperiodic tilings like the \textsc{Penrose} ones are
%not in the capabilities of this package. For more complex needs, we must used
%low level and more painfull technics, with the basic \cs{multido}
%and \cs{multirput} macros.
%
%\section{Package history and description of it two different modes}
%
%  As already said, this package was written in 1994 by Timothy \textsc{van
%Zandt}. Two modes were defined, called respectively \emph{manual} and
%\emph{automatic}. For both, the pattern is generated on contiguous positions in
%a rather large area which include the region to fill, later cut to the
%required dimensions by clipping mechanism. In the first mode, the pattern is
%explicitely inserted in the PostScript file each time. In the second one, the
%result is the same but with an unique explicit insertion of the pattern and a
%repetition done by PostScript. Nevertheless, in this method, the control of
%the starting point was loosed, so it allowed only to \emph{fill} a region and
%not to \emph{tile} it.
%
%  See the difference between the two modes, \emph{tiling}:
% {\psset{unit=0.5cm}%
% \psboxfill{\begin{pspicture}(1,1)\psframe[dimen=middle](1,1)\end{pspicture}}
% \begin{pspicture}(3,3.3)
%   \psframe[fillstyle=boxfill](3,3)
% \end{pspicture}}
% and \emph{filling}:
%{%
% \makeatletter
%\pst@def{BoxFill}<
%  gsave
%    gsave \tx@STV CM grestore dtransform CM idtransform
%    abs /h ED abs /w ED
%    pathbbox
%    h div round 2 add cvi /y2 ED
%    w div round 2 add cvi /x2 ED
%    h div round 2 sub cvi /y1 ED
%    w div round 2 sub cvi /x1 ED
%    /y2 y2 y1 sub def
%    /x2 x2 x1 sub def
%    CP
%    y1 h mul sub neg /y1 ED
%    x1 w mul sub neg /x1 ED
%    clip
%    y2 {
%      /x x1 def
%      x2 {
%        save CP x y1 T moveto Box restore
%        /x x w add def
%      } repeat
%      /y1 y1 h add def
%    } repeat
% currentpoint currentfont grestore setfont moveto>
% \makeatother
%
% \psset{unit=0.5}
% \psboxfill{\begin{pspicture}(1,1)\psframe[dimen=middle](1,1)\end{pspicture}}
% \begin{pspicture}(3,3.3)
%   \psframe[fillstyle=boxfill](3,3)
% \end{pspicture}
% or
% \begin{pspicture}(3,3.3)
%   \psframe[fillstyle=boxfill](3,3)
% \end{pspicture}
%}
%as we can see that initial position is arbitrary and dependent of
%the current point.
%
%
% It's clear that usage of filling is very restrictive comparing to tiling,
%as desired effects required very often the possibility to control the starting 
%point. So, this mode was of limited interest, but unfortunately the
%\emph{manual} one has the very big disadvantage to require very huge amounts
%of ressources, mainly in disk space and consequently in printing time.
%A small tiling can require sometimes several megabytes in \emph{manual} mode!
%So, it was very often not really usable in practice.
%
%It is why I modified the code, to allow tilings in \emph{automatic} mode,
%controlling in this mode too the starting point. And most of the time, that is
%to say if some special options are not used, the tiling is done exactly in the
%region described, which make it faster. So there is no more reason to use the
%\emph{manual} mode, apart very special cases where \emph{automatic} one cannot
%work, as explained later -- currently, I know only one case.
%
%  To load this modified \emph{automatic} mode, with \LaTeX{} use
%simply:\newline 
%\verb+\usepackage[tiling]{pst-fill}+\newline
%and in plain \TeX{} after:\newline
%\verb+\input{pst-fill}+\newline
%add the following definition:\newline
%\verb+\def\PstTiling{true}+
%
%  To obtain the original behaviour, just don't use the \emph{tiling} optional
%keyword at loading.
%
%  Take care than in \emph{tiling} mode, I introduce also some other changes.
%First I define aliases on some parameter names for consistancy (all specific
%parameters will begin by the \texttt{fill} prefix in this case) and I change
%some default values, which were not well adapted for tilings (\texttt{fillsep}
%is set to 0 and as explained \texttt{fillsize} set to \texttt{auto}). I rename 
%\texttt{fillcycle} to \texttt{fillcyclex}. I also restore normal way so that
%the frame of the area is drawn and all line (\texttt{linestyle},
%\texttt{linecolor}, \texttt{doubleline}, etc.) parameters are now active (but
%there are not in non \emph{tiling} mode). And I also introduce new parameters
%to control the tilings (see below).
%
%  \textbf{In all the following examples, we will consider only the
% \emph{tiling} mode.}
%
%  To do a tiling, we have just to define the pattern with the
% \verb+\psboxfill+ macro and to use the new \texttt{fillstyle}
% \verb+boxfill+.
%
%  Note that tilings are drawn from left to right and top to bottom, which can
%have an importance in some circonstances.
%
%  PostScript programmers can be also interested to know that, even in the
%\emph{automatic} mode, the iterations of the pattern are managed directly by
%the PostScript code of the package which used only PostScript Level 1
%operators. The special ones introduced in Level 2 for drawing of patterns
%\cite[section 4.9]{PostScript95} are not used.
%
%  And first, for conveniance, we define a simple \cs{Tiling} macro, which
%will simplify our examples:
%
%\begin{verbatim}
%  \newcommand{\Tiling}[2][]{%
%    \edef\Temp{#1}%
%    \begin{pspicture}#2
%      \ifx\Temp\empty
%        \psframe[fillstyle=boxfill]#2
%      \else
%        \psframe[fillstyle=boxfill,#1]#2
%      \fi
%    \end{pspicture}}
%\end{verbatim}
%
%
%\newcommand{\Tiling}[2][]{%
%  \edef\Temp{#1}%
%  \begin{pspicture}#2
%    \ifx\Temp\empty
%      \psframe[fillstyle=boxfill]#2
%    \else
%      \psframe[fillstyle=boxfill,#1]#2
%    \fi
% \end{pspicture}}
%
%\subsection{Parameters}
%
%  There are \textbf{14} specific parameters available to change the way the
% filling/tiling is defined, and one debugging option.
%
% \begin{Description}{2cm}
%  \item [fillangle (real)\hfill :] the value of the rotation
%  applied to the patterns (\emph{Default:~0}).
% \end{Description}
%
%
%   In this case, we must force the tiling area to be notably larger than the
% area to cover, to be sure that the defined area will be covered after rotation.
% \lstset{gobble=2}
% \begin{LTXexample}
% \newcommand{\Square}{%
%   \begin{pspicture}(1,1)
%     \psframe[dimen=middle](1,1)
%   \end{pspicture}}
% \psset{unit=0.5}
% \psboxfill{\Square}
% \Tiling[fillangle=45]{(3,3)}\quad
% \Tiling[fillangle=-60]{(3,3)}
% \end{LTXexample}
% 
% \newcommand{\Square}{\begin{pspicture}(1,1)\psframe[dimen=middle](1,1)\end{pspicture}}
% 
% \begin{Description}{2cm}
%   \setcounter{footnote}{1}
%   \item[\texttt{fillsepx} (real$\|$dim) :] value of the horizontal
%   separation between consecutive patterns (\emph{Default:~0 for
%   tilings\footnotemark, 2pt otherwise}).  \footnotetext{This option was added
%   by me, is not part of the original package and is available only if the
%   \texttt{tiling} keyword is used when loading the package.}
%   \setcounter{footnote}{1}
%   \item [\texttt{fillsepy} (real$\|$dim)\hfill :] value of the vertical
%   separation between consecutive patterns (\emph{Default:~0 for
%   ti\-lings\footnotemark, 2pt otherwise}).
%   \setcounter{footnote}{1}
%   \item [\texttt{fillsep} (real$\|$dim)\hfill :] value of horizontal and
%   vertical separations between consecutive patterns (\emph{Default:~0 for
%   tilings\footnotemark, 2pt otherwise}).
% \end{Description}
% 
%   These values can be negative, which allow the tiles to overlap.
% 
% \begin{LTXexample}
% \psset{unit=0.5}
% \psboxfill{\Square}
% \Tiling[fillsepx=2mm]{(3,3)} 
% \Tiling[fillsepy=1mm]{(3,3)}\\
% \Tiling[fillsep=0.5]{(3,3)} 
% \Tiling[fillsep=-0.5]{(3,3)}
% \end{LTXexample}
% 
% \begin{Description}{2cm}
%   \item [\texttt{fillcyclex}\footnotemark\ (integer)\hfill :] Shift
%   coefficient applied to each row (\emph{Default:~0}).
%   \footnotetext{It was \texttt{fillcycle} in the original version.}
%   \setcounter{footnote}{1}
%   \item [\texttt{fillcycley}\footnotemark\ (integer)\hfill :] Same thing for
%   columns (\emph{Default:~0}).
%   \setcounter{footnote}{1}
%   \item [\texttt{fillcycle}\footnotemark\ (integer)\hfill :] Allow to fix
%   both \texttt{fillcyclex} and \texttt{fillcycley} directly to the same value
%   (\emph{Default:~0}).
% \end{Description}
% 
%   For instance, if \texttt{fillcyclex} is 2, the second row of patterns will
% be horizontally shifted by a factor of $\frac{1}{2}=0.5$, and by a factor of
% 0.333 if \texttt{fillcyclex} is 3, etc.). These values can be negative.
% 
% \begin{LTXexample}[width=0.35\linewidth]
% \psset{unit=0.5}
% \psboxfill{\Square}
% \newcommand{\TilingA}[1]{\Tiling[fillcyclex=#1]{(3,3)}}
% \TilingA{0} \TilingA{1}\\
% \TilingA{2} \TilingA{3}\\[3mm]
% \TilingA{4} \TilingA{5}\\
% \TilingA{6} \TilingA{-3}\\[3mm]
% \Tiling[fillcycley=2]{(3,3)}
% \Tiling[fillcycley=3]{(3,3)}\\
% \Tiling[fillcycley=-3]{(3,3)}
% \Tiling[fillcycle=2]{(3,3)}
% \end{LTXexample}
% 
% \begin{Description}{2cm}
%   \setcounter{footnote}{1}
%   \item [\texttt{fillmovex}\footnotemark\ (real$\|$dim)\hfill :] value of the
%   horizontal moves between consecutive patterns (\emph{Default:~0}).
%   \setcounter{footnote}{1}
%   \item [\texttt{fillmovey}\footnotemark\ (real$\|$dim)\hfill :] value of the
%   vertical moves between consecutive patterns (\emph{Default:~0}).
%   \setcounter{footnote}{1}
%   \item [\texttt{fillmove}\footnotemark\ (real$\|$dim)\hfill :] value of
%   horizontal and vertical moves between consecutive patterns
%   (\emph{Default:~0}).
% \end{Description}
% 
%   These parameters allow the patterns to overlap and to draw some special
% kinds of tilings. They are implemented only for the \emph{automatic} and
% \emph{tiling} modes and their values can be negative.
% 
%   In some cases, the effect of these parameters will be the same that with the 
% \texttt{fillcycle?} ones, but you can see that it is not true for some other
% values.
% 
% \begin{LTXexample}
% \psset{unit=0.5}
% \psboxfill{\Square}
% \Tiling[fillmovex=0.5]{(3,3)} 
% \Tiling[fillmovey=0.5]{(3,3)}\\
% \Tiling[fillmove=0.5]{(3,3)}
% \Tiling[fillmove=-0.5]{(3,3)}
% \end{LTXexample}
% 
% \begin{Description}{2cm}
%   \item [\texttt{fillsize}
%   (auto$\|$\{(real$\|$dim,real$\|$dim)(real$\|$dim,real$\|$dim)\}) :] The
%   choice of \emph{automatic} mode or the size of the area in \emph{manual}
%   mode. If first pair values are not given, (0,0) is used. (\emph{Default:
%   auto when \emph{tiling} mode is used, {(-15cm,-15cm)(15cm,15cm)}
%   otherwise}).
% \end{Description}
% 
%   As explained in the introduction, the \emph{manual} mode can require very
% huge amount of computer ressources. So, it usage is to discourage in front off
% the \emph{automatic} mode. It seems only useful in special circonstances, in
% fact when the \emph{automatic} mode failed, which is known only in one case,
% for some kinds of EPS files, as the ones produce by dump of portions of
% screens (see \ref{sec:GraphicFiles}).
% 
% \begin{Description}{2cm}
%   \setcounter{footnote}{1}
%   \item [\texttt{fillloopaddx}\footnotemark\ (integer)\hfill :] number of
%   times the pattern is added on left and right positions (\emph{Default:~0}).
%   \setcounter{footnote}{1}
%   \item [\texttt{fillloopaddy}\footnotemark\ (integer)\hfill :] number of
%   times the pattern is added on top and bottom positions (\emph{Default:~0}).
%   \setcounter{footnote}{1}
%   \item [\texttt{fillloopadd}\footnotemark\ (integer)\hfill :] number of
%   times the pattern is added on left, right, top and bottom positions
%   (\emph{Default:~0}).
% \end{Description}
% 
%   These parameters are only useful in special circonstances, as for complex
% patterns when the size of the rectangular box used to tile the area doesn't 
% correspond to the pattern itself (see an example in Figure~\ref{fig:Sheeps})
% and also sometimes when the size of the pattern is not a divisor of the size
% of the area to fill and that the number of loop repeats is not properly
% computed, which can occur.
% 
%   They are implemented only for the \emph{tiling} mode.
% 
% \begin{Description}{2cm}
%   \setcounter{footnote}{1}
%   \item [\texttt{PstDebug}\footnotemark\ (integer, 0 or 1)\hfill :] to
%   require to see the exact tiling done, without clipping (\emph{Default:~0}).
% \end{Description}
% 
%   It's mainly useful for debugging or to understand better how the tilings
% are done. It is implemented only for the \emph{tiling} mode.
% 
% \begin{LTXexample}
% \psset{unit=0.3,PstDebug=1}
% \psboxfill{\Square}
% \psset{linewidth=1mm}
% \Tiling{(2,2)}\\[5mm]
% \Tiling[fillcyclex=2]{(2,2)}\\[1cm]
% \Tiling[fillmove=0.5]{(2,2)}
% \end{LTXexample}
% 
% \vspace{3cm}
% \section{Examples}
% 
%   In fact this unique \cs{psboxfill} macro allow a lot a variations and
% different usages. We will try here to demonstrate this.
% 
% \subsection{Kind of tiles}
% \label{sec:KindTiles}
% 
%   Of course, we can access to all the power of PSTricks macros to define the
% \emph{tiles} (\emph{patterns}) used. So, we can define complicated ones.
% 
%   Here we give four other Archimedian tilings (those built with only some
% regular polygons) among the twelve existing, first discovered completely by
% Johanes \textsc{Kepler} at the beginning of 17th century \cite{GS87}, the two
% other \emph{regular} ones with the tiling by squares, formed by a unique
% regular polygon, and two other formed by two different regular polygons.
% 
% \begin{LTXexample}[pos=t]
%   \newcommand{\Triangle}{%
%     \begin{pspicture}(1,1)
%       \pstriangle[dimen=middle](0.5,0)(1,1)
%     \end{pspicture}}
%   \newcommand{\Hexagon}{
% ^^A sin(60)=0.866
%     \begin{pspicture}(0.866,0.75)
%       \SpecialCoor
% ^^A  Hexagon  
%       \pspolygon[dimen=middle]%
%         (0.5;30)(0.5;90)(0.5;150)(0.5;210)(0.5;270)(0.5;330)
%     \end{pspicture}}
% 
%   \psset{unit=0.5}
%   \psboxfill{\Triangle}
%   \Tiling{(4,4)}\hfill
% ^^A The two other regular tilings
%   \Tiling[fillcyclex=2]{(4,4)}\hfill
%   \psboxfill{\Hexagon}
%   \Tiling[fillcyclex=2,fillloopaddy=1]{(5,5)}
% \end{LTXexample}
% 
% \begin{LTXexample}[pos=t]
%   \newcommand{\ArchimedianA}{%
%      ^^A Archimedian tiling 3^2.4.3.4
%     \psset{dimen=middle}
%      ^^A sin(60)=0.866
%     \begin{pspicture}(1.866,1.866)
%       \psframe(1,1)
%       \psline(1,0)(1.866,0.5)(1,1)(0.5,1.866)(0,1)(-0.866,0.5)
%       \psline(0,0)(0.5,-0.866)
%     \end{pspicture}}
%   \newcommand{\ArchimedianB}{%
%      ^^A Archimedian tiling 4.8^2
%     \psset{dimen=middle,unit=1.5}
%      ^^A sin(22.5)=0.3827 ; cos(22.5)=0.9239
%     \begin{pspicture}(1.3066,0.6533)
%       \SpecialCoor
%      ^^A Octogon
%       \pspolygon(0.5;22.5)(0.5;67.5)(0.5;112.5)(0.5;157.5)
%                 (0.5;202.5)(0.5;247.5)(0.5;292.5)(0.5;337.5)
%     \end{pspicture}}
% 
%   \psset{unit=0.5}
%   \psboxfill{\ArchimedianA}
%   \Tiling[fillmove=0.5]{(7,7)}\hfill
%   \psboxfill{\ArchimedianB}
%   \Tiling[fillcyclex=2,fillloopaddy=1]{(7,7)}
% \end{LTXexample}
% 
%   \setcounter{footnote}{3}
%   We can of course tile an area arbitrarily defined. And with the
% \texttt{addfillstyle} parameter\footnote{Introduced in PSTricks 97.}, we can
% easily mix the \texttt{boxfill} style with another one.
% 
% \begin{LTXexample}[width=6cm]
%   \psset{unit=0.5,dimen=middle}
%   \psboxfill{%
%     \begin{pspicture}(1,1)
%       \psframe(1,1)
%       \pscircle(0.5,0.5){0.25}
%     \end{pspicture}}
%   \begin{pspicture}(4,6)
%     \pspolygon[fillstyle=boxfill,fillsep=0.25](0,1)(1,4)(4,6)(4,0)(2,1)
%   \end{pspicture}\hspace{1em}
%   \begin{pspicture}(4,4)
%%     \pscircle[linestyle=none,fillstyle=solid,fillcolor=yellow,fillsep=0.5,
%%               addfillstyle=boxfill](2,2){2}
%   \end{pspicture}
% \end{LTXexample}
%
%   Various effects can be obtained, sometimes complicated ones very easily, as
% in this example reproduced from one shown by Slavik \textsc{Jablan} in the
% field of \emph{OpTiles}, inspired by the \emph{Op-art}:
% 
% \begin{LTXexample}[pos=t]
% \newcommand{\ProtoTile}{%
%  \begin{pspicture}(1,1)%%% 1/12=0.08333
%   \psset{linestyle=none,linewidth=0,
%     hatchwidth=0.08333\psunit,hatchsep=0.08333\psunit}
%   \psframe[fillstyle=solid,fillcolor=black,addfillstyle=hlines,hatchcolor=white](1,1)
%   \pswedge[fillstyle=solid,fillcolor=white,addfillstyle=hlines]{1}{0}{90}
%  \end{pspicture}}
% \newcommand{\BasicTile}{%
%  \begin{pspicture}(2,1)
%    \rput[lb](0,0){\ProtoTile}\rput[lb](1,0){\psrotateleft{\ProtoTile}}
%  \end{pspicture}}
% \ProtoTile\hfill\BasicTile\hfill
% \psboxfill{\BasicTile}
% \Tiling[fillcyclex=2]{(4,4)}
% \end{LTXexample}
% 
%   It is also directly possible to surimpose several different tilings. Here is
% the splendid visual proof of the \textsc{Pytha\-gore} theorem done by the arab
% mathematician \textsc{Annairizi} around the year 900, given by superposition
% of two tilings by squares of different sizes.
% 
% \begin{LTXexample}[pos=t]
% \psset{unit=1.5,dimen=middle}
% \begin{pspicture*}(3,3)
%   \psboxfill{\begin{pspicture}(1,1)
%     \psframe(1,1)\end{pspicture}}
%   \psframe[fillstyle=boxfill](3,3)
%   \psboxfill{\begin{pspicture}(1,1)
%     \rput{-37}{\psframe[linecolor=red](0.8,0.8)}
%   \end{pspicture}}
%   \psframe[fillstyle=boxfill](3,4)
%   \pspolygon[fillstyle=hlines,hatchangle=90](1,2)(1.64,1.53)(2,2)
% \end{pspicture*}
% \end{LTXexample}
% 
%   In a same way, it is possible to build tilings based on figurative patterns,
% in the style of the famous \textsc{Escher} ones. Following an example of
% Andr\'e \textsc{Deledicq} \cite{Deledicq97}, we first show a simple tiling of
% the \emph{p1} category (according to the international classification of the
% 17~symmetry groups of the plane first discovered by the russian
% crystalographer Jevgraf \textsc{Fedorov} at the end of the 19th century):
% 
% \begin{LTXexample}[pos=t]
%  \newcommand{\SheepHead}[1]{%
%    \begin{pspicture}(3,1.5)
%      \pscustom[liftpen=2,fillstyle=solid,fillcolor=#1]{%
%        \pscurve(0.5,-0.2)(0.6,0.5)(0.2,1.3)(0,1.5)(0,1.5)
%          (0.4,1.3)(0.8,1.5)(2.2,1.9)(3,1.5)(3,1.5)(3.2,1.3)
%          (3.6,0.5)(3.4,-0.3)(3,0)(2.2,0.4)(0.5,-0.2)}
%      \pscircle*(2.65,1.25){0.12\psunit} % Eye
%      \psccurve*(3.5,0.3)(3.35,0.45)(3.5,0.6)(3.6,0.4)% Muzzle
%     ^^A   % Mouth
%       \pscurve(3,0.35)(3.3,0.1)(3.6,0.05)
%     ^^A   % Ear
%       \pscurve(2.3,1.3)(2.1,1.5)(2.15,1.7)\pscurve(2.1,1.7)(2.35,1.6)(2.45,1.4)
%   \end{pspicture}}
%  \psboxfill{\psset{unit=0.5}\SheepHead{yellow}\SheepHead{cyan}}
%  \Tiling[fillcyclex=2,fillloopadd=1]{(10,5)}
% \end{LTXexample}
% \label{fig:Sheeps}
% 
%   Now a tiling of the \emph{pg} category (the code for the kangaroo itself is
% too long to be shown here, but has no difficulties ; the kangaroo is reproduce
% from an original picture from Raoul \textsc{Raba} and here is a translation in
% PSTricks from the one drawn by Emmanuel \textsc{Chailloux} and Guy
% \textsc{Cousineau} for their MLgraph system \cite{MLgraphTSI}):
% 
% \begin{LTXexample}[pos=t]
% \psboxfill{\psset{unit=0.4}
%   \Kangaroo{yellow}\Kangaroo{red}\Kangaroo{cyan}\Kangaroo{green}%
%   \psscalebox{-1 1}{%
%     \rput(1.235,4.8){\Kangaroo{green}\Kangaroo{cyan}\Kangaroo{red}\Kangaroo{yellow}}}}
%   \Tiling[fillloopadd=1]{(10,6)}
% \end{LTXexample}
% 
%   And here a \textsc{Wang} tiling \cite{Wang65}, \cite[chapter
% 11]{GS87}, based on very simple tiles of the form of a square and composed
% of four colored triangles. Such tilings are built with only a matching color
% constraint. Despite of it simplicity, it is an important kind of tilings, as
% \textsc{Wang} and others used them to study the special class of
% \emph{aperiodic} tilings, and also because it was shown that surprisingly this 
% tiling is similar to a \textsc{Turing} machine.
% 
% \begin{LTXexample}[pos=t]
%   \newcommand{\WangTile}[4]{%
%     \begin{pspicture}(1,1)
%       \pspolygon*[linecolor=#1](0,0)(0,1)(0.5,0.5)
%       \pspolygon*[linecolor=#2](0,1)(1,1)(0.5,0.5)
%       \pspolygon*[linecolor=#3](1,1)(1,0)(0.5,0.5)
%       \pspolygon*[linecolor=#4](1,0)(0,0)(0.5,0.5)
%     \end{pspicture}}
%   \newcommand{\WangTileA}{\WangTile{cyan}{yellow}{cyan}{cyan}}
%   \newcommand{\WangTileB}{\WangTile{yellow}{cyan}{cyan}{red}}
%   \newcommand{\WangTileC}{\WangTile{cyan}{red}{yellow}{yellow}}
%   \newcommand{\WangTiles}[1][]{%
%     \begin{pspicture}(3,3) \psset{ref=lb}
%       \rput(0,2){\WangTileB}  \rput(1,2){\WangTileA}%
%       \rput(2,2){\WangTileC}  \rput(0,1){\WangTileC}%
%       \rput(1,1){\WangTileB}  \rput(2,1){\WangTileA}
%       \rput(0,0){\WangTileA}  \rput(1,0){\WangTileC}%
%       \rput(2,0){\WangTileB}
%       #1
%     \end{pspicture}}
%   \WangTileA\hfill\WangTileB\hfill\WangTileC\hfill
%   \WangTiles[{\psgrid[subgriddiv=0,gridlabels=0](3,3)}]\hfill
%   \psset{unit=0.4} \psboxfill{\WangTiles} \Tiling{(12,12)}
% \end{LTXexample}
% 
% \subsection{External graphic files}
% \label{sec:GraphicFiles}
% 
%   We can also fill an arbitrary area with an external image. We have only, 
% as usual, to matter of the \emph{BoundingBox} definition if there is no one
% provided or if it is not the accurate one, as for the well known
% \texttt{tiger} picture part of the \texttt{ghostscript} distribution.
% 
% \begin{LTXexample}[pos=t]
%   \psboxfill{%% Strangely require x1=x2...
%     \begin{pspicture}(0,1)(0,4.1)
%       \includegraphics[bb=17 176 560 74,width=3cm]{tiger}
%     \end{pspicture}}
%   \Tiling{(6,6.2)}
% \end{LTXexample}
% 
%   Nevertheless, there are some special files for which the \emph{automatic}
% mode doesn't work, specially for some files obtained by a screen dump, as in
% the next example, where a picture was reduced before it conversion in the
% \emph{Encapsulated PostScript} format by a screen dump utility. In this case,
% usage of the \emph{manual} mode is the only alternative, at the price of the
% real multiple inclusion of the EPS file. We must take care to specify the
% correct \texttt{fillsize} parameter, because otherwise the default values are
% large and will load the file many times, perhaps just really using few
% occurrences as the other ones would be clipped...
% 
% \begin{LTXexample}[pos=t]
%   \psboxfill{\includegraphics{flowers}}
%   \begin{pspicture}(8,4)
%     \psellipse[fillstyle=boxfill,fillsize={(8,4)}](4,2)(4,2)
%   \end{pspicture}
% \end{LTXexample}
% 
% \subsection{Tiling of characters}
% 
%   We can also use the \cs{psboxfill} macro to fill the interior of characters
% for special effects like these ones:
% 
% \begin{LTXexample}[pos=t]
%   \DeclareFixedFont{\bigsf}{T1}{phv}{b}{n}{4.5cm}
%   \DeclareFixedFont{\smallrm}{T1}{ptm}{m}{n}{3mm}
%   \psboxfill{\smallrm Since 182 days...}
%   \begin{pspicture*}(8,4)
%     \centerline{%
%       \pscharpath[fillstyle=gradient,gradangle=-45,
%                   gradmidpoint=0.5,addfillstyle=boxfill,
%                   fillangle=45,fillsep=0.7mm]
%                  {\rput[b](0,0.1){\bigsf 2000}}}
%   \end{pspicture*}
% \end{LTXexample}
% 
% \begin{LTXexample}[pos=t]
%   \DeclareFixedFont{\mediumrm}{T1}{ptm}{m}{n}{2cm}
%   \psboxfill{%
%     \psset{unit=0.1,linewidth=0.2pt}
%     \Kangaroo{PeachPuff}\Kangaroo{PaleGreen}%
%       \Kangaroo{LightBlue}\Kangaroo{LemonChiffon}%
%     \psscalebox{-1 1}{%
%       \rput(1.235,4.8){%
%         \Kangaroo{LemonChiffon}\Kangaroo{LightBlue}%
%           \Kangaroo{PaleGreen}\Kangaroo{PeachPuff}}}}
% ^^A   % A kangaroo of kangaroos...
%   \begin{pspicture}(8,2)
%     \pscharpath[linestyle=none,fillstyle=boxfill,fillloopadd=1]
%                {\rput[b](4,0){\mediumrm Kangaroo}}
%   \end{pspicture}
% \end{LTXexample}
% 
% \subsection{Other kinds of usage}
% 
%   Other kinds of usage can be imagined. For instance, we can use tilings in a
% sort of degenerated way to draw some special lines made by a unique or
% multiple repeating patterns. But it can be only a special dashed line, as here
% with three different dashes:
% 
% \begin{LTXexample}[pos=t]
%   \newcommand{\Dashes}{%
%     \psset{dimen=middle}
%     \begin{pspicture}(0,-0.5\pslinewidth)(1,0.5\pslinewidth)
%       \rput(0,0){\psline(0.4,0)}%
%         \rput(0.5,0){\psline(0.2,0)}%
%         \rput(0.8,0){\psline(0.1,0)}
%     \end{pspicture}}
% 
%   \newcommand{\SpecialDashedLine}[3]{%
%     \psboxfill{#3}
%     \Tiling[linestyle=none]
%            {(#1,-0.5\pslinewidth)(#2,0.5\pslinewidth)}}
% 
%   \SpecialDashedLine{0}{7}{\Dashes}
% 
%   \psset{unit=0.5,linewidth=1mm,linecolor=red}
%   \SpecialDashedLine{0}{10}{\Dashes}
% \end{LTXexample}
% 
%   It allow also to use special patterns in business graphics, as in the
% following example generated by \texttt{PstChart}\footnote{A personal
% development to draw business charts with PSTricks, not distributed.}.
% 
% \vspace{3mm}
% \begin{figure}[!ht]
% \centering
% \psset{unit=0.75}
% ^^A % Generated by pstchart.sh version 0.21 (11/28/97)
% {\psset{dimen=middle}
% \psset{xunit=2,yunit=0.005}
% \begin{pspicture}(-0.6,-200)(6.6,2300)
% ^^A   % Title
%   \rput(3,2200){\shortstack{Fantaisist repartition of kangaroos\\
%                             in the world (in thousands)}}
% ^^A   % Frame background
%   \psframe[fillstyle=solid,fillcolor=LemonChiffon](0,0)(6,2000)
% ^^A   % Graduations
%   \multido{\n=0+500}{5}{\rput[r](-0.12,\n){\psscalebox{0.8}{\n}}}
% ^^A   % Minor ticks
%   \multips(0,100)(0,100){19}{\psline[unit=4.8pt](1,0)}
%   \multips(6,100)(0,100){19}{\psline[unit=4.8pt](-1,0)}
% ^^A   % Major ticks
%   \multips(0,500)(0,500){3}{\psline[unit=9.6pt](1,0)}
%   \multips(6,500)(0,500){3}{\psline[unit=9.6pt](-1,0)}
% ^^A   % Lines from major ticks marks
%   \multips(0,500)(0,500){3}{\psline[linestyle=dotted,linewidth=0.6pt](6,0)}
% ^^A   % Drawing for the data
%   \psboxfill{\psset{unit=0.78\psxunit}\KangarooPstChart{red}}
%   \psframe[linestyle=none,fillstyle=boxfill,fillloopaddy=1](0.61,0)(1.39,1800)
%   \psboxfill{\psset{unit=0.78\psxunit}\KangarooPstChart{yellow}}
%   \psframe[linestyle=none,fillstyle=boxfill,fillloopaddy=1](1.61,0)(2.39,800)
%   \psboxfill{\psset{unit=0.78\psxunit}\KangarooPstChart{cyan}}
%   \psframe[linestyle=none,fillstyle=boxfill,fillloopaddy=1](2.61,0)(3.39,550)
%   \psboxfill{\psset{unit=0.78\psxunit}\KangarooPstChart{magenta}}
%   \psframe[linestyle=none,fillstyle=boxfill,fillloopaddy=1](3.61,0)(4.39,500)
%   \psboxfill{\psset{unit=0.78\psxunit}\KangarooPstChart{green}}
%   \psframe[linestyle=none,fillstyle=boxfill,fillloopaddy=1](4.61,0)(5.39,200)
% ^^A   % Bottom labels
%   \uput{0.2}[270]{0}(1,0){\psscalebox{0.7}{Oceania}}
%   \uput{0.2}[270]{0}(2,0){\psscalebox{0.7}{Africa}}
%   \uput{0.2}[270]{0}(3,0){\psscalebox{0.7}{Asia}}
%   \uput{0.2}[270]{0}(4,0){\psscalebox{0.7}{America}}
%   \uput{0.2}[270]{0}(5,0){\psscalebox{0.7}{Europe}}
% ^^A   % Frame box around the chart
%   \psframe[linestyle=solid](0,0)(6,2000)
% \end{pspicture}}
%   \caption{Bar chart generated by PstChart, with bars filled by patterns}
%   \label{fig:PstChart}
% \end{figure}
% 
% \section{``Dynamic'' tilings}
% 
%   In some cases, tilings used non \emph{static} tiles, that is to say that the 
% \emph{prototile(s)}, even if unique, can have several forms, by instance
% specified by different colors or rotations, not fixed before generation or
% varying each time.
% 
% \subsection{Lewthwaite-Pickover-Truchet tiling}
% 
%   We give here for example the so-called \emph{Truchet} tiling, which much be
% in fact better called \emph{Lewthwaite-Pick\-over-Truchet (LPT)} tiling%
% \footnote{For description of the context, history and references about
% S\'ebastien \textsc{Truchet} and this tiling, see \cite{EsperetGirou98}.}.
% 
%   The unique prototile is only a square with two opposite circle arcs.
% This tile has obviously two positions, if we rotate it from 90 degrees (see
% the two tiles on the next figure). A \emph{LPT tiling} is a tiling with
% randomly oriented LPT tiles. We can see that even if it is very simple in it
% principle, it draw sophisticated curves with strange properties.
% 
%   Nevertheless, in the straightforward way \FillPackage{} does not work,
% because the \cs{psboxfill} macro store the content of the tile used in a
% \TeX{} box, which is static. So the calling to the random function is done
% only one time, which explain that only one rotation of the tile is used for
% all the tiling. It's only the one of the two rotations which could differ from
% one drawing to the next one...
% 
% ^^A % Truchet (Lewthwaite-Pickover-Truchet) tiling
% ^^A % --------------------------------------------
% 
% \begin{LTXexample}[pos=t]
% ^^A   % LPT prototile
%   \newcommand{\ProtoTileLPT}{%
%     \psset{dimen=middle}
%     \begin{pspicture}(1,1)
%       \psframe(1,1)
%       \psarc(0,0){0.5}{0}{90}
%       \psarc(1,1){0.5}{-180}{-90}
%     \end{pspicture}}
% 
% ^^A   % LPT tile
%   \newcount\Boolean
%   \newcommand{\BasicTileLPT}{%
% ^^A     % From random.tex by Donald Arseneau
%     \setrannum{\Boolean}{0}{1}%
%     \ifnum\Boolean=0
%       \ProtoTileLPT%
%     \else
%       \psrotateleft{\ProtoTileLPT}%
%     \fi}
% 
%   \ProtoTileLPT\hfill\psrotateleft{\ProtoTileLPT}\hfill
%   \psset{unit=0.5}
%   \psboxfill{\BasicTileLPT}
%   \Tiling{(5,5)}
% \end{LTXexample}
% 
%   But, for simple cases, there is a solution to this problem using a mixture
% of PSTricks and PostScript programming. Here the PSTricks
% construction \verb+\pscustom{\code{...}}+ allow to insert PostScript code
% inside the \LaTeX{} + PSTricks one.
% 
%   Programmation is less straightforward, but it has also the advantage to be
% notably faster, as all the tilings operations are done in PostScript, and
% mainly to not be limited by \TeX{} memory (the \TeX{} + PSTricks solution
% I wrote in 1995 for the colored problem was limited to small sizes for this
% reason). Just note also that \cs{pslbrace} and \cs{psrbrace} are two
% PSTricks macros to define and be able to insert the \verb+{+ and \verb+}+
% characters.
% 
% \begin{LTXexample}[pos=t]
% ^^A   % LPT prototile
%   \newcommand{\ProtoTileLPT}{%
%     \psset{dimen=middle}
%     \psframe(1,1)
%     \psarc(0,0){0.5}{0}{90}
%     \psarc(1,1){0.5}{-180}{-90}}
% 
% ^^A   % Counter to change the random seed
%   \newcount\InitCounter
% ^^A   % LPT tile
%   \newcommand{\BasicTileLPT}{%
%     \InitCounter=\the\time
%     \pscustom{\code{%
%       rand \the\InitCounter\space sub 2 mod 0 eq \pslbrace}}
%     \begin{pspicture}(1,1)
%       \ProtoTileLPT
%     \end{pspicture}%
%     \pscustom{\code{\psrbrace \pslbrace}}
%     \psrotateleft{\ProtoTileLPT}%
%     \pscustom{\code{\psrbrace ifelse}}}
% 
%   \psset{unit=0.4,linewidth=0.4pt}
%   \psboxfill{\BasicTileLPT}
%   \Tiling{(15,15)}
% \end{LTXexample}
% 
%   Using the very surprising fact (see \cite{EsperetGirou98}) that
% coloration of these tiles do not depend of their neighbors (even if it is
% difficult to believe as the opposite seems obvious!) but only of the parity of
% the value of row and column positions, we can directly program in the same way
% a colored version of the LPT tiling.
% 
% \setcounter{footnote}{1}
%   We have also introduce in the \FillPackage{} code for \emph{tiling} mode two
% new accessible Post\-Script variables, \texttt{row} and
% \texttt{column}\footnotemark, which can be useful in some circonstances, like
% this one.
% 
% \begin{LTXexample}[pos=t]
% ^^A   % LPT prototile
%   \newcommand{\ProtoTileLPT}[2]{%
%     \psset{dimen=middle,linestyle=none,fillstyle=solid}
%     \psframe[fillcolor=#1](1,1)
%     \psset{fillcolor=#2}
%     \pswedge(0,0){0.5}{0}{90} \pswedge(1,1){0.5}{-180}{-90}}
% ^^A   % Counter to change the random seed
%   \newcount\InitCounter
% ^^A   % LPT tile
%   \newcommand{\BasicTileLPT}[2]{%
%     \InitCounter=\the\time
%     \pscustom{\code{%
%       rand \the\InitCounter\space sub 2 mod 0 eq \pslbrace
%       row column add 2 mod 0 eq \pslbrace}}
%     \begin{pspicture}(1,1)\ProtoTileLPT{#1}{#2}\end{pspicture}%
%     \pscustom{\code{\psrbrace \pslbrace}}
%     \ProtoTileLPT{#2}{#1}%
%     \pscustom{\code{%
%       \psrbrace ifelse \psrbrace \pslbrace row column add 2 mod 0 eq \pslbrace}}
%     \psrotateleft{\ProtoTileLPT{#2}{#1}}\pscustom{\code{\psrbrace \pslbrace}}
%     \psrotateleft{\ProtoTileLPT{#1}{#2}}\pscustom{\code{\psrbrace ifelse \psrbrace ifelse}}}
%   \psboxfill{\BasicTileLPT{red}{yellow}}
%   \Tiling{(4,4)}\hfill
%   \psset{unit=0.4}\psboxfill{\BasicTileLPT{blue}{cyan}}
%   \Tiling{(15,15)}
% \end{LTXexample}
% 
%   Another classic example is to generate coordinates and numerotation for a
% grid. Of course, it is possible to do it directly in PSTricks using nested
% \cs{multido} commands. It would be clearly easy to program, but, nevertheless, 
% for users who have a little knowledge of PostScript programming, this offer
% an alternative which is useful for large cases, because on this way it will
% be notably faster and less computer ressources consuming.
% 
%   Remember here that the tiling is drawn from left to right, and top to
% bottom, and note that the PostScript variable \texttt{x2} give the total
% number of columns.
% 
% \begin{LTXexample}[pos=t]
% ^^A   % \Escape will be the \ character
%   {\catcode`\!=0\catcode`\\=11!gdef!Escape{\}}
%   \newcommand{\ProtoTile}{%
%     \Square\pscustom{%
%       \moveto(-0.9,0.75) % In PSTricks units
%       \code{ /Times-Italic findfont 8 scalefont setfont
%         (\Escape() show row 3 string cvs show (,) show 
%         column 3 string cvs show (\Escape)) show}
%       \moveto(-0.5,0.25) % In PSTricks units
%       \code{ /Times-Bold findfont 18 scalefont setfont
%         1 0 0 setrgbcolor % Red color
%         /center {dup stringwidth pop 2 div neg 0 rmoveto} def
%         row 1 sub x2 mul column add 3 string cvs center show}}}
%   \psboxfill{\ProtoTile}
%   \Tiling{(6,4)}
% \end{LTXexample}
% 
% \subsection{A complete example: the Poisson equation}
% 
%   To finish, we will show a complete real example, a drawing to explain the
% method used to solve the \textsc{Poisson} equation by a domain
% decomposition method, adapted to distributed memory computers. The
% objective is to show the communications required between processes and the
% position of the data to exchange. This code also show some useful and powerful
% technics for PSTricks programming (look specially at the way some higher level
% macros are defined, and how the same object is used to draw the four
% neighbors).
%
%\psset{unit=1cm}
%\newcommand{\Pattern}[1]{%
%   \begin{pspicture}(-0.25,-0.25)(0.25,0.25)\rput{*0}{\psdot[dotstyle=#1]}
%   \end{pspicture}}
%\newcommand{\West}{\Pattern{o}}   \newcommand{\South}{\Pattern{x}}
%\newcommand{\Central}{\Pattern{+}}\newcommand{\North}{\Pattern{square}}
%\newcommand{\East}{\Pattern{triangle}}
%\newcommand{\Cross}{%
%  \pspolygon[unit=0.5,linewidth=0.2,linecolor=red](0,0)(0,1)(1,1)(1,2)(2,2)(2,1)%
%              (3,1)(3,0)(2,0)(2,-1)(1,-1)(1,0)}
%\newcommand{\StylePosition}[1]{\LARGE\textcolor{red}{\textbf{#1}}}
%\newcommand{\SubDomain}[4]{%
%    \psboxfill{#4}\begin{psclip}{\psframe[linestyle=none]#1}%
%      \psframe[linestyle=#3](5,5)\psframe[fillstyle=boxfill]#2%
%    \end{psclip}}
%\newcommand{\SendArea}[1]{\psframe[fillstyle=solid,fillcolor=cyan]#1}
%\newcommand{\ReceiveData}[2]{%
%  \psboxfill{#2}\psframe[fillstyle=solid,fillcolor=yellow,addfillstyle=boxfill]#1}%
%\newcommand{\Neighbor}[2]{%
%    \begin{pspicture}(5,5)
%      \rput{*0}(2.5,2.5){\StylePosition{#1}}
%      \ReceiveData{(0.5,0)(4.5,0.5)}{\Central}\SendArea{(0.5,0.5)(4.5,1)}%
%      \SubDomain{(5,2)}{(0.5,0.5)(4.5,3)}{dashed}{#2}%
%      \pcarc[arcangle=45,arrows=->](0.5,-1.25)(0.5,0.25)%
%      \pcarc[arcangle=45,arrows=->,linestyle=dotted,dotsep=2pt](4.5,0.75)(4.5,-0.75)%
%    \end{pspicture}}%
%  \psset{dimen=middle,dotscale=2,fillloopadd=2}
%\begin{pspicture}(-5.7,-5.7)(5.7,5.7)
%  \rput(0,0){%
%      \begin{pspicture}(5,5)
%        \ReceiveData{(0,0.5)(0.5,4.5)}{\West} \ReceiveData{(4.5,0.5)(5,4.5)}{\East}
%        \ReceiveData{(0.5,4.5)(4.5,5)}{\North}\ReceiveData{(0.5,0)(4.5,0.5)}{\South}
%        \SendArea{(0.5,0.5)(1,4.5)}\SendArea{(4,0.5)(4.5,4.5)}
%        \SendArea{(0.5,0.5)(4.5,1)}\SendArea{(0.5,4)(4.5,4.5)}
%        \SubDomain{(5,5)}{(0.5,0.5)(4.5,4.5)}{solid}{\Central}
%        \psline(1,0.5)(1,4.5)\psline(4,0.5)(4,4.5)%
%        \rput(1.5,4){\Cross}\rput(2,2){\Cross}%
%      \end{pspicture}}%
%  \rput(0,5.5){\Neighbor{N}{\North}}\rput{-90}(5.5,0){\Neighbor{E}{\East}}%
%  \rput{90}(-5.5,0){\Neighbor{W}{\West}}\rput{180}(0,-5.5){\Neighbor{S}{\South}}%
%\end{pspicture}
%
% \begin{lstlisting}
%   \newcommand{\Pattern}[1]{%
%     \begin{pspicture}(-0.25,-0.25)(0.25,0.25)\rput{*0}{\psdot[dotstyle=#1]}
%     \end{pspicture}}
%   \newcommand{\West}{\Pattern{o}}   \newcommand{\South}{\Pattern{x}}
%   \newcommand{\Central}{\Pattern{+}}\newcommand{\North}{\Pattern{square}}
%   \newcommand{\East}{\Pattern{triangle}}
%   \newcommand{\Cross}{%
%     \pspolygon[unit=0.5,linewidth=0.2,linecolor=red](0,0)(0,1)(1,1)(1,2)(2,2)(2,1)
%               (3,1)(3,0)(2,0)(2,-1)(1,-1)(1,0)}
%   \newcommand{\StylePosition}[1]{\LARGE\textcolor{red}{\textbf{#1}}}
%   \newcommand{\SubDomain}[4]{%
%     \psboxfill{#4}
%     \begin{psclip}{\psframe[linestyle=none]#1}
%       \psframe[linestyle=#3](5,5)\psframe[fillstyle=boxfill]#2
%     \end{psclip}}
%   \newcommand{\SendArea}[1]{\psframe[fillstyle=solid,fillcolor=cyan]#1}
%   \newcommand{\ReceiveData}[2]{%
%     \psboxfill{#2}
%     \psframe[fillstyle=solid,fillcolor=yellow,addfillstyle=boxfill]#1}
%   \newcommand{\Neighbor}[2]{%
%     \begin{pspicture}(5,5)
%       \rput{*0}(2.5,2.5){\StylePosition{#1}}
%       \ReceiveData{(0.5,0)(4.5,0.5)}{\Central}\SendArea{(0.5,0.5)(4.5,1)}
%       \SubDomain{(5,2)}{(0.5,0.5)(4.5,3)}{dashed}{#2}%
% ^^A       % Receive and send arrows
%       \pcarc[arcangle=45,arrows=->](0.5,-1.25)(0.5,0.25)
%       \pcarc[arcangle=45,arrows=->,linestyle=dotted,dotsep=2pt](4.5,0.75)(4.5,-0.75)
%     \end{pspicture}}
%   \psset{dimen=middle,dotscale=2,fillloopadd=2}
%   \begin{pspicture}(-5.7,-5.7)(5.7,5.7)
% ^^A     % Central domain
%     \rput(0,0){%
%       \begin{pspicture}(5,5)
% ^^A         % Receive from West, East, North and South
%         \ReceiveData{(0,0.5)(0.5,4.5)}{\West} \ReceiveData{(4.5,0.5)(5,4.5)}{\East}
%         \ReceiveData{(0.5,4.5)(4.5,5)}{\North}\ReceiveData{(0.5,0)(4.5,0.5)}{\South}
% ^^A         % send area for West, East, North and South
%         \SendArea{(0.5,0.5)(1,4.5)} \SendArea{(4,0.5)(4.5,4.5)}
%         \SendArea{(0.5,0.5)(4.5,1)} \SendArea{(0.5,4)(4.5,4.5)}
% ^^A         % Central domain
%         \SubDomain{(5,5)}{(0.5,0.5)(4.5,4.5)}{solid}{\Central}
% ^^A         % Redraw overlapped linesY
%         \psline(1,0.5)(1,4.5)  \psline(4,0.5)(4,4.5)
% ^^A         % Two crossesY
%         \rput(1.5,4){\Cross}  \rput(2,2){\Cross}
%       \end{pspicture}}
% ^^A     % The four neighborsY
%     \rput(0,5.5){\Neighbor{N}{\North}}     \rput{-90}(5.5,0){\Neighbor{E}{\East}}
%     \rput{90}(-5.5,0){\Neighbor{W}{\West}} \rput{180}(0,-5.5){\Neighbor{S}{\South}}
%   \end{pspicture}
% \end{lstlisting}
%
%
%
% Bibliography
% \begin{thebibliography}{99}
% \bibitem{PostScript95} Adobe, Systems~Incorporated, \emph{PostScript Language
% Reference Manual}, Addison-Wesley, 2~edition, 1995.
%
% \bibitem{Bolek98} Piotr Bolek, \MP{} and patterns, \emph{\TUB}, Volume~19,
% Number~3, pages 276--283, September 1998, \CTANref{mpattern}.
%
% \bibitem{MLgraphTSI} Emmanuel Chailloux, Guy Cousineau and Asc\'ander
% Su\'arez, Programmation fonctionnelle de graphismes pour la production
% d'illustrations techniques, \emph{Technique et science informatique},
% Volume~15, Number~7, pages 977--1007, 1996 (in french).
%
% \bibitem{Deledicq97} Andr\'e Deledicq, \emph{Le monde des pavages}, ACL
% \'Editions, 1997 (in french).
%
% \bibitem{EsperetGirou98} Philippe Esperet and Denis Girou,
% Coloriage du pavage dit de Truchet, Cahiers GUTenberg, Number~31,
% pages 5--18, December~1998  (in french).
%
% \bibitem{Girou94} Denis Girou, Pr\'esentation de PSTricks, \emph{Cahiers
% GUTenberg}, Number~16, pages 21--70, February~1994 (in french).
%
% \bibitem{LGC97} Michel Goossens, Sebastian Rahtz and Frank Mittelbach,
% \emph{The \LaTeX{} Graphics Companion}, Addison-Wesley, 2005.
%
% \bibitem{GS87} Branko Gr\"unbaum and Geoffrey Shephard, \emph{Tilings and
% Patterns}, Freeman and Company, 1987.
%
% \bibitem{Hoenig97} Alan Hoenig, \emph{\TeX{} Unbound: \LaTeX{} \& \TeX{}
% Strategies, Fonts, Graphics, and More}, Oxford University Press, 1997.
%
% \bibitem{XYpic} Kristoffer~H. Rose and Ross Moore, \XYpic. Pattern and Tile
% extension, available from \CTAN, 1991-1998, \CTANref{xypic}.
%
% \bibitem{LAAN96} Kees van der Laan, Paradigms: Just a little bit of PostScript,
% \emph{MAPS}, Volume~17, pages 137--150, 1996.
%
% \bibitem{LAAN97} Kees van der Laan, Tiling in PostScript and \MF{} -- Escher's
% wink, \emph{MAPS}, Volume~19, Number~2, pages 39--67, 1997.
%
% \bibitem{vanZandt93} Timothy Van Zandt, PSTricks. PostScript macros for
% Generic \TeX, available from \CTAN, 1993, \CTANref{pstricks}.
%
% \bibitem{vanZandtGirou94} Timothy Van Zandt and Denis Girou, Inside PSTricks,
% \emph{\TUB}, Volume~15, Number~3, pages 239--246, September 1994.
%
%
% \bibitem{voss07} Herbert Vo\ss, PSTricks -- Graphics for \TeX\ and \LaTeX, DANTE/Lehmanns, 4th ed., 2007.
% \bibitem{Wang65} Hao Wang, Games, Logic and Computers, \emph{Scientific
% American}, pages 98--106, November 1965.
% \end{thebibliography}
%
%
% \StopEventually{}
%
% ^^A .................... End of the documentation part ....................
%
% \section{Driver file}
%
%   The next bit of code contains the documentation driver file for \TeX{},
% i.e., the file that will produce the documentation you are currently
% reading. It will be extracted from this file by the \texttt{docstrip}
% program.
%
%    \begin{macrocode}
%<*driver>
\documentclass{ltxdoc}
\GetFileInfo{pst-fill.dtx}
%
\usepackage[T1]{fontenc}
\usepackage{lmodern}               % For PDF
\usepackage{graphicx}              % `graphicx' LaTeX standard package
\usepackage{showexpl}
\usepackage{mflogo}                % For the MetaFont and MetaPost logos
\input{random.tex}                 % Random macros from Donald Arseneau
\usepackage{url}                   % URLs convenient typesetting
\usepackage{multido}               % General loop macro
\usepackage[dvipsnames]{pstricks}  % PSTricks with the `color' extension
\usepackage{pst-text}              % PSTricks package for character path
\usepackage{pst-grad}              % PSTricks package for gradient filling
\usepackage{pst-node}              % PSTricks package for nodes
\usepackage[tiling]{pst-fill}      % PSTricks package for filling/tiling
%
\AtBeginDocument{%
%  \OnlyDescription % comment out for implementation details
  \EnableCrossrefs
  \CodelineIndex
  \RecordChanges}
\AtEndDocument{%
  \PrintIndex
  \setcounter{IndexColumns}{1}
  \PrintChanges}
\hbadness=7000            % Over and under full box warnings
\hfuzz=3pt
\begin{document}
  \DocInput{pst-fill.dtx}
\end{document}
%</driver>
%    \end{macrocode}
%
% \section{\texttt{pst-fill} \LaTeX{} wrapper}
%
%    \begin{macrocode}
%<*latex-wrapper>
\RequirePackage{pstricks}
\ProvidesPackage{pst-fill}[2005/09/13 package wrapper for 
  pst-fill.tex (hv)]
\DeclareOption{tiling}{\def\PstTiling{true}}
\ProcessOptions\relax
\input{pst-fill.tex}
\ProvidesFile{pst-fill.tex}
  [\filedate\space v\fileversion\space `PST-fill' (tvz,dg)]
%</latex-wrapper>
%    \end{macrocode}
%
%
% \section{Pst-Fill Package{} code}
%
%    \begin{macrocode}
%<*pst-fill>
%    \end{macrocode}
%
% \subsection{Preamble}
%
%   Who we are.
%
%    \begin{macrocode}
\def\fileversion{1.01}
\def\filedate{2007/03/10}
\message{`PST-Fill' v\fileversion, \filedate\space (tvz,dg,hv)}
\csname PSTboxfillLoaded\endcsname
\let\PSTboxfillLoaded\endinput
%    \end{macrocode}
%
%   Require the main PSTricks package.
%
%    \begin{macrocode}
\ifx\PSTricksLoaded\endinput\else\input pstricks.tex\fi
%    \end{macrocode}
%
%   interface to the extended `\textsf{keyval}' package.
%
%    \begin{macrocode}
\ifx\PSTXKeyLoaded\endinput\else\input pst-xkey\fi
%
%    \end{macrocode}
%
%   Catcodes changes and defining the family name for xkeyval.
%
%    \begin{macrocode}
\edef\PstAtCode{\the\catcode`\@}\catcode`\@=11\relax

\pst@addfams{pst-fill}
%
%    \end{macrocode}
%
%
% \subsection{The size of the box}
% \begin{macro}{pst@@boxfillsize}
%    \begin{macrocode}
%
\def\pst@@boxfillsize#1(#2,#3)#4(#5,#6)#7(#8\@nil{%
  \begingroup
    \ifx\@empty#7\relax
      \pst@dima\z@
      \pst@dimb\z@
      \pssetxlength\pst@dimc{#2}%
      \pssetylength\pst@dimd{#3}%
    \else
      \pssetxlength\pst@dima{#2}%
      \pssetylength\pst@dimb{#3}%
      \pssetxlength\pst@dimc{#5}%
      \pssetylength\pst@dimd{#6}%
    \fi
    \xdef\pst@tempg{%
      \pst@dima=\number\pst@dima sp
      \pst@dimb=\number\pst@dimb sp
      \pst@dimc=\number\pst@dimc sp
      \pst@dimd=\number\pst@dimd sp }%
  \endgroup
  \let\psk@boxfillsize\pst@tempg}
%    \end{macrocode}
% \end{macro}
%

% \subsection{Definition of the parameters}
%
%    \begin{macrocode}
\define@key[psset]{pst-fill}{boxfillsize}{%
  \def\pst@tempg{#1}\def\pst@temph{auto}%
  \ifx\pst@tempg\pst@temph
    \let\psk@boxfillsize\relax
  \else
    \pst@@boxfillsize#1(\z@,\z@)\@empty(\z@,\z@)(\@nil
  \fi}
\psset{boxfillsize={(-15cm,-15cm)(15cm,15cm)}}
\define@key[psset]{pst-fill}{boxfillcolor}{\pst@getcolor{#1}\psboxfillcolor}
\psset{boxfillcolor=black}% hv
\define@key[psset]{pst-fill}{boxfillangle}{\pst@getangle{#1}\psk@boxfillangle}
\psset{boxfillangle=0}
\define@key[psset]{pst-fill}{fillsepx}{%
  \pst@getlength{#1}\psk@fillsepx}
\define@key[psset]{pst-fill}{fillsepy}{%
  \pst@getlength{#1}\psk@fillsepy}
\define@key[psset]{pst-fill}{fillsep}{%
  \pst@getlength{#1}\psk@fillsepx%
  \let\psk@fillsepy\psk@fillsepx}
\psset{fillsep=2pt}

\ifx\PstTiling\@undefined
  \define@key[psset]{pst-fill}{fillcycle}{\pst@getint{#1}\psk@fillcycle}
  \psset{fillcycle=0}
\else
  \define@key[psset]{pst-fill}{fillangle}{\pst@getangle{#1}\psk@boxfillangle}
  \define@key[psset]{pst-fill}{fillsize}{%
      \def\pst@tempg{#1}\def\pst@temph{auto}%
      \ifx\pst@tempg\pst@temph\let\psk@boxfillsize\relax
      \else\pst@@boxfillsize#1(\z@,\z@)\@empty(\z@,\z@)(\@nil\fi}
  \psset{fillsep=0,fillsize=auto}
  \define@key[psset]{pst-fill}{fillcyclex}{\pst@getint{#1}\psk@fillcyclex}
  \define@key[psset]{pst-fill}{fillcycley}{\pst@getint{#1}\psk@fillcycley}
  \define@key[psset]{pst-fill}{fillcycle}{%
    \pst@getint{#1}\psk@fillcyclex\let\psk@fillcycley\psk@fillcyclex}
  \psset{fillcycle=0}
  \define@key[psset]{pst-fill}{fillmovex}{\pst@getlength{#1}\psk@fillmovex}
  \define@key[psset]{pst-fill}{fillmovey}{\pst@getlength{#1}\psk@fillmovey}
  \define@key[psset]{pst-fill}{fillmove}{%
      \pst@getlength{#1}\psk@fillmovex\let\psk@fillmovey\psk@fillmovex}
  \psset{fillmove=0pt}
  \define@key[psset]{pst-fill}{fillloopaddx}{\pst@getint{#1}\psk@fillloopaddx}
  \define@key[psset]{pst-fill}{fillloopaddy}{\pst@getint{#1}\psk@fillloopaddy}
  \define@key[psset]{pst-fill}{fillloopadd}{%
    \pst@getint{#1}\psk@fillloopaddx\let\psk@fillloopaddy\psk@fillloopaddx}
  \psset{fillloopadd=0}
%    \end{macrocode}
%
%    \begin{macrocode}
% For debugging (to debug, set PstDebug=1)
% we now use the one from pstricks to prevent a clash with package
% pstricks                        2004-06-22
%%    \define@key[psset]{pst-fill}{PstDebug}{\pst@getint{#1}\psk@PstDebug}
    \psset{PstDebug=0}
\fi
% DG addition end
%    \end{macrocode}

% \subsection{Definition of the fill box}
% \begin{macro}{psboxfill}
%    \begin{macrocode}
\newbox\pst@fillbox
\def\psboxfill{\pst@killglue\pst@makebox\psboxfill@i}
\def\psboxfill@i{\setbox\pst@fillbox\box\pst@hbox\ignorespaces}
%    \end{macrocode}
% \end{macro}
% \subsection{The main macros}
%
% \begin{macro}{psfs@boxfill}
%    \begin{macrocode}
\def\psfs@boxfill{%
  \ifvoid\pst@fillbox
    \@pstrickserr{Fill box is empty. Use \string\psboxfill\space first.}\@ehpa
  \else
    \ifx\psk@boxfillsize\relax \pst@AutoBoxFill
    \else\pst@ManualBoxFill\fi
  \fi}
%    \end{macrocode}
% \end{macro}
%
% \begin{macro}{pst@ManualBoxFill}
%    \begin{macrocode}
\def\pst@ManualBoxFill{%
  \leavevmode
  \begingroup
    \pst@FlushCode
    \begin@psclip
    \pstVerb{clip}%
    \expandafter\pst@AddFillBox\psk@boxfillsize
    \end@psclip
  \endgroup}
%    \end{macrocode}
% \end{macro}
%
% \begin{macro}{pst@FlushCode}
%    \begin{macrocode}
\def\pst@FlushCode{%
  \pst@Verb{%
    /mtrxc CM def
    CP CP T
    \tx@STV
    \psk@origin
    \psk@swapaxes
    \pst@newpath
    \pst@code
    mtrxc setmatrix
    moveto
    0 setgray}%
  \gdef\pst@code{}}
%    \end{macrocode}
% \end{macro}
%
% \begin{macro}{pst@AddFillBox}
%    \begin{macrocode}
\def\pst@AddFillBox#1 #2 #3 #4 {%
  \begingroup
    \setbox\pst@fillbox=\vbox{%
      \hbox{\unhcopy\pst@fillbox\kern\psk@fillsepx\p@}%
      \vskip\psk@fillsepy\p@}%
    \psk@boxfillsize
    \pst@cnta=\pst@dimc
    \advance\pst@cnta-\pst@dima
    \divide\pst@cnta\wd\pst@fillbox
    \pst@cntb=\pst@dimd
    \advance\pst@cntb-\pst@dimb
    \pst@dimd=\ht\pst@fillbox
    \divide\pst@cntb\pst@dimd
    \def\pst@tempa{%
      \pst@tempg
      \copy\pst@fillbox
      \advance\pst@cntc\@ne
      \ifnum\pst@cntc<\pst@cntd\expandafter\pst@tempa\fi}%
    \let\pst@tempg\relax
    \pst@cntc-\tw@
    \pst@cntd\pst@cnta
    \setbox\pst@fillbox=\hbox to \z@{%
      \kern\pst@dima
      \kern-\wd\pst@fillbox
      \pst@tempa
      \hss}%
    \pst@cntd\pst@cntb
%% DG modification begin - Dec. 11, 1997 - Patch 2
    \ifx\PstTiling\@undefined
      \ifnum\psk@fillcycle=\z@\pst@ManualFillCycle\fi
    \else
      \ifnum\psk@fillcyclex=\z@\pst@ManualFillCycle\fi
    \fi
%% DG modification end
    \global\setbox\pst@boxg=\vbox to\z@{%
      \offinterlineskip
      \vss
      \pst@tempa
      \vskip\pst@dimb}%
  \endgroup
  \setbox\pst@fillbox\box\pst@boxg
  \pst@rotate\psk@boxfillangle\pst@fillbox
  \box\pst@fillbox}
%    \end{macrocode}
% \end{macro}
%
% \begin{macro}{pst@ManualFillCycle}
%    \begin{macrocode}
\def\pst@ManualFillCycle{%
  \ifx\PstTiling\@undefined
    \pst@cntg=\psk@fillcycle
  \else
    \pst@cntg=\psk@fillcyclex
  \fi
  \pst@dimg=\wd\pst@fillbox
  \ifnum\pst@cntg=\z@
  \else
  \divide\pst@dimg\pst@cntg
  \fi
  \ifnum\pst@cntg<\z@\pst@cntg=-\pst@cntg\fi
  \advance\pst@cntg\m@ne
  \pst@cnth=\pst@cntg
  \def\pst@tempg{%
    \ifnum\pst@cnth<\pst@cntg\advance\pst@cnth\@ne\else\pst@cnth\z@\fi
    \moveright\pst@cnth\pst@dimg}}
%    \end{macrocode}
% \end{macro}
%
%% Auto box fill:        !! Fix dictionary
%
% \subsection{The PostScript subroutines}
%
%    \begin{macrocode}
%% DG addition begin - Apr. 8, 1997 and Dec. 1997 - Patch 2
\ifx\PstTiling\@undefined
\pst@def{AutoFillCycle}<%
  /c ED
  /n 0 def
  /s {
    /x x w c div n mul add def
    /n n c abs 1 sub lt { n 1 add } { 0 } ifelse def
  } def>

\pst@def{BoxFill}<%
  gsave
    gsave \tx@STV CM grestore dtransform CM idtransform
    abs /h ED abs /w ED
    pathbbox
    h div round 2 add cvi /y2 ED
    w div round 2 add cvi /x2 ED
    h div round 2 sub cvi /y1 ED
    w div round 2 sub cvi /x1 ED
    /y2 y2 y1 sub def
    /x2 x2 x1 sub def
    CP
    y1 h mul sub neg /y1 ED
    x1 w mul sub neg /x1 ED
    clip
    y2 {
      /x x1 def
      s
      x2 {
        save CP x y1
%% patch 4   hv --------------
        \ifx\VTeXversion\undefined
        \else
%%============ mv: 09-10-01 ??? this is likely to be a right change
        neg
%%============
        \fi
%% end patch 4
T moveto Box restore
        /x x w add def
      } repeat
      /y1 y1 h add def
    } repeat
    % Next line not useful... To see that, suppress clipping (DG)
    CP x y1 T moveto Box
  currentpoint currentfont grestore setfont moveto>
\else
%% DG modification begin - Apr. 8, 1997 and Nov. / Dec. 1997 - Patch 2
\pst@def{AutoFillCycleX}<%
  /cX ED
  /nX 0 def
  /CycleX {
    /x x w cX div nX mul add def
    /nX nX cX abs 1 sub lt { nX 1 add } { 0 } ifelse def
  } def>
\pst@def{AutoFillCycleY}<%
  /cY ED
  /mY 0 def
  /nY 0 def
  /CycleY {
    /y1 y1 h cY div mY mul sub def
    nY cY abs 1 sub lt { /nY nY 1 add def /mY 1 def }
                       { /nY 0 def        /mY cY abs 1 sub neg def } ifelse
  } def>

\pst@def{BoxFill}<%
  gsave
    gsave \tx@STV CM grestore dtransform CM idtransform
    abs /h ED abs /w ED
    pathbbox
    h div round 2 add cvi /y2 ED
    w div round 2 add cvi /x2 ED
    h div round 2 sub cvi /y1 ED
    w div round 2 sub cvi /x1 ED
    /CoefLoopX 0 def
    /CoefLoopY 0 def
    /CoefMoveX 0 def
    /CoefMoveY 0 def
    \psk@boxfillangle\space 0 ne {/CoefLoopX 8 def /CoefLoopY 8 def} if
    \psk@fillcyclex\space 0 ne {/CoefLoopX CoefLoopX 1 add def} if
    \psk@fillcycley\space 0 ne {/CoefLoopY CoefLoopY 1 add def} if
    \psk@fillmovex\space 0 ne
      {/CoefLoopX CoefLoopX 2 add def
       \psk@fillmovex\space 0 gt {/CoefMoveX CoefLoopX def}
                           {/CoefMoveX CoefLoopX neg def} ifelse} if
    \psk@fillmovey\space 0 ne
      {/CoefLoopY CoefLoopY 2 add def
       \psk@fillmovey\space 0 gt {/CoefMoveY CoefLoopY def}
                           {/CoefMoveY CoefLoopY neg def} ifelse} if
    \psk@fillsepx\space 0 ne {/CoefLoopX CoefLoopX 1 add def} if
    \psk@fillsepy\space 0 ne {/CoefLoopY CoefLoopY 1 add def} if
    /CoefLoopX CoefLoopX \psk@fillloopaddx\space add def
    /CoefLoopY CoefLoopY \psk@fillloopaddy\space add def
    /x2 x2 x1 sub 4 sub CoefLoopX 2 mul add def
    /y2 y2 y1 sub 4 sub CoefLoopY 2 mul add def
%% We must fix the origin of tiling, as it must not vary according other stuff
%% in the page!
    w x1 CoefLoopX add CoefMoveX add mul
      h y1 y2 add 1 sub CoefLoopY sub CoefMoveY sub mul moveto
    CP
    y1 h mul sub neg /y1 ED
    x1 w mul sub neg /x1 ED
%%  hv 2004-06-22   to prevent clash with pst-gr3d
%%    \psk@PstDebug 0 eq {clip} if
    \Pst@Debug 0 eq {clip} if
%% end hv
    \psk@fillmovex\space \psk@fillmovey
    gsave \tx@STV CM grestore dtransform CM idtransform
    /hmove ED /wmove ED
    /row 0 def
   y2 {
       /row row 1 add def
       /column 0 def
       /x x1 def
       CycleX
       save
       x2 {
          /column column 1 add def
          CycleY
          save CP x y1
%% patch 4   hv --------------
          \ifx\VTeXversion\undefined
          \else
%%============ mv: 09-10-01 ??? this is likely to be a right change
          neg
%%============
          \fi
  T moveto Box restore
          /x x w add def
          0 hmove translate
          } repeat
       restore
       /y1 y1 h add def
       wmove 0 translate
       } repeat
  currentpoint currentfont grestore setfont moveto>
\fi
%    \end{macrocode}

%    \begin{macrocode}
\def\pst@AutoBoxFill{%
  \leavevmode
  \begingroup
    \pst@stroke
    \pst@FlushCode
    \pst@Verb{\psk@boxfillangle\space \tx@RotBegin}%
    \pstVerb{\pst@dict /Box \pslbrace end}%
    \ifx\PstTiling\@undefined
    \else
      \ifx\pst@tempa\@undefined % Undefined for instance for \pscharpath
      \else\ifx\pst@tempa\@empty\else
        \def\pst@temph{0}%
        \ifx\pst@tempa\pst@temph
        \else
          \pstVerb{/TR {pop pop currentpoint translate \pst@tempa\space translate } def}%
        \fi
      \fi\fi
    \fi
    \hbox to \z@{\vbox to\z@{\vss\copy\pst@fillbox\vskip-\dp\pst@fillbox}\hss}%
    \ifx\PstTiling\@undefined
      \pstVerb{%
        tx@Dict begin \psrbrace def
        \ifnum\psk@fillcycle=\z@
          /s {} def
        \else
          \psk@fillcycle \tx@AutoFillCycle
        \fi
        \pst@number{\wd\pst@fillbox}%
        \psk@fillsepx\space add
        \pst@number{\ht\pst@fillbox}%
        \pst@number{\dp\pst@fillbox}%
        \psk@fillsepy\space add add
        \tx@BoxFill
        end}%
      \else
      \pstVerb{%
        tx@Dict begin \psrbrace def
        \ifnum\psk@fillcyclex=\z@
          /CycleX {} def
        \else
          \psk@fillcyclex\space \tx@AutoFillCycleX
        \fi
        \ifnum\psk@fillcycley=\z@
          /CycleY {} def
        \else
          \psk@fillcycley\space \tx@AutoFillCycleY
        \fi
        \pst@number{\wd\pst@fillbox}%
        \psk@fillsepx\space add
        \pst@number{\ht\pst@fillbox}%
        \pst@number{\dp\pst@fillbox}%
        \psk@fillsepy\space add add
        \tx@BoxFill
        end}%
    \fi
    \pst@Verb{\tx@RotEnd}%
  \endgroup}
%    \end{macrocode}
% \subsection{Closing}
%
%   Catcodes restoration.
%
%    \begin{macrocode}
\catcode`\@=\PstAtCode\relax
%    \end{macrocode}
%
%    \begin{macrocode}
%</pst-fill>
%    \end{macrocode}
%
% \Finale
%
\endinput
%%
%% End of file `pst-fill.dtx'

\ProvidesFile{pst-fill.tex}
  [\filedate\space v\fileversion\space `PST-fill' (tvz,dg)]
%</latex-wrapper>
%    \end{macrocode}
%
%
% \section{Pst-Fill Package{} code}
%
%    \begin{macrocode}
%<*pst-fill>
%    \end{macrocode}
%
% \subsection{Preamble}
%
%   Who we are.
%
%    \begin{macrocode}
\def\fileversion{1.01}
\def\filedate{2007/03/10}
\message{`PST-Fill' v\fileversion, \filedate\space (tvz,dg,hv)}
\csname PSTboxfillLoaded\endcsname
\let\PSTboxfillLoaded\endinput
%    \end{macrocode}
%
%   Require the main PSTricks package.
%
%    \begin{macrocode}
\ifx\PSTricksLoaded\endinput\else\input pstricks.tex\fi
%    \end{macrocode}
%
%   interface to the extended `\textsf{keyval}' package.
%
%    \begin{macrocode}
\ifx\PSTXKeyLoaded\endinput\else\input pst-xkey\fi
%
%    \end{macrocode}
%
%   Catcodes changes and defining the family name for xkeyval.
%
%    \begin{macrocode}
\edef\PstAtCode{\the\catcode`\@}\catcode`\@=11\relax

\pst@addfams{pst-fill}
%
%    \end{macrocode}
%
%
% \subsection{The size of the box}
% \begin{macro}{pst@@boxfillsize}
%    \begin{macrocode}
%
\def\pst@@boxfillsize#1(#2,#3)#4(#5,#6)#7(#8\@nil{%
  \begingroup
    \ifx\@empty#7\relax
      \pst@dima\z@
      \pst@dimb\z@
      \pssetxlength\pst@dimc{#2}%
      \pssetylength\pst@dimd{#3}%
    \else
      \pssetxlength\pst@dima{#2}%
      \pssetylength\pst@dimb{#3}%
      \pssetxlength\pst@dimc{#5}%
      \pssetylength\pst@dimd{#6}%
    \fi
    \xdef\pst@tempg{%
      \pst@dima=\number\pst@dima sp
      \pst@dimb=\number\pst@dimb sp
      \pst@dimc=\number\pst@dimc sp
      \pst@dimd=\number\pst@dimd sp }%
  \endgroup
  \let\psk@boxfillsize\pst@tempg}
%    \end{macrocode}
% \end{macro}
%

% \subsection{Definition of the parameters}
%
%    \begin{macrocode}
\define@key[psset]{pst-fill}{boxfillsize}{%
  \def\pst@tempg{#1}\def\pst@temph{auto}%
  \ifx\pst@tempg\pst@temph
    \let\psk@boxfillsize\relax
  \else
    \pst@@boxfillsize#1(\z@,\z@)\@empty(\z@,\z@)(\@nil
  \fi}
\psset{boxfillsize={(-15cm,-15cm)(15cm,15cm)}}
\define@key[psset]{pst-fill}{boxfillcolor}{\pst@getcolor{#1}\psboxfillcolor}
\psset{boxfillcolor=black}% hv
\define@key[psset]{pst-fill}{boxfillangle}{\pst@getangle{#1}\psk@boxfillangle}
\psset{boxfillangle=0}
\define@key[psset]{pst-fill}{fillsepx}{%
  \pst@getlength{#1}\psk@fillsepx}
\define@key[psset]{pst-fill}{fillsepy}{%
  \pst@getlength{#1}\psk@fillsepy}
\define@key[psset]{pst-fill}{fillsep}{%
  \pst@getlength{#1}\psk@fillsepx%
  \let\psk@fillsepy\psk@fillsepx}
\psset{fillsep=2pt}

\ifx\PstTiling\@undefined
  \define@key[psset]{pst-fill}{fillcycle}{\pst@getint{#1}\psk@fillcycle}
  \psset{fillcycle=0}
\else
  \define@key[psset]{pst-fill}{fillangle}{\pst@getangle{#1}\psk@boxfillangle}
  \define@key[psset]{pst-fill}{fillsize}{%
      \def\pst@tempg{#1}\def\pst@temph{auto}%
      \ifx\pst@tempg\pst@temph\let\psk@boxfillsize\relax
      \else\pst@@boxfillsize#1(\z@,\z@)\@empty(\z@,\z@)(\@nil\fi}
  \psset{fillsep=0,fillsize=auto}
  \define@key[psset]{pst-fill}{fillcyclex}{\pst@getint{#1}\psk@fillcyclex}
  \define@key[psset]{pst-fill}{fillcycley}{\pst@getint{#1}\psk@fillcycley}
  \define@key[psset]{pst-fill}{fillcycle}{%
    \pst@getint{#1}\psk@fillcyclex\let\psk@fillcycley\psk@fillcyclex}
  \psset{fillcycle=0}
  \define@key[psset]{pst-fill}{fillmovex}{\pst@getlength{#1}\psk@fillmovex}
  \define@key[psset]{pst-fill}{fillmovey}{\pst@getlength{#1}\psk@fillmovey}
  \define@key[psset]{pst-fill}{fillmove}{%
      \pst@getlength{#1}\psk@fillmovex\let\psk@fillmovey\psk@fillmovex}
  \psset{fillmove=0pt}
  \define@key[psset]{pst-fill}{fillloopaddx}{\pst@getint{#1}\psk@fillloopaddx}
  \define@key[psset]{pst-fill}{fillloopaddy}{\pst@getint{#1}\psk@fillloopaddy}
  \define@key[psset]{pst-fill}{fillloopadd}{%
    \pst@getint{#1}\psk@fillloopaddx\let\psk@fillloopaddy\psk@fillloopaddx}
  \psset{fillloopadd=0}
%    \end{macrocode}
%
%    \begin{macrocode}
% For debugging (to debug, set PstDebug=1)
% we now use the one from pstricks to prevent a clash with package
% pstricks                        2004-06-22
%%    \define@key[psset]{pst-fill}{PstDebug}{\pst@getint{#1}\psk@PstDebug}
    \psset{PstDebug=0}
\fi
% DG addition end
%    \end{macrocode}

% \subsection{Definition of the fill box}
% \begin{macro}{psboxfill}
%    \begin{macrocode}
\newbox\pst@fillbox
\def\psboxfill{\pst@killglue\pst@makebox\psboxfill@i}
\def\psboxfill@i{\setbox\pst@fillbox\box\pst@hbox\ignorespaces}
%    \end{macrocode}
% \end{macro}
% \subsection{The main macros}
%
% \begin{macro}{psfs@boxfill}
%    \begin{macrocode}
\def\psfs@boxfill{%
  \ifvoid\pst@fillbox
    \@pstrickserr{Fill box is empty. Use \string\psboxfill\space first.}\@ehpa
  \else
    \ifx\psk@boxfillsize\relax \pst@AutoBoxFill
    \else\pst@ManualBoxFill\fi
  \fi}
%    \end{macrocode}
% \end{macro}
%
% \begin{macro}{pst@ManualBoxFill}
%    \begin{macrocode}
\def\pst@ManualBoxFill{%
  \leavevmode
  \begingroup
    \pst@FlushCode
    \begin@psclip
    \pstVerb{clip}%
    \expandafter\pst@AddFillBox\psk@boxfillsize
    \end@psclip
  \endgroup}
%    \end{macrocode}
% \end{macro}
%
% \begin{macro}{pst@FlushCode}
%    \begin{macrocode}
\def\pst@FlushCode{%
  \pst@Verb{%
    /mtrxc CM def
    CP CP T
    \tx@STV
    \psk@origin
    \psk@swapaxes
    \pst@newpath
    \pst@code
    mtrxc setmatrix
    moveto
    0 setgray}%
  \gdef\pst@code{}}
%    \end{macrocode}
% \end{macro}
%
% \begin{macro}{pst@AddFillBox}
%    \begin{macrocode}
\def\pst@AddFillBox#1 #2 #3 #4 {%
  \begingroup
    \setbox\pst@fillbox=\vbox{%
      \hbox{\unhcopy\pst@fillbox\kern\psk@fillsepx\p@}%
      \vskip\psk@fillsepy\p@}%
    \psk@boxfillsize
    \pst@cnta=\pst@dimc
    \advance\pst@cnta-\pst@dima
    \divide\pst@cnta\wd\pst@fillbox
    \pst@cntb=\pst@dimd
    \advance\pst@cntb-\pst@dimb
    \pst@dimd=\ht\pst@fillbox
    \divide\pst@cntb\pst@dimd
    \def\pst@tempa{%
      \pst@tempg
      \copy\pst@fillbox
      \advance\pst@cntc\@ne
      \ifnum\pst@cntc<\pst@cntd\expandafter\pst@tempa\fi}%
    \let\pst@tempg\relax
    \pst@cntc-\tw@
    \pst@cntd\pst@cnta
    \setbox\pst@fillbox=\hbox to \z@{%
      \kern\pst@dima
      \kern-\wd\pst@fillbox
      \pst@tempa
      \hss}%
    \pst@cntd\pst@cntb
%% DG modification begin - Dec. 11, 1997 - Patch 2
    \ifx\PstTiling\@undefined
      \ifnum\psk@fillcycle=\z@\pst@ManualFillCycle\fi
    \else
      \ifnum\psk@fillcyclex=\z@\pst@ManualFillCycle\fi
    \fi
%% DG modification end
    \global\setbox\pst@boxg=\vbox to\z@{%
      \offinterlineskip
      \vss
      \pst@tempa
      \vskip\pst@dimb}%
  \endgroup
  \setbox\pst@fillbox\box\pst@boxg
  \pst@rotate\psk@boxfillangle\pst@fillbox
  \box\pst@fillbox}
%    \end{macrocode}
% \end{macro}
%
% \begin{macro}{pst@ManualFillCycle}
%    \begin{macrocode}
\def\pst@ManualFillCycle{%
  \ifx\PstTiling\@undefined
    \pst@cntg=\psk@fillcycle
  \else
    \pst@cntg=\psk@fillcyclex
  \fi
  \pst@dimg=\wd\pst@fillbox
  \ifnum\pst@cntg=\z@
  \else
  \divide\pst@dimg\pst@cntg
  \fi
  \ifnum\pst@cntg<\z@\pst@cntg=-\pst@cntg\fi
  \advance\pst@cntg\m@ne
  \pst@cnth=\pst@cntg
  \def\pst@tempg{%
    \ifnum\pst@cnth<\pst@cntg\advance\pst@cnth\@ne\else\pst@cnth\z@\fi
    \moveright\pst@cnth\pst@dimg}}
%    \end{macrocode}
% \end{macro}
%
%% Auto box fill:        !! Fix dictionary
%
% \subsection{The PostScript subroutines}
%
%    \begin{macrocode}
%% DG addition begin - Apr. 8, 1997 and Dec. 1997 - Patch 2
\ifx\PstTiling\@undefined
\pst@def{AutoFillCycle}<%
  /c ED
  /n 0 def
  /s {
    /x x w c div n mul add def
    /n n c abs 1 sub lt { n 1 add } { 0 } ifelse def
  } def>

\pst@def{BoxFill}<%
  gsave
    gsave \tx@STV CM grestore dtransform CM idtransform
    abs /h ED abs /w ED
    pathbbox
    h div round 2 add cvi /y2 ED
    w div round 2 add cvi /x2 ED
    h div round 2 sub cvi /y1 ED
    w div round 2 sub cvi /x1 ED
    /y2 y2 y1 sub def
    /x2 x2 x1 sub def
    CP
    y1 h mul sub neg /y1 ED
    x1 w mul sub neg /x1 ED
    clip
    y2 {
      /x x1 def
      s
      x2 {
        save CP x y1
%% patch 4   hv --------------
        \ifx\VTeXversion\undefined
        \else
%%============ mv: 09-10-01 ??? this is likely to be a right change
        neg
%%============
        \fi
%% end patch 4
T moveto Box restore
        /x x w add def
      } repeat
      /y1 y1 h add def
    } repeat
    % Next line not useful... To see that, suppress clipping (DG)
    CP x y1 T moveto Box
  currentpoint currentfont grestore setfont moveto>
\else
%% DG modification begin - Apr. 8, 1997 and Nov. / Dec. 1997 - Patch 2
\pst@def{AutoFillCycleX}<%
  /cX ED
  /nX 0 def
  /CycleX {
    /x x w cX div nX mul add def
    /nX nX cX abs 1 sub lt { nX 1 add } { 0 } ifelse def
  } def>
\pst@def{AutoFillCycleY}<%
  /cY ED
  /mY 0 def
  /nY 0 def
  /CycleY {
    /y1 y1 h cY div mY mul sub def
    nY cY abs 1 sub lt { /nY nY 1 add def /mY 1 def }
                       { /nY 0 def        /mY cY abs 1 sub neg def } ifelse
  } def>

\pst@def{BoxFill}<%
  gsave
    gsave \tx@STV CM grestore dtransform CM idtransform
    abs /h ED abs /w ED
    pathbbox
    h div round 2 add cvi /y2 ED
    w div round 2 add cvi /x2 ED
    h div round 2 sub cvi /y1 ED
    w div round 2 sub cvi /x1 ED
    /CoefLoopX 0 def
    /CoefLoopY 0 def
    /CoefMoveX 0 def
    /CoefMoveY 0 def
    \psk@boxfillangle\space 0 ne {/CoefLoopX 8 def /CoefLoopY 8 def} if
    \psk@fillcyclex\space 0 ne {/CoefLoopX CoefLoopX 1 add def} if
    \psk@fillcycley\space 0 ne {/CoefLoopY CoefLoopY 1 add def} if
    \psk@fillmovex\space 0 ne
      {/CoefLoopX CoefLoopX 2 add def
       \psk@fillmovex\space 0 gt {/CoefMoveX CoefLoopX def}
                           {/CoefMoveX CoefLoopX neg def} ifelse} if
    \psk@fillmovey\space 0 ne
      {/CoefLoopY CoefLoopY 2 add def
       \psk@fillmovey\space 0 gt {/CoefMoveY CoefLoopY def}
                           {/CoefMoveY CoefLoopY neg def} ifelse} if
    \psk@fillsepx\space 0 ne {/CoefLoopX CoefLoopX 1 add def} if
    \psk@fillsepy\space 0 ne {/CoefLoopY CoefLoopY 1 add def} if
    /CoefLoopX CoefLoopX \psk@fillloopaddx\space add def
    /CoefLoopY CoefLoopY \psk@fillloopaddy\space add def
    /x2 x2 x1 sub 4 sub CoefLoopX 2 mul add def
    /y2 y2 y1 sub 4 sub CoefLoopY 2 mul add def
%% We must fix the origin of tiling, as it must not vary according other stuff
%% in the page!
    w x1 CoefLoopX add CoefMoveX add mul
      h y1 y2 add 1 sub CoefLoopY sub CoefMoveY sub mul moveto
    CP
    y1 h mul sub neg /y1 ED
    x1 w mul sub neg /x1 ED
%%  hv 2004-06-22   to prevent clash with pst-gr3d
%%    \psk@PstDebug 0 eq {clip} if
    \Pst@Debug 0 eq {clip} if
%% end hv
    \psk@fillmovex\space \psk@fillmovey
    gsave \tx@STV CM grestore dtransform CM idtransform
    /hmove ED /wmove ED
    /row 0 def
   y2 {
       /row row 1 add def
       /column 0 def
       /x x1 def
       CycleX
       save
       x2 {
          /column column 1 add def
          CycleY
          save CP x y1
%% patch 4   hv --------------
          \ifx\VTeXversion\undefined
          \else
%%============ mv: 09-10-01 ??? this is likely to be a right change
          neg
%%============
          \fi
  T moveto Box restore
          /x x w add def
          0 hmove translate
          } repeat
       restore
       /y1 y1 h add def
       wmove 0 translate
       } repeat
  currentpoint currentfont grestore setfont moveto>
\fi
%    \end{macrocode}

%    \begin{macrocode}
\def\pst@AutoBoxFill{%
  \leavevmode
  \begingroup
    \pst@stroke
    \pst@FlushCode
    \pst@Verb{\psk@boxfillangle\space \tx@RotBegin}%
    \pstVerb{\pst@dict /Box \pslbrace end}%
    \ifx\PstTiling\@undefined
    \else
      \ifx\pst@tempa\@undefined % Undefined for instance for \pscharpath
      \else\ifx\pst@tempa\@empty\else
        \def\pst@temph{0}%
        \ifx\pst@tempa\pst@temph
        \else
          \pstVerb{/TR {pop pop currentpoint translate \pst@tempa\space translate } def}%
        \fi
      \fi\fi
    \fi
    \hbox to \z@{\vbox to\z@{\vss\copy\pst@fillbox\vskip-\dp\pst@fillbox}\hss}%
    \ifx\PstTiling\@undefined
      \pstVerb{%
        tx@Dict begin \psrbrace def
        \ifnum\psk@fillcycle=\z@
          /s {} def
        \else
          \psk@fillcycle \tx@AutoFillCycle
        \fi
        \pst@number{\wd\pst@fillbox}%
        \psk@fillsepx\space add
        \pst@number{\ht\pst@fillbox}%
        \pst@number{\dp\pst@fillbox}%
        \psk@fillsepy\space add add
        \tx@BoxFill
        end}%
      \else
      \pstVerb{%
        tx@Dict begin \psrbrace def
        \ifnum\psk@fillcyclex=\z@
          /CycleX {} def
        \else
          \psk@fillcyclex\space \tx@AutoFillCycleX
        \fi
        \ifnum\psk@fillcycley=\z@
          /CycleY {} def
        \else
          \psk@fillcycley\space \tx@AutoFillCycleY
        \fi
        \pst@number{\wd\pst@fillbox}%
        \psk@fillsepx\space add
        \pst@number{\ht\pst@fillbox}%
        \pst@number{\dp\pst@fillbox}%
        \psk@fillsepy\space add add
        \tx@BoxFill
        end}%
    \fi
    \pst@Verb{\tx@RotEnd}%
  \endgroup}
%    \end{macrocode}
% \subsection{Closing}
%
%   Catcodes restoration.
%
%    \begin{macrocode}
\catcode`\@=\PstAtCode\relax
%    \end{macrocode}
%
%    \begin{macrocode}
%</pst-fill>
%    \end{macrocode}
%
% \Finale
%
\endinput
%%
%% End of file `pst-fill.dtx'
+\newline
%add the following definition:\newline
%\verb+\def\PstTiling{true}+
%
%  To obtain the original behaviour, just don't use the \emph{tiling} optional
%keyword at loading.
%
%  Take care than in \emph{tiling} mode, I introduce also some other changes.
%First I define aliases on some parameter names for consistancy (all specific
%parameters will begin by the \texttt{fill} prefix in this case) and I change
%some default values, which were not well adapted for tilings (\texttt{fillsep}
%is set to 0 and as explained \texttt{fillsize} set to \texttt{auto}). I rename 
%\texttt{fillcycle} to \texttt{fillcyclex}. I also restore normal way so that
%the frame of the area is drawn and all line (\texttt{linestyle},
%\texttt{linecolor}, \texttt{doubleline}, etc.) parameters are now active (but
%there are not in non \emph{tiling} mode). And I also introduce new parameters
%to control the tilings (see below).
%
%  \textbf{In all the following examples, we will consider only the
% \emph{tiling} mode.}
%
%  To do a tiling, we have just to define the pattern with the
% \verb+\psboxfill+ macro and to use the new \texttt{fillstyle}
% \verb+boxfill+.
%
%  Note that tilings are drawn from left to right and top to bottom, which can
%have an importance in some circonstances.
%
%  PostScript programmers can be also interested to know that, even in the
%\emph{automatic} mode, the iterations of the pattern are managed directly by
%the PostScript code of the package which used only PostScript Level 1
%operators. The special ones introduced in Level 2 for drawing of patterns
%\cite[section 4.9]{PostScript95} are not used.
%
%  And first, for conveniance, we define a simple \cs{Tiling} macro, which
%will simplify our examples:
%
%\begin{verbatim}
%  \newcommand{\Tiling}[2][]{%
%    \edef\Temp{#1}%
%    \begin{pspicture}#2
%      \ifx\Temp\empty
%        \psframe[fillstyle=boxfill]#2
%      \else
%        \psframe[fillstyle=boxfill,#1]#2
%      \fi
%    \end{pspicture}}
%\end{verbatim}
%
%
%\newcommand{\Tiling}[2][]{%
%  \edef\Temp{#1}%
%  \begin{pspicture}#2
%    \ifx\Temp\empty
%      \psframe[fillstyle=boxfill]#2
%    \else
%      \psframe[fillstyle=boxfill,#1]#2
%    \fi
% \end{pspicture}}
%
%\subsection{Parameters}
%
%  There are \textbf{14} specific parameters available to change the way the
% filling/tiling is defined, and one debugging option.
%
% \begin{Description}{2cm}
%  \item [fillangle (real)\hfill :] the value of the rotation
%  applied to the patterns (\emph{Default:~0}).
% \end{Description}
%
%
%   In this case, we must force the tiling area to be notably larger than the
% area to cover, to be sure that the defined area will be covered after rotation.
% \lstset{gobble=2}
% \begin{LTXexample}
% \newcommand{\Square}{%
%   \begin{pspicture}(1,1)
%     \psframe[dimen=middle](1,1)
%   \end{pspicture}}
% \psset{unit=0.5}
% \psboxfill{\Square}
% \Tiling[fillangle=45]{(3,3)}\quad
% \Tiling[fillangle=-60]{(3,3)}
% \end{LTXexample}
% 
% \newcommand{\Square}{\begin{pspicture}(1,1)\psframe[dimen=middle](1,1)\end{pspicture}}
% 
% \begin{Description}{2cm}
%   \setcounter{footnote}{1}
%   \item[\texttt{fillsepx} (real$\|$dim) :] value of the horizontal
%   separation between consecutive patterns (\emph{Default:~0 for
%   tilings\footnotemark, 2pt otherwise}).  \footnotetext{This option was added
%   by me, is not part of the original package and is available only if the
%   \texttt{tiling} keyword is used when loading the package.}
%   \setcounter{footnote}{1}
%   \item [\texttt{fillsepy} (real$\|$dim)\hfill :] value of the vertical
%   separation between consecutive patterns (\emph{Default:~0 for
%   ti\-lings\footnotemark, 2pt otherwise}).
%   \setcounter{footnote}{1}
%   \item [\texttt{fillsep} (real$\|$dim)\hfill :] value of horizontal and
%   vertical separations between consecutive patterns (\emph{Default:~0 for
%   tilings\footnotemark, 2pt otherwise}).
% \end{Description}
% 
%   These values can be negative, which allow the tiles to overlap.
% 
% \begin{LTXexample}
% \psset{unit=0.5}
% \psboxfill{\Square}
% \Tiling[fillsepx=2mm]{(3,3)} 
% \Tiling[fillsepy=1mm]{(3,3)}\\
% \Tiling[fillsep=0.5]{(3,3)} 
% \Tiling[fillsep=-0.5]{(3,3)}
% \end{LTXexample}
% 
% \begin{Description}{2cm}
%   \item [\texttt{fillcyclex}\footnotemark\ (integer)\hfill :] Shift
%   coefficient applied to each row (\emph{Default:~0}).
%   \footnotetext{It was \texttt{fillcycle} in the original version.}
%   \setcounter{footnote}{1}
%   \item [\texttt{fillcycley}\footnotemark\ (integer)\hfill :] Same thing for
%   columns (\emph{Default:~0}).
%   \setcounter{footnote}{1}
%   \item [\texttt{fillcycle}\footnotemark\ (integer)\hfill :] Allow to fix
%   both \texttt{fillcyclex} and \texttt{fillcycley} directly to the same value
%   (\emph{Default:~0}).
% \end{Description}
% 
%   For instance, if \texttt{fillcyclex} is 2, the second row of patterns will
% be horizontally shifted by a factor of $\frac{1}{2}=0.5$, and by a factor of
% 0.333 if \texttt{fillcyclex} is 3, etc.). These values can be negative.
% 
% \begin{LTXexample}[width=0.35\linewidth]
% \psset{unit=0.5}
% \psboxfill{\Square}
% \newcommand{\TilingA}[1]{\Tiling[fillcyclex=#1]{(3,3)}}
% \TilingA{0} \TilingA{1}\\
% \TilingA{2} \TilingA{3}\\[3mm]
% \TilingA{4} \TilingA{5}\\
% \TilingA{6} \TilingA{-3}\\[3mm]
% \Tiling[fillcycley=2]{(3,3)}
% \Tiling[fillcycley=3]{(3,3)}\\
% \Tiling[fillcycley=-3]{(3,3)}
% \Tiling[fillcycle=2]{(3,3)}
% \end{LTXexample}
% 
% \begin{Description}{2cm}
%   \setcounter{footnote}{1}
%   \item [\texttt{fillmovex}\footnotemark\ (real$\|$dim)\hfill :] value of the
%   horizontal moves between consecutive patterns (\emph{Default:~0}).
%   \setcounter{footnote}{1}
%   \item [\texttt{fillmovey}\footnotemark\ (real$\|$dim)\hfill :] value of the
%   vertical moves between consecutive patterns (\emph{Default:~0}).
%   \setcounter{footnote}{1}
%   \item [\texttt{fillmove}\footnotemark\ (real$\|$dim)\hfill :] value of
%   horizontal and vertical moves between consecutive patterns
%   (\emph{Default:~0}).
% \end{Description}
% 
%   These parameters allow the patterns to overlap and to draw some special
% kinds of tilings. They are implemented only for the \emph{automatic} and
% \emph{tiling} modes and their values can be negative.
% 
%   In some cases, the effect of these parameters will be the same that with the 
% \texttt{fillcycle?} ones, but you can see that it is not true for some other
% values.
% 
% \begin{LTXexample}
% \psset{unit=0.5}
% \psboxfill{\Square}
% \Tiling[fillmovex=0.5]{(3,3)} 
% \Tiling[fillmovey=0.5]{(3,3)}\\
% \Tiling[fillmove=0.5]{(3,3)}
% \Tiling[fillmove=-0.5]{(3,3)}
% \end{LTXexample}
% 
% \begin{Description}{2cm}
%   \item [\texttt{fillsize}
%   (auto$\|$\{(real$\|$dim,real$\|$dim)(real$\|$dim,real$\|$dim)\}) :] The
%   choice of \emph{automatic} mode or the size of the area in \emph{manual}
%   mode. If first pair values are not given, (0,0) is used. (\emph{Default:
%   auto when \emph{tiling} mode is used, {(-15cm,-15cm)(15cm,15cm)}
%   otherwise}).
% \end{Description}
% 
%   As explained in the introduction, the \emph{manual} mode can require very
% huge amount of computer ressources. So, it usage is to discourage in front off
% the \emph{automatic} mode. It seems only useful in special circonstances, in
% fact when the \emph{automatic} mode failed, which is known only in one case,
% for some kinds of EPS files, as the ones produce by dump of portions of
% screens (see \ref{sec:GraphicFiles}).
% 
% \begin{Description}{2cm}
%   \setcounter{footnote}{1}
%   \item [\texttt{fillloopaddx}\footnotemark\ (integer)\hfill :] number of
%   times the pattern is added on left and right positions (\emph{Default:~0}).
%   \setcounter{footnote}{1}
%   \item [\texttt{fillloopaddy}\footnotemark\ (integer)\hfill :] number of
%   times the pattern is added on top and bottom positions (\emph{Default:~0}).
%   \setcounter{footnote}{1}
%   \item [\texttt{fillloopadd}\footnotemark\ (integer)\hfill :] number of
%   times the pattern is added on left, right, top and bottom positions
%   (\emph{Default:~0}).
% \end{Description}
% 
%   These parameters are only useful in special circonstances, as for complex
% patterns when the size of the rectangular box used to tile the area doesn't 
% correspond to the pattern itself (see an example in Figure~\ref{fig:Sheeps})
% and also sometimes when the size of the pattern is not a divisor of the size
% of the area to fill and that the number of loop repeats is not properly
% computed, which can occur.
% 
%   They are implemented only for the \emph{tiling} mode.
% 
% \begin{Description}{2cm}
%   \setcounter{footnote}{1}
%   \item [\texttt{PstDebug}\footnotemark\ (integer, 0 or 1)\hfill :] to
%   require to see the exact tiling done, without clipping (\emph{Default:~0}).
% \end{Description}
% 
%   It's mainly useful for debugging or to understand better how the tilings
% are done. It is implemented only for the \emph{tiling} mode.
% 
% \begin{LTXexample}
% \psset{unit=0.3,PstDebug=1}
% \psboxfill{\Square}
% \psset{linewidth=1mm}
% \Tiling{(2,2)}\\[5mm]
% \Tiling[fillcyclex=2]{(2,2)}\\[1cm]
% \Tiling[fillmove=0.5]{(2,2)}
% \end{LTXexample}
% 
% \vspace{3cm}
% \section{Examples}
% 
%   In fact this unique \cs{psboxfill} macro allow a lot a variations and
% different usages. We will try here to demonstrate this.
% 
% \subsection{Kind of tiles}
% \label{sec:KindTiles}
% 
%   Of course, we can access to all the power of PSTricks macros to define the
% \emph{tiles} (\emph{patterns}) used. So, we can define complicated ones.
% 
%   Here we give four other Archimedian tilings (those built with only some
% regular polygons) among the twelve existing, first discovered completely by
% Johanes \textsc{Kepler} at the beginning of 17th century \cite{GS87}, the two
% other \emph{regular} ones with the tiling by squares, formed by a unique
% regular polygon, and two other formed by two different regular polygons.
% 
% \begin{LTXexample}[pos=t]
%   \newcommand{\Triangle}{%
%     \begin{pspicture}(1,1)
%       \pstriangle[dimen=middle](0.5,0)(1,1)
%     \end{pspicture}}
%   \newcommand{\Hexagon}{
% ^^A sin(60)=0.866
%     \begin{pspicture}(0.866,0.75)
%       \SpecialCoor
% ^^A  Hexagon  
%       \pspolygon[dimen=middle]%
%         (0.5;30)(0.5;90)(0.5;150)(0.5;210)(0.5;270)(0.5;330)
%     \end{pspicture}}
% 
%   \psset{unit=0.5}
%   \psboxfill{\Triangle}
%   \Tiling{(4,4)}\hfill
% ^^A The two other regular tilings
%   \Tiling[fillcyclex=2]{(4,4)}\hfill
%   \psboxfill{\Hexagon}
%   \Tiling[fillcyclex=2,fillloopaddy=1]{(5,5)}
% \end{LTXexample}
% 
% \begin{LTXexample}[pos=t]
%   \newcommand{\ArchimedianA}{%
%      ^^A Archimedian tiling 3^2.4.3.4
%     \psset{dimen=middle}
%      ^^A sin(60)=0.866
%     \begin{pspicture}(1.866,1.866)
%       \psframe(1,1)
%       \psline(1,0)(1.866,0.5)(1,1)(0.5,1.866)(0,1)(-0.866,0.5)
%       \psline(0,0)(0.5,-0.866)
%     \end{pspicture}}
%   \newcommand{\ArchimedianB}{%
%      ^^A Archimedian tiling 4.8^2
%     \psset{dimen=middle,unit=1.5}
%      ^^A sin(22.5)=0.3827 ; cos(22.5)=0.9239
%     \begin{pspicture}(1.3066,0.6533)
%       \SpecialCoor
%      ^^A Octogon
%       \pspolygon(0.5;22.5)(0.5;67.5)(0.5;112.5)(0.5;157.5)
%                 (0.5;202.5)(0.5;247.5)(0.5;292.5)(0.5;337.5)
%     \end{pspicture}}
% 
%   \psset{unit=0.5}
%   \psboxfill{\ArchimedianA}
%   \Tiling[fillmove=0.5]{(7,7)}\hfill
%   \psboxfill{\ArchimedianB}
%   \Tiling[fillcyclex=2,fillloopaddy=1]{(7,7)}
% \end{LTXexample}
% 
%   \setcounter{footnote}{3}
%   We can of course tile an area arbitrarily defined. And with the
% \texttt{addfillstyle} parameter\footnote{Introduced in PSTricks 97.}, we can
% easily mix the \texttt{boxfill} style with another one.
% 
% \begin{LTXexample}[width=6cm]
%   \psset{unit=0.5,dimen=middle}
%   \psboxfill{%
%     \begin{pspicture}(1,1)
%       \psframe(1,1)
%       \pscircle(0.5,0.5){0.25}
%     \end{pspicture}}
%   \begin{pspicture}(4,6)
%     \pspolygon[fillstyle=boxfill,fillsep=0.25](0,1)(1,4)(4,6)(4,0)(2,1)
%   \end{pspicture}\hspace{1em}
%   \begin{pspicture}(4,4)
%%     \pscircle[linestyle=none,fillstyle=solid,fillcolor=yellow,fillsep=0.5,
%%               addfillstyle=boxfill](2,2){2}
%   \end{pspicture}
% \end{LTXexample}
%
%   Various effects can be obtained, sometimes complicated ones very easily, as
% in this example reproduced from one shown by Slavik \textsc{Jablan} in the
% field of \emph{OpTiles}, inspired by the \emph{Op-art}:
% 
% \begin{LTXexample}[pos=t]
% \newcommand{\ProtoTile}{%
%  \begin{pspicture}(1,1)%%% 1/12=0.08333
%   \psset{linestyle=none,linewidth=0,
%     hatchwidth=0.08333\psunit,hatchsep=0.08333\psunit}
%   \psframe[fillstyle=solid,fillcolor=black,addfillstyle=hlines,hatchcolor=white](1,1)
%   \pswedge[fillstyle=solid,fillcolor=white,addfillstyle=hlines]{1}{0}{90}
%  \end{pspicture}}
% \newcommand{\BasicTile}{%
%  \begin{pspicture}(2,1)
%    \rput[lb](0,0){\ProtoTile}\rput[lb](1,0){\psrotateleft{\ProtoTile}}
%  \end{pspicture}}
% \ProtoTile\hfill\BasicTile\hfill
% \psboxfill{\BasicTile}
% \Tiling[fillcyclex=2]{(4,4)}
% \end{LTXexample}
% 
%   It is also directly possible to surimpose several different tilings. Here is
% the splendid visual proof of the \textsc{Pytha\-gore} theorem done by the arab
% mathematician \textsc{Annairizi} around the year 900, given by superposition
% of two tilings by squares of different sizes.
% 
% \begin{LTXexample}[pos=t]
% \psset{unit=1.5,dimen=middle}
% \begin{pspicture*}(3,3)
%   \psboxfill{\begin{pspicture}(1,1)
%     \psframe(1,1)\end{pspicture}}
%   \psframe[fillstyle=boxfill](3,3)
%   \psboxfill{\begin{pspicture}(1,1)
%     \rput{-37}{\psframe[linecolor=red](0.8,0.8)}
%   \end{pspicture}}
%   \psframe[fillstyle=boxfill](3,4)
%   \pspolygon[fillstyle=hlines,hatchangle=90](1,2)(1.64,1.53)(2,2)
% \end{pspicture*}
% \end{LTXexample}
% 
%   In a same way, it is possible to build tilings based on figurative patterns,
% in the style of the famous \textsc{Escher} ones. Following an example of
% Andr\'e \textsc{Deledicq} \cite{Deledicq97}, we first show a simple tiling of
% the \emph{p1} category (according to the international classification of the
% 17~symmetry groups of the plane first discovered by the russian
% crystalographer Jevgraf \textsc{Fedorov} at the end of the 19th century):
% 
% \begin{LTXexample}[pos=t]
%  \newcommand{\SheepHead}[1]{%
%    \begin{pspicture}(3,1.5)
%      \pscustom[liftpen=2,fillstyle=solid,fillcolor=#1]{%
%        \pscurve(0.5,-0.2)(0.6,0.5)(0.2,1.3)(0,1.5)(0,1.5)
%          (0.4,1.3)(0.8,1.5)(2.2,1.9)(3,1.5)(3,1.5)(3.2,1.3)
%          (3.6,0.5)(3.4,-0.3)(3,0)(2.2,0.4)(0.5,-0.2)}
%      \pscircle*(2.65,1.25){0.12\psunit} % Eye
%      \psccurve*(3.5,0.3)(3.35,0.45)(3.5,0.6)(3.6,0.4)% Muzzle
%     ^^A   % Mouth
%       \pscurve(3,0.35)(3.3,0.1)(3.6,0.05)
%     ^^A   % Ear
%       \pscurve(2.3,1.3)(2.1,1.5)(2.15,1.7)\pscurve(2.1,1.7)(2.35,1.6)(2.45,1.4)
%   \end{pspicture}}
%  \psboxfill{\psset{unit=0.5}\SheepHead{yellow}\SheepHead{cyan}}
%  \Tiling[fillcyclex=2,fillloopadd=1]{(10,5)}
% \end{LTXexample}
% \label{fig:Sheeps}
% 
%   Now a tiling of the \emph{pg} category (the code for the kangaroo itself is
% too long to be shown here, but has no difficulties ; the kangaroo is reproduce
% from an original picture from Raoul \textsc{Raba} and here is a translation in
% PSTricks from the one drawn by Emmanuel \textsc{Chailloux} and Guy
% \textsc{Cousineau} for their MLgraph system \cite{MLgraphTSI}):
% 
% \begin{LTXexample}[pos=t]
% \psboxfill{\psset{unit=0.4}
%   \Kangaroo{yellow}\Kangaroo{red}\Kangaroo{cyan}\Kangaroo{green}%
%   \psscalebox{-1 1}{%
%     \rput(1.235,4.8){\Kangaroo{green}\Kangaroo{cyan}\Kangaroo{red}\Kangaroo{yellow}}}}
%   \Tiling[fillloopadd=1]{(10,6)}
% \end{LTXexample}
% 
%   And here a \textsc{Wang} tiling \cite{Wang65}, \cite[chapter
% 11]{GS87}, based on very simple tiles of the form of a square and composed
% of four colored triangles. Such tilings are built with only a matching color
% constraint. Despite of it simplicity, it is an important kind of tilings, as
% \textsc{Wang} and others used them to study the special class of
% \emph{aperiodic} tilings, and also because it was shown that surprisingly this 
% tiling is similar to a \textsc{Turing} machine.
% 
% \begin{LTXexample}[pos=t]
%   \newcommand{\WangTile}[4]{%
%     \begin{pspicture}(1,1)
%       \pspolygon*[linecolor=#1](0,0)(0,1)(0.5,0.5)
%       \pspolygon*[linecolor=#2](0,1)(1,1)(0.5,0.5)
%       \pspolygon*[linecolor=#3](1,1)(1,0)(0.5,0.5)
%       \pspolygon*[linecolor=#4](1,0)(0,0)(0.5,0.5)
%     \end{pspicture}}
%   \newcommand{\WangTileA}{\WangTile{cyan}{yellow}{cyan}{cyan}}
%   \newcommand{\WangTileB}{\WangTile{yellow}{cyan}{cyan}{red}}
%   \newcommand{\WangTileC}{\WangTile{cyan}{red}{yellow}{yellow}}
%   \newcommand{\WangTiles}[1][]{%
%     \begin{pspicture}(3,3) \psset{ref=lb}
%       \rput(0,2){\WangTileB}  \rput(1,2){\WangTileA}%
%       \rput(2,2){\WangTileC}  \rput(0,1){\WangTileC}%
%       \rput(1,1){\WangTileB}  \rput(2,1){\WangTileA}
%       \rput(0,0){\WangTileA}  \rput(1,0){\WangTileC}%
%       \rput(2,0){\WangTileB}
%       #1
%     \end{pspicture}}
%   \WangTileA\hfill\WangTileB\hfill\WangTileC\hfill
%   \WangTiles[{\psgrid[subgriddiv=0,gridlabels=0](3,3)}]\hfill
%   \psset{unit=0.4} \psboxfill{\WangTiles} \Tiling{(12,12)}
% \end{LTXexample}
% 
% \subsection{External graphic files}
% \label{sec:GraphicFiles}
% 
%   We can also fill an arbitrary area with an external image. We have only, 
% as usual, to matter of the \emph{BoundingBox} definition if there is no one
% provided or if it is not the accurate one, as for the well known
% \texttt{tiger} picture part of the \texttt{ghostscript} distribution.
% 
% \begin{LTXexample}[pos=t]
%   \psboxfill{%% Strangely require x1=x2...
%     \begin{pspicture}(0,1)(0,4.1)
%       \includegraphics[bb=17 176 560 74,width=3cm]{tiger}
%     \end{pspicture}}
%   \Tiling{(6,6.2)}
% \end{LTXexample}
% 
%   Nevertheless, there are some special files for which the \emph{automatic}
% mode doesn't work, specially for some files obtained by a screen dump, as in
% the next example, where a picture was reduced before it conversion in the
% \emph{Encapsulated PostScript} format by a screen dump utility. In this case,
% usage of the \emph{manual} mode is the only alternative, at the price of the
% real multiple inclusion of the EPS file. We must take care to specify the
% correct \texttt{fillsize} parameter, because otherwise the default values are
% large and will load the file many times, perhaps just really using few
% occurrences as the other ones would be clipped...
% 
% \begin{LTXexample}[pos=t]
%   \psboxfill{\includegraphics{flowers}}
%   \begin{pspicture}(8,4)
%     \psellipse[fillstyle=boxfill,fillsize={(8,4)}](4,2)(4,2)
%   \end{pspicture}
% \end{LTXexample}
% 
% \subsection{Tiling of characters}
% 
%   We can also use the \cs{psboxfill} macro to fill the interior of characters
% for special effects like these ones:
% 
% \begin{LTXexample}[pos=t]
%   \DeclareFixedFont{\bigsf}{T1}{phv}{b}{n}{4.5cm}
%   \DeclareFixedFont{\smallrm}{T1}{ptm}{m}{n}{3mm}
%   \psboxfill{\smallrm Since 182 days...}
%   \begin{pspicture*}(8,4)
%     \centerline{%
%       \pscharpath[fillstyle=gradient,gradangle=-45,
%                   gradmidpoint=0.5,addfillstyle=boxfill,
%                   fillangle=45,fillsep=0.7mm]
%                  {\rput[b](0,0.1){\bigsf 2000}}}
%   \end{pspicture*}
% \end{LTXexample}
% 
% \begin{LTXexample}[pos=t]
%   \DeclareFixedFont{\mediumrm}{T1}{ptm}{m}{n}{2cm}
%   \psboxfill{%
%     \psset{unit=0.1,linewidth=0.2pt}
%     \Kangaroo{PeachPuff}\Kangaroo{PaleGreen}%
%       \Kangaroo{LightBlue}\Kangaroo{LemonChiffon}%
%     \psscalebox{-1 1}{%
%       \rput(1.235,4.8){%
%         \Kangaroo{LemonChiffon}\Kangaroo{LightBlue}%
%           \Kangaroo{PaleGreen}\Kangaroo{PeachPuff}}}}
% ^^A   % A kangaroo of kangaroos...
%   \begin{pspicture}(8,2)
%     \pscharpath[linestyle=none,fillstyle=boxfill,fillloopadd=1]
%                {\rput[b](4,0){\mediumrm Kangaroo}}
%   \end{pspicture}
% \end{LTXexample}
% 
% \subsection{Other kinds of usage}
% 
%   Other kinds of usage can be imagined. For instance, we can use tilings in a
% sort of degenerated way to draw some special lines made by a unique or
% multiple repeating patterns. But it can be only a special dashed line, as here
% with three different dashes:
% 
% \begin{LTXexample}[pos=t]
%   \newcommand{\Dashes}{%
%     \psset{dimen=middle}
%     \begin{pspicture}(0,-0.5\pslinewidth)(1,0.5\pslinewidth)
%       \rput(0,0){\psline(0.4,0)}%
%         \rput(0.5,0){\psline(0.2,0)}%
%         \rput(0.8,0){\psline(0.1,0)}
%     \end{pspicture}}
% 
%   \newcommand{\SpecialDashedLine}[3]{%
%     \psboxfill{#3}
%     \Tiling[linestyle=none]
%            {(#1,-0.5\pslinewidth)(#2,0.5\pslinewidth)}}
% 
%   \SpecialDashedLine{0}{7}{\Dashes}
% 
%   \psset{unit=0.5,linewidth=1mm,linecolor=red}
%   \SpecialDashedLine{0}{10}{\Dashes}
% \end{LTXexample}
% 
%   It allow also to use special patterns in business graphics, as in the
% following example generated by \texttt{PstChart}\footnote{A personal
% development to draw business charts with PSTricks, not distributed.}.
% 
% \vspace{3mm}
% \begin{figure}[!ht]
% \centering
% \psset{unit=0.75}
% ^^A % Generated by pstchart.sh version 0.21 (11/28/97)
% {\psset{dimen=middle}
% \psset{xunit=2,yunit=0.005}
% \begin{pspicture}(-0.6,-200)(6.6,2300)
% ^^A   % Title
%   \rput(3,2200){\shortstack{Fantaisist repartition of kangaroos\\
%                             in the world (in thousands)}}
% ^^A   % Frame background
%   \psframe[fillstyle=solid,fillcolor=LemonChiffon](0,0)(6,2000)
% ^^A   % Graduations
%   \multido{\n=0+500}{5}{\rput[r](-0.12,\n){\psscalebox{0.8}{\n}}}
% ^^A   % Minor ticks
%   \multips(0,100)(0,100){19}{\psline[unit=4.8pt](1,0)}
%   \multips(6,100)(0,100){19}{\psline[unit=4.8pt](-1,0)}
% ^^A   % Major ticks
%   \multips(0,500)(0,500){3}{\psline[unit=9.6pt](1,0)}
%   \multips(6,500)(0,500){3}{\psline[unit=9.6pt](-1,0)}
% ^^A   % Lines from major ticks marks
%   \multips(0,500)(0,500){3}{\psline[linestyle=dotted,linewidth=0.6pt](6,0)}
% ^^A   % Drawing for the data
%   \psboxfill{\psset{unit=0.78\psxunit}\KangarooPstChart{red}}
%   \psframe[linestyle=none,fillstyle=boxfill,fillloopaddy=1](0.61,0)(1.39,1800)
%   \psboxfill{\psset{unit=0.78\psxunit}\KangarooPstChart{yellow}}
%   \psframe[linestyle=none,fillstyle=boxfill,fillloopaddy=1](1.61,0)(2.39,800)
%   \psboxfill{\psset{unit=0.78\psxunit}\KangarooPstChart{cyan}}
%   \psframe[linestyle=none,fillstyle=boxfill,fillloopaddy=1](2.61,0)(3.39,550)
%   \psboxfill{\psset{unit=0.78\psxunit}\KangarooPstChart{magenta}}
%   \psframe[linestyle=none,fillstyle=boxfill,fillloopaddy=1](3.61,0)(4.39,500)
%   \psboxfill{\psset{unit=0.78\psxunit}\KangarooPstChart{green}}
%   \psframe[linestyle=none,fillstyle=boxfill,fillloopaddy=1](4.61,0)(5.39,200)
% ^^A   % Bottom labels
%   \uput{0.2}[270]{0}(1,0){\psscalebox{0.7}{Oceania}}
%   \uput{0.2}[270]{0}(2,0){\psscalebox{0.7}{Africa}}
%   \uput{0.2}[270]{0}(3,0){\psscalebox{0.7}{Asia}}
%   \uput{0.2}[270]{0}(4,0){\psscalebox{0.7}{America}}
%   \uput{0.2}[270]{0}(5,0){\psscalebox{0.7}{Europe}}
% ^^A   % Frame box around the chart
%   \psframe[linestyle=solid](0,0)(6,2000)
% \end{pspicture}}
%   \caption{Bar chart generated by PstChart, with bars filled by patterns}
%   \label{fig:PstChart}
% \end{figure}
% 
% \section{``Dynamic'' tilings}
% 
%   In some cases, tilings used non \emph{static} tiles, that is to say that the 
% \emph{prototile(s)}, even if unique, can have several forms, by instance
% specified by different colors or rotations, not fixed before generation or
% varying each time.
% 
% \subsection{Lewthwaite-Pickover-Truchet tiling}
% 
%   We give here for example the so-called \emph{Truchet} tiling, which much be
% in fact better called \emph{Lewthwaite-Pick\-over-Truchet (LPT)} tiling%
% \footnote{For description of the context, history and references about
% S\'ebastien \textsc{Truchet} and this tiling, see \cite{EsperetGirou98}.}.
% 
%   The unique prototile is only a square with two opposite circle arcs.
% This tile has obviously two positions, if we rotate it from 90 degrees (see
% the two tiles on the next figure). A \emph{LPT tiling} is a tiling with
% randomly oriented LPT tiles. We can see that even if it is very simple in it
% principle, it draw sophisticated curves with strange properties.
% 
%   Nevertheless, in the straightforward way \FillPackage{} does not work,
% because the \cs{psboxfill} macro store the content of the tile used in a
% \TeX{} box, which is static. So the calling to the random function is done
% only one time, which explain that only one rotation of the tile is used for
% all the tiling. It's only the one of the two rotations which could differ from
% one drawing to the next one...
% 
% ^^A % Truchet (Lewthwaite-Pickover-Truchet) tiling
% ^^A % --------------------------------------------
% 
% \begin{LTXexample}[pos=t]
% ^^A   % LPT prototile
%   \newcommand{\ProtoTileLPT}{%
%     \psset{dimen=middle}
%     \begin{pspicture}(1,1)
%       \psframe(1,1)
%       \psarc(0,0){0.5}{0}{90}
%       \psarc(1,1){0.5}{-180}{-90}
%     \end{pspicture}}
% 
% ^^A   % LPT tile
%   \newcount\Boolean
%   \newcommand{\BasicTileLPT}{%
% ^^A     % From random.tex by Donald Arseneau
%     \setrannum{\Boolean}{0}{1}%
%     \ifnum\Boolean=0
%       \ProtoTileLPT%
%     \else
%       \psrotateleft{\ProtoTileLPT}%
%     \fi}
% 
%   \ProtoTileLPT\hfill\psrotateleft{\ProtoTileLPT}\hfill
%   \psset{unit=0.5}
%   \psboxfill{\BasicTileLPT}
%   \Tiling{(5,5)}
% \end{LTXexample}
% 
%   But, for simple cases, there is a solution to this problem using a mixture
% of PSTricks and PostScript programming. Here the PSTricks
% construction \verb+\pscustom{\code{...}}+ allow to insert PostScript code
% inside the \LaTeX{} + PSTricks one.
% 
%   Programmation is less straightforward, but it has also the advantage to be
% notably faster, as all the tilings operations are done in PostScript, and
% mainly to not be limited by \TeX{} memory (the \TeX{} + PSTricks solution
% I wrote in 1995 for the colored problem was limited to small sizes for this
% reason). Just note also that \cs{pslbrace} and \cs{psrbrace} are two
% PSTricks macros to define and be able to insert the \verb+{+ and \verb+}+
% characters.
% 
% \begin{LTXexample}[pos=t]
% ^^A   % LPT prototile
%   \newcommand{\ProtoTileLPT}{%
%     \psset{dimen=middle}
%     \psframe(1,1)
%     \psarc(0,0){0.5}{0}{90}
%     \psarc(1,1){0.5}{-180}{-90}}
% 
% ^^A   % Counter to change the random seed
%   \newcount\InitCounter
% ^^A   % LPT tile
%   \newcommand{\BasicTileLPT}{%
%     \InitCounter=\the\time
%     \pscustom{\code{%
%       rand \the\InitCounter\space sub 2 mod 0 eq \pslbrace}}
%     \begin{pspicture}(1,1)
%       \ProtoTileLPT
%     \end{pspicture}%
%     \pscustom{\code{\psrbrace \pslbrace}}
%     \psrotateleft{\ProtoTileLPT}%
%     \pscustom{\code{\psrbrace ifelse}}}
% 
%   \psset{unit=0.4,linewidth=0.4pt}
%   \psboxfill{\BasicTileLPT}
%   \Tiling{(15,15)}
% \end{LTXexample}
% 
%   Using the very surprising fact (see \cite{EsperetGirou98}) that
% coloration of these tiles do not depend of their neighbors (even if it is
% difficult to believe as the opposite seems obvious!) but only of the parity of
% the value of row and column positions, we can directly program in the same way
% a colored version of the LPT tiling.
% 
% \setcounter{footnote}{1}
%   We have also introduce in the \FillPackage{} code for \emph{tiling} mode two
% new accessible Post\-Script variables, \texttt{row} and
% \texttt{column}\footnotemark, which can be useful in some circonstances, like
% this one.
% 
% \begin{LTXexample}[pos=t]
% ^^A   % LPT prototile
%   \newcommand{\ProtoTileLPT}[2]{%
%     \psset{dimen=middle,linestyle=none,fillstyle=solid}
%     \psframe[fillcolor=#1](1,1)
%     \psset{fillcolor=#2}
%     \pswedge(0,0){0.5}{0}{90} \pswedge(1,1){0.5}{-180}{-90}}
% ^^A   % Counter to change the random seed
%   \newcount\InitCounter
% ^^A   % LPT tile
%   \newcommand{\BasicTileLPT}[2]{%
%     \InitCounter=\the\time
%     \pscustom{\code{%
%       rand \the\InitCounter\space sub 2 mod 0 eq \pslbrace
%       row column add 2 mod 0 eq \pslbrace}}
%     \begin{pspicture}(1,1)\ProtoTileLPT{#1}{#2}\end{pspicture}%
%     \pscustom{\code{\psrbrace \pslbrace}}
%     \ProtoTileLPT{#2}{#1}%
%     \pscustom{\code{%
%       \psrbrace ifelse \psrbrace \pslbrace row column add 2 mod 0 eq \pslbrace}}
%     \psrotateleft{\ProtoTileLPT{#2}{#1}}\pscustom{\code{\psrbrace \pslbrace}}
%     \psrotateleft{\ProtoTileLPT{#1}{#2}}\pscustom{\code{\psrbrace ifelse \psrbrace ifelse}}}
%   \psboxfill{\BasicTileLPT{red}{yellow}}
%   \Tiling{(4,4)}\hfill
%   \psset{unit=0.4}\psboxfill{\BasicTileLPT{blue}{cyan}}
%   \Tiling{(15,15)}
% \end{LTXexample}
% 
%   Another classic example is to generate coordinates and numerotation for a
% grid. Of course, it is possible to do it directly in PSTricks using nested
% \cs{multido} commands. It would be clearly easy to program, but, nevertheless, 
% for users who have a little knowledge of PostScript programming, this offer
% an alternative which is useful for large cases, because on this way it will
% be notably faster and less computer ressources consuming.
% 
%   Remember here that the tiling is drawn from left to right, and top to
% bottom, and note that the PostScript variable \texttt{x2} give the total
% number of columns.
% 
% \begin{LTXexample}[pos=t]
% ^^A   % \Escape will be the \ character
%   {\catcode`\!=0\catcode`\\=11!gdef!Escape{\}}
%   \newcommand{\ProtoTile}{%
%     \Square\pscustom{%
%       \moveto(-0.9,0.75) % In PSTricks units
%       \code{ /Times-Italic findfont 8 scalefont setfont
%         (\Escape() show row 3 string cvs show (,) show 
%         column 3 string cvs show (\Escape)) show}
%       \moveto(-0.5,0.25) % In PSTricks units
%       \code{ /Times-Bold findfont 18 scalefont setfont
%         1 0 0 setrgbcolor % Red color
%         /center {dup stringwidth pop 2 div neg 0 rmoveto} def
%         row 1 sub x2 mul column add 3 string cvs center show}}}
%   \psboxfill{\ProtoTile}
%   \Tiling{(6,4)}
% \end{LTXexample}
% 
% \subsection{A complete example: the Poisson equation}
% 
%   To finish, we will show a complete real example, a drawing to explain the
% method used to solve the \textsc{Poisson} equation by a domain
% decomposition method, adapted to distributed memory computers. The
% objective is to show the communications required between processes and the
% position of the data to exchange. This code also show some useful and powerful
% technics for PSTricks programming (look specially at the way some higher level
% macros are defined, and how the same object is used to draw the four
% neighbors).
%
%\psset{unit=1cm}
%\newcommand{\Pattern}[1]{%
%   \begin{pspicture}(-0.25,-0.25)(0.25,0.25)\rput{*0}{\psdot[dotstyle=#1]}
%   \end{pspicture}}
%\newcommand{\West}{\Pattern{o}}   \newcommand{\South}{\Pattern{x}}
%\newcommand{\Central}{\Pattern{+}}\newcommand{\North}{\Pattern{square}}
%\newcommand{\East}{\Pattern{triangle}}
%\newcommand{\Cross}{%
%  \pspolygon[unit=0.5,linewidth=0.2,linecolor=red](0,0)(0,1)(1,1)(1,2)(2,2)(2,1)%
%              (3,1)(3,0)(2,0)(2,-1)(1,-1)(1,0)}
%\newcommand{\StylePosition}[1]{\LARGE\textcolor{red}{\textbf{#1}}}
%\newcommand{\SubDomain}[4]{%
%    \psboxfill{#4}\begin{psclip}{\psframe[linestyle=none]#1}%
%      \psframe[linestyle=#3](5,5)\psframe[fillstyle=boxfill]#2%
%    \end{psclip}}
%\newcommand{\SendArea}[1]{\psframe[fillstyle=solid,fillcolor=cyan]#1}
%\newcommand{\ReceiveData}[2]{%
%  \psboxfill{#2}\psframe[fillstyle=solid,fillcolor=yellow,addfillstyle=boxfill]#1}%
%\newcommand{\Neighbor}[2]{%
%    \begin{pspicture}(5,5)
%      \rput{*0}(2.5,2.5){\StylePosition{#1}}
%      \ReceiveData{(0.5,0)(4.5,0.5)}{\Central}\SendArea{(0.5,0.5)(4.5,1)}%
%      \SubDomain{(5,2)}{(0.5,0.5)(4.5,3)}{dashed}{#2}%
%      \pcarc[arcangle=45,arrows=->](0.5,-1.25)(0.5,0.25)%
%      \pcarc[arcangle=45,arrows=->,linestyle=dotted,dotsep=2pt](4.5,0.75)(4.5,-0.75)%
%    \end{pspicture}}%
%  \psset{dimen=middle,dotscale=2,fillloopadd=2}
%\begin{pspicture}(-5.7,-5.7)(5.7,5.7)
%  \rput(0,0){%
%      \begin{pspicture}(5,5)
%        \ReceiveData{(0,0.5)(0.5,4.5)}{\West} \ReceiveData{(4.5,0.5)(5,4.5)}{\East}
%        \ReceiveData{(0.5,4.5)(4.5,5)}{\North}\ReceiveData{(0.5,0)(4.5,0.5)}{\South}
%        \SendArea{(0.5,0.5)(1,4.5)}\SendArea{(4,0.5)(4.5,4.5)}
%        \SendArea{(0.5,0.5)(4.5,1)}\SendArea{(0.5,4)(4.5,4.5)}
%        \SubDomain{(5,5)}{(0.5,0.5)(4.5,4.5)}{solid}{\Central}
%        \psline(1,0.5)(1,4.5)\psline(4,0.5)(4,4.5)%
%        \rput(1.5,4){\Cross}\rput(2,2){\Cross}%
%      \end{pspicture}}%
%  \rput(0,5.5){\Neighbor{N}{\North}}\rput{-90}(5.5,0){\Neighbor{E}{\East}}%
%  \rput{90}(-5.5,0){\Neighbor{W}{\West}}\rput{180}(0,-5.5){\Neighbor{S}{\South}}%
%\end{pspicture}
%
% \begin{lstlisting}
%   \newcommand{\Pattern}[1]{%
%     \begin{pspicture}(-0.25,-0.25)(0.25,0.25)\rput{*0}{\psdot[dotstyle=#1]}
%     \end{pspicture}}
%   \newcommand{\West}{\Pattern{o}}   \newcommand{\South}{\Pattern{x}}
%   \newcommand{\Central}{\Pattern{+}}\newcommand{\North}{\Pattern{square}}
%   \newcommand{\East}{\Pattern{triangle}}
%   \newcommand{\Cross}{%
%     \pspolygon[unit=0.5,linewidth=0.2,linecolor=red](0,0)(0,1)(1,1)(1,2)(2,2)(2,1)
%               (3,1)(3,0)(2,0)(2,-1)(1,-1)(1,0)}
%   \newcommand{\StylePosition}[1]{\LARGE\textcolor{red}{\textbf{#1}}}
%   \newcommand{\SubDomain}[4]{%
%     \psboxfill{#4}
%     \begin{psclip}{\psframe[linestyle=none]#1}
%       \psframe[linestyle=#3](5,5)\psframe[fillstyle=boxfill]#2
%     \end{psclip}}
%   \newcommand{\SendArea}[1]{\psframe[fillstyle=solid,fillcolor=cyan]#1}
%   \newcommand{\ReceiveData}[2]{%
%     \psboxfill{#2}
%     \psframe[fillstyle=solid,fillcolor=yellow,addfillstyle=boxfill]#1}
%   \newcommand{\Neighbor}[2]{%
%     \begin{pspicture}(5,5)
%       \rput{*0}(2.5,2.5){\StylePosition{#1}}
%       \ReceiveData{(0.5,0)(4.5,0.5)}{\Central}\SendArea{(0.5,0.5)(4.5,1)}
%       \SubDomain{(5,2)}{(0.5,0.5)(4.5,3)}{dashed}{#2}%
% ^^A       % Receive and send arrows
%       \pcarc[arcangle=45,arrows=->](0.5,-1.25)(0.5,0.25)
%       \pcarc[arcangle=45,arrows=->,linestyle=dotted,dotsep=2pt](4.5,0.75)(4.5,-0.75)
%     \end{pspicture}}
%   \psset{dimen=middle,dotscale=2,fillloopadd=2}
%   \begin{pspicture}(-5.7,-5.7)(5.7,5.7)
% ^^A     % Central domain
%     \rput(0,0){%
%       \begin{pspicture}(5,5)
% ^^A         % Receive from West, East, North and South
%         \ReceiveData{(0,0.5)(0.5,4.5)}{\West} \ReceiveData{(4.5,0.5)(5,4.5)}{\East}
%         \ReceiveData{(0.5,4.5)(4.5,5)}{\North}\ReceiveData{(0.5,0)(4.5,0.5)}{\South}
% ^^A         % send area for West, East, North and South
%         \SendArea{(0.5,0.5)(1,4.5)} \SendArea{(4,0.5)(4.5,4.5)}
%         \SendArea{(0.5,0.5)(4.5,1)} \SendArea{(0.5,4)(4.5,4.5)}
% ^^A         % Central domain
%         \SubDomain{(5,5)}{(0.5,0.5)(4.5,4.5)}{solid}{\Central}
% ^^A         % Redraw overlapped linesY
%         \psline(1,0.5)(1,4.5)  \psline(4,0.5)(4,4.5)
% ^^A         % Two crossesY
%         \rput(1.5,4){\Cross}  \rput(2,2){\Cross}
%       \end{pspicture}}
% ^^A     % The four neighborsY
%     \rput(0,5.5){\Neighbor{N}{\North}}     \rput{-90}(5.5,0){\Neighbor{E}{\East}}
%     \rput{90}(-5.5,0){\Neighbor{W}{\West}} \rput{180}(0,-5.5){\Neighbor{S}{\South}}
%   \end{pspicture}
% \end{lstlisting}
%
%
%
% Bibliography
% \begin{thebibliography}{99}
% \bibitem{PostScript95} Adobe, Systems~Incorporated, \emph{PostScript Language
% Reference Manual}, Addison-Wesley, 2~edition, 1995.
%
% \bibitem{Bolek98} Piotr Bolek, \MP{} and patterns, \emph{\TUB}, Volume~19,
% Number~3, pages 276--283, September 1998, \CTANref{mpattern}.
%
% \bibitem{MLgraphTSI} Emmanuel Chailloux, Guy Cousineau and Asc\'ander
% Su\'arez, Programmation fonctionnelle de graphismes pour la production
% d'illustrations techniques, \emph{Technique et science informatique},
% Volume~15, Number~7, pages 977--1007, 1996 (in french).
%
% \bibitem{Deledicq97} Andr\'e Deledicq, \emph{Le monde des pavages}, ACL
% \'Editions, 1997 (in french).
%
% \bibitem{EsperetGirou98} Philippe Esperet and Denis Girou,
% Coloriage du pavage dit de Truchet, Cahiers GUTenberg, Number~31,
% pages 5--18, December~1998  (in french).
%
% \bibitem{Girou94} Denis Girou, Pr\'esentation de PSTricks, \emph{Cahiers
% GUTenberg}, Number~16, pages 21--70, February~1994 (in french).
%
% \bibitem{LGC97} Michel Goossens, Sebastian Rahtz and Frank Mittelbach,
% \emph{The \LaTeX{} Graphics Companion}, Addison-Wesley, 2005.
%
% \bibitem{GS87} Branko Gr\"unbaum and Geoffrey Shephard, \emph{Tilings and
% Patterns}, Freeman and Company, 1987.
%
% \bibitem{Hoenig97} Alan Hoenig, \emph{\TeX{} Unbound: \LaTeX{} \& \TeX{}
% Strategies, Fonts, Graphics, and More}, Oxford University Press, 1997.
%
% \bibitem{XYpic} Kristoffer~H. Rose and Ross Moore, \XYpic. Pattern and Tile
% extension, available from \CTAN, 1991-1998, \CTANref{xypic}.
%
% \bibitem{LAAN96} Kees van der Laan, Paradigms: Just a little bit of PostScript,
% \emph{MAPS}, Volume~17, pages 137--150, 1996.
%
% \bibitem{LAAN97} Kees van der Laan, Tiling in PostScript and \MF{} -- Escher's
% wink, \emph{MAPS}, Volume~19, Number~2, pages 39--67, 1997.
%
% \bibitem{vanZandt93} Timothy Van Zandt, PSTricks. PostScript macros for
% Generic \TeX, available from \CTAN, 1993, \CTANref{pstricks}.
%
% \bibitem{vanZandtGirou94} Timothy Van Zandt and Denis Girou, Inside PSTricks,
% \emph{\TUB}, Volume~15, Number~3, pages 239--246, September 1994.
%
%
% \bibitem{voss07} Herbert Vo\ss, PSTricks -- Graphics for \TeX\ and \LaTeX, DANTE/Lehmanns, 4th ed., 2007.
% \bibitem{Wang65} Hao Wang, Games, Logic and Computers, \emph{Scientific
% American}, pages 98--106, November 1965.
% \end{thebibliography}
%
%
% \StopEventually{}
%
% ^^A .................... End of the documentation part ....................
%
% \section{Driver file}
%
%   The next bit of code contains the documentation driver file for \TeX{},
% i.e., the file that will produce the documentation you are currently
% reading. It will be extracted from this file by the \texttt{docstrip}
% program.
%
%    \begin{macrocode}
%<*driver>
\documentclass{ltxdoc}
\GetFileInfo{pst-fill.dtx}
%
\usepackage[T1]{fontenc}
\usepackage{lmodern}               % For PDF
\usepackage{graphicx}              % `graphicx' LaTeX standard package
\usepackage{showexpl}
\usepackage{mflogo}                % For the MetaFont and MetaPost logos
\input{random.tex}                 % Random macros from Donald Arseneau
\usepackage{url}                   % URLs convenient typesetting
\usepackage{multido}               % General loop macro
\usepackage[dvipsnames]{pstricks}  % PSTricks with the `color' extension
\usepackage{pst-text}              % PSTricks package for character path
\usepackage{pst-grad}              % PSTricks package for gradient filling
\usepackage{pst-node}              % PSTricks package for nodes
\usepackage[tiling]{pst-fill}      % PSTricks package for filling/tiling
%
\AtBeginDocument{%
%  \OnlyDescription % comment out for implementation details
  \EnableCrossrefs
  \CodelineIndex
  \RecordChanges}
\AtEndDocument{%
  \PrintIndex
  \setcounter{IndexColumns}{1}
  \PrintChanges}
\hbadness=7000            % Over and under full box warnings
\hfuzz=3pt
\begin{document}
  \DocInput{pst-fill.dtx}
\end{document}
%</driver>
%    \end{macrocode}
%
% \section{\texttt{pst-fill} \LaTeX{} wrapper}
%
%    \begin{macrocode}
%<*latex-wrapper>
\RequirePackage{pstricks}
\ProvidesPackage{pst-fill}[2005/09/13 package wrapper for 
  pst-fill.tex (hv)]
\DeclareOption{tiling}{\def\PstTiling{true}}
\ProcessOptions\relax
% \iffalse meta-comment, etc.
%%
%% Package `pst-fill.dtx'
%%
%% Denis Girou (CNRS/IDRIS - France) <Denis.Girou@idris.fr>
%% Herbert Voss <voss@pstricks.de>
%%
%% This program can be redistributed and/or modified under the terms
%% of the LaTeX Project Public License Distributed from CTAN archives
%% in directory macros/latex/base/lppl.txt.
%%
%% DESCRIPTION:
%%   `pst-fill' is a PSTricks package for filling and tiling areas 
%%
% \fi
% \changes{v1.01}{2007/03/10}{bugfix for incomplete ifx (hv)}
% \changes{v1.00}{2006/11/06}{use pst-xkey for extend keys (hv)}
% \changes{v0.99}{2004/08/17}{merge the VTeX and TeX versions (patch 4) (hv)}
% \changes{v0.98}{2004/06/22}{delete the Pst@Debug option and use the
%   the one from pstricks to prevent a clash with pst-gr3d (hv)}
% \changes{v0.97}{2001/10/09}{make it work with VTeX (mv)}
% \changes{v0.94}{1997/04/08}{With a \PstTiling macro defined (or "tiling" optional parameter
%   on \textbackslash usepackage[tiling]{pst-fill}), this file run exactly as
%   the original boxfill.tex file from Timothy, version 0.94,
%   except a correction in \textbackslash pst@ManualFillCycle to avoid a division by 0.
%   It's the default.}
% \changes{v0.93}{1997/04/07}{With a \textbackslash PstTiling macro defined (or "tiling" optional parameter
%   on \textbackslash usepackage[tiling]{pst-fill}) there are several add-ons
%   and changes to do `tiling' rather than `filling' in "automatic" mode :
%     - we fix the position of the beginning of tiling,
%     - we allow normally the framing of the area as expected, using
%       the line.... parameters
%     - we define move parameters fillmovex, fillmovey and fillmove,
%     - we define fillcyclex as previous fillcycle parameter, and add the
%       fillcycley and fillcycle (both fillcyclex and fillcycley) ones
%     - we can extend the tiling area using fillloopaddx, fillloopaddy and
%       fillloopadd parameters,
%     - we can debug and see the whole tiling area without clipping using
%       PstDebug parameter,
%     - for names consistancy, we can use fillangle in place of boxfillangle
%       and fillsize in place of boxfillsize,
%     - default value for fillsep is 0 and for fillsize is auto.}
%
% \DoNotIndex{\!,\",\#,\$,\%,\&,\',\(,\+,\*,\,,\-,\.,\/,\:,\;,\<,\=,\>,\?}
% \DoNotIndex{\@,\@B,\@K,\@cTq,\@f,\@fPl,\@ifnextchar,\@nameuse,\@oVk}
% \DoNotIndex{\[,\\,\],\^,\_,\ }
% \DoNotIndex{\^,\\^,\\\^,$\^$,$\\^$,$\\^$}
% \DoNotIndex{\0,\2,\4,\5,\6,\7,\8,}
% \DoNotIndex{\A,\a}
% \DoNotIndex{\B,\b,\Bc,\begin,\Bq,\Bqc}
% \DoNotIndex{\C,\c,\catcode,\cJA,\CodelineIndex,\csname}
% \DoNotIndex{\D,\def,\define@key,\Df,\divide,\DocInput,\documentclass,\pst@addfams}
% \DoNotIndex{\eCN,\edef,\else,\eHd,\eMcj,\EnableCrossrefs,\end,\endcsname}
% \DoNotIndex{\endCenterExample,\endExample,\endinput,\endpsclip}
% \DoNotIndex{\PrintIndex,\PrintChanges,\ProvidesFile}
% \DoNotIndex{\endpspicture,\endSideBySideExample,\Example}
% \DoNotIndex{\F,\f,\FdUrr,\fi,\filedate,\fileversion,\FV@Environment}
% \DoNotIndex{\FV@UseKeyValues,\FV@XRightMargin,\FVB@Example,\fvset}
% \DoNotIndex{\G,\g,\GetFileInfo,\gr,\GradientLoaded,\gsFKrbK@o,\gsj,\gsOX}
% \DoNotIndex{\hbadness,\hfuzz,\HLEmphasize,\HLMacro,\HLMacro@i}
% \DoNotIndex{\HLReverse,\HLReverse@i,\hqcu,\HqY}
% \DoNotIndex{\I,\i,\ifx,\input,\Ir,\IU}
% \DoNotIndex{\j,\jl,\JT,\JVodH}
% \DoNotIndex{\K,\k,\kfSlL}
% \DoNotIndex{\L,\let}
% \DoNotIndex{\message,\mHNa,\mIU}
% \DoNotIndex{\N,\nB,\newcmykcolor,\newdimen,\newif,\nW}
% \DoNotIndex{\O,\oCDJDo,\ocQhVI,\OnlyDescription,\oRKJ}
% \DoNotIndex{\P,\p,\ProvidesPackage,\psframe,\pslinewidth,\psset}
% \DoNotIndex{\PstAtCode,\PSTricksLoaded}
% \DoNotIndex{\q,\Qr,\qssRXq,\qu,\qXjFQp,\qYL}
% \DoNotIndex{\R,\r,\RecordChanges,\relax,\RlaYI,\rN,\Rp,\rp,\RPDXNn,\rput}
% \DoNotIndex{\S,\scalebox,\SgY,\SideBySide@Example,\SideBySideExample}
% \DoNotIndex{\SgY,\sk,\Sp,\space,\sZb}
% \DoNotIndex{\T,\the,\tw@}
% \DoNotIndex{\u,\UiSWGEf@,\uJi,\usepackage,\uVQdMM,\UYj}
% \DoNotIndex{\VerbatimEnvironment,\VerbatimInput,\VrC@}
% \DoNotIndex{\WhZ,\WjKCYb,\WNs}
% \DoNotIndex{\XkN,\XW}
% \DoNotIndex{\Z,\ZCM,\Ze}
% \DoNotIndex{\addtocounter,\advance,\alph,\arabic,\AtBeginDocument,\AtEndDocument}
% \DoNotIndex{\AtEndOfPackage,\begingroup,\bfseries,\bgroup,\box,\csname}
% \DoNotIndex{\else,\endcsname,\endgroup,\endinput,\expandafter,\fi}
% \DoNotIndex{\TeX,\z@,\p@,\@one,\xdef,\thr@@,\string,\sixt@@n,\reset,\or,\multiply,\repeat,\RequirePackage}
% \DoNotIndex{\@cclvi,\@ne,\@ehpa,\@nil,\copy,\dp,\global,\hbox,\hss,\ht,\ifodd,\ifdim,\ifcase,\kern}
% \DoNotIndex{\chardef,\loop,\leavevmode,\ifnum,\lower}
% \setcounter{IndexColumns}{2}
%
% ^^A To extend the height used for the text
%
% ^^A  Aligned labels in a description environment
%\newenvironment{Description}[1]{%
%\begin{list}{nothing}{\setlength{\leftmargin}{#1}
%\setlength{\labelwidth}{\leftmargin}\setlength{\labelsep}{1mm}}}
%{\end{list}}
%
% ^^A For macro names
%\DeclareRobustCommand\cs[1]{\texttt{\char`\\#1}}
%
%
% ^^A From ltugboat.cls
% ^^A For references
%\makeatletter
%\newcommand\acro[1]{\textsc{#1}\@}
%\def\CTAN{\acro{CTAN}}
%\let\texttub\textsl              % ^^A redefined in other situations
%\def\TUB{\texttub{TUGboat}}
%\def\TUG{\TeX\ \UG}
%\def\tug{\acro{TUG}}
%\def\UG{Users Group}
% ^^A For the bibliography 
%\let\@internalcite\cite
%\def\cite{\def\@citeseppen{-1000}%
%    \def\@cite##1##2{(##1\if@tempswa , ##2\fi)}%
%    \def\citeauthoryear##1##2##3{##1, ##3}\@internalcite}
%\def\etal{et\,al.\@}
%\newcommand\CTANdirectory[1]{\expandafter\urldef
%  \csname CTAN@#1\endcsname\path}
%\newcommand\CTANfile[1]{\expandafter\urldef
%  \csname CTAN@#1\endcsname\path}
%\newcommand\CTANref[1]{\expandafter\@setref\csname CTAN@#1\endcsname
%  \relax{#1}}
%\makeatother
% ^^A Define CTAN addresses 
%\CTANdirectory{mpattern}{graphics/metapost/macros/mpattern}
%\CTANdirectory{pstricks}{graphics/pstricks}
%\CTANdirectory{pst-fill.sty}{graphics/pstricks/latex/pst-fill.sty}
%\CTANdirectory{pst-fill}{graphics/pstricks/generic/pst-fill.tex}
%\CTANdirectory{Roegel}{graphics/metapost/contrib/macros/truchet}
%\CTANdirectory{xypic}{macros/generic/diagrams/xypic}
%
% ^^A Personal macros (D.G.)
% ^^A ----------------------
%
% ^^A Some colors used
%\definecolor{LemonChiffon}{rgb}{1.,0.98,0.8}
%\definecolor{LightBlue}   {rgb}{0.8,0.85,0.95}
%\definecolor{PaleGreen}   {rgb}{0.88,1,0.88}
%\definecolor{PeachPuff}   {rgb}{1.0,0.85,0.73}
%
% ^^A To define a unique string for TeX and LaTeX
%\newcommand{\AllTeX}{%
%{\rm(L\kern-.36em\raise.3ex\hbox{\sc a}\kern-.15em)%
%T\kern-.1667em\lower.7ex\hbox{E}\kern-.125emX}}
%
% ^^A Bibliography style
%\bibliographystyle{ltugbib}
%
% ^^A Name macros
%\newcommand{\FillPackage}{\textsf{`pst-fill'}}
%\newcommand{\XYpic}{%
%\leavevmode\hbox{\kern-.1em X\kern-.3em\lower.4ex\hbox{Y\kern-.15em}-pic}}
%
%\makeatletter
%
% ^^A Example environments
% ^^A (do not use in them the four JXYZ characters, that we will use
% ^^A as escape characters!)
%
% ^^A Default PSTricks parameters
%  \psset{dimen=middle}
%
% ^^A Translation in PSTricks from the one drawn by Emmanuel Chailloux and
% ^^A Guy Cousineau for the MLgraph system
% ^^A (see /ftp.ens.fr:/pub/unix/lang/MLgraph/version-2.1/MLgraph-refman.ps.gz)
% ^^A The kangaroo itself is reproduce from an original picture from Raoul Raba
% \newcommand{\DimX}{2.47}
% \newcommand{\DimY}{4.8}
% \newcommand{\DimXDivTwo}{1.235}
%
% \newcommand{\KangarooItself}[1]{%
% ^^A Body
% \pspolygon[fillstyle=solid,fillcolor=#1]%
%  (52.5,68)(55,72.5)(55.8,76.5)(56.8,79.8)(58.2,83)(60,85.8)(61.5,86.5)
% (64,87)(66,87.5)(67.8,87.3)(70,87)(71.5,87.3)(73,88)(74.7,88.5)
% (76,90.3)(77,91.5)(72.8,93.8)(69,96)(64.5,99)(59.4,103)(56.2,106.3)
% (53,110.5)(49.5,115.5)(47.2,119.9)(45.7,126)(43.2,123)(41.5,121)(37.5,125)
% (37,122.5)(36.8,120)(37,117)(37.6,113.5)(38.6,110)(40,106.3)(42,102.3)
%  (43.5,99.5)(45,97)(46.2,94)(46.8,91.7)(47.2,88)(47,83.5)(46.3,80.8)
%  (45.3,78.5)(42.5,76.5)(39.5,75.8)(36,75.9)(33,75.9)(29,76.2)(26,77)
%  (22.3,77.5)(18,78.4)(12.8,79.3)(8.6,80)(5.5,80.3)(3,80.5)(0,80)
%  (-5.2,78.5)(-9,76.3)(-11.2,74.8)(-13,72.5)(-16.5,68)(-16.5,68)(-19.5,62.5)
%  (-22,58)(-25.5,53)(-29,48.5)(-32.5,45)(-36,42)(-39,39.5)(-44,37)
%  (-49,35)(-51,34)(-53.5,34.5)(-55.5,36)(-56.5,38)(-56.5,40.5)(-55,41.5)
%  (-53.5,41)(-51.5,41)(-50.5,43)(-50.5,44.5)(-51,47)(-51.5,47.2)(-56.5,47)
%  (-58.5,46.5)(-60,44.7)(-62,42.3)(-63,39.5)(-63.5,36.3)(-63.5,33)(-63.1,29.5)
%  (-61.5,26)(-58,23.6)(-54,22.2)(-50.7,22)(-47.5,22)(-44.5,22.3)(-41,23.5)
%  (-36.8,25.8)(-33,28)(-28.5,31)(-23.4,35)(-20.2,38.3)(-17,42.5)(-13.5,47.5)
%  (-11.2,51.9)(-9.7,58)(-7.2,55)(-5.5,53)(-1.5,57)(-1,54.5)(-0.8,52)
%  (-1,49)(-1.6,45.5)(-2.6,42)(-4,38.3)(-6,34.3)(-7.5,31.5)(-9,29)
%  (-10.2,26)(-10.8,23.7)(-11.2,20)(-11,15.5)(-10.3,12.8)(-9.3,10.5)(-6.5,8.5)
%  (-3.5,7.8)(0,7.9)(3,7.9)(7,8.2)(10,9)(13.7,9.5)(18,10.4)
%  (23.2,11.3)(27.4,12)(30.5,12.3)(33,12.5)(36,12)(41.2,10.5)(45,8.3)
%  (47.2,6.8)(49,4.5)(52.5,0)(50,4.5)(49.2,8.5)(48.2,11.8)(46.8,15)
%  (45,17.8)(43.5,18.5)(41,19)(39,19.5)(37.2,19.3)(35,19)(33.5,19.3)
%  (32,20)(30.3,20.5)(29,22.3)(28,23.5)(28,23.5)(24.5,22.3)(21.5,22)
%  (18.3,22)(15,22.2)(11,23.6)(7.5,26)(5.9,29.5)(5.5,33)(5.5,36.3)
%  (6,39.5)(7,42.3)(9,44.7)(10.5,46.5)(12.5,47)(17.5,47.2)(18,47)
%  (18.5,44.5)(18.5,43)(17.5,41)(15.5,41)(14,41.5)(12.5,40.5)(12.5,38)
%  (13.5,36)(15.5,34.5)(18,34)(20,35)(25,37)(30,39.5)(33,42)
%  (36.5,45)(40,48.5)(43.5,53)(47,58)(49.5,62.5)(52.5,68)
% ^^A Eye
% \pscircle*[linecolor=white](58.2,98.3){2\psxunit}
% \pscircle*(58.2,97.3){\psxunit}
% ^^A Mouth
% \psline(71.5,88)(70,89.3)(68.5,90.3)(67,91.9)
% ^^A Tear
% \psline(42,121)(45,118)(47,115.3)(48.5,112.7)(50,110)(51.8,106.5)
%       (52.5,103.7)(53,100.5)
% \pspolygon(41.2,115.8)(43.2,114.7)(45,112.5)(47,109.8)(48,107)(49.5,104.2)%
%       (50.5,101.6)(51,98.5)(47.7,100.6)(46,102.2)(44.8,104)(43.5,106)
%       (42.5,108)(41.7,110.5)(41,113.2)}
%
% \newcommand{\Kangaroo}[1]{%
%   \begin{pspicture}(\DimX,\DimY)
%   \psset{unit=0.035278}
%   \KangarooItself{#1}
%   \end{pspicture}}
%
% \newcommand{\KangarooPstChart}[1]{{%
%   \psset{xunit=0.006784,yunit=0.00735,linewidth=0.01}
%   \begin{pspicture}(-65.5,0)(82,126)
%     \KangarooItself{#1}
%   \end{pspicture}}}
%
%
% ^^A For the possible index and changes log
% \setlength{\columnseprule}{0.6pt}
%
% ^^A Beginning of the documentation itself
%\title{\texttt{pst-fill}\\A PSTricks package for filling and tiling areas}
%\author{Timothy Van Zandt\thanks{\protect\url{tvz@econ.insead.fr}. (documentation by
% Denis Girou (\protect\url{Denis.Girou@idris.fr}) and Herbert Vo\ss (\protect\url{hvoss@tug.org}).}}
%
%\date{\shortstack{\today --- Version 1.00\\
%                  {\small Documentation revised \today}}}
% \maketitle
% \tableofcontents
%
%\begin{abstract}
%  \FillPackage{} is a PSTricks \cite{vanZandt93},\cite{Girou94},\cite{vanZandtGirou94}, 
%\cite{Hoenig97},\cite{LGC97} package to draw easily
%  various kinds of filling and tiling of areas. It is also a good example of
%  the great power and flexibility of PSTricks, as in fact it is a very short
%  program (it body is around 200~lines long) but nevertheless really powerful.
%
%  \hspace{5mm} It was written in 1994 by Timothy \textsc{van Zandt} but
%  publicly available only in PSTricks 97 and without any documentation.
%  We describe here the version \emph{97 patch 2} of December 12, 1997, which
%  is the original one modified by myself to manage \emph{tilings} in the
%  so-called \emph{automatic} mode. This article would like to serve both of
%  reference manual and of user's guide.
%
%This package is available on \CTAN{} in the
%  \texttt{graphics/pstricks} directory (files \texttt{latex/pst-fill.sty} and
%  \texttt{generic/pst-fill.tex}).
%\end{abstract}
%
%\section{Introduction}
%
%  Here we will refer as \emph{filling} as the operation which consist to fill
%a defined area by a pattern (or a composition of patterns). We will refer as
%\emph{tiling} as the operation which consist to do the same thing, but with
%the control of the starting point, which is here the upper left corner.
%The pattern is positioned relatively to this point. This make an essential
%difference between the two modes, as without control of the starting point we
%can't draw \emph{tilings} (sometimes  called \emph{tesselations}) as used in
%many fields of Art and Science%
%\footnote{For an extensive presentation of tilings, in their history and usage
%in many fields, see the reference book \cite{GS87}.
%
%  In the \TeX{} world, few work was done on tilings. You can look at the
%\emph{tile} extension of the \XYpic{} package \cite{XYpic}, at the articles of
%Kees \textsc{van der Laan} \cite[paragraph 7]{LAAN96} (the tiling was in
%fact directly done in PostScript) and \cite{LAAN97}, at the \MP{} program
%(available on \CTANref{Roegel}) by Denis \textsc{Roegel} for the
%\textsc{Truchet} contest in 1995 \cite{EsperetGirou98} and at the \MP{}
%package \cite{Bolek98} to draw patterns, which have a strong connection with
%tilings.}.
%
%  Nevertheless, as tilings are a wide and difficult field in mathematics, this
%package is limited to simple ones, mainly \emph{monohedral} tilings with one
%prototile (which can be composite, see section \ref{sec:KindTiles}). With some
%experience and wiliness we can do more and obtained easily rather
%sophisticated results, but obviously hyperbolic tilings like the famous
%\textsc{Escher} ones or aperiodic tilings like the \textsc{Penrose} ones are
%not in the capabilities of this package. For more complex needs, we must used
%low level and more painfull technics, with the basic \cs{multido}
%and \cs{multirput} macros.
%
%\section{Package history and description of it two different modes}
%
%  As already said, this package was written in 1994 by Timothy \textsc{van
%Zandt}. Two modes were defined, called respectively \emph{manual} and
%\emph{automatic}. For both, the pattern is generated on contiguous positions in
%a rather large area which include the region to fill, later cut to the
%required dimensions by clipping mechanism. In the first mode, the pattern is
%explicitely inserted in the PostScript file each time. In the second one, the
%result is the same but with an unique explicit insertion of the pattern and a
%repetition done by PostScript. Nevertheless, in this method, the control of
%the starting point was loosed, so it allowed only to \emph{fill} a region and
%not to \emph{tile} it.
%
%  See the difference between the two modes, \emph{tiling}:
% {\psset{unit=0.5cm}%
% \psboxfill{\begin{pspicture}(1,1)\psframe[dimen=middle](1,1)\end{pspicture}}
% \begin{pspicture}(3,3.3)
%   \psframe[fillstyle=boxfill](3,3)
% \end{pspicture}}
% and \emph{filling}:
%{%
% \makeatletter
%\pst@def{BoxFill}<
%  gsave
%    gsave \tx@STV CM grestore dtransform CM idtransform
%    abs /h ED abs /w ED
%    pathbbox
%    h div round 2 add cvi /y2 ED
%    w div round 2 add cvi /x2 ED
%    h div round 2 sub cvi /y1 ED
%    w div round 2 sub cvi /x1 ED
%    /y2 y2 y1 sub def
%    /x2 x2 x1 sub def
%    CP
%    y1 h mul sub neg /y1 ED
%    x1 w mul sub neg /x1 ED
%    clip
%    y2 {
%      /x x1 def
%      x2 {
%        save CP x y1 T moveto Box restore
%        /x x w add def
%      } repeat
%      /y1 y1 h add def
%    } repeat
% currentpoint currentfont grestore setfont moveto>
% \makeatother
%
% \psset{unit=0.5}
% \psboxfill{\begin{pspicture}(1,1)\psframe[dimen=middle](1,1)\end{pspicture}}
% \begin{pspicture}(3,3.3)
%   \psframe[fillstyle=boxfill](3,3)
% \end{pspicture}
% or
% \begin{pspicture}(3,3.3)
%   \psframe[fillstyle=boxfill](3,3)
% \end{pspicture}
%}
%as we can see that initial position is arbitrary and dependent of
%the current point.
%
%
% It's clear that usage of filling is very restrictive comparing to tiling,
%as desired effects required very often the possibility to control the starting 
%point. So, this mode was of limited interest, but unfortunately the
%\emph{manual} one has the very big disadvantage to require very huge amounts
%of ressources, mainly in disk space and consequently in printing time.
%A small tiling can require sometimes several megabytes in \emph{manual} mode!
%So, it was very often not really usable in practice.
%
%It is why I modified the code, to allow tilings in \emph{automatic} mode,
%controlling in this mode too the starting point. And most of the time, that is
%to say if some special options are not used, the tiling is done exactly in the
%region described, which make it faster. So there is no more reason to use the
%\emph{manual} mode, apart very special cases where \emph{automatic} one cannot
%work, as explained later -- currently, I know only one case.
%
%  To load this modified \emph{automatic} mode, with \LaTeX{} use
%simply:\newline 
%\verb+\usepackage[tiling]{pst-fill}+\newline
%and in plain \TeX{} after:\newline
%\verb+% \iffalse meta-comment, etc.
%%
%% Package `pst-fill.dtx'
%%
%% Denis Girou (CNRS/IDRIS - France) <Denis.Girou@idris.fr>
%% Herbert Voss <voss@pstricks.de>
%%
%% This program can be redistributed and/or modified under the terms
%% of the LaTeX Project Public License Distributed from CTAN archives
%% in directory macros/latex/base/lppl.txt.
%%
%% DESCRIPTION:
%%   `pst-fill' is a PSTricks package for filling and tiling areas 
%%
% \fi
% \changes{v1.01}{2007/03/10}{bugfix for incomplete ifx (hv)}
% \changes{v1.00}{2006/11/06}{use pst-xkey for extend keys (hv)}
% \changes{v0.99}{2004/08/17}{merge the VTeX and TeX versions (patch 4) (hv)}
% \changes{v0.98}{2004/06/22}{delete the Pst@Debug option and use the
%   the one from pstricks to prevent a clash with pst-gr3d (hv)}
% \changes{v0.97}{2001/10/09}{make it work with VTeX (mv)}
% \changes{v0.94}{1997/04/08}{With a \PstTiling macro defined (or "tiling" optional parameter
%   on \textbackslash usepackage[tiling]{pst-fill}), this file run exactly as
%   the original boxfill.tex file from Timothy, version 0.94,
%   except a correction in \textbackslash pst@ManualFillCycle to avoid a division by 0.
%   It's the default.}
% \changes{v0.93}{1997/04/07}{With a \textbackslash PstTiling macro defined (or "tiling" optional parameter
%   on \textbackslash usepackage[tiling]{pst-fill}) there are several add-ons
%   and changes to do `tiling' rather than `filling' in "automatic" mode :
%     - we fix the position of the beginning of tiling,
%     - we allow normally the framing of the area as expected, using
%       the line.... parameters
%     - we define move parameters fillmovex, fillmovey and fillmove,
%     - we define fillcyclex as previous fillcycle parameter, and add the
%       fillcycley and fillcycle (both fillcyclex and fillcycley) ones
%     - we can extend the tiling area using fillloopaddx, fillloopaddy and
%       fillloopadd parameters,
%     - we can debug and see the whole tiling area without clipping using
%       PstDebug parameter,
%     - for names consistancy, we can use fillangle in place of boxfillangle
%       and fillsize in place of boxfillsize,
%     - default value for fillsep is 0 and for fillsize is auto.}
%
% \DoNotIndex{\!,\",\#,\$,\%,\&,\',\(,\+,\*,\,,\-,\.,\/,\:,\;,\<,\=,\>,\?}
% \DoNotIndex{\@,\@B,\@K,\@cTq,\@f,\@fPl,\@ifnextchar,\@nameuse,\@oVk}
% \DoNotIndex{\[,\\,\],\^,\_,\ }
% \DoNotIndex{\^,\\^,\\\^,$\^$,$\\^$,$\\^$}
% \DoNotIndex{\0,\2,\4,\5,\6,\7,\8,}
% \DoNotIndex{\A,\a}
% \DoNotIndex{\B,\b,\Bc,\begin,\Bq,\Bqc}
% \DoNotIndex{\C,\c,\catcode,\cJA,\CodelineIndex,\csname}
% \DoNotIndex{\D,\def,\define@key,\Df,\divide,\DocInput,\documentclass,\pst@addfams}
% \DoNotIndex{\eCN,\edef,\else,\eHd,\eMcj,\EnableCrossrefs,\end,\endcsname}
% \DoNotIndex{\endCenterExample,\endExample,\endinput,\endpsclip}
% \DoNotIndex{\PrintIndex,\PrintChanges,\ProvidesFile}
% \DoNotIndex{\endpspicture,\endSideBySideExample,\Example}
% \DoNotIndex{\F,\f,\FdUrr,\fi,\filedate,\fileversion,\FV@Environment}
% \DoNotIndex{\FV@UseKeyValues,\FV@XRightMargin,\FVB@Example,\fvset}
% \DoNotIndex{\G,\g,\GetFileInfo,\gr,\GradientLoaded,\gsFKrbK@o,\gsj,\gsOX}
% \DoNotIndex{\hbadness,\hfuzz,\HLEmphasize,\HLMacro,\HLMacro@i}
% \DoNotIndex{\HLReverse,\HLReverse@i,\hqcu,\HqY}
% \DoNotIndex{\I,\i,\ifx,\input,\Ir,\IU}
% \DoNotIndex{\j,\jl,\JT,\JVodH}
% \DoNotIndex{\K,\k,\kfSlL}
% \DoNotIndex{\L,\let}
% \DoNotIndex{\message,\mHNa,\mIU}
% \DoNotIndex{\N,\nB,\newcmykcolor,\newdimen,\newif,\nW}
% \DoNotIndex{\O,\oCDJDo,\ocQhVI,\OnlyDescription,\oRKJ}
% \DoNotIndex{\P,\p,\ProvidesPackage,\psframe,\pslinewidth,\psset}
% \DoNotIndex{\PstAtCode,\PSTricksLoaded}
% \DoNotIndex{\q,\Qr,\qssRXq,\qu,\qXjFQp,\qYL}
% \DoNotIndex{\R,\r,\RecordChanges,\relax,\RlaYI,\rN,\Rp,\rp,\RPDXNn,\rput}
% \DoNotIndex{\S,\scalebox,\SgY,\SideBySide@Example,\SideBySideExample}
% \DoNotIndex{\SgY,\sk,\Sp,\space,\sZb}
% \DoNotIndex{\T,\the,\tw@}
% \DoNotIndex{\u,\UiSWGEf@,\uJi,\usepackage,\uVQdMM,\UYj}
% \DoNotIndex{\VerbatimEnvironment,\VerbatimInput,\VrC@}
% \DoNotIndex{\WhZ,\WjKCYb,\WNs}
% \DoNotIndex{\XkN,\XW}
% \DoNotIndex{\Z,\ZCM,\Ze}
% \DoNotIndex{\addtocounter,\advance,\alph,\arabic,\AtBeginDocument,\AtEndDocument}
% \DoNotIndex{\AtEndOfPackage,\begingroup,\bfseries,\bgroup,\box,\csname}
% \DoNotIndex{\else,\endcsname,\endgroup,\endinput,\expandafter,\fi}
% \DoNotIndex{\TeX,\z@,\p@,\@one,\xdef,\thr@@,\string,\sixt@@n,\reset,\or,\multiply,\repeat,\RequirePackage}
% \DoNotIndex{\@cclvi,\@ne,\@ehpa,\@nil,\copy,\dp,\global,\hbox,\hss,\ht,\ifodd,\ifdim,\ifcase,\kern}
% \DoNotIndex{\chardef,\loop,\leavevmode,\ifnum,\lower}
% \setcounter{IndexColumns}{2}
%
% ^^A To extend the height used for the text
%
% ^^A  Aligned labels in a description environment
%\newenvironment{Description}[1]{%
%\begin{list}{nothing}{\setlength{\leftmargin}{#1}
%\setlength{\labelwidth}{\leftmargin}\setlength{\labelsep}{1mm}}}
%{\end{list}}
%
% ^^A For macro names
%\DeclareRobustCommand\cs[1]{\texttt{\char`\\#1}}
%
%
% ^^A From ltugboat.cls
% ^^A For references
%\makeatletter
%\newcommand\acro[1]{\textsc{#1}\@}
%\def\CTAN{\acro{CTAN}}
%\let\texttub\textsl              % ^^A redefined in other situations
%\def\TUB{\texttub{TUGboat}}
%\def\TUG{\TeX\ \UG}
%\def\tug{\acro{TUG}}
%\def\UG{Users Group}
% ^^A For the bibliography 
%\let\@internalcite\cite
%\def\cite{\def\@citeseppen{-1000}%
%    \def\@cite##1##2{(##1\if@tempswa , ##2\fi)}%
%    \def\citeauthoryear##1##2##3{##1, ##3}\@internalcite}
%\def\etal{et\,al.\@}
%\newcommand\CTANdirectory[1]{\expandafter\urldef
%  \csname CTAN@#1\endcsname\path}
%\newcommand\CTANfile[1]{\expandafter\urldef
%  \csname CTAN@#1\endcsname\path}
%\newcommand\CTANref[1]{\expandafter\@setref\csname CTAN@#1\endcsname
%  \relax{#1}}
%\makeatother
% ^^A Define CTAN addresses 
%\CTANdirectory{mpattern}{graphics/metapost/macros/mpattern}
%\CTANdirectory{pstricks}{graphics/pstricks}
%\CTANdirectory{pst-fill.sty}{graphics/pstricks/latex/pst-fill.sty}
%\CTANdirectory{pst-fill}{graphics/pstricks/generic/pst-fill.tex}
%\CTANdirectory{Roegel}{graphics/metapost/contrib/macros/truchet}
%\CTANdirectory{xypic}{macros/generic/diagrams/xypic}
%
% ^^A Personal macros (D.G.)
% ^^A ----------------------
%
% ^^A Some colors used
%\definecolor{LemonChiffon}{rgb}{1.,0.98,0.8}
%\definecolor{LightBlue}   {rgb}{0.8,0.85,0.95}
%\definecolor{PaleGreen}   {rgb}{0.88,1,0.88}
%\definecolor{PeachPuff}   {rgb}{1.0,0.85,0.73}
%
% ^^A To define a unique string for TeX and LaTeX
%\newcommand{\AllTeX}{%
%{\rm(L\kern-.36em\raise.3ex\hbox{\sc a}\kern-.15em)%
%T\kern-.1667em\lower.7ex\hbox{E}\kern-.125emX}}
%
% ^^A Bibliography style
%\bibliographystyle{ltugbib}
%
% ^^A Name macros
%\newcommand{\FillPackage}{\textsf{`pst-fill'}}
%\newcommand{\XYpic}{%
%\leavevmode\hbox{\kern-.1em X\kern-.3em\lower.4ex\hbox{Y\kern-.15em}-pic}}
%
%\makeatletter
%
% ^^A Example environments
% ^^A (do not use in them the four JXYZ characters, that we will use
% ^^A as escape characters!)
%
% ^^A Default PSTricks parameters
%  \psset{dimen=middle}
%
% ^^A Translation in PSTricks from the one drawn by Emmanuel Chailloux and
% ^^A Guy Cousineau for the MLgraph system
% ^^A (see /ftp.ens.fr:/pub/unix/lang/MLgraph/version-2.1/MLgraph-refman.ps.gz)
% ^^A The kangaroo itself is reproduce from an original picture from Raoul Raba
% \newcommand{\DimX}{2.47}
% \newcommand{\DimY}{4.8}
% \newcommand{\DimXDivTwo}{1.235}
%
% \newcommand{\KangarooItself}[1]{%
% ^^A Body
% \pspolygon[fillstyle=solid,fillcolor=#1]%
%  (52.5,68)(55,72.5)(55.8,76.5)(56.8,79.8)(58.2,83)(60,85.8)(61.5,86.5)
% (64,87)(66,87.5)(67.8,87.3)(70,87)(71.5,87.3)(73,88)(74.7,88.5)
% (76,90.3)(77,91.5)(72.8,93.8)(69,96)(64.5,99)(59.4,103)(56.2,106.3)
% (53,110.5)(49.5,115.5)(47.2,119.9)(45.7,126)(43.2,123)(41.5,121)(37.5,125)
% (37,122.5)(36.8,120)(37,117)(37.6,113.5)(38.6,110)(40,106.3)(42,102.3)
%  (43.5,99.5)(45,97)(46.2,94)(46.8,91.7)(47.2,88)(47,83.5)(46.3,80.8)
%  (45.3,78.5)(42.5,76.5)(39.5,75.8)(36,75.9)(33,75.9)(29,76.2)(26,77)
%  (22.3,77.5)(18,78.4)(12.8,79.3)(8.6,80)(5.5,80.3)(3,80.5)(0,80)
%  (-5.2,78.5)(-9,76.3)(-11.2,74.8)(-13,72.5)(-16.5,68)(-16.5,68)(-19.5,62.5)
%  (-22,58)(-25.5,53)(-29,48.5)(-32.5,45)(-36,42)(-39,39.5)(-44,37)
%  (-49,35)(-51,34)(-53.5,34.5)(-55.5,36)(-56.5,38)(-56.5,40.5)(-55,41.5)
%  (-53.5,41)(-51.5,41)(-50.5,43)(-50.5,44.5)(-51,47)(-51.5,47.2)(-56.5,47)
%  (-58.5,46.5)(-60,44.7)(-62,42.3)(-63,39.5)(-63.5,36.3)(-63.5,33)(-63.1,29.5)
%  (-61.5,26)(-58,23.6)(-54,22.2)(-50.7,22)(-47.5,22)(-44.5,22.3)(-41,23.5)
%  (-36.8,25.8)(-33,28)(-28.5,31)(-23.4,35)(-20.2,38.3)(-17,42.5)(-13.5,47.5)
%  (-11.2,51.9)(-9.7,58)(-7.2,55)(-5.5,53)(-1.5,57)(-1,54.5)(-0.8,52)
%  (-1,49)(-1.6,45.5)(-2.6,42)(-4,38.3)(-6,34.3)(-7.5,31.5)(-9,29)
%  (-10.2,26)(-10.8,23.7)(-11.2,20)(-11,15.5)(-10.3,12.8)(-9.3,10.5)(-6.5,8.5)
%  (-3.5,7.8)(0,7.9)(3,7.9)(7,8.2)(10,9)(13.7,9.5)(18,10.4)
%  (23.2,11.3)(27.4,12)(30.5,12.3)(33,12.5)(36,12)(41.2,10.5)(45,8.3)
%  (47.2,6.8)(49,4.5)(52.5,0)(50,4.5)(49.2,8.5)(48.2,11.8)(46.8,15)
%  (45,17.8)(43.5,18.5)(41,19)(39,19.5)(37.2,19.3)(35,19)(33.5,19.3)
%  (32,20)(30.3,20.5)(29,22.3)(28,23.5)(28,23.5)(24.5,22.3)(21.5,22)
%  (18.3,22)(15,22.2)(11,23.6)(7.5,26)(5.9,29.5)(5.5,33)(5.5,36.3)
%  (6,39.5)(7,42.3)(9,44.7)(10.5,46.5)(12.5,47)(17.5,47.2)(18,47)
%  (18.5,44.5)(18.5,43)(17.5,41)(15.5,41)(14,41.5)(12.5,40.5)(12.5,38)
%  (13.5,36)(15.5,34.5)(18,34)(20,35)(25,37)(30,39.5)(33,42)
%  (36.5,45)(40,48.5)(43.5,53)(47,58)(49.5,62.5)(52.5,68)
% ^^A Eye
% \pscircle*[linecolor=white](58.2,98.3){2\psxunit}
% \pscircle*(58.2,97.3){\psxunit}
% ^^A Mouth
% \psline(71.5,88)(70,89.3)(68.5,90.3)(67,91.9)
% ^^A Tear
% \psline(42,121)(45,118)(47,115.3)(48.5,112.7)(50,110)(51.8,106.5)
%       (52.5,103.7)(53,100.5)
% \pspolygon(41.2,115.8)(43.2,114.7)(45,112.5)(47,109.8)(48,107)(49.5,104.2)%
%       (50.5,101.6)(51,98.5)(47.7,100.6)(46,102.2)(44.8,104)(43.5,106)
%       (42.5,108)(41.7,110.5)(41,113.2)}
%
% \newcommand{\Kangaroo}[1]{%
%   \begin{pspicture}(\DimX,\DimY)
%   \psset{unit=0.035278}
%   \KangarooItself{#1}
%   \end{pspicture}}
%
% \newcommand{\KangarooPstChart}[1]{{%
%   \psset{xunit=0.006784,yunit=0.00735,linewidth=0.01}
%   \begin{pspicture}(-65.5,0)(82,126)
%     \KangarooItself{#1}
%   \end{pspicture}}}
%
%
% ^^A For the possible index and changes log
% \setlength{\columnseprule}{0.6pt}
%
% ^^A Beginning of the documentation itself
%\title{\texttt{pst-fill}\\A PSTricks package for filling and tiling areas}
%\author{Timothy Van Zandt\thanks{\protect\url{tvz@econ.insead.fr}. (documentation by
% Denis Girou (\protect\url{Denis.Girou@idris.fr}) and Herbert Vo\ss (\protect\url{hvoss@tug.org}).}}
%
%\date{\shortstack{\today --- Version 1.00\\
%                  {\small Documentation revised \today}}}
% \maketitle
% \tableofcontents
%
%\begin{abstract}
%  \FillPackage{} is a PSTricks \cite{vanZandt93},\cite{Girou94},\cite{vanZandtGirou94}, 
%\cite{Hoenig97},\cite{LGC97} package to draw easily
%  various kinds of filling and tiling of areas. It is also a good example of
%  the great power and flexibility of PSTricks, as in fact it is a very short
%  program (it body is around 200~lines long) but nevertheless really powerful.
%
%  \hspace{5mm} It was written in 1994 by Timothy \textsc{van Zandt} but
%  publicly available only in PSTricks 97 and without any documentation.
%  We describe here the version \emph{97 patch 2} of December 12, 1997, which
%  is the original one modified by myself to manage \emph{tilings} in the
%  so-called \emph{automatic} mode. This article would like to serve both of
%  reference manual and of user's guide.
%
%This package is available on \CTAN{} in the
%  \texttt{graphics/pstricks} directory (files \texttt{latex/pst-fill.sty} and
%  \texttt{generic/pst-fill.tex}).
%\end{abstract}
%
%\section{Introduction}
%
%  Here we will refer as \emph{filling} as the operation which consist to fill
%a defined area by a pattern (or a composition of patterns). We will refer as
%\emph{tiling} as the operation which consist to do the same thing, but with
%the control of the starting point, which is here the upper left corner.
%The pattern is positioned relatively to this point. This make an essential
%difference between the two modes, as without control of the starting point we
%can't draw \emph{tilings} (sometimes  called \emph{tesselations}) as used in
%many fields of Art and Science%
%\footnote{For an extensive presentation of tilings, in their history and usage
%in many fields, see the reference book \cite{GS87}.
%
%  In the \TeX{} world, few work was done on tilings. You can look at the
%\emph{tile} extension of the \XYpic{} package \cite{XYpic}, at the articles of
%Kees \textsc{van der Laan} \cite[paragraph 7]{LAAN96} (the tiling was in
%fact directly done in PostScript) and \cite{LAAN97}, at the \MP{} program
%(available on \CTANref{Roegel}) by Denis \textsc{Roegel} for the
%\textsc{Truchet} contest in 1995 \cite{EsperetGirou98} and at the \MP{}
%package \cite{Bolek98} to draw patterns, which have a strong connection with
%tilings.}.
%
%  Nevertheless, as tilings are a wide and difficult field in mathematics, this
%package is limited to simple ones, mainly \emph{monohedral} tilings with one
%prototile (which can be composite, see section \ref{sec:KindTiles}). With some
%experience and wiliness we can do more and obtained easily rather
%sophisticated results, but obviously hyperbolic tilings like the famous
%\textsc{Escher} ones or aperiodic tilings like the \textsc{Penrose} ones are
%not in the capabilities of this package. For more complex needs, we must used
%low level and more painfull technics, with the basic \cs{multido}
%and \cs{multirput} macros.
%
%\section{Package history and description of it two different modes}
%
%  As already said, this package was written in 1994 by Timothy \textsc{van
%Zandt}. Two modes were defined, called respectively \emph{manual} and
%\emph{automatic}. For both, the pattern is generated on contiguous positions in
%a rather large area which include the region to fill, later cut to the
%required dimensions by clipping mechanism. In the first mode, the pattern is
%explicitely inserted in the PostScript file each time. In the second one, the
%result is the same but with an unique explicit insertion of the pattern and a
%repetition done by PostScript. Nevertheless, in this method, the control of
%the starting point was loosed, so it allowed only to \emph{fill} a region and
%not to \emph{tile} it.
%
%  See the difference between the two modes, \emph{tiling}:
% {\psset{unit=0.5cm}%
% \psboxfill{\begin{pspicture}(1,1)\psframe[dimen=middle](1,1)\end{pspicture}}
% \begin{pspicture}(3,3.3)
%   \psframe[fillstyle=boxfill](3,3)
% \end{pspicture}}
% and \emph{filling}:
%{%
% \makeatletter
%\pst@def{BoxFill}<
%  gsave
%    gsave \tx@STV CM grestore dtransform CM idtransform
%    abs /h ED abs /w ED
%    pathbbox
%    h div round 2 add cvi /y2 ED
%    w div round 2 add cvi /x2 ED
%    h div round 2 sub cvi /y1 ED
%    w div round 2 sub cvi /x1 ED
%    /y2 y2 y1 sub def
%    /x2 x2 x1 sub def
%    CP
%    y1 h mul sub neg /y1 ED
%    x1 w mul sub neg /x1 ED
%    clip
%    y2 {
%      /x x1 def
%      x2 {
%        save CP x y1 T moveto Box restore
%        /x x w add def
%      } repeat
%      /y1 y1 h add def
%    } repeat
% currentpoint currentfont grestore setfont moveto>
% \makeatother
%
% \psset{unit=0.5}
% \psboxfill{\begin{pspicture}(1,1)\psframe[dimen=middle](1,1)\end{pspicture}}
% \begin{pspicture}(3,3.3)
%   \psframe[fillstyle=boxfill](3,3)
% \end{pspicture}
% or
% \begin{pspicture}(3,3.3)
%   \psframe[fillstyle=boxfill](3,3)
% \end{pspicture}
%}
%as we can see that initial position is arbitrary and dependent of
%the current point.
%
%
% It's clear that usage of filling is very restrictive comparing to tiling,
%as desired effects required very often the possibility to control the starting 
%point. So, this mode was of limited interest, but unfortunately the
%\emph{manual} one has the very big disadvantage to require very huge amounts
%of ressources, mainly in disk space and consequently in printing time.
%A small tiling can require sometimes several megabytes in \emph{manual} mode!
%So, it was very often not really usable in practice.
%
%It is why I modified the code, to allow tilings in \emph{automatic} mode,
%controlling in this mode too the starting point. And most of the time, that is
%to say if some special options are not used, the tiling is done exactly in the
%region described, which make it faster. So there is no more reason to use the
%\emph{manual} mode, apart very special cases where \emph{automatic} one cannot
%work, as explained later -- currently, I know only one case.
%
%  To load this modified \emph{automatic} mode, with \LaTeX{} use
%simply:\newline 
%\verb+\usepackage[tiling]{pst-fill}+\newline
%and in plain \TeX{} after:\newline
%\verb+\input{pst-fill}+\newline
%add the following definition:\newline
%\verb+\def\PstTiling{true}+
%
%  To obtain the original behaviour, just don't use the \emph{tiling} optional
%keyword at loading.
%
%  Take care than in \emph{tiling} mode, I introduce also some other changes.
%First I define aliases on some parameter names for consistancy (all specific
%parameters will begin by the \texttt{fill} prefix in this case) and I change
%some default values, which were not well adapted for tilings (\texttt{fillsep}
%is set to 0 and as explained \texttt{fillsize} set to \texttt{auto}). I rename 
%\texttt{fillcycle} to \texttt{fillcyclex}. I also restore normal way so that
%the frame of the area is drawn and all line (\texttt{linestyle},
%\texttt{linecolor}, \texttt{doubleline}, etc.) parameters are now active (but
%there are not in non \emph{tiling} mode). And I also introduce new parameters
%to control the tilings (see below).
%
%  \textbf{In all the following examples, we will consider only the
% \emph{tiling} mode.}
%
%  To do a tiling, we have just to define the pattern with the
% \verb+\psboxfill+ macro and to use the new \texttt{fillstyle}
% \verb+boxfill+.
%
%  Note that tilings are drawn from left to right and top to bottom, which can
%have an importance in some circonstances.
%
%  PostScript programmers can be also interested to know that, even in the
%\emph{automatic} mode, the iterations of the pattern are managed directly by
%the PostScript code of the package which used only PostScript Level 1
%operators. The special ones introduced in Level 2 for drawing of patterns
%\cite[section 4.9]{PostScript95} are not used.
%
%  And first, for conveniance, we define a simple \cs{Tiling} macro, which
%will simplify our examples:
%
%\begin{verbatim}
%  \newcommand{\Tiling}[2][]{%
%    \edef\Temp{#1}%
%    \begin{pspicture}#2
%      \ifx\Temp\empty
%        \psframe[fillstyle=boxfill]#2
%      \else
%        \psframe[fillstyle=boxfill,#1]#2
%      \fi
%    \end{pspicture}}
%\end{verbatim}
%
%
%\newcommand{\Tiling}[2][]{%
%  \edef\Temp{#1}%
%  \begin{pspicture}#2
%    \ifx\Temp\empty
%      \psframe[fillstyle=boxfill]#2
%    \else
%      \psframe[fillstyle=boxfill,#1]#2
%    \fi
% \end{pspicture}}
%
%\subsection{Parameters}
%
%  There are \textbf{14} specific parameters available to change the way the
% filling/tiling is defined, and one debugging option.
%
% \begin{Description}{2cm}
%  \item [fillangle (real)\hfill :] the value of the rotation
%  applied to the patterns (\emph{Default:~0}).
% \end{Description}
%
%
%   In this case, we must force the tiling area to be notably larger than the
% area to cover, to be sure that the defined area will be covered after rotation.
% \lstset{gobble=2}
% \begin{LTXexample}
% \newcommand{\Square}{%
%   \begin{pspicture}(1,1)
%     \psframe[dimen=middle](1,1)
%   \end{pspicture}}
% \psset{unit=0.5}
% \psboxfill{\Square}
% \Tiling[fillangle=45]{(3,3)}\quad
% \Tiling[fillangle=-60]{(3,3)}
% \end{LTXexample}
% 
% \newcommand{\Square}{\begin{pspicture}(1,1)\psframe[dimen=middle](1,1)\end{pspicture}}
% 
% \begin{Description}{2cm}
%   \setcounter{footnote}{1}
%   \item[\texttt{fillsepx} (real$\|$dim) :] value of the horizontal
%   separation between consecutive patterns (\emph{Default:~0 for
%   tilings\footnotemark, 2pt otherwise}).  \footnotetext{This option was added
%   by me, is not part of the original package and is available only if the
%   \texttt{tiling} keyword is used when loading the package.}
%   \setcounter{footnote}{1}
%   \item [\texttt{fillsepy} (real$\|$dim)\hfill :] value of the vertical
%   separation between consecutive patterns (\emph{Default:~0 for
%   ti\-lings\footnotemark, 2pt otherwise}).
%   \setcounter{footnote}{1}
%   \item [\texttt{fillsep} (real$\|$dim)\hfill :] value of horizontal and
%   vertical separations between consecutive patterns (\emph{Default:~0 for
%   tilings\footnotemark, 2pt otherwise}).
% \end{Description}
% 
%   These values can be negative, which allow the tiles to overlap.
% 
% \begin{LTXexample}
% \psset{unit=0.5}
% \psboxfill{\Square}
% \Tiling[fillsepx=2mm]{(3,3)} 
% \Tiling[fillsepy=1mm]{(3,3)}\\
% \Tiling[fillsep=0.5]{(3,3)} 
% \Tiling[fillsep=-0.5]{(3,3)}
% \end{LTXexample}
% 
% \begin{Description}{2cm}
%   \item [\texttt{fillcyclex}\footnotemark\ (integer)\hfill :] Shift
%   coefficient applied to each row (\emph{Default:~0}).
%   \footnotetext{It was \texttt{fillcycle} in the original version.}
%   \setcounter{footnote}{1}
%   \item [\texttt{fillcycley}\footnotemark\ (integer)\hfill :] Same thing for
%   columns (\emph{Default:~0}).
%   \setcounter{footnote}{1}
%   \item [\texttt{fillcycle}\footnotemark\ (integer)\hfill :] Allow to fix
%   both \texttt{fillcyclex} and \texttt{fillcycley} directly to the same value
%   (\emph{Default:~0}).
% \end{Description}
% 
%   For instance, if \texttt{fillcyclex} is 2, the second row of patterns will
% be horizontally shifted by a factor of $\frac{1}{2}=0.5$, and by a factor of
% 0.333 if \texttt{fillcyclex} is 3, etc.). These values can be negative.
% 
% \begin{LTXexample}[width=0.35\linewidth]
% \psset{unit=0.5}
% \psboxfill{\Square}
% \newcommand{\TilingA}[1]{\Tiling[fillcyclex=#1]{(3,3)}}
% \TilingA{0} \TilingA{1}\\
% \TilingA{2} \TilingA{3}\\[3mm]
% \TilingA{4} \TilingA{5}\\
% \TilingA{6} \TilingA{-3}\\[3mm]
% \Tiling[fillcycley=2]{(3,3)}
% \Tiling[fillcycley=3]{(3,3)}\\
% \Tiling[fillcycley=-3]{(3,3)}
% \Tiling[fillcycle=2]{(3,3)}
% \end{LTXexample}
% 
% \begin{Description}{2cm}
%   \setcounter{footnote}{1}
%   \item [\texttt{fillmovex}\footnotemark\ (real$\|$dim)\hfill :] value of the
%   horizontal moves between consecutive patterns (\emph{Default:~0}).
%   \setcounter{footnote}{1}
%   \item [\texttt{fillmovey}\footnotemark\ (real$\|$dim)\hfill :] value of the
%   vertical moves between consecutive patterns (\emph{Default:~0}).
%   \setcounter{footnote}{1}
%   \item [\texttt{fillmove}\footnotemark\ (real$\|$dim)\hfill :] value of
%   horizontal and vertical moves between consecutive patterns
%   (\emph{Default:~0}).
% \end{Description}
% 
%   These parameters allow the patterns to overlap and to draw some special
% kinds of tilings. They are implemented only for the \emph{automatic} and
% \emph{tiling} modes and their values can be negative.
% 
%   In some cases, the effect of these parameters will be the same that with the 
% \texttt{fillcycle?} ones, but you can see that it is not true for some other
% values.
% 
% \begin{LTXexample}
% \psset{unit=0.5}
% \psboxfill{\Square}
% \Tiling[fillmovex=0.5]{(3,3)} 
% \Tiling[fillmovey=0.5]{(3,3)}\\
% \Tiling[fillmove=0.5]{(3,3)}
% \Tiling[fillmove=-0.5]{(3,3)}
% \end{LTXexample}
% 
% \begin{Description}{2cm}
%   \item [\texttt{fillsize}
%   (auto$\|$\{(real$\|$dim,real$\|$dim)(real$\|$dim,real$\|$dim)\}) :] The
%   choice of \emph{automatic} mode or the size of the area in \emph{manual}
%   mode. If first pair values are not given, (0,0) is used. (\emph{Default:
%   auto when \emph{tiling} mode is used, {(-15cm,-15cm)(15cm,15cm)}
%   otherwise}).
% \end{Description}
% 
%   As explained in the introduction, the \emph{manual} mode can require very
% huge amount of computer ressources. So, it usage is to discourage in front off
% the \emph{automatic} mode. It seems only useful in special circonstances, in
% fact when the \emph{automatic} mode failed, which is known only in one case,
% for some kinds of EPS files, as the ones produce by dump of portions of
% screens (see \ref{sec:GraphicFiles}).
% 
% \begin{Description}{2cm}
%   \setcounter{footnote}{1}
%   \item [\texttt{fillloopaddx}\footnotemark\ (integer)\hfill :] number of
%   times the pattern is added on left and right positions (\emph{Default:~0}).
%   \setcounter{footnote}{1}
%   \item [\texttt{fillloopaddy}\footnotemark\ (integer)\hfill :] number of
%   times the pattern is added on top and bottom positions (\emph{Default:~0}).
%   \setcounter{footnote}{1}
%   \item [\texttt{fillloopadd}\footnotemark\ (integer)\hfill :] number of
%   times the pattern is added on left, right, top and bottom positions
%   (\emph{Default:~0}).
% \end{Description}
% 
%   These parameters are only useful in special circonstances, as for complex
% patterns when the size of the rectangular box used to tile the area doesn't 
% correspond to the pattern itself (see an example in Figure~\ref{fig:Sheeps})
% and also sometimes when the size of the pattern is not a divisor of the size
% of the area to fill and that the number of loop repeats is not properly
% computed, which can occur.
% 
%   They are implemented only for the \emph{tiling} mode.
% 
% \begin{Description}{2cm}
%   \setcounter{footnote}{1}
%   \item [\texttt{PstDebug}\footnotemark\ (integer, 0 or 1)\hfill :] to
%   require to see the exact tiling done, without clipping (\emph{Default:~0}).
% \end{Description}
% 
%   It's mainly useful for debugging or to understand better how the tilings
% are done. It is implemented only for the \emph{tiling} mode.
% 
% \begin{LTXexample}
% \psset{unit=0.3,PstDebug=1}
% \psboxfill{\Square}
% \psset{linewidth=1mm}
% \Tiling{(2,2)}\\[5mm]
% \Tiling[fillcyclex=2]{(2,2)}\\[1cm]
% \Tiling[fillmove=0.5]{(2,2)}
% \end{LTXexample}
% 
% \vspace{3cm}
% \section{Examples}
% 
%   In fact this unique \cs{psboxfill} macro allow a lot a variations and
% different usages. We will try here to demonstrate this.
% 
% \subsection{Kind of tiles}
% \label{sec:KindTiles}
% 
%   Of course, we can access to all the power of PSTricks macros to define the
% \emph{tiles} (\emph{patterns}) used. So, we can define complicated ones.
% 
%   Here we give four other Archimedian tilings (those built with only some
% regular polygons) among the twelve existing, first discovered completely by
% Johanes \textsc{Kepler} at the beginning of 17th century \cite{GS87}, the two
% other \emph{regular} ones with the tiling by squares, formed by a unique
% regular polygon, and two other formed by two different regular polygons.
% 
% \begin{LTXexample}[pos=t]
%   \newcommand{\Triangle}{%
%     \begin{pspicture}(1,1)
%       \pstriangle[dimen=middle](0.5,0)(1,1)
%     \end{pspicture}}
%   \newcommand{\Hexagon}{
% ^^A sin(60)=0.866
%     \begin{pspicture}(0.866,0.75)
%       \SpecialCoor
% ^^A  Hexagon  
%       \pspolygon[dimen=middle]%
%         (0.5;30)(0.5;90)(0.5;150)(0.5;210)(0.5;270)(0.5;330)
%     \end{pspicture}}
% 
%   \psset{unit=0.5}
%   \psboxfill{\Triangle}
%   \Tiling{(4,4)}\hfill
% ^^A The two other regular tilings
%   \Tiling[fillcyclex=2]{(4,4)}\hfill
%   \psboxfill{\Hexagon}
%   \Tiling[fillcyclex=2,fillloopaddy=1]{(5,5)}
% \end{LTXexample}
% 
% \begin{LTXexample}[pos=t]
%   \newcommand{\ArchimedianA}{%
%      ^^A Archimedian tiling 3^2.4.3.4
%     \psset{dimen=middle}
%      ^^A sin(60)=0.866
%     \begin{pspicture}(1.866,1.866)
%       \psframe(1,1)
%       \psline(1,0)(1.866,0.5)(1,1)(0.5,1.866)(0,1)(-0.866,0.5)
%       \psline(0,0)(0.5,-0.866)
%     \end{pspicture}}
%   \newcommand{\ArchimedianB}{%
%      ^^A Archimedian tiling 4.8^2
%     \psset{dimen=middle,unit=1.5}
%      ^^A sin(22.5)=0.3827 ; cos(22.5)=0.9239
%     \begin{pspicture}(1.3066,0.6533)
%       \SpecialCoor
%      ^^A Octogon
%       \pspolygon(0.5;22.5)(0.5;67.5)(0.5;112.5)(0.5;157.5)
%                 (0.5;202.5)(0.5;247.5)(0.5;292.5)(0.5;337.5)
%     \end{pspicture}}
% 
%   \psset{unit=0.5}
%   \psboxfill{\ArchimedianA}
%   \Tiling[fillmove=0.5]{(7,7)}\hfill
%   \psboxfill{\ArchimedianB}
%   \Tiling[fillcyclex=2,fillloopaddy=1]{(7,7)}
% \end{LTXexample}
% 
%   \setcounter{footnote}{3}
%   We can of course tile an area arbitrarily defined. And with the
% \texttt{addfillstyle} parameter\footnote{Introduced in PSTricks 97.}, we can
% easily mix the \texttt{boxfill} style with another one.
% 
% \begin{LTXexample}[width=6cm]
%   \psset{unit=0.5,dimen=middle}
%   \psboxfill{%
%     \begin{pspicture}(1,1)
%       \psframe(1,1)
%       \pscircle(0.5,0.5){0.25}
%     \end{pspicture}}
%   \begin{pspicture}(4,6)
%     \pspolygon[fillstyle=boxfill,fillsep=0.25](0,1)(1,4)(4,6)(4,0)(2,1)
%   \end{pspicture}\hspace{1em}
%   \begin{pspicture}(4,4)
%%     \pscircle[linestyle=none,fillstyle=solid,fillcolor=yellow,fillsep=0.5,
%%               addfillstyle=boxfill](2,2){2}
%   \end{pspicture}
% \end{LTXexample}
%
%   Various effects can be obtained, sometimes complicated ones very easily, as
% in this example reproduced from one shown by Slavik \textsc{Jablan} in the
% field of \emph{OpTiles}, inspired by the \emph{Op-art}:
% 
% \begin{LTXexample}[pos=t]
% \newcommand{\ProtoTile}{%
%  \begin{pspicture}(1,1)%%% 1/12=0.08333
%   \psset{linestyle=none,linewidth=0,
%     hatchwidth=0.08333\psunit,hatchsep=0.08333\psunit}
%   \psframe[fillstyle=solid,fillcolor=black,addfillstyle=hlines,hatchcolor=white](1,1)
%   \pswedge[fillstyle=solid,fillcolor=white,addfillstyle=hlines]{1}{0}{90}
%  \end{pspicture}}
% \newcommand{\BasicTile}{%
%  \begin{pspicture}(2,1)
%    \rput[lb](0,0){\ProtoTile}\rput[lb](1,0){\psrotateleft{\ProtoTile}}
%  \end{pspicture}}
% \ProtoTile\hfill\BasicTile\hfill
% \psboxfill{\BasicTile}
% \Tiling[fillcyclex=2]{(4,4)}
% \end{LTXexample}
% 
%   It is also directly possible to surimpose several different tilings. Here is
% the splendid visual proof of the \textsc{Pytha\-gore} theorem done by the arab
% mathematician \textsc{Annairizi} around the year 900, given by superposition
% of two tilings by squares of different sizes.
% 
% \begin{LTXexample}[pos=t]
% \psset{unit=1.5,dimen=middle}
% \begin{pspicture*}(3,3)
%   \psboxfill{\begin{pspicture}(1,1)
%     \psframe(1,1)\end{pspicture}}
%   \psframe[fillstyle=boxfill](3,3)
%   \psboxfill{\begin{pspicture}(1,1)
%     \rput{-37}{\psframe[linecolor=red](0.8,0.8)}
%   \end{pspicture}}
%   \psframe[fillstyle=boxfill](3,4)
%   \pspolygon[fillstyle=hlines,hatchangle=90](1,2)(1.64,1.53)(2,2)
% \end{pspicture*}
% \end{LTXexample}
% 
%   In a same way, it is possible to build tilings based on figurative patterns,
% in the style of the famous \textsc{Escher} ones. Following an example of
% Andr\'e \textsc{Deledicq} \cite{Deledicq97}, we first show a simple tiling of
% the \emph{p1} category (according to the international classification of the
% 17~symmetry groups of the plane first discovered by the russian
% crystalographer Jevgraf \textsc{Fedorov} at the end of the 19th century):
% 
% \begin{LTXexample}[pos=t]
%  \newcommand{\SheepHead}[1]{%
%    \begin{pspicture}(3,1.5)
%      \pscustom[liftpen=2,fillstyle=solid,fillcolor=#1]{%
%        \pscurve(0.5,-0.2)(0.6,0.5)(0.2,1.3)(0,1.5)(0,1.5)
%          (0.4,1.3)(0.8,1.5)(2.2,1.9)(3,1.5)(3,1.5)(3.2,1.3)
%          (3.6,0.5)(3.4,-0.3)(3,0)(2.2,0.4)(0.5,-0.2)}
%      \pscircle*(2.65,1.25){0.12\psunit} % Eye
%      \psccurve*(3.5,0.3)(3.35,0.45)(3.5,0.6)(3.6,0.4)% Muzzle
%     ^^A   % Mouth
%       \pscurve(3,0.35)(3.3,0.1)(3.6,0.05)
%     ^^A   % Ear
%       \pscurve(2.3,1.3)(2.1,1.5)(2.15,1.7)\pscurve(2.1,1.7)(2.35,1.6)(2.45,1.4)
%   \end{pspicture}}
%  \psboxfill{\psset{unit=0.5}\SheepHead{yellow}\SheepHead{cyan}}
%  \Tiling[fillcyclex=2,fillloopadd=1]{(10,5)}
% \end{LTXexample}
% \label{fig:Sheeps}
% 
%   Now a tiling of the \emph{pg} category (the code for the kangaroo itself is
% too long to be shown here, but has no difficulties ; the kangaroo is reproduce
% from an original picture from Raoul \textsc{Raba} and here is a translation in
% PSTricks from the one drawn by Emmanuel \textsc{Chailloux} and Guy
% \textsc{Cousineau} for their MLgraph system \cite{MLgraphTSI}):
% 
% \begin{LTXexample}[pos=t]
% \psboxfill{\psset{unit=0.4}
%   \Kangaroo{yellow}\Kangaroo{red}\Kangaroo{cyan}\Kangaroo{green}%
%   \psscalebox{-1 1}{%
%     \rput(1.235,4.8){\Kangaroo{green}\Kangaroo{cyan}\Kangaroo{red}\Kangaroo{yellow}}}}
%   \Tiling[fillloopadd=1]{(10,6)}
% \end{LTXexample}
% 
%   And here a \textsc{Wang} tiling \cite{Wang65}, \cite[chapter
% 11]{GS87}, based on very simple tiles of the form of a square and composed
% of four colored triangles. Such tilings are built with only a matching color
% constraint. Despite of it simplicity, it is an important kind of tilings, as
% \textsc{Wang} and others used them to study the special class of
% \emph{aperiodic} tilings, and also because it was shown that surprisingly this 
% tiling is similar to a \textsc{Turing} machine.
% 
% \begin{LTXexample}[pos=t]
%   \newcommand{\WangTile}[4]{%
%     \begin{pspicture}(1,1)
%       \pspolygon*[linecolor=#1](0,0)(0,1)(0.5,0.5)
%       \pspolygon*[linecolor=#2](0,1)(1,1)(0.5,0.5)
%       \pspolygon*[linecolor=#3](1,1)(1,0)(0.5,0.5)
%       \pspolygon*[linecolor=#4](1,0)(0,0)(0.5,0.5)
%     \end{pspicture}}
%   \newcommand{\WangTileA}{\WangTile{cyan}{yellow}{cyan}{cyan}}
%   \newcommand{\WangTileB}{\WangTile{yellow}{cyan}{cyan}{red}}
%   \newcommand{\WangTileC}{\WangTile{cyan}{red}{yellow}{yellow}}
%   \newcommand{\WangTiles}[1][]{%
%     \begin{pspicture}(3,3) \psset{ref=lb}
%       \rput(0,2){\WangTileB}  \rput(1,2){\WangTileA}%
%       \rput(2,2){\WangTileC}  \rput(0,1){\WangTileC}%
%       \rput(1,1){\WangTileB}  \rput(2,1){\WangTileA}
%       \rput(0,0){\WangTileA}  \rput(1,0){\WangTileC}%
%       \rput(2,0){\WangTileB}
%       #1
%     \end{pspicture}}
%   \WangTileA\hfill\WangTileB\hfill\WangTileC\hfill
%   \WangTiles[{\psgrid[subgriddiv=0,gridlabels=0](3,3)}]\hfill
%   \psset{unit=0.4} \psboxfill{\WangTiles} \Tiling{(12,12)}
% \end{LTXexample}
% 
% \subsection{External graphic files}
% \label{sec:GraphicFiles}
% 
%   We can also fill an arbitrary area with an external image. We have only, 
% as usual, to matter of the \emph{BoundingBox} definition if there is no one
% provided or if it is not the accurate one, as for the well known
% \texttt{tiger} picture part of the \texttt{ghostscript} distribution.
% 
% \begin{LTXexample}[pos=t]
%   \psboxfill{%% Strangely require x1=x2...
%     \begin{pspicture}(0,1)(0,4.1)
%       \includegraphics[bb=17 176 560 74,width=3cm]{tiger}
%     \end{pspicture}}
%   \Tiling{(6,6.2)}
% \end{LTXexample}
% 
%   Nevertheless, there are some special files for which the \emph{automatic}
% mode doesn't work, specially for some files obtained by a screen dump, as in
% the next example, where a picture was reduced before it conversion in the
% \emph{Encapsulated PostScript} format by a screen dump utility. In this case,
% usage of the \emph{manual} mode is the only alternative, at the price of the
% real multiple inclusion of the EPS file. We must take care to specify the
% correct \texttt{fillsize} parameter, because otherwise the default values are
% large and will load the file many times, perhaps just really using few
% occurrences as the other ones would be clipped...
% 
% \begin{LTXexample}[pos=t]
%   \psboxfill{\includegraphics{flowers}}
%   \begin{pspicture}(8,4)
%     \psellipse[fillstyle=boxfill,fillsize={(8,4)}](4,2)(4,2)
%   \end{pspicture}
% \end{LTXexample}
% 
% \subsection{Tiling of characters}
% 
%   We can also use the \cs{psboxfill} macro to fill the interior of characters
% for special effects like these ones:
% 
% \begin{LTXexample}[pos=t]
%   \DeclareFixedFont{\bigsf}{T1}{phv}{b}{n}{4.5cm}
%   \DeclareFixedFont{\smallrm}{T1}{ptm}{m}{n}{3mm}
%   \psboxfill{\smallrm Since 182 days...}
%   \begin{pspicture*}(8,4)
%     \centerline{%
%       \pscharpath[fillstyle=gradient,gradangle=-45,
%                   gradmidpoint=0.5,addfillstyle=boxfill,
%                   fillangle=45,fillsep=0.7mm]
%                  {\rput[b](0,0.1){\bigsf 2000}}}
%   \end{pspicture*}
% \end{LTXexample}
% 
% \begin{LTXexample}[pos=t]
%   \DeclareFixedFont{\mediumrm}{T1}{ptm}{m}{n}{2cm}
%   \psboxfill{%
%     \psset{unit=0.1,linewidth=0.2pt}
%     \Kangaroo{PeachPuff}\Kangaroo{PaleGreen}%
%       \Kangaroo{LightBlue}\Kangaroo{LemonChiffon}%
%     \psscalebox{-1 1}{%
%       \rput(1.235,4.8){%
%         \Kangaroo{LemonChiffon}\Kangaroo{LightBlue}%
%           \Kangaroo{PaleGreen}\Kangaroo{PeachPuff}}}}
% ^^A   % A kangaroo of kangaroos...
%   \begin{pspicture}(8,2)
%     \pscharpath[linestyle=none,fillstyle=boxfill,fillloopadd=1]
%                {\rput[b](4,0){\mediumrm Kangaroo}}
%   \end{pspicture}
% \end{LTXexample}
% 
% \subsection{Other kinds of usage}
% 
%   Other kinds of usage can be imagined. For instance, we can use tilings in a
% sort of degenerated way to draw some special lines made by a unique or
% multiple repeating patterns. But it can be only a special dashed line, as here
% with three different dashes:
% 
% \begin{LTXexample}[pos=t]
%   \newcommand{\Dashes}{%
%     \psset{dimen=middle}
%     \begin{pspicture}(0,-0.5\pslinewidth)(1,0.5\pslinewidth)
%       \rput(0,0){\psline(0.4,0)}%
%         \rput(0.5,0){\psline(0.2,0)}%
%         \rput(0.8,0){\psline(0.1,0)}
%     \end{pspicture}}
% 
%   \newcommand{\SpecialDashedLine}[3]{%
%     \psboxfill{#3}
%     \Tiling[linestyle=none]
%            {(#1,-0.5\pslinewidth)(#2,0.5\pslinewidth)}}
% 
%   \SpecialDashedLine{0}{7}{\Dashes}
% 
%   \psset{unit=0.5,linewidth=1mm,linecolor=red}
%   \SpecialDashedLine{0}{10}{\Dashes}
% \end{LTXexample}
% 
%   It allow also to use special patterns in business graphics, as in the
% following example generated by \texttt{PstChart}\footnote{A personal
% development to draw business charts with PSTricks, not distributed.}.
% 
% \vspace{3mm}
% \begin{figure}[!ht]
% \centering
% \psset{unit=0.75}
% ^^A % Generated by pstchart.sh version 0.21 (11/28/97)
% {\psset{dimen=middle}
% \psset{xunit=2,yunit=0.005}
% \begin{pspicture}(-0.6,-200)(6.6,2300)
% ^^A   % Title
%   \rput(3,2200){\shortstack{Fantaisist repartition of kangaroos\\
%                             in the world (in thousands)}}
% ^^A   % Frame background
%   \psframe[fillstyle=solid,fillcolor=LemonChiffon](0,0)(6,2000)
% ^^A   % Graduations
%   \multido{\n=0+500}{5}{\rput[r](-0.12,\n){\psscalebox{0.8}{\n}}}
% ^^A   % Minor ticks
%   \multips(0,100)(0,100){19}{\psline[unit=4.8pt](1,0)}
%   \multips(6,100)(0,100){19}{\psline[unit=4.8pt](-1,0)}
% ^^A   % Major ticks
%   \multips(0,500)(0,500){3}{\psline[unit=9.6pt](1,0)}
%   \multips(6,500)(0,500){3}{\psline[unit=9.6pt](-1,0)}
% ^^A   % Lines from major ticks marks
%   \multips(0,500)(0,500){3}{\psline[linestyle=dotted,linewidth=0.6pt](6,0)}
% ^^A   % Drawing for the data
%   \psboxfill{\psset{unit=0.78\psxunit}\KangarooPstChart{red}}
%   \psframe[linestyle=none,fillstyle=boxfill,fillloopaddy=1](0.61,0)(1.39,1800)
%   \psboxfill{\psset{unit=0.78\psxunit}\KangarooPstChart{yellow}}
%   \psframe[linestyle=none,fillstyle=boxfill,fillloopaddy=1](1.61,0)(2.39,800)
%   \psboxfill{\psset{unit=0.78\psxunit}\KangarooPstChart{cyan}}
%   \psframe[linestyle=none,fillstyle=boxfill,fillloopaddy=1](2.61,0)(3.39,550)
%   \psboxfill{\psset{unit=0.78\psxunit}\KangarooPstChart{magenta}}
%   \psframe[linestyle=none,fillstyle=boxfill,fillloopaddy=1](3.61,0)(4.39,500)
%   \psboxfill{\psset{unit=0.78\psxunit}\KangarooPstChart{green}}
%   \psframe[linestyle=none,fillstyle=boxfill,fillloopaddy=1](4.61,0)(5.39,200)
% ^^A   % Bottom labels
%   \uput{0.2}[270]{0}(1,0){\psscalebox{0.7}{Oceania}}
%   \uput{0.2}[270]{0}(2,0){\psscalebox{0.7}{Africa}}
%   \uput{0.2}[270]{0}(3,0){\psscalebox{0.7}{Asia}}
%   \uput{0.2}[270]{0}(4,0){\psscalebox{0.7}{America}}
%   \uput{0.2}[270]{0}(5,0){\psscalebox{0.7}{Europe}}
% ^^A   % Frame box around the chart
%   \psframe[linestyle=solid](0,0)(6,2000)
% \end{pspicture}}
%   \caption{Bar chart generated by PstChart, with bars filled by patterns}
%   \label{fig:PstChart}
% \end{figure}
% 
% \section{``Dynamic'' tilings}
% 
%   In some cases, tilings used non \emph{static} tiles, that is to say that the 
% \emph{prototile(s)}, even if unique, can have several forms, by instance
% specified by different colors or rotations, not fixed before generation or
% varying each time.
% 
% \subsection{Lewthwaite-Pickover-Truchet tiling}
% 
%   We give here for example the so-called \emph{Truchet} tiling, which much be
% in fact better called \emph{Lewthwaite-Pick\-over-Truchet (LPT)} tiling%
% \footnote{For description of the context, history and references about
% S\'ebastien \textsc{Truchet} and this tiling, see \cite{EsperetGirou98}.}.
% 
%   The unique prototile is only a square with two opposite circle arcs.
% This tile has obviously two positions, if we rotate it from 90 degrees (see
% the two tiles on the next figure). A \emph{LPT tiling} is a tiling with
% randomly oriented LPT tiles. We can see that even if it is very simple in it
% principle, it draw sophisticated curves with strange properties.
% 
%   Nevertheless, in the straightforward way \FillPackage{} does not work,
% because the \cs{psboxfill} macro store the content of the tile used in a
% \TeX{} box, which is static. So the calling to the random function is done
% only one time, which explain that only one rotation of the tile is used for
% all the tiling. It's only the one of the two rotations which could differ from
% one drawing to the next one...
% 
% ^^A % Truchet (Lewthwaite-Pickover-Truchet) tiling
% ^^A % --------------------------------------------
% 
% \begin{LTXexample}[pos=t]
% ^^A   % LPT prototile
%   \newcommand{\ProtoTileLPT}{%
%     \psset{dimen=middle}
%     \begin{pspicture}(1,1)
%       \psframe(1,1)
%       \psarc(0,0){0.5}{0}{90}
%       \psarc(1,1){0.5}{-180}{-90}
%     \end{pspicture}}
% 
% ^^A   % LPT tile
%   \newcount\Boolean
%   \newcommand{\BasicTileLPT}{%
% ^^A     % From random.tex by Donald Arseneau
%     \setrannum{\Boolean}{0}{1}%
%     \ifnum\Boolean=0
%       \ProtoTileLPT%
%     \else
%       \psrotateleft{\ProtoTileLPT}%
%     \fi}
% 
%   \ProtoTileLPT\hfill\psrotateleft{\ProtoTileLPT}\hfill
%   \psset{unit=0.5}
%   \psboxfill{\BasicTileLPT}
%   \Tiling{(5,5)}
% \end{LTXexample}
% 
%   But, for simple cases, there is a solution to this problem using a mixture
% of PSTricks and PostScript programming. Here the PSTricks
% construction \verb+\pscustom{\code{...}}+ allow to insert PostScript code
% inside the \LaTeX{} + PSTricks one.
% 
%   Programmation is less straightforward, but it has also the advantage to be
% notably faster, as all the tilings operations are done in PostScript, and
% mainly to not be limited by \TeX{} memory (the \TeX{} + PSTricks solution
% I wrote in 1995 for the colored problem was limited to small sizes for this
% reason). Just note also that \cs{pslbrace} and \cs{psrbrace} are two
% PSTricks macros to define and be able to insert the \verb+{+ and \verb+}+
% characters.
% 
% \begin{LTXexample}[pos=t]
% ^^A   % LPT prototile
%   \newcommand{\ProtoTileLPT}{%
%     \psset{dimen=middle}
%     \psframe(1,1)
%     \psarc(0,0){0.5}{0}{90}
%     \psarc(1,1){0.5}{-180}{-90}}
% 
% ^^A   % Counter to change the random seed
%   \newcount\InitCounter
% ^^A   % LPT tile
%   \newcommand{\BasicTileLPT}{%
%     \InitCounter=\the\time
%     \pscustom{\code{%
%       rand \the\InitCounter\space sub 2 mod 0 eq \pslbrace}}
%     \begin{pspicture}(1,1)
%       \ProtoTileLPT
%     \end{pspicture}%
%     \pscustom{\code{\psrbrace \pslbrace}}
%     \psrotateleft{\ProtoTileLPT}%
%     \pscustom{\code{\psrbrace ifelse}}}
% 
%   \psset{unit=0.4,linewidth=0.4pt}
%   \psboxfill{\BasicTileLPT}
%   \Tiling{(15,15)}
% \end{LTXexample}
% 
%   Using the very surprising fact (see \cite{EsperetGirou98}) that
% coloration of these tiles do not depend of their neighbors (even if it is
% difficult to believe as the opposite seems obvious!) but only of the parity of
% the value of row and column positions, we can directly program in the same way
% a colored version of the LPT tiling.
% 
% \setcounter{footnote}{1}
%   We have also introduce in the \FillPackage{} code for \emph{tiling} mode two
% new accessible Post\-Script variables, \texttt{row} and
% \texttt{column}\footnotemark, which can be useful in some circonstances, like
% this one.
% 
% \begin{LTXexample}[pos=t]
% ^^A   % LPT prototile
%   \newcommand{\ProtoTileLPT}[2]{%
%     \psset{dimen=middle,linestyle=none,fillstyle=solid}
%     \psframe[fillcolor=#1](1,1)
%     \psset{fillcolor=#2}
%     \pswedge(0,0){0.5}{0}{90} \pswedge(1,1){0.5}{-180}{-90}}
% ^^A   % Counter to change the random seed
%   \newcount\InitCounter
% ^^A   % LPT tile
%   \newcommand{\BasicTileLPT}[2]{%
%     \InitCounter=\the\time
%     \pscustom{\code{%
%       rand \the\InitCounter\space sub 2 mod 0 eq \pslbrace
%       row column add 2 mod 0 eq \pslbrace}}
%     \begin{pspicture}(1,1)\ProtoTileLPT{#1}{#2}\end{pspicture}%
%     \pscustom{\code{\psrbrace \pslbrace}}
%     \ProtoTileLPT{#2}{#1}%
%     \pscustom{\code{%
%       \psrbrace ifelse \psrbrace \pslbrace row column add 2 mod 0 eq \pslbrace}}
%     \psrotateleft{\ProtoTileLPT{#2}{#1}}\pscustom{\code{\psrbrace \pslbrace}}
%     \psrotateleft{\ProtoTileLPT{#1}{#2}}\pscustom{\code{\psrbrace ifelse \psrbrace ifelse}}}
%   \psboxfill{\BasicTileLPT{red}{yellow}}
%   \Tiling{(4,4)}\hfill
%   \psset{unit=0.4}\psboxfill{\BasicTileLPT{blue}{cyan}}
%   \Tiling{(15,15)}
% \end{LTXexample}
% 
%   Another classic example is to generate coordinates and numerotation for a
% grid. Of course, it is possible to do it directly in PSTricks using nested
% \cs{multido} commands. It would be clearly easy to program, but, nevertheless, 
% for users who have a little knowledge of PostScript programming, this offer
% an alternative which is useful for large cases, because on this way it will
% be notably faster and less computer ressources consuming.
% 
%   Remember here that the tiling is drawn from left to right, and top to
% bottom, and note that the PostScript variable \texttt{x2} give the total
% number of columns.
% 
% \begin{LTXexample}[pos=t]
% ^^A   % \Escape will be the \ character
%   {\catcode`\!=0\catcode`\\=11!gdef!Escape{\}}
%   \newcommand{\ProtoTile}{%
%     \Square\pscustom{%
%       \moveto(-0.9,0.75) % In PSTricks units
%       \code{ /Times-Italic findfont 8 scalefont setfont
%         (\Escape() show row 3 string cvs show (,) show 
%         column 3 string cvs show (\Escape)) show}
%       \moveto(-0.5,0.25) % In PSTricks units
%       \code{ /Times-Bold findfont 18 scalefont setfont
%         1 0 0 setrgbcolor % Red color
%         /center {dup stringwidth pop 2 div neg 0 rmoveto} def
%         row 1 sub x2 mul column add 3 string cvs center show}}}
%   \psboxfill{\ProtoTile}
%   \Tiling{(6,4)}
% \end{LTXexample}
% 
% \subsection{A complete example: the Poisson equation}
% 
%   To finish, we will show a complete real example, a drawing to explain the
% method used to solve the \textsc{Poisson} equation by a domain
% decomposition method, adapted to distributed memory computers. The
% objective is to show the communications required between processes and the
% position of the data to exchange. This code also show some useful and powerful
% technics for PSTricks programming (look specially at the way some higher level
% macros are defined, and how the same object is used to draw the four
% neighbors).
%
%\psset{unit=1cm}
%\newcommand{\Pattern}[1]{%
%   \begin{pspicture}(-0.25,-0.25)(0.25,0.25)\rput{*0}{\psdot[dotstyle=#1]}
%   \end{pspicture}}
%\newcommand{\West}{\Pattern{o}}   \newcommand{\South}{\Pattern{x}}
%\newcommand{\Central}{\Pattern{+}}\newcommand{\North}{\Pattern{square}}
%\newcommand{\East}{\Pattern{triangle}}
%\newcommand{\Cross}{%
%  \pspolygon[unit=0.5,linewidth=0.2,linecolor=red](0,0)(0,1)(1,1)(1,2)(2,2)(2,1)%
%              (3,1)(3,0)(2,0)(2,-1)(1,-1)(1,0)}
%\newcommand{\StylePosition}[1]{\LARGE\textcolor{red}{\textbf{#1}}}
%\newcommand{\SubDomain}[4]{%
%    \psboxfill{#4}\begin{psclip}{\psframe[linestyle=none]#1}%
%      \psframe[linestyle=#3](5,5)\psframe[fillstyle=boxfill]#2%
%    \end{psclip}}
%\newcommand{\SendArea}[1]{\psframe[fillstyle=solid,fillcolor=cyan]#1}
%\newcommand{\ReceiveData}[2]{%
%  \psboxfill{#2}\psframe[fillstyle=solid,fillcolor=yellow,addfillstyle=boxfill]#1}%
%\newcommand{\Neighbor}[2]{%
%    \begin{pspicture}(5,5)
%      \rput{*0}(2.5,2.5){\StylePosition{#1}}
%      \ReceiveData{(0.5,0)(4.5,0.5)}{\Central}\SendArea{(0.5,0.5)(4.5,1)}%
%      \SubDomain{(5,2)}{(0.5,0.5)(4.5,3)}{dashed}{#2}%
%      \pcarc[arcangle=45,arrows=->](0.5,-1.25)(0.5,0.25)%
%      \pcarc[arcangle=45,arrows=->,linestyle=dotted,dotsep=2pt](4.5,0.75)(4.5,-0.75)%
%    \end{pspicture}}%
%  \psset{dimen=middle,dotscale=2,fillloopadd=2}
%\begin{pspicture}(-5.7,-5.7)(5.7,5.7)
%  \rput(0,0){%
%      \begin{pspicture}(5,5)
%        \ReceiveData{(0,0.5)(0.5,4.5)}{\West} \ReceiveData{(4.5,0.5)(5,4.5)}{\East}
%        \ReceiveData{(0.5,4.5)(4.5,5)}{\North}\ReceiveData{(0.5,0)(4.5,0.5)}{\South}
%        \SendArea{(0.5,0.5)(1,4.5)}\SendArea{(4,0.5)(4.5,4.5)}
%        \SendArea{(0.5,0.5)(4.5,1)}\SendArea{(0.5,4)(4.5,4.5)}
%        \SubDomain{(5,5)}{(0.5,0.5)(4.5,4.5)}{solid}{\Central}
%        \psline(1,0.5)(1,4.5)\psline(4,0.5)(4,4.5)%
%        \rput(1.5,4){\Cross}\rput(2,2){\Cross}%
%      \end{pspicture}}%
%  \rput(0,5.5){\Neighbor{N}{\North}}\rput{-90}(5.5,0){\Neighbor{E}{\East}}%
%  \rput{90}(-5.5,0){\Neighbor{W}{\West}}\rput{180}(0,-5.5){\Neighbor{S}{\South}}%
%\end{pspicture}
%
% \begin{lstlisting}
%   \newcommand{\Pattern}[1]{%
%     \begin{pspicture}(-0.25,-0.25)(0.25,0.25)\rput{*0}{\psdot[dotstyle=#1]}
%     \end{pspicture}}
%   \newcommand{\West}{\Pattern{o}}   \newcommand{\South}{\Pattern{x}}
%   \newcommand{\Central}{\Pattern{+}}\newcommand{\North}{\Pattern{square}}
%   \newcommand{\East}{\Pattern{triangle}}
%   \newcommand{\Cross}{%
%     \pspolygon[unit=0.5,linewidth=0.2,linecolor=red](0,0)(0,1)(1,1)(1,2)(2,2)(2,1)
%               (3,1)(3,0)(2,0)(2,-1)(1,-1)(1,0)}
%   \newcommand{\StylePosition}[1]{\LARGE\textcolor{red}{\textbf{#1}}}
%   \newcommand{\SubDomain}[4]{%
%     \psboxfill{#4}
%     \begin{psclip}{\psframe[linestyle=none]#1}
%       \psframe[linestyle=#3](5,5)\psframe[fillstyle=boxfill]#2
%     \end{psclip}}
%   \newcommand{\SendArea}[1]{\psframe[fillstyle=solid,fillcolor=cyan]#1}
%   \newcommand{\ReceiveData}[2]{%
%     \psboxfill{#2}
%     \psframe[fillstyle=solid,fillcolor=yellow,addfillstyle=boxfill]#1}
%   \newcommand{\Neighbor}[2]{%
%     \begin{pspicture}(5,5)
%       \rput{*0}(2.5,2.5){\StylePosition{#1}}
%       \ReceiveData{(0.5,0)(4.5,0.5)}{\Central}\SendArea{(0.5,0.5)(4.5,1)}
%       \SubDomain{(5,2)}{(0.5,0.5)(4.5,3)}{dashed}{#2}%
% ^^A       % Receive and send arrows
%       \pcarc[arcangle=45,arrows=->](0.5,-1.25)(0.5,0.25)
%       \pcarc[arcangle=45,arrows=->,linestyle=dotted,dotsep=2pt](4.5,0.75)(4.5,-0.75)
%     \end{pspicture}}
%   \psset{dimen=middle,dotscale=2,fillloopadd=2}
%   \begin{pspicture}(-5.7,-5.7)(5.7,5.7)
% ^^A     % Central domain
%     \rput(0,0){%
%       \begin{pspicture}(5,5)
% ^^A         % Receive from West, East, North and South
%         \ReceiveData{(0,0.5)(0.5,4.5)}{\West} \ReceiveData{(4.5,0.5)(5,4.5)}{\East}
%         \ReceiveData{(0.5,4.5)(4.5,5)}{\North}\ReceiveData{(0.5,0)(4.5,0.5)}{\South}
% ^^A         % send area for West, East, North and South
%         \SendArea{(0.5,0.5)(1,4.5)} \SendArea{(4,0.5)(4.5,4.5)}
%         \SendArea{(0.5,0.5)(4.5,1)} \SendArea{(0.5,4)(4.5,4.5)}
% ^^A         % Central domain
%         \SubDomain{(5,5)}{(0.5,0.5)(4.5,4.5)}{solid}{\Central}
% ^^A         % Redraw overlapped linesY
%         \psline(1,0.5)(1,4.5)  \psline(4,0.5)(4,4.5)
% ^^A         % Two crossesY
%         \rput(1.5,4){\Cross}  \rput(2,2){\Cross}
%       \end{pspicture}}
% ^^A     % The four neighborsY
%     \rput(0,5.5){\Neighbor{N}{\North}}     \rput{-90}(5.5,0){\Neighbor{E}{\East}}
%     \rput{90}(-5.5,0){\Neighbor{W}{\West}} \rput{180}(0,-5.5){\Neighbor{S}{\South}}
%   \end{pspicture}
% \end{lstlisting}
%
%
%
% Bibliography
% \begin{thebibliography}{99}
% \bibitem{PostScript95} Adobe, Systems~Incorporated, \emph{PostScript Language
% Reference Manual}, Addison-Wesley, 2~edition, 1995.
%
% \bibitem{Bolek98} Piotr Bolek, \MP{} and patterns, \emph{\TUB}, Volume~19,
% Number~3, pages 276--283, September 1998, \CTANref{mpattern}.
%
% \bibitem{MLgraphTSI} Emmanuel Chailloux, Guy Cousineau and Asc\'ander
% Su\'arez, Programmation fonctionnelle de graphismes pour la production
% d'illustrations techniques, \emph{Technique et science informatique},
% Volume~15, Number~7, pages 977--1007, 1996 (in french).
%
% \bibitem{Deledicq97} Andr\'e Deledicq, \emph{Le monde des pavages}, ACL
% \'Editions, 1997 (in french).
%
% \bibitem{EsperetGirou98} Philippe Esperet and Denis Girou,
% Coloriage du pavage dit de Truchet, Cahiers GUTenberg, Number~31,
% pages 5--18, December~1998  (in french).
%
% \bibitem{Girou94} Denis Girou, Pr\'esentation de PSTricks, \emph{Cahiers
% GUTenberg}, Number~16, pages 21--70, February~1994 (in french).
%
% \bibitem{LGC97} Michel Goossens, Sebastian Rahtz and Frank Mittelbach,
% \emph{The \LaTeX{} Graphics Companion}, Addison-Wesley, 2005.
%
% \bibitem{GS87} Branko Gr\"unbaum and Geoffrey Shephard, \emph{Tilings and
% Patterns}, Freeman and Company, 1987.
%
% \bibitem{Hoenig97} Alan Hoenig, \emph{\TeX{} Unbound: \LaTeX{} \& \TeX{}
% Strategies, Fonts, Graphics, and More}, Oxford University Press, 1997.
%
% \bibitem{XYpic} Kristoffer~H. Rose and Ross Moore, \XYpic. Pattern and Tile
% extension, available from \CTAN, 1991-1998, \CTANref{xypic}.
%
% \bibitem{LAAN96} Kees van der Laan, Paradigms: Just a little bit of PostScript,
% \emph{MAPS}, Volume~17, pages 137--150, 1996.
%
% \bibitem{LAAN97} Kees van der Laan, Tiling in PostScript and \MF{} -- Escher's
% wink, \emph{MAPS}, Volume~19, Number~2, pages 39--67, 1997.
%
% \bibitem{vanZandt93} Timothy Van Zandt, PSTricks. PostScript macros for
% Generic \TeX, available from \CTAN, 1993, \CTANref{pstricks}.
%
% \bibitem{vanZandtGirou94} Timothy Van Zandt and Denis Girou, Inside PSTricks,
% \emph{\TUB}, Volume~15, Number~3, pages 239--246, September 1994.
%
%
% \bibitem{voss07} Herbert Vo\ss, PSTricks -- Graphics for \TeX\ and \LaTeX, DANTE/Lehmanns, 4th ed., 2007.
% \bibitem{Wang65} Hao Wang, Games, Logic and Computers, \emph{Scientific
% American}, pages 98--106, November 1965.
% \end{thebibliography}
%
%
% \StopEventually{}
%
% ^^A .................... End of the documentation part ....................
%
% \section{Driver file}
%
%   The next bit of code contains the documentation driver file for \TeX{},
% i.e., the file that will produce the documentation you are currently
% reading. It will be extracted from this file by the \texttt{docstrip}
% program.
%
%    \begin{macrocode}
%<*driver>
\documentclass{ltxdoc}
\GetFileInfo{pst-fill.dtx}
%
\usepackage[T1]{fontenc}
\usepackage{lmodern}               % For PDF
\usepackage{graphicx}              % `graphicx' LaTeX standard package
\usepackage{showexpl}
\usepackage{mflogo}                % For the MetaFont and MetaPost logos
\input{random.tex}                 % Random macros from Donald Arseneau
\usepackage{url}                   % URLs convenient typesetting
\usepackage{multido}               % General loop macro
\usepackage[dvipsnames]{pstricks}  % PSTricks with the `color' extension
\usepackage{pst-text}              % PSTricks package for character path
\usepackage{pst-grad}              % PSTricks package for gradient filling
\usepackage{pst-node}              % PSTricks package for nodes
\usepackage[tiling]{pst-fill}      % PSTricks package for filling/tiling
%
\AtBeginDocument{%
%  \OnlyDescription % comment out for implementation details
  \EnableCrossrefs
  \CodelineIndex
  \RecordChanges}
\AtEndDocument{%
  \PrintIndex
  \setcounter{IndexColumns}{1}
  \PrintChanges}
\hbadness=7000            % Over and under full box warnings
\hfuzz=3pt
\begin{document}
  \DocInput{pst-fill.dtx}
\end{document}
%</driver>
%    \end{macrocode}
%
% \section{\texttt{pst-fill} \LaTeX{} wrapper}
%
%    \begin{macrocode}
%<*latex-wrapper>
\RequirePackage{pstricks}
\ProvidesPackage{pst-fill}[2005/09/13 package wrapper for 
  pst-fill.tex (hv)]
\DeclareOption{tiling}{\def\PstTiling{true}}
\ProcessOptions\relax
\input{pst-fill.tex}
\ProvidesFile{pst-fill.tex}
  [\filedate\space v\fileversion\space `PST-fill' (tvz,dg)]
%</latex-wrapper>
%    \end{macrocode}
%
%
% \section{Pst-Fill Package{} code}
%
%    \begin{macrocode}
%<*pst-fill>
%    \end{macrocode}
%
% \subsection{Preamble}
%
%   Who we are.
%
%    \begin{macrocode}
\def\fileversion{1.01}
\def\filedate{2007/03/10}
\message{`PST-Fill' v\fileversion, \filedate\space (tvz,dg,hv)}
\csname PSTboxfillLoaded\endcsname
\let\PSTboxfillLoaded\endinput
%    \end{macrocode}
%
%   Require the main PSTricks package.
%
%    \begin{macrocode}
\ifx\PSTricksLoaded\endinput\else\input pstricks.tex\fi
%    \end{macrocode}
%
%   interface to the extended `\textsf{keyval}' package.
%
%    \begin{macrocode}
\ifx\PSTXKeyLoaded\endinput\else\input pst-xkey\fi
%
%    \end{macrocode}
%
%   Catcodes changes and defining the family name for xkeyval.
%
%    \begin{macrocode}
\edef\PstAtCode{\the\catcode`\@}\catcode`\@=11\relax

\pst@addfams{pst-fill}
%
%    \end{macrocode}
%
%
% \subsection{The size of the box}
% \begin{macro}{pst@@boxfillsize}
%    \begin{macrocode}
%
\def\pst@@boxfillsize#1(#2,#3)#4(#5,#6)#7(#8\@nil{%
  \begingroup
    \ifx\@empty#7\relax
      \pst@dima\z@
      \pst@dimb\z@
      \pssetxlength\pst@dimc{#2}%
      \pssetylength\pst@dimd{#3}%
    \else
      \pssetxlength\pst@dima{#2}%
      \pssetylength\pst@dimb{#3}%
      \pssetxlength\pst@dimc{#5}%
      \pssetylength\pst@dimd{#6}%
    \fi
    \xdef\pst@tempg{%
      \pst@dima=\number\pst@dima sp
      \pst@dimb=\number\pst@dimb sp
      \pst@dimc=\number\pst@dimc sp
      \pst@dimd=\number\pst@dimd sp }%
  \endgroup
  \let\psk@boxfillsize\pst@tempg}
%    \end{macrocode}
% \end{macro}
%

% \subsection{Definition of the parameters}
%
%    \begin{macrocode}
\define@key[psset]{pst-fill}{boxfillsize}{%
  \def\pst@tempg{#1}\def\pst@temph{auto}%
  \ifx\pst@tempg\pst@temph
    \let\psk@boxfillsize\relax
  \else
    \pst@@boxfillsize#1(\z@,\z@)\@empty(\z@,\z@)(\@nil
  \fi}
\psset{boxfillsize={(-15cm,-15cm)(15cm,15cm)}}
\define@key[psset]{pst-fill}{boxfillcolor}{\pst@getcolor{#1}\psboxfillcolor}
\psset{boxfillcolor=black}% hv
\define@key[psset]{pst-fill}{boxfillangle}{\pst@getangle{#1}\psk@boxfillangle}
\psset{boxfillangle=0}
\define@key[psset]{pst-fill}{fillsepx}{%
  \pst@getlength{#1}\psk@fillsepx}
\define@key[psset]{pst-fill}{fillsepy}{%
  \pst@getlength{#1}\psk@fillsepy}
\define@key[psset]{pst-fill}{fillsep}{%
  \pst@getlength{#1}\psk@fillsepx%
  \let\psk@fillsepy\psk@fillsepx}
\psset{fillsep=2pt}

\ifx\PstTiling\@undefined
  \define@key[psset]{pst-fill}{fillcycle}{\pst@getint{#1}\psk@fillcycle}
  \psset{fillcycle=0}
\else
  \define@key[psset]{pst-fill}{fillangle}{\pst@getangle{#1}\psk@boxfillangle}
  \define@key[psset]{pst-fill}{fillsize}{%
      \def\pst@tempg{#1}\def\pst@temph{auto}%
      \ifx\pst@tempg\pst@temph\let\psk@boxfillsize\relax
      \else\pst@@boxfillsize#1(\z@,\z@)\@empty(\z@,\z@)(\@nil\fi}
  \psset{fillsep=0,fillsize=auto}
  \define@key[psset]{pst-fill}{fillcyclex}{\pst@getint{#1}\psk@fillcyclex}
  \define@key[psset]{pst-fill}{fillcycley}{\pst@getint{#1}\psk@fillcycley}
  \define@key[psset]{pst-fill}{fillcycle}{%
    \pst@getint{#1}\psk@fillcyclex\let\psk@fillcycley\psk@fillcyclex}
  \psset{fillcycle=0}
  \define@key[psset]{pst-fill}{fillmovex}{\pst@getlength{#1}\psk@fillmovex}
  \define@key[psset]{pst-fill}{fillmovey}{\pst@getlength{#1}\psk@fillmovey}
  \define@key[psset]{pst-fill}{fillmove}{%
      \pst@getlength{#1}\psk@fillmovex\let\psk@fillmovey\psk@fillmovex}
  \psset{fillmove=0pt}
  \define@key[psset]{pst-fill}{fillloopaddx}{\pst@getint{#1}\psk@fillloopaddx}
  \define@key[psset]{pst-fill}{fillloopaddy}{\pst@getint{#1}\psk@fillloopaddy}
  \define@key[psset]{pst-fill}{fillloopadd}{%
    \pst@getint{#1}\psk@fillloopaddx\let\psk@fillloopaddy\psk@fillloopaddx}
  \psset{fillloopadd=0}
%    \end{macrocode}
%
%    \begin{macrocode}
% For debugging (to debug, set PstDebug=1)
% we now use the one from pstricks to prevent a clash with package
% pstricks                        2004-06-22
%%    \define@key[psset]{pst-fill}{PstDebug}{\pst@getint{#1}\psk@PstDebug}
    \psset{PstDebug=0}
\fi
% DG addition end
%    \end{macrocode}

% \subsection{Definition of the fill box}
% \begin{macro}{psboxfill}
%    \begin{macrocode}
\newbox\pst@fillbox
\def\psboxfill{\pst@killglue\pst@makebox\psboxfill@i}
\def\psboxfill@i{\setbox\pst@fillbox\box\pst@hbox\ignorespaces}
%    \end{macrocode}
% \end{macro}
% \subsection{The main macros}
%
% \begin{macro}{psfs@boxfill}
%    \begin{macrocode}
\def\psfs@boxfill{%
  \ifvoid\pst@fillbox
    \@pstrickserr{Fill box is empty. Use \string\psboxfill\space first.}\@ehpa
  \else
    \ifx\psk@boxfillsize\relax \pst@AutoBoxFill
    \else\pst@ManualBoxFill\fi
  \fi}
%    \end{macrocode}
% \end{macro}
%
% \begin{macro}{pst@ManualBoxFill}
%    \begin{macrocode}
\def\pst@ManualBoxFill{%
  \leavevmode
  \begingroup
    \pst@FlushCode
    \begin@psclip
    \pstVerb{clip}%
    \expandafter\pst@AddFillBox\psk@boxfillsize
    \end@psclip
  \endgroup}
%    \end{macrocode}
% \end{macro}
%
% \begin{macro}{pst@FlushCode}
%    \begin{macrocode}
\def\pst@FlushCode{%
  \pst@Verb{%
    /mtrxc CM def
    CP CP T
    \tx@STV
    \psk@origin
    \psk@swapaxes
    \pst@newpath
    \pst@code
    mtrxc setmatrix
    moveto
    0 setgray}%
  \gdef\pst@code{}}
%    \end{macrocode}
% \end{macro}
%
% \begin{macro}{pst@AddFillBox}
%    \begin{macrocode}
\def\pst@AddFillBox#1 #2 #3 #4 {%
  \begingroup
    \setbox\pst@fillbox=\vbox{%
      \hbox{\unhcopy\pst@fillbox\kern\psk@fillsepx\p@}%
      \vskip\psk@fillsepy\p@}%
    \psk@boxfillsize
    \pst@cnta=\pst@dimc
    \advance\pst@cnta-\pst@dima
    \divide\pst@cnta\wd\pst@fillbox
    \pst@cntb=\pst@dimd
    \advance\pst@cntb-\pst@dimb
    \pst@dimd=\ht\pst@fillbox
    \divide\pst@cntb\pst@dimd
    \def\pst@tempa{%
      \pst@tempg
      \copy\pst@fillbox
      \advance\pst@cntc\@ne
      \ifnum\pst@cntc<\pst@cntd\expandafter\pst@tempa\fi}%
    \let\pst@tempg\relax
    \pst@cntc-\tw@
    \pst@cntd\pst@cnta
    \setbox\pst@fillbox=\hbox to \z@{%
      \kern\pst@dima
      \kern-\wd\pst@fillbox
      \pst@tempa
      \hss}%
    \pst@cntd\pst@cntb
%% DG modification begin - Dec. 11, 1997 - Patch 2
    \ifx\PstTiling\@undefined
      \ifnum\psk@fillcycle=\z@\pst@ManualFillCycle\fi
    \else
      \ifnum\psk@fillcyclex=\z@\pst@ManualFillCycle\fi
    \fi
%% DG modification end
    \global\setbox\pst@boxg=\vbox to\z@{%
      \offinterlineskip
      \vss
      \pst@tempa
      \vskip\pst@dimb}%
  \endgroup
  \setbox\pst@fillbox\box\pst@boxg
  \pst@rotate\psk@boxfillangle\pst@fillbox
  \box\pst@fillbox}
%    \end{macrocode}
% \end{macro}
%
% \begin{macro}{pst@ManualFillCycle}
%    \begin{macrocode}
\def\pst@ManualFillCycle{%
  \ifx\PstTiling\@undefined
    \pst@cntg=\psk@fillcycle
  \else
    \pst@cntg=\psk@fillcyclex
  \fi
  \pst@dimg=\wd\pst@fillbox
  \ifnum\pst@cntg=\z@
  \else
  \divide\pst@dimg\pst@cntg
  \fi
  \ifnum\pst@cntg<\z@\pst@cntg=-\pst@cntg\fi
  \advance\pst@cntg\m@ne
  \pst@cnth=\pst@cntg
  \def\pst@tempg{%
    \ifnum\pst@cnth<\pst@cntg\advance\pst@cnth\@ne\else\pst@cnth\z@\fi
    \moveright\pst@cnth\pst@dimg}}
%    \end{macrocode}
% \end{macro}
%
%% Auto box fill:        !! Fix dictionary
%
% \subsection{The PostScript subroutines}
%
%    \begin{macrocode}
%% DG addition begin - Apr. 8, 1997 and Dec. 1997 - Patch 2
\ifx\PstTiling\@undefined
\pst@def{AutoFillCycle}<%
  /c ED
  /n 0 def
  /s {
    /x x w c div n mul add def
    /n n c abs 1 sub lt { n 1 add } { 0 } ifelse def
  } def>

\pst@def{BoxFill}<%
  gsave
    gsave \tx@STV CM grestore dtransform CM idtransform
    abs /h ED abs /w ED
    pathbbox
    h div round 2 add cvi /y2 ED
    w div round 2 add cvi /x2 ED
    h div round 2 sub cvi /y1 ED
    w div round 2 sub cvi /x1 ED
    /y2 y2 y1 sub def
    /x2 x2 x1 sub def
    CP
    y1 h mul sub neg /y1 ED
    x1 w mul sub neg /x1 ED
    clip
    y2 {
      /x x1 def
      s
      x2 {
        save CP x y1
%% patch 4   hv --------------
        \ifx\VTeXversion\undefined
        \else
%%============ mv: 09-10-01 ??? this is likely to be a right change
        neg
%%============
        \fi
%% end patch 4
T moveto Box restore
        /x x w add def
      } repeat
      /y1 y1 h add def
    } repeat
    % Next line not useful... To see that, suppress clipping (DG)
    CP x y1 T moveto Box
  currentpoint currentfont grestore setfont moveto>
\else
%% DG modification begin - Apr. 8, 1997 and Nov. / Dec. 1997 - Patch 2
\pst@def{AutoFillCycleX}<%
  /cX ED
  /nX 0 def
  /CycleX {
    /x x w cX div nX mul add def
    /nX nX cX abs 1 sub lt { nX 1 add } { 0 } ifelse def
  } def>
\pst@def{AutoFillCycleY}<%
  /cY ED
  /mY 0 def
  /nY 0 def
  /CycleY {
    /y1 y1 h cY div mY mul sub def
    nY cY abs 1 sub lt { /nY nY 1 add def /mY 1 def }
                       { /nY 0 def        /mY cY abs 1 sub neg def } ifelse
  } def>

\pst@def{BoxFill}<%
  gsave
    gsave \tx@STV CM grestore dtransform CM idtransform
    abs /h ED abs /w ED
    pathbbox
    h div round 2 add cvi /y2 ED
    w div round 2 add cvi /x2 ED
    h div round 2 sub cvi /y1 ED
    w div round 2 sub cvi /x1 ED
    /CoefLoopX 0 def
    /CoefLoopY 0 def
    /CoefMoveX 0 def
    /CoefMoveY 0 def
    \psk@boxfillangle\space 0 ne {/CoefLoopX 8 def /CoefLoopY 8 def} if
    \psk@fillcyclex\space 0 ne {/CoefLoopX CoefLoopX 1 add def} if
    \psk@fillcycley\space 0 ne {/CoefLoopY CoefLoopY 1 add def} if
    \psk@fillmovex\space 0 ne
      {/CoefLoopX CoefLoopX 2 add def
       \psk@fillmovex\space 0 gt {/CoefMoveX CoefLoopX def}
                           {/CoefMoveX CoefLoopX neg def} ifelse} if
    \psk@fillmovey\space 0 ne
      {/CoefLoopY CoefLoopY 2 add def
       \psk@fillmovey\space 0 gt {/CoefMoveY CoefLoopY def}
                           {/CoefMoveY CoefLoopY neg def} ifelse} if
    \psk@fillsepx\space 0 ne {/CoefLoopX CoefLoopX 1 add def} if
    \psk@fillsepy\space 0 ne {/CoefLoopY CoefLoopY 1 add def} if
    /CoefLoopX CoefLoopX \psk@fillloopaddx\space add def
    /CoefLoopY CoefLoopY \psk@fillloopaddy\space add def
    /x2 x2 x1 sub 4 sub CoefLoopX 2 mul add def
    /y2 y2 y1 sub 4 sub CoefLoopY 2 mul add def
%% We must fix the origin of tiling, as it must not vary according other stuff
%% in the page!
    w x1 CoefLoopX add CoefMoveX add mul
      h y1 y2 add 1 sub CoefLoopY sub CoefMoveY sub mul moveto
    CP
    y1 h mul sub neg /y1 ED
    x1 w mul sub neg /x1 ED
%%  hv 2004-06-22   to prevent clash with pst-gr3d
%%    \psk@PstDebug 0 eq {clip} if
    \Pst@Debug 0 eq {clip} if
%% end hv
    \psk@fillmovex\space \psk@fillmovey
    gsave \tx@STV CM grestore dtransform CM idtransform
    /hmove ED /wmove ED
    /row 0 def
   y2 {
       /row row 1 add def
       /column 0 def
       /x x1 def
       CycleX
       save
       x2 {
          /column column 1 add def
          CycleY
          save CP x y1
%% patch 4   hv --------------
          \ifx\VTeXversion\undefined
          \else
%%============ mv: 09-10-01 ??? this is likely to be a right change
          neg
%%============
          \fi
  T moveto Box restore
          /x x w add def
          0 hmove translate
          } repeat
       restore
       /y1 y1 h add def
       wmove 0 translate
       } repeat
  currentpoint currentfont grestore setfont moveto>
\fi
%    \end{macrocode}

%    \begin{macrocode}
\def\pst@AutoBoxFill{%
  \leavevmode
  \begingroup
    \pst@stroke
    \pst@FlushCode
    \pst@Verb{\psk@boxfillangle\space \tx@RotBegin}%
    \pstVerb{\pst@dict /Box \pslbrace end}%
    \ifx\PstTiling\@undefined
    \else
      \ifx\pst@tempa\@undefined % Undefined for instance for \pscharpath
      \else\ifx\pst@tempa\@empty\else
        \def\pst@temph{0}%
        \ifx\pst@tempa\pst@temph
        \else
          \pstVerb{/TR {pop pop currentpoint translate \pst@tempa\space translate } def}%
        \fi
      \fi\fi
    \fi
    \hbox to \z@{\vbox to\z@{\vss\copy\pst@fillbox\vskip-\dp\pst@fillbox}\hss}%
    \ifx\PstTiling\@undefined
      \pstVerb{%
        tx@Dict begin \psrbrace def
        \ifnum\psk@fillcycle=\z@
          /s {} def
        \else
          \psk@fillcycle \tx@AutoFillCycle
        \fi
        \pst@number{\wd\pst@fillbox}%
        \psk@fillsepx\space add
        \pst@number{\ht\pst@fillbox}%
        \pst@number{\dp\pst@fillbox}%
        \psk@fillsepy\space add add
        \tx@BoxFill
        end}%
      \else
      \pstVerb{%
        tx@Dict begin \psrbrace def
        \ifnum\psk@fillcyclex=\z@
          /CycleX {} def
        \else
          \psk@fillcyclex\space \tx@AutoFillCycleX
        \fi
        \ifnum\psk@fillcycley=\z@
          /CycleY {} def
        \else
          \psk@fillcycley\space \tx@AutoFillCycleY
        \fi
        \pst@number{\wd\pst@fillbox}%
        \psk@fillsepx\space add
        \pst@number{\ht\pst@fillbox}%
        \pst@number{\dp\pst@fillbox}%
        \psk@fillsepy\space add add
        \tx@BoxFill
        end}%
    \fi
    \pst@Verb{\tx@RotEnd}%
  \endgroup}
%    \end{macrocode}
% \subsection{Closing}
%
%   Catcodes restoration.
%
%    \begin{macrocode}
\catcode`\@=\PstAtCode\relax
%    \end{macrocode}
%
%    \begin{macrocode}
%</pst-fill>
%    \end{macrocode}
%
% \Finale
%
\endinput
%%
%% End of file `pst-fill.dtx'
+\newline
%add the following definition:\newline
%\verb+\def\PstTiling{true}+
%
%  To obtain the original behaviour, just don't use the \emph{tiling} optional
%keyword at loading.
%
%  Take care than in \emph{tiling} mode, I introduce also some other changes.
%First I define aliases on some parameter names for consistancy (all specific
%parameters will begin by the \texttt{fill} prefix in this case) and I change
%some default values, which were not well adapted for tilings (\texttt{fillsep}
%is set to 0 and as explained \texttt{fillsize} set to \texttt{auto}). I rename 
%\texttt{fillcycle} to \texttt{fillcyclex}. I also restore normal way so that
%the frame of the area is drawn and all line (\texttt{linestyle},
%\texttt{linecolor}, \texttt{doubleline}, etc.) parameters are now active (but
%there are not in non \emph{tiling} mode). And I also introduce new parameters
%to control the tilings (see below).
%
%  \textbf{In all the following examples, we will consider only the
% \emph{tiling} mode.}
%
%  To do a tiling, we have just to define the pattern with the
% \verb+\psboxfill+ macro and to use the new \texttt{fillstyle}
% \verb+boxfill+.
%
%  Note that tilings are drawn from left to right and top to bottom, which can
%have an importance in some circonstances.
%
%  PostScript programmers can be also interested to know that, even in the
%\emph{automatic} mode, the iterations of the pattern are managed directly by
%the PostScript code of the package which used only PostScript Level 1
%operators. The special ones introduced in Level 2 for drawing of patterns
%\cite[section 4.9]{PostScript95} are not used.
%
%  And first, for conveniance, we define a simple \cs{Tiling} macro, which
%will simplify our examples:
%
%\begin{verbatim}
%  \newcommand{\Tiling}[2][]{%
%    \edef\Temp{#1}%
%    \begin{pspicture}#2
%      \ifx\Temp\empty
%        \psframe[fillstyle=boxfill]#2
%      \else
%        \psframe[fillstyle=boxfill,#1]#2
%      \fi
%    \end{pspicture}}
%\end{verbatim}
%
%
%\newcommand{\Tiling}[2][]{%
%  \edef\Temp{#1}%
%  \begin{pspicture}#2
%    \ifx\Temp\empty
%      \psframe[fillstyle=boxfill]#2
%    \else
%      \psframe[fillstyle=boxfill,#1]#2
%    \fi
% \end{pspicture}}
%
%\subsection{Parameters}
%
%  There are \textbf{14} specific parameters available to change the way the
% filling/tiling is defined, and one debugging option.
%
% \begin{Description}{2cm}
%  \item [fillangle (real)\hfill :] the value of the rotation
%  applied to the patterns (\emph{Default:~0}).
% \end{Description}
%
%
%   In this case, we must force the tiling area to be notably larger than the
% area to cover, to be sure that the defined area will be covered after rotation.
% \lstset{gobble=2}
% \begin{LTXexample}
% \newcommand{\Square}{%
%   \begin{pspicture}(1,1)
%     \psframe[dimen=middle](1,1)
%   \end{pspicture}}
% \psset{unit=0.5}
% \psboxfill{\Square}
% \Tiling[fillangle=45]{(3,3)}\quad
% \Tiling[fillangle=-60]{(3,3)}
% \end{LTXexample}
% 
% \newcommand{\Square}{\begin{pspicture}(1,1)\psframe[dimen=middle](1,1)\end{pspicture}}
% 
% \begin{Description}{2cm}
%   \setcounter{footnote}{1}
%   \item[\texttt{fillsepx} (real$\|$dim) :] value of the horizontal
%   separation between consecutive patterns (\emph{Default:~0 for
%   tilings\footnotemark, 2pt otherwise}).  \footnotetext{This option was added
%   by me, is not part of the original package and is available only if the
%   \texttt{tiling} keyword is used when loading the package.}
%   \setcounter{footnote}{1}
%   \item [\texttt{fillsepy} (real$\|$dim)\hfill :] value of the vertical
%   separation between consecutive patterns (\emph{Default:~0 for
%   ti\-lings\footnotemark, 2pt otherwise}).
%   \setcounter{footnote}{1}
%   \item [\texttt{fillsep} (real$\|$dim)\hfill :] value of horizontal and
%   vertical separations between consecutive patterns (\emph{Default:~0 for
%   tilings\footnotemark, 2pt otherwise}).
% \end{Description}
% 
%   These values can be negative, which allow the tiles to overlap.
% 
% \begin{LTXexample}
% \psset{unit=0.5}
% \psboxfill{\Square}
% \Tiling[fillsepx=2mm]{(3,3)} 
% \Tiling[fillsepy=1mm]{(3,3)}\\
% \Tiling[fillsep=0.5]{(3,3)} 
% \Tiling[fillsep=-0.5]{(3,3)}
% \end{LTXexample}
% 
% \begin{Description}{2cm}
%   \item [\texttt{fillcyclex}\footnotemark\ (integer)\hfill :] Shift
%   coefficient applied to each row (\emph{Default:~0}).
%   \footnotetext{It was \texttt{fillcycle} in the original version.}
%   \setcounter{footnote}{1}
%   \item [\texttt{fillcycley}\footnotemark\ (integer)\hfill :] Same thing for
%   columns (\emph{Default:~0}).
%   \setcounter{footnote}{1}
%   \item [\texttt{fillcycle}\footnotemark\ (integer)\hfill :] Allow to fix
%   both \texttt{fillcyclex} and \texttt{fillcycley} directly to the same value
%   (\emph{Default:~0}).
% \end{Description}
% 
%   For instance, if \texttt{fillcyclex} is 2, the second row of patterns will
% be horizontally shifted by a factor of $\frac{1}{2}=0.5$, and by a factor of
% 0.333 if \texttt{fillcyclex} is 3, etc.). These values can be negative.
% 
% \begin{LTXexample}[width=0.35\linewidth]
% \psset{unit=0.5}
% \psboxfill{\Square}
% \newcommand{\TilingA}[1]{\Tiling[fillcyclex=#1]{(3,3)}}
% \TilingA{0} \TilingA{1}\\
% \TilingA{2} \TilingA{3}\\[3mm]
% \TilingA{4} \TilingA{5}\\
% \TilingA{6} \TilingA{-3}\\[3mm]
% \Tiling[fillcycley=2]{(3,3)}
% \Tiling[fillcycley=3]{(3,3)}\\
% \Tiling[fillcycley=-3]{(3,3)}
% \Tiling[fillcycle=2]{(3,3)}
% \end{LTXexample}
% 
% \begin{Description}{2cm}
%   \setcounter{footnote}{1}
%   \item [\texttt{fillmovex}\footnotemark\ (real$\|$dim)\hfill :] value of the
%   horizontal moves between consecutive patterns (\emph{Default:~0}).
%   \setcounter{footnote}{1}
%   \item [\texttt{fillmovey}\footnotemark\ (real$\|$dim)\hfill :] value of the
%   vertical moves between consecutive patterns (\emph{Default:~0}).
%   \setcounter{footnote}{1}
%   \item [\texttt{fillmove}\footnotemark\ (real$\|$dim)\hfill :] value of
%   horizontal and vertical moves between consecutive patterns
%   (\emph{Default:~0}).
% \end{Description}
% 
%   These parameters allow the patterns to overlap and to draw some special
% kinds of tilings. They are implemented only for the \emph{automatic} and
% \emph{tiling} modes and their values can be negative.
% 
%   In some cases, the effect of these parameters will be the same that with the 
% \texttt{fillcycle?} ones, but you can see that it is not true for some other
% values.
% 
% \begin{LTXexample}
% \psset{unit=0.5}
% \psboxfill{\Square}
% \Tiling[fillmovex=0.5]{(3,3)} 
% \Tiling[fillmovey=0.5]{(3,3)}\\
% \Tiling[fillmove=0.5]{(3,3)}
% \Tiling[fillmove=-0.5]{(3,3)}
% \end{LTXexample}
% 
% \begin{Description}{2cm}
%   \item [\texttt{fillsize}
%   (auto$\|$\{(real$\|$dim,real$\|$dim)(real$\|$dim,real$\|$dim)\}) :] The
%   choice of \emph{automatic} mode or the size of the area in \emph{manual}
%   mode. If first pair values are not given, (0,0) is used. (\emph{Default:
%   auto when \emph{tiling} mode is used, {(-15cm,-15cm)(15cm,15cm)}
%   otherwise}).
% \end{Description}
% 
%   As explained in the introduction, the \emph{manual} mode can require very
% huge amount of computer ressources. So, it usage is to discourage in front off
% the \emph{automatic} mode. It seems only useful in special circonstances, in
% fact when the \emph{automatic} mode failed, which is known only in one case,
% for some kinds of EPS files, as the ones produce by dump of portions of
% screens (see \ref{sec:GraphicFiles}).
% 
% \begin{Description}{2cm}
%   \setcounter{footnote}{1}
%   \item [\texttt{fillloopaddx}\footnotemark\ (integer)\hfill :] number of
%   times the pattern is added on left and right positions (\emph{Default:~0}).
%   \setcounter{footnote}{1}
%   \item [\texttt{fillloopaddy}\footnotemark\ (integer)\hfill :] number of
%   times the pattern is added on top and bottom positions (\emph{Default:~0}).
%   \setcounter{footnote}{1}
%   \item [\texttt{fillloopadd}\footnotemark\ (integer)\hfill :] number of
%   times the pattern is added on left, right, top and bottom positions
%   (\emph{Default:~0}).
% \end{Description}
% 
%   These parameters are only useful in special circonstances, as for complex
% patterns when the size of the rectangular box used to tile the area doesn't 
% correspond to the pattern itself (see an example in Figure~\ref{fig:Sheeps})
% and also sometimes when the size of the pattern is not a divisor of the size
% of the area to fill and that the number of loop repeats is not properly
% computed, which can occur.
% 
%   They are implemented only for the \emph{tiling} mode.
% 
% \begin{Description}{2cm}
%   \setcounter{footnote}{1}
%   \item [\texttt{PstDebug}\footnotemark\ (integer, 0 or 1)\hfill :] to
%   require to see the exact tiling done, without clipping (\emph{Default:~0}).
% \end{Description}
% 
%   It's mainly useful for debugging or to understand better how the tilings
% are done. It is implemented only for the \emph{tiling} mode.
% 
% \begin{LTXexample}
% \psset{unit=0.3,PstDebug=1}
% \psboxfill{\Square}
% \psset{linewidth=1mm}
% \Tiling{(2,2)}\\[5mm]
% \Tiling[fillcyclex=2]{(2,2)}\\[1cm]
% \Tiling[fillmove=0.5]{(2,2)}
% \end{LTXexample}
% 
% \vspace{3cm}
% \section{Examples}
% 
%   In fact this unique \cs{psboxfill} macro allow a lot a variations and
% different usages. We will try here to demonstrate this.
% 
% \subsection{Kind of tiles}
% \label{sec:KindTiles}
% 
%   Of course, we can access to all the power of PSTricks macros to define the
% \emph{tiles} (\emph{patterns}) used. So, we can define complicated ones.
% 
%   Here we give four other Archimedian tilings (those built with only some
% regular polygons) among the twelve existing, first discovered completely by
% Johanes \textsc{Kepler} at the beginning of 17th century \cite{GS87}, the two
% other \emph{regular} ones with the tiling by squares, formed by a unique
% regular polygon, and two other formed by two different regular polygons.
% 
% \begin{LTXexample}[pos=t]
%   \newcommand{\Triangle}{%
%     \begin{pspicture}(1,1)
%       \pstriangle[dimen=middle](0.5,0)(1,1)
%     \end{pspicture}}
%   \newcommand{\Hexagon}{
% ^^A sin(60)=0.866
%     \begin{pspicture}(0.866,0.75)
%       \SpecialCoor
% ^^A  Hexagon  
%       \pspolygon[dimen=middle]%
%         (0.5;30)(0.5;90)(0.5;150)(0.5;210)(0.5;270)(0.5;330)
%     \end{pspicture}}
% 
%   \psset{unit=0.5}
%   \psboxfill{\Triangle}
%   \Tiling{(4,4)}\hfill
% ^^A The two other regular tilings
%   \Tiling[fillcyclex=2]{(4,4)}\hfill
%   \psboxfill{\Hexagon}
%   \Tiling[fillcyclex=2,fillloopaddy=1]{(5,5)}
% \end{LTXexample}
% 
% \begin{LTXexample}[pos=t]
%   \newcommand{\ArchimedianA}{%
%      ^^A Archimedian tiling 3^2.4.3.4
%     \psset{dimen=middle}
%      ^^A sin(60)=0.866
%     \begin{pspicture}(1.866,1.866)
%       \psframe(1,1)
%       \psline(1,0)(1.866,0.5)(1,1)(0.5,1.866)(0,1)(-0.866,0.5)
%       \psline(0,0)(0.5,-0.866)
%     \end{pspicture}}
%   \newcommand{\ArchimedianB}{%
%      ^^A Archimedian tiling 4.8^2
%     \psset{dimen=middle,unit=1.5}
%      ^^A sin(22.5)=0.3827 ; cos(22.5)=0.9239
%     \begin{pspicture}(1.3066,0.6533)
%       \SpecialCoor
%      ^^A Octogon
%       \pspolygon(0.5;22.5)(0.5;67.5)(0.5;112.5)(0.5;157.5)
%                 (0.5;202.5)(0.5;247.5)(0.5;292.5)(0.5;337.5)
%     \end{pspicture}}
% 
%   \psset{unit=0.5}
%   \psboxfill{\ArchimedianA}
%   \Tiling[fillmove=0.5]{(7,7)}\hfill
%   \psboxfill{\ArchimedianB}
%   \Tiling[fillcyclex=2,fillloopaddy=1]{(7,7)}
% \end{LTXexample}
% 
%   \setcounter{footnote}{3}
%   We can of course tile an area arbitrarily defined. And with the
% \texttt{addfillstyle} parameter\footnote{Introduced in PSTricks 97.}, we can
% easily mix the \texttt{boxfill} style with another one.
% 
% \begin{LTXexample}[width=6cm]
%   \psset{unit=0.5,dimen=middle}
%   \psboxfill{%
%     \begin{pspicture}(1,1)
%       \psframe(1,1)
%       \pscircle(0.5,0.5){0.25}
%     \end{pspicture}}
%   \begin{pspicture}(4,6)
%     \pspolygon[fillstyle=boxfill,fillsep=0.25](0,1)(1,4)(4,6)(4,0)(2,1)
%   \end{pspicture}\hspace{1em}
%   \begin{pspicture}(4,4)
%%     \pscircle[linestyle=none,fillstyle=solid,fillcolor=yellow,fillsep=0.5,
%%               addfillstyle=boxfill](2,2){2}
%   \end{pspicture}
% \end{LTXexample}
%
%   Various effects can be obtained, sometimes complicated ones very easily, as
% in this example reproduced from one shown by Slavik \textsc{Jablan} in the
% field of \emph{OpTiles}, inspired by the \emph{Op-art}:
% 
% \begin{LTXexample}[pos=t]
% \newcommand{\ProtoTile}{%
%  \begin{pspicture}(1,1)%%% 1/12=0.08333
%   \psset{linestyle=none,linewidth=0,
%     hatchwidth=0.08333\psunit,hatchsep=0.08333\psunit}
%   \psframe[fillstyle=solid,fillcolor=black,addfillstyle=hlines,hatchcolor=white](1,1)
%   \pswedge[fillstyle=solid,fillcolor=white,addfillstyle=hlines]{1}{0}{90}
%  \end{pspicture}}
% \newcommand{\BasicTile}{%
%  \begin{pspicture}(2,1)
%    \rput[lb](0,0){\ProtoTile}\rput[lb](1,0){\psrotateleft{\ProtoTile}}
%  \end{pspicture}}
% \ProtoTile\hfill\BasicTile\hfill
% \psboxfill{\BasicTile}
% \Tiling[fillcyclex=2]{(4,4)}
% \end{LTXexample}
% 
%   It is also directly possible to surimpose several different tilings. Here is
% the splendid visual proof of the \textsc{Pytha\-gore} theorem done by the arab
% mathematician \textsc{Annairizi} around the year 900, given by superposition
% of two tilings by squares of different sizes.
% 
% \begin{LTXexample}[pos=t]
% \psset{unit=1.5,dimen=middle}
% \begin{pspicture*}(3,3)
%   \psboxfill{\begin{pspicture}(1,1)
%     \psframe(1,1)\end{pspicture}}
%   \psframe[fillstyle=boxfill](3,3)
%   \psboxfill{\begin{pspicture}(1,1)
%     \rput{-37}{\psframe[linecolor=red](0.8,0.8)}
%   \end{pspicture}}
%   \psframe[fillstyle=boxfill](3,4)
%   \pspolygon[fillstyle=hlines,hatchangle=90](1,2)(1.64,1.53)(2,2)
% \end{pspicture*}
% \end{LTXexample}
% 
%   In a same way, it is possible to build tilings based on figurative patterns,
% in the style of the famous \textsc{Escher} ones. Following an example of
% Andr\'e \textsc{Deledicq} \cite{Deledicq97}, we first show a simple tiling of
% the \emph{p1} category (according to the international classification of the
% 17~symmetry groups of the plane first discovered by the russian
% crystalographer Jevgraf \textsc{Fedorov} at the end of the 19th century):
% 
% \begin{LTXexample}[pos=t]
%  \newcommand{\SheepHead}[1]{%
%    \begin{pspicture}(3,1.5)
%      \pscustom[liftpen=2,fillstyle=solid,fillcolor=#1]{%
%        \pscurve(0.5,-0.2)(0.6,0.5)(0.2,1.3)(0,1.5)(0,1.5)
%          (0.4,1.3)(0.8,1.5)(2.2,1.9)(3,1.5)(3,1.5)(3.2,1.3)
%          (3.6,0.5)(3.4,-0.3)(3,0)(2.2,0.4)(0.5,-0.2)}
%      \pscircle*(2.65,1.25){0.12\psunit} % Eye
%      \psccurve*(3.5,0.3)(3.35,0.45)(3.5,0.6)(3.6,0.4)% Muzzle
%     ^^A   % Mouth
%       \pscurve(3,0.35)(3.3,0.1)(3.6,0.05)
%     ^^A   % Ear
%       \pscurve(2.3,1.3)(2.1,1.5)(2.15,1.7)\pscurve(2.1,1.7)(2.35,1.6)(2.45,1.4)
%   \end{pspicture}}
%  \psboxfill{\psset{unit=0.5}\SheepHead{yellow}\SheepHead{cyan}}
%  \Tiling[fillcyclex=2,fillloopadd=1]{(10,5)}
% \end{LTXexample}
% \label{fig:Sheeps}
% 
%   Now a tiling of the \emph{pg} category (the code for the kangaroo itself is
% too long to be shown here, but has no difficulties ; the kangaroo is reproduce
% from an original picture from Raoul \textsc{Raba} and here is a translation in
% PSTricks from the one drawn by Emmanuel \textsc{Chailloux} and Guy
% \textsc{Cousineau} for their MLgraph system \cite{MLgraphTSI}):
% 
% \begin{LTXexample}[pos=t]
% \psboxfill{\psset{unit=0.4}
%   \Kangaroo{yellow}\Kangaroo{red}\Kangaroo{cyan}\Kangaroo{green}%
%   \psscalebox{-1 1}{%
%     \rput(1.235,4.8){\Kangaroo{green}\Kangaroo{cyan}\Kangaroo{red}\Kangaroo{yellow}}}}
%   \Tiling[fillloopadd=1]{(10,6)}
% \end{LTXexample}
% 
%   And here a \textsc{Wang} tiling \cite{Wang65}, \cite[chapter
% 11]{GS87}, based on very simple tiles of the form of a square and composed
% of four colored triangles. Such tilings are built with only a matching color
% constraint. Despite of it simplicity, it is an important kind of tilings, as
% \textsc{Wang} and others used them to study the special class of
% \emph{aperiodic} tilings, and also because it was shown that surprisingly this 
% tiling is similar to a \textsc{Turing} machine.
% 
% \begin{LTXexample}[pos=t]
%   \newcommand{\WangTile}[4]{%
%     \begin{pspicture}(1,1)
%       \pspolygon*[linecolor=#1](0,0)(0,1)(0.5,0.5)
%       \pspolygon*[linecolor=#2](0,1)(1,1)(0.5,0.5)
%       \pspolygon*[linecolor=#3](1,1)(1,0)(0.5,0.5)
%       \pspolygon*[linecolor=#4](1,0)(0,0)(0.5,0.5)
%     \end{pspicture}}
%   \newcommand{\WangTileA}{\WangTile{cyan}{yellow}{cyan}{cyan}}
%   \newcommand{\WangTileB}{\WangTile{yellow}{cyan}{cyan}{red}}
%   \newcommand{\WangTileC}{\WangTile{cyan}{red}{yellow}{yellow}}
%   \newcommand{\WangTiles}[1][]{%
%     \begin{pspicture}(3,3) \psset{ref=lb}
%       \rput(0,2){\WangTileB}  \rput(1,2){\WangTileA}%
%       \rput(2,2){\WangTileC}  \rput(0,1){\WangTileC}%
%       \rput(1,1){\WangTileB}  \rput(2,1){\WangTileA}
%       \rput(0,0){\WangTileA}  \rput(1,0){\WangTileC}%
%       \rput(2,0){\WangTileB}
%       #1
%     \end{pspicture}}
%   \WangTileA\hfill\WangTileB\hfill\WangTileC\hfill
%   \WangTiles[{\psgrid[subgriddiv=0,gridlabels=0](3,3)}]\hfill
%   \psset{unit=0.4} \psboxfill{\WangTiles} \Tiling{(12,12)}
% \end{LTXexample}
% 
% \subsection{External graphic files}
% \label{sec:GraphicFiles}
% 
%   We can also fill an arbitrary area with an external image. We have only, 
% as usual, to matter of the \emph{BoundingBox} definition if there is no one
% provided or if it is not the accurate one, as for the well known
% \texttt{tiger} picture part of the \texttt{ghostscript} distribution.
% 
% \begin{LTXexample}[pos=t]
%   \psboxfill{%% Strangely require x1=x2...
%     \begin{pspicture}(0,1)(0,4.1)
%       \includegraphics[bb=17 176 560 74,width=3cm]{tiger}
%     \end{pspicture}}
%   \Tiling{(6,6.2)}
% \end{LTXexample}
% 
%   Nevertheless, there are some special files for which the \emph{automatic}
% mode doesn't work, specially for some files obtained by a screen dump, as in
% the next example, where a picture was reduced before it conversion in the
% \emph{Encapsulated PostScript} format by a screen dump utility. In this case,
% usage of the \emph{manual} mode is the only alternative, at the price of the
% real multiple inclusion of the EPS file. We must take care to specify the
% correct \texttt{fillsize} parameter, because otherwise the default values are
% large and will load the file many times, perhaps just really using few
% occurrences as the other ones would be clipped...
% 
% \begin{LTXexample}[pos=t]
%   \psboxfill{\includegraphics{flowers}}
%   \begin{pspicture}(8,4)
%     \psellipse[fillstyle=boxfill,fillsize={(8,4)}](4,2)(4,2)
%   \end{pspicture}
% \end{LTXexample}
% 
% \subsection{Tiling of characters}
% 
%   We can also use the \cs{psboxfill} macro to fill the interior of characters
% for special effects like these ones:
% 
% \begin{LTXexample}[pos=t]
%   \DeclareFixedFont{\bigsf}{T1}{phv}{b}{n}{4.5cm}
%   \DeclareFixedFont{\smallrm}{T1}{ptm}{m}{n}{3mm}
%   \psboxfill{\smallrm Since 182 days...}
%   \begin{pspicture*}(8,4)
%     \centerline{%
%       \pscharpath[fillstyle=gradient,gradangle=-45,
%                   gradmidpoint=0.5,addfillstyle=boxfill,
%                   fillangle=45,fillsep=0.7mm]
%                  {\rput[b](0,0.1){\bigsf 2000}}}
%   \end{pspicture*}
% \end{LTXexample}
% 
% \begin{LTXexample}[pos=t]
%   \DeclareFixedFont{\mediumrm}{T1}{ptm}{m}{n}{2cm}
%   \psboxfill{%
%     \psset{unit=0.1,linewidth=0.2pt}
%     \Kangaroo{PeachPuff}\Kangaroo{PaleGreen}%
%       \Kangaroo{LightBlue}\Kangaroo{LemonChiffon}%
%     \psscalebox{-1 1}{%
%       \rput(1.235,4.8){%
%         \Kangaroo{LemonChiffon}\Kangaroo{LightBlue}%
%           \Kangaroo{PaleGreen}\Kangaroo{PeachPuff}}}}
% ^^A   % A kangaroo of kangaroos...
%   \begin{pspicture}(8,2)
%     \pscharpath[linestyle=none,fillstyle=boxfill,fillloopadd=1]
%                {\rput[b](4,0){\mediumrm Kangaroo}}
%   \end{pspicture}
% \end{LTXexample}
% 
% \subsection{Other kinds of usage}
% 
%   Other kinds of usage can be imagined. For instance, we can use tilings in a
% sort of degenerated way to draw some special lines made by a unique or
% multiple repeating patterns. But it can be only a special dashed line, as here
% with three different dashes:
% 
% \begin{LTXexample}[pos=t]
%   \newcommand{\Dashes}{%
%     \psset{dimen=middle}
%     \begin{pspicture}(0,-0.5\pslinewidth)(1,0.5\pslinewidth)
%       \rput(0,0){\psline(0.4,0)}%
%         \rput(0.5,0){\psline(0.2,0)}%
%         \rput(0.8,0){\psline(0.1,0)}
%     \end{pspicture}}
% 
%   \newcommand{\SpecialDashedLine}[3]{%
%     \psboxfill{#3}
%     \Tiling[linestyle=none]
%            {(#1,-0.5\pslinewidth)(#2,0.5\pslinewidth)}}
% 
%   \SpecialDashedLine{0}{7}{\Dashes}
% 
%   \psset{unit=0.5,linewidth=1mm,linecolor=red}
%   \SpecialDashedLine{0}{10}{\Dashes}
% \end{LTXexample}
% 
%   It allow also to use special patterns in business graphics, as in the
% following example generated by \texttt{PstChart}\footnote{A personal
% development to draw business charts with PSTricks, not distributed.}.
% 
% \vspace{3mm}
% \begin{figure}[!ht]
% \centering
% \psset{unit=0.75}
% ^^A % Generated by pstchart.sh version 0.21 (11/28/97)
% {\psset{dimen=middle}
% \psset{xunit=2,yunit=0.005}
% \begin{pspicture}(-0.6,-200)(6.6,2300)
% ^^A   % Title
%   \rput(3,2200){\shortstack{Fantaisist repartition of kangaroos\\
%                             in the world (in thousands)}}
% ^^A   % Frame background
%   \psframe[fillstyle=solid,fillcolor=LemonChiffon](0,0)(6,2000)
% ^^A   % Graduations
%   \multido{\n=0+500}{5}{\rput[r](-0.12,\n){\psscalebox{0.8}{\n}}}
% ^^A   % Minor ticks
%   \multips(0,100)(0,100){19}{\psline[unit=4.8pt](1,0)}
%   \multips(6,100)(0,100){19}{\psline[unit=4.8pt](-1,0)}
% ^^A   % Major ticks
%   \multips(0,500)(0,500){3}{\psline[unit=9.6pt](1,0)}
%   \multips(6,500)(0,500){3}{\psline[unit=9.6pt](-1,0)}
% ^^A   % Lines from major ticks marks
%   \multips(0,500)(0,500){3}{\psline[linestyle=dotted,linewidth=0.6pt](6,0)}
% ^^A   % Drawing for the data
%   \psboxfill{\psset{unit=0.78\psxunit}\KangarooPstChart{red}}
%   \psframe[linestyle=none,fillstyle=boxfill,fillloopaddy=1](0.61,0)(1.39,1800)
%   \psboxfill{\psset{unit=0.78\psxunit}\KangarooPstChart{yellow}}
%   \psframe[linestyle=none,fillstyle=boxfill,fillloopaddy=1](1.61,0)(2.39,800)
%   \psboxfill{\psset{unit=0.78\psxunit}\KangarooPstChart{cyan}}
%   \psframe[linestyle=none,fillstyle=boxfill,fillloopaddy=1](2.61,0)(3.39,550)
%   \psboxfill{\psset{unit=0.78\psxunit}\KangarooPstChart{magenta}}
%   \psframe[linestyle=none,fillstyle=boxfill,fillloopaddy=1](3.61,0)(4.39,500)
%   \psboxfill{\psset{unit=0.78\psxunit}\KangarooPstChart{green}}
%   \psframe[linestyle=none,fillstyle=boxfill,fillloopaddy=1](4.61,0)(5.39,200)
% ^^A   % Bottom labels
%   \uput{0.2}[270]{0}(1,0){\psscalebox{0.7}{Oceania}}
%   \uput{0.2}[270]{0}(2,0){\psscalebox{0.7}{Africa}}
%   \uput{0.2}[270]{0}(3,0){\psscalebox{0.7}{Asia}}
%   \uput{0.2}[270]{0}(4,0){\psscalebox{0.7}{America}}
%   \uput{0.2}[270]{0}(5,0){\psscalebox{0.7}{Europe}}
% ^^A   % Frame box around the chart
%   \psframe[linestyle=solid](0,0)(6,2000)
% \end{pspicture}}
%   \caption{Bar chart generated by PstChart, with bars filled by patterns}
%   \label{fig:PstChart}
% \end{figure}
% 
% \section{``Dynamic'' tilings}
% 
%   In some cases, tilings used non \emph{static} tiles, that is to say that the 
% \emph{prototile(s)}, even if unique, can have several forms, by instance
% specified by different colors or rotations, not fixed before generation or
% varying each time.
% 
% \subsection{Lewthwaite-Pickover-Truchet tiling}
% 
%   We give here for example the so-called \emph{Truchet} tiling, which much be
% in fact better called \emph{Lewthwaite-Pick\-over-Truchet (LPT)} tiling%
% \footnote{For description of the context, history and references about
% S\'ebastien \textsc{Truchet} and this tiling, see \cite{EsperetGirou98}.}.
% 
%   The unique prototile is only a square with two opposite circle arcs.
% This tile has obviously two positions, if we rotate it from 90 degrees (see
% the two tiles on the next figure). A \emph{LPT tiling} is a tiling with
% randomly oriented LPT tiles. We can see that even if it is very simple in it
% principle, it draw sophisticated curves with strange properties.
% 
%   Nevertheless, in the straightforward way \FillPackage{} does not work,
% because the \cs{psboxfill} macro store the content of the tile used in a
% \TeX{} box, which is static. So the calling to the random function is done
% only one time, which explain that only one rotation of the tile is used for
% all the tiling. It's only the one of the two rotations which could differ from
% one drawing to the next one...
% 
% ^^A % Truchet (Lewthwaite-Pickover-Truchet) tiling
% ^^A % --------------------------------------------
% 
% \begin{LTXexample}[pos=t]
% ^^A   % LPT prototile
%   \newcommand{\ProtoTileLPT}{%
%     \psset{dimen=middle}
%     \begin{pspicture}(1,1)
%       \psframe(1,1)
%       \psarc(0,0){0.5}{0}{90}
%       \psarc(1,1){0.5}{-180}{-90}
%     \end{pspicture}}
% 
% ^^A   % LPT tile
%   \newcount\Boolean
%   \newcommand{\BasicTileLPT}{%
% ^^A     % From random.tex by Donald Arseneau
%     \setrannum{\Boolean}{0}{1}%
%     \ifnum\Boolean=0
%       \ProtoTileLPT%
%     \else
%       \psrotateleft{\ProtoTileLPT}%
%     \fi}
% 
%   \ProtoTileLPT\hfill\psrotateleft{\ProtoTileLPT}\hfill
%   \psset{unit=0.5}
%   \psboxfill{\BasicTileLPT}
%   \Tiling{(5,5)}
% \end{LTXexample}
% 
%   But, for simple cases, there is a solution to this problem using a mixture
% of PSTricks and PostScript programming. Here the PSTricks
% construction \verb+\pscustom{\code{...}}+ allow to insert PostScript code
% inside the \LaTeX{} + PSTricks one.
% 
%   Programmation is less straightforward, but it has also the advantage to be
% notably faster, as all the tilings operations are done in PostScript, and
% mainly to not be limited by \TeX{} memory (the \TeX{} + PSTricks solution
% I wrote in 1995 for the colored problem was limited to small sizes for this
% reason). Just note also that \cs{pslbrace} and \cs{psrbrace} are two
% PSTricks macros to define and be able to insert the \verb+{+ and \verb+}+
% characters.
% 
% \begin{LTXexample}[pos=t]
% ^^A   % LPT prototile
%   \newcommand{\ProtoTileLPT}{%
%     \psset{dimen=middle}
%     \psframe(1,1)
%     \psarc(0,0){0.5}{0}{90}
%     \psarc(1,1){0.5}{-180}{-90}}
% 
% ^^A   % Counter to change the random seed
%   \newcount\InitCounter
% ^^A   % LPT tile
%   \newcommand{\BasicTileLPT}{%
%     \InitCounter=\the\time
%     \pscustom{\code{%
%       rand \the\InitCounter\space sub 2 mod 0 eq \pslbrace}}
%     \begin{pspicture}(1,1)
%       \ProtoTileLPT
%     \end{pspicture}%
%     \pscustom{\code{\psrbrace \pslbrace}}
%     \psrotateleft{\ProtoTileLPT}%
%     \pscustom{\code{\psrbrace ifelse}}}
% 
%   \psset{unit=0.4,linewidth=0.4pt}
%   \psboxfill{\BasicTileLPT}
%   \Tiling{(15,15)}
% \end{LTXexample}
% 
%   Using the very surprising fact (see \cite{EsperetGirou98}) that
% coloration of these tiles do not depend of their neighbors (even if it is
% difficult to believe as the opposite seems obvious!) but only of the parity of
% the value of row and column positions, we can directly program in the same way
% a colored version of the LPT tiling.
% 
% \setcounter{footnote}{1}
%   We have also introduce in the \FillPackage{} code for \emph{tiling} mode two
% new accessible Post\-Script variables, \texttt{row} and
% \texttt{column}\footnotemark, which can be useful in some circonstances, like
% this one.
% 
% \begin{LTXexample}[pos=t]
% ^^A   % LPT prototile
%   \newcommand{\ProtoTileLPT}[2]{%
%     \psset{dimen=middle,linestyle=none,fillstyle=solid}
%     \psframe[fillcolor=#1](1,1)
%     \psset{fillcolor=#2}
%     \pswedge(0,0){0.5}{0}{90} \pswedge(1,1){0.5}{-180}{-90}}
% ^^A   % Counter to change the random seed
%   \newcount\InitCounter
% ^^A   % LPT tile
%   \newcommand{\BasicTileLPT}[2]{%
%     \InitCounter=\the\time
%     \pscustom{\code{%
%       rand \the\InitCounter\space sub 2 mod 0 eq \pslbrace
%       row column add 2 mod 0 eq \pslbrace}}
%     \begin{pspicture}(1,1)\ProtoTileLPT{#1}{#2}\end{pspicture}%
%     \pscustom{\code{\psrbrace \pslbrace}}
%     \ProtoTileLPT{#2}{#1}%
%     \pscustom{\code{%
%       \psrbrace ifelse \psrbrace \pslbrace row column add 2 mod 0 eq \pslbrace}}
%     \psrotateleft{\ProtoTileLPT{#2}{#1}}\pscustom{\code{\psrbrace \pslbrace}}
%     \psrotateleft{\ProtoTileLPT{#1}{#2}}\pscustom{\code{\psrbrace ifelse \psrbrace ifelse}}}
%   \psboxfill{\BasicTileLPT{red}{yellow}}
%   \Tiling{(4,4)}\hfill
%   \psset{unit=0.4}\psboxfill{\BasicTileLPT{blue}{cyan}}
%   \Tiling{(15,15)}
% \end{LTXexample}
% 
%   Another classic example is to generate coordinates and numerotation for a
% grid. Of course, it is possible to do it directly in PSTricks using nested
% \cs{multido} commands. It would be clearly easy to program, but, nevertheless, 
% for users who have a little knowledge of PostScript programming, this offer
% an alternative which is useful for large cases, because on this way it will
% be notably faster and less computer ressources consuming.
% 
%   Remember here that the tiling is drawn from left to right, and top to
% bottom, and note that the PostScript variable \texttt{x2} give the total
% number of columns.
% 
% \begin{LTXexample}[pos=t]
% ^^A   % \Escape will be the \ character
%   {\catcode`\!=0\catcode`\\=11!gdef!Escape{\}}
%   \newcommand{\ProtoTile}{%
%     \Square\pscustom{%
%       \moveto(-0.9,0.75) % In PSTricks units
%       \code{ /Times-Italic findfont 8 scalefont setfont
%         (\Escape() show row 3 string cvs show (,) show 
%         column 3 string cvs show (\Escape)) show}
%       \moveto(-0.5,0.25) % In PSTricks units
%       \code{ /Times-Bold findfont 18 scalefont setfont
%         1 0 0 setrgbcolor % Red color
%         /center {dup stringwidth pop 2 div neg 0 rmoveto} def
%         row 1 sub x2 mul column add 3 string cvs center show}}}
%   \psboxfill{\ProtoTile}
%   \Tiling{(6,4)}
% \end{LTXexample}
% 
% \subsection{A complete example: the Poisson equation}
% 
%   To finish, we will show a complete real example, a drawing to explain the
% method used to solve the \textsc{Poisson} equation by a domain
% decomposition method, adapted to distributed memory computers. The
% objective is to show the communications required between processes and the
% position of the data to exchange. This code also show some useful and powerful
% technics for PSTricks programming (look specially at the way some higher level
% macros are defined, and how the same object is used to draw the four
% neighbors).
%
%\psset{unit=1cm}
%\newcommand{\Pattern}[1]{%
%   \begin{pspicture}(-0.25,-0.25)(0.25,0.25)\rput{*0}{\psdot[dotstyle=#1]}
%   \end{pspicture}}
%\newcommand{\West}{\Pattern{o}}   \newcommand{\South}{\Pattern{x}}
%\newcommand{\Central}{\Pattern{+}}\newcommand{\North}{\Pattern{square}}
%\newcommand{\East}{\Pattern{triangle}}
%\newcommand{\Cross}{%
%  \pspolygon[unit=0.5,linewidth=0.2,linecolor=red](0,0)(0,1)(1,1)(1,2)(2,2)(2,1)%
%              (3,1)(3,0)(2,0)(2,-1)(1,-1)(1,0)}
%\newcommand{\StylePosition}[1]{\LARGE\textcolor{red}{\textbf{#1}}}
%\newcommand{\SubDomain}[4]{%
%    \psboxfill{#4}\begin{psclip}{\psframe[linestyle=none]#1}%
%      \psframe[linestyle=#3](5,5)\psframe[fillstyle=boxfill]#2%
%    \end{psclip}}
%\newcommand{\SendArea}[1]{\psframe[fillstyle=solid,fillcolor=cyan]#1}
%\newcommand{\ReceiveData}[2]{%
%  \psboxfill{#2}\psframe[fillstyle=solid,fillcolor=yellow,addfillstyle=boxfill]#1}%
%\newcommand{\Neighbor}[2]{%
%    \begin{pspicture}(5,5)
%      \rput{*0}(2.5,2.5){\StylePosition{#1}}
%      \ReceiveData{(0.5,0)(4.5,0.5)}{\Central}\SendArea{(0.5,0.5)(4.5,1)}%
%      \SubDomain{(5,2)}{(0.5,0.5)(4.5,3)}{dashed}{#2}%
%      \pcarc[arcangle=45,arrows=->](0.5,-1.25)(0.5,0.25)%
%      \pcarc[arcangle=45,arrows=->,linestyle=dotted,dotsep=2pt](4.5,0.75)(4.5,-0.75)%
%    \end{pspicture}}%
%  \psset{dimen=middle,dotscale=2,fillloopadd=2}
%\begin{pspicture}(-5.7,-5.7)(5.7,5.7)
%  \rput(0,0){%
%      \begin{pspicture}(5,5)
%        \ReceiveData{(0,0.5)(0.5,4.5)}{\West} \ReceiveData{(4.5,0.5)(5,4.5)}{\East}
%        \ReceiveData{(0.5,4.5)(4.5,5)}{\North}\ReceiveData{(0.5,0)(4.5,0.5)}{\South}
%        \SendArea{(0.5,0.5)(1,4.5)}\SendArea{(4,0.5)(4.5,4.5)}
%        \SendArea{(0.5,0.5)(4.5,1)}\SendArea{(0.5,4)(4.5,4.5)}
%        \SubDomain{(5,5)}{(0.5,0.5)(4.5,4.5)}{solid}{\Central}
%        \psline(1,0.5)(1,4.5)\psline(4,0.5)(4,4.5)%
%        \rput(1.5,4){\Cross}\rput(2,2){\Cross}%
%      \end{pspicture}}%
%  \rput(0,5.5){\Neighbor{N}{\North}}\rput{-90}(5.5,0){\Neighbor{E}{\East}}%
%  \rput{90}(-5.5,0){\Neighbor{W}{\West}}\rput{180}(0,-5.5){\Neighbor{S}{\South}}%
%\end{pspicture}
%
% \begin{lstlisting}
%   \newcommand{\Pattern}[1]{%
%     \begin{pspicture}(-0.25,-0.25)(0.25,0.25)\rput{*0}{\psdot[dotstyle=#1]}
%     \end{pspicture}}
%   \newcommand{\West}{\Pattern{o}}   \newcommand{\South}{\Pattern{x}}
%   \newcommand{\Central}{\Pattern{+}}\newcommand{\North}{\Pattern{square}}
%   \newcommand{\East}{\Pattern{triangle}}
%   \newcommand{\Cross}{%
%     \pspolygon[unit=0.5,linewidth=0.2,linecolor=red](0,0)(0,1)(1,1)(1,2)(2,2)(2,1)
%               (3,1)(3,0)(2,0)(2,-1)(1,-1)(1,0)}
%   \newcommand{\StylePosition}[1]{\LARGE\textcolor{red}{\textbf{#1}}}
%   \newcommand{\SubDomain}[4]{%
%     \psboxfill{#4}
%     \begin{psclip}{\psframe[linestyle=none]#1}
%       \psframe[linestyle=#3](5,5)\psframe[fillstyle=boxfill]#2
%     \end{psclip}}
%   \newcommand{\SendArea}[1]{\psframe[fillstyle=solid,fillcolor=cyan]#1}
%   \newcommand{\ReceiveData}[2]{%
%     \psboxfill{#2}
%     \psframe[fillstyle=solid,fillcolor=yellow,addfillstyle=boxfill]#1}
%   \newcommand{\Neighbor}[2]{%
%     \begin{pspicture}(5,5)
%       \rput{*0}(2.5,2.5){\StylePosition{#1}}
%       \ReceiveData{(0.5,0)(4.5,0.5)}{\Central}\SendArea{(0.5,0.5)(4.5,1)}
%       \SubDomain{(5,2)}{(0.5,0.5)(4.5,3)}{dashed}{#2}%
% ^^A       % Receive and send arrows
%       \pcarc[arcangle=45,arrows=->](0.5,-1.25)(0.5,0.25)
%       \pcarc[arcangle=45,arrows=->,linestyle=dotted,dotsep=2pt](4.5,0.75)(4.5,-0.75)
%     \end{pspicture}}
%   \psset{dimen=middle,dotscale=2,fillloopadd=2}
%   \begin{pspicture}(-5.7,-5.7)(5.7,5.7)
% ^^A     % Central domain
%     \rput(0,0){%
%       \begin{pspicture}(5,5)
% ^^A         % Receive from West, East, North and South
%         \ReceiveData{(0,0.5)(0.5,4.5)}{\West} \ReceiveData{(4.5,0.5)(5,4.5)}{\East}
%         \ReceiveData{(0.5,4.5)(4.5,5)}{\North}\ReceiveData{(0.5,0)(4.5,0.5)}{\South}
% ^^A         % send area for West, East, North and South
%         \SendArea{(0.5,0.5)(1,4.5)} \SendArea{(4,0.5)(4.5,4.5)}
%         \SendArea{(0.5,0.5)(4.5,1)} \SendArea{(0.5,4)(4.5,4.5)}
% ^^A         % Central domain
%         \SubDomain{(5,5)}{(0.5,0.5)(4.5,4.5)}{solid}{\Central}
% ^^A         % Redraw overlapped linesY
%         \psline(1,0.5)(1,4.5)  \psline(4,0.5)(4,4.5)
% ^^A         % Two crossesY
%         \rput(1.5,4){\Cross}  \rput(2,2){\Cross}
%       \end{pspicture}}
% ^^A     % The four neighborsY
%     \rput(0,5.5){\Neighbor{N}{\North}}     \rput{-90}(5.5,0){\Neighbor{E}{\East}}
%     \rput{90}(-5.5,0){\Neighbor{W}{\West}} \rput{180}(0,-5.5){\Neighbor{S}{\South}}
%   \end{pspicture}
% \end{lstlisting}
%
%
%
% Bibliography
% \begin{thebibliography}{99}
% \bibitem{PostScript95} Adobe, Systems~Incorporated, \emph{PostScript Language
% Reference Manual}, Addison-Wesley, 2~edition, 1995.
%
% \bibitem{Bolek98} Piotr Bolek, \MP{} and patterns, \emph{\TUB}, Volume~19,
% Number~3, pages 276--283, September 1998, \CTANref{mpattern}.
%
% \bibitem{MLgraphTSI} Emmanuel Chailloux, Guy Cousineau and Asc\'ander
% Su\'arez, Programmation fonctionnelle de graphismes pour la production
% d'illustrations techniques, \emph{Technique et science informatique},
% Volume~15, Number~7, pages 977--1007, 1996 (in french).
%
% \bibitem{Deledicq97} Andr\'e Deledicq, \emph{Le monde des pavages}, ACL
% \'Editions, 1997 (in french).
%
% \bibitem{EsperetGirou98} Philippe Esperet and Denis Girou,
% Coloriage du pavage dit de Truchet, Cahiers GUTenberg, Number~31,
% pages 5--18, December~1998  (in french).
%
% \bibitem{Girou94} Denis Girou, Pr\'esentation de PSTricks, \emph{Cahiers
% GUTenberg}, Number~16, pages 21--70, February~1994 (in french).
%
% \bibitem{LGC97} Michel Goossens, Sebastian Rahtz and Frank Mittelbach,
% \emph{The \LaTeX{} Graphics Companion}, Addison-Wesley, 2005.
%
% \bibitem{GS87} Branko Gr\"unbaum and Geoffrey Shephard, \emph{Tilings and
% Patterns}, Freeman and Company, 1987.
%
% \bibitem{Hoenig97} Alan Hoenig, \emph{\TeX{} Unbound: \LaTeX{} \& \TeX{}
% Strategies, Fonts, Graphics, and More}, Oxford University Press, 1997.
%
% \bibitem{XYpic} Kristoffer~H. Rose and Ross Moore, \XYpic. Pattern and Tile
% extension, available from \CTAN, 1991-1998, \CTANref{xypic}.
%
% \bibitem{LAAN96} Kees van der Laan, Paradigms: Just a little bit of PostScript,
% \emph{MAPS}, Volume~17, pages 137--150, 1996.
%
% \bibitem{LAAN97} Kees van der Laan, Tiling in PostScript and \MF{} -- Escher's
% wink, \emph{MAPS}, Volume~19, Number~2, pages 39--67, 1997.
%
% \bibitem{vanZandt93} Timothy Van Zandt, PSTricks. PostScript macros for
% Generic \TeX, available from \CTAN, 1993, \CTANref{pstricks}.
%
% \bibitem{vanZandtGirou94} Timothy Van Zandt and Denis Girou, Inside PSTricks,
% \emph{\TUB}, Volume~15, Number~3, pages 239--246, September 1994.
%
%
% \bibitem{voss07} Herbert Vo\ss, PSTricks -- Graphics for \TeX\ and \LaTeX, DANTE/Lehmanns, 4th ed., 2007.
% \bibitem{Wang65} Hao Wang, Games, Logic and Computers, \emph{Scientific
% American}, pages 98--106, November 1965.
% \end{thebibliography}
%
%
% \StopEventually{}
%
% ^^A .................... End of the documentation part ....................
%
% \section{Driver file}
%
%   The next bit of code contains the documentation driver file for \TeX{},
% i.e., the file that will produce the documentation you are currently
% reading. It will be extracted from this file by the \texttt{docstrip}
% program.
%
%    \begin{macrocode}
%<*driver>
\documentclass{ltxdoc}
\GetFileInfo{pst-fill.dtx}
%
\usepackage[T1]{fontenc}
\usepackage{lmodern}               % For PDF
\usepackage{graphicx}              % `graphicx' LaTeX standard package
\usepackage{showexpl}
\usepackage{mflogo}                % For the MetaFont and MetaPost logos
\input{random.tex}                 % Random macros from Donald Arseneau
\usepackage{url}                   % URLs convenient typesetting
\usepackage{multido}               % General loop macro
\usepackage[dvipsnames]{pstricks}  % PSTricks with the `color' extension
\usepackage{pst-text}              % PSTricks package for character path
\usepackage{pst-grad}              % PSTricks package for gradient filling
\usepackage{pst-node}              % PSTricks package for nodes
\usepackage[tiling]{pst-fill}      % PSTricks package for filling/tiling
%
\AtBeginDocument{%
%  \OnlyDescription % comment out for implementation details
  \EnableCrossrefs
  \CodelineIndex
  \RecordChanges}
\AtEndDocument{%
  \PrintIndex
  \setcounter{IndexColumns}{1}
  \PrintChanges}
\hbadness=7000            % Over and under full box warnings
\hfuzz=3pt
\begin{document}
  \DocInput{pst-fill.dtx}
\end{document}
%</driver>
%    \end{macrocode}
%
% \section{\texttt{pst-fill} \LaTeX{} wrapper}
%
%    \begin{macrocode}
%<*latex-wrapper>
\RequirePackage{pstricks}
\ProvidesPackage{pst-fill}[2005/09/13 package wrapper for 
  pst-fill.tex (hv)]
\DeclareOption{tiling}{\def\PstTiling{true}}
\ProcessOptions\relax
% \iffalse meta-comment, etc.
%%
%% Package `pst-fill.dtx'
%%
%% Denis Girou (CNRS/IDRIS - France) <Denis.Girou@idris.fr>
%% Herbert Voss <voss@pstricks.de>
%%
%% This program can be redistributed and/or modified under the terms
%% of the LaTeX Project Public License Distributed from CTAN archives
%% in directory macros/latex/base/lppl.txt.
%%
%% DESCRIPTION:
%%   `pst-fill' is a PSTricks package for filling and tiling areas 
%%
% \fi
% \changes{v1.01}{2007/03/10}{bugfix for incomplete ifx (hv)}
% \changes{v1.00}{2006/11/06}{use pst-xkey for extend keys (hv)}
% \changes{v0.99}{2004/08/17}{merge the VTeX and TeX versions (patch 4) (hv)}
% \changes{v0.98}{2004/06/22}{delete the Pst@Debug option and use the
%   the one from pstricks to prevent a clash with pst-gr3d (hv)}
% \changes{v0.97}{2001/10/09}{make it work with VTeX (mv)}
% \changes{v0.94}{1997/04/08}{With a \PstTiling macro defined (or "tiling" optional parameter
%   on \textbackslash usepackage[tiling]{pst-fill}), this file run exactly as
%   the original boxfill.tex file from Timothy, version 0.94,
%   except a correction in \textbackslash pst@ManualFillCycle to avoid a division by 0.
%   It's the default.}
% \changes{v0.93}{1997/04/07}{With a \textbackslash PstTiling macro defined (or "tiling" optional parameter
%   on \textbackslash usepackage[tiling]{pst-fill}) there are several add-ons
%   and changes to do `tiling' rather than `filling' in "automatic" mode :
%     - we fix the position of the beginning of tiling,
%     - we allow normally the framing of the area as expected, using
%       the line.... parameters
%     - we define move parameters fillmovex, fillmovey and fillmove,
%     - we define fillcyclex as previous fillcycle parameter, and add the
%       fillcycley and fillcycle (both fillcyclex and fillcycley) ones
%     - we can extend the tiling area using fillloopaddx, fillloopaddy and
%       fillloopadd parameters,
%     - we can debug and see the whole tiling area without clipping using
%       PstDebug parameter,
%     - for names consistancy, we can use fillangle in place of boxfillangle
%       and fillsize in place of boxfillsize,
%     - default value for fillsep is 0 and for fillsize is auto.}
%
% \DoNotIndex{\!,\",\#,\$,\%,\&,\',\(,\+,\*,\,,\-,\.,\/,\:,\;,\<,\=,\>,\?}
% \DoNotIndex{\@,\@B,\@K,\@cTq,\@f,\@fPl,\@ifnextchar,\@nameuse,\@oVk}
% \DoNotIndex{\[,\\,\],\^,\_,\ }
% \DoNotIndex{\^,\\^,\\\^,$\^$,$\\^$,$\\^$}
% \DoNotIndex{\0,\2,\4,\5,\6,\7,\8,}
% \DoNotIndex{\A,\a}
% \DoNotIndex{\B,\b,\Bc,\begin,\Bq,\Bqc}
% \DoNotIndex{\C,\c,\catcode,\cJA,\CodelineIndex,\csname}
% \DoNotIndex{\D,\def,\define@key,\Df,\divide,\DocInput,\documentclass,\pst@addfams}
% \DoNotIndex{\eCN,\edef,\else,\eHd,\eMcj,\EnableCrossrefs,\end,\endcsname}
% \DoNotIndex{\endCenterExample,\endExample,\endinput,\endpsclip}
% \DoNotIndex{\PrintIndex,\PrintChanges,\ProvidesFile}
% \DoNotIndex{\endpspicture,\endSideBySideExample,\Example}
% \DoNotIndex{\F,\f,\FdUrr,\fi,\filedate,\fileversion,\FV@Environment}
% \DoNotIndex{\FV@UseKeyValues,\FV@XRightMargin,\FVB@Example,\fvset}
% \DoNotIndex{\G,\g,\GetFileInfo,\gr,\GradientLoaded,\gsFKrbK@o,\gsj,\gsOX}
% \DoNotIndex{\hbadness,\hfuzz,\HLEmphasize,\HLMacro,\HLMacro@i}
% \DoNotIndex{\HLReverse,\HLReverse@i,\hqcu,\HqY}
% \DoNotIndex{\I,\i,\ifx,\input,\Ir,\IU}
% \DoNotIndex{\j,\jl,\JT,\JVodH}
% \DoNotIndex{\K,\k,\kfSlL}
% \DoNotIndex{\L,\let}
% \DoNotIndex{\message,\mHNa,\mIU}
% \DoNotIndex{\N,\nB,\newcmykcolor,\newdimen,\newif,\nW}
% \DoNotIndex{\O,\oCDJDo,\ocQhVI,\OnlyDescription,\oRKJ}
% \DoNotIndex{\P,\p,\ProvidesPackage,\psframe,\pslinewidth,\psset}
% \DoNotIndex{\PstAtCode,\PSTricksLoaded}
% \DoNotIndex{\q,\Qr,\qssRXq,\qu,\qXjFQp,\qYL}
% \DoNotIndex{\R,\r,\RecordChanges,\relax,\RlaYI,\rN,\Rp,\rp,\RPDXNn,\rput}
% \DoNotIndex{\S,\scalebox,\SgY,\SideBySide@Example,\SideBySideExample}
% \DoNotIndex{\SgY,\sk,\Sp,\space,\sZb}
% \DoNotIndex{\T,\the,\tw@}
% \DoNotIndex{\u,\UiSWGEf@,\uJi,\usepackage,\uVQdMM,\UYj}
% \DoNotIndex{\VerbatimEnvironment,\VerbatimInput,\VrC@}
% \DoNotIndex{\WhZ,\WjKCYb,\WNs}
% \DoNotIndex{\XkN,\XW}
% \DoNotIndex{\Z,\ZCM,\Ze}
% \DoNotIndex{\addtocounter,\advance,\alph,\arabic,\AtBeginDocument,\AtEndDocument}
% \DoNotIndex{\AtEndOfPackage,\begingroup,\bfseries,\bgroup,\box,\csname}
% \DoNotIndex{\else,\endcsname,\endgroup,\endinput,\expandafter,\fi}
% \DoNotIndex{\TeX,\z@,\p@,\@one,\xdef,\thr@@,\string,\sixt@@n,\reset,\or,\multiply,\repeat,\RequirePackage}
% \DoNotIndex{\@cclvi,\@ne,\@ehpa,\@nil,\copy,\dp,\global,\hbox,\hss,\ht,\ifodd,\ifdim,\ifcase,\kern}
% \DoNotIndex{\chardef,\loop,\leavevmode,\ifnum,\lower}
% \setcounter{IndexColumns}{2}
%
% ^^A To extend the height used for the text
%
% ^^A  Aligned labels in a description environment
%\newenvironment{Description}[1]{%
%\begin{list}{nothing}{\setlength{\leftmargin}{#1}
%\setlength{\labelwidth}{\leftmargin}\setlength{\labelsep}{1mm}}}
%{\end{list}}
%
% ^^A For macro names
%\DeclareRobustCommand\cs[1]{\texttt{\char`\\#1}}
%
%
% ^^A From ltugboat.cls
% ^^A For references
%\makeatletter
%\newcommand\acro[1]{\textsc{#1}\@}
%\def\CTAN{\acro{CTAN}}
%\let\texttub\textsl              % ^^A redefined in other situations
%\def\TUB{\texttub{TUGboat}}
%\def\TUG{\TeX\ \UG}
%\def\tug{\acro{TUG}}
%\def\UG{Users Group}
% ^^A For the bibliography 
%\let\@internalcite\cite
%\def\cite{\def\@citeseppen{-1000}%
%    \def\@cite##1##2{(##1\if@tempswa , ##2\fi)}%
%    \def\citeauthoryear##1##2##3{##1, ##3}\@internalcite}
%\def\etal{et\,al.\@}
%\newcommand\CTANdirectory[1]{\expandafter\urldef
%  \csname CTAN@#1\endcsname\path}
%\newcommand\CTANfile[1]{\expandafter\urldef
%  \csname CTAN@#1\endcsname\path}
%\newcommand\CTANref[1]{\expandafter\@setref\csname CTAN@#1\endcsname
%  \relax{#1}}
%\makeatother
% ^^A Define CTAN addresses 
%\CTANdirectory{mpattern}{graphics/metapost/macros/mpattern}
%\CTANdirectory{pstricks}{graphics/pstricks}
%\CTANdirectory{pst-fill.sty}{graphics/pstricks/latex/pst-fill.sty}
%\CTANdirectory{pst-fill}{graphics/pstricks/generic/pst-fill.tex}
%\CTANdirectory{Roegel}{graphics/metapost/contrib/macros/truchet}
%\CTANdirectory{xypic}{macros/generic/diagrams/xypic}
%
% ^^A Personal macros (D.G.)
% ^^A ----------------------
%
% ^^A Some colors used
%\definecolor{LemonChiffon}{rgb}{1.,0.98,0.8}
%\definecolor{LightBlue}   {rgb}{0.8,0.85,0.95}
%\definecolor{PaleGreen}   {rgb}{0.88,1,0.88}
%\definecolor{PeachPuff}   {rgb}{1.0,0.85,0.73}
%
% ^^A To define a unique string for TeX and LaTeX
%\newcommand{\AllTeX}{%
%{\rm(L\kern-.36em\raise.3ex\hbox{\sc a}\kern-.15em)%
%T\kern-.1667em\lower.7ex\hbox{E}\kern-.125emX}}
%
% ^^A Bibliography style
%\bibliographystyle{ltugbib}
%
% ^^A Name macros
%\newcommand{\FillPackage}{\textsf{`pst-fill'}}
%\newcommand{\XYpic}{%
%\leavevmode\hbox{\kern-.1em X\kern-.3em\lower.4ex\hbox{Y\kern-.15em}-pic}}
%
%\makeatletter
%
% ^^A Example environments
% ^^A (do not use in them the four JXYZ characters, that we will use
% ^^A as escape characters!)
%
% ^^A Default PSTricks parameters
%  \psset{dimen=middle}
%
% ^^A Translation in PSTricks from the one drawn by Emmanuel Chailloux and
% ^^A Guy Cousineau for the MLgraph system
% ^^A (see /ftp.ens.fr:/pub/unix/lang/MLgraph/version-2.1/MLgraph-refman.ps.gz)
% ^^A The kangaroo itself is reproduce from an original picture from Raoul Raba
% \newcommand{\DimX}{2.47}
% \newcommand{\DimY}{4.8}
% \newcommand{\DimXDivTwo}{1.235}
%
% \newcommand{\KangarooItself}[1]{%
% ^^A Body
% \pspolygon[fillstyle=solid,fillcolor=#1]%
%  (52.5,68)(55,72.5)(55.8,76.5)(56.8,79.8)(58.2,83)(60,85.8)(61.5,86.5)
% (64,87)(66,87.5)(67.8,87.3)(70,87)(71.5,87.3)(73,88)(74.7,88.5)
% (76,90.3)(77,91.5)(72.8,93.8)(69,96)(64.5,99)(59.4,103)(56.2,106.3)
% (53,110.5)(49.5,115.5)(47.2,119.9)(45.7,126)(43.2,123)(41.5,121)(37.5,125)
% (37,122.5)(36.8,120)(37,117)(37.6,113.5)(38.6,110)(40,106.3)(42,102.3)
%  (43.5,99.5)(45,97)(46.2,94)(46.8,91.7)(47.2,88)(47,83.5)(46.3,80.8)
%  (45.3,78.5)(42.5,76.5)(39.5,75.8)(36,75.9)(33,75.9)(29,76.2)(26,77)
%  (22.3,77.5)(18,78.4)(12.8,79.3)(8.6,80)(5.5,80.3)(3,80.5)(0,80)
%  (-5.2,78.5)(-9,76.3)(-11.2,74.8)(-13,72.5)(-16.5,68)(-16.5,68)(-19.5,62.5)
%  (-22,58)(-25.5,53)(-29,48.5)(-32.5,45)(-36,42)(-39,39.5)(-44,37)
%  (-49,35)(-51,34)(-53.5,34.5)(-55.5,36)(-56.5,38)(-56.5,40.5)(-55,41.5)
%  (-53.5,41)(-51.5,41)(-50.5,43)(-50.5,44.5)(-51,47)(-51.5,47.2)(-56.5,47)
%  (-58.5,46.5)(-60,44.7)(-62,42.3)(-63,39.5)(-63.5,36.3)(-63.5,33)(-63.1,29.5)
%  (-61.5,26)(-58,23.6)(-54,22.2)(-50.7,22)(-47.5,22)(-44.5,22.3)(-41,23.5)
%  (-36.8,25.8)(-33,28)(-28.5,31)(-23.4,35)(-20.2,38.3)(-17,42.5)(-13.5,47.5)
%  (-11.2,51.9)(-9.7,58)(-7.2,55)(-5.5,53)(-1.5,57)(-1,54.5)(-0.8,52)
%  (-1,49)(-1.6,45.5)(-2.6,42)(-4,38.3)(-6,34.3)(-7.5,31.5)(-9,29)
%  (-10.2,26)(-10.8,23.7)(-11.2,20)(-11,15.5)(-10.3,12.8)(-9.3,10.5)(-6.5,8.5)
%  (-3.5,7.8)(0,7.9)(3,7.9)(7,8.2)(10,9)(13.7,9.5)(18,10.4)
%  (23.2,11.3)(27.4,12)(30.5,12.3)(33,12.5)(36,12)(41.2,10.5)(45,8.3)
%  (47.2,6.8)(49,4.5)(52.5,0)(50,4.5)(49.2,8.5)(48.2,11.8)(46.8,15)
%  (45,17.8)(43.5,18.5)(41,19)(39,19.5)(37.2,19.3)(35,19)(33.5,19.3)
%  (32,20)(30.3,20.5)(29,22.3)(28,23.5)(28,23.5)(24.5,22.3)(21.5,22)
%  (18.3,22)(15,22.2)(11,23.6)(7.5,26)(5.9,29.5)(5.5,33)(5.5,36.3)
%  (6,39.5)(7,42.3)(9,44.7)(10.5,46.5)(12.5,47)(17.5,47.2)(18,47)
%  (18.5,44.5)(18.5,43)(17.5,41)(15.5,41)(14,41.5)(12.5,40.5)(12.5,38)
%  (13.5,36)(15.5,34.5)(18,34)(20,35)(25,37)(30,39.5)(33,42)
%  (36.5,45)(40,48.5)(43.5,53)(47,58)(49.5,62.5)(52.5,68)
% ^^A Eye
% \pscircle*[linecolor=white](58.2,98.3){2\psxunit}
% \pscircle*(58.2,97.3){\psxunit}
% ^^A Mouth
% \psline(71.5,88)(70,89.3)(68.5,90.3)(67,91.9)
% ^^A Tear
% \psline(42,121)(45,118)(47,115.3)(48.5,112.7)(50,110)(51.8,106.5)
%       (52.5,103.7)(53,100.5)
% \pspolygon(41.2,115.8)(43.2,114.7)(45,112.5)(47,109.8)(48,107)(49.5,104.2)%
%       (50.5,101.6)(51,98.5)(47.7,100.6)(46,102.2)(44.8,104)(43.5,106)
%       (42.5,108)(41.7,110.5)(41,113.2)}
%
% \newcommand{\Kangaroo}[1]{%
%   \begin{pspicture}(\DimX,\DimY)
%   \psset{unit=0.035278}
%   \KangarooItself{#1}
%   \end{pspicture}}
%
% \newcommand{\KangarooPstChart}[1]{{%
%   \psset{xunit=0.006784,yunit=0.00735,linewidth=0.01}
%   \begin{pspicture}(-65.5,0)(82,126)
%     \KangarooItself{#1}
%   \end{pspicture}}}
%
%
% ^^A For the possible index and changes log
% \setlength{\columnseprule}{0.6pt}
%
% ^^A Beginning of the documentation itself
%\title{\texttt{pst-fill}\\A PSTricks package for filling and tiling areas}
%\author{Timothy Van Zandt\thanks{\protect\url{tvz@econ.insead.fr}. (documentation by
% Denis Girou (\protect\url{Denis.Girou@idris.fr}) and Herbert Vo\ss (\protect\url{hvoss@tug.org}).}}
%
%\date{\shortstack{\today --- Version 1.00\\
%                  {\small Documentation revised \today}}}
% \maketitle
% \tableofcontents
%
%\begin{abstract}
%  \FillPackage{} is a PSTricks \cite{vanZandt93},\cite{Girou94},\cite{vanZandtGirou94}, 
%\cite{Hoenig97},\cite{LGC97} package to draw easily
%  various kinds of filling and tiling of areas. It is also a good example of
%  the great power and flexibility of PSTricks, as in fact it is a very short
%  program (it body is around 200~lines long) but nevertheless really powerful.
%
%  \hspace{5mm} It was written in 1994 by Timothy \textsc{van Zandt} but
%  publicly available only in PSTricks 97 and without any documentation.
%  We describe here the version \emph{97 patch 2} of December 12, 1997, which
%  is the original one modified by myself to manage \emph{tilings} in the
%  so-called \emph{automatic} mode. This article would like to serve both of
%  reference manual and of user's guide.
%
%This package is available on \CTAN{} in the
%  \texttt{graphics/pstricks} directory (files \texttt{latex/pst-fill.sty} and
%  \texttt{generic/pst-fill.tex}).
%\end{abstract}
%
%\section{Introduction}
%
%  Here we will refer as \emph{filling} as the operation which consist to fill
%a defined area by a pattern (or a composition of patterns). We will refer as
%\emph{tiling} as the operation which consist to do the same thing, but with
%the control of the starting point, which is here the upper left corner.
%The pattern is positioned relatively to this point. This make an essential
%difference between the two modes, as without control of the starting point we
%can't draw \emph{tilings} (sometimes  called \emph{tesselations}) as used in
%many fields of Art and Science%
%\footnote{For an extensive presentation of tilings, in their history and usage
%in many fields, see the reference book \cite{GS87}.
%
%  In the \TeX{} world, few work was done on tilings. You can look at the
%\emph{tile} extension of the \XYpic{} package \cite{XYpic}, at the articles of
%Kees \textsc{van der Laan} \cite[paragraph 7]{LAAN96} (the tiling was in
%fact directly done in PostScript) and \cite{LAAN97}, at the \MP{} program
%(available on \CTANref{Roegel}) by Denis \textsc{Roegel} for the
%\textsc{Truchet} contest in 1995 \cite{EsperetGirou98} and at the \MP{}
%package \cite{Bolek98} to draw patterns, which have a strong connection with
%tilings.}.
%
%  Nevertheless, as tilings are a wide and difficult field in mathematics, this
%package is limited to simple ones, mainly \emph{monohedral} tilings with one
%prototile (which can be composite, see section \ref{sec:KindTiles}). With some
%experience and wiliness we can do more and obtained easily rather
%sophisticated results, but obviously hyperbolic tilings like the famous
%\textsc{Escher} ones or aperiodic tilings like the \textsc{Penrose} ones are
%not in the capabilities of this package. For more complex needs, we must used
%low level and more painfull technics, with the basic \cs{multido}
%and \cs{multirput} macros.
%
%\section{Package history and description of it two different modes}
%
%  As already said, this package was written in 1994 by Timothy \textsc{van
%Zandt}. Two modes were defined, called respectively \emph{manual} and
%\emph{automatic}. For both, the pattern is generated on contiguous positions in
%a rather large area which include the region to fill, later cut to the
%required dimensions by clipping mechanism. In the first mode, the pattern is
%explicitely inserted in the PostScript file each time. In the second one, the
%result is the same but with an unique explicit insertion of the pattern and a
%repetition done by PostScript. Nevertheless, in this method, the control of
%the starting point was loosed, so it allowed only to \emph{fill} a region and
%not to \emph{tile} it.
%
%  See the difference between the two modes, \emph{tiling}:
% {\psset{unit=0.5cm}%
% \psboxfill{\begin{pspicture}(1,1)\psframe[dimen=middle](1,1)\end{pspicture}}
% \begin{pspicture}(3,3.3)
%   \psframe[fillstyle=boxfill](3,3)
% \end{pspicture}}
% and \emph{filling}:
%{%
% \makeatletter
%\pst@def{BoxFill}<
%  gsave
%    gsave \tx@STV CM grestore dtransform CM idtransform
%    abs /h ED abs /w ED
%    pathbbox
%    h div round 2 add cvi /y2 ED
%    w div round 2 add cvi /x2 ED
%    h div round 2 sub cvi /y1 ED
%    w div round 2 sub cvi /x1 ED
%    /y2 y2 y1 sub def
%    /x2 x2 x1 sub def
%    CP
%    y1 h mul sub neg /y1 ED
%    x1 w mul sub neg /x1 ED
%    clip
%    y2 {
%      /x x1 def
%      x2 {
%        save CP x y1 T moveto Box restore
%        /x x w add def
%      } repeat
%      /y1 y1 h add def
%    } repeat
% currentpoint currentfont grestore setfont moveto>
% \makeatother
%
% \psset{unit=0.5}
% \psboxfill{\begin{pspicture}(1,1)\psframe[dimen=middle](1,1)\end{pspicture}}
% \begin{pspicture}(3,3.3)
%   \psframe[fillstyle=boxfill](3,3)
% \end{pspicture}
% or
% \begin{pspicture}(3,3.3)
%   \psframe[fillstyle=boxfill](3,3)
% \end{pspicture}
%}
%as we can see that initial position is arbitrary and dependent of
%the current point.
%
%
% It's clear that usage of filling is very restrictive comparing to tiling,
%as desired effects required very often the possibility to control the starting 
%point. So, this mode was of limited interest, but unfortunately the
%\emph{manual} one has the very big disadvantage to require very huge amounts
%of ressources, mainly in disk space and consequently in printing time.
%A small tiling can require sometimes several megabytes in \emph{manual} mode!
%So, it was very often not really usable in practice.
%
%It is why I modified the code, to allow tilings in \emph{automatic} mode,
%controlling in this mode too the starting point. And most of the time, that is
%to say if some special options are not used, the tiling is done exactly in the
%region described, which make it faster. So there is no more reason to use the
%\emph{manual} mode, apart very special cases where \emph{automatic} one cannot
%work, as explained later -- currently, I know only one case.
%
%  To load this modified \emph{automatic} mode, with \LaTeX{} use
%simply:\newline 
%\verb+\usepackage[tiling]{pst-fill}+\newline
%and in plain \TeX{} after:\newline
%\verb+\input{pst-fill}+\newline
%add the following definition:\newline
%\verb+\def\PstTiling{true}+
%
%  To obtain the original behaviour, just don't use the \emph{tiling} optional
%keyword at loading.
%
%  Take care than in \emph{tiling} mode, I introduce also some other changes.
%First I define aliases on some parameter names for consistancy (all specific
%parameters will begin by the \texttt{fill} prefix in this case) and I change
%some default values, which were not well adapted for tilings (\texttt{fillsep}
%is set to 0 and as explained \texttt{fillsize} set to \texttt{auto}). I rename 
%\texttt{fillcycle} to \texttt{fillcyclex}. I also restore normal way so that
%the frame of the area is drawn and all line (\texttt{linestyle},
%\texttt{linecolor}, \texttt{doubleline}, etc.) parameters are now active (but
%there are not in non \emph{tiling} mode). And I also introduce new parameters
%to control the tilings (see below).
%
%  \textbf{In all the following examples, we will consider only the
% \emph{tiling} mode.}
%
%  To do a tiling, we have just to define the pattern with the
% \verb+\psboxfill+ macro and to use the new \texttt{fillstyle}
% \verb+boxfill+.
%
%  Note that tilings are drawn from left to right and top to bottom, which can
%have an importance in some circonstances.
%
%  PostScript programmers can be also interested to know that, even in the
%\emph{automatic} mode, the iterations of the pattern are managed directly by
%the PostScript code of the package which used only PostScript Level 1
%operators. The special ones introduced in Level 2 for drawing of patterns
%\cite[section 4.9]{PostScript95} are not used.
%
%  And first, for conveniance, we define a simple \cs{Tiling} macro, which
%will simplify our examples:
%
%\begin{verbatim}
%  \newcommand{\Tiling}[2][]{%
%    \edef\Temp{#1}%
%    \begin{pspicture}#2
%      \ifx\Temp\empty
%        \psframe[fillstyle=boxfill]#2
%      \else
%        \psframe[fillstyle=boxfill,#1]#2
%      \fi
%    \end{pspicture}}
%\end{verbatim}
%
%
%\newcommand{\Tiling}[2][]{%
%  \edef\Temp{#1}%
%  \begin{pspicture}#2
%    \ifx\Temp\empty
%      \psframe[fillstyle=boxfill]#2
%    \else
%      \psframe[fillstyle=boxfill,#1]#2
%    \fi
% \end{pspicture}}
%
%\subsection{Parameters}
%
%  There are \textbf{14} specific parameters available to change the way the
% filling/tiling is defined, and one debugging option.
%
% \begin{Description}{2cm}
%  \item [fillangle (real)\hfill :] the value of the rotation
%  applied to the patterns (\emph{Default:~0}).
% \end{Description}
%
%
%   In this case, we must force the tiling area to be notably larger than the
% area to cover, to be sure that the defined area will be covered after rotation.
% \lstset{gobble=2}
% \begin{LTXexample}
% \newcommand{\Square}{%
%   \begin{pspicture}(1,1)
%     \psframe[dimen=middle](1,1)
%   \end{pspicture}}
% \psset{unit=0.5}
% \psboxfill{\Square}
% \Tiling[fillangle=45]{(3,3)}\quad
% \Tiling[fillangle=-60]{(3,3)}
% \end{LTXexample}
% 
% \newcommand{\Square}{\begin{pspicture}(1,1)\psframe[dimen=middle](1,1)\end{pspicture}}
% 
% \begin{Description}{2cm}
%   \setcounter{footnote}{1}
%   \item[\texttt{fillsepx} (real$\|$dim) :] value of the horizontal
%   separation between consecutive patterns (\emph{Default:~0 for
%   tilings\footnotemark, 2pt otherwise}).  \footnotetext{This option was added
%   by me, is not part of the original package and is available only if the
%   \texttt{tiling} keyword is used when loading the package.}
%   \setcounter{footnote}{1}
%   \item [\texttt{fillsepy} (real$\|$dim)\hfill :] value of the vertical
%   separation between consecutive patterns (\emph{Default:~0 for
%   ti\-lings\footnotemark, 2pt otherwise}).
%   \setcounter{footnote}{1}
%   \item [\texttt{fillsep} (real$\|$dim)\hfill :] value of horizontal and
%   vertical separations between consecutive patterns (\emph{Default:~0 for
%   tilings\footnotemark, 2pt otherwise}).
% \end{Description}
% 
%   These values can be negative, which allow the tiles to overlap.
% 
% \begin{LTXexample}
% \psset{unit=0.5}
% \psboxfill{\Square}
% \Tiling[fillsepx=2mm]{(3,3)} 
% \Tiling[fillsepy=1mm]{(3,3)}\\
% \Tiling[fillsep=0.5]{(3,3)} 
% \Tiling[fillsep=-0.5]{(3,3)}
% \end{LTXexample}
% 
% \begin{Description}{2cm}
%   \item [\texttt{fillcyclex}\footnotemark\ (integer)\hfill :] Shift
%   coefficient applied to each row (\emph{Default:~0}).
%   \footnotetext{It was \texttt{fillcycle} in the original version.}
%   \setcounter{footnote}{1}
%   \item [\texttt{fillcycley}\footnotemark\ (integer)\hfill :] Same thing for
%   columns (\emph{Default:~0}).
%   \setcounter{footnote}{1}
%   \item [\texttt{fillcycle}\footnotemark\ (integer)\hfill :] Allow to fix
%   both \texttt{fillcyclex} and \texttt{fillcycley} directly to the same value
%   (\emph{Default:~0}).
% \end{Description}
% 
%   For instance, if \texttt{fillcyclex} is 2, the second row of patterns will
% be horizontally shifted by a factor of $\frac{1}{2}=0.5$, and by a factor of
% 0.333 if \texttt{fillcyclex} is 3, etc.). These values can be negative.
% 
% \begin{LTXexample}[width=0.35\linewidth]
% \psset{unit=0.5}
% \psboxfill{\Square}
% \newcommand{\TilingA}[1]{\Tiling[fillcyclex=#1]{(3,3)}}
% \TilingA{0} \TilingA{1}\\
% \TilingA{2} \TilingA{3}\\[3mm]
% \TilingA{4} \TilingA{5}\\
% \TilingA{6} \TilingA{-3}\\[3mm]
% \Tiling[fillcycley=2]{(3,3)}
% \Tiling[fillcycley=3]{(3,3)}\\
% \Tiling[fillcycley=-3]{(3,3)}
% \Tiling[fillcycle=2]{(3,3)}
% \end{LTXexample}
% 
% \begin{Description}{2cm}
%   \setcounter{footnote}{1}
%   \item [\texttt{fillmovex}\footnotemark\ (real$\|$dim)\hfill :] value of the
%   horizontal moves between consecutive patterns (\emph{Default:~0}).
%   \setcounter{footnote}{1}
%   \item [\texttt{fillmovey}\footnotemark\ (real$\|$dim)\hfill :] value of the
%   vertical moves between consecutive patterns (\emph{Default:~0}).
%   \setcounter{footnote}{1}
%   \item [\texttt{fillmove}\footnotemark\ (real$\|$dim)\hfill :] value of
%   horizontal and vertical moves between consecutive patterns
%   (\emph{Default:~0}).
% \end{Description}
% 
%   These parameters allow the patterns to overlap and to draw some special
% kinds of tilings. They are implemented only for the \emph{automatic} and
% \emph{tiling} modes and their values can be negative.
% 
%   In some cases, the effect of these parameters will be the same that with the 
% \texttt{fillcycle?} ones, but you can see that it is not true for some other
% values.
% 
% \begin{LTXexample}
% \psset{unit=0.5}
% \psboxfill{\Square}
% \Tiling[fillmovex=0.5]{(3,3)} 
% \Tiling[fillmovey=0.5]{(3,3)}\\
% \Tiling[fillmove=0.5]{(3,3)}
% \Tiling[fillmove=-0.5]{(3,3)}
% \end{LTXexample}
% 
% \begin{Description}{2cm}
%   \item [\texttt{fillsize}
%   (auto$\|$\{(real$\|$dim,real$\|$dim)(real$\|$dim,real$\|$dim)\}) :] The
%   choice of \emph{automatic} mode or the size of the area in \emph{manual}
%   mode. If first pair values are not given, (0,0) is used. (\emph{Default:
%   auto when \emph{tiling} mode is used, {(-15cm,-15cm)(15cm,15cm)}
%   otherwise}).
% \end{Description}
% 
%   As explained in the introduction, the \emph{manual} mode can require very
% huge amount of computer ressources. So, it usage is to discourage in front off
% the \emph{automatic} mode. It seems only useful in special circonstances, in
% fact when the \emph{automatic} mode failed, which is known only in one case,
% for some kinds of EPS files, as the ones produce by dump of portions of
% screens (see \ref{sec:GraphicFiles}).
% 
% \begin{Description}{2cm}
%   \setcounter{footnote}{1}
%   \item [\texttt{fillloopaddx}\footnotemark\ (integer)\hfill :] number of
%   times the pattern is added on left and right positions (\emph{Default:~0}).
%   \setcounter{footnote}{1}
%   \item [\texttt{fillloopaddy}\footnotemark\ (integer)\hfill :] number of
%   times the pattern is added on top and bottom positions (\emph{Default:~0}).
%   \setcounter{footnote}{1}
%   \item [\texttt{fillloopadd}\footnotemark\ (integer)\hfill :] number of
%   times the pattern is added on left, right, top and bottom positions
%   (\emph{Default:~0}).
% \end{Description}
% 
%   These parameters are only useful in special circonstances, as for complex
% patterns when the size of the rectangular box used to tile the area doesn't 
% correspond to the pattern itself (see an example in Figure~\ref{fig:Sheeps})
% and also sometimes when the size of the pattern is not a divisor of the size
% of the area to fill and that the number of loop repeats is not properly
% computed, which can occur.
% 
%   They are implemented only for the \emph{tiling} mode.
% 
% \begin{Description}{2cm}
%   \setcounter{footnote}{1}
%   \item [\texttt{PstDebug}\footnotemark\ (integer, 0 or 1)\hfill :] to
%   require to see the exact tiling done, without clipping (\emph{Default:~0}).
% \end{Description}
% 
%   It's mainly useful for debugging or to understand better how the tilings
% are done. It is implemented only for the \emph{tiling} mode.
% 
% \begin{LTXexample}
% \psset{unit=0.3,PstDebug=1}
% \psboxfill{\Square}
% \psset{linewidth=1mm}
% \Tiling{(2,2)}\\[5mm]
% \Tiling[fillcyclex=2]{(2,2)}\\[1cm]
% \Tiling[fillmove=0.5]{(2,2)}
% \end{LTXexample}
% 
% \vspace{3cm}
% \section{Examples}
% 
%   In fact this unique \cs{psboxfill} macro allow a lot a variations and
% different usages. We will try here to demonstrate this.
% 
% \subsection{Kind of tiles}
% \label{sec:KindTiles}
% 
%   Of course, we can access to all the power of PSTricks macros to define the
% \emph{tiles} (\emph{patterns}) used. So, we can define complicated ones.
% 
%   Here we give four other Archimedian tilings (those built with only some
% regular polygons) among the twelve existing, first discovered completely by
% Johanes \textsc{Kepler} at the beginning of 17th century \cite{GS87}, the two
% other \emph{regular} ones with the tiling by squares, formed by a unique
% regular polygon, and two other formed by two different regular polygons.
% 
% \begin{LTXexample}[pos=t]
%   \newcommand{\Triangle}{%
%     \begin{pspicture}(1,1)
%       \pstriangle[dimen=middle](0.5,0)(1,1)
%     \end{pspicture}}
%   \newcommand{\Hexagon}{
% ^^A sin(60)=0.866
%     \begin{pspicture}(0.866,0.75)
%       \SpecialCoor
% ^^A  Hexagon  
%       \pspolygon[dimen=middle]%
%         (0.5;30)(0.5;90)(0.5;150)(0.5;210)(0.5;270)(0.5;330)
%     \end{pspicture}}
% 
%   \psset{unit=0.5}
%   \psboxfill{\Triangle}
%   \Tiling{(4,4)}\hfill
% ^^A The two other regular tilings
%   \Tiling[fillcyclex=2]{(4,4)}\hfill
%   \psboxfill{\Hexagon}
%   \Tiling[fillcyclex=2,fillloopaddy=1]{(5,5)}
% \end{LTXexample}
% 
% \begin{LTXexample}[pos=t]
%   \newcommand{\ArchimedianA}{%
%      ^^A Archimedian tiling 3^2.4.3.4
%     \psset{dimen=middle}
%      ^^A sin(60)=0.866
%     \begin{pspicture}(1.866,1.866)
%       \psframe(1,1)
%       \psline(1,0)(1.866,0.5)(1,1)(0.5,1.866)(0,1)(-0.866,0.5)
%       \psline(0,0)(0.5,-0.866)
%     \end{pspicture}}
%   \newcommand{\ArchimedianB}{%
%      ^^A Archimedian tiling 4.8^2
%     \psset{dimen=middle,unit=1.5}
%      ^^A sin(22.5)=0.3827 ; cos(22.5)=0.9239
%     \begin{pspicture}(1.3066,0.6533)
%       \SpecialCoor
%      ^^A Octogon
%       \pspolygon(0.5;22.5)(0.5;67.5)(0.5;112.5)(0.5;157.5)
%                 (0.5;202.5)(0.5;247.5)(0.5;292.5)(0.5;337.5)
%     \end{pspicture}}
% 
%   \psset{unit=0.5}
%   \psboxfill{\ArchimedianA}
%   \Tiling[fillmove=0.5]{(7,7)}\hfill
%   \psboxfill{\ArchimedianB}
%   \Tiling[fillcyclex=2,fillloopaddy=1]{(7,7)}
% \end{LTXexample}
% 
%   \setcounter{footnote}{3}
%   We can of course tile an area arbitrarily defined. And with the
% \texttt{addfillstyle} parameter\footnote{Introduced in PSTricks 97.}, we can
% easily mix the \texttt{boxfill} style with another one.
% 
% \begin{LTXexample}[width=6cm]
%   \psset{unit=0.5,dimen=middle}
%   \psboxfill{%
%     \begin{pspicture}(1,1)
%       \psframe(1,1)
%       \pscircle(0.5,0.5){0.25}
%     \end{pspicture}}
%   \begin{pspicture}(4,6)
%     \pspolygon[fillstyle=boxfill,fillsep=0.25](0,1)(1,4)(4,6)(4,0)(2,1)
%   \end{pspicture}\hspace{1em}
%   \begin{pspicture}(4,4)
%%     \pscircle[linestyle=none,fillstyle=solid,fillcolor=yellow,fillsep=0.5,
%%               addfillstyle=boxfill](2,2){2}
%   \end{pspicture}
% \end{LTXexample}
%
%   Various effects can be obtained, sometimes complicated ones very easily, as
% in this example reproduced from one shown by Slavik \textsc{Jablan} in the
% field of \emph{OpTiles}, inspired by the \emph{Op-art}:
% 
% \begin{LTXexample}[pos=t]
% \newcommand{\ProtoTile}{%
%  \begin{pspicture}(1,1)%%% 1/12=0.08333
%   \psset{linestyle=none,linewidth=0,
%     hatchwidth=0.08333\psunit,hatchsep=0.08333\psunit}
%   \psframe[fillstyle=solid,fillcolor=black,addfillstyle=hlines,hatchcolor=white](1,1)
%   \pswedge[fillstyle=solid,fillcolor=white,addfillstyle=hlines]{1}{0}{90}
%  \end{pspicture}}
% \newcommand{\BasicTile}{%
%  \begin{pspicture}(2,1)
%    \rput[lb](0,0){\ProtoTile}\rput[lb](1,0){\psrotateleft{\ProtoTile}}
%  \end{pspicture}}
% \ProtoTile\hfill\BasicTile\hfill
% \psboxfill{\BasicTile}
% \Tiling[fillcyclex=2]{(4,4)}
% \end{LTXexample}
% 
%   It is also directly possible to surimpose several different tilings. Here is
% the splendid visual proof of the \textsc{Pytha\-gore} theorem done by the arab
% mathematician \textsc{Annairizi} around the year 900, given by superposition
% of two tilings by squares of different sizes.
% 
% \begin{LTXexample}[pos=t]
% \psset{unit=1.5,dimen=middle}
% \begin{pspicture*}(3,3)
%   \psboxfill{\begin{pspicture}(1,1)
%     \psframe(1,1)\end{pspicture}}
%   \psframe[fillstyle=boxfill](3,3)
%   \psboxfill{\begin{pspicture}(1,1)
%     \rput{-37}{\psframe[linecolor=red](0.8,0.8)}
%   \end{pspicture}}
%   \psframe[fillstyle=boxfill](3,4)
%   \pspolygon[fillstyle=hlines,hatchangle=90](1,2)(1.64,1.53)(2,2)
% \end{pspicture*}
% \end{LTXexample}
% 
%   In a same way, it is possible to build tilings based on figurative patterns,
% in the style of the famous \textsc{Escher} ones. Following an example of
% Andr\'e \textsc{Deledicq} \cite{Deledicq97}, we first show a simple tiling of
% the \emph{p1} category (according to the international classification of the
% 17~symmetry groups of the plane first discovered by the russian
% crystalographer Jevgraf \textsc{Fedorov} at the end of the 19th century):
% 
% \begin{LTXexample}[pos=t]
%  \newcommand{\SheepHead}[1]{%
%    \begin{pspicture}(3,1.5)
%      \pscustom[liftpen=2,fillstyle=solid,fillcolor=#1]{%
%        \pscurve(0.5,-0.2)(0.6,0.5)(0.2,1.3)(0,1.5)(0,1.5)
%          (0.4,1.3)(0.8,1.5)(2.2,1.9)(3,1.5)(3,1.5)(3.2,1.3)
%          (3.6,0.5)(3.4,-0.3)(3,0)(2.2,0.4)(0.5,-0.2)}
%      \pscircle*(2.65,1.25){0.12\psunit} % Eye
%      \psccurve*(3.5,0.3)(3.35,0.45)(3.5,0.6)(3.6,0.4)% Muzzle
%     ^^A   % Mouth
%       \pscurve(3,0.35)(3.3,0.1)(3.6,0.05)
%     ^^A   % Ear
%       \pscurve(2.3,1.3)(2.1,1.5)(2.15,1.7)\pscurve(2.1,1.7)(2.35,1.6)(2.45,1.4)
%   \end{pspicture}}
%  \psboxfill{\psset{unit=0.5}\SheepHead{yellow}\SheepHead{cyan}}
%  \Tiling[fillcyclex=2,fillloopadd=1]{(10,5)}
% \end{LTXexample}
% \label{fig:Sheeps}
% 
%   Now a tiling of the \emph{pg} category (the code for the kangaroo itself is
% too long to be shown here, but has no difficulties ; the kangaroo is reproduce
% from an original picture from Raoul \textsc{Raba} and here is a translation in
% PSTricks from the one drawn by Emmanuel \textsc{Chailloux} and Guy
% \textsc{Cousineau} for their MLgraph system \cite{MLgraphTSI}):
% 
% \begin{LTXexample}[pos=t]
% \psboxfill{\psset{unit=0.4}
%   \Kangaroo{yellow}\Kangaroo{red}\Kangaroo{cyan}\Kangaroo{green}%
%   \psscalebox{-1 1}{%
%     \rput(1.235,4.8){\Kangaroo{green}\Kangaroo{cyan}\Kangaroo{red}\Kangaroo{yellow}}}}
%   \Tiling[fillloopadd=1]{(10,6)}
% \end{LTXexample}
% 
%   And here a \textsc{Wang} tiling \cite{Wang65}, \cite[chapter
% 11]{GS87}, based on very simple tiles of the form of a square and composed
% of four colored triangles. Such tilings are built with only a matching color
% constraint. Despite of it simplicity, it is an important kind of tilings, as
% \textsc{Wang} and others used them to study the special class of
% \emph{aperiodic} tilings, and also because it was shown that surprisingly this 
% tiling is similar to a \textsc{Turing} machine.
% 
% \begin{LTXexample}[pos=t]
%   \newcommand{\WangTile}[4]{%
%     \begin{pspicture}(1,1)
%       \pspolygon*[linecolor=#1](0,0)(0,1)(0.5,0.5)
%       \pspolygon*[linecolor=#2](0,1)(1,1)(0.5,0.5)
%       \pspolygon*[linecolor=#3](1,1)(1,0)(0.5,0.5)
%       \pspolygon*[linecolor=#4](1,0)(0,0)(0.5,0.5)
%     \end{pspicture}}
%   \newcommand{\WangTileA}{\WangTile{cyan}{yellow}{cyan}{cyan}}
%   \newcommand{\WangTileB}{\WangTile{yellow}{cyan}{cyan}{red}}
%   \newcommand{\WangTileC}{\WangTile{cyan}{red}{yellow}{yellow}}
%   \newcommand{\WangTiles}[1][]{%
%     \begin{pspicture}(3,3) \psset{ref=lb}
%       \rput(0,2){\WangTileB}  \rput(1,2){\WangTileA}%
%       \rput(2,2){\WangTileC}  \rput(0,1){\WangTileC}%
%       \rput(1,1){\WangTileB}  \rput(2,1){\WangTileA}
%       \rput(0,0){\WangTileA}  \rput(1,0){\WangTileC}%
%       \rput(2,0){\WangTileB}
%       #1
%     \end{pspicture}}
%   \WangTileA\hfill\WangTileB\hfill\WangTileC\hfill
%   \WangTiles[{\psgrid[subgriddiv=0,gridlabels=0](3,3)}]\hfill
%   \psset{unit=0.4} \psboxfill{\WangTiles} \Tiling{(12,12)}
% \end{LTXexample}
% 
% \subsection{External graphic files}
% \label{sec:GraphicFiles}
% 
%   We can also fill an arbitrary area with an external image. We have only, 
% as usual, to matter of the \emph{BoundingBox} definition if there is no one
% provided or if it is not the accurate one, as for the well known
% \texttt{tiger} picture part of the \texttt{ghostscript} distribution.
% 
% \begin{LTXexample}[pos=t]
%   \psboxfill{%% Strangely require x1=x2...
%     \begin{pspicture}(0,1)(0,4.1)
%       \includegraphics[bb=17 176 560 74,width=3cm]{tiger}
%     \end{pspicture}}
%   \Tiling{(6,6.2)}
% \end{LTXexample}
% 
%   Nevertheless, there are some special files for which the \emph{automatic}
% mode doesn't work, specially for some files obtained by a screen dump, as in
% the next example, where a picture was reduced before it conversion in the
% \emph{Encapsulated PostScript} format by a screen dump utility. In this case,
% usage of the \emph{manual} mode is the only alternative, at the price of the
% real multiple inclusion of the EPS file. We must take care to specify the
% correct \texttt{fillsize} parameter, because otherwise the default values are
% large and will load the file many times, perhaps just really using few
% occurrences as the other ones would be clipped...
% 
% \begin{LTXexample}[pos=t]
%   \psboxfill{\includegraphics{flowers}}
%   \begin{pspicture}(8,4)
%     \psellipse[fillstyle=boxfill,fillsize={(8,4)}](4,2)(4,2)
%   \end{pspicture}
% \end{LTXexample}
% 
% \subsection{Tiling of characters}
% 
%   We can also use the \cs{psboxfill} macro to fill the interior of characters
% for special effects like these ones:
% 
% \begin{LTXexample}[pos=t]
%   \DeclareFixedFont{\bigsf}{T1}{phv}{b}{n}{4.5cm}
%   \DeclareFixedFont{\smallrm}{T1}{ptm}{m}{n}{3mm}
%   \psboxfill{\smallrm Since 182 days...}
%   \begin{pspicture*}(8,4)
%     \centerline{%
%       \pscharpath[fillstyle=gradient,gradangle=-45,
%                   gradmidpoint=0.5,addfillstyle=boxfill,
%                   fillangle=45,fillsep=0.7mm]
%                  {\rput[b](0,0.1){\bigsf 2000}}}
%   \end{pspicture*}
% \end{LTXexample}
% 
% \begin{LTXexample}[pos=t]
%   \DeclareFixedFont{\mediumrm}{T1}{ptm}{m}{n}{2cm}
%   \psboxfill{%
%     \psset{unit=0.1,linewidth=0.2pt}
%     \Kangaroo{PeachPuff}\Kangaroo{PaleGreen}%
%       \Kangaroo{LightBlue}\Kangaroo{LemonChiffon}%
%     \psscalebox{-1 1}{%
%       \rput(1.235,4.8){%
%         \Kangaroo{LemonChiffon}\Kangaroo{LightBlue}%
%           \Kangaroo{PaleGreen}\Kangaroo{PeachPuff}}}}
% ^^A   % A kangaroo of kangaroos...
%   \begin{pspicture}(8,2)
%     \pscharpath[linestyle=none,fillstyle=boxfill,fillloopadd=1]
%                {\rput[b](4,0){\mediumrm Kangaroo}}
%   \end{pspicture}
% \end{LTXexample}
% 
% \subsection{Other kinds of usage}
% 
%   Other kinds of usage can be imagined. For instance, we can use tilings in a
% sort of degenerated way to draw some special lines made by a unique or
% multiple repeating patterns. But it can be only a special dashed line, as here
% with three different dashes:
% 
% \begin{LTXexample}[pos=t]
%   \newcommand{\Dashes}{%
%     \psset{dimen=middle}
%     \begin{pspicture}(0,-0.5\pslinewidth)(1,0.5\pslinewidth)
%       \rput(0,0){\psline(0.4,0)}%
%         \rput(0.5,0){\psline(0.2,0)}%
%         \rput(0.8,0){\psline(0.1,0)}
%     \end{pspicture}}
% 
%   \newcommand{\SpecialDashedLine}[3]{%
%     \psboxfill{#3}
%     \Tiling[linestyle=none]
%            {(#1,-0.5\pslinewidth)(#2,0.5\pslinewidth)}}
% 
%   \SpecialDashedLine{0}{7}{\Dashes}
% 
%   \psset{unit=0.5,linewidth=1mm,linecolor=red}
%   \SpecialDashedLine{0}{10}{\Dashes}
% \end{LTXexample}
% 
%   It allow also to use special patterns in business graphics, as in the
% following example generated by \texttt{PstChart}\footnote{A personal
% development to draw business charts with PSTricks, not distributed.}.
% 
% \vspace{3mm}
% \begin{figure}[!ht]
% \centering
% \psset{unit=0.75}
% ^^A % Generated by pstchart.sh version 0.21 (11/28/97)
% {\psset{dimen=middle}
% \psset{xunit=2,yunit=0.005}
% \begin{pspicture}(-0.6,-200)(6.6,2300)
% ^^A   % Title
%   \rput(3,2200){\shortstack{Fantaisist repartition of kangaroos\\
%                             in the world (in thousands)}}
% ^^A   % Frame background
%   \psframe[fillstyle=solid,fillcolor=LemonChiffon](0,0)(6,2000)
% ^^A   % Graduations
%   \multido{\n=0+500}{5}{\rput[r](-0.12,\n){\psscalebox{0.8}{\n}}}
% ^^A   % Minor ticks
%   \multips(0,100)(0,100){19}{\psline[unit=4.8pt](1,0)}
%   \multips(6,100)(0,100){19}{\psline[unit=4.8pt](-1,0)}
% ^^A   % Major ticks
%   \multips(0,500)(0,500){3}{\psline[unit=9.6pt](1,0)}
%   \multips(6,500)(0,500){3}{\psline[unit=9.6pt](-1,0)}
% ^^A   % Lines from major ticks marks
%   \multips(0,500)(0,500){3}{\psline[linestyle=dotted,linewidth=0.6pt](6,0)}
% ^^A   % Drawing for the data
%   \psboxfill{\psset{unit=0.78\psxunit}\KangarooPstChart{red}}
%   \psframe[linestyle=none,fillstyle=boxfill,fillloopaddy=1](0.61,0)(1.39,1800)
%   \psboxfill{\psset{unit=0.78\psxunit}\KangarooPstChart{yellow}}
%   \psframe[linestyle=none,fillstyle=boxfill,fillloopaddy=1](1.61,0)(2.39,800)
%   \psboxfill{\psset{unit=0.78\psxunit}\KangarooPstChart{cyan}}
%   \psframe[linestyle=none,fillstyle=boxfill,fillloopaddy=1](2.61,0)(3.39,550)
%   \psboxfill{\psset{unit=0.78\psxunit}\KangarooPstChart{magenta}}
%   \psframe[linestyle=none,fillstyle=boxfill,fillloopaddy=1](3.61,0)(4.39,500)
%   \psboxfill{\psset{unit=0.78\psxunit}\KangarooPstChart{green}}
%   \psframe[linestyle=none,fillstyle=boxfill,fillloopaddy=1](4.61,0)(5.39,200)
% ^^A   % Bottom labels
%   \uput{0.2}[270]{0}(1,0){\psscalebox{0.7}{Oceania}}
%   \uput{0.2}[270]{0}(2,0){\psscalebox{0.7}{Africa}}
%   \uput{0.2}[270]{0}(3,0){\psscalebox{0.7}{Asia}}
%   \uput{0.2}[270]{0}(4,0){\psscalebox{0.7}{America}}
%   \uput{0.2}[270]{0}(5,0){\psscalebox{0.7}{Europe}}
% ^^A   % Frame box around the chart
%   \psframe[linestyle=solid](0,0)(6,2000)
% \end{pspicture}}
%   \caption{Bar chart generated by PstChart, with bars filled by patterns}
%   \label{fig:PstChart}
% \end{figure}
% 
% \section{``Dynamic'' tilings}
% 
%   In some cases, tilings used non \emph{static} tiles, that is to say that the 
% \emph{prototile(s)}, even if unique, can have several forms, by instance
% specified by different colors or rotations, not fixed before generation or
% varying each time.
% 
% \subsection{Lewthwaite-Pickover-Truchet tiling}
% 
%   We give here for example the so-called \emph{Truchet} tiling, which much be
% in fact better called \emph{Lewthwaite-Pick\-over-Truchet (LPT)} tiling%
% \footnote{For description of the context, history and references about
% S\'ebastien \textsc{Truchet} and this tiling, see \cite{EsperetGirou98}.}.
% 
%   The unique prototile is only a square with two opposite circle arcs.
% This tile has obviously two positions, if we rotate it from 90 degrees (see
% the two tiles on the next figure). A \emph{LPT tiling} is a tiling with
% randomly oriented LPT tiles. We can see that even if it is very simple in it
% principle, it draw sophisticated curves with strange properties.
% 
%   Nevertheless, in the straightforward way \FillPackage{} does not work,
% because the \cs{psboxfill} macro store the content of the tile used in a
% \TeX{} box, which is static. So the calling to the random function is done
% only one time, which explain that only one rotation of the tile is used for
% all the tiling. It's only the one of the two rotations which could differ from
% one drawing to the next one...
% 
% ^^A % Truchet (Lewthwaite-Pickover-Truchet) tiling
% ^^A % --------------------------------------------
% 
% \begin{LTXexample}[pos=t]
% ^^A   % LPT prototile
%   \newcommand{\ProtoTileLPT}{%
%     \psset{dimen=middle}
%     \begin{pspicture}(1,1)
%       \psframe(1,1)
%       \psarc(0,0){0.5}{0}{90}
%       \psarc(1,1){0.5}{-180}{-90}
%     \end{pspicture}}
% 
% ^^A   % LPT tile
%   \newcount\Boolean
%   \newcommand{\BasicTileLPT}{%
% ^^A     % From random.tex by Donald Arseneau
%     \setrannum{\Boolean}{0}{1}%
%     \ifnum\Boolean=0
%       \ProtoTileLPT%
%     \else
%       \psrotateleft{\ProtoTileLPT}%
%     \fi}
% 
%   \ProtoTileLPT\hfill\psrotateleft{\ProtoTileLPT}\hfill
%   \psset{unit=0.5}
%   \psboxfill{\BasicTileLPT}
%   \Tiling{(5,5)}
% \end{LTXexample}
% 
%   But, for simple cases, there is a solution to this problem using a mixture
% of PSTricks and PostScript programming. Here the PSTricks
% construction \verb+\pscustom{\code{...}}+ allow to insert PostScript code
% inside the \LaTeX{} + PSTricks one.
% 
%   Programmation is less straightforward, but it has also the advantage to be
% notably faster, as all the tilings operations are done in PostScript, and
% mainly to not be limited by \TeX{} memory (the \TeX{} + PSTricks solution
% I wrote in 1995 for the colored problem was limited to small sizes for this
% reason). Just note also that \cs{pslbrace} and \cs{psrbrace} are two
% PSTricks macros to define and be able to insert the \verb+{+ and \verb+}+
% characters.
% 
% \begin{LTXexample}[pos=t]
% ^^A   % LPT prototile
%   \newcommand{\ProtoTileLPT}{%
%     \psset{dimen=middle}
%     \psframe(1,1)
%     \psarc(0,0){0.5}{0}{90}
%     \psarc(1,1){0.5}{-180}{-90}}
% 
% ^^A   % Counter to change the random seed
%   \newcount\InitCounter
% ^^A   % LPT tile
%   \newcommand{\BasicTileLPT}{%
%     \InitCounter=\the\time
%     \pscustom{\code{%
%       rand \the\InitCounter\space sub 2 mod 0 eq \pslbrace}}
%     \begin{pspicture}(1,1)
%       \ProtoTileLPT
%     \end{pspicture}%
%     \pscustom{\code{\psrbrace \pslbrace}}
%     \psrotateleft{\ProtoTileLPT}%
%     \pscustom{\code{\psrbrace ifelse}}}
% 
%   \psset{unit=0.4,linewidth=0.4pt}
%   \psboxfill{\BasicTileLPT}
%   \Tiling{(15,15)}
% \end{LTXexample}
% 
%   Using the very surprising fact (see \cite{EsperetGirou98}) that
% coloration of these tiles do not depend of their neighbors (even if it is
% difficult to believe as the opposite seems obvious!) but only of the parity of
% the value of row and column positions, we can directly program in the same way
% a colored version of the LPT tiling.
% 
% \setcounter{footnote}{1}
%   We have also introduce in the \FillPackage{} code for \emph{tiling} mode two
% new accessible Post\-Script variables, \texttt{row} and
% \texttt{column}\footnotemark, which can be useful in some circonstances, like
% this one.
% 
% \begin{LTXexample}[pos=t]
% ^^A   % LPT prototile
%   \newcommand{\ProtoTileLPT}[2]{%
%     \psset{dimen=middle,linestyle=none,fillstyle=solid}
%     \psframe[fillcolor=#1](1,1)
%     \psset{fillcolor=#2}
%     \pswedge(0,0){0.5}{0}{90} \pswedge(1,1){0.5}{-180}{-90}}
% ^^A   % Counter to change the random seed
%   \newcount\InitCounter
% ^^A   % LPT tile
%   \newcommand{\BasicTileLPT}[2]{%
%     \InitCounter=\the\time
%     \pscustom{\code{%
%       rand \the\InitCounter\space sub 2 mod 0 eq \pslbrace
%       row column add 2 mod 0 eq \pslbrace}}
%     \begin{pspicture}(1,1)\ProtoTileLPT{#1}{#2}\end{pspicture}%
%     \pscustom{\code{\psrbrace \pslbrace}}
%     \ProtoTileLPT{#2}{#1}%
%     \pscustom{\code{%
%       \psrbrace ifelse \psrbrace \pslbrace row column add 2 mod 0 eq \pslbrace}}
%     \psrotateleft{\ProtoTileLPT{#2}{#1}}\pscustom{\code{\psrbrace \pslbrace}}
%     \psrotateleft{\ProtoTileLPT{#1}{#2}}\pscustom{\code{\psrbrace ifelse \psrbrace ifelse}}}
%   \psboxfill{\BasicTileLPT{red}{yellow}}
%   \Tiling{(4,4)}\hfill
%   \psset{unit=0.4}\psboxfill{\BasicTileLPT{blue}{cyan}}
%   \Tiling{(15,15)}
% \end{LTXexample}
% 
%   Another classic example is to generate coordinates and numerotation for a
% grid. Of course, it is possible to do it directly in PSTricks using nested
% \cs{multido} commands. It would be clearly easy to program, but, nevertheless, 
% for users who have a little knowledge of PostScript programming, this offer
% an alternative which is useful for large cases, because on this way it will
% be notably faster and less computer ressources consuming.
% 
%   Remember here that the tiling is drawn from left to right, and top to
% bottom, and note that the PostScript variable \texttt{x2} give the total
% number of columns.
% 
% \begin{LTXexample}[pos=t]
% ^^A   % \Escape will be the \ character
%   {\catcode`\!=0\catcode`\\=11!gdef!Escape{\}}
%   \newcommand{\ProtoTile}{%
%     \Square\pscustom{%
%       \moveto(-0.9,0.75) % In PSTricks units
%       \code{ /Times-Italic findfont 8 scalefont setfont
%         (\Escape() show row 3 string cvs show (,) show 
%         column 3 string cvs show (\Escape)) show}
%       \moveto(-0.5,0.25) % In PSTricks units
%       \code{ /Times-Bold findfont 18 scalefont setfont
%         1 0 0 setrgbcolor % Red color
%         /center {dup stringwidth pop 2 div neg 0 rmoveto} def
%         row 1 sub x2 mul column add 3 string cvs center show}}}
%   \psboxfill{\ProtoTile}
%   \Tiling{(6,4)}
% \end{LTXexample}
% 
% \subsection{A complete example: the Poisson equation}
% 
%   To finish, we will show a complete real example, a drawing to explain the
% method used to solve the \textsc{Poisson} equation by a domain
% decomposition method, adapted to distributed memory computers. The
% objective is to show the communications required between processes and the
% position of the data to exchange. This code also show some useful and powerful
% technics for PSTricks programming (look specially at the way some higher level
% macros are defined, and how the same object is used to draw the four
% neighbors).
%
%\psset{unit=1cm}
%\newcommand{\Pattern}[1]{%
%   \begin{pspicture}(-0.25,-0.25)(0.25,0.25)\rput{*0}{\psdot[dotstyle=#1]}
%   \end{pspicture}}
%\newcommand{\West}{\Pattern{o}}   \newcommand{\South}{\Pattern{x}}
%\newcommand{\Central}{\Pattern{+}}\newcommand{\North}{\Pattern{square}}
%\newcommand{\East}{\Pattern{triangle}}
%\newcommand{\Cross}{%
%  \pspolygon[unit=0.5,linewidth=0.2,linecolor=red](0,0)(0,1)(1,1)(1,2)(2,2)(2,1)%
%              (3,1)(3,0)(2,0)(2,-1)(1,-1)(1,0)}
%\newcommand{\StylePosition}[1]{\LARGE\textcolor{red}{\textbf{#1}}}
%\newcommand{\SubDomain}[4]{%
%    \psboxfill{#4}\begin{psclip}{\psframe[linestyle=none]#1}%
%      \psframe[linestyle=#3](5,5)\psframe[fillstyle=boxfill]#2%
%    \end{psclip}}
%\newcommand{\SendArea}[1]{\psframe[fillstyle=solid,fillcolor=cyan]#1}
%\newcommand{\ReceiveData}[2]{%
%  \psboxfill{#2}\psframe[fillstyle=solid,fillcolor=yellow,addfillstyle=boxfill]#1}%
%\newcommand{\Neighbor}[2]{%
%    \begin{pspicture}(5,5)
%      \rput{*0}(2.5,2.5){\StylePosition{#1}}
%      \ReceiveData{(0.5,0)(4.5,0.5)}{\Central}\SendArea{(0.5,0.5)(4.5,1)}%
%      \SubDomain{(5,2)}{(0.5,0.5)(4.5,3)}{dashed}{#2}%
%      \pcarc[arcangle=45,arrows=->](0.5,-1.25)(0.5,0.25)%
%      \pcarc[arcangle=45,arrows=->,linestyle=dotted,dotsep=2pt](4.5,0.75)(4.5,-0.75)%
%    \end{pspicture}}%
%  \psset{dimen=middle,dotscale=2,fillloopadd=2}
%\begin{pspicture}(-5.7,-5.7)(5.7,5.7)
%  \rput(0,0){%
%      \begin{pspicture}(5,5)
%        \ReceiveData{(0,0.5)(0.5,4.5)}{\West} \ReceiveData{(4.5,0.5)(5,4.5)}{\East}
%        \ReceiveData{(0.5,4.5)(4.5,5)}{\North}\ReceiveData{(0.5,0)(4.5,0.5)}{\South}
%        \SendArea{(0.5,0.5)(1,4.5)}\SendArea{(4,0.5)(4.5,4.5)}
%        \SendArea{(0.5,0.5)(4.5,1)}\SendArea{(0.5,4)(4.5,4.5)}
%        \SubDomain{(5,5)}{(0.5,0.5)(4.5,4.5)}{solid}{\Central}
%        \psline(1,0.5)(1,4.5)\psline(4,0.5)(4,4.5)%
%        \rput(1.5,4){\Cross}\rput(2,2){\Cross}%
%      \end{pspicture}}%
%  \rput(0,5.5){\Neighbor{N}{\North}}\rput{-90}(5.5,0){\Neighbor{E}{\East}}%
%  \rput{90}(-5.5,0){\Neighbor{W}{\West}}\rput{180}(0,-5.5){\Neighbor{S}{\South}}%
%\end{pspicture}
%
% \begin{lstlisting}
%   \newcommand{\Pattern}[1]{%
%     \begin{pspicture}(-0.25,-0.25)(0.25,0.25)\rput{*0}{\psdot[dotstyle=#1]}
%     \end{pspicture}}
%   \newcommand{\West}{\Pattern{o}}   \newcommand{\South}{\Pattern{x}}
%   \newcommand{\Central}{\Pattern{+}}\newcommand{\North}{\Pattern{square}}
%   \newcommand{\East}{\Pattern{triangle}}
%   \newcommand{\Cross}{%
%     \pspolygon[unit=0.5,linewidth=0.2,linecolor=red](0,0)(0,1)(1,1)(1,2)(2,2)(2,1)
%               (3,1)(3,0)(2,0)(2,-1)(1,-1)(1,0)}
%   \newcommand{\StylePosition}[1]{\LARGE\textcolor{red}{\textbf{#1}}}
%   \newcommand{\SubDomain}[4]{%
%     \psboxfill{#4}
%     \begin{psclip}{\psframe[linestyle=none]#1}
%       \psframe[linestyle=#3](5,5)\psframe[fillstyle=boxfill]#2
%     \end{psclip}}
%   \newcommand{\SendArea}[1]{\psframe[fillstyle=solid,fillcolor=cyan]#1}
%   \newcommand{\ReceiveData}[2]{%
%     \psboxfill{#2}
%     \psframe[fillstyle=solid,fillcolor=yellow,addfillstyle=boxfill]#1}
%   \newcommand{\Neighbor}[2]{%
%     \begin{pspicture}(5,5)
%       \rput{*0}(2.5,2.5){\StylePosition{#1}}
%       \ReceiveData{(0.5,0)(4.5,0.5)}{\Central}\SendArea{(0.5,0.5)(4.5,1)}
%       \SubDomain{(5,2)}{(0.5,0.5)(4.5,3)}{dashed}{#2}%
% ^^A       % Receive and send arrows
%       \pcarc[arcangle=45,arrows=->](0.5,-1.25)(0.5,0.25)
%       \pcarc[arcangle=45,arrows=->,linestyle=dotted,dotsep=2pt](4.5,0.75)(4.5,-0.75)
%     \end{pspicture}}
%   \psset{dimen=middle,dotscale=2,fillloopadd=2}
%   \begin{pspicture}(-5.7,-5.7)(5.7,5.7)
% ^^A     % Central domain
%     \rput(0,0){%
%       \begin{pspicture}(5,5)
% ^^A         % Receive from West, East, North and South
%         \ReceiveData{(0,0.5)(0.5,4.5)}{\West} \ReceiveData{(4.5,0.5)(5,4.5)}{\East}
%         \ReceiveData{(0.5,4.5)(4.5,5)}{\North}\ReceiveData{(0.5,0)(4.5,0.5)}{\South}
% ^^A         % send area for West, East, North and South
%         \SendArea{(0.5,0.5)(1,4.5)} \SendArea{(4,0.5)(4.5,4.5)}
%         \SendArea{(0.5,0.5)(4.5,1)} \SendArea{(0.5,4)(4.5,4.5)}
% ^^A         % Central domain
%         \SubDomain{(5,5)}{(0.5,0.5)(4.5,4.5)}{solid}{\Central}
% ^^A         % Redraw overlapped linesY
%         \psline(1,0.5)(1,4.5)  \psline(4,0.5)(4,4.5)
% ^^A         % Two crossesY
%         \rput(1.5,4){\Cross}  \rput(2,2){\Cross}
%       \end{pspicture}}
% ^^A     % The four neighborsY
%     \rput(0,5.5){\Neighbor{N}{\North}}     \rput{-90}(5.5,0){\Neighbor{E}{\East}}
%     \rput{90}(-5.5,0){\Neighbor{W}{\West}} \rput{180}(0,-5.5){\Neighbor{S}{\South}}
%   \end{pspicture}
% \end{lstlisting}
%
%
%
% Bibliography
% \begin{thebibliography}{99}
% \bibitem{PostScript95} Adobe, Systems~Incorporated, \emph{PostScript Language
% Reference Manual}, Addison-Wesley, 2~edition, 1995.
%
% \bibitem{Bolek98} Piotr Bolek, \MP{} and patterns, \emph{\TUB}, Volume~19,
% Number~3, pages 276--283, September 1998, \CTANref{mpattern}.
%
% \bibitem{MLgraphTSI} Emmanuel Chailloux, Guy Cousineau and Asc\'ander
% Su\'arez, Programmation fonctionnelle de graphismes pour la production
% d'illustrations techniques, \emph{Technique et science informatique},
% Volume~15, Number~7, pages 977--1007, 1996 (in french).
%
% \bibitem{Deledicq97} Andr\'e Deledicq, \emph{Le monde des pavages}, ACL
% \'Editions, 1997 (in french).
%
% \bibitem{EsperetGirou98} Philippe Esperet and Denis Girou,
% Coloriage du pavage dit de Truchet, Cahiers GUTenberg, Number~31,
% pages 5--18, December~1998  (in french).
%
% \bibitem{Girou94} Denis Girou, Pr\'esentation de PSTricks, \emph{Cahiers
% GUTenberg}, Number~16, pages 21--70, February~1994 (in french).
%
% \bibitem{LGC97} Michel Goossens, Sebastian Rahtz and Frank Mittelbach,
% \emph{The \LaTeX{} Graphics Companion}, Addison-Wesley, 2005.
%
% \bibitem{GS87} Branko Gr\"unbaum and Geoffrey Shephard, \emph{Tilings and
% Patterns}, Freeman and Company, 1987.
%
% \bibitem{Hoenig97} Alan Hoenig, \emph{\TeX{} Unbound: \LaTeX{} \& \TeX{}
% Strategies, Fonts, Graphics, and More}, Oxford University Press, 1997.
%
% \bibitem{XYpic} Kristoffer~H. Rose and Ross Moore, \XYpic. Pattern and Tile
% extension, available from \CTAN, 1991-1998, \CTANref{xypic}.
%
% \bibitem{LAAN96} Kees van der Laan, Paradigms: Just a little bit of PostScript,
% \emph{MAPS}, Volume~17, pages 137--150, 1996.
%
% \bibitem{LAAN97} Kees van der Laan, Tiling in PostScript and \MF{} -- Escher's
% wink, \emph{MAPS}, Volume~19, Number~2, pages 39--67, 1997.
%
% \bibitem{vanZandt93} Timothy Van Zandt, PSTricks. PostScript macros for
% Generic \TeX, available from \CTAN, 1993, \CTANref{pstricks}.
%
% \bibitem{vanZandtGirou94} Timothy Van Zandt and Denis Girou, Inside PSTricks,
% \emph{\TUB}, Volume~15, Number~3, pages 239--246, September 1994.
%
%
% \bibitem{voss07} Herbert Vo\ss, PSTricks -- Graphics for \TeX\ and \LaTeX, DANTE/Lehmanns, 4th ed., 2007.
% \bibitem{Wang65} Hao Wang, Games, Logic and Computers, \emph{Scientific
% American}, pages 98--106, November 1965.
% \end{thebibliography}
%
%
% \StopEventually{}
%
% ^^A .................... End of the documentation part ....................
%
% \section{Driver file}
%
%   The next bit of code contains the documentation driver file for \TeX{},
% i.e., the file that will produce the documentation you are currently
% reading. It will be extracted from this file by the \texttt{docstrip}
% program.
%
%    \begin{macrocode}
%<*driver>
\documentclass{ltxdoc}
\GetFileInfo{pst-fill.dtx}
%
\usepackage[T1]{fontenc}
\usepackage{lmodern}               % For PDF
\usepackage{graphicx}              % `graphicx' LaTeX standard package
\usepackage{showexpl}
\usepackage{mflogo}                % For the MetaFont and MetaPost logos
\input{random.tex}                 % Random macros from Donald Arseneau
\usepackage{url}                   % URLs convenient typesetting
\usepackage{multido}               % General loop macro
\usepackage[dvipsnames]{pstricks}  % PSTricks with the `color' extension
\usepackage{pst-text}              % PSTricks package for character path
\usepackage{pst-grad}              % PSTricks package for gradient filling
\usepackage{pst-node}              % PSTricks package for nodes
\usepackage[tiling]{pst-fill}      % PSTricks package for filling/tiling
%
\AtBeginDocument{%
%  \OnlyDescription % comment out for implementation details
  \EnableCrossrefs
  \CodelineIndex
  \RecordChanges}
\AtEndDocument{%
  \PrintIndex
  \setcounter{IndexColumns}{1}
  \PrintChanges}
\hbadness=7000            % Over and under full box warnings
\hfuzz=3pt
\begin{document}
  \DocInput{pst-fill.dtx}
\end{document}
%</driver>
%    \end{macrocode}
%
% \section{\texttt{pst-fill} \LaTeX{} wrapper}
%
%    \begin{macrocode}
%<*latex-wrapper>
\RequirePackage{pstricks}
\ProvidesPackage{pst-fill}[2005/09/13 package wrapper for 
  pst-fill.tex (hv)]
\DeclareOption{tiling}{\def\PstTiling{true}}
\ProcessOptions\relax
\input{pst-fill.tex}
\ProvidesFile{pst-fill.tex}
  [\filedate\space v\fileversion\space `PST-fill' (tvz,dg)]
%</latex-wrapper>
%    \end{macrocode}
%
%
% \section{Pst-Fill Package{} code}
%
%    \begin{macrocode}
%<*pst-fill>
%    \end{macrocode}
%
% \subsection{Preamble}
%
%   Who we are.
%
%    \begin{macrocode}
\def\fileversion{1.01}
\def\filedate{2007/03/10}
\message{`PST-Fill' v\fileversion, \filedate\space (tvz,dg,hv)}
\csname PSTboxfillLoaded\endcsname
\let\PSTboxfillLoaded\endinput
%    \end{macrocode}
%
%   Require the main PSTricks package.
%
%    \begin{macrocode}
\ifx\PSTricksLoaded\endinput\else\input pstricks.tex\fi
%    \end{macrocode}
%
%   interface to the extended `\textsf{keyval}' package.
%
%    \begin{macrocode}
\ifx\PSTXKeyLoaded\endinput\else\input pst-xkey\fi
%
%    \end{macrocode}
%
%   Catcodes changes and defining the family name for xkeyval.
%
%    \begin{macrocode}
\edef\PstAtCode{\the\catcode`\@}\catcode`\@=11\relax

\pst@addfams{pst-fill}
%
%    \end{macrocode}
%
%
% \subsection{The size of the box}
% \begin{macro}{pst@@boxfillsize}
%    \begin{macrocode}
%
\def\pst@@boxfillsize#1(#2,#3)#4(#5,#6)#7(#8\@nil{%
  \begingroup
    \ifx\@empty#7\relax
      \pst@dima\z@
      \pst@dimb\z@
      \pssetxlength\pst@dimc{#2}%
      \pssetylength\pst@dimd{#3}%
    \else
      \pssetxlength\pst@dima{#2}%
      \pssetylength\pst@dimb{#3}%
      \pssetxlength\pst@dimc{#5}%
      \pssetylength\pst@dimd{#6}%
    \fi
    \xdef\pst@tempg{%
      \pst@dima=\number\pst@dima sp
      \pst@dimb=\number\pst@dimb sp
      \pst@dimc=\number\pst@dimc sp
      \pst@dimd=\number\pst@dimd sp }%
  \endgroup
  \let\psk@boxfillsize\pst@tempg}
%    \end{macrocode}
% \end{macro}
%

% \subsection{Definition of the parameters}
%
%    \begin{macrocode}
\define@key[psset]{pst-fill}{boxfillsize}{%
  \def\pst@tempg{#1}\def\pst@temph{auto}%
  \ifx\pst@tempg\pst@temph
    \let\psk@boxfillsize\relax
  \else
    \pst@@boxfillsize#1(\z@,\z@)\@empty(\z@,\z@)(\@nil
  \fi}
\psset{boxfillsize={(-15cm,-15cm)(15cm,15cm)}}
\define@key[psset]{pst-fill}{boxfillcolor}{\pst@getcolor{#1}\psboxfillcolor}
\psset{boxfillcolor=black}% hv
\define@key[psset]{pst-fill}{boxfillangle}{\pst@getangle{#1}\psk@boxfillangle}
\psset{boxfillangle=0}
\define@key[psset]{pst-fill}{fillsepx}{%
  \pst@getlength{#1}\psk@fillsepx}
\define@key[psset]{pst-fill}{fillsepy}{%
  \pst@getlength{#1}\psk@fillsepy}
\define@key[psset]{pst-fill}{fillsep}{%
  \pst@getlength{#1}\psk@fillsepx%
  \let\psk@fillsepy\psk@fillsepx}
\psset{fillsep=2pt}

\ifx\PstTiling\@undefined
  \define@key[psset]{pst-fill}{fillcycle}{\pst@getint{#1}\psk@fillcycle}
  \psset{fillcycle=0}
\else
  \define@key[psset]{pst-fill}{fillangle}{\pst@getangle{#1}\psk@boxfillangle}
  \define@key[psset]{pst-fill}{fillsize}{%
      \def\pst@tempg{#1}\def\pst@temph{auto}%
      \ifx\pst@tempg\pst@temph\let\psk@boxfillsize\relax
      \else\pst@@boxfillsize#1(\z@,\z@)\@empty(\z@,\z@)(\@nil\fi}
  \psset{fillsep=0,fillsize=auto}
  \define@key[psset]{pst-fill}{fillcyclex}{\pst@getint{#1}\psk@fillcyclex}
  \define@key[psset]{pst-fill}{fillcycley}{\pst@getint{#1}\psk@fillcycley}
  \define@key[psset]{pst-fill}{fillcycle}{%
    \pst@getint{#1}\psk@fillcyclex\let\psk@fillcycley\psk@fillcyclex}
  \psset{fillcycle=0}
  \define@key[psset]{pst-fill}{fillmovex}{\pst@getlength{#1}\psk@fillmovex}
  \define@key[psset]{pst-fill}{fillmovey}{\pst@getlength{#1}\psk@fillmovey}
  \define@key[psset]{pst-fill}{fillmove}{%
      \pst@getlength{#1}\psk@fillmovex\let\psk@fillmovey\psk@fillmovex}
  \psset{fillmove=0pt}
  \define@key[psset]{pst-fill}{fillloopaddx}{\pst@getint{#1}\psk@fillloopaddx}
  \define@key[psset]{pst-fill}{fillloopaddy}{\pst@getint{#1}\psk@fillloopaddy}
  \define@key[psset]{pst-fill}{fillloopadd}{%
    \pst@getint{#1}\psk@fillloopaddx\let\psk@fillloopaddy\psk@fillloopaddx}
  \psset{fillloopadd=0}
%    \end{macrocode}
%
%    \begin{macrocode}
% For debugging (to debug, set PstDebug=1)
% we now use the one from pstricks to prevent a clash with package
% pstricks                        2004-06-22
%%    \define@key[psset]{pst-fill}{PstDebug}{\pst@getint{#1}\psk@PstDebug}
    \psset{PstDebug=0}
\fi
% DG addition end
%    \end{macrocode}

% \subsection{Definition of the fill box}
% \begin{macro}{psboxfill}
%    \begin{macrocode}
\newbox\pst@fillbox
\def\psboxfill{\pst@killglue\pst@makebox\psboxfill@i}
\def\psboxfill@i{\setbox\pst@fillbox\box\pst@hbox\ignorespaces}
%    \end{macrocode}
% \end{macro}
% \subsection{The main macros}
%
% \begin{macro}{psfs@boxfill}
%    \begin{macrocode}
\def\psfs@boxfill{%
  \ifvoid\pst@fillbox
    \@pstrickserr{Fill box is empty. Use \string\psboxfill\space first.}\@ehpa
  \else
    \ifx\psk@boxfillsize\relax \pst@AutoBoxFill
    \else\pst@ManualBoxFill\fi
  \fi}
%    \end{macrocode}
% \end{macro}
%
% \begin{macro}{pst@ManualBoxFill}
%    \begin{macrocode}
\def\pst@ManualBoxFill{%
  \leavevmode
  \begingroup
    \pst@FlushCode
    \begin@psclip
    \pstVerb{clip}%
    \expandafter\pst@AddFillBox\psk@boxfillsize
    \end@psclip
  \endgroup}
%    \end{macrocode}
% \end{macro}
%
% \begin{macro}{pst@FlushCode}
%    \begin{macrocode}
\def\pst@FlushCode{%
  \pst@Verb{%
    /mtrxc CM def
    CP CP T
    \tx@STV
    \psk@origin
    \psk@swapaxes
    \pst@newpath
    \pst@code
    mtrxc setmatrix
    moveto
    0 setgray}%
  \gdef\pst@code{}}
%    \end{macrocode}
% \end{macro}
%
% \begin{macro}{pst@AddFillBox}
%    \begin{macrocode}
\def\pst@AddFillBox#1 #2 #3 #4 {%
  \begingroup
    \setbox\pst@fillbox=\vbox{%
      \hbox{\unhcopy\pst@fillbox\kern\psk@fillsepx\p@}%
      \vskip\psk@fillsepy\p@}%
    \psk@boxfillsize
    \pst@cnta=\pst@dimc
    \advance\pst@cnta-\pst@dima
    \divide\pst@cnta\wd\pst@fillbox
    \pst@cntb=\pst@dimd
    \advance\pst@cntb-\pst@dimb
    \pst@dimd=\ht\pst@fillbox
    \divide\pst@cntb\pst@dimd
    \def\pst@tempa{%
      \pst@tempg
      \copy\pst@fillbox
      \advance\pst@cntc\@ne
      \ifnum\pst@cntc<\pst@cntd\expandafter\pst@tempa\fi}%
    \let\pst@tempg\relax
    \pst@cntc-\tw@
    \pst@cntd\pst@cnta
    \setbox\pst@fillbox=\hbox to \z@{%
      \kern\pst@dima
      \kern-\wd\pst@fillbox
      \pst@tempa
      \hss}%
    \pst@cntd\pst@cntb
%% DG modification begin - Dec. 11, 1997 - Patch 2
    \ifx\PstTiling\@undefined
      \ifnum\psk@fillcycle=\z@\pst@ManualFillCycle\fi
    \else
      \ifnum\psk@fillcyclex=\z@\pst@ManualFillCycle\fi
    \fi
%% DG modification end
    \global\setbox\pst@boxg=\vbox to\z@{%
      \offinterlineskip
      \vss
      \pst@tempa
      \vskip\pst@dimb}%
  \endgroup
  \setbox\pst@fillbox\box\pst@boxg
  \pst@rotate\psk@boxfillangle\pst@fillbox
  \box\pst@fillbox}
%    \end{macrocode}
% \end{macro}
%
% \begin{macro}{pst@ManualFillCycle}
%    \begin{macrocode}
\def\pst@ManualFillCycle{%
  \ifx\PstTiling\@undefined
    \pst@cntg=\psk@fillcycle
  \else
    \pst@cntg=\psk@fillcyclex
  \fi
  \pst@dimg=\wd\pst@fillbox
  \ifnum\pst@cntg=\z@
  \else
  \divide\pst@dimg\pst@cntg
  \fi
  \ifnum\pst@cntg<\z@\pst@cntg=-\pst@cntg\fi
  \advance\pst@cntg\m@ne
  \pst@cnth=\pst@cntg
  \def\pst@tempg{%
    \ifnum\pst@cnth<\pst@cntg\advance\pst@cnth\@ne\else\pst@cnth\z@\fi
    \moveright\pst@cnth\pst@dimg}}
%    \end{macrocode}
% \end{macro}
%
%% Auto box fill:        !! Fix dictionary
%
% \subsection{The PostScript subroutines}
%
%    \begin{macrocode}
%% DG addition begin - Apr. 8, 1997 and Dec. 1997 - Patch 2
\ifx\PstTiling\@undefined
\pst@def{AutoFillCycle}<%
  /c ED
  /n 0 def
  /s {
    /x x w c div n mul add def
    /n n c abs 1 sub lt { n 1 add } { 0 } ifelse def
  } def>

\pst@def{BoxFill}<%
  gsave
    gsave \tx@STV CM grestore dtransform CM idtransform
    abs /h ED abs /w ED
    pathbbox
    h div round 2 add cvi /y2 ED
    w div round 2 add cvi /x2 ED
    h div round 2 sub cvi /y1 ED
    w div round 2 sub cvi /x1 ED
    /y2 y2 y1 sub def
    /x2 x2 x1 sub def
    CP
    y1 h mul sub neg /y1 ED
    x1 w mul sub neg /x1 ED
    clip
    y2 {
      /x x1 def
      s
      x2 {
        save CP x y1
%% patch 4   hv --------------
        \ifx\VTeXversion\undefined
        \else
%%============ mv: 09-10-01 ??? this is likely to be a right change
        neg
%%============
        \fi
%% end patch 4
T moveto Box restore
        /x x w add def
      } repeat
      /y1 y1 h add def
    } repeat
    % Next line not useful... To see that, suppress clipping (DG)
    CP x y1 T moveto Box
  currentpoint currentfont grestore setfont moveto>
\else
%% DG modification begin - Apr. 8, 1997 and Nov. / Dec. 1997 - Patch 2
\pst@def{AutoFillCycleX}<%
  /cX ED
  /nX 0 def
  /CycleX {
    /x x w cX div nX mul add def
    /nX nX cX abs 1 sub lt { nX 1 add } { 0 } ifelse def
  } def>
\pst@def{AutoFillCycleY}<%
  /cY ED
  /mY 0 def
  /nY 0 def
  /CycleY {
    /y1 y1 h cY div mY mul sub def
    nY cY abs 1 sub lt { /nY nY 1 add def /mY 1 def }
                       { /nY 0 def        /mY cY abs 1 sub neg def } ifelse
  } def>

\pst@def{BoxFill}<%
  gsave
    gsave \tx@STV CM grestore dtransform CM idtransform
    abs /h ED abs /w ED
    pathbbox
    h div round 2 add cvi /y2 ED
    w div round 2 add cvi /x2 ED
    h div round 2 sub cvi /y1 ED
    w div round 2 sub cvi /x1 ED
    /CoefLoopX 0 def
    /CoefLoopY 0 def
    /CoefMoveX 0 def
    /CoefMoveY 0 def
    \psk@boxfillangle\space 0 ne {/CoefLoopX 8 def /CoefLoopY 8 def} if
    \psk@fillcyclex\space 0 ne {/CoefLoopX CoefLoopX 1 add def} if
    \psk@fillcycley\space 0 ne {/CoefLoopY CoefLoopY 1 add def} if
    \psk@fillmovex\space 0 ne
      {/CoefLoopX CoefLoopX 2 add def
       \psk@fillmovex\space 0 gt {/CoefMoveX CoefLoopX def}
                           {/CoefMoveX CoefLoopX neg def} ifelse} if
    \psk@fillmovey\space 0 ne
      {/CoefLoopY CoefLoopY 2 add def
       \psk@fillmovey\space 0 gt {/CoefMoveY CoefLoopY def}
                           {/CoefMoveY CoefLoopY neg def} ifelse} if
    \psk@fillsepx\space 0 ne {/CoefLoopX CoefLoopX 1 add def} if
    \psk@fillsepy\space 0 ne {/CoefLoopY CoefLoopY 1 add def} if
    /CoefLoopX CoefLoopX \psk@fillloopaddx\space add def
    /CoefLoopY CoefLoopY \psk@fillloopaddy\space add def
    /x2 x2 x1 sub 4 sub CoefLoopX 2 mul add def
    /y2 y2 y1 sub 4 sub CoefLoopY 2 mul add def
%% We must fix the origin of tiling, as it must not vary according other stuff
%% in the page!
    w x1 CoefLoopX add CoefMoveX add mul
      h y1 y2 add 1 sub CoefLoopY sub CoefMoveY sub mul moveto
    CP
    y1 h mul sub neg /y1 ED
    x1 w mul sub neg /x1 ED
%%  hv 2004-06-22   to prevent clash with pst-gr3d
%%    \psk@PstDebug 0 eq {clip} if
    \Pst@Debug 0 eq {clip} if
%% end hv
    \psk@fillmovex\space \psk@fillmovey
    gsave \tx@STV CM grestore dtransform CM idtransform
    /hmove ED /wmove ED
    /row 0 def
   y2 {
       /row row 1 add def
       /column 0 def
       /x x1 def
       CycleX
       save
       x2 {
          /column column 1 add def
          CycleY
          save CP x y1
%% patch 4   hv --------------
          \ifx\VTeXversion\undefined
          \else
%%============ mv: 09-10-01 ??? this is likely to be a right change
          neg
%%============
          \fi
  T moveto Box restore
          /x x w add def
          0 hmove translate
          } repeat
       restore
       /y1 y1 h add def
       wmove 0 translate
       } repeat
  currentpoint currentfont grestore setfont moveto>
\fi
%    \end{macrocode}

%    \begin{macrocode}
\def\pst@AutoBoxFill{%
  \leavevmode
  \begingroup
    \pst@stroke
    \pst@FlushCode
    \pst@Verb{\psk@boxfillangle\space \tx@RotBegin}%
    \pstVerb{\pst@dict /Box \pslbrace end}%
    \ifx\PstTiling\@undefined
    \else
      \ifx\pst@tempa\@undefined % Undefined for instance for \pscharpath
      \else\ifx\pst@tempa\@empty\else
        \def\pst@temph{0}%
        \ifx\pst@tempa\pst@temph
        \else
          \pstVerb{/TR {pop pop currentpoint translate \pst@tempa\space translate } def}%
        \fi
      \fi\fi
    \fi
    \hbox to \z@{\vbox to\z@{\vss\copy\pst@fillbox\vskip-\dp\pst@fillbox}\hss}%
    \ifx\PstTiling\@undefined
      \pstVerb{%
        tx@Dict begin \psrbrace def
        \ifnum\psk@fillcycle=\z@
          /s {} def
        \else
          \psk@fillcycle \tx@AutoFillCycle
        \fi
        \pst@number{\wd\pst@fillbox}%
        \psk@fillsepx\space add
        \pst@number{\ht\pst@fillbox}%
        \pst@number{\dp\pst@fillbox}%
        \psk@fillsepy\space add add
        \tx@BoxFill
        end}%
      \else
      \pstVerb{%
        tx@Dict begin \psrbrace def
        \ifnum\psk@fillcyclex=\z@
          /CycleX {} def
        \else
          \psk@fillcyclex\space \tx@AutoFillCycleX
        \fi
        \ifnum\psk@fillcycley=\z@
          /CycleY {} def
        \else
          \psk@fillcycley\space \tx@AutoFillCycleY
        \fi
        \pst@number{\wd\pst@fillbox}%
        \psk@fillsepx\space add
        \pst@number{\ht\pst@fillbox}%
        \pst@number{\dp\pst@fillbox}%
        \psk@fillsepy\space add add
        \tx@BoxFill
        end}%
    \fi
    \pst@Verb{\tx@RotEnd}%
  \endgroup}
%    \end{macrocode}
% \subsection{Closing}
%
%   Catcodes restoration.
%
%    \begin{macrocode}
\catcode`\@=\PstAtCode\relax
%    \end{macrocode}
%
%    \begin{macrocode}
%</pst-fill>
%    \end{macrocode}
%
% \Finale
%
\endinput
%%
%% End of file `pst-fill.dtx'

\ProvidesFile{pst-fill.tex}
  [\filedate\space v\fileversion\space `PST-fill' (tvz,dg)]
%</latex-wrapper>
%    \end{macrocode}
%
%
% \section{Pst-Fill Package{} code}
%
%    \begin{macrocode}
%<*pst-fill>
%    \end{macrocode}
%
% \subsection{Preamble}
%
%   Who we are.
%
%    \begin{macrocode}
\def\fileversion{1.01}
\def\filedate{2007/03/10}
\message{`PST-Fill' v\fileversion, \filedate\space (tvz,dg,hv)}
\csname PSTboxfillLoaded\endcsname
\let\PSTboxfillLoaded\endinput
%    \end{macrocode}
%
%   Require the main PSTricks package.
%
%    \begin{macrocode}
\ifx\PSTricksLoaded\endinput\else\input pstricks.tex\fi
%    \end{macrocode}
%
%   interface to the extended `\textsf{keyval}' package.
%
%    \begin{macrocode}
\ifx\PSTXKeyLoaded\endinput\else\input pst-xkey\fi
%
%    \end{macrocode}
%
%   Catcodes changes and defining the family name for xkeyval.
%
%    \begin{macrocode}
\edef\PstAtCode{\the\catcode`\@}\catcode`\@=11\relax

\pst@addfams{pst-fill}
%
%    \end{macrocode}
%
%
% \subsection{The size of the box}
% \begin{macro}{pst@@boxfillsize}
%    \begin{macrocode}
%
\def\pst@@boxfillsize#1(#2,#3)#4(#5,#6)#7(#8\@nil{%
  \begingroup
    \ifx\@empty#7\relax
      \pst@dima\z@
      \pst@dimb\z@
      \pssetxlength\pst@dimc{#2}%
      \pssetylength\pst@dimd{#3}%
    \else
      \pssetxlength\pst@dima{#2}%
      \pssetylength\pst@dimb{#3}%
      \pssetxlength\pst@dimc{#5}%
      \pssetylength\pst@dimd{#6}%
    \fi
    \xdef\pst@tempg{%
      \pst@dima=\number\pst@dima sp
      \pst@dimb=\number\pst@dimb sp
      \pst@dimc=\number\pst@dimc sp
      \pst@dimd=\number\pst@dimd sp }%
  \endgroup
  \let\psk@boxfillsize\pst@tempg}
%    \end{macrocode}
% \end{macro}
%

% \subsection{Definition of the parameters}
%
%    \begin{macrocode}
\define@key[psset]{pst-fill}{boxfillsize}{%
  \def\pst@tempg{#1}\def\pst@temph{auto}%
  \ifx\pst@tempg\pst@temph
    \let\psk@boxfillsize\relax
  \else
    \pst@@boxfillsize#1(\z@,\z@)\@empty(\z@,\z@)(\@nil
  \fi}
\psset{boxfillsize={(-15cm,-15cm)(15cm,15cm)}}
\define@key[psset]{pst-fill}{boxfillcolor}{\pst@getcolor{#1}\psboxfillcolor}
\psset{boxfillcolor=black}% hv
\define@key[psset]{pst-fill}{boxfillangle}{\pst@getangle{#1}\psk@boxfillangle}
\psset{boxfillangle=0}
\define@key[psset]{pst-fill}{fillsepx}{%
  \pst@getlength{#1}\psk@fillsepx}
\define@key[psset]{pst-fill}{fillsepy}{%
  \pst@getlength{#1}\psk@fillsepy}
\define@key[psset]{pst-fill}{fillsep}{%
  \pst@getlength{#1}\psk@fillsepx%
  \let\psk@fillsepy\psk@fillsepx}
\psset{fillsep=2pt}

\ifx\PstTiling\@undefined
  \define@key[psset]{pst-fill}{fillcycle}{\pst@getint{#1}\psk@fillcycle}
  \psset{fillcycle=0}
\else
  \define@key[psset]{pst-fill}{fillangle}{\pst@getangle{#1}\psk@boxfillangle}
  \define@key[psset]{pst-fill}{fillsize}{%
      \def\pst@tempg{#1}\def\pst@temph{auto}%
      \ifx\pst@tempg\pst@temph\let\psk@boxfillsize\relax
      \else\pst@@boxfillsize#1(\z@,\z@)\@empty(\z@,\z@)(\@nil\fi}
  \psset{fillsep=0,fillsize=auto}
  \define@key[psset]{pst-fill}{fillcyclex}{\pst@getint{#1}\psk@fillcyclex}
  \define@key[psset]{pst-fill}{fillcycley}{\pst@getint{#1}\psk@fillcycley}
  \define@key[psset]{pst-fill}{fillcycle}{%
    \pst@getint{#1}\psk@fillcyclex\let\psk@fillcycley\psk@fillcyclex}
  \psset{fillcycle=0}
  \define@key[psset]{pst-fill}{fillmovex}{\pst@getlength{#1}\psk@fillmovex}
  \define@key[psset]{pst-fill}{fillmovey}{\pst@getlength{#1}\psk@fillmovey}
  \define@key[psset]{pst-fill}{fillmove}{%
      \pst@getlength{#1}\psk@fillmovex\let\psk@fillmovey\psk@fillmovex}
  \psset{fillmove=0pt}
  \define@key[psset]{pst-fill}{fillloopaddx}{\pst@getint{#1}\psk@fillloopaddx}
  \define@key[psset]{pst-fill}{fillloopaddy}{\pst@getint{#1}\psk@fillloopaddy}
  \define@key[psset]{pst-fill}{fillloopadd}{%
    \pst@getint{#1}\psk@fillloopaddx\let\psk@fillloopaddy\psk@fillloopaddx}
  \psset{fillloopadd=0}
%    \end{macrocode}
%
%    \begin{macrocode}
% For debugging (to debug, set PstDebug=1)
% we now use the one from pstricks to prevent a clash with package
% pstricks                        2004-06-22
%%    \define@key[psset]{pst-fill}{PstDebug}{\pst@getint{#1}\psk@PstDebug}
    \psset{PstDebug=0}
\fi
% DG addition end
%    \end{macrocode}

% \subsection{Definition of the fill box}
% \begin{macro}{psboxfill}
%    \begin{macrocode}
\newbox\pst@fillbox
\def\psboxfill{\pst@killglue\pst@makebox\psboxfill@i}
\def\psboxfill@i{\setbox\pst@fillbox\box\pst@hbox\ignorespaces}
%    \end{macrocode}
% \end{macro}
% \subsection{The main macros}
%
% \begin{macro}{psfs@boxfill}
%    \begin{macrocode}
\def\psfs@boxfill{%
  \ifvoid\pst@fillbox
    \@pstrickserr{Fill box is empty. Use \string\psboxfill\space first.}\@ehpa
  \else
    \ifx\psk@boxfillsize\relax \pst@AutoBoxFill
    \else\pst@ManualBoxFill\fi
  \fi}
%    \end{macrocode}
% \end{macro}
%
% \begin{macro}{pst@ManualBoxFill}
%    \begin{macrocode}
\def\pst@ManualBoxFill{%
  \leavevmode
  \begingroup
    \pst@FlushCode
    \begin@psclip
    \pstVerb{clip}%
    \expandafter\pst@AddFillBox\psk@boxfillsize
    \end@psclip
  \endgroup}
%    \end{macrocode}
% \end{macro}
%
% \begin{macro}{pst@FlushCode}
%    \begin{macrocode}
\def\pst@FlushCode{%
  \pst@Verb{%
    /mtrxc CM def
    CP CP T
    \tx@STV
    \psk@origin
    \psk@swapaxes
    \pst@newpath
    \pst@code
    mtrxc setmatrix
    moveto
    0 setgray}%
  \gdef\pst@code{}}
%    \end{macrocode}
% \end{macro}
%
% \begin{macro}{pst@AddFillBox}
%    \begin{macrocode}
\def\pst@AddFillBox#1 #2 #3 #4 {%
  \begingroup
    \setbox\pst@fillbox=\vbox{%
      \hbox{\unhcopy\pst@fillbox\kern\psk@fillsepx\p@}%
      \vskip\psk@fillsepy\p@}%
    \psk@boxfillsize
    \pst@cnta=\pst@dimc
    \advance\pst@cnta-\pst@dima
    \divide\pst@cnta\wd\pst@fillbox
    \pst@cntb=\pst@dimd
    \advance\pst@cntb-\pst@dimb
    \pst@dimd=\ht\pst@fillbox
    \divide\pst@cntb\pst@dimd
    \def\pst@tempa{%
      \pst@tempg
      \copy\pst@fillbox
      \advance\pst@cntc\@ne
      \ifnum\pst@cntc<\pst@cntd\expandafter\pst@tempa\fi}%
    \let\pst@tempg\relax
    \pst@cntc-\tw@
    \pst@cntd\pst@cnta
    \setbox\pst@fillbox=\hbox to \z@{%
      \kern\pst@dima
      \kern-\wd\pst@fillbox
      \pst@tempa
      \hss}%
    \pst@cntd\pst@cntb
%% DG modification begin - Dec. 11, 1997 - Patch 2
    \ifx\PstTiling\@undefined
      \ifnum\psk@fillcycle=\z@\pst@ManualFillCycle\fi
    \else
      \ifnum\psk@fillcyclex=\z@\pst@ManualFillCycle\fi
    \fi
%% DG modification end
    \global\setbox\pst@boxg=\vbox to\z@{%
      \offinterlineskip
      \vss
      \pst@tempa
      \vskip\pst@dimb}%
  \endgroup
  \setbox\pst@fillbox\box\pst@boxg
  \pst@rotate\psk@boxfillangle\pst@fillbox
  \box\pst@fillbox}
%    \end{macrocode}
% \end{macro}
%
% \begin{macro}{pst@ManualFillCycle}
%    \begin{macrocode}
\def\pst@ManualFillCycle{%
  \ifx\PstTiling\@undefined
    \pst@cntg=\psk@fillcycle
  \else
    \pst@cntg=\psk@fillcyclex
  \fi
  \pst@dimg=\wd\pst@fillbox
  \ifnum\pst@cntg=\z@
  \else
  \divide\pst@dimg\pst@cntg
  \fi
  \ifnum\pst@cntg<\z@\pst@cntg=-\pst@cntg\fi
  \advance\pst@cntg\m@ne
  \pst@cnth=\pst@cntg
  \def\pst@tempg{%
    \ifnum\pst@cnth<\pst@cntg\advance\pst@cnth\@ne\else\pst@cnth\z@\fi
    \moveright\pst@cnth\pst@dimg}}
%    \end{macrocode}
% \end{macro}
%
%% Auto box fill:        !! Fix dictionary
%
% \subsection{The PostScript subroutines}
%
%    \begin{macrocode}
%% DG addition begin - Apr. 8, 1997 and Dec. 1997 - Patch 2
\ifx\PstTiling\@undefined
\pst@def{AutoFillCycle}<%
  /c ED
  /n 0 def
  /s {
    /x x w c div n mul add def
    /n n c abs 1 sub lt { n 1 add } { 0 } ifelse def
  } def>

\pst@def{BoxFill}<%
  gsave
    gsave \tx@STV CM grestore dtransform CM idtransform
    abs /h ED abs /w ED
    pathbbox
    h div round 2 add cvi /y2 ED
    w div round 2 add cvi /x2 ED
    h div round 2 sub cvi /y1 ED
    w div round 2 sub cvi /x1 ED
    /y2 y2 y1 sub def
    /x2 x2 x1 sub def
    CP
    y1 h mul sub neg /y1 ED
    x1 w mul sub neg /x1 ED
    clip
    y2 {
      /x x1 def
      s
      x2 {
        save CP x y1
%% patch 4   hv --------------
        \ifx\VTeXversion\undefined
        \else
%%============ mv: 09-10-01 ??? this is likely to be a right change
        neg
%%============
        \fi
%% end patch 4
T moveto Box restore
        /x x w add def
      } repeat
      /y1 y1 h add def
    } repeat
    % Next line not useful... To see that, suppress clipping (DG)
    CP x y1 T moveto Box
  currentpoint currentfont grestore setfont moveto>
\else
%% DG modification begin - Apr. 8, 1997 and Nov. / Dec. 1997 - Patch 2
\pst@def{AutoFillCycleX}<%
  /cX ED
  /nX 0 def
  /CycleX {
    /x x w cX div nX mul add def
    /nX nX cX abs 1 sub lt { nX 1 add } { 0 } ifelse def
  } def>
\pst@def{AutoFillCycleY}<%
  /cY ED
  /mY 0 def
  /nY 0 def
  /CycleY {
    /y1 y1 h cY div mY mul sub def
    nY cY abs 1 sub lt { /nY nY 1 add def /mY 1 def }
                       { /nY 0 def        /mY cY abs 1 sub neg def } ifelse
  } def>

\pst@def{BoxFill}<%
  gsave
    gsave \tx@STV CM grestore dtransform CM idtransform
    abs /h ED abs /w ED
    pathbbox
    h div round 2 add cvi /y2 ED
    w div round 2 add cvi /x2 ED
    h div round 2 sub cvi /y1 ED
    w div round 2 sub cvi /x1 ED
    /CoefLoopX 0 def
    /CoefLoopY 0 def
    /CoefMoveX 0 def
    /CoefMoveY 0 def
    \psk@boxfillangle\space 0 ne {/CoefLoopX 8 def /CoefLoopY 8 def} if
    \psk@fillcyclex\space 0 ne {/CoefLoopX CoefLoopX 1 add def} if
    \psk@fillcycley\space 0 ne {/CoefLoopY CoefLoopY 1 add def} if
    \psk@fillmovex\space 0 ne
      {/CoefLoopX CoefLoopX 2 add def
       \psk@fillmovex\space 0 gt {/CoefMoveX CoefLoopX def}
                           {/CoefMoveX CoefLoopX neg def} ifelse} if
    \psk@fillmovey\space 0 ne
      {/CoefLoopY CoefLoopY 2 add def
       \psk@fillmovey\space 0 gt {/CoefMoveY CoefLoopY def}
                           {/CoefMoveY CoefLoopY neg def} ifelse} if
    \psk@fillsepx\space 0 ne {/CoefLoopX CoefLoopX 1 add def} if
    \psk@fillsepy\space 0 ne {/CoefLoopY CoefLoopY 1 add def} if
    /CoefLoopX CoefLoopX \psk@fillloopaddx\space add def
    /CoefLoopY CoefLoopY \psk@fillloopaddy\space add def
    /x2 x2 x1 sub 4 sub CoefLoopX 2 mul add def
    /y2 y2 y1 sub 4 sub CoefLoopY 2 mul add def
%% We must fix the origin of tiling, as it must not vary according other stuff
%% in the page!
    w x1 CoefLoopX add CoefMoveX add mul
      h y1 y2 add 1 sub CoefLoopY sub CoefMoveY sub mul moveto
    CP
    y1 h mul sub neg /y1 ED
    x1 w mul sub neg /x1 ED
%%  hv 2004-06-22   to prevent clash with pst-gr3d
%%    \psk@PstDebug 0 eq {clip} if
    \Pst@Debug 0 eq {clip} if
%% end hv
    \psk@fillmovex\space \psk@fillmovey
    gsave \tx@STV CM grestore dtransform CM idtransform
    /hmove ED /wmove ED
    /row 0 def
   y2 {
       /row row 1 add def
       /column 0 def
       /x x1 def
       CycleX
       save
       x2 {
          /column column 1 add def
          CycleY
          save CP x y1
%% patch 4   hv --------------
          \ifx\VTeXversion\undefined
          \else
%%============ mv: 09-10-01 ??? this is likely to be a right change
          neg
%%============
          \fi
  T moveto Box restore
          /x x w add def
          0 hmove translate
          } repeat
       restore
       /y1 y1 h add def
       wmove 0 translate
       } repeat
  currentpoint currentfont grestore setfont moveto>
\fi
%    \end{macrocode}

%    \begin{macrocode}
\def\pst@AutoBoxFill{%
  \leavevmode
  \begingroup
    \pst@stroke
    \pst@FlushCode
    \pst@Verb{\psk@boxfillangle\space \tx@RotBegin}%
    \pstVerb{\pst@dict /Box \pslbrace end}%
    \ifx\PstTiling\@undefined
    \else
      \ifx\pst@tempa\@undefined % Undefined for instance for \pscharpath
      \else\ifx\pst@tempa\@empty\else
        \def\pst@temph{0}%
        \ifx\pst@tempa\pst@temph
        \else
          \pstVerb{/TR {pop pop currentpoint translate \pst@tempa\space translate } def}%
        \fi
      \fi\fi
    \fi
    \hbox to \z@{\vbox to\z@{\vss\copy\pst@fillbox\vskip-\dp\pst@fillbox}\hss}%
    \ifx\PstTiling\@undefined
      \pstVerb{%
        tx@Dict begin \psrbrace def
        \ifnum\psk@fillcycle=\z@
          /s {} def
        \else
          \psk@fillcycle \tx@AutoFillCycle
        \fi
        \pst@number{\wd\pst@fillbox}%
        \psk@fillsepx\space add
        \pst@number{\ht\pst@fillbox}%
        \pst@number{\dp\pst@fillbox}%
        \psk@fillsepy\space add add
        \tx@BoxFill
        end}%
      \else
      \pstVerb{%
        tx@Dict begin \psrbrace def
        \ifnum\psk@fillcyclex=\z@
          /CycleX {} def
        \else
          \psk@fillcyclex\space \tx@AutoFillCycleX
        \fi
        \ifnum\psk@fillcycley=\z@
          /CycleY {} def
        \else
          \psk@fillcycley\space \tx@AutoFillCycleY
        \fi
        \pst@number{\wd\pst@fillbox}%
        \psk@fillsepx\space add
        \pst@number{\ht\pst@fillbox}%
        \pst@number{\dp\pst@fillbox}%
        \psk@fillsepy\space add add
        \tx@BoxFill
        end}%
    \fi
    \pst@Verb{\tx@RotEnd}%
  \endgroup}
%    \end{macrocode}
% \subsection{Closing}
%
%   Catcodes restoration.
%
%    \begin{macrocode}
\catcode`\@=\PstAtCode\relax
%    \end{macrocode}
%
%    \begin{macrocode}
%</pst-fill>
%    \end{macrocode}
%
% \Finale
%
\endinput
%%
%% End of file `pst-fill.dtx'

\ProvidesFile{pst-fill.tex}
  [\filedate\space v\fileversion\space `PST-fill' (tvz,dg)]
%</latex-wrapper>
%    \end{macrocode}
%
%
% \section{Pst-Fill Package{} code}
%
%    \begin{macrocode}
%<*pst-fill>
%    \end{macrocode}
%
% \subsection{Preamble}
%
%   Who we are.
%
%    \begin{macrocode}
\def\fileversion{1.01}
\def\filedate{2007/03/10}
\message{`PST-Fill' v\fileversion, \filedate\space (tvz,dg,hv)}
\csname PSTboxfillLoaded\endcsname
\let\PSTboxfillLoaded\endinput
%    \end{macrocode}
%
%   Require the main PSTricks package.
%
%    \begin{macrocode}
\ifx\PSTricksLoaded\endinput\else\input pstricks.tex\fi
%    \end{macrocode}
%
%   interface to the extended `\textsf{keyval}' package.
%
%    \begin{macrocode}
\ifx\PSTXKeyLoaded\endinput\else\input pst-xkey\fi
%
%    \end{macrocode}
%
%   Catcodes changes and defining the family name for xkeyval.
%
%    \begin{macrocode}
\edef\PstAtCode{\the\catcode`\@}\catcode`\@=11\relax

\pst@addfams{pst-fill}
%
%    \end{macrocode}
%
%
% \subsection{The size of the box}
% \begin{macro}{pst@@boxfillsize}
%    \begin{macrocode}
%
\def\pst@@boxfillsize#1(#2,#3)#4(#5,#6)#7(#8\@nil{%
  \begingroup
    \ifx\@empty#7\relax
      \pst@dima\z@
      \pst@dimb\z@
      \pssetxlength\pst@dimc{#2}%
      \pssetylength\pst@dimd{#3}%
    \else
      \pssetxlength\pst@dima{#2}%
      \pssetylength\pst@dimb{#3}%
      \pssetxlength\pst@dimc{#5}%
      \pssetylength\pst@dimd{#6}%
    \fi
    \xdef\pst@tempg{%
      \pst@dima=\number\pst@dima sp
      \pst@dimb=\number\pst@dimb sp
      \pst@dimc=\number\pst@dimc sp
      \pst@dimd=\number\pst@dimd sp }%
  \endgroup
  \let\psk@boxfillsize\pst@tempg}
%    \end{macrocode}
% \end{macro}
%

% \subsection{Definition of the parameters}
%
%    \begin{macrocode}
\define@key[psset]{pst-fill}{boxfillsize}{%
  \def\pst@tempg{#1}\def\pst@temph{auto}%
  \ifx\pst@tempg\pst@temph
    \let\psk@boxfillsize\relax
  \else
    \pst@@boxfillsize#1(\z@,\z@)\@empty(\z@,\z@)(\@nil
  \fi}
\psset{boxfillsize={(-15cm,-15cm)(15cm,15cm)}}
\define@key[psset]{pst-fill}{boxfillcolor}{\pst@getcolor{#1}\psboxfillcolor}
\psset{boxfillcolor=black}% hv
\define@key[psset]{pst-fill}{boxfillangle}{\pst@getangle{#1}\psk@boxfillangle}
\psset{boxfillangle=0}
\define@key[psset]{pst-fill}{fillsepx}{%
  \pst@getlength{#1}\psk@fillsepx}
\define@key[psset]{pst-fill}{fillsepy}{%
  \pst@getlength{#1}\psk@fillsepy}
\define@key[psset]{pst-fill}{fillsep}{%
  \pst@getlength{#1}\psk@fillsepx%
  \let\psk@fillsepy\psk@fillsepx}
\psset{fillsep=2pt}

\ifx\PstTiling\@undefined
  \define@key[psset]{pst-fill}{fillcycle}{\pst@getint{#1}\psk@fillcycle}
  \psset{fillcycle=0}
\else
  \define@key[psset]{pst-fill}{fillangle}{\pst@getangle{#1}\psk@boxfillangle}
  \define@key[psset]{pst-fill}{fillsize}{%
      \def\pst@tempg{#1}\def\pst@temph{auto}%
      \ifx\pst@tempg\pst@temph\let\psk@boxfillsize\relax
      \else\pst@@boxfillsize#1(\z@,\z@)\@empty(\z@,\z@)(\@nil\fi}
  \psset{fillsep=0,fillsize=auto}
  \define@key[psset]{pst-fill}{fillcyclex}{\pst@getint{#1}\psk@fillcyclex}
  \define@key[psset]{pst-fill}{fillcycley}{\pst@getint{#1}\psk@fillcycley}
  \define@key[psset]{pst-fill}{fillcycle}{%
    \pst@getint{#1}\psk@fillcyclex\let\psk@fillcycley\psk@fillcyclex}
  \psset{fillcycle=0}
  \define@key[psset]{pst-fill}{fillmovex}{\pst@getlength{#1}\psk@fillmovex}
  \define@key[psset]{pst-fill}{fillmovey}{\pst@getlength{#1}\psk@fillmovey}
  \define@key[psset]{pst-fill}{fillmove}{%
      \pst@getlength{#1}\psk@fillmovex\let\psk@fillmovey\psk@fillmovex}
  \psset{fillmove=0pt}
  \define@key[psset]{pst-fill}{fillloopaddx}{\pst@getint{#1}\psk@fillloopaddx}
  \define@key[psset]{pst-fill}{fillloopaddy}{\pst@getint{#1}\psk@fillloopaddy}
  \define@key[psset]{pst-fill}{fillloopadd}{%
    \pst@getint{#1}\psk@fillloopaddx\let\psk@fillloopaddy\psk@fillloopaddx}
  \psset{fillloopadd=0}
%    \end{macrocode}
%
%    \begin{macrocode}
% For debugging (to debug, set PstDebug=1)
% we now use the one from pstricks to prevent a clash with package
% pstricks                        2004-06-22
%%    \define@key[psset]{pst-fill}{PstDebug}{\pst@getint{#1}\psk@PstDebug}
    \psset{PstDebug=0}
\fi
% DG addition end
%    \end{macrocode}

% \subsection{Definition of the fill box}
% \begin{macro}{psboxfill}
%    \begin{macrocode}
\newbox\pst@fillbox
\def\psboxfill{\pst@killglue\pst@makebox\psboxfill@i}
\def\psboxfill@i{\setbox\pst@fillbox\box\pst@hbox\ignorespaces}
%    \end{macrocode}
% \end{macro}
% \subsection{The main macros}
%
% \begin{macro}{psfs@boxfill}
%    \begin{macrocode}
\def\psfs@boxfill{%
  \ifvoid\pst@fillbox
    \@pstrickserr{Fill box is empty. Use \string\psboxfill\space first.}\@ehpa
  \else
    \ifx\psk@boxfillsize\relax \pst@AutoBoxFill
    \else\pst@ManualBoxFill\fi
  \fi}
%    \end{macrocode}
% \end{macro}
%
% \begin{macro}{pst@ManualBoxFill}
%    \begin{macrocode}
\def\pst@ManualBoxFill{%
  \leavevmode
  \begingroup
    \pst@FlushCode
    \begin@psclip
    \pstVerb{clip}%
    \expandafter\pst@AddFillBox\psk@boxfillsize
    \end@psclip
  \endgroup}
%    \end{macrocode}
% \end{macro}
%
% \begin{macro}{pst@FlushCode}
%    \begin{macrocode}
\def\pst@FlushCode{%
  \pst@Verb{%
    /mtrxc CM def
    CP CP T
    \tx@STV
    \psk@origin
    \psk@swapaxes
    \pst@newpath
    \pst@code
    mtrxc setmatrix
    moveto
    0 setgray}%
  \gdef\pst@code{}}
%    \end{macrocode}
% \end{macro}
%
% \begin{macro}{pst@AddFillBox}
%    \begin{macrocode}
\def\pst@AddFillBox#1 #2 #3 #4 {%
  \begingroup
    \setbox\pst@fillbox=\vbox{%
      \hbox{\unhcopy\pst@fillbox\kern\psk@fillsepx\p@}%
      \vskip\psk@fillsepy\p@}%
    \psk@boxfillsize
    \pst@cnta=\pst@dimc
    \advance\pst@cnta-\pst@dima
    \divide\pst@cnta\wd\pst@fillbox
    \pst@cntb=\pst@dimd
    \advance\pst@cntb-\pst@dimb
    \pst@dimd=\ht\pst@fillbox
    \divide\pst@cntb\pst@dimd
    \def\pst@tempa{%
      \pst@tempg
      \copy\pst@fillbox
      \advance\pst@cntc\@ne
      \ifnum\pst@cntc<\pst@cntd\expandafter\pst@tempa\fi}%
    \let\pst@tempg\relax
    \pst@cntc-\tw@
    \pst@cntd\pst@cnta
    \setbox\pst@fillbox=\hbox to \z@{%
      \kern\pst@dima
      \kern-\wd\pst@fillbox
      \pst@tempa
      \hss}%
    \pst@cntd\pst@cntb
%% DG modification begin - Dec. 11, 1997 - Patch 2
    \ifx\PstTiling\@undefined
      \ifnum\psk@fillcycle=\z@\pst@ManualFillCycle\fi
    \else
      \ifnum\psk@fillcyclex=\z@\pst@ManualFillCycle\fi
    \fi
%% DG modification end
    \global\setbox\pst@boxg=\vbox to\z@{%
      \offinterlineskip
      \vss
      \pst@tempa
      \vskip\pst@dimb}%
  \endgroup
  \setbox\pst@fillbox\box\pst@boxg
  \pst@rotate\psk@boxfillangle\pst@fillbox
  \box\pst@fillbox}
%    \end{macrocode}
% \end{macro}
%
% \begin{macro}{pst@ManualFillCycle}
%    \begin{macrocode}
\def\pst@ManualFillCycle{%
  \ifx\PstTiling\@undefined
    \pst@cntg=\psk@fillcycle
  \else
    \pst@cntg=\psk@fillcyclex
  \fi
  \pst@dimg=\wd\pst@fillbox
  \ifnum\pst@cntg=\z@
  \else
  \divide\pst@dimg\pst@cntg
  \fi
  \ifnum\pst@cntg<\z@\pst@cntg=-\pst@cntg\fi
  \advance\pst@cntg\m@ne
  \pst@cnth=\pst@cntg
  \def\pst@tempg{%
    \ifnum\pst@cnth<\pst@cntg\advance\pst@cnth\@ne\else\pst@cnth\z@\fi
    \moveright\pst@cnth\pst@dimg}}
%    \end{macrocode}
% \end{macro}
%
%% Auto box fill:        !! Fix dictionary
%
% \subsection{The PostScript subroutines}
%
%    \begin{macrocode}
%% DG addition begin - Apr. 8, 1997 and Dec. 1997 - Patch 2
\ifx\PstTiling\@undefined
\pst@def{AutoFillCycle}<%
  /c ED
  /n 0 def
  /s {
    /x x w c div n mul add def
    /n n c abs 1 sub lt { n 1 add } { 0 } ifelse def
  } def>

\pst@def{BoxFill}<%
  gsave
    gsave \tx@STV CM grestore dtransform CM idtransform
    abs /h ED abs /w ED
    pathbbox
    h div round 2 add cvi /y2 ED
    w div round 2 add cvi /x2 ED
    h div round 2 sub cvi /y1 ED
    w div round 2 sub cvi /x1 ED
    /y2 y2 y1 sub def
    /x2 x2 x1 sub def
    CP
    y1 h mul sub neg /y1 ED
    x1 w mul sub neg /x1 ED
    clip
    y2 {
      /x x1 def
      s
      x2 {
        save CP x y1
%% patch 4   hv --------------
        \ifx\VTeXversion\undefined
        \else
%%============ mv: 09-10-01 ??? this is likely to be a right change
        neg
%%============
        \fi
%% end patch 4
T moveto Box restore
        /x x w add def
      } repeat
      /y1 y1 h add def
    } repeat
    % Next line not useful... To see that, suppress clipping (DG)
    CP x y1 T moveto Box
  currentpoint currentfont grestore setfont moveto>
\else
%% DG modification begin - Apr. 8, 1997 and Nov. / Dec. 1997 - Patch 2
\pst@def{AutoFillCycleX}<%
  /cX ED
  /nX 0 def
  /CycleX {
    /x x w cX div nX mul add def
    /nX nX cX abs 1 sub lt { nX 1 add } { 0 } ifelse def
  } def>
\pst@def{AutoFillCycleY}<%
  /cY ED
  /mY 0 def
  /nY 0 def
  /CycleY {
    /y1 y1 h cY div mY mul sub def
    nY cY abs 1 sub lt { /nY nY 1 add def /mY 1 def }
                       { /nY 0 def        /mY cY abs 1 sub neg def } ifelse
  } def>

\pst@def{BoxFill}<%
  gsave
    gsave \tx@STV CM grestore dtransform CM idtransform
    abs /h ED abs /w ED
    pathbbox
    h div round 2 add cvi /y2 ED
    w div round 2 add cvi /x2 ED
    h div round 2 sub cvi /y1 ED
    w div round 2 sub cvi /x1 ED
    /CoefLoopX 0 def
    /CoefLoopY 0 def
    /CoefMoveX 0 def
    /CoefMoveY 0 def
    \psk@boxfillangle\space 0 ne {/CoefLoopX 8 def /CoefLoopY 8 def} if
    \psk@fillcyclex\space 0 ne {/CoefLoopX CoefLoopX 1 add def} if
    \psk@fillcycley\space 0 ne {/CoefLoopY CoefLoopY 1 add def} if
    \psk@fillmovex\space 0 ne
      {/CoefLoopX CoefLoopX 2 add def
       \psk@fillmovex\space 0 gt {/CoefMoveX CoefLoopX def}
                           {/CoefMoveX CoefLoopX neg def} ifelse} if
    \psk@fillmovey\space 0 ne
      {/CoefLoopY CoefLoopY 2 add def
       \psk@fillmovey\space 0 gt {/CoefMoveY CoefLoopY def}
                           {/CoefMoveY CoefLoopY neg def} ifelse} if
    \psk@fillsepx\space 0 ne {/CoefLoopX CoefLoopX 1 add def} if
    \psk@fillsepy\space 0 ne {/CoefLoopY CoefLoopY 1 add def} if
    /CoefLoopX CoefLoopX \psk@fillloopaddx\space add def
    /CoefLoopY CoefLoopY \psk@fillloopaddy\space add def
    /x2 x2 x1 sub 4 sub CoefLoopX 2 mul add def
    /y2 y2 y1 sub 4 sub CoefLoopY 2 mul add def
%% We must fix the origin of tiling, as it must not vary according other stuff
%% in the page!
    w x1 CoefLoopX add CoefMoveX add mul
      h y1 y2 add 1 sub CoefLoopY sub CoefMoveY sub mul moveto
    CP
    y1 h mul sub neg /y1 ED
    x1 w mul sub neg /x1 ED
%%  hv 2004-06-22   to prevent clash with pst-gr3d
%%    \psk@PstDebug 0 eq {clip} if
    \Pst@Debug 0 eq {clip} if
%% end hv
    \psk@fillmovex\space \psk@fillmovey
    gsave \tx@STV CM grestore dtransform CM idtransform
    /hmove ED /wmove ED
    /row 0 def
   y2 {
       /row row 1 add def
       /column 0 def
       /x x1 def
       CycleX
       save
       x2 {
          /column column 1 add def
          CycleY
          save CP x y1
%% patch 4   hv --------------
          \ifx\VTeXversion\undefined
          \else
%%============ mv: 09-10-01 ??? this is likely to be a right change
          neg
%%============
          \fi
  T moveto Box restore
          /x x w add def
          0 hmove translate
          } repeat
       restore
       /y1 y1 h add def
       wmove 0 translate
       } repeat
  currentpoint currentfont grestore setfont moveto>
\fi
%    \end{macrocode}

%    \begin{macrocode}
\def\pst@AutoBoxFill{%
  \leavevmode
  \begingroup
    \pst@stroke
    \pst@FlushCode
    \pst@Verb{\psk@boxfillangle\space \tx@RotBegin}%
    \pstVerb{\pst@dict /Box \pslbrace end}%
    \ifx\PstTiling\@undefined
    \else
      \ifx\pst@tempa\@undefined % Undefined for instance for \pscharpath
      \else\ifx\pst@tempa\@empty\else
        \def\pst@temph{0}%
        \ifx\pst@tempa\pst@temph
        \else
          \pstVerb{/TR {pop pop currentpoint translate \pst@tempa\space translate } def}%
        \fi
      \fi\fi
    \fi
    \hbox to \z@{\vbox to\z@{\vss\copy\pst@fillbox\vskip-\dp\pst@fillbox}\hss}%
    \ifx\PstTiling\@undefined
      \pstVerb{%
        tx@Dict begin \psrbrace def
        \ifnum\psk@fillcycle=\z@
          /s {} def
        \else
          \psk@fillcycle \tx@AutoFillCycle
        \fi
        \pst@number{\wd\pst@fillbox}%
        \psk@fillsepx\space add
        \pst@number{\ht\pst@fillbox}%
        \pst@number{\dp\pst@fillbox}%
        \psk@fillsepy\space add add
        \tx@BoxFill
        end}%
      \else
      \pstVerb{%
        tx@Dict begin \psrbrace def
        \ifnum\psk@fillcyclex=\z@
          /CycleX {} def
        \else
          \psk@fillcyclex\space \tx@AutoFillCycleX
        \fi
        \ifnum\psk@fillcycley=\z@
          /CycleY {} def
        \else
          \psk@fillcycley\space \tx@AutoFillCycleY
        \fi
        \pst@number{\wd\pst@fillbox}%
        \psk@fillsepx\space add
        \pst@number{\ht\pst@fillbox}%
        \pst@number{\dp\pst@fillbox}%
        \psk@fillsepy\space add add
        \tx@BoxFill
        end}%
    \fi
    \pst@Verb{\tx@RotEnd}%
  \endgroup}
%    \end{macrocode}
% \subsection{Closing}
%
%   Catcodes restoration.
%
%    \begin{macrocode}
\catcode`\@=\PstAtCode\relax
%    \end{macrocode}
%
%    \begin{macrocode}
%</pst-fill>
%    \end{macrocode}
%
% \Finale
%
\endinput
%%
%% End of file `pst-fill.dtx'

\ProvidesFile{pst-fill.tex}
  [\filedate\space v\fileversion\space `PST-fill' (tvz,dg)]
%</latex-wrapper>
%    \end{macrocode}
%
%
% \section{Pst-Fill Package{} code}
%
%    \begin{macrocode}
%<*pst-fill>
%    \end{macrocode}
%
% \subsection{Preamble}
%
%   Who we are.
%
%    \begin{macrocode}
\def\fileversion{1.01}
\def\filedate{2007/03/10}
\message{`PST-Fill' v\fileversion, \filedate\space (tvz,dg,hv)}
\csname PSTboxfillLoaded\endcsname
\let\PSTboxfillLoaded\endinput
%    \end{macrocode}
%
%   Require the main PSTricks package.
%
%    \begin{macrocode}
\ifx\PSTricksLoaded\endinput\else\input pstricks.tex\fi
%    \end{macrocode}
%
%   interface to the extended `\textsf{keyval}' package.
%
%    \begin{macrocode}
\ifx\PSTXKeyLoaded\endinput\else\input pst-xkey\fi
%
%    \end{macrocode}
%
%   Catcodes changes and defining the family name for xkeyval.
%
%    \begin{macrocode}
\edef\PstAtCode{\the\catcode`\@}\catcode`\@=11\relax

\pst@addfams{pst-fill}
%
%    \end{macrocode}
%
%
% \subsection{The size of the box}
% \begin{macro}{pst@@boxfillsize}
%    \begin{macrocode}
%
\def\pst@@boxfillsize#1(#2,#3)#4(#5,#6)#7(#8\@nil{%
  \begingroup
    \ifx\@empty#7\relax
      \pst@dima\z@
      \pst@dimb\z@
      \pssetxlength\pst@dimc{#2}%
      \pssetylength\pst@dimd{#3}%
    \else
      \pssetxlength\pst@dima{#2}%
      \pssetylength\pst@dimb{#3}%
      \pssetxlength\pst@dimc{#5}%
      \pssetylength\pst@dimd{#6}%
    \fi
    \xdef\pst@tempg{%
      \pst@dima=\number\pst@dima sp
      \pst@dimb=\number\pst@dimb sp
      \pst@dimc=\number\pst@dimc sp
      \pst@dimd=\number\pst@dimd sp }%
  \endgroup
  \let\psk@boxfillsize\pst@tempg}
%    \end{macrocode}
% \end{macro}
%

% \subsection{Definition of the parameters}
%
%    \begin{macrocode}
\define@key[psset]{pst-fill}{boxfillsize}{%
  \def\pst@tempg{#1}\def\pst@temph{auto}%
  \ifx\pst@tempg\pst@temph
    \let\psk@boxfillsize\relax
  \else
    \pst@@boxfillsize#1(\z@,\z@)\@empty(\z@,\z@)(\@nil
  \fi}
\psset{boxfillsize={(-15cm,-15cm)(15cm,15cm)}}
\define@key[psset]{pst-fill}{boxfillcolor}{\pst@getcolor{#1}\psboxfillcolor}
\psset{boxfillcolor=black}% hv
\define@key[psset]{pst-fill}{boxfillangle}{\pst@getangle{#1}\psk@boxfillangle}
\psset{boxfillangle=0}
\define@key[psset]{pst-fill}{fillsepx}{%
  \pst@getlength{#1}\psk@fillsepx}
\define@key[psset]{pst-fill}{fillsepy}{%
  \pst@getlength{#1}\psk@fillsepy}
\define@key[psset]{pst-fill}{fillsep}{%
  \pst@getlength{#1}\psk@fillsepx%
  \let\psk@fillsepy\psk@fillsepx}
\psset{fillsep=2pt}

\ifx\PstTiling\@undefined
  \define@key[psset]{pst-fill}{fillcycle}{\pst@getint{#1}\psk@fillcycle}
  \psset{fillcycle=0}
\else
  \define@key[psset]{pst-fill}{fillangle}{\pst@getangle{#1}\psk@boxfillangle}
  \define@key[psset]{pst-fill}{fillsize}{%
      \def\pst@tempg{#1}\def\pst@temph{auto}%
      \ifx\pst@tempg\pst@temph\let\psk@boxfillsize\relax
      \else\pst@@boxfillsize#1(\z@,\z@)\@empty(\z@,\z@)(\@nil\fi}
  \psset{fillsep=0,fillsize=auto}
  \define@key[psset]{pst-fill}{fillcyclex}{\pst@getint{#1}\psk@fillcyclex}
  \define@key[psset]{pst-fill}{fillcycley}{\pst@getint{#1}\psk@fillcycley}
  \define@key[psset]{pst-fill}{fillcycle}{%
    \pst@getint{#1}\psk@fillcyclex\let\psk@fillcycley\psk@fillcyclex}
  \psset{fillcycle=0}
  \define@key[psset]{pst-fill}{fillmovex}{\pst@getlength{#1}\psk@fillmovex}
  \define@key[psset]{pst-fill}{fillmovey}{\pst@getlength{#1}\psk@fillmovey}
  \define@key[psset]{pst-fill}{fillmove}{%
      \pst@getlength{#1}\psk@fillmovex\let\psk@fillmovey\psk@fillmovex}
  \psset{fillmove=0pt}
  \define@key[psset]{pst-fill}{fillloopaddx}{\pst@getint{#1}\psk@fillloopaddx}
  \define@key[psset]{pst-fill}{fillloopaddy}{\pst@getint{#1}\psk@fillloopaddy}
  \define@key[psset]{pst-fill}{fillloopadd}{%
    \pst@getint{#1}\psk@fillloopaddx\let\psk@fillloopaddy\psk@fillloopaddx}
  \psset{fillloopadd=0}
%    \end{macrocode}
%
%    \begin{macrocode}
% For debugging (to debug, set PstDebug=1)
% we now use the one from pstricks to prevent a clash with package
% pstricks                        2004-06-22
%%    \define@key[psset]{pst-fill}{PstDebug}{\pst@getint{#1}\psk@PstDebug}
    \psset{PstDebug=0}
\fi
% DG addition end
%    \end{macrocode}

% \subsection{Definition of the fill box}
% \begin{macro}{psboxfill}
%    \begin{macrocode}
\newbox\pst@fillbox
\def\psboxfill{\pst@killglue\pst@makebox\psboxfill@i}
\def\psboxfill@i{\setbox\pst@fillbox\box\pst@hbox\ignorespaces}
%    \end{macrocode}
% \end{macro}
% \subsection{The main macros}
%
% \begin{macro}{psfs@boxfill}
%    \begin{macrocode}
\def\psfs@boxfill{%
  \ifvoid\pst@fillbox
    \@pstrickserr{Fill box is empty. Use \string\psboxfill\space first.}\@ehpa
  \else
    \ifx\psk@boxfillsize\relax \pst@AutoBoxFill
    \else\pst@ManualBoxFill\fi
  \fi}
%    \end{macrocode}
% \end{macro}
%
% \begin{macro}{pst@ManualBoxFill}
%    \begin{macrocode}
\def\pst@ManualBoxFill{%
  \leavevmode
  \begingroup
    \pst@FlushCode
    \begin@psclip
    \pstVerb{clip}%
    \expandafter\pst@AddFillBox\psk@boxfillsize
    \end@psclip
  \endgroup}
%    \end{macrocode}
% \end{macro}
%
% \begin{macro}{pst@FlushCode}
%    \begin{macrocode}
\def\pst@FlushCode{%
  \pst@Verb{%
    /mtrxc CM def
    CP CP T
    \tx@STV
    \psk@origin
    \psk@swapaxes
    \pst@newpath
    \pst@code
    mtrxc setmatrix
    moveto
    0 setgray}%
  \gdef\pst@code{}}
%    \end{macrocode}
% \end{macro}
%
% \begin{macro}{pst@AddFillBox}
%    \begin{macrocode}
\def\pst@AddFillBox#1 #2 #3 #4 {%
  \begingroup
    \setbox\pst@fillbox=\vbox{%
      \hbox{\unhcopy\pst@fillbox\kern\psk@fillsepx\p@}%
      \vskip\psk@fillsepy\p@}%
    \psk@boxfillsize
    \pst@cnta=\pst@dimc
    \advance\pst@cnta-\pst@dima
    \divide\pst@cnta\wd\pst@fillbox
    \pst@cntb=\pst@dimd
    \advance\pst@cntb-\pst@dimb
    \pst@dimd=\ht\pst@fillbox
    \divide\pst@cntb\pst@dimd
    \def\pst@tempa{%
      \pst@tempg
      \copy\pst@fillbox
      \advance\pst@cntc\@ne
      \ifnum\pst@cntc<\pst@cntd\expandafter\pst@tempa\fi}%
    \let\pst@tempg\relax
    \pst@cntc-\tw@
    \pst@cntd\pst@cnta
    \setbox\pst@fillbox=\hbox to \z@{%
      \kern\pst@dima
      \kern-\wd\pst@fillbox
      \pst@tempa
      \hss}%
    \pst@cntd\pst@cntb
%% DG modification begin - Dec. 11, 1997 - Patch 2
    \ifx\PstTiling\@undefined
      \ifnum\psk@fillcycle=\z@\pst@ManualFillCycle\fi
    \else
      \ifnum\psk@fillcyclex=\z@\pst@ManualFillCycle\fi
    \fi
%% DG modification end
    \global\setbox\pst@boxg=\vbox to\z@{%
      \offinterlineskip
      \vss
      \pst@tempa
      \vskip\pst@dimb}%
  \endgroup
  \setbox\pst@fillbox\box\pst@boxg
  \pst@rotate\psk@boxfillangle\pst@fillbox
  \box\pst@fillbox}
%    \end{macrocode}
% \end{macro}
%
% \begin{macro}{pst@ManualFillCycle}
%    \begin{macrocode}
\def\pst@ManualFillCycle{%
  \ifx\PstTiling\@undefined
    \pst@cntg=\psk@fillcycle
  \else
    \pst@cntg=\psk@fillcyclex
  \fi
  \pst@dimg=\wd\pst@fillbox
  \ifnum\pst@cntg=\z@
  \else
  \divide\pst@dimg\pst@cntg
  \fi
  \ifnum\pst@cntg<\z@\pst@cntg=-\pst@cntg\fi
  \advance\pst@cntg\m@ne
  \pst@cnth=\pst@cntg
  \def\pst@tempg{%
    \ifnum\pst@cnth<\pst@cntg\advance\pst@cnth\@ne\else\pst@cnth\z@\fi
    \moveright\pst@cnth\pst@dimg}}
%    \end{macrocode}
% \end{macro}
%
%% Auto box fill:        !! Fix dictionary
%
% \subsection{The PostScript subroutines}
%
%    \begin{macrocode}
%% DG addition begin - Apr. 8, 1997 and Dec. 1997 - Patch 2
\ifx\PstTiling\@undefined
\pst@def{AutoFillCycle}<%
  /c ED
  /n 0 def
  /s {
    /x x w c div n mul add def
    /n n c abs 1 sub lt { n 1 add } { 0 } ifelse def
  } def>

\pst@def{BoxFill}<%
  gsave
    gsave \tx@STV CM grestore dtransform CM idtransform
    abs /h ED abs /w ED
    pathbbox
    h div round 2 add cvi /y2 ED
    w div round 2 add cvi /x2 ED
    h div round 2 sub cvi /y1 ED
    w div round 2 sub cvi /x1 ED
    /y2 y2 y1 sub def
    /x2 x2 x1 sub def
    CP
    y1 h mul sub neg /y1 ED
    x1 w mul sub neg /x1 ED
    clip
    y2 {
      /x x1 def
      s
      x2 {
        save CP x y1
%% patch 4   hv --------------
        \ifx\VTeXversion\undefined
        \else
%%============ mv: 09-10-01 ??? this is likely to be a right change
        neg
%%============
        \fi
%% end patch 4
T moveto Box restore
        /x x w add def
      } repeat
      /y1 y1 h add def
    } repeat
    % Next line not useful... To see that, suppress clipping (DG)
    CP x y1 T moveto Box
  currentpoint currentfont grestore setfont moveto>
\else
%% DG modification begin - Apr. 8, 1997 and Nov. / Dec. 1997 - Patch 2
\pst@def{AutoFillCycleX}<%
  /cX ED
  /nX 0 def
  /CycleX {
    /x x w cX div nX mul add def
    /nX nX cX abs 1 sub lt { nX 1 add } { 0 } ifelse def
  } def>
\pst@def{AutoFillCycleY}<%
  /cY ED
  /mY 0 def
  /nY 0 def
  /CycleY {
    /y1 y1 h cY div mY mul sub def
    nY cY abs 1 sub lt { /nY nY 1 add def /mY 1 def }
                       { /nY 0 def        /mY cY abs 1 sub neg def } ifelse
  } def>

\pst@def{BoxFill}<%
  gsave
    gsave \tx@STV CM grestore dtransform CM idtransform
    abs /h ED abs /w ED
    pathbbox
    h div round 2 add cvi /y2 ED
    w div round 2 add cvi /x2 ED
    h div round 2 sub cvi /y1 ED
    w div round 2 sub cvi /x1 ED
    /CoefLoopX 0 def
    /CoefLoopY 0 def
    /CoefMoveX 0 def
    /CoefMoveY 0 def
    \psk@boxfillangle\space 0 ne {/CoefLoopX 8 def /CoefLoopY 8 def} if
    \psk@fillcyclex\space 0 ne {/CoefLoopX CoefLoopX 1 add def} if
    \psk@fillcycley\space 0 ne {/CoefLoopY CoefLoopY 1 add def} if
    \psk@fillmovex\space 0 ne
      {/CoefLoopX CoefLoopX 2 add def
       \psk@fillmovex\space 0 gt {/CoefMoveX CoefLoopX def}
                           {/CoefMoveX CoefLoopX neg def} ifelse} if
    \psk@fillmovey\space 0 ne
      {/CoefLoopY CoefLoopY 2 add def
       \psk@fillmovey\space 0 gt {/CoefMoveY CoefLoopY def}
                           {/CoefMoveY CoefLoopY neg def} ifelse} if
    \psk@fillsepx\space 0 ne {/CoefLoopX CoefLoopX 1 add def} if
    \psk@fillsepy\space 0 ne {/CoefLoopY CoefLoopY 1 add def} if
    /CoefLoopX CoefLoopX \psk@fillloopaddx\space add def
    /CoefLoopY CoefLoopY \psk@fillloopaddy\space add def
    /x2 x2 x1 sub 4 sub CoefLoopX 2 mul add def
    /y2 y2 y1 sub 4 sub CoefLoopY 2 mul add def
%% We must fix the origin of tiling, as it must not vary according other stuff
%% in the page!
    w x1 CoefLoopX add CoefMoveX add mul
      h y1 y2 add 1 sub CoefLoopY sub CoefMoveY sub mul moveto
    CP
    y1 h mul sub neg /y1 ED
    x1 w mul sub neg /x1 ED
%%  hv 2004-06-22   to prevent clash with pst-gr3d
%%    \psk@PstDebug 0 eq {clip} if
    \Pst@Debug 0 eq {clip} if
%% end hv
    \psk@fillmovex\space \psk@fillmovey
    gsave \tx@STV CM grestore dtransform CM idtransform
    /hmove ED /wmove ED
    /row 0 def
   y2 {
       /row row 1 add def
       /column 0 def
       /x x1 def
       CycleX
       save
       x2 {
          /column column 1 add def
          CycleY
          save CP x y1
%% patch 4   hv --------------
          \ifx\VTeXversion\undefined
          \else
%%============ mv: 09-10-01 ??? this is likely to be a right change
          neg
%%============
          \fi
  T moveto Box restore
          /x x w add def
          0 hmove translate
          } repeat
       restore
       /y1 y1 h add def
       wmove 0 translate
       } repeat
  currentpoint currentfont grestore setfont moveto>
\fi
%    \end{macrocode}

%    \begin{macrocode}
\def\pst@AutoBoxFill{%
  \leavevmode
  \begingroup
    \pst@stroke
    \pst@FlushCode
    \pst@Verb{\psk@boxfillangle\space \tx@RotBegin}%
    \pstVerb{\pst@dict /Box \pslbrace end}%
    \ifx\PstTiling\@undefined
    \else
      \ifx\pst@tempa\@undefined % Undefined for instance for \pscharpath
      \else\ifx\pst@tempa\@empty\else
        \def\pst@temph{0}%
        \ifx\pst@tempa\pst@temph
        \else
          \pstVerb{/TR {pop pop currentpoint translate \pst@tempa\space translate } def}%
        \fi
      \fi\fi
    \fi
    \hbox to \z@{\vbox to\z@{\vss\copy\pst@fillbox\vskip-\dp\pst@fillbox}\hss}%
    \ifx\PstTiling\@undefined
      \pstVerb{%
        tx@Dict begin \psrbrace def
        \ifnum\psk@fillcycle=\z@
          /s {} def
        \else
          \psk@fillcycle \tx@AutoFillCycle
        \fi
        \pst@number{\wd\pst@fillbox}%
        \psk@fillsepx\space add
        \pst@number{\ht\pst@fillbox}%
        \pst@number{\dp\pst@fillbox}%
        \psk@fillsepy\space add add
        \tx@BoxFill
        end}%
      \else
      \pstVerb{%
        tx@Dict begin \psrbrace def
        \ifnum\psk@fillcyclex=\z@
          /CycleX {} def
        \else
          \psk@fillcyclex\space \tx@AutoFillCycleX
        \fi
        \ifnum\psk@fillcycley=\z@
          /CycleY {} def
        \else
          \psk@fillcycley\space \tx@AutoFillCycleY
        \fi
        \pst@number{\wd\pst@fillbox}%
        \psk@fillsepx\space add
        \pst@number{\ht\pst@fillbox}%
        \pst@number{\dp\pst@fillbox}%
        \psk@fillsepy\space add add
        \tx@BoxFill
        end}%
    \fi
    \pst@Verb{\tx@RotEnd}%
  \endgroup}
%    \end{macrocode}
% \subsection{Closing}
%
%   Catcodes restoration.
%
%    \begin{macrocode}
\catcode`\@=\PstAtCode\relax
%    \end{macrocode}
%
%    \begin{macrocode}
%</pst-fill>
%    \end{macrocode}
%
% \Finale
%
\endinput
%%
%% End of file `pst-fill.dtx'

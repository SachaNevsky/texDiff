% \iffalse meta-comment
%
% File: footnoterange.dtx
% Version: 2019/11/24 v1.0c
%
% Copyright (C) 2012 - 2019 by
%    H.-Martin M"unch <Martin dot Muench at Uni-Bonn dot de>
% Portions of code copyrighted by other people as marked;
% lines marked "FMi": code by Frank Mittelbach, thanks!
%
% This work may be distributed and/or modified under the
% conditions of the LaTeX Project Public License, either
% version 1.3c of this license or (at your option) any later
% version. This version of this license is in
%    https://www.latex-project.org/lppl/lppl-1-3c.txt
% and the latest version of this license is in
%    https://www.latex-project.org/lppl.txt
% and version 1.3c or later is part of all distributions of
% LaTeX version 2005/12/01 or later.
%
% This work has the LPPL maintenance status "maintained".
%
% The Current Maintainer of this work is H.-Martin Muench.
%
% This work consists of the main source file footnoterange.dtx,
% the README, and the derived files
%    footnoterange.sty, footnoterange.pdf,
%    footnoterange.ins, footnoterange.drv,
%    footnoterange-example.tex, footnoterange-example.pdf.
%
% 'footnoterange' is available on CTAN:
% https://www.ctan.org/pkg/footnoterange
%
% Also a TDS.ZIP file is provided that contains all the files
% already sorted in a TDS tree:
% http://mirror.ctan.org/install/macros/latex/contrib/footnoterange.tds.zip
%
% Unpacking:
%    (a) If footnoterange.ins is present:
%           tex footnoterange.ins
%    (b) Without footnoterange.ins:
%           tex footnoterange.dtx
%    (c) If you insist on using LaTeX
%           latex \let\install=y% \iffalse meta-comment
%
% File: footnoterange.dtx
% Version: 2019/11/24 v1.0c
%
% Copyright (C) 2012 - 2019 by
%    H.-Martin M"unch <Martin dot Muench at Uni-Bonn dot de>
% Portions of code copyrighted by other people as marked;
% lines marked "FMi": code by Frank Mittelbach, thanks!
%
% This work may be distributed and/or modified under the
% conditions of the LaTeX Project Public License, either
% version 1.3c of this license or (at your option) any later
% version. This version of this license is in
%    https://www.latex-project.org/lppl/lppl-1-3c.txt
% and the latest version of this license is in
%    https://www.latex-project.org/lppl.txt
% and version 1.3c or later is part of all distributions of
% LaTeX version 2005/12/01 or later.
%
% This work has the LPPL maintenance status "maintained".
%
% The Current Maintainer of this work is H.-Martin Muench.
%
% This work consists of the main source file footnoterange.dtx,
% the README, and the derived files
%    footnoterange.sty, footnoterange.pdf,
%    footnoterange.ins, footnoterange.drv,
%    footnoterange-example.tex, footnoterange-example.pdf.
%
% 'footnoterange' is available on CTAN:
% https://www.ctan.org/pkg/footnoterange
%
% Also a TDS.ZIP file is provided that contains all the files
% already sorted in a TDS tree:
% http://mirror.ctan.org/install/macros/latex/contrib/footnoterange.tds.zip
%
% Unpacking:
%    (a) If footnoterange.ins is present:
%           tex footnoterange.ins
%    (b) Without footnoterange.ins:
%           tex footnoterange.dtx
%    (c) If you insist on using LaTeX
%           latex \let\install=y% \iffalse meta-comment
%
% File: footnoterange.dtx
% Version: 2019/11/24 v1.0c
%
% Copyright (C) 2012 - 2019 by
%    H.-Martin M"unch <Martin dot Muench at Uni-Bonn dot de>
% Portions of code copyrighted by other people as marked;
% lines marked "FMi": code by Frank Mittelbach, thanks!
%
% This work may be distributed and/or modified under the
% conditions of the LaTeX Project Public License, either
% version 1.3c of this license or (at your option) any later
% version. This version of this license is in
%    https://www.latex-project.org/lppl/lppl-1-3c.txt
% and the latest version of this license is in
%    https://www.latex-project.org/lppl.txt
% and version 1.3c or later is part of all distributions of
% LaTeX version 2005/12/01 or later.
%
% This work has the LPPL maintenance status "maintained".
%
% The Current Maintainer of this work is H.-Martin Muench.
%
% This work consists of the main source file footnoterange.dtx,
% the README, and the derived files
%    footnoterange.sty, footnoterange.pdf,
%    footnoterange.ins, footnoterange.drv,
%    footnoterange-example.tex, footnoterange-example.pdf.
%
% 'footnoterange' is available on CTAN:
% https://www.ctan.org/pkg/footnoterange
%
% Also a TDS.ZIP file is provided that contains all the files
% already sorted in a TDS tree:
% http://mirror.ctan.org/install/macros/latex/contrib/footnoterange.tds.zip
%
% Unpacking:
%    (a) If footnoterange.ins is present:
%           tex footnoterange.ins
%    (b) Without footnoterange.ins:
%           tex footnoterange.dtx
%    (c) If you insist on using LaTeX
%           latex \let\install=y% \iffalse meta-comment
%
% File: footnoterange.dtx
% Version: 2019/11/24 v1.0c
%
% Copyright (C) 2012 - 2019 by
%    H.-Martin M"unch <Martin dot Muench at Uni-Bonn dot de>
% Portions of code copyrighted by other people as marked;
% lines marked "FMi": code by Frank Mittelbach, thanks!
%
% This work may be distributed and/or modified under the
% conditions of the LaTeX Project Public License, either
% version 1.3c of this license or (at your option) any later
% version. This version of this license is in
%    https://www.latex-project.org/lppl/lppl-1-3c.txt
% and the latest version of this license is in
%    https://www.latex-project.org/lppl.txt
% and version 1.3c or later is part of all distributions of
% LaTeX version 2005/12/01 or later.
%
% This work has the LPPL maintenance status "maintained".
%
% The Current Maintainer of this work is H.-Martin Muench.
%
% This work consists of the main source file footnoterange.dtx,
% the README, and the derived files
%    footnoterange.sty, footnoterange.pdf,
%    footnoterange.ins, footnoterange.drv,
%    footnoterange-example.tex, footnoterange-example.pdf.
%
% 'footnoterange' is available on CTAN:
% https://www.ctan.org/pkg/footnoterange
%
% Also a TDS.ZIP file is provided that contains all the files
% already sorted in a TDS tree:
% http://mirror.ctan.org/install/macros/latex/contrib/footnoterange.tds.zip
%
% Unpacking:
%    (a) If footnoterange.ins is present:
%           tex footnoterange.ins
%    (b) Without footnoterange.ins:
%           tex footnoterange.dtx
%    (c) If you insist on using LaTeX
%           latex \let\install=y\input{footnoterange.dtx}
%        (quote the arguments according to the demands of your shell)
%
% Documentation:
%    (a) If footnoterange.drv is present:
%           (pdf)latex footnoterange.drv
%           makeindex -s gind.ist footnoterange.idx
%           (pdf)latex footnoterange.drv
%           makeindex -s gind.ist footnoterange.idx
%           (pdf)latex footnoterange.drv
%    (b) Without footnoterange.drv:
%           (pdf)latex footnoterange.dtx
%           makeindex -s gind.ist footnoterange.idx
%           (pdf)latex footnoterange.dtx
%           makeindex -s gind.ist footnoterange.idx
%           (pdf)latex footnoterange.dtx
%
%    The class ltxdoc loads the configuration file ltxdoc.cfg
%    if available. Here you can specify further options, e.g.
%    use DIN A4 as paper format:
%       \PassOptionsToClass{a4paper}{article}
%
% Installation:
%    TDS:tex/latex/footnoterange/footnoterange.sty
%    TDS:doc/latex/footnoterange/footnoterange.pdf
%    TDS:doc/latex/footnoterange/footnoterange-example.tex
%    TDS:doc/latex/footnoterange/footnoterange-example.pdf
%    TDS:source/latex/footnoterange/footnoterange.dtx
%
%<*ignore>
\begingroup
  \catcode123=1 %
  \catcode125=2 %
  \def\x{LaTeX2e}%
\expandafter\endgroup
\ifcase 0\ifx\install y1\fi\expandafter
         \ifx\csname processbatchFile\endcsname\relax\else1\fi
         \ifx\fmtname\x\else 1\fi\relax
\else\csname fi\endcsname
%</ignore>
%<*install>
\input docstrip.tex
\Msg{*********************************************************************************}
\Msg{* Installation}
\Msg{* Package: footnoterange 2019/11/24 v1.0c References to ranges of footnotes (HMM)}
\Msg{*********************************************************************************}

\keepsilent
\askforoverwritefalse

\let\MetaPrefix\relax
\preamble

This is a generated file.

Project: footnoterange
Version: 2019/11/24 v1.0c

Copyright (C) 2012 - 2019 by
    H.-Martin M"unch <Martin dot Muench at Uni-Bonn dot de>
Portions of code copyrighted by other people as marked;
lines marked "FMi": code by Frank Mittelbach, thanks!

The usual disclaimer applies:
If it doesn't work right that's your problem.
(Nevertheless, please send an e-mail to the maintainer
 when you find an error in this package.)

This work may be distributed and/or modified under the
conditions of the LaTeX Project Public License, either
version 1.3c of this license or (at your option) any later
version. This version of this license is in
   https://www.latex-project.org/lppl/lppl-1-3c.txt
and the latest version of this license is in
   https://www.latex-project.org/lppl.txt
and version 1.3c or later is part of all distributions of
LaTeX version 2005/12/01 or later.

This work has the LPPL maintenance status "maintained".

The Current Maintainer of this work is H.-Martin Muench.

This work consists of the main source file footnoterange.dtx,
the README, and the derived files
   footnoterange.sty, footnoterange.pdf,
   footnoterange.ins, footnoterange.drv,
   footnoterange-example.tex, footnoterange-example.pdf.

In memoriam
 Claudia Simone Barth + 1996/01/30
 Tommy Muench + 2014/01/02
 Hans-Klaus Muench + 2014/08/24

\endpreamble
\let\MetaPrefix\DoubleperCent

\generate{%
  \file{footnoterange.ins}{\from{footnoterange.dtx}{install}}%
  \file{footnoterange.drv}{\from{footnoterange.dtx}{driver}}%
  \usedir{tex/latex/footnoterange}%
  \file{footnoterange.sty}{\from{footnoterange.dtx}{package}}%
  \usedir{doc/latex/footnoterange}%
  \file{footnoterange-example.tex}{\from{footnoterange.dtx}{example}}%
}

\catcode32=13\relax% active space
\let =\space%
\Msg{************************************************************************}
\Msg{*}
\Msg{* To finish the installation you have to move the following}
\Msg{* file into a directory searched by TeX:}
\Msg{*}
\Msg{* footnoterange.sty}
\Msg{*}
\Msg{* To produce the documentation run the file `footnoterange.drv'}
\Msg{* through (pdf)LaTeX, e.g.}
\Msg{*  pdflatex footnoterange.drv}
\Msg{*  makeindex -s gind.ist footnoterange.idx}
\Msg{*  pdflatex footnoterange.drv}
\Msg{*  makeindex -s gind.ist footnoterange.idx}
\Msg{*  pdflatex footnoterange.drv}
\Msg{*}
\Msg{* At least three runs are necessary e.g. to get the}
\Msg{*  references right!}
\Msg{*}
\Msg{* Happy TeXing!}
\Msg{*}
\Msg{************************************************************************}

\endbatchfile
%</install>
%<*ignore>
\fi
%</ignore>
%
% \section{The documentation driver file}
%
% The next bit of code contains the documentation driver file for
% \TeX , i.\,e., the file that will produce the documentation you
% are currently reading. It will be extracted from this file by the
% \verb|docstrip| programme. That is, run \LaTeX{} on \verb|docstrip|
% and specify the \verb|driver| option when \verb|docstrip|
% asks for options.
%
%    \begin{macrocode}
%<*driver>
\NeedsTeXFormat{LaTeX2e}[2015/01/01]
\ProvidesFile{footnoterange.drv}[2019/11/24 v1.0c
            References to ranges of footnotes (HMM)]
\documentclass{ltxdoc}[2015/03/26]% v2.0w
\usepackage{holtxdoc}[2012/03/21]%  v0.24
%% footnoterange may work with earlier versions of LaTeX2e and those
%% class and package, but this was not tested.
%% Please consider updating your LaTeX, class, and package
%% to the most recent version (if they are not already the most
%% recent version).
\hypersetup{%
 pdfsubject={References to ranges of footnotes (HMM)},%
 pdfkeywords={LaTeX, footnoterange, footnotes, footnote, reference, H.-Martin Muench},%
 pdfencoding=auto,%
 pdflang={en},%
 breaklinks=true,%
 linktoc=all,%
 pdfstartview=FitH,%
 pdfpagelayout=OneColumn,%
 bookmarksnumbered=true,%
 bookmarksopen=true,%
 bookmarksopenlevel=3,%
 pdfmenubar=true,%
 pdftoolbar=true,%
 pdfwindowui=true,%
 pdfnewwindow=true%
}
\CodelineIndex
\hyphenation{printing docu-ment}
\gdef\unit#1{\mathord{\thinspace\mathrm{#1}}}
\begin{document}
  \DocInput{footnoterange.dtx}%
\end{document}
%</driver>
%    \end{macrocode}
%
% \fi
%
% \CheckSum{321}
%
% \CharacterTable
%  {Upper-case    \A\B\C\D\E\F\G\H\I\J\K\L\M\N\O\P\Q\R\S\T\U\V\W\X\Y\Z
%   Lower-case    \a\b\c\d\e\f\g\h\i\j\k\l\m\n\o\p\q\r\s\t\u\v\w\x\y\z
%   Digits        \0\1\2\3\4\5\6\7\8\9
%   Exclamation   \!     Double quote  \"     Hash (number) \#
%   Dollar        \$     Percent       \%     Ampersand     \&
%   Acute accent  \'     Left paren    \(     Right paren   \)
%   Asterisk      \*     Plus          \+     Comma         \,
%   Minus         \-     Point         \.     Solidus       \/
%   Colon         \:     Semicolon     \;     Less than     \<
%   Equals        \=     Greater than  \>     Question mark \?
%   Commercial at \@     Left bracket  \[     Backslash     \\
%   Right bracket \]     Circumflex    \^     Underscore    \_
%   Grave accent  \`     Left brace    \{     Vertical bar  \|
%   Right brace   \}     Tilde         \~}
%
% \GetFileInfo{footnoterange.drv}
%
% \begingroup
%   \def\x{\#,\$,\^,\_,\~,\ ,\&,\{,\},\%}%
%   \makeatletter
%   \@onelevel@sanitize\x
% \expandafter\endgroup
% \expandafter\DoNotIndex\expandafter{\x}
% \expandafter\DoNotIndex\expandafter{\string\ }
% \begingroup
%   \makeatletter
%     \lccode`9=32\relax
%     \lowercase{%^^A
%       \edef\x{\noexpand\DoNotIndex{\@backslashchar9}}%^^A
%     }%^^A
%   \expandafter\endgroup\x
% \DoNotIndex{\,,\\}
% \DoNotIndex{\documentclass,\usepackage,\ProvidesPackage,\begin,\end}
% \DoNotIndex{\MessageBreak}
% \DoNotIndex{\NeedsTeXFormat,\DoNotIndex,\verb}
% \DoNotIndex{\def,\edef,\gdef,\global,\xdef}
% \DoNotIndex{\ifx,\listfiles,\mathord,\mathrm}
% \DoNotIndex{\bigskip,\space,\thinspace,\ldots}
% \DoNotIndex{\indent,\noindent,\newline,\linebreak,\pagebreak,\newpage}
% \DoNotIndex{\textbf,\textit,\textsf,\textsc,\texttt,\textquotedblleft,\textquotedblright}
% \DoNotIndex{\plainTeX,\TeX,\LaTeX,\pdfLaTeX}
% \DoNotIndex{\section,\pagenumbering,\smallskip}
% \DoNotIndex{\@ifpackageloaded}
% \DoNotIndex{\arabic, \Alph, \alph, \alphalph, \Roman, \roman, \fnsymbol}
%
% \title{The \xpackage{footnoterange} package}
% \date{2019/11/24 v1.0c}
% \author{H.-Martin M\"{u}nch\\\xemail{Martin.Muench at Uni-Bonn.de}}
%
% \maketitle
%
% \begin{abstract}
% \noindent This package provides the environments |footnoterange| and
% |footnoterange*|. Multiple footnotes inside these environments
% are not referenced as e.\,g.
% \makeatletter%
% \hbox{\textquotedblleft \@textsuperscript {\normalfont 1 2 3}\textquotedblright}
% but as \hbox{\textquotedblleft \@textsuperscript {\normalfont 1--3}\textquotedblright ,\ }
% \makeatother%
% i.\,e. the range (from first to last referred footnote at that place) is given.
% Depending on loading of \xpackage{hyperref} and use of its |hyperfootnotes|-option
% the references are hyperlinked. (References to footnotes in |footnoterange*|
% are never hyperlinked.)
% \end{abstract}
%
% \bigskip
%
% \noindent Disclaimer for web links: The author is not responsible for any contents
% referred to in this work unless if having full knowledge of illegal contents.
% If any damage occurs by the use of information presented there, only the
% author of the respective pages might be liable, not the one who has referred
% to these pages.
%
% \noindent Save per page about $200\unit{ml}$ water,
% $2\unit{g}$ CO$_{2}$ and $2\unit{g}$ wood:
% Therefore please print only if this is really necessary.
%
% \medskip
%
% \tableofcontents
%
%\newpage
%
% \section{Introduction}
% \indent This \LaTeXe{} package provides the environments |footnoterange| and\newline
% |footnoterange*|. Multiple footnotes inside these environments are not referenced
% as e.\,g.
% \makeatletter%
% \hbox{\textquotedblleft \@textsuperscript {\normalfont 1 2 3}\textquotedblright}
% but as \hbox{\textquotedblleft \@textsuperscript {\normalfont 1--3}\textquotedblright ,\ }
% \makeatother%
% i.\,e. the range (from first to last
% referred footnote at that place) is given. Depending on loading of
% \xpackage{hyperref} and use of its |hyperfootnotes|-option the references are
% hyperlinked. References to footnotes in |footnoterange*| are never hyperlinked.\newline
% |\footnotemark\footnotetext| instead of |\footnote| is not supported inside
% |footnoterange| and |footnoterange*| environments.
%
% \bigskip
%
% \section{Usage}
%
% \indent Just load the package placing
% \begin{quote}
%   |\usepackage{footnoterange}|
% \end{quote}
% \noindent in the preamble of your \LaTeXe{} source file. There are no options to
% be used. The package should be loaded after \xpackage{hyperref} (if that package is used)
% and after any other package redefining footnotes. Usage of the |footnoterange| (or
% |footnoterange*|) environment is simple: Just encapsulate the footnotes, which shall
% be referred as range instead of referencing as single ones (see the example file).
%
% About hyperlinks:\newline
% Footnote-references in a |footnoterange*| environment are never hyperlinked.\newline
% Footnote-references in a |footnoterange| environment may be hyperlinked:
% \begin{itemize}
%  \item When the \xpackage{hyperref} package is not loaded, they are not
%         hyperlinked (naturally).
%  \item When the \xpackage{hyperref} package is loaded, it depends on the
%         |hyperfootnotes| option of the \xpackage{hyperref} package:
%  \begin{itemize}
%   \item With |hyperfootnotes=true|, the references are hyperlinked.
%   \item With |hyperfootnotes=false|, the references are not hyperlinked.
%  \end{itemize}
% \end{itemize}
% |\footnote|s inside of the |footnoterange|/{}|footnoterange*| environment
% can use the optional parameter, i.\,e. |\footnote[42]{text}| even if this is the
% third footnote. When giving ranges as reference, this does not seem to be
% such a good idea, if this is not done consistently (i.\,e. footnotes
% |[42], [43], [44]| resulting in 42--44 is OK,
% but |[42]| (original~2), unchanged~3, unchanged~4 resulting in
% 42--4 is confusing). A warning about these changed footnotemarks is written
% into the \xfile{.log}-file.
%
% When there is only a single footnote in a |footnoterange| environment,
% this does not produce some reference like %
% \makeatletter%
% \hbox{\textquotedblleft \@textsuperscript {\normalfont 7--7}\textquotedblright{},\ }%
% \makeatother%
% but only one footnotemark is used automatically. Because it is strange to use
% a |footnoterange| environment without a range of footnotes, an info is written
% into the \xfile{.log}-file about this.
%
% The usage of other footnote-number-representations
% (e.\,g.~\verb|\Roman|, \verb|\roman|, \verb|\Alph|, \verb|\alph|,
% \verb|\fnsymbol|) is also possible, but two things must be taken
% into account: The number of footnote references is restricted (for
% example with \verb|\Alph| only references A to Z are possible), which
% can be fixed e.\,g. with the \url{https://ctan.org/pkg/alphalph} package
% (but not automatically), and references to footnote-symbol-ranges (\verb|\fnsymbol|)
% are probably not very clear.
%
% \subsection*{Options}
% \indent The \xpackage{footnoterange} package takes no options.
%
% \section{Alternatives\label{sec:Alternatives}}
%
% The \xpackage{fnpct} (\url{https://www.ctan.org/pkg/fnpct}) package allows to set
% multiple footnotes (|\multfootnote|), suppress a footnotemark \hbox{(|*|),}
% and customize the multi-footnote-separator \hbox{(|\setfnpct{mult-fn-sep={|
% \emph{something} |}}|)}.
% Combining these with stepping of the footnotecounter gives a footnoterange:
% 
% \begin{verbatim}
% \documentclass{article}
% \usepackage{hyperref}
% \usepackage{fnpct}
% \begin{document}
% The three little pigs built their houses
% out of straw\footnote{not to be confused with hay},
% sticks\footnote{or lumber according to some sources}
% and bricks%
% \makeatletter%
% \setfnpct{mult-fn-sep={--}}%
% \multfootnote{probably fired clay bricks\addtocounter\@mpfn{+1};%
% *or something else;%
% what do I know}. %
% \makeatother%
% \end{document}
% \end{verbatim}
% 
% \noindent If you programmed or found another alternative, which is available at
% \url{https://www.CTAN.org/}, please send an e-mail to me with the name,
% location at CTAN, and a short notice, and I will probably list it here.
%
% \section{Example}
%
%    \begin{macrocode}
%<*example>
\documentclass[british]{article}[2014/09/29]% v1.4h
%%%%%%%%%%%%%%%%%%%%%%%%%%%%%%%%%%%%%%%%%%%%%%%%%%%%%%%%%%%%%%%%%%%%%
\usepackage[%
 extension=pdf,%
 plainpages=false,%
 pdfpagelabels=true,%
 hyperindex=false,%
 pdflang={en},%
 pdftitle={footnoterange package example},%
 pdfauthor={H.-Martin Muench},%
 pdfsubject={Example for the footnoterange package},%
 pdfkeywords={LaTeX, footnoterange, H.-Martin Muench},%
 pdfview=FitH,%
 pdfstartview=FitH,%
 pdfpagelayout=OneColumn,%
 hyperfootnotes=true%
]{hyperref}[2012/11/06]% v6.83m

\usepackage{footnoterange}[2019/11/24]% v1.0c

\gdef\unit#1{\mathord{\thinspace\mathrm{#1}}}%
\listfiles
\begin{document}
\pagenumbering{arabic}
\section*{Example for footnoterange}

This example demonstrates the use of package\newline
\textsf{footnoterange}, v1.0c as of 2019/11/24 (HMM).\newline
The package does not use options.\newline
\verb|footnoterange-example.pdf| can be created by\newline
\verb|pdflatex footnoterange-example.tex|,\newline
and \verb|footnoterange-example.tex| can be created by\newline
\verb|tex footnoterange.dtx|.\newline
\verb|footnoterange.dtx| is at
\url{https://www.ctan.org/pkg/footnoterange}.\newline

\noindent For more details please see the documentation!\newline

\noindent Save per page about $200\unit{ml}$ water,
$2\unit{g}$ CO$_{2}$ and $2\unit{g}$ wood:\newline
Therefore please print only if this is really necessary.\newline

This\footnote{Lorem} text bears a multiplicity of footnotes.
Because the \verb|hyperref| package is used with option
\verb|hyperfootnotes=true|, the footnotes
\begin{footnoterange}
\footnote{ipsum}%
\footnote{dolor}%
\footnote{sit}%
\footnote{amet,}%
\footnote{consectetur}%
\footnote{adipisicing}%
\footnote{elit,}%
\footnote{sed}%
\end{footnoterange}
are hyperlinked.\footnote{do eiusmod\ldots} Using
\verb|hyperfootnotes=false| or not using \verb|hyperref|
will remove the hyperlinks to the footnotes.\medskip

This text also bears a multiplicity of footnotes,
but due to the use of the starred
\begin{footnoterange*}
\footnote{Ut purus elit,}%
\footnote{vestibulum ut,}%
\footnote{placerat ac,}%
\footnote{adipiscing vitae,}%
\footnote{felis.}%
\footnote{Curabitur}%
\footnote{dictum}%
\footnote{gravida}%
\end{footnoterange*}
form of the environment they are not
hyperlinked.\footnote{mauris\ldots}\medskip

The use of the optional parameter for footnotes
\begin{footnoterange}
\footnote[101]{Nam arcu libero,}%
\footnote[102]{nonummy eget,}%
\footnote[103]{consectetuer id,}%
\footnote[104]{vulputate a,}%
\footnote[105]{magna.}
\end{footnoterange}
is also possible, but should be used consistently.

\pagebreak

When there is only one footnote
\begin{footnoterange}
\footnote{a single footnote}
\end{footnoterange}
in the \verb|footnoterange| environment, only one
footnotemark is used and an info is written into the
\verb|log|-file.\medskip

The usage of other footnote-number-representations
(e.\,g.~\verb|\Roman|, \verb|\roman|, \verb|\Alph|, \verb|\alph|,
\verb|\fnsymbol|) is also possible, but two things must be taken
into account: The number of footnote references is restricted
(for example with \verb|\Alph| only references A to Z are possible),
which can be fixed e.\,g. with the
\href{https://www.ctan.org/pkg/alphalph}{\texttt{alphalph} package},
and references to footnote-symbol-ranges (\verb|\fnsymbol|)
are probably not very clear.

\end{document}
%</example>
%    \end{macrocode}
%
% \StopEventually{}
%
% \section{The implementation}
%
% We start off by checking that we are loading into \LaTeXe{} and
% announcing the name and version of this package.
%
%    \begin{macrocode}
%<*package>
%    \end{macrocode}
%
%    \begin{macrocode}
\NeedsTeXFormat{LaTeX2e}[2011/06/27]
\ProvidesPackage{footnoterange}[2019/11/24 v1.0c
            References to ranges of footnotes (HMM)]
%    \end{macrocode}
%
% A short description of the \xpackage{footnoterange} package:
%
%    \begin{macrocode}
%% Provides the footnoterange and footnoterange* environments to
%% refer to ranges of footnotes.
%    \end{macrocode}
%
% We need the \xpackage{ltxcmds} package:
%
%    \begin{macrocode}
\RequirePackage{ltxcmds}[2011/11/09]% v1.22; for "\ltx@ifpackageloaded"
\RequirePackage{letltxmacro}[2010/09/02]% v1.4; for "\GlobalLetLtxMacro"

%    \end{macrocode}
%
% We use |\ifHy@hyperfootnotes ... \else ... \fi|, but when the \xpackage{hyperref}
% package has not been loaded, |\ifHy@hyperfootnotes| is not defined, and \LaTeX
% relates the |\else| and/or |\fi| (wrongly) to other |\if|s or groups.
% Therefore the |\ifHy@hyperfootnotes| are placed in their own commands,
% which are not looked at when the \xpackage{hyperref} package has not been loaded.
%
% \pagebreak
%
% \begin{macro}{\fnr@ifhyperrfn}
%    \begin{macrocode}
\newcommand{\fnr@ifhyperrfn}{%
  \ifHy@hyperfootnotes% option hyperfootnotes=true
    \addtocounter{Hfootnote}{+1}%
    % Code from the hyperref package
      \global\let\Hy@saved@currentHref\@currentHref%
      \hyper@makecurrent{Hfootnote}%
      \global\let\Hy@footnote@currentHref\@currentHref%
      \global\let\@currentHref\Hy@saved@currentHref%
    % End of code from the hyperref package
  \fi%
}

%    \end{macrocode}
% \end{macro}
%
% \begin{macro}{\fnr@rangefootnote}
% The |\fnr@rangefootnote| is the internal replacement for the original\newline
% |\footnote| command inside of a |footnoterange| environment.
%
%    \begin{macrocode}
\newcommand{\fnr@rangefootnote}[2][\empty]{%
%    \end{macrocode}
%
% |#1| contains the optional argument, which is predefined as |\empty|.
%
%    \begin{macrocode}
  \nolinebreak%
  % no \footnotemark here
  \addtocounter\@mpfn{+1}%
  \ltx@ifpackageloaded{hyperref}{\fnr@ifhyperrfn}{\relax}%
%    \end{macrocode}
%
% See the explanation before the declaration of the |\fnr@ifhyperrfn| command!
%
%    \begin{macrocode}
  \xdef\fnr@opt{#1}% contains the optional argument
  \xdef\fnr@arabic{\arabic\@mpfn}%
  \protected@edef\fnr@formated{\thefootnote}%
%    \end{macrocode}
%
% The |\footnotemark| is not set (yet), but the |\footnotetext|,
% and it gets a label.
%
%    \begin{macrocode}
  \ifx\fnr@opt\empty%
    \footnotetext{\label{fnr:\fnr@arabic}#2}%
  \else%
    \PackageWarning{footnoterange}{%
      Custom footnote-numbers used in a footnoterange\MessageBreak%
      environment: Footnote with natural number\MessageBreak%
      \ifx\fnr@formated\fnr@arabic%
        \fnr@arabic\space%
      \else%
        \space\fnr@arabic\space (\fnr@formated )%
      \fi%
      \MessageBreak%
      was renumbered to \fnr@opt\MessageBreak%
     }%
    \ltx@ifpackageloaded{hyperref}{% hyperref loaded
      \footnotetext[#1]{\phantomsection\label{fnr:\fnr@arabic}#2}%
     }{% hyperref not loaded
      \footnotetext[#1]{\label{fnr:\fnr@arabic}#2}%
     }%
  \fi%
}

%    \end{macrocode}
% \end{macro}
%
%    \begin{macrocode}

%    \end{macrocode}
% \pagebreak
%
% |\ifHy@hyperfootnotes| \textquotedblleft encapsulation\textquotedblright{} again:
%
% \begin{macro}{\fnr@ifhyperfnrA}
%    \begin{macrocode}
\newcommand{\fnr@ifhyperfnrA}{%
  \ifHy@hyperfootnotes% option hyperfootnotes=true
    \hbox{\hyperref[fnr:\fnr@first]{%
          \@textsuperscript{\normalfont \ref*{fnr:\fnr@first}}}}%
  \else% option hyperfootnotes=false
    \hbox{\@textsuperscript{\normalfont \ref*{fnr:\fnr@first}}}%
  \fi%
}

%    \end{macrocode}
% \end{macro}
%
% \begin{macro}{\fnr@ifhyperfnrN}
%    \begin{macrocode}
\newcommand{\fnr@ifhyperfnrN}{%
  \ifHy@hyperfootnotes% option hyperfootnotes=true
    \hbox{\hyperref[fnr:\fnr@first]{%
            \@textsuperscript{\normalfont \ref*{fnr:\fnr@first}}}%
          \@textsuperscript{\normalfont --}%
          \hyperref[fnr:\arabic\@mpfn]{%
            \@textsuperscript{\normalfont \ref*{fnr:\arabic\@mpfn}}}}%
  \else% option hyperfootnotes=false
    \hbox{\@textsuperscript{\normalfont \ref*{fnr:\fnr@first}%
                                        --\ref*{fnr:\arabic\@mpfn}}}%
  \fi%
}

%    \end{macrocode}
% \end{macro}
%
% \begin{environment}{footnoterange}
% The |footnoterange| environment remembers the next footnote number as\newline
% |\fnr@first| and the current |\footnote| definition as |\fnr@origfn|
% and uses |\fnr@rangefootnote| instead.
%
%    \begin{macrocode}
\newenvironment{footnoterange}{%
  % at begin of the footnoterange environment
  \ifhmode \unskip \fi% FMi
  \addtocounter\@mpfn{+1}%
  \xdef\fnr@first{\arabic\@mpfn}%
  \addtocounter\@mpfn{-1}%
  \let\fnr@origfn\footnote%
  \let\footnote\fnr@rangefootnote%
  \GlobalLetLtxMacro{\fnr@origfnmark}{\footnotemark}%
  \GlobalLetLtxMacro{\footnotemark}{\fnr@rangefootnotemark}%
  \ignorespaces% FMi
%    \end{macrocode}
%
% End of the environment:
%
%    \begin{macrocode}
}{% at end of the footnoterange environment
  \ifhmode% FMi 
    \unskip% FMi
%    \end{macrocode}
%
% Remember the spacefactor for restoring after the environment has ended
% (only in |hmode|!). \% FMi
%
%    \begin{macrocode}
    \xdef\fnr@aftergroup{\spacefactor\the\spacefactor\relax}% FMi
  \else% FMi
%    \end{macrocode}
%
% Otherwise do something harmless. \% FMi
%
%    \begin{macrocode}
    \xdef\fnr@aftergroup{\relax}%
  \fi% FMi
%    \end{macrocode}
%
% At the end of the environment the previous definition of |\footnote| is restored
%
%    \begin{macrocode}
  \let\footnote\fnr@origfn%
  \GlobalLetLtxMacro{\footnotemark}{\fnr@origfnmark}%
%    \end{macrocode}
%
% and the references are placed:
%
%    \begin{macrocode}
  \ifx\fnr@first\fnr@arabic% There is only one reference in the environment.
    \PackageInfo{footnoterange}{%
      Environment footnoterange used,\MessageBreak%
      but only one footnote (number \fnr@first ) is contained%
      \MessageBreak%
     }%
    \ltx@ifpackageloaded{hyperref}{\fnr@ifhyperfnrA% hyperref package loaded
    }{% hyperref package not loaded
      \hbox{\@textsuperscript{\normalfont \ref{fnr:\fnr@first}}}%
     }%
  \else%
    \ltx@ifpackageloaded{hyperref}{\fnr@ifhyperfnrN% hyperref package loaded
    }{% hyperref package not loaded
      \hbox{\@textsuperscript{\normalfont \ref{fnr:\fnr@first}%
                                        --\ref{fnr:\arabic\@mpfn}}}%
     }%
  \fi%
%    \end{macrocode}
%
% Restore the spacefactor after the environment has ended. \% FMi
%
%    \begin{macrocode}
  \aftergroup\fnr@aftergroup% FMi
}

%    \end{macrocode}
% \end{environment}
%
% \begin{environment}{footnoterange*}
% The |footnoterange*| environment does the same as the |footnoterange|
% environment but suppresses all hyperlinks to footnotes inside.
%
%    \begin{macrocode}
\newenvironment{footnoterange*}{%
  % at begin of the footnoterange* environment
  \ifhmode \unskip \fi% FMi
  \addtocounter\@mpfn{+1}%
  \xdef\fnr@first{\arabic\@mpfn}%
  \addtocounter\@mpfn{-1}%
  \let\fnr@origfn\footnote%
  \let\footnote\fnr@rangefootnote%
  \GlobalLetLtxMacro{\fnr@origfnmark}{\footnotemark}%
  \GlobalLetLtxMacro{\footnotemark}{\fnr@rangefootnotemark}%
  \ignorespaces% FMi
}{% at end of the footnoterange* environment
  \ifhmode% FMi 
    \unskip% FMi
    \xdef\fnr@aftergroup{\spacefactor\the\spacefactor\relax}% FMi
  \else% FMi
    \xdef\fnr@aftergroup{\relax}%
  \fi% FMi
  \let\footnote\fnr@origfn%
  \GlobalLetLtxMacro{\footnotemark}{\fnr@origfnmark}%
  \ifx\fnr@first\fnr@arabic% There is only one reference in the environment.
    \PackageInfo{footnoterange}{%
      Environment footnoterange used,\MessageBreak%
      but only one footnote (number \fnr@first ) is contained%
      \MessageBreak%
     }%
%    \end{macrocode}
% \pagebreak
%    \begin{macrocode}
    \ltx@ifpackageloaded{hyperref}{% hyperref package loaded
      \hbox{\@textsuperscript{\normalfont \ref*{fnr:\fnr@first}}}%
    }{% hyperref package not loaded
      \hbox{\@textsuperscript{\normalfont \ref{fnr:\fnr@first}}}%
     }%
  \else%
    \ltx@ifpackageloaded{hyperref}{% hyperref package loaded
      \hbox{\@textsuperscript{\normalfont \ref*{fnr:\fnr@first}%
                                        --\ref*{fnr:\arabic\@mpfn}}}%
    }{% hyperref package not loaded
      \hbox{\@textsuperscript{\normalfont \ref{fnr:\fnr@first}%
                                        --\ref{fnr:\arabic\@mpfn}}}%
     }%
  \fi%
  \aftergroup\fnr@aftergroup% FMi
}

%    \end{macrocode}
% \end{environment}
%
% The \xpackage{footnoterange} package does not handle |\footnotemark|s,
% but at least the user is made aware of this:
%
% \begin{macro}{\fnr@rangefootnotemark}
%    \begin{macrocode}

\newcommand{\fnr@rangefootnotemark}[1][\empty]{%
  \PackageError{footnoterange}{%
     \string\footnotemark\space not supported in footnoterange(*) environment%
   }{The footnoterange package does not handle a \string\footnotemark.\MessageBreak%
     Either use \string\footnote\string{...\string}\space instead of%
     \string\footnotemark\space and \string\footnotetext\string{...\string}\MessageBreak%
     in footnoterange and footnoterange* environments,\MessageBreak%
     or do not use those environments around a \string\footnotemark\MessageBreak%
     (or \string\renewcommand\string{\string\fnr@rangefootnotemark\string}\MessageBreak%
     with your own fixing code).%
    }%
  \xdef\fnr@mopt{#1}%
  \ifx\fnr@mopt\empty%
    \fnr@origfnmark%
  \else%
    \fnr@origfnmark[#1]%
  \fi%
}

%    \end{macrocode}
% \end{macro}
%
%    \begin{macrocode}
%</package>
%    \end{macrocode}
%
% \pagebreak
%
% \section{Installation}
%
% \subsection{Downloads\label{ss:Downloads}}
%
% Everything is available on CTAN,
% \url{https://www.ctan.org/tex-archive/},
% but may need additional packages themselves.
%
% \DescribeMacro{footnoterange.dtx}
% For unpacking the \xfile{footnoterange.dtx} file
% and constructing the documentation it is required:
% \begin{description}
% \item[-] \TeX{} Format \LaTeXe{}, \url{https://www.CTAN.org/}
%
% \item[-] document class \xpackage{ltxdoc}, 2015/03/26, v2.0w,
%           \url{https://ctan.org/pkg/ltxdoc}
%
% \item[-] package \xpackage{holtxdoc}, 2012/03/21, v0.24,
%           \url{https://ctan.org/pkg/holtxdoc}
% \end{description}
%
% \DescribeMacro{footnoterange.sty}
% The |footnoterange.sty| for \LaTeXe{} (i.\,e. each document using
% the \xpackage{footnoterange} package) requires:
% \begin{description}
% \item[-] \TeX{} Format \LaTeXe{}, \url{https://www.CTAN.org/}
%
% \item[-] package \xpackage{ltxcmds}, 2011/04/18, v1.20,
%           \url{https://www.ctan.org/pkg/ltxcmds}
% \end{description}
%
% \DescribeMacro{footnoterange-example.tex}
% The |footnoterange-example.tex| requires the same files as all
% documents using the \xpackage{footnoterange} package and additionally:
% \begin{description}
% \item[-] class \xpackage{article}, 2007/10/19, v1.4h, from \xpackage{classes.dtx}:\newline
%           \url{https://www.ctan.org/tex-archive/macros/latex/base/}
%
% \item[-] package \xpackage{footnoterange}, 2019/11/24, v1.0c,\newline
%           \url{https://www.ctan.org/pkg/footnoterange}\newline
%           (Well, it is the example file for this package, and because you are
%            reading the documentation for the \xpackage{footnoterange} package,
%            it can be assumed that you already have some version of it --
%            is it the current one?)
%
% \item[-] package \xpackage{hyperref}, 2012/11/06, v6.83m,\newline
%           \url{https://www.ctan.org/pkg/hyperref}
% \end{description}
%
% \DescribeMacro{Oberdiek}
% \DescribeMacro{holtxdoc}
% \DescribeMacro{ltxcmds}
% \DescribeMacro{alphalph}
% All packages of \textsc{Heiko Oberdiek}'s bundle `oberdiek'
% (especially \xpackage{holtxdoc}, \xpackage{ltxcmds}, and \xpackage{alphalph})
% are also available in a TDS compliant ZIP archive:\newline
% \url{http://mirror.ctan.org/install/macros/latex/contrib/oberdiek.tds.zip}.\newline
% It is probably best to download and use this, because the packages in there
% should be both recent and compatible.
%
% \DescribeMacro{hyperref}
% \noindent \xpackage{hyperref} is not included in that bundle and needs to be
% downloaded separately,\newline
% \url{http://mirrors.ctan.org/install/macros/latex/contrib/hyperref.tds.zip}.
%
% \DescribeMacro{M\"{u}nch}
% A list of my packages can be found at \url{https://www.ctan.org/author/muench-hm}.
%
% \pagebreak
%
% \subsection{Package, unpacking TDS}
% \paragraph{Package.} This package is available on \url{https://www.CTAN.org}.
% \begin{description}
% \item[\url{http://mirror.ctan.org/macros/latex/contrib/footnoterange/footnoterange.dtx}]\hspace*{0.1cm}
%       The source file.
% \item[\url{http://mirror.ctan.org/macros/latex/contrib/footnoterange/footnoterange.pdf}]\hspace*{0.1cm}
%       The documentation.
% \item[\url{http://mirror.ctan.org/macros/latex/contrib/footnoterange/footnoterange-example.pdf}]\hspace*{0.1cm}
%       The compiled example file, as it should look like.
% \item[\url{http://mirror.ctan.org/macros/latex/contrib/footnoterange/README}]\hspace*{0.1cm}\newline\hspace*{0.1cm}
%       The README file.
% \end{description}
% There is also a footnoterange.tds.zip available:
% \begin{description}
% \item[\url{http://mirror.ctan.org/install/macros/latex/contrib/footnoterange.tds.zip}]\hspace*{0.1cm}
%       Everything in TDS compliant, compiled format.
% \end{description}
% which additionally contains\newline
% \begin{tabular}{ll}
% footnoterange.ins & The installation file.\\
% footnoterange.drv & The driver to generate the documentation.\\
% footnoterange.sty & The \xext{sty}le file.
% \end{tabular}
%
% \bigskip
%
% \noindent For required other packages, please see the preceding subsection.
%
% \paragraph{Unpacking.} The \xfile{.dtx} file is a self-extracting
% \docstrip{} archive. The files are extracted by running the
% \xfile{.dtx} through \plainTeX:
% \begin{quote}
%   \verb|tex footnoterange.dtx|
% \end{quote}
%
% About generating the documentation see paragraph~\ref{GenDoc} below.
%
% \paragraph{TDS.} Now the different files must be moved into
% the different directories in your installation TDS tree
% (also known as \xfile{texmf} tree):
% \begin{quote}
% \def\t{^^A
% \begin{tabular}{@{}>{\ttfamily}l@{ $\rightarrow$ }>{\ttfamily}l@{}}
%   footnoterange.sty & tex/latex/footnoterange/footnoterange.sty\\
%   footnoterange.pdf & doc/latex/footnoterange/footnoterange.pdf\\
%   footnoterange-example.tex & doc/latex/footnoterange/footnoterange-example.tex\\
%   footnoterange-example.pdf & doc/latex/footnoterange/footnoterange-example.pdf\\
%   footnoterange.dtx & source/latex/footnoterange/footnoterange.dtx\\
% \end{tabular}^^A
% }^^A
% \sbox0{\t}^^A
% \ifdim\wd0>\linewidth
%   \begingroup
%     \advance\linewidth by\leftmargin
%     \advance\linewidth by\rightmargin
%   \edef\x{\endgroup
%     \def\noexpand\lw{\the\linewidth}^^A
%   }\x
%   \def\lwbox{^^A
%     \leavevmode
%     \hbox to \linewidth{^^A
%       \kern-\leftmargin\relax
%       \hss
%       \usebox0
%       \hss
%       \kern-\rightmargin\relax
%     }^^A
%   }^^A
%   \ifdim\wd0>\lw
%     \sbox0{\small\t}^^A
%     \ifdim\wd0>\linewidth
%       \ifdim\wd0>\lw
%         \sbox0{\footnotesize\t}^^A
%         \ifdim\wd0>\linewidth
%           \ifdim\wd0>\lw
%             \sbox0{\scriptsize\t}^^A
%             \ifdim\wd0>\linewidth
%               \ifdim\wd0>\lw
%                 \sbox0{\tiny\t}^^A
%                 \ifdim\wd0>\linewidth
%                   \lwbox
%                 \else
%                   \usebox0
%                 \fi
%               \else
%                 \lwbox
%               \fi
%             \else
%               \usebox0
%             \fi
%           \else
%             \lwbox
%           \fi
%         \else
%           \usebox0
%         \fi
%       \else
%         \lwbox
%       \fi
%     \else
%       \usebox0
%     \fi
%   \else
%     \lwbox
%   \fi
% \else
%   \usebox0
% \fi
% \end{quote}
% If you have a \xfile{docstrip.cfg} that configures and enables \docstrip's
% TDS installing feature, then some files can already be in the right
% place, see the documentation of \docstrip.
%
% \subsection{Refresh file name databases}
%
% If your \TeX~distribution (\TeX{} Live, \mikTeX, \teTeX, \dots) relies on
% file name databases, you must refresh these. For example, \teTeX{} users run
% \verb|texhash| or \verb|mktexlsr|.
%
% \subsection{Some details for the interested}
%
% \paragraph{Unpacking with \LaTeX{}.}
% The \xfile{.dtx} chooses its action depending on the format:
% \begin{description}
% \item[\plainTeX:] Run \docstrip{} and extract the files.
% \item[\LaTeX:] Generate the documentation.
% \end{description}
% If you insist on using \LaTeX{} for \docstrip{} (really,
% \docstrip{} does not need \LaTeX ), then inform the autodetect routine
% about your intention:
% \begin{quote}
%   \verb|latex \let\install=y\input{footnoterange.dtx}|
% \end{quote}
% Do not forget to quote the argument according to the demands
% of your shell.
%
% \paragraph{Generating the documentation.\label{GenDoc}}
% You can use both the \xfile{.dtx} or the \xfile{.drv} to generate
% the documentation. The process can be configured by a
% configuration file \xfile{ltxdoc.cfg}. For instance, put the following
% line into this file, if you want to have A4 as paper format:
% \begin{quote}
%   \verb|\PassOptionsToClass{a4paper}{article}|
% \end{quote}
%
% \noindent An example follows how to generate the
% documentation with \pdfLaTeX :
%
% \begin{quote}
%\begin{verbatim}
%pdflatex footnoterange.drv
%makeindex -s gind.ist footnoterange.idx
%pdflatex footnoterange.drv
%makeindex -s gind.ist footnoterange.idx
%pdflatex footnoterange.drv
%\end{verbatim}
% \end{quote}
%
% \subsection{Compiling the example}
%
% The example file, \textsf{footnoterange-example.tex}, can be compiled via\newline
% \indent \verb|(pdf)latex footnoterange-example.tex|\newline
% but will need probably two compiler runs to get everything right.\newline
% \verb|footnoterange-example.tex| can be created by\newline
% \indent \verb|tex footnoterange.dtx|.\newline
% \verb|footnoterange.dtx| is at
% \url{https://www.ctan.org/pkg/footnoterange}.\newline
%
% \section{Acknowledgements}
%
% I (H.-Martin M\"{u}nch) would like to thank \textsc{Heiko Oberdiek}
% for providing a~lot~(!) of useful packages (from
% which I also got everything I know about creating a file in \xext{dtx}
% format, ok, say it: copying), \textsc{Frank Mittelbach} for several bug reports,
% for code for improving the package, and for pointing me to the \xpackage{fnpct}
% package as alternative, Willie Wong for reporting a bug and its fix,
% everybody of the CTAN team for managing CTAN,
% the \Newsgroup{comp.text.tex} and \Newsgroup{de.comp.text.tex} newsgroups,
% and everybody at \url{https://tex.stackexchange.com/}
% for their help in all things \hbox{\TeX{}.}
% This package was created as response to question
% \textquotedblleft Compressing consecutive footnote
% marks\textquotedblright{} by user001\newline
% (\url{https://tex.stackexchange.com/users/10127/user001}) on\newline
% \url{https://tex.stackexchange.com/q/44165/6865}.
%
% \newpage
% \phantomsection
% \begin{History}\label{History}
%   \begin{Version}{2011/02/11 v0.1ff.}
%     \item Some versions at \url{https://tex.stackexchange.com/q/44165/6865}.
%   \end{Version}
%   \begin{Version}{2012/02/17 v1.0a}
%     \item Package created and uploaded to CTAN.
%   \end{Version}
%   \begin{Version}{2018/10/15 v1.0b}
%     \item Fixed/updated minor details.
%     \item Bug fix: Did not work without \xpackage{hyperref}, wrong spacefactor.
%     \item Listing the \href{https://www.ctan.org/pkg/fnpct}{\xpackage{fnpct}}
%           package as alternative.
%   \end{Version}
%   \begin{Version}{2019/11/24 v1.0c}
%     \item Bug fix: Issue with footnote symbols.
%   \end{Version}
% \end{History}
%
% \bigskip
%
% When you find a mistake or have a suggestion for an improvement of this package,
% please send an e-mail to the maintainer, thanks! (Please see BUG REPORTS in the README.)
%
% \PrintIndex
%
% \Finale
\endinput
%        (quote the arguments according to the demands of your shell)
%
% Documentation:
%    (a) If footnoterange.drv is present:
%           (pdf)latex footnoterange.drv
%           makeindex -s gind.ist footnoterange.idx
%           (pdf)latex footnoterange.drv
%           makeindex -s gind.ist footnoterange.idx
%           (pdf)latex footnoterange.drv
%    (b) Without footnoterange.drv:
%           (pdf)latex footnoterange.dtx
%           makeindex -s gind.ist footnoterange.idx
%           (pdf)latex footnoterange.dtx
%           makeindex -s gind.ist footnoterange.idx
%           (pdf)latex footnoterange.dtx
%
%    The class ltxdoc loads the configuration file ltxdoc.cfg
%    if available. Here you can specify further options, e.g.
%    use DIN A4 as paper format:
%       \PassOptionsToClass{a4paper}{article}
%
% Installation:
%    TDS:tex/latex/footnoterange/footnoterange.sty
%    TDS:doc/latex/footnoterange/footnoterange.pdf
%    TDS:doc/latex/footnoterange/footnoterange-example.tex
%    TDS:doc/latex/footnoterange/footnoterange-example.pdf
%    TDS:source/latex/footnoterange/footnoterange.dtx
%
%<*ignore>
\begingroup
  \catcode123=1 %
  \catcode125=2 %
  \def\x{LaTeX2e}%
\expandafter\endgroup
\ifcase 0\ifx\install y1\fi\expandafter
         \ifx\csname processbatchFile\endcsname\relax\else1\fi
         \ifx\fmtname\x\else 1\fi\relax
\else\csname fi\endcsname
%</ignore>
%<*install>
\input docstrip.tex
\Msg{*********************************************************************************}
\Msg{* Installation}
\Msg{* Package: footnoterange 2019/11/24 v1.0c References to ranges of footnotes (HMM)}
\Msg{*********************************************************************************}

\keepsilent
\askforoverwritefalse

\let\MetaPrefix\relax
\preamble

This is a generated file.

Project: footnoterange
Version: 2019/11/24 v1.0c

Copyright (C) 2012 - 2019 by
    H.-Martin M"unch <Martin dot Muench at Uni-Bonn dot de>
Portions of code copyrighted by other people as marked;
lines marked "FMi": code by Frank Mittelbach, thanks!

The usual disclaimer applies:
If it doesn't work right that's your problem.
(Nevertheless, please send an e-mail to the maintainer
 when you find an error in this package.)

This work may be distributed and/or modified under the
conditions of the LaTeX Project Public License, either
version 1.3c of this license or (at your option) any later
version. This version of this license is in
   https://www.latex-project.org/lppl/lppl-1-3c.txt
and the latest version of this license is in
   https://www.latex-project.org/lppl.txt
and version 1.3c or later is part of all distributions of
LaTeX version 2005/12/01 or later.

This work has the LPPL maintenance status "maintained".

The Current Maintainer of this work is H.-Martin Muench.

This work consists of the main source file footnoterange.dtx,
the README, and the derived files
   footnoterange.sty, footnoterange.pdf,
   footnoterange.ins, footnoterange.drv,
   footnoterange-example.tex, footnoterange-example.pdf.

In memoriam
 Claudia Simone Barth + 1996/01/30
 Tommy Muench + 2014/01/02
 Hans-Klaus Muench + 2014/08/24

\endpreamble
\let\MetaPrefix\DoubleperCent

\generate{%
  \file{footnoterange.ins}{\from{footnoterange.dtx}{install}}%
  \file{footnoterange.drv}{\from{footnoterange.dtx}{driver}}%
  \usedir{tex/latex/footnoterange}%
  \file{footnoterange.sty}{\from{footnoterange.dtx}{package}}%
  \usedir{doc/latex/footnoterange}%
  \file{footnoterange-example.tex}{\from{footnoterange.dtx}{example}}%
}

\catcode32=13\relax% active space
\let =\space%
\Msg{************************************************************************}
\Msg{*}
\Msg{* To finish the installation you have to move the following}
\Msg{* file into a directory searched by TeX:}
\Msg{*}
\Msg{* footnoterange.sty}
\Msg{*}
\Msg{* To produce the documentation run the file `footnoterange.drv'}
\Msg{* through (pdf)LaTeX, e.g.}
\Msg{*  pdflatex footnoterange.drv}
\Msg{*  makeindex -s gind.ist footnoterange.idx}
\Msg{*  pdflatex footnoterange.drv}
\Msg{*  makeindex -s gind.ist footnoterange.idx}
\Msg{*  pdflatex footnoterange.drv}
\Msg{*}
\Msg{* At least three runs are necessary e.g. to get the}
\Msg{*  references right!}
\Msg{*}
\Msg{* Happy TeXing!}
\Msg{*}
\Msg{************************************************************************}

\endbatchfile
%</install>
%<*ignore>
\fi
%</ignore>
%
% \section{The documentation driver file}
%
% The next bit of code contains the documentation driver file for
% \TeX , i.\,e., the file that will produce the documentation you
% are currently reading. It will be extracted from this file by the
% \verb|docstrip| programme. That is, run \LaTeX{} on \verb|docstrip|
% and specify the \verb|driver| option when \verb|docstrip|
% asks for options.
%
%    \begin{macrocode}
%<*driver>
\NeedsTeXFormat{LaTeX2e}[2015/01/01]
\ProvidesFile{footnoterange.drv}[2019/11/24 v1.0c
            References to ranges of footnotes (HMM)]
\documentclass{ltxdoc}[2015/03/26]% v2.0w
\usepackage{holtxdoc}[2012/03/21]%  v0.24
%% footnoterange may work with earlier versions of LaTeX2e and those
%% class and package, but this was not tested.
%% Please consider updating your LaTeX, class, and package
%% to the most recent version (if they are not already the most
%% recent version).
\hypersetup{%
 pdfsubject={References to ranges of footnotes (HMM)},%
 pdfkeywords={LaTeX, footnoterange, footnotes, footnote, reference, H.-Martin Muench},%
 pdfencoding=auto,%
 pdflang={en},%
 breaklinks=true,%
 linktoc=all,%
 pdfstartview=FitH,%
 pdfpagelayout=OneColumn,%
 bookmarksnumbered=true,%
 bookmarksopen=true,%
 bookmarksopenlevel=3,%
 pdfmenubar=true,%
 pdftoolbar=true,%
 pdfwindowui=true,%
 pdfnewwindow=true%
}
\CodelineIndex
\hyphenation{printing docu-ment}
\gdef\unit#1{\mathord{\thinspace\mathrm{#1}}}
\begin{document}
  \DocInput{footnoterange.dtx}%
\end{document}
%</driver>
%    \end{macrocode}
%
% \fi
%
% \CheckSum{321}
%
% \CharacterTable
%  {Upper-case    \A\B\C\D\E\F\G\H\I\J\K\L\M\N\O\P\Q\R\S\T\U\V\W\X\Y\Z
%   Lower-case    \a\b\c\d\e\f\g\h\i\j\k\l\m\n\o\p\q\r\s\t\u\v\w\x\y\z
%   Digits        \0\1\2\3\4\5\6\7\8\9
%   Exclamation   \!     Double quote  \"     Hash (number) \#
%   Dollar        \$     Percent       \%     Ampersand     \&
%   Acute accent  \'     Left paren    \(     Right paren   \)
%   Asterisk      \*     Plus          \+     Comma         \,
%   Minus         \-     Point         \.     Solidus       \/
%   Colon         \:     Semicolon     \;     Less than     \<
%   Equals        \=     Greater than  \>     Question mark \?
%   Commercial at \@     Left bracket  \[     Backslash     \\
%   Right bracket \]     Circumflex    \^     Underscore    \_
%   Grave accent  \`     Left brace    \{     Vertical bar  \|
%   Right brace   \}     Tilde         \~}
%
% \GetFileInfo{footnoterange.drv}
%
% \begingroup
%   \def\x{\#,\$,\^,\_,\~,\ ,\&,\{,\},\%}%
%   \makeatletter
%   \@onelevel@sanitize\x
% \expandafter\endgroup
% \expandafter\DoNotIndex\expandafter{\x}
% \expandafter\DoNotIndex\expandafter{\string\ }
% \begingroup
%   \makeatletter
%     \lccode`9=32\relax
%     \lowercase{%^^A
%       \edef\x{\noexpand\DoNotIndex{\@backslashchar9}}%^^A
%     }%^^A
%   \expandafter\endgroup\x
% \DoNotIndex{\,,\\}
% \DoNotIndex{\documentclass,\usepackage,\ProvidesPackage,\begin,\end}
% \DoNotIndex{\MessageBreak}
% \DoNotIndex{\NeedsTeXFormat,\DoNotIndex,\verb}
% \DoNotIndex{\def,\edef,\gdef,\global,\xdef}
% \DoNotIndex{\ifx,\listfiles,\mathord,\mathrm}
% \DoNotIndex{\bigskip,\space,\thinspace,\ldots}
% \DoNotIndex{\indent,\noindent,\newline,\linebreak,\pagebreak,\newpage}
% \DoNotIndex{\textbf,\textit,\textsf,\textsc,\texttt,\textquotedblleft,\textquotedblright}
% \DoNotIndex{\plainTeX,\TeX,\LaTeX,\pdfLaTeX}
% \DoNotIndex{\section,\pagenumbering,\smallskip}
% \DoNotIndex{\@ifpackageloaded}
% \DoNotIndex{\arabic, \Alph, \alph, \alphalph, \Roman, \roman, \fnsymbol}
%
% \title{The \xpackage{footnoterange} package}
% \date{2019/11/24 v1.0c}
% \author{H.-Martin M\"{u}nch\\\xemail{Martin.Muench at Uni-Bonn.de}}
%
% \maketitle
%
% \begin{abstract}
% \noindent This package provides the environments |footnoterange| and
% |footnoterange*|. Multiple footnotes inside these environments
% are not referenced as e.\,g.
% \makeatletter%
% \hbox{\textquotedblleft \@textsuperscript {\normalfont 1 2 3}\textquotedblright}
% but as \hbox{\textquotedblleft \@textsuperscript {\normalfont 1--3}\textquotedblright ,\ }
% \makeatother%
% i.\,e. the range (from first to last referred footnote at that place) is given.
% Depending on loading of \xpackage{hyperref} and use of its |hyperfootnotes|-option
% the references are hyperlinked. (References to footnotes in |footnoterange*|
% are never hyperlinked.)
% \end{abstract}
%
% \bigskip
%
% \noindent Disclaimer for web links: The author is not responsible for any contents
% referred to in this work unless if having full knowledge of illegal contents.
% If any damage occurs by the use of information presented there, only the
% author of the respective pages might be liable, not the one who has referred
% to these pages.
%
% \noindent Save per page about $200\unit{ml}$ water,
% $2\unit{g}$ CO$_{2}$ and $2\unit{g}$ wood:
% Therefore please print only if this is really necessary.
%
% \medskip
%
% \tableofcontents
%
%\newpage
%
% \section{Introduction}
% \indent This \LaTeXe{} package provides the environments |footnoterange| and\newline
% |footnoterange*|. Multiple footnotes inside these environments are not referenced
% as e.\,g.
% \makeatletter%
% \hbox{\textquotedblleft \@textsuperscript {\normalfont 1 2 3}\textquotedblright}
% but as \hbox{\textquotedblleft \@textsuperscript {\normalfont 1--3}\textquotedblright ,\ }
% \makeatother%
% i.\,e. the range (from first to last
% referred footnote at that place) is given. Depending on loading of
% \xpackage{hyperref} and use of its |hyperfootnotes|-option the references are
% hyperlinked. References to footnotes in |footnoterange*| are never hyperlinked.\newline
% |\footnotemark\footnotetext| instead of |\footnote| is not supported inside
% |footnoterange| and |footnoterange*| environments.
%
% \bigskip
%
% \section{Usage}
%
% \indent Just load the package placing
% \begin{quote}
%   |\usepackage{footnoterange}|
% \end{quote}
% \noindent in the preamble of your \LaTeXe{} source file. There are no options to
% be used. The package should be loaded after \xpackage{hyperref} (if that package is used)
% and after any other package redefining footnotes. Usage of the |footnoterange| (or
% |footnoterange*|) environment is simple: Just encapsulate the footnotes, which shall
% be referred as range instead of referencing as single ones (see the example file).
%
% About hyperlinks:\newline
% Footnote-references in a |footnoterange*| environment are never hyperlinked.\newline
% Footnote-references in a |footnoterange| environment may be hyperlinked:
% \begin{itemize}
%  \item When the \xpackage{hyperref} package is not loaded, they are not
%         hyperlinked (naturally).
%  \item When the \xpackage{hyperref} package is loaded, it depends on the
%         |hyperfootnotes| option of the \xpackage{hyperref} package:
%  \begin{itemize}
%   \item With |hyperfootnotes=true|, the references are hyperlinked.
%   \item With |hyperfootnotes=false|, the references are not hyperlinked.
%  \end{itemize}
% \end{itemize}
% |\footnote|s inside of the |footnoterange|/{}|footnoterange*| environment
% can use the optional parameter, i.\,e. |\footnote[42]{text}| even if this is the
% third footnote. When giving ranges as reference, this does not seem to be
% such a good idea, if this is not done consistently (i.\,e. footnotes
% |[42], [43], [44]| resulting in 42--44 is OK,
% but |[42]| (original~2), unchanged~3, unchanged~4 resulting in
% 42--4 is confusing). A warning about these changed footnotemarks is written
% into the \xfile{.log}-file.
%
% When there is only a single footnote in a |footnoterange| environment,
% this does not produce some reference like %
% \makeatletter%
% \hbox{\textquotedblleft \@textsuperscript {\normalfont 7--7}\textquotedblright{},\ }%
% \makeatother%
% but only one footnotemark is used automatically. Because it is strange to use
% a |footnoterange| environment without a range of footnotes, an info is written
% into the \xfile{.log}-file about this.
%
% The usage of other footnote-number-representations
% (e.\,g.~\verb|\Roman|, \verb|\roman|, \verb|\Alph|, \verb|\alph|,
% \verb|\fnsymbol|) is also possible, but two things must be taken
% into account: The number of footnote references is restricted (for
% example with \verb|\Alph| only references A to Z are possible), which
% can be fixed e.\,g. with the \url{https://ctan.org/pkg/alphalph} package
% (but not automatically), and references to footnote-symbol-ranges (\verb|\fnsymbol|)
% are probably not very clear.
%
% \subsection*{Options}
% \indent The \xpackage{footnoterange} package takes no options.
%
% \section{Alternatives\label{sec:Alternatives}}
%
% The \xpackage{fnpct} (\url{https://www.ctan.org/pkg/fnpct}) package allows to set
% multiple footnotes (|\multfootnote|), suppress a footnotemark \hbox{(|*|),}
% and customize the multi-footnote-separator \hbox{(|\setfnpct{mult-fn-sep={|
% \emph{something} |}}|)}.
% Combining these with stepping of the footnotecounter gives a footnoterange:
% 
% \begin{verbatim}
% \documentclass{article}
% \usepackage{hyperref}
% \usepackage{fnpct}
% \begin{document}
% The three little pigs built their houses
% out of straw\footnote{not to be confused with hay},
% sticks\footnote{or lumber according to some sources}
% and bricks%
% \makeatletter%
% \setfnpct{mult-fn-sep={--}}%
% \multfootnote{probably fired clay bricks\addtocounter\@mpfn{+1};%
% *or something else;%
% what do I know}. %
% \makeatother%
% \end{document}
% \end{verbatim}
% 
% \noindent If you programmed or found another alternative, which is available at
% \url{https://www.CTAN.org/}, please send an e-mail to me with the name,
% location at CTAN, and a short notice, and I will probably list it here.
%
% \section{Example}
%
%    \begin{macrocode}
%<*example>
\documentclass[british]{article}[2014/09/29]% v1.4h
%%%%%%%%%%%%%%%%%%%%%%%%%%%%%%%%%%%%%%%%%%%%%%%%%%%%%%%%%%%%%%%%%%%%%
\usepackage[%
 extension=pdf,%
 plainpages=false,%
 pdfpagelabels=true,%
 hyperindex=false,%
 pdflang={en},%
 pdftitle={footnoterange package example},%
 pdfauthor={H.-Martin Muench},%
 pdfsubject={Example for the footnoterange package},%
 pdfkeywords={LaTeX, footnoterange, H.-Martin Muench},%
 pdfview=FitH,%
 pdfstartview=FitH,%
 pdfpagelayout=OneColumn,%
 hyperfootnotes=true%
]{hyperref}[2012/11/06]% v6.83m

\usepackage{footnoterange}[2019/11/24]% v1.0c

\gdef\unit#1{\mathord{\thinspace\mathrm{#1}}}%
\listfiles
\begin{document}
\pagenumbering{arabic}
\section*{Example for footnoterange}

This example demonstrates the use of package\newline
\textsf{footnoterange}, v1.0c as of 2019/11/24 (HMM).\newline
The package does not use options.\newline
\verb|footnoterange-example.pdf| can be created by\newline
\verb|pdflatex footnoterange-example.tex|,\newline
and \verb|footnoterange-example.tex| can be created by\newline
\verb|tex footnoterange.dtx|.\newline
\verb|footnoterange.dtx| is at
\url{https://www.ctan.org/pkg/footnoterange}.\newline

\noindent For more details please see the documentation!\newline

\noindent Save per page about $200\unit{ml}$ water,
$2\unit{g}$ CO$_{2}$ and $2\unit{g}$ wood:\newline
Therefore please print only if this is really necessary.\newline

This\footnote{Lorem} text bears a multiplicity of footnotes.
Because the \verb|hyperref| package is used with option
\verb|hyperfootnotes=true|, the footnotes
\begin{footnoterange}
\footnote{ipsum}%
\footnote{dolor}%
\footnote{sit}%
\footnote{amet,}%
\footnote{consectetur}%
\footnote{adipisicing}%
\footnote{elit,}%
\footnote{sed}%
\end{footnoterange}
are hyperlinked.\footnote{do eiusmod\ldots} Using
\verb|hyperfootnotes=false| or not using \verb|hyperref|
will remove the hyperlinks to the footnotes.\medskip

This text also bears a multiplicity of footnotes,
but due to the use of the starred
\begin{footnoterange*}
\footnote{Ut purus elit,}%
\footnote{vestibulum ut,}%
\footnote{placerat ac,}%
\footnote{adipiscing vitae,}%
\footnote{felis.}%
\footnote{Curabitur}%
\footnote{dictum}%
\footnote{gravida}%
\end{footnoterange*}
form of the environment they are not
hyperlinked.\footnote{mauris\ldots}\medskip

The use of the optional parameter for footnotes
\begin{footnoterange}
\footnote[101]{Nam arcu libero,}%
\footnote[102]{nonummy eget,}%
\footnote[103]{consectetuer id,}%
\footnote[104]{vulputate a,}%
\footnote[105]{magna.}
\end{footnoterange}
is also possible, but should be used consistently.

\pagebreak

When there is only one footnote
\begin{footnoterange}
\footnote{a single footnote}
\end{footnoterange}
in the \verb|footnoterange| environment, only one
footnotemark is used and an info is written into the
\verb|log|-file.\medskip

The usage of other footnote-number-representations
(e.\,g.~\verb|\Roman|, \verb|\roman|, \verb|\Alph|, \verb|\alph|,
\verb|\fnsymbol|) is also possible, but two things must be taken
into account: The number of footnote references is restricted
(for example with \verb|\Alph| only references A to Z are possible),
which can be fixed e.\,g. with the
\href{https://www.ctan.org/pkg/alphalph}{\texttt{alphalph} package},
and references to footnote-symbol-ranges (\verb|\fnsymbol|)
are probably not very clear.

\end{document}
%</example>
%    \end{macrocode}
%
% \StopEventually{}
%
% \section{The implementation}
%
% We start off by checking that we are loading into \LaTeXe{} and
% announcing the name and version of this package.
%
%    \begin{macrocode}
%<*package>
%    \end{macrocode}
%
%    \begin{macrocode}
\NeedsTeXFormat{LaTeX2e}[2011/06/27]
\ProvidesPackage{footnoterange}[2019/11/24 v1.0c
            References to ranges of footnotes (HMM)]
%    \end{macrocode}
%
% A short description of the \xpackage{footnoterange} package:
%
%    \begin{macrocode}
%% Provides the footnoterange and footnoterange* environments to
%% refer to ranges of footnotes.
%    \end{macrocode}
%
% We need the \xpackage{ltxcmds} package:
%
%    \begin{macrocode}
\RequirePackage{ltxcmds}[2011/11/09]% v1.22; for "\ltx@ifpackageloaded"
\RequirePackage{letltxmacro}[2010/09/02]% v1.4; for "\GlobalLetLtxMacro"

%    \end{macrocode}
%
% We use |\ifHy@hyperfootnotes ... \else ... \fi|, but when the \xpackage{hyperref}
% package has not been loaded, |\ifHy@hyperfootnotes| is not defined, and \LaTeX
% relates the |\else| and/or |\fi| (wrongly) to other |\if|s or groups.
% Therefore the |\ifHy@hyperfootnotes| are placed in their own commands,
% which are not looked at when the \xpackage{hyperref} package has not been loaded.
%
% \pagebreak
%
% \begin{macro}{\fnr@ifhyperrfn}
%    \begin{macrocode}
\newcommand{\fnr@ifhyperrfn}{%
  \ifHy@hyperfootnotes% option hyperfootnotes=true
    \addtocounter{Hfootnote}{+1}%
    % Code from the hyperref package
      \global\let\Hy@saved@currentHref\@currentHref%
      \hyper@makecurrent{Hfootnote}%
      \global\let\Hy@footnote@currentHref\@currentHref%
      \global\let\@currentHref\Hy@saved@currentHref%
    % End of code from the hyperref package
  \fi%
}

%    \end{macrocode}
% \end{macro}
%
% \begin{macro}{\fnr@rangefootnote}
% The |\fnr@rangefootnote| is the internal replacement for the original\newline
% |\footnote| command inside of a |footnoterange| environment.
%
%    \begin{macrocode}
\newcommand{\fnr@rangefootnote}[2][\empty]{%
%    \end{macrocode}
%
% |#1| contains the optional argument, which is predefined as |\empty|.
%
%    \begin{macrocode}
  \nolinebreak%
  % no \footnotemark here
  \addtocounter\@mpfn{+1}%
  \ltx@ifpackageloaded{hyperref}{\fnr@ifhyperrfn}{\relax}%
%    \end{macrocode}
%
% See the explanation before the declaration of the |\fnr@ifhyperrfn| command!
%
%    \begin{macrocode}
  \xdef\fnr@opt{#1}% contains the optional argument
  \xdef\fnr@arabic{\arabic\@mpfn}%
  \protected@edef\fnr@formated{\thefootnote}%
%    \end{macrocode}
%
% The |\footnotemark| is not set (yet), but the |\footnotetext|,
% and it gets a label.
%
%    \begin{macrocode}
  \ifx\fnr@opt\empty%
    \footnotetext{\label{fnr:\fnr@arabic}#2}%
  \else%
    \PackageWarning{footnoterange}{%
      Custom footnote-numbers used in a footnoterange\MessageBreak%
      environment: Footnote with natural number\MessageBreak%
      \ifx\fnr@formated\fnr@arabic%
        \fnr@arabic\space%
      \else%
        \space\fnr@arabic\space (\fnr@formated )%
      \fi%
      \MessageBreak%
      was renumbered to \fnr@opt\MessageBreak%
     }%
    \ltx@ifpackageloaded{hyperref}{% hyperref loaded
      \footnotetext[#1]{\phantomsection\label{fnr:\fnr@arabic}#2}%
     }{% hyperref not loaded
      \footnotetext[#1]{\label{fnr:\fnr@arabic}#2}%
     }%
  \fi%
}

%    \end{macrocode}
% \end{macro}
%
%    \begin{macrocode}

%    \end{macrocode}
% \pagebreak
%
% |\ifHy@hyperfootnotes| \textquotedblleft encapsulation\textquotedblright{} again:
%
% \begin{macro}{\fnr@ifhyperfnrA}
%    \begin{macrocode}
\newcommand{\fnr@ifhyperfnrA}{%
  \ifHy@hyperfootnotes% option hyperfootnotes=true
    \hbox{\hyperref[fnr:\fnr@first]{%
          \@textsuperscript{\normalfont \ref*{fnr:\fnr@first}}}}%
  \else% option hyperfootnotes=false
    \hbox{\@textsuperscript{\normalfont \ref*{fnr:\fnr@first}}}%
  \fi%
}

%    \end{macrocode}
% \end{macro}
%
% \begin{macro}{\fnr@ifhyperfnrN}
%    \begin{macrocode}
\newcommand{\fnr@ifhyperfnrN}{%
  \ifHy@hyperfootnotes% option hyperfootnotes=true
    \hbox{\hyperref[fnr:\fnr@first]{%
            \@textsuperscript{\normalfont \ref*{fnr:\fnr@first}}}%
          \@textsuperscript{\normalfont --}%
          \hyperref[fnr:\arabic\@mpfn]{%
            \@textsuperscript{\normalfont \ref*{fnr:\arabic\@mpfn}}}}%
  \else% option hyperfootnotes=false
    \hbox{\@textsuperscript{\normalfont \ref*{fnr:\fnr@first}%
                                        --\ref*{fnr:\arabic\@mpfn}}}%
  \fi%
}

%    \end{macrocode}
% \end{macro}
%
% \begin{environment}{footnoterange}
% The |footnoterange| environment remembers the next footnote number as\newline
% |\fnr@first| and the current |\footnote| definition as |\fnr@origfn|
% and uses |\fnr@rangefootnote| instead.
%
%    \begin{macrocode}
\newenvironment{footnoterange}{%
  % at begin of the footnoterange environment
  \ifhmode \unskip \fi% FMi
  \addtocounter\@mpfn{+1}%
  \xdef\fnr@first{\arabic\@mpfn}%
  \addtocounter\@mpfn{-1}%
  \let\fnr@origfn\footnote%
  \let\footnote\fnr@rangefootnote%
  \GlobalLetLtxMacro{\fnr@origfnmark}{\footnotemark}%
  \GlobalLetLtxMacro{\footnotemark}{\fnr@rangefootnotemark}%
  \ignorespaces% FMi
%    \end{macrocode}
%
% End of the environment:
%
%    \begin{macrocode}
}{% at end of the footnoterange environment
  \ifhmode% FMi 
    \unskip% FMi
%    \end{macrocode}
%
% Remember the spacefactor for restoring after the environment has ended
% (only in |hmode|!). \% FMi
%
%    \begin{macrocode}
    \xdef\fnr@aftergroup{\spacefactor\the\spacefactor\relax}% FMi
  \else% FMi
%    \end{macrocode}
%
% Otherwise do something harmless. \% FMi
%
%    \begin{macrocode}
    \xdef\fnr@aftergroup{\relax}%
  \fi% FMi
%    \end{macrocode}
%
% At the end of the environment the previous definition of |\footnote| is restored
%
%    \begin{macrocode}
  \let\footnote\fnr@origfn%
  \GlobalLetLtxMacro{\footnotemark}{\fnr@origfnmark}%
%    \end{macrocode}
%
% and the references are placed:
%
%    \begin{macrocode}
  \ifx\fnr@first\fnr@arabic% There is only one reference in the environment.
    \PackageInfo{footnoterange}{%
      Environment footnoterange used,\MessageBreak%
      but only one footnote (number \fnr@first ) is contained%
      \MessageBreak%
     }%
    \ltx@ifpackageloaded{hyperref}{\fnr@ifhyperfnrA% hyperref package loaded
    }{% hyperref package not loaded
      \hbox{\@textsuperscript{\normalfont \ref{fnr:\fnr@first}}}%
     }%
  \else%
    \ltx@ifpackageloaded{hyperref}{\fnr@ifhyperfnrN% hyperref package loaded
    }{% hyperref package not loaded
      \hbox{\@textsuperscript{\normalfont \ref{fnr:\fnr@first}%
                                        --\ref{fnr:\arabic\@mpfn}}}%
     }%
  \fi%
%    \end{macrocode}
%
% Restore the spacefactor after the environment has ended. \% FMi
%
%    \begin{macrocode}
  \aftergroup\fnr@aftergroup% FMi
}

%    \end{macrocode}
% \end{environment}
%
% \begin{environment}{footnoterange*}
% The |footnoterange*| environment does the same as the |footnoterange|
% environment but suppresses all hyperlinks to footnotes inside.
%
%    \begin{macrocode}
\newenvironment{footnoterange*}{%
  % at begin of the footnoterange* environment
  \ifhmode \unskip \fi% FMi
  \addtocounter\@mpfn{+1}%
  \xdef\fnr@first{\arabic\@mpfn}%
  \addtocounter\@mpfn{-1}%
  \let\fnr@origfn\footnote%
  \let\footnote\fnr@rangefootnote%
  \GlobalLetLtxMacro{\fnr@origfnmark}{\footnotemark}%
  \GlobalLetLtxMacro{\footnotemark}{\fnr@rangefootnotemark}%
  \ignorespaces% FMi
}{% at end of the footnoterange* environment
  \ifhmode% FMi 
    \unskip% FMi
    \xdef\fnr@aftergroup{\spacefactor\the\spacefactor\relax}% FMi
  \else% FMi
    \xdef\fnr@aftergroup{\relax}%
  \fi% FMi
  \let\footnote\fnr@origfn%
  \GlobalLetLtxMacro{\footnotemark}{\fnr@origfnmark}%
  \ifx\fnr@first\fnr@arabic% There is only one reference in the environment.
    \PackageInfo{footnoterange}{%
      Environment footnoterange used,\MessageBreak%
      but only one footnote (number \fnr@first ) is contained%
      \MessageBreak%
     }%
%    \end{macrocode}
% \pagebreak
%    \begin{macrocode}
    \ltx@ifpackageloaded{hyperref}{% hyperref package loaded
      \hbox{\@textsuperscript{\normalfont \ref*{fnr:\fnr@first}}}%
    }{% hyperref package not loaded
      \hbox{\@textsuperscript{\normalfont \ref{fnr:\fnr@first}}}%
     }%
  \else%
    \ltx@ifpackageloaded{hyperref}{% hyperref package loaded
      \hbox{\@textsuperscript{\normalfont \ref*{fnr:\fnr@first}%
                                        --\ref*{fnr:\arabic\@mpfn}}}%
    }{% hyperref package not loaded
      \hbox{\@textsuperscript{\normalfont \ref{fnr:\fnr@first}%
                                        --\ref{fnr:\arabic\@mpfn}}}%
     }%
  \fi%
  \aftergroup\fnr@aftergroup% FMi
}

%    \end{macrocode}
% \end{environment}
%
% The \xpackage{footnoterange} package does not handle |\footnotemark|s,
% but at least the user is made aware of this:
%
% \begin{macro}{\fnr@rangefootnotemark}
%    \begin{macrocode}

\newcommand{\fnr@rangefootnotemark}[1][\empty]{%
  \PackageError{footnoterange}{%
     \string\footnotemark\space not supported in footnoterange(*) environment%
   }{The footnoterange package does not handle a \string\footnotemark.\MessageBreak%
     Either use \string\footnote\string{...\string}\space instead of%
     \string\footnotemark\space and \string\footnotetext\string{...\string}\MessageBreak%
     in footnoterange and footnoterange* environments,\MessageBreak%
     or do not use those environments around a \string\footnotemark\MessageBreak%
     (or \string\renewcommand\string{\string\fnr@rangefootnotemark\string}\MessageBreak%
     with your own fixing code).%
    }%
  \xdef\fnr@mopt{#1}%
  \ifx\fnr@mopt\empty%
    \fnr@origfnmark%
  \else%
    \fnr@origfnmark[#1]%
  \fi%
}

%    \end{macrocode}
% \end{macro}
%
%    \begin{macrocode}
%</package>
%    \end{macrocode}
%
% \pagebreak
%
% \section{Installation}
%
% \subsection{Downloads\label{ss:Downloads}}
%
% Everything is available on CTAN,
% \url{https://www.ctan.org/tex-archive/},
% but may need additional packages themselves.
%
% \DescribeMacro{footnoterange.dtx}
% For unpacking the \xfile{footnoterange.dtx} file
% and constructing the documentation it is required:
% \begin{description}
% \item[-] \TeX{} Format \LaTeXe{}, \url{https://www.CTAN.org/}
%
% \item[-] document class \xpackage{ltxdoc}, 2015/03/26, v2.0w,
%           \url{https://ctan.org/pkg/ltxdoc}
%
% \item[-] package \xpackage{holtxdoc}, 2012/03/21, v0.24,
%           \url{https://ctan.org/pkg/holtxdoc}
% \end{description}
%
% \DescribeMacro{footnoterange.sty}
% The |footnoterange.sty| for \LaTeXe{} (i.\,e. each document using
% the \xpackage{footnoterange} package) requires:
% \begin{description}
% \item[-] \TeX{} Format \LaTeXe{}, \url{https://www.CTAN.org/}
%
% \item[-] package \xpackage{ltxcmds}, 2011/04/18, v1.20,
%           \url{https://www.ctan.org/pkg/ltxcmds}
% \end{description}
%
% \DescribeMacro{footnoterange-example.tex}
% The |footnoterange-example.tex| requires the same files as all
% documents using the \xpackage{footnoterange} package and additionally:
% \begin{description}
% \item[-] class \xpackage{article}, 2007/10/19, v1.4h, from \xpackage{classes.dtx}:\newline
%           \url{https://www.ctan.org/tex-archive/macros/latex/base/}
%
% \item[-] package \xpackage{footnoterange}, 2019/11/24, v1.0c,\newline
%           \url{https://www.ctan.org/pkg/footnoterange}\newline
%           (Well, it is the example file for this package, and because you are
%            reading the documentation for the \xpackage{footnoterange} package,
%            it can be assumed that you already have some version of it --
%            is it the current one?)
%
% \item[-] package \xpackage{hyperref}, 2012/11/06, v6.83m,\newline
%           \url{https://www.ctan.org/pkg/hyperref}
% \end{description}
%
% \DescribeMacro{Oberdiek}
% \DescribeMacro{holtxdoc}
% \DescribeMacro{ltxcmds}
% \DescribeMacro{alphalph}
% All packages of \textsc{Heiko Oberdiek}'s bundle `oberdiek'
% (especially \xpackage{holtxdoc}, \xpackage{ltxcmds}, and \xpackage{alphalph})
% are also available in a TDS compliant ZIP archive:\newline
% \url{http://mirror.ctan.org/install/macros/latex/contrib/oberdiek.tds.zip}.\newline
% It is probably best to download and use this, because the packages in there
% should be both recent and compatible.
%
% \DescribeMacro{hyperref}
% \noindent \xpackage{hyperref} is not included in that bundle and needs to be
% downloaded separately,\newline
% \url{http://mirrors.ctan.org/install/macros/latex/contrib/hyperref.tds.zip}.
%
% \DescribeMacro{M\"{u}nch}
% A list of my packages can be found at \url{https://www.ctan.org/author/muench-hm}.
%
% \pagebreak
%
% \subsection{Package, unpacking TDS}
% \paragraph{Package.} This package is available on \url{https://www.CTAN.org}.
% \begin{description}
% \item[\url{http://mirror.ctan.org/macros/latex/contrib/footnoterange/footnoterange.dtx}]\hspace*{0.1cm}
%       The source file.
% \item[\url{http://mirror.ctan.org/macros/latex/contrib/footnoterange/footnoterange.pdf}]\hspace*{0.1cm}
%       The documentation.
% \item[\url{http://mirror.ctan.org/macros/latex/contrib/footnoterange/footnoterange-example.pdf}]\hspace*{0.1cm}
%       The compiled example file, as it should look like.
% \item[\url{http://mirror.ctan.org/macros/latex/contrib/footnoterange/README}]\hspace*{0.1cm}\newline\hspace*{0.1cm}
%       The README file.
% \end{description}
% There is also a footnoterange.tds.zip available:
% \begin{description}
% \item[\url{http://mirror.ctan.org/install/macros/latex/contrib/footnoterange.tds.zip}]\hspace*{0.1cm}
%       Everything in TDS compliant, compiled format.
% \end{description}
% which additionally contains\newline
% \begin{tabular}{ll}
% footnoterange.ins & The installation file.\\
% footnoterange.drv & The driver to generate the documentation.\\
% footnoterange.sty & The \xext{sty}le file.
% \end{tabular}
%
% \bigskip
%
% \noindent For required other packages, please see the preceding subsection.
%
% \paragraph{Unpacking.} The \xfile{.dtx} file is a self-extracting
% \docstrip{} archive. The files are extracted by running the
% \xfile{.dtx} through \plainTeX:
% \begin{quote}
%   \verb|tex footnoterange.dtx|
% \end{quote}
%
% About generating the documentation see paragraph~\ref{GenDoc} below.
%
% \paragraph{TDS.} Now the different files must be moved into
% the different directories in your installation TDS tree
% (also known as \xfile{texmf} tree):
% \begin{quote}
% \def\t{^^A
% \begin{tabular}{@{}>{\ttfamily}l@{ $\rightarrow$ }>{\ttfamily}l@{}}
%   footnoterange.sty & tex/latex/footnoterange/footnoterange.sty\\
%   footnoterange.pdf & doc/latex/footnoterange/footnoterange.pdf\\
%   footnoterange-example.tex & doc/latex/footnoterange/footnoterange-example.tex\\
%   footnoterange-example.pdf & doc/latex/footnoterange/footnoterange-example.pdf\\
%   footnoterange.dtx & source/latex/footnoterange/footnoterange.dtx\\
% \end{tabular}^^A
% }^^A
% \sbox0{\t}^^A
% \ifdim\wd0>\linewidth
%   \begingroup
%     \advance\linewidth by\leftmargin
%     \advance\linewidth by\rightmargin
%   \edef\x{\endgroup
%     \def\noexpand\lw{\the\linewidth}^^A
%   }\x
%   \def\lwbox{^^A
%     \leavevmode
%     \hbox to \linewidth{^^A
%       \kern-\leftmargin\relax
%       \hss
%       \usebox0
%       \hss
%       \kern-\rightmargin\relax
%     }^^A
%   }^^A
%   \ifdim\wd0>\lw
%     \sbox0{\small\t}^^A
%     \ifdim\wd0>\linewidth
%       \ifdim\wd0>\lw
%         \sbox0{\footnotesize\t}^^A
%         \ifdim\wd0>\linewidth
%           \ifdim\wd0>\lw
%             \sbox0{\scriptsize\t}^^A
%             \ifdim\wd0>\linewidth
%               \ifdim\wd0>\lw
%                 \sbox0{\tiny\t}^^A
%                 \ifdim\wd0>\linewidth
%                   \lwbox
%                 \else
%                   \usebox0
%                 \fi
%               \else
%                 \lwbox
%               \fi
%             \else
%               \usebox0
%             \fi
%           \else
%             \lwbox
%           \fi
%         \else
%           \usebox0
%         \fi
%       \else
%         \lwbox
%       \fi
%     \else
%       \usebox0
%     \fi
%   \else
%     \lwbox
%   \fi
% \else
%   \usebox0
% \fi
% \end{quote}
% If you have a \xfile{docstrip.cfg} that configures and enables \docstrip's
% TDS installing feature, then some files can already be in the right
% place, see the documentation of \docstrip.
%
% \subsection{Refresh file name databases}
%
% If your \TeX~distribution (\TeX{} Live, \mikTeX, \teTeX, \dots) relies on
% file name databases, you must refresh these. For example, \teTeX{} users run
% \verb|texhash| or \verb|mktexlsr|.
%
% \subsection{Some details for the interested}
%
% \paragraph{Unpacking with \LaTeX{}.}
% The \xfile{.dtx} chooses its action depending on the format:
% \begin{description}
% \item[\plainTeX:] Run \docstrip{} and extract the files.
% \item[\LaTeX:] Generate the documentation.
% \end{description}
% If you insist on using \LaTeX{} for \docstrip{} (really,
% \docstrip{} does not need \LaTeX ), then inform the autodetect routine
% about your intention:
% \begin{quote}
%   \verb|latex \let\install=y% \iffalse meta-comment
%
% File: footnoterange.dtx
% Version: 2019/11/24 v1.0c
%
% Copyright (C) 2012 - 2019 by
%    H.-Martin M"unch <Martin dot Muench at Uni-Bonn dot de>
% Portions of code copyrighted by other people as marked;
% lines marked "FMi": code by Frank Mittelbach, thanks!
%
% This work may be distributed and/or modified under the
% conditions of the LaTeX Project Public License, either
% version 1.3c of this license or (at your option) any later
% version. This version of this license is in
%    https://www.latex-project.org/lppl/lppl-1-3c.txt
% and the latest version of this license is in
%    https://www.latex-project.org/lppl.txt
% and version 1.3c or later is part of all distributions of
% LaTeX version 2005/12/01 or later.
%
% This work has the LPPL maintenance status "maintained".
%
% The Current Maintainer of this work is H.-Martin Muench.
%
% This work consists of the main source file footnoterange.dtx,
% the README, and the derived files
%    footnoterange.sty, footnoterange.pdf,
%    footnoterange.ins, footnoterange.drv,
%    footnoterange-example.tex, footnoterange-example.pdf.
%
% 'footnoterange' is available on CTAN:
% https://www.ctan.org/pkg/footnoterange
%
% Also a TDS.ZIP file is provided that contains all the files
% already sorted in a TDS tree:
% http://mirror.ctan.org/install/macros/latex/contrib/footnoterange.tds.zip
%
% Unpacking:
%    (a) If footnoterange.ins is present:
%           tex footnoterange.ins
%    (b) Without footnoterange.ins:
%           tex footnoterange.dtx
%    (c) If you insist on using LaTeX
%           latex \let\install=y\input{footnoterange.dtx}
%        (quote the arguments according to the demands of your shell)
%
% Documentation:
%    (a) If footnoterange.drv is present:
%           (pdf)latex footnoterange.drv
%           makeindex -s gind.ist footnoterange.idx
%           (pdf)latex footnoterange.drv
%           makeindex -s gind.ist footnoterange.idx
%           (pdf)latex footnoterange.drv
%    (b) Without footnoterange.drv:
%           (pdf)latex footnoterange.dtx
%           makeindex -s gind.ist footnoterange.idx
%           (pdf)latex footnoterange.dtx
%           makeindex -s gind.ist footnoterange.idx
%           (pdf)latex footnoterange.dtx
%
%    The class ltxdoc loads the configuration file ltxdoc.cfg
%    if available. Here you can specify further options, e.g.
%    use DIN A4 as paper format:
%       \PassOptionsToClass{a4paper}{article}
%
% Installation:
%    TDS:tex/latex/footnoterange/footnoterange.sty
%    TDS:doc/latex/footnoterange/footnoterange.pdf
%    TDS:doc/latex/footnoterange/footnoterange-example.tex
%    TDS:doc/latex/footnoterange/footnoterange-example.pdf
%    TDS:source/latex/footnoterange/footnoterange.dtx
%
%<*ignore>
\begingroup
  \catcode123=1 %
  \catcode125=2 %
  \def\x{LaTeX2e}%
\expandafter\endgroup
\ifcase 0\ifx\install y1\fi\expandafter
         \ifx\csname processbatchFile\endcsname\relax\else1\fi
         \ifx\fmtname\x\else 1\fi\relax
\else\csname fi\endcsname
%</ignore>
%<*install>
\input docstrip.tex
\Msg{*********************************************************************************}
\Msg{* Installation}
\Msg{* Package: footnoterange 2019/11/24 v1.0c References to ranges of footnotes (HMM)}
\Msg{*********************************************************************************}

\keepsilent
\askforoverwritefalse

\let\MetaPrefix\relax
\preamble

This is a generated file.

Project: footnoterange
Version: 2019/11/24 v1.0c

Copyright (C) 2012 - 2019 by
    H.-Martin M"unch <Martin dot Muench at Uni-Bonn dot de>
Portions of code copyrighted by other people as marked;
lines marked "FMi": code by Frank Mittelbach, thanks!

The usual disclaimer applies:
If it doesn't work right that's your problem.
(Nevertheless, please send an e-mail to the maintainer
 when you find an error in this package.)

This work may be distributed and/or modified under the
conditions of the LaTeX Project Public License, either
version 1.3c of this license or (at your option) any later
version. This version of this license is in
   https://www.latex-project.org/lppl/lppl-1-3c.txt
and the latest version of this license is in
   https://www.latex-project.org/lppl.txt
and version 1.3c or later is part of all distributions of
LaTeX version 2005/12/01 or later.

This work has the LPPL maintenance status "maintained".

The Current Maintainer of this work is H.-Martin Muench.

This work consists of the main source file footnoterange.dtx,
the README, and the derived files
   footnoterange.sty, footnoterange.pdf,
   footnoterange.ins, footnoterange.drv,
   footnoterange-example.tex, footnoterange-example.pdf.

In memoriam
 Claudia Simone Barth + 1996/01/30
 Tommy Muench + 2014/01/02
 Hans-Klaus Muench + 2014/08/24

\endpreamble
\let\MetaPrefix\DoubleperCent

\generate{%
  \file{footnoterange.ins}{\from{footnoterange.dtx}{install}}%
  \file{footnoterange.drv}{\from{footnoterange.dtx}{driver}}%
  \usedir{tex/latex/footnoterange}%
  \file{footnoterange.sty}{\from{footnoterange.dtx}{package}}%
  \usedir{doc/latex/footnoterange}%
  \file{footnoterange-example.tex}{\from{footnoterange.dtx}{example}}%
}

\catcode32=13\relax% active space
\let =\space%
\Msg{************************************************************************}
\Msg{*}
\Msg{* To finish the installation you have to move the following}
\Msg{* file into a directory searched by TeX:}
\Msg{*}
\Msg{* footnoterange.sty}
\Msg{*}
\Msg{* To produce the documentation run the file `footnoterange.drv'}
\Msg{* through (pdf)LaTeX, e.g.}
\Msg{*  pdflatex footnoterange.drv}
\Msg{*  makeindex -s gind.ist footnoterange.idx}
\Msg{*  pdflatex footnoterange.drv}
\Msg{*  makeindex -s gind.ist footnoterange.idx}
\Msg{*  pdflatex footnoterange.drv}
\Msg{*}
\Msg{* At least three runs are necessary e.g. to get the}
\Msg{*  references right!}
\Msg{*}
\Msg{* Happy TeXing!}
\Msg{*}
\Msg{************************************************************************}

\endbatchfile
%</install>
%<*ignore>
\fi
%</ignore>
%
% \section{The documentation driver file}
%
% The next bit of code contains the documentation driver file for
% \TeX , i.\,e., the file that will produce the documentation you
% are currently reading. It will be extracted from this file by the
% \verb|docstrip| programme. That is, run \LaTeX{} on \verb|docstrip|
% and specify the \verb|driver| option when \verb|docstrip|
% asks for options.
%
%    \begin{macrocode}
%<*driver>
\NeedsTeXFormat{LaTeX2e}[2015/01/01]
\ProvidesFile{footnoterange.drv}[2019/11/24 v1.0c
            References to ranges of footnotes (HMM)]
\documentclass{ltxdoc}[2015/03/26]% v2.0w
\usepackage{holtxdoc}[2012/03/21]%  v0.24
%% footnoterange may work with earlier versions of LaTeX2e and those
%% class and package, but this was not tested.
%% Please consider updating your LaTeX, class, and package
%% to the most recent version (if they are not already the most
%% recent version).
\hypersetup{%
 pdfsubject={References to ranges of footnotes (HMM)},%
 pdfkeywords={LaTeX, footnoterange, footnotes, footnote, reference, H.-Martin Muench},%
 pdfencoding=auto,%
 pdflang={en},%
 breaklinks=true,%
 linktoc=all,%
 pdfstartview=FitH,%
 pdfpagelayout=OneColumn,%
 bookmarksnumbered=true,%
 bookmarksopen=true,%
 bookmarksopenlevel=3,%
 pdfmenubar=true,%
 pdftoolbar=true,%
 pdfwindowui=true,%
 pdfnewwindow=true%
}
\CodelineIndex
\hyphenation{printing docu-ment}
\gdef\unit#1{\mathord{\thinspace\mathrm{#1}}}
\begin{document}
  \DocInput{footnoterange.dtx}%
\end{document}
%</driver>
%    \end{macrocode}
%
% \fi
%
% \CheckSum{321}
%
% \CharacterTable
%  {Upper-case    \A\B\C\D\E\F\G\H\I\J\K\L\M\N\O\P\Q\R\S\T\U\V\W\X\Y\Z
%   Lower-case    \a\b\c\d\e\f\g\h\i\j\k\l\m\n\o\p\q\r\s\t\u\v\w\x\y\z
%   Digits        \0\1\2\3\4\5\6\7\8\9
%   Exclamation   \!     Double quote  \"     Hash (number) \#
%   Dollar        \$     Percent       \%     Ampersand     \&
%   Acute accent  \'     Left paren    \(     Right paren   \)
%   Asterisk      \*     Plus          \+     Comma         \,
%   Minus         \-     Point         \.     Solidus       \/
%   Colon         \:     Semicolon     \;     Less than     \<
%   Equals        \=     Greater than  \>     Question mark \?
%   Commercial at \@     Left bracket  \[     Backslash     \\
%   Right bracket \]     Circumflex    \^     Underscore    \_
%   Grave accent  \`     Left brace    \{     Vertical bar  \|
%   Right brace   \}     Tilde         \~}
%
% \GetFileInfo{footnoterange.drv}
%
% \begingroup
%   \def\x{\#,\$,\^,\_,\~,\ ,\&,\{,\},\%}%
%   \makeatletter
%   \@onelevel@sanitize\x
% \expandafter\endgroup
% \expandafter\DoNotIndex\expandafter{\x}
% \expandafter\DoNotIndex\expandafter{\string\ }
% \begingroup
%   \makeatletter
%     \lccode`9=32\relax
%     \lowercase{%^^A
%       \edef\x{\noexpand\DoNotIndex{\@backslashchar9}}%^^A
%     }%^^A
%   \expandafter\endgroup\x
% \DoNotIndex{\,,\\}
% \DoNotIndex{\documentclass,\usepackage,\ProvidesPackage,\begin,\end}
% \DoNotIndex{\MessageBreak}
% \DoNotIndex{\NeedsTeXFormat,\DoNotIndex,\verb}
% \DoNotIndex{\def,\edef,\gdef,\global,\xdef}
% \DoNotIndex{\ifx,\listfiles,\mathord,\mathrm}
% \DoNotIndex{\bigskip,\space,\thinspace,\ldots}
% \DoNotIndex{\indent,\noindent,\newline,\linebreak,\pagebreak,\newpage}
% \DoNotIndex{\textbf,\textit,\textsf,\textsc,\texttt,\textquotedblleft,\textquotedblright}
% \DoNotIndex{\plainTeX,\TeX,\LaTeX,\pdfLaTeX}
% \DoNotIndex{\section,\pagenumbering,\smallskip}
% \DoNotIndex{\@ifpackageloaded}
% \DoNotIndex{\arabic, \Alph, \alph, \alphalph, \Roman, \roman, \fnsymbol}
%
% \title{The \xpackage{footnoterange} package}
% \date{2019/11/24 v1.0c}
% \author{H.-Martin M\"{u}nch\\\xemail{Martin.Muench at Uni-Bonn.de}}
%
% \maketitle
%
% \begin{abstract}
% \noindent This package provides the environments |footnoterange| and
% |footnoterange*|. Multiple footnotes inside these environments
% are not referenced as e.\,g.
% \makeatletter%
% \hbox{\textquotedblleft \@textsuperscript {\normalfont 1 2 3}\textquotedblright}
% but as \hbox{\textquotedblleft \@textsuperscript {\normalfont 1--3}\textquotedblright ,\ }
% \makeatother%
% i.\,e. the range (from first to last referred footnote at that place) is given.
% Depending on loading of \xpackage{hyperref} and use of its |hyperfootnotes|-option
% the references are hyperlinked. (References to footnotes in |footnoterange*|
% are never hyperlinked.)
% \end{abstract}
%
% \bigskip
%
% \noindent Disclaimer for web links: The author is not responsible for any contents
% referred to in this work unless if having full knowledge of illegal contents.
% If any damage occurs by the use of information presented there, only the
% author of the respective pages might be liable, not the one who has referred
% to these pages.
%
% \noindent Save per page about $200\unit{ml}$ water,
% $2\unit{g}$ CO$_{2}$ and $2\unit{g}$ wood:
% Therefore please print only if this is really necessary.
%
% \medskip
%
% \tableofcontents
%
%\newpage
%
% \section{Introduction}
% \indent This \LaTeXe{} package provides the environments |footnoterange| and\newline
% |footnoterange*|. Multiple footnotes inside these environments are not referenced
% as e.\,g.
% \makeatletter%
% \hbox{\textquotedblleft \@textsuperscript {\normalfont 1 2 3}\textquotedblright}
% but as \hbox{\textquotedblleft \@textsuperscript {\normalfont 1--3}\textquotedblright ,\ }
% \makeatother%
% i.\,e. the range (from first to last
% referred footnote at that place) is given. Depending on loading of
% \xpackage{hyperref} and use of its |hyperfootnotes|-option the references are
% hyperlinked. References to footnotes in |footnoterange*| are never hyperlinked.\newline
% |\footnotemark\footnotetext| instead of |\footnote| is not supported inside
% |footnoterange| and |footnoterange*| environments.
%
% \bigskip
%
% \section{Usage}
%
% \indent Just load the package placing
% \begin{quote}
%   |\usepackage{footnoterange}|
% \end{quote}
% \noindent in the preamble of your \LaTeXe{} source file. There are no options to
% be used. The package should be loaded after \xpackage{hyperref} (if that package is used)
% and after any other package redefining footnotes. Usage of the |footnoterange| (or
% |footnoterange*|) environment is simple: Just encapsulate the footnotes, which shall
% be referred as range instead of referencing as single ones (see the example file).
%
% About hyperlinks:\newline
% Footnote-references in a |footnoterange*| environment are never hyperlinked.\newline
% Footnote-references in a |footnoterange| environment may be hyperlinked:
% \begin{itemize}
%  \item When the \xpackage{hyperref} package is not loaded, they are not
%         hyperlinked (naturally).
%  \item When the \xpackage{hyperref} package is loaded, it depends on the
%         |hyperfootnotes| option of the \xpackage{hyperref} package:
%  \begin{itemize}
%   \item With |hyperfootnotes=true|, the references are hyperlinked.
%   \item With |hyperfootnotes=false|, the references are not hyperlinked.
%  \end{itemize}
% \end{itemize}
% |\footnote|s inside of the |footnoterange|/{}|footnoterange*| environment
% can use the optional parameter, i.\,e. |\footnote[42]{text}| even if this is the
% third footnote. When giving ranges as reference, this does not seem to be
% such a good idea, if this is not done consistently (i.\,e. footnotes
% |[42], [43], [44]| resulting in 42--44 is OK,
% but |[42]| (original~2), unchanged~3, unchanged~4 resulting in
% 42--4 is confusing). A warning about these changed footnotemarks is written
% into the \xfile{.log}-file.
%
% When there is only a single footnote in a |footnoterange| environment,
% this does not produce some reference like %
% \makeatletter%
% \hbox{\textquotedblleft \@textsuperscript {\normalfont 7--7}\textquotedblright{},\ }%
% \makeatother%
% but only one footnotemark is used automatically. Because it is strange to use
% a |footnoterange| environment without a range of footnotes, an info is written
% into the \xfile{.log}-file about this.
%
% The usage of other footnote-number-representations
% (e.\,g.~\verb|\Roman|, \verb|\roman|, \verb|\Alph|, \verb|\alph|,
% \verb|\fnsymbol|) is also possible, but two things must be taken
% into account: The number of footnote references is restricted (for
% example with \verb|\Alph| only references A to Z are possible), which
% can be fixed e.\,g. with the \url{https://ctan.org/pkg/alphalph} package
% (but not automatically), and references to footnote-symbol-ranges (\verb|\fnsymbol|)
% are probably not very clear.
%
% \subsection*{Options}
% \indent The \xpackage{footnoterange} package takes no options.
%
% \section{Alternatives\label{sec:Alternatives}}
%
% The \xpackage{fnpct} (\url{https://www.ctan.org/pkg/fnpct}) package allows to set
% multiple footnotes (|\multfootnote|), suppress a footnotemark \hbox{(|*|),}
% and customize the multi-footnote-separator \hbox{(|\setfnpct{mult-fn-sep={|
% \emph{something} |}}|)}.
% Combining these with stepping of the footnotecounter gives a footnoterange:
% 
% \begin{verbatim}
% \documentclass{article}
% \usepackage{hyperref}
% \usepackage{fnpct}
% \begin{document}
% The three little pigs built their houses
% out of straw\footnote{not to be confused with hay},
% sticks\footnote{or lumber according to some sources}
% and bricks%
% \makeatletter%
% \setfnpct{mult-fn-sep={--}}%
% \multfootnote{probably fired clay bricks\addtocounter\@mpfn{+1};%
% *or something else;%
% what do I know}. %
% \makeatother%
% \end{document}
% \end{verbatim}
% 
% \noindent If you programmed or found another alternative, which is available at
% \url{https://www.CTAN.org/}, please send an e-mail to me with the name,
% location at CTAN, and a short notice, and I will probably list it here.
%
% \section{Example}
%
%    \begin{macrocode}
%<*example>
\documentclass[british]{article}[2014/09/29]% v1.4h
%%%%%%%%%%%%%%%%%%%%%%%%%%%%%%%%%%%%%%%%%%%%%%%%%%%%%%%%%%%%%%%%%%%%%
\usepackage[%
 extension=pdf,%
 plainpages=false,%
 pdfpagelabels=true,%
 hyperindex=false,%
 pdflang={en},%
 pdftitle={footnoterange package example},%
 pdfauthor={H.-Martin Muench},%
 pdfsubject={Example for the footnoterange package},%
 pdfkeywords={LaTeX, footnoterange, H.-Martin Muench},%
 pdfview=FitH,%
 pdfstartview=FitH,%
 pdfpagelayout=OneColumn,%
 hyperfootnotes=true%
]{hyperref}[2012/11/06]% v6.83m

\usepackage{footnoterange}[2019/11/24]% v1.0c

\gdef\unit#1{\mathord{\thinspace\mathrm{#1}}}%
\listfiles
\begin{document}
\pagenumbering{arabic}
\section*{Example for footnoterange}

This example demonstrates the use of package\newline
\textsf{footnoterange}, v1.0c as of 2019/11/24 (HMM).\newline
The package does not use options.\newline
\verb|footnoterange-example.pdf| can be created by\newline
\verb|pdflatex footnoterange-example.tex|,\newline
and \verb|footnoterange-example.tex| can be created by\newline
\verb|tex footnoterange.dtx|.\newline
\verb|footnoterange.dtx| is at
\url{https://www.ctan.org/pkg/footnoterange}.\newline

\noindent For more details please see the documentation!\newline

\noindent Save per page about $200\unit{ml}$ water,
$2\unit{g}$ CO$_{2}$ and $2\unit{g}$ wood:\newline
Therefore please print only if this is really necessary.\newline

This\footnote{Lorem} text bears a multiplicity of footnotes.
Because the \verb|hyperref| package is used with option
\verb|hyperfootnotes=true|, the footnotes
\begin{footnoterange}
\footnote{ipsum}%
\footnote{dolor}%
\footnote{sit}%
\footnote{amet,}%
\footnote{consectetur}%
\footnote{adipisicing}%
\footnote{elit,}%
\footnote{sed}%
\end{footnoterange}
are hyperlinked.\footnote{do eiusmod\ldots} Using
\verb|hyperfootnotes=false| or not using \verb|hyperref|
will remove the hyperlinks to the footnotes.\medskip

This text also bears a multiplicity of footnotes,
but due to the use of the starred
\begin{footnoterange*}
\footnote{Ut purus elit,}%
\footnote{vestibulum ut,}%
\footnote{placerat ac,}%
\footnote{adipiscing vitae,}%
\footnote{felis.}%
\footnote{Curabitur}%
\footnote{dictum}%
\footnote{gravida}%
\end{footnoterange*}
form of the environment they are not
hyperlinked.\footnote{mauris\ldots}\medskip

The use of the optional parameter for footnotes
\begin{footnoterange}
\footnote[101]{Nam arcu libero,}%
\footnote[102]{nonummy eget,}%
\footnote[103]{consectetuer id,}%
\footnote[104]{vulputate a,}%
\footnote[105]{magna.}
\end{footnoterange}
is also possible, but should be used consistently.

\pagebreak

When there is only one footnote
\begin{footnoterange}
\footnote{a single footnote}
\end{footnoterange}
in the \verb|footnoterange| environment, only one
footnotemark is used and an info is written into the
\verb|log|-file.\medskip

The usage of other footnote-number-representations
(e.\,g.~\verb|\Roman|, \verb|\roman|, \verb|\Alph|, \verb|\alph|,
\verb|\fnsymbol|) is also possible, but two things must be taken
into account: The number of footnote references is restricted
(for example with \verb|\Alph| only references A to Z are possible),
which can be fixed e.\,g. with the
\href{https://www.ctan.org/pkg/alphalph}{\texttt{alphalph} package},
and references to footnote-symbol-ranges (\verb|\fnsymbol|)
are probably not very clear.

\end{document}
%</example>
%    \end{macrocode}
%
% \StopEventually{}
%
% \section{The implementation}
%
% We start off by checking that we are loading into \LaTeXe{} and
% announcing the name and version of this package.
%
%    \begin{macrocode}
%<*package>
%    \end{macrocode}
%
%    \begin{macrocode}
\NeedsTeXFormat{LaTeX2e}[2011/06/27]
\ProvidesPackage{footnoterange}[2019/11/24 v1.0c
            References to ranges of footnotes (HMM)]
%    \end{macrocode}
%
% A short description of the \xpackage{footnoterange} package:
%
%    \begin{macrocode}
%% Provides the footnoterange and footnoterange* environments to
%% refer to ranges of footnotes.
%    \end{macrocode}
%
% We need the \xpackage{ltxcmds} package:
%
%    \begin{macrocode}
\RequirePackage{ltxcmds}[2011/11/09]% v1.22; for "\ltx@ifpackageloaded"
\RequirePackage{letltxmacro}[2010/09/02]% v1.4; for "\GlobalLetLtxMacro"

%    \end{macrocode}
%
% We use |\ifHy@hyperfootnotes ... \else ... \fi|, but when the \xpackage{hyperref}
% package has not been loaded, |\ifHy@hyperfootnotes| is not defined, and \LaTeX
% relates the |\else| and/or |\fi| (wrongly) to other |\if|s or groups.
% Therefore the |\ifHy@hyperfootnotes| are placed in their own commands,
% which are not looked at when the \xpackage{hyperref} package has not been loaded.
%
% \pagebreak
%
% \begin{macro}{\fnr@ifhyperrfn}
%    \begin{macrocode}
\newcommand{\fnr@ifhyperrfn}{%
  \ifHy@hyperfootnotes% option hyperfootnotes=true
    \addtocounter{Hfootnote}{+1}%
    % Code from the hyperref package
      \global\let\Hy@saved@currentHref\@currentHref%
      \hyper@makecurrent{Hfootnote}%
      \global\let\Hy@footnote@currentHref\@currentHref%
      \global\let\@currentHref\Hy@saved@currentHref%
    % End of code from the hyperref package
  \fi%
}

%    \end{macrocode}
% \end{macro}
%
% \begin{macro}{\fnr@rangefootnote}
% The |\fnr@rangefootnote| is the internal replacement for the original\newline
% |\footnote| command inside of a |footnoterange| environment.
%
%    \begin{macrocode}
\newcommand{\fnr@rangefootnote}[2][\empty]{%
%    \end{macrocode}
%
% |#1| contains the optional argument, which is predefined as |\empty|.
%
%    \begin{macrocode}
  \nolinebreak%
  % no \footnotemark here
  \addtocounter\@mpfn{+1}%
  \ltx@ifpackageloaded{hyperref}{\fnr@ifhyperrfn}{\relax}%
%    \end{macrocode}
%
% See the explanation before the declaration of the |\fnr@ifhyperrfn| command!
%
%    \begin{macrocode}
  \xdef\fnr@opt{#1}% contains the optional argument
  \xdef\fnr@arabic{\arabic\@mpfn}%
  \protected@edef\fnr@formated{\thefootnote}%
%    \end{macrocode}
%
% The |\footnotemark| is not set (yet), but the |\footnotetext|,
% and it gets a label.
%
%    \begin{macrocode}
  \ifx\fnr@opt\empty%
    \footnotetext{\label{fnr:\fnr@arabic}#2}%
  \else%
    \PackageWarning{footnoterange}{%
      Custom footnote-numbers used in a footnoterange\MessageBreak%
      environment: Footnote with natural number\MessageBreak%
      \ifx\fnr@formated\fnr@arabic%
        \fnr@arabic\space%
      \else%
        \space\fnr@arabic\space (\fnr@formated )%
      \fi%
      \MessageBreak%
      was renumbered to \fnr@opt\MessageBreak%
     }%
    \ltx@ifpackageloaded{hyperref}{% hyperref loaded
      \footnotetext[#1]{\phantomsection\label{fnr:\fnr@arabic}#2}%
     }{% hyperref not loaded
      \footnotetext[#1]{\label{fnr:\fnr@arabic}#2}%
     }%
  \fi%
}

%    \end{macrocode}
% \end{macro}
%
%    \begin{macrocode}

%    \end{macrocode}
% \pagebreak
%
% |\ifHy@hyperfootnotes| \textquotedblleft encapsulation\textquotedblright{} again:
%
% \begin{macro}{\fnr@ifhyperfnrA}
%    \begin{macrocode}
\newcommand{\fnr@ifhyperfnrA}{%
  \ifHy@hyperfootnotes% option hyperfootnotes=true
    \hbox{\hyperref[fnr:\fnr@first]{%
          \@textsuperscript{\normalfont \ref*{fnr:\fnr@first}}}}%
  \else% option hyperfootnotes=false
    \hbox{\@textsuperscript{\normalfont \ref*{fnr:\fnr@first}}}%
  \fi%
}

%    \end{macrocode}
% \end{macro}
%
% \begin{macro}{\fnr@ifhyperfnrN}
%    \begin{macrocode}
\newcommand{\fnr@ifhyperfnrN}{%
  \ifHy@hyperfootnotes% option hyperfootnotes=true
    \hbox{\hyperref[fnr:\fnr@first]{%
            \@textsuperscript{\normalfont \ref*{fnr:\fnr@first}}}%
          \@textsuperscript{\normalfont --}%
          \hyperref[fnr:\arabic\@mpfn]{%
            \@textsuperscript{\normalfont \ref*{fnr:\arabic\@mpfn}}}}%
  \else% option hyperfootnotes=false
    \hbox{\@textsuperscript{\normalfont \ref*{fnr:\fnr@first}%
                                        --\ref*{fnr:\arabic\@mpfn}}}%
  \fi%
}

%    \end{macrocode}
% \end{macro}
%
% \begin{environment}{footnoterange}
% The |footnoterange| environment remembers the next footnote number as\newline
% |\fnr@first| and the current |\footnote| definition as |\fnr@origfn|
% and uses |\fnr@rangefootnote| instead.
%
%    \begin{macrocode}
\newenvironment{footnoterange}{%
  % at begin of the footnoterange environment
  \ifhmode \unskip \fi% FMi
  \addtocounter\@mpfn{+1}%
  \xdef\fnr@first{\arabic\@mpfn}%
  \addtocounter\@mpfn{-1}%
  \let\fnr@origfn\footnote%
  \let\footnote\fnr@rangefootnote%
  \GlobalLetLtxMacro{\fnr@origfnmark}{\footnotemark}%
  \GlobalLetLtxMacro{\footnotemark}{\fnr@rangefootnotemark}%
  \ignorespaces% FMi
%    \end{macrocode}
%
% End of the environment:
%
%    \begin{macrocode}
}{% at end of the footnoterange environment
  \ifhmode% FMi 
    \unskip% FMi
%    \end{macrocode}
%
% Remember the spacefactor for restoring after the environment has ended
% (only in |hmode|!). \% FMi
%
%    \begin{macrocode}
    \xdef\fnr@aftergroup{\spacefactor\the\spacefactor\relax}% FMi
  \else% FMi
%    \end{macrocode}
%
% Otherwise do something harmless. \% FMi
%
%    \begin{macrocode}
    \xdef\fnr@aftergroup{\relax}%
  \fi% FMi
%    \end{macrocode}
%
% At the end of the environment the previous definition of |\footnote| is restored
%
%    \begin{macrocode}
  \let\footnote\fnr@origfn%
  \GlobalLetLtxMacro{\footnotemark}{\fnr@origfnmark}%
%    \end{macrocode}
%
% and the references are placed:
%
%    \begin{macrocode}
  \ifx\fnr@first\fnr@arabic% There is only one reference in the environment.
    \PackageInfo{footnoterange}{%
      Environment footnoterange used,\MessageBreak%
      but only one footnote (number \fnr@first ) is contained%
      \MessageBreak%
     }%
    \ltx@ifpackageloaded{hyperref}{\fnr@ifhyperfnrA% hyperref package loaded
    }{% hyperref package not loaded
      \hbox{\@textsuperscript{\normalfont \ref{fnr:\fnr@first}}}%
     }%
  \else%
    \ltx@ifpackageloaded{hyperref}{\fnr@ifhyperfnrN% hyperref package loaded
    }{% hyperref package not loaded
      \hbox{\@textsuperscript{\normalfont \ref{fnr:\fnr@first}%
                                        --\ref{fnr:\arabic\@mpfn}}}%
     }%
  \fi%
%    \end{macrocode}
%
% Restore the spacefactor after the environment has ended. \% FMi
%
%    \begin{macrocode}
  \aftergroup\fnr@aftergroup% FMi
}

%    \end{macrocode}
% \end{environment}
%
% \begin{environment}{footnoterange*}
% The |footnoterange*| environment does the same as the |footnoterange|
% environment but suppresses all hyperlinks to footnotes inside.
%
%    \begin{macrocode}
\newenvironment{footnoterange*}{%
  % at begin of the footnoterange* environment
  \ifhmode \unskip \fi% FMi
  \addtocounter\@mpfn{+1}%
  \xdef\fnr@first{\arabic\@mpfn}%
  \addtocounter\@mpfn{-1}%
  \let\fnr@origfn\footnote%
  \let\footnote\fnr@rangefootnote%
  \GlobalLetLtxMacro{\fnr@origfnmark}{\footnotemark}%
  \GlobalLetLtxMacro{\footnotemark}{\fnr@rangefootnotemark}%
  \ignorespaces% FMi
}{% at end of the footnoterange* environment
  \ifhmode% FMi 
    \unskip% FMi
    \xdef\fnr@aftergroup{\spacefactor\the\spacefactor\relax}% FMi
  \else% FMi
    \xdef\fnr@aftergroup{\relax}%
  \fi% FMi
  \let\footnote\fnr@origfn%
  \GlobalLetLtxMacro{\footnotemark}{\fnr@origfnmark}%
  \ifx\fnr@first\fnr@arabic% There is only one reference in the environment.
    \PackageInfo{footnoterange}{%
      Environment footnoterange used,\MessageBreak%
      but only one footnote (number \fnr@first ) is contained%
      \MessageBreak%
     }%
%    \end{macrocode}
% \pagebreak
%    \begin{macrocode}
    \ltx@ifpackageloaded{hyperref}{% hyperref package loaded
      \hbox{\@textsuperscript{\normalfont \ref*{fnr:\fnr@first}}}%
    }{% hyperref package not loaded
      \hbox{\@textsuperscript{\normalfont \ref{fnr:\fnr@first}}}%
     }%
  \else%
    \ltx@ifpackageloaded{hyperref}{% hyperref package loaded
      \hbox{\@textsuperscript{\normalfont \ref*{fnr:\fnr@first}%
                                        --\ref*{fnr:\arabic\@mpfn}}}%
    }{% hyperref package not loaded
      \hbox{\@textsuperscript{\normalfont \ref{fnr:\fnr@first}%
                                        --\ref{fnr:\arabic\@mpfn}}}%
     }%
  \fi%
  \aftergroup\fnr@aftergroup% FMi
}

%    \end{macrocode}
% \end{environment}
%
% The \xpackage{footnoterange} package does not handle |\footnotemark|s,
% but at least the user is made aware of this:
%
% \begin{macro}{\fnr@rangefootnotemark}
%    \begin{macrocode}

\newcommand{\fnr@rangefootnotemark}[1][\empty]{%
  \PackageError{footnoterange}{%
     \string\footnotemark\space not supported in footnoterange(*) environment%
   }{The footnoterange package does not handle a \string\footnotemark.\MessageBreak%
     Either use \string\footnote\string{...\string}\space instead of%
     \string\footnotemark\space and \string\footnotetext\string{...\string}\MessageBreak%
     in footnoterange and footnoterange* environments,\MessageBreak%
     or do not use those environments around a \string\footnotemark\MessageBreak%
     (or \string\renewcommand\string{\string\fnr@rangefootnotemark\string}\MessageBreak%
     with your own fixing code).%
    }%
  \xdef\fnr@mopt{#1}%
  \ifx\fnr@mopt\empty%
    \fnr@origfnmark%
  \else%
    \fnr@origfnmark[#1]%
  \fi%
}

%    \end{macrocode}
% \end{macro}
%
%    \begin{macrocode}
%</package>
%    \end{macrocode}
%
% \pagebreak
%
% \section{Installation}
%
% \subsection{Downloads\label{ss:Downloads}}
%
% Everything is available on CTAN,
% \url{https://www.ctan.org/tex-archive/},
% but may need additional packages themselves.
%
% \DescribeMacro{footnoterange.dtx}
% For unpacking the \xfile{footnoterange.dtx} file
% and constructing the documentation it is required:
% \begin{description}
% \item[-] \TeX{} Format \LaTeXe{}, \url{https://www.CTAN.org/}
%
% \item[-] document class \xpackage{ltxdoc}, 2015/03/26, v2.0w,
%           \url{https://ctan.org/pkg/ltxdoc}
%
% \item[-] package \xpackage{holtxdoc}, 2012/03/21, v0.24,
%           \url{https://ctan.org/pkg/holtxdoc}
% \end{description}
%
% \DescribeMacro{footnoterange.sty}
% The |footnoterange.sty| for \LaTeXe{} (i.\,e. each document using
% the \xpackage{footnoterange} package) requires:
% \begin{description}
% \item[-] \TeX{} Format \LaTeXe{}, \url{https://www.CTAN.org/}
%
% \item[-] package \xpackage{ltxcmds}, 2011/04/18, v1.20,
%           \url{https://www.ctan.org/pkg/ltxcmds}
% \end{description}
%
% \DescribeMacro{footnoterange-example.tex}
% The |footnoterange-example.tex| requires the same files as all
% documents using the \xpackage{footnoterange} package and additionally:
% \begin{description}
% \item[-] class \xpackage{article}, 2007/10/19, v1.4h, from \xpackage{classes.dtx}:\newline
%           \url{https://www.ctan.org/tex-archive/macros/latex/base/}
%
% \item[-] package \xpackage{footnoterange}, 2019/11/24, v1.0c,\newline
%           \url{https://www.ctan.org/pkg/footnoterange}\newline
%           (Well, it is the example file for this package, and because you are
%            reading the documentation for the \xpackage{footnoterange} package,
%            it can be assumed that you already have some version of it --
%            is it the current one?)
%
% \item[-] package \xpackage{hyperref}, 2012/11/06, v6.83m,\newline
%           \url{https://www.ctan.org/pkg/hyperref}
% \end{description}
%
% \DescribeMacro{Oberdiek}
% \DescribeMacro{holtxdoc}
% \DescribeMacro{ltxcmds}
% \DescribeMacro{alphalph}
% All packages of \textsc{Heiko Oberdiek}'s bundle `oberdiek'
% (especially \xpackage{holtxdoc}, \xpackage{ltxcmds}, and \xpackage{alphalph})
% are also available in a TDS compliant ZIP archive:\newline
% \url{http://mirror.ctan.org/install/macros/latex/contrib/oberdiek.tds.zip}.\newline
% It is probably best to download and use this, because the packages in there
% should be both recent and compatible.
%
% \DescribeMacro{hyperref}
% \noindent \xpackage{hyperref} is not included in that bundle and needs to be
% downloaded separately,\newline
% \url{http://mirrors.ctan.org/install/macros/latex/contrib/hyperref.tds.zip}.
%
% \DescribeMacro{M\"{u}nch}
% A list of my packages can be found at \url{https://www.ctan.org/author/muench-hm}.
%
% \pagebreak
%
% \subsection{Package, unpacking TDS}
% \paragraph{Package.} This package is available on \url{https://www.CTAN.org}.
% \begin{description}
% \item[\url{http://mirror.ctan.org/macros/latex/contrib/footnoterange/footnoterange.dtx}]\hspace*{0.1cm}
%       The source file.
% \item[\url{http://mirror.ctan.org/macros/latex/contrib/footnoterange/footnoterange.pdf}]\hspace*{0.1cm}
%       The documentation.
% \item[\url{http://mirror.ctan.org/macros/latex/contrib/footnoterange/footnoterange-example.pdf}]\hspace*{0.1cm}
%       The compiled example file, as it should look like.
% \item[\url{http://mirror.ctan.org/macros/latex/contrib/footnoterange/README}]\hspace*{0.1cm}\newline\hspace*{0.1cm}
%       The README file.
% \end{description}
% There is also a footnoterange.tds.zip available:
% \begin{description}
% \item[\url{http://mirror.ctan.org/install/macros/latex/contrib/footnoterange.tds.zip}]\hspace*{0.1cm}
%       Everything in TDS compliant, compiled format.
% \end{description}
% which additionally contains\newline
% \begin{tabular}{ll}
% footnoterange.ins & The installation file.\\
% footnoterange.drv & The driver to generate the documentation.\\
% footnoterange.sty & The \xext{sty}le file.
% \end{tabular}
%
% \bigskip
%
% \noindent For required other packages, please see the preceding subsection.
%
% \paragraph{Unpacking.} The \xfile{.dtx} file is a self-extracting
% \docstrip{} archive. The files are extracted by running the
% \xfile{.dtx} through \plainTeX:
% \begin{quote}
%   \verb|tex footnoterange.dtx|
% \end{quote}
%
% About generating the documentation see paragraph~\ref{GenDoc} below.
%
% \paragraph{TDS.} Now the different files must be moved into
% the different directories in your installation TDS tree
% (also known as \xfile{texmf} tree):
% \begin{quote}
% \def\t{^^A
% \begin{tabular}{@{}>{\ttfamily}l@{ $\rightarrow$ }>{\ttfamily}l@{}}
%   footnoterange.sty & tex/latex/footnoterange/footnoterange.sty\\
%   footnoterange.pdf & doc/latex/footnoterange/footnoterange.pdf\\
%   footnoterange-example.tex & doc/latex/footnoterange/footnoterange-example.tex\\
%   footnoterange-example.pdf & doc/latex/footnoterange/footnoterange-example.pdf\\
%   footnoterange.dtx & source/latex/footnoterange/footnoterange.dtx\\
% \end{tabular}^^A
% }^^A
% \sbox0{\t}^^A
% \ifdim\wd0>\linewidth
%   \begingroup
%     \advance\linewidth by\leftmargin
%     \advance\linewidth by\rightmargin
%   \edef\x{\endgroup
%     \def\noexpand\lw{\the\linewidth}^^A
%   }\x
%   \def\lwbox{^^A
%     \leavevmode
%     \hbox to \linewidth{^^A
%       \kern-\leftmargin\relax
%       \hss
%       \usebox0
%       \hss
%       \kern-\rightmargin\relax
%     }^^A
%   }^^A
%   \ifdim\wd0>\lw
%     \sbox0{\small\t}^^A
%     \ifdim\wd0>\linewidth
%       \ifdim\wd0>\lw
%         \sbox0{\footnotesize\t}^^A
%         \ifdim\wd0>\linewidth
%           \ifdim\wd0>\lw
%             \sbox0{\scriptsize\t}^^A
%             \ifdim\wd0>\linewidth
%               \ifdim\wd0>\lw
%                 \sbox0{\tiny\t}^^A
%                 \ifdim\wd0>\linewidth
%                   \lwbox
%                 \else
%                   \usebox0
%                 \fi
%               \else
%                 \lwbox
%               \fi
%             \else
%               \usebox0
%             \fi
%           \else
%             \lwbox
%           \fi
%         \else
%           \usebox0
%         \fi
%       \else
%         \lwbox
%       \fi
%     \else
%       \usebox0
%     \fi
%   \else
%     \lwbox
%   \fi
% \else
%   \usebox0
% \fi
% \end{quote}
% If you have a \xfile{docstrip.cfg} that configures and enables \docstrip's
% TDS installing feature, then some files can already be in the right
% place, see the documentation of \docstrip.
%
% \subsection{Refresh file name databases}
%
% If your \TeX~distribution (\TeX{} Live, \mikTeX, \teTeX, \dots) relies on
% file name databases, you must refresh these. For example, \teTeX{} users run
% \verb|texhash| or \verb|mktexlsr|.
%
% \subsection{Some details for the interested}
%
% \paragraph{Unpacking with \LaTeX{}.}
% The \xfile{.dtx} chooses its action depending on the format:
% \begin{description}
% \item[\plainTeX:] Run \docstrip{} and extract the files.
% \item[\LaTeX:] Generate the documentation.
% \end{description}
% If you insist on using \LaTeX{} for \docstrip{} (really,
% \docstrip{} does not need \LaTeX ), then inform the autodetect routine
% about your intention:
% \begin{quote}
%   \verb|latex \let\install=y\input{footnoterange.dtx}|
% \end{quote}
% Do not forget to quote the argument according to the demands
% of your shell.
%
% \paragraph{Generating the documentation.\label{GenDoc}}
% You can use both the \xfile{.dtx} or the \xfile{.drv} to generate
% the documentation. The process can be configured by a
% configuration file \xfile{ltxdoc.cfg}. For instance, put the following
% line into this file, if you want to have A4 as paper format:
% \begin{quote}
%   \verb|\PassOptionsToClass{a4paper}{article}|
% \end{quote}
%
% \noindent An example follows how to generate the
% documentation with \pdfLaTeX :
%
% \begin{quote}
%\begin{verbatim}
%pdflatex footnoterange.drv
%makeindex -s gind.ist footnoterange.idx
%pdflatex footnoterange.drv
%makeindex -s gind.ist footnoterange.idx
%pdflatex footnoterange.drv
%\end{verbatim}
% \end{quote}
%
% \subsection{Compiling the example}
%
% The example file, \textsf{footnoterange-example.tex}, can be compiled via\newline
% \indent \verb|(pdf)latex footnoterange-example.tex|\newline
% but will need probably two compiler runs to get everything right.\newline
% \verb|footnoterange-example.tex| can be created by\newline
% \indent \verb|tex footnoterange.dtx|.\newline
% \verb|footnoterange.dtx| is at
% \url{https://www.ctan.org/pkg/footnoterange}.\newline
%
% \section{Acknowledgements}
%
% I (H.-Martin M\"{u}nch) would like to thank \textsc{Heiko Oberdiek}
% for providing a~lot~(!) of useful packages (from
% which I also got everything I know about creating a file in \xext{dtx}
% format, ok, say it: copying), \textsc{Frank Mittelbach} for several bug reports,
% for code for improving the package, and for pointing me to the \xpackage{fnpct}
% package as alternative, Willie Wong for reporting a bug and its fix,
% everybody of the CTAN team for managing CTAN,
% the \Newsgroup{comp.text.tex} and \Newsgroup{de.comp.text.tex} newsgroups,
% and everybody at \url{https://tex.stackexchange.com/}
% for their help in all things \hbox{\TeX{}.}
% This package was created as response to question
% \textquotedblleft Compressing consecutive footnote
% marks\textquotedblright{} by user001\newline
% (\url{https://tex.stackexchange.com/users/10127/user001}) on\newline
% \url{https://tex.stackexchange.com/q/44165/6865}.
%
% \newpage
% \phantomsection
% \begin{History}\label{History}
%   \begin{Version}{2011/02/11 v0.1ff.}
%     \item Some versions at \url{https://tex.stackexchange.com/q/44165/6865}.
%   \end{Version}
%   \begin{Version}{2012/02/17 v1.0a}
%     \item Package created and uploaded to CTAN.
%   \end{Version}
%   \begin{Version}{2018/10/15 v1.0b}
%     \item Fixed/updated minor details.
%     \item Bug fix: Did not work without \xpackage{hyperref}, wrong spacefactor.
%     \item Listing the \href{https://www.ctan.org/pkg/fnpct}{\xpackage{fnpct}}
%           package as alternative.
%   \end{Version}
%   \begin{Version}{2019/11/24 v1.0c}
%     \item Bug fix: Issue with footnote symbols.
%   \end{Version}
% \end{History}
%
% \bigskip
%
% When you find a mistake or have a suggestion for an improvement of this package,
% please send an e-mail to the maintainer, thanks! (Please see BUG REPORTS in the README.)
%
% \PrintIndex
%
% \Finale
\endinput|
% \end{quote}
% Do not forget to quote the argument according to the demands
% of your shell.
%
% \paragraph{Generating the documentation.\label{GenDoc}}
% You can use both the \xfile{.dtx} or the \xfile{.drv} to generate
% the documentation. The process can be configured by a
% configuration file \xfile{ltxdoc.cfg}. For instance, put the following
% line into this file, if you want to have A4 as paper format:
% \begin{quote}
%   \verb|\PassOptionsToClass{a4paper}{article}|
% \end{quote}
%
% \noindent An example follows how to generate the
% documentation with \pdfLaTeX :
%
% \begin{quote}
%\begin{verbatim}
%pdflatex footnoterange.drv
%makeindex -s gind.ist footnoterange.idx
%pdflatex footnoterange.drv
%makeindex -s gind.ist footnoterange.idx
%pdflatex footnoterange.drv
%\end{verbatim}
% \end{quote}
%
% \subsection{Compiling the example}
%
% The example file, \textsf{footnoterange-example.tex}, can be compiled via\newline
% \indent \verb|(pdf)latex footnoterange-example.tex|\newline
% but will need probably two compiler runs to get everything right.\newline
% \verb|footnoterange-example.tex| can be created by\newline
% \indent \verb|tex footnoterange.dtx|.\newline
% \verb|footnoterange.dtx| is at
% \url{https://www.ctan.org/pkg/footnoterange}.\newline
%
% \section{Acknowledgements}
%
% I (H.-Martin M\"{u}nch) would like to thank \textsc{Heiko Oberdiek}
% for providing a~lot~(!) of useful packages (from
% which I also got everything I know about creating a file in \xext{dtx}
% format, ok, say it: copying), \textsc{Frank Mittelbach} for several bug reports,
% for code for improving the package, and for pointing me to the \xpackage{fnpct}
% package as alternative, Willie Wong for reporting a bug and its fix,
% everybody of the CTAN team for managing CTAN,
% the \Newsgroup{comp.text.tex} and \Newsgroup{de.comp.text.tex} newsgroups,
% and everybody at \url{https://tex.stackexchange.com/}
% for their help in all things \hbox{\TeX{}.}
% This package was created as response to question
% \textquotedblleft Compressing consecutive footnote
% marks\textquotedblright{} by user001\newline
% (\url{https://tex.stackexchange.com/users/10127/user001}) on\newline
% \url{https://tex.stackexchange.com/q/44165/6865}.
%
% \newpage
% \phantomsection
% \begin{History}\label{History}
%   \begin{Version}{2011/02/11 v0.1ff.}
%     \item Some versions at \url{https://tex.stackexchange.com/q/44165/6865}.
%   \end{Version}
%   \begin{Version}{2012/02/17 v1.0a}
%     \item Package created and uploaded to CTAN.
%   \end{Version}
%   \begin{Version}{2018/10/15 v1.0b}
%     \item Fixed/updated minor details.
%     \item Bug fix: Did not work without \xpackage{hyperref}, wrong spacefactor.
%     \item Listing the \href{https://www.ctan.org/pkg/fnpct}{\xpackage{fnpct}}
%           package as alternative.
%   \end{Version}
%   \begin{Version}{2019/11/24 v1.0c}
%     \item Bug fix: Issue with footnote symbols.
%   \end{Version}
% \end{History}
%
% \bigskip
%
% When you find a mistake or have a suggestion for an improvement of this package,
% please send an e-mail to the maintainer, thanks! (Please see BUG REPORTS in the README.)
%
% \PrintIndex
%
% \Finale
\endinput
%        (quote the arguments according to the demands of your shell)
%
% Documentation:
%    (a) If footnoterange.drv is present:
%           (pdf)latex footnoterange.drv
%           makeindex -s gind.ist footnoterange.idx
%           (pdf)latex footnoterange.drv
%           makeindex -s gind.ist footnoterange.idx
%           (pdf)latex footnoterange.drv
%    (b) Without footnoterange.drv:
%           (pdf)latex footnoterange.dtx
%           makeindex -s gind.ist footnoterange.idx
%           (pdf)latex footnoterange.dtx
%           makeindex -s gind.ist footnoterange.idx
%           (pdf)latex footnoterange.dtx
%
%    The class ltxdoc loads the configuration file ltxdoc.cfg
%    if available. Here you can specify further options, e.g.
%    use DIN A4 as paper format:
%       \PassOptionsToClass{a4paper}{article}
%
% Installation:
%    TDS:tex/latex/footnoterange/footnoterange.sty
%    TDS:doc/latex/footnoterange/footnoterange.pdf
%    TDS:doc/latex/footnoterange/footnoterange-example.tex
%    TDS:doc/latex/footnoterange/footnoterange-example.pdf
%    TDS:source/latex/footnoterange/footnoterange.dtx
%
%<*ignore>
\begingroup
  \catcode123=1 %
  \catcode125=2 %
  \def\x{LaTeX2e}%
\expandafter\endgroup
\ifcase 0\ifx\install y1\fi\expandafter
         \ifx\csname processbatchFile\endcsname\relax\else1\fi
         \ifx\fmtname\x\else 1\fi\relax
\else\csname fi\endcsname
%</ignore>
%<*install>
\input docstrip.tex
\Msg{*********************************************************************************}
\Msg{* Installation}
\Msg{* Package: footnoterange 2019/11/24 v1.0c References to ranges of footnotes (HMM)}
\Msg{*********************************************************************************}

\keepsilent
\askforoverwritefalse

\let\MetaPrefix\relax
\preamble

This is a generated file.

Project: footnoterange
Version: 2019/11/24 v1.0c

Copyright (C) 2012 - 2019 by
    H.-Martin M"unch <Martin dot Muench at Uni-Bonn dot de>
Portions of code copyrighted by other people as marked;
lines marked "FMi": code by Frank Mittelbach, thanks!

The usual disclaimer applies:
If it doesn't work right that's your problem.
(Nevertheless, please send an e-mail to the maintainer
 when you find an error in this package.)

This work may be distributed and/or modified under the
conditions of the LaTeX Project Public License, either
version 1.3c of this license or (at your option) any later
version. This version of this license is in
   https://www.latex-project.org/lppl/lppl-1-3c.txt
and the latest version of this license is in
   https://www.latex-project.org/lppl.txt
and version 1.3c or later is part of all distributions of
LaTeX version 2005/12/01 or later.

This work has the LPPL maintenance status "maintained".

The Current Maintainer of this work is H.-Martin Muench.

This work consists of the main source file footnoterange.dtx,
the README, and the derived files
   footnoterange.sty, footnoterange.pdf,
   footnoterange.ins, footnoterange.drv,
   footnoterange-example.tex, footnoterange-example.pdf.

In memoriam
 Claudia Simone Barth + 1996/01/30
 Tommy Muench + 2014/01/02
 Hans-Klaus Muench + 2014/08/24

\endpreamble
\let\MetaPrefix\DoubleperCent

\generate{%
  \file{footnoterange.ins}{\from{footnoterange.dtx}{install}}%
  \file{footnoterange.drv}{\from{footnoterange.dtx}{driver}}%
  \usedir{tex/latex/footnoterange}%
  \file{footnoterange.sty}{\from{footnoterange.dtx}{package}}%
  \usedir{doc/latex/footnoterange}%
  \file{footnoterange-example.tex}{\from{footnoterange.dtx}{example}}%
}

\catcode32=13\relax% active space
\let =\space%
\Msg{************************************************************************}
\Msg{*}
\Msg{* To finish the installation you have to move the following}
\Msg{* file into a directory searched by TeX:}
\Msg{*}
\Msg{* footnoterange.sty}
\Msg{*}
\Msg{* To produce the documentation run the file `footnoterange.drv'}
\Msg{* through (pdf)LaTeX, e.g.}
\Msg{*  pdflatex footnoterange.drv}
\Msg{*  makeindex -s gind.ist footnoterange.idx}
\Msg{*  pdflatex footnoterange.drv}
\Msg{*  makeindex -s gind.ist footnoterange.idx}
\Msg{*  pdflatex footnoterange.drv}
\Msg{*}
\Msg{* At least three runs are necessary e.g. to get the}
\Msg{*  references right!}
\Msg{*}
\Msg{* Happy TeXing!}
\Msg{*}
\Msg{************************************************************************}

\endbatchfile
%</install>
%<*ignore>
\fi
%</ignore>
%
% \section{The documentation driver file}
%
% The next bit of code contains the documentation driver file for
% \TeX , i.\,e., the file that will produce the documentation you
% are currently reading. It will be extracted from this file by the
% \verb|docstrip| programme. That is, run \LaTeX{} on \verb|docstrip|
% and specify the \verb|driver| option when \verb|docstrip|
% asks for options.
%
%    \begin{macrocode}
%<*driver>
\NeedsTeXFormat{LaTeX2e}[2015/01/01]
\ProvidesFile{footnoterange.drv}[2019/11/24 v1.0c
            References to ranges of footnotes (HMM)]
\documentclass{ltxdoc}[2015/03/26]% v2.0w
\usepackage{holtxdoc}[2012/03/21]%  v0.24
%% footnoterange may work with earlier versions of LaTeX2e and those
%% class and package, but this was not tested.
%% Please consider updating your LaTeX, class, and package
%% to the most recent version (if they are not already the most
%% recent version).
\hypersetup{%
 pdfsubject={References to ranges of footnotes (HMM)},%
 pdfkeywords={LaTeX, footnoterange, footnotes, footnote, reference, H.-Martin Muench},%
 pdfencoding=auto,%
 pdflang={en},%
 breaklinks=true,%
 linktoc=all,%
 pdfstartview=FitH,%
 pdfpagelayout=OneColumn,%
 bookmarksnumbered=true,%
 bookmarksopen=true,%
 bookmarksopenlevel=3,%
 pdfmenubar=true,%
 pdftoolbar=true,%
 pdfwindowui=true,%
 pdfnewwindow=true%
}
\CodelineIndex
\hyphenation{printing docu-ment}
\gdef\unit#1{\mathord{\thinspace\mathrm{#1}}}
\begin{document}
  \DocInput{footnoterange.dtx}%
\end{document}
%</driver>
%    \end{macrocode}
%
% \fi
%
% \CheckSum{321}
%
% \CharacterTable
%  {Upper-case    \A\B\C\D\E\F\G\H\I\J\K\L\M\N\O\P\Q\R\S\T\U\V\W\X\Y\Z
%   Lower-case    \a\b\c\d\e\f\g\h\i\j\k\l\m\n\o\p\q\r\s\t\u\v\w\x\y\z
%   Digits        \0\1\2\3\4\5\6\7\8\9
%   Exclamation   \!     Double quote  \"     Hash (number) \#
%   Dollar        \$     Percent       \%     Ampersand     \&
%   Acute accent  \'     Left paren    \(     Right paren   \)
%   Asterisk      \*     Plus          \+     Comma         \,
%   Minus         \-     Point         \.     Solidus       \/
%   Colon         \:     Semicolon     \;     Less than     \<
%   Equals        \=     Greater than  \>     Question mark \?
%   Commercial at \@     Left bracket  \[     Backslash     \\
%   Right bracket \]     Circumflex    \^     Underscore    \_
%   Grave accent  \`     Left brace    \{     Vertical bar  \|
%   Right brace   \}     Tilde         \~}
%
% \GetFileInfo{footnoterange.drv}
%
% \begingroup
%   \def\x{\#,\$,\^,\_,\~,\ ,\&,\{,\},\%}%
%   \makeatletter
%   \@onelevel@sanitize\x
% \expandafter\endgroup
% \expandafter\DoNotIndex\expandafter{\x}
% \expandafter\DoNotIndex\expandafter{\string\ }
% \begingroup
%   \makeatletter
%     \lccode`9=32\relax
%     \lowercase{%^^A
%       \edef\x{\noexpand\DoNotIndex{\@backslashchar9}}%^^A
%     }%^^A
%   \expandafter\endgroup\x
% \DoNotIndex{\,,\\}
% \DoNotIndex{\documentclass,\usepackage,\ProvidesPackage,\begin,\end}
% \DoNotIndex{\MessageBreak}
% \DoNotIndex{\NeedsTeXFormat,\DoNotIndex,\verb}
% \DoNotIndex{\def,\edef,\gdef,\global,\xdef}
% \DoNotIndex{\ifx,\listfiles,\mathord,\mathrm}
% \DoNotIndex{\bigskip,\space,\thinspace,\ldots}
% \DoNotIndex{\indent,\noindent,\newline,\linebreak,\pagebreak,\newpage}
% \DoNotIndex{\textbf,\textit,\textsf,\textsc,\texttt,\textquotedblleft,\textquotedblright}
% \DoNotIndex{\plainTeX,\TeX,\LaTeX,\pdfLaTeX}
% \DoNotIndex{\section,\pagenumbering,\smallskip}
% \DoNotIndex{\@ifpackageloaded}
% \DoNotIndex{\arabic, \Alph, \alph, \alphalph, \Roman, \roman, \fnsymbol}
%
% \title{The \xpackage{footnoterange} package}
% \date{2019/11/24 v1.0c}
% \author{H.-Martin M\"{u}nch\\\xemail{Martin.Muench at Uni-Bonn.de}}
%
% \maketitle
%
% \begin{abstract}
% \noindent This package provides the environments |footnoterange| and
% |footnoterange*|. Multiple footnotes inside these environments
% are not referenced as e.\,g.
% \makeatletter%
% \hbox{\textquotedblleft \@textsuperscript {\normalfont 1 2 3}\textquotedblright}
% but as \hbox{\textquotedblleft \@textsuperscript {\normalfont 1--3}\textquotedblright ,\ }
% \makeatother%
% i.\,e. the range (from first to last referred footnote at that place) is given.
% Depending on loading of \xpackage{hyperref} and use of its |hyperfootnotes|-option
% the references are hyperlinked. (References to footnotes in |footnoterange*|
% are never hyperlinked.)
% \end{abstract}
%
% \bigskip
%
% \noindent Disclaimer for web links: The author is not responsible for any contents
% referred to in this work unless if having full knowledge of illegal contents.
% If any damage occurs by the use of information presented there, only the
% author of the respective pages might be liable, not the one who has referred
% to these pages.
%
% \noindent Save per page about $200\unit{ml}$ water,
% $2\unit{g}$ CO$_{2}$ and $2\unit{g}$ wood:
% Therefore please print only if this is really necessary.
%
% \medskip
%
% \tableofcontents
%
%\newpage
%
% \section{Introduction}
% \indent This \LaTeXe{} package provides the environments |footnoterange| and\newline
% |footnoterange*|. Multiple footnotes inside these environments are not referenced
% as e.\,g.
% \makeatletter%
% \hbox{\textquotedblleft \@textsuperscript {\normalfont 1 2 3}\textquotedblright}
% but as \hbox{\textquotedblleft \@textsuperscript {\normalfont 1--3}\textquotedblright ,\ }
% \makeatother%
% i.\,e. the range (from first to last
% referred footnote at that place) is given. Depending on loading of
% \xpackage{hyperref} and use of its |hyperfootnotes|-option the references are
% hyperlinked. References to footnotes in |footnoterange*| are never hyperlinked.\newline
% |\footnotemark\footnotetext| instead of |\footnote| is not supported inside
% |footnoterange| and |footnoterange*| environments.
%
% \bigskip
%
% \section{Usage}
%
% \indent Just load the package placing
% \begin{quote}
%   |\usepackage{footnoterange}|
% \end{quote}
% \noindent in the preamble of your \LaTeXe{} source file. There are no options to
% be used. The package should be loaded after \xpackage{hyperref} (if that package is used)
% and after any other package redefining footnotes. Usage of the |footnoterange| (or
% |footnoterange*|) environment is simple: Just encapsulate the footnotes, which shall
% be referred as range instead of referencing as single ones (see the example file).
%
% About hyperlinks:\newline
% Footnote-references in a |footnoterange*| environment are never hyperlinked.\newline
% Footnote-references in a |footnoterange| environment may be hyperlinked:
% \begin{itemize}
%  \item When the \xpackage{hyperref} package is not loaded, they are not
%         hyperlinked (naturally).
%  \item When the \xpackage{hyperref} package is loaded, it depends on the
%         |hyperfootnotes| option of the \xpackage{hyperref} package:
%  \begin{itemize}
%   \item With |hyperfootnotes=true|, the references are hyperlinked.
%   \item With |hyperfootnotes=false|, the references are not hyperlinked.
%  \end{itemize}
% \end{itemize}
% |\footnote|s inside of the |footnoterange|/{}|footnoterange*| environment
% can use the optional parameter, i.\,e. |\footnote[42]{text}| even if this is the
% third footnote. When giving ranges as reference, this does not seem to be
% such a good idea, if this is not done consistently (i.\,e. footnotes
% |[42], [43], [44]| resulting in 42--44 is OK,
% but |[42]| (original~2), unchanged~3, unchanged~4 resulting in
% 42--4 is confusing). A warning about these changed footnotemarks is written
% into the \xfile{.log}-file.
%
% When there is only a single footnote in a |footnoterange| environment,
% this does not produce some reference like %
% \makeatletter%
% \hbox{\textquotedblleft \@textsuperscript {\normalfont 7--7}\textquotedblright{},\ }%
% \makeatother%
% but only one footnotemark is used automatically. Because it is strange to use
% a |footnoterange| environment without a range of footnotes, an info is written
% into the \xfile{.log}-file about this.
%
% The usage of other footnote-number-representations
% (e.\,g.~\verb|\Roman|, \verb|\roman|, \verb|\Alph|, \verb|\alph|,
% \verb|\fnsymbol|) is also possible, but two things must be taken
% into account: The number of footnote references is restricted (for
% example with \verb|\Alph| only references A to Z are possible), which
% can be fixed e.\,g. with the \url{https://ctan.org/pkg/alphalph} package
% (but not automatically), and references to footnote-symbol-ranges (\verb|\fnsymbol|)
% are probably not very clear.
%
% \subsection*{Options}
% \indent The \xpackage{footnoterange} package takes no options.
%
% \section{Alternatives\label{sec:Alternatives}}
%
% The \xpackage{fnpct} (\url{https://www.ctan.org/pkg/fnpct}) package allows to set
% multiple footnotes (|\multfootnote|), suppress a footnotemark \hbox{(|*|),}
% and customize the multi-footnote-separator \hbox{(|\setfnpct{mult-fn-sep={|
% \emph{something} |}}|)}.
% Combining these with stepping of the footnotecounter gives a footnoterange:
% 
% \begin{verbatim}
% \documentclass{article}
% \usepackage{hyperref}
% \usepackage{fnpct}
% \begin{document}
% The three little pigs built their houses
% out of straw\footnote{not to be confused with hay},
% sticks\footnote{or lumber according to some sources}
% and bricks%
% \makeatletter%
% \setfnpct{mult-fn-sep={--}}%
% \multfootnote{probably fired clay bricks\addtocounter\@mpfn{+1};%
% *or something else;%
% what do I know}. %
% \makeatother%
% \end{document}
% \end{verbatim}
% 
% \noindent If you programmed or found another alternative, which is available at
% \url{https://www.CTAN.org/}, please send an e-mail to me with the name,
% location at CTAN, and a short notice, and I will probably list it here.
%
% \section{Example}
%
%    \begin{macrocode}
%<*example>
\documentclass[british]{article}[2014/09/29]% v1.4h
%%%%%%%%%%%%%%%%%%%%%%%%%%%%%%%%%%%%%%%%%%%%%%%%%%%%%%%%%%%%%%%%%%%%%
\usepackage[%
 extension=pdf,%
 plainpages=false,%
 pdfpagelabels=true,%
 hyperindex=false,%
 pdflang={en},%
 pdftitle={footnoterange package example},%
 pdfauthor={H.-Martin Muench},%
 pdfsubject={Example for the footnoterange package},%
 pdfkeywords={LaTeX, footnoterange, H.-Martin Muench},%
 pdfview=FitH,%
 pdfstartview=FitH,%
 pdfpagelayout=OneColumn,%
 hyperfootnotes=true%
]{hyperref}[2012/11/06]% v6.83m

\usepackage{footnoterange}[2019/11/24]% v1.0c

\gdef\unit#1{\mathord{\thinspace\mathrm{#1}}}%
\listfiles
\begin{document}
\pagenumbering{arabic}
\section*{Example for footnoterange}

This example demonstrates the use of package\newline
\textsf{footnoterange}, v1.0c as of 2019/11/24 (HMM).\newline
The package does not use options.\newline
\verb|footnoterange-example.pdf| can be created by\newline
\verb|pdflatex footnoterange-example.tex|,\newline
and \verb|footnoterange-example.tex| can be created by\newline
\verb|tex footnoterange.dtx|.\newline
\verb|footnoterange.dtx| is at
\url{https://www.ctan.org/pkg/footnoterange}.\newline

\noindent For more details please see the documentation!\newline

\noindent Save per page about $200\unit{ml}$ water,
$2\unit{g}$ CO$_{2}$ and $2\unit{g}$ wood:\newline
Therefore please print only if this is really necessary.\newline

This\footnote{Lorem} text bears a multiplicity of footnotes.
Because the \verb|hyperref| package is used with option
\verb|hyperfootnotes=true|, the footnotes
\begin{footnoterange}
\footnote{ipsum}%
\footnote{dolor}%
\footnote{sit}%
\footnote{amet,}%
\footnote{consectetur}%
\footnote{adipisicing}%
\footnote{elit,}%
\footnote{sed}%
\end{footnoterange}
are hyperlinked.\footnote{do eiusmod\ldots} Using
\verb|hyperfootnotes=false| or not using \verb|hyperref|
will remove the hyperlinks to the footnotes.\medskip

This text also bears a multiplicity of footnotes,
but due to the use of the starred
\begin{footnoterange*}
\footnote{Ut purus elit,}%
\footnote{vestibulum ut,}%
\footnote{placerat ac,}%
\footnote{adipiscing vitae,}%
\footnote{felis.}%
\footnote{Curabitur}%
\footnote{dictum}%
\footnote{gravida}%
\end{footnoterange*}
form of the environment they are not
hyperlinked.\footnote{mauris\ldots}\medskip

The use of the optional parameter for footnotes
\begin{footnoterange}
\footnote[101]{Nam arcu libero,}%
\footnote[102]{nonummy eget,}%
\footnote[103]{consectetuer id,}%
\footnote[104]{vulputate a,}%
\footnote[105]{magna.}
\end{footnoterange}
is also possible, but should be used consistently.

\pagebreak

When there is only one footnote
\begin{footnoterange}
\footnote{a single footnote}
\end{footnoterange}
in the \verb|footnoterange| environment, only one
footnotemark is used and an info is written into the
\verb|log|-file.\medskip

The usage of other footnote-number-representations
(e.\,g.~\verb|\Roman|, \verb|\roman|, \verb|\Alph|, \verb|\alph|,
\verb|\fnsymbol|) is also possible, but two things must be taken
into account: The number of footnote references is restricted
(for example with \verb|\Alph| only references A to Z are possible),
which can be fixed e.\,g. with the
\href{https://www.ctan.org/pkg/alphalph}{\texttt{alphalph} package},
and references to footnote-symbol-ranges (\verb|\fnsymbol|)
are probably not very clear.

\end{document}
%</example>
%    \end{macrocode}
%
% \StopEventually{}
%
% \section{The implementation}
%
% We start off by checking that we are loading into \LaTeXe{} and
% announcing the name and version of this package.
%
%    \begin{macrocode}
%<*package>
%    \end{macrocode}
%
%    \begin{macrocode}
\NeedsTeXFormat{LaTeX2e}[2011/06/27]
\ProvidesPackage{footnoterange}[2019/11/24 v1.0c
            References to ranges of footnotes (HMM)]
%    \end{macrocode}
%
% A short description of the \xpackage{footnoterange} package:
%
%    \begin{macrocode}
%% Provides the footnoterange and footnoterange* environments to
%% refer to ranges of footnotes.
%    \end{macrocode}
%
% We need the \xpackage{ltxcmds} package:
%
%    \begin{macrocode}
\RequirePackage{ltxcmds}[2011/11/09]% v1.22; for "\ltx@ifpackageloaded"
\RequirePackage{letltxmacro}[2010/09/02]% v1.4; for "\GlobalLetLtxMacro"

%    \end{macrocode}
%
% We use |\ifHy@hyperfootnotes ... \else ... \fi|, but when the \xpackage{hyperref}
% package has not been loaded, |\ifHy@hyperfootnotes| is not defined, and \LaTeX
% relates the |\else| and/or |\fi| (wrongly) to other |\if|s or groups.
% Therefore the |\ifHy@hyperfootnotes| are placed in their own commands,
% which are not looked at when the \xpackage{hyperref} package has not been loaded.
%
% \pagebreak
%
% \begin{macro}{\fnr@ifhyperrfn}
%    \begin{macrocode}
\newcommand{\fnr@ifhyperrfn}{%
  \ifHy@hyperfootnotes% option hyperfootnotes=true
    \addtocounter{Hfootnote}{+1}%
    % Code from the hyperref package
      \global\let\Hy@saved@currentHref\@currentHref%
      \hyper@makecurrent{Hfootnote}%
      \global\let\Hy@footnote@currentHref\@currentHref%
      \global\let\@currentHref\Hy@saved@currentHref%
    % End of code from the hyperref package
  \fi%
}

%    \end{macrocode}
% \end{macro}
%
% \begin{macro}{\fnr@rangefootnote}
% The |\fnr@rangefootnote| is the internal replacement for the original\newline
% |\footnote| command inside of a |footnoterange| environment.
%
%    \begin{macrocode}
\newcommand{\fnr@rangefootnote}[2][\empty]{%
%    \end{macrocode}
%
% |#1| contains the optional argument, which is predefined as |\empty|.
%
%    \begin{macrocode}
  \nolinebreak%
  % no \footnotemark here
  \addtocounter\@mpfn{+1}%
  \ltx@ifpackageloaded{hyperref}{\fnr@ifhyperrfn}{\relax}%
%    \end{macrocode}
%
% See the explanation before the declaration of the |\fnr@ifhyperrfn| command!
%
%    \begin{macrocode}
  \xdef\fnr@opt{#1}% contains the optional argument
  \xdef\fnr@arabic{\arabic\@mpfn}%
  \protected@edef\fnr@formated{\thefootnote}%
%    \end{macrocode}
%
% The |\footnotemark| is not set (yet), but the |\footnotetext|,
% and it gets a label.
%
%    \begin{macrocode}
  \ifx\fnr@opt\empty%
    \footnotetext{\label{fnr:\fnr@arabic}#2}%
  \else%
    \PackageWarning{footnoterange}{%
      Custom footnote-numbers used in a footnoterange\MessageBreak%
      environment: Footnote with natural number\MessageBreak%
      \ifx\fnr@formated\fnr@arabic%
        \fnr@arabic\space%
      \else%
        \space\fnr@arabic\space (\fnr@formated )%
      \fi%
      \MessageBreak%
      was renumbered to \fnr@opt\MessageBreak%
     }%
    \ltx@ifpackageloaded{hyperref}{% hyperref loaded
      \footnotetext[#1]{\phantomsection\label{fnr:\fnr@arabic}#2}%
     }{% hyperref not loaded
      \footnotetext[#1]{\label{fnr:\fnr@arabic}#2}%
     }%
  \fi%
}

%    \end{macrocode}
% \end{macro}
%
%    \begin{macrocode}

%    \end{macrocode}
% \pagebreak
%
% |\ifHy@hyperfootnotes| \textquotedblleft encapsulation\textquotedblright{} again:
%
% \begin{macro}{\fnr@ifhyperfnrA}
%    \begin{macrocode}
\newcommand{\fnr@ifhyperfnrA}{%
  \ifHy@hyperfootnotes% option hyperfootnotes=true
    \hbox{\hyperref[fnr:\fnr@first]{%
          \@textsuperscript{\normalfont \ref*{fnr:\fnr@first}}}}%
  \else% option hyperfootnotes=false
    \hbox{\@textsuperscript{\normalfont \ref*{fnr:\fnr@first}}}%
  \fi%
}

%    \end{macrocode}
% \end{macro}
%
% \begin{macro}{\fnr@ifhyperfnrN}
%    \begin{macrocode}
\newcommand{\fnr@ifhyperfnrN}{%
  \ifHy@hyperfootnotes% option hyperfootnotes=true
    \hbox{\hyperref[fnr:\fnr@first]{%
            \@textsuperscript{\normalfont \ref*{fnr:\fnr@first}}}%
          \@textsuperscript{\normalfont --}%
          \hyperref[fnr:\arabic\@mpfn]{%
            \@textsuperscript{\normalfont \ref*{fnr:\arabic\@mpfn}}}}%
  \else% option hyperfootnotes=false
    \hbox{\@textsuperscript{\normalfont \ref*{fnr:\fnr@first}%
                                        --\ref*{fnr:\arabic\@mpfn}}}%
  \fi%
}

%    \end{macrocode}
% \end{macro}
%
% \begin{environment}{footnoterange}
% The |footnoterange| environment remembers the next footnote number as\newline
% |\fnr@first| and the current |\footnote| definition as |\fnr@origfn|
% and uses |\fnr@rangefootnote| instead.
%
%    \begin{macrocode}
\newenvironment{footnoterange}{%
  % at begin of the footnoterange environment
  \ifhmode \unskip \fi% FMi
  \addtocounter\@mpfn{+1}%
  \xdef\fnr@first{\arabic\@mpfn}%
  \addtocounter\@mpfn{-1}%
  \let\fnr@origfn\footnote%
  \let\footnote\fnr@rangefootnote%
  \GlobalLetLtxMacro{\fnr@origfnmark}{\footnotemark}%
  \GlobalLetLtxMacro{\footnotemark}{\fnr@rangefootnotemark}%
  \ignorespaces% FMi
%    \end{macrocode}
%
% End of the environment:
%
%    \begin{macrocode}
}{% at end of the footnoterange environment
  \ifhmode% FMi 
    \unskip% FMi
%    \end{macrocode}
%
% Remember the spacefactor for restoring after the environment has ended
% (only in |hmode|!). \% FMi
%
%    \begin{macrocode}
    \xdef\fnr@aftergroup{\spacefactor\the\spacefactor\relax}% FMi
  \else% FMi
%    \end{macrocode}
%
% Otherwise do something harmless. \% FMi
%
%    \begin{macrocode}
    \xdef\fnr@aftergroup{\relax}%
  \fi% FMi
%    \end{macrocode}
%
% At the end of the environment the previous definition of |\footnote| is restored
%
%    \begin{macrocode}
  \let\footnote\fnr@origfn%
  \GlobalLetLtxMacro{\footnotemark}{\fnr@origfnmark}%
%    \end{macrocode}
%
% and the references are placed:
%
%    \begin{macrocode}
  \ifx\fnr@first\fnr@arabic% There is only one reference in the environment.
    \PackageInfo{footnoterange}{%
      Environment footnoterange used,\MessageBreak%
      but only one footnote (number \fnr@first ) is contained%
      \MessageBreak%
     }%
    \ltx@ifpackageloaded{hyperref}{\fnr@ifhyperfnrA% hyperref package loaded
    }{% hyperref package not loaded
      \hbox{\@textsuperscript{\normalfont \ref{fnr:\fnr@first}}}%
     }%
  \else%
    \ltx@ifpackageloaded{hyperref}{\fnr@ifhyperfnrN% hyperref package loaded
    }{% hyperref package not loaded
      \hbox{\@textsuperscript{\normalfont \ref{fnr:\fnr@first}%
                                        --\ref{fnr:\arabic\@mpfn}}}%
     }%
  \fi%
%    \end{macrocode}
%
% Restore the spacefactor after the environment has ended. \% FMi
%
%    \begin{macrocode}
  \aftergroup\fnr@aftergroup% FMi
}

%    \end{macrocode}
% \end{environment}
%
% \begin{environment}{footnoterange*}
% The |footnoterange*| environment does the same as the |footnoterange|
% environment but suppresses all hyperlinks to footnotes inside.
%
%    \begin{macrocode}
\newenvironment{footnoterange*}{%
  % at begin of the footnoterange* environment
  \ifhmode \unskip \fi% FMi
  \addtocounter\@mpfn{+1}%
  \xdef\fnr@first{\arabic\@mpfn}%
  \addtocounter\@mpfn{-1}%
  \let\fnr@origfn\footnote%
  \let\footnote\fnr@rangefootnote%
  \GlobalLetLtxMacro{\fnr@origfnmark}{\footnotemark}%
  \GlobalLetLtxMacro{\footnotemark}{\fnr@rangefootnotemark}%
  \ignorespaces% FMi
}{% at end of the footnoterange* environment
  \ifhmode% FMi 
    \unskip% FMi
    \xdef\fnr@aftergroup{\spacefactor\the\spacefactor\relax}% FMi
  \else% FMi
    \xdef\fnr@aftergroup{\relax}%
  \fi% FMi
  \let\footnote\fnr@origfn%
  \GlobalLetLtxMacro{\footnotemark}{\fnr@origfnmark}%
  \ifx\fnr@first\fnr@arabic% There is only one reference in the environment.
    \PackageInfo{footnoterange}{%
      Environment footnoterange used,\MessageBreak%
      but only one footnote (number \fnr@first ) is contained%
      \MessageBreak%
     }%
%    \end{macrocode}
% \pagebreak
%    \begin{macrocode}
    \ltx@ifpackageloaded{hyperref}{% hyperref package loaded
      \hbox{\@textsuperscript{\normalfont \ref*{fnr:\fnr@first}}}%
    }{% hyperref package not loaded
      \hbox{\@textsuperscript{\normalfont \ref{fnr:\fnr@first}}}%
     }%
  \else%
    \ltx@ifpackageloaded{hyperref}{% hyperref package loaded
      \hbox{\@textsuperscript{\normalfont \ref*{fnr:\fnr@first}%
                                        --\ref*{fnr:\arabic\@mpfn}}}%
    }{% hyperref package not loaded
      \hbox{\@textsuperscript{\normalfont \ref{fnr:\fnr@first}%
                                        --\ref{fnr:\arabic\@mpfn}}}%
     }%
  \fi%
  \aftergroup\fnr@aftergroup% FMi
}

%    \end{macrocode}
% \end{environment}
%
% The \xpackage{footnoterange} package does not handle |\footnotemark|s,
% but at least the user is made aware of this:
%
% \begin{macro}{\fnr@rangefootnotemark}
%    \begin{macrocode}

\newcommand{\fnr@rangefootnotemark}[1][\empty]{%
  \PackageError{footnoterange}{%
     \string\footnotemark\space not supported in footnoterange(*) environment%
   }{The footnoterange package does not handle a \string\footnotemark.\MessageBreak%
     Either use \string\footnote\string{...\string}\space instead of%
     \string\footnotemark\space and \string\footnotetext\string{...\string}\MessageBreak%
     in footnoterange and footnoterange* environments,\MessageBreak%
     or do not use those environments around a \string\footnotemark\MessageBreak%
     (or \string\renewcommand\string{\string\fnr@rangefootnotemark\string}\MessageBreak%
     with your own fixing code).%
    }%
  \xdef\fnr@mopt{#1}%
  \ifx\fnr@mopt\empty%
    \fnr@origfnmark%
  \else%
    \fnr@origfnmark[#1]%
  \fi%
}

%    \end{macrocode}
% \end{macro}
%
%    \begin{macrocode}
%</package>
%    \end{macrocode}
%
% \pagebreak
%
% \section{Installation}
%
% \subsection{Downloads\label{ss:Downloads}}
%
% Everything is available on CTAN,
% \url{https://www.ctan.org/tex-archive/},
% but may need additional packages themselves.
%
% \DescribeMacro{footnoterange.dtx}
% For unpacking the \xfile{footnoterange.dtx} file
% and constructing the documentation it is required:
% \begin{description}
% \item[-] \TeX{} Format \LaTeXe{}, \url{https://www.CTAN.org/}
%
% \item[-] document class \xpackage{ltxdoc}, 2015/03/26, v2.0w,
%           \url{https://ctan.org/pkg/ltxdoc}
%
% \item[-] package \xpackage{holtxdoc}, 2012/03/21, v0.24,
%           \url{https://ctan.org/pkg/holtxdoc}
% \end{description}
%
% \DescribeMacro{footnoterange.sty}
% The |footnoterange.sty| for \LaTeXe{} (i.\,e. each document using
% the \xpackage{footnoterange} package) requires:
% \begin{description}
% \item[-] \TeX{} Format \LaTeXe{}, \url{https://www.CTAN.org/}
%
% \item[-] package \xpackage{ltxcmds}, 2011/04/18, v1.20,
%           \url{https://www.ctan.org/pkg/ltxcmds}
% \end{description}
%
% \DescribeMacro{footnoterange-example.tex}
% The |footnoterange-example.tex| requires the same files as all
% documents using the \xpackage{footnoterange} package and additionally:
% \begin{description}
% \item[-] class \xpackage{article}, 2007/10/19, v1.4h, from \xpackage{classes.dtx}:\newline
%           \url{https://www.ctan.org/tex-archive/macros/latex/base/}
%
% \item[-] package \xpackage{footnoterange}, 2019/11/24, v1.0c,\newline
%           \url{https://www.ctan.org/pkg/footnoterange}\newline
%           (Well, it is the example file for this package, and because you are
%            reading the documentation for the \xpackage{footnoterange} package,
%            it can be assumed that you already have some version of it --
%            is it the current one?)
%
% \item[-] package \xpackage{hyperref}, 2012/11/06, v6.83m,\newline
%           \url{https://www.ctan.org/pkg/hyperref}
% \end{description}
%
% \DescribeMacro{Oberdiek}
% \DescribeMacro{holtxdoc}
% \DescribeMacro{ltxcmds}
% \DescribeMacro{alphalph}
% All packages of \textsc{Heiko Oberdiek}'s bundle `oberdiek'
% (especially \xpackage{holtxdoc}, \xpackage{ltxcmds}, and \xpackage{alphalph})
% are also available in a TDS compliant ZIP archive:\newline
% \url{http://mirror.ctan.org/install/macros/latex/contrib/oberdiek.tds.zip}.\newline
% It is probably best to download and use this, because the packages in there
% should be both recent and compatible.
%
% \DescribeMacro{hyperref}
% \noindent \xpackage{hyperref} is not included in that bundle and needs to be
% downloaded separately,\newline
% \url{http://mirrors.ctan.org/install/macros/latex/contrib/hyperref.tds.zip}.
%
% \DescribeMacro{M\"{u}nch}
% A list of my packages can be found at \url{https://www.ctan.org/author/muench-hm}.
%
% \pagebreak
%
% \subsection{Package, unpacking TDS}
% \paragraph{Package.} This package is available on \url{https://www.CTAN.org}.
% \begin{description}
% \item[\url{http://mirror.ctan.org/macros/latex/contrib/footnoterange/footnoterange.dtx}]\hspace*{0.1cm}
%       The source file.
% \item[\url{http://mirror.ctan.org/macros/latex/contrib/footnoterange/footnoterange.pdf}]\hspace*{0.1cm}
%       The documentation.
% \item[\url{http://mirror.ctan.org/macros/latex/contrib/footnoterange/footnoterange-example.pdf}]\hspace*{0.1cm}
%       The compiled example file, as it should look like.
% \item[\url{http://mirror.ctan.org/macros/latex/contrib/footnoterange/README}]\hspace*{0.1cm}\newline\hspace*{0.1cm}
%       The README file.
% \end{description}
% There is also a footnoterange.tds.zip available:
% \begin{description}
% \item[\url{http://mirror.ctan.org/install/macros/latex/contrib/footnoterange.tds.zip}]\hspace*{0.1cm}
%       Everything in TDS compliant, compiled format.
% \end{description}
% which additionally contains\newline
% \begin{tabular}{ll}
% footnoterange.ins & The installation file.\\
% footnoterange.drv & The driver to generate the documentation.\\
% footnoterange.sty & The \xext{sty}le file.
% \end{tabular}
%
% \bigskip
%
% \noindent For required other packages, please see the preceding subsection.
%
% \paragraph{Unpacking.} The \xfile{.dtx} file is a self-extracting
% \docstrip{} archive. The files are extracted by running the
% \xfile{.dtx} through \plainTeX:
% \begin{quote}
%   \verb|tex footnoterange.dtx|
% \end{quote}
%
% About generating the documentation see paragraph~\ref{GenDoc} below.
%
% \paragraph{TDS.} Now the different files must be moved into
% the different directories in your installation TDS tree
% (also known as \xfile{texmf} tree):
% \begin{quote}
% \def\t{^^A
% \begin{tabular}{@{}>{\ttfamily}l@{ $\rightarrow$ }>{\ttfamily}l@{}}
%   footnoterange.sty & tex/latex/footnoterange/footnoterange.sty\\
%   footnoterange.pdf & doc/latex/footnoterange/footnoterange.pdf\\
%   footnoterange-example.tex & doc/latex/footnoterange/footnoterange-example.tex\\
%   footnoterange-example.pdf & doc/latex/footnoterange/footnoterange-example.pdf\\
%   footnoterange.dtx & source/latex/footnoterange/footnoterange.dtx\\
% \end{tabular}^^A
% }^^A
% \sbox0{\t}^^A
% \ifdim\wd0>\linewidth
%   \begingroup
%     \advance\linewidth by\leftmargin
%     \advance\linewidth by\rightmargin
%   \edef\x{\endgroup
%     \def\noexpand\lw{\the\linewidth}^^A
%   }\x
%   \def\lwbox{^^A
%     \leavevmode
%     \hbox to \linewidth{^^A
%       \kern-\leftmargin\relax
%       \hss
%       \usebox0
%       \hss
%       \kern-\rightmargin\relax
%     }^^A
%   }^^A
%   \ifdim\wd0>\lw
%     \sbox0{\small\t}^^A
%     \ifdim\wd0>\linewidth
%       \ifdim\wd0>\lw
%         \sbox0{\footnotesize\t}^^A
%         \ifdim\wd0>\linewidth
%           \ifdim\wd0>\lw
%             \sbox0{\scriptsize\t}^^A
%             \ifdim\wd0>\linewidth
%               \ifdim\wd0>\lw
%                 \sbox0{\tiny\t}^^A
%                 \ifdim\wd0>\linewidth
%                   \lwbox
%                 \else
%                   \usebox0
%                 \fi
%               \else
%                 \lwbox
%               \fi
%             \else
%               \usebox0
%             \fi
%           \else
%             \lwbox
%           \fi
%         \else
%           \usebox0
%         \fi
%       \else
%         \lwbox
%       \fi
%     \else
%       \usebox0
%     \fi
%   \else
%     \lwbox
%   \fi
% \else
%   \usebox0
% \fi
% \end{quote}
% If you have a \xfile{docstrip.cfg} that configures and enables \docstrip's
% TDS installing feature, then some files can already be in the right
% place, see the documentation of \docstrip.
%
% \subsection{Refresh file name databases}
%
% If your \TeX~distribution (\TeX{} Live, \mikTeX, \teTeX, \dots) relies on
% file name databases, you must refresh these. For example, \teTeX{} users run
% \verb|texhash| or \verb|mktexlsr|.
%
% \subsection{Some details for the interested}
%
% \paragraph{Unpacking with \LaTeX{}.}
% The \xfile{.dtx} chooses its action depending on the format:
% \begin{description}
% \item[\plainTeX:] Run \docstrip{} and extract the files.
% \item[\LaTeX:] Generate the documentation.
% \end{description}
% If you insist on using \LaTeX{} for \docstrip{} (really,
% \docstrip{} does not need \LaTeX ), then inform the autodetect routine
% about your intention:
% \begin{quote}
%   \verb|latex \let\install=y% \iffalse meta-comment
%
% File: footnoterange.dtx
% Version: 2019/11/24 v1.0c
%
% Copyright (C) 2012 - 2019 by
%    H.-Martin M"unch <Martin dot Muench at Uni-Bonn dot de>
% Portions of code copyrighted by other people as marked;
% lines marked "FMi": code by Frank Mittelbach, thanks!
%
% This work may be distributed and/or modified under the
% conditions of the LaTeX Project Public License, either
% version 1.3c of this license or (at your option) any later
% version. This version of this license is in
%    https://www.latex-project.org/lppl/lppl-1-3c.txt
% and the latest version of this license is in
%    https://www.latex-project.org/lppl.txt
% and version 1.3c or later is part of all distributions of
% LaTeX version 2005/12/01 or later.
%
% This work has the LPPL maintenance status "maintained".
%
% The Current Maintainer of this work is H.-Martin Muench.
%
% This work consists of the main source file footnoterange.dtx,
% the README, and the derived files
%    footnoterange.sty, footnoterange.pdf,
%    footnoterange.ins, footnoterange.drv,
%    footnoterange-example.tex, footnoterange-example.pdf.
%
% 'footnoterange' is available on CTAN:
% https://www.ctan.org/pkg/footnoterange
%
% Also a TDS.ZIP file is provided that contains all the files
% already sorted in a TDS tree:
% http://mirror.ctan.org/install/macros/latex/contrib/footnoterange.tds.zip
%
% Unpacking:
%    (a) If footnoterange.ins is present:
%           tex footnoterange.ins
%    (b) Without footnoterange.ins:
%           tex footnoterange.dtx
%    (c) If you insist on using LaTeX
%           latex \let\install=y% \iffalse meta-comment
%
% File: footnoterange.dtx
% Version: 2019/11/24 v1.0c
%
% Copyright (C) 2012 - 2019 by
%    H.-Martin M"unch <Martin dot Muench at Uni-Bonn dot de>
% Portions of code copyrighted by other people as marked;
% lines marked "FMi": code by Frank Mittelbach, thanks!
%
% This work may be distributed and/or modified under the
% conditions of the LaTeX Project Public License, either
% version 1.3c of this license or (at your option) any later
% version. This version of this license is in
%    https://www.latex-project.org/lppl/lppl-1-3c.txt
% and the latest version of this license is in
%    https://www.latex-project.org/lppl.txt
% and version 1.3c or later is part of all distributions of
% LaTeX version 2005/12/01 or later.
%
% This work has the LPPL maintenance status "maintained".
%
% The Current Maintainer of this work is H.-Martin Muench.
%
% This work consists of the main source file footnoterange.dtx,
% the README, and the derived files
%    footnoterange.sty, footnoterange.pdf,
%    footnoterange.ins, footnoterange.drv,
%    footnoterange-example.tex, footnoterange-example.pdf.
%
% 'footnoterange' is available on CTAN:
% https://www.ctan.org/pkg/footnoterange
%
% Also a TDS.ZIP file is provided that contains all the files
% already sorted in a TDS tree:
% http://mirror.ctan.org/install/macros/latex/contrib/footnoterange.tds.zip
%
% Unpacking:
%    (a) If footnoterange.ins is present:
%           tex footnoterange.ins
%    (b) Without footnoterange.ins:
%           tex footnoterange.dtx
%    (c) If you insist on using LaTeX
%           latex \let\install=y\input{footnoterange.dtx}
%        (quote the arguments according to the demands of your shell)
%
% Documentation:
%    (a) If footnoterange.drv is present:
%           (pdf)latex footnoterange.drv
%           makeindex -s gind.ist footnoterange.idx
%           (pdf)latex footnoterange.drv
%           makeindex -s gind.ist footnoterange.idx
%           (pdf)latex footnoterange.drv
%    (b) Without footnoterange.drv:
%           (pdf)latex footnoterange.dtx
%           makeindex -s gind.ist footnoterange.idx
%           (pdf)latex footnoterange.dtx
%           makeindex -s gind.ist footnoterange.idx
%           (pdf)latex footnoterange.dtx
%
%    The class ltxdoc loads the configuration file ltxdoc.cfg
%    if available. Here you can specify further options, e.g.
%    use DIN A4 as paper format:
%       \PassOptionsToClass{a4paper}{article}
%
% Installation:
%    TDS:tex/latex/footnoterange/footnoterange.sty
%    TDS:doc/latex/footnoterange/footnoterange.pdf
%    TDS:doc/latex/footnoterange/footnoterange-example.tex
%    TDS:doc/latex/footnoterange/footnoterange-example.pdf
%    TDS:source/latex/footnoterange/footnoterange.dtx
%
%<*ignore>
\begingroup
  \catcode123=1 %
  \catcode125=2 %
  \def\x{LaTeX2e}%
\expandafter\endgroup
\ifcase 0\ifx\install y1\fi\expandafter
         \ifx\csname processbatchFile\endcsname\relax\else1\fi
         \ifx\fmtname\x\else 1\fi\relax
\else\csname fi\endcsname
%</ignore>
%<*install>
\input docstrip.tex
\Msg{*********************************************************************************}
\Msg{* Installation}
\Msg{* Package: footnoterange 2019/11/24 v1.0c References to ranges of footnotes (HMM)}
\Msg{*********************************************************************************}

\keepsilent
\askforoverwritefalse

\let\MetaPrefix\relax
\preamble

This is a generated file.

Project: footnoterange
Version: 2019/11/24 v1.0c

Copyright (C) 2012 - 2019 by
    H.-Martin M"unch <Martin dot Muench at Uni-Bonn dot de>
Portions of code copyrighted by other people as marked;
lines marked "FMi": code by Frank Mittelbach, thanks!

The usual disclaimer applies:
If it doesn't work right that's your problem.
(Nevertheless, please send an e-mail to the maintainer
 when you find an error in this package.)

This work may be distributed and/or modified under the
conditions of the LaTeX Project Public License, either
version 1.3c of this license or (at your option) any later
version. This version of this license is in
   https://www.latex-project.org/lppl/lppl-1-3c.txt
and the latest version of this license is in
   https://www.latex-project.org/lppl.txt
and version 1.3c or later is part of all distributions of
LaTeX version 2005/12/01 or later.

This work has the LPPL maintenance status "maintained".

The Current Maintainer of this work is H.-Martin Muench.

This work consists of the main source file footnoterange.dtx,
the README, and the derived files
   footnoterange.sty, footnoterange.pdf,
   footnoterange.ins, footnoterange.drv,
   footnoterange-example.tex, footnoterange-example.pdf.

In memoriam
 Claudia Simone Barth + 1996/01/30
 Tommy Muench + 2014/01/02
 Hans-Klaus Muench + 2014/08/24

\endpreamble
\let\MetaPrefix\DoubleperCent

\generate{%
  \file{footnoterange.ins}{\from{footnoterange.dtx}{install}}%
  \file{footnoterange.drv}{\from{footnoterange.dtx}{driver}}%
  \usedir{tex/latex/footnoterange}%
  \file{footnoterange.sty}{\from{footnoterange.dtx}{package}}%
  \usedir{doc/latex/footnoterange}%
  \file{footnoterange-example.tex}{\from{footnoterange.dtx}{example}}%
}

\catcode32=13\relax% active space
\let =\space%
\Msg{************************************************************************}
\Msg{*}
\Msg{* To finish the installation you have to move the following}
\Msg{* file into a directory searched by TeX:}
\Msg{*}
\Msg{* footnoterange.sty}
\Msg{*}
\Msg{* To produce the documentation run the file `footnoterange.drv'}
\Msg{* through (pdf)LaTeX, e.g.}
\Msg{*  pdflatex footnoterange.drv}
\Msg{*  makeindex -s gind.ist footnoterange.idx}
\Msg{*  pdflatex footnoterange.drv}
\Msg{*  makeindex -s gind.ist footnoterange.idx}
\Msg{*  pdflatex footnoterange.drv}
\Msg{*}
\Msg{* At least three runs are necessary e.g. to get the}
\Msg{*  references right!}
\Msg{*}
\Msg{* Happy TeXing!}
\Msg{*}
\Msg{************************************************************************}

\endbatchfile
%</install>
%<*ignore>
\fi
%</ignore>
%
% \section{The documentation driver file}
%
% The next bit of code contains the documentation driver file for
% \TeX , i.\,e., the file that will produce the documentation you
% are currently reading. It will be extracted from this file by the
% \verb|docstrip| programme. That is, run \LaTeX{} on \verb|docstrip|
% and specify the \verb|driver| option when \verb|docstrip|
% asks for options.
%
%    \begin{macrocode}
%<*driver>
\NeedsTeXFormat{LaTeX2e}[2015/01/01]
\ProvidesFile{footnoterange.drv}[2019/11/24 v1.0c
            References to ranges of footnotes (HMM)]
\documentclass{ltxdoc}[2015/03/26]% v2.0w
\usepackage{holtxdoc}[2012/03/21]%  v0.24
%% footnoterange may work with earlier versions of LaTeX2e and those
%% class and package, but this was not tested.
%% Please consider updating your LaTeX, class, and package
%% to the most recent version (if they are not already the most
%% recent version).
\hypersetup{%
 pdfsubject={References to ranges of footnotes (HMM)},%
 pdfkeywords={LaTeX, footnoterange, footnotes, footnote, reference, H.-Martin Muench},%
 pdfencoding=auto,%
 pdflang={en},%
 breaklinks=true,%
 linktoc=all,%
 pdfstartview=FitH,%
 pdfpagelayout=OneColumn,%
 bookmarksnumbered=true,%
 bookmarksopen=true,%
 bookmarksopenlevel=3,%
 pdfmenubar=true,%
 pdftoolbar=true,%
 pdfwindowui=true,%
 pdfnewwindow=true%
}
\CodelineIndex
\hyphenation{printing docu-ment}
\gdef\unit#1{\mathord{\thinspace\mathrm{#1}}}
\begin{document}
  \DocInput{footnoterange.dtx}%
\end{document}
%</driver>
%    \end{macrocode}
%
% \fi
%
% \CheckSum{321}
%
% \CharacterTable
%  {Upper-case    \A\B\C\D\E\F\G\H\I\J\K\L\M\N\O\P\Q\R\S\T\U\V\W\X\Y\Z
%   Lower-case    \a\b\c\d\e\f\g\h\i\j\k\l\m\n\o\p\q\r\s\t\u\v\w\x\y\z
%   Digits        \0\1\2\3\4\5\6\7\8\9
%   Exclamation   \!     Double quote  \"     Hash (number) \#
%   Dollar        \$     Percent       \%     Ampersand     \&
%   Acute accent  \'     Left paren    \(     Right paren   \)
%   Asterisk      \*     Plus          \+     Comma         \,
%   Minus         \-     Point         \.     Solidus       \/
%   Colon         \:     Semicolon     \;     Less than     \<
%   Equals        \=     Greater than  \>     Question mark \?
%   Commercial at \@     Left bracket  \[     Backslash     \\
%   Right bracket \]     Circumflex    \^     Underscore    \_
%   Grave accent  \`     Left brace    \{     Vertical bar  \|
%   Right brace   \}     Tilde         \~}
%
% \GetFileInfo{footnoterange.drv}
%
% \begingroup
%   \def\x{\#,\$,\^,\_,\~,\ ,\&,\{,\},\%}%
%   \makeatletter
%   \@onelevel@sanitize\x
% \expandafter\endgroup
% \expandafter\DoNotIndex\expandafter{\x}
% \expandafter\DoNotIndex\expandafter{\string\ }
% \begingroup
%   \makeatletter
%     \lccode`9=32\relax
%     \lowercase{%^^A
%       \edef\x{\noexpand\DoNotIndex{\@backslashchar9}}%^^A
%     }%^^A
%   \expandafter\endgroup\x
% \DoNotIndex{\,,\\}
% \DoNotIndex{\documentclass,\usepackage,\ProvidesPackage,\begin,\end}
% \DoNotIndex{\MessageBreak}
% \DoNotIndex{\NeedsTeXFormat,\DoNotIndex,\verb}
% \DoNotIndex{\def,\edef,\gdef,\global,\xdef}
% \DoNotIndex{\ifx,\listfiles,\mathord,\mathrm}
% \DoNotIndex{\bigskip,\space,\thinspace,\ldots}
% \DoNotIndex{\indent,\noindent,\newline,\linebreak,\pagebreak,\newpage}
% \DoNotIndex{\textbf,\textit,\textsf,\textsc,\texttt,\textquotedblleft,\textquotedblright}
% \DoNotIndex{\plainTeX,\TeX,\LaTeX,\pdfLaTeX}
% \DoNotIndex{\section,\pagenumbering,\smallskip}
% \DoNotIndex{\@ifpackageloaded}
% \DoNotIndex{\arabic, \Alph, \alph, \alphalph, \Roman, \roman, \fnsymbol}
%
% \title{The \xpackage{footnoterange} package}
% \date{2019/11/24 v1.0c}
% \author{H.-Martin M\"{u}nch\\\xemail{Martin.Muench at Uni-Bonn.de}}
%
% \maketitle
%
% \begin{abstract}
% \noindent This package provides the environments |footnoterange| and
% |footnoterange*|. Multiple footnotes inside these environments
% are not referenced as e.\,g.
% \makeatletter%
% \hbox{\textquotedblleft \@textsuperscript {\normalfont 1 2 3}\textquotedblright}
% but as \hbox{\textquotedblleft \@textsuperscript {\normalfont 1--3}\textquotedblright ,\ }
% \makeatother%
% i.\,e. the range (from first to last referred footnote at that place) is given.
% Depending on loading of \xpackage{hyperref} and use of its |hyperfootnotes|-option
% the references are hyperlinked. (References to footnotes in |footnoterange*|
% are never hyperlinked.)
% \end{abstract}
%
% \bigskip
%
% \noindent Disclaimer for web links: The author is not responsible for any contents
% referred to in this work unless if having full knowledge of illegal contents.
% If any damage occurs by the use of information presented there, only the
% author of the respective pages might be liable, not the one who has referred
% to these pages.
%
% \noindent Save per page about $200\unit{ml}$ water,
% $2\unit{g}$ CO$_{2}$ and $2\unit{g}$ wood:
% Therefore please print only if this is really necessary.
%
% \medskip
%
% \tableofcontents
%
%\newpage
%
% \section{Introduction}
% \indent This \LaTeXe{} package provides the environments |footnoterange| and\newline
% |footnoterange*|. Multiple footnotes inside these environments are not referenced
% as e.\,g.
% \makeatletter%
% \hbox{\textquotedblleft \@textsuperscript {\normalfont 1 2 3}\textquotedblright}
% but as \hbox{\textquotedblleft \@textsuperscript {\normalfont 1--3}\textquotedblright ,\ }
% \makeatother%
% i.\,e. the range (from first to last
% referred footnote at that place) is given. Depending on loading of
% \xpackage{hyperref} and use of its |hyperfootnotes|-option the references are
% hyperlinked. References to footnotes in |footnoterange*| are never hyperlinked.\newline
% |\footnotemark\footnotetext| instead of |\footnote| is not supported inside
% |footnoterange| and |footnoterange*| environments.
%
% \bigskip
%
% \section{Usage}
%
% \indent Just load the package placing
% \begin{quote}
%   |\usepackage{footnoterange}|
% \end{quote}
% \noindent in the preamble of your \LaTeXe{} source file. There are no options to
% be used. The package should be loaded after \xpackage{hyperref} (if that package is used)
% and after any other package redefining footnotes. Usage of the |footnoterange| (or
% |footnoterange*|) environment is simple: Just encapsulate the footnotes, which shall
% be referred as range instead of referencing as single ones (see the example file).
%
% About hyperlinks:\newline
% Footnote-references in a |footnoterange*| environment are never hyperlinked.\newline
% Footnote-references in a |footnoterange| environment may be hyperlinked:
% \begin{itemize}
%  \item When the \xpackage{hyperref} package is not loaded, they are not
%         hyperlinked (naturally).
%  \item When the \xpackage{hyperref} package is loaded, it depends on the
%         |hyperfootnotes| option of the \xpackage{hyperref} package:
%  \begin{itemize}
%   \item With |hyperfootnotes=true|, the references are hyperlinked.
%   \item With |hyperfootnotes=false|, the references are not hyperlinked.
%  \end{itemize}
% \end{itemize}
% |\footnote|s inside of the |footnoterange|/{}|footnoterange*| environment
% can use the optional parameter, i.\,e. |\footnote[42]{text}| even if this is the
% third footnote. When giving ranges as reference, this does not seem to be
% such a good idea, if this is not done consistently (i.\,e. footnotes
% |[42], [43], [44]| resulting in 42--44 is OK,
% but |[42]| (original~2), unchanged~3, unchanged~4 resulting in
% 42--4 is confusing). A warning about these changed footnotemarks is written
% into the \xfile{.log}-file.
%
% When there is only a single footnote in a |footnoterange| environment,
% this does not produce some reference like %
% \makeatletter%
% \hbox{\textquotedblleft \@textsuperscript {\normalfont 7--7}\textquotedblright{},\ }%
% \makeatother%
% but only one footnotemark is used automatically. Because it is strange to use
% a |footnoterange| environment without a range of footnotes, an info is written
% into the \xfile{.log}-file about this.
%
% The usage of other footnote-number-representations
% (e.\,g.~\verb|\Roman|, \verb|\roman|, \verb|\Alph|, \verb|\alph|,
% \verb|\fnsymbol|) is also possible, but two things must be taken
% into account: The number of footnote references is restricted (for
% example with \verb|\Alph| only references A to Z are possible), which
% can be fixed e.\,g. with the \url{https://ctan.org/pkg/alphalph} package
% (but not automatically), and references to footnote-symbol-ranges (\verb|\fnsymbol|)
% are probably not very clear.
%
% \subsection*{Options}
% \indent The \xpackage{footnoterange} package takes no options.
%
% \section{Alternatives\label{sec:Alternatives}}
%
% The \xpackage{fnpct} (\url{https://www.ctan.org/pkg/fnpct}) package allows to set
% multiple footnotes (|\multfootnote|), suppress a footnotemark \hbox{(|*|),}
% and customize the multi-footnote-separator \hbox{(|\setfnpct{mult-fn-sep={|
% \emph{something} |}}|)}.
% Combining these with stepping of the footnotecounter gives a footnoterange:
% 
% \begin{verbatim}
% \documentclass{article}
% \usepackage{hyperref}
% \usepackage{fnpct}
% \begin{document}
% The three little pigs built their houses
% out of straw\footnote{not to be confused with hay},
% sticks\footnote{or lumber according to some sources}
% and bricks%
% \makeatletter%
% \setfnpct{mult-fn-sep={--}}%
% \multfootnote{probably fired clay bricks\addtocounter\@mpfn{+1};%
% *or something else;%
% what do I know}. %
% \makeatother%
% \end{document}
% \end{verbatim}
% 
% \noindent If you programmed or found another alternative, which is available at
% \url{https://www.CTAN.org/}, please send an e-mail to me with the name,
% location at CTAN, and a short notice, and I will probably list it here.
%
% \section{Example}
%
%    \begin{macrocode}
%<*example>
\documentclass[british]{article}[2014/09/29]% v1.4h
%%%%%%%%%%%%%%%%%%%%%%%%%%%%%%%%%%%%%%%%%%%%%%%%%%%%%%%%%%%%%%%%%%%%%
\usepackage[%
 extension=pdf,%
 plainpages=false,%
 pdfpagelabels=true,%
 hyperindex=false,%
 pdflang={en},%
 pdftitle={footnoterange package example},%
 pdfauthor={H.-Martin Muench},%
 pdfsubject={Example for the footnoterange package},%
 pdfkeywords={LaTeX, footnoterange, H.-Martin Muench},%
 pdfview=FitH,%
 pdfstartview=FitH,%
 pdfpagelayout=OneColumn,%
 hyperfootnotes=true%
]{hyperref}[2012/11/06]% v6.83m

\usepackage{footnoterange}[2019/11/24]% v1.0c

\gdef\unit#1{\mathord{\thinspace\mathrm{#1}}}%
\listfiles
\begin{document}
\pagenumbering{arabic}
\section*{Example for footnoterange}

This example demonstrates the use of package\newline
\textsf{footnoterange}, v1.0c as of 2019/11/24 (HMM).\newline
The package does not use options.\newline
\verb|footnoterange-example.pdf| can be created by\newline
\verb|pdflatex footnoterange-example.tex|,\newline
and \verb|footnoterange-example.tex| can be created by\newline
\verb|tex footnoterange.dtx|.\newline
\verb|footnoterange.dtx| is at
\url{https://www.ctan.org/pkg/footnoterange}.\newline

\noindent For more details please see the documentation!\newline

\noindent Save per page about $200\unit{ml}$ water,
$2\unit{g}$ CO$_{2}$ and $2\unit{g}$ wood:\newline
Therefore please print only if this is really necessary.\newline

This\footnote{Lorem} text bears a multiplicity of footnotes.
Because the \verb|hyperref| package is used with option
\verb|hyperfootnotes=true|, the footnotes
\begin{footnoterange}
\footnote{ipsum}%
\footnote{dolor}%
\footnote{sit}%
\footnote{amet,}%
\footnote{consectetur}%
\footnote{adipisicing}%
\footnote{elit,}%
\footnote{sed}%
\end{footnoterange}
are hyperlinked.\footnote{do eiusmod\ldots} Using
\verb|hyperfootnotes=false| or not using \verb|hyperref|
will remove the hyperlinks to the footnotes.\medskip

This text also bears a multiplicity of footnotes,
but due to the use of the starred
\begin{footnoterange*}
\footnote{Ut purus elit,}%
\footnote{vestibulum ut,}%
\footnote{placerat ac,}%
\footnote{adipiscing vitae,}%
\footnote{felis.}%
\footnote{Curabitur}%
\footnote{dictum}%
\footnote{gravida}%
\end{footnoterange*}
form of the environment they are not
hyperlinked.\footnote{mauris\ldots}\medskip

The use of the optional parameter for footnotes
\begin{footnoterange}
\footnote[101]{Nam arcu libero,}%
\footnote[102]{nonummy eget,}%
\footnote[103]{consectetuer id,}%
\footnote[104]{vulputate a,}%
\footnote[105]{magna.}
\end{footnoterange}
is also possible, but should be used consistently.

\pagebreak

When there is only one footnote
\begin{footnoterange}
\footnote{a single footnote}
\end{footnoterange}
in the \verb|footnoterange| environment, only one
footnotemark is used and an info is written into the
\verb|log|-file.\medskip

The usage of other footnote-number-representations
(e.\,g.~\verb|\Roman|, \verb|\roman|, \verb|\Alph|, \verb|\alph|,
\verb|\fnsymbol|) is also possible, but two things must be taken
into account: The number of footnote references is restricted
(for example with \verb|\Alph| only references A to Z are possible),
which can be fixed e.\,g. with the
\href{https://www.ctan.org/pkg/alphalph}{\texttt{alphalph} package},
and references to footnote-symbol-ranges (\verb|\fnsymbol|)
are probably not very clear.

\end{document}
%</example>
%    \end{macrocode}
%
% \StopEventually{}
%
% \section{The implementation}
%
% We start off by checking that we are loading into \LaTeXe{} and
% announcing the name and version of this package.
%
%    \begin{macrocode}
%<*package>
%    \end{macrocode}
%
%    \begin{macrocode}
\NeedsTeXFormat{LaTeX2e}[2011/06/27]
\ProvidesPackage{footnoterange}[2019/11/24 v1.0c
            References to ranges of footnotes (HMM)]
%    \end{macrocode}
%
% A short description of the \xpackage{footnoterange} package:
%
%    \begin{macrocode}
%% Provides the footnoterange and footnoterange* environments to
%% refer to ranges of footnotes.
%    \end{macrocode}
%
% We need the \xpackage{ltxcmds} package:
%
%    \begin{macrocode}
\RequirePackage{ltxcmds}[2011/11/09]% v1.22; for "\ltx@ifpackageloaded"
\RequirePackage{letltxmacro}[2010/09/02]% v1.4; for "\GlobalLetLtxMacro"

%    \end{macrocode}
%
% We use |\ifHy@hyperfootnotes ... \else ... \fi|, but when the \xpackage{hyperref}
% package has not been loaded, |\ifHy@hyperfootnotes| is not defined, and \LaTeX
% relates the |\else| and/or |\fi| (wrongly) to other |\if|s or groups.
% Therefore the |\ifHy@hyperfootnotes| are placed in their own commands,
% which are not looked at when the \xpackage{hyperref} package has not been loaded.
%
% \pagebreak
%
% \begin{macro}{\fnr@ifhyperrfn}
%    \begin{macrocode}
\newcommand{\fnr@ifhyperrfn}{%
  \ifHy@hyperfootnotes% option hyperfootnotes=true
    \addtocounter{Hfootnote}{+1}%
    % Code from the hyperref package
      \global\let\Hy@saved@currentHref\@currentHref%
      \hyper@makecurrent{Hfootnote}%
      \global\let\Hy@footnote@currentHref\@currentHref%
      \global\let\@currentHref\Hy@saved@currentHref%
    % End of code from the hyperref package
  \fi%
}

%    \end{macrocode}
% \end{macro}
%
% \begin{macro}{\fnr@rangefootnote}
% The |\fnr@rangefootnote| is the internal replacement for the original\newline
% |\footnote| command inside of a |footnoterange| environment.
%
%    \begin{macrocode}
\newcommand{\fnr@rangefootnote}[2][\empty]{%
%    \end{macrocode}
%
% |#1| contains the optional argument, which is predefined as |\empty|.
%
%    \begin{macrocode}
  \nolinebreak%
  % no \footnotemark here
  \addtocounter\@mpfn{+1}%
  \ltx@ifpackageloaded{hyperref}{\fnr@ifhyperrfn}{\relax}%
%    \end{macrocode}
%
% See the explanation before the declaration of the |\fnr@ifhyperrfn| command!
%
%    \begin{macrocode}
  \xdef\fnr@opt{#1}% contains the optional argument
  \xdef\fnr@arabic{\arabic\@mpfn}%
  \protected@edef\fnr@formated{\thefootnote}%
%    \end{macrocode}
%
% The |\footnotemark| is not set (yet), but the |\footnotetext|,
% and it gets a label.
%
%    \begin{macrocode}
  \ifx\fnr@opt\empty%
    \footnotetext{\label{fnr:\fnr@arabic}#2}%
  \else%
    \PackageWarning{footnoterange}{%
      Custom footnote-numbers used in a footnoterange\MessageBreak%
      environment: Footnote with natural number\MessageBreak%
      \ifx\fnr@formated\fnr@arabic%
        \fnr@arabic\space%
      \else%
        \space\fnr@arabic\space (\fnr@formated )%
      \fi%
      \MessageBreak%
      was renumbered to \fnr@opt\MessageBreak%
     }%
    \ltx@ifpackageloaded{hyperref}{% hyperref loaded
      \footnotetext[#1]{\phantomsection\label{fnr:\fnr@arabic}#2}%
     }{% hyperref not loaded
      \footnotetext[#1]{\label{fnr:\fnr@arabic}#2}%
     }%
  \fi%
}

%    \end{macrocode}
% \end{macro}
%
%    \begin{macrocode}

%    \end{macrocode}
% \pagebreak
%
% |\ifHy@hyperfootnotes| \textquotedblleft encapsulation\textquotedblright{} again:
%
% \begin{macro}{\fnr@ifhyperfnrA}
%    \begin{macrocode}
\newcommand{\fnr@ifhyperfnrA}{%
  \ifHy@hyperfootnotes% option hyperfootnotes=true
    \hbox{\hyperref[fnr:\fnr@first]{%
          \@textsuperscript{\normalfont \ref*{fnr:\fnr@first}}}}%
  \else% option hyperfootnotes=false
    \hbox{\@textsuperscript{\normalfont \ref*{fnr:\fnr@first}}}%
  \fi%
}

%    \end{macrocode}
% \end{macro}
%
% \begin{macro}{\fnr@ifhyperfnrN}
%    \begin{macrocode}
\newcommand{\fnr@ifhyperfnrN}{%
  \ifHy@hyperfootnotes% option hyperfootnotes=true
    \hbox{\hyperref[fnr:\fnr@first]{%
            \@textsuperscript{\normalfont \ref*{fnr:\fnr@first}}}%
          \@textsuperscript{\normalfont --}%
          \hyperref[fnr:\arabic\@mpfn]{%
            \@textsuperscript{\normalfont \ref*{fnr:\arabic\@mpfn}}}}%
  \else% option hyperfootnotes=false
    \hbox{\@textsuperscript{\normalfont \ref*{fnr:\fnr@first}%
                                        --\ref*{fnr:\arabic\@mpfn}}}%
  \fi%
}

%    \end{macrocode}
% \end{macro}
%
% \begin{environment}{footnoterange}
% The |footnoterange| environment remembers the next footnote number as\newline
% |\fnr@first| and the current |\footnote| definition as |\fnr@origfn|
% and uses |\fnr@rangefootnote| instead.
%
%    \begin{macrocode}
\newenvironment{footnoterange}{%
  % at begin of the footnoterange environment
  \ifhmode \unskip \fi% FMi
  \addtocounter\@mpfn{+1}%
  \xdef\fnr@first{\arabic\@mpfn}%
  \addtocounter\@mpfn{-1}%
  \let\fnr@origfn\footnote%
  \let\footnote\fnr@rangefootnote%
  \GlobalLetLtxMacro{\fnr@origfnmark}{\footnotemark}%
  \GlobalLetLtxMacro{\footnotemark}{\fnr@rangefootnotemark}%
  \ignorespaces% FMi
%    \end{macrocode}
%
% End of the environment:
%
%    \begin{macrocode}
}{% at end of the footnoterange environment
  \ifhmode% FMi 
    \unskip% FMi
%    \end{macrocode}
%
% Remember the spacefactor for restoring after the environment has ended
% (only in |hmode|!). \% FMi
%
%    \begin{macrocode}
    \xdef\fnr@aftergroup{\spacefactor\the\spacefactor\relax}% FMi
  \else% FMi
%    \end{macrocode}
%
% Otherwise do something harmless. \% FMi
%
%    \begin{macrocode}
    \xdef\fnr@aftergroup{\relax}%
  \fi% FMi
%    \end{macrocode}
%
% At the end of the environment the previous definition of |\footnote| is restored
%
%    \begin{macrocode}
  \let\footnote\fnr@origfn%
  \GlobalLetLtxMacro{\footnotemark}{\fnr@origfnmark}%
%    \end{macrocode}
%
% and the references are placed:
%
%    \begin{macrocode}
  \ifx\fnr@first\fnr@arabic% There is only one reference in the environment.
    \PackageInfo{footnoterange}{%
      Environment footnoterange used,\MessageBreak%
      but only one footnote (number \fnr@first ) is contained%
      \MessageBreak%
     }%
    \ltx@ifpackageloaded{hyperref}{\fnr@ifhyperfnrA% hyperref package loaded
    }{% hyperref package not loaded
      \hbox{\@textsuperscript{\normalfont \ref{fnr:\fnr@first}}}%
     }%
  \else%
    \ltx@ifpackageloaded{hyperref}{\fnr@ifhyperfnrN% hyperref package loaded
    }{% hyperref package not loaded
      \hbox{\@textsuperscript{\normalfont \ref{fnr:\fnr@first}%
                                        --\ref{fnr:\arabic\@mpfn}}}%
     }%
  \fi%
%    \end{macrocode}
%
% Restore the spacefactor after the environment has ended. \% FMi
%
%    \begin{macrocode}
  \aftergroup\fnr@aftergroup% FMi
}

%    \end{macrocode}
% \end{environment}
%
% \begin{environment}{footnoterange*}
% The |footnoterange*| environment does the same as the |footnoterange|
% environment but suppresses all hyperlinks to footnotes inside.
%
%    \begin{macrocode}
\newenvironment{footnoterange*}{%
  % at begin of the footnoterange* environment
  \ifhmode \unskip \fi% FMi
  \addtocounter\@mpfn{+1}%
  \xdef\fnr@first{\arabic\@mpfn}%
  \addtocounter\@mpfn{-1}%
  \let\fnr@origfn\footnote%
  \let\footnote\fnr@rangefootnote%
  \GlobalLetLtxMacro{\fnr@origfnmark}{\footnotemark}%
  \GlobalLetLtxMacro{\footnotemark}{\fnr@rangefootnotemark}%
  \ignorespaces% FMi
}{% at end of the footnoterange* environment
  \ifhmode% FMi 
    \unskip% FMi
    \xdef\fnr@aftergroup{\spacefactor\the\spacefactor\relax}% FMi
  \else% FMi
    \xdef\fnr@aftergroup{\relax}%
  \fi% FMi
  \let\footnote\fnr@origfn%
  \GlobalLetLtxMacro{\footnotemark}{\fnr@origfnmark}%
  \ifx\fnr@first\fnr@arabic% There is only one reference in the environment.
    \PackageInfo{footnoterange}{%
      Environment footnoterange used,\MessageBreak%
      but only one footnote (number \fnr@first ) is contained%
      \MessageBreak%
     }%
%    \end{macrocode}
% \pagebreak
%    \begin{macrocode}
    \ltx@ifpackageloaded{hyperref}{% hyperref package loaded
      \hbox{\@textsuperscript{\normalfont \ref*{fnr:\fnr@first}}}%
    }{% hyperref package not loaded
      \hbox{\@textsuperscript{\normalfont \ref{fnr:\fnr@first}}}%
     }%
  \else%
    \ltx@ifpackageloaded{hyperref}{% hyperref package loaded
      \hbox{\@textsuperscript{\normalfont \ref*{fnr:\fnr@first}%
                                        --\ref*{fnr:\arabic\@mpfn}}}%
    }{% hyperref package not loaded
      \hbox{\@textsuperscript{\normalfont \ref{fnr:\fnr@first}%
                                        --\ref{fnr:\arabic\@mpfn}}}%
     }%
  \fi%
  \aftergroup\fnr@aftergroup% FMi
}

%    \end{macrocode}
% \end{environment}
%
% The \xpackage{footnoterange} package does not handle |\footnotemark|s,
% but at least the user is made aware of this:
%
% \begin{macro}{\fnr@rangefootnotemark}
%    \begin{macrocode}

\newcommand{\fnr@rangefootnotemark}[1][\empty]{%
  \PackageError{footnoterange}{%
     \string\footnotemark\space not supported in footnoterange(*) environment%
   }{The footnoterange package does not handle a \string\footnotemark.\MessageBreak%
     Either use \string\footnote\string{...\string}\space instead of%
     \string\footnotemark\space and \string\footnotetext\string{...\string}\MessageBreak%
     in footnoterange and footnoterange* environments,\MessageBreak%
     or do not use those environments around a \string\footnotemark\MessageBreak%
     (or \string\renewcommand\string{\string\fnr@rangefootnotemark\string}\MessageBreak%
     with your own fixing code).%
    }%
  \xdef\fnr@mopt{#1}%
  \ifx\fnr@mopt\empty%
    \fnr@origfnmark%
  \else%
    \fnr@origfnmark[#1]%
  \fi%
}

%    \end{macrocode}
% \end{macro}
%
%    \begin{macrocode}
%</package>
%    \end{macrocode}
%
% \pagebreak
%
% \section{Installation}
%
% \subsection{Downloads\label{ss:Downloads}}
%
% Everything is available on CTAN,
% \url{https://www.ctan.org/tex-archive/},
% but may need additional packages themselves.
%
% \DescribeMacro{footnoterange.dtx}
% For unpacking the \xfile{footnoterange.dtx} file
% and constructing the documentation it is required:
% \begin{description}
% \item[-] \TeX{} Format \LaTeXe{}, \url{https://www.CTAN.org/}
%
% \item[-] document class \xpackage{ltxdoc}, 2015/03/26, v2.0w,
%           \url{https://ctan.org/pkg/ltxdoc}
%
% \item[-] package \xpackage{holtxdoc}, 2012/03/21, v0.24,
%           \url{https://ctan.org/pkg/holtxdoc}
% \end{description}
%
% \DescribeMacro{footnoterange.sty}
% The |footnoterange.sty| for \LaTeXe{} (i.\,e. each document using
% the \xpackage{footnoterange} package) requires:
% \begin{description}
% \item[-] \TeX{} Format \LaTeXe{}, \url{https://www.CTAN.org/}
%
% \item[-] package \xpackage{ltxcmds}, 2011/04/18, v1.20,
%           \url{https://www.ctan.org/pkg/ltxcmds}
% \end{description}
%
% \DescribeMacro{footnoterange-example.tex}
% The |footnoterange-example.tex| requires the same files as all
% documents using the \xpackage{footnoterange} package and additionally:
% \begin{description}
% \item[-] class \xpackage{article}, 2007/10/19, v1.4h, from \xpackage{classes.dtx}:\newline
%           \url{https://www.ctan.org/tex-archive/macros/latex/base/}
%
% \item[-] package \xpackage{footnoterange}, 2019/11/24, v1.0c,\newline
%           \url{https://www.ctan.org/pkg/footnoterange}\newline
%           (Well, it is the example file for this package, and because you are
%            reading the documentation for the \xpackage{footnoterange} package,
%            it can be assumed that you already have some version of it --
%            is it the current one?)
%
% \item[-] package \xpackage{hyperref}, 2012/11/06, v6.83m,\newline
%           \url{https://www.ctan.org/pkg/hyperref}
% \end{description}
%
% \DescribeMacro{Oberdiek}
% \DescribeMacro{holtxdoc}
% \DescribeMacro{ltxcmds}
% \DescribeMacro{alphalph}
% All packages of \textsc{Heiko Oberdiek}'s bundle `oberdiek'
% (especially \xpackage{holtxdoc}, \xpackage{ltxcmds}, and \xpackage{alphalph})
% are also available in a TDS compliant ZIP archive:\newline
% \url{http://mirror.ctan.org/install/macros/latex/contrib/oberdiek.tds.zip}.\newline
% It is probably best to download and use this, because the packages in there
% should be both recent and compatible.
%
% \DescribeMacro{hyperref}
% \noindent \xpackage{hyperref} is not included in that bundle and needs to be
% downloaded separately,\newline
% \url{http://mirrors.ctan.org/install/macros/latex/contrib/hyperref.tds.zip}.
%
% \DescribeMacro{M\"{u}nch}
% A list of my packages can be found at \url{https://www.ctan.org/author/muench-hm}.
%
% \pagebreak
%
% \subsection{Package, unpacking TDS}
% \paragraph{Package.} This package is available on \url{https://www.CTAN.org}.
% \begin{description}
% \item[\url{http://mirror.ctan.org/macros/latex/contrib/footnoterange/footnoterange.dtx}]\hspace*{0.1cm}
%       The source file.
% \item[\url{http://mirror.ctan.org/macros/latex/contrib/footnoterange/footnoterange.pdf}]\hspace*{0.1cm}
%       The documentation.
% \item[\url{http://mirror.ctan.org/macros/latex/contrib/footnoterange/footnoterange-example.pdf}]\hspace*{0.1cm}
%       The compiled example file, as it should look like.
% \item[\url{http://mirror.ctan.org/macros/latex/contrib/footnoterange/README}]\hspace*{0.1cm}\newline\hspace*{0.1cm}
%       The README file.
% \end{description}
% There is also a footnoterange.tds.zip available:
% \begin{description}
% \item[\url{http://mirror.ctan.org/install/macros/latex/contrib/footnoterange.tds.zip}]\hspace*{0.1cm}
%       Everything in TDS compliant, compiled format.
% \end{description}
% which additionally contains\newline
% \begin{tabular}{ll}
% footnoterange.ins & The installation file.\\
% footnoterange.drv & The driver to generate the documentation.\\
% footnoterange.sty & The \xext{sty}le file.
% \end{tabular}
%
% \bigskip
%
% \noindent For required other packages, please see the preceding subsection.
%
% \paragraph{Unpacking.} The \xfile{.dtx} file is a self-extracting
% \docstrip{} archive. The files are extracted by running the
% \xfile{.dtx} through \plainTeX:
% \begin{quote}
%   \verb|tex footnoterange.dtx|
% \end{quote}
%
% About generating the documentation see paragraph~\ref{GenDoc} below.
%
% \paragraph{TDS.} Now the different files must be moved into
% the different directories in your installation TDS tree
% (also known as \xfile{texmf} tree):
% \begin{quote}
% \def\t{^^A
% \begin{tabular}{@{}>{\ttfamily}l@{ $\rightarrow$ }>{\ttfamily}l@{}}
%   footnoterange.sty & tex/latex/footnoterange/footnoterange.sty\\
%   footnoterange.pdf & doc/latex/footnoterange/footnoterange.pdf\\
%   footnoterange-example.tex & doc/latex/footnoterange/footnoterange-example.tex\\
%   footnoterange-example.pdf & doc/latex/footnoterange/footnoterange-example.pdf\\
%   footnoterange.dtx & source/latex/footnoterange/footnoterange.dtx\\
% \end{tabular}^^A
% }^^A
% \sbox0{\t}^^A
% \ifdim\wd0>\linewidth
%   \begingroup
%     \advance\linewidth by\leftmargin
%     \advance\linewidth by\rightmargin
%   \edef\x{\endgroup
%     \def\noexpand\lw{\the\linewidth}^^A
%   }\x
%   \def\lwbox{^^A
%     \leavevmode
%     \hbox to \linewidth{^^A
%       \kern-\leftmargin\relax
%       \hss
%       \usebox0
%       \hss
%       \kern-\rightmargin\relax
%     }^^A
%   }^^A
%   \ifdim\wd0>\lw
%     \sbox0{\small\t}^^A
%     \ifdim\wd0>\linewidth
%       \ifdim\wd0>\lw
%         \sbox0{\footnotesize\t}^^A
%         \ifdim\wd0>\linewidth
%           \ifdim\wd0>\lw
%             \sbox0{\scriptsize\t}^^A
%             \ifdim\wd0>\linewidth
%               \ifdim\wd0>\lw
%                 \sbox0{\tiny\t}^^A
%                 \ifdim\wd0>\linewidth
%                   \lwbox
%                 \else
%                   \usebox0
%                 \fi
%               \else
%                 \lwbox
%               \fi
%             \else
%               \usebox0
%             \fi
%           \else
%             \lwbox
%           \fi
%         \else
%           \usebox0
%         \fi
%       \else
%         \lwbox
%       \fi
%     \else
%       \usebox0
%     \fi
%   \else
%     \lwbox
%   \fi
% \else
%   \usebox0
% \fi
% \end{quote}
% If you have a \xfile{docstrip.cfg} that configures and enables \docstrip's
% TDS installing feature, then some files can already be in the right
% place, see the documentation of \docstrip.
%
% \subsection{Refresh file name databases}
%
% If your \TeX~distribution (\TeX{} Live, \mikTeX, \teTeX, \dots) relies on
% file name databases, you must refresh these. For example, \teTeX{} users run
% \verb|texhash| or \verb|mktexlsr|.
%
% \subsection{Some details for the interested}
%
% \paragraph{Unpacking with \LaTeX{}.}
% The \xfile{.dtx} chooses its action depending on the format:
% \begin{description}
% \item[\plainTeX:] Run \docstrip{} and extract the files.
% \item[\LaTeX:] Generate the documentation.
% \end{description}
% If you insist on using \LaTeX{} for \docstrip{} (really,
% \docstrip{} does not need \LaTeX ), then inform the autodetect routine
% about your intention:
% \begin{quote}
%   \verb|latex \let\install=y\input{footnoterange.dtx}|
% \end{quote}
% Do not forget to quote the argument according to the demands
% of your shell.
%
% \paragraph{Generating the documentation.\label{GenDoc}}
% You can use both the \xfile{.dtx} or the \xfile{.drv} to generate
% the documentation. The process can be configured by a
% configuration file \xfile{ltxdoc.cfg}. For instance, put the following
% line into this file, if you want to have A4 as paper format:
% \begin{quote}
%   \verb|\PassOptionsToClass{a4paper}{article}|
% \end{quote}
%
% \noindent An example follows how to generate the
% documentation with \pdfLaTeX :
%
% \begin{quote}
%\begin{verbatim}
%pdflatex footnoterange.drv
%makeindex -s gind.ist footnoterange.idx
%pdflatex footnoterange.drv
%makeindex -s gind.ist footnoterange.idx
%pdflatex footnoterange.drv
%\end{verbatim}
% \end{quote}
%
% \subsection{Compiling the example}
%
% The example file, \textsf{footnoterange-example.tex}, can be compiled via\newline
% \indent \verb|(pdf)latex footnoterange-example.tex|\newline
% but will need probably two compiler runs to get everything right.\newline
% \verb|footnoterange-example.tex| can be created by\newline
% \indent \verb|tex footnoterange.dtx|.\newline
% \verb|footnoterange.dtx| is at
% \url{https://www.ctan.org/pkg/footnoterange}.\newline
%
% \section{Acknowledgements}
%
% I (H.-Martin M\"{u}nch) would like to thank \textsc{Heiko Oberdiek}
% for providing a~lot~(!) of useful packages (from
% which I also got everything I know about creating a file in \xext{dtx}
% format, ok, say it: copying), \textsc{Frank Mittelbach} for several bug reports,
% for code for improving the package, and for pointing me to the \xpackage{fnpct}
% package as alternative, Willie Wong for reporting a bug and its fix,
% everybody of the CTAN team for managing CTAN,
% the \Newsgroup{comp.text.tex} and \Newsgroup{de.comp.text.tex} newsgroups,
% and everybody at \url{https://tex.stackexchange.com/}
% for their help in all things \hbox{\TeX{}.}
% This package was created as response to question
% \textquotedblleft Compressing consecutive footnote
% marks\textquotedblright{} by user001\newline
% (\url{https://tex.stackexchange.com/users/10127/user001}) on\newline
% \url{https://tex.stackexchange.com/q/44165/6865}.
%
% \newpage
% \phantomsection
% \begin{History}\label{History}
%   \begin{Version}{2011/02/11 v0.1ff.}
%     \item Some versions at \url{https://tex.stackexchange.com/q/44165/6865}.
%   \end{Version}
%   \begin{Version}{2012/02/17 v1.0a}
%     \item Package created and uploaded to CTAN.
%   \end{Version}
%   \begin{Version}{2018/10/15 v1.0b}
%     \item Fixed/updated minor details.
%     \item Bug fix: Did not work without \xpackage{hyperref}, wrong spacefactor.
%     \item Listing the \href{https://www.ctan.org/pkg/fnpct}{\xpackage{fnpct}}
%           package as alternative.
%   \end{Version}
%   \begin{Version}{2019/11/24 v1.0c}
%     \item Bug fix: Issue with footnote symbols.
%   \end{Version}
% \end{History}
%
% \bigskip
%
% When you find a mistake or have a suggestion for an improvement of this package,
% please send an e-mail to the maintainer, thanks! (Please see BUG REPORTS in the README.)
%
% \PrintIndex
%
% \Finale
\endinput
%        (quote the arguments according to the demands of your shell)
%
% Documentation:
%    (a) If footnoterange.drv is present:
%           (pdf)latex footnoterange.drv
%           makeindex -s gind.ist footnoterange.idx
%           (pdf)latex footnoterange.drv
%           makeindex -s gind.ist footnoterange.idx
%           (pdf)latex footnoterange.drv
%    (b) Without footnoterange.drv:
%           (pdf)latex footnoterange.dtx
%           makeindex -s gind.ist footnoterange.idx
%           (pdf)latex footnoterange.dtx
%           makeindex -s gind.ist footnoterange.idx
%           (pdf)latex footnoterange.dtx
%
%    The class ltxdoc loads the configuration file ltxdoc.cfg
%    if available. Here you can specify further options, e.g.
%    use DIN A4 as paper format:
%       \PassOptionsToClass{a4paper}{article}
%
% Installation:
%    TDS:tex/latex/footnoterange/footnoterange.sty
%    TDS:doc/latex/footnoterange/footnoterange.pdf
%    TDS:doc/latex/footnoterange/footnoterange-example.tex
%    TDS:doc/latex/footnoterange/footnoterange-example.pdf
%    TDS:source/latex/footnoterange/footnoterange.dtx
%
%<*ignore>
\begingroup
  \catcode123=1 %
  \catcode125=2 %
  \def\x{LaTeX2e}%
\expandafter\endgroup
\ifcase 0\ifx\install y1\fi\expandafter
         \ifx\csname processbatchFile\endcsname\relax\else1\fi
         \ifx\fmtname\x\else 1\fi\relax
\else\csname fi\endcsname
%</ignore>
%<*install>
\input docstrip.tex
\Msg{*********************************************************************************}
\Msg{* Installation}
\Msg{* Package: footnoterange 2019/11/24 v1.0c References to ranges of footnotes (HMM)}
\Msg{*********************************************************************************}

\keepsilent
\askforoverwritefalse

\let\MetaPrefix\relax
\preamble

This is a generated file.

Project: footnoterange
Version: 2019/11/24 v1.0c

Copyright (C) 2012 - 2019 by
    H.-Martin M"unch <Martin dot Muench at Uni-Bonn dot de>
Portions of code copyrighted by other people as marked;
lines marked "FMi": code by Frank Mittelbach, thanks!

The usual disclaimer applies:
If it doesn't work right that's your problem.
(Nevertheless, please send an e-mail to the maintainer
 when you find an error in this package.)

This work may be distributed and/or modified under the
conditions of the LaTeX Project Public License, either
version 1.3c of this license or (at your option) any later
version. This version of this license is in
   https://www.latex-project.org/lppl/lppl-1-3c.txt
and the latest version of this license is in
   https://www.latex-project.org/lppl.txt
and version 1.3c or later is part of all distributions of
LaTeX version 2005/12/01 or later.

This work has the LPPL maintenance status "maintained".

The Current Maintainer of this work is H.-Martin Muench.

This work consists of the main source file footnoterange.dtx,
the README, and the derived files
   footnoterange.sty, footnoterange.pdf,
   footnoterange.ins, footnoterange.drv,
   footnoterange-example.tex, footnoterange-example.pdf.

In memoriam
 Claudia Simone Barth + 1996/01/30
 Tommy Muench + 2014/01/02
 Hans-Klaus Muench + 2014/08/24

\endpreamble
\let\MetaPrefix\DoubleperCent

\generate{%
  \file{footnoterange.ins}{\from{footnoterange.dtx}{install}}%
  \file{footnoterange.drv}{\from{footnoterange.dtx}{driver}}%
  \usedir{tex/latex/footnoterange}%
  \file{footnoterange.sty}{\from{footnoterange.dtx}{package}}%
  \usedir{doc/latex/footnoterange}%
  \file{footnoterange-example.tex}{\from{footnoterange.dtx}{example}}%
}

\catcode32=13\relax% active space
\let =\space%
\Msg{************************************************************************}
\Msg{*}
\Msg{* To finish the installation you have to move the following}
\Msg{* file into a directory searched by TeX:}
\Msg{*}
\Msg{* footnoterange.sty}
\Msg{*}
\Msg{* To produce the documentation run the file `footnoterange.drv'}
\Msg{* through (pdf)LaTeX, e.g.}
\Msg{*  pdflatex footnoterange.drv}
\Msg{*  makeindex -s gind.ist footnoterange.idx}
\Msg{*  pdflatex footnoterange.drv}
\Msg{*  makeindex -s gind.ist footnoterange.idx}
\Msg{*  pdflatex footnoterange.drv}
\Msg{*}
\Msg{* At least three runs are necessary e.g. to get the}
\Msg{*  references right!}
\Msg{*}
\Msg{* Happy TeXing!}
\Msg{*}
\Msg{************************************************************************}

\endbatchfile
%</install>
%<*ignore>
\fi
%</ignore>
%
% \section{The documentation driver file}
%
% The next bit of code contains the documentation driver file for
% \TeX , i.\,e., the file that will produce the documentation you
% are currently reading. It will be extracted from this file by the
% \verb|docstrip| programme. That is, run \LaTeX{} on \verb|docstrip|
% and specify the \verb|driver| option when \verb|docstrip|
% asks for options.
%
%    \begin{macrocode}
%<*driver>
\NeedsTeXFormat{LaTeX2e}[2015/01/01]
\ProvidesFile{footnoterange.drv}[2019/11/24 v1.0c
            References to ranges of footnotes (HMM)]
\documentclass{ltxdoc}[2015/03/26]% v2.0w
\usepackage{holtxdoc}[2012/03/21]%  v0.24
%% footnoterange may work with earlier versions of LaTeX2e and those
%% class and package, but this was not tested.
%% Please consider updating your LaTeX, class, and package
%% to the most recent version (if they are not already the most
%% recent version).
\hypersetup{%
 pdfsubject={References to ranges of footnotes (HMM)},%
 pdfkeywords={LaTeX, footnoterange, footnotes, footnote, reference, H.-Martin Muench},%
 pdfencoding=auto,%
 pdflang={en},%
 breaklinks=true,%
 linktoc=all,%
 pdfstartview=FitH,%
 pdfpagelayout=OneColumn,%
 bookmarksnumbered=true,%
 bookmarksopen=true,%
 bookmarksopenlevel=3,%
 pdfmenubar=true,%
 pdftoolbar=true,%
 pdfwindowui=true,%
 pdfnewwindow=true%
}
\CodelineIndex
\hyphenation{printing docu-ment}
\gdef\unit#1{\mathord{\thinspace\mathrm{#1}}}
\begin{document}
  \DocInput{footnoterange.dtx}%
\end{document}
%</driver>
%    \end{macrocode}
%
% \fi
%
% \CheckSum{321}
%
% \CharacterTable
%  {Upper-case    \A\B\C\D\E\F\G\H\I\J\K\L\M\N\O\P\Q\R\S\T\U\V\W\X\Y\Z
%   Lower-case    \a\b\c\d\e\f\g\h\i\j\k\l\m\n\o\p\q\r\s\t\u\v\w\x\y\z
%   Digits        \0\1\2\3\4\5\6\7\8\9
%   Exclamation   \!     Double quote  \"     Hash (number) \#
%   Dollar        \$     Percent       \%     Ampersand     \&
%   Acute accent  \'     Left paren    \(     Right paren   \)
%   Asterisk      \*     Plus          \+     Comma         \,
%   Minus         \-     Point         \.     Solidus       \/
%   Colon         \:     Semicolon     \;     Less than     \<
%   Equals        \=     Greater than  \>     Question mark \?
%   Commercial at \@     Left bracket  \[     Backslash     \\
%   Right bracket \]     Circumflex    \^     Underscore    \_
%   Grave accent  \`     Left brace    \{     Vertical bar  \|
%   Right brace   \}     Tilde         \~}
%
% \GetFileInfo{footnoterange.drv}
%
% \begingroup
%   \def\x{\#,\$,\^,\_,\~,\ ,\&,\{,\},\%}%
%   \makeatletter
%   \@onelevel@sanitize\x
% \expandafter\endgroup
% \expandafter\DoNotIndex\expandafter{\x}
% \expandafter\DoNotIndex\expandafter{\string\ }
% \begingroup
%   \makeatletter
%     \lccode`9=32\relax
%     \lowercase{%^^A
%       \edef\x{\noexpand\DoNotIndex{\@backslashchar9}}%^^A
%     }%^^A
%   \expandafter\endgroup\x
% \DoNotIndex{\,,\\}
% \DoNotIndex{\documentclass,\usepackage,\ProvidesPackage,\begin,\end}
% \DoNotIndex{\MessageBreak}
% \DoNotIndex{\NeedsTeXFormat,\DoNotIndex,\verb}
% \DoNotIndex{\def,\edef,\gdef,\global,\xdef}
% \DoNotIndex{\ifx,\listfiles,\mathord,\mathrm}
% \DoNotIndex{\bigskip,\space,\thinspace,\ldots}
% \DoNotIndex{\indent,\noindent,\newline,\linebreak,\pagebreak,\newpage}
% \DoNotIndex{\textbf,\textit,\textsf,\textsc,\texttt,\textquotedblleft,\textquotedblright}
% \DoNotIndex{\plainTeX,\TeX,\LaTeX,\pdfLaTeX}
% \DoNotIndex{\section,\pagenumbering,\smallskip}
% \DoNotIndex{\@ifpackageloaded}
% \DoNotIndex{\arabic, \Alph, \alph, \alphalph, \Roman, \roman, \fnsymbol}
%
% \title{The \xpackage{footnoterange} package}
% \date{2019/11/24 v1.0c}
% \author{H.-Martin M\"{u}nch\\\xemail{Martin.Muench at Uni-Bonn.de}}
%
% \maketitle
%
% \begin{abstract}
% \noindent This package provides the environments |footnoterange| and
% |footnoterange*|. Multiple footnotes inside these environments
% are not referenced as e.\,g.
% \makeatletter%
% \hbox{\textquotedblleft \@textsuperscript {\normalfont 1 2 3}\textquotedblright}
% but as \hbox{\textquotedblleft \@textsuperscript {\normalfont 1--3}\textquotedblright ,\ }
% \makeatother%
% i.\,e. the range (from first to last referred footnote at that place) is given.
% Depending on loading of \xpackage{hyperref} and use of its |hyperfootnotes|-option
% the references are hyperlinked. (References to footnotes in |footnoterange*|
% are never hyperlinked.)
% \end{abstract}
%
% \bigskip
%
% \noindent Disclaimer for web links: The author is not responsible for any contents
% referred to in this work unless if having full knowledge of illegal contents.
% If any damage occurs by the use of information presented there, only the
% author of the respective pages might be liable, not the one who has referred
% to these pages.
%
% \noindent Save per page about $200\unit{ml}$ water,
% $2\unit{g}$ CO$_{2}$ and $2\unit{g}$ wood:
% Therefore please print only if this is really necessary.
%
% \medskip
%
% \tableofcontents
%
%\newpage
%
% \section{Introduction}
% \indent This \LaTeXe{} package provides the environments |footnoterange| and\newline
% |footnoterange*|. Multiple footnotes inside these environments are not referenced
% as e.\,g.
% \makeatletter%
% \hbox{\textquotedblleft \@textsuperscript {\normalfont 1 2 3}\textquotedblright}
% but as \hbox{\textquotedblleft \@textsuperscript {\normalfont 1--3}\textquotedblright ,\ }
% \makeatother%
% i.\,e. the range (from first to last
% referred footnote at that place) is given. Depending on loading of
% \xpackage{hyperref} and use of its |hyperfootnotes|-option the references are
% hyperlinked. References to footnotes in |footnoterange*| are never hyperlinked.\newline
% |\footnotemark\footnotetext| instead of |\footnote| is not supported inside
% |footnoterange| and |footnoterange*| environments.
%
% \bigskip
%
% \section{Usage}
%
% \indent Just load the package placing
% \begin{quote}
%   |\usepackage{footnoterange}|
% \end{quote}
% \noindent in the preamble of your \LaTeXe{} source file. There are no options to
% be used. The package should be loaded after \xpackage{hyperref} (if that package is used)
% and after any other package redefining footnotes. Usage of the |footnoterange| (or
% |footnoterange*|) environment is simple: Just encapsulate the footnotes, which shall
% be referred as range instead of referencing as single ones (see the example file).
%
% About hyperlinks:\newline
% Footnote-references in a |footnoterange*| environment are never hyperlinked.\newline
% Footnote-references in a |footnoterange| environment may be hyperlinked:
% \begin{itemize}
%  \item When the \xpackage{hyperref} package is not loaded, they are not
%         hyperlinked (naturally).
%  \item When the \xpackage{hyperref} package is loaded, it depends on the
%         |hyperfootnotes| option of the \xpackage{hyperref} package:
%  \begin{itemize}
%   \item With |hyperfootnotes=true|, the references are hyperlinked.
%   \item With |hyperfootnotes=false|, the references are not hyperlinked.
%  \end{itemize}
% \end{itemize}
% |\footnote|s inside of the |footnoterange|/{}|footnoterange*| environment
% can use the optional parameter, i.\,e. |\footnote[42]{text}| even if this is the
% third footnote. When giving ranges as reference, this does not seem to be
% such a good idea, if this is not done consistently (i.\,e. footnotes
% |[42], [43], [44]| resulting in 42--44 is OK,
% but |[42]| (original~2), unchanged~3, unchanged~4 resulting in
% 42--4 is confusing). A warning about these changed footnotemarks is written
% into the \xfile{.log}-file.
%
% When there is only a single footnote in a |footnoterange| environment,
% this does not produce some reference like %
% \makeatletter%
% \hbox{\textquotedblleft \@textsuperscript {\normalfont 7--7}\textquotedblright{},\ }%
% \makeatother%
% but only one footnotemark is used automatically. Because it is strange to use
% a |footnoterange| environment without a range of footnotes, an info is written
% into the \xfile{.log}-file about this.
%
% The usage of other footnote-number-representations
% (e.\,g.~\verb|\Roman|, \verb|\roman|, \verb|\Alph|, \verb|\alph|,
% \verb|\fnsymbol|) is also possible, but two things must be taken
% into account: The number of footnote references is restricted (for
% example with \verb|\Alph| only references A to Z are possible), which
% can be fixed e.\,g. with the \url{https://ctan.org/pkg/alphalph} package
% (but not automatically), and references to footnote-symbol-ranges (\verb|\fnsymbol|)
% are probably not very clear.
%
% \subsection*{Options}
% \indent The \xpackage{footnoterange} package takes no options.
%
% \section{Alternatives\label{sec:Alternatives}}
%
% The \xpackage{fnpct} (\url{https://www.ctan.org/pkg/fnpct}) package allows to set
% multiple footnotes (|\multfootnote|), suppress a footnotemark \hbox{(|*|),}
% and customize the multi-footnote-separator \hbox{(|\setfnpct{mult-fn-sep={|
% \emph{something} |}}|)}.
% Combining these with stepping of the footnotecounter gives a footnoterange:
% 
% \begin{verbatim}
% \documentclass{article}
% \usepackage{hyperref}
% \usepackage{fnpct}
% \begin{document}
% The three little pigs built their houses
% out of straw\footnote{not to be confused with hay},
% sticks\footnote{or lumber according to some sources}
% and bricks%
% \makeatletter%
% \setfnpct{mult-fn-sep={--}}%
% \multfootnote{probably fired clay bricks\addtocounter\@mpfn{+1};%
% *or something else;%
% what do I know}. %
% \makeatother%
% \end{document}
% \end{verbatim}
% 
% \noindent If you programmed or found another alternative, which is available at
% \url{https://www.CTAN.org/}, please send an e-mail to me with the name,
% location at CTAN, and a short notice, and I will probably list it here.
%
% \section{Example}
%
%    \begin{macrocode}
%<*example>
\documentclass[british]{article}[2014/09/29]% v1.4h
%%%%%%%%%%%%%%%%%%%%%%%%%%%%%%%%%%%%%%%%%%%%%%%%%%%%%%%%%%%%%%%%%%%%%
\usepackage[%
 extension=pdf,%
 plainpages=false,%
 pdfpagelabels=true,%
 hyperindex=false,%
 pdflang={en},%
 pdftitle={footnoterange package example},%
 pdfauthor={H.-Martin Muench},%
 pdfsubject={Example for the footnoterange package},%
 pdfkeywords={LaTeX, footnoterange, H.-Martin Muench},%
 pdfview=FitH,%
 pdfstartview=FitH,%
 pdfpagelayout=OneColumn,%
 hyperfootnotes=true%
]{hyperref}[2012/11/06]% v6.83m

\usepackage{footnoterange}[2019/11/24]% v1.0c

\gdef\unit#1{\mathord{\thinspace\mathrm{#1}}}%
\listfiles
\begin{document}
\pagenumbering{arabic}
\section*{Example for footnoterange}

This example demonstrates the use of package\newline
\textsf{footnoterange}, v1.0c as of 2019/11/24 (HMM).\newline
The package does not use options.\newline
\verb|footnoterange-example.pdf| can be created by\newline
\verb|pdflatex footnoterange-example.tex|,\newline
and \verb|footnoterange-example.tex| can be created by\newline
\verb|tex footnoterange.dtx|.\newline
\verb|footnoterange.dtx| is at
\url{https://www.ctan.org/pkg/footnoterange}.\newline

\noindent For more details please see the documentation!\newline

\noindent Save per page about $200\unit{ml}$ water,
$2\unit{g}$ CO$_{2}$ and $2\unit{g}$ wood:\newline
Therefore please print only if this is really necessary.\newline

This\footnote{Lorem} text bears a multiplicity of footnotes.
Because the \verb|hyperref| package is used with option
\verb|hyperfootnotes=true|, the footnotes
\begin{footnoterange}
\footnote{ipsum}%
\footnote{dolor}%
\footnote{sit}%
\footnote{amet,}%
\footnote{consectetur}%
\footnote{adipisicing}%
\footnote{elit,}%
\footnote{sed}%
\end{footnoterange}
are hyperlinked.\footnote{do eiusmod\ldots} Using
\verb|hyperfootnotes=false| or not using \verb|hyperref|
will remove the hyperlinks to the footnotes.\medskip

This text also bears a multiplicity of footnotes,
but due to the use of the starred
\begin{footnoterange*}
\footnote{Ut purus elit,}%
\footnote{vestibulum ut,}%
\footnote{placerat ac,}%
\footnote{adipiscing vitae,}%
\footnote{felis.}%
\footnote{Curabitur}%
\footnote{dictum}%
\footnote{gravida}%
\end{footnoterange*}
form of the environment they are not
hyperlinked.\footnote{mauris\ldots}\medskip

The use of the optional parameter for footnotes
\begin{footnoterange}
\footnote[101]{Nam arcu libero,}%
\footnote[102]{nonummy eget,}%
\footnote[103]{consectetuer id,}%
\footnote[104]{vulputate a,}%
\footnote[105]{magna.}
\end{footnoterange}
is also possible, but should be used consistently.

\pagebreak

When there is only one footnote
\begin{footnoterange}
\footnote{a single footnote}
\end{footnoterange}
in the \verb|footnoterange| environment, only one
footnotemark is used and an info is written into the
\verb|log|-file.\medskip

The usage of other footnote-number-representations
(e.\,g.~\verb|\Roman|, \verb|\roman|, \verb|\Alph|, \verb|\alph|,
\verb|\fnsymbol|) is also possible, but two things must be taken
into account: The number of footnote references is restricted
(for example with \verb|\Alph| only references A to Z are possible),
which can be fixed e.\,g. with the
\href{https://www.ctan.org/pkg/alphalph}{\texttt{alphalph} package},
and references to footnote-symbol-ranges (\verb|\fnsymbol|)
are probably not very clear.

\end{document}
%</example>
%    \end{macrocode}
%
% \StopEventually{}
%
% \section{The implementation}
%
% We start off by checking that we are loading into \LaTeXe{} and
% announcing the name and version of this package.
%
%    \begin{macrocode}
%<*package>
%    \end{macrocode}
%
%    \begin{macrocode}
\NeedsTeXFormat{LaTeX2e}[2011/06/27]
\ProvidesPackage{footnoterange}[2019/11/24 v1.0c
            References to ranges of footnotes (HMM)]
%    \end{macrocode}
%
% A short description of the \xpackage{footnoterange} package:
%
%    \begin{macrocode}
%% Provides the footnoterange and footnoterange* environments to
%% refer to ranges of footnotes.
%    \end{macrocode}
%
% We need the \xpackage{ltxcmds} package:
%
%    \begin{macrocode}
\RequirePackage{ltxcmds}[2011/11/09]% v1.22; for "\ltx@ifpackageloaded"
\RequirePackage{letltxmacro}[2010/09/02]% v1.4; for "\GlobalLetLtxMacro"

%    \end{macrocode}
%
% We use |\ifHy@hyperfootnotes ... \else ... \fi|, but when the \xpackage{hyperref}
% package has not been loaded, |\ifHy@hyperfootnotes| is not defined, and \LaTeX
% relates the |\else| and/or |\fi| (wrongly) to other |\if|s or groups.
% Therefore the |\ifHy@hyperfootnotes| are placed in their own commands,
% which are not looked at when the \xpackage{hyperref} package has not been loaded.
%
% \pagebreak
%
% \begin{macro}{\fnr@ifhyperrfn}
%    \begin{macrocode}
\newcommand{\fnr@ifhyperrfn}{%
  \ifHy@hyperfootnotes% option hyperfootnotes=true
    \addtocounter{Hfootnote}{+1}%
    % Code from the hyperref package
      \global\let\Hy@saved@currentHref\@currentHref%
      \hyper@makecurrent{Hfootnote}%
      \global\let\Hy@footnote@currentHref\@currentHref%
      \global\let\@currentHref\Hy@saved@currentHref%
    % End of code from the hyperref package
  \fi%
}

%    \end{macrocode}
% \end{macro}
%
% \begin{macro}{\fnr@rangefootnote}
% The |\fnr@rangefootnote| is the internal replacement for the original\newline
% |\footnote| command inside of a |footnoterange| environment.
%
%    \begin{macrocode}
\newcommand{\fnr@rangefootnote}[2][\empty]{%
%    \end{macrocode}
%
% |#1| contains the optional argument, which is predefined as |\empty|.
%
%    \begin{macrocode}
  \nolinebreak%
  % no \footnotemark here
  \addtocounter\@mpfn{+1}%
  \ltx@ifpackageloaded{hyperref}{\fnr@ifhyperrfn}{\relax}%
%    \end{macrocode}
%
% See the explanation before the declaration of the |\fnr@ifhyperrfn| command!
%
%    \begin{macrocode}
  \xdef\fnr@opt{#1}% contains the optional argument
  \xdef\fnr@arabic{\arabic\@mpfn}%
  \protected@edef\fnr@formated{\thefootnote}%
%    \end{macrocode}
%
% The |\footnotemark| is not set (yet), but the |\footnotetext|,
% and it gets a label.
%
%    \begin{macrocode}
  \ifx\fnr@opt\empty%
    \footnotetext{\label{fnr:\fnr@arabic}#2}%
  \else%
    \PackageWarning{footnoterange}{%
      Custom footnote-numbers used in a footnoterange\MessageBreak%
      environment: Footnote with natural number\MessageBreak%
      \ifx\fnr@formated\fnr@arabic%
        \fnr@arabic\space%
      \else%
        \space\fnr@arabic\space (\fnr@formated )%
      \fi%
      \MessageBreak%
      was renumbered to \fnr@opt\MessageBreak%
     }%
    \ltx@ifpackageloaded{hyperref}{% hyperref loaded
      \footnotetext[#1]{\phantomsection\label{fnr:\fnr@arabic}#2}%
     }{% hyperref not loaded
      \footnotetext[#1]{\label{fnr:\fnr@arabic}#2}%
     }%
  \fi%
}

%    \end{macrocode}
% \end{macro}
%
%    \begin{macrocode}

%    \end{macrocode}
% \pagebreak
%
% |\ifHy@hyperfootnotes| \textquotedblleft encapsulation\textquotedblright{} again:
%
% \begin{macro}{\fnr@ifhyperfnrA}
%    \begin{macrocode}
\newcommand{\fnr@ifhyperfnrA}{%
  \ifHy@hyperfootnotes% option hyperfootnotes=true
    \hbox{\hyperref[fnr:\fnr@first]{%
          \@textsuperscript{\normalfont \ref*{fnr:\fnr@first}}}}%
  \else% option hyperfootnotes=false
    \hbox{\@textsuperscript{\normalfont \ref*{fnr:\fnr@first}}}%
  \fi%
}

%    \end{macrocode}
% \end{macro}
%
% \begin{macro}{\fnr@ifhyperfnrN}
%    \begin{macrocode}
\newcommand{\fnr@ifhyperfnrN}{%
  \ifHy@hyperfootnotes% option hyperfootnotes=true
    \hbox{\hyperref[fnr:\fnr@first]{%
            \@textsuperscript{\normalfont \ref*{fnr:\fnr@first}}}%
          \@textsuperscript{\normalfont --}%
          \hyperref[fnr:\arabic\@mpfn]{%
            \@textsuperscript{\normalfont \ref*{fnr:\arabic\@mpfn}}}}%
  \else% option hyperfootnotes=false
    \hbox{\@textsuperscript{\normalfont \ref*{fnr:\fnr@first}%
                                        --\ref*{fnr:\arabic\@mpfn}}}%
  \fi%
}

%    \end{macrocode}
% \end{macro}
%
% \begin{environment}{footnoterange}
% The |footnoterange| environment remembers the next footnote number as\newline
% |\fnr@first| and the current |\footnote| definition as |\fnr@origfn|
% and uses |\fnr@rangefootnote| instead.
%
%    \begin{macrocode}
\newenvironment{footnoterange}{%
  % at begin of the footnoterange environment
  \ifhmode \unskip \fi% FMi
  \addtocounter\@mpfn{+1}%
  \xdef\fnr@first{\arabic\@mpfn}%
  \addtocounter\@mpfn{-1}%
  \let\fnr@origfn\footnote%
  \let\footnote\fnr@rangefootnote%
  \GlobalLetLtxMacro{\fnr@origfnmark}{\footnotemark}%
  \GlobalLetLtxMacro{\footnotemark}{\fnr@rangefootnotemark}%
  \ignorespaces% FMi
%    \end{macrocode}
%
% End of the environment:
%
%    \begin{macrocode}
}{% at end of the footnoterange environment
  \ifhmode% FMi 
    \unskip% FMi
%    \end{macrocode}
%
% Remember the spacefactor for restoring after the environment has ended
% (only in |hmode|!). \% FMi
%
%    \begin{macrocode}
    \xdef\fnr@aftergroup{\spacefactor\the\spacefactor\relax}% FMi
  \else% FMi
%    \end{macrocode}
%
% Otherwise do something harmless. \% FMi
%
%    \begin{macrocode}
    \xdef\fnr@aftergroup{\relax}%
  \fi% FMi
%    \end{macrocode}
%
% At the end of the environment the previous definition of |\footnote| is restored
%
%    \begin{macrocode}
  \let\footnote\fnr@origfn%
  \GlobalLetLtxMacro{\footnotemark}{\fnr@origfnmark}%
%    \end{macrocode}
%
% and the references are placed:
%
%    \begin{macrocode}
  \ifx\fnr@first\fnr@arabic% There is only one reference in the environment.
    \PackageInfo{footnoterange}{%
      Environment footnoterange used,\MessageBreak%
      but only one footnote (number \fnr@first ) is contained%
      \MessageBreak%
     }%
    \ltx@ifpackageloaded{hyperref}{\fnr@ifhyperfnrA% hyperref package loaded
    }{% hyperref package not loaded
      \hbox{\@textsuperscript{\normalfont \ref{fnr:\fnr@first}}}%
     }%
  \else%
    \ltx@ifpackageloaded{hyperref}{\fnr@ifhyperfnrN% hyperref package loaded
    }{% hyperref package not loaded
      \hbox{\@textsuperscript{\normalfont \ref{fnr:\fnr@first}%
                                        --\ref{fnr:\arabic\@mpfn}}}%
     }%
  \fi%
%    \end{macrocode}
%
% Restore the spacefactor after the environment has ended. \% FMi
%
%    \begin{macrocode}
  \aftergroup\fnr@aftergroup% FMi
}

%    \end{macrocode}
% \end{environment}
%
% \begin{environment}{footnoterange*}
% The |footnoterange*| environment does the same as the |footnoterange|
% environment but suppresses all hyperlinks to footnotes inside.
%
%    \begin{macrocode}
\newenvironment{footnoterange*}{%
  % at begin of the footnoterange* environment
  \ifhmode \unskip \fi% FMi
  \addtocounter\@mpfn{+1}%
  \xdef\fnr@first{\arabic\@mpfn}%
  \addtocounter\@mpfn{-1}%
  \let\fnr@origfn\footnote%
  \let\footnote\fnr@rangefootnote%
  \GlobalLetLtxMacro{\fnr@origfnmark}{\footnotemark}%
  \GlobalLetLtxMacro{\footnotemark}{\fnr@rangefootnotemark}%
  \ignorespaces% FMi
}{% at end of the footnoterange* environment
  \ifhmode% FMi 
    \unskip% FMi
    \xdef\fnr@aftergroup{\spacefactor\the\spacefactor\relax}% FMi
  \else% FMi
    \xdef\fnr@aftergroup{\relax}%
  \fi% FMi
  \let\footnote\fnr@origfn%
  \GlobalLetLtxMacro{\footnotemark}{\fnr@origfnmark}%
  \ifx\fnr@first\fnr@arabic% There is only one reference in the environment.
    \PackageInfo{footnoterange}{%
      Environment footnoterange used,\MessageBreak%
      but only one footnote (number \fnr@first ) is contained%
      \MessageBreak%
     }%
%    \end{macrocode}
% \pagebreak
%    \begin{macrocode}
    \ltx@ifpackageloaded{hyperref}{% hyperref package loaded
      \hbox{\@textsuperscript{\normalfont \ref*{fnr:\fnr@first}}}%
    }{% hyperref package not loaded
      \hbox{\@textsuperscript{\normalfont \ref{fnr:\fnr@first}}}%
     }%
  \else%
    \ltx@ifpackageloaded{hyperref}{% hyperref package loaded
      \hbox{\@textsuperscript{\normalfont \ref*{fnr:\fnr@first}%
                                        --\ref*{fnr:\arabic\@mpfn}}}%
    }{% hyperref package not loaded
      \hbox{\@textsuperscript{\normalfont \ref{fnr:\fnr@first}%
                                        --\ref{fnr:\arabic\@mpfn}}}%
     }%
  \fi%
  \aftergroup\fnr@aftergroup% FMi
}

%    \end{macrocode}
% \end{environment}
%
% The \xpackage{footnoterange} package does not handle |\footnotemark|s,
% but at least the user is made aware of this:
%
% \begin{macro}{\fnr@rangefootnotemark}
%    \begin{macrocode}

\newcommand{\fnr@rangefootnotemark}[1][\empty]{%
  \PackageError{footnoterange}{%
     \string\footnotemark\space not supported in footnoterange(*) environment%
   }{The footnoterange package does not handle a \string\footnotemark.\MessageBreak%
     Either use \string\footnote\string{...\string}\space instead of%
     \string\footnotemark\space and \string\footnotetext\string{...\string}\MessageBreak%
     in footnoterange and footnoterange* environments,\MessageBreak%
     or do not use those environments around a \string\footnotemark\MessageBreak%
     (or \string\renewcommand\string{\string\fnr@rangefootnotemark\string}\MessageBreak%
     with your own fixing code).%
    }%
  \xdef\fnr@mopt{#1}%
  \ifx\fnr@mopt\empty%
    \fnr@origfnmark%
  \else%
    \fnr@origfnmark[#1]%
  \fi%
}

%    \end{macrocode}
% \end{macro}
%
%    \begin{macrocode}
%</package>
%    \end{macrocode}
%
% \pagebreak
%
% \section{Installation}
%
% \subsection{Downloads\label{ss:Downloads}}
%
% Everything is available on CTAN,
% \url{https://www.ctan.org/tex-archive/},
% but may need additional packages themselves.
%
% \DescribeMacro{footnoterange.dtx}
% For unpacking the \xfile{footnoterange.dtx} file
% and constructing the documentation it is required:
% \begin{description}
% \item[-] \TeX{} Format \LaTeXe{}, \url{https://www.CTAN.org/}
%
% \item[-] document class \xpackage{ltxdoc}, 2015/03/26, v2.0w,
%           \url{https://ctan.org/pkg/ltxdoc}
%
% \item[-] package \xpackage{holtxdoc}, 2012/03/21, v0.24,
%           \url{https://ctan.org/pkg/holtxdoc}
% \end{description}
%
% \DescribeMacro{footnoterange.sty}
% The |footnoterange.sty| for \LaTeXe{} (i.\,e. each document using
% the \xpackage{footnoterange} package) requires:
% \begin{description}
% \item[-] \TeX{} Format \LaTeXe{}, \url{https://www.CTAN.org/}
%
% \item[-] package \xpackage{ltxcmds}, 2011/04/18, v1.20,
%           \url{https://www.ctan.org/pkg/ltxcmds}
% \end{description}
%
% \DescribeMacro{footnoterange-example.tex}
% The |footnoterange-example.tex| requires the same files as all
% documents using the \xpackage{footnoterange} package and additionally:
% \begin{description}
% \item[-] class \xpackage{article}, 2007/10/19, v1.4h, from \xpackage{classes.dtx}:\newline
%           \url{https://www.ctan.org/tex-archive/macros/latex/base/}
%
% \item[-] package \xpackage{footnoterange}, 2019/11/24, v1.0c,\newline
%           \url{https://www.ctan.org/pkg/footnoterange}\newline
%           (Well, it is the example file for this package, and because you are
%            reading the documentation for the \xpackage{footnoterange} package,
%            it can be assumed that you already have some version of it --
%            is it the current one?)
%
% \item[-] package \xpackage{hyperref}, 2012/11/06, v6.83m,\newline
%           \url{https://www.ctan.org/pkg/hyperref}
% \end{description}
%
% \DescribeMacro{Oberdiek}
% \DescribeMacro{holtxdoc}
% \DescribeMacro{ltxcmds}
% \DescribeMacro{alphalph}
% All packages of \textsc{Heiko Oberdiek}'s bundle `oberdiek'
% (especially \xpackage{holtxdoc}, \xpackage{ltxcmds}, and \xpackage{alphalph})
% are also available in a TDS compliant ZIP archive:\newline
% \url{http://mirror.ctan.org/install/macros/latex/contrib/oberdiek.tds.zip}.\newline
% It is probably best to download and use this, because the packages in there
% should be both recent and compatible.
%
% \DescribeMacro{hyperref}
% \noindent \xpackage{hyperref} is not included in that bundle and needs to be
% downloaded separately,\newline
% \url{http://mirrors.ctan.org/install/macros/latex/contrib/hyperref.tds.zip}.
%
% \DescribeMacro{M\"{u}nch}
% A list of my packages can be found at \url{https://www.ctan.org/author/muench-hm}.
%
% \pagebreak
%
% \subsection{Package, unpacking TDS}
% \paragraph{Package.} This package is available on \url{https://www.CTAN.org}.
% \begin{description}
% \item[\url{http://mirror.ctan.org/macros/latex/contrib/footnoterange/footnoterange.dtx}]\hspace*{0.1cm}
%       The source file.
% \item[\url{http://mirror.ctan.org/macros/latex/contrib/footnoterange/footnoterange.pdf}]\hspace*{0.1cm}
%       The documentation.
% \item[\url{http://mirror.ctan.org/macros/latex/contrib/footnoterange/footnoterange-example.pdf}]\hspace*{0.1cm}
%       The compiled example file, as it should look like.
% \item[\url{http://mirror.ctan.org/macros/latex/contrib/footnoterange/README}]\hspace*{0.1cm}\newline\hspace*{0.1cm}
%       The README file.
% \end{description}
% There is also a footnoterange.tds.zip available:
% \begin{description}
% \item[\url{http://mirror.ctan.org/install/macros/latex/contrib/footnoterange.tds.zip}]\hspace*{0.1cm}
%       Everything in TDS compliant, compiled format.
% \end{description}
% which additionally contains\newline
% \begin{tabular}{ll}
% footnoterange.ins & The installation file.\\
% footnoterange.drv & The driver to generate the documentation.\\
% footnoterange.sty & The \xext{sty}le file.
% \end{tabular}
%
% \bigskip
%
% \noindent For required other packages, please see the preceding subsection.
%
% \paragraph{Unpacking.} The \xfile{.dtx} file is a self-extracting
% \docstrip{} archive. The files are extracted by running the
% \xfile{.dtx} through \plainTeX:
% \begin{quote}
%   \verb|tex footnoterange.dtx|
% \end{quote}
%
% About generating the documentation see paragraph~\ref{GenDoc} below.
%
% \paragraph{TDS.} Now the different files must be moved into
% the different directories in your installation TDS tree
% (also known as \xfile{texmf} tree):
% \begin{quote}
% \def\t{^^A
% \begin{tabular}{@{}>{\ttfamily}l@{ $\rightarrow$ }>{\ttfamily}l@{}}
%   footnoterange.sty & tex/latex/footnoterange/footnoterange.sty\\
%   footnoterange.pdf & doc/latex/footnoterange/footnoterange.pdf\\
%   footnoterange-example.tex & doc/latex/footnoterange/footnoterange-example.tex\\
%   footnoterange-example.pdf & doc/latex/footnoterange/footnoterange-example.pdf\\
%   footnoterange.dtx & source/latex/footnoterange/footnoterange.dtx\\
% \end{tabular}^^A
% }^^A
% \sbox0{\t}^^A
% \ifdim\wd0>\linewidth
%   \begingroup
%     \advance\linewidth by\leftmargin
%     \advance\linewidth by\rightmargin
%   \edef\x{\endgroup
%     \def\noexpand\lw{\the\linewidth}^^A
%   }\x
%   \def\lwbox{^^A
%     \leavevmode
%     \hbox to \linewidth{^^A
%       \kern-\leftmargin\relax
%       \hss
%       \usebox0
%       \hss
%       \kern-\rightmargin\relax
%     }^^A
%   }^^A
%   \ifdim\wd0>\lw
%     \sbox0{\small\t}^^A
%     \ifdim\wd0>\linewidth
%       \ifdim\wd0>\lw
%         \sbox0{\footnotesize\t}^^A
%         \ifdim\wd0>\linewidth
%           \ifdim\wd0>\lw
%             \sbox0{\scriptsize\t}^^A
%             \ifdim\wd0>\linewidth
%               \ifdim\wd0>\lw
%                 \sbox0{\tiny\t}^^A
%                 \ifdim\wd0>\linewidth
%                   \lwbox
%                 \else
%                   \usebox0
%                 \fi
%               \else
%                 \lwbox
%               \fi
%             \else
%               \usebox0
%             \fi
%           \else
%             \lwbox
%           \fi
%         \else
%           \usebox0
%         \fi
%       \else
%         \lwbox
%       \fi
%     \else
%       \usebox0
%     \fi
%   \else
%     \lwbox
%   \fi
% \else
%   \usebox0
% \fi
% \end{quote}
% If you have a \xfile{docstrip.cfg} that configures and enables \docstrip's
% TDS installing feature, then some files can already be in the right
% place, see the documentation of \docstrip.
%
% \subsection{Refresh file name databases}
%
% If your \TeX~distribution (\TeX{} Live, \mikTeX, \teTeX, \dots) relies on
% file name databases, you must refresh these. For example, \teTeX{} users run
% \verb|texhash| or \verb|mktexlsr|.
%
% \subsection{Some details for the interested}
%
% \paragraph{Unpacking with \LaTeX{}.}
% The \xfile{.dtx} chooses its action depending on the format:
% \begin{description}
% \item[\plainTeX:] Run \docstrip{} and extract the files.
% \item[\LaTeX:] Generate the documentation.
% \end{description}
% If you insist on using \LaTeX{} for \docstrip{} (really,
% \docstrip{} does not need \LaTeX ), then inform the autodetect routine
% about your intention:
% \begin{quote}
%   \verb|latex \let\install=y% \iffalse meta-comment
%
% File: footnoterange.dtx
% Version: 2019/11/24 v1.0c
%
% Copyright (C) 2012 - 2019 by
%    H.-Martin M"unch <Martin dot Muench at Uni-Bonn dot de>
% Portions of code copyrighted by other people as marked;
% lines marked "FMi": code by Frank Mittelbach, thanks!
%
% This work may be distributed and/or modified under the
% conditions of the LaTeX Project Public License, either
% version 1.3c of this license or (at your option) any later
% version. This version of this license is in
%    https://www.latex-project.org/lppl/lppl-1-3c.txt
% and the latest version of this license is in
%    https://www.latex-project.org/lppl.txt
% and version 1.3c or later is part of all distributions of
% LaTeX version 2005/12/01 or later.
%
% This work has the LPPL maintenance status "maintained".
%
% The Current Maintainer of this work is H.-Martin Muench.
%
% This work consists of the main source file footnoterange.dtx,
% the README, and the derived files
%    footnoterange.sty, footnoterange.pdf,
%    footnoterange.ins, footnoterange.drv,
%    footnoterange-example.tex, footnoterange-example.pdf.
%
% 'footnoterange' is available on CTAN:
% https://www.ctan.org/pkg/footnoterange
%
% Also a TDS.ZIP file is provided that contains all the files
% already sorted in a TDS tree:
% http://mirror.ctan.org/install/macros/latex/contrib/footnoterange.tds.zip
%
% Unpacking:
%    (a) If footnoterange.ins is present:
%           tex footnoterange.ins
%    (b) Without footnoterange.ins:
%           tex footnoterange.dtx
%    (c) If you insist on using LaTeX
%           latex \let\install=y\input{footnoterange.dtx}
%        (quote the arguments according to the demands of your shell)
%
% Documentation:
%    (a) If footnoterange.drv is present:
%           (pdf)latex footnoterange.drv
%           makeindex -s gind.ist footnoterange.idx
%           (pdf)latex footnoterange.drv
%           makeindex -s gind.ist footnoterange.idx
%           (pdf)latex footnoterange.drv
%    (b) Without footnoterange.drv:
%           (pdf)latex footnoterange.dtx
%           makeindex -s gind.ist footnoterange.idx
%           (pdf)latex footnoterange.dtx
%           makeindex -s gind.ist footnoterange.idx
%           (pdf)latex footnoterange.dtx
%
%    The class ltxdoc loads the configuration file ltxdoc.cfg
%    if available. Here you can specify further options, e.g.
%    use DIN A4 as paper format:
%       \PassOptionsToClass{a4paper}{article}
%
% Installation:
%    TDS:tex/latex/footnoterange/footnoterange.sty
%    TDS:doc/latex/footnoterange/footnoterange.pdf
%    TDS:doc/latex/footnoterange/footnoterange-example.tex
%    TDS:doc/latex/footnoterange/footnoterange-example.pdf
%    TDS:source/latex/footnoterange/footnoterange.dtx
%
%<*ignore>
\begingroup
  \catcode123=1 %
  \catcode125=2 %
  \def\x{LaTeX2e}%
\expandafter\endgroup
\ifcase 0\ifx\install y1\fi\expandafter
         \ifx\csname processbatchFile\endcsname\relax\else1\fi
         \ifx\fmtname\x\else 1\fi\relax
\else\csname fi\endcsname
%</ignore>
%<*install>
\input docstrip.tex
\Msg{*********************************************************************************}
\Msg{* Installation}
\Msg{* Package: footnoterange 2019/11/24 v1.0c References to ranges of footnotes (HMM)}
\Msg{*********************************************************************************}

\keepsilent
\askforoverwritefalse

\let\MetaPrefix\relax
\preamble

This is a generated file.

Project: footnoterange
Version: 2019/11/24 v1.0c

Copyright (C) 2012 - 2019 by
    H.-Martin M"unch <Martin dot Muench at Uni-Bonn dot de>
Portions of code copyrighted by other people as marked;
lines marked "FMi": code by Frank Mittelbach, thanks!

The usual disclaimer applies:
If it doesn't work right that's your problem.
(Nevertheless, please send an e-mail to the maintainer
 when you find an error in this package.)

This work may be distributed and/or modified under the
conditions of the LaTeX Project Public License, either
version 1.3c of this license or (at your option) any later
version. This version of this license is in
   https://www.latex-project.org/lppl/lppl-1-3c.txt
and the latest version of this license is in
   https://www.latex-project.org/lppl.txt
and version 1.3c or later is part of all distributions of
LaTeX version 2005/12/01 or later.

This work has the LPPL maintenance status "maintained".

The Current Maintainer of this work is H.-Martin Muench.

This work consists of the main source file footnoterange.dtx,
the README, and the derived files
   footnoterange.sty, footnoterange.pdf,
   footnoterange.ins, footnoterange.drv,
   footnoterange-example.tex, footnoterange-example.pdf.

In memoriam
 Claudia Simone Barth + 1996/01/30
 Tommy Muench + 2014/01/02
 Hans-Klaus Muench + 2014/08/24

\endpreamble
\let\MetaPrefix\DoubleperCent

\generate{%
  \file{footnoterange.ins}{\from{footnoterange.dtx}{install}}%
  \file{footnoterange.drv}{\from{footnoterange.dtx}{driver}}%
  \usedir{tex/latex/footnoterange}%
  \file{footnoterange.sty}{\from{footnoterange.dtx}{package}}%
  \usedir{doc/latex/footnoterange}%
  \file{footnoterange-example.tex}{\from{footnoterange.dtx}{example}}%
}

\catcode32=13\relax% active space
\let =\space%
\Msg{************************************************************************}
\Msg{*}
\Msg{* To finish the installation you have to move the following}
\Msg{* file into a directory searched by TeX:}
\Msg{*}
\Msg{* footnoterange.sty}
\Msg{*}
\Msg{* To produce the documentation run the file `footnoterange.drv'}
\Msg{* through (pdf)LaTeX, e.g.}
\Msg{*  pdflatex footnoterange.drv}
\Msg{*  makeindex -s gind.ist footnoterange.idx}
\Msg{*  pdflatex footnoterange.drv}
\Msg{*  makeindex -s gind.ist footnoterange.idx}
\Msg{*  pdflatex footnoterange.drv}
\Msg{*}
\Msg{* At least three runs are necessary e.g. to get the}
\Msg{*  references right!}
\Msg{*}
\Msg{* Happy TeXing!}
\Msg{*}
\Msg{************************************************************************}

\endbatchfile
%</install>
%<*ignore>
\fi
%</ignore>
%
% \section{The documentation driver file}
%
% The next bit of code contains the documentation driver file for
% \TeX , i.\,e., the file that will produce the documentation you
% are currently reading. It will be extracted from this file by the
% \verb|docstrip| programme. That is, run \LaTeX{} on \verb|docstrip|
% and specify the \verb|driver| option when \verb|docstrip|
% asks for options.
%
%    \begin{macrocode}
%<*driver>
\NeedsTeXFormat{LaTeX2e}[2015/01/01]
\ProvidesFile{footnoterange.drv}[2019/11/24 v1.0c
            References to ranges of footnotes (HMM)]
\documentclass{ltxdoc}[2015/03/26]% v2.0w
\usepackage{holtxdoc}[2012/03/21]%  v0.24
%% footnoterange may work with earlier versions of LaTeX2e and those
%% class and package, but this was not tested.
%% Please consider updating your LaTeX, class, and package
%% to the most recent version (if they are not already the most
%% recent version).
\hypersetup{%
 pdfsubject={References to ranges of footnotes (HMM)},%
 pdfkeywords={LaTeX, footnoterange, footnotes, footnote, reference, H.-Martin Muench},%
 pdfencoding=auto,%
 pdflang={en},%
 breaklinks=true,%
 linktoc=all,%
 pdfstartview=FitH,%
 pdfpagelayout=OneColumn,%
 bookmarksnumbered=true,%
 bookmarksopen=true,%
 bookmarksopenlevel=3,%
 pdfmenubar=true,%
 pdftoolbar=true,%
 pdfwindowui=true,%
 pdfnewwindow=true%
}
\CodelineIndex
\hyphenation{printing docu-ment}
\gdef\unit#1{\mathord{\thinspace\mathrm{#1}}}
\begin{document}
  \DocInput{footnoterange.dtx}%
\end{document}
%</driver>
%    \end{macrocode}
%
% \fi
%
% \CheckSum{321}
%
% \CharacterTable
%  {Upper-case    \A\B\C\D\E\F\G\H\I\J\K\L\M\N\O\P\Q\R\S\T\U\V\W\X\Y\Z
%   Lower-case    \a\b\c\d\e\f\g\h\i\j\k\l\m\n\o\p\q\r\s\t\u\v\w\x\y\z
%   Digits        \0\1\2\3\4\5\6\7\8\9
%   Exclamation   \!     Double quote  \"     Hash (number) \#
%   Dollar        \$     Percent       \%     Ampersand     \&
%   Acute accent  \'     Left paren    \(     Right paren   \)
%   Asterisk      \*     Plus          \+     Comma         \,
%   Minus         \-     Point         \.     Solidus       \/
%   Colon         \:     Semicolon     \;     Less than     \<
%   Equals        \=     Greater than  \>     Question mark \?
%   Commercial at \@     Left bracket  \[     Backslash     \\
%   Right bracket \]     Circumflex    \^     Underscore    \_
%   Grave accent  \`     Left brace    \{     Vertical bar  \|
%   Right brace   \}     Tilde         \~}
%
% \GetFileInfo{footnoterange.drv}
%
% \begingroup
%   \def\x{\#,\$,\^,\_,\~,\ ,\&,\{,\},\%}%
%   \makeatletter
%   \@onelevel@sanitize\x
% \expandafter\endgroup
% \expandafter\DoNotIndex\expandafter{\x}
% \expandafter\DoNotIndex\expandafter{\string\ }
% \begingroup
%   \makeatletter
%     \lccode`9=32\relax
%     \lowercase{%^^A
%       \edef\x{\noexpand\DoNotIndex{\@backslashchar9}}%^^A
%     }%^^A
%   \expandafter\endgroup\x
% \DoNotIndex{\,,\\}
% \DoNotIndex{\documentclass,\usepackage,\ProvidesPackage,\begin,\end}
% \DoNotIndex{\MessageBreak}
% \DoNotIndex{\NeedsTeXFormat,\DoNotIndex,\verb}
% \DoNotIndex{\def,\edef,\gdef,\global,\xdef}
% \DoNotIndex{\ifx,\listfiles,\mathord,\mathrm}
% \DoNotIndex{\bigskip,\space,\thinspace,\ldots}
% \DoNotIndex{\indent,\noindent,\newline,\linebreak,\pagebreak,\newpage}
% \DoNotIndex{\textbf,\textit,\textsf,\textsc,\texttt,\textquotedblleft,\textquotedblright}
% \DoNotIndex{\plainTeX,\TeX,\LaTeX,\pdfLaTeX}
% \DoNotIndex{\section,\pagenumbering,\smallskip}
% \DoNotIndex{\@ifpackageloaded}
% \DoNotIndex{\arabic, \Alph, \alph, \alphalph, \Roman, \roman, \fnsymbol}
%
% \title{The \xpackage{footnoterange} package}
% \date{2019/11/24 v1.0c}
% \author{H.-Martin M\"{u}nch\\\xemail{Martin.Muench at Uni-Bonn.de}}
%
% \maketitle
%
% \begin{abstract}
% \noindent This package provides the environments |footnoterange| and
% |footnoterange*|. Multiple footnotes inside these environments
% are not referenced as e.\,g.
% \makeatletter%
% \hbox{\textquotedblleft \@textsuperscript {\normalfont 1 2 3}\textquotedblright}
% but as \hbox{\textquotedblleft \@textsuperscript {\normalfont 1--3}\textquotedblright ,\ }
% \makeatother%
% i.\,e. the range (from first to last referred footnote at that place) is given.
% Depending on loading of \xpackage{hyperref} and use of its |hyperfootnotes|-option
% the references are hyperlinked. (References to footnotes in |footnoterange*|
% are never hyperlinked.)
% \end{abstract}
%
% \bigskip
%
% \noindent Disclaimer for web links: The author is not responsible for any contents
% referred to in this work unless if having full knowledge of illegal contents.
% If any damage occurs by the use of information presented there, only the
% author of the respective pages might be liable, not the one who has referred
% to these pages.
%
% \noindent Save per page about $200\unit{ml}$ water,
% $2\unit{g}$ CO$_{2}$ and $2\unit{g}$ wood:
% Therefore please print only if this is really necessary.
%
% \medskip
%
% \tableofcontents
%
%\newpage
%
% \section{Introduction}
% \indent This \LaTeXe{} package provides the environments |footnoterange| and\newline
% |footnoterange*|. Multiple footnotes inside these environments are not referenced
% as e.\,g.
% \makeatletter%
% \hbox{\textquotedblleft \@textsuperscript {\normalfont 1 2 3}\textquotedblright}
% but as \hbox{\textquotedblleft \@textsuperscript {\normalfont 1--3}\textquotedblright ,\ }
% \makeatother%
% i.\,e. the range (from first to last
% referred footnote at that place) is given. Depending on loading of
% \xpackage{hyperref} and use of its |hyperfootnotes|-option the references are
% hyperlinked. References to footnotes in |footnoterange*| are never hyperlinked.\newline
% |\footnotemark\footnotetext| instead of |\footnote| is not supported inside
% |footnoterange| and |footnoterange*| environments.
%
% \bigskip
%
% \section{Usage}
%
% \indent Just load the package placing
% \begin{quote}
%   |\usepackage{footnoterange}|
% \end{quote}
% \noindent in the preamble of your \LaTeXe{} source file. There are no options to
% be used. The package should be loaded after \xpackage{hyperref} (if that package is used)
% and after any other package redefining footnotes. Usage of the |footnoterange| (or
% |footnoterange*|) environment is simple: Just encapsulate the footnotes, which shall
% be referred as range instead of referencing as single ones (see the example file).
%
% About hyperlinks:\newline
% Footnote-references in a |footnoterange*| environment are never hyperlinked.\newline
% Footnote-references in a |footnoterange| environment may be hyperlinked:
% \begin{itemize}
%  \item When the \xpackage{hyperref} package is not loaded, they are not
%         hyperlinked (naturally).
%  \item When the \xpackage{hyperref} package is loaded, it depends on the
%         |hyperfootnotes| option of the \xpackage{hyperref} package:
%  \begin{itemize}
%   \item With |hyperfootnotes=true|, the references are hyperlinked.
%   \item With |hyperfootnotes=false|, the references are not hyperlinked.
%  \end{itemize}
% \end{itemize}
% |\footnote|s inside of the |footnoterange|/{}|footnoterange*| environment
% can use the optional parameter, i.\,e. |\footnote[42]{text}| even if this is the
% third footnote. When giving ranges as reference, this does not seem to be
% such a good idea, if this is not done consistently (i.\,e. footnotes
% |[42], [43], [44]| resulting in 42--44 is OK,
% but |[42]| (original~2), unchanged~3, unchanged~4 resulting in
% 42--4 is confusing). A warning about these changed footnotemarks is written
% into the \xfile{.log}-file.
%
% When there is only a single footnote in a |footnoterange| environment,
% this does not produce some reference like %
% \makeatletter%
% \hbox{\textquotedblleft \@textsuperscript {\normalfont 7--7}\textquotedblright{},\ }%
% \makeatother%
% but only one footnotemark is used automatically. Because it is strange to use
% a |footnoterange| environment without a range of footnotes, an info is written
% into the \xfile{.log}-file about this.
%
% The usage of other footnote-number-representations
% (e.\,g.~\verb|\Roman|, \verb|\roman|, \verb|\Alph|, \verb|\alph|,
% \verb|\fnsymbol|) is also possible, but two things must be taken
% into account: The number of footnote references is restricted (for
% example with \verb|\Alph| only references A to Z are possible), which
% can be fixed e.\,g. with the \url{https://ctan.org/pkg/alphalph} package
% (but not automatically), and references to footnote-symbol-ranges (\verb|\fnsymbol|)
% are probably not very clear.
%
% \subsection*{Options}
% \indent The \xpackage{footnoterange} package takes no options.
%
% \section{Alternatives\label{sec:Alternatives}}
%
% The \xpackage{fnpct} (\url{https://www.ctan.org/pkg/fnpct}) package allows to set
% multiple footnotes (|\multfootnote|), suppress a footnotemark \hbox{(|*|),}
% and customize the multi-footnote-separator \hbox{(|\setfnpct{mult-fn-sep={|
% \emph{something} |}}|)}.
% Combining these with stepping of the footnotecounter gives a footnoterange:
% 
% \begin{verbatim}
% \documentclass{article}
% \usepackage{hyperref}
% \usepackage{fnpct}
% \begin{document}
% The three little pigs built their houses
% out of straw\footnote{not to be confused with hay},
% sticks\footnote{or lumber according to some sources}
% and bricks%
% \makeatletter%
% \setfnpct{mult-fn-sep={--}}%
% \multfootnote{probably fired clay bricks\addtocounter\@mpfn{+1};%
% *or something else;%
% what do I know}. %
% \makeatother%
% \end{document}
% \end{verbatim}
% 
% \noindent If you programmed or found another alternative, which is available at
% \url{https://www.CTAN.org/}, please send an e-mail to me with the name,
% location at CTAN, and a short notice, and I will probably list it here.
%
% \section{Example}
%
%    \begin{macrocode}
%<*example>
\documentclass[british]{article}[2014/09/29]% v1.4h
%%%%%%%%%%%%%%%%%%%%%%%%%%%%%%%%%%%%%%%%%%%%%%%%%%%%%%%%%%%%%%%%%%%%%
\usepackage[%
 extension=pdf,%
 plainpages=false,%
 pdfpagelabels=true,%
 hyperindex=false,%
 pdflang={en},%
 pdftitle={footnoterange package example},%
 pdfauthor={H.-Martin Muench},%
 pdfsubject={Example for the footnoterange package},%
 pdfkeywords={LaTeX, footnoterange, H.-Martin Muench},%
 pdfview=FitH,%
 pdfstartview=FitH,%
 pdfpagelayout=OneColumn,%
 hyperfootnotes=true%
]{hyperref}[2012/11/06]% v6.83m

\usepackage{footnoterange}[2019/11/24]% v1.0c

\gdef\unit#1{\mathord{\thinspace\mathrm{#1}}}%
\listfiles
\begin{document}
\pagenumbering{arabic}
\section*{Example for footnoterange}

This example demonstrates the use of package\newline
\textsf{footnoterange}, v1.0c as of 2019/11/24 (HMM).\newline
The package does not use options.\newline
\verb|footnoterange-example.pdf| can be created by\newline
\verb|pdflatex footnoterange-example.tex|,\newline
and \verb|footnoterange-example.tex| can be created by\newline
\verb|tex footnoterange.dtx|.\newline
\verb|footnoterange.dtx| is at
\url{https://www.ctan.org/pkg/footnoterange}.\newline

\noindent For more details please see the documentation!\newline

\noindent Save per page about $200\unit{ml}$ water,
$2\unit{g}$ CO$_{2}$ and $2\unit{g}$ wood:\newline
Therefore please print only if this is really necessary.\newline

This\footnote{Lorem} text bears a multiplicity of footnotes.
Because the \verb|hyperref| package is used with option
\verb|hyperfootnotes=true|, the footnotes
\begin{footnoterange}
\footnote{ipsum}%
\footnote{dolor}%
\footnote{sit}%
\footnote{amet,}%
\footnote{consectetur}%
\footnote{adipisicing}%
\footnote{elit,}%
\footnote{sed}%
\end{footnoterange}
are hyperlinked.\footnote{do eiusmod\ldots} Using
\verb|hyperfootnotes=false| or not using \verb|hyperref|
will remove the hyperlinks to the footnotes.\medskip

This text also bears a multiplicity of footnotes,
but due to the use of the starred
\begin{footnoterange*}
\footnote{Ut purus elit,}%
\footnote{vestibulum ut,}%
\footnote{placerat ac,}%
\footnote{adipiscing vitae,}%
\footnote{felis.}%
\footnote{Curabitur}%
\footnote{dictum}%
\footnote{gravida}%
\end{footnoterange*}
form of the environment they are not
hyperlinked.\footnote{mauris\ldots}\medskip

The use of the optional parameter for footnotes
\begin{footnoterange}
\footnote[101]{Nam arcu libero,}%
\footnote[102]{nonummy eget,}%
\footnote[103]{consectetuer id,}%
\footnote[104]{vulputate a,}%
\footnote[105]{magna.}
\end{footnoterange}
is also possible, but should be used consistently.

\pagebreak

When there is only one footnote
\begin{footnoterange}
\footnote{a single footnote}
\end{footnoterange}
in the \verb|footnoterange| environment, only one
footnotemark is used and an info is written into the
\verb|log|-file.\medskip

The usage of other footnote-number-representations
(e.\,g.~\verb|\Roman|, \verb|\roman|, \verb|\Alph|, \verb|\alph|,
\verb|\fnsymbol|) is also possible, but two things must be taken
into account: The number of footnote references is restricted
(for example with \verb|\Alph| only references A to Z are possible),
which can be fixed e.\,g. with the
\href{https://www.ctan.org/pkg/alphalph}{\texttt{alphalph} package},
and references to footnote-symbol-ranges (\verb|\fnsymbol|)
are probably not very clear.

\end{document}
%</example>
%    \end{macrocode}
%
% \StopEventually{}
%
% \section{The implementation}
%
% We start off by checking that we are loading into \LaTeXe{} and
% announcing the name and version of this package.
%
%    \begin{macrocode}
%<*package>
%    \end{macrocode}
%
%    \begin{macrocode}
\NeedsTeXFormat{LaTeX2e}[2011/06/27]
\ProvidesPackage{footnoterange}[2019/11/24 v1.0c
            References to ranges of footnotes (HMM)]
%    \end{macrocode}
%
% A short description of the \xpackage{footnoterange} package:
%
%    \begin{macrocode}
%% Provides the footnoterange and footnoterange* environments to
%% refer to ranges of footnotes.
%    \end{macrocode}
%
% We need the \xpackage{ltxcmds} package:
%
%    \begin{macrocode}
\RequirePackage{ltxcmds}[2011/11/09]% v1.22; for "\ltx@ifpackageloaded"
\RequirePackage{letltxmacro}[2010/09/02]% v1.4; for "\GlobalLetLtxMacro"

%    \end{macrocode}
%
% We use |\ifHy@hyperfootnotes ... \else ... \fi|, but when the \xpackage{hyperref}
% package has not been loaded, |\ifHy@hyperfootnotes| is not defined, and \LaTeX
% relates the |\else| and/or |\fi| (wrongly) to other |\if|s or groups.
% Therefore the |\ifHy@hyperfootnotes| are placed in their own commands,
% which are not looked at when the \xpackage{hyperref} package has not been loaded.
%
% \pagebreak
%
% \begin{macro}{\fnr@ifhyperrfn}
%    \begin{macrocode}
\newcommand{\fnr@ifhyperrfn}{%
  \ifHy@hyperfootnotes% option hyperfootnotes=true
    \addtocounter{Hfootnote}{+1}%
    % Code from the hyperref package
      \global\let\Hy@saved@currentHref\@currentHref%
      \hyper@makecurrent{Hfootnote}%
      \global\let\Hy@footnote@currentHref\@currentHref%
      \global\let\@currentHref\Hy@saved@currentHref%
    % End of code from the hyperref package
  \fi%
}

%    \end{macrocode}
% \end{macro}
%
% \begin{macro}{\fnr@rangefootnote}
% The |\fnr@rangefootnote| is the internal replacement for the original\newline
% |\footnote| command inside of a |footnoterange| environment.
%
%    \begin{macrocode}
\newcommand{\fnr@rangefootnote}[2][\empty]{%
%    \end{macrocode}
%
% |#1| contains the optional argument, which is predefined as |\empty|.
%
%    \begin{macrocode}
  \nolinebreak%
  % no \footnotemark here
  \addtocounter\@mpfn{+1}%
  \ltx@ifpackageloaded{hyperref}{\fnr@ifhyperrfn}{\relax}%
%    \end{macrocode}
%
% See the explanation before the declaration of the |\fnr@ifhyperrfn| command!
%
%    \begin{macrocode}
  \xdef\fnr@opt{#1}% contains the optional argument
  \xdef\fnr@arabic{\arabic\@mpfn}%
  \protected@edef\fnr@formated{\thefootnote}%
%    \end{macrocode}
%
% The |\footnotemark| is not set (yet), but the |\footnotetext|,
% and it gets a label.
%
%    \begin{macrocode}
  \ifx\fnr@opt\empty%
    \footnotetext{\label{fnr:\fnr@arabic}#2}%
  \else%
    \PackageWarning{footnoterange}{%
      Custom footnote-numbers used in a footnoterange\MessageBreak%
      environment: Footnote with natural number\MessageBreak%
      \ifx\fnr@formated\fnr@arabic%
        \fnr@arabic\space%
      \else%
        \space\fnr@arabic\space (\fnr@formated )%
      \fi%
      \MessageBreak%
      was renumbered to \fnr@opt\MessageBreak%
     }%
    \ltx@ifpackageloaded{hyperref}{% hyperref loaded
      \footnotetext[#1]{\phantomsection\label{fnr:\fnr@arabic}#2}%
     }{% hyperref not loaded
      \footnotetext[#1]{\label{fnr:\fnr@arabic}#2}%
     }%
  \fi%
}

%    \end{macrocode}
% \end{macro}
%
%    \begin{macrocode}

%    \end{macrocode}
% \pagebreak
%
% |\ifHy@hyperfootnotes| \textquotedblleft encapsulation\textquotedblright{} again:
%
% \begin{macro}{\fnr@ifhyperfnrA}
%    \begin{macrocode}
\newcommand{\fnr@ifhyperfnrA}{%
  \ifHy@hyperfootnotes% option hyperfootnotes=true
    \hbox{\hyperref[fnr:\fnr@first]{%
          \@textsuperscript{\normalfont \ref*{fnr:\fnr@first}}}}%
  \else% option hyperfootnotes=false
    \hbox{\@textsuperscript{\normalfont \ref*{fnr:\fnr@first}}}%
  \fi%
}

%    \end{macrocode}
% \end{macro}
%
% \begin{macro}{\fnr@ifhyperfnrN}
%    \begin{macrocode}
\newcommand{\fnr@ifhyperfnrN}{%
  \ifHy@hyperfootnotes% option hyperfootnotes=true
    \hbox{\hyperref[fnr:\fnr@first]{%
            \@textsuperscript{\normalfont \ref*{fnr:\fnr@first}}}%
          \@textsuperscript{\normalfont --}%
          \hyperref[fnr:\arabic\@mpfn]{%
            \@textsuperscript{\normalfont \ref*{fnr:\arabic\@mpfn}}}}%
  \else% option hyperfootnotes=false
    \hbox{\@textsuperscript{\normalfont \ref*{fnr:\fnr@first}%
                                        --\ref*{fnr:\arabic\@mpfn}}}%
  \fi%
}

%    \end{macrocode}
% \end{macro}
%
% \begin{environment}{footnoterange}
% The |footnoterange| environment remembers the next footnote number as\newline
% |\fnr@first| and the current |\footnote| definition as |\fnr@origfn|
% and uses |\fnr@rangefootnote| instead.
%
%    \begin{macrocode}
\newenvironment{footnoterange}{%
  % at begin of the footnoterange environment
  \ifhmode \unskip \fi% FMi
  \addtocounter\@mpfn{+1}%
  \xdef\fnr@first{\arabic\@mpfn}%
  \addtocounter\@mpfn{-1}%
  \let\fnr@origfn\footnote%
  \let\footnote\fnr@rangefootnote%
  \GlobalLetLtxMacro{\fnr@origfnmark}{\footnotemark}%
  \GlobalLetLtxMacro{\footnotemark}{\fnr@rangefootnotemark}%
  \ignorespaces% FMi
%    \end{macrocode}
%
% End of the environment:
%
%    \begin{macrocode}
}{% at end of the footnoterange environment
  \ifhmode% FMi 
    \unskip% FMi
%    \end{macrocode}
%
% Remember the spacefactor for restoring after the environment has ended
% (only in |hmode|!). \% FMi
%
%    \begin{macrocode}
    \xdef\fnr@aftergroup{\spacefactor\the\spacefactor\relax}% FMi
  \else% FMi
%    \end{macrocode}
%
% Otherwise do something harmless. \% FMi
%
%    \begin{macrocode}
    \xdef\fnr@aftergroup{\relax}%
  \fi% FMi
%    \end{macrocode}
%
% At the end of the environment the previous definition of |\footnote| is restored
%
%    \begin{macrocode}
  \let\footnote\fnr@origfn%
  \GlobalLetLtxMacro{\footnotemark}{\fnr@origfnmark}%
%    \end{macrocode}
%
% and the references are placed:
%
%    \begin{macrocode}
  \ifx\fnr@first\fnr@arabic% There is only one reference in the environment.
    \PackageInfo{footnoterange}{%
      Environment footnoterange used,\MessageBreak%
      but only one footnote (number \fnr@first ) is contained%
      \MessageBreak%
     }%
    \ltx@ifpackageloaded{hyperref}{\fnr@ifhyperfnrA% hyperref package loaded
    }{% hyperref package not loaded
      \hbox{\@textsuperscript{\normalfont \ref{fnr:\fnr@first}}}%
     }%
  \else%
    \ltx@ifpackageloaded{hyperref}{\fnr@ifhyperfnrN% hyperref package loaded
    }{% hyperref package not loaded
      \hbox{\@textsuperscript{\normalfont \ref{fnr:\fnr@first}%
                                        --\ref{fnr:\arabic\@mpfn}}}%
     }%
  \fi%
%    \end{macrocode}
%
% Restore the spacefactor after the environment has ended. \% FMi
%
%    \begin{macrocode}
  \aftergroup\fnr@aftergroup% FMi
}

%    \end{macrocode}
% \end{environment}
%
% \begin{environment}{footnoterange*}
% The |footnoterange*| environment does the same as the |footnoterange|
% environment but suppresses all hyperlinks to footnotes inside.
%
%    \begin{macrocode}
\newenvironment{footnoterange*}{%
  % at begin of the footnoterange* environment
  \ifhmode \unskip \fi% FMi
  \addtocounter\@mpfn{+1}%
  \xdef\fnr@first{\arabic\@mpfn}%
  \addtocounter\@mpfn{-1}%
  \let\fnr@origfn\footnote%
  \let\footnote\fnr@rangefootnote%
  \GlobalLetLtxMacro{\fnr@origfnmark}{\footnotemark}%
  \GlobalLetLtxMacro{\footnotemark}{\fnr@rangefootnotemark}%
  \ignorespaces% FMi
}{% at end of the footnoterange* environment
  \ifhmode% FMi 
    \unskip% FMi
    \xdef\fnr@aftergroup{\spacefactor\the\spacefactor\relax}% FMi
  \else% FMi
    \xdef\fnr@aftergroup{\relax}%
  \fi% FMi
  \let\footnote\fnr@origfn%
  \GlobalLetLtxMacro{\footnotemark}{\fnr@origfnmark}%
  \ifx\fnr@first\fnr@arabic% There is only one reference in the environment.
    \PackageInfo{footnoterange}{%
      Environment footnoterange used,\MessageBreak%
      but only one footnote (number \fnr@first ) is contained%
      \MessageBreak%
     }%
%    \end{macrocode}
% \pagebreak
%    \begin{macrocode}
    \ltx@ifpackageloaded{hyperref}{% hyperref package loaded
      \hbox{\@textsuperscript{\normalfont \ref*{fnr:\fnr@first}}}%
    }{% hyperref package not loaded
      \hbox{\@textsuperscript{\normalfont \ref{fnr:\fnr@first}}}%
     }%
  \else%
    \ltx@ifpackageloaded{hyperref}{% hyperref package loaded
      \hbox{\@textsuperscript{\normalfont \ref*{fnr:\fnr@first}%
                                        --\ref*{fnr:\arabic\@mpfn}}}%
    }{% hyperref package not loaded
      \hbox{\@textsuperscript{\normalfont \ref{fnr:\fnr@first}%
                                        --\ref{fnr:\arabic\@mpfn}}}%
     }%
  \fi%
  \aftergroup\fnr@aftergroup% FMi
}

%    \end{macrocode}
% \end{environment}
%
% The \xpackage{footnoterange} package does not handle |\footnotemark|s,
% but at least the user is made aware of this:
%
% \begin{macro}{\fnr@rangefootnotemark}
%    \begin{macrocode}

\newcommand{\fnr@rangefootnotemark}[1][\empty]{%
  \PackageError{footnoterange}{%
     \string\footnotemark\space not supported in footnoterange(*) environment%
   }{The footnoterange package does not handle a \string\footnotemark.\MessageBreak%
     Either use \string\footnote\string{...\string}\space instead of%
     \string\footnotemark\space and \string\footnotetext\string{...\string}\MessageBreak%
     in footnoterange and footnoterange* environments,\MessageBreak%
     or do not use those environments around a \string\footnotemark\MessageBreak%
     (or \string\renewcommand\string{\string\fnr@rangefootnotemark\string}\MessageBreak%
     with your own fixing code).%
    }%
  \xdef\fnr@mopt{#1}%
  \ifx\fnr@mopt\empty%
    \fnr@origfnmark%
  \else%
    \fnr@origfnmark[#1]%
  \fi%
}

%    \end{macrocode}
% \end{macro}
%
%    \begin{macrocode}
%</package>
%    \end{macrocode}
%
% \pagebreak
%
% \section{Installation}
%
% \subsection{Downloads\label{ss:Downloads}}
%
% Everything is available on CTAN,
% \url{https://www.ctan.org/tex-archive/},
% but may need additional packages themselves.
%
% \DescribeMacro{footnoterange.dtx}
% For unpacking the \xfile{footnoterange.dtx} file
% and constructing the documentation it is required:
% \begin{description}
% \item[-] \TeX{} Format \LaTeXe{}, \url{https://www.CTAN.org/}
%
% \item[-] document class \xpackage{ltxdoc}, 2015/03/26, v2.0w,
%           \url{https://ctan.org/pkg/ltxdoc}
%
% \item[-] package \xpackage{holtxdoc}, 2012/03/21, v0.24,
%           \url{https://ctan.org/pkg/holtxdoc}
% \end{description}
%
% \DescribeMacro{footnoterange.sty}
% The |footnoterange.sty| for \LaTeXe{} (i.\,e. each document using
% the \xpackage{footnoterange} package) requires:
% \begin{description}
% \item[-] \TeX{} Format \LaTeXe{}, \url{https://www.CTAN.org/}
%
% \item[-] package \xpackage{ltxcmds}, 2011/04/18, v1.20,
%           \url{https://www.ctan.org/pkg/ltxcmds}
% \end{description}
%
% \DescribeMacro{footnoterange-example.tex}
% The |footnoterange-example.tex| requires the same files as all
% documents using the \xpackage{footnoterange} package and additionally:
% \begin{description}
% \item[-] class \xpackage{article}, 2007/10/19, v1.4h, from \xpackage{classes.dtx}:\newline
%           \url{https://www.ctan.org/tex-archive/macros/latex/base/}
%
% \item[-] package \xpackage{footnoterange}, 2019/11/24, v1.0c,\newline
%           \url{https://www.ctan.org/pkg/footnoterange}\newline
%           (Well, it is the example file for this package, and because you are
%            reading the documentation for the \xpackage{footnoterange} package,
%            it can be assumed that you already have some version of it --
%            is it the current one?)
%
% \item[-] package \xpackage{hyperref}, 2012/11/06, v6.83m,\newline
%           \url{https://www.ctan.org/pkg/hyperref}
% \end{description}
%
% \DescribeMacro{Oberdiek}
% \DescribeMacro{holtxdoc}
% \DescribeMacro{ltxcmds}
% \DescribeMacro{alphalph}
% All packages of \textsc{Heiko Oberdiek}'s bundle `oberdiek'
% (especially \xpackage{holtxdoc}, \xpackage{ltxcmds}, and \xpackage{alphalph})
% are also available in a TDS compliant ZIP archive:\newline
% \url{http://mirror.ctan.org/install/macros/latex/contrib/oberdiek.tds.zip}.\newline
% It is probably best to download and use this, because the packages in there
% should be both recent and compatible.
%
% \DescribeMacro{hyperref}
% \noindent \xpackage{hyperref} is not included in that bundle and needs to be
% downloaded separately,\newline
% \url{http://mirrors.ctan.org/install/macros/latex/contrib/hyperref.tds.zip}.
%
% \DescribeMacro{M\"{u}nch}
% A list of my packages can be found at \url{https://www.ctan.org/author/muench-hm}.
%
% \pagebreak
%
% \subsection{Package, unpacking TDS}
% \paragraph{Package.} This package is available on \url{https://www.CTAN.org}.
% \begin{description}
% \item[\url{http://mirror.ctan.org/macros/latex/contrib/footnoterange/footnoterange.dtx}]\hspace*{0.1cm}
%       The source file.
% \item[\url{http://mirror.ctan.org/macros/latex/contrib/footnoterange/footnoterange.pdf}]\hspace*{0.1cm}
%       The documentation.
% \item[\url{http://mirror.ctan.org/macros/latex/contrib/footnoterange/footnoterange-example.pdf}]\hspace*{0.1cm}
%       The compiled example file, as it should look like.
% \item[\url{http://mirror.ctan.org/macros/latex/contrib/footnoterange/README}]\hspace*{0.1cm}\newline\hspace*{0.1cm}
%       The README file.
% \end{description}
% There is also a footnoterange.tds.zip available:
% \begin{description}
% \item[\url{http://mirror.ctan.org/install/macros/latex/contrib/footnoterange.tds.zip}]\hspace*{0.1cm}
%       Everything in TDS compliant, compiled format.
% \end{description}
% which additionally contains\newline
% \begin{tabular}{ll}
% footnoterange.ins & The installation file.\\
% footnoterange.drv & The driver to generate the documentation.\\
% footnoterange.sty & The \xext{sty}le file.
% \end{tabular}
%
% \bigskip
%
% \noindent For required other packages, please see the preceding subsection.
%
% \paragraph{Unpacking.} The \xfile{.dtx} file is a self-extracting
% \docstrip{} archive. The files are extracted by running the
% \xfile{.dtx} through \plainTeX:
% \begin{quote}
%   \verb|tex footnoterange.dtx|
% \end{quote}
%
% About generating the documentation see paragraph~\ref{GenDoc} below.
%
% \paragraph{TDS.} Now the different files must be moved into
% the different directories in your installation TDS tree
% (also known as \xfile{texmf} tree):
% \begin{quote}
% \def\t{^^A
% \begin{tabular}{@{}>{\ttfamily}l@{ $\rightarrow$ }>{\ttfamily}l@{}}
%   footnoterange.sty & tex/latex/footnoterange/footnoterange.sty\\
%   footnoterange.pdf & doc/latex/footnoterange/footnoterange.pdf\\
%   footnoterange-example.tex & doc/latex/footnoterange/footnoterange-example.tex\\
%   footnoterange-example.pdf & doc/latex/footnoterange/footnoterange-example.pdf\\
%   footnoterange.dtx & source/latex/footnoterange/footnoterange.dtx\\
% \end{tabular}^^A
% }^^A
% \sbox0{\t}^^A
% \ifdim\wd0>\linewidth
%   \begingroup
%     \advance\linewidth by\leftmargin
%     \advance\linewidth by\rightmargin
%   \edef\x{\endgroup
%     \def\noexpand\lw{\the\linewidth}^^A
%   }\x
%   \def\lwbox{^^A
%     \leavevmode
%     \hbox to \linewidth{^^A
%       \kern-\leftmargin\relax
%       \hss
%       \usebox0
%       \hss
%       \kern-\rightmargin\relax
%     }^^A
%   }^^A
%   \ifdim\wd0>\lw
%     \sbox0{\small\t}^^A
%     \ifdim\wd0>\linewidth
%       \ifdim\wd0>\lw
%         \sbox0{\footnotesize\t}^^A
%         \ifdim\wd0>\linewidth
%           \ifdim\wd0>\lw
%             \sbox0{\scriptsize\t}^^A
%             \ifdim\wd0>\linewidth
%               \ifdim\wd0>\lw
%                 \sbox0{\tiny\t}^^A
%                 \ifdim\wd0>\linewidth
%                   \lwbox
%                 \else
%                   \usebox0
%                 \fi
%               \else
%                 \lwbox
%               \fi
%             \else
%               \usebox0
%             \fi
%           \else
%             \lwbox
%           \fi
%         \else
%           \usebox0
%         \fi
%       \else
%         \lwbox
%       \fi
%     \else
%       \usebox0
%     \fi
%   \else
%     \lwbox
%   \fi
% \else
%   \usebox0
% \fi
% \end{quote}
% If you have a \xfile{docstrip.cfg} that configures and enables \docstrip's
% TDS installing feature, then some files can already be in the right
% place, see the documentation of \docstrip.
%
% \subsection{Refresh file name databases}
%
% If your \TeX~distribution (\TeX{} Live, \mikTeX, \teTeX, \dots) relies on
% file name databases, you must refresh these. For example, \teTeX{} users run
% \verb|texhash| or \verb|mktexlsr|.
%
% \subsection{Some details for the interested}
%
% \paragraph{Unpacking with \LaTeX{}.}
% The \xfile{.dtx} chooses its action depending on the format:
% \begin{description}
% \item[\plainTeX:] Run \docstrip{} and extract the files.
% \item[\LaTeX:] Generate the documentation.
% \end{description}
% If you insist on using \LaTeX{} for \docstrip{} (really,
% \docstrip{} does not need \LaTeX ), then inform the autodetect routine
% about your intention:
% \begin{quote}
%   \verb|latex \let\install=y\input{footnoterange.dtx}|
% \end{quote}
% Do not forget to quote the argument according to the demands
% of your shell.
%
% \paragraph{Generating the documentation.\label{GenDoc}}
% You can use both the \xfile{.dtx} or the \xfile{.drv} to generate
% the documentation. The process can be configured by a
% configuration file \xfile{ltxdoc.cfg}. For instance, put the following
% line into this file, if you want to have A4 as paper format:
% \begin{quote}
%   \verb|\PassOptionsToClass{a4paper}{article}|
% \end{quote}
%
% \noindent An example follows how to generate the
% documentation with \pdfLaTeX :
%
% \begin{quote}
%\begin{verbatim}
%pdflatex footnoterange.drv
%makeindex -s gind.ist footnoterange.idx
%pdflatex footnoterange.drv
%makeindex -s gind.ist footnoterange.idx
%pdflatex footnoterange.drv
%\end{verbatim}
% \end{quote}
%
% \subsection{Compiling the example}
%
% The example file, \textsf{footnoterange-example.tex}, can be compiled via\newline
% \indent \verb|(pdf)latex footnoterange-example.tex|\newline
% but will need probably two compiler runs to get everything right.\newline
% \verb|footnoterange-example.tex| can be created by\newline
% \indent \verb|tex footnoterange.dtx|.\newline
% \verb|footnoterange.dtx| is at
% \url{https://www.ctan.org/pkg/footnoterange}.\newline
%
% \section{Acknowledgements}
%
% I (H.-Martin M\"{u}nch) would like to thank \textsc{Heiko Oberdiek}
% for providing a~lot~(!) of useful packages (from
% which I also got everything I know about creating a file in \xext{dtx}
% format, ok, say it: copying), \textsc{Frank Mittelbach} for several bug reports,
% for code for improving the package, and for pointing me to the \xpackage{fnpct}
% package as alternative, Willie Wong for reporting a bug and its fix,
% everybody of the CTAN team for managing CTAN,
% the \Newsgroup{comp.text.tex} and \Newsgroup{de.comp.text.tex} newsgroups,
% and everybody at \url{https://tex.stackexchange.com/}
% for their help in all things \hbox{\TeX{}.}
% This package was created as response to question
% \textquotedblleft Compressing consecutive footnote
% marks\textquotedblright{} by user001\newline
% (\url{https://tex.stackexchange.com/users/10127/user001}) on\newline
% \url{https://tex.stackexchange.com/q/44165/6865}.
%
% \newpage
% \phantomsection
% \begin{History}\label{History}
%   \begin{Version}{2011/02/11 v0.1ff.}
%     \item Some versions at \url{https://tex.stackexchange.com/q/44165/6865}.
%   \end{Version}
%   \begin{Version}{2012/02/17 v1.0a}
%     \item Package created and uploaded to CTAN.
%   \end{Version}
%   \begin{Version}{2018/10/15 v1.0b}
%     \item Fixed/updated minor details.
%     \item Bug fix: Did not work without \xpackage{hyperref}, wrong spacefactor.
%     \item Listing the \href{https://www.ctan.org/pkg/fnpct}{\xpackage{fnpct}}
%           package as alternative.
%   \end{Version}
%   \begin{Version}{2019/11/24 v1.0c}
%     \item Bug fix: Issue with footnote symbols.
%   \end{Version}
% \end{History}
%
% \bigskip
%
% When you find a mistake or have a suggestion for an improvement of this package,
% please send an e-mail to the maintainer, thanks! (Please see BUG REPORTS in the README.)
%
% \PrintIndex
%
% \Finale
\endinput|
% \end{quote}
% Do not forget to quote the argument according to the demands
% of your shell.
%
% \paragraph{Generating the documentation.\label{GenDoc}}
% You can use both the \xfile{.dtx} or the \xfile{.drv} to generate
% the documentation. The process can be configured by a
% configuration file \xfile{ltxdoc.cfg}. For instance, put the following
% line into this file, if you want to have A4 as paper format:
% \begin{quote}
%   \verb|\PassOptionsToClass{a4paper}{article}|
% \end{quote}
%
% \noindent An example follows how to generate the
% documentation with \pdfLaTeX :
%
% \begin{quote}
%\begin{verbatim}
%pdflatex footnoterange.drv
%makeindex -s gind.ist footnoterange.idx
%pdflatex footnoterange.drv
%makeindex -s gind.ist footnoterange.idx
%pdflatex footnoterange.drv
%\end{verbatim}
% \end{quote}
%
% \subsection{Compiling the example}
%
% The example file, \textsf{footnoterange-example.tex}, can be compiled via\newline
% \indent \verb|(pdf)latex footnoterange-example.tex|\newline
% but will need probably two compiler runs to get everything right.\newline
% \verb|footnoterange-example.tex| can be created by\newline
% \indent \verb|tex footnoterange.dtx|.\newline
% \verb|footnoterange.dtx| is at
% \url{https://www.ctan.org/pkg/footnoterange}.\newline
%
% \section{Acknowledgements}
%
% I (H.-Martin M\"{u}nch) would like to thank \textsc{Heiko Oberdiek}
% for providing a~lot~(!) of useful packages (from
% which I also got everything I know about creating a file in \xext{dtx}
% format, ok, say it: copying), \textsc{Frank Mittelbach} for several bug reports,
% for code for improving the package, and for pointing me to the \xpackage{fnpct}
% package as alternative, Willie Wong for reporting a bug and its fix,
% everybody of the CTAN team for managing CTAN,
% the \Newsgroup{comp.text.tex} and \Newsgroup{de.comp.text.tex} newsgroups,
% and everybody at \url{https://tex.stackexchange.com/}
% for their help in all things \hbox{\TeX{}.}
% This package was created as response to question
% \textquotedblleft Compressing consecutive footnote
% marks\textquotedblright{} by user001\newline
% (\url{https://tex.stackexchange.com/users/10127/user001}) on\newline
% \url{https://tex.stackexchange.com/q/44165/6865}.
%
% \newpage
% \phantomsection
% \begin{History}\label{History}
%   \begin{Version}{2011/02/11 v0.1ff.}
%     \item Some versions at \url{https://tex.stackexchange.com/q/44165/6865}.
%   \end{Version}
%   \begin{Version}{2012/02/17 v1.0a}
%     \item Package created and uploaded to CTAN.
%   \end{Version}
%   \begin{Version}{2018/10/15 v1.0b}
%     \item Fixed/updated minor details.
%     \item Bug fix: Did not work without \xpackage{hyperref}, wrong spacefactor.
%     \item Listing the \href{https://www.ctan.org/pkg/fnpct}{\xpackage{fnpct}}
%           package as alternative.
%   \end{Version}
%   \begin{Version}{2019/11/24 v1.0c}
%     \item Bug fix: Issue with footnote symbols.
%   \end{Version}
% \end{History}
%
% \bigskip
%
% When you find a mistake or have a suggestion for an improvement of this package,
% please send an e-mail to the maintainer, thanks! (Please see BUG REPORTS in the README.)
%
% \PrintIndex
%
% \Finale
\endinput|
% \end{quote}
% Do not forget to quote the argument according to the demands
% of your shell.
%
% \paragraph{Generating the documentation.\label{GenDoc}}
% You can use both the \xfile{.dtx} or the \xfile{.drv} to generate
% the documentation. The process can be configured by a
% configuration file \xfile{ltxdoc.cfg}. For instance, put the following
% line into this file, if you want to have A4 as paper format:
% \begin{quote}
%   \verb|\PassOptionsToClass{a4paper}{article}|
% \end{quote}
%
% \noindent An example follows how to generate the
% documentation with \pdfLaTeX :
%
% \begin{quote}
%\begin{verbatim}
%pdflatex footnoterange.drv
%makeindex -s gind.ist footnoterange.idx
%pdflatex footnoterange.drv
%makeindex -s gind.ist footnoterange.idx
%pdflatex footnoterange.drv
%\end{verbatim}
% \end{quote}
%
% \subsection{Compiling the example}
%
% The example file, \textsf{footnoterange-example.tex}, can be compiled via\newline
% \indent \verb|(pdf)latex footnoterange-example.tex|\newline
% but will need probably two compiler runs to get everything right.\newline
% \verb|footnoterange-example.tex| can be created by\newline
% \indent \verb|tex footnoterange.dtx|.\newline
% \verb|footnoterange.dtx| is at
% \url{https://www.ctan.org/pkg/footnoterange}.\newline
%
% \section{Acknowledgements}
%
% I (H.-Martin M\"{u}nch) would like to thank \textsc{Heiko Oberdiek}
% for providing a~lot~(!) of useful packages (from
% which I also got everything I know about creating a file in \xext{dtx}
% format, ok, say it: copying), \textsc{Frank Mittelbach} for several bug reports,
% for code for improving the package, and for pointing me to the \xpackage{fnpct}
% package as alternative, Willie Wong for reporting a bug and its fix,
% everybody of the CTAN team for managing CTAN,
% the \Newsgroup{comp.text.tex} and \Newsgroup{de.comp.text.tex} newsgroups,
% and everybody at \url{https://tex.stackexchange.com/}
% for their help in all things \hbox{\TeX{}.}
% This package was created as response to question
% \textquotedblleft Compressing consecutive footnote
% marks\textquotedblright{} by user001\newline
% (\url{https://tex.stackexchange.com/users/10127/user001}) on\newline
% \url{https://tex.stackexchange.com/q/44165/6865}.
%
% \newpage
% \phantomsection
% \begin{History}\label{History}
%   \begin{Version}{2011/02/11 v0.1ff.}
%     \item Some versions at \url{https://tex.stackexchange.com/q/44165/6865}.
%   \end{Version}
%   \begin{Version}{2012/02/17 v1.0a}
%     \item Package created and uploaded to CTAN.
%   \end{Version}
%   \begin{Version}{2018/10/15 v1.0b}
%     \item Fixed/updated minor details.
%     \item Bug fix: Did not work without \xpackage{hyperref}, wrong spacefactor.
%     \item Listing the \href{https://www.ctan.org/pkg/fnpct}{\xpackage{fnpct}}
%           package as alternative.
%   \end{Version}
%   \begin{Version}{2019/11/24 v1.0c}
%     \item Bug fix: Issue with footnote symbols.
%   \end{Version}
% \end{History}
%
% \bigskip
%
% When you find a mistake or have a suggestion for an improvement of this package,
% please send an e-mail to the maintainer, thanks! (Please see BUG REPORTS in the README.)
%
% \PrintIndex
%
% \Finale
\endinput
%        (quote the arguments according to the demands of your shell)
%
% Documentation:
%    (a) If footnoterange.drv is present:
%           (pdf)latex footnoterange.drv
%           makeindex -s gind.ist footnoterange.idx
%           (pdf)latex footnoterange.drv
%           makeindex -s gind.ist footnoterange.idx
%           (pdf)latex footnoterange.drv
%    (b) Without footnoterange.drv:
%           (pdf)latex footnoterange.dtx
%           makeindex -s gind.ist footnoterange.idx
%           (pdf)latex footnoterange.dtx
%           makeindex -s gind.ist footnoterange.idx
%           (pdf)latex footnoterange.dtx
%
%    The class ltxdoc loads the configuration file ltxdoc.cfg
%    if available. Here you can specify further options, e.g.
%    use DIN A4 as paper format:
%       \PassOptionsToClass{a4paper}{article}
%
% Installation:
%    TDS:tex/latex/footnoterange/footnoterange.sty
%    TDS:doc/latex/footnoterange/footnoterange.pdf
%    TDS:doc/latex/footnoterange/footnoterange-example.tex
%    TDS:doc/latex/footnoterange/footnoterange-example.pdf
%    TDS:source/latex/footnoterange/footnoterange.dtx
%
%<*ignore>
\begingroup
  \catcode123=1 %
  \catcode125=2 %
  \def\x{LaTeX2e}%
\expandafter\endgroup
\ifcase 0\ifx\install y1\fi\expandafter
         \ifx\csname processbatchFile\endcsname\relax\else1\fi
         \ifx\fmtname\x\else 1\fi\relax
\else\csname fi\endcsname
%</ignore>
%<*install>
\input docstrip.tex
\Msg{*********************************************************************************}
\Msg{* Installation}
\Msg{* Package: footnoterange 2019/11/24 v1.0c References to ranges of footnotes (HMM)}
\Msg{*********************************************************************************}

\keepsilent
\askforoverwritefalse

\let\MetaPrefix\relax
\preamble

This is a generated file.

Project: footnoterange
Version: 2019/11/24 v1.0c

Copyright (C) 2012 - 2019 by
    H.-Martin M"unch <Martin dot Muench at Uni-Bonn dot de>
Portions of code copyrighted by other people as marked;
lines marked "FMi": code by Frank Mittelbach, thanks!

The usual disclaimer applies:
If it doesn't work right that's your problem.
(Nevertheless, please send an e-mail to the maintainer
 when you find an error in this package.)

This work may be distributed and/or modified under the
conditions of the LaTeX Project Public License, either
version 1.3c of this license or (at your option) any later
version. This version of this license is in
   https://www.latex-project.org/lppl/lppl-1-3c.txt
and the latest version of this license is in
   https://www.latex-project.org/lppl.txt
and version 1.3c or later is part of all distributions of
LaTeX version 2005/12/01 or later.

This work has the LPPL maintenance status "maintained".

The Current Maintainer of this work is H.-Martin Muench.

This work consists of the main source file footnoterange.dtx,
the README, and the derived files
   footnoterange.sty, footnoterange.pdf,
   footnoterange.ins, footnoterange.drv,
   footnoterange-example.tex, footnoterange-example.pdf.

In memoriam
 Claudia Simone Barth + 1996/01/30
 Tommy Muench + 2014/01/02
 Hans-Klaus Muench + 2014/08/24

\endpreamble
\let\MetaPrefix\DoubleperCent

\generate{%
  \file{footnoterange.ins}{\from{footnoterange.dtx}{install}}%
  \file{footnoterange.drv}{\from{footnoterange.dtx}{driver}}%
  \usedir{tex/latex/footnoterange}%
  \file{footnoterange.sty}{\from{footnoterange.dtx}{package}}%
  \usedir{doc/latex/footnoterange}%
  \file{footnoterange-example.tex}{\from{footnoterange.dtx}{example}}%
}

\catcode32=13\relax% active space
\let =\space%
\Msg{************************************************************************}
\Msg{*}
\Msg{* To finish the installation you have to move the following}
\Msg{* file into a directory searched by TeX:}
\Msg{*}
\Msg{* footnoterange.sty}
\Msg{*}
\Msg{* To produce the documentation run the file `footnoterange.drv'}
\Msg{* through (pdf)LaTeX, e.g.}
\Msg{*  pdflatex footnoterange.drv}
\Msg{*  makeindex -s gind.ist footnoterange.idx}
\Msg{*  pdflatex footnoterange.drv}
\Msg{*  makeindex -s gind.ist footnoterange.idx}
\Msg{*  pdflatex footnoterange.drv}
\Msg{*}
\Msg{* At least three runs are necessary e.g. to get the}
\Msg{*  references right!}
\Msg{*}
\Msg{* Happy TeXing!}
\Msg{*}
\Msg{************************************************************************}

\endbatchfile
%</install>
%<*ignore>
\fi
%</ignore>
%
% \section{The documentation driver file}
%
% The next bit of code contains the documentation driver file for
% \TeX , i.\,e., the file that will produce the documentation you
% are currently reading. It will be extracted from this file by the
% \verb|docstrip| programme. That is, run \LaTeX{} on \verb|docstrip|
% and specify the \verb|driver| option when \verb|docstrip|
% asks for options.
%
%    \begin{macrocode}
%<*driver>
\NeedsTeXFormat{LaTeX2e}[2015/01/01]
\ProvidesFile{footnoterange.drv}[2019/11/24 v1.0c
            References to ranges of footnotes (HMM)]
\documentclass{ltxdoc}[2015/03/26]% v2.0w
\usepackage{holtxdoc}[2012/03/21]%  v0.24
%% footnoterange may work with earlier versions of LaTeX2e and those
%% class and package, but this was not tested.
%% Please consider updating your LaTeX, class, and package
%% to the most recent version (if they are not already the most
%% recent version).
\hypersetup{%
 pdfsubject={References to ranges of footnotes (HMM)},%
 pdfkeywords={LaTeX, footnoterange, footnotes, footnote, reference, H.-Martin Muench},%
 pdfencoding=auto,%
 pdflang={en},%
 breaklinks=true,%
 linktoc=all,%
 pdfstartview=FitH,%
 pdfpagelayout=OneColumn,%
 bookmarksnumbered=true,%
 bookmarksopen=true,%
 bookmarksopenlevel=3,%
 pdfmenubar=true,%
 pdftoolbar=true,%
 pdfwindowui=true,%
 pdfnewwindow=true%
}
\CodelineIndex
\hyphenation{printing docu-ment}
\gdef\unit#1{\mathord{\thinspace\mathrm{#1}}}
\begin{document}
  \DocInput{footnoterange.dtx}%
\end{document}
%</driver>
%    \end{macrocode}
%
% \fi
%
% \CheckSum{321}
%
% \CharacterTable
%  {Upper-case    \A\B\C\D\E\F\G\H\I\J\K\L\M\N\O\P\Q\R\S\T\U\V\W\X\Y\Z
%   Lower-case    \a\b\c\d\e\f\g\h\i\j\k\l\m\n\o\p\q\r\s\t\u\v\w\x\y\z
%   Digits        \0\1\2\3\4\5\6\7\8\9
%   Exclamation   \!     Double quote  \"     Hash (number) \#
%   Dollar        \$     Percent       \%     Ampersand     \&
%   Acute accent  \'     Left paren    \(     Right paren   \)
%   Asterisk      \*     Plus          \+     Comma         \,
%   Minus         \-     Point         \.     Solidus       \/
%   Colon         \:     Semicolon     \;     Less than     \<
%   Equals        \=     Greater than  \>     Question mark \?
%   Commercial at \@     Left bracket  \[     Backslash     \\
%   Right bracket \]     Circumflex    \^     Underscore    \_
%   Grave accent  \`     Left brace    \{     Vertical bar  \|
%   Right brace   \}     Tilde         \~}
%
% \GetFileInfo{footnoterange.drv}
%
% \begingroup
%   \def\x{\#,\$,\^,\_,\~,\ ,\&,\{,\},\%}%
%   \makeatletter
%   \@onelevel@sanitize\x
% \expandafter\endgroup
% \expandafter\DoNotIndex\expandafter{\x}
% \expandafter\DoNotIndex\expandafter{\string\ }
% \begingroup
%   \makeatletter
%     \lccode`9=32\relax
%     \lowercase{%^^A
%       \edef\x{\noexpand\DoNotIndex{\@backslashchar9}}%^^A
%     }%^^A
%   \expandafter\endgroup\x
% \DoNotIndex{\,,\\}
% \DoNotIndex{\documentclass,\usepackage,\ProvidesPackage,\begin,\end}
% \DoNotIndex{\MessageBreak}
% \DoNotIndex{\NeedsTeXFormat,\DoNotIndex,\verb}
% \DoNotIndex{\def,\edef,\gdef,\global,\xdef}
% \DoNotIndex{\ifx,\listfiles,\mathord,\mathrm}
% \DoNotIndex{\bigskip,\space,\thinspace,\ldots}
% \DoNotIndex{\indent,\noindent,\newline,\linebreak,\pagebreak,\newpage}
% \DoNotIndex{\textbf,\textit,\textsf,\textsc,\texttt,\textquotedblleft,\textquotedblright}
% \DoNotIndex{\plainTeX,\TeX,\LaTeX,\pdfLaTeX}
% \DoNotIndex{\section,\pagenumbering,\smallskip}
% \DoNotIndex{\@ifpackageloaded}
% \DoNotIndex{\arabic, \Alph, \alph, \alphalph, \Roman, \roman, \fnsymbol}
%
% \title{The \xpackage{footnoterange} package}
% \date{2019/11/24 v1.0c}
% \author{H.-Martin M\"{u}nch\\\xemail{Martin.Muench at Uni-Bonn.de}}
%
% \maketitle
%
% \begin{abstract}
% \noindent This package provides the environments |footnoterange| and
% |footnoterange*|. Multiple footnotes inside these environments
% are not referenced as e.\,g.
% \makeatletter%
% \hbox{\textquotedblleft \@textsuperscript {\normalfont 1 2 3}\textquotedblright}
% but as \hbox{\textquotedblleft \@textsuperscript {\normalfont 1--3}\textquotedblright ,\ }
% \makeatother%
% i.\,e. the range (from first to last referred footnote at that place) is given.
% Depending on loading of \xpackage{hyperref} and use of its |hyperfootnotes|-option
% the references are hyperlinked. (References to footnotes in |footnoterange*|
% are never hyperlinked.)
% \end{abstract}
%
% \bigskip
%
% \noindent Disclaimer for web links: The author is not responsible for any contents
% referred to in this work unless if having full knowledge of illegal contents.
% If any damage occurs by the use of information presented there, only the
% author of the respective pages might be liable, not the one who has referred
% to these pages.
%
% \noindent Save per page about $200\unit{ml}$ water,
% $2\unit{g}$ CO$_{2}$ and $2\unit{g}$ wood:
% Therefore please print only if this is really necessary.
%
% \medskip
%
% \tableofcontents
%
%\newpage
%
% \section{Introduction}
% \indent This \LaTeXe{} package provides the environments |footnoterange| and\newline
% |footnoterange*|. Multiple footnotes inside these environments are not referenced
% as e.\,g.
% \makeatletter%
% \hbox{\textquotedblleft \@textsuperscript {\normalfont 1 2 3}\textquotedblright}
% but as \hbox{\textquotedblleft \@textsuperscript {\normalfont 1--3}\textquotedblright ,\ }
% \makeatother%
% i.\,e. the range (from first to last
% referred footnote at that place) is given. Depending on loading of
% \xpackage{hyperref} and use of its |hyperfootnotes|-option the references are
% hyperlinked. References to footnotes in |footnoterange*| are never hyperlinked.\newline
% |\footnotemark\footnotetext| instead of |\footnote| is not supported inside
% |footnoterange| and |footnoterange*| environments.
%
% \bigskip
%
% \section{Usage}
%
% \indent Just load the package placing
% \begin{quote}
%   |\usepackage{footnoterange}|
% \end{quote}
% \noindent in the preamble of your \LaTeXe{} source file. There are no options to
% be used. The package should be loaded after \xpackage{hyperref} (if that package is used)
% and after any other package redefining footnotes. Usage of the |footnoterange| (or
% |footnoterange*|) environment is simple: Just encapsulate the footnotes, which shall
% be referred as range instead of referencing as single ones (see the example file).
%
% About hyperlinks:\newline
% Footnote-references in a |footnoterange*| environment are never hyperlinked.\newline
% Footnote-references in a |footnoterange| environment may be hyperlinked:
% \begin{itemize}
%  \item When the \xpackage{hyperref} package is not loaded, they are not
%         hyperlinked (naturally).
%  \item When the \xpackage{hyperref} package is loaded, it depends on the
%         |hyperfootnotes| option of the \xpackage{hyperref} package:
%  \begin{itemize}
%   \item With |hyperfootnotes=true|, the references are hyperlinked.
%   \item With |hyperfootnotes=false|, the references are not hyperlinked.
%  \end{itemize}
% \end{itemize}
% |\footnote|s inside of the |footnoterange|/{}|footnoterange*| environment
% can use the optional parameter, i.\,e. |\footnote[42]{text}| even if this is the
% third footnote. When giving ranges as reference, this does not seem to be
% such a good idea, if this is not done consistently (i.\,e. footnotes
% |[42], [43], [44]| resulting in 42--44 is OK,
% but |[42]| (original~2), unchanged~3, unchanged~4 resulting in
% 42--4 is confusing). A warning about these changed footnotemarks is written
% into the \xfile{.log}-file.
%
% When there is only a single footnote in a |footnoterange| environment,
% this does not produce some reference like %
% \makeatletter%
% \hbox{\textquotedblleft \@textsuperscript {\normalfont 7--7}\textquotedblright{},\ }%
% \makeatother%
% but only one footnotemark is used automatically. Because it is strange to use
% a |footnoterange| environment without a range of footnotes, an info is written
% into the \xfile{.log}-file about this.
%
% The usage of other footnote-number-representations
% (e.\,g.~\verb|\Roman|, \verb|\roman|, \verb|\Alph|, \verb|\alph|,
% \verb|\fnsymbol|) is also possible, but two things must be taken
% into account: The number of footnote references is restricted (for
% example with \verb|\Alph| only references A to Z are possible), which
% can be fixed e.\,g. with the \url{https://ctan.org/pkg/alphalph} package
% (but not automatically), and references to footnote-symbol-ranges (\verb|\fnsymbol|)
% are probably not very clear.
%
% \subsection*{Options}
% \indent The \xpackage{footnoterange} package takes no options.
%
% \section{Alternatives\label{sec:Alternatives}}
%
% The \xpackage{fnpct} (\url{https://www.ctan.org/pkg/fnpct}) package allows to set
% multiple footnotes (|\multfootnote|), suppress a footnotemark \hbox{(|*|),}
% and customize the multi-footnote-separator \hbox{(|\setfnpct{mult-fn-sep={|
% \emph{something} |}}|)}.
% Combining these with stepping of the footnotecounter gives a footnoterange:
% 
% \begin{verbatim}
% \documentclass{article}
% \usepackage{hyperref}
% \usepackage{fnpct}
% \begin{document}
% The three little pigs built their houses
% out of straw\footnote{not to be confused with hay},
% sticks\footnote{or lumber according to some sources}
% and bricks%
% \makeatletter%
% \setfnpct{mult-fn-sep={--}}%
% \multfootnote{probably fired clay bricks\addtocounter\@mpfn{+1};%
% *or something else;%
% what do I know}. %
% \makeatother%
% \end{document}
% \end{verbatim}
% 
% \noindent If you programmed or found another alternative, which is available at
% \url{https://www.CTAN.org/}, please send an e-mail to me with the name,
% location at CTAN, and a short notice, and I will probably list it here.
%
% \section{Example}
%
%    \begin{macrocode}
%<*example>
\documentclass[british]{article}[2014/09/29]% v1.4h
%%%%%%%%%%%%%%%%%%%%%%%%%%%%%%%%%%%%%%%%%%%%%%%%%%%%%%%%%%%%%%%%%%%%%
\usepackage[%
 extension=pdf,%
 plainpages=false,%
 pdfpagelabels=true,%
 hyperindex=false,%
 pdflang={en},%
 pdftitle={footnoterange package example},%
 pdfauthor={H.-Martin Muench},%
 pdfsubject={Example for the footnoterange package},%
 pdfkeywords={LaTeX, footnoterange, H.-Martin Muench},%
 pdfview=FitH,%
 pdfstartview=FitH,%
 pdfpagelayout=OneColumn,%
 hyperfootnotes=true%
]{hyperref}[2012/11/06]% v6.83m

\usepackage{footnoterange}[2019/11/24]% v1.0c

\gdef\unit#1{\mathord{\thinspace\mathrm{#1}}}%
\listfiles
\begin{document}
\pagenumbering{arabic}
\section*{Example for footnoterange}

This example demonstrates the use of package\newline
\textsf{footnoterange}, v1.0c as of 2019/11/24 (HMM).\newline
The package does not use options.\newline
\verb|footnoterange-example.pdf| can be created by\newline
\verb|pdflatex footnoterange-example.tex|,\newline
and \verb|footnoterange-example.tex| can be created by\newline
\verb|tex footnoterange.dtx|.\newline
\verb|footnoterange.dtx| is at
\url{https://www.ctan.org/pkg/footnoterange}.\newline

\noindent For more details please see the documentation!\newline

\noindent Save per page about $200\unit{ml}$ water,
$2\unit{g}$ CO$_{2}$ and $2\unit{g}$ wood:\newline
Therefore please print only if this is really necessary.\newline

This\footnote{Lorem} text bears a multiplicity of footnotes.
Because the \verb|hyperref| package is used with option
\verb|hyperfootnotes=true|, the footnotes
\begin{footnoterange}
\footnote{ipsum}%
\footnote{dolor}%
\footnote{sit}%
\footnote{amet,}%
\footnote{consectetur}%
\footnote{adipisicing}%
\footnote{elit,}%
\footnote{sed}%
\end{footnoterange}
are hyperlinked.\footnote{do eiusmod\ldots} Using
\verb|hyperfootnotes=false| or not using \verb|hyperref|
will remove the hyperlinks to the footnotes.\medskip

This text also bears a multiplicity of footnotes,
but due to the use of the starred
\begin{footnoterange*}
\footnote{Ut purus elit,}%
\footnote{vestibulum ut,}%
\footnote{placerat ac,}%
\footnote{adipiscing vitae,}%
\footnote{felis.}%
\footnote{Curabitur}%
\footnote{dictum}%
\footnote{gravida}%
\end{footnoterange*}
form of the environment they are not
hyperlinked.\footnote{mauris\ldots}\medskip

The use of the optional parameter for footnotes
\begin{footnoterange}
\footnote[101]{Nam arcu libero,}%
\footnote[102]{nonummy eget,}%
\footnote[103]{consectetuer id,}%
\footnote[104]{vulputate a,}%
\footnote[105]{magna.}
\end{footnoterange}
is also possible, but should be used consistently.

\pagebreak

When there is only one footnote
\begin{footnoterange}
\footnote{a single footnote}
\end{footnoterange}
in the \verb|footnoterange| environment, only one
footnotemark is used and an info is written into the
\verb|log|-file.\medskip

The usage of other footnote-number-representations
(e.\,g.~\verb|\Roman|, \verb|\roman|, \verb|\Alph|, \verb|\alph|,
\verb|\fnsymbol|) is also possible, but two things must be taken
into account: The number of footnote references is restricted
(for example with \verb|\Alph| only references A to Z are possible),
which can be fixed e.\,g. with the
\href{https://www.ctan.org/pkg/alphalph}{\texttt{alphalph} package},
and references to footnote-symbol-ranges (\verb|\fnsymbol|)
are probably not very clear.

\end{document}
%</example>
%    \end{macrocode}
%
% \StopEventually{}
%
% \section{The implementation}
%
% We start off by checking that we are loading into \LaTeXe{} and
% announcing the name and version of this package.
%
%    \begin{macrocode}
%<*package>
%    \end{macrocode}
%
%    \begin{macrocode}
\NeedsTeXFormat{LaTeX2e}[2011/06/27]
\ProvidesPackage{footnoterange}[2019/11/24 v1.0c
            References to ranges of footnotes (HMM)]
%    \end{macrocode}
%
% A short description of the \xpackage{footnoterange} package:
%
%    \begin{macrocode}
%% Provides the footnoterange and footnoterange* environments to
%% refer to ranges of footnotes.
%    \end{macrocode}
%
% We need the \xpackage{ltxcmds} package:
%
%    \begin{macrocode}
\RequirePackage{ltxcmds}[2011/11/09]% v1.22; for "\ltx@ifpackageloaded"
\RequirePackage{letltxmacro}[2010/09/02]% v1.4; for "\GlobalLetLtxMacro"

%    \end{macrocode}
%
% We use |\ifHy@hyperfootnotes ... \else ... \fi|, but when the \xpackage{hyperref}
% package has not been loaded, |\ifHy@hyperfootnotes| is not defined, and \LaTeX
% relates the |\else| and/or |\fi| (wrongly) to other |\if|s or groups.
% Therefore the |\ifHy@hyperfootnotes| are placed in their own commands,
% which are not looked at when the \xpackage{hyperref} package has not been loaded.
%
% \pagebreak
%
% \begin{macro}{\fnr@ifhyperrfn}
%    \begin{macrocode}
\newcommand{\fnr@ifhyperrfn}{%
  \ifHy@hyperfootnotes% option hyperfootnotes=true
    \addtocounter{Hfootnote}{+1}%
    % Code from the hyperref package
      \global\let\Hy@saved@currentHref\@currentHref%
      \hyper@makecurrent{Hfootnote}%
      \global\let\Hy@footnote@currentHref\@currentHref%
      \global\let\@currentHref\Hy@saved@currentHref%
    % End of code from the hyperref package
  \fi%
}

%    \end{macrocode}
% \end{macro}
%
% \begin{macro}{\fnr@rangefootnote}
% The |\fnr@rangefootnote| is the internal replacement for the original\newline
% |\footnote| command inside of a |footnoterange| environment.
%
%    \begin{macrocode}
\newcommand{\fnr@rangefootnote}[2][\empty]{%
%    \end{macrocode}
%
% |#1| contains the optional argument, which is predefined as |\empty|.
%
%    \begin{macrocode}
  \nolinebreak%
  % no \footnotemark here
  \addtocounter\@mpfn{+1}%
  \ltx@ifpackageloaded{hyperref}{\fnr@ifhyperrfn}{\relax}%
%    \end{macrocode}
%
% See the explanation before the declaration of the |\fnr@ifhyperrfn| command!
%
%    \begin{macrocode}
  \xdef\fnr@opt{#1}% contains the optional argument
  \xdef\fnr@arabic{\arabic\@mpfn}%
  \protected@edef\fnr@formated{\thefootnote}%
%    \end{macrocode}
%
% The |\footnotemark| is not set (yet), but the |\footnotetext|,
% and it gets a label.
%
%    \begin{macrocode}
  \ifx\fnr@opt\empty%
    \footnotetext{\label{fnr:\fnr@arabic}#2}%
  \else%
    \PackageWarning{footnoterange}{%
      Custom footnote-numbers used in a footnoterange\MessageBreak%
      environment: Footnote with natural number\MessageBreak%
      \ifx\fnr@formated\fnr@arabic%
        \fnr@arabic\space%
      \else%
        \space\fnr@arabic\space (\fnr@formated )%
      \fi%
      \MessageBreak%
      was renumbered to \fnr@opt\MessageBreak%
     }%
    \ltx@ifpackageloaded{hyperref}{% hyperref loaded
      \footnotetext[#1]{\phantomsection\label{fnr:\fnr@arabic}#2}%
     }{% hyperref not loaded
      \footnotetext[#1]{\label{fnr:\fnr@arabic}#2}%
     }%
  \fi%
}

%    \end{macrocode}
% \end{macro}
%
%    \begin{macrocode}

%    \end{macrocode}
% \pagebreak
%
% |\ifHy@hyperfootnotes| \textquotedblleft encapsulation\textquotedblright{} again:
%
% \begin{macro}{\fnr@ifhyperfnrA}
%    \begin{macrocode}
\newcommand{\fnr@ifhyperfnrA}{%
  \ifHy@hyperfootnotes% option hyperfootnotes=true
    \hbox{\hyperref[fnr:\fnr@first]{%
          \@textsuperscript{\normalfont \ref*{fnr:\fnr@first}}}}%
  \else% option hyperfootnotes=false
    \hbox{\@textsuperscript{\normalfont \ref*{fnr:\fnr@first}}}%
  \fi%
}

%    \end{macrocode}
% \end{macro}
%
% \begin{macro}{\fnr@ifhyperfnrN}
%    \begin{macrocode}
\newcommand{\fnr@ifhyperfnrN}{%
  \ifHy@hyperfootnotes% option hyperfootnotes=true
    \hbox{\hyperref[fnr:\fnr@first]{%
            \@textsuperscript{\normalfont \ref*{fnr:\fnr@first}}}%
          \@textsuperscript{\normalfont --}%
          \hyperref[fnr:\arabic\@mpfn]{%
            \@textsuperscript{\normalfont \ref*{fnr:\arabic\@mpfn}}}}%
  \else% option hyperfootnotes=false
    \hbox{\@textsuperscript{\normalfont \ref*{fnr:\fnr@first}%
                                        --\ref*{fnr:\arabic\@mpfn}}}%
  \fi%
}

%    \end{macrocode}
% \end{macro}
%
% \begin{environment}{footnoterange}
% The |footnoterange| environment remembers the next footnote number as\newline
% |\fnr@first| and the current |\footnote| definition as |\fnr@origfn|
% and uses |\fnr@rangefootnote| instead.
%
%    \begin{macrocode}
\newenvironment{footnoterange}{%
  % at begin of the footnoterange environment
  \ifhmode \unskip \fi% FMi
  \addtocounter\@mpfn{+1}%
  \xdef\fnr@first{\arabic\@mpfn}%
  \addtocounter\@mpfn{-1}%
  \let\fnr@origfn\footnote%
  \let\footnote\fnr@rangefootnote%
  \GlobalLetLtxMacro{\fnr@origfnmark}{\footnotemark}%
  \GlobalLetLtxMacro{\footnotemark}{\fnr@rangefootnotemark}%
  \ignorespaces% FMi
%    \end{macrocode}
%
% End of the environment:
%
%    \begin{macrocode}
}{% at end of the footnoterange environment
  \ifhmode% FMi 
    \unskip% FMi
%    \end{macrocode}
%
% Remember the spacefactor for restoring after the environment has ended
% (only in |hmode|!). \% FMi
%
%    \begin{macrocode}
    \xdef\fnr@aftergroup{\spacefactor\the\spacefactor\relax}% FMi
  \else% FMi
%    \end{macrocode}
%
% Otherwise do something harmless. \% FMi
%
%    \begin{macrocode}
    \xdef\fnr@aftergroup{\relax}%
  \fi% FMi
%    \end{macrocode}
%
% At the end of the environment the previous definition of |\footnote| is restored
%
%    \begin{macrocode}
  \let\footnote\fnr@origfn%
  \GlobalLetLtxMacro{\footnotemark}{\fnr@origfnmark}%
%    \end{macrocode}
%
% and the references are placed:
%
%    \begin{macrocode}
  \ifx\fnr@first\fnr@arabic% There is only one reference in the environment.
    \PackageInfo{footnoterange}{%
      Environment footnoterange used,\MessageBreak%
      but only one footnote (number \fnr@first ) is contained%
      \MessageBreak%
     }%
    \ltx@ifpackageloaded{hyperref}{\fnr@ifhyperfnrA% hyperref package loaded
    }{% hyperref package not loaded
      \hbox{\@textsuperscript{\normalfont \ref{fnr:\fnr@first}}}%
     }%
  \else%
    \ltx@ifpackageloaded{hyperref}{\fnr@ifhyperfnrN% hyperref package loaded
    }{% hyperref package not loaded
      \hbox{\@textsuperscript{\normalfont \ref{fnr:\fnr@first}%
                                        --\ref{fnr:\arabic\@mpfn}}}%
     }%
  \fi%
%    \end{macrocode}
%
% Restore the spacefactor after the environment has ended. \% FMi
%
%    \begin{macrocode}
  \aftergroup\fnr@aftergroup% FMi
}

%    \end{macrocode}
% \end{environment}
%
% \begin{environment}{footnoterange*}
% The |footnoterange*| environment does the same as the |footnoterange|
% environment but suppresses all hyperlinks to footnotes inside.
%
%    \begin{macrocode}
\newenvironment{footnoterange*}{%
  % at begin of the footnoterange* environment
  \ifhmode \unskip \fi% FMi
  \addtocounter\@mpfn{+1}%
  \xdef\fnr@first{\arabic\@mpfn}%
  \addtocounter\@mpfn{-1}%
  \let\fnr@origfn\footnote%
  \let\footnote\fnr@rangefootnote%
  \GlobalLetLtxMacro{\fnr@origfnmark}{\footnotemark}%
  \GlobalLetLtxMacro{\footnotemark}{\fnr@rangefootnotemark}%
  \ignorespaces% FMi
}{% at end of the footnoterange* environment
  \ifhmode% FMi 
    \unskip% FMi
    \xdef\fnr@aftergroup{\spacefactor\the\spacefactor\relax}% FMi
  \else% FMi
    \xdef\fnr@aftergroup{\relax}%
  \fi% FMi
  \let\footnote\fnr@origfn%
  \GlobalLetLtxMacro{\footnotemark}{\fnr@origfnmark}%
  \ifx\fnr@first\fnr@arabic% There is only one reference in the environment.
    \PackageInfo{footnoterange}{%
      Environment footnoterange used,\MessageBreak%
      but only one footnote (number \fnr@first ) is contained%
      \MessageBreak%
     }%
%    \end{macrocode}
% \pagebreak
%    \begin{macrocode}
    \ltx@ifpackageloaded{hyperref}{% hyperref package loaded
      \hbox{\@textsuperscript{\normalfont \ref*{fnr:\fnr@first}}}%
    }{% hyperref package not loaded
      \hbox{\@textsuperscript{\normalfont \ref{fnr:\fnr@first}}}%
     }%
  \else%
    \ltx@ifpackageloaded{hyperref}{% hyperref package loaded
      \hbox{\@textsuperscript{\normalfont \ref*{fnr:\fnr@first}%
                                        --\ref*{fnr:\arabic\@mpfn}}}%
    }{% hyperref package not loaded
      \hbox{\@textsuperscript{\normalfont \ref{fnr:\fnr@first}%
                                        --\ref{fnr:\arabic\@mpfn}}}%
     }%
  \fi%
  \aftergroup\fnr@aftergroup% FMi
}

%    \end{macrocode}
% \end{environment}
%
% The \xpackage{footnoterange} package does not handle |\footnotemark|s,
% but at least the user is made aware of this:
%
% \begin{macro}{\fnr@rangefootnotemark}
%    \begin{macrocode}

\newcommand{\fnr@rangefootnotemark}[1][\empty]{%
  \PackageError{footnoterange}{%
     \string\footnotemark\space not supported in footnoterange(*) environment%
   }{The footnoterange package does not handle a \string\footnotemark.\MessageBreak%
     Either use \string\footnote\string{...\string}\space instead of%
     \string\footnotemark\space and \string\footnotetext\string{...\string}\MessageBreak%
     in footnoterange and footnoterange* environments,\MessageBreak%
     or do not use those environments around a \string\footnotemark\MessageBreak%
     (or \string\renewcommand\string{\string\fnr@rangefootnotemark\string}\MessageBreak%
     with your own fixing code).%
    }%
  \xdef\fnr@mopt{#1}%
  \ifx\fnr@mopt\empty%
    \fnr@origfnmark%
  \else%
    \fnr@origfnmark[#1]%
  \fi%
}

%    \end{macrocode}
% \end{macro}
%
%    \begin{macrocode}
%</package>
%    \end{macrocode}
%
% \pagebreak
%
% \section{Installation}
%
% \subsection{Downloads\label{ss:Downloads}}
%
% Everything is available on CTAN,
% \url{https://www.ctan.org/tex-archive/},
% but may need additional packages themselves.
%
% \DescribeMacro{footnoterange.dtx}
% For unpacking the \xfile{footnoterange.dtx} file
% and constructing the documentation it is required:
% \begin{description}
% \item[-] \TeX{} Format \LaTeXe{}, \url{https://www.CTAN.org/}
%
% \item[-] document class \xpackage{ltxdoc}, 2015/03/26, v2.0w,
%           \url{https://ctan.org/pkg/ltxdoc}
%
% \item[-] package \xpackage{holtxdoc}, 2012/03/21, v0.24,
%           \url{https://ctan.org/pkg/holtxdoc}
% \end{description}
%
% \DescribeMacro{footnoterange.sty}
% The |footnoterange.sty| for \LaTeXe{} (i.\,e. each document using
% the \xpackage{footnoterange} package) requires:
% \begin{description}
% \item[-] \TeX{} Format \LaTeXe{}, \url{https://www.CTAN.org/}
%
% \item[-] package \xpackage{ltxcmds}, 2011/04/18, v1.20,
%           \url{https://www.ctan.org/pkg/ltxcmds}
% \end{description}
%
% \DescribeMacro{footnoterange-example.tex}
% The |footnoterange-example.tex| requires the same files as all
% documents using the \xpackage{footnoterange} package and additionally:
% \begin{description}
% \item[-] class \xpackage{article}, 2007/10/19, v1.4h, from \xpackage{classes.dtx}:\newline
%           \url{https://www.ctan.org/tex-archive/macros/latex/base/}
%
% \item[-] package \xpackage{footnoterange}, 2019/11/24, v1.0c,\newline
%           \url{https://www.ctan.org/pkg/footnoterange}\newline
%           (Well, it is the example file for this package, and because you are
%            reading the documentation for the \xpackage{footnoterange} package,
%            it can be assumed that you already have some version of it --
%            is it the current one?)
%
% \item[-] package \xpackage{hyperref}, 2012/11/06, v6.83m,\newline
%           \url{https://www.ctan.org/pkg/hyperref}
% \end{description}
%
% \DescribeMacro{Oberdiek}
% \DescribeMacro{holtxdoc}
% \DescribeMacro{ltxcmds}
% \DescribeMacro{alphalph}
% All packages of \textsc{Heiko Oberdiek}'s bundle `oberdiek'
% (especially \xpackage{holtxdoc}, \xpackage{ltxcmds}, and \xpackage{alphalph})
% are also available in a TDS compliant ZIP archive:\newline
% \url{http://mirror.ctan.org/install/macros/latex/contrib/oberdiek.tds.zip}.\newline
% It is probably best to download and use this, because the packages in there
% should be both recent and compatible.
%
% \DescribeMacro{hyperref}
% \noindent \xpackage{hyperref} is not included in that bundle and needs to be
% downloaded separately,\newline
% \url{http://mirrors.ctan.org/install/macros/latex/contrib/hyperref.tds.zip}.
%
% \DescribeMacro{M\"{u}nch}
% A list of my packages can be found at \url{https://www.ctan.org/author/muench-hm}.
%
% \pagebreak
%
% \subsection{Package, unpacking TDS}
% \paragraph{Package.} This package is available on \url{https://www.CTAN.org}.
% \begin{description}
% \item[\url{http://mirror.ctan.org/macros/latex/contrib/footnoterange/footnoterange.dtx}]\hspace*{0.1cm}
%       The source file.
% \item[\url{http://mirror.ctan.org/macros/latex/contrib/footnoterange/footnoterange.pdf}]\hspace*{0.1cm}
%       The documentation.
% \item[\url{http://mirror.ctan.org/macros/latex/contrib/footnoterange/footnoterange-example.pdf}]\hspace*{0.1cm}
%       The compiled example file, as it should look like.
% \item[\url{http://mirror.ctan.org/macros/latex/contrib/footnoterange/README}]\hspace*{0.1cm}\newline\hspace*{0.1cm}
%       The README file.
% \end{description}
% There is also a footnoterange.tds.zip available:
% \begin{description}
% \item[\url{http://mirror.ctan.org/install/macros/latex/contrib/footnoterange.tds.zip}]\hspace*{0.1cm}
%       Everything in TDS compliant, compiled format.
% \end{description}
% which additionally contains\newline
% \begin{tabular}{ll}
% footnoterange.ins & The installation file.\\
% footnoterange.drv & The driver to generate the documentation.\\
% footnoterange.sty & The \xext{sty}le file.
% \end{tabular}
%
% \bigskip
%
% \noindent For required other packages, please see the preceding subsection.
%
% \paragraph{Unpacking.} The \xfile{.dtx} file is a self-extracting
% \docstrip{} archive. The files are extracted by running the
% \xfile{.dtx} through \plainTeX:
% \begin{quote}
%   \verb|tex footnoterange.dtx|
% \end{quote}
%
% About generating the documentation see paragraph~\ref{GenDoc} below.
%
% \paragraph{TDS.} Now the different files must be moved into
% the different directories in your installation TDS tree
% (also known as \xfile{texmf} tree):
% \begin{quote}
% \def\t{^^A
% \begin{tabular}{@{}>{\ttfamily}l@{ $\rightarrow$ }>{\ttfamily}l@{}}
%   footnoterange.sty & tex/latex/footnoterange/footnoterange.sty\\
%   footnoterange.pdf & doc/latex/footnoterange/footnoterange.pdf\\
%   footnoterange-example.tex & doc/latex/footnoterange/footnoterange-example.tex\\
%   footnoterange-example.pdf & doc/latex/footnoterange/footnoterange-example.pdf\\
%   footnoterange.dtx & source/latex/footnoterange/footnoterange.dtx\\
% \end{tabular}^^A
% }^^A
% \sbox0{\t}^^A
% \ifdim\wd0>\linewidth
%   \begingroup
%     \advance\linewidth by\leftmargin
%     \advance\linewidth by\rightmargin
%   \edef\x{\endgroup
%     \def\noexpand\lw{\the\linewidth}^^A
%   }\x
%   \def\lwbox{^^A
%     \leavevmode
%     \hbox to \linewidth{^^A
%       \kern-\leftmargin\relax
%       \hss
%       \usebox0
%       \hss
%       \kern-\rightmargin\relax
%     }^^A
%   }^^A
%   \ifdim\wd0>\lw
%     \sbox0{\small\t}^^A
%     \ifdim\wd0>\linewidth
%       \ifdim\wd0>\lw
%         \sbox0{\footnotesize\t}^^A
%         \ifdim\wd0>\linewidth
%           \ifdim\wd0>\lw
%             \sbox0{\scriptsize\t}^^A
%             \ifdim\wd0>\linewidth
%               \ifdim\wd0>\lw
%                 \sbox0{\tiny\t}^^A
%                 \ifdim\wd0>\linewidth
%                   \lwbox
%                 \else
%                   \usebox0
%                 \fi
%               \else
%                 \lwbox
%               \fi
%             \else
%               \usebox0
%             \fi
%           \else
%             \lwbox
%           \fi
%         \else
%           \usebox0
%         \fi
%       \else
%         \lwbox
%       \fi
%     \else
%       \usebox0
%     \fi
%   \else
%     \lwbox
%   \fi
% \else
%   \usebox0
% \fi
% \end{quote}
% If you have a \xfile{docstrip.cfg} that configures and enables \docstrip's
% TDS installing feature, then some files can already be in the right
% place, see the documentation of \docstrip.
%
% \subsection{Refresh file name databases}
%
% If your \TeX~distribution (\TeX{} Live, \mikTeX, \teTeX, \dots) relies on
% file name databases, you must refresh these. For example, \teTeX{} users run
% \verb|texhash| or \verb|mktexlsr|.
%
% \subsection{Some details for the interested}
%
% \paragraph{Unpacking with \LaTeX{}.}
% The \xfile{.dtx} chooses its action depending on the format:
% \begin{description}
% \item[\plainTeX:] Run \docstrip{} and extract the files.
% \item[\LaTeX:] Generate the documentation.
% \end{description}
% If you insist on using \LaTeX{} for \docstrip{} (really,
% \docstrip{} does not need \LaTeX ), then inform the autodetect routine
% about your intention:
% \begin{quote}
%   \verb|latex \let\install=y% \iffalse meta-comment
%
% File: footnoterange.dtx
% Version: 2019/11/24 v1.0c
%
% Copyright (C) 2012 - 2019 by
%    H.-Martin M"unch <Martin dot Muench at Uni-Bonn dot de>
% Portions of code copyrighted by other people as marked;
% lines marked "FMi": code by Frank Mittelbach, thanks!
%
% This work may be distributed and/or modified under the
% conditions of the LaTeX Project Public License, either
% version 1.3c of this license or (at your option) any later
% version. This version of this license is in
%    https://www.latex-project.org/lppl/lppl-1-3c.txt
% and the latest version of this license is in
%    https://www.latex-project.org/lppl.txt
% and version 1.3c or later is part of all distributions of
% LaTeX version 2005/12/01 or later.
%
% This work has the LPPL maintenance status "maintained".
%
% The Current Maintainer of this work is H.-Martin Muench.
%
% This work consists of the main source file footnoterange.dtx,
% the README, and the derived files
%    footnoterange.sty, footnoterange.pdf,
%    footnoterange.ins, footnoterange.drv,
%    footnoterange-example.tex, footnoterange-example.pdf.
%
% 'footnoterange' is available on CTAN:
% https://www.ctan.org/pkg/footnoterange
%
% Also a TDS.ZIP file is provided that contains all the files
% already sorted in a TDS tree:
% http://mirror.ctan.org/install/macros/latex/contrib/footnoterange.tds.zip
%
% Unpacking:
%    (a) If footnoterange.ins is present:
%           tex footnoterange.ins
%    (b) Without footnoterange.ins:
%           tex footnoterange.dtx
%    (c) If you insist on using LaTeX
%           latex \let\install=y% \iffalse meta-comment
%
% File: footnoterange.dtx
% Version: 2019/11/24 v1.0c
%
% Copyright (C) 2012 - 2019 by
%    H.-Martin M"unch <Martin dot Muench at Uni-Bonn dot de>
% Portions of code copyrighted by other people as marked;
% lines marked "FMi": code by Frank Mittelbach, thanks!
%
% This work may be distributed and/or modified under the
% conditions of the LaTeX Project Public License, either
% version 1.3c of this license or (at your option) any later
% version. This version of this license is in
%    https://www.latex-project.org/lppl/lppl-1-3c.txt
% and the latest version of this license is in
%    https://www.latex-project.org/lppl.txt
% and version 1.3c or later is part of all distributions of
% LaTeX version 2005/12/01 or later.
%
% This work has the LPPL maintenance status "maintained".
%
% The Current Maintainer of this work is H.-Martin Muench.
%
% This work consists of the main source file footnoterange.dtx,
% the README, and the derived files
%    footnoterange.sty, footnoterange.pdf,
%    footnoterange.ins, footnoterange.drv,
%    footnoterange-example.tex, footnoterange-example.pdf.
%
% 'footnoterange' is available on CTAN:
% https://www.ctan.org/pkg/footnoterange
%
% Also a TDS.ZIP file is provided that contains all the files
% already sorted in a TDS tree:
% http://mirror.ctan.org/install/macros/latex/contrib/footnoterange.tds.zip
%
% Unpacking:
%    (a) If footnoterange.ins is present:
%           tex footnoterange.ins
%    (b) Without footnoterange.ins:
%           tex footnoterange.dtx
%    (c) If you insist on using LaTeX
%           latex \let\install=y% \iffalse meta-comment
%
% File: footnoterange.dtx
% Version: 2019/11/24 v1.0c
%
% Copyright (C) 2012 - 2019 by
%    H.-Martin M"unch <Martin dot Muench at Uni-Bonn dot de>
% Portions of code copyrighted by other people as marked;
% lines marked "FMi": code by Frank Mittelbach, thanks!
%
% This work may be distributed and/or modified under the
% conditions of the LaTeX Project Public License, either
% version 1.3c of this license or (at your option) any later
% version. This version of this license is in
%    https://www.latex-project.org/lppl/lppl-1-3c.txt
% and the latest version of this license is in
%    https://www.latex-project.org/lppl.txt
% and version 1.3c or later is part of all distributions of
% LaTeX version 2005/12/01 or later.
%
% This work has the LPPL maintenance status "maintained".
%
% The Current Maintainer of this work is H.-Martin Muench.
%
% This work consists of the main source file footnoterange.dtx,
% the README, and the derived files
%    footnoterange.sty, footnoterange.pdf,
%    footnoterange.ins, footnoterange.drv,
%    footnoterange-example.tex, footnoterange-example.pdf.
%
% 'footnoterange' is available on CTAN:
% https://www.ctan.org/pkg/footnoterange
%
% Also a TDS.ZIP file is provided that contains all the files
% already sorted in a TDS tree:
% http://mirror.ctan.org/install/macros/latex/contrib/footnoterange.tds.zip
%
% Unpacking:
%    (a) If footnoterange.ins is present:
%           tex footnoterange.ins
%    (b) Without footnoterange.ins:
%           tex footnoterange.dtx
%    (c) If you insist on using LaTeX
%           latex \let\install=y\input{footnoterange.dtx}
%        (quote the arguments according to the demands of your shell)
%
% Documentation:
%    (a) If footnoterange.drv is present:
%           (pdf)latex footnoterange.drv
%           makeindex -s gind.ist footnoterange.idx
%           (pdf)latex footnoterange.drv
%           makeindex -s gind.ist footnoterange.idx
%           (pdf)latex footnoterange.drv
%    (b) Without footnoterange.drv:
%           (pdf)latex footnoterange.dtx
%           makeindex -s gind.ist footnoterange.idx
%           (pdf)latex footnoterange.dtx
%           makeindex -s gind.ist footnoterange.idx
%           (pdf)latex footnoterange.dtx
%
%    The class ltxdoc loads the configuration file ltxdoc.cfg
%    if available. Here you can specify further options, e.g.
%    use DIN A4 as paper format:
%       \PassOptionsToClass{a4paper}{article}
%
% Installation:
%    TDS:tex/latex/footnoterange/footnoterange.sty
%    TDS:doc/latex/footnoterange/footnoterange.pdf
%    TDS:doc/latex/footnoterange/footnoterange-example.tex
%    TDS:doc/latex/footnoterange/footnoterange-example.pdf
%    TDS:source/latex/footnoterange/footnoterange.dtx
%
%<*ignore>
\begingroup
  \catcode123=1 %
  \catcode125=2 %
  \def\x{LaTeX2e}%
\expandafter\endgroup
\ifcase 0\ifx\install y1\fi\expandafter
         \ifx\csname processbatchFile\endcsname\relax\else1\fi
         \ifx\fmtname\x\else 1\fi\relax
\else\csname fi\endcsname
%</ignore>
%<*install>
\input docstrip.tex
\Msg{*********************************************************************************}
\Msg{* Installation}
\Msg{* Package: footnoterange 2019/11/24 v1.0c References to ranges of footnotes (HMM)}
\Msg{*********************************************************************************}

\keepsilent
\askforoverwritefalse

\let\MetaPrefix\relax
\preamble

This is a generated file.

Project: footnoterange
Version: 2019/11/24 v1.0c

Copyright (C) 2012 - 2019 by
    H.-Martin M"unch <Martin dot Muench at Uni-Bonn dot de>
Portions of code copyrighted by other people as marked;
lines marked "FMi": code by Frank Mittelbach, thanks!

The usual disclaimer applies:
If it doesn't work right that's your problem.
(Nevertheless, please send an e-mail to the maintainer
 when you find an error in this package.)

This work may be distributed and/or modified under the
conditions of the LaTeX Project Public License, either
version 1.3c of this license or (at your option) any later
version. This version of this license is in
   https://www.latex-project.org/lppl/lppl-1-3c.txt
and the latest version of this license is in
   https://www.latex-project.org/lppl.txt
and version 1.3c or later is part of all distributions of
LaTeX version 2005/12/01 or later.

This work has the LPPL maintenance status "maintained".

The Current Maintainer of this work is H.-Martin Muench.

This work consists of the main source file footnoterange.dtx,
the README, and the derived files
   footnoterange.sty, footnoterange.pdf,
   footnoterange.ins, footnoterange.drv,
   footnoterange-example.tex, footnoterange-example.pdf.

In memoriam
 Claudia Simone Barth + 1996/01/30
 Tommy Muench + 2014/01/02
 Hans-Klaus Muench + 2014/08/24

\endpreamble
\let\MetaPrefix\DoubleperCent

\generate{%
  \file{footnoterange.ins}{\from{footnoterange.dtx}{install}}%
  \file{footnoterange.drv}{\from{footnoterange.dtx}{driver}}%
  \usedir{tex/latex/footnoterange}%
  \file{footnoterange.sty}{\from{footnoterange.dtx}{package}}%
  \usedir{doc/latex/footnoterange}%
  \file{footnoterange-example.tex}{\from{footnoterange.dtx}{example}}%
}

\catcode32=13\relax% active space
\let =\space%
\Msg{************************************************************************}
\Msg{*}
\Msg{* To finish the installation you have to move the following}
\Msg{* file into a directory searched by TeX:}
\Msg{*}
\Msg{* footnoterange.sty}
\Msg{*}
\Msg{* To produce the documentation run the file `footnoterange.drv'}
\Msg{* through (pdf)LaTeX, e.g.}
\Msg{*  pdflatex footnoterange.drv}
\Msg{*  makeindex -s gind.ist footnoterange.idx}
\Msg{*  pdflatex footnoterange.drv}
\Msg{*  makeindex -s gind.ist footnoterange.idx}
\Msg{*  pdflatex footnoterange.drv}
\Msg{*}
\Msg{* At least three runs are necessary e.g. to get the}
\Msg{*  references right!}
\Msg{*}
\Msg{* Happy TeXing!}
\Msg{*}
\Msg{************************************************************************}

\endbatchfile
%</install>
%<*ignore>
\fi
%</ignore>
%
% \section{The documentation driver file}
%
% The next bit of code contains the documentation driver file for
% \TeX , i.\,e., the file that will produce the documentation you
% are currently reading. It will be extracted from this file by the
% \verb|docstrip| programme. That is, run \LaTeX{} on \verb|docstrip|
% and specify the \verb|driver| option when \verb|docstrip|
% asks for options.
%
%    \begin{macrocode}
%<*driver>
\NeedsTeXFormat{LaTeX2e}[2015/01/01]
\ProvidesFile{footnoterange.drv}[2019/11/24 v1.0c
            References to ranges of footnotes (HMM)]
\documentclass{ltxdoc}[2015/03/26]% v2.0w
\usepackage{holtxdoc}[2012/03/21]%  v0.24
%% footnoterange may work with earlier versions of LaTeX2e and those
%% class and package, but this was not tested.
%% Please consider updating your LaTeX, class, and package
%% to the most recent version (if they are not already the most
%% recent version).
\hypersetup{%
 pdfsubject={References to ranges of footnotes (HMM)},%
 pdfkeywords={LaTeX, footnoterange, footnotes, footnote, reference, H.-Martin Muench},%
 pdfencoding=auto,%
 pdflang={en},%
 breaklinks=true,%
 linktoc=all,%
 pdfstartview=FitH,%
 pdfpagelayout=OneColumn,%
 bookmarksnumbered=true,%
 bookmarksopen=true,%
 bookmarksopenlevel=3,%
 pdfmenubar=true,%
 pdftoolbar=true,%
 pdfwindowui=true,%
 pdfnewwindow=true%
}
\CodelineIndex
\hyphenation{printing docu-ment}
\gdef\unit#1{\mathord{\thinspace\mathrm{#1}}}
\begin{document}
  \DocInput{footnoterange.dtx}%
\end{document}
%</driver>
%    \end{macrocode}
%
% \fi
%
% \CheckSum{321}
%
% \CharacterTable
%  {Upper-case    \A\B\C\D\E\F\G\H\I\J\K\L\M\N\O\P\Q\R\S\T\U\V\W\X\Y\Z
%   Lower-case    \a\b\c\d\e\f\g\h\i\j\k\l\m\n\o\p\q\r\s\t\u\v\w\x\y\z
%   Digits        \0\1\2\3\4\5\6\7\8\9
%   Exclamation   \!     Double quote  \"     Hash (number) \#
%   Dollar        \$     Percent       \%     Ampersand     \&
%   Acute accent  \'     Left paren    \(     Right paren   \)
%   Asterisk      \*     Plus          \+     Comma         \,
%   Minus         \-     Point         \.     Solidus       \/
%   Colon         \:     Semicolon     \;     Less than     \<
%   Equals        \=     Greater than  \>     Question mark \?
%   Commercial at \@     Left bracket  \[     Backslash     \\
%   Right bracket \]     Circumflex    \^     Underscore    \_
%   Grave accent  \`     Left brace    \{     Vertical bar  \|
%   Right brace   \}     Tilde         \~}
%
% \GetFileInfo{footnoterange.drv}
%
% \begingroup
%   \def\x{\#,\$,\^,\_,\~,\ ,\&,\{,\},\%}%
%   \makeatletter
%   \@onelevel@sanitize\x
% \expandafter\endgroup
% \expandafter\DoNotIndex\expandafter{\x}
% \expandafter\DoNotIndex\expandafter{\string\ }
% \begingroup
%   \makeatletter
%     \lccode`9=32\relax
%     \lowercase{%^^A
%       \edef\x{\noexpand\DoNotIndex{\@backslashchar9}}%^^A
%     }%^^A
%   \expandafter\endgroup\x
% \DoNotIndex{\,,\\}
% \DoNotIndex{\documentclass,\usepackage,\ProvidesPackage,\begin,\end}
% \DoNotIndex{\MessageBreak}
% \DoNotIndex{\NeedsTeXFormat,\DoNotIndex,\verb}
% \DoNotIndex{\def,\edef,\gdef,\global,\xdef}
% \DoNotIndex{\ifx,\listfiles,\mathord,\mathrm}
% \DoNotIndex{\bigskip,\space,\thinspace,\ldots}
% \DoNotIndex{\indent,\noindent,\newline,\linebreak,\pagebreak,\newpage}
% \DoNotIndex{\textbf,\textit,\textsf,\textsc,\texttt,\textquotedblleft,\textquotedblright}
% \DoNotIndex{\plainTeX,\TeX,\LaTeX,\pdfLaTeX}
% \DoNotIndex{\section,\pagenumbering,\smallskip}
% \DoNotIndex{\@ifpackageloaded}
% \DoNotIndex{\arabic, \Alph, \alph, \alphalph, \Roman, \roman, \fnsymbol}
%
% \title{The \xpackage{footnoterange} package}
% \date{2019/11/24 v1.0c}
% \author{H.-Martin M\"{u}nch\\\xemail{Martin.Muench at Uni-Bonn.de}}
%
% \maketitle
%
% \begin{abstract}
% \noindent This package provides the environments |footnoterange| and
% |footnoterange*|. Multiple footnotes inside these environments
% are not referenced as e.\,g.
% \makeatletter%
% \hbox{\textquotedblleft \@textsuperscript {\normalfont 1 2 3}\textquotedblright}
% but as \hbox{\textquotedblleft \@textsuperscript {\normalfont 1--3}\textquotedblright ,\ }
% \makeatother%
% i.\,e. the range (from first to last referred footnote at that place) is given.
% Depending on loading of \xpackage{hyperref} and use of its |hyperfootnotes|-option
% the references are hyperlinked. (References to footnotes in |footnoterange*|
% are never hyperlinked.)
% \end{abstract}
%
% \bigskip
%
% \noindent Disclaimer for web links: The author is not responsible for any contents
% referred to in this work unless if having full knowledge of illegal contents.
% If any damage occurs by the use of information presented there, only the
% author of the respective pages might be liable, not the one who has referred
% to these pages.
%
% \noindent Save per page about $200\unit{ml}$ water,
% $2\unit{g}$ CO$_{2}$ and $2\unit{g}$ wood:
% Therefore please print only if this is really necessary.
%
% \medskip
%
% \tableofcontents
%
%\newpage
%
% \section{Introduction}
% \indent This \LaTeXe{} package provides the environments |footnoterange| and\newline
% |footnoterange*|. Multiple footnotes inside these environments are not referenced
% as e.\,g.
% \makeatletter%
% \hbox{\textquotedblleft \@textsuperscript {\normalfont 1 2 3}\textquotedblright}
% but as \hbox{\textquotedblleft \@textsuperscript {\normalfont 1--3}\textquotedblright ,\ }
% \makeatother%
% i.\,e. the range (from first to last
% referred footnote at that place) is given. Depending on loading of
% \xpackage{hyperref} and use of its |hyperfootnotes|-option the references are
% hyperlinked. References to footnotes in |footnoterange*| are never hyperlinked.\newline
% |\footnotemark\footnotetext| instead of |\footnote| is not supported inside
% |footnoterange| and |footnoterange*| environments.
%
% \bigskip
%
% \section{Usage}
%
% \indent Just load the package placing
% \begin{quote}
%   |\usepackage{footnoterange}|
% \end{quote}
% \noindent in the preamble of your \LaTeXe{} source file. There are no options to
% be used. The package should be loaded after \xpackage{hyperref} (if that package is used)
% and after any other package redefining footnotes. Usage of the |footnoterange| (or
% |footnoterange*|) environment is simple: Just encapsulate the footnotes, which shall
% be referred as range instead of referencing as single ones (see the example file).
%
% About hyperlinks:\newline
% Footnote-references in a |footnoterange*| environment are never hyperlinked.\newline
% Footnote-references in a |footnoterange| environment may be hyperlinked:
% \begin{itemize}
%  \item When the \xpackage{hyperref} package is not loaded, they are not
%         hyperlinked (naturally).
%  \item When the \xpackage{hyperref} package is loaded, it depends on the
%         |hyperfootnotes| option of the \xpackage{hyperref} package:
%  \begin{itemize}
%   \item With |hyperfootnotes=true|, the references are hyperlinked.
%   \item With |hyperfootnotes=false|, the references are not hyperlinked.
%  \end{itemize}
% \end{itemize}
% |\footnote|s inside of the |footnoterange|/{}|footnoterange*| environment
% can use the optional parameter, i.\,e. |\footnote[42]{text}| even if this is the
% third footnote. When giving ranges as reference, this does not seem to be
% such a good idea, if this is not done consistently (i.\,e. footnotes
% |[42], [43], [44]| resulting in 42--44 is OK,
% but |[42]| (original~2), unchanged~3, unchanged~4 resulting in
% 42--4 is confusing). A warning about these changed footnotemarks is written
% into the \xfile{.log}-file.
%
% When there is only a single footnote in a |footnoterange| environment,
% this does not produce some reference like %
% \makeatletter%
% \hbox{\textquotedblleft \@textsuperscript {\normalfont 7--7}\textquotedblright{},\ }%
% \makeatother%
% but only one footnotemark is used automatically. Because it is strange to use
% a |footnoterange| environment without a range of footnotes, an info is written
% into the \xfile{.log}-file about this.
%
% The usage of other footnote-number-representations
% (e.\,g.~\verb|\Roman|, \verb|\roman|, \verb|\Alph|, \verb|\alph|,
% \verb|\fnsymbol|) is also possible, but two things must be taken
% into account: The number of footnote references is restricted (for
% example with \verb|\Alph| only references A to Z are possible), which
% can be fixed e.\,g. with the \url{https://ctan.org/pkg/alphalph} package
% (but not automatically), and references to footnote-symbol-ranges (\verb|\fnsymbol|)
% are probably not very clear.
%
% \subsection*{Options}
% \indent The \xpackage{footnoterange} package takes no options.
%
% \section{Alternatives\label{sec:Alternatives}}
%
% The \xpackage{fnpct} (\url{https://www.ctan.org/pkg/fnpct}) package allows to set
% multiple footnotes (|\multfootnote|), suppress a footnotemark \hbox{(|*|),}
% and customize the multi-footnote-separator \hbox{(|\setfnpct{mult-fn-sep={|
% \emph{something} |}}|)}.
% Combining these with stepping of the footnotecounter gives a footnoterange:
% 
% \begin{verbatim}
% \documentclass{article}
% \usepackage{hyperref}
% \usepackage{fnpct}
% \begin{document}
% The three little pigs built their houses
% out of straw\footnote{not to be confused with hay},
% sticks\footnote{or lumber according to some sources}
% and bricks%
% \makeatletter%
% \setfnpct{mult-fn-sep={--}}%
% \multfootnote{probably fired clay bricks\addtocounter\@mpfn{+1};%
% *or something else;%
% what do I know}. %
% \makeatother%
% \end{document}
% \end{verbatim}
% 
% \noindent If you programmed or found another alternative, which is available at
% \url{https://www.CTAN.org/}, please send an e-mail to me with the name,
% location at CTAN, and a short notice, and I will probably list it here.
%
% \section{Example}
%
%    \begin{macrocode}
%<*example>
\documentclass[british]{article}[2014/09/29]% v1.4h
%%%%%%%%%%%%%%%%%%%%%%%%%%%%%%%%%%%%%%%%%%%%%%%%%%%%%%%%%%%%%%%%%%%%%
\usepackage[%
 extension=pdf,%
 plainpages=false,%
 pdfpagelabels=true,%
 hyperindex=false,%
 pdflang={en},%
 pdftitle={footnoterange package example},%
 pdfauthor={H.-Martin Muench},%
 pdfsubject={Example for the footnoterange package},%
 pdfkeywords={LaTeX, footnoterange, H.-Martin Muench},%
 pdfview=FitH,%
 pdfstartview=FitH,%
 pdfpagelayout=OneColumn,%
 hyperfootnotes=true%
]{hyperref}[2012/11/06]% v6.83m

\usepackage{footnoterange}[2019/11/24]% v1.0c

\gdef\unit#1{\mathord{\thinspace\mathrm{#1}}}%
\listfiles
\begin{document}
\pagenumbering{arabic}
\section*{Example for footnoterange}

This example demonstrates the use of package\newline
\textsf{footnoterange}, v1.0c as of 2019/11/24 (HMM).\newline
The package does not use options.\newline
\verb|footnoterange-example.pdf| can be created by\newline
\verb|pdflatex footnoterange-example.tex|,\newline
and \verb|footnoterange-example.tex| can be created by\newline
\verb|tex footnoterange.dtx|.\newline
\verb|footnoterange.dtx| is at
\url{https://www.ctan.org/pkg/footnoterange}.\newline

\noindent For more details please see the documentation!\newline

\noindent Save per page about $200\unit{ml}$ water,
$2\unit{g}$ CO$_{2}$ and $2\unit{g}$ wood:\newline
Therefore please print only if this is really necessary.\newline

This\footnote{Lorem} text bears a multiplicity of footnotes.
Because the \verb|hyperref| package is used with option
\verb|hyperfootnotes=true|, the footnotes
\begin{footnoterange}
\footnote{ipsum}%
\footnote{dolor}%
\footnote{sit}%
\footnote{amet,}%
\footnote{consectetur}%
\footnote{adipisicing}%
\footnote{elit,}%
\footnote{sed}%
\end{footnoterange}
are hyperlinked.\footnote{do eiusmod\ldots} Using
\verb|hyperfootnotes=false| or not using \verb|hyperref|
will remove the hyperlinks to the footnotes.\medskip

This text also bears a multiplicity of footnotes,
but due to the use of the starred
\begin{footnoterange*}
\footnote{Ut purus elit,}%
\footnote{vestibulum ut,}%
\footnote{placerat ac,}%
\footnote{adipiscing vitae,}%
\footnote{felis.}%
\footnote{Curabitur}%
\footnote{dictum}%
\footnote{gravida}%
\end{footnoterange*}
form of the environment they are not
hyperlinked.\footnote{mauris\ldots}\medskip

The use of the optional parameter for footnotes
\begin{footnoterange}
\footnote[101]{Nam arcu libero,}%
\footnote[102]{nonummy eget,}%
\footnote[103]{consectetuer id,}%
\footnote[104]{vulputate a,}%
\footnote[105]{magna.}
\end{footnoterange}
is also possible, but should be used consistently.

\pagebreak

When there is only one footnote
\begin{footnoterange}
\footnote{a single footnote}
\end{footnoterange}
in the \verb|footnoterange| environment, only one
footnotemark is used and an info is written into the
\verb|log|-file.\medskip

The usage of other footnote-number-representations
(e.\,g.~\verb|\Roman|, \verb|\roman|, \verb|\Alph|, \verb|\alph|,
\verb|\fnsymbol|) is also possible, but two things must be taken
into account: The number of footnote references is restricted
(for example with \verb|\Alph| only references A to Z are possible),
which can be fixed e.\,g. with the
\href{https://www.ctan.org/pkg/alphalph}{\texttt{alphalph} package},
and references to footnote-symbol-ranges (\verb|\fnsymbol|)
are probably not very clear.

\end{document}
%</example>
%    \end{macrocode}
%
% \StopEventually{}
%
% \section{The implementation}
%
% We start off by checking that we are loading into \LaTeXe{} and
% announcing the name and version of this package.
%
%    \begin{macrocode}
%<*package>
%    \end{macrocode}
%
%    \begin{macrocode}
\NeedsTeXFormat{LaTeX2e}[2011/06/27]
\ProvidesPackage{footnoterange}[2019/11/24 v1.0c
            References to ranges of footnotes (HMM)]
%    \end{macrocode}
%
% A short description of the \xpackage{footnoterange} package:
%
%    \begin{macrocode}
%% Provides the footnoterange and footnoterange* environments to
%% refer to ranges of footnotes.
%    \end{macrocode}
%
% We need the \xpackage{ltxcmds} package:
%
%    \begin{macrocode}
\RequirePackage{ltxcmds}[2011/11/09]% v1.22; for "\ltx@ifpackageloaded"
\RequirePackage{letltxmacro}[2010/09/02]% v1.4; for "\GlobalLetLtxMacro"

%    \end{macrocode}
%
% We use |\ifHy@hyperfootnotes ... \else ... \fi|, but when the \xpackage{hyperref}
% package has not been loaded, |\ifHy@hyperfootnotes| is not defined, and \LaTeX
% relates the |\else| and/or |\fi| (wrongly) to other |\if|s or groups.
% Therefore the |\ifHy@hyperfootnotes| are placed in their own commands,
% which are not looked at when the \xpackage{hyperref} package has not been loaded.
%
% \pagebreak
%
% \begin{macro}{\fnr@ifhyperrfn}
%    \begin{macrocode}
\newcommand{\fnr@ifhyperrfn}{%
  \ifHy@hyperfootnotes% option hyperfootnotes=true
    \addtocounter{Hfootnote}{+1}%
    % Code from the hyperref package
      \global\let\Hy@saved@currentHref\@currentHref%
      \hyper@makecurrent{Hfootnote}%
      \global\let\Hy@footnote@currentHref\@currentHref%
      \global\let\@currentHref\Hy@saved@currentHref%
    % End of code from the hyperref package
  \fi%
}

%    \end{macrocode}
% \end{macro}
%
% \begin{macro}{\fnr@rangefootnote}
% The |\fnr@rangefootnote| is the internal replacement for the original\newline
% |\footnote| command inside of a |footnoterange| environment.
%
%    \begin{macrocode}
\newcommand{\fnr@rangefootnote}[2][\empty]{%
%    \end{macrocode}
%
% |#1| contains the optional argument, which is predefined as |\empty|.
%
%    \begin{macrocode}
  \nolinebreak%
  % no \footnotemark here
  \addtocounter\@mpfn{+1}%
  \ltx@ifpackageloaded{hyperref}{\fnr@ifhyperrfn}{\relax}%
%    \end{macrocode}
%
% See the explanation before the declaration of the |\fnr@ifhyperrfn| command!
%
%    \begin{macrocode}
  \xdef\fnr@opt{#1}% contains the optional argument
  \xdef\fnr@arabic{\arabic\@mpfn}%
  \protected@edef\fnr@formated{\thefootnote}%
%    \end{macrocode}
%
% The |\footnotemark| is not set (yet), but the |\footnotetext|,
% and it gets a label.
%
%    \begin{macrocode}
  \ifx\fnr@opt\empty%
    \footnotetext{\label{fnr:\fnr@arabic}#2}%
  \else%
    \PackageWarning{footnoterange}{%
      Custom footnote-numbers used in a footnoterange\MessageBreak%
      environment: Footnote with natural number\MessageBreak%
      \ifx\fnr@formated\fnr@arabic%
        \fnr@arabic\space%
      \else%
        \space\fnr@arabic\space (\fnr@formated )%
      \fi%
      \MessageBreak%
      was renumbered to \fnr@opt\MessageBreak%
     }%
    \ltx@ifpackageloaded{hyperref}{% hyperref loaded
      \footnotetext[#1]{\phantomsection\label{fnr:\fnr@arabic}#2}%
     }{% hyperref not loaded
      \footnotetext[#1]{\label{fnr:\fnr@arabic}#2}%
     }%
  \fi%
}

%    \end{macrocode}
% \end{macro}
%
%    \begin{macrocode}

%    \end{macrocode}
% \pagebreak
%
% |\ifHy@hyperfootnotes| \textquotedblleft encapsulation\textquotedblright{} again:
%
% \begin{macro}{\fnr@ifhyperfnrA}
%    \begin{macrocode}
\newcommand{\fnr@ifhyperfnrA}{%
  \ifHy@hyperfootnotes% option hyperfootnotes=true
    \hbox{\hyperref[fnr:\fnr@first]{%
          \@textsuperscript{\normalfont \ref*{fnr:\fnr@first}}}}%
  \else% option hyperfootnotes=false
    \hbox{\@textsuperscript{\normalfont \ref*{fnr:\fnr@first}}}%
  \fi%
}

%    \end{macrocode}
% \end{macro}
%
% \begin{macro}{\fnr@ifhyperfnrN}
%    \begin{macrocode}
\newcommand{\fnr@ifhyperfnrN}{%
  \ifHy@hyperfootnotes% option hyperfootnotes=true
    \hbox{\hyperref[fnr:\fnr@first]{%
            \@textsuperscript{\normalfont \ref*{fnr:\fnr@first}}}%
          \@textsuperscript{\normalfont --}%
          \hyperref[fnr:\arabic\@mpfn]{%
            \@textsuperscript{\normalfont \ref*{fnr:\arabic\@mpfn}}}}%
  \else% option hyperfootnotes=false
    \hbox{\@textsuperscript{\normalfont \ref*{fnr:\fnr@first}%
                                        --\ref*{fnr:\arabic\@mpfn}}}%
  \fi%
}

%    \end{macrocode}
% \end{macro}
%
% \begin{environment}{footnoterange}
% The |footnoterange| environment remembers the next footnote number as\newline
% |\fnr@first| and the current |\footnote| definition as |\fnr@origfn|
% and uses |\fnr@rangefootnote| instead.
%
%    \begin{macrocode}
\newenvironment{footnoterange}{%
  % at begin of the footnoterange environment
  \ifhmode \unskip \fi% FMi
  \addtocounter\@mpfn{+1}%
  \xdef\fnr@first{\arabic\@mpfn}%
  \addtocounter\@mpfn{-1}%
  \let\fnr@origfn\footnote%
  \let\footnote\fnr@rangefootnote%
  \GlobalLetLtxMacro{\fnr@origfnmark}{\footnotemark}%
  \GlobalLetLtxMacro{\footnotemark}{\fnr@rangefootnotemark}%
  \ignorespaces% FMi
%    \end{macrocode}
%
% End of the environment:
%
%    \begin{macrocode}
}{% at end of the footnoterange environment
  \ifhmode% FMi 
    \unskip% FMi
%    \end{macrocode}
%
% Remember the spacefactor for restoring after the environment has ended
% (only in |hmode|!). \% FMi
%
%    \begin{macrocode}
    \xdef\fnr@aftergroup{\spacefactor\the\spacefactor\relax}% FMi
  \else% FMi
%    \end{macrocode}
%
% Otherwise do something harmless. \% FMi
%
%    \begin{macrocode}
    \xdef\fnr@aftergroup{\relax}%
  \fi% FMi
%    \end{macrocode}
%
% At the end of the environment the previous definition of |\footnote| is restored
%
%    \begin{macrocode}
  \let\footnote\fnr@origfn%
  \GlobalLetLtxMacro{\footnotemark}{\fnr@origfnmark}%
%    \end{macrocode}
%
% and the references are placed:
%
%    \begin{macrocode}
  \ifx\fnr@first\fnr@arabic% There is only one reference in the environment.
    \PackageInfo{footnoterange}{%
      Environment footnoterange used,\MessageBreak%
      but only one footnote (number \fnr@first ) is contained%
      \MessageBreak%
     }%
    \ltx@ifpackageloaded{hyperref}{\fnr@ifhyperfnrA% hyperref package loaded
    }{% hyperref package not loaded
      \hbox{\@textsuperscript{\normalfont \ref{fnr:\fnr@first}}}%
     }%
  \else%
    \ltx@ifpackageloaded{hyperref}{\fnr@ifhyperfnrN% hyperref package loaded
    }{% hyperref package not loaded
      \hbox{\@textsuperscript{\normalfont \ref{fnr:\fnr@first}%
                                        --\ref{fnr:\arabic\@mpfn}}}%
     }%
  \fi%
%    \end{macrocode}
%
% Restore the spacefactor after the environment has ended. \% FMi
%
%    \begin{macrocode}
  \aftergroup\fnr@aftergroup% FMi
}

%    \end{macrocode}
% \end{environment}
%
% \begin{environment}{footnoterange*}
% The |footnoterange*| environment does the same as the |footnoterange|
% environment but suppresses all hyperlinks to footnotes inside.
%
%    \begin{macrocode}
\newenvironment{footnoterange*}{%
  % at begin of the footnoterange* environment
  \ifhmode \unskip \fi% FMi
  \addtocounter\@mpfn{+1}%
  \xdef\fnr@first{\arabic\@mpfn}%
  \addtocounter\@mpfn{-1}%
  \let\fnr@origfn\footnote%
  \let\footnote\fnr@rangefootnote%
  \GlobalLetLtxMacro{\fnr@origfnmark}{\footnotemark}%
  \GlobalLetLtxMacro{\footnotemark}{\fnr@rangefootnotemark}%
  \ignorespaces% FMi
}{% at end of the footnoterange* environment
  \ifhmode% FMi 
    \unskip% FMi
    \xdef\fnr@aftergroup{\spacefactor\the\spacefactor\relax}% FMi
  \else% FMi
    \xdef\fnr@aftergroup{\relax}%
  \fi% FMi
  \let\footnote\fnr@origfn%
  \GlobalLetLtxMacro{\footnotemark}{\fnr@origfnmark}%
  \ifx\fnr@first\fnr@arabic% There is only one reference in the environment.
    \PackageInfo{footnoterange}{%
      Environment footnoterange used,\MessageBreak%
      but only one footnote (number \fnr@first ) is contained%
      \MessageBreak%
     }%
%    \end{macrocode}
% \pagebreak
%    \begin{macrocode}
    \ltx@ifpackageloaded{hyperref}{% hyperref package loaded
      \hbox{\@textsuperscript{\normalfont \ref*{fnr:\fnr@first}}}%
    }{% hyperref package not loaded
      \hbox{\@textsuperscript{\normalfont \ref{fnr:\fnr@first}}}%
     }%
  \else%
    \ltx@ifpackageloaded{hyperref}{% hyperref package loaded
      \hbox{\@textsuperscript{\normalfont \ref*{fnr:\fnr@first}%
                                        --\ref*{fnr:\arabic\@mpfn}}}%
    }{% hyperref package not loaded
      \hbox{\@textsuperscript{\normalfont \ref{fnr:\fnr@first}%
                                        --\ref{fnr:\arabic\@mpfn}}}%
     }%
  \fi%
  \aftergroup\fnr@aftergroup% FMi
}

%    \end{macrocode}
% \end{environment}
%
% The \xpackage{footnoterange} package does not handle |\footnotemark|s,
% but at least the user is made aware of this:
%
% \begin{macro}{\fnr@rangefootnotemark}
%    \begin{macrocode}

\newcommand{\fnr@rangefootnotemark}[1][\empty]{%
  \PackageError{footnoterange}{%
     \string\footnotemark\space not supported in footnoterange(*) environment%
   }{The footnoterange package does not handle a \string\footnotemark.\MessageBreak%
     Either use \string\footnote\string{...\string}\space instead of%
     \string\footnotemark\space and \string\footnotetext\string{...\string}\MessageBreak%
     in footnoterange and footnoterange* environments,\MessageBreak%
     or do not use those environments around a \string\footnotemark\MessageBreak%
     (or \string\renewcommand\string{\string\fnr@rangefootnotemark\string}\MessageBreak%
     with your own fixing code).%
    }%
  \xdef\fnr@mopt{#1}%
  \ifx\fnr@mopt\empty%
    \fnr@origfnmark%
  \else%
    \fnr@origfnmark[#1]%
  \fi%
}

%    \end{macrocode}
% \end{macro}
%
%    \begin{macrocode}
%</package>
%    \end{macrocode}
%
% \pagebreak
%
% \section{Installation}
%
% \subsection{Downloads\label{ss:Downloads}}
%
% Everything is available on CTAN,
% \url{https://www.ctan.org/tex-archive/},
% but may need additional packages themselves.
%
% \DescribeMacro{footnoterange.dtx}
% For unpacking the \xfile{footnoterange.dtx} file
% and constructing the documentation it is required:
% \begin{description}
% \item[-] \TeX{} Format \LaTeXe{}, \url{https://www.CTAN.org/}
%
% \item[-] document class \xpackage{ltxdoc}, 2015/03/26, v2.0w,
%           \url{https://ctan.org/pkg/ltxdoc}
%
% \item[-] package \xpackage{holtxdoc}, 2012/03/21, v0.24,
%           \url{https://ctan.org/pkg/holtxdoc}
% \end{description}
%
% \DescribeMacro{footnoterange.sty}
% The |footnoterange.sty| for \LaTeXe{} (i.\,e. each document using
% the \xpackage{footnoterange} package) requires:
% \begin{description}
% \item[-] \TeX{} Format \LaTeXe{}, \url{https://www.CTAN.org/}
%
% \item[-] package \xpackage{ltxcmds}, 2011/04/18, v1.20,
%           \url{https://www.ctan.org/pkg/ltxcmds}
% \end{description}
%
% \DescribeMacro{footnoterange-example.tex}
% The |footnoterange-example.tex| requires the same files as all
% documents using the \xpackage{footnoterange} package and additionally:
% \begin{description}
% \item[-] class \xpackage{article}, 2007/10/19, v1.4h, from \xpackage{classes.dtx}:\newline
%           \url{https://www.ctan.org/tex-archive/macros/latex/base/}
%
% \item[-] package \xpackage{footnoterange}, 2019/11/24, v1.0c,\newline
%           \url{https://www.ctan.org/pkg/footnoterange}\newline
%           (Well, it is the example file for this package, and because you are
%            reading the documentation for the \xpackage{footnoterange} package,
%            it can be assumed that you already have some version of it --
%            is it the current one?)
%
% \item[-] package \xpackage{hyperref}, 2012/11/06, v6.83m,\newline
%           \url{https://www.ctan.org/pkg/hyperref}
% \end{description}
%
% \DescribeMacro{Oberdiek}
% \DescribeMacro{holtxdoc}
% \DescribeMacro{ltxcmds}
% \DescribeMacro{alphalph}
% All packages of \textsc{Heiko Oberdiek}'s bundle `oberdiek'
% (especially \xpackage{holtxdoc}, \xpackage{ltxcmds}, and \xpackage{alphalph})
% are also available in a TDS compliant ZIP archive:\newline
% \url{http://mirror.ctan.org/install/macros/latex/contrib/oberdiek.tds.zip}.\newline
% It is probably best to download and use this, because the packages in there
% should be both recent and compatible.
%
% \DescribeMacro{hyperref}
% \noindent \xpackage{hyperref} is not included in that bundle and needs to be
% downloaded separately,\newline
% \url{http://mirrors.ctan.org/install/macros/latex/contrib/hyperref.tds.zip}.
%
% \DescribeMacro{M\"{u}nch}
% A list of my packages can be found at \url{https://www.ctan.org/author/muench-hm}.
%
% \pagebreak
%
% \subsection{Package, unpacking TDS}
% \paragraph{Package.} This package is available on \url{https://www.CTAN.org}.
% \begin{description}
% \item[\url{http://mirror.ctan.org/macros/latex/contrib/footnoterange/footnoterange.dtx}]\hspace*{0.1cm}
%       The source file.
% \item[\url{http://mirror.ctan.org/macros/latex/contrib/footnoterange/footnoterange.pdf}]\hspace*{0.1cm}
%       The documentation.
% \item[\url{http://mirror.ctan.org/macros/latex/contrib/footnoterange/footnoterange-example.pdf}]\hspace*{0.1cm}
%       The compiled example file, as it should look like.
% \item[\url{http://mirror.ctan.org/macros/latex/contrib/footnoterange/README}]\hspace*{0.1cm}\newline\hspace*{0.1cm}
%       The README file.
% \end{description}
% There is also a footnoterange.tds.zip available:
% \begin{description}
% \item[\url{http://mirror.ctan.org/install/macros/latex/contrib/footnoterange.tds.zip}]\hspace*{0.1cm}
%       Everything in TDS compliant, compiled format.
% \end{description}
% which additionally contains\newline
% \begin{tabular}{ll}
% footnoterange.ins & The installation file.\\
% footnoterange.drv & The driver to generate the documentation.\\
% footnoterange.sty & The \xext{sty}le file.
% \end{tabular}
%
% \bigskip
%
% \noindent For required other packages, please see the preceding subsection.
%
% \paragraph{Unpacking.} The \xfile{.dtx} file is a self-extracting
% \docstrip{} archive. The files are extracted by running the
% \xfile{.dtx} through \plainTeX:
% \begin{quote}
%   \verb|tex footnoterange.dtx|
% \end{quote}
%
% About generating the documentation see paragraph~\ref{GenDoc} below.
%
% \paragraph{TDS.} Now the different files must be moved into
% the different directories in your installation TDS tree
% (also known as \xfile{texmf} tree):
% \begin{quote}
% \def\t{^^A
% \begin{tabular}{@{}>{\ttfamily}l@{ $\rightarrow$ }>{\ttfamily}l@{}}
%   footnoterange.sty & tex/latex/footnoterange/footnoterange.sty\\
%   footnoterange.pdf & doc/latex/footnoterange/footnoterange.pdf\\
%   footnoterange-example.tex & doc/latex/footnoterange/footnoterange-example.tex\\
%   footnoterange-example.pdf & doc/latex/footnoterange/footnoterange-example.pdf\\
%   footnoterange.dtx & source/latex/footnoterange/footnoterange.dtx\\
% \end{tabular}^^A
% }^^A
% \sbox0{\t}^^A
% \ifdim\wd0>\linewidth
%   \begingroup
%     \advance\linewidth by\leftmargin
%     \advance\linewidth by\rightmargin
%   \edef\x{\endgroup
%     \def\noexpand\lw{\the\linewidth}^^A
%   }\x
%   \def\lwbox{^^A
%     \leavevmode
%     \hbox to \linewidth{^^A
%       \kern-\leftmargin\relax
%       \hss
%       \usebox0
%       \hss
%       \kern-\rightmargin\relax
%     }^^A
%   }^^A
%   \ifdim\wd0>\lw
%     \sbox0{\small\t}^^A
%     \ifdim\wd0>\linewidth
%       \ifdim\wd0>\lw
%         \sbox0{\footnotesize\t}^^A
%         \ifdim\wd0>\linewidth
%           \ifdim\wd0>\lw
%             \sbox0{\scriptsize\t}^^A
%             \ifdim\wd0>\linewidth
%               \ifdim\wd0>\lw
%                 \sbox0{\tiny\t}^^A
%                 \ifdim\wd0>\linewidth
%                   \lwbox
%                 \else
%                   \usebox0
%                 \fi
%               \else
%                 \lwbox
%               \fi
%             \else
%               \usebox0
%             \fi
%           \else
%             \lwbox
%           \fi
%         \else
%           \usebox0
%         \fi
%       \else
%         \lwbox
%       \fi
%     \else
%       \usebox0
%     \fi
%   \else
%     \lwbox
%   \fi
% \else
%   \usebox0
% \fi
% \end{quote}
% If you have a \xfile{docstrip.cfg} that configures and enables \docstrip's
% TDS installing feature, then some files can already be in the right
% place, see the documentation of \docstrip.
%
% \subsection{Refresh file name databases}
%
% If your \TeX~distribution (\TeX{} Live, \mikTeX, \teTeX, \dots) relies on
% file name databases, you must refresh these. For example, \teTeX{} users run
% \verb|texhash| or \verb|mktexlsr|.
%
% \subsection{Some details for the interested}
%
% \paragraph{Unpacking with \LaTeX{}.}
% The \xfile{.dtx} chooses its action depending on the format:
% \begin{description}
% \item[\plainTeX:] Run \docstrip{} and extract the files.
% \item[\LaTeX:] Generate the documentation.
% \end{description}
% If you insist on using \LaTeX{} for \docstrip{} (really,
% \docstrip{} does not need \LaTeX ), then inform the autodetect routine
% about your intention:
% \begin{quote}
%   \verb|latex \let\install=y\input{footnoterange.dtx}|
% \end{quote}
% Do not forget to quote the argument according to the demands
% of your shell.
%
% \paragraph{Generating the documentation.\label{GenDoc}}
% You can use both the \xfile{.dtx} or the \xfile{.drv} to generate
% the documentation. The process can be configured by a
% configuration file \xfile{ltxdoc.cfg}. For instance, put the following
% line into this file, if you want to have A4 as paper format:
% \begin{quote}
%   \verb|\PassOptionsToClass{a4paper}{article}|
% \end{quote}
%
% \noindent An example follows how to generate the
% documentation with \pdfLaTeX :
%
% \begin{quote}
%\begin{verbatim}
%pdflatex footnoterange.drv
%makeindex -s gind.ist footnoterange.idx
%pdflatex footnoterange.drv
%makeindex -s gind.ist footnoterange.idx
%pdflatex footnoterange.drv
%\end{verbatim}
% \end{quote}
%
% \subsection{Compiling the example}
%
% The example file, \textsf{footnoterange-example.tex}, can be compiled via\newline
% \indent \verb|(pdf)latex footnoterange-example.tex|\newline
% but will need probably two compiler runs to get everything right.\newline
% \verb|footnoterange-example.tex| can be created by\newline
% \indent \verb|tex footnoterange.dtx|.\newline
% \verb|footnoterange.dtx| is at
% \url{https://www.ctan.org/pkg/footnoterange}.\newline
%
% \section{Acknowledgements}
%
% I (H.-Martin M\"{u}nch) would like to thank \textsc{Heiko Oberdiek}
% for providing a~lot~(!) of useful packages (from
% which I also got everything I know about creating a file in \xext{dtx}
% format, ok, say it: copying), \textsc{Frank Mittelbach} for several bug reports,
% for code for improving the package, and for pointing me to the \xpackage{fnpct}
% package as alternative, Willie Wong for reporting a bug and its fix,
% everybody of the CTAN team for managing CTAN,
% the \Newsgroup{comp.text.tex} and \Newsgroup{de.comp.text.tex} newsgroups,
% and everybody at \url{https://tex.stackexchange.com/}
% for their help in all things \hbox{\TeX{}.}
% This package was created as response to question
% \textquotedblleft Compressing consecutive footnote
% marks\textquotedblright{} by user001\newline
% (\url{https://tex.stackexchange.com/users/10127/user001}) on\newline
% \url{https://tex.stackexchange.com/q/44165/6865}.
%
% \newpage
% \phantomsection
% \begin{History}\label{History}
%   \begin{Version}{2011/02/11 v0.1ff.}
%     \item Some versions at \url{https://tex.stackexchange.com/q/44165/6865}.
%   \end{Version}
%   \begin{Version}{2012/02/17 v1.0a}
%     \item Package created and uploaded to CTAN.
%   \end{Version}
%   \begin{Version}{2018/10/15 v1.0b}
%     \item Fixed/updated minor details.
%     \item Bug fix: Did not work without \xpackage{hyperref}, wrong spacefactor.
%     \item Listing the \href{https://www.ctan.org/pkg/fnpct}{\xpackage{fnpct}}
%           package as alternative.
%   \end{Version}
%   \begin{Version}{2019/11/24 v1.0c}
%     \item Bug fix: Issue with footnote symbols.
%   \end{Version}
% \end{History}
%
% \bigskip
%
% When you find a mistake or have a suggestion for an improvement of this package,
% please send an e-mail to the maintainer, thanks! (Please see BUG REPORTS in the README.)
%
% \PrintIndex
%
% \Finale
\endinput
%        (quote the arguments according to the demands of your shell)
%
% Documentation:
%    (a) If footnoterange.drv is present:
%           (pdf)latex footnoterange.drv
%           makeindex -s gind.ist footnoterange.idx
%           (pdf)latex footnoterange.drv
%           makeindex -s gind.ist footnoterange.idx
%           (pdf)latex footnoterange.drv
%    (b) Without footnoterange.drv:
%           (pdf)latex footnoterange.dtx
%           makeindex -s gind.ist footnoterange.idx
%           (pdf)latex footnoterange.dtx
%           makeindex -s gind.ist footnoterange.idx
%           (pdf)latex footnoterange.dtx
%
%    The class ltxdoc loads the configuration file ltxdoc.cfg
%    if available. Here you can specify further options, e.g.
%    use DIN A4 as paper format:
%       \PassOptionsToClass{a4paper}{article}
%
% Installation:
%    TDS:tex/latex/footnoterange/footnoterange.sty
%    TDS:doc/latex/footnoterange/footnoterange.pdf
%    TDS:doc/latex/footnoterange/footnoterange-example.tex
%    TDS:doc/latex/footnoterange/footnoterange-example.pdf
%    TDS:source/latex/footnoterange/footnoterange.dtx
%
%<*ignore>
\begingroup
  \catcode123=1 %
  \catcode125=2 %
  \def\x{LaTeX2e}%
\expandafter\endgroup
\ifcase 0\ifx\install y1\fi\expandafter
         \ifx\csname processbatchFile\endcsname\relax\else1\fi
         \ifx\fmtname\x\else 1\fi\relax
\else\csname fi\endcsname
%</ignore>
%<*install>
\input docstrip.tex
\Msg{*********************************************************************************}
\Msg{* Installation}
\Msg{* Package: footnoterange 2019/11/24 v1.0c References to ranges of footnotes (HMM)}
\Msg{*********************************************************************************}

\keepsilent
\askforoverwritefalse

\let\MetaPrefix\relax
\preamble

This is a generated file.

Project: footnoterange
Version: 2019/11/24 v1.0c

Copyright (C) 2012 - 2019 by
    H.-Martin M"unch <Martin dot Muench at Uni-Bonn dot de>
Portions of code copyrighted by other people as marked;
lines marked "FMi": code by Frank Mittelbach, thanks!

The usual disclaimer applies:
If it doesn't work right that's your problem.
(Nevertheless, please send an e-mail to the maintainer
 when you find an error in this package.)

This work may be distributed and/or modified under the
conditions of the LaTeX Project Public License, either
version 1.3c of this license or (at your option) any later
version. This version of this license is in
   https://www.latex-project.org/lppl/lppl-1-3c.txt
and the latest version of this license is in
   https://www.latex-project.org/lppl.txt
and version 1.3c or later is part of all distributions of
LaTeX version 2005/12/01 or later.

This work has the LPPL maintenance status "maintained".

The Current Maintainer of this work is H.-Martin Muench.

This work consists of the main source file footnoterange.dtx,
the README, and the derived files
   footnoterange.sty, footnoterange.pdf,
   footnoterange.ins, footnoterange.drv,
   footnoterange-example.tex, footnoterange-example.pdf.

In memoriam
 Claudia Simone Barth + 1996/01/30
 Tommy Muench + 2014/01/02
 Hans-Klaus Muench + 2014/08/24

\endpreamble
\let\MetaPrefix\DoubleperCent

\generate{%
  \file{footnoterange.ins}{\from{footnoterange.dtx}{install}}%
  \file{footnoterange.drv}{\from{footnoterange.dtx}{driver}}%
  \usedir{tex/latex/footnoterange}%
  \file{footnoterange.sty}{\from{footnoterange.dtx}{package}}%
  \usedir{doc/latex/footnoterange}%
  \file{footnoterange-example.tex}{\from{footnoterange.dtx}{example}}%
}

\catcode32=13\relax% active space
\let =\space%
\Msg{************************************************************************}
\Msg{*}
\Msg{* To finish the installation you have to move the following}
\Msg{* file into a directory searched by TeX:}
\Msg{*}
\Msg{* footnoterange.sty}
\Msg{*}
\Msg{* To produce the documentation run the file `footnoterange.drv'}
\Msg{* through (pdf)LaTeX, e.g.}
\Msg{*  pdflatex footnoterange.drv}
\Msg{*  makeindex -s gind.ist footnoterange.idx}
\Msg{*  pdflatex footnoterange.drv}
\Msg{*  makeindex -s gind.ist footnoterange.idx}
\Msg{*  pdflatex footnoterange.drv}
\Msg{*}
\Msg{* At least three runs are necessary e.g. to get the}
\Msg{*  references right!}
\Msg{*}
\Msg{* Happy TeXing!}
\Msg{*}
\Msg{************************************************************************}

\endbatchfile
%</install>
%<*ignore>
\fi
%</ignore>
%
% \section{The documentation driver file}
%
% The next bit of code contains the documentation driver file for
% \TeX , i.\,e., the file that will produce the documentation you
% are currently reading. It will be extracted from this file by the
% \verb|docstrip| programme. That is, run \LaTeX{} on \verb|docstrip|
% and specify the \verb|driver| option when \verb|docstrip|
% asks for options.
%
%    \begin{macrocode}
%<*driver>
\NeedsTeXFormat{LaTeX2e}[2015/01/01]
\ProvidesFile{footnoterange.drv}[2019/11/24 v1.0c
            References to ranges of footnotes (HMM)]
\documentclass{ltxdoc}[2015/03/26]% v2.0w
\usepackage{holtxdoc}[2012/03/21]%  v0.24
%% footnoterange may work with earlier versions of LaTeX2e and those
%% class and package, but this was not tested.
%% Please consider updating your LaTeX, class, and package
%% to the most recent version (if they are not already the most
%% recent version).
\hypersetup{%
 pdfsubject={References to ranges of footnotes (HMM)},%
 pdfkeywords={LaTeX, footnoterange, footnotes, footnote, reference, H.-Martin Muench},%
 pdfencoding=auto,%
 pdflang={en},%
 breaklinks=true,%
 linktoc=all,%
 pdfstartview=FitH,%
 pdfpagelayout=OneColumn,%
 bookmarksnumbered=true,%
 bookmarksopen=true,%
 bookmarksopenlevel=3,%
 pdfmenubar=true,%
 pdftoolbar=true,%
 pdfwindowui=true,%
 pdfnewwindow=true%
}
\CodelineIndex
\hyphenation{printing docu-ment}
\gdef\unit#1{\mathord{\thinspace\mathrm{#1}}}
\begin{document}
  \DocInput{footnoterange.dtx}%
\end{document}
%</driver>
%    \end{macrocode}
%
% \fi
%
% \CheckSum{321}
%
% \CharacterTable
%  {Upper-case    \A\B\C\D\E\F\G\H\I\J\K\L\M\N\O\P\Q\R\S\T\U\V\W\X\Y\Z
%   Lower-case    \a\b\c\d\e\f\g\h\i\j\k\l\m\n\o\p\q\r\s\t\u\v\w\x\y\z
%   Digits        \0\1\2\3\4\5\6\7\8\9
%   Exclamation   \!     Double quote  \"     Hash (number) \#
%   Dollar        \$     Percent       \%     Ampersand     \&
%   Acute accent  \'     Left paren    \(     Right paren   \)
%   Asterisk      \*     Plus          \+     Comma         \,
%   Minus         \-     Point         \.     Solidus       \/
%   Colon         \:     Semicolon     \;     Less than     \<
%   Equals        \=     Greater than  \>     Question mark \?
%   Commercial at \@     Left bracket  \[     Backslash     \\
%   Right bracket \]     Circumflex    \^     Underscore    \_
%   Grave accent  \`     Left brace    \{     Vertical bar  \|
%   Right brace   \}     Tilde         \~}
%
% \GetFileInfo{footnoterange.drv}
%
% \begingroup
%   \def\x{\#,\$,\^,\_,\~,\ ,\&,\{,\},\%}%
%   \makeatletter
%   \@onelevel@sanitize\x
% \expandafter\endgroup
% \expandafter\DoNotIndex\expandafter{\x}
% \expandafter\DoNotIndex\expandafter{\string\ }
% \begingroup
%   \makeatletter
%     \lccode`9=32\relax
%     \lowercase{%^^A
%       \edef\x{\noexpand\DoNotIndex{\@backslashchar9}}%^^A
%     }%^^A
%   \expandafter\endgroup\x
% \DoNotIndex{\,,\\}
% \DoNotIndex{\documentclass,\usepackage,\ProvidesPackage,\begin,\end}
% \DoNotIndex{\MessageBreak}
% \DoNotIndex{\NeedsTeXFormat,\DoNotIndex,\verb}
% \DoNotIndex{\def,\edef,\gdef,\global,\xdef}
% \DoNotIndex{\ifx,\listfiles,\mathord,\mathrm}
% \DoNotIndex{\bigskip,\space,\thinspace,\ldots}
% \DoNotIndex{\indent,\noindent,\newline,\linebreak,\pagebreak,\newpage}
% \DoNotIndex{\textbf,\textit,\textsf,\textsc,\texttt,\textquotedblleft,\textquotedblright}
% \DoNotIndex{\plainTeX,\TeX,\LaTeX,\pdfLaTeX}
% \DoNotIndex{\section,\pagenumbering,\smallskip}
% \DoNotIndex{\@ifpackageloaded}
% \DoNotIndex{\arabic, \Alph, \alph, \alphalph, \Roman, \roman, \fnsymbol}
%
% \title{The \xpackage{footnoterange} package}
% \date{2019/11/24 v1.0c}
% \author{H.-Martin M\"{u}nch\\\xemail{Martin.Muench at Uni-Bonn.de}}
%
% \maketitle
%
% \begin{abstract}
% \noindent This package provides the environments |footnoterange| and
% |footnoterange*|. Multiple footnotes inside these environments
% are not referenced as e.\,g.
% \makeatletter%
% \hbox{\textquotedblleft \@textsuperscript {\normalfont 1 2 3}\textquotedblright}
% but as \hbox{\textquotedblleft \@textsuperscript {\normalfont 1--3}\textquotedblright ,\ }
% \makeatother%
% i.\,e. the range (from first to last referred footnote at that place) is given.
% Depending on loading of \xpackage{hyperref} and use of its |hyperfootnotes|-option
% the references are hyperlinked. (References to footnotes in |footnoterange*|
% are never hyperlinked.)
% \end{abstract}
%
% \bigskip
%
% \noindent Disclaimer for web links: The author is not responsible for any contents
% referred to in this work unless if having full knowledge of illegal contents.
% If any damage occurs by the use of information presented there, only the
% author of the respective pages might be liable, not the one who has referred
% to these pages.
%
% \noindent Save per page about $200\unit{ml}$ water,
% $2\unit{g}$ CO$_{2}$ and $2\unit{g}$ wood:
% Therefore please print only if this is really necessary.
%
% \medskip
%
% \tableofcontents
%
%\newpage
%
% \section{Introduction}
% \indent This \LaTeXe{} package provides the environments |footnoterange| and\newline
% |footnoterange*|. Multiple footnotes inside these environments are not referenced
% as e.\,g.
% \makeatletter%
% \hbox{\textquotedblleft \@textsuperscript {\normalfont 1 2 3}\textquotedblright}
% but as \hbox{\textquotedblleft \@textsuperscript {\normalfont 1--3}\textquotedblright ,\ }
% \makeatother%
% i.\,e. the range (from first to last
% referred footnote at that place) is given. Depending on loading of
% \xpackage{hyperref} and use of its |hyperfootnotes|-option the references are
% hyperlinked. References to footnotes in |footnoterange*| are never hyperlinked.\newline
% |\footnotemark\footnotetext| instead of |\footnote| is not supported inside
% |footnoterange| and |footnoterange*| environments.
%
% \bigskip
%
% \section{Usage}
%
% \indent Just load the package placing
% \begin{quote}
%   |\usepackage{footnoterange}|
% \end{quote}
% \noindent in the preamble of your \LaTeXe{} source file. There are no options to
% be used. The package should be loaded after \xpackage{hyperref} (if that package is used)
% and after any other package redefining footnotes. Usage of the |footnoterange| (or
% |footnoterange*|) environment is simple: Just encapsulate the footnotes, which shall
% be referred as range instead of referencing as single ones (see the example file).
%
% About hyperlinks:\newline
% Footnote-references in a |footnoterange*| environment are never hyperlinked.\newline
% Footnote-references in a |footnoterange| environment may be hyperlinked:
% \begin{itemize}
%  \item When the \xpackage{hyperref} package is not loaded, they are not
%         hyperlinked (naturally).
%  \item When the \xpackage{hyperref} package is loaded, it depends on the
%         |hyperfootnotes| option of the \xpackage{hyperref} package:
%  \begin{itemize}
%   \item With |hyperfootnotes=true|, the references are hyperlinked.
%   \item With |hyperfootnotes=false|, the references are not hyperlinked.
%  \end{itemize}
% \end{itemize}
% |\footnote|s inside of the |footnoterange|/{}|footnoterange*| environment
% can use the optional parameter, i.\,e. |\footnote[42]{text}| even if this is the
% third footnote. When giving ranges as reference, this does not seem to be
% such a good idea, if this is not done consistently (i.\,e. footnotes
% |[42], [43], [44]| resulting in 42--44 is OK,
% but |[42]| (original~2), unchanged~3, unchanged~4 resulting in
% 42--4 is confusing). A warning about these changed footnotemarks is written
% into the \xfile{.log}-file.
%
% When there is only a single footnote in a |footnoterange| environment,
% this does not produce some reference like %
% \makeatletter%
% \hbox{\textquotedblleft \@textsuperscript {\normalfont 7--7}\textquotedblright{},\ }%
% \makeatother%
% but only one footnotemark is used automatically. Because it is strange to use
% a |footnoterange| environment without a range of footnotes, an info is written
% into the \xfile{.log}-file about this.
%
% The usage of other footnote-number-representations
% (e.\,g.~\verb|\Roman|, \verb|\roman|, \verb|\Alph|, \verb|\alph|,
% \verb|\fnsymbol|) is also possible, but two things must be taken
% into account: The number of footnote references is restricted (for
% example with \verb|\Alph| only references A to Z are possible), which
% can be fixed e.\,g. with the \url{https://ctan.org/pkg/alphalph} package
% (but not automatically), and references to footnote-symbol-ranges (\verb|\fnsymbol|)
% are probably not very clear.
%
% \subsection*{Options}
% \indent The \xpackage{footnoterange} package takes no options.
%
% \section{Alternatives\label{sec:Alternatives}}
%
% The \xpackage{fnpct} (\url{https://www.ctan.org/pkg/fnpct}) package allows to set
% multiple footnotes (|\multfootnote|), suppress a footnotemark \hbox{(|*|),}
% and customize the multi-footnote-separator \hbox{(|\setfnpct{mult-fn-sep={|
% \emph{something} |}}|)}.
% Combining these with stepping of the footnotecounter gives a footnoterange:
% 
% \begin{verbatim}
% \documentclass{article}
% \usepackage{hyperref}
% \usepackage{fnpct}
% \begin{document}
% The three little pigs built their houses
% out of straw\footnote{not to be confused with hay},
% sticks\footnote{or lumber according to some sources}
% and bricks%
% \makeatletter%
% \setfnpct{mult-fn-sep={--}}%
% \multfootnote{probably fired clay bricks\addtocounter\@mpfn{+1};%
% *or something else;%
% what do I know}. %
% \makeatother%
% \end{document}
% \end{verbatim}
% 
% \noindent If you programmed or found another alternative, which is available at
% \url{https://www.CTAN.org/}, please send an e-mail to me with the name,
% location at CTAN, and a short notice, and I will probably list it here.
%
% \section{Example}
%
%    \begin{macrocode}
%<*example>
\documentclass[british]{article}[2014/09/29]% v1.4h
%%%%%%%%%%%%%%%%%%%%%%%%%%%%%%%%%%%%%%%%%%%%%%%%%%%%%%%%%%%%%%%%%%%%%
\usepackage[%
 extension=pdf,%
 plainpages=false,%
 pdfpagelabels=true,%
 hyperindex=false,%
 pdflang={en},%
 pdftitle={footnoterange package example},%
 pdfauthor={H.-Martin Muench},%
 pdfsubject={Example for the footnoterange package},%
 pdfkeywords={LaTeX, footnoterange, H.-Martin Muench},%
 pdfview=FitH,%
 pdfstartview=FitH,%
 pdfpagelayout=OneColumn,%
 hyperfootnotes=true%
]{hyperref}[2012/11/06]% v6.83m

\usepackage{footnoterange}[2019/11/24]% v1.0c

\gdef\unit#1{\mathord{\thinspace\mathrm{#1}}}%
\listfiles
\begin{document}
\pagenumbering{arabic}
\section*{Example for footnoterange}

This example demonstrates the use of package\newline
\textsf{footnoterange}, v1.0c as of 2019/11/24 (HMM).\newline
The package does not use options.\newline
\verb|footnoterange-example.pdf| can be created by\newline
\verb|pdflatex footnoterange-example.tex|,\newline
and \verb|footnoterange-example.tex| can be created by\newline
\verb|tex footnoterange.dtx|.\newline
\verb|footnoterange.dtx| is at
\url{https://www.ctan.org/pkg/footnoterange}.\newline

\noindent For more details please see the documentation!\newline

\noindent Save per page about $200\unit{ml}$ water,
$2\unit{g}$ CO$_{2}$ and $2\unit{g}$ wood:\newline
Therefore please print only if this is really necessary.\newline

This\footnote{Lorem} text bears a multiplicity of footnotes.
Because the \verb|hyperref| package is used with option
\verb|hyperfootnotes=true|, the footnotes
\begin{footnoterange}
\footnote{ipsum}%
\footnote{dolor}%
\footnote{sit}%
\footnote{amet,}%
\footnote{consectetur}%
\footnote{adipisicing}%
\footnote{elit,}%
\footnote{sed}%
\end{footnoterange}
are hyperlinked.\footnote{do eiusmod\ldots} Using
\verb|hyperfootnotes=false| or not using \verb|hyperref|
will remove the hyperlinks to the footnotes.\medskip

This text also bears a multiplicity of footnotes,
but due to the use of the starred
\begin{footnoterange*}
\footnote{Ut purus elit,}%
\footnote{vestibulum ut,}%
\footnote{placerat ac,}%
\footnote{adipiscing vitae,}%
\footnote{felis.}%
\footnote{Curabitur}%
\footnote{dictum}%
\footnote{gravida}%
\end{footnoterange*}
form of the environment they are not
hyperlinked.\footnote{mauris\ldots}\medskip

The use of the optional parameter for footnotes
\begin{footnoterange}
\footnote[101]{Nam arcu libero,}%
\footnote[102]{nonummy eget,}%
\footnote[103]{consectetuer id,}%
\footnote[104]{vulputate a,}%
\footnote[105]{magna.}
\end{footnoterange}
is also possible, but should be used consistently.

\pagebreak

When there is only one footnote
\begin{footnoterange}
\footnote{a single footnote}
\end{footnoterange}
in the \verb|footnoterange| environment, only one
footnotemark is used and an info is written into the
\verb|log|-file.\medskip

The usage of other footnote-number-representations
(e.\,g.~\verb|\Roman|, \verb|\roman|, \verb|\Alph|, \verb|\alph|,
\verb|\fnsymbol|) is also possible, but two things must be taken
into account: The number of footnote references is restricted
(for example with \verb|\Alph| only references A to Z are possible),
which can be fixed e.\,g. with the
\href{https://www.ctan.org/pkg/alphalph}{\texttt{alphalph} package},
and references to footnote-symbol-ranges (\verb|\fnsymbol|)
are probably not very clear.

\end{document}
%</example>
%    \end{macrocode}
%
% \StopEventually{}
%
% \section{The implementation}
%
% We start off by checking that we are loading into \LaTeXe{} and
% announcing the name and version of this package.
%
%    \begin{macrocode}
%<*package>
%    \end{macrocode}
%
%    \begin{macrocode}
\NeedsTeXFormat{LaTeX2e}[2011/06/27]
\ProvidesPackage{footnoterange}[2019/11/24 v1.0c
            References to ranges of footnotes (HMM)]
%    \end{macrocode}
%
% A short description of the \xpackage{footnoterange} package:
%
%    \begin{macrocode}
%% Provides the footnoterange and footnoterange* environments to
%% refer to ranges of footnotes.
%    \end{macrocode}
%
% We need the \xpackage{ltxcmds} package:
%
%    \begin{macrocode}
\RequirePackage{ltxcmds}[2011/11/09]% v1.22; for "\ltx@ifpackageloaded"
\RequirePackage{letltxmacro}[2010/09/02]% v1.4; for "\GlobalLetLtxMacro"

%    \end{macrocode}
%
% We use |\ifHy@hyperfootnotes ... \else ... \fi|, but when the \xpackage{hyperref}
% package has not been loaded, |\ifHy@hyperfootnotes| is not defined, and \LaTeX
% relates the |\else| and/or |\fi| (wrongly) to other |\if|s or groups.
% Therefore the |\ifHy@hyperfootnotes| are placed in their own commands,
% which are not looked at when the \xpackage{hyperref} package has not been loaded.
%
% \pagebreak
%
% \begin{macro}{\fnr@ifhyperrfn}
%    \begin{macrocode}
\newcommand{\fnr@ifhyperrfn}{%
  \ifHy@hyperfootnotes% option hyperfootnotes=true
    \addtocounter{Hfootnote}{+1}%
    % Code from the hyperref package
      \global\let\Hy@saved@currentHref\@currentHref%
      \hyper@makecurrent{Hfootnote}%
      \global\let\Hy@footnote@currentHref\@currentHref%
      \global\let\@currentHref\Hy@saved@currentHref%
    % End of code from the hyperref package
  \fi%
}

%    \end{macrocode}
% \end{macro}
%
% \begin{macro}{\fnr@rangefootnote}
% The |\fnr@rangefootnote| is the internal replacement for the original\newline
% |\footnote| command inside of a |footnoterange| environment.
%
%    \begin{macrocode}
\newcommand{\fnr@rangefootnote}[2][\empty]{%
%    \end{macrocode}
%
% |#1| contains the optional argument, which is predefined as |\empty|.
%
%    \begin{macrocode}
  \nolinebreak%
  % no \footnotemark here
  \addtocounter\@mpfn{+1}%
  \ltx@ifpackageloaded{hyperref}{\fnr@ifhyperrfn}{\relax}%
%    \end{macrocode}
%
% See the explanation before the declaration of the |\fnr@ifhyperrfn| command!
%
%    \begin{macrocode}
  \xdef\fnr@opt{#1}% contains the optional argument
  \xdef\fnr@arabic{\arabic\@mpfn}%
  \protected@edef\fnr@formated{\thefootnote}%
%    \end{macrocode}
%
% The |\footnotemark| is not set (yet), but the |\footnotetext|,
% and it gets a label.
%
%    \begin{macrocode}
  \ifx\fnr@opt\empty%
    \footnotetext{\label{fnr:\fnr@arabic}#2}%
  \else%
    \PackageWarning{footnoterange}{%
      Custom footnote-numbers used in a footnoterange\MessageBreak%
      environment: Footnote with natural number\MessageBreak%
      \ifx\fnr@formated\fnr@arabic%
        \fnr@arabic\space%
      \else%
        \space\fnr@arabic\space (\fnr@formated )%
      \fi%
      \MessageBreak%
      was renumbered to \fnr@opt\MessageBreak%
     }%
    \ltx@ifpackageloaded{hyperref}{% hyperref loaded
      \footnotetext[#1]{\phantomsection\label{fnr:\fnr@arabic}#2}%
     }{% hyperref not loaded
      \footnotetext[#1]{\label{fnr:\fnr@arabic}#2}%
     }%
  \fi%
}

%    \end{macrocode}
% \end{macro}
%
%    \begin{macrocode}

%    \end{macrocode}
% \pagebreak
%
% |\ifHy@hyperfootnotes| \textquotedblleft encapsulation\textquotedblright{} again:
%
% \begin{macro}{\fnr@ifhyperfnrA}
%    \begin{macrocode}
\newcommand{\fnr@ifhyperfnrA}{%
  \ifHy@hyperfootnotes% option hyperfootnotes=true
    \hbox{\hyperref[fnr:\fnr@first]{%
          \@textsuperscript{\normalfont \ref*{fnr:\fnr@first}}}}%
  \else% option hyperfootnotes=false
    \hbox{\@textsuperscript{\normalfont \ref*{fnr:\fnr@first}}}%
  \fi%
}

%    \end{macrocode}
% \end{macro}
%
% \begin{macro}{\fnr@ifhyperfnrN}
%    \begin{macrocode}
\newcommand{\fnr@ifhyperfnrN}{%
  \ifHy@hyperfootnotes% option hyperfootnotes=true
    \hbox{\hyperref[fnr:\fnr@first]{%
            \@textsuperscript{\normalfont \ref*{fnr:\fnr@first}}}%
          \@textsuperscript{\normalfont --}%
          \hyperref[fnr:\arabic\@mpfn]{%
            \@textsuperscript{\normalfont \ref*{fnr:\arabic\@mpfn}}}}%
  \else% option hyperfootnotes=false
    \hbox{\@textsuperscript{\normalfont \ref*{fnr:\fnr@first}%
                                        --\ref*{fnr:\arabic\@mpfn}}}%
  \fi%
}

%    \end{macrocode}
% \end{macro}
%
% \begin{environment}{footnoterange}
% The |footnoterange| environment remembers the next footnote number as\newline
% |\fnr@first| and the current |\footnote| definition as |\fnr@origfn|
% and uses |\fnr@rangefootnote| instead.
%
%    \begin{macrocode}
\newenvironment{footnoterange}{%
  % at begin of the footnoterange environment
  \ifhmode \unskip \fi% FMi
  \addtocounter\@mpfn{+1}%
  \xdef\fnr@first{\arabic\@mpfn}%
  \addtocounter\@mpfn{-1}%
  \let\fnr@origfn\footnote%
  \let\footnote\fnr@rangefootnote%
  \GlobalLetLtxMacro{\fnr@origfnmark}{\footnotemark}%
  \GlobalLetLtxMacro{\footnotemark}{\fnr@rangefootnotemark}%
  \ignorespaces% FMi
%    \end{macrocode}
%
% End of the environment:
%
%    \begin{macrocode}
}{% at end of the footnoterange environment
  \ifhmode% FMi 
    \unskip% FMi
%    \end{macrocode}
%
% Remember the spacefactor for restoring after the environment has ended
% (only in |hmode|!). \% FMi
%
%    \begin{macrocode}
    \xdef\fnr@aftergroup{\spacefactor\the\spacefactor\relax}% FMi
  \else% FMi
%    \end{macrocode}
%
% Otherwise do something harmless. \% FMi
%
%    \begin{macrocode}
    \xdef\fnr@aftergroup{\relax}%
  \fi% FMi
%    \end{macrocode}
%
% At the end of the environment the previous definition of |\footnote| is restored
%
%    \begin{macrocode}
  \let\footnote\fnr@origfn%
  \GlobalLetLtxMacro{\footnotemark}{\fnr@origfnmark}%
%    \end{macrocode}
%
% and the references are placed:
%
%    \begin{macrocode}
  \ifx\fnr@first\fnr@arabic% There is only one reference in the environment.
    \PackageInfo{footnoterange}{%
      Environment footnoterange used,\MessageBreak%
      but only one footnote (number \fnr@first ) is contained%
      \MessageBreak%
     }%
    \ltx@ifpackageloaded{hyperref}{\fnr@ifhyperfnrA% hyperref package loaded
    }{% hyperref package not loaded
      \hbox{\@textsuperscript{\normalfont \ref{fnr:\fnr@first}}}%
     }%
  \else%
    \ltx@ifpackageloaded{hyperref}{\fnr@ifhyperfnrN% hyperref package loaded
    }{% hyperref package not loaded
      \hbox{\@textsuperscript{\normalfont \ref{fnr:\fnr@first}%
                                        --\ref{fnr:\arabic\@mpfn}}}%
     }%
  \fi%
%    \end{macrocode}
%
% Restore the spacefactor after the environment has ended. \% FMi
%
%    \begin{macrocode}
  \aftergroup\fnr@aftergroup% FMi
}

%    \end{macrocode}
% \end{environment}
%
% \begin{environment}{footnoterange*}
% The |footnoterange*| environment does the same as the |footnoterange|
% environment but suppresses all hyperlinks to footnotes inside.
%
%    \begin{macrocode}
\newenvironment{footnoterange*}{%
  % at begin of the footnoterange* environment
  \ifhmode \unskip \fi% FMi
  \addtocounter\@mpfn{+1}%
  \xdef\fnr@first{\arabic\@mpfn}%
  \addtocounter\@mpfn{-1}%
  \let\fnr@origfn\footnote%
  \let\footnote\fnr@rangefootnote%
  \GlobalLetLtxMacro{\fnr@origfnmark}{\footnotemark}%
  \GlobalLetLtxMacro{\footnotemark}{\fnr@rangefootnotemark}%
  \ignorespaces% FMi
}{% at end of the footnoterange* environment
  \ifhmode% FMi 
    \unskip% FMi
    \xdef\fnr@aftergroup{\spacefactor\the\spacefactor\relax}% FMi
  \else% FMi
    \xdef\fnr@aftergroup{\relax}%
  \fi% FMi
  \let\footnote\fnr@origfn%
  \GlobalLetLtxMacro{\footnotemark}{\fnr@origfnmark}%
  \ifx\fnr@first\fnr@arabic% There is only one reference in the environment.
    \PackageInfo{footnoterange}{%
      Environment footnoterange used,\MessageBreak%
      but only one footnote (number \fnr@first ) is contained%
      \MessageBreak%
     }%
%    \end{macrocode}
% \pagebreak
%    \begin{macrocode}
    \ltx@ifpackageloaded{hyperref}{% hyperref package loaded
      \hbox{\@textsuperscript{\normalfont \ref*{fnr:\fnr@first}}}%
    }{% hyperref package not loaded
      \hbox{\@textsuperscript{\normalfont \ref{fnr:\fnr@first}}}%
     }%
  \else%
    \ltx@ifpackageloaded{hyperref}{% hyperref package loaded
      \hbox{\@textsuperscript{\normalfont \ref*{fnr:\fnr@first}%
                                        --\ref*{fnr:\arabic\@mpfn}}}%
    }{% hyperref package not loaded
      \hbox{\@textsuperscript{\normalfont \ref{fnr:\fnr@first}%
                                        --\ref{fnr:\arabic\@mpfn}}}%
     }%
  \fi%
  \aftergroup\fnr@aftergroup% FMi
}

%    \end{macrocode}
% \end{environment}
%
% The \xpackage{footnoterange} package does not handle |\footnotemark|s,
% but at least the user is made aware of this:
%
% \begin{macro}{\fnr@rangefootnotemark}
%    \begin{macrocode}

\newcommand{\fnr@rangefootnotemark}[1][\empty]{%
  \PackageError{footnoterange}{%
     \string\footnotemark\space not supported in footnoterange(*) environment%
   }{The footnoterange package does not handle a \string\footnotemark.\MessageBreak%
     Either use \string\footnote\string{...\string}\space instead of%
     \string\footnotemark\space and \string\footnotetext\string{...\string}\MessageBreak%
     in footnoterange and footnoterange* environments,\MessageBreak%
     or do not use those environments around a \string\footnotemark\MessageBreak%
     (or \string\renewcommand\string{\string\fnr@rangefootnotemark\string}\MessageBreak%
     with your own fixing code).%
    }%
  \xdef\fnr@mopt{#1}%
  \ifx\fnr@mopt\empty%
    \fnr@origfnmark%
  \else%
    \fnr@origfnmark[#1]%
  \fi%
}

%    \end{macrocode}
% \end{macro}
%
%    \begin{macrocode}
%</package>
%    \end{macrocode}
%
% \pagebreak
%
% \section{Installation}
%
% \subsection{Downloads\label{ss:Downloads}}
%
% Everything is available on CTAN,
% \url{https://www.ctan.org/tex-archive/},
% but may need additional packages themselves.
%
% \DescribeMacro{footnoterange.dtx}
% For unpacking the \xfile{footnoterange.dtx} file
% and constructing the documentation it is required:
% \begin{description}
% \item[-] \TeX{} Format \LaTeXe{}, \url{https://www.CTAN.org/}
%
% \item[-] document class \xpackage{ltxdoc}, 2015/03/26, v2.0w,
%           \url{https://ctan.org/pkg/ltxdoc}
%
% \item[-] package \xpackage{holtxdoc}, 2012/03/21, v0.24,
%           \url{https://ctan.org/pkg/holtxdoc}
% \end{description}
%
% \DescribeMacro{footnoterange.sty}
% The |footnoterange.sty| for \LaTeXe{} (i.\,e. each document using
% the \xpackage{footnoterange} package) requires:
% \begin{description}
% \item[-] \TeX{} Format \LaTeXe{}, \url{https://www.CTAN.org/}
%
% \item[-] package \xpackage{ltxcmds}, 2011/04/18, v1.20,
%           \url{https://www.ctan.org/pkg/ltxcmds}
% \end{description}
%
% \DescribeMacro{footnoterange-example.tex}
% The |footnoterange-example.tex| requires the same files as all
% documents using the \xpackage{footnoterange} package and additionally:
% \begin{description}
% \item[-] class \xpackage{article}, 2007/10/19, v1.4h, from \xpackage{classes.dtx}:\newline
%           \url{https://www.ctan.org/tex-archive/macros/latex/base/}
%
% \item[-] package \xpackage{footnoterange}, 2019/11/24, v1.0c,\newline
%           \url{https://www.ctan.org/pkg/footnoterange}\newline
%           (Well, it is the example file for this package, and because you are
%            reading the documentation for the \xpackage{footnoterange} package,
%            it can be assumed that you already have some version of it --
%            is it the current one?)
%
% \item[-] package \xpackage{hyperref}, 2012/11/06, v6.83m,\newline
%           \url{https://www.ctan.org/pkg/hyperref}
% \end{description}
%
% \DescribeMacro{Oberdiek}
% \DescribeMacro{holtxdoc}
% \DescribeMacro{ltxcmds}
% \DescribeMacro{alphalph}
% All packages of \textsc{Heiko Oberdiek}'s bundle `oberdiek'
% (especially \xpackage{holtxdoc}, \xpackage{ltxcmds}, and \xpackage{alphalph})
% are also available in a TDS compliant ZIP archive:\newline
% \url{http://mirror.ctan.org/install/macros/latex/contrib/oberdiek.tds.zip}.\newline
% It is probably best to download and use this, because the packages in there
% should be both recent and compatible.
%
% \DescribeMacro{hyperref}
% \noindent \xpackage{hyperref} is not included in that bundle and needs to be
% downloaded separately,\newline
% \url{http://mirrors.ctan.org/install/macros/latex/contrib/hyperref.tds.zip}.
%
% \DescribeMacro{M\"{u}nch}
% A list of my packages can be found at \url{https://www.ctan.org/author/muench-hm}.
%
% \pagebreak
%
% \subsection{Package, unpacking TDS}
% \paragraph{Package.} This package is available on \url{https://www.CTAN.org}.
% \begin{description}
% \item[\url{http://mirror.ctan.org/macros/latex/contrib/footnoterange/footnoterange.dtx}]\hspace*{0.1cm}
%       The source file.
% \item[\url{http://mirror.ctan.org/macros/latex/contrib/footnoterange/footnoterange.pdf}]\hspace*{0.1cm}
%       The documentation.
% \item[\url{http://mirror.ctan.org/macros/latex/contrib/footnoterange/footnoterange-example.pdf}]\hspace*{0.1cm}
%       The compiled example file, as it should look like.
% \item[\url{http://mirror.ctan.org/macros/latex/contrib/footnoterange/README}]\hspace*{0.1cm}\newline\hspace*{0.1cm}
%       The README file.
% \end{description}
% There is also a footnoterange.tds.zip available:
% \begin{description}
% \item[\url{http://mirror.ctan.org/install/macros/latex/contrib/footnoterange.tds.zip}]\hspace*{0.1cm}
%       Everything in TDS compliant, compiled format.
% \end{description}
% which additionally contains\newline
% \begin{tabular}{ll}
% footnoterange.ins & The installation file.\\
% footnoterange.drv & The driver to generate the documentation.\\
% footnoterange.sty & The \xext{sty}le file.
% \end{tabular}
%
% \bigskip
%
% \noindent For required other packages, please see the preceding subsection.
%
% \paragraph{Unpacking.} The \xfile{.dtx} file is a self-extracting
% \docstrip{} archive. The files are extracted by running the
% \xfile{.dtx} through \plainTeX:
% \begin{quote}
%   \verb|tex footnoterange.dtx|
% \end{quote}
%
% About generating the documentation see paragraph~\ref{GenDoc} below.
%
% \paragraph{TDS.} Now the different files must be moved into
% the different directories in your installation TDS tree
% (also known as \xfile{texmf} tree):
% \begin{quote}
% \def\t{^^A
% \begin{tabular}{@{}>{\ttfamily}l@{ $\rightarrow$ }>{\ttfamily}l@{}}
%   footnoterange.sty & tex/latex/footnoterange/footnoterange.sty\\
%   footnoterange.pdf & doc/latex/footnoterange/footnoterange.pdf\\
%   footnoterange-example.tex & doc/latex/footnoterange/footnoterange-example.tex\\
%   footnoterange-example.pdf & doc/latex/footnoterange/footnoterange-example.pdf\\
%   footnoterange.dtx & source/latex/footnoterange/footnoterange.dtx\\
% \end{tabular}^^A
% }^^A
% \sbox0{\t}^^A
% \ifdim\wd0>\linewidth
%   \begingroup
%     \advance\linewidth by\leftmargin
%     \advance\linewidth by\rightmargin
%   \edef\x{\endgroup
%     \def\noexpand\lw{\the\linewidth}^^A
%   }\x
%   \def\lwbox{^^A
%     \leavevmode
%     \hbox to \linewidth{^^A
%       \kern-\leftmargin\relax
%       \hss
%       \usebox0
%       \hss
%       \kern-\rightmargin\relax
%     }^^A
%   }^^A
%   \ifdim\wd0>\lw
%     \sbox0{\small\t}^^A
%     \ifdim\wd0>\linewidth
%       \ifdim\wd0>\lw
%         \sbox0{\footnotesize\t}^^A
%         \ifdim\wd0>\linewidth
%           \ifdim\wd0>\lw
%             \sbox0{\scriptsize\t}^^A
%             \ifdim\wd0>\linewidth
%               \ifdim\wd0>\lw
%                 \sbox0{\tiny\t}^^A
%                 \ifdim\wd0>\linewidth
%                   \lwbox
%                 \else
%                   \usebox0
%                 \fi
%               \else
%                 \lwbox
%               \fi
%             \else
%               \usebox0
%             \fi
%           \else
%             \lwbox
%           \fi
%         \else
%           \usebox0
%         \fi
%       \else
%         \lwbox
%       \fi
%     \else
%       \usebox0
%     \fi
%   \else
%     \lwbox
%   \fi
% \else
%   \usebox0
% \fi
% \end{quote}
% If you have a \xfile{docstrip.cfg} that configures and enables \docstrip's
% TDS installing feature, then some files can already be in the right
% place, see the documentation of \docstrip.
%
% \subsection{Refresh file name databases}
%
% If your \TeX~distribution (\TeX{} Live, \mikTeX, \teTeX, \dots) relies on
% file name databases, you must refresh these. For example, \teTeX{} users run
% \verb|texhash| or \verb|mktexlsr|.
%
% \subsection{Some details for the interested}
%
% \paragraph{Unpacking with \LaTeX{}.}
% The \xfile{.dtx} chooses its action depending on the format:
% \begin{description}
% \item[\plainTeX:] Run \docstrip{} and extract the files.
% \item[\LaTeX:] Generate the documentation.
% \end{description}
% If you insist on using \LaTeX{} for \docstrip{} (really,
% \docstrip{} does not need \LaTeX ), then inform the autodetect routine
% about your intention:
% \begin{quote}
%   \verb|latex \let\install=y% \iffalse meta-comment
%
% File: footnoterange.dtx
% Version: 2019/11/24 v1.0c
%
% Copyright (C) 2012 - 2019 by
%    H.-Martin M"unch <Martin dot Muench at Uni-Bonn dot de>
% Portions of code copyrighted by other people as marked;
% lines marked "FMi": code by Frank Mittelbach, thanks!
%
% This work may be distributed and/or modified under the
% conditions of the LaTeX Project Public License, either
% version 1.3c of this license or (at your option) any later
% version. This version of this license is in
%    https://www.latex-project.org/lppl/lppl-1-3c.txt
% and the latest version of this license is in
%    https://www.latex-project.org/lppl.txt
% and version 1.3c or later is part of all distributions of
% LaTeX version 2005/12/01 or later.
%
% This work has the LPPL maintenance status "maintained".
%
% The Current Maintainer of this work is H.-Martin Muench.
%
% This work consists of the main source file footnoterange.dtx,
% the README, and the derived files
%    footnoterange.sty, footnoterange.pdf,
%    footnoterange.ins, footnoterange.drv,
%    footnoterange-example.tex, footnoterange-example.pdf.
%
% 'footnoterange' is available on CTAN:
% https://www.ctan.org/pkg/footnoterange
%
% Also a TDS.ZIP file is provided that contains all the files
% already sorted in a TDS tree:
% http://mirror.ctan.org/install/macros/latex/contrib/footnoterange.tds.zip
%
% Unpacking:
%    (a) If footnoterange.ins is present:
%           tex footnoterange.ins
%    (b) Without footnoterange.ins:
%           tex footnoterange.dtx
%    (c) If you insist on using LaTeX
%           latex \let\install=y\input{footnoterange.dtx}
%        (quote the arguments according to the demands of your shell)
%
% Documentation:
%    (a) If footnoterange.drv is present:
%           (pdf)latex footnoterange.drv
%           makeindex -s gind.ist footnoterange.idx
%           (pdf)latex footnoterange.drv
%           makeindex -s gind.ist footnoterange.idx
%           (pdf)latex footnoterange.drv
%    (b) Without footnoterange.drv:
%           (pdf)latex footnoterange.dtx
%           makeindex -s gind.ist footnoterange.idx
%           (pdf)latex footnoterange.dtx
%           makeindex -s gind.ist footnoterange.idx
%           (pdf)latex footnoterange.dtx
%
%    The class ltxdoc loads the configuration file ltxdoc.cfg
%    if available. Here you can specify further options, e.g.
%    use DIN A4 as paper format:
%       \PassOptionsToClass{a4paper}{article}
%
% Installation:
%    TDS:tex/latex/footnoterange/footnoterange.sty
%    TDS:doc/latex/footnoterange/footnoterange.pdf
%    TDS:doc/latex/footnoterange/footnoterange-example.tex
%    TDS:doc/latex/footnoterange/footnoterange-example.pdf
%    TDS:source/latex/footnoterange/footnoterange.dtx
%
%<*ignore>
\begingroup
  \catcode123=1 %
  \catcode125=2 %
  \def\x{LaTeX2e}%
\expandafter\endgroup
\ifcase 0\ifx\install y1\fi\expandafter
         \ifx\csname processbatchFile\endcsname\relax\else1\fi
         \ifx\fmtname\x\else 1\fi\relax
\else\csname fi\endcsname
%</ignore>
%<*install>
\input docstrip.tex
\Msg{*********************************************************************************}
\Msg{* Installation}
\Msg{* Package: footnoterange 2019/11/24 v1.0c References to ranges of footnotes (HMM)}
\Msg{*********************************************************************************}

\keepsilent
\askforoverwritefalse

\let\MetaPrefix\relax
\preamble

This is a generated file.

Project: footnoterange
Version: 2019/11/24 v1.0c

Copyright (C) 2012 - 2019 by
    H.-Martin M"unch <Martin dot Muench at Uni-Bonn dot de>
Portions of code copyrighted by other people as marked;
lines marked "FMi": code by Frank Mittelbach, thanks!

The usual disclaimer applies:
If it doesn't work right that's your problem.
(Nevertheless, please send an e-mail to the maintainer
 when you find an error in this package.)

This work may be distributed and/or modified under the
conditions of the LaTeX Project Public License, either
version 1.3c of this license or (at your option) any later
version. This version of this license is in
   https://www.latex-project.org/lppl/lppl-1-3c.txt
and the latest version of this license is in
   https://www.latex-project.org/lppl.txt
and version 1.3c or later is part of all distributions of
LaTeX version 2005/12/01 or later.

This work has the LPPL maintenance status "maintained".

The Current Maintainer of this work is H.-Martin Muench.

This work consists of the main source file footnoterange.dtx,
the README, and the derived files
   footnoterange.sty, footnoterange.pdf,
   footnoterange.ins, footnoterange.drv,
   footnoterange-example.tex, footnoterange-example.pdf.

In memoriam
 Claudia Simone Barth + 1996/01/30
 Tommy Muench + 2014/01/02
 Hans-Klaus Muench + 2014/08/24

\endpreamble
\let\MetaPrefix\DoubleperCent

\generate{%
  \file{footnoterange.ins}{\from{footnoterange.dtx}{install}}%
  \file{footnoterange.drv}{\from{footnoterange.dtx}{driver}}%
  \usedir{tex/latex/footnoterange}%
  \file{footnoterange.sty}{\from{footnoterange.dtx}{package}}%
  \usedir{doc/latex/footnoterange}%
  \file{footnoterange-example.tex}{\from{footnoterange.dtx}{example}}%
}

\catcode32=13\relax% active space
\let =\space%
\Msg{************************************************************************}
\Msg{*}
\Msg{* To finish the installation you have to move the following}
\Msg{* file into a directory searched by TeX:}
\Msg{*}
\Msg{* footnoterange.sty}
\Msg{*}
\Msg{* To produce the documentation run the file `footnoterange.drv'}
\Msg{* through (pdf)LaTeX, e.g.}
\Msg{*  pdflatex footnoterange.drv}
\Msg{*  makeindex -s gind.ist footnoterange.idx}
\Msg{*  pdflatex footnoterange.drv}
\Msg{*  makeindex -s gind.ist footnoterange.idx}
\Msg{*  pdflatex footnoterange.drv}
\Msg{*}
\Msg{* At least three runs are necessary e.g. to get the}
\Msg{*  references right!}
\Msg{*}
\Msg{* Happy TeXing!}
\Msg{*}
\Msg{************************************************************************}

\endbatchfile
%</install>
%<*ignore>
\fi
%</ignore>
%
% \section{The documentation driver file}
%
% The next bit of code contains the documentation driver file for
% \TeX , i.\,e., the file that will produce the documentation you
% are currently reading. It will be extracted from this file by the
% \verb|docstrip| programme. That is, run \LaTeX{} on \verb|docstrip|
% and specify the \verb|driver| option when \verb|docstrip|
% asks for options.
%
%    \begin{macrocode}
%<*driver>
\NeedsTeXFormat{LaTeX2e}[2015/01/01]
\ProvidesFile{footnoterange.drv}[2019/11/24 v1.0c
            References to ranges of footnotes (HMM)]
\documentclass{ltxdoc}[2015/03/26]% v2.0w
\usepackage{holtxdoc}[2012/03/21]%  v0.24
%% footnoterange may work with earlier versions of LaTeX2e and those
%% class and package, but this was not tested.
%% Please consider updating your LaTeX, class, and package
%% to the most recent version (if they are not already the most
%% recent version).
\hypersetup{%
 pdfsubject={References to ranges of footnotes (HMM)},%
 pdfkeywords={LaTeX, footnoterange, footnotes, footnote, reference, H.-Martin Muench},%
 pdfencoding=auto,%
 pdflang={en},%
 breaklinks=true,%
 linktoc=all,%
 pdfstartview=FitH,%
 pdfpagelayout=OneColumn,%
 bookmarksnumbered=true,%
 bookmarksopen=true,%
 bookmarksopenlevel=3,%
 pdfmenubar=true,%
 pdftoolbar=true,%
 pdfwindowui=true,%
 pdfnewwindow=true%
}
\CodelineIndex
\hyphenation{printing docu-ment}
\gdef\unit#1{\mathord{\thinspace\mathrm{#1}}}
\begin{document}
  \DocInput{footnoterange.dtx}%
\end{document}
%</driver>
%    \end{macrocode}
%
% \fi
%
% \CheckSum{321}
%
% \CharacterTable
%  {Upper-case    \A\B\C\D\E\F\G\H\I\J\K\L\M\N\O\P\Q\R\S\T\U\V\W\X\Y\Z
%   Lower-case    \a\b\c\d\e\f\g\h\i\j\k\l\m\n\o\p\q\r\s\t\u\v\w\x\y\z
%   Digits        \0\1\2\3\4\5\6\7\8\9
%   Exclamation   \!     Double quote  \"     Hash (number) \#
%   Dollar        \$     Percent       \%     Ampersand     \&
%   Acute accent  \'     Left paren    \(     Right paren   \)
%   Asterisk      \*     Plus          \+     Comma         \,
%   Minus         \-     Point         \.     Solidus       \/
%   Colon         \:     Semicolon     \;     Less than     \<
%   Equals        \=     Greater than  \>     Question mark \?
%   Commercial at \@     Left bracket  \[     Backslash     \\
%   Right bracket \]     Circumflex    \^     Underscore    \_
%   Grave accent  \`     Left brace    \{     Vertical bar  \|
%   Right brace   \}     Tilde         \~}
%
% \GetFileInfo{footnoterange.drv}
%
% \begingroup
%   \def\x{\#,\$,\^,\_,\~,\ ,\&,\{,\},\%}%
%   \makeatletter
%   \@onelevel@sanitize\x
% \expandafter\endgroup
% \expandafter\DoNotIndex\expandafter{\x}
% \expandafter\DoNotIndex\expandafter{\string\ }
% \begingroup
%   \makeatletter
%     \lccode`9=32\relax
%     \lowercase{%^^A
%       \edef\x{\noexpand\DoNotIndex{\@backslashchar9}}%^^A
%     }%^^A
%   \expandafter\endgroup\x
% \DoNotIndex{\,,\\}
% \DoNotIndex{\documentclass,\usepackage,\ProvidesPackage,\begin,\end}
% \DoNotIndex{\MessageBreak}
% \DoNotIndex{\NeedsTeXFormat,\DoNotIndex,\verb}
% \DoNotIndex{\def,\edef,\gdef,\global,\xdef}
% \DoNotIndex{\ifx,\listfiles,\mathord,\mathrm}
% \DoNotIndex{\bigskip,\space,\thinspace,\ldots}
% \DoNotIndex{\indent,\noindent,\newline,\linebreak,\pagebreak,\newpage}
% \DoNotIndex{\textbf,\textit,\textsf,\textsc,\texttt,\textquotedblleft,\textquotedblright}
% \DoNotIndex{\plainTeX,\TeX,\LaTeX,\pdfLaTeX}
% \DoNotIndex{\section,\pagenumbering,\smallskip}
% \DoNotIndex{\@ifpackageloaded}
% \DoNotIndex{\arabic, \Alph, \alph, \alphalph, \Roman, \roman, \fnsymbol}
%
% \title{The \xpackage{footnoterange} package}
% \date{2019/11/24 v1.0c}
% \author{H.-Martin M\"{u}nch\\\xemail{Martin.Muench at Uni-Bonn.de}}
%
% \maketitle
%
% \begin{abstract}
% \noindent This package provides the environments |footnoterange| and
% |footnoterange*|. Multiple footnotes inside these environments
% are not referenced as e.\,g.
% \makeatletter%
% \hbox{\textquotedblleft \@textsuperscript {\normalfont 1 2 3}\textquotedblright}
% but as \hbox{\textquotedblleft \@textsuperscript {\normalfont 1--3}\textquotedblright ,\ }
% \makeatother%
% i.\,e. the range (from first to last referred footnote at that place) is given.
% Depending on loading of \xpackage{hyperref} and use of its |hyperfootnotes|-option
% the references are hyperlinked. (References to footnotes in |footnoterange*|
% are never hyperlinked.)
% \end{abstract}
%
% \bigskip
%
% \noindent Disclaimer for web links: The author is not responsible for any contents
% referred to in this work unless if having full knowledge of illegal contents.
% If any damage occurs by the use of information presented there, only the
% author of the respective pages might be liable, not the one who has referred
% to these pages.
%
% \noindent Save per page about $200\unit{ml}$ water,
% $2\unit{g}$ CO$_{2}$ and $2\unit{g}$ wood:
% Therefore please print only if this is really necessary.
%
% \medskip
%
% \tableofcontents
%
%\newpage
%
% \section{Introduction}
% \indent This \LaTeXe{} package provides the environments |footnoterange| and\newline
% |footnoterange*|. Multiple footnotes inside these environments are not referenced
% as e.\,g.
% \makeatletter%
% \hbox{\textquotedblleft \@textsuperscript {\normalfont 1 2 3}\textquotedblright}
% but as \hbox{\textquotedblleft \@textsuperscript {\normalfont 1--3}\textquotedblright ,\ }
% \makeatother%
% i.\,e. the range (from first to last
% referred footnote at that place) is given. Depending on loading of
% \xpackage{hyperref} and use of its |hyperfootnotes|-option the references are
% hyperlinked. References to footnotes in |footnoterange*| are never hyperlinked.\newline
% |\footnotemark\footnotetext| instead of |\footnote| is not supported inside
% |footnoterange| and |footnoterange*| environments.
%
% \bigskip
%
% \section{Usage}
%
% \indent Just load the package placing
% \begin{quote}
%   |\usepackage{footnoterange}|
% \end{quote}
% \noindent in the preamble of your \LaTeXe{} source file. There are no options to
% be used. The package should be loaded after \xpackage{hyperref} (if that package is used)
% and after any other package redefining footnotes. Usage of the |footnoterange| (or
% |footnoterange*|) environment is simple: Just encapsulate the footnotes, which shall
% be referred as range instead of referencing as single ones (see the example file).
%
% About hyperlinks:\newline
% Footnote-references in a |footnoterange*| environment are never hyperlinked.\newline
% Footnote-references in a |footnoterange| environment may be hyperlinked:
% \begin{itemize}
%  \item When the \xpackage{hyperref} package is not loaded, they are not
%         hyperlinked (naturally).
%  \item When the \xpackage{hyperref} package is loaded, it depends on the
%         |hyperfootnotes| option of the \xpackage{hyperref} package:
%  \begin{itemize}
%   \item With |hyperfootnotes=true|, the references are hyperlinked.
%   \item With |hyperfootnotes=false|, the references are not hyperlinked.
%  \end{itemize}
% \end{itemize}
% |\footnote|s inside of the |footnoterange|/{}|footnoterange*| environment
% can use the optional parameter, i.\,e. |\footnote[42]{text}| even if this is the
% third footnote. When giving ranges as reference, this does not seem to be
% such a good idea, if this is not done consistently (i.\,e. footnotes
% |[42], [43], [44]| resulting in 42--44 is OK,
% but |[42]| (original~2), unchanged~3, unchanged~4 resulting in
% 42--4 is confusing). A warning about these changed footnotemarks is written
% into the \xfile{.log}-file.
%
% When there is only a single footnote in a |footnoterange| environment,
% this does not produce some reference like %
% \makeatletter%
% \hbox{\textquotedblleft \@textsuperscript {\normalfont 7--7}\textquotedblright{},\ }%
% \makeatother%
% but only one footnotemark is used automatically. Because it is strange to use
% a |footnoterange| environment without a range of footnotes, an info is written
% into the \xfile{.log}-file about this.
%
% The usage of other footnote-number-representations
% (e.\,g.~\verb|\Roman|, \verb|\roman|, \verb|\Alph|, \verb|\alph|,
% \verb|\fnsymbol|) is also possible, but two things must be taken
% into account: The number of footnote references is restricted (for
% example with \verb|\Alph| only references A to Z are possible), which
% can be fixed e.\,g. with the \url{https://ctan.org/pkg/alphalph} package
% (but not automatically), and references to footnote-symbol-ranges (\verb|\fnsymbol|)
% are probably not very clear.
%
% \subsection*{Options}
% \indent The \xpackage{footnoterange} package takes no options.
%
% \section{Alternatives\label{sec:Alternatives}}
%
% The \xpackage{fnpct} (\url{https://www.ctan.org/pkg/fnpct}) package allows to set
% multiple footnotes (|\multfootnote|), suppress a footnotemark \hbox{(|*|),}
% and customize the multi-footnote-separator \hbox{(|\setfnpct{mult-fn-sep={|
% \emph{something} |}}|)}.
% Combining these with stepping of the footnotecounter gives a footnoterange:
% 
% \begin{verbatim}
% \documentclass{article}
% \usepackage{hyperref}
% \usepackage{fnpct}
% \begin{document}
% The three little pigs built their houses
% out of straw\footnote{not to be confused with hay},
% sticks\footnote{or lumber according to some sources}
% and bricks%
% \makeatletter%
% \setfnpct{mult-fn-sep={--}}%
% \multfootnote{probably fired clay bricks\addtocounter\@mpfn{+1};%
% *or something else;%
% what do I know}. %
% \makeatother%
% \end{document}
% \end{verbatim}
% 
% \noindent If you programmed or found another alternative, which is available at
% \url{https://www.CTAN.org/}, please send an e-mail to me with the name,
% location at CTAN, and a short notice, and I will probably list it here.
%
% \section{Example}
%
%    \begin{macrocode}
%<*example>
\documentclass[british]{article}[2014/09/29]% v1.4h
%%%%%%%%%%%%%%%%%%%%%%%%%%%%%%%%%%%%%%%%%%%%%%%%%%%%%%%%%%%%%%%%%%%%%
\usepackage[%
 extension=pdf,%
 plainpages=false,%
 pdfpagelabels=true,%
 hyperindex=false,%
 pdflang={en},%
 pdftitle={footnoterange package example},%
 pdfauthor={H.-Martin Muench},%
 pdfsubject={Example for the footnoterange package},%
 pdfkeywords={LaTeX, footnoterange, H.-Martin Muench},%
 pdfview=FitH,%
 pdfstartview=FitH,%
 pdfpagelayout=OneColumn,%
 hyperfootnotes=true%
]{hyperref}[2012/11/06]% v6.83m

\usepackage{footnoterange}[2019/11/24]% v1.0c

\gdef\unit#1{\mathord{\thinspace\mathrm{#1}}}%
\listfiles
\begin{document}
\pagenumbering{arabic}
\section*{Example for footnoterange}

This example demonstrates the use of package\newline
\textsf{footnoterange}, v1.0c as of 2019/11/24 (HMM).\newline
The package does not use options.\newline
\verb|footnoterange-example.pdf| can be created by\newline
\verb|pdflatex footnoterange-example.tex|,\newline
and \verb|footnoterange-example.tex| can be created by\newline
\verb|tex footnoterange.dtx|.\newline
\verb|footnoterange.dtx| is at
\url{https://www.ctan.org/pkg/footnoterange}.\newline

\noindent For more details please see the documentation!\newline

\noindent Save per page about $200\unit{ml}$ water,
$2\unit{g}$ CO$_{2}$ and $2\unit{g}$ wood:\newline
Therefore please print only if this is really necessary.\newline

This\footnote{Lorem} text bears a multiplicity of footnotes.
Because the \verb|hyperref| package is used with option
\verb|hyperfootnotes=true|, the footnotes
\begin{footnoterange}
\footnote{ipsum}%
\footnote{dolor}%
\footnote{sit}%
\footnote{amet,}%
\footnote{consectetur}%
\footnote{adipisicing}%
\footnote{elit,}%
\footnote{sed}%
\end{footnoterange}
are hyperlinked.\footnote{do eiusmod\ldots} Using
\verb|hyperfootnotes=false| or not using \verb|hyperref|
will remove the hyperlinks to the footnotes.\medskip

This text also bears a multiplicity of footnotes,
but due to the use of the starred
\begin{footnoterange*}
\footnote{Ut purus elit,}%
\footnote{vestibulum ut,}%
\footnote{placerat ac,}%
\footnote{adipiscing vitae,}%
\footnote{felis.}%
\footnote{Curabitur}%
\footnote{dictum}%
\footnote{gravida}%
\end{footnoterange*}
form of the environment they are not
hyperlinked.\footnote{mauris\ldots}\medskip

The use of the optional parameter for footnotes
\begin{footnoterange}
\footnote[101]{Nam arcu libero,}%
\footnote[102]{nonummy eget,}%
\footnote[103]{consectetuer id,}%
\footnote[104]{vulputate a,}%
\footnote[105]{magna.}
\end{footnoterange}
is also possible, but should be used consistently.

\pagebreak

When there is only one footnote
\begin{footnoterange}
\footnote{a single footnote}
\end{footnoterange}
in the \verb|footnoterange| environment, only one
footnotemark is used and an info is written into the
\verb|log|-file.\medskip

The usage of other footnote-number-representations
(e.\,g.~\verb|\Roman|, \verb|\roman|, \verb|\Alph|, \verb|\alph|,
\verb|\fnsymbol|) is also possible, but two things must be taken
into account: The number of footnote references is restricted
(for example with \verb|\Alph| only references A to Z are possible),
which can be fixed e.\,g. with the
\href{https://www.ctan.org/pkg/alphalph}{\texttt{alphalph} package},
and references to footnote-symbol-ranges (\verb|\fnsymbol|)
are probably not very clear.

\end{document}
%</example>
%    \end{macrocode}
%
% \StopEventually{}
%
% \section{The implementation}
%
% We start off by checking that we are loading into \LaTeXe{} and
% announcing the name and version of this package.
%
%    \begin{macrocode}
%<*package>
%    \end{macrocode}
%
%    \begin{macrocode}
\NeedsTeXFormat{LaTeX2e}[2011/06/27]
\ProvidesPackage{footnoterange}[2019/11/24 v1.0c
            References to ranges of footnotes (HMM)]
%    \end{macrocode}
%
% A short description of the \xpackage{footnoterange} package:
%
%    \begin{macrocode}
%% Provides the footnoterange and footnoterange* environments to
%% refer to ranges of footnotes.
%    \end{macrocode}
%
% We need the \xpackage{ltxcmds} package:
%
%    \begin{macrocode}
\RequirePackage{ltxcmds}[2011/11/09]% v1.22; for "\ltx@ifpackageloaded"
\RequirePackage{letltxmacro}[2010/09/02]% v1.4; for "\GlobalLetLtxMacro"

%    \end{macrocode}
%
% We use |\ifHy@hyperfootnotes ... \else ... \fi|, but when the \xpackage{hyperref}
% package has not been loaded, |\ifHy@hyperfootnotes| is not defined, and \LaTeX
% relates the |\else| and/or |\fi| (wrongly) to other |\if|s or groups.
% Therefore the |\ifHy@hyperfootnotes| are placed in their own commands,
% which are not looked at when the \xpackage{hyperref} package has not been loaded.
%
% \pagebreak
%
% \begin{macro}{\fnr@ifhyperrfn}
%    \begin{macrocode}
\newcommand{\fnr@ifhyperrfn}{%
  \ifHy@hyperfootnotes% option hyperfootnotes=true
    \addtocounter{Hfootnote}{+1}%
    % Code from the hyperref package
      \global\let\Hy@saved@currentHref\@currentHref%
      \hyper@makecurrent{Hfootnote}%
      \global\let\Hy@footnote@currentHref\@currentHref%
      \global\let\@currentHref\Hy@saved@currentHref%
    % End of code from the hyperref package
  \fi%
}

%    \end{macrocode}
% \end{macro}
%
% \begin{macro}{\fnr@rangefootnote}
% The |\fnr@rangefootnote| is the internal replacement for the original\newline
% |\footnote| command inside of a |footnoterange| environment.
%
%    \begin{macrocode}
\newcommand{\fnr@rangefootnote}[2][\empty]{%
%    \end{macrocode}
%
% |#1| contains the optional argument, which is predefined as |\empty|.
%
%    \begin{macrocode}
  \nolinebreak%
  % no \footnotemark here
  \addtocounter\@mpfn{+1}%
  \ltx@ifpackageloaded{hyperref}{\fnr@ifhyperrfn}{\relax}%
%    \end{macrocode}
%
% See the explanation before the declaration of the |\fnr@ifhyperrfn| command!
%
%    \begin{macrocode}
  \xdef\fnr@opt{#1}% contains the optional argument
  \xdef\fnr@arabic{\arabic\@mpfn}%
  \protected@edef\fnr@formated{\thefootnote}%
%    \end{macrocode}
%
% The |\footnotemark| is not set (yet), but the |\footnotetext|,
% and it gets a label.
%
%    \begin{macrocode}
  \ifx\fnr@opt\empty%
    \footnotetext{\label{fnr:\fnr@arabic}#2}%
  \else%
    \PackageWarning{footnoterange}{%
      Custom footnote-numbers used in a footnoterange\MessageBreak%
      environment: Footnote with natural number\MessageBreak%
      \ifx\fnr@formated\fnr@arabic%
        \fnr@arabic\space%
      \else%
        \space\fnr@arabic\space (\fnr@formated )%
      \fi%
      \MessageBreak%
      was renumbered to \fnr@opt\MessageBreak%
     }%
    \ltx@ifpackageloaded{hyperref}{% hyperref loaded
      \footnotetext[#1]{\phantomsection\label{fnr:\fnr@arabic}#2}%
     }{% hyperref not loaded
      \footnotetext[#1]{\label{fnr:\fnr@arabic}#2}%
     }%
  \fi%
}

%    \end{macrocode}
% \end{macro}
%
%    \begin{macrocode}

%    \end{macrocode}
% \pagebreak
%
% |\ifHy@hyperfootnotes| \textquotedblleft encapsulation\textquotedblright{} again:
%
% \begin{macro}{\fnr@ifhyperfnrA}
%    \begin{macrocode}
\newcommand{\fnr@ifhyperfnrA}{%
  \ifHy@hyperfootnotes% option hyperfootnotes=true
    \hbox{\hyperref[fnr:\fnr@first]{%
          \@textsuperscript{\normalfont \ref*{fnr:\fnr@first}}}}%
  \else% option hyperfootnotes=false
    \hbox{\@textsuperscript{\normalfont \ref*{fnr:\fnr@first}}}%
  \fi%
}

%    \end{macrocode}
% \end{macro}
%
% \begin{macro}{\fnr@ifhyperfnrN}
%    \begin{macrocode}
\newcommand{\fnr@ifhyperfnrN}{%
  \ifHy@hyperfootnotes% option hyperfootnotes=true
    \hbox{\hyperref[fnr:\fnr@first]{%
            \@textsuperscript{\normalfont \ref*{fnr:\fnr@first}}}%
          \@textsuperscript{\normalfont --}%
          \hyperref[fnr:\arabic\@mpfn]{%
            \@textsuperscript{\normalfont \ref*{fnr:\arabic\@mpfn}}}}%
  \else% option hyperfootnotes=false
    \hbox{\@textsuperscript{\normalfont \ref*{fnr:\fnr@first}%
                                        --\ref*{fnr:\arabic\@mpfn}}}%
  \fi%
}

%    \end{macrocode}
% \end{macro}
%
% \begin{environment}{footnoterange}
% The |footnoterange| environment remembers the next footnote number as\newline
% |\fnr@first| and the current |\footnote| definition as |\fnr@origfn|
% and uses |\fnr@rangefootnote| instead.
%
%    \begin{macrocode}
\newenvironment{footnoterange}{%
  % at begin of the footnoterange environment
  \ifhmode \unskip \fi% FMi
  \addtocounter\@mpfn{+1}%
  \xdef\fnr@first{\arabic\@mpfn}%
  \addtocounter\@mpfn{-1}%
  \let\fnr@origfn\footnote%
  \let\footnote\fnr@rangefootnote%
  \GlobalLetLtxMacro{\fnr@origfnmark}{\footnotemark}%
  \GlobalLetLtxMacro{\footnotemark}{\fnr@rangefootnotemark}%
  \ignorespaces% FMi
%    \end{macrocode}
%
% End of the environment:
%
%    \begin{macrocode}
}{% at end of the footnoterange environment
  \ifhmode% FMi 
    \unskip% FMi
%    \end{macrocode}
%
% Remember the spacefactor for restoring after the environment has ended
% (only in |hmode|!). \% FMi
%
%    \begin{macrocode}
    \xdef\fnr@aftergroup{\spacefactor\the\spacefactor\relax}% FMi
  \else% FMi
%    \end{macrocode}
%
% Otherwise do something harmless. \% FMi
%
%    \begin{macrocode}
    \xdef\fnr@aftergroup{\relax}%
  \fi% FMi
%    \end{macrocode}
%
% At the end of the environment the previous definition of |\footnote| is restored
%
%    \begin{macrocode}
  \let\footnote\fnr@origfn%
  \GlobalLetLtxMacro{\footnotemark}{\fnr@origfnmark}%
%    \end{macrocode}
%
% and the references are placed:
%
%    \begin{macrocode}
  \ifx\fnr@first\fnr@arabic% There is only one reference in the environment.
    \PackageInfo{footnoterange}{%
      Environment footnoterange used,\MessageBreak%
      but only one footnote (number \fnr@first ) is contained%
      \MessageBreak%
     }%
    \ltx@ifpackageloaded{hyperref}{\fnr@ifhyperfnrA% hyperref package loaded
    }{% hyperref package not loaded
      \hbox{\@textsuperscript{\normalfont \ref{fnr:\fnr@first}}}%
     }%
  \else%
    \ltx@ifpackageloaded{hyperref}{\fnr@ifhyperfnrN% hyperref package loaded
    }{% hyperref package not loaded
      \hbox{\@textsuperscript{\normalfont \ref{fnr:\fnr@first}%
                                        --\ref{fnr:\arabic\@mpfn}}}%
     }%
  \fi%
%    \end{macrocode}
%
% Restore the spacefactor after the environment has ended. \% FMi
%
%    \begin{macrocode}
  \aftergroup\fnr@aftergroup% FMi
}

%    \end{macrocode}
% \end{environment}
%
% \begin{environment}{footnoterange*}
% The |footnoterange*| environment does the same as the |footnoterange|
% environment but suppresses all hyperlinks to footnotes inside.
%
%    \begin{macrocode}
\newenvironment{footnoterange*}{%
  % at begin of the footnoterange* environment
  \ifhmode \unskip \fi% FMi
  \addtocounter\@mpfn{+1}%
  \xdef\fnr@first{\arabic\@mpfn}%
  \addtocounter\@mpfn{-1}%
  \let\fnr@origfn\footnote%
  \let\footnote\fnr@rangefootnote%
  \GlobalLetLtxMacro{\fnr@origfnmark}{\footnotemark}%
  \GlobalLetLtxMacro{\footnotemark}{\fnr@rangefootnotemark}%
  \ignorespaces% FMi
}{% at end of the footnoterange* environment
  \ifhmode% FMi 
    \unskip% FMi
    \xdef\fnr@aftergroup{\spacefactor\the\spacefactor\relax}% FMi
  \else% FMi
    \xdef\fnr@aftergroup{\relax}%
  \fi% FMi
  \let\footnote\fnr@origfn%
  \GlobalLetLtxMacro{\footnotemark}{\fnr@origfnmark}%
  \ifx\fnr@first\fnr@arabic% There is only one reference in the environment.
    \PackageInfo{footnoterange}{%
      Environment footnoterange used,\MessageBreak%
      but only one footnote (number \fnr@first ) is contained%
      \MessageBreak%
     }%
%    \end{macrocode}
% \pagebreak
%    \begin{macrocode}
    \ltx@ifpackageloaded{hyperref}{% hyperref package loaded
      \hbox{\@textsuperscript{\normalfont \ref*{fnr:\fnr@first}}}%
    }{% hyperref package not loaded
      \hbox{\@textsuperscript{\normalfont \ref{fnr:\fnr@first}}}%
     }%
  \else%
    \ltx@ifpackageloaded{hyperref}{% hyperref package loaded
      \hbox{\@textsuperscript{\normalfont \ref*{fnr:\fnr@first}%
                                        --\ref*{fnr:\arabic\@mpfn}}}%
    }{% hyperref package not loaded
      \hbox{\@textsuperscript{\normalfont \ref{fnr:\fnr@first}%
                                        --\ref{fnr:\arabic\@mpfn}}}%
     }%
  \fi%
  \aftergroup\fnr@aftergroup% FMi
}

%    \end{macrocode}
% \end{environment}
%
% The \xpackage{footnoterange} package does not handle |\footnotemark|s,
% but at least the user is made aware of this:
%
% \begin{macro}{\fnr@rangefootnotemark}
%    \begin{macrocode}

\newcommand{\fnr@rangefootnotemark}[1][\empty]{%
  \PackageError{footnoterange}{%
     \string\footnotemark\space not supported in footnoterange(*) environment%
   }{The footnoterange package does not handle a \string\footnotemark.\MessageBreak%
     Either use \string\footnote\string{...\string}\space instead of%
     \string\footnotemark\space and \string\footnotetext\string{...\string}\MessageBreak%
     in footnoterange and footnoterange* environments,\MessageBreak%
     or do not use those environments around a \string\footnotemark\MessageBreak%
     (or \string\renewcommand\string{\string\fnr@rangefootnotemark\string}\MessageBreak%
     with your own fixing code).%
    }%
  \xdef\fnr@mopt{#1}%
  \ifx\fnr@mopt\empty%
    \fnr@origfnmark%
  \else%
    \fnr@origfnmark[#1]%
  \fi%
}

%    \end{macrocode}
% \end{macro}
%
%    \begin{macrocode}
%</package>
%    \end{macrocode}
%
% \pagebreak
%
% \section{Installation}
%
% \subsection{Downloads\label{ss:Downloads}}
%
% Everything is available on CTAN,
% \url{https://www.ctan.org/tex-archive/},
% but may need additional packages themselves.
%
% \DescribeMacro{footnoterange.dtx}
% For unpacking the \xfile{footnoterange.dtx} file
% and constructing the documentation it is required:
% \begin{description}
% \item[-] \TeX{} Format \LaTeXe{}, \url{https://www.CTAN.org/}
%
% \item[-] document class \xpackage{ltxdoc}, 2015/03/26, v2.0w,
%           \url{https://ctan.org/pkg/ltxdoc}
%
% \item[-] package \xpackage{holtxdoc}, 2012/03/21, v0.24,
%           \url{https://ctan.org/pkg/holtxdoc}
% \end{description}
%
% \DescribeMacro{footnoterange.sty}
% The |footnoterange.sty| for \LaTeXe{} (i.\,e. each document using
% the \xpackage{footnoterange} package) requires:
% \begin{description}
% \item[-] \TeX{} Format \LaTeXe{}, \url{https://www.CTAN.org/}
%
% \item[-] package \xpackage{ltxcmds}, 2011/04/18, v1.20,
%           \url{https://www.ctan.org/pkg/ltxcmds}
% \end{description}
%
% \DescribeMacro{footnoterange-example.tex}
% The |footnoterange-example.tex| requires the same files as all
% documents using the \xpackage{footnoterange} package and additionally:
% \begin{description}
% \item[-] class \xpackage{article}, 2007/10/19, v1.4h, from \xpackage{classes.dtx}:\newline
%           \url{https://www.ctan.org/tex-archive/macros/latex/base/}
%
% \item[-] package \xpackage{footnoterange}, 2019/11/24, v1.0c,\newline
%           \url{https://www.ctan.org/pkg/footnoterange}\newline
%           (Well, it is the example file for this package, and because you are
%            reading the documentation for the \xpackage{footnoterange} package,
%            it can be assumed that you already have some version of it --
%            is it the current one?)
%
% \item[-] package \xpackage{hyperref}, 2012/11/06, v6.83m,\newline
%           \url{https://www.ctan.org/pkg/hyperref}
% \end{description}
%
% \DescribeMacro{Oberdiek}
% \DescribeMacro{holtxdoc}
% \DescribeMacro{ltxcmds}
% \DescribeMacro{alphalph}
% All packages of \textsc{Heiko Oberdiek}'s bundle `oberdiek'
% (especially \xpackage{holtxdoc}, \xpackage{ltxcmds}, and \xpackage{alphalph})
% are also available in a TDS compliant ZIP archive:\newline
% \url{http://mirror.ctan.org/install/macros/latex/contrib/oberdiek.tds.zip}.\newline
% It is probably best to download and use this, because the packages in there
% should be both recent and compatible.
%
% \DescribeMacro{hyperref}
% \noindent \xpackage{hyperref} is not included in that bundle and needs to be
% downloaded separately,\newline
% \url{http://mirrors.ctan.org/install/macros/latex/contrib/hyperref.tds.zip}.
%
% \DescribeMacro{M\"{u}nch}
% A list of my packages can be found at \url{https://www.ctan.org/author/muench-hm}.
%
% \pagebreak
%
% \subsection{Package, unpacking TDS}
% \paragraph{Package.} This package is available on \url{https://www.CTAN.org}.
% \begin{description}
% \item[\url{http://mirror.ctan.org/macros/latex/contrib/footnoterange/footnoterange.dtx}]\hspace*{0.1cm}
%       The source file.
% \item[\url{http://mirror.ctan.org/macros/latex/contrib/footnoterange/footnoterange.pdf}]\hspace*{0.1cm}
%       The documentation.
% \item[\url{http://mirror.ctan.org/macros/latex/contrib/footnoterange/footnoterange-example.pdf}]\hspace*{0.1cm}
%       The compiled example file, as it should look like.
% \item[\url{http://mirror.ctan.org/macros/latex/contrib/footnoterange/README}]\hspace*{0.1cm}\newline\hspace*{0.1cm}
%       The README file.
% \end{description}
% There is also a footnoterange.tds.zip available:
% \begin{description}
% \item[\url{http://mirror.ctan.org/install/macros/latex/contrib/footnoterange.tds.zip}]\hspace*{0.1cm}
%       Everything in TDS compliant, compiled format.
% \end{description}
% which additionally contains\newline
% \begin{tabular}{ll}
% footnoterange.ins & The installation file.\\
% footnoterange.drv & The driver to generate the documentation.\\
% footnoterange.sty & The \xext{sty}le file.
% \end{tabular}
%
% \bigskip
%
% \noindent For required other packages, please see the preceding subsection.
%
% \paragraph{Unpacking.} The \xfile{.dtx} file is a self-extracting
% \docstrip{} archive. The files are extracted by running the
% \xfile{.dtx} through \plainTeX:
% \begin{quote}
%   \verb|tex footnoterange.dtx|
% \end{quote}
%
% About generating the documentation see paragraph~\ref{GenDoc} below.
%
% \paragraph{TDS.} Now the different files must be moved into
% the different directories in your installation TDS tree
% (also known as \xfile{texmf} tree):
% \begin{quote}
% \def\t{^^A
% \begin{tabular}{@{}>{\ttfamily}l@{ $\rightarrow$ }>{\ttfamily}l@{}}
%   footnoterange.sty & tex/latex/footnoterange/footnoterange.sty\\
%   footnoterange.pdf & doc/latex/footnoterange/footnoterange.pdf\\
%   footnoterange-example.tex & doc/latex/footnoterange/footnoterange-example.tex\\
%   footnoterange-example.pdf & doc/latex/footnoterange/footnoterange-example.pdf\\
%   footnoterange.dtx & source/latex/footnoterange/footnoterange.dtx\\
% \end{tabular}^^A
% }^^A
% \sbox0{\t}^^A
% \ifdim\wd0>\linewidth
%   \begingroup
%     \advance\linewidth by\leftmargin
%     \advance\linewidth by\rightmargin
%   \edef\x{\endgroup
%     \def\noexpand\lw{\the\linewidth}^^A
%   }\x
%   \def\lwbox{^^A
%     \leavevmode
%     \hbox to \linewidth{^^A
%       \kern-\leftmargin\relax
%       \hss
%       \usebox0
%       \hss
%       \kern-\rightmargin\relax
%     }^^A
%   }^^A
%   \ifdim\wd0>\lw
%     \sbox0{\small\t}^^A
%     \ifdim\wd0>\linewidth
%       \ifdim\wd0>\lw
%         \sbox0{\footnotesize\t}^^A
%         \ifdim\wd0>\linewidth
%           \ifdim\wd0>\lw
%             \sbox0{\scriptsize\t}^^A
%             \ifdim\wd0>\linewidth
%               \ifdim\wd0>\lw
%                 \sbox0{\tiny\t}^^A
%                 \ifdim\wd0>\linewidth
%                   \lwbox
%                 \else
%                   \usebox0
%                 \fi
%               \else
%                 \lwbox
%               \fi
%             \else
%               \usebox0
%             \fi
%           \else
%             \lwbox
%           \fi
%         \else
%           \usebox0
%         \fi
%       \else
%         \lwbox
%       \fi
%     \else
%       \usebox0
%     \fi
%   \else
%     \lwbox
%   \fi
% \else
%   \usebox0
% \fi
% \end{quote}
% If you have a \xfile{docstrip.cfg} that configures and enables \docstrip's
% TDS installing feature, then some files can already be in the right
% place, see the documentation of \docstrip.
%
% \subsection{Refresh file name databases}
%
% If your \TeX~distribution (\TeX{} Live, \mikTeX, \teTeX, \dots) relies on
% file name databases, you must refresh these. For example, \teTeX{} users run
% \verb|texhash| or \verb|mktexlsr|.
%
% \subsection{Some details for the interested}
%
% \paragraph{Unpacking with \LaTeX{}.}
% The \xfile{.dtx} chooses its action depending on the format:
% \begin{description}
% \item[\plainTeX:] Run \docstrip{} and extract the files.
% \item[\LaTeX:] Generate the documentation.
% \end{description}
% If you insist on using \LaTeX{} for \docstrip{} (really,
% \docstrip{} does not need \LaTeX ), then inform the autodetect routine
% about your intention:
% \begin{quote}
%   \verb|latex \let\install=y\input{footnoterange.dtx}|
% \end{quote}
% Do not forget to quote the argument according to the demands
% of your shell.
%
% \paragraph{Generating the documentation.\label{GenDoc}}
% You can use both the \xfile{.dtx} or the \xfile{.drv} to generate
% the documentation. The process can be configured by a
% configuration file \xfile{ltxdoc.cfg}. For instance, put the following
% line into this file, if you want to have A4 as paper format:
% \begin{quote}
%   \verb|\PassOptionsToClass{a4paper}{article}|
% \end{quote}
%
% \noindent An example follows how to generate the
% documentation with \pdfLaTeX :
%
% \begin{quote}
%\begin{verbatim}
%pdflatex footnoterange.drv
%makeindex -s gind.ist footnoterange.idx
%pdflatex footnoterange.drv
%makeindex -s gind.ist footnoterange.idx
%pdflatex footnoterange.drv
%\end{verbatim}
% \end{quote}
%
% \subsection{Compiling the example}
%
% The example file, \textsf{footnoterange-example.tex}, can be compiled via\newline
% \indent \verb|(pdf)latex footnoterange-example.tex|\newline
% but will need probably two compiler runs to get everything right.\newline
% \verb|footnoterange-example.tex| can be created by\newline
% \indent \verb|tex footnoterange.dtx|.\newline
% \verb|footnoterange.dtx| is at
% \url{https://www.ctan.org/pkg/footnoterange}.\newline
%
% \section{Acknowledgements}
%
% I (H.-Martin M\"{u}nch) would like to thank \textsc{Heiko Oberdiek}
% for providing a~lot~(!) of useful packages (from
% which I also got everything I know about creating a file in \xext{dtx}
% format, ok, say it: copying), \textsc{Frank Mittelbach} for several bug reports,
% for code for improving the package, and for pointing me to the \xpackage{fnpct}
% package as alternative, Willie Wong for reporting a bug and its fix,
% everybody of the CTAN team for managing CTAN,
% the \Newsgroup{comp.text.tex} and \Newsgroup{de.comp.text.tex} newsgroups,
% and everybody at \url{https://tex.stackexchange.com/}
% for their help in all things \hbox{\TeX{}.}
% This package was created as response to question
% \textquotedblleft Compressing consecutive footnote
% marks\textquotedblright{} by user001\newline
% (\url{https://tex.stackexchange.com/users/10127/user001}) on\newline
% \url{https://tex.stackexchange.com/q/44165/6865}.
%
% \newpage
% \phantomsection
% \begin{History}\label{History}
%   \begin{Version}{2011/02/11 v0.1ff.}
%     \item Some versions at \url{https://tex.stackexchange.com/q/44165/6865}.
%   \end{Version}
%   \begin{Version}{2012/02/17 v1.0a}
%     \item Package created and uploaded to CTAN.
%   \end{Version}
%   \begin{Version}{2018/10/15 v1.0b}
%     \item Fixed/updated minor details.
%     \item Bug fix: Did not work without \xpackage{hyperref}, wrong spacefactor.
%     \item Listing the \href{https://www.ctan.org/pkg/fnpct}{\xpackage{fnpct}}
%           package as alternative.
%   \end{Version}
%   \begin{Version}{2019/11/24 v1.0c}
%     \item Bug fix: Issue with footnote symbols.
%   \end{Version}
% \end{History}
%
% \bigskip
%
% When you find a mistake or have a suggestion for an improvement of this package,
% please send an e-mail to the maintainer, thanks! (Please see BUG REPORTS in the README.)
%
% \PrintIndex
%
% \Finale
\endinput|
% \end{quote}
% Do not forget to quote the argument according to the demands
% of your shell.
%
% \paragraph{Generating the documentation.\label{GenDoc}}
% You can use both the \xfile{.dtx} or the \xfile{.drv} to generate
% the documentation. The process can be configured by a
% configuration file \xfile{ltxdoc.cfg}. For instance, put the following
% line into this file, if you want to have A4 as paper format:
% \begin{quote}
%   \verb|\PassOptionsToClass{a4paper}{article}|
% \end{quote}
%
% \noindent An example follows how to generate the
% documentation with \pdfLaTeX :
%
% \begin{quote}
%\begin{verbatim}
%pdflatex footnoterange.drv
%makeindex -s gind.ist footnoterange.idx
%pdflatex footnoterange.drv
%makeindex -s gind.ist footnoterange.idx
%pdflatex footnoterange.drv
%\end{verbatim}
% \end{quote}
%
% \subsection{Compiling the example}
%
% The example file, \textsf{footnoterange-example.tex}, can be compiled via\newline
% \indent \verb|(pdf)latex footnoterange-example.tex|\newline
% but will need probably two compiler runs to get everything right.\newline
% \verb|footnoterange-example.tex| can be created by\newline
% \indent \verb|tex footnoterange.dtx|.\newline
% \verb|footnoterange.dtx| is at
% \url{https://www.ctan.org/pkg/footnoterange}.\newline
%
% \section{Acknowledgements}
%
% I (H.-Martin M\"{u}nch) would like to thank \textsc{Heiko Oberdiek}
% for providing a~lot~(!) of useful packages (from
% which I also got everything I know about creating a file in \xext{dtx}
% format, ok, say it: copying), \textsc{Frank Mittelbach} for several bug reports,
% for code for improving the package, and for pointing me to the \xpackage{fnpct}
% package as alternative, Willie Wong for reporting a bug and its fix,
% everybody of the CTAN team for managing CTAN,
% the \Newsgroup{comp.text.tex} and \Newsgroup{de.comp.text.tex} newsgroups,
% and everybody at \url{https://tex.stackexchange.com/}
% for their help in all things \hbox{\TeX{}.}
% This package was created as response to question
% \textquotedblleft Compressing consecutive footnote
% marks\textquotedblright{} by user001\newline
% (\url{https://tex.stackexchange.com/users/10127/user001}) on\newline
% \url{https://tex.stackexchange.com/q/44165/6865}.
%
% \newpage
% \phantomsection
% \begin{History}\label{History}
%   \begin{Version}{2011/02/11 v0.1ff.}
%     \item Some versions at \url{https://tex.stackexchange.com/q/44165/6865}.
%   \end{Version}
%   \begin{Version}{2012/02/17 v1.0a}
%     \item Package created and uploaded to CTAN.
%   \end{Version}
%   \begin{Version}{2018/10/15 v1.0b}
%     \item Fixed/updated minor details.
%     \item Bug fix: Did not work without \xpackage{hyperref}, wrong spacefactor.
%     \item Listing the \href{https://www.ctan.org/pkg/fnpct}{\xpackage{fnpct}}
%           package as alternative.
%   \end{Version}
%   \begin{Version}{2019/11/24 v1.0c}
%     \item Bug fix: Issue with footnote symbols.
%   \end{Version}
% \end{History}
%
% \bigskip
%
% When you find a mistake or have a suggestion for an improvement of this package,
% please send an e-mail to the maintainer, thanks! (Please see BUG REPORTS in the README.)
%
% \PrintIndex
%
% \Finale
\endinput
%        (quote the arguments according to the demands of your shell)
%
% Documentation:
%    (a) If footnoterange.drv is present:
%           (pdf)latex footnoterange.drv
%           makeindex -s gind.ist footnoterange.idx
%           (pdf)latex footnoterange.drv
%           makeindex -s gind.ist footnoterange.idx
%           (pdf)latex footnoterange.drv
%    (b) Without footnoterange.drv:
%           (pdf)latex footnoterange.dtx
%           makeindex -s gind.ist footnoterange.idx
%           (pdf)latex footnoterange.dtx
%           makeindex -s gind.ist footnoterange.idx
%           (pdf)latex footnoterange.dtx
%
%    The class ltxdoc loads the configuration file ltxdoc.cfg
%    if available. Here you can specify further options, e.g.
%    use DIN A4 as paper format:
%       \PassOptionsToClass{a4paper}{article}
%
% Installation:
%    TDS:tex/latex/footnoterange/footnoterange.sty
%    TDS:doc/latex/footnoterange/footnoterange.pdf
%    TDS:doc/latex/footnoterange/footnoterange-example.tex
%    TDS:doc/latex/footnoterange/footnoterange-example.pdf
%    TDS:source/latex/footnoterange/footnoterange.dtx
%
%<*ignore>
\begingroup
  \catcode123=1 %
  \catcode125=2 %
  \def\x{LaTeX2e}%
\expandafter\endgroup
\ifcase 0\ifx\install y1\fi\expandafter
         \ifx\csname processbatchFile\endcsname\relax\else1\fi
         \ifx\fmtname\x\else 1\fi\relax
\else\csname fi\endcsname
%</ignore>
%<*install>
\input docstrip.tex
\Msg{*********************************************************************************}
\Msg{* Installation}
\Msg{* Package: footnoterange 2019/11/24 v1.0c References to ranges of footnotes (HMM)}
\Msg{*********************************************************************************}

\keepsilent
\askforoverwritefalse

\let\MetaPrefix\relax
\preamble

This is a generated file.

Project: footnoterange
Version: 2019/11/24 v1.0c

Copyright (C) 2012 - 2019 by
    H.-Martin M"unch <Martin dot Muench at Uni-Bonn dot de>
Portions of code copyrighted by other people as marked;
lines marked "FMi": code by Frank Mittelbach, thanks!

The usual disclaimer applies:
If it doesn't work right that's your problem.
(Nevertheless, please send an e-mail to the maintainer
 when you find an error in this package.)

This work may be distributed and/or modified under the
conditions of the LaTeX Project Public License, either
version 1.3c of this license or (at your option) any later
version. This version of this license is in
   https://www.latex-project.org/lppl/lppl-1-3c.txt
and the latest version of this license is in
   https://www.latex-project.org/lppl.txt
and version 1.3c or later is part of all distributions of
LaTeX version 2005/12/01 or later.

This work has the LPPL maintenance status "maintained".

The Current Maintainer of this work is H.-Martin Muench.

This work consists of the main source file footnoterange.dtx,
the README, and the derived files
   footnoterange.sty, footnoterange.pdf,
   footnoterange.ins, footnoterange.drv,
   footnoterange-example.tex, footnoterange-example.pdf.

In memoriam
 Claudia Simone Barth + 1996/01/30
 Tommy Muench + 2014/01/02
 Hans-Klaus Muench + 2014/08/24

\endpreamble
\let\MetaPrefix\DoubleperCent

\generate{%
  \file{footnoterange.ins}{\from{footnoterange.dtx}{install}}%
  \file{footnoterange.drv}{\from{footnoterange.dtx}{driver}}%
  \usedir{tex/latex/footnoterange}%
  \file{footnoterange.sty}{\from{footnoterange.dtx}{package}}%
  \usedir{doc/latex/footnoterange}%
  \file{footnoterange-example.tex}{\from{footnoterange.dtx}{example}}%
}

\catcode32=13\relax% active space
\let =\space%
\Msg{************************************************************************}
\Msg{*}
\Msg{* To finish the installation you have to move the following}
\Msg{* file into a directory searched by TeX:}
\Msg{*}
\Msg{* footnoterange.sty}
\Msg{*}
\Msg{* To produce the documentation run the file `footnoterange.drv'}
\Msg{* through (pdf)LaTeX, e.g.}
\Msg{*  pdflatex footnoterange.drv}
\Msg{*  makeindex -s gind.ist footnoterange.idx}
\Msg{*  pdflatex footnoterange.drv}
\Msg{*  makeindex -s gind.ist footnoterange.idx}
\Msg{*  pdflatex footnoterange.drv}
\Msg{*}
\Msg{* At least three runs are necessary e.g. to get the}
\Msg{*  references right!}
\Msg{*}
\Msg{* Happy TeXing!}
\Msg{*}
\Msg{************************************************************************}

\endbatchfile
%</install>
%<*ignore>
\fi
%</ignore>
%
% \section{The documentation driver file}
%
% The next bit of code contains the documentation driver file for
% \TeX , i.\,e., the file that will produce the documentation you
% are currently reading. It will be extracted from this file by the
% \verb|docstrip| programme. That is, run \LaTeX{} on \verb|docstrip|
% and specify the \verb|driver| option when \verb|docstrip|
% asks for options.
%
%    \begin{macrocode}
%<*driver>
\NeedsTeXFormat{LaTeX2e}[2015/01/01]
\ProvidesFile{footnoterange.drv}[2019/11/24 v1.0c
            References to ranges of footnotes (HMM)]
\documentclass{ltxdoc}[2015/03/26]% v2.0w
\usepackage{holtxdoc}[2012/03/21]%  v0.24
%% footnoterange may work with earlier versions of LaTeX2e and those
%% class and package, but this was not tested.
%% Please consider updating your LaTeX, class, and package
%% to the most recent version (if they are not already the most
%% recent version).
\hypersetup{%
 pdfsubject={References to ranges of footnotes (HMM)},%
 pdfkeywords={LaTeX, footnoterange, footnotes, footnote, reference, H.-Martin Muench},%
 pdfencoding=auto,%
 pdflang={en},%
 breaklinks=true,%
 linktoc=all,%
 pdfstartview=FitH,%
 pdfpagelayout=OneColumn,%
 bookmarksnumbered=true,%
 bookmarksopen=true,%
 bookmarksopenlevel=3,%
 pdfmenubar=true,%
 pdftoolbar=true,%
 pdfwindowui=true,%
 pdfnewwindow=true%
}
\CodelineIndex
\hyphenation{printing docu-ment}
\gdef\unit#1{\mathord{\thinspace\mathrm{#1}}}
\begin{document}
  \DocInput{footnoterange.dtx}%
\end{document}
%</driver>
%    \end{macrocode}
%
% \fi
%
% \CheckSum{321}
%
% \CharacterTable
%  {Upper-case    \A\B\C\D\E\F\G\H\I\J\K\L\M\N\O\P\Q\R\S\T\U\V\W\X\Y\Z
%   Lower-case    \a\b\c\d\e\f\g\h\i\j\k\l\m\n\o\p\q\r\s\t\u\v\w\x\y\z
%   Digits        \0\1\2\3\4\5\6\7\8\9
%   Exclamation   \!     Double quote  \"     Hash (number) \#
%   Dollar        \$     Percent       \%     Ampersand     \&
%   Acute accent  \'     Left paren    \(     Right paren   \)
%   Asterisk      \*     Plus          \+     Comma         \,
%   Minus         \-     Point         \.     Solidus       \/
%   Colon         \:     Semicolon     \;     Less than     \<
%   Equals        \=     Greater than  \>     Question mark \?
%   Commercial at \@     Left bracket  \[     Backslash     \\
%   Right bracket \]     Circumflex    \^     Underscore    \_
%   Grave accent  \`     Left brace    \{     Vertical bar  \|
%   Right brace   \}     Tilde         \~}
%
% \GetFileInfo{footnoterange.drv}
%
% \begingroup
%   \def\x{\#,\$,\^,\_,\~,\ ,\&,\{,\},\%}%
%   \makeatletter
%   \@onelevel@sanitize\x
% \expandafter\endgroup
% \expandafter\DoNotIndex\expandafter{\x}
% \expandafter\DoNotIndex\expandafter{\string\ }
% \begingroup
%   \makeatletter
%     \lccode`9=32\relax
%     \lowercase{%^^A
%       \edef\x{\noexpand\DoNotIndex{\@backslashchar9}}%^^A
%     }%^^A
%   \expandafter\endgroup\x
% \DoNotIndex{\,,\\}
% \DoNotIndex{\documentclass,\usepackage,\ProvidesPackage,\begin,\end}
% \DoNotIndex{\MessageBreak}
% \DoNotIndex{\NeedsTeXFormat,\DoNotIndex,\verb}
% \DoNotIndex{\def,\edef,\gdef,\global,\xdef}
% \DoNotIndex{\ifx,\listfiles,\mathord,\mathrm}
% \DoNotIndex{\bigskip,\space,\thinspace,\ldots}
% \DoNotIndex{\indent,\noindent,\newline,\linebreak,\pagebreak,\newpage}
% \DoNotIndex{\textbf,\textit,\textsf,\textsc,\texttt,\textquotedblleft,\textquotedblright}
% \DoNotIndex{\plainTeX,\TeX,\LaTeX,\pdfLaTeX}
% \DoNotIndex{\section,\pagenumbering,\smallskip}
% \DoNotIndex{\@ifpackageloaded}
% \DoNotIndex{\arabic, \Alph, \alph, \alphalph, \Roman, \roman, \fnsymbol}
%
% \title{The \xpackage{footnoterange} package}
% \date{2019/11/24 v1.0c}
% \author{H.-Martin M\"{u}nch\\\xemail{Martin.Muench at Uni-Bonn.de}}
%
% \maketitle
%
% \begin{abstract}
% \noindent This package provides the environments |footnoterange| and
% |footnoterange*|. Multiple footnotes inside these environments
% are not referenced as e.\,g.
% \makeatletter%
% \hbox{\textquotedblleft \@textsuperscript {\normalfont 1 2 3}\textquotedblright}
% but as \hbox{\textquotedblleft \@textsuperscript {\normalfont 1--3}\textquotedblright ,\ }
% \makeatother%
% i.\,e. the range (from first to last referred footnote at that place) is given.
% Depending on loading of \xpackage{hyperref} and use of its |hyperfootnotes|-option
% the references are hyperlinked. (References to footnotes in |footnoterange*|
% are never hyperlinked.)
% \end{abstract}
%
% \bigskip
%
% \noindent Disclaimer for web links: The author is not responsible for any contents
% referred to in this work unless if having full knowledge of illegal contents.
% If any damage occurs by the use of information presented there, only the
% author of the respective pages might be liable, not the one who has referred
% to these pages.
%
% \noindent Save per page about $200\unit{ml}$ water,
% $2\unit{g}$ CO$_{2}$ and $2\unit{g}$ wood:
% Therefore please print only if this is really necessary.
%
% \medskip
%
% \tableofcontents
%
%\newpage
%
% \section{Introduction}
% \indent This \LaTeXe{} package provides the environments |footnoterange| and\newline
% |footnoterange*|. Multiple footnotes inside these environments are not referenced
% as e.\,g.
% \makeatletter%
% \hbox{\textquotedblleft \@textsuperscript {\normalfont 1 2 3}\textquotedblright}
% but as \hbox{\textquotedblleft \@textsuperscript {\normalfont 1--3}\textquotedblright ,\ }
% \makeatother%
% i.\,e. the range (from first to last
% referred footnote at that place) is given. Depending on loading of
% \xpackage{hyperref} and use of its |hyperfootnotes|-option the references are
% hyperlinked. References to footnotes in |footnoterange*| are never hyperlinked.\newline
% |\footnotemark\footnotetext| instead of |\footnote| is not supported inside
% |footnoterange| and |footnoterange*| environments.
%
% \bigskip
%
% \section{Usage}
%
% \indent Just load the package placing
% \begin{quote}
%   |\usepackage{footnoterange}|
% \end{quote}
% \noindent in the preamble of your \LaTeXe{} source file. There are no options to
% be used. The package should be loaded after \xpackage{hyperref} (if that package is used)
% and after any other package redefining footnotes. Usage of the |footnoterange| (or
% |footnoterange*|) environment is simple: Just encapsulate the footnotes, which shall
% be referred as range instead of referencing as single ones (see the example file).
%
% About hyperlinks:\newline
% Footnote-references in a |footnoterange*| environment are never hyperlinked.\newline
% Footnote-references in a |footnoterange| environment may be hyperlinked:
% \begin{itemize}
%  \item When the \xpackage{hyperref} package is not loaded, they are not
%         hyperlinked (naturally).
%  \item When the \xpackage{hyperref} package is loaded, it depends on the
%         |hyperfootnotes| option of the \xpackage{hyperref} package:
%  \begin{itemize}
%   \item With |hyperfootnotes=true|, the references are hyperlinked.
%   \item With |hyperfootnotes=false|, the references are not hyperlinked.
%  \end{itemize}
% \end{itemize}
% |\footnote|s inside of the |footnoterange|/{}|footnoterange*| environment
% can use the optional parameter, i.\,e. |\footnote[42]{text}| even if this is the
% third footnote. When giving ranges as reference, this does not seem to be
% such a good idea, if this is not done consistently (i.\,e. footnotes
% |[42], [43], [44]| resulting in 42--44 is OK,
% but |[42]| (original~2), unchanged~3, unchanged~4 resulting in
% 42--4 is confusing). A warning about these changed footnotemarks is written
% into the \xfile{.log}-file.
%
% When there is only a single footnote in a |footnoterange| environment,
% this does not produce some reference like %
% \makeatletter%
% \hbox{\textquotedblleft \@textsuperscript {\normalfont 7--7}\textquotedblright{},\ }%
% \makeatother%
% but only one footnotemark is used automatically. Because it is strange to use
% a |footnoterange| environment without a range of footnotes, an info is written
% into the \xfile{.log}-file about this.
%
% The usage of other footnote-number-representations
% (e.\,g.~\verb|\Roman|, \verb|\roman|, \verb|\Alph|, \verb|\alph|,
% \verb|\fnsymbol|) is also possible, but two things must be taken
% into account: The number of footnote references is restricted (for
% example with \verb|\Alph| only references A to Z are possible), which
% can be fixed e.\,g. with the \url{https://ctan.org/pkg/alphalph} package
% (but not automatically), and references to footnote-symbol-ranges (\verb|\fnsymbol|)
% are probably not very clear.
%
% \subsection*{Options}
% \indent The \xpackage{footnoterange} package takes no options.
%
% \section{Alternatives\label{sec:Alternatives}}
%
% The \xpackage{fnpct} (\url{https://www.ctan.org/pkg/fnpct}) package allows to set
% multiple footnotes (|\multfootnote|), suppress a footnotemark \hbox{(|*|),}
% and customize the multi-footnote-separator \hbox{(|\setfnpct{mult-fn-sep={|
% \emph{something} |}}|)}.
% Combining these with stepping of the footnotecounter gives a footnoterange:
% 
% \begin{verbatim}
% \documentclass{article}
% \usepackage{hyperref}
% \usepackage{fnpct}
% \begin{document}
% The three little pigs built their houses
% out of straw\footnote{not to be confused with hay},
% sticks\footnote{or lumber according to some sources}
% and bricks%
% \makeatletter%
% \setfnpct{mult-fn-sep={--}}%
% \multfootnote{probably fired clay bricks\addtocounter\@mpfn{+1};%
% *or something else;%
% what do I know}. %
% \makeatother%
% \end{document}
% \end{verbatim}
% 
% \noindent If you programmed or found another alternative, which is available at
% \url{https://www.CTAN.org/}, please send an e-mail to me with the name,
% location at CTAN, and a short notice, and I will probably list it here.
%
% \section{Example}
%
%    \begin{macrocode}
%<*example>
\documentclass[british]{article}[2014/09/29]% v1.4h
%%%%%%%%%%%%%%%%%%%%%%%%%%%%%%%%%%%%%%%%%%%%%%%%%%%%%%%%%%%%%%%%%%%%%
\usepackage[%
 extension=pdf,%
 plainpages=false,%
 pdfpagelabels=true,%
 hyperindex=false,%
 pdflang={en},%
 pdftitle={footnoterange package example},%
 pdfauthor={H.-Martin Muench},%
 pdfsubject={Example for the footnoterange package},%
 pdfkeywords={LaTeX, footnoterange, H.-Martin Muench},%
 pdfview=FitH,%
 pdfstartview=FitH,%
 pdfpagelayout=OneColumn,%
 hyperfootnotes=true%
]{hyperref}[2012/11/06]% v6.83m

\usepackage{footnoterange}[2019/11/24]% v1.0c

\gdef\unit#1{\mathord{\thinspace\mathrm{#1}}}%
\listfiles
\begin{document}
\pagenumbering{arabic}
\section*{Example for footnoterange}

This example demonstrates the use of package\newline
\textsf{footnoterange}, v1.0c as of 2019/11/24 (HMM).\newline
The package does not use options.\newline
\verb|footnoterange-example.pdf| can be created by\newline
\verb|pdflatex footnoterange-example.tex|,\newline
and \verb|footnoterange-example.tex| can be created by\newline
\verb|tex footnoterange.dtx|.\newline
\verb|footnoterange.dtx| is at
\url{https://www.ctan.org/pkg/footnoterange}.\newline

\noindent For more details please see the documentation!\newline

\noindent Save per page about $200\unit{ml}$ water,
$2\unit{g}$ CO$_{2}$ and $2\unit{g}$ wood:\newline
Therefore please print only if this is really necessary.\newline

This\footnote{Lorem} text bears a multiplicity of footnotes.
Because the \verb|hyperref| package is used with option
\verb|hyperfootnotes=true|, the footnotes
\begin{footnoterange}
\footnote{ipsum}%
\footnote{dolor}%
\footnote{sit}%
\footnote{amet,}%
\footnote{consectetur}%
\footnote{adipisicing}%
\footnote{elit,}%
\footnote{sed}%
\end{footnoterange}
are hyperlinked.\footnote{do eiusmod\ldots} Using
\verb|hyperfootnotes=false| or not using \verb|hyperref|
will remove the hyperlinks to the footnotes.\medskip

This text also bears a multiplicity of footnotes,
but due to the use of the starred
\begin{footnoterange*}
\footnote{Ut purus elit,}%
\footnote{vestibulum ut,}%
\footnote{placerat ac,}%
\footnote{adipiscing vitae,}%
\footnote{felis.}%
\footnote{Curabitur}%
\footnote{dictum}%
\footnote{gravida}%
\end{footnoterange*}
form of the environment they are not
hyperlinked.\footnote{mauris\ldots}\medskip

The use of the optional parameter for footnotes
\begin{footnoterange}
\footnote[101]{Nam arcu libero,}%
\footnote[102]{nonummy eget,}%
\footnote[103]{consectetuer id,}%
\footnote[104]{vulputate a,}%
\footnote[105]{magna.}
\end{footnoterange}
is also possible, but should be used consistently.

\pagebreak

When there is only one footnote
\begin{footnoterange}
\footnote{a single footnote}
\end{footnoterange}
in the \verb|footnoterange| environment, only one
footnotemark is used and an info is written into the
\verb|log|-file.\medskip

The usage of other footnote-number-representations
(e.\,g.~\verb|\Roman|, \verb|\roman|, \verb|\Alph|, \verb|\alph|,
\verb|\fnsymbol|) is also possible, but two things must be taken
into account: The number of footnote references is restricted
(for example with \verb|\Alph| only references A to Z are possible),
which can be fixed e.\,g. with the
\href{https://www.ctan.org/pkg/alphalph}{\texttt{alphalph} package},
and references to footnote-symbol-ranges (\verb|\fnsymbol|)
are probably not very clear.

\end{document}
%</example>
%    \end{macrocode}
%
% \StopEventually{}
%
% \section{The implementation}
%
% We start off by checking that we are loading into \LaTeXe{} and
% announcing the name and version of this package.
%
%    \begin{macrocode}
%<*package>
%    \end{macrocode}
%
%    \begin{macrocode}
\NeedsTeXFormat{LaTeX2e}[2011/06/27]
\ProvidesPackage{footnoterange}[2019/11/24 v1.0c
            References to ranges of footnotes (HMM)]
%    \end{macrocode}
%
% A short description of the \xpackage{footnoterange} package:
%
%    \begin{macrocode}
%% Provides the footnoterange and footnoterange* environments to
%% refer to ranges of footnotes.
%    \end{macrocode}
%
% We need the \xpackage{ltxcmds} package:
%
%    \begin{macrocode}
\RequirePackage{ltxcmds}[2011/11/09]% v1.22; for "\ltx@ifpackageloaded"
\RequirePackage{letltxmacro}[2010/09/02]% v1.4; for "\GlobalLetLtxMacro"

%    \end{macrocode}
%
% We use |\ifHy@hyperfootnotes ... \else ... \fi|, but when the \xpackage{hyperref}
% package has not been loaded, |\ifHy@hyperfootnotes| is not defined, and \LaTeX
% relates the |\else| and/or |\fi| (wrongly) to other |\if|s or groups.
% Therefore the |\ifHy@hyperfootnotes| are placed in their own commands,
% which are not looked at when the \xpackage{hyperref} package has not been loaded.
%
% \pagebreak
%
% \begin{macro}{\fnr@ifhyperrfn}
%    \begin{macrocode}
\newcommand{\fnr@ifhyperrfn}{%
  \ifHy@hyperfootnotes% option hyperfootnotes=true
    \addtocounter{Hfootnote}{+1}%
    % Code from the hyperref package
      \global\let\Hy@saved@currentHref\@currentHref%
      \hyper@makecurrent{Hfootnote}%
      \global\let\Hy@footnote@currentHref\@currentHref%
      \global\let\@currentHref\Hy@saved@currentHref%
    % End of code from the hyperref package
  \fi%
}

%    \end{macrocode}
% \end{macro}
%
% \begin{macro}{\fnr@rangefootnote}
% The |\fnr@rangefootnote| is the internal replacement for the original\newline
% |\footnote| command inside of a |footnoterange| environment.
%
%    \begin{macrocode}
\newcommand{\fnr@rangefootnote}[2][\empty]{%
%    \end{macrocode}
%
% |#1| contains the optional argument, which is predefined as |\empty|.
%
%    \begin{macrocode}
  \nolinebreak%
  % no \footnotemark here
  \addtocounter\@mpfn{+1}%
  \ltx@ifpackageloaded{hyperref}{\fnr@ifhyperrfn}{\relax}%
%    \end{macrocode}
%
% See the explanation before the declaration of the |\fnr@ifhyperrfn| command!
%
%    \begin{macrocode}
  \xdef\fnr@opt{#1}% contains the optional argument
  \xdef\fnr@arabic{\arabic\@mpfn}%
  \protected@edef\fnr@formated{\thefootnote}%
%    \end{macrocode}
%
% The |\footnotemark| is not set (yet), but the |\footnotetext|,
% and it gets a label.
%
%    \begin{macrocode}
  \ifx\fnr@opt\empty%
    \footnotetext{\label{fnr:\fnr@arabic}#2}%
  \else%
    \PackageWarning{footnoterange}{%
      Custom footnote-numbers used in a footnoterange\MessageBreak%
      environment: Footnote with natural number\MessageBreak%
      \ifx\fnr@formated\fnr@arabic%
        \fnr@arabic\space%
      \else%
        \space\fnr@arabic\space (\fnr@formated )%
      \fi%
      \MessageBreak%
      was renumbered to \fnr@opt\MessageBreak%
     }%
    \ltx@ifpackageloaded{hyperref}{% hyperref loaded
      \footnotetext[#1]{\phantomsection\label{fnr:\fnr@arabic}#2}%
     }{% hyperref not loaded
      \footnotetext[#1]{\label{fnr:\fnr@arabic}#2}%
     }%
  \fi%
}

%    \end{macrocode}
% \end{macro}
%
%    \begin{macrocode}

%    \end{macrocode}
% \pagebreak
%
% |\ifHy@hyperfootnotes| \textquotedblleft encapsulation\textquotedblright{} again:
%
% \begin{macro}{\fnr@ifhyperfnrA}
%    \begin{macrocode}
\newcommand{\fnr@ifhyperfnrA}{%
  \ifHy@hyperfootnotes% option hyperfootnotes=true
    \hbox{\hyperref[fnr:\fnr@first]{%
          \@textsuperscript{\normalfont \ref*{fnr:\fnr@first}}}}%
  \else% option hyperfootnotes=false
    \hbox{\@textsuperscript{\normalfont \ref*{fnr:\fnr@first}}}%
  \fi%
}

%    \end{macrocode}
% \end{macro}
%
% \begin{macro}{\fnr@ifhyperfnrN}
%    \begin{macrocode}
\newcommand{\fnr@ifhyperfnrN}{%
  \ifHy@hyperfootnotes% option hyperfootnotes=true
    \hbox{\hyperref[fnr:\fnr@first]{%
            \@textsuperscript{\normalfont \ref*{fnr:\fnr@first}}}%
          \@textsuperscript{\normalfont --}%
          \hyperref[fnr:\arabic\@mpfn]{%
            \@textsuperscript{\normalfont \ref*{fnr:\arabic\@mpfn}}}}%
  \else% option hyperfootnotes=false
    \hbox{\@textsuperscript{\normalfont \ref*{fnr:\fnr@first}%
                                        --\ref*{fnr:\arabic\@mpfn}}}%
  \fi%
}

%    \end{macrocode}
% \end{macro}
%
% \begin{environment}{footnoterange}
% The |footnoterange| environment remembers the next footnote number as\newline
% |\fnr@first| and the current |\footnote| definition as |\fnr@origfn|
% and uses |\fnr@rangefootnote| instead.
%
%    \begin{macrocode}
\newenvironment{footnoterange}{%
  % at begin of the footnoterange environment
  \ifhmode \unskip \fi% FMi
  \addtocounter\@mpfn{+1}%
  \xdef\fnr@first{\arabic\@mpfn}%
  \addtocounter\@mpfn{-1}%
  \let\fnr@origfn\footnote%
  \let\footnote\fnr@rangefootnote%
  \GlobalLetLtxMacro{\fnr@origfnmark}{\footnotemark}%
  \GlobalLetLtxMacro{\footnotemark}{\fnr@rangefootnotemark}%
  \ignorespaces% FMi
%    \end{macrocode}
%
% End of the environment:
%
%    \begin{macrocode}
}{% at end of the footnoterange environment
  \ifhmode% FMi 
    \unskip% FMi
%    \end{macrocode}
%
% Remember the spacefactor for restoring after the environment has ended
% (only in |hmode|!). \% FMi
%
%    \begin{macrocode}
    \xdef\fnr@aftergroup{\spacefactor\the\spacefactor\relax}% FMi
  \else% FMi
%    \end{macrocode}
%
% Otherwise do something harmless. \% FMi
%
%    \begin{macrocode}
    \xdef\fnr@aftergroup{\relax}%
  \fi% FMi
%    \end{macrocode}
%
% At the end of the environment the previous definition of |\footnote| is restored
%
%    \begin{macrocode}
  \let\footnote\fnr@origfn%
  \GlobalLetLtxMacro{\footnotemark}{\fnr@origfnmark}%
%    \end{macrocode}
%
% and the references are placed:
%
%    \begin{macrocode}
  \ifx\fnr@first\fnr@arabic% There is only one reference in the environment.
    \PackageInfo{footnoterange}{%
      Environment footnoterange used,\MessageBreak%
      but only one footnote (number \fnr@first ) is contained%
      \MessageBreak%
     }%
    \ltx@ifpackageloaded{hyperref}{\fnr@ifhyperfnrA% hyperref package loaded
    }{% hyperref package not loaded
      \hbox{\@textsuperscript{\normalfont \ref{fnr:\fnr@first}}}%
     }%
  \else%
    \ltx@ifpackageloaded{hyperref}{\fnr@ifhyperfnrN% hyperref package loaded
    }{% hyperref package not loaded
      \hbox{\@textsuperscript{\normalfont \ref{fnr:\fnr@first}%
                                        --\ref{fnr:\arabic\@mpfn}}}%
     }%
  \fi%
%    \end{macrocode}
%
% Restore the spacefactor after the environment has ended. \% FMi
%
%    \begin{macrocode}
  \aftergroup\fnr@aftergroup% FMi
}

%    \end{macrocode}
% \end{environment}
%
% \begin{environment}{footnoterange*}
% The |footnoterange*| environment does the same as the |footnoterange|
% environment but suppresses all hyperlinks to footnotes inside.
%
%    \begin{macrocode}
\newenvironment{footnoterange*}{%
  % at begin of the footnoterange* environment
  \ifhmode \unskip \fi% FMi
  \addtocounter\@mpfn{+1}%
  \xdef\fnr@first{\arabic\@mpfn}%
  \addtocounter\@mpfn{-1}%
  \let\fnr@origfn\footnote%
  \let\footnote\fnr@rangefootnote%
  \GlobalLetLtxMacro{\fnr@origfnmark}{\footnotemark}%
  \GlobalLetLtxMacro{\footnotemark}{\fnr@rangefootnotemark}%
  \ignorespaces% FMi
}{% at end of the footnoterange* environment
  \ifhmode% FMi 
    \unskip% FMi
    \xdef\fnr@aftergroup{\spacefactor\the\spacefactor\relax}% FMi
  \else% FMi
    \xdef\fnr@aftergroup{\relax}%
  \fi% FMi
  \let\footnote\fnr@origfn%
  \GlobalLetLtxMacro{\footnotemark}{\fnr@origfnmark}%
  \ifx\fnr@first\fnr@arabic% There is only one reference in the environment.
    \PackageInfo{footnoterange}{%
      Environment footnoterange used,\MessageBreak%
      but only one footnote (number \fnr@first ) is contained%
      \MessageBreak%
     }%
%    \end{macrocode}
% \pagebreak
%    \begin{macrocode}
    \ltx@ifpackageloaded{hyperref}{% hyperref package loaded
      \hbox{\@textsuperscript{\normalfont \ref*{fnr:\fnr@first}}}%
    }{% hyperref package not loaded
      \hbox{\@textsuperscript{\normalfont \ref{fnr:\fnr@first}}}%
     }%
  \else%
    \ltx@ifpackageloaded{hyperref}{% hyperref package loaded
      \hbox{\@textsuperscript{\normalfont \ref*{fnr:\fnr@first}%
                                        --\ref*{fnr:\arabic\@mpfn}}}%
    }{% hyperref package not loaded
      \hbox{\@textsuperscript{\normalfont \ref{fnr:\fnr@first}%
                                        --\ref{fnr:\arabic\@mpfn}}}%
     }%
  \fi%
  \aftergroup\fnr@aftergroup% FMi
}

%    \end{macrocode}
% \end{environment}
%
% The \xpackage{footnoterange} package does not handle |\footnotemark|s,
% but at least the user is made aware of this:
%
% \begin{macro}{\fnr@rangefootnotemark}
%    \begin{macrocode}

\newcommand{\fnr@rangefootnotemark}[1][\empty]{%
  \PackageError{footnoterange}{%
     \string\footnotemark\space not supported in footnoterange(*) environment%
   }{The footnoterange package does not handle a \string\footnotemark.\MessageBreak%
     Either use \string\footnote\string{...\string}\space instead of%
     \string\footnotemark\space and \string\footnotetext\string{...\string}\MessageBreak%
     in footnoterange and footnoterange* environments,\MessageBreak%
     or do not use those environments around a \string\footnotemark\MessageBreak%
     (or \string\renewcommand\string{\string\fnr@rangefootnotemark\string}\MessageBreak%
     with your own fixing code).%
    }%
  \xdef\fnr@mopt{#1}%
  \ifx\fnr@mopt\empty%
    \fnr@origfnmark%
  \else%
    \fnr@origfnmark[#1]%
  \fi%
}

%    \end{macrocode}
% \end{macro}
%
%    \begin{macrocode}
%</package>
%    \end{macrocode}
%
% \pagebreak
%
% \section{Installation}
%
% \subsection{Downloads\label{ss:Downloads}}
%
% Everything is available on CTAN,
% \url{https://www.ctan.org/tex-archive/},
% but may need additional packages themselves.
%
% \DescribeMacro{footnoterange.dtx}
% For unpacking the \xfile{footnoterange.dtx} file
% and constructing the documentation it is required:
% \begin{description}
% \item[-] \TeX{} Format \LaTeXe{}, \url{https://www.CTAN.org/}
%
% \item[-] document class \xpackage{ltxdoc}, 2015/03/26, v2.0w,
%           \url{https://ctan.org/pkg/ltxdoc}
%
% \item[-] package \xpackage{holtxdoc}, 2012/03/21, v0.24,
%           \url{https://ctan.org/pkg/holtxdoc}
% \end{description}
%
% \DescribeMacro{footnoterange.sty}
% The |footnoterange.sty| for \LaTeXe{} (i.\,e. each document using
% the \xpackage{footnoterange} package) requires:
% \begin{description}
% \item[-] \TeX{} Format \LaTeXe{}, \url{https://www.CTAN.org/}
%
% \item[-] package \xpackage{ltxcmds}, 2011/04/18, v1.20,
%           \url{https://www.ctan.org/pkg/ltxcmds}
% \end{description}
%
% \DescribeMacro{footnoterange-example.tex}
% The |footnoterange-example.tex| requires the same files as all
% documents using the \xpackage{footnoterange} package and additionally:
% \begin{description}
% \item[-] class \xpackage{article}, 2007/10/19, v1.4h, from \xpackage{classes.dtx}:\newline
%           \url{https://www.ctan.org/tex-archive/macros/latex/base/}
%
% \item[-] package \xpackage{footnoterange}, 2019/11/24, v1.0c,\newline
%           \url{https://www.ctan.org/pkg/footnoterange}\newline
%           (Well, it is the example file for this package, and because you are
%            reading the documentation for the \xpackage{footnoterange} package,
%            it can be assumed that you already have some version of it --
%            is it the current one?)
%
% \item[-] package \xpackage{hyperref}, 2012/11/06, v6.83m,\newline
%           \url{https://www.ctan.org/pkg/hyperref}
% \end{description}
%
% \DescribeMacro{Oberdiek}
% \DescribeMacro{holtxdoc}
% \DescribeMacro{ltxcmds}
% \DescribeMacro{alphalph}
% All packages of \textsc{Heiko Oberdiek}'s bundle `oberdiek'
% (especially \xpackage{holtxdoc}, \xpackage{ltxcmds}, and \xpackage{alphalph})
% are also available in a TDS compliant ZIP archive:\newline
% \url{http://mirror.ctan.org/install/macros/latex/contrib/oberdiek.tds.zip}.\newline
% It is probably best to download and use this, because the packages in there
% should be both recent and compatible.
%
% \DescribeMacro{hyperref}
% \noindent \xpackage{hyperref} is not included in that bundle and needs to be
% downloaded separately,\newline
% \url{http://mirrors.ctan.org/install/macros/latex/contrib/hyperref.tds.zip}.
%
% \DescribeMacro{M\"{u}nch}
% A list of my packages can be found at \url{https://www.ctan.org/author/muench-hm}.
%
% \pagebreak
%
% \subsection{Package, unpacking TDS}
% \paragraph{Package.} This package is available on \url{https://www.CTAN.org}.
% \begin{description}
% \item[\url{http://mirror.ctan.org/macros/latex/contrib/footnoterange/footnoterange.dtx}]\hspace*{0.1cm}
%       The source file.
% \item[\url{http://mirror.ctan.org/macros/latex/contrib/footnoterange/footnoterange.pdf}]\hspace*{0.1cm}
%       The documentation.
% \item[\url{http://mirror.ctan.org/macros/latex/contrib/footnoterange/footnoterange-example.pdf}]\hspace*{0.1cm}
%       The compiled example file, as it should look like.
% \item[\url{http://mirror.ctan.org/macros/latex/contrib/footnoterange/README}]\hspace*{0.1cm}\newline\hspace*{0.1cm}
%       The README file.
% \end{description}
% There is also a footnoterange.tds.zip available:
% \begin{description}
% \item[\url{http://mirror.ctan.org/install/macros/latex/contrib/footnoterange.tds.zip}]\hspace*{0.1cm}
%       Everything in TDS compliant, compiled format.
% \end{description}
% which additionally contains\newline
% \begin{tabular}{ll}
% footnoterange.ins & The installation file.\\
% footnoterange.drv & The driver to generate the documentation.\\
% footnoterange.sty & The \xext{sty}le file.
% \end{tabular}
%
% \bigskip
%
% \noindent For required other packages, please see the preceding subsection.
%
% \paragraph{Unpacking.} The \xfile{.dtx} file is a self-extracting
% \docstrip{} archive. The files are extracted by running the
% \xfile{.dtx} through \plainTeX:
% \begin{quote}
%   \verb|tex footnoterange.dtx|
% \end{quote}
%
% About generating the documentation see paragraph~\ref{GenDoc} below.
%
% \paragraph{TDS.} Now the different files must be moved into
% the different directories in your installation TDS tree
% (also known as \xfile{texmf} tree):
% \begin{quote}
% \def\t{^^A
% \begin{tabular}{@{}>{\ttfamily}l@{ $\rightarrow$ }>{\ttfamily}l@{}}
%   footnoterange.sty & tex/latex/footnoterange/footnoterange.sty\\
%   footnoterange.pdf & doc/latex/footnoterange/footnoterange.pdf\\
%   footnoterange-example.tex & doc/latex/footnoterange/footnoterange-example.tex\\
%   footnoterange-example.pdf & doc/latex/footnoterange/footnoterange-example.pdf\\
%   footnoterange.dtx & source/latex/footnoterange/footnoterange.dtx\\
% \end{tabular}^^A
% }^^A
% \sbox0{\t}^^A
% \ifdim\wd0>\linewidth
%   \begingroup
%     \advance\linewidth by\leftmargin
%     \advance\linewidth by\rightmargin
%   \edef\x{\endgroup
%     \def\noexpand\lw{\the\linewidth}^^A
%   }\x
%   \def\lwbox{^^A
%     \leavevmode
%     \hbox to \linewidth{^^A
%       \kern-\leftmargin\relax
%       \hss
%       \usebox0
%       \hss
%       \kern-\rightmargin\relax
%     }^^A
%   }^^A
%   \ifdim\wd0>\lw
%     \sbox0{\small\t}^^A
%     \ifdim\wd0>\linewidth
%       \ifdim\wd0>\lw
%         \sbox0{\footnotesize\t}^^A
%         \ifdim\wd0>\linewidth
%           \ifdim\wd0>\lw
%             \sbox0{\scriptsize\t}^^A
%             \ifdim\wd0>\linewidth
%               \ifdim\wd0>\lw
%                 \sbox0{\tiny\t}^^A
%                 \ifdim\wd0>\linewidth
%                   \lwbox
%                 \else
%                   \usebox0
%                 \fi
%               \else
%                 \lwbox
%               \fi
%             \else
%               \usebox0
%             \fi
%           \else
%             \lwbox
%           \fi
%         \else
%           \usebox0
%         \fi
%       \else
%         \lwbox
%       \fi
%     \else
%       \usebox0
%     \fi
%   \else
%     \lwbox
%   \fi
% \else
%   \usebox0
% \fi
% \end{quote}
% If you have a \xfile{docstrip.cfg} that configures and enables \docstrip's
% TDS installing feature, then some files can already be in the right
% place, see the documentation of \docstrip.
%
% \subsection{Refresh file name databases}
%
% If your \TeX~distribution (\TeX{} Live, \mikTeX, \teTeX, \dots) relies on
% file name databases, you must refresh these. For example, \teTeX{} users run
% \verb|texhash| or \verb|mktexlsr|.
%
% \subsection{Some details for the interested}
%
% \paragraph{Unpacking with \LaTeX{}.}
% The \xfile{.dtx} chooses its action depending on the format:
% \begin{description}
% \item[\plainTeX:] Run \docstrip{} and extract the files.
% \item[\LaTeX:] Generate the documentation.
% \end{description}
% If you insist on using \LaTeX{} for \docstrip{} (really,
% \docstrip{} does not need \LaTeX ), then inform the autodetect routine
% about your intention:
% \begin{quote}
%   \verb|latex \let\install=y% \iffalse meta-comment
%
% File: footnoterange.dtx
% Version: 2019/11/24 v1.0c
%
% Copyright (C) 2012 - 2019 by
%    H.-Martin M"unch <Martin dot Muench at Uni-Bonn dot de>
% Portions of code copyrighted by other people as marked;
% lines marked "FMi": code by Frank Mittelbach, thanks!
%
% This work may be distributed and/or modified under the
% conditions of the LaTeX Project Public License, either
% version 1.3c of this license or (at your option) any later
% version. This version of this license is in
%    https://www.latex-project.org/lppl/lppl-1-3c.txt
% and the latest version of this license is in
%    https://www.latex-project.org/lppl.txt
% and version 1.3c or later is part of all distributions of
% LaTeX version 2005/12/01 or later.
%
% This work has the LPPL maintenance status "maintained".
%
% The Current Maintainer of this work is H.-Martin Muench.
%
% This work consists of the main source file footnoterange.dtx,
% the README, and the derived files
%    footnoterange.sty, footnoterange.pdf,
%    footnoterange.ins, footnoterange.drv,
%    footnoterange-example.tex, footnoterange-example.pdf.
%
% 'footnoterange' is available on CTAN:
% https://www.ctan.org/pkg/footnoterange
%
% Also a TDS.ZIP file is provided that contains all the files
% already sorted in a TDS tree:
% http://mirror.ctan.org/install/macros/latex/contrib/footnoterange.tds.zip
%
% Unpacking:
%    (a) If footnoterange.ins is present:
%           tex footnoterange.ins
%    (b) Without footnoterange.ins:
%           tex footnoterange.dtx
%    (c) If you insist on using LaTeX
%           latex \let\install=y% \iffalse meta-comment
%
% File: footnoterange.dtx
% Version: 2019/11/24 v1.0c
%
% Copyright (C) 2012 - 2019 by
%    H.-Martin M"unch <Martin dot Muench at Uni-Bonn dot de>
% Portions of code copyrighted by other people as marked;
% lines marked "FMi": code by Frank Mittelbach, thanks!
%
% This work may be distributed and/or modified under the
% conditions of the LaTeX Project Public License, either
% version 1.3c of this license or (at your option) any later
% version. This version of this license is in
%    https://www.latex-project.org/lppl/lppl-1-3c.txt
% and the latest version of this license is in
%    https://www.latex-project.org/lppl.txt
% and version 1.3c or later is part of all distributions of
% LaTeX version 2005/12/01 or later.
%
% This work has the LPPL maintenance status "maintained".
%
% The Current Maintainer of this work is H.-Martin Muench.
%
% This work consists of the main source file footnoterange.dtx,
% the README, and the derived files
%    footnoterange.sty, footnoterange.pdf,
%    footnoterange.ins, footnoterange.drv,
%    footnoterange-example.tex, footnoterange-example.pdf.
%
% 'footnoterange' is available on CTAN:
% https://www.ctan.org/pkg/footnoterange
%
% Also a TDS.ZIP file is provided that contains all the files
% already sorted in a TDS tree:
% http://mirror.ctan.org/install/macros/latex/contrib/footnoterange.tds.zip
%
% Unpacking:
%    (a) If footnoterange.ins is present:
%           tex footnoterange.ins
%    (b) Without footnoterange.ins:
%           tex footnoterange.dtx
%    (c) If you insist on using LaTeX
%           latex \let\install=y\input{footnoterange.dtx}
%        (quote the arguments according to the demands of your shell)
%
% Documentation:
%    (a) If footnoterange.drv is present:
%           (pdf)latex footnoterange.drv
%           makeindex -s gind.ist footnoterange.idx
%           (pdf)latex footnoterange.drv
%           makeindex -s gind.ist footnoterange.idx
%           (pdf)latex footnoterange.drv
%    (b) Without footnoterange.drv:
%           (pdf)latex footnoterange.dtx
%           makeindex -s gind.ist footnoterange.idx
%           (pdf)latex footnoterange.dtx
%           makeindex -s gind.ist footnoterange.idx
%           (pdf)latex footnoterange.dtx
%
%    The class ltxdoc loads the configuration file ltxdoc.cfg
%    if available. Here you can specify further options, e.g.
%    use DIN A4 as paper format:
%       \PassOptionsToClass{a4paper}{article}
%
% Installation:
%    TDS:tex/latex/footnoterange/footnoterange.sty
%    TDS:doc/latex/footnoterange/footnoterange.pdf
%    TDS:doc/latex/footnoterange/footnoterange-example.tex
%    TDS:doc/latex/footnoterange/footnoterange-example.pdf
%    TDS:source/latex/footnoterange/footnoterange.dtx
%
%<*ignore>
\begingroup
  \catcode123=1 %
  \catcode125=2 %
  \def\x{LaTeX2e}%
\expandafter\endgroup
\ifcase 0\ifx\install y1\fi\expandafter
         \ifx\csname processbatchFile\endcsname\relax\else1\fi
         \ifx\fmtname\x\else 1\fi\relax
\else\csname fi\endcsname
%</ignore>
%<*install>
\input docstrip.tex
\Msg{*********************************************************************************}
\Msg{* Installation}
\Msg{* Package: footnoterange 2019/11/24 v1.0c References to ranges of footnotes (HMM)}
\Msg{*********************************************************************************}

\keepsilent
\askforoverwritefalse

\let\MetaPrefix\relax
\preamble

This is a generated file.

Project: footnoterange
Version: 2019/11/24 v1.0c

Copyright (C) 2012 - 2019 by
    H.-Martin M"unch <Martin dot Muench at Uni-Bonn dot de>
Portions of code copyrighted by other people as marked;
lines marked "FMi": code by Frank Mittelbach, thanks!

The usual disclaimer applies:
If it doesn't work right that's your problem.
(Nevertheless, please send an e-mail to the maintainer
 when you find an error in this package.)

This work may be distributed and/or modified under the
conditions of the LaTeX Project Public License, either
version 1.3c of this license or (at your option) any later
version. This version of this license is in
   https://www.latex-project.org/lppl/lppl-1-3c.txt
and the latest version of this license is in
   https://www.latex-project.org/lppl.txt
and version 1.3c or later is part of all distributions of
LaTeX version 2005/12/01 or later.

This work has the LPPL maintenance status "maintained".

The Current Maintainer of this work is H.-Martin Muench.

This work consists of the main source file footnoterange.dtx,
the README, and the derived files
   footnoterange.sty, footnoterange.pdf,
   footnoterange.ins, footnoterange.drv,
   footnoterange-example.tex, footnoterange-example.pdf.

In memoriam
 Claudia Simone Barth + 1996/01/30
 Tommy Muench + 2014/01/02
 Hans-Klaus Muench + 2014/08/24

\endpreamble
\let\MetaPrefix\DoubleperCent

\generate{%
  \file{footnoterange.ins}{\from{footnoterange.dtx}{install}}%
  \file{footnoterange.drv}{\from{footnoterange.dtx}{driver}}%
  \usedir{tex/latex/footnoterange}%
  \file{footnoterange.sty}{\from{footnoterange.dtx}{package}}%
  \usedir{doc/latex/footnoterange}%
  \file{footnoterange-example.tex}{\from{footnoterange.dtx}{example}}%
}

\catcode32=13\relax% active space
\let =\space%
\Msg{************************************************************************}
\Msg{*}
\Msg{* To finish the installation you have to move the following}
\Msg{* file into a directory searched by TeX:}
\Msg{*}
\Msg{* footnoterange.sty}
\Msg{*}
\Msg{* To produce the documentation run the file `footnoterange.drv'}
\Msg{* through (pdf)LaTeX, e.g.}
\Msg{*  pdflatex footnoterange.drv}
\Msg{*  makeindex -s gind.ist footnoterange.idx}
\Msg{*  pdflatex footnoterange.drv}
\Msg{*  makeindex -s gind.ist footnoterange.idx}
\Msg{*  pdflatex footnoterange.drv}
\Msg{*}
\Msg{* At least three runs are necessary e.g. to get the}
\Msg{*  references right!}
\Msg{*}
\Msg{* Happy TeXing!}
\Msg{*}
\Msg{************************************************************************}

\endbatchfile
%</install>
%<*ignore>
\fi
%</ignore>
%
% \section{The documentation driver file}
%
% The next bit of code contains the documentation driver file for
% \TeX , i.\,e., the file that will produce the documentation you
% are currently reading. It will be extracted from this file by the
% \verb|docstrip| programme. That is, run \LaTeX{} on \verb|docstrip|
% and specify the \verb|driver| option when \verb|docstrip|
% asks for options.
%
%    \begin{macrocode}
%<*driver>
\NeedsTeXFormat{LaTeX2e}[2015/01/01]
\ProvidesFile{footnoterange.drv}[2019/11/24 v1.0c
            References to ranges of footnotes (HMM)]
\documentclass{ltxdoc}[2015/03/26]% v2.0w
\usepackage{holtxdoc}[2012/03/21]%  v0.24
%% footnoterange may work with earlier versions of LaTeX2e and those
%% class and package, but this was not tested.
%% Please consider updating your LaTeX, class, and package
%% to the most recent version (if they are not already the most
%% recent version).
\hypersetup{%
 pdfsubject={References to ranges of footnotes (HMM)},%
 pdfkeywords={LaTeX, footnoterange, footnotes, footnote, reference, H.-Martin Muench},%
 pdfencoding=auto,%
 pdflang={en},%
 breaklinks=true,%
 linktoc=all,%
 pdfstartview=FitH,%
 pdfpagelayout=OneColumn,%
 bookmarksnumbered=true,%
 bookmarksopen=true,%
 bookmarksopenlevel=3,%
 pdfmenubar=true,%
 pdftoolbar=true,%
 pdfwindowui=true,%
 pdfnewwindow=true%
}
\CodelineIndex
\hyphenation{printing docu-ment}
\gdef\unit#1{\mathord{\thinspace\mathrm{#1}}}
\begin{document}
  \DocInput{footnoterange.dtx}%
\end{document}
%</driver>
%    \end{macrocode}
%
% \fi
%
% \CheckSum{321}
%
% \CharacterTable
%  {Upper-case    \A\B\C\D\E\F\G\H\I\J\K\L\M\N\O\P\Q\R\S\T\U\V\W\X\Y\Z
%   Lower-case    \a\b\c\d\e\f\g\h\i\j\k\l\m\n\o\p\q\r\s\t\u\v\w\x\y\z
%   Digits        \0\1\2\3\4\5\6\7\8\9
%   Exclamation   \!     Double quote  \"     Hash (number) \#
%   Dollar        \$     Percent       \%     Ampersand     \&
%   Acute accent  \'     Left paren    \(     Right paren   \)
%   Asterisk      \*     Plus          \+     Comma         \,
%   Minus         \-     Point         \.     Solidus       \/
%   Colon         \:     Semicolon     \;     Less than     \<
%   Equals        \=     Greater than  \>     Question mark \?
%   Commercial at \@     Left bracket  \[     Backslash     \\
%   Right bracket \]     Circumflex    \^     Underscore    \_
%   Grave accent  \`     Left brace    \{     Vertical bar  \|
%   Right brace   \}     Tilde         \~}
%
% \GetFileInfo{footnoterange.drv}
%
% \begingroup
%   \def\x{\#,\$,\^,\_,\~,\ ,\&,\{,\},\%}%
%   \makeatletter
%   \@onelevel@sanitize\x
% \expandafter\endgroup
% \expandafter\DoNotIndex\expandafter{\x}
% \expandafter\DoNotIndex\expandafter{\string\ }
% \begingroup
%   \makeatletter
%     \lccode`9=32\relax
%     \lowercase{%^^A
%       \edef\x{\noexpand\DoNotIndex{\@backslashchar9}}%^^A
%     }%^^A
%   \expandafter\endgroup\x
% \DoNotIndex{\,,\\}
% \DoNotIndex{\documentclass,\usepackage,\ProvidesPackage,\begin,\end}
% \DoNotIndex{\MessageBreak}
% \DoNotIndex{\NeedsTeXFormat,\DoNotIndex,\verb}
% \DoNotIndex{\def,\edef,\gdef,\global,\xdef}
% \DoNotIndex{\ifx,\listfiles,\mathord,\mathrm}
% \DoNotIndex{\bigskip,\space,\thinspace,\ldots}
% \DoNotIndex{\indent,\noindent,\newline,\linebreak,\pagebreak,\newpage}
% \DoNotIndex{\textbf,\textit,\textsf,\textsc,\texttt,\textquotedblleft,\textquotedblright}
% \DoNotIndex{\plainTeX,\TeX,\LaTeX,\pdfLaTeX}
% \DoNotIndex{\section,\pagenumbering,\smallskip}
% \DoNotIndex{\@ifpackageloaded}
% \DoNotIndex{\arabic, \Alph, \alph, \alphalph, \Roman, \roman, \fnsymbol}
%
% \title{The \xpackage{footnoterange} package}
% \date{2019/11/24 v1.0c}
% \author{H.-Martin M\"{u}nch\\\xemail{Martin.Muench at Uni-Bonn.de}}
%
% \maketitle
%
% \begin{abstract}
% \noindent This package provides the environments |footnoterange| and
% |footnoterange*|. Multiple footnotes inside these environments
% are not referenced as e.\,g.
% \makeatletter%
% \hbox{\textquotedblleft \@textsuperscript {\normalfont 1 2 3}\textquotedblright}
% but as \hbox{\textquotedblleft \@textsuperscript {\normalfont 1--3}\textquotedblright ,\ }
% \makeatother%
% i.\,e. the range (from first to last referred footnote at that place) is given.
% Depending on loading of \xpackage{hyperref} and use of its |hyperfootnotes|-option
% the references are hyperlinked. (References to footnotes in |footnoterange*|
% are never hyperlinked.)
% \end{abstract}
%
% \bigskip
%
% \noindent Disclaimer for web links: The author is not responsible for any contents
% referred to in this work unless if having full knowledge of illegal contents.
% If any damage occurs by the use of information presented there, only the
% author of the respective pages might be liable, not the one who has referred
% to these pages.
%
% \noindent Save per page about $200\unit{ml}$ water,
% $2\unit{g}$ CO$_{2}$ and $2\unit{g}$ wood:
% Therefore please print only if this is really necessary.
%
% \medskip
%
% \tableofcontents
%
%\newpage
%
% \section{Introduction}
% \indent This \LaTeXe{} package provides the environments |footnoterange| and\newline
% |footnoterange*|. Multiple footnotes inside these environments are not referenced
% as e.\,g.
% \makeatletter%
% \hbox{\textquotedblleft \@textsuperscript {\normalfont 1 2 3}\textquotedblright}
% but as \hbox{\textquotedblleft \@textsuperscript {\normalfont 1--3}\textquotedblright ,\ }
% \makeatother%
% i.\,e. the range (from first to last
% referred footnote at that place) is given. Depending on loading of
% \xpackage{hyperref} and use of its |hyperfootnotes|-option the references are
% hyperlinked. References to footnotes in |footnoterange*| are never hyperlinked.\newline
% |\footnotemark\footnotetext| instead of |\footnote| is not supported inside
% |footnoterange| and |footnoterange*| environments.
%
% \bigskip
%
% \section{Usage}
%
% \indent Just load the package placing
% \begin{quote}
%   |\usepackage{footnoterange}|
% \end{quote}
% \noindent in the preamble of your \LaTeXe{} source file. There are no options to
% be used. The package should be loaded after \xpackage{hyperref} (if that package is used)
% and after any other package redefining footnotes. Usage of the |footnoterange| (or
% |footnoterange*|) environment is simple: Just encapsulate the footnotes, which shall
% be referred as range instead of referencing as single ones (see the example file).
%
% About hyperlinks:\newline
% Footnote-references in a |footnoterange*| environment are never hyperlinked.\newline
% Footnote-references in a |footnoterange| environment may be hyperlinked:
% \begin{itemize}
%  \item When the \xpackage{hyperref} package is not loaded, they are not
%         hyperlinked (naturally).
%  \item When the \xpackage{hyperref} package is loaded, it depends on the
%         |hyperfootnotes| option of the \xpackage{hyperref} package:
%  \begin{itemize}
%   \item With |hyperfootnotes=true|, the references are hyperlinked.
%   \item With |hyperfootnotes=false|, the references are not hyperlinked.
%  \end{itemize}
% \end{itemize}
% |\footnote|s inside of the |footnoterange|/{}|footnoterange*| environment
% can use the optional parameter, i.\,e. |\footnote[42]{text}| even if this is the
% third footnote. When giving ranges as reference, this does not seem to be
% such a good idea, if this is not done consistently (i.\,e. footnotes
% |[42], [43], [44]| resulting in 42--44 is OK,
% but |[42]| (original~2), unchanged~3, unchanged~4 resulting in
% 42--4 is confusing). A warning about these changed footnotemarks is written
% into the \xfile{.log}-file.
%
% When there is only a single footnote in a |footnoterange| environment,
% this does not produce some reference like %
% \makeatletter%
% \hbox{\textquotedblleft \@textsuperscript {\normalfont 7--7}\textquotedblright{},\ }%
% \makeatother%
% but only one footnotemark is used automatically. Because it is strange to use
% a |footnoterange| environment without a range of footnotes, an info is written
% into the \xfile{.log}-file about this.
%
% The usage of other footnote-number-representations
% (e.\,g.~\verb|\Roman|, \verb|\roman|, \verb|\Alph|, \verb|\alph|,
% \verb|\fnsymbol|) is also possible, but two things must be taken
% into account: The number of footnote references is restricted (for
% example with \verb|\Alph| only references A to Z are possible), which
% can be fixed e.\,g. with the \url{https://ctan.org/pkg/alphalph} package
% (but not automatically), and references to footnote-symbol-ranges (\verb|\fnsymbol|)
% are probably not very clear.
%
% \subsection*{Options}
% \indent The \xpackage{footnoterange} package takes no options.
%
% \section{Alternatives\label{sec:Alternatives}}
%
% The \xpackage{fnpct} (\url{https://www.ctan.org/pkg/fnpct}) package allows to set
% multiple footnotes (|\multfootnote|), suppress a footnotemark \hbox{(|*|),}
% and customize the multi-footnote-separator \hbox{(|\setfnpct{mult-fn-sep={|
% \emph{something} |}}|)}.
% Combining these with stepping of the footnotecounter gives a footnoterange:
% 
% \begin{verbatim}
% \documentclass{article}
% \usepackage{hyperref}
% \usepackage{fnpct}
% \begin{document}
% The three little pigs built their houses
% out of straw\footnote{not to be confused with hay},
% sticks\footnote{or lumber according to some sources}
% and bricks%
% \makeatletter%
% \setfnpct{mult-fn-sep={--}}%
% \multfootnote{probably fired clay bricks\addtocounter\@mpfn{+1};%
% *or something else;%
% what do I know}. %
% \makeatother%
% \end{document}
% \end{verbatim}
% 
% \noindent If you programmed or found another alternative, which is available at
% \url{https://www.CTAN.org/}, please send an e-mail to me with the name,
% location at CTAN, and a short notice, and I will probably list it here.
%
% \section{Example}
%
%    \begin{macrocode}
%<*example>
\documentclass[british]{article}[2014/09/29]% v1.4h
%%%%%%%%%%%%%%%%%%%%%%%%%%%%%%%%%%%%%%%%%%%%%%%%%%%%%%%%%%%%%%%%%%%%%
\usepackage[%
 extension=pdf,%
 plainpages=false,%
 pdfpagelabels=true,%
 hyperindex=false,%
 pdflang={en},%
 pdftitle={footnoterange package example},%
 pdfauthor={H.-Martin Muench},%
 pdfsubject={Example for the footnoterange package},%
 pdfkeywords={LaTeX, footnoterange, H.-Martin Muench},%
 pdfview=FitH,%
 pdfstartview=FitH,%
 pdfpagelayout=OneColumn,%
 hyperfootnotes=true%
]{hyperref}[2012/11/06]% v6.83m

\usepackage{footnoterange}[2019/11/24]% v1.0c

\gdef\unit#1{\mathord{\thinspace\mathrm{#1}}}%
\listfiles
\begin{document}
\pagenumbering{arabic}
\section*{Example for footnoterange}

This example demonstrates the use of package\newline
\textsf{footnoterange}, v1.0c as of 2019/11/24 (HMM).\newline
The package does not use options.\newline
\verb|footnoterange-example.pdf| can be created by\newline
\verb|pdflatex footnoterange-example.tex|,\newline
and \verb|footnoterange-example.tex| can be created by\newline
\verb|tex footnoterange.dtx|.\newline
\verb|footnoterange.dtx| is at
\url{https://www.ctan.org/pkg/footnoterange}.\newline

\noindent For more details please see the documentation!\newline

\noindent Save per page about $200\unit{ml}$ water,
$2\unit{g}$ CO$_{2}$ and $2\unit{g}$ wood:\newline
Therefore please print only if this is really necessary.\newline

This\footnote{Lorem} text bears a multiplicity of footnotes.
Because the \verb|hyperref| package is used with option
\verb|hyperfootnotes=true|, the footnotes
\begin{footnoterange}
\footnote{ipsum}%
\footnote{dolor}%
\footnote{sit}%
\footnote{amet,}%
\footnote{consectetur}%
\footnote{adipisicing}%
\footnote{elit,}%
\footnote{sed}%
\end{footnoterange}
are hyperlinked.\footnote{do eiusmod\ldots} Using
\verb|hyperfootnotes=false| or not using \verb|hyperref|
will remove the hyperlinks to the footnotes.\medskip

This text also bears a multiplicity of footnotes,
but due to the use of the starred
\begin{footnoterange*}
\footnote{Ut purus elit,}%
\footnote{vestibulum ut,}%
\footnote{placerat ac,}%
\footnote{adipiscing vitae,}%
\footnote{felis.}%
\footnote{Curabitur}%
\footnote{dictum}%
\footnote{gravida}%
\end{footnoterange*}
form of the environment they are not
hyperlinked.\footnote{mauris\ldots}\medskip

The use of the optional parameter for footnotes
\begin{footnoterange}
\footnote[101]{Nam arcu libero,}%
\footnote[102]{nonummy eget,}%
\footnote[103]{consectetuer id,}%
\footnote[104]{vulputate a,}%
\footnote[105]{magna.}
\end{footnoterange}
is also possible, but should be used consistently.

\pagebreak

When there is only one footnote
\begin{footnoterange}
\footnote{a single footnote}
\end{footnoterange}
in the \verb|footnoterange| environment, only one
footnotemark is used and an info is written into the
\verb|log|-file.\medskip

The usage of other footnote-number-representations
(e.\,g.~\verb|\Roman|, \verb|\roman|, \verb|\Alph|, \verb|\alph|,
\verb|\fnsymbol|) is also possible, but two things must be taken
into account: The number of footnote references is restricted
(for example with \verb|\Alph| only references A to Z are possible),
which can be fixed e.\,g. with the
\href{https://www.ctan.org/pkg/alphalph}{\texttt{alphalph} package},
and references to footnote-symbol-ranges (\verb|\fnsymbol|)
are probably not very clear.

\end{document}
%</example>
%    \end{macrocode}
%
% \StopEventually{}
%
% \section{The implementation}
%
% We start off by checking that we are loading into \LaTeXe{} and
% announcing the name and version of this package.
%
%    \begin{macrocode}
%<*package>
%    \end{macrocode}
%
%    \begin{macrocode}
\NeedsTeXFormat{LaTeX2e}[2011/06/27]
\ProvidesPackage{footnoterange}[2019/11/24 v1.0c
            References to ranges of footnotes (HMM)]
%    \end{macrocode}
%
% A short description of the \xpackage{footnoterange} package:
%
%    \begin{macrocode}
%% Provides the footnoterange and footnoterange* environments to
%% refer to ranges of footnotes.
%    \end{macrocode}
%
% We need the \xpackage{ltxcmds} package:
%
%    \begin{macrocode}
\RequirePackage{ltxcmds}[2011/11/09]% v1.22; for "\ltx@ifpackageloaded"
\RequirePackage{letltxmacro}[2010/09/02]% v1.4; for "\GlobalLetLtxMacro"

%    \end{macrocode}
%
% We use |\ifHy@hyperfootnotes ... \else ... \fi|, but when the \xpackage{hyperref}
% package has not been loaded, |\ifHy@hyperfootnotes| is not defined, and \LaTeX
% relates the |\else| and/or |\fi| (wrongly) to other |\if|s or groups.
% Therefore the |\ifHy@hyperfootnotes| are placed in their own commands,
% which are not looked at when the \xpackage{hyperref} package has not been loaded.
%
% \pagebreak
%
% \begin{macro}{\fnr@ifhyperrfn}
%    \begin{macrocode}
\newcommand{\fnr@ifhyperrfn}{%
  \ifHy@hyperfootnotes% option hyperfootnotes=true
    \addtocounter{Hfootnote}{+1}%
    % Code from the hyperref package
      \global\let\Hy@saved@currentHref\@currentHref%
      \hyper@makecurrent{Hfootnote}%
      \global\let\Hy@footnote@currentHref\@currentHref%
      \global\let\@currentHref\Hy@saved@currentHref%
    % End of code from the hyperref package
  \fi%
}

%    \end{macrocode}
% \end{macro}
%
% \begin{macro}{\fnr@rangefootnote}
% The |\fnr@rangefootnote| is the internal replacement for the original\newline
% |\footnote| command inside of a |footnoterange| environment.
%
%    \begin{macrocode}
\newcommand{\fnr@rangefootnote}[2][\empty]{%
%    \end{macrocode}
%
% |#1| contains the optional argument, which is predefined as |\empty|.
%
%    \begin{macrocode}
  \nolinebreak%
  % no \footnotemark here
  \addtocounter\@mpfn{+1}%
  \ltx@ifpackageloaded{hyperref}{\fnr@ifhyperrfn}{\relax}%
%    \end{macrocode}
%
% See the explanation before the declaration of the |\fnr@ifhyperrfn| command!
%
%    \begin{macrocode}
  \xdef\fnr@opt{#1}% contains the optional argument
  \xdef\fnr@arabic{\arabic\@mpfn}%
  \protected@edef\fnr@formated{\thefootnote}%
%    \end{macrocode}
%
% The |\footnotemark| is not set (yet), but the |\footnotetext|,
% and it gets a label.
%
%    \begin{macrocode}
  \ifx\fnr@opt\empty%
    \footnotetext{\label{fnr:\fnr@arabic}#2}%
  \else%
    \PackageWarning{footnoterange}{%
      Custom footnote-numbers used in a footnoterange\MessageBreak%
      environment: Footnote with natural number\MessageBreak%
      \ifx\fnr@formated\fnr@arabic%
        \fnr@arabic\space%
      \else%
        \space\fnr@arabic\space (\fnr@formated )%
      \fi%
      \MessageBreak%
      was renumbered to \fnr@opt\MessageBreak%
     }%
    \ltx@ifpackageloaded{hyperref}{% hyperref loaded
      \footnotetext[#1]{\phantomsection\label{fnr:\fnr@arabic}#2}%
     }{% hyperref not loaded
      \footnotetext[#1]{\label{fnr:\fnr@arabic}#2}%
     }%
  \fi%
}

%    \end{macrocode}
% \end{macro}
%
%    \begin{macrocode}

%    \end{macrocode}
% \pagebreak
%
% |\ifHy@hyperfootnotes| \textquotedblleft encapsulation\textquotedblright{} again:
%
% \begin{macro}{\fnr@ifhyperfnrA}
%    \begin{macrocode}
\newcommand{\fnr@ifhyperfnrA}{%
  \ifHy@hyperfootnotes% option hyperfootnotes=true
    \hbox{\hyperref[fnr:\fnr@first]{%
          \@textsuperscript{\normalfont \ref*{fnr:\fnr@first}}}}%
  \else% option hyperfootnotes=false
    \hbox{\@textsuperscript{\normalfont \ref*{fnr:\fnr@first}}}%
  \fi%
}

%    \end{macrocode}
% \end{macro}
%
% \begin{macro}{\fnr@ifhyperfnrN}
%    \begin{macrocode}
\newcommand{\fnr@ifhyperfnrN}{%
  \ifHy@hyperfootnotes% option hyperfootnotes=true
    \hbox{\hyperref[fnr:\fnr@first]{%
            \@textsuperscript{\normalfont \ref*{fnr:\fnr@first}}}%
          \@textsuperscript{\normalfont --}%
          \hyperref[fnr:\arabic\@mpfn]{%
            \@textsuperscript{\normalfont \ref*{fnr:\arabic\@mpfn}}}}%
  \else% option hyperfootnotes=false
    \hbox{\@textsuperscript{\normalfont \ref*{fnr:\fnr@first}%
                                        --\ref*{fnr:\arabic\@mpfn}}}%
  \fi%
}

%    \end{macrocode}
% \end{macro}
%
% \begin{environment}{footnoterange}
% The |footnoterange| environment remembers the next footnote number as\newline
% |\fnr@first| and the current |\footnote| definition as |\fnr@origfn|
% and uses |\fnr@rangefootnote| instead.
%
%    \begin{macrocode}
\newenvironment{footnoterange}{%
  % at begin of the footnoterange environment
  \ifhmode \unskip \fi% FMi
  \addtocounter\@mpfn{+1}%
  \xdef\fnr@first{\arabic\@mpfn}%
  \addtocounter\@mpfn{-1}%
  \let\fnr@origfn\footnote%
  \let\footnote\fnr@rangefootnote%
  \GlobalLetLtxMacro{\fnr@origfnmark}{\footnotemark}%
  \GlobalLetLtxMacro{\footnotemark}{\fnr@rangefootnotemark}%
  \ignorespaces% FMi
%    \end{macrocode}
%
% End of the environment:
%
%    \begin{macrocode}
}{% at end of the footnoterange environment
  \ifhmode% FMi 
    \unskip% FMi
%    \end{macrocode}
%
% Remember the spacefactor for restoring after the environment has ended
% (only in |hmode|!). \% FMi
%
%    \begin{macrocode}
    \xdef\fnr@aftergroup{\spacefactor\the\spacefactor\relax}% FMi
  \else% FMi
%    \end{macrocode}
%
% Otherwise do something harmless. \% FMi
%
%    \begin{macrocode}
    \xdef\fnr@aftergroup{\relax}%
  \fi% FMi
%    \end{macrocode}
%
% At the end of the environment the previous definition of |\footnote| is restored
%
%    \begin{macrocode}
  \let\footnote\fnr@origfn%
  \GlobalLetLtxMacro{\footnotemark}{\fnr@origfnmark}%
%    \end{macrocode}
%
% and the references are placed:
%
%    \begin{macrocode}
  \ifx\fnr@first\fnr@arabic% There is only one reference in the environment.
    \PackageInfo{footnoterange}{%
      Environment footnoterange used,\MessageBreak%
      but only one footnote (number \fnr@first ) is contained%
      \MessageBreak%
     }%
    \ltx@ifpackageloaded{hyperref}{\fnr@ifhyperfnrA% hyperref package loaded
    }{% hyperref package not loaded
      \hbox{\@textsuperscript{\normalfont \ref{fnr:\fnr@first}}}%
     }%
  \else%
    \ltx@ifpackageloaded{hyperref}{\fnr@ifhyperfnrN% hyperref package loaded
    }{% hyperref package not loaded
      \hbox{\@textsuperscript{\normalfont \ref{fnr:\fnr@first}%
                                        --\ref{fnr:\arabic\@mpfn}}}%
     }%
  \fi%
%    \end{macrocode}
%
% Restore the spacefactor after the environment has ended. \% FMi
%
%    \begin{macrocode}
  \aftergroup\fnr@aftergroup% FMi
}

%    \end{macrocode}
% \end{environment}
%
% \begin{environment}{footnoterange*}
% The |footnoterange*| environment does the same as the |footnoterange|
% environment but suppresses all hyperlinks to footnotes inside.
%
%    \begin{macrocode}
\newenvironment{footnoterange*}{%
  % at begin of the footnoterange* environment
  \ifhmode \unskip \fi% FMi
  \addtocounter\@mpfn{+1}%
  \xdef\fnr@first{\arabic\@mpfn}%
  \addtocounter\@mpfn{-1}%
  \let\fnr@origfn\footnote%
  \let\footnote\fnr@rangefootnote%
  \GlobalLetLtxMacro{\fnr@origfnmark}{\footnotemark}%
  \GlobalLetLtxMacro{\footnotemark}{\fnr@rangefootnotemark}%
  \ignorespaces% FMi
}{% at end of the footnoterange* environment
  \ifhmode% FMi 
    \unskip% FMi
    \xdef\fnr@aftergroup{\spacefactor\the\spacefactor\relax}% FMi
  \else% FMi
    \xdef\fnr@aftergroup{\relax}%
  \fi% FMi
  \let\footnote\fnr@origfn%
  \GlobalLetLtxMacro{\footnotemark}{\fnr@origfnmark}%
  \ifx\fnr@first\fnr@arabic% There is only one reference in the environment.
    \PackageInfo{footnoterange}{%
      Environment footnoterange used,\MessageBreak%
      but only one footnote (number \fnr@first ) is contained%
      \MessageBreak%
     }%
%    \end{macrocode}
% \pagebreak
%    \begin{macrocode}
    \ltx@ifpackageloaded{hyperref}{% hyperref package loaded
      \hbox{\@textsuperscript{\normalfont \ref*{fnr:\fnr@first}}}%
    }{% hyperref package not loaded
      \hbox{\@textsuperscript{\normalfont \ref{fnr:\fnr@first}}}%
     }%
  \else%
    \ltx@ifpackageloaded{hyperref}{% hyperref package loaded
      \hbox{\@textsuperscript{\normalfont \ref*{fnr:\fnr@first}%
                                        --\ref*{fnr:\arabic\@mpfn}}}%
    }{% hyperref package not loaded
      \hbox{\@textsuperscript{\normalfont \ref{fnr:\fnr@first}%
                                        --\ref{fnr:\arabic\@mpfn}}}%
     }%
  \fi%
  \aftergroup\fnr@aftergroup% FMi
}

%    \end{macrocode}
% \end{environment}
%
% The \xpackage{footnoterange} package does not handle |\footnotemark|s,
% but at least the user is made aware of this:
%
% \begin{macro}{\fnr@rangefootnotemark}
%    \begin{macrocode}

\newcommand{\fnr@rangefootnotemark}[1][\empty]{%
  \PackageError{footnoterange}{%
     \string\footnotemark\space not supported in footnoterange(*) environment%
   }{The footnoterange package does not handle a \string\footnotemark.\MessageBreak%
     Either use \string\footnote\string{...\string}\space instead of%
     \string\footnotemark\space and \string\footnotetext\string{...\string}\MessageBreak%
     in footnoterange and footnoterange* environments,\MessageBreak%
     or do not use those environments around a \string\footnotemark\MessageBreak%
     (or \string\renewcommand\string{\string\fnr@rangefootnotemark\string}\MessageBreak%
     with your own fixing code).%
    }%
  \xdef\fnr@mopt{#1}%
  \ifx\fnr@mopt\empty%
    \fnr@origfnmark%
  \else%
    \fnr@origfnmark[#1]%
  \fi%
}

%    \end{macrocode}
% \end{macro}
%
%    \begin{macrocode}
%</package>
%    \end{macrocode}
%
% \pagebreak
%
% \section{Installation}
%
% \subsection{Downloads\label{ss:Downloads}}
%
% Everything is available on CTAN,
% \url{https://www.ctan.org/tex-archive/},
% but may need additional packages themselves.
%
% \DescribeMacro{footnoterange.dtx}
% For unpacking the \xfile{footnoterange.dtx} file
% and constructing the documentation it is required:
% \begin{description}
% \item[-] \TeX{} Format \LaTeXe{}, \url{https://www.CTAN.org/}
%
% \item[-] document class \xpackage{ltxdoc}, 2015/03/26, v2.0w,
%           \url{https://ctan.org/pkg/ltxdoc}
%
% \item[-] package \xpackage{holtxdoc}, 2012/03/21, v0.24,
%           \url{https://ctan.org/pkg/holtxdoc}
% \end{description}
%
% \DescribeMacro{footnoterange.sty}
% The |footnoterange.sty| for \LaTeXe{} (i.\,e. each document using
% the \xpackage{footnoterange} package) requires:
% \begin{description}
% \item[-] \TeX{} Format \LaTeXe{}, \url{https://www.CTAN.org/}
%
% \item[-] package \xpackage{ltxcmds}, 2011/04/18, v1.20,
%           \url{https://www.ctan.org/pkg/ltxcmds}
% \end{description}
%
% \DescribeMacro{footnoterange-example.tex}
% The |footnoterange-example.tex| requires the same files as all
% documents using the \xpackage{footnoterange} package and additionally:
% \begin{description}
% \item[-] class \xpackage{article}, 2007/10/19, v1.4h, from \xpackage{classes.dtx}:\newline
%           \url{https://www.ctan.org/tex-archive/macros/latex/base/}
%
% \item[-] package \xpackage{footnoterange}, 2019/11/24, v1.0c,\newline
%           \url{https://www.ctan.org/pkg/footnoterange}\newline
%           (Well, it is the example file for this package, and because you are
%            reading the documentation for the \xpackage{footnoterange} package,
%            it can be assumed that you already have some version of it --
%            is it the current one?)
%
% \item[-] package \xpackage{hyperref}, 2012/11/06, v6.83m,\newline
%           \url{https://www.ctan.org/pkg/hyperref}
% \end{description}
%
% \DescribeMacro{Oberdiek}
% \DescribeMacro{holtxdoc}
% \DescribeMacro{ltxcmds}
% \DescribeMacro{alphalph}
% All packages of \textsc{Heiko Oberdiek}'s bundle `oberdiek'
% (especially \xpackage{holtxdoc}, \xpackage{ltxcmds}, and \xpackage{alphalph})
% are also available in a TDS compliant ZIP archive:\newline
% \url{http://mirror.ctan.org/install/macros/latex/contrib/oberdiek.tds.zip}.\newline
% It is probably best to download and use this, because the packages in there
% should be both recent and compatible.
%
% \DescribeMacro{hyperref}
% \noindent \xpackage{hyperref} is not included in that bundle and needs to be
% downloaded separately,\newline
% \url{http://mirrors.ctan.org/install/macros/latex/contrib/hyperref.tds.zip}.
%
% \DescribeMacro{M\"{u}nch}
% A list of my packages can be found at \url{https://www.ctan.org/author/muench-hm}.
%
% \pagebreak
%
% \subsection{Package, unpacking TDS}
% \paragraph{Package.} This package is available on \url{https://www.CTAN.org}.
% \begin{description}
% \item[\url{http://mirror.ctan.org/macros/latex/contrib/footnoterange/footnoterange.dtx}]\hspace*{0.1cm}
%       The source file.
% \item[\url{http://mirror.ctan.org/macros/latex/contrib/footnoterange/footnoterange.pdf}]\hspace*{0.1cm}
%       The documentation.
% \item[\url{http://mirror.ctan.org/macros/latex/contrib/footnoterange/footnoterange-example.pdf}]\hspace*{0.1cm}
%       The compiled example file, as it should look like.
% \item[\url{http://mirror.ctan.org/macros/latex/contrib/footnoterange/README}]\hspace*{0.1cm}\newline\hspace*{0.1cm}
%       The README file.
% \end{description}
% There is also a footnoterange.tds.zip available:
% \begin{description}
% \item[\url{http://mirror.ctan.org/install/macros/latex/contrib/footnoterange.tds.zip}]\hspace*{0.1cm}
%       Everything in TDS compliant, compiled format.
% \end{description}
% which additionally contains\newline
% \begin{tabular}{ll}
% footnoterange.ins & The installation file.\\
% footnoterange.drv & The driver to generate the documentation.\\
% footnoterange.sty & The \xext{sty}le file.
% \end{tabular}
%
% \bigskip
%
% \noindent For required other packages, please see the preceding subsection.
%
% \paragraph{Unpacking.} The \xfile{.dtx} file is a self-extracting
% \docstrip{} archive. The files are extracted by running the
% \xfile{.dtx} through \plainTeX:
% \begin{quote}
%   \verb|tex footnoterange.dtx|
% \end{quote}
%
% About generating the documentation see paragraph~\ref{GenDoc} below.
%
% \paragraph{TDS.} Now the different files must be moved into
% the different directories in your installation TDS tree
% (also known as \xfile{texmf} tree):
% \begin{quote}
% \def\t{^^A
% \begin{tabular}{@{}>{\ttfamily}l@{ $\rightarrow$ }>{\ttfamily}l@{}}
%   footnoterange.sty & tex/latex/footnoterange/footnoterange.sty\\
%   footnoterange.pdf & doc/latex/footnoterange/footnoterange.pdf\\
%   footnoterange-example.tex & doc/latex/footnoterange/footnoterange-example.tex\\
%   footnoterange-example.pdf & doc/latex/footnoterange/footnoterange-example.pdf\\
%   footnoterange.dtx & source/latex/footnoterange/footnoterange.dtx\\
% \end{tabular}^^A
% }^^A
% \sbox0{\t}^^A
% \ifdim\wd0>\linewidth
%   \begingroup
%     \advance\linewidth by\leftmargin
%     \advance\linewidth by\rightmargin
%   \edef\x{\endgroup
%     \def\noexpand\lw{\the\linewidth}^^A
%   }\x
%   \def\lwbox{^^A
%     \leavevmode
%     \hbox to \linewidth{^^A
%       \kern-\leftmargin\relax
%       \hss
%       \usebox0
%       \hss
%       \kern-\rightmargin\relax
%     }^^A
%   }^^A
%   \ifdim\wd0>\lw
%     \sbox0{\small\t}^^A
%     \ifdim\wd0>\linewidth
%       \ifdim\wd0>\lw
%         \sbox0{\footnotesize\t}^^A
%         \ifdim\wd0>\linewidth
%           \ifdim\wd0>\lw
%             \sbox0{\scriptsize\t}^^A
%             \ifdim\wd0>\linewidth
%               \ifdim\wd0>\lw
%                 \sbox0{\tiny\t}^^A
%                 \ifdim\wd0>\linewidth
%                   \lwbox
%                 \else
%                   \usebox0
%                 \fi
%               \else
%                 \lwbox
%               \fi
%             \else
%               \usebox0
%             \fi
%           \else
%             \lwbox
%           \fi
%         \else
%           \usebox0
%         \fi
%       \else
%         \lwbox
%       \fi
%     \else
%       \usebox0
%     \fi
%   \else
%     \lwbox
%   \fi
% \else
%   \usebox0
% \fi
% \end{quote}
% If you have a \xfile{docstrip.cfg} that configures and enables \docstrip's
% TDS installing feature, then some files can already be in the right
% place, see the documentation of \docstrip.
%
% \subsection{Refresh file name databases}
%
% If your \TeX~distribution (\TeX{} Live, \mikTeX, \teTeX, \dots) relies on
% file name databases, you must refresh these. For example, \teTeX{} users run
% \verb|texhash| or \verb|mktexlsr|.
%
% \subsection{Some details for the interested}
%
% \paragraph{Unpacking with \LaTeX{}.}
% The \xfile{.dtx} chooses its action depending on the format:
% \begin{description}
% \item[\plainTeX:] Run \docstrip{} and extract the files.
% \item[\LaTeX:] Generate the documentation.
% \end{description}
% If you insist on using \LaTeX{} for \docstrip{} (really,
% \docstrip{} does not need \LaTeX ), then inform the autodetect routine
% about your intention:
% \begin{quote}
%   \verb|latex \let\install=y\input{footnoterange.dtx}|
% \end{quote}
% Do not forget to quote the argument according to the demands
% of your shell.
%
% \paragraph{Generating the documentation.\label{GenDoc}}
% You can use both the \xfile{.dtx} or the \xfile{.drv} to generate
% the documentation. The process can be configured by a
% configuration file \xfile{ltxdoc.cfg}. For instance, put the following
% line into this file, if you want to have A4 as paper format:
% \begin{quote}
%   \verb|\PassOptionsToClass{a4paper}{article}|
% \end{quote}
%
% \noindent An example follows how to generate the
% documentation with \pdfLaTeX :
%
% \begin{quote}
%\begin{verbatim}
%pdflatex footnoterange.drv
%makeindex -s gind.ist footnoterange.idx
%pdflatex footnoterange.drv
%makeindex -s gind.ist footnoterange.idx
%pdflatex footnoterange.drv
%\end{verbatim}
% \end{quote}
%
% \subsection{Compiling the example}
%
% The example file, \textsf{footnoterange-example.tex}, can be compiled via\newline
% \indent \verb|(pdf)latex footnoterange-example.tex|\newline
% but will need probably two compiler runs to get everything right.\newline
% \verb|footnoterange-example.tex| can be created by\newline
% \indent \verb|tex footnoterange.dtx|.\newline
% \verb|footnoterange.dtx| is at
% \url{https://www.ctan.org/pkg/footnoterange}.\newline
%
% \section{Acknowledgements}
%
% I (H.-Martin M\"{u}nch) would like to thank \textsc{Heiko Oberdiek}
% for providing a~lot~(!) of useful packages (from
% which I also got everything I know about creating a file in \xext{dtx}
% format, ok, say it: copying), \textsc{Frank Mittelbach} for several bug reports,
% for code for improving the package, and for pointing me to the \xpackage{fnpct}
% package as alternative, Willie Wong for reporting a bug and its fix,
% everybody of the CTAN team for managing CTAN,
% the \Newsgroup{comp.text.tex} and \Newsgroup{de.comp.text.tex} newsgroups,
% and everybody at \url{https://tex.stackexchange.com/}
% for their help in all things \hbox{\TeX{}.}
% This package was created as response to question
% \textquotedblleft Compressing consecutive footnote
% marks\textquotedblright{} by user001\newline
% (\url{https://tex.stackexchange.com/users/10127/user001}) on\newline
% \url{https://tex.stackexchange.com/q/44165/6865}.
%
% \newpage
% \phantomsection
% \begin{History}\label{History}
%   \begin{Version}{2011/02/11 v0.1ff.}
%     \item Some versions at \url{https://tex.stackexchange.com/q/44165/6865}.
%   \end{Version}
%   \begin{Version}{2012/02/17 v1.0a}
%     \item Package created and uploaded to CTAN.
%   \end{Version}
%   \begin{Version}{2018/10/15 v1.0b}
%     \item Fixed/updated minor details.
%     \item Bug fix: Did not work without \xpackage{hyperref}, wrong spacefactor.
%     \item Listing the \href{https://www.ctan.org/pkg/fnpct}{\xpackage{fnpct}}
%           package as alternative.
%   \end{Version}
%   \begin{Version}{2019/11/24 v1.0c}
%     \item Bug fix: Issue with footnote symbols.
%   \end{Version}
% \end{History}
%
% \bigskip
%
% When you find a mistake or have a suggestion for an improvement of this package,
% please send an e-mail to the maintainer, thanks! (Please see BUG REPORTS in the README.)
%
% \PrintIndex
%
% \Finale
\endinput
%        (quote the arguments according to the demands of your shell)
%
% Documentation:
%    (a) If footnoterange.drv is present:
%           (pdf)latex footnoterange.drv
%           makeindex -s gind.ist footnoterange.idx
%           (pdf)latex footnoterange.drv
%           makeindex -s gind.ist footnoterange.idx
%           (pdf)latex footnoterange.drv
%    (b) Without footnoterange.drv:
%           (pdf)latex footnoterange.dtx
%           makeindex -s gind.ist footnoterange.idx
%           (pdf)latex footnoterange.dtx
%           makeindex -s gind.ist footnoterange.idx
%           (pdf)latex footnoterange.dtx
%
%    The class ltxdoc loads the configuration file ltxdoc.cfg
%    if available. Here you can specify further options, e.g.
%    use DIN A4 as paper format:
%       \PassOptionsToClass{a4paper}{article}
%
% Installation:
%    TDS:tex/latex/footnoterange/footnoterange.sty
%    TDS:doc/latex/footnoterange/footnoterange.pdf
%    TDS:doc/latex/footnoterange/footnoterange-example.tex
%    TDS:doc/latex/footnoterange/footnoterange-example.pdf
%    TDS:source/latex/footnoterange/footnoterange.dtx
%
%<*ignore>
\begingroup
  \catcode123=1 %
  \catcode125=2 %
  \def\x{LaTeX2e}%
\expandafter\endgroup
\ifcase 0\ifx\install y1\fi\expandafter
         \ifx\csname processbatchFile\endcsname\relax\else1\fi
         \ifx\fmtname\x\else 1\fi\relax
\else\csname fi\endcsname
%</ignore>
%<*install>
\input docstrip.tex
\Msg{*********************************************************************************}
\Msg{* Installation}
\Msg{* Package: footnoterange 2019/11/24 v1.0c References to ranges of footnotes (HMM)}
\Msg{*********************************************************************************}

\keepsilent
\askforoverwritefalse

\let\MetaPrefix\relax
\preamble

This is a generated file.

Project: footnoterange
Version: 2019/11/24 v1.0c

Copyright (C) 2012 - 2019 by
    H.-Martin M"unch <Martin dot Muench at Uni-Bonn dot de>
Portions of code copyrighted by other people as marked;
lines marked "FMi": code by Frank Mittelbach, thanks!

The usual disclaimer applies:
If it doesn't work right that's your problem.
(Nevertheless, please send an e-mail to the maintainer
 when you find an error in this package.)

This work may be distributed and/or modified under the
conditions of the LaTeX Project Public License, either
version 1.3c of this license or (at your option) any later
version. This version of this license is in
   https://www.latex-project.org/lppl/lppl-1-3c.txt
and the latest version of this license is in
   https://www.latex-project.org/lppl.txt
and version 1.3c or later is part of all distributions of
LaTeX version 2005/12/01 or later.

This work has the LPPL maintenance status "maintained".

The Current Maintainer of this work is H.-Martin Muench.

This work consists of the main source file footnoterange.dtx,
the README, and the derived files
   footnoterange.sty, footnoterange.pdf,
   footnoterange.ins, footnoterange.drv,
   footnoterange-example.tex, footnoterange-example.pdf.

In memoriam
 Claudia Simone Barth + 1996/01/30
 Tommy Muench + 2014/01/02
 Hans-Klaus Muench + 2014/08/24

\endpreamble
\let\MetaPrefix\DoubleperCent

\generate{%
  \file{footnoterange.ins}{\from{footnoterange.dtx}{install}}%
  \file{footnoterange.drv}{\from{footnoterange.dtx}{driver}}%
  \usedir{tex/latex/footnoterange}%
  \file{footnoterange.sty}{\from{footnoterange.dtx}{package}}%
  \usedir{doc/latex/footnoterange}%
  \file{footnoterange-example.tex}{\from{footnoterange.dtx}{example}}%
}

\catcode32=13\relax% active space
\let =\space%
\Msg{************************************************************************}
\Msg{*}
\Msg{* To finish the installation you have to move the following}
\Msg{* file into a directory searched by TeX:}
\Msg{*}
\Msg{* footnoterange.sty}
\Msg{*}
\Msg{* To produce the documentation run the file `footnoterange.drv'}
\Msg{* through (pdf)LaTeX, e.g.}
\Msg{*  pdflatex footnoterange.drv}
\Msg{*  makeindex -s gind.ist footnoterange.idx}
\Msg{*  pdflatex footnoterange.drv}
\Msg{*  makeindex -s gind.ist footnoterange.idx}
\Msg{*  pdflatex footnoterange.drv}
\Msg{*}
\Msg{* At least three runs are necessary e.g. to get the}
\Msg{*  references right!}
\Msg{*}
\Msg{* Happy TeXing!}
\Msg{*}
\Msg{************************************************************************}

\endbatchfile
%</install>
%<*ignore>
\fi
%</ignore>
%
% \section{The documentation driver file}
%
% The next bit of code contains the documentation driver file for
% \TeX , i.\,e., the file that will produce the documentation you
% are currently reading. It will be extracted from this file by the
% \verb|docstrip| programme. That is, run \LaTeX{} on \verb|docstrip|
% and specify the \verb|driver| option when \verb|docstrip|
% asks for options.
%
%    \begin{macrocode}
%<*driver>
\NeedsTeXFormat{LaTeX2e}[2015/01/01]
\ProvidesFile{footnoterange.drv}[2019/11/24 v1.0c
            References to ranges of footnotes (HMM)]
\documentclass{ltxdoc}[2015/03/26]% v2.0w
\usepackage{holtxdoc}[2012/03/21]%  v0.24
%% footnoterange may work with earlier versions of LaTeX2e and those
%% class and package, but this was not tested.
%% Please consider updating your LaTeX, class, and package
%% to the most recent version (if they are not already the most
%% recent version).
\hypersetup{%
 pdfsubject={References to ranges of footnotes (HMM)},%
 pdfkeywords={LaTeX, footnoterange, footnotes, footnote, reference, H.-Martin Muench},%
 pdfencoding=auto,%
 pdflang={en},%
 breaklinks=true,%
 linktoc=all,%
 pdfstartview=FitH,%
 pdfpagelayout=OneColumn,%
 bookmarksnumbered=true,%
 bookmarksopen=true,%
 bookmarksopenlevel=3,%
 pdfmenubar=true,%
 pdftoolbar=true,%
 pdfwindowui=true,%
 pdfnewwindow=true%
}
\CodelineIndex
\hyphenation{printing docu-ment}
\gdef\unit#1{\mathord{\thinspace\mathrm{#1}}}
\begin{document}
  \DocInput{footnoterange.dtx}%
\end{document}
%</driver>
%    \end{macrocode}
%
% \fi
%
% \CheckSum{321}
%
% \CharacterTable
%  {Upper-case    \A\B\C\D\E\F\G\H\I\J\K\L\M\N\O\P\Q\R\S\T\U\V\W\X\Y\Z
%   Lower-case    \a\b\c\d\e\f\g\h\i\j\k\l\m\n\o\p\q\r\s\t\u\v\w\x\y\z
%   Digits        \0\1\2\3\4\5\6\7\8\9
%   Exclamation   \!     Double quote  \"     Hash (number) \#
%   Dollar        \$     Percent       \%     Ampersand     \&
%   Acute accent  \'     Left paren    \(     Right paren   \)
%   Asterisk      \*     Plus          \+     Comma         \,
%   Minus         \-     Point         \.     Solidus       \/
%   Colon         \:     Semicolon     \;     Less than     \<
%   Equals        \=     Greater than  \>     Question mark \?
%   Commercial at \@     Left bracket  \[     Backslash     \\
%   Right bracket \]     Circumflex    \^     Underscore    \_
%   Grave accent  \`     Left brace    \{     Vertical bar  \|
%   Right brace   \}     Tilde         \~}
%
% \GetFileInfo{footnoterange.drv}
%
% \begingroup
%   \def\x{\#,\$,\^,\_,\~,\ ,\&,\{,\},\%}%
%   \makeatletter
%   \@onelevel@sanitize\x
% \expandafter\endgroup
% \expandafter\DoNotIndex\expandafter{\x}
% \expandafter\DoNotIndex\expandafter{\string\ }
% \begingroup
%   \makeatletter
%     \lccode`9=32\relax
%     \lowercase{%^^A
%       \edef\x{\noexpand\DoNotIndex{\@backslashchar9}}%^^A
%     }%^^A
%   \expandafter\endgroup\x
% \DoNotIndex{\,,\\}
% \DoNotIndex{\documentclass,\usepackage,\ProvidesPackage,\begin,\end}
% \DoNotIndex{\MessageBreak}
% \DoNotIndex{\NeedsTeXFormat,\DoNotIndex,\verb}
% \DoNotIndex{\def,\edef,\gdef,\global,\xdef}
% \DoNotIndex{\ifx,\listfiles,\mathord,\mathrm}
% \DoNotIndex{\bigskip,\space,\thinspace,\ldots}
% \DoNotIndex{\indent,\noindent,\newline,\linebreak,\pagebreak,\newpage}
% \DoNotIndex{\textbf,\textit,\textsf,\textsc,\texttt,\textquotedblleft,\textquotedblright}
% \DoNotIndex{\plainTeX,\TeX,\LaTeX,\pdfLaTeX}
% \DoNotIndex{\section,\pagenumbering,\smallskip}
% \DoNotIndex{\@ifpackageloaded}
% \DoNotIndex{\arabic, \Alph, \alph, \alphalph, \Roman, \roman, \fnsymbol}
%
% \title{The \xpackage{footnoterange} package}
% \date{2019/11/24 v1.0c}
% \author{H.-Martin M\"{u}nch\\\xemail{Martin.Muench at Uni-Bonn.de}}
%
% \maketitle
%
% \begin{abstract}
% \noindent This package provides the environments |footnoterange| and
% |footnoterange*|. Multiple footnotes inside these environments
% are not referenced as e.\,g.
% \makeatletter%
% \hbox{\textquotedblleft \@textsuperscript {\normalfont 1 2 3}\textquotedblright}
% but as \hbox{\textquotedblleft \@textsuperscript {\normalfont 1--3}\textquotedblright ,\ }
% \makeatother%
% i.\,e. the range (from first to last referred footnote at that place) is given.
% Depending on loading of \xpackage{hyperref} and use of its |hyperfootnotes|-option
% the references are hyperlinked. (References to footnotes in |footnoterange*|
% are never hyperlinked.)
% \end{abstract}
%
% \bigskip
%
% \noindent Disclaimer for web links: The author is not responsible for any contents
% referred to in this work unless if having full knowledge of illegal contents.
% If any damage occurs by the use of information presented there, only the
% author of the respective pages might be liable, not the one who has referred
% to these pages.
%
% \noindent Save per page about $200\unit{ml}$ water,
% $2\unit{g}$ CO$_{2}$ and $2\unit{g}$ wood:
% Therefore please print only if this is really necessary.
%
% \medskip
%
% \tableofcontents
%
%\newpage
%
% \section{Introduction}
% \indent This \LaTeXe{} package provides the environments |footnoterange| and\newline
% |footnoterange*|. Multiple footnotes inside these environments are not referenced
% as e.\,g.
% \makeatletter%
% \hbox{\textquotedblleft \@textsuperscript {\normalfont 1 2 3}\textquotedblright}
% but as \hbox{\textquotedblleft \@textsuperscript {\normalfont 1--3}\textquotedblright ,\ }
% \makeatother%
% i.\,e. the range (from first to last
% referred footnote at that place) is given. Depending on loading of
% \xpackage{hyperref} and use of its |hyperfootnotes|-option the references are
% hyperlinked. References to footnotes in |footnoterange*| are never hyperlinked.\newline
% |\footnotemark\footnotetext| instead of |\footnote| is not supported inside
% |footnoterange| and |footnoterange*| environments.
%
% \bigskip
%
% \section{Usage}
%
% \indent Just load the package placing
% \begin{quote}
%   |\usepackage{footnoterange}|
% \end{quote}
% \noindent in the preamble of your \LaTeXe{} source file. There are no options to
% be used. The package should be loaded after \xpackage{hyperref} (if that package is used)
% and after any other package redefining footnotes. Usage of the |footnoterange| (or
% |footnoterange*|) environment is simple: Just encapsulate the footnotes, which shall
% be referred as range instead of referencing as single ones (see the example file).
%
% About hyperlinks:\newline
% Footnote-references in a |footnoterange*| environment are never hyperlinked.\newline
% Footnote-references in a |footnoterange| environment may be hyperlinked:
% \begin{itemize}
%  \item When the \xpackage{hyperref} package is not loaded, they are not
%         hyperlinked (naturally).
%  \item When the \xpackage{hyperref} package is loaded, it depends on the
%         |hyperfootnotes| option of the \xpackage{hyperref} package:
%  \begin{itemize}
%   \item With |hyperfootnotes=true|, the references are hyperlinked.
%   \item With |hyperfootnotes=false|, the references are not hyperlinked.
%  \end{itemize}
% \end{itemize}
% |\footnote|s inside of the |footnoterange|/{}|footnoterange*| environment
% can use the optional parameter, i.\,e. |\footnote[42]{text}| even if this is the
% third footnote. When giving ranges as reference, this does not seem to be
% such a good idea, if this is not done consistently (i.\,e. footnotes
% |[42], [43], [44]| resulting in 42--44 is OK,
% but |[42]| (original~2), unchanged~3, unchanged~4 resulting in
% 42--4 is confusing). A warning about these changed footnotemarks is written
% into the \xfile{.log}-file.
%
% When there is only a single footnote in a |footnoterange| environment,
% this does not produce some reference like %
% \makeatletter%
% \hbox{\textquotedblleft \@textsuperscript {\normalfont 7--7}\textquotedblright{},\ }%
% \makeatother%
% but only one footnotemark is used automatically. Because it is strange to use
% a |footnoterange| environment without a range of footnotes, an info is written
% into the \xfile{.log}-file about this.
%
% The usage of other footnote-number-representations
% (e.\,g.~\verb|\Roman|, \verb|\roman|, \verb|\Alph|, \verb|\alph|,
% \verb|\fnsymbol|) is also possible, but two things must be taken
% into account: The number of footnote references is restricted (for
% example with \verb|\Alph| only references A to Z are possible), which
% can be fixed e.\,g. with the \url{https://ctan.org/pkg/alphalph} package
% (but not automatically), and references to footnote-symbol-ranges (\verb|\fnsymbol|)
% are probably not very clear.
%
% \subsection*{Options}
% \indent The \xpackage{footnoterange} package takes no options.
%
% \section{Alternatives\label{sec:Alternatives}}
%
% The \xpackage{fnpct} (\url{https://www.ctan.org/pkg/fnpct}) package allows to set
% multiple footnotes (|\multfootnote|), suppress a footnotemark \hbox{(|*|),}
% and customize the multi-footnote-separator \hbox{(|\setfnpct{mult-fn-sep={|
% \emph{something} |}}|)}.
% Combining these with stepping of the footnotecounter gives a footnoterange:
% 
% \begin{verbatim}
% \documentclass{article}
% \usepackage{hyperref}
% \usepackage{fnpct}
% \begin{document}
% The three little pigs built their houses
% out of straw\footnote{not to be confused with hay},
% sticks\footnote{or lumber according to some sources}
% and bricks%
% \makeatletter%
% \setfnpct{mult-fn-sep={--}}%
% \multfootnote{probably fired clay bricks\addtocounter\@mpfn{+1};%
% *or something else;%
% what do I know}. %
% \makeatother%
% \end{document}
% \end{verbatim}
% 
% \noindent If you programmed or found another alternative, which is available at
% \url{https://www.CTAN.org/}, please send an e-mail to me with the name,
% location at CTAN, and a short notice, and I will probably list it here.
%
% \section{Example}
%
%    \begin{macrocode}
%<*example>
\documentclass[british]{article}[2014/09/29]% v1.4h
%%%%%%%%%%%%%%%%%%%%%%%%%%%%%%%%%%%%%%%%%%%%%%%%%%%%%%%%%%%%%%%%%%%%%
\usepackage[%
 extension=pdf,%
 plainpages=false,%
 pdfpagelabels=true,%
 hyperindex=false,%
 pdflang={en},%
 pdftitle={footnoterange package example},%
 pdfauthor={H.-Martin Muench},%
 pdfsubject={Example for the footnoterange package},%
 pdfkeywords={LaTeX, footnoterange, H.-Martin Muench},%
 pdfview=FitH,%
 pdfstartview=FitH,%
 pdfpagelayout=OneColumn,%
 hyperfootnotes=true%
]{hyperref}[2012/11/06]% v6.83m

\usepackage{footnoterange}[2019/11/24]% v1.0c

\gdef\unit#1{\mathord{\thinspace\mathrm{#1}}}%
\listfiles
\begin{document}
\pagenumbering{arabic}
\section*{Example for footnoterange}

This example demonstrates the use of package\newline
\textsf{footnoterange}, v1.0c as of 2019/11/24 (HMM).\newline
The package does not use options.\newline
\verb|footnoterange-example.pdf| can be created by\newline
\verb|pdflatex footnoterange-example.tex|,\newline
and \verb|footnoterange-example.tex| can be created by\newline
\verb|tex footnoterange.dtx|.\newline
\verb|footnoterange.dtx| is at
\url{https://www.ctan.org/pkg/footnoterange}.\newline

\noindent For more details please see the documentation!\newline

\noindent Save per page about $200\unit{ml}$ water,
$2\unit{g}$ CO$_{2}$ and $2\unit{g}$ wood:\newline
Therefore please print only if this is really necessary.\newline

This\footnote{Lorem} text bears a multiplicity of footnotes.
Because the \verb|hyperref| package is used with option
\verb|hyperfootnotes=true|, the footnotes
\begin{footnoterange}
\footnote{ipsum}%
\footnote{dolor}%
\footnote{sit}%
\footnote{amet,}%
\footnote{consectetur}%
\footnote{adipisicing}%
\footnote{elit,}%
\footnote{sed}%
\end{footnoterange}
are hyperlinked.\footnote{do eiusmod\ldots} Using
\verb|hyperfootnotes=false| or not using \verb|hyperref|
will remove the hyperlinks to the footnotes.\medskip

This text also bears a multiplicity of footnotes,
but due to the use of the starred
\begin{footnoterange*}
\footnote{Ut purus elit,}%
\footnote{vestibulum ut,}%
\footnote{placerat ac,}%
\footnote{adipiscing vitae,}%
\footnote{felis.}%
\footnote{Curabitur}%
\footnote{dictum}%
\footnote{gravida}%
\end{footnoterange*}
form of the environment they are not
hyperlinked.\footnote{mauris\ldots}\medskip

The use of the optional parameter for footnotes
\begin{footnoterange}
\footnote[101]{Nam arcu libero,}%
\footnote[102]{nonummy eget,}%
\footnote[103]{consectetuer id,}%
\footnote[104]{vulputate a,}%
\footnote[105]{magna.}
\end{footnoterange}
is also possible, but should be used consistently.

\pagebreak

When there is only one footnote
\begin{footnoterange}
\footnote{a single footnote}
\end{footnoterange}
in the \verb|footnoterange| environment, only one
footnotemark is used and an info is written into the
\verb|log|-file.\medskip

The usage of other footnote-number-representations
(e.\,g.~\verb|\Roman|, \verb|\roman|, \verb|\Alph|, \verb|\alph|,
\verb|\fnsymbol|) is also possible, but two things must be taken
into account: The number of footnote references is restricted
(for example with \verb|\Alph| only references A to Z are possible),
which can be fixed e.\,g. with the
\href{https://www.ctan.org/pkg/alphalph}{\texttt{alphalph} package},
and references to footnote-symbol-ranges (\verb|\fnsymbol|)
are probably not very clear.

\end{document}
%</example>
%    \end{macrocode}
%
% \StopEventually{}
%
% \section{The implementation}
%
% We start off by checking that we are loading into \LaTeXe{} and
% announcing the name and version of this package.
%
%    \begin{macrocode}
%<*package>
%    \end{macrocode}
%
%    \begin{macrocode}
\NeedsTeXFormat{LaTeX2e}[2011/06/27]
\ProvidesPackage{footnoterange}[2019/11/24 v1.0c
            References to ranges of footnotes (HMM)]
%    \end{macrocode}
%
% A short description of the \xpackage{footnoterange} package:
%
%    \begin{macrocode}
%% Provides the footnoterange and footnoterange* environments to
%% refer to ranges of footnotes.
%    \end{macrocode}
%
% We need the \xpackage{ltxcmds} package:
%
%    \begin{macrocode}
\RequirePackage{ltxcmds}[2011/11/09]% v1.22; for "\ltx@ifpackageloaded"
\RequirePackage{letltxmacro}[2010/09/02]% v1.4; for "\GlobalLetLtxMacro"

%    \end{macrocode}
%
% We use |\ifHy@hyperfootnotes ... \else ... \fi|, but when the \xpackage{hyperref}
% package has not been loaded, |\ifHy@hyperfootnotes| is not defined, and \LaTeX
% relates the |\else| and/or |\fi| (wrongly) to other |\if|s or groups.
% Therefore the |\ifHy@hyperfootnotes| are placed in their own commands,
% which are not looked at when the \xpackage{hyperref} package has not been loaded.
%
% \pagebreak
%
% \begin{macro}{\fnr@ifhyperrfn}
%    \begin{macrocode}
\newcommand{\fnr@ifhyperrfn}{%
  \ifHy@hyperfootnotes% option hyperfootnotes=true
    \addtocounter{Hfootnote}{+1}%
    % Code from the hyperref package
      \global\let\Hy@saved@currentHref\@currentHref%
      \hyper@makecurrent{Hfootnote}%
      \global\let\Hy@footnote@currentHref\@currentHref%
      \global\let\@currentHref\Hy@saved@currentHref%
    % End of code from the hyperref package
  \fi%
}

%    \end{macrocode}
% \end{macro}
%
% \begin{macro}{\fnr@rangefootnote}
% The |\fnr@rangefootnote| is the internal replacement for the original\newline
% |\footnote| command inside of a |footnoterange| environment.
%
%    \begin{macrocode}
\newcommand{\fnr@rangefootnote}[2][\empty]{%
%    \end{macrocode}
%
% |#1| contains the optional argument, which is predefined as |\empty|.
%
%    \begin{macrocode}
  \nolinebreak%
  % no \footnotemark here
  \addtocounter\@mpfn{+1}%
  \ltx@ifpackageloaded{hyperref}{\fnr@ifhyperrfn}{\relax}%
%    \end{macrocode}
%
% See the explanation before the declaration of the |\fnr@ifhyperrfn| command!
%
%    \begin{macrocode}
  \xdef\fnr@opt{#1}% contains the optional argument
  \xdef\fnr@arabic{\arabic\@mpfn}%
  \protected@edef\fnr@formated{\thefootnote}%
%    \end{macrocode}
%
% The |\footnotemark| is not set (yet), but the |\footnotetext|,
% and it gets a label.
%
%    \begin{macrocode}
  \ifx\fnr@opt\empty%
    \footnotetext{\label{fnr:\fnr@arabic}#2}%
  \else%
    \PackageWarning{footnoterange}{%
      Custom footnote-numbers used in a footnoterange\MessageBreak%
      environment: Footnote with natural number\MessageBreak%
      \ifx\fnr@formated\fnr@arabic%
        \fnr@arabic\space%
      \else%
        \space\fnr@arabic\space (\fnr@formated )%
      \fi%
      \MessageBreak%
      was renumbered to \fnr@opt\MessageBreak%
     }%
    \ltx@ifpackageloaded{hyperref}{% hyperref loaded
      \footnotetext[#1]{\phantomsection\label{fnr:\fnr@arabic}#2}%
     }{% hyperref not loaded
      \footnotetext[#1]{\label{fnr:\fnr@arabic}#2}%
     }%
  \fi%
}

%    \end{macrocode}
% \end{macro}
%
%    \begin{macrocode}

%    \end{macrocode}
% \pagebreak
%
% |\ifHy@hyperfootnotes| \textquotedblleft encapsulation\textquotedblright{} again:
%
% \begin{macro}{\fnr@ifhyperfnrA}
%    \begin{macrocode}
\newcommand{\fnr@ifhyperfnrA}{%
  \ifHy@hyperfootnotes% option hyperfootnotes=true
    \hbox{\hyperref[fnr:\fnr@first]{%
          \@textsuperscript{\normalfont \ref*{fnr:\fnr@first}}}}%
  \else% option hyperfootnotes=false
    \hbox{\@textsuperscript{\normalfont \ref*{fnr:\fnr@first}}}%
  \fi%
}

%    \end{macrocode}
% \end{macro}
%
% \begin{macro}{\fnr@ifhyperfnrN}
%    \begin{macrocode}
\newcommand{\fnr@ifhyperfnrN}{%
  \ifHy@hyperfootnotes% option hyperfootnotes=true
    \hbox{\hyperref[fnr:\fnr@first]{%
            \@textsuperscript{\normalfont \ref*{fnr:\fnr@first}}}%
          \@textsuperscript{\normalfont --}%
          \hyperref[fnr:\arabic\@mpfn]{%
            \@textsuperscript{\normalfont \ref*{fnr:\arabic\@mpfn}}}}%
  \else% option hyperfootnotes=false
    \hbox{\@textsuperscript{\normalfont \ref*{fnr:\fnr@first}%
                                        --\ref*{fnr:\arabic\@mpfn}}}%
  \fi%
}

%    \end{macrocode}
% \end{macro}
%
% \begin{environment}{footnoterange}
% The |footnoterange| environment remembers the next footnote number as\newline
% |\fnr@first| and the current |\footnote| definition as |\fnr@origfn|
% and uses |\fnr@rangefootnote| instead.
%
%    \begin{macrocode}
\newenvironment{footnoterange}{%
  % at begin of the footnoterange environment
  \ifhmode \unskip \fi% FMi
  \addtocounter\@mpfn{+1}%
  \xdef\fnr@first{\arabic\@mpfn}%
  \addtocounter\@mpfn{-1}%
  \let\fnr@origfn\footnote%
  \let\footnote\fnr@rangefootnote%
  \GlobalLetLtxMacro{\fnr@origfnmark}{\footnotemark}%
  \GlobalLetLtxMacro{\footnotemark}{\fnr@rangefootnotemark}%
  \ignorespaces% FMi
%    \end{macrocode}
%
% End of the environment:
%
%    \begin{macrocode}
}{% at end of the footnoterange environment
  \ifhmode% FMi 
    \unskip% FMi
%    \end{macrocode}
%
% Remember the spacefactor for restoring after the environment has ended
% (only in |hmode|!). \% FMi
%
%    \begin{macrocode}
    \xdef\fnr@aftergroup{\spacefactor\the\spacefactor\relax}% FMi
  \else% FMi
%    \end{macrocode}
%
% Otherwise do something harmless. \% FMi
%
%    \begin{macrocode}
    \xdef\fnr@aftergroup{\relax}%
  \fi% FMi
%    \end{macrocode}
%
% At the end of the environment the previous definition of |\footnote| is restored
%
%    \begin{macrocode}
  \let\footnote\fnr@origfn%
  \GlobalLetLtxMacro{\footnotemark}{\fnr@origfnmark}%
%    \end{macrocode}
%
% and the references are placed:
%
%    \begin{macrocode}
  \ifx\fnr@first\fnr@arabic% There is only one reference in the environment.
    \PackageInfo{footnoterange}{%
      Environment footnoterange used,\MessageBreak%
      but only one footnote (number \fnr@first ) is contained%
      \MessageBreak%
     }%
    \ltx@ifpackageloaded{hyperref}{\fnr@ifhyperfnrA% hyperref package loaded
    }{% hyperref package not loaded
      \hbox{\@textsuperscript{\normalfont \ref{fnr:\fnr@first}}}%
     }%
  \else%
    \ltx@ifpackageloaded{hyperref}{\fnr@ifhyperfnrN% hyperref package loaded
    }{% hyperref package not loaded
      \hbox{\@textsuperscript{\normalfont \ref{fnr:\fnr@first}%
                                        --\ref{fnr:\arabic\@mpfn}}}%
     }%
  \fi%
%    \end{macrocode}
%
% Restore the spacefactor after the environment has ended. \% FMi
%
%    \begin{macrocode}
  \aftergroup\fnr@aftergroup% FMi
}

%    \end{macrocode}
% \end{environment}
%
% \begin{environment}{footnoterange*}
% The |footnoterange*| environment does the same as the |footnoterange|
% environment but suppresses all hyperlinks to footnotes inside.
%
%    \begin{macrocode}
\newenvironment{footnoterange*}{%
  % at begin of the footnoterange* environment
  \ifhmode \unskip \fi% FMi
  \addtocounter\@mpfn{+1}%
  \xdef\fnr@first{\arabic\@mpfn}%
  \addtocounter\@mpfn{-1}%
  \let\fnr@origfn\footnote%
  \let\footnote\fnr@rangefootnote%
  \GlobalLetLtxMacro{\fnr@origfnmark}{\footnotemark}%
  \GlobalLetLtxMacro{\footnotemark}{\fnr@rangefootnotemark}%
  \ignorespaces% FMi
}{% at end of the footnoterange* environment
  \ifhmode% FMi 
    \unskip% FMi
    \xdef\fnr@aftergroup{\spacefactor\the\spacefactor\relax}% FMi
  \else% FMi
    \xdef\fnr@aftergroup{\relax}%
  \fi% FMi
  \let\footnote\fnr@origfn%
  \GlobalLetLtxMacro{\footnotemark}{\fnr@origfnmark}%
  \ifx\fnr@first\fnr@arabic% There is only one reference in the environment.
    \PackageInfo{footnoterange}{%
      Environment footnoterange used,\MessageBreak%
      but only one footnote (number \fnr@first ) is contained%
      \MessageBreak%
     }%
%    \end{macrocode}
% \pagebreak
%    \begin{macrocode}
    \ltx@ifpackageloaded{hyperref}{% hyperref package loaded
      \hbox{\@textsuperscript{\normalfont \ref*{fnr:\fnr@first}}}%
    }{% hyperref package not loaded
      \hbox{\@textsuperscript{\normalfont \ref{fnr:\fnr@first}}}%
     }%
  \else%
    \ltx@ifpackageloaded{hyperref}{% hyperref package loaded
      \hbox{\@textsuperscript{\normalfont \ref*{fnr:\fnr@first}%
                                        --\ref*{fnr:\arabic\@mpfn}}}%
    }{% hyperref package not loaded
      \hbox{\@textsuperscript{\normalfont \ref{fnr:\fnr@first}%
                                        --\ref{fnr:\arabic\@mpfn}}}%
     }%
  \fi%
  \aftergroup\fnr@aftergroup% FMi
}

%    \end{macrocode}
% \end{environment}
%
% The \xpackage{footnoterange} package does not handle |\footnotemark|s,
% but at least the user is made aware of this:
%
% \begin{macro}{\fnr@rangefootnotemark}
%    \begin{macrocode}

\newcommand{\fnr@rangefootnotemark}[1][\empty]{%
  \PackageError{footnoterange}{%
     \string\footnotemark\space not supported in footnoterange(*) environment%
   }{The footnoterange package does not handle a \string\footnotemark.\MessageBreak%
     Either use \string\footnote\string{...\string}\space instead of%
     \string\footnotemark\space and \string\footnotetext\string{...\string}\MessageBreak%
     in footnoterange and footnoterange* environments,\MessageBreak%
     or do not use those environments around a \string\footnotemark\MessageBreak%
     (or \string\renewcommand\string{\string\fnr@rangefootnotemark\string}\MessageBreak%
     with your own fixing code).%
    }%
  \xdef\fnr@mopt{#1}%
  \ifx\fnr@mopt\empty%
    \fnr@origfnmark%
  \else%
    \fnr@origfnmark[#1]%
  \fi%
}

%    \end{macrocode}
% \end{macro}
%
%    \begin{macrocode}
%</package>
%    \end{macrocode}
%
% \pagebreak
%
% \section{Installation}
%
% \subsection{Downloads\label{ss:Downloads}}
%
% Everything is available on CTAN,
% \url{https://www.ctan.org/tex-archive/},
% but may need additional packages themselves.
%
% \DescribeMacro{footnoterange.dtx}
% For unpacking the \xfile{footnoterange.dtx} file
% and constructing the documentation it is required:
% \begin{description}
% \item[-] \TeX{} Format \LaTeXe{}, \url{https://www.CTAN.org/}
%
% \item[-] document class \xpackage{ltxdoc}, 2015/03/26, v2.0w,
%           \url{https://ctan.org/pkg/ltxdoc}
%
% \item[-] package \xpackage{holtxdoc}, 2012/03/21, v0.24,
%           \url{https://ctan.org/pkg/holtxdoc}
% \end{description}
%
% \DescribeMacro{footnoterange.sty}
% The |footnoterange.sty| for \LaTeXe{} (i.\,e. each document using
% the \xpackage{footnoterange} package) requires:
% \begin{description}
% \item[-] \TeX{} Format \LaTeXe{}, \url{https://www.CTAN.org/}
%
% \item[-] package \xpackage{ltxcmds}, 2011/04/18, v1.20,
%           \url{https://www.ctan.org/pkg/ltxcmds}
% \end{description}
%
% \DescribeMacro{footnoterange-example.tex}
% The |footnoterange-example.tex| requires the same files as all
% documents using the \xpackage{footnoterange} package and additionally:
% \begin{description}
% \item[-] class \xpackage{article}, 2007/10/19, v1.4h, from \xpackage{classes.dtx}:\newline
%           \url{https://www.ctan.org/tex-archive/macros/latex/base/}
%
% \item[-] package \xpackage{footnoterange}, 2019/11/24, v1.0c,\newline
%           \url{https://www.ctan.org/pkg/footnoterange}\newline
%           (Well, it is the example file for this package, and because you are
%            reading the documentation for the \xpackage{footnoterange} package,
%            it can be assumed that you already have some version of it --
%            is it the current one?)
%
% \item[-] package \xpackage{hyperref}, 2012/11/06, v6.83m,\newline
%           \url{https://www.ctan.org/pkg/hyperref}
% \end{description}
%
% \DescribeMacro{Oberdiek}
% \DescribeMacro{holtxdoc}
% \DescribeMacro{ltxcmds}
% \DescribeMacro{alphalph}
% All packages of \textsc{Heiko Oberdiek}'s bundle `oberdiek'
% (especially \xpackage{holtxdoc}, \xpackage{ltxcmds}, and \xpackage{alphalph})
% are also available in a TDS compliant ZIP archive:\newline
% \url{http://mirror.ctan.org/install/macros/latex/contrib/oberdiek.tds.zip}.\newline
% It is probably best to download and use this, because the packages in there
% should be both recent and compatible.
%
% \DescribeMacro{hyperref}
% \noindent \xpackage{hyperref} is not included in that bundle and needs to be
% downloaded separately,\newline
% \url{http://mirrors.ctan.org/install/macros/latex/contrib/hyperref.tds.zip}.
%
% \DescribeMacro{M\"{u}nch}
% A list of my packages can be found at \url{https://www.ctan.org/author/muench-hm}.
%
% \pagebreak
%
% \subsection{Package, unpacking TDS}
% \paragraph{Package.} This package is available on \url{https://www.CTAN.org}.
% \begin{description}
% \item[\url{http://mirror.ctan.org/macros/latex/contrib/footnoterange/footnoterange.dtx}]\hspace*{0.1cm}
%       The source file.
% \item[\url{http://mirror.ctan.org/macros/latex/contrib/footnoterange/footnoterange.pdf}]\hspace*{0.1cm}
%       The documentation.
% \item[\url{http://mirror.ctan.org/macros/latex/contrib/footnoterange/footnoterange-example.pdf}]\hspace*{0.1cm}
%       The compiled example file, as it should look like.
% \item[\url{http://mirror.ctan.org/macros/latex/contrib/footnoterange/README}]\hspace*{0.1cm}\newline\hspace*{0.1cm}
%       The README file.
% \end{description}
% There is also a footnoterange.tds.zip available:
% \begin{description}
% \item[\url{http://mirror.ctan.org/install/macros/latex/contrib/footnoterange.tds.zip}]\hspace*{0.1cm}
%       Everything in TDS compliant, compiled format.
% \end{description}
% which additionally contains\newline
% \begin{tabular}{ll}
% footnoterange.ins & The installation file.\\
% footnoterange.drv & The driver to generate the documentation.\\
% footnoterange.sty & The \xext{sty}le file.
% \end{tabular}
%
% \bigskip
%
% \noindent For required other packages, please see the preceding subsection.
%
% \paragraph{Unpacking.} The \xfile{.dtx} file is a self-extracting
% \docstrip{} archive. The files are extracted by running the
% \xfile{.dtx} through \plainTeX:
% \begin{quote}
%   \verb|tex footnoterange.dtx|
% \end{quote}
%
% About generating the documentation see paragraph~\ref{GenDoc} below.
%
% \paragraph{TDS.} Now the different files must be moved into
% the different directories in your installation TDS tree
% (also known as \xfile{texmf} tree):
% \begin{quote}
% \def\t{^^A
% \begin{tabular}{@{}>{\ttfamily}l@{ $\rightarrow$ }>{\ttfamily}l@{}}
%   footnoterange.sty & tex/latex/footnoterange/footnoterange.sty\\
%   footnoterange.pdf & doc/latex/footnoterange/footnoterange.pdf\\
%   footnoterange-example.tex & doc/latex/footnoterange/footnoterange-example.tex\\
%   footnoterange-example.pdf & doc/latex/footnoterange/footnoterange-example.pdf\\
%   footnoterange.dtx & source/latex/footnoterange/footnoterange.dtx\\
% \end{tabular}^^A
% }^^A
% \sbox0{\t}^^A
% \ifdim\wd0>\linewidth
%   \begingroup
%     \advance\linewidth by\leftmargin
%     \advance\linewidth by\rightmargin
%   \edef\x{\endgroup
%     \def\noexpand\lw{\the\linewidth}^^A
%   }\x
%   \def\lwbox{^^A
%     \leavevmode
%     \hbox to \linewidth{^^A
%       \kern-\leftmargin\relax
%       \hss
%       \usebox0
%       \hss
%       \kern-\rightmargin\relax
%     }^^A
%   }^^A
%   \ifdim\wd0>\lw
%     \sbox0{\small\t}^^A
%     \ifdim\wd0>\linewidth
%       \ifdim\wd0>\lw
%         \sbox0{\footnotesize\t}^^A
%         \ifdim\wd0>\linewidth
%           \ifdim\wd0>\lw
%             \sbox0{\scriptsize\t}^^A
%             \ifdim\wd0>\linewidth
%               \ifdim\wd0>\lw
%                 \sbox0{\tiny\t}^^A
%                 \ifdim\wd0>\linewidth
%                   \lwbox
%                 \else
%                   \usebox0
%                 \fi
%               \else
%                 \lwbox
%               \fi
%             \else
%               \usebox0
%             \fi
%           \else
%             \lwbox
%           \fi
%         \else
%           \usebox0
%         \fi
%       \else
%         \lwbox
%       \fi
%     \else
%       \usebox0
%     \fi
%   \else
%     \lwbox
%   \fi
% \else
%   \usebox0
% \fi
% \end{quote}
% If you have a \xfile{docstrip.cfg} that configures and enables \docstrip's
% TDS installing feature, then some files can already be in the right
% place, see the documentation of \docstrip.
%
% \subsection{Refresh file name databases}
%
% If your \TeX~distribution (\TeX{} Live, \mikTeX, \teTeX, \dots) relies on
% file name databases, you must refresh these. For example, \teTeX{} users run
% \verb|texhash| or \verb|mktexlsr|.
%
% \subsection{Some details for the interested}
%
% \paragraph{Unpacking with \LaTeX{}.}
% The \xfile{.dtx} chooses its action depending on the format:
% \begin{description}
% \item[\plainTeX:] Run \docstrip{} and extract the files.
% \item[\LaTeX:] Generate the documentation.
% \end{description}
% If you insist on using \LaTeX{} for \docstrip{} (really,
% \docstrip{} does not need \LaTeX ), then inform the autodetect routine
% about your intention:
% \begin{quote}
%   \verb|latex \let\install=y% \iffalse meta-comment
%
% File: footnoterange.dtx
% Version: 2019/11/24 v1.0c
%
% Copyright (C) 2012 - 2019 by
%    H.-Martin M"unch <Martin dot Muench at Uni-Bonn dot de>
% Portions of code copyrighted by other people as marked;
% lines marked "FMi": code by Frank Mittelbach, thanks!
%
% This work may be distributed and/or modified under the
% conditions of the LaTeX Project Public License, either
% version 1.3c of this license or (at your option) any later
% version. This version of this license is in
%    https://www.latex-project.org/lppl/lppl-1-3c.txt
% and the latest version of this license is in
%    https://www.latex-project.org/lppl.txt
% and version 1.3c or later is part of all distributions of
% LaTeX version 2005/12/01 or later.
%
% This work has the LPPL maintenance status "maintained".
%
% The Current Maintainer of this work is H.-Martin Muench.
%
% This work consists of the main source file footnoterange.dtx,
% the README, and the derived files
%    footnoterange.sty, footnoterange.pdf,
%    footnoterange.ins, footnoterange.drv,
%    footnoterange-example.tex, footnoterange-example.pdf.
%
% 'footnoterange' is available on CTAN:
% https://www.ctan.org/pkg/footnoterange
%
% Also a TDS.ZIP file is provided that contains all the files
% already sorted in a TDS tree:
% http://mirror.ctan.org/install/macros/latex/contrib/footnoterange.tds.zip
%
% Unpacking:
%    (a) If footnoterange.ins is present:
%           tex footnoterange.ins
%    (b) Without footnoterange.ins:
%           tex footnoterange.dtx
%    (c) If you insist on using LaTeX
%           latex \let\install=y\input{footnoterange.dtx}
%        (quote the arguments according to the demands of your shell)
%
% Documentation:
%    (a) If footnoterange.drv is present:
%           (pdf)latex footnoterange.drv
%           makeindex -s gind.ist footnoterange.idx
%           (pdf)latex footnoterange.drv
%           makeindex -s gind.ist footnoterange.idx
%           (pdf)latex footnoterange.drv
%    (b) Without footnoterange.drv:
%           (pdf)latex footnoterange.dtx
%           makeindex -s gind.ist footnoterange.idx
%           (pdf)latex footnoterange.dtx
%           makeindex -s gind.ist footnoterange.idx
%           (pdf)latex footnoterange.dtx
%
%    The class ltxdoc loads the configuration file ltxdoc.cfg
%    if available. Here you can specify further options, e.g.
%    use DIN A4 as paper format:
%       \PassOptionsToClass{a4paper}{article}
%
% Installation:
%    TDS:tex/latex/footnoterange/footnoterange.sty
%    TDS:doc/latex/footnoterange/footnoterange.pdf
%    TDS:doc/latex/footnoterange/footnoterange-example.tex
%    TDS:doc/latex/footnoterange/footnoterange-example.pdf
%    TDS:source/latex/footnoterange/footnoterange.dtx
%
%<*ignore>
\begingroup
  \catcode123=1 %
  \catcode125=2 %
  \def\x{LaTeX2e}%
\expandafter\endgroup
\ifcase 0\ifx\install y1\fi\expandafter
         \ifx\csname processbatchFile\endcsname\relax\else1\fi
         \ifx\fmtname\x\else 1\fi\relax
\else\csname fi\endcsname
%</ignore>
%<*install>
\input docstrip.tex
\Msg{*********************************************************************************}
\Msg{* Installation}
\Msg{* Package: footnoterange 2019/11/24 v1.0c References to ranges of footnotes (HMM)}
\Msg{*********************************************************************************}

\keepsilent
\askforoverwritefalse

\let\MetaPrefix\relax
\preamble

This is a generated file.

Project: footnoterange
Version: 2019/11/24 v1.0c

Copyright (C) 2012 - 2019 by
    H.-Martin M"unch <Martin dot Muench at Uni-Bonn dot de>
Portions of code copyrighted by other people as marked;
lines marked "FMi": code by Frank Mittelbach, thanks!

The usual disclaimer applies:
If it doesn't work right that's your problem.
(Nevertheless, please send an e-mail to the maintainer
 when you find an error in this package.)

This work may be distributed and/or modified under the
conditions of the LaTeX Project Public License, either
version 1.3c of this license or (at your option) any later
version. This version of this license is in
   https://www.latex-project.org/lppl/lppl-1-3c.txt
and the latest version of this license is in
   https://www.latex-project.org/lppl.txt
and version 1.3c or later is part of all distributions of
LaTeX version 2005/12/01 or later.

This work has the LPPL maintenance status "maintained".

The Current Maintainer of this work is H.-Martin Muench.

This work consists of the main source file footnoterange.dtx,
the README, and the derived files
   footnoterange.sty, footnoterange.pdf,
   footnoterange.ins, footnoterange.drv,
   footnoterange-example.tex, footnoterange-example.pdf.

In memoriam
 Claudia Simone Barth + 1996/01/30
 Tommy Muench + 2014/01/02
 Hans-Klaus Muench + 2014/08/24

\endpreamble
\let\MetaPrefix\DoubleperCent

\generate{%
  \file{footnoterange.ins}{\from{footnoterange.dtx}{install}}%
  \file{footnoterange.drv}{\from{footnoterange.dtx}{driver}}%
  \usedir{tex/latex/footnoterange}%
  \file{footnoterange.sty}{\from{footnoterange.dtx}{package}}%
  \usedir{doc/latex/footnoterange}%
  \file{footnoterange-example.tex}{\from{footnoterange.dtx}{example}}%
}

\catcode32=13\relax% active space
\let =\space%
\Msg{************************************************************************}
\Msg{*}
\Msg{* To finish the installation you have to move the following}
\Msg{* file into a directory searched by TeX:}
\Msg{*}
\Msg{* footnoterange.sty}
\Msg{*}
\Msg{* To produce the documentation run the file `footnoterange.drv'}
\Msg{* through (pdf)LaTeX, e.g.}
\Msg{*  pdflatex footnoterange.drv}
\Msg{*  makeindex -s gind.ist footnoterange.idx}
\Msg{*  pdflatex footnoterange.drv}
\Msg{*  makeindex -s gind.ist footnoterange.idx}
\Msg{*  pdflatex footnoterange.drv}
\Msg{*}
\Msg{* At least three runs are necessary e.g. to get the}
\Msg{*  references right!}
\Msg{*}
\Msg{* Happy TeXing!}
\Msg{*}
\Msg{************************************************************************}

\endbatchfile
%</install>
%<*ignore>
\fi
%</ignore>
%
% \section{The documentation driver file}
%
% The next bit of code contains the documentation driver file for
% \TeX , i.\,e., the file that will produce the documentation you
% are currently reading. It will be extracted from this file by the
% \verb|docstrip| programme. That is, run \LaTeX{} on \verb|docstrip|
% and specify the \verb|driver| option when \verb|docstrip|
% asks for options.
%
%    \begin{macrocode}
%<*driver>
\NeedsTeXFormat{LaTeX2e}[2015/01/01]
\ProvidesFile{footnoterange.drv}[2019/11/24 v1.0c
            References to ranges of footnotes (HMM)]
\documentclass{ltxdoc}[2015/03/26]% v2.0w
\usepackage{holtxdoc}[2012/03/21]%  v0.24
%% footnoterange may work with earlier versions of LaTeX2e and those
%% class and package, but this was not tested.
%% Please consider updating your LaTeX, class, and package
%% to the most recent version (if they are not already the most
%% recent version).
\hypersetup{%
 pdfsubject={References to ranges of footnotes (HMM)},%
 pdfkeywords={LaTeX, footnoterange, footnotes, footnote, reference, H.-Martin Muench},%
 pdfencoding=auto,%
 pdflang={en},%
 breaklinks=true,%
 linktoc=all,%
 pdfstartview=FitH,%
 pdfpagelayout=OneColumn,%
 bookmarksnumbered=true,%
 bookmarksopen=true,%
 bookmarksopenlevel=3,%
 pdfmenubar=true,%
 pdftoolbar=true,%
 pdfwindowui=true,%
 pdfnewwindow=true%
}
\CodelineIndex
\hyphenation{printing docu-ment}
\gdef\unit#1{\mathord{\thinspace\mathrm{#1}}}
\begin{document}
  \DocInput{footnoterange.dtx}%
\end{document}
%</driver>
%    \end{macrocode}
%
% \fi
%
% \CheckSum{321}
%
% \CharacterTable
%  {Upper-case    \A\B\C\D\E\F\G\H\I\J\K\L\M\N\O\P\Q\R\S\T\U\V\W\X\Y\Z
%   Lower-case    \a\b\c\d\e\f\g\h\i\j\k\l\m\n\o\p\q\r\s\t\u\v\w\x\y\z
%   Digits        \0\1\2\3\4\5\6\7\8\9
%   Exclamation   \!     Double quote  \"     Hash (number) \#
%   Dollar        \$     Percent       \%     Ampersand     \&
%   Acute accent  \'     Left paren    \(     Right paren   \)
%   Asterisk      \*     Plus          \+     Comma         \,
%   Minus         \-     Point         \.     Solidus       \/
%   Colon         \:     Semicolon     \;     Less than     \<
%   Equals        \=     Greater than  \>     Question mark \?
%   Commercial at \@     Left bracket  \[     Backslash     \\
%   Right bracket \]     Circumflex    \^     Underscore    \_
%   Grave accent  \`     Left brace    \{     Vertical bar  \|
%   Right brace   \}     Tilde         \~}
%
% \GetFileInfo{footnoterange.drv}
%
% \begingroup
%   \def\x{\#,\$,\^,\_,\~,\ ,\&,\{,\},\%}%
%   \makeatletter
%   \@onelevel@sanitize\x
% \expandafter\endgroup
% \expandafter\DoNotIndex\expandafter{\x}
% \expandafter\DoNotIndex\expandafter{\string\ }
% \begingroup
%   \makeatletter
%     \lccode`9=32\relax
%     \lowercase{%^^A
%       \edef\x{\noexpand\DoNotIndex{\@backslashchar9}}%^^A
%     }%^^A
%   \expandafter\endgroup\x
% \DoNotIndex{\,,\\}
% \DoNotIndex{\documentclass,\usepackage,\ProvidesPackage,\begin,\end}
% \DoNotIndex{\MessageBreak}
% \DoNotIndex{\NeedsTeXFormat,\DoNotIndex,\verb}
% \DoNotIndex{\def,\edef,\gdef,\global,\xdef}
% \DoNotIndex{\ifx,\listfiles,\mathord,\mathrm}
% \DoNotIndex{\bigskip,\space,\thinspace,\ldots}
% \DoNotIndex{\indent,\noindent,\newline,\linebreak,\pagebreak,\newpage}
% \DoNotIndex{\textbf,\textit,\textsf,\textsc,\texttt,\textquotedblleft,\textquotedblright}
% \DoNotIndex{\plainTeX,\TeX,\LaTeX,\pdfLaTeX}
% \DoNotIndex{\section,\pagenumbering,\smallskip}
% \DoNotIndex{\@ifpackageloaded}
% \DoNotIndex{\arabic, \Alph, \alph, \alphalph, \Roman, \roman, \fnsymbol}
%
% \title{The \xpackage{footnoterange} package}
% \date{2019/11/24 v1.0c}
% \author{H.-Martin M\"{u}nch\\\xemail{Martin.Muench at Uni-Bonn.de}}
%
% \maketitle
%
% \begin{abstract}
% \noindent This package provides the environments |footnoterange| and
% |footnoterange*|. Multiple footnotes inside these environments
% are not referenced as e.\,g.
% \makeatletter%
% \hbox{\textquotedblleft \@textsuperscript {\normalfont 1 2 3}\textquotedblright}
% but as \hbox{\textquotedblleft \@textsuperscript {\normalfont 1--3}\textquotedblright ,\ }
% \makeatother%
% i.\,e. the range (from first to last referred footnote at that place) is given.
% Depending on loading of \xpackage{hyperref} and use of its |hyperfootnotes|-option
% the references are hyperlinked. (References to footnotes in |footnoterange*|
% are never hyperlinked.)
% \end{abstract}
%
% \bigskip
%
% \noindent Disclaimer for web links: The author is not responsible for any contents
% referred to in this work unless if having full knowledge of illegal contents.
% If any damage occurs by the use of information presented there, only the
% author of the respective pages might be liable, not the one who has referred
% to these pages.
%
% \noindent Save per page about $200\unit{ml}$ water,
% $2\unit{g}$ CO$_{2}$ and $2\unit{g}$ wood:
% Therefore please print only if this is really necessary.
%
% \medskip
%
% \tableofcontents
%
%\newpage
%
% \section{Introduction}
% \indent This \LaTeXe{} package provides the environments |footnoterange| and\newline
% |footnoterange*|. Multiple footnotes inside these environments are not referenced
% as e.\,g.
% \makeatletter%
% \hbox{\textquotedblleft \@textsuperscript {\normalfont 1 2 3}\textquotedblright}
% but as \hbox{\textquotedblleft \@textsuperscript {\normalfont 1--3}\textquotedblright ,\ }
% \makeatother%
% i.\,e. the range (from first to last
% referred footnote at that place) is given. Depending on loading of
% \xpackage{hyperref} and use of its |hyperfootnotes|-option the references are
% hyperlinked. References to footnotes in |footnoterange*| are never hyperlinked.\newline
% |\footnotemark\footnotetext| instead of |\footnote| is not supported inside
% |footnoterange| and |footnoterange*| environments.
%
% \bigskip
%
% \section{Usage}
%
% \indent Just load the package placing
% \begin{quote}
%   |\usepackage{footnoterange}|
% \end{quote}
% \noindent in the preamble of your \LaTeXe{} source file. There are no options to
% be used. The package should be loaded after \xpackage{hyperref} (if that package is used)
% and after any other package redefining footnotes. Usage of the |footnoterange| (or
% |footnoterange*|) environment is simple: Just encapsulate the footnotes, which shall
% be referred as range instead of referencing as single ones (see the example file).
%
% About hyperlinks:\newline
% Footnote-references in a |footnoterange*| environment are never hyperlinked.\newline
% Footnote-references in a |footnoterange| environment may be hyperlinked:
% \begin{itemize}
%  \item When the \xpackage{hyperref} package is not loaded, they are not
%         hyperlinked (naturally).
%  \item When the \xpackage{hyperref} package is loaded, it depends on the
%         |hyperfootnotes| option of the \xpackage{hyperref} package:
%  \begin{itemize}
%   \item With |hyperfootnotes=true|, the references are hyperlinked.
%   \item With |hyperfootnotes=false|, the references are not hyperlinked.
%  \end{itemize}
% \end{itemize}
% |\footnote|s inside of the |footnoterange|/{}|footnoterange*| environment
% can use the optional parameter, i.\,e. |\footnote[42]{text}| even if this is the
% third footnote. When giving ranges as reference, this does not seem to be
% such a good idea, if this is not done consistently (i.\,e. footnotes
% |[42], [43], [44]| resulting in 42--44 is OK,
% but |[42]| (original~2), unchanged~3, unchanged~4 resulting in
% 42--4 is confusing). A warning about these changed footnotemarks is written
% into the \xfile{.log}-file.
%
% When there is only a single footnote in a |footnoterange| environment,
% this does not produce some reference like %
% \makeatletter%
% \hbox{\textquotedblleft \@textsuperscript {\normalfont 7--7}\textquotedblright{},\ }%
% \makeatother%
% but only one footnotemark is used automatically. Because it is strange to use
% a |footnoterange| environment without a range of footnotes, an info is written
% into the \xfile{.log}-file about this.
%
% The usage of other footnote-number-representations
% (e.\,g.~\verb|\Roman|, \verb|\roman|, \verb|\Alph|, \verb|\alph|,
% \verb|\fnsymbol|) is also possible, but two things must be taken
% into account: The number of footnote references is restricted (for
% example with \verb|\Alph| only references A to Z are possible), which
% can be fixed e.\,g. with the \url{https://ctan.org/pkg/alphalph} package
% (but not automatically), and references to footnote-symbol-ranges (\verb|\fnsymbol|)
% are probably not very clear.
%
% \subsection*{Options}
% \indent The \xpackage{footnoterange} package takes no options.
%
% \section{Alternatives\label{sec:Alternatives}}
%
% The \xpackage{fnpct} (\url{https://www.ctan.org/pkg/fnpct}) package allows to set
% multiple footnotes (|\multfootnote|), suppress a footnotemark \hbox{(|*|),}
% and customize the multi-footnote-separator \hbox{(|\setfnpct{mult-fn-sep={|
% \emph{something} |}}|)}.
% Combining these with stepping of the footnotecounter gives a footnoterange:
% 
% \begin{verbatim}
% \documentclass{article}
% \usepackage{hyperref}
% \usepackage{fnpct}
% \begin{document}
% The three little pigs built their houses
% out of straw\footnote{not to be confused with hay},
% sticks\footnote{or lumber according to some sources}
% and bricks%
% \makeatletter%
% \setfnpct{mult-fn-sep={--}}%
% \multfootnote{probably fired clay bricks\addtocounter\@mpfn{+1};%
% *or something else;%
% what do I know}. %
% \makeatother%
% \end{document}
% \end{verbatim}
% 
% \noindent If you programmed or found another alternative, which is available at
% \url{https://www.CTAN.org/}, please send an e-mail to me with the name,
% location at CTAN, and a short notice, and I will probably list it here.
%
% \section{Example}
%
%    \begin{macrocode}
%<*example>
\documentclass[british]{article}[2014/09/29]% v1.4h
%%%%%%%%%%%%%%%%%%%%%%%%%%%%%%%%%%%%%%%%%%%%%%%%%%%%%%%%%%%%%%%%%%%%%
\usepackage[%
 extension=pdf,%
 plainpages=false,%
 pdfpagelabels=true,%
 hyperindex=false,%
 pdflang={en},%
 pdftitle={footnoterange package example},%
 pdfauthor={H.-Martin Muench},%
 pdfsubject={Example for the footnoterange package},%
 pdfkeywords={LaTeX, footnoterange, H.-Martin Muench},%
 pdfview=FitH,%
 pdfstartview=FitH,%
 pdfpagelayout=OneColumn,%
 hyperfootnotes=true%
]{hyperref}[2012/11/06]% v6.83m

\usepackage{footnoterange}[2019/11/24]% v1.0c

\gdef\unit#1{\mathord{\thinspace\mathrm{#1}}}%
\listfiles
\begin{document}
\pagenumbering{arabic}
\section*{Example for footnoterange}

This example demonstrates the use of package\newline
\textsf{footnoterange}, v1.0c as of 2019/11/24 (HMM).\newline
The package does not use options.\newline
\verb|footnoterange-example.pdf| can be created by\newline
\verb|pdflatex footnoterange-example.tex|,\newline
and \verb|footnoterange-example.tex| can be created by\newline
\verb|tex footnoterange.dtx|.\newline
\verb|footnoterange.dtx| is at
\url{https://www.ctan.org/pkg/footnoterange}.\newline

\noindent For more details please see the documentation!\newline

\noindent Save per page about $200\unit{ml}$ water,
$2\unit{g}$ CO$_{2}$ and $2\unit{g}$ wood:\newline
Therefore please print only if this is really necessary.\newline

This\footnote{Lorem} text bears a multiplicity of footnotes.
Because the \verb|hyperref| package is used with option
\verb|hyperfootnotes=true|, the footnotes
\begin{footnoterange}
\footnote{ipsum}%
\footnote{dolor}%
\footnote{sit}%
\footnote{amet,}%
\footnote{consectetur}%
\footnote{adipisicing}%
\footnote{elit,}%
\footnote{sed}%
\end{footnoterange}
are hyperlinked.\footnote{do eiusmod\ldots} Using
\verb|hyperfootnotes=false| or not using \verb|hyperref|
will remove the hyperlinks to the footnotes.\medskip

This text also bears a multiplicity of footnotes,
but due to the use of the starred
\begin{footnoterange*}
\footnote{Ut purus elit,}%
\footnote{vestibulum ut,}%
\footnote{placerat ac,}%
\footnote{adipiscing vitae,}%
\footnote{felis.}%
\footnote{Curabitur}%
\footnote{dictum}%
\footnote{gravida}%
\end{footnoterange*}
form of the environment they are not
hyperlinked.\footnote{mauris\ldots}\medskip

The use of the optional parameter for footnotes
\begin{footnoterange}
\footnote[101]{Nam arcu libero,}%
\footnote[102]{nonummy eget,}%
\footnote[103]{consectetuer id,}%
\footnote[104]{vulputate a,}%
\footnote[105]{magna.}
\end{footnoterange}
is also possible, but should be used consistently.

\pagebreak

When there is only one footnote
\begin{footnoterange}
\footnote{a single footnote}
\end{footnoterange}
in the \verb|footnoterange| environment, only one
footnotemark is used and an info is written into the
\verb|log|-file.\medskip

The usage of other footnote-number-representations
(e.\,g.~\verb|\Roman|, \verb|\roman|, \verb|\Alph|, \verb|\alph|,
\verb|\fnsymbol|) is also possible, but two things must be taken
into account: The number of footnote references is restricted
(for example with \verb|\Alph| only references A to Z are possible),
which can be fixed e.\,g. with the
\href{https://www.ctan.org/pkg/alphalph}{\texttt{alphalph} package},
and references to footnote-symbol-ranges (\verb|\fnsymbol|)
are probably not very clear.

\end{document}
%</example>
%    \end{macrocode}
%
% \StopEventually{}
%
% \section{The implementation}
%
% We start off by checking that we are loading into \LaTeXe{} and
% announcing the name and version of this package.
%
%    \begin{macrocode}
%<*package>
%    \end{macrocode}
%
%    \begin{macrocode}
\NeedsTeXFormat{LaTeX2e}[2011/06/27]
\ProvidesPackage{footnoterange}[2019/11/24 v1.0c
            References to ranges of footnotes (HMM)]
%    \end{macrocode}
%
% A short description of the \xpackage{footnoterange} package:
%
%    \begin{macrocode}
%% Provides the footnoterange and footnoterange* environments to
%% refer to ranges of footnotes.
%    \end{macrocode}
%
% We need the \xpackage{ltxcmds} package:
%
%    \begin{macrocode}
\RequirePackage{ltxcmds}[2011/11/09]% v1.22; for "\ltx@ifpackageloaded"
\RequirePackage{letltxmacro}[2010/09/02]% v1.4; for "\GlobalLetLtxMacro"

%    \end{macrocode}
%
% We use |\ifHy@hyperfootnotes ... \else ... \fi|, but when the \xpackage{hyperref}
% package has not been loaded, |\ifHy@hyperfootnotes| is not defined, and \LaTeX
% relates the |\else| and/or |\fi| (wrongly) to other |\if|s or groups.
% Therefore the |\ifHy@hyperfootnotes| are placed in their own commands,
% which are not looked at when the \xpackage{hyperref} package has not been loaded.
%
% \pagebreak
%
% \begin{macro}{\fnr@ifhyperrfn}
%    \begin{macrocode}
\newcommand{\fnr@ifhyperrfn}{%
  \ifHy@hyperfootnotes% option hyperfootnotes=true
    \addtocounter{Hfootnote}{+1}%
    % Code from the hyperref package
      \global\let\Hy@saved@currentHref\@currentHref%
      \hyper@makecurrent{Hfootnote}%
      \global\let\Hy@footnote@currentHref\@currentHref%
      \global\let\@currentHref\Hy@saved@currentHref%
    % End of code from the hyperref package
  \fi%
}

%    \end{macrocode}
% \end{macro}
%
% \begin{macro}{\fnr@rangefootnote}
% The |\fnr@rangefootnote| is the internal replacement for the original\newline
% |\footnote| command inside of a |footnoterange| environment.
%
%    \begin{macrocode}
\newcommand{\fnr@rangefootnote}[2][\empty]{%
%    \end{macrocode}
%
% |#1| contains the optional argument, which is predefined as |\empty|.
%
%    \begin{macrocode}
  \nolinebreak%
  % no \footnotemark here
  \addtocounter\@mpfn{+1}%
  \ltx@ifpackageloaded{hyperref}{\fnr@ifhyperrfn}{\relax}%
%    \end{macrocode}
%
% See the explanation before the declaration of the |\fnr@ifhyperrfn| command!
%
%    \begin{macrocode}
  \xdef\fnr@opt{#1}% contains the optional argument
  \xdef\fnr@arabic{\arabic\@mpfn}%
  \protected@edef\fnr@formated{\thefootnote}%
%    \end{macrocode}
%
% The |\footnotemark| is not set (yet), but the |\footnotetext|,
% and it gets a label.
%
%    \begin{macrocode}
  \ifx\fnr@opt\empty%
    \footnotetext{\label{fnr:\fnr@arabic}#2}%
  \else%
    \PackageWarning{footnoterange}{%
      Custom footnote-numbers used in a footnoterange\MessageBreak%
      environment: Footnote with natural number\MessageBreak%
      \ifx\fnr@formated\fnr@arabic%
        \fnr@arabic\space%
      \else%
        \space\fnr@arabic\space (\fnr@formated )%
      \fi%
      \MessageBreak%
      was renumbered to \fnr@opt\MessageBreak%
     }%
    \ltx@ifpackageloaded{hyperref}{% hyperref loaded
      \footnotetext[#1]{\phantomsection\label{fnr:\fnr@arabic}#2}%
     }{% hyperref not loaded
      \footnotetext[#1]{\label{fnr:\fnr@arabic}#2}%
     }%
  \fi%
}

%    \end{macrocode}
% \end{macro}
%
%    \begin{macrocode}

%    \end{macrocode}
% \pagebreak
%
% |\ifHy@hyperfootnotes| \textquotedblleft encapsulation\textquotedblright{} again:
%
% \begin{macro}{\fnr@ifhyperfnrA}
%    \begin{macrocode}
\newcommand{\fnr@ifhyperfnrA}{%
  \ifHy@hyperfootnotes% option hyperfootnotes=true
    \hbox{\hyperref[fnr:\fnr@first]{%
          \@textsuperscript{\normalfont \ref*{fnr:\fnr@first}}}}%
  \else% option hyperfootnotes=false
    \hbox{\@textsuperscript{\normalfont \ref*{fnr:\fnr@first}}}%
  \fi%
}

%    \end{macrocode}
% \end{macro}
%
% \begin{macro}{\fnr@ifhyperfnrN}
%    \begin{macrocode}
\newcommand{\fnr@ifhyperfnrN}{%
  \ifHy@hyperfootnotes% option hyperfootnotes=true
    \hbox{\hyperref[fnr:\fnr@first]{%
            \@textsuperscript{\normalfont \ref*{fnr:\fnr@first}}}%
          \@textsuperscript{\normalfont --}%
          \hyperref[fnr:\arabic\@mpfn]{%
            \@textsuperscript{\normalfont \ref*{fnr:\arabic\@mpfn}}}}%
  \else% option hyperfootnotes=false
    \hbox{\@textsuperscript{\normalfont \ref*{fnr:\fnr@first}%
                                        --\ref*{fnr:\arabic\@mpfn}}}%
  \fi%
}

%    \end{macrocode}
% \end{macro}
%
% \begin{environment}{footnoterange}
% The |footnoterange| environment remembers the next footnote number as\newline
% |\fnr@first| and the current |\footnote| definition as |\fnr@origfn|
% and uses |\fnr@rangefootnote| instead.
%
%    \begin{macrocode}
\newenvironment{footnoterange}{%
  % at begin of the footnoterange environment
  \ifhmode \unskip \fi% FMi
  \addtocounter\@mpfn{+1}%
  \xdef\fnr@first{\arabic\@mpfn}%
  \addtocounter\@mpfn{-1}%
  \let\fnr@origfn\footnote%
  \let\footnote\fnr@rangefootnote%
  \GlobalLetLtxMacro{\fnr@origfnmark}{\footnotemark}%
  \GlobalLetLtxMacro{\footnotemark}{\fnr@rangefootnotemark}%
  \ignorespaces% FMi
%    \end{macrocode}
%
% End of the environment:
%
%    \begin{macrocode}
}{% at end of the footnoterange environment
  \ifhmode% FMi 
    \unskip% FMi
%    \end{macrocode}
%
% Remember the spacefactor for restoring after the environment has ended
% (only in |hmode|!). \% FMi
%
%    \begin{macrocode}
    \xdef\fnr@aftergroup{\spacefactor\the\spacefactor\relax}% FMi
  \else% FMi
%    \end{macrocode}
%
% Otherwise do something harmless. \% FMi
%
%    \begin{macrocode}
    \xdef\fnr@aftergroup{\relax}%
  \fi% FMi
%    \end{macrocode}
%
% At the end of the environment the previous definition of |\footnote| is restored
%
%    \begin{macrocode}
  \let\footnote\fnr@origfn%
  \GlobalLetLtxMacro{\footnotemark}{\fnr@origfnmark}%
%    \end{macrocode}
%
% and the references are placed:
%
%    \begin{macrocode}
  \ifx\fnr@first\fnr@arabic% There is only one reference in the environment.
    \PackageInfo{footnoterange}{%
      Environment footnoterange used,\MessageBreak%
      but only one footnote (number \fnr@first ) is contained%
      \MessageBreak%
     }%
    \ltx@ifpackageloaded{hyperref}{\fnr@ifhyperfnrA% hyperref package loaded
    }{% hyperref package not loaded
      \hbox{\@textsuperscript{\normalfont \ref{fnr:\fnr@first}}}%
     }%
  \else%
    \ltx@ifpackageloaded{hyperref}{\fnr@ifhyperfnrN% hyperref package loaded
    }{% hyperref package not loaded
      \hbox{\@textsuperscript{\normalfont \ref{fnr:\fnr@first}%
                                        --\ref{fnr:\arabic\@mpfn}}}%
     }%
  \fi%
%    \end{macrocode}
%
% Restore the spacefactor after the environment has ended. \% FMi
%
%    \begin{macrocode}
  \aftergroup\fnr@aftergroup% FMi
}

%    \end{macrocode}
% \end{environment}
%
% \begin{environment}{footnoterange*}
% The |footnoterange*| environment does the same as the |footnoterange|
% environment but suppresses all hyperlinks to footnotes inside.
%
%    \begin{macrocode}
\newenvironment{footnoterange*}{%
  % at begin of the footnoterange* environment
  \ifhmode \unskip \fi% FMi
  \addtocounter\@mpfn{+1}%
  \xdef\fnr@first{\arabic\@mpfn}%
  \addtocounter\@mpfn{-1}%
  \let\fnr@origfn\footnote%
  \let\footnote\fnr@rangefootnote%
  \GlobalLetLtxMacro{\fnr@origfnmark}{\footnotemark}%
  \GlobalLetLtxMacro{\footnotemark}{\fnr@rangefootnotemark}%
  \ignorespaces% FMi
}{% at end of the footnoterange* environment
  \ifhmode% FMi 
    \unskip% FMi
    \xdef\fnr@aftergroup{\spacefactor\the\spacefactor\relax}% FMi
  \else% FMi
    \xdef\fnr@aftergroup{\relax}%
  \fi% FMi
  \let\footnote\fnr@origfn%
  \GlobalLetLtxMacro{\footnotemark}{\fnr@origfnmark}%
  \ifx\fnr@first\fnr@arabic% There is only one reference in the environment.
    \PackageInfo{footnoterange}{%
      Environment footnoterange used,\MessageBreak%
      but only one footnote (number \fnr@first ) is contained%
      \MessageBreak%
     }%
%    \end{macrocode}
% \pagebreak
%    \begin{macrocode}
    \ltx@ifpackageloaded{hyperref}{% hyperref package loaded
      \hbox{\@textsuperscript{\normalfont \ref*{fnr:\fnr@first}}}%
    }{% hyperref package not loaded
      \hbox{\@textsuperscript{\normalfont \ref{fnr:\fnr@first}}}%
     }%
  \else%
    \ltx@ifpackageloaded{hyperref}{% hyperref package loaded
      \hbox{\@textsuperscript{\normalfont \ref*{fnr:\fnr@first}%
                                        --\ref*{fnr:\arabic\@mpfn}}}%
    }{% hyperref package not loaded
      \hbox{\@textsuperscript{\normalfont \ref{fnr:\fnr@first}%
                                        --\ref{fnr:\arabic\@mpfn}}}%
     }%
  \fi%
  \aftergroup\fnr@aftergroup% FMi
}

%    \end{macrocode}
% \end{environment}
%
% The \xpackage{footnoterange} package does not handle |\footnotemark|s,
% but at least the user is made aware of this:
%
% \begin{macro}{\fnr@rangefootnotemark}
%    \begin{macrocode}

\newcommand{\fnr@rangefootnotemark}[1][\empty]{%
  \PackageError{footnoterange}{%
     \string\footnotemark\space not supported in footnoterange(*) environment%
   }{The footnoterange package does not handle a \string\footnotemark.\MessageBreak%
     Either use \string\footnote\string{...\string}\space instead of%
     \string\footnotemark\space and \string\footnotetext\string{...\string}\MessageBreak%
     in footnoterange and footnoterange* environments,\MessageBreak%
     or do not use those environments around a \string\footnotemark\MessageBreak%
     (or \string\renewcommand\string{\string\fnr@rangefootnotemark\string}\MessageBreak%
     with your own fixing code).%
    }%
  \xdef\fnr@mopt{#1}%
  \ifx\fnr@mopt\empty%
    \fnr@origfnmark%
  \else%
    \fnr@origfnmark[#1]%
  \fi%
}

%    \end{macrocode}
% \end{macro}
%
%    \begin{macrocode}
%</package>
%    \end{macrocode}
%
% \pagebreak
%
% \section{Installation}
%
% \subsection{Downloads\label{ss:Downloads}}
%
% Everything is available on CTAN,
% \url{https://www.ctan.org/tex-archive/},
% but may need additional packages themselves.
%
% \DescribeMacro{footnoterange.dtx}
% For unpacking the \xfile{footnoterange.dtx} file
% and constructing the documentation it is required:
% \begin{description}
% \item[-] \TeX{} Format \LaTeXe{}, \url{https://www.CTAN.org/}
%
% \item[-] document class \xpackage{ltxdoc}, 2015/03/26, v2.0w,
%           \url{https://ctan.org/pkg/ltxdoc}
%
% \item[-] package \xpackage{holtxdoc}, 2012/03/21, v0.24,
%           \url{https://ctan.org/pkg/holtxdoc}
% \end{description}
%
% \DescribeMacro{footnoterange.sty}
% The |footnoterange.sty| for \LaTeXe{} (i.\,e. each document using
% the \xpackage{footnoterange} package) requires:
% \begin{description}
% \item[-] \TeX{} Format \LaTeXe{}, \url{https://www.CTAN.org/}
%
% \item[-] package \xpackage{ltxcmds}, 2011/04/18, v1.20,
%           \url{https://www.ctan.org/pkg/ltxcmds}
% \end{description}
%
% \DescribeMacro{footnoterange-example.tex}
% The |footnoterange-example.tex| requires the same files as all
% documents using the \xpackage{footnoterange} package and additionally:
% \begin{description}
% \item[-] class \xpackage{article}, 2007/10/19, v1.4h, from \xpackage{classes.dtx}:\newline
%           \url{https://www.ctan.org/tex-archive/macros/latex/base/}
%
% \item[-] package \xpackage{footnoterange}, 2019/11/24, v1.0c,\newline
%           \url{https://www.ctan.org/pkg/footnoterange}\newline
%           (Well, it is the example file for this package, and because you are
%            reading the documentation for the \xpackage{footnoterange} package,
%            it can be assumed that you already have some version of it --
%            is it the current one?)
%
% \item[-] package \xpackage{hyperref}, 2012/11/06, v6.83m,\newline
%           \url{https://www.ctan.org/pkg/hyperref}
% \end{description}
%
% \DescribeMacro{Oberdiek}
% \DescribeMacro{holtxdoc}
% \DescribeMacro{ltxcmds}
% \DescribeMacro{alphalph}
% All packages of \textsc{Heiko Oberdiek}'s bundle `oberdiek'
% (especially \xpackage{holtxdoc}, \xpackage{ltxcmds}, and \xpackage{alphalph})
% are also available in a TDS compliant ZIP archive:\newline
% \url{http://mirror.ctan.org/install/macros/latex/contrib/oberdiek.tds.zip}.\newline
% It is probably best to download and use this, because the packages in there
% should be both recent and compatible.
%
% \DescribeMacro{hyperref}
% \noindent \xpackage{hyperref} is not included in that bundle and needs to be
% downloaded separately,\newline
% \url{http://mirrors.ctan.org/install/macros/latex/contrib/hyperref.tds.zip}.
%
% \DescribeMacro{M\"{u}nch}
% A list of my packages can be found at \url{https://www.ctan.org/author/muench-hm}.
%
% \pagebreak
%
% \subsection{Package, unpacking TDS}
% \paragraph{Package.} This package is available on \url{https://www.CTAN.org}.
% \begin{description}
% \item[\url{http://mirror.ctan.org/macros/latex/contrib/footnoterange/footnoterange.dtx}]\hspace*{0.1cm}
%       The source file.
% \item[\url{http://mirror.ctan.org/macros/latex/contrib/footnoterange/footnoterange.pdf}]\hspace*{0.1cm}
%       The documentation.
% \item[\url{http://mirror.ctan.org/macros/latex/contrib/footnoterange/footnoterange-example.pdf}]\hspace*{0.1cm}
%       The compiled example file, as it should look like.
% \item[\url{http://mirror.ctan.org/macros/latex/contrib/footnoterange/README}]\hspace*{0.1cm}\newline\hspace*{0.1cm}
%       The README file.
% \end{description}
% There is also a footnoterange.tds.zip available:
% \begin{description}
% \item[\url{http://mirror.ctan.org/install/macros/latex/contrib/footnoterange.tds.zip}]\hspace*{0.1cm}
%       Everything in TDS compliant, compiled format.
% \end{description}
% which additionally contains\newline
% \begin{tabular}{ll}
% footnoterange.ins & The installation file.\\
% footnoterange.drv & The driver to generate the documentation.\\
% footnoterange.sty & The \xext{sty}le file.
% \end{tabular}
%
% \bigskip
%
% \noindent For required other packages, please see the preceding subsection.
%
% \paragraph{Unpacking.} The \xfile{.dtx} file is a self-extracting
% \docstrip{} archive. The files are extracted by running the
% \xfile{.dtx} through \plainTeX:
% \begin{quote}
%   \verb|tex footnoterange.dtx|
% \end{quote}
%
% About generating the documentation see paragraph~\ref{GenDoc} below.
%
% \paragraph{TDS.} Now the different files must be moved into
% the different directories in your installation TDS tree
% (also known as \xfile{texmf} tree):
% \begin{quote}
% \def\t{^^A
% \begin{tabular}{@{}>{\ttfamily}l@{ $\rightarrow$ }>{\ttfamily}l@{}}
%   footnoterange.sty & tex/latex/footnoterange/footnoterange.sty\\
%   footnoterange.pdf & doc/latex/footnoterange/footnoterange.pdf\\
%   footnoterange-example.tex & doc/latex/footnoterange/footnoterange-example.tex\\
%   footnoterange-example.pdf & doc/latex/footnoterange/footnoterange-example.pdf\\
%   footnoterange.dtx & source/latex/footnoterange/footnoterange.dtx\\
% \end{tabular}^^A
% }^^A
% \sbox0{\t}^^A
% \ifdim\wd0>\linewidth
%   \begingroup
%     \advance\linewidth by\leftmargin
%     \advance\linewidth by\rightmargin
%   \edef\x{\endgroup
%     \def\noexpand\lw{\the\linewidth}^^A
%   }\x
%   \def\lwbox{^^A
%     \leavevmode
%     \hbox to \linewidth{^^A
%       \kern-\leftmargin\relax
%       \hss
%       \usebox0
%       \hss
%       \kern-\rightmargin\relax
%     }^^A
%   }^^A
%   \ifdim\wd0>\lw
%     \sbox0{\small\t}^^A
%     \ifdim\wd0>\linewidth
%       \ifdim\wd0>\lw
%         \sbox0{\footnotesize\t}^^A
%         \ifdim\wd0>\linewidth
%           \ifdim\wd0>\lw
%             \sbox0{\scriptsize\t}^^A
%             \ifdim\wd0>\linewidth
%               \ifdim\wd0>\lw
%                 \sbox0{\tiny\t}^^A
%                 \ifdim\wd0>\linewidth
%                   \lwbox
%                 \else
%                   \usebox0
%                 \fi
%               \else
%                 \lwbox
%               \fi
%             \else
%               \usebox0
%             \fi
%           \else
%             \lwbox
%           \fi
%         \else
%           \usebox0
%         \fi
%       \else
%         \lwbox
%       \fi
%     \else
%       \usebox0
%     \fi
%   \else
%     \lwbox
%   \fi
% \else
%   \usebox0
% \fi
% \end{quote}
% If you have a \xfile{docstrip.cfg} that configures and enables \docstrip's
% TDS installing feature, then some files can already be in the right
% place, see the documentation of \docstrip.
%
% \subsection{Refresh file name databases}
%
% If your \TeX~distribution (\TeX{} Live, \mikTeX, \teTeX, \dots) relies on
% file name databases, you must refresh these. For example, \teTeX{} users run
% \verb|texhash| or \verb|mktexlsr|.
%
% \subsection{Some details for the interested}
%
% \paragraph{Unpacking with \LaTeX{}.}
% The \xfile{.dtx} chooses its action depending on the format:
% \begin{description}
% \item[\plainTeX:] Run \docstrip{} and extract the files.
% \item[\LaTeX:] Generate the documentation.
% \end{description}
% If you insist on using \LaTeX{} for \docstrip{} (really,
% \docstrip{} does not need \LaTeX ), then inform the autodetect routine
% about your intention:
% \begin{quote}
%   \verb|latex \let\install=y\input{footnoterange.dtx}|
% \end{quote}
% Do not forget to quote the argument according to the demands
% of your shell.
%
% \paragraph{Generating the documentation.\label{GenDoc}}
% You can use both the \xfile{.dtx} or the \xfile{.drv} to generate
% the documentation. The process can be configured by a
% configuration file \xfile{ltxdoc.cfg}. For instance, put the following
% line into this file, if you want to have A4 as paper format:
% \begin{quote}
%   \verb|\PassOptionsToClass{a4paper}{article}|
% \end{quote}
%
% \noindent An example follows how to generate the
% documentation with \pdfLaTeX :
%
% \begin{quote}
%\begin{verbatim}
%pdflatex footnoterange.drv
%makeindex -s gind.ist footnoterange.idx
%pdflatex footnoterange.drv
%makeindex -s gind.ist footnoterange.idx
%pdflatex footnoterange.drv
%\end{verbatim}
% \end{quote}
%
% \subsection{Compiling the example}
%
% The example file, \textsf{footnoterange-example.tex}, can be compiled via\newline
% \indent \verb|(pdf)latex footnoterange-example.tex|\newline
% but will need probably two compiler runs to get everything right.\newline
% \verb|footnoterange-example.tex| can be created by\newline
% \indent \verb|tex footnoterange.dtx|.\newline
% \verb|footnoterange.dtx| is at
% \url{https://www.ctan.org/pkg/footnoterange}.\newline
%
% \section{Acknowledgements}
%
% I (H.-Martin M\"{u}nch) would like to thank \textsc{Heiko Oberdiek}
% for providing a~lot~(!) of useful packages (from
% which I also got everything I know about creating a file in \xext{dtx}
% format, ok, say it: copying), \textsc{Frank Mittelbach} for several bug reports,
% for code for improving the package, and for pointing me to the \xpackage{fnpct}
% package as alternative, Willie Wong for reporting a bug and its fix,
% everybody of the CTAN team for managing CTAN,
% the \Newsgroup{comp.text.tex} and \Newsgroup{de.comp.text.tex} newsgroups,
% and everybody at \url{https://tex.stackexchange.com/}
% for their help in all things \hbox{\TeX{}.}
% This package was created as response to question
% \textquotedblleft Compressing consecutive footnote
% marks\textquotedblright{} by user001\newline
% (\url{https://tex.stackexchange.com/users/10127/user001}) on\newline
% \url{https://tex.stackexchange.com/q/44165/6865}.
%
% \newpage
% \phantomsection
% \begin{History}\label{History}
%   \begin{Version}{2011/02/11 v0.1ff.}
%     \item Some versions at \url{https://tex.stackexchange.com/q/44165/6865}.
%   \end{Version}
%   \begin{Version}{2012/02/17 v1.0a}
%     \item Package created and uploaded to CTAN.
%   \end{Version}
%   \begin{Version}{2018/10/15 v1.0b}
%     \item Fixed/updated minor details.
%     \item Bug fix: Did not work without \xpackage{hyperref}, wrong spacefactor.
%     \item Listing the \href{https://www.ctan.org/pkg/fnpct}{\xpackage{fnpct}}
%           package as alternative.
%   \end{Version}
%   \begin{Version}{2019/11/24 v1.0c}
%     \item Bug fix: Issue with footnote symbols.
%   \end{Version}
% \end{History}
%
% \bigskip
%
% When you find a mistake or have a suggestion for an improvement of this package,
% please send an e-mail to the maintainer, thanks! (Please see BUG REPORTS in the README.)
%
% \PrintIndex
%
% \Finale
\endinput|
% \end{quote}
% Do not forget to quote the argument according to the demands
% of your shell.
%
% \paragraph{Generating the documentation.\label{GenDoc}}
% You can use both the \xfile{.dtx} or the \xfile{.drv} to generate
% the documentation. The process can be configured by a
% configuration file \xfile{ltxdoc.cfg}. For instance, put the following
% line into this file, if you want to have A4 as paper format:
% \begin{quote}
%   \verb|\PassOptionsToClass{a4paper}{article}|
% \end{quote}
%
% \noindent An example follows how to generate the
% documentation with \pdfLaTeX :
%
% \begin{quote}
%\begin{verbatim}
%pdflatex footnoterange.drv
%makeindex -s gind.ist footnoterange.idx
%pdflatex footnoterange.drv
%makeindex -s gind.ist footnoterange.idx
%pdflatex footnoterange.drv
%\end{verbatim}
% \end{quote}
%
% \subsection{Compiling the example}
%
% The example file, \textsf{footnoterange-example.tex}, can be compiled via\newline
% \indent \verb|(pdf)latex footnoterange-example.tex|\newline
% but will need probably two compiler runs to get everything right.\newline
% \verb|footnoterange-example.tex| can be created by\newline
% \indent \verb|tex footnoterange.dtx|.\newline
% \verb|footnoterange.dtx| is at
% \url{https://www.ctan.org/pkg/footnoterange}.\newline
%
% \section{Acknowledgements}
%
% I (H.-Martin M\"{u}nch) would like to thank \textsc{Heiko Oberdiek}
% for providing a~lot~(!) of useful packages (from
% which I also got everything I know about creating a file in \xext{dtx}
% format, ok, say it: copying), \textsc{Frank Mittelbach} for several bug reports,
% for code for improving the package, and for pointing me to the \xpackage{fnpct}
% package as alternative, Willie Wong for reporting a bug and its fix,
% everybody of the CTAN team for managing CTAN,
% the \Newsgroup{comp.text.tex} and \Newsgroup{de.comp.text.tex} newsgroups,
% and everybody at \url{https://tex.stackexchange.com/}
% for their help in all things \hbox{\TeX{}.}
% This package was created as response to question
% \textquotedblleft Compressing consecutive footnote
% marks\textquotedblright{} by user001\newline
% (\url{https://tex.stackexchange.com/users/10127/user001}) on\newline
% \url{https://tex.stackexchange.com/q/44165/6865}.
%
% \newpage
% \phantomsection
% \begin{History}\label{History}
%   \begin{Version}{2011/02/11 v0.1ff.}
%     \item Some versions at \url{https://tex.stackexchange.com/q/44165/6865}.
%   \end{Version}
%   \begin{Version}{2012/02/17 v1.0a}
%     \item Package created and uploaded to CTAN.
%   \end{Version}
%   \begin{Version}{2018/10/15 v1.0b}
%     \item Fixed/updated minor details.
%     \item Bug fix: Did not work without \xpackage{hyperref}, wrong spacefactor.
%     \item Listing the \href{https://www.ctan.org/pkg/fnpct}{\xpackage{fnpct}}
%           package as alternative.
%   \end{Version}
%   \begin{Version}{2019/11/24 v1.0c}
%     \item Bug fix: Issue with footnote symbols.
%   \end{Version}
% \end{History}
%
% \bigskip
%
% When you find a mistake or have a suggestion for an improvement of this package,
% please send an e-mail to the maintainer, thanks! (Please see BUG REPORTS in the README.)
%
% \PrintIndex
%
% \Finale
\endinput|
% \end{quote}
% Do not forget to quote the argument according to the demands
% of your shell.
%
% \paragraph{Generating the documentation.\label{GenDoc}}
% You can use both the \xfile{.dtx} or the \xfile{.drv} to generate
% the documentation. The process can be configured by a
% configuration file \xfile{ltxdoc.cfg}. For instance, put the following
% line into this file, if you want to have A4 as paper format:
% \begin{quote}
%   \verb|\PassOptionsToClass{a4paper}{article}|
% \end{quote}
%
% \noindent An example follows how to generate the
% documentation with \pdfLaTeX :
%
% \begin{quote}
%\begin{verbatim}
%pdflatex footnoterange.drv
%makeindex -s gind.ist footnoterange.idx
%pdflatex footnoterange.drv
%makeindex -s gind.ist footnoterange.idx
%pdflatex footnoterange.drv
%\end{verbatim}
% \end{quote}
%
% \subsection{Compiling the example}
%
% The example file, \textsf{footnoterange-example.tex}, can be compiled via\newline
% \indent \verb|(pdf)latex footnoterange-example.tex|\newline
% but will need probably two compiler runs to get everything right.\newline
% \verb|footnoterange-example.tex| can be created by\newline
% \indent \verb|tex footnoterange.dtx|.\newline
% \verb|footnoterange.dtx| is at
% \url{https://www.ctan.org/pkg/footnoterange}.\newline
%
% \section{Acknowledgements}
%
% I (H.-Martin M\"{u}nch) would like to thank \textsc{Heiko Oberdiek}
% for providing a~lot~(!) of useful packages (from
% which I also got everything I know about creating a file in \xext{dtx}
% format, ok, say it: copying), \textsc{Frank Mittelbach} for several bug reports,
% for code for improving the package, and for pointing me to the \xpackage{fnpct}
% package as alternative, Willie Wong for reporting a bug and its fix,
% everybody of the CTAN team for managing CTAN,
% the \Newsgroup{comp.text.tex} and \Newsgroup{de.comp.text.tex} newsgroups,
% and everybody at \url{https://tex.stackexchange.com/}
% for their help in all things \hbox{\TeX{}.}
% This package was created as response to question
% \textquotedblleft Compressing consecutive footnote
% marks\textquotedblright{} by user001\newline
% (\url{https://tex.stackexchange.com/users/10127/user001}) on\newline
% \url{https://tex.stackexchange.com/q/44165/6865}.
%
% \newpage
% \phantomsection
% \begin{History}\label{History}
%   \begin{Version}{2011/02/11 v0.1ff.}
%     \item Some versions at \url{https://tex.stackexchange.com/q/44165/6865}.
%   \end{Version}
%   \begin{Version}{2012/02/17 v1.0a}
%     \item Package created and uploaded to CTAN.
%   \end{Version}
%   \begin{Version}{2018/10/15 v1.0b}
%     \item Fixed/updated minor details.
%     \item Bug fix: Did not work without \xpackage{hyperref}, wrong spacefactor.
%     \item Listing the \href{https://www.ctan.org/pkg/fnpct}{\xpackage{fnpct}}
%           package as alternative.
%   \end{Version}
%   \begin{Version}{2019/11/24 v1.0c}
%     \item Bug fix: Issue with footnote symbols.
%   \end{Version}
% \end{History}
%
% \bigskip
%
% When you find a mistake or have a suggestion for an improvement of this package,
% please send an e-mail to the maintainer, thanks! (Please see BUG REPORTS in the README.)
%
% \PrintIndex
%
% \Finale
\endinput|
% \end{quote}
% Do not forget to quote the argument according to the demands
% of your shell.
%
% \paragraph{Generating the documentation.\label{GenDoc}}
% You can use both the \xfile{.dtx} or the \xfile{.drv} to generate
% the documentation. The process can be configured by a
% configuration file \xfile{ltxdoc.cfg}. For instance, put the following
% line into this file, if you want to have A4 as paper format:
% \begin{quote}
%   \verb|\PassOptionsToClass{a4paper}{article}|
% \end{quote}
%
% \noindent An example follows how to generate the
% documentation with \pdfLaTeX :
%
% \begin{quote}
%\begin{verbatim}
%pdflatex footnoterange.drv
%makeindex -s gind.ist footnoterange.idx
%pdflatex footnoterange.drv
%makeindex -s gind.ist footnoterange.idx
%pdflatex footnoterange.drv
%\end{verbatim}
% \end{quote}
%
% \subsection{Compiling the example}
%
% The example file, \textsf{footnoterange-example.tex}, can be compiled via\newline
% \indent \verb|(pdf)latex footnoterange-example.tex|\newline
% but will need probably two compiler runs to get everything right.\newline
% \verb|footnoterange-example.tex| can be created by\newline
% \indent \verb|tex footnoterange.dtx|.\newline
% \verb|footnoterange.dtx| is at
% \url{https://www.ctan.org/pkg/footnoterange}.\newline
%
% \section{Acknowledgements}
%
% I (H.-Martin M\"{u}nch) would like to thank \textsc{Heiko Oberdiek}
% for providing a~lot~(!) of useful packages (from
% which I also got everything I know about creating a file in \xext{dtx}
% format, ok, say it: copying), \textsc{Frank Mittelbach} for several bug reports,
% for code for improving the package, and for pointing me to the \xpackage{fnpct}
% package as alternative, Willie Wong for reporting a bug and its fix,
% everybody of the CTAN team for managing CTAN,
% the \Newsgroup{comp.text.tex} and \Newsgroup{de.comp.text.tex} newsgroups,
% and everybody at \url{https://tex.stackexchange.com/}
% for their help in all things \hbox{\TeX{}.}
% This package was created as response to question
% \textquotedblleft Compressing consecutive footnote
% marks\textquotedblright{} by user001\newline
% (\url{https://tex.stackexchange.com/users/10127/user001}) on\newline
% \url{https://tex.stackexchange.com/q/44165/6865}.
%
% \newpage
% \phantomsection
% \begin{History}\label{History}
%   \begin{Version}{2011/02/11 v0.1ff.}
%     \item Some versions at \url{https://tex.stackexchange.com/q/44165/6865}.
%   \end{Version}
%   \begin{Version}{2012/02/17 v1.0a}
%     \item Package created and uploaded to CTAN.
%   \end{Version}
%   \begin{Version}{2018/10/15 v1.0b}
%     \item Fixed/updated minor details.
%     \item Bug fix: Did not work without \xpackage{hyperref}, wrong spacefactor.
%     \item Listing the \href{https://www.ctan.org/pkg/fnpct}{\xpackage{fnpct}}
%           package as alternative.
%   \end{Version}
%   \begin{Version}{2019/11/24 v1.0c}
%     \item Bug fix: Issue with footnote symbols.
%   \end{Version}
% \end{History}
%
% \bigskip
%
% When you find a mistake or have a suggestion for an improvement of this package,
% please send an e-mail to the maintainer, thanks! (Please see BUG REPORTS in the README.)
%
% \PrintIndex
%
% \Finale
\endinput
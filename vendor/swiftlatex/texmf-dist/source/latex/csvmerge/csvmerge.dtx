% \iffalse meta-comment
%
% Copyright (C) 2019 by Laurence R Taylor
% -----------------------------------
%
% This file may be distributed and/or modified under the
% conditions of the LaTeX Project Public License, either version 1.3 
% of this license or (at your option) any later version.
% The latest version of this license is in:
%
% http://www.latex-project.org/lppl.txt
%
% and version 1.3 or later is part of all distributions of LaTeX
% version 2005/12/01 or later.
%
% \fi
%
% \iffalse
%
%<*driver>
\documentclass{ltxdoc}
\usepackage{csvmerge}
\usepackage{hyperref}
\hypersetup{
    colorlinks=true,
    linkcolor=blue,
    filecolor=magenta,      
    urlcolor=cyan,
}
\DisableCrossrefs
\CodelineNumbered
\RecordChanges
\CheckSum{525}
\begin{document}
\DocInput{csvmerge.dtx} 
\end{document}
%</driver>
% \fi
%
% 
% \CharacterTable
%  {Upper-case \A\B\C\D\E\F\G\H\I\J\K\L\M\N\O\P\Q\R\S\T\U\V\W\X\Y\Z
%  Lower-case \a\b\c\d\e\f\g\h\i\j\k\l\m\n\o\p\q\r\s\t\u\v\w\x\y\z
%  Digits \0\1\2\3\4\5\6\7\8\9
%   Exclamation   \!     Double quote  \"     Hash (number) \#
%   Dollar        \$     Percent       \%     Ampersand     \&
%   Acute accent  \'    Left paren    \(     Right paren   \)
%  Asterisk \*  Plus \+  Comma \,
%  Minus \-  Point \.  Solidus   \/
%   Colon \:  Semicolon  \;  Less than     \<
%   Equals \=  Greater than  \>  Question mark \?
%   Commercial at \@   Left bracket  \[   Backslash     \\
%   Right bracket \]  Circumflex    \^  Underscore    \_
%   Grave accent  \`    Left brace    \{     Vertical bar  \|  
%   Right brace   \}     Tilde         \~}
%   
%
%
% \changes{v1.0}{?2019?/?07?/?17?}{Initial version} 
%    \title{csvmerge}
%    \author{Laurence R Taylor}
%    \date{August 8, 2019}
% \maketitle
%
% \begin{abstract}
%  The package |csvmerge.sty| is designed to take a bit of \TeX\ code such as a letter or a line
%    in a table and produce
%    multiple copies of it with parts of each copy filled in from a data file. 
%    It is invoked in the usual manner from a master-file, that is a standard
%    \LaTeX\ file with 
%    
%    \null\noindent\hskip 1in|\documentclass{whatever}|\\
%    \null\hskip 1in|\usepackage{csvmerge}|\\
%    \null\hskip 1.5in$\vdots$
%
%    The data source is a standard csv file. 
%    The bit of \TeX\ code resides in a blueprint file. 
%    The |\usepackage{csvmerge}| can go anywhere in the preamble. 
% \end{abstract}
%   \tableofcontents
%   \def\Hyperlink#1{\hyperlink{#1Description}{\tt \mm #1}}
%   \def\Hypertarget#1#2{\hypertarget{#1Description}{#2}}
%   \xdef\mm{$\backslash\hskip-5pt\backslash$}
% \section{Introduction}
%
%    The |csvmerge.sty| merges \TeX\ code with a data file. 
%    \begin{itemize}
%    \item 
%    The \TeX\ code comes from a blueprint file, which looks like
%    a standard \TeX\ file except it contains no code which should 
%    go into the preamble of a normal \TeX\ file. 
%    The blueprint file should also not contain a |\begin{document}| 
%    nor an |\end{document}|. This stuff goes in the master-file. 
%    \item The data file is a standard csv file. 
%    \end{itemize}
%    One difference between |csvmerge.sty| and other packages which 
%    merge |csv| data into a \TeX\ file is that the user does not need to do
%    much with the |csv|-file. As long as the first row contains reasonable 
%    entries describing the contents of its column you are good to go.
%    Extra columns are no problem; subsequent rows can be too short
%    or too long; characters which are special to \TeX\ such as |$| and |&|
%    can appear in the data; \dots\ 
%    \noindent More details concerning the |csv|-file can be found in subsection 2.1. 
%    
%    The |tex|-file consists of standard \TeX\ code. Each row of the |csv|-file is processed
%    and you have access to the values of all the entries on that row as well as the row
%    number. You also can tell if the entry in a particular column is empty.
%    \noindent More details concerning the |tex|-file can be found in subsection 2.2. 
%
%    The main macro, |\mergeFields|, supports two modes. 
%    In one mode, \emph{stream mode}, the \TeX\ code from the |tex|-file  
%    is added to the input stream at the location from which |\mergeFields| is invoked. 
%    In the second mode, \emph{storage mode}, the code is written to a 
%    temporary file. You can then use the standard \LaTeX\ |\input| to read the 
%    temporary file back into the master-file. 
%    
%    Good models for the two modes are the following. Most mail-merges can be
%    done in streaming mode. Because \TeX\ is doing rather complicated things behind
%    the scenes when it is setting a table or a matrix, you are usually better off using
%    storage mode with your blueprint file containing the code for one line of your table/matrix. 
%    After the data is generated you can read it back into your table. 
%    See \hyperlink{subsubsection.B.3.1}{here} for a typical example. 
%    
%  \newpage
% \section{Usage}
%
% \DescribeMacro{\mergeFields}
%    The main macro is |\mergeFields| which takes one argument. 
% The format is  
%\begin{verbatim}
% \mergeFields{
%    tex={<tex-file>}
%    csv = {<csv-file>}
%    tmp = {<tmp-file>}
%   }
%\end{verbatim}
%    The |tmp| entry is not required. If there is none, |mergeFields| operates in 
%    \emph{streaming} mode, otherwise it acts in \emph{storage} mode. 
%    The |<prefix-file>| entry should be a path to the corresponding sort of file. 
%    If the file name has spaces you will need to quote it: life is easier if there are no
%    spaces in file names. 
%    The format |prefix = {<prefix-file>}| is quite forgiving: all leading and trailing spaces 
%    are stripped out so |prefix  =  { <prefix-file>   }| works. 
%    The three lines can be in any order and do not even need to be on separate lines. 
%    The extensions of the three files are irrelevant: the |tex|-file contains \TeX\ code  
%    but it can have an extension of |.txt| (or anything else or none at all) if you prefer.  
%    All that matters is that 
%    when \TeX\ reads the file the contents have the proper content. 
%    If you have a simple project you may be able to complete it just by looking at 
%    the examples in \hyperlink{appendix.B}{appendix B}. 
%
%    \subsection{The data file}
%    The data file is a |csv| spreadsheet file.
%    By default the field delimiter is the comma and the 
%    text delimiter is the double-quotes character. 
%    This is default behavior for most spreadsheets when
%    asked to produce a |cvs| file. 
%    If there are no {\tt "}'s in your |csv|-file there should be no problems.
%    In some cases you may need a different set of delimiters
%    and this is supported. 
%    See \Hyperlink{setDelimitersTabQuote}\ below for more details. 
%
%    The entries in the first row should consist solely of \emph{alphanumeric characters, 
%    punctuation and spaces}. 
%    Each entry is a \emph{label} for that column. 
%    Labels need not be unique although they usually are except sometimes there are
%    several columns with blank labels.
%
%    A \emph{field-name} is extracted from a label by deleting all spaces and making all
%    characters lowercase: e.g. a column labeled |First Name| has a field-name 
%    |firstname|. Commas are also deleted so a column label of |Last, First Name| 
%    becomes a field-name of |lastfirstname|. 
%    Distinct labels can have the same field-name although again 
%    this seems unlikely except for blank labels. 
%
%    Associated to each field-name is a \emph{field-name-macro}. 
%    This macro is private: its name is found by prepending |@LRT@@| to the field-name. 
%    When |\mergeFields| is invoked, the data file is read one row at a time. 
%    All rows after the first are data. 
%    When a row is read, each entry is peeled off and placed in the field-name-macro 
%    whose field-name is the one associated to the label for that column 
%    with |@LRT@| prepended. 
%    For example, if row one of some column contains |First Name|, the field-name is |firstname| and
%    the field-name-macro is |\@LRT@@firstname|. When a row after the first is read, 
%    |\@LRT@@firstname| will expand to the entry in the |First Name| column on that row. 
%
%    When |\mergeFields| has read a row and set the field-name-macros the |tex|-file is 
%    read and the \TeX\ code evaluated. 
%    In streaming mode the result is added to the input stream at
%    the current location whereas in storage mode the result is written to the |tmp|-file. 
%
% \DescribeMacro{End of csv-file}\Hypertarget{End of csv-file}
%    This process continues until the last line in the |csv|-file is processed OR until an entirely 
%    blank line is encountered. 
%    \vskip10pt
%    The request for only alphanumeric, punctuation and blanks for entries in row one is 
%    not actually enforced but the macro |caselower| from |stringstrings| is used to 
%    produce lowercase letters and may do weird things to characters not on the requested list. 
%    Without studying the |stringstrings| manual carefully %
%    you may not be able to identify the field-names if you insist on weird characters. 
%    
%    It is quite possible that several columns can have the same field-name, and hence 
%    the same field-name-macro. In that case the macro will expand to the contents of the 
%    column with that field-name which is furthest to the right.
%
%    Because of the prepended |@LRT@@| you can not use these field-name-macros directly 
%    in your |tex|-file. 
%    The solution is to use \Hyperlink{Field}{}|{field-name}| which will expand correctly. 
%    See the examples in appendix B.
%    \subsubsection{Entry issues}
%    Except for issues discussed here, you should be able to use any 
%    standard |csv|-file.  
%    
%    The data file is read with the usual \TeX\ issues turned off: 
%    |#|, |$|, |&|, |@|, |%|
%    , |^|, |~|,  and \verb!|!
%    can all be in your data file.
%    \begin{itemize}
%    \item 
%    In streaming mode, only |#|, |$|, |&|, |@|, |%|, |^|, |~| print correctly. 
%    The characters \$, \#, \%, @ and \& often occur in data and they print correctly. 
%    \TeX\ code will not be rendered correctly so for example, |\$| in the data file produces 
%    {$``$\tt\$} when \TeX ed. 
%    As long as you are not too creative in your entries you should get the results you want. 
%    \item
%    In storage mode the characters above end up printing correctly in the |tmp|-file
%    BUT when you |\input| the |tmp|-file back into your master-file these characters
%    will resume their usual \TeX\ meaning and wreak havoc with your code.  
%    If you absolutely must have them, writing |\&|, |\$|, |\#|,
%    etc. in the |csv|-file may save you. 
%    \end{itemize}
%    
%    \DescribeMacro{Delimiters}\null\hskip-\parindent
%    There is one more issue. For most csv files, the field separator is the comma. 
%    When a field has a comma in it, the entire field is delimited by double quotes ({\tt"}). 
%    When the field has a double quote it is converted to a pair of double quotes ({\tt""}) and so on.
%    If you have both commas and quotes in a field the best case scenario is that 
%    you will get  |""|'s where you were hoping for |"|'s. This probably is not what you want.  
%    Even worse, a |",| in a data entry will end the entry prematurely at this point 
%    and things deteriorate from there. 
%    
%    There is a mechanism to deal with this. Most spreadsheets can save csv files with other
%    field (and text) delimiters. The second most common version of csv has a tab as the 
%    field delimiter and quotes for fields which have a tab in a field. 
%
%    These two cases are supported out of the box. By default you get the comma-quotes version.
%    If you add |\setDelimitersTabQuote| before calling |\mergeFields| you will get 
%    the tab-quotes version. 
%    If you have a second call to |\mergeFields| and the data file is comma-quotes, 
%    adding |\setDelimitersCommaQuote| before |\mergeFields| will restore comma-quotes.
%    See \hyperlink{setDelimitersCommaQuoteDescription}{setDelimters\dots} 
%    for a more thorough discussion. 
%
%
%    \subsection{The blueprint file}
%    
%    The blueprint file is a normal \TeX\ file with macros defined as above from the |csv|-file. 
%    It should not contain a \verb!\begin{document}! or an \verb!\end{document}! 
%    nor anything that needs to be expanded in the preamble. 
%    Such things belong in the master file. 
%    
%    For mail-merges, the |tex|-file should end with a \verb!\newpage! 
%    to guarantee that each copy of the message appears on a 
%    different page. 
%    If you need custom sizes or headers or you have page numbers that you want to 
%    reset for each copy, you should do this in the |tex|-file. 
%    
%    To use the data from the |csv|-file you need to write |\Field{some-field-name}| wherever 
%    you want the data in your |tex|-file.
%    Some field-names may be rather long so if you are going to use them a lot in your 
%    |tex|-file, feel free to write something like |\def\sfn{\Field{some-field-name}}| at the
%    top of your |tex|-file and then use |\sfn| thereafter.
%    
%    One additional issue occurs in storage mode. You may very well want \TeX\ code in your
%    |tex|-file. When |mergeFields| reads a line of data, you want the field-name-macros to 
%    expand to their current values but you often do not want your \TeX\ code to expand. 
%    It is good practice stick a |\noexpand| in front of each \TeX\ macro. 
%    Each line is expanded once when the line is written to the |tmp| file, so the |\noexpand|
%    disappears and you are left with just your \TeX\ macros. 
%    
%    If you do define macros like |\sfn| as above, you will need 
%    to write\\ |\noexpand\def\noexpand\sfn{\Field{some-field-name}}|. 
% \StopEventually{\PrintIndex}
%
% \section{List of public macros}
%    There are the 6  public macros: links are to descriptions of the |@LRT@|-versions which
%    actually do the work.
%
%   \null\hskip-10pt \begin{tabular}{lll}
%    \Hyperlink{mergeFields}%
%    &\Hyperlink{Field}%
%    &\hbox to 0pt{\Hyperlink{ifFieldEmpty}\hss}%
%    \\
%    \Hyperlink{setDelimitersCommaQuote}&\Hyperlink{setDelimitersTabQuote}%
%    &\Hyperlink{makeMePublic}\\
%    \end{tabular}
%   \begin{itemize}
%    \item |\mergeFields{tex={tex-file}csv={csv-file}}| or
%    \\|\mergeFields{tex={tex-file}csv={csv-file}{tmp={tmp-file}}|\\
%     sets up to do the merge. 
%     \hyperlink{section.2}{Section 2} is devoted to a detailed discussion. 
%     \item |\Field{field-name}| expands to the current value of the field-name-macro.
%     \item |\ifFieldEmpty{field-name to test}{do if empty}{do if not}|
%     \item The two |\setDelimiters| set the delimiters to match those used in the |csv| file: 
%     comma-quotes is the default.
%     \item |\makeMePublic| must be immediately followed by |\@LRT@| or you will get a 
%     |Use of \makeMePublic doesn't match its definition.| error. Then you can write 
%     |\makeMePublic \@LRT@foo\becomes\Foo| to make the public macro |\Foo| that 
%     you just define equal to the private macro |\@LRT@foo|. You may not write 
%     |\makeMePublic{\@LRT@foo}\becomes\Foo| or other variants: the first non-white-space
%     character \TeX\ sees after |\makeMePublic| needs to be the start of |\@LRT@|. 
%
%     As a useful example, 
%     |\makeMePublic\@LRT@Row\becomes\Row| means you can get your hands on
%     the row number using |\Row|.
%    \end{itemize}
%
% \section{Implementation}
%
% \begin{macro}{Header}
% First we have the standard package header stuff. 
% The line numbers in the typeset \jobname.dtx file match the line numbers 
%  in the \jobname.sty file 
% generated by \TeX ing the \jobname.ins file.
%   \setcounter{CodelineNo}{15}
%    \begin{macrocode}
\NeedsTeXFormat{LaTeX2e}[2005/12/01]
\ProvidesPackage{csvmerge}
      [2019/07/17 v1.0 merges TeX code with csv data]

%    \end{macrocode}
% \end{macro}
% \begin{macro}{Required Packages}
% These required packages are part of the standard \href{https://www.tug.org/texlive/}{{\tt TeX Live}} 
%and \href{https://miktex.org}{{\tt MiKTEX}} distributions. 
%    \begin{macrocode}
\RequirePackage{stringstrings}
\RequirePackage{trimspaces}
\RequirePackage{etoolbox}

%    \end{macrocode}
% \end{macro}
% \begin{macro}{Options}
% |csvmerge| supports an option which is rarely needed. All the private macros 
%    (except \hyperlink{exceptionDescription}{one}) in
%    this package have |@LRT@| or |@lrt@| somewhere in their name so are very 
%    unlikely to produce name conflicts. In the very unlikely chance that one or more of
%    the six public macros has a name conflict, 
%    this can be alleviated by putting |\usepackage[foo]{csvmerge}|
%    in your master file. Here |foo| can be any string of alphabetical characters. 
%    Then the names of all six public macros are invoked with a leading |foo|:
%    |\mergeFields| no longer exists but becomes |\foomergeFields|. The evident change happens
%    with the other five macros. 
%    
%    The next two lines of code merely save the option for use 
%    \hyperlink{optionsDescription}{later} at the end.
%    \begin{macrocode}
\DeclareOption*{\xdef\@LRT@CO{\CurrentOption}}
\ProcessOptions\relax

%    \end{macrocode}
% \end{macro}
%    \begin{macrocode}
\makeatletter

%    \end{macrocode}
% \begin{macro}{\mergeFields}
%    \Hypertarget{mergeFields}
%    |\@LRT@mergeFields| actually initiates the work. 
%    It initializes some macros for later use. 
%    Then it collects the |prefix={<prefix-file>}| arguments using |\@LRT@getFilePaths|
%    and checks to be sure it has a |csv|-file
%    and a |tex|-file. If not it bails. 
%    Then it checks if it has a |tmp| file and sets the file reading subroutine, 
%    |\@LRT@Input| to |\@LRT@streaming| is there is no |tmp| file and to 
%    |\@LRT@storage| if there is. If there is a |tmp| file, it is opened. 
%   Then |\@LRT@mergeData| is called which actually does the merging. 
%   Finally, if there is a |tmp| file, it is closed. 
%    \begin{macrocode}
\newcommand{\@LRT@mergeFields}[1]{%
\gdef\@LRT@csv{}%
\gdef\@LRT@tex{}%
\gdef\@LRT@tmp{}%
\setcounter{\@LRT@macroPrefix Row}{2}\relax%
\gdef\@LRT@csvBlankLine{}%
\@LRT@getFilePaths#1=@%
\if@LRT@empty{\@LRT@csv}{\@LRT@missingArgumentMessage{csv}}{}%
\if@LRT@empty{\@LRT@tex}{\@LRT@missingArgumentMessage{tex}}{}%
\if@LRT@empty{\@LRT@tmp}%
{\global\let\@LRT@Input\@LRT@streaming}%
{\global\let\@LRT@Input\@LRT@storage}%
\if@LRT@empty{\@LRT@tmp}{}{%
\immediate\openout\@lrt@tmpFile=\@LRT@tmp\relax}%
\@LRT@mergeData%
\if@LRT@empty{\@LRT@tmp}{}{\immediate\closeout\@lrt@tmpFile\relax}%
}

%    \end{macrocode}
% \end{macro}
%  \Hypertarget{macroPrefix}{}
% \begin{macro}{\@LRT@macroPrefix}
%    \begin{macrocode}
\gdef\@LRT@macroPrefix{@LRT@@}

%    \end{macrocode}
% \end{macro}
%  \Hypertarget{Field}{}
% \begin{macro}{\Field}
%    |\@LRT@Field| is just a wrapper for a |\csname ... \endcsname|. 
%    Notice it adds the |\@LRT@macroPrefix| prefix to the argument so 
%    the input is always the field-name and the output is always the value of 
%    the corresponding field-name-macro. 
%    Because of the third |@| there is no danger that these macros defined from
%    entries in the |csv|-file will conflict with other macros defined in this package 
%    since none of them have a |@@| in their names. 
%    \begin{macrocode}
\def\@LRT@Field#1{%
\csname\@LRT@macroPrefix#1\endcsname%
}

%    \end{macrocode}
% \end{macro}
%
% \begin{macro}{\ifFieldEmpty}
%  \Hypertarget{ifFieldEmpty}{}
%    |\ifFieldEmpty| takes three arguments. 
%    The first argument is a field-name and if the content of the corresponding 
%    macro is empty then the second argument is executed: otherwise the 
%    third argument is executed.
%    
%    \begin{macrocode}
\long\gdef\@LRT@ifFieldEmpty#1#2#3{\ifbool{\@LRT@macroPrefix#1}{#2}{#3}}

%    \end{macrocode}
% \end{macro}
%
% \begin{macro}{\setDelimitersCommaQuote}
%\end{macro}
% \begin{macro}{\setDelimitersTabQuote}
%  \Hypertarget{setDelimitersCommaQuote}{}
%  \Hypertarget{setDelimitersTabQuote}{}
%    These two |setDelimiters| are similar. 
%    Recall how delimiters work. A row in a |csv|-file with comma delimiters 
%    looks like |foo1,foo2,foo3,...| so the delimiter is the comma and if
%    |\@LRT@getFieldDelimited| is put in front of this string it will peal off 
%    |foo1|. Since |^^I| is \TeX\ for the tab, 
%    a row in a csv file with tab delimiters 
%    looks like |foo1^^Ifoo2^^Ifoo3^^I...| and so the first two lines of 
%    |\@LRT@setDelimitersCommaQuote| and |\@LRT@setDelimitersTabQuote| 
%    look the same after switching |,| and |^^I|. 
%    
%    There is a wrinkle in the tab case. The tab normally looks to \TeX\ like white space and 
%    so by the time the line reaches |\@LRT@getFieldDelimited| the line looks like
%    |foo1foo2foo3...| and there is no way to break off |foo1|. The |\catcode`\^^I12\relax|
%    insures  |\@LRT@getFieldDelimited| sees the line as |foo1^^Ifoo2^^Ifoo3^^I...|..
%    This means that the catcode of |^^I| must also be 12 when each line is read and the line
%    and the macro |\gdef\@LRT@catcodes{|\\|\catcode`\^^I12\relax}| insures this. 
%    In the comma delimited version no catcode heroics are needed so |\@LRT@catcodes|
%    is empty. 
%    
%    If some field contains one or more |\@LRT@fieldDelimiterBack|'s then the |csv|-file will use 
%    the text-delimiter at the front and back. When quote is the text delimiter and |foo2| contains 
%    a comma but |foo1| and |foo3| do not, the comma delimited string looks like 
%    |foo1,"foo2",foo3,...| and the tab delimited string looks like 
%     |foo1^^I"foo2"^^Ifoo3^^I...|. 
%    
%    The text delimiters have a front one, the double-quotes in both versions, and the back one which is 
%    the front text delimiter followed by the back field delimiter. 
%    
%    The macro |\@LRT@blankFieldItem| is used to construct what a blank line looks like.
%    In the comma delimited case, you need a string of commas of the same length as  
%    the number of entries on line one of the |csv|-file. In the tab delimited case, the blank line
%    turns out to be a blank macro.
%    
%    The |\@LRT@catcodes| macro is a collection of catcodes that need to be changed: 
%    nothing for the comma delimited case and |^^I| for the tab delimited case.
%
%    See \hyperlink{newDelimiterAppendix}{Appendix A} for more discussion on writing your own delimiters. 
%    \begin{macrocode}
{{%
\gdef\@LRT@setDelimitersCommaQuote{
\gdef\@LRT@fieldDelimiterBack{,}
\gdef\@LRT@getFieldDelimited##1,{\@LRT@getFirstArg{##1}}
\gdef\@LRT@textDelimiterFront{"}
\gdef\@LRT@getTextDelimited"##1",{\@LRT@getFirstArg{##1}}
\gdef\@LRT@blankFieldItem{,}
\gdef\@LRT@catcodes{}
}%
}}

{{\catcode`\^^I12\relax%
\gdef\@LRT@setDelimitersTabQuote{
\gdef\@LRT@fieldDelimiterBack{^^I}
\gdef\@LRT@getFieldDelimited##1^^I{\@LRT@getFirstArg{##1}}
\gdef\@LRT@textDelimiterFront{"}
\gdef\@LRT@getTextDelimited"##1"^^I{\@LRT@getFirstArg{##1}}
\gdef\@LRT@blankFieldItem{}
\gdef\@LRT@catcodes{\catcode`\^^I12\relax}
}%
}}

%    \end{macrocode}
% \end{macro}
%
% \begin{macro}{\@LRT@makeMePublic}
%    \hypertarget{exceptionDescription}{}
%    \hypertarget{makeMePublicDescription}{}
%    The next line of code defines the only macro which does not have |@LRT@|
%    in its name. It seems sufficiently weird that there should be no name conflicts. 
%    Notice that |\makeatletter| is still in effect so |@| has catcode 11.
%    
%    Then we switch to |\makeatother| so |@| has catcode 12. 
%    The next line expands to |\@LRT@makeMePublic| with the initial 
%    |\@LRT@| having its |@|'s of catcode 11 so it is a private macro. 
%    In the |\@LRT@| that you see, the |@|'s have catcode 12 so it is really
%    the macro |\@| with extra stuff. Since it should never be expanded 
%    it doesn't really matter. Then we drop back into |\makeatletter| to define
%    more private macros.
%    
%    The net result is that you can write \\
%    \null\hskip1in |\makeMePublic \@LRT@foo\becomes\newMacro| \\in your master-file and
%    the private macro |\@LRT@foo| becomes the same as |\newMacro| and you can
%    use |\newMacro| in your |tex|-file. 
%    \begin{macrocode}
\expandafter\gdef\csname1238LRTLRTsvbneLRT\endcsname{@LRT@}
\makeatother
\expandafter\gdef\csname\csname1238LRTLRTsvbneLRT\endcsname 
makeMePublic\endcsname\@LRT@#1\becomes#2{\global\letcs{#2}{@LRT@#1}}%
\makeatletter

%    \end{macrocode}
% \end{macro}
% \begin{macro}{private macros}
% \end{macro}
% \begin{macro}{\@LRT@Row}
%  \Hypertarget{Row}{}
%    The macro |\@LRT@Row| expands to the number of the row of the |csv|-file currently
%    being read. To get the |\xdef| to expand correctly we needed to define (and did)
%    |\@LRT@macroPrefix| first. Recall that the first line of actual data is row 2.
%    \begin{macrocode}
\xdef\@LRT@Row{\noexpand\number\noexpand\value{\@LRT@macroPrefix Row}}

%    \end{macrocode}
% \end{macro}
% \begin{macro}{\if@LRT@empty}\Hypertarget{if@LRT@empty}{}
%  Takes three arguments: if |#1| expands to empty do the code in |#2|; 
%  otherwise do the code in |#3|. 
%    \begin{macrocode}
\newcommand{\if@LRT@empty}[3]{\ifthenelse{\equal{#1}{}}{#2}{#3}}

%    \end{macrocode}
% \end{macro}
% \begin{macro}{\@LRT@fieldName}
%    Takes an alphanumeric string, uses |\caselower| from the |stringstrings| package
%    to convert the string to all lower case; then it strips the spaces from the answer
%    using |\convertchar|, also from the |stringstrings| package. Next it strips commas from
%    the answer. 
%    The final |\@LRT@TheString| makes the answer global. 
%    \begin{macrocode}
\gdef\@LRT@fieldName#1{%
\caselower[q]{#1}%
\convertchar[q]{\thestring}{ }{}%
\convertchar[q]{\thestring}{,}{}%
\xdef\@LRT@TheString{\thestring}%
}

%    \end{macrocode}
% \end{macro}
% \begin{macro}{\@LRT@getFilePaths}
%    |\@LRT@getFilePaths#1=#2@| is a recursive macro which works through the 
%    |prefix={<prefix-file>}| argument list. 
%    The first argument, |#1|, picks up one  |prefix={<prefix-file>}| and |#2| contains the rest of the arguments. 
%    If |#1| is not empty it is passed to |\@LRT@extractKeyValue| which collects the 
%    prefix and file path.
%    \begin{macrocode}
\gdef\@LRT@getFilePaths#1=#2@{%
\xdef\@LRT@hashKey{\trim@spaces{#1}}%
\if@LRT@empty{\@LRT@hashKey}{}{
\def\@LRT@test{tex}\ifthenelse{\equal{\@LRT@hashKey}{\@LRT@test}}{}%
{\def\@LRT@test{csv}\ifthenelse{\equal{\@LRT@hashKey}{\@LRT@test}}{}%
{\def\@LRT@test{tmp}\ifthenelse{\equal{\@LRT@hashKey}{\@LRT@test}}{}%
{\@LRT@unknownArgumentMessage{#1}}}}%
\@LRT@extractKeyValue#2@%
\expandafter\@LRT@getFilePaths\@LRT@doWeStillHaveStuffToDo @}}

%    \end{macrocode}
% \end{macro}
% \begin{macro}{\@LRT@extractKeyValue}
%    |\@LRT@hashKey| contains the prefix and |\@LRT@hashValue| contains the file path. 
%    If the prefix is not one of |tex|, |csv| or |tmp| you will be warned. 
%    \begin{macrocode}
\def\@LRT@extractKeyValue#1#2@{%
\edef\@LRT@hashValue{\trim@spaces{#1}}%
\xdef\@LRT@doWeStillHaveStuffToDo{\trim@spaces{#2}}%
\expandafter\xdef\csname @LRT@\@LRT@hashKey\endcsname{\@LRT@hashValue}%
}

%    \end{macrocode}
% \end{macro}
% \begin{macro}{\@LRT@SecondifFieldEmpty}
%    |\@LRT@SecondifFieldEmpty| checks for an empty macro directly. 
%    The previously defined \Hyperlink{ifFieldEmpty} uses a different method so both are needed at
%    various points in the code.
%    \begin{macrocode}
\newcommand{\@LRT@SecondifFieldEmpty}[3]{%
\edef\@LRT@fieldValue{#1}%
\ifx\@LRT@fieldValue\@empty\relax{#2}\else{#3}\fi\relax}

%    \end{macrocode}
% \end{macro}
% \begin{macro}{\@LRT@getCommaFieldDelimited}
%    The macros |\@LRT@getCommaFieldDelimited|and 
%    |\@LRT@getFieldDelimited| are the same when the field delimiter is the comma. 
%   The list of field-names is always comma delimited and so
%   we need a |\@LRT@getCommaFieldDelimited| to iterate through it.
%    \begin{macrocode}
\gdef\@LRT@getCommaFieldDelimited#1,{\@LRT@getFirstArg{#1}}

%    \end{macrocode}
% \end{macro}
% \begin{macro}{New counter} The counter containing the row number. 
%    \begin{macrocode}
\newcounter{\@LRT@macroPrefix Row}

%    \end{macrocode}
% \end{macro}
% \begin{macro}{New read/write's}
%    \begin{macrocode}
\newread\@lrt@csvFile%
\newread\@lrt@texFile%
\newwrite\@lrt@tmpFile%

%    \end{macrocode}
% \end{macro}
% \begin{macro}{Messages}
%    \begin{macrocode}
\gdef\@LRT@justProcessedMessage{%
\immediate\typeout{}%
\immediate\typeout{Just processed row %
\@LRT@Row\space of the data file.}%
\immediate\typeout{}%
\stepcounter{\@LRT@macroPrefix Row}\relax%
}

\gdef\@LRT@failedToOpenMessage#1{%
\immediate\typeout{}%
\immediate\typeout{File #1 failed to open.}%
\immediate\typeout{}\fi%
\end{document}%
}

\gdef\@LRT@NoMoreDataMessage{%
\immediate\typeout{No more data.}%
\immediate\typeout{}%
}

\gdef\@LRT@missingArgumentMessage#1{
\immediate\typeout{}%
\immediate\typeout{I was never given a path to the #1 file.}%
\immediate\typeout{}%
\end{document}%
}

\gdef\@LRT@unknownArgumentMessage#1{
\immediate\typeout{}%
\immediate\typeout{I do not recognize an argument of type #1.}%
\immediate\typeout{}%
}

%    \end{macrocode}
% \end{macro}
% \begin{macro}{Inputs}
%  When |mergeFields| starts up, it knows immediately whether it is in streaming or in storage
%  mode. It sets |\@LRT@Input| to |\@LRT@streaming| or  |\@LRT@storage|. 
%  Each time a new line of data in the |csv|-file is read the 
%  the macros are updated to reflect that data on the current line.  
%  |\@LRT@streaming| uses the \LaTeX\ command |\input| to add the |tex|-file to the 
%  stream. 
%  |\@LRT@storage| works through the data in the |tex|-file. 
%  First it opens the |tex| file; then it 
%  reads the |tex|-file one line at a time; expands that line; and 
%  writes the answer to a the |tmp|-file. When the |eof| for the |tex|-file is reached, 
%  the |tex|-file is closed. 
%    
%    |\@LRT@catcodeMagic| is used in two places to set the catcodes of certain characters 
%    to values that won't cause trouble. 
%    
%    |\@LRT@texNotAtEOF{#1}| is used to tighten up the code looping over the lines in
%    the |tex|-file in |\@LRT@storage|.  
%    \begin{macrocode}
\gdef\@LRT@streaming{\input{\@LRT@tex}}

\gdef\@LRT@catcodeMagic{%
\catcode`\^^M9\relax%
\catcode`\#12\relax
\catcode`\$12\relax
\catcode`\&12\relax
\catcode`\^12\relax
\catcode`\_12\relax
\catcode`\%12\relax
}

\gdef\@LRT@texNotAtEOF#1{%
\ifeof\@lrt@texFile{\global\booltrue{@LRT@texEOF}}\else{#1}\fi%
}

\gdef\@LRT@storage{{%
\@LRT@catcodeMagic%
\openin\@lrt@texFile=\@LRT@tex\relax%
\@LRT@texNotAtEOF{\global\boolfalse{@LRT@texEOF}}%
\unlessboolexpr{bool{@LRT@texEOF}}{%
\read\@lrt@texFile to \@LRT@preItem%
\@LRT@texNotAtEOF{%
\immediate\write\@lrt@tmpFile{\@LRT@preItem}\relax}%
}{}%
\immediate\closein\@lrt@texFile%
}}

%    \end{macrocode}
% \end{macro}
% \begin{macro}{\@LRT@dataLineIterator}
%    This macro looks to see if the next character is |\@LRT@textDelimiterFront|. 
%    If it is the entry is collected using |\@LRT@getTextDelimited| and if not we collect it 
%    using |\@LRT@getFieldDelimited|.
%    \begin{macrocode}
\gdef\@LRT@dataLineIterator{%
\expandafter\@ifnextchar\@LRT@textDelimiterFront%
{\@LRT@getTextDelimited}{\@LRT@getFieldDelimited}}

%    \end{macrocode}
% \end{macro}
% \begin{macro}{\@LRT@getFirstArg}
%    This is the |\@LRT@getFirstArg| from the |\@LRT@getFieldDelimited| and \\
%    |\@LRT@getTextDelimited| macros. It collects the first field and then 
%    uses |\@LRT@getSecondArg| to collect the rest of the line from the |csv|-file.
%    \begin{macrocode}
\gdef\@LRT@getFirstArg#1{\gdef\@LRT@firstArg{#1}\@LRT@getSecondArg}
\gdef\@LRT@getSecondArg#1@{\gdef\@LRT@restArg{#1}}

%    \end{macrocode}
% \end{macro}
% \begin{macro}{\@LRT@getLOFirstArg}
%  These next two macros do the same thing with the first line of the |csv|-file. 
%  The difference is that these two macros put their answers in a different pair of
%  macros than the previous two so we can iterate along both the first line (field-names)
%  and the current line of the |csv|-file. 
%    \begin{macrocode}
\gdef\@LRT@getLOFirstArg#1{\gdef\@LRT@ffirstArg{#1}\@LRT@getLOSecondArg}
\gdef\@LRT@getLOSecondArg#1@{\gdef\@LRT@frestArg{#1}}

%    \end{macrocode}
% \end{macro}
% \begin{macro}{\@LRT@getLineOne}
%  This is the version of |\@LRT@getFirstArg| which is used on the first line of the |csv|-file.
%    \begin{macrocode}
\gdef\@LRT@fieldNameIterator#1,{\@LRT@getLOFirstArg{#1}}

%    \end{macrocode}
% \end{macro}
% \begin{macro}{\@LRT@setFirstLineAndBlankLine}
%  This macro collects the first line of the |csv|-file and packages it for later use:
%  |\@lrt@firstline| is the first line when we are done. 
%  We also construct |\@LRT@csvBlankLine| which is used to check if
%  a line of the |csv|-file is blank. 
%    \begin{macrocode}
\providebool{@LRT@iterator}
\gdef\@LRT@setFirstLineAndBlankLine#1{%
\gdef\@lrt@firstline{}%
\gdef\@LRT@csvBlankLine{}
\edef\@LRT@restArg{#1\@LRT@fieldDelimiterBack}
\booltrue{@LRT@iterator}%
\whileboolexpr{bool{@LRT@iterator}}{%
\expandafter\@LRT@dataLineIterator\@LRT@restArg @%
\@LRT@fieldName{\@LRT@firstArg}%
\xdef\@lrt@firstline{\@lrt@firstline\@LRT@TheString,}%
\if@LRT@empty{\@LRT@restArg}%
{\boolfalse{@LRT@iterator}}{\booltrue{@LRT@iterator}}%
\ifboolexpr{bool{@LRT@iterator}}%
{\xdef\@LRT@csvBlankLine{\@LRT@csvBlankLine\@LRT@blankFieldItem}}{}%
}{}%
}

%    \end{macrocode}
% \end{macro}
% \begin{macro}{\@LRT@getAndProcessNextLine}
%  This is the recursive macro which processes each line of the |csv|-file 
%  until there are no more. 
%  The code in the  first |{{| \dots |}}| pair 
%  reads a line from the |csv|-file: the last line here converts 
%  the answer from local to global. 
%  The code in the second |{{| \dots |}}| pair checks 
%  if the line is \hyperlink{End of csv-fileDescription}{blank} or not. 
%  |\@LRT@exitGetAndProcessNextLine| exits the recursion and 
%  |\@LRT@repeatGetAndProcessNextLine| reloops. 
%    \begin{macrocode}
\gdef\@LRT@getAndProcessNextLine{%
\unlessboolexpr{bool{@LRT@csvEOF}}{%
{{%
\@LRT@catcodeMagic%
\@LRT@catcodes
\read\@lrt@csvFile to\@LRT@x@nextLine%
\ifeof\@lrt@csvFile\relax{\global\booltrue{@LRT@csvEOF}}\else%
{\global\boolfalse{@LRT@csvEOF}}\fi%
\xdef\@LRT@nextLine{\@LRT@x@nextLine}%
}}%
{{%
\if@LRT@empty{\@LRT@nextLine}%
{\global\let\@LRT@next\@LRT@exitGetAndProcessNextLine\relax}%
{\ifx\@LRT@csvBlankLine\@LRT@nextLine\relax%
{\global\let\@LRT@next\@LRT@exitGetAndProcessNextLine\relax}\else%
{\global\let\@LRT@next\@LRT@repeatGetAndProcessNextLine\relax}\fi}%
\@LRT@next%
}}%
}{}%
}

%    \end{macrocode}
% \end{macro}
% \begin{macro}{\@LRT@exitGetAndProcessNextLine}
%  This just exits.
%    \begin{macrocode}
\gdef\@LRT@exitGetAndProcessNextLine{%
\closein\@lrt@csvFile\relax\global\booltrue{@LRT@csvEOF}}

%    \end{macrocode}
% \end{macro}
% \begin{macro}{\@LRT@repeatGetAndProcessNextLine}
%  This sets up to process the line we just read.
%    \begin{macrocode}
\gdef\@LRT@repeatGetAndProcessNextLine{%
\@LRT@processNextLine{\@LRT@nextLine\@LRT@fieldDelimiterBack}}

%    \end{macrocode}
% \end{macro}
% \begin{macro}{\@LRT@setMacrosForOneEntry}
%  This bit of code sets the macro and the if-empty boolean for a single item:
%  \#1 is the field-name and \#2 is the entry in that (row,column).
%    \begin{macrocode}
\gdef\@LRT@setMacrosForOneEntry#1#2{%
\expandafter\xdef\csname\@LRT@macroPrefix#1\endcsname{#2}%
\@LRT@SecondifFieldEmpty{#2}%
{\global\booltrue{\@LRT@macroPrefix#1}}%
{\global\boolfalse{\@LRT@macroPrefix#1}}%
}

%    \end{macrocode}
% \end{macro}
% \begin{macro}{\@LRT@processNextLine}
%    This code actually processes the line from the |csv|-file. 
%    The first three lines are setup. 
%    The two |\edef|'s do the following: |\@LRT@restArg{#1}| has the entire current row; and
%    |\@LRT@frestArg| has a comma separated list of field-names. 
%    Setting |\booltrue{@LRT@iterator}| means the iteration will start.
%    The two |\expandafter| lines do the iteration: the current values are in |\ffirstArg| for
%    the field-name and |\firstArg| for the entry in the current row. 
%   The line |\@LRT@setMacrosForOneEntry| sets the contents of the  field-name-macro. 
%   Finally the 
%   |\if@LRT@empty| line checks if we are done. 
%    
%    The |\edef| between the two |\expandafter|'s adds a blank entry to the back of 
%    |\@LRT@restArg| so if the row is too short in the |csv|-file it will still scan correctly.
%
%  The |\@LRT@Input| outputs the result either to the stream or to the |tmp|-file
%  and |\@LRT@justProcessedMessage| writes to the terminal which row was just processed. 
%    If something suddenly goes wrong you know which row you were working on 
%    when things messed up.
%    \begin{macrocode}
\gdef\@LRT@processNextLine#1{%
\edef\@LRT@restArg{#1}%
\edef\@LRT@frestArg{\@lrt@firstline}%
\booltrue{@LRT@iterator}%
\whileboolexpr{bool{@LRT@iterator}}{%
\expandafter\@LRT@fieldNameIterator\@LRT@frestArg @%
\edef\@LRT@restArgA{\@LRT@restArg\@LRT@fieldDelimiterBack}
\expandafter\@LRT@dataLineIterator\@LRT@restArgA @%
\@LRT@setMacrosForOneEntry{\@LRT@ffirstArg}{\@LRT@firstArg}%
\if@LRT@empty{\@LRT@frestArg}%
{\boolfalse{@LRT@iterator}}{\booltrue{@LRT@iterator}}%
}{}%
\@LRT@Input\relax%
\@LRT@justProcessedMessage%
}

%    \end{macrocode}
% \end{macro}
% \begin{macro}{\@LRT@mergeData}
%    This macro opens the |csv|-file, reads the first row.  
%    The code between the |{{ ... }}| makes a comma separated list of field-names
%    in the order in which they occur in the |csv|-file. It also creates a macro which 
%    expands to what a blank line in the |csv|-file will look like. It also defines one 
%    boolean for each field-name. 
%    This requires an iteration over |\@lrtfirstline|.
%    
%    Because \TeX\ |\newif|'s are not global, we need to define the ones we need as soon
%    as possible, which is right here.
%
%    Then |\@LRT@mergeData| calls |\@LRT@getAndProcessNextLine| which  
%    reads and processes the subsequent lines of the |csv| file. 
%
%    Finally it puts out a no-more-data message to let you know |\mergeFields| is finished.
%    \begin{macrocode}
\gdef\@LRT@mergeData{%
\providebool{@LRT@texEOF}%
\immediate\openin\@lrt@csvFile=\@LRT@csv\relax%
\providebool{@LRT@csvEOF}%
\boolfalse{@LRT@csvEOF}%
\ifeof\@lrt@csvFile\@LRT@failedToOpenMessage{\@LRT@csv}\fi%
{{\@LRT@catcodes
\read\@lrt@csvFile to\@lrt@firstlineX%
\@LRT@setFirstLineAndBlankLine{\@lrt@firstlineX}
}}%
\booltrue{@LRT@iterator}
\edef\@LRT@restArg{\@lrt@firstline}%
\whileboolexpr{bool{@LRT@iterator}}{%
\expandafter\@LRT@getCommaFieldDelimited\@LRT@restArg @%
\providebool{\@LRT@macroPrefix\@LRT@firstArg}%
\if@LRT@empty{\@LRT@restArg}%
{\boolfalse{@LRT@iterator}}{\booltrue{@LRT@iterator}}%
}{}%
\@LRT@getAndProcessNextLine%
\@LRT@NoMoreDataMessage%
}

%    \end{macrocode}
% \end{macro}
%  \Hypertarget{options}{}
% \begin{macro}{Options code}
%   Here is the code which 
%    examines the option string and maps the public macro names to the
%    private codes which actually do the work. 
%    \begin{macrocode}
\gdef\@LRT@weirdOption#1{\@ifundefined{@LRT@CO}%
{\global\csletcs{#1}{@LRT@#1}}{{\global\csletcs{\@LRT@CO #1}{@LRT@#1}}}%
}

\@LRT@weirdOption{mergeFields}%
\@LRT@weirdOption{Field}%
\@LRT@weirdOption{ifFieldEmpty}%
\@LRT@weirdOption{setDelimitersCommaQuote}%
\@LRT@weirdOption{setDelimitersTabQuote}%
\@LRT@weirdOption{makeMePublic}%

%    \end{macrocode}
% \end{macro}
% \begin{macro}{Set the default delimiters}
%   Here  \Hypertarget{defaults}{is} 
%   where the default delimiters are set. 
%    \begin{macrocode}
\@LRT@setDelimitersCommaQuote

%    \end{macrocode}
% \end{macro}
% \begin{macro}{Finish up}
%    \begin{macrocode}
\makeatother
\endinput
%    \end{macrocode}
% \end{macro}
% \Finale
%  \newpage
% \appendix
% \section{Writing new delimiter macros}\hypertarget{newDelimiterAppendix}{}
%    If the field delimiter is not something \TeX\ sees as white space, life is fairly easy.
%    In your master-file, copy and paste one of the code examples below. 
%    Change to suit your needs.
%
%    \noindent***********************************************************************
%
%    \begin{verbatim}
%    {{%
%    \makeatletter
%    \gdef\setMyDelimiters{%
%    \gdef\@LRT@fieldDelimiterBack{,}
%    \gdef\@LRT@getFieldDelimited##1,{\@LRT@getFirstArg{##1}}
%    \gdef\@LRT@textDelimiterFront{"}
%    \gdef\@LRT@textDelimiterBack{",}
%    \gdef\@LRT@getTextDelimited"##1",{\@LRT@getFirstArg{##1}}
%    \gdef\@LRT@blankFieldItem{,}
%    \gdef\@LRT@catcodes{}
%    }%
%    \makeatother
%    }}
%    \end{verbatim}
%
%    \noindent***********************************************************************
%
%    Replace all occurrences of {\tt ,} by your field delimiter. Replace all occurrences of {\tt "} 
%    with your text deliminator
%    and you should be ready to go. 
%    Just add |\setMyDelimiters| in your master-file before using |\mergeFields|.
%    
%    If the field delimiter is a white space character such as a tab, then copy and paste
%    the next example
%
%    \noindent***********************************************************************
%    
%    \begin{verbatim}
%    \makeatletter
%    {{\makeatletter
%    \catcode`\^^I12\relax%
%    \gdef\@LRT@setDelimitersTabQuote{
%    \gdef\@LRT@fieldDelimiterBack{^^I}
%    \gdef\@LRT@getFieldDelimited##1^^I{\@LRT@getFirstArg{##1}}
%    \gdef\@LRT@textDelimiterFront{"}
%    \gdef\@LRT@textDelimiterBack{"^^I}
%    \gdef\@LRT@getTextDelimited"##1"^^I{\@LRT@getFirstArg{##1}}
%    \gdef\@LRT@blankFieldItem{}
%    \gdef\@LRT@catcodes{\catcode`\^^I12\relax}
%    }%
%    }}
%    \makeatother
%    }}
%    \end{verbatim}
%
%    \noindent***********************************************************************
%
%    Use Chapter 8 of the \TeX\ book to find how to write your delimiter and replace 
%    all occurrences of |\^^I| with the code for your delimiter. 
%    Then replace all occurrences of {\tt "}  with your text deliminator 
%     and hope for the best. 
%    Try leaving |\@LRT@blankFieldItem| blank and hope for the best. 
%    Again add |\setMyDelimiters| in your master-file before using |\mergeFields|.
%
% \section{Some examples}
% \subsection{\TeX\ code in a table}
%  A typical line in a tabular environment looks like |item1 & item2& item3\\|.
%    Suppose you have 4 columns in your |csv|-file with entries |Last Name|, 
%    |First name|, |Numerator| and |Denominator| and you want |item1| to be 
%    |Last Name|, |item2| to be |First name| and |item3| to be |\frac{Numerator}{Denominator}|.
%    Here is the needed code in your |tex|-file:
%
%    \begin{verbatim}
%    \Field{lastname}&\Field{firstname}&%
%    $\noexpand\frac{\Field{numerator}}{\Field{denominator}}$\noexpand\\
%    \end{verbatim}
%    You might have expected to see a |\noexpand| in front of the \$'s but remember \$ is one of 
%    those characters which goes through unscathed to the |tmp|-file. 
%    It certainly doesn't hurt to write |\noexpand$|. When you read the |tmp|-file back in
%    you do want |$| to resume its meaning of making math mode and it does. 
%    
%    The |\noexpand| in front of the |\frac| and the |\\| are needed. 
%    You don't want to try to expand either one
%    until the |tmp|-file is read back in. 
%    
%    Assuming that the file |merge.tex| contains the code above and that the file |data.csv| contains your data,
%    the following code typesets the table.
%    \begin{verbatim}
%    \documentclass[12pt]{article}
%    \RequirePackage{csvmerge}
%    \begin{document} 
%    \mergeFields{
%       tmp={example.tmp}
%       tex  ={merge.tex}
%       csv =  {data.csv}
%       }
%    \begin{tabular}{lll}
%    %%
%% This is file `example.tex',
%% generated with the docstrip utility.
%%
%% The original source files were:
%%
%% semproc.dtx  (with options: `exampletex')
%% ----------------------------------------------------------------
%% semproc --- LaTeX class for seminar proceedings
%% E-mail: tobias.columbus@kit.edu
%% Released under the LaTeX Project Public License v1.3c or later
%% See http://www.latex-project.org/lppl.txt
%% ----------------------------------------------------------------
%% 
\documentclass[a4paper,openright,parskip=half-]{semproc}
\addglobalbib{example.bib}
\usepackage[ngerman,english]{babel}
\begin{document}
\author{Max Mustermann}
\title{Example}
\maketitle
\globaltoc
\chapter{Introduction}
This document serves as an example for the \verb|semproc| class.

The reader should note that this introduction, which is typeset as a
\verb|\chapter| shows up in the table of contents. However, the numbering scheme
of talks and chapters is confusing at best. \verb|semproc| therefore offers a
macro \verb|\nonumchapter| for unnumbered chapters that show up in the table of
contents and may be used e.\,g.\ for introductions as in the following second
introduction.

\nonumchapter{Second Introduction}
This is a second introduction with no number but still showing up in the table
of contents.

\talk[speaker={Donald Knuth}]{Notation}
\begin{summary}
  This could be a manuscript for a talk entitled ``Notation'' given by
  Donald Knuth at Stanford university. A video recording is available at
  \cite{video}.
\end{summary}
\localtoc
\section{Introduction}
An introduction could go here.
\section{Examples of Notation}
\subsection{Notation for integrals}
One could compare Leibniz' and Newton's notation for integrals as Knuth does in
his talk.
\subsection{Frege's Grundgesetze}
Even if Knuth does not talk about the notation in \citeauthor{frege}'s
\cite{frege}, there is even a paper~\cite{onfrege} on it.
\section{Outlook}
A final section with final words.
\localbib

\talk[speaker={Somebody},notes={Somebody else}]{On Something}
\begin{summary}
  Some Summary
\end{summary}
\localtoc
\section{Section A}
\section{Section B}

\selectlanguage{ngerman}
\talk[speaker={Hans}]{\"Uber etwas Wichtiges}
\begin{summary}
  Hier sieht man, dass \verb|semproc| an andere Sprachen, wie zum Beispiel
  \verb|ngerman| angepasst werden kann. Momentan werden aber nur \verb|english|
  und \verb|ngerman| unterstützt.
\end{summary}
\end{document}
%% 
%% Copyright (C) 2015 by Tobias Columbus <tobias.columbus@kit.edu>
%% 
%% This work may be distributed and/or modified under the
%% conditions of the LaTeX Project Public License (LPPL), either
%% version 1.3c of this license or (at your option) any later
%% version.  The latest version of this license is in the file:
%% 
%% http://www.latex-project.org/lppl.txt
%% 
%% This work is "maintained" (as per LPPL maintenance status) by
%% Tobias Columbus.
%% 
%% This work consists of the files semproc.dtx
%% and the derived files           semproc.ins,
%%                                 semproc.pdf,
%%                                 semproc.cls,
%%                                 example.tex,
%%                                 example.bib,
%%                                 example.pdf
%% 
%%
%% End of file `example.tex'.

%    \end{tabular}
%    \end{document}
%    \end{verbatim}
%    The |csv|-file needs to contain the four field-names that appear in the |tex|-file, 
%    but it is fine if it contains additional columns.
%
%    \subsection{\TeX\ code in a mail-merge}
%    Here is a snippet of code to do a header in a mail-merge business letter.
%    The field-names from the |csv|-file should be obvious. The interesting point
%    is the usage of |\ifFieldEmpty|.  
%    \begin{verbatim}
%    \Field{firstname} \ifFieldEmpty{middleinitial}{}
%    {\Field{middleinitial}.~}\Field{lastname}\par
%    \ifFieldEmpty{streetline1}{}{\Field{streetline1}\par}
%    \ifFieldEmpty{streetline2}{}{\Field{streetline2}\par}
%    \ifFieldEmpty{streetline3}{}{\Field{streetline3}\par}
%    \Field{city}, \Field{stateprovince} \Field{postalcode}
%    \end{verbatim}
%    
%    The |middleinitial| column is a middle initial with no period. Some recipients will have no middle initial 
%    but you don't want just a floating period. Also you don't want an additional space. 
%    Knuth also suggests that names should be \TeX ed with |First M.~Last| [The \TeX\ book, Ch. 6]. 
%    The |\ifFieldEmpty| with |streetline1| is probably redundant but often |streetline2| will be empty.
%    If it is not, we want a new line after it, but if it is empty we don't want a blank new line. 
%    
%    In storage mode you need a |\noexpand| before the |~|, as well as before the |\par|'s.
%    \subsection{Some master-files}
%    \subsubsection{A table}
%    We are doing a table so we want storage mode. 
%    We also want a horizontal line after each row of the table. 
%    The master-file:
%    \begin{verbatim}
%    \documentclass[12pt]{amsart}
%    \RequirePackage{csvmerge} 
%    \begin{document} 
%    \mergeFields{
%    tmp= {	./table.tmp}
%    csv =  {./data.csv}
%    tex  ={./table.tex  }
%    }
%    \begin{tabular}{|l|l|l|l|l|}
%    \hline%
%    \input{./table.tmp}
%    \end{tabular}
%    \end{document}
%    \end{verbatim}
%    
%    The |tex|-file |table.tmp|:
%    \begin{verbatim}
%    % one table line
%    \Field{firstname} &\Field{id} & \Field{city}  &\Field{lastname}&   
%    \Field{postalcode}  \noexpand\\
%    \noexpand\hline
%    \end{verbatim}
%
%    \subsubsection{A mail-merge}
%    We are doing a mail-merge so we can use streaming mode. 
%    Since the |tex|-file name has spaces, we have quoted it.
%    The master-file:
%    \begin{verbatim}
%    \documentclass[12pt]{letter}
%    \RequirePackage{./csvmerge} 
%    \begin{document}
%    \mergeFields{ tex  = {"./Second letter.template"  } 
%    csv ={  ./Invites.csv}}
%    \end{document}
%    \end{verbatim}
%    
%    The |tex|-file |Second letter.template|:
% 
%    \begin{verbatim}
%    %First page of letter
%    \setcounter{page}{1}
%     
%    \Field{firstname} \ifFieldEmpty{middleinitial}{}%
%    {\Field{middleinitial}.~}\Field{lastname}\par
%    \ifFieldEmpty{streetline1}{}{\Field{streetline1} \par}
%    \ifFieldEmpty{streetline2}{}{\Field{streetline2}\par}
%    \ifFieldEmpty{streetline3}{}{\Field{streetline3}\par}
%    \Field{city}, \Field{stateprovince} \Field{postalcode}
%    
%    \vskip.25in
%    Dear \Field{preferredfirstname},\par
%    \vskip10pt
%    How come no RSVP?
%    \vskip10pt
%    Bilbo Baggins
%   \newpage
%    \end{verbatim}
%
%    Here is an equivalent version using storage mode.
%    The master-file:
%    \begin{verbatim}
%    \documentclass[12pt]{letter}
%    \RequirePackage{./csvmerge} 
%    \begin{document}
%    \mergeFields{ tex  = {"./Second letter.template"  } 
%    csv ={  ./Invites.csv}tmp={letters}}
%    \input{letters}
%    \end{document}
%    \end{verbatim}
%    
%    We must adjust the |tex|-file |Second letter.template|:
% 
%    \begin{verbatim}
%    %First page of letter
%    \noexpand\setcounter{page}{1}
%     
%    \Field{firstname} \ifFieldEmpty{middleinitial}{}%
%    {\Field{middleinitial}.\noexpand~}\Field{lastname}\noexpand\par
%    \ifFieldEmpty{streetline1}{}{\Field{streetline1} \noexpand\par}
%    \ifFieldEmpty{streetline2}{}{\Field{streetline2}\noexpand\par}
%    \ifFieldEmpty{streetline3}{}{\Field{streetline3}\noexpand\par}
%    \Field{city}, \Field{stateprovince} \Field{postalcode}
%    
%    \noexpand\vskip.25in
%    Dear \Field{preferredfirstname},\par
%    \noexpand\vskip10pt
%    How come no RSVP?
%    \noexpand\vskip10pt
%    Bilbo Baggins
%   \noexpand\newpage
%    \end{verbatim}
%
%    Usually using storage mode when you don't need it is wasteful since you have to create and
%    save a |tmp|-file. It can come in handy if your letter code has lots of field macros and 
%    you can't get one |tex|-file to not have overfull/underfull boxes and the like after the 
%    various expansions. The |tmp|-file contains all the letters and you can fine tune each one and
%    then just do 
%    \begin{verbatim}
%    \documentclass[12pt]{letter}
%    \begin{document}
%    \input{letters}
%    \end{document}
%    \end{verbatim}







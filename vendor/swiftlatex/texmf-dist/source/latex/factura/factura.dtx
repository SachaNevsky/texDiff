%\iffalse meta-comment
%
% factura.dtx
% Copyright 2019 Emilio Augusto Lazo Zaia <emiliolazozaia@gmail.com>
%
% This work may be distributed and/or modified under the
% conditions of the LaTeX Project Public License, either
% version 1.3c of this license or (at your option) any later
% version. This version of this license is in
%   http://www.latex-project.org/lppl/lppl-1-3c.{html|txt|tex|pdf}
% and the latest version of this license is in
%   http://www.latex-project.org/lppl.txt
% and version 1.3c or later is part of all distributions of
% LaTeX version 2008/05/04 or later.
%
% This work has the LPPL maintenance status `maintained'.
%
% The Current Maintainer of this work is Emilio Augusto Lazo Zaia.
%
% This work consists of the files README and factura.dtx
% with its derived files factura.ins, factura.cls, factura.def,
% factura-ejemplo-aux.tex, factura-ejemplo-firma.tex,
% factura-ejemplo-fondo.tex, factura-ejemplo-membrete.tex,
% factura-ejemplo-cotizacion.tex, factura-ejemplo-prefactura.tex,
% factura-ejemplo-factura01.tex, factura-ejemplo-factura02.tex,
% factura-ejemplo-factura03.tex, factura-ejemplo-factura04.tex,
% factura-ejemplo-factura05.tex, factura-ejemplo-factura06.tex,
% factura-ejemplo-factura07.tex, factura-ejemplo-factura08.tex,
% factura-ejemplo-factura09.tex, factura-ejemplo-factura10.tex,
% factura-ejemplo-reporte1.tex, factura-ejemplo-reporte2.tex,
% factura-ejemplo-reporte3.tex and factura-beispiel-rechnung.tex.
%<*internalbatchfile>
\def\nameofplainTeX{plain}
\ifx\fmtname\nameofplainTeX
\else
 \expandafter\begingroup
\fi
%</internalbatchfile>
%<*batchfile>
\input docstrip.tex
\preamble

Copyright 2019 Emilio Augusto Lazo Zaia <emiliolazozaia@gmail.com>

This work may be distributed and/or modified under the
conditions of the LaTeX Project Public License, either
version 1.3c of this license or (at your option) any later
version. This version of this license is in
 http://www.latex-project.org/lppl/lppl-1-3c.{html|txt|tex|pdf}
and the latest version of this license is in
 http://www.latex-project.org/lppl.txt
and version 1.3c or later is part of all distributions of
LaTeX version 2008/05/04 or later.

This work has the LPPL maintenance status `maintained'.

The Current Maintainer of this work is Emilio Augusto Lazo Zaia.

This work consists of the files README and factura.dtx
with its derived files factura.ins, factura.cls, factura.def,
factura-ejemplo-aux.tex, factura-ejemplo-firma.tex,
factura-ejemplo-fondo.tex, factura-ejemplo-membrete.tex,
factura-ejemplo-cotizacion.tex, factura-ejemplo-prefactura.tex,
factura-ejemplo-factura01.tex, factura-ejemplo-factura02.tex,
factura-ejemplo-factura03.tex, factura-ejemplo-factura04.tex,
factura-ejemplo-factura05.tex, factura-ejemplo-factura06.tex,
factura-ejemplo-factura07.tex, factura-ejemplo-factura08.tex,
factura-ejemplo-factura09.tex, factura-ejemplo-factura10.tex,
factura-ejemplo-reporte1.tex, factura-ejemplo-reporte2.tex,
factura-ejemplo-reporte3.tex and factura-beispiel-rechnung.tex.

\endpreamble
\keepsilent
\askforoverwritefalse
\obeyspaces
\Msg{*****************************************************************}
\Msg{**************************** factura ****************************}
\Msg{*****************************************************************}
\generate{\file{\jobname.cls}{\from{\jobname.dtx}{class}}}
\generate{\file{\jobname.def}{\from{\jobname.dtx}{def}}}
\Msg{\jobname.cls and \jobname.def written.}
\Msg{}
\postamble
\endpostamble
%</batchfile>
%<*internalbatchfile>
\generate{\file{\jobname.ins}{\from{\jobname.dtx}{batchfile}}}
\Msg{\jobname.ins written.}
%</internalbatchfile>
%<*batchfile>
\def\factuno{factura-ejemplo-factura01.tex}
\def\factdos{factura-ejemplo-factura02.tex}
\def\facttres{factura-ejemplo-factura03.tex}
\def\factcuatro{factura-ejemplo-factura04.tex}
\def\factcinco{factura-ejemplo-factura05.tex}
\def\factseis{factura-ejemplo-factura06.tex}
\def\factsiete{factura-ejemplo-factura07.tex}
\def\factocho{factura-ejemplo-factura08.tex}
\def\factnueve{factura-ejemplo-factura09.tex}
\def\factdiez{factura-ejemplo-factura10.tex}
\def\repuno{factura-ejemplo-reporte1.tex}
\def\repdos{factura-ejemplo-reporte2.tex}
\def\reptres{factura-ejemplo-reporte3.tex}
\def\cotiza{factura-ejemplo-cotizacion.tex}
\def\prefac{factura-ejemplo-prefactura.tex}
\def\beispiel{factura-beispiel-rechnung.tex}
\def\auxejemplo{factura-ejemplo-aux.tex}
\def\firma{factura-ejemplo-firma.tex}
\def\fondo{factura-ejemplo-fondo.tex}
\def\membrete{factura-ejemplo-membrete.tex}
\generate{\file{\factuno}{\from{\jobname.dtx}{factuno}}}
\generate{\file{\factdos}{\from{\jobname.dtx}{factdos}}}
\generate{\file{\facttres}{\from{\jobname.dtx}{facttres}}}
\generate{\file{\factcuatro}{\from{\jobname.dtx}{factcuatro}}}
\generate{\file{\factcinco}{\from{\jobname.dtx}{factcinco}}}
\generate{\file{\factseis}{\from{\jobname.dtx}{factseis}}}
\generate{\file{\factsiete}{\from{\jobname.dtx}{factsiete}}}
\generate{\file{\factocho}{\from{\jobname.dtx}{factocho}}}
\generate{\file{\factnueve}{\from{\jobname.dtx}{factnueve}}}
\generate{\file{\factdiez}{\from{\jobname.dtx}{factdiez}}}
\generate{\file{\repuno}{\from{\jobname.dtx}{repuno}}}
\generate{\file{\repdos}{\from{\jobname.dtx}{repdos}}}
\generate{\file{\reptres}{\from{\jobname.dtx}{reptres}}}
\generate{\file{\cotiza}{\from{\jobname.dtx}{cotiza}}}
\generate{\file{\prefac}{\from{\jobname.dtx}{prefac}}}
\generate{\file{\beispiel}{\from{\jobname.dtx}{beispiel}}}
\generate{\file{\auxejemplo}{\from{\jobname.dtx}{aux}}}
\generate{\file{\firma}{\from{\jobname.dtx}{firma}}}
\generate{\file{\fondo}{\from{\jobname.dtx}{fondo}}}
\generate{\file{\membrete}{\from{\jobname.dtx}{membrete}}}
\Msg{.tex examples written.}
\Msg{}
\Msg{*****************************************************************}
\immediate\write18{makeindex -s gind.ist -o \jobname.ind \jobname.idx}
\immediate\write18{makeindex -s gglo.ist -o \jobname.gls \jobname.glo}
\Msg{}
\Msg{Change history and index updated.}
\Msg{*****************************************************************}
\Msg{To produce only package files, type}
\Msg{  pdftex \jobname.dtx}
\Msg{To produce package files and pdf documentation, type}
\Msg{  xelatex \jobname.dtx}
\Msg{  pdflatex factura-ejemplo-fondo.tex}
\Msg{  pdflatex factura-ejemplo-membrete.tex}
\Msg{  xelatex factura-ejemplo-firma.tex}
\Msg{}
\Msg{Also compile all other examples and recompile \jobname.dtx.}
\Msg{(factura-ejemplo-aux.tex is not an example!)}
\Msg{}
\Msg{To manual update index and changelog, type}
\Msg{  makeindex -s gind.ist -o \jobname.ind \jobname.idx}
\Msg{  makeindex -s gglo.ist -o \jobname.gls \jobname.glo}
\Msg{}
\Msg{More information (in spanish) about compiling on README and documentation.}
\Msg{}
\Msg{*****************************************************************}
\Msg{**************************** factura ****************************}
\Msg{*****************************************************************}
\Msg{}
%</batchfile>
%<batchfile>\endbatchfile
%<*internalbatchfile>
\ifx\fmtname\nameofplainTeX
 \expandafter\endbatchfile
\else
 \expandafter\endgroup
\fi
%</internalbatchfile>
%<*driver>
\ProvidesFile{factura.dtx}
%</driver>
%<class>\NeedsTeXFormat{LaTeX2e}[1995/12/01]
%<class>\ProvidesClass{factura}
%<def>\ProvidesFile{factura.def}
%<*class|def>
[2019/08/16 v3.70
%<class>Clase para elaborar facturas legales
%<class>en la Repu'blica Bolivariana de Venezuela.
%<def>Archivo de valores iniciales y condiciones de borde para factura.cls
]

%</class|def>
%<*driver>
\documentclass{ltxdoc}
\usepackage[a4paper,headheight=0.6cm,headsep=0.77cm,footskip=1.2cm,footnotesep=0.54cm,marginparsep=10pt,vmargin={3cm,2cm},hmargin={4cm,2cm}]{geometry}
\usepackage[dvipsnames]{xcolor}
\usepackage[spanish]{babel}
\usepackage{
 array,
 calc,
 catchfilebetweentags,
 etoolbox,
 footnote,
 graphicx,
 ifluatex,
 ifxetex,
 metalogo,
 paralist,
 pgffor,
 regexpatch,
 setspace,
 tikz,
 todo,
 verbatim,
 xparse,
 xspace,
 xstring,
 hypdoc}

% Necesario ejecutar \newif de esta forma:
\csname newif\expandafter\endcsname\csname ifmotorunicode\endcsname
\csname newif\expandafter\endcsname\csname ifejemplos\endcsname

\ejemplostrue

\ifxetex
 \motorunicodetrue
\else
 \ifluatex \motorunicodetrue \fi
\fi

\ifmotorunicode
 \usepackage{fontspec}
 \setmainfont[Ligatures=TeX,Mapping=TeX]{Linux Libertine O}
\else
 \usepackage[T1]{fontenc}
 \usepackage[utf8]{inputenc}
\fi

\overfullrule15pt

\newcommand{\VERBATIM}{\verbatimcolor\setstretch{1.0}}

\xspaceaddexceptions{*}

\hypersetup{
 plainpages=false,
 colorlinks=true,
 linkcolor=blue,
 unicode=true}

\makesavenoteenv{tabular}
\makesavenoteenv{table}

\newcommand\flechaCR{%
 \raisebox{-0.4ex}{%
  \begin{tikzpicture}
  \draw[line width=0.075ex] (0,0) arc (20:70:0.75ex) + (0,0) arc (-70:-20:0.75ex) ++ (0.8ex,0.55ex) arc (0:-100:1ex);
 \end{tikzpicture}}}

\EnableCrossrefs
\CodelineIndex
\OnlyDescription
\RecordChanges

\setlength\parskip{7pt}
\setlength\IndexMin{100pt}

\widowpenalty10000

\newbool{Ejemplo}
\newbool{EjemploIzq}
\newbool{EjemploDer}
\newbool{DescripEjemplo}

\def\ejemplooptcolor{\color{blue}}
\def\ejemplomacrocolor{\color{purple}}
\def\verbatimcolor{\color{OrangeRed}}
\def\macromandcolor{\color{Red}}
\def\macrooptcolor{\color{RoyalBlue}}

\def\tikz{T\emph{i}kZ\xspace}

\catcode`\%=11
\catcode`\|=14
\NewDocumentCommand{\Ejemplo}{s O{0.5} m s O{} G{1} o}{|
 \IfFileExists{factura-#3.pdf}{|
  \catcode`\<=12
  \catcode`\>=12
  \IfBooleanTF{#4}
   {\IfBooleanTF{#1}
    {\booltrue{DescripEjemplo}}
    {\booltrue{Ejemplo}\booltrue{EjemploDer}}}
   {\IfBooleanTF{#1}
    {\booltrue{Ejemplo}\booltrue{EjemploIzq}}
    {\booltrue{Ejemplo}\booltrue{EjemploIzq}\booltrue{DescripEjemplo}}}
  \ifbool{Ejemplo}
   {\ifnum #6=1 \def\Ancho{#2}{\phantomsection\par\mbox{}\hfill{\color{RoyalBlue}\ttfamily \large factura-#3}\hfill\mbox{}\par}{}\fi}
   {\def\Ancho{0.05}}|
  \ifbool{EjemploDer}{\hfill}{}|
  \ifbool{Ejemplo}
   {\begin{minipage}{\Ancho\linewidth}
     \fbox{\includegraphics[page=#6,width=\linewidth-2\fboxrule-2\fboxsep]{factura-#3}}
     \ifbool{DescripEjemplo}{\addcontentsline{toc}{subsection}{factura-#3}}{}
    \end{minipage}}
   {}|
  \ifbool{EjemploIzq}{\hfill}{}|
  \ifbool{DescripEjemplo}
   {\hfill
   | Gracias a cfr y a egreg por el esclarecimiento en cuanto a la necesidad de \makeatletter en \CatchFileBetweenDelims y antes de \regexpatchcmd*.
   | Thanks to cfr and egreg for the clarification on the use of \makeatletter on \CatchFileBetweenDelims and before \regexpatchcmd*.
   | (https://tex.stackexchange.com/questions/380796/how-to-patch-a-macro-with-control-sequences-using-regexpatch)
   \CatchFileBetweenDelims{\DescEjemplo}{factura-#3.tex}{<desc>}{</desc>}[\catcode37=11\catcode13=14\makeatletter]|
   \catcode`\%=11
   \makeatletter
   | Gracias a Werner y a egreg por el uso de \begingroup\ttfamily\string ... \endgroup.
   | Thanks to Werner and egreg for the use of \begingroup\ttfamily\string ... \endgroup.
   | (http://tex.stackexchange.com/questions/342572/error-using-verb-as-the-replace-string-on-xpatchcmd)
   \regexpatchcmd*{\DescEjemplo}{\cC.\cO[*]*}{\c{begingroup}\c{ejemplomacrocolor}\c{ttfamily}\c{string}\0\c{endgroup}\c{xspace}}{}{}|
   \regexpatchcmd*{\DescEjemplo}{(\()(\cL[a-zG]*)(\))}{(\c{begingroup}\c{ejemplooptcolor}\c{ttfamily}\2\c{endgroup})}{}{}|
   \xpatchcmd*{\DescEjemplo}{%% }{\item}{}{}|
   \xpatchcmd*{\DescEjemplo}{}{}{}|
   \makeatother
   \begin{minipage}{\linewidth-\Ancho\linewidth}
    \begin{itemize}
     \setlength{\itemsep}{0pt}
     #5\DescEjemplo
    \end{itemize}
   \end{minipage}
   \IfValueT{#7}{{#7}}}
   {}|
  \boolfalse{Ejemplo}\boolfalse{EjemploIzq}\boolfalse{EjemploDer}\boolfalse{DescripEjemplo}|
  \catcode`\%=9
  \catcode`\<=13
  \catcode`\>=13
 }{\message{^^JLaTeX Warning: No existe factura-#4.pdf; hay que compilarlo antes de compilar este documento.^^J}}}
\catcode`\%=14
\catcode`\|=12

% No quiero que las notas pendientes tengan los números de página:
\patchcmd{\todoitem}{(p.~\pageref{todopage:\thetodo}):}{-}{}{}

\renewcommand{\todoname}{Pendiente}
\renewcommand{\todomark}{Pendiente}
\newcommand{\changesname}{Lista de cambios en las versiones}
\newcommand{\glosname}{Índice de comandos, entornos y opciones}

\makeatletter
 \def\SpecialEnvIndex#1{\@bsphack
  \index{#1\actualchar{\protect\ttfamily#1}
   (entorno)\encapchar usage}%
  \index{entornos:\levelchar#1\actualchar
   {\protect\ttfamily#1}\encapchar usage}\@esphack}
 \def\es@yearl{l año}

 % Definimos el comando para describir opciones, que inexplicablemente falta en doc.sty:
 \def\DescribeOption{\leavevmode\@bsphack\begingroup\MakePrivateLetters
  \Describe@Option}
 \def\Describe@Option#1{\endgroup
  \marginpar{\raggedleft\PrintDescribeOption{#1}}%
  \SpecialOptionIndex{#1}\@esphack\ignorespaces}
 \def\SpecialOptionIndex#1{\@bsphack
  \index{#1\actualchar{\protect\ttfamily#1}
   (opción)\encapchar usage}%
  \index{opciones:\levelchar#1\actualchar{\protect\ttfamily#1}\encapchar
   usage}\@esphack}
 \def\PrintDescribeOption#1{\strut \MacroFont #1\ }

 % Redefinimos el comando del encabezado del índice:
 \def\index@prologue{%
  \section*{\glosname}%
  \markboth{\glosname}{\glosname}}%

 % Adaptamos un poco la lista de cambios:
 \def\changes@#1#2#3{%
  \protected@edef\@tempa{%
   \noexpand\glossary{%
    {\bfseries #1 --- #2}%
    \ifx\saved@macroname\@empty
     \relax\actualchar
    \else
     \saved@indexname
     \actualchar
     \string\verb\quotechar*%
     \verbatimchar\saved@macroname
     \verbatimchar
    \fi
    \levelchar #3}}%
  \@tempa\endgroup\@esphack}
 
 \def\@wrglossary#1{%
  \protected@write\@glossaryfile{}%
  {\string\glossaryentry{#1}{1}}%
  \endgroup\@esphack}%

 \patchcmd{\glossary@prologue}{Change History}{\changesname}{}{}
\makeatother

\title{The \textsf{\jobname} class\thanks{Esta documentación corresponde a la clase \textsf{\jobname}~\fileversion, de fecha \filedate.}}
\author{Emilio Augusto Lazo Zaia\\ \texttt{<emiliolazozaia at gmail.com>}}

\GetFileInfo{\jobname.dtx}

\begin{document}
 \maketitle
 \DocInput{\jobname.dtx}
 \clearpage
 \phantomsection
 \addcontentsline{toc}{section}{\todoname}
 \todos
 \section*{Información específica de esta versión + Reconversión Monetaria}
 % v2.82 Desde la versión v2.82 de esta clase, para poder compilar con \LuaLaTeX{} es necesaria la versión v2.26 del paquete |zref|, de fecha 2016/05/21, y la versión v1.18 de |atbegshi| de fecha 2016/06/09.

 % v2.82 -> Esta versión de |factura| se sacó principalmente para corregir un error introducido en la versión anterior v2.72 en cuanto a las referencias, para agregar la opción |numitem| que enumera los ítems en la factura y para que la línea diagonal (opción |linea|) funcione correctamente siempre.
 
 % v3.00 -> Desde la versión v3.00, todos los ejemplos están incorporados a la documentación, con la descripción de cada uno, de manera automatizada.
 
 % v3.02 -> Esta versión v3.02 se saca porque la versión 3.9 de |fancyhdr| hizo necesaria esta actualización por un error que daba con los reportes por el uso en esta clase de la macro |\fancy@setoffs|. Lo que hacíamos con esa macro de |fancyhdr| lo hicimos ahora de otra forma mejor, dada la incompatibilidad con la nueva versión de ese paquete.%v3.02
 
 % v3.40 -> La versión 3.40 se saca por los cambios en la política impositiva decretados por el Presidente en Decreto número 3085 relativos a descuentos del 3\% y 5\% en el IVA general cuando los pagos se hacen por medios electrónicos; también se corrigió la tasa de IVA adicional.
 
 % v3.40 -> En la versión 3.40 se agregaron nuevas opciones en cuanto al manejo de la fecha de emisión y su formato, posibilidad de facturas a crédito con fecha de vencimiento, campos nuevos adicionales para agregar más textos en la factura, posibilidad de imprimir información en el reverso de la hoja de la factura, posibilidades nuevas en cuanto al posicionamiento de los elementos fijos, diseño \tikz más versátil en los cuadros con bordes, mejor manejo del archivo |.csv|, cambios en los nombres de algunos comandos y opciones.
 
 % v3.40 -> En la misma versión 3.40 se cambió la estructura de la sección de los comandos en esta documentación y se agregó una página de descripción al final de cada ejemplo.
 
 % v3.44 -> El cambio en la política impositiva reflejado en la versión 3.40 queda sin efecto para fines legales, ya no está vigente dicho descuento; sin embargo, no eliminamos el código que lo produce sino que la clase arroja una advertencia cuando se pide usar la opción de descuentos por pago electrónico.
 
 % v3.44 -> Desde la versión 3.44 podemos activar cualquier opción de la clase a través de condicionales en el preámbulo y no solamente como una opción al cargar la clase; también se puede ahora quitar el efecto de cualquier opción previamente dada al cargar la clase, esto con el propósito de hacer más versátil el uso de las opciones y así poder incluirlas también en archivos |.tex| o |.def| personalizados.
 
 % v3.44 -> Adicionalmente desde la versión 3.44 el archivo |.log| de las compilaciones tiene como comentario todas las opciones que fueron pedidas al cargar la clase.
 
 % v3.47 -> La versión 3.47 corrige problemas con la compilación en \LuaLaTeX{} y escribe la versión de la clase en el campo \texttt{Subject} de la {\slshape metadata} del archivo |.pdf| compilado.
 
 % v3.47 -> La versión 3.47 agrega el comando |\Moneda| para cambiar fácilmente la moneda.
 
 % v3.56 -> {\bfseries La versión 3.56 se saca para corregir la versión 3.47 en cuanto a dar cumplimiento al Decreto Presidencial 3.332 sobre la Reconvesión Monetaria.} Desde el 4 de junio del año 2018 la moneda pasará a llamarse Bolívar Soberano (Bs.S) y luego en una fecha aún no definida (la asumimos como el 1 de enero del año 2019) la moneda volverá a llamarse Bolívar (Bs.). Eventualmente el Banco Central de Venezuela emitirá un comunicado para establecer desde cuándo la moneda de curso legal vuelve a llamarse simplemente Bolívar. A partír de esa fecha deberá usarse el comando |\Moneda| para cambiar la moneda a su denominación definitiva mientras no se haya emitido la actualización de esta clase. Esa fecha aún no está determinada, así que la suponemos -y hacemos el cambio de moneda automático en esta clase- como el inicio del año 2019. Antes del 4 de junio, se establece una nota interna automática que escribe el total también en Bolívares Soberanos.
 
 % v3.56 -> Por otra parte, la versión 3.56 elimina la inclusión del paquete |lmodern| y discrimina la inclusión de |fontenc| solamente para la compilación con pdf\LaTeX, que es lo correcto. Además, en esta versión se agrega un campo extra opcional en los ítems de la factura para poder agregar una descripción más larga a cada ítem que podrá ser referenciada en otras partes del documento (por ejemplo en el reporte) pero que no aparece en el cuadro de la factura.

 % v3.56 -> En la versión 3.56 fueron corregidos errores en cuanto al posicionamiento de la línea diagonal y se agregaron las opciones: \texttt{sinlineahni}, \texttt{sinlineahtit} y \texttt{sinlineahtot} para poder quitar las líneas horizontales internas una a una y no como antes, todas a la vez, con \texttt{sinlineashi}.

 % v3.57 -> {\bfseries Esta versión extraordinaria v3.57 se saca para dar cumplimiento al Decreto Presidencial 3.548 sobre la Reconversión Monetaria, publicado en Gaceta Oficial 41.446. Dicho decreto establece que la nueva moneda Bolívar Soberano no tendrá 3 ceros menos como lo establecía el derogado decreto 3.332 sino 5 ceros menos que el Bolívar Fuerte. La Reconversión Monetaria comienza el 20 de agosto del 2018\footnote{Salvo que se postergue otra vez.} y antes de esa fecha ya los precios tienen que estar especificados en ambas monedas, y esta clase está adaptada a esa exigencia.}

 % v3.57 -> Si el usuario especifica a mano la moneda con el comando |\Moneda| -introducido en una reciente versión-, la clase |factura| no interviene en nada en cuanto a la Reconversión Monetaria, ahora, si no se especifica la moneda, puede ocurrir lo siguiente:

 % v3.57 -> Al emitirse una factura con fecha anterior a la reconversión (20 de agosto del 2018), los montos quedarán expresados en Bolívares Fuertes, simbolizado como {\slshape Bs.}, y la clase automáticamente agregará un comentario como {\slshape nota interna} indicando cuál sería el monto en Bolívares Soberanos; esto para dar cumplimiento al Decreto Presidencial que indica que los montos deben estar en expresados en ambas monedas antes de la entrada de la nueva moneda.

 % v3.57 -> Si se emite una factura con fecha posterior a la reconversión, los montos se asumirán como expresados en Bolívares Soberanos, es decir, el usuario debe darlos ya con los 5 ceros quitados, la clase va a simbolizar la moneda como {\slshape Bs.S} y agregará un comentario similar al que acabamos de mencionar pero para explícitar el monto en Bolívares Fuertes, es decir, con 5 ceros más; esto no forma parte del Decreto pero lo quisimos agregar.
 
 % v3.57 -> Luego, la moneda volverá a llamarse Bolívar simplemente y se simbolizará otra vez como {\slshape Bs.}; ese momento aún no lo sabemos pero para los efectos de esta clase vamos a asumirlo como que ocurrirá el 1 de enero del año 2019.
 
 % v3.57 -> Esta será la última versión que sacamos solamente para corregir fechas de la Reconversión Monetaria; si el Gobierno Nacional cambia las fechas nuevamente, el usuario tendrá que hacer uso de la interfaz del paquete |datetime2| para ajustar el funcionamiento de esta clase a los nuevos plazos; para esto aclaramos que el identificador para la entrada de la Reconversión Monetaria se llama {\slshape FechaBsS}\footnote{recordemos que antes de esta fecha, los montos se expresan en Bolívares Fuertes y la nota automática aclara cuánto es en Bolívares Soberanos, y después de esta fecha es al contrario.} y el identificador para el momento (que aún no se ha dicho) desde el cual la moneda vuelve a llamarse Bolívar es {\slshape FechaBs}. Para ajustarlos se usa el comando \cs{DTMsavedate} y las fechas se dan en formato {\slshape yyyy-mm-dd}. Damos como ejemplo lo que actualmente está programado en esta clase:
 
 Desde el 15/02/2019 la moneda nacional se vuelve a llamar “Bolívar” abreviado “Bs” sin “soberano”; esto según Gaceta Oficial 41.587. Quitamos toda referencia a la Reconversión Monetaria.
 
 También en esta versión establecemos la tasa de IVA general como 16\% tal como ya está vigente desde agosto del año 2018.
 
 Ya no está activa la opción |electro| para descuentos en pagos por medios electrónicos porque ya no está vigente el decreto.
 
 Ahora en los totales aparece el signo monetario siempre a menos de que se pida en la clase la opción |sinmonedatotales|.
 
 En esta misma versión se agregó un “gancho” llamado \cs{BeforeEndPreamble} para usos muy específicos.
 
 Las “condiciones del documento” ya no son “condiciones” sino que ahora lo llamamos “nota final” porque es otra nota más aparte de las demás, y va al final del documento; no tiene sentido hablar de condiciones del documento para referirse a ese espacio. Con eso se renombran los 5 comandos, pero dejando por compatibilidad unos alias a los viejos; los nuevos son: \cs{NotaFinal}, \cs{PosNotaFinal}, \cs{SepNotaFinal}, \cs{LineaNotaFinal}, \cs{LetraNotaFinal}.

 La macro \cs{PosNotaFinal} se comporta ahora como \cs{PosFirmas}, es decir, acepta un valor inicial en “x” y uno final.

 \section*{Agradecimientos}
 
 Por pruebas y sugerencias que dieron lugar a mejoras agradezco a Ritguey%
 \ifmotorunicode\footnote{\setmainfont{Symbola}🐧}\else\message{^^J^^JLaTeX Warning: Compila con XeLaTeX.^^J}\fi{}
 Flores Esteves y a Alberto Mijares. Por contribuciones a través de \texttt{StackExchange} agradezco a Alan Munn, cfr, David Carlisle, egreg, Gustavo Mezzetti, Heiko Oberdiek, jfbu, Joseph Wright, Werner y wipet. El motivo por el que se da cada agradecimiento está como comentario en el archivo |.cls| o en el |.dtx|.
 
 \addcontentsline{toc}{section}{\changesname}
 {\makeatletter\def\pfill{\unskip\hfill\@gobble}\makeatother
  \setlength{\columnsep}{0pt}
  \PrintChanges}
 \phantomsection
 \addcontentsline{toc}{section}{\glosname}
 \PrintIndex
\end{document}
%</driver>
% \fi
%
%^^A Gracias a jfbu y a Joseph Wright por \starttodo, \stoptodo, \endorgo.
%^^A Thanks to jfbu and Joseph Wright for \starttodo, \stoptodo, \endorgo.
%^^A (http://tex.stackexchange.com/questions/235680/add-something-to-the-driver-section-inside-the-package-code-using-onlydescripti)
% \long\def\stoptodo#1\starttodo {}%
% \long\def\starttodo#1\stoptodo {}%
%
% \makeatletter
% \def\endorgo #1{%
%  \ifx #1\endinput\expandafter\stoptodo
%  \else #1\expandafter\@gobble\fi}%
% \makeatother
%
%^^A No se puede calcular el checksum con \OnlyDescription.
% \CheckSum{0}
%
% \CharacterTable
%  {Upper-case    \A\B\C\D\E\F\G\H\I\J\K\L\M\N\O\P\Q\R\S\T\U\V\W\X\Y\Z
%   Lower-case    \a\b\c\d\e\f\g\h\i\j\k\l\m\n\o\p\q\r\s\t\u\v\w\x\y\z
%   Digits        \0\1\2\3\4\5\6\7\8\9
%   Exclamation   \!     Double quote  \"     Hash (number) \#
%   Dollar        \$     Percent       \%     Ampersand     \&
%   Acute accent  \'     Left paren    \(     Right paren   \)
%   Asterisk      \*     Plus          \+     Comma         \,
%   Minus         \-     Point         \.     Solidus       \/
%   Colon         \:     Semicolon     \;     Less than     \<
%   Equals        \=     Greater than  \>     Question mark \?
%   Commercial at \@     Left bracket  \[     Backslash     \\
%   Right bracket \]     Circumflex    \^     Underscore    \_
%   Grave accent  \`     Left brace    \{     Vertical bar  \|
%   Right brace   \}     Tilde         \~}
%
% \changes{v1.00}{2013/04/29}{Primera versión.}
% \changes{v2.00}{2015/03/30}{Segunda versión: errores corregidos, comandos y opciones nuevas, ambiente de reporte, documentación completa, ejemplos y convertido todo a formato \texttt{.dtx}.}
% \changes{v2.05}{2015/04/13}{Se cambió de \texttt{4pt} a \texttt{6pt} en \texttt{factura.def} para el margen interno predeterminado en las cajas del encabezado y en la descripción.}
% \changes{v2.05}{2015/04/13}{Se quitó \texttt{UTF8} de \texttt{inputenc} como un paquete que se cargue de forma predeterminada.}
% \changes{v2.05}{2015/04/13}{Advierte que no son compatibles los comandos \cs{Descuento} con la opción \texttt{filas}.}
% \changes{v2.05}{2015/04/13}{Se puso un parche en \cs{@CargaFondoReporte} para arreglar que \cs{topskip} aplicara bien en \cs{@EncabezadoReporte}.}
% \changes{v2.05}{2015/04/13}{Arreglada la interacción entre las opciones \texttt{ivanoexplicito} y \texttt{filas}, y se quitó la línea del total cuando se usa \texttt{ivanoexplicito}.}
% \changes{v2.05}{2015/04/13}{Se renombró la opción \texttt{relleno} a \texttt{exparriba}, a la vez que se quitó \cs{Descripcion*} porque se agregó la opción \texttt{sinexpandir}.}
% \changes{v2.05}{2015/04/13}{Se renombró la opción \texttt{ivanoexplicito} por \texttt{sinivaexpresado}.}
% \changes{v2.05}{2015/04/13}{Se cambiaron los \cs{AtBeginDocument} por \cs{AtBeginShipoutNext} de la factura porque daba problemas al especificar \texttt{conreporte} sin luego agregar el ambiente de reporte.}
% \changes{v2.05}{2015/04/13}{Se colocó \cs{@SeparaMinTotales} dentro de otro \cs{AtBeginDocument} más adecuado.}
% \changes{v2.05}{2015/04/13}{Modificaciones en la documentación y en los ejemplos.}
% \changes{v2.05}{2015/04/13}{Los comandos \cs{ItemX} ahora colocan la línea bien con la opción \texttt{lineaitem}.}
% \changes{v2.1}{2015/08/26}{Advierte cuando la fecha del documento es distinta de la fecha del dia.}
% \changes{v2.1}{2015/08/26}{Los comandos de descuentos totales aceptan un * como argumento para que el descuento pueda ser en moneda y no en porcentaje.}
% \changes{v2.1}{2015/08/26}{Quitados los alias de macros viejas agregados por compatibilidad.}
% \changes{v2.1}{2015/08/26}{Corregida la forma como sale la denominación del impuesto para todos los casos.}
% \changes{v2.1}{2015/08/26}{Corregida y mejorada la documentación.}
% \changes{v2.1}{2015/08/26}{Correcciones en los ejemplos.}
% \changes{v2.1}{2015/08/26}{Mejor manejo de \cs{InicioReporte}.}
% \changes{v2.6}{2016/03/07}{Cambiado el lugar donde se fija el nombre del documento (FACTURA o COTIZACIÓN) porque antes no era correcto y no se podía cambiar el nombre de la cotización a algo distinto como por ejemplo PRESUPUESTO.}
% \changes{v2.6}{2016/03/07}{Se renombró la opción \texttt{sincond} en favor de \texttt{sincondiciones}.}
% \changes{v2.6}{2016/03/07}{Cambiado “condiciones de pago” por “condiciones del documento”.}
% \changes{v2.6}{2016/03/07}{Los descuentos globales aceptan decimales bien; antes no.}
% \changes{v2.6}{2016/03/07}{Se cambió la forma como se llama al ambiente \texttt{tabularx} en la factura creando una nueva macro que se evalúa completamente resultando en la definición de las columnas.}
% \changes{v2.6}{2016/03/07}{Agregadas cuatro opciones: \texttt{sinlineasvi}, \texttt{sinlineasve}, \texttt{sinlineash}, \texttt{sinlineashi}, \texttt{sinlineashe} y renombrada la opción \texttt{lineaitem} por \texttt{conlineasha} aludiendo a “con líneas horizontales para los artículos”.}
% \changes{v2.6}{2016/03/07}{Se arregló la línea oblicua que en algunos casos seguía sin salir.}
% \changes{v2.6}{2016/03/07}{Textos “PAGADO” y “Por el emisor” para las facturas y cotizaciones cuando no se define el emisor del documento.}
% \changes{v2.6}{2016/03/07}{Nueva opción \texttt{dosfirmas} para poner las dos firmas en casos en los que lo predeterminado sea una o ninguna firma (cotizaciones o prefacturas).}
% \changes{v2.6}{2016/03/07}{Nuevo comando \cs{Conforme} para el nombre de quien recibe la factura.}
% \changes{v2.6}{2016/03/07}{Cambiado el lugar en el código donde se llama a la macro que dibuja la línea inutilizadora del espacio en blanco y se hace ahora con \cs{AddToShipoutPicture} para que no se genere un espacio en blanco extra.}
% \changes{v2.6}{2016/03/07}{El ejemplo de la documentación es de una cotización y no de una prefactura; actualizada la documentación y los ejemplos.}
% \changes{v2.6}{2016/03/07}{En el índice ponemos “entorno” en lugar de “environment”.}
% \changes{v2.6}{2016/03/07}{El verbo “poner” es digno para la documentación. No sólo las gallinas ponen.}
% \changes{v2.6}{2016/03/07}{Renombrado el comando \cs{EstiloPag} por \cs{InfoPagina}; también \cs{LetraEncReporte} por \cs{LetraInfoPagina}; por otra parte se cambió \cs{LetraDatosReporte} en favor de \cs{LetraEncReporte}, y por último se cambió \cs{LetraTitRenglon} por \cs{LetraTitColumnas}.}
% \changes{v2.6}{2016/03/07}{Renombrados los estilos de página: \texttt{soloenc} a \texttt{soloarriba}, y \texttt{solopie} a \texttt{soloabajo}.}
% \changes{v2.6}{2016/03/07}{Se mejoró el código relativo al estilo de página del reporte y agregada la macro \cs{EstiloPagReporte} para establecer el estilo de página del reporte.}
% \changes{v2.6}{2016/03/07}{Se agregó una forma para que más de una línea en la descripción de la factura sea mejor manejada.}
% \changes{v2.6}{2016/03/07}{Mejorado el código de la macro interna \cs{@EscribeTipoDeImpuesto}.}
% \changes{v2.6}{2016/03/07}{Cuando la descripción en el cuadro de la factura ocupa más que una línea, agrega al final de cada línea el carácter $\hookleftarrow$.}
% \changes{v2.6}{2016/03/07}{Corregido que salga la “G” obligatoria de IVA General cuando hay un elemento exento y uno con IVA General.}
% \changes{v2.6}{2016/03/07}{Cuatro nuevas definiciones dentro del archivo \texttt{factura.def} para establecer algunos caracteres que antes estaban fijados dentro del archivo \texttt{factura.cls}}
% \changes{v2.6}{2016/03/07}{{\slshape parbox}es para cada elemento en la tabla, para asegurar que la denominación del gravamen quede alineada correctamente.}
% \changes{v2.6}{2016/03/07}{Quitados los paquetes \texttt{xifthen} y \texttt{pbox}. Arreglada la presentación de los datos del cliente cuando el resumen es más grande.}
% \changes{v2.6}{2016/03/07}{Cuando había descuentos totales y se pedía modo de totalización en filas, este modo no se apagaba correctamente y la línea oblicua daba problemas. Arreglado.}
% \changes{v2.72}{2016/05/16}{Las macros internas \cs{@DescripcionAuxCmd} y \cs{@EscribeTipoDeImpuesto} fueron modificadas.}
% \changes{v2.72}{2016/05/16}{Arreglada la alineación a la derecha en algunos casos en los precios.}
% \changes{v2.72}{2016/05/16}{Se incluyó dentro de la macro interna \cs{@ITEM} la opción para soportar \cs{ItemX}.}
% \changes{v2.72}{2016/05/16}{Ahora se procesa el entorno de la factura aún cuando se pide sólo el reporte, para poder tomar las referencias y los totales. Para eso se cambiaron de sitio varias macros internas.}
% \changes{v2.72}{2016/05/16}{Agregadas las etiquetas y las referencias.}
% \changes{v2.82}{2016/06/25}{Agregada la opción \texttt{numitem} para enumerar cada ítem en la factura.}
% \changes{v2.82}{2016/06/25}{Corregido problema de etiquetas repetidas introducido en la versión 2.72.}
% \changes{v2.82}{2016/06/25}{Corregida la alineación vertical entre las distintas columnas agregando \texttt{[t]} al abrir los “{\slshape parboxes}” en los comandos auxiliares que ponen cada celda en un {\slshape parbox} del tamaño de la celda más grande, porque si no se centra verticalmente la celda y se desalinea respecto a la columna de al lado.}
% \changes{v2.82}{2016/06/25}{Ahora dentro de la definición de \cs{@CaracterCeldasVacias} está el código que centra el texto, para poder usar comandos que llenen la caja completa como \cs{dotfill}.}
% \changes{v2.82}{2016/06/25}{Reimplementada la línea diagonal para que funcione en todos los casos.}
% \changes{v2.82}{2016/06/25}{Nueva macro interna \cs{@ColumnasMultiColumnTotalesAuxCmd} para hacer más general la colocación de los totales en el modo de columnas dentro de un \cs{multicolumn}.}
% \changes{v2.82}{2016/06/25}{Quitado el paquete \texttt{MnSymbol} en favor del comando \cs{flechaCR} para la flecha que indica que la descripción ocupa más de una línea.}
% \changes{v3.00}{2017/03/11}{Se agregó la opción \texttt{sinmarcas} para pedir que no se escriba el carácter de fin de linea cuando la descripción ocupa más que una línea y también para evitar que salga el carácter que marca las celdas vacías también cuando la decripción es larga.}
% \changes{v3.00}{2017/03/11}{Nueva macro \cs{SepItemsExtra} para agregar espacio extra entre los ítems en la factura.}
% \changes{v3.00}{2017/03/11}{Opción nueva \texttt{sintotales} para que no se escriba ningún total de los elementos. Útil para cotizaciones de por ejemplo varias formas de un mismo producto.}
% \changes{v3.00}{2017/03/11}{Opción nueva \texttt{centradesc} para centrar el texto en la descripción de cada elemento en la factura.}
% \changes{v3.00}{2017/03/11}{Cambiado el interlineado en la documentación, el tamaño del papel y agregados a la documentación todos los ejemplos que están embebidos en el \texttt{.dtx}, de forma automática.}
% \changes{v3.00}{2017/03/11}{El descuento individual ahora es opcional al pedir la opción, es decir, no hace falta establecer en '0' los elementos donde no haya descuento cuando se usa la opción \texttt{descuentos}.}
% \changes{v3.00}{2017/03/11}{Opción \texttt{subtotal} para agregar una columna extra después del precio unitario y antes del descuento, para reflejar el precio antes de aplicar el descuento individual.}
% \changes{v3.02}{2017/07/16}{Ya no usamos la macro \cs{fancy@setoffs} de \texttt{fancyhdr} en el reporte para establecer valores iniciales por ejemplo de la longitud de la línea en los encabezados; ahora lo hacemos de una forma diferente, incluyendo en el \texttt{.def} una macro llamada \cs{@GrosorLineaInfoPag} que alberga el valor que \texttt{fancyhdr} asigna preterminadamente al grosor de dicha línea. No agregamos un comando para cambiar ese valor por no ser algo de relevancia.}
% \changes{v3.02}{2017/07/16}{Ajustada ligeramente esta lista de cambios y la inclusión de los ejemplos en la documentación.}
% \changes{v3.40}{2017/12/11}{Se agregó la opción \texttt{electro} para dar cumplimiento con el Decreto Presidencial 3085 y la Providencia Administrativa 0048 que establece descuentos en el IVA.}
% \changes{v3.40}{2017/12/11}{Macros nuevas para notas internas al cuadro de la factura y externas a él.}
% \changes{v3.40}{2017/12/11}{Ahora \cs{PosCond} acepta \texttt{*} para que el texto pueda abrir hacia arriba en la posición establecida y no solamente hacia abajo.}
% \changes{v3.40}{2017/12/11}{Agregado el comando \cs{Credito} para la posibilidad de facturas a crédito donde es calculada la fecha de vencimiento de la factura que aparecerá debajo de la fecha de emisión.}
% \changes{v3.40}{2017/12/11}{Ahora \cs{PosFecha} también acepta \texttt{*} para que las fechas (emisión y vencimiento) puedan fijarse también respecto a una posición final y no solamente inicial.}
% \changes{v3.40}{2017/12/11}{El comando \cs{Fecha} acepta \texttt{*} para poder especificar la fecha en formato \texttt{iso} de forma más estándar para que sea tomada correctamente por el paquete \texttt{datetime2}, usado para calcular la fecha de vencimiento.}
% \changes{v3.40}{2017/12/11}{Se agregó la macro \cs{FormatoFecha} para poder cambiar el formato en el que el usuario da la fecha de emisión y cómo ésta es presentada.}
% \changes{v3.40}{2017/12/11}{Si no se especifica la fecha, será tomada la fecha actual y generará una advertencia indicando eso y no un error indicando que no se ha especificado la fecha, como era antes.}
% \changes{v3.40}{2017/12/11}{Actualizada la documentación para reflejar los cambios.}
% \changes{v3.40}{2017/12/11}{Se renombró \cs{@EstiloFecha} por \cs{@EstiloFechas}, y \cs{@PosicionaFecha} por \cs{@PosicionaFechas}.}
% \changes{v3.40}{2017/12/11}{Reestructuración de la documentación.}
% \changes{v3.40}{2017/12/11}{Nueva opción \texttt{vencimiento} para poner la fecha de vencimiento aún en facturas a contado.}
% \changes{v3.40}{2017/12/11}{También \cs{PosDenom} y \cs{PosControl} aceptan \texttt{*}. Cambió en 2mm el margen superior predeterminado, así como también hubo un cambio en la coordenada vertical de la fecha, la denominación del documento y el número de control, ya que antes los predeterminados estaban especificados sin \texttt{*} y luego no.}
% \changes{v3.40}{2017/12/11}{No poner el teléfono ahora produce una advertencia y no un error como antes.}
% \changes{v3.40}{2017/12/11}{Funciones internas para los mensajes informaticos, advertencias y errores de la clase.}
% \changes{v3.40}{2017/12/11}{Se le agregó la posibilidad de \texttt{*} a la macro interna \cs{@DatosAux} para que pudiera ser utilizada la misma macro por los casos donde hay resumen y donde no lo hay.}
% \changes{v3.40}{2017/12/11}{A los cuadros dibujados con \tikz les fue separada la definición del nodo para ponerla dentro del \texttt{.def} en lugar de estar en el \texttt{.cls}.}
% \changes{v3.40}{2017/12/11}{Cuando se especifica 0 como grosor de línea para la descripción o el encabezado, desaparece la línea completamente y la estructura \tikz.}
% \changes{v3.40}{2017/12/11}{Renombrada \cs{PosCond} a \cs{PosCondiciones}.}
% \changes{v3.40}{2017/12/11}{Ahora las condiciones aceptan línea en \tikz, es decir que hay dos macros nuevas: \cs{LineaCondiciones} y \cs{SepCondiciones}. Predeterminadamente las condiciones quedaron como antes, o sea sin diseño de \tikz.}
% \changes{v3.40}{2017/12/11}{Quitada la macro interna \cs{@TamDescrip} porque no hacía nada.}
% \changes{v3.40}{2017/12/11}{Renombrada \cs{SepEncabez} a \cs{SepEncabezado}, tambień \cs{SepDescrip} a \cs{SepDescripcion}, \cs{LineaEncabez} en favor de \cs{LineaEncabezado}, y \cs{LineaDescrip} en favor de \cs{LineaDescripcion}, así como algunas macros internas asociadas a ellas.}
% \changes{v3.40}{2017/12/11}{Macro nueva para una nota previa al cuadro de la factura.}
% \changes{v3.40}{2017/12/11}{La macro \cs{SepEncabezado} fue dividida entre \cs{SepDatosResumen} y \cs{SepEncabezado}. La macro \cs{SepElementos} fue renombrada a \cs{SepVertical} y \cs{LetraTitCuadro} a \cs{LetraTitColumnas}.}
% \changes{v3.40}{2017/12/11}{Contador \texttt{@NumItem} renombrado a \texttt{NumItem}. Las macros para los parámetros de las líneas ahora aceptan dar solamente el primer argumento, por ejemplo para definir que no salga la línea, sólo basta con por ejemplo \cs{LineaEncabezado}{0pt}.}
% \changes{v3.40}{2017/12/11}{Nuevo parámetro opcional después de la descripción en los \cs{Item***} para cambiar el campo que va a salir en el \texttt{.csv}, y ahora se puede especificar un descuento individual en los ítems, que será ignorado si no se carga la opción para descuentos individuales.}
% \changes{v3.40}{2017/12/11}{Actualizada la tasa de IVA adicional.}
% \changes{v3.40}{2017/12/11}{Nueva opción \texttt{iva} para agregar una columna adicional que muestra el IVA de cada item.}
% \changes{v3.40}{2017/12/11}{Modificada la macro interna \cs{@DescripcionAuxCmd} por problemas que daba en algunos casos.}
% \changes{v3.40}{2017/12/11}{Se renombró la opción \texttt{exparriba} por \texttt{expandecuadro}.}
% \DoNotIndex{}
% \changes{v3.40}{2017/12/11}{Modificada la definición de \cs{ItemX} para que maneje la descripción especial para el archivo \texttt{csv}.}
% \changes{v3.40}{2017/12/11}{Nuevo comando \cs{ivaref} para referenciar el IVA del ítem etiquetado.}
% \changes{v3.40}{2017/12/11}{Se cambiaron unos \cs{AtEndDocument} por \cs{AtBeginShipoutNext}. Se hizo necesario siempre el paquete \texttt{eso-pic}.}
% \changes{v3.40}{2017/12/11}{Se quitó la fijación del estilo de página de un \cs{AtBeginDocument} innecesario. Nuevo entorno \texttt{reverso} para el reverso de la factura y nuevo comando para la letra predeterminada del entorno: \cs{LetraReverso}.}
% \changes{v3.40}{2017/12/11}{Cada ejemplo tiene una página al final con su explicación correspondiente.}
% \changes{v3.40}{2017/12/11}{Ahora en los ejemplos, primero va la opción o la macro entre paréntesis y después va la explicación y no al revés como era antes.}
% \changes{v3.40}{2017/12/11}{Quitada la opción \texttt{sincondiciones}. Si no se establecen las condiciones produce una advertencia.}
% \changes{v3.40}{2017/12/11}{Quitada la macro \cs{mostrar}. Ahora cada texto a mostrar se hace con el mismo comando con el que se definió.}
% \changes{v3.40}{2017/12/11}{Algunos condicionales cambiados.}
% \changes{v3.40}{2017/12/11}{Otra vez corregida la macro interna \cs{@DescripcionAuxCmd} y arreglado el espaciado entre los ítems; ojalá que esta vez sí esté bien.}
% \changes{v3.44}{2018/03/13}{Reestructurado el archivo \texttt{.cls} para aceptar prender y apagar las opciones de la clase a través de condicionales de \TeX{} con el mismo nombre de la opción.}
% \changes{v3.44}{2018/03/13}{Se cambió un \cs{AtBeginDocument} que escribía texto en la hoja por un \cs{AfterEndPreamble}.}
% \changes{v3.44}{2018/03/13}{Se acabó el descuento del IVA. Ahora advierte cuando se usa, posibilitando su uso después si llegan a volver a activar la medida.}
% \changes{v3.44}{2018/03/13}{Cada opción cargada escribe un comentario en el archivo \texttt{.log} para depuración.}
% \changes{v3.47}{2018/04/17}{No compilaba con \LuaLaTeX{} por diferencias en el motor. Ya compila.}
% \changes{v3.47}{2018/04/17}{Agrega la versión de la clase bajo \texttt{Subject} en la metadata de los PDF compilados.}
% \changes{v3.47}{2018/04/17}{Se agrega la macro \cs{Moneda} para cambiar el signo monetario a usar.}
% \changes{v3.47}{2018/04/17}{Se da cumplimiento al Decreto Presidencial 3.332 de la reconversión monetaria.}
% \changes{v3.47}{2018/04/17}{Cambiados algunos de los \cs{immediate}\cs{write} por \cs{write}.}
% \changes{v3.56}{2018/05/09}{Quitamos \texttt{lmodern} de todas partes y mejoramos la inclusión discrecional de \texttt{fontenc}.}
% \changes{v3.56}{2018/05/21}{Agregada la opción de una descripción larga para cada ítem de la factura únicamente para referenciarla luego en otra parte del documento.}
% \changes{v3.56}{2018/05/21}{La implementación anterior del Bolívar Soberano tenía defectos; ya están corregidos.}
% \changes{v3.56}{2018/05/21}{Volvimos a agregar la macro interna \cs{@TamDescripcion} quitada en la versión v3.40. Sin ella la línea no salía bien con la opción \texttt{filas}.}
% \changes{v3.56}{2018/05/21}{Creamos condicionales internos: \cs{if@ExpandeCuadro} y otros para evaluar la línea inutilizadora del espacio en blanco, así como también unas macros que guardan los valores de \cs{baselineskip} y \cs{tabcolsep} y otros cambios en la línea.}
% \changes{v3.56}{2018/05/21}{Nuevas opciones \texttt{sinlineahni}, \texttt{sinlineahtit} y \texttt{sinlineahtot} para no poner la línea horizontal antes de la nota interna, después del título de las columnas y antes de los totales.}
% \changes{v3.56}{2018/05/21}{Agregamos el paquete \texttt{anyfontsize}.}
% \changes{v3.57}{2018/08/02}{El Gobierno Nacional posterga la Reconversión Monetaria. Cambiamos un poco la forma como estaba programado esto. El usuario podrá cambiar las fechas que intervienen en eso, por si ocurre otro cambio.}
% \changes{v3.70}{2019/08/17}{Nueva tasa de IVA general del 16\% vigente desde agosto del 2018.}
% \changes{v3.70}{2019/08/17}{Establecido el 15/02/2019 como la fecha en la que ocurre que la nueva moneda pierde el adjetivo “soberano” para llamarse “bolívar” simplemente.}
% \changes{v3.70}{2019/08/17}{Nueva macro interna \cs{@EstableceOpciones} para poder evaluarla doblemente, una antes del -nuevo también- gancho \cs{BeforeEndPreamble} y otra después de él, porque en ese gancho se puede escribir código que altere las opciones y deberán ser evaluadas nuevamente para la consistencia entre ellas.}
% \changes{v3.70}{2019/08/17}{Inhabilitada la opción \texttt{electro} pero se deja el código todavía, aunque comentado.}
% \changes{v3.70}{2019/08/17}{Se comentó todo el código referente a la Reconversión Monetaria.}
% \changes{v3.70}{2019/08/17}{Implementada la función de que agregue el signo monetario en los totales, y una opción \texttt{sinmonedatotales} para evitarlo.}
% \changes{v3.70}{2019/08/17}{Ya no hablamos de condiciones sino de “nota final”, que es aquello que va después de la descripción larga y que tiene posición fija. Ese espacio se puede usar para distintas cosas, por lo que hablar de ese espacio como el de las condiciones del documento no tiene mucho sentido. De manera que los comandos renombrados pasaron a ser \cs{NotaFinal}, \cs{PosNotaFinal}, \cs{SepNotaFinal}, \cs{LineaNotaFinal}, \cs{LetraNotaFinal}. Se mantienen los comandos anteriores por compatibilidad.}
% \changes{v3.70}{2019/08/17}{Corregido el posicionamiento en “x” de la nota final.}
% \changes{v3.70}{2019/08/17}{Ahora la macro \cs{PosNotaFinal} acepta un intervalo en “x” para posicionar la nota final, es decir que funciona como \cs{PosFirmas}.}
% \changes{v3.70}{2019/08/17}{Mejorada la macro interna que escribe la fecha en la metadata; ya no usa \texttt{xstring} sino \LaTeX3. Quitado código espúreo de \texttt{xstring} en las macros que permiten fijar las posiciones de los elementos fijos.}
%
%
% \tableofcontents
%
%
% \section*{Brief description in english}
%
% |factura| is the spanish word for `invoice', so this is a \LaTeX{} class for typesetting and calculating invoices, made taking into account requirements of {\slshape SENIAT} legislation (tax collector entity on the Bolivarian Republic of Venezuela), but its use is not restricted to Venezuela because all variables and displayed text can be redefined by invoking commands or editing.
%
% \newpage
% \setstretch{1.4}
%
%
% \section{Introducción}
%
% La clase de \LaTeX{} |factura| está diseñada para su uso en la República Bolivariana de Venezuela, conforme a los requerimientos fijados por las providencias del {\slshape SENIAT} 0591, 0257, 0071 y 0048, válidas éstas para agosto del año 2019.
%
% Con |factura| se pueden elaborar facturas en formas libres fiscales\footnote{En cuanto a las formas libres, cada hoja del talonario de facturas legales tiene un encabezado con los datos del emisor de la factura y un pie de página con la información de identificación de la imprenta, en tamaño de hoja carta entera; esto es lo que se llama “forma libre”.}, cotizaciones y reportes de servicios, dejando todos los cálculos a \TeX{}.
%
% El diseño de las facturas hechas con esta clase consiste en un encabezado con los datos que identifican al cliente, con al lado un -opcional- resumen; estas partes de la factura están encerradas en recuadros. Si no se especifica el resumen saldrá solamente una de estas cajas en el encabezado que es la de los datos del cliente.
%
% Luego de esto viene un -opcional- recuadro con una nota donde puede ir la forma de pago, luego un cuadro con los campos: cantidad, descripción, precio unitario y precio total\footnote{Pueden haber variaciones en estos campos controlables con las opciones de la clase (véanse subsecciones \ref{subsec:columnas}, \ref{subsec:descuentos} y \ref{subsec:sincantidad}).}, con los elementos a facturar, cada uno en una línea, luego los totales, luego una -opcional- nota dentro del cuadro de la factura, otra nota (opcional también) después del cuadro de la factura y al final viene una opcional descripción larga que sale en un recuadro. La factura puede tener también un reverso con más información.
%
% En fin, la factura consta de:
% \begin{enumerate}[\bfseries i)]
%  \item el encabezado con los datos del cliente y el -opcional- resumen,
%  \item la opcional nota previa al cuadro de la factura,
%  \item el cuadro con los elementos que se están facturando y las totalizaciones,
%  \item la opcional nota interna al cuadro de la factura,
%  \item la opcional nota externa al cuadro de la factura,
%  \item una opcional descripción larga, e
%  \item información opcional en el reverso de la hoja.
% \end{enumerate}
%
% Aparte, el usuario puede escribir un reporte largo asociado a la factura, es decir, esta clase sirve también para escribir reportes.
%
% \newpage
%
%
% \section{Instalación}
%
% El paquete que está en CTAN consta de un archivo |README| en formato |UTF-8|, esta documentación compilada como |factura.pdf| y el archivo |factura.dtx|, del cual se generan todos los demás archivos.
%
% Al compilar |factura.dtx| va a producirse |factura.cls| y |factura.def|, además de todos los ejemplos en archivos |.tex| y |factura.ins|.
%
% \noindent Si queremos solamente generar los archivos sin compilar la documentación, debemos ejecutar:
%
% \noindent {\color{RoyalBlue}|pdftex factura.dtx|}
%
% \noindent Para generar los archivos y también compilar la documentación, debemos ejecutar los siguientes pasos:
%
% \begin{enumerate}[\bfseries 1. ]
%
% \item {\color{RoyalBlue}|xelatex factura.dtx|}\footnote{Es preferible compilar el |.dtx| con \XeLaTeX{} o con \LuaLaTeX pero también puede hacerse con pdf\LaTeX.}
%
% Luego de compilar |factura.dtx| hay que compilar el fondo de ejemplo, el membrete de ejemplo y la firma de ejemplo; estos tres archivos emulan haber digitalizado una hoja de una factura del talonario fiscal, un membrete para los reportes y también una firma que puede agregarse a los documentos a producir. Estos ejemplos se usarán en algunos de los ejemplos de formatos que incluímos y por eso es necesario tenerlos compilados en |.pdf| previamente, ya que su inclusión en los ejemplos es en formato compilado |.pdf|.
%
% El fondo y membrete de ejemplo los debemos compilar así:
%
% \item {\color{RoyalBlue}|pdflatex factura-ejemplo-fondo.tex|}
% \item {\color{RoyalBlue}|pdflatex factura-ejemplo-membrete.tex|}
%
% La firma escaneada de ejemplo necesita compilarse con \XeLaTeX{} porque son códigos en PSTricks y debe producirse un |.pdf|. Se compila así:
%
% \item {\color{RoyalBlue}|xelatex factura-ejemplo-firma.tex|}
%
% Al tener el fondo, el membrete y la firma compilados en |.pdf| según los pasos anteriores, podremos compilar cualquiera de los ejemplos en |.tex| que el archivo |factura.dtx| genera, y eso lo debemos hacer, ya que todos ellos (en su versión en |.pdf|) forman parte de la documentación. Los podemos compilar así:
%
% \item {\color{RoyalBlue}|pdflatex factura-ejemplo-prefactura.tex|}
%
% De esta forma como compilamos {\ttfamily factura-ejemplo-prefactura} compilamos todos los demás ejemplos: la cotización, las facturas del 1 hasta la última, el reporte del 1 hasta el último, y finalmente la factura para otra legislación.\footnote{Podemos también compilar con \LuaLaTeX{} o con \XeLaTeX{}.}
%
% Al tener todo esto hecho volvemos a compilar |factura.dtx| para producir esta documentación, puesto que el |.dtx| necesita ser compilado más de una vez para que la documentación quede correctamente elaborada:
%
% \item {\color{RoyalBlue}|xelatex factura.dtx|}
%
% Los comandos para generar el índice y la lista de cambios, que automáticamente se ejecutan al compilar |factura.dtx| son:
%
% \item {\color{RoyalBlue}|makeindex -s gind.ist -o factura.ind factura.idx|}
% \item {\color{RoyalBlue}|makeindex -s gglo.ist -o factura.gls factura.glo|}
%
% {\slshape No es necesario actualizar los índices con los dos comandos anteriores porque esos dos comandos los ejecuta el motor de compilación al actuar sobre }|factura.dtx|.
%
% \end{enumerate}
%
%
% \section{Archivos incluídos en el paquete}
%
% Este paquete incluye varios ejemplos, entre ellos la emulación de un diseño de factura entregado por alguna imprenta, que nos sirve para nuestros ejemplos de prefacturas, un diseño de membrete que nos sirve para los reportes, una firma que emula ser una firma real escaneada para su inclusión en algunos casos, así como varios ejemplos de facturas, prefacturas, cotizaciones y reportes. Cada uno de ellos tiene al principio del documento fuente la aclaratoria de qué es lo que hace, para diferenciarlo de los demás ejemplos.
%
% \noindent \begin{tabular}{rl}
%  \bfseries Archivo & \bfseries Descripción \\ \hline
%  |factura.dtx| & Archivo principal del que todos los demás se generan, \\
%  |factura.pdf| & Esta documentación, \\
%  |README| & Archivo con la descripción corta, \\
%  |factura-ejemplo-firma.tex| & Ejemplo de firma digitalizada, \\
%  |factura-ejemplo-fondo.tex| &  Elemplo de una forma libre fiscal digitalizada, \\
%  |factura-ejemplo-membrete.tex| & Ejemplo de membrete para los reportes, \\
%  |factura-ejemplo-aux.tex| & Archivo auxiliar que se incluye en todos los ejemplos y \\
%                            & sirve para que cada ejemplo conste de su descripción, \\
%  |factura-ejemplo-prefactura.tex| & Ejemplo de “{\slshape prefactura}”, \\
%  |factura-ejemplo-cotizacion.tex| & Ejemplo de cotización, \\
%  |factura-ejemplo-factura*.tex| & Ejemplos de facturas, \\
%  |factura-ejemplo-reporte*.tex| & Ejemplos de reportes, \\
%  |factura-beispiel-rechnung.tex| & Un ejemplo de una factura adaptada a otra legislación. \\ \hline
% \end{tabular}
%
% \IfFileExists{factura-ejemplo-prefactura.pdf}{%
%  \newpage\section{Ejemplo preliminar\label{sec:ejemplo}}
%  Lo que sigue es un ejemplo del aspecto final de una {\slshape “prefactura”}, que está bajo el nombre de “|factura-ejemplo-prefactura.tex|” y más adelante explicaremos en qué consiste:
%
%  \noindent
%  \mbox{}\hfill
%  \vfil
%  \fbox{\includegraphics[width=0.9\linewidth]{factura-ejemplo-prefactura.pdf}}
%  \hfill\mbox{}
%  \vfil\mbox{}
%  \newpage}
% {\message{^^JLaTeX Warning: No existe factura-ejemplo-prefactura.pdf; hay que compilarlo antes de compilar este documento.^^J}}
%
%
% \section{Geometría de la hoja\label{sec:geohoja}}
%
% El espacio que la imprenta deja en blanco en la forma libre está todo disponible para la impresión de la factura con esta clase. Este espacio se delimita con el paquete |geometry|. Los valores predeterminados que tienen que ver con la geometría de la hoja están todos definidos dentro del archivo |factura.def|.
%
% El margen superior que se da con |geometry| está contabilizado desde el inicio del papel\footnote{El papel predeterminado es tamaño carta.} hasta donde empieza el encabezado con los datos del cliente, es decir que el encabezado es el primer elemento desde arriba hacia abajo en una factura.
%
% El margen inferior de |geometry| empieza donde termina la descripción del servicio -o el fin del cuadro de la factura, en caso de que no se agregue un texto de descripción larga-, hasta el final del papel.
%
% Para aclarar lo anterior es necesario decir que hay textos con posiciones fijas que están fuera de estos márgenes, como la fecha de emisión, la “denominación” del documento (que es “FACTURA”, “COTIZACIÓN” u otro especificable), el número de la factura, el número de control (cuando aplica), las nota final y los espacios para las firmas; todas estas partes del documento están fijas en posiciones arbitrarias fuera de los márgenes dejados por |geometry|. Estos textos son fijos para darles posiciones que no dependan del contenido particular de cada factura sino del diseño del talonario que ya tenemos impreso, que siempre será igual, y para el cual estamos adaptando esta clase |factura|.
%
% Las ubicaciones de los textos fijos también tienen valores predefinidos dentro de |factura.def|. Cualquiera de esos valores puede cambiarse mediante comandos, tal como se fijan los valores predeterminados en dicho archivo |.def|; esto significa que se puede personalizar a cualquier diseño del talonario entregado por la imprenta, basta con entender qué hace cada comando (explicado en este documento), consultar el |.def| o ver los ejemplos.
%
% En algunos casos las formas libres tienen impreso el número de la factura y no es controlable esto por el emisor (como ocurre con el número de control que la legislación obliga a que siempre venga impreso en cada hoja del talonario). En otros casos el emisor tiene la libertad de escribir el número de la factura y controlarlo él mismo. Se puede hacer funcionar la clase según cualquiera de estas dos formas con una opción en |\documentclass|, es decir, podemos imprimir el número de la factura donde queramos, o también podemos asumirlo impreso en el talonario, según sea el caso del talonario que tengamos.
%
% Al usarse esta clase se deben primero hacer medidas en las facturas que la imprenta nos entrega para fijar los márgenes y determinar dónde poner los textos fijos que se deseen cambiar de la posición que viene predeterminada.
%
% El archivo |factura.def| tiene todas aquellas definiciones y valores que pueden ser ajustados; basta usarlos en nuestros documentos de la misma forma como están predefinidos allí; eso aplica no sólo para los de la geometría de la hoja sino para cualquier ajuste que esté en ese archivo o cualquier campo que tenga algún texto o palabra predefinida para la legislación venezolana.
%
%
% \section{Uso de la clase}
%
% El documento debe comenzarse con |\documentclass{factura}| con las opciones necesarias, que se explicarán en la sección \ref{sec:opciones}. Después se definen a través de macros {\bfseries en el preámbulo} todas las variables y dimensiones para personalizar el documento, esto es las posiciones de los textos fijos, tipos de letra (opcional), datos del cliente, un resumen (opcional), fecha y demás. Despues de |\begin{document}| va el cuadro de la factura propiamente, para lo cual la clase define un entorno llamado |factura|
% \DescribeEnv{factura}
% (|\begin{factura}...\end{factura}|), en cuyo interior se escriben únicamente los elementos a facturar a través de comandos del tipo |\Item|\footnote{Es posible incluir elementos gravados en IVA con alícuota general (12\%), alícuota reducida (8\%), alícuota adicional (27\%), así como también elementos exentos de impuesto al valor agregado, con \cs{Item}, \cs{ItemR}, \cs{ItemA} e \cs{ItemE} respectivamente.}, especificando su cantidad, descripción y precio unitario como argumentos, en ese orden.
%
% Finalizado el entorno se puede escribir (opcionalmente) una descripción larga o una aclaratoria de lo que se factura; esto se hace con el comando |\Descripcion|.
%
% Algunas de las posibilidades de cambio de estilo de esta clase consisten en que puede cambiarse la forma como es presentado cualquiera de los textos, se pueden suprimir las líneas verticales y horizontales del cuadro de la factura, se pueden agregar líneas horizontales separando cada ítem, también se puede dibujar una línea oblicua que proteja el cuadro de la factura de adiciones posteriores de texto, entre otras prestaciones más.
%
% Si el emisor lo desea puede digitalizar su firma personal y colocarla en el espacio ya dispuesto para firmar, con un ajuste mínimo para que salga correctamente sobre la línea.
%
% También el usuario tiene la posibilidad de producir una {\slshape “prefactura”}; así le llamamos a un documento sin validez legal pero que puede ser enviado por fax o por correo electrónico. Este documento tiene el mismo aspecto de la factura fiscal si se digitaliza la forma libre vacía\footnote{Quitándole las numeraciones obviamente. También puede diseñarse en un |.tex| una forma libre vacía idéntica a la que se tiene, así como se hace en el ejemplo incluído.}, se incluye la imagen como fondo de la página y se definen las posiciones fijas de las numeraciones impresas en el talonario, es decir el número de factura y el número de control, de los que hablamos previamente.
%
% Con esta misma clase tiene se la posibilidad de imprimir un información en el reverso de la hoja de la factura que saldrá escrita utilizando toda la hoja (los márgenes son ajustables). Esa información se encierra en un ambiente destinado para ese fin (|\begin{reverso}...\end{reverso}|),
% \DescribeEnv{reverso}
% y al momento de la compilación puede decidirse si se imprime esa parte o no.
%
% Otra posibilidad que da esta clase es la de escribir un reporte largo asociado a la factura\footnote{Si lo hubiera, así como también se puede usar estar clase para elaborar un reporte sin asociarlo a una factura en particular.}. Tal reporte se escribe dentro de un ambiente en el mismo documento en el que se escribe la factura asociada a él (|\begin{reporte}...\end{reporte}|),
% \DescribeEnv{reporte}
% y con dar una opción al inicio se puede hacer que se impirma el reporte en las páginas siguientes a la factura, o también se puede hacer que se imprima sólo el reporte ignorando todo lo que no constituye el reporte, de la misma manera como también se puede hacer que se imprima sólo la factura ignorando el entorno |reporte|. Esto permite tener todo en un mismo documento e imprimir lo que se desee en el momento, o todo a la vez.
%
%
% \section{Compilación de un documento\label{sec:compila}}
%
% Para compilar una |factura| puede usar cualquiera de los motores \LaTeX, pdf\LaTeX, \XeLaTeX{} o \LuaLaTeX, naturalmente con las restricciones que impone cada uno.
%
% Hay casos en los que hace falta compilar más de una vez el documento. Dos veces necesita ser compilado un documento cuando alguna descripción en el cuadro de la factura ocupa más que una línea; también dos veces cuando se incluye una firma en el reporte, y tres o cuatro veces cuando se pide que se dibuje una línea oblicua que inutilice el espacio restante en el cuadro de la factura o cuando se usan etiquetas (sección \ref{sec:etiquetas}).
%
% Todo documento |.pdf| compilado con la clase |factura| escribe en la {\slshape metadata} del |.pdf| un texto con la versión y la fecha de la clase con la que se compiló; ese texto va en el campo |Subject| de dicha {\slshape metadata}.
%
%
% \section{Opciones al cargar la clase\label{sec:opciones}}
%
%  Al cargar la clase con |\documentclass| podemos dar distintas opciones según el comportamiento que queramos; dichas opciones están descritas en esta sección y cualquier opción cargada se verá como comentario en el archivo |.log| de la compilación bajo el texto “{\ttfamily Opción cargada: }” con el nombre de la opción pedida.
%
%
% \subsection{Sobre el tipo de documento a producir}
%
% {\setlength{\parindent}{0pt}
%
% \DescribeOption{prefactura}
%  La {\slshape “prefactura”} es un documento no legal, preliminar, que no se imprime sobre papel de factura sino que está concebido para enviarse por correo o fax. Para eso se define la imagen que contiene el diseño de la factura con el comando |\FondoFactura|, luego la clase escribe la denominación “FACTURA” con su número\footnote{Lo predeterminado cuando no es una {\slshape “prefactura”} sino una factura final es no escribir la denominación ni el número, ya que eso generalmente viene impreso en cada hoja del talonario.}. Esta opción implica la opción |sinfirmas|, ya que si es un documento preliminar sin validez legal, las firmas no son necesarias; sin embargo se puede cambiar este comportamiento con las opciones |dosfirmas| y |unafirma|.
%
% \DescribeOption{cotizacion}
%  La cotización es un documento similar a una factura pero no es un documento legal en el sentido de que no se imprime sobre talonario fiscal alguno, así que no lleva número de control y puede o no tener su propia numeración, que es cosa del emisor de la cotización. Esta opción implica la opción |unafirma|, para que firme el emisor de la factura, y si su nombre no se especifica saldrá “Por el emisor” debajo de la línea para firmar. También puede cambiarse esto con la opción |dosfirmas| o con |sinfirmas|.
%
% \DescribeOption{sinreverso}
%  No escribe la información -opcional- que se escribió dentro del entorno |reverso| a salir en el reverso de la factura.
%
% \DescribeOption{conreporte}
%  Genera la factura (prefactura o cotización) y luego el reporte.
%
% \DescribeOption{soloreporte}
%  Genera solamente el reporte, ignorando lo relacionado con la factura. Si no se especifica |conreporte| ni |soloreporte| se va a imprimir únicamente la factura.
%
%
% \subsection{Relativo a las columnas que puede tener el cuadro de la factura\label{subsec:columnas}}
%
% \DescribeOption{numitem}
%  Con la opción |numitem| una nueva columna aparece en el cuadro de la factura donde se enumera cada ítem en el órden en el que se escriben dentro del entorno para la factura. Esto es útil por ejemplo cuando se hace referencia al número del ítem con el comando |\itemref|. Véase la sección \ref{sec:etiquetas}.
%
% \DescribeOption{sincantidad}
%  El ambiente |factura| tiene también la posibilidad de no mostrar el campo de la cantidad, ni aceptarlo como argumento de cada ítem; en ese caso los comandos |\Item| -de los que hablaremos en la sección \ref{sec:factura}- aceptan como argumentos solamente la descripción del elemento y su precio, en ese orden.
%
% \DescribeOption{descuentos}
%  Permite agregar un descuento en porcentaje en cada elemento de la factura, siendo dicho porcentaje el último argumento de los comandos |\Item|. Si no aplica descuento para un ítem, puede este dejarse sin ese argumento, dando el mismo resultado que indicar “0” como argumento para ese descuento. (Explicado también en la sección \ref{subsec:descuentos}).
%
% \DescribeOption{subtotal}
%  Agrega una columna extra en la factura; sólo aplica cuando también se usa la opción |descuentos|. La nueva columna escribe el subtotal que cada ítem produce, es decir, el precio unitario multiplicado por la cantidad sin aplicar el descuento. Esta opción sobrecarga la factura (precio unitario, descuento, subtotal y precio total), y su uso es recomendable cuando también se usa |sincantidad|, porque de esta forma se tiene el subtotal, el descuento, y el total.
%
% \DescribeOption{iva}
%  Agrega una nueva columna donde se expresa el porcentaje del IVA para cada elemento en la factura. Sin especificar esta opción queda claro según lo establecido por nuestra legislación cuál será el gravamen que aplica para cada elemento por una letra que lo denota al lado del precio o de la descripción.
%
% \DescribeOption{duc}
%  Cambia el orden de los campos en el documento impreso a: descripción, precio unitario, cantidad y precio total. {\itshape Esto no cambia la forma como deben especificarse los argumentos de |\Item| sino solamente cambiará cómo serán mostrados}.
%
% \DescribeOption{dcu}
%  Cambia el orden de los campos en el documento impreso a: descripción, cantidad, precio unitario y precio total\footnote{Sin \texttt{duc} ni \texttt{dcu}, el modo predeterminado es \texttt{cdu}: cantidad, descripción, precio unitario.}.
%
%
% \subsection{Relativo al IVA}
%
% \DescribeOption{siniva}
%  No se cobra IVA, para los casos en los que eso aplique.
%
% \DescribeOption{sinivaexpresado}
%  Al no cobrar IVA, tampoco aparece en la totalización los IVA ni las bases imponibles.
%
% \DescribeOption{todosiva}
%  Agrega el monto exento y todos los tipos de IVA en los totales así no se hayan incluído elementos gravados con los distintos tipos de impuesto.
%
% \DescribeOption{electro}
%  {\bfseries Esta opción ya no está vigente en la legislación y se quitó de esta clase.} Se agregó en su momento esta opción para dar cumplimiento con las Gacetas Oficiales números 41239, 41242 y 41244\footnote{De fecha 19, 21 y 26 de septiembre del año 2017, respectivamente. En la última de estas gacetas está la Providencia Administrativa PA0048 del SENIAT.} donde se establece que las operaciones de compra (salvo excepciones establecidas en la Gaceta) gravadas con IVA general gozarán de un descuento cuando son pagadas mediante medios electrónicos. El descuento es del 3\% cuando la factura es de un monto menor a Bs.F 2.000.000 y del 5\% cuando el monto de la factura es de Bs.F 2.000.000 o más. Al dar esta opción el descuento queda establecido según el Decreto Presidencial número 3085.
%
% {\itshape Esta opción no es compatible con el modo de totalización en filas.}
%
% \DescribeOption{G}
%  Agrega una “|G|” al lado de todos los items gravados con IVA general así sea el general el único IVA existente en la factura.
%
% \DescribeOption{impdesc}
%  Con esta opción, la letra que indica la denominación del impuesto aplicado a cada elemento, en lugar de escribirse en el precio unitario y en el total, que es la forma predeterminada, sale escrito al final de la descripción. Puede preferirse de esta forma, y ahorra un poco de espacio usable en la descripción porque ya no sale en dos lugares dicha letra sino en uno solo.
%
%
% \subsection{Sobre las líneas en el cuadro de la factura}
%
% \DescribeOption{sinlineasve}
% \DescribeOption{sinlineasvi}
% \DescribeOption{sinlineasv}
% \DescribeOption{sinlineashe}
% \DescribeOption{sinlineashi}
% \DescribeOption{sinlineash}
%  La clase provee varias opciones para no dibujar las líneas del cuadro de la factura. La clase diferencia entre las líneas internas y las líneas externas. Las líneas externas son las que conforman el contorno externo del cuadro de la factura. Las líneas internas en el caso de las horizontales son la línea que va después del título de las columnas (cantidad, descripción, ...), la línea que separa los ítems de los totales y la línea que separa los totales de la nota interna; en el caso de las verticales, las líneas internas son las que separan las columnas.
%
%  Podemos apagar cada conjunto de línea; ya explicado qué es interno y qué es externo, queda aclarar que “ve” significa vertical externa, “vi” vertical interna, “he” horizontal externa y “hi” horizontal interna. Usando |sinlineasv| o |sinlineash| apagamos todas las líneas verticales u horizontales.
%
% \DescribeOption{sinlineahtit}
% \DescribeOption{sinlineahtot}
% \DescribeOption{sinlineahni}
%  Adicionalmente podemos apagar las líneas horizontales internas por separado, es decir, la línea que va después del título de las columnas la podemos quitar con \texttt{sinlineahtit}, la línea que separa los elementos a facturar de los totales generales se puede quitar con \texttt{sinlineahtot} y la línea que separa los totales de la -opcional- nota interna la podemos quitar con \texttt{sinlineahni}. 
% 
% \DescribeOption{conlineasha}
%  Agrega una línea horizontal separando cada artículo o ítem de los demás dentro del cuadro de la factura, como las facturas hechas a mano que tienen un espacio determinado para cada artículo.
%
%  Si no se especifica nada sobre las líneas, saldrán todas ellas (verticales y horizontales) salvo las que separan  los ítems a facturar entre sí.
%
%
% \subsection{Elementos a mostrar o no mostrar en la factura}
%
% \DescribeOption{sinmarcas}
%  Con |sinmarcas| evitamos que el carácter de fin de línea aparezca cuando la descripción ocupa más que una línea; ese carácter indica que la línea continúa abajo, y predeterminadamente es la flecha “\flechaCR”. También con esta opción no aparece el carácter “$\cdot$” en las celdas vacías en este mismo caso.
%
% \DescribeOption{sinmonedatotales}
%  Esta opción |sinmonedatotales| permite quitar el signo monetario de los totales, que antes nunca salía y ahora sale de manera predeterminada.
%
% \DescribeOption{unafirma}
%  No pone el espacio para que firme quien recibe el documento sino sólo el emisor; esto es lo predeterminado en las cotizaciones.
%
% \DescribeOption{sinfirmas}
%  No pone ninguna línea ni espacio para firmar; esto es lo predeterminado para las prefacturas.
%
% \DescribeOption{dosfirmas}
% Coloca las dos líneas para que firme quien emite y quien recibe. Esta opción es la predeterminada salvo en las prefacturas y en las cotizaciones.
%
% \DescribeOption{vencimiento}
%  Cuando establecemos los días del crédito de la factura, se imprime una línea también con la fecha de vencimiento de la factura. Con esta opción podemos pedir que la fecha de vencimiento salga siempre en la facturas, inclusive cuando no sean facturas a crédito.
%
% \DescribeOption{denom}
%  Con esta opción escribimos también el texto “FACTURA” con el número. La razón de esta opción es que hay formas libres que tienen la numeración de la factura y otras que solamente tienen el número de control; para las que tienen sólo el número de control existe esta opción que también escribe el número en cada factura impresa, en la posición que se defina.
%
% \DescribeOption{sinnumero}
%  Escribe el texto de la denominación del documento en su posición (e.g. “COTIZACIÓN”) pero sin escribir el número; por ejemplo para las cotizaciones que no son documentos que legalmente lleven número, aunque el número puede ser para control interno. Esta opción implica la opción anterior |denom|.
%
% \DescribeOption{sintotales}
%  Con esta opción evitamos que se totalicen los elementos en la factura; sólo se presenta el cuadro sin los totales. La utilidad de esto no es en las facturas sino en las cotizaciones para casos donde se muestran variantes del mismo producto para que el cliente elija entre ellos. No tendría sentido totalizar allí sino que funcione como un listado de precios. {\itshape Debe usarse en las cotizaciones porque una factura sin totales no es válida.}
%
% \DescribeOption{linea}
%  Agrega una línea diagonal que inutiliza el espacio vacío en el cuadro de la factura, para evitar posibles futuras adiciones de texto. Esta opción requiere de la opción |expandecuadro|, explicada a continuación.
%
%
% \subsection{Disposición de algunos elementos en la factura}
%
% \DescribeOption{expandecuadro}
%  Expande el cuadro de la factura con espacio en blanco, es decir, rellena el espacio restante del cuadro de la factura con espacio en blanco. Lo predeterminado es expandir la parte de la descripción larga cuando la hay.
%
% \DescribeOption{sinexpandir}
%  No expande ninguna parte en la factura, es decir, deja tanto el cuadro de la factura como la descripción en su tamaño mínimo según el contenido que tenga.
%
% \DescribeOption{centradesc}
%  Podemos centrar el texto de la descripción de cada ítem en la factura con este comando en lugar de justificarlo.
%
% \DescribeOption{filas}
%  Muesta los totales finales en modo de filas; esto los presenta uno al lado del otro en lugar de uno debajo del otro. Esta opción tiene limitaciones, por ejemplo no es compatible con la opción |sintotales|, |sinivaexpresado|, y por razones de espacio tampoco es compatible con los comandos de descuentos en los totales (subsección \ref{subsec:dctototales}).
%
% \DescribeOption{centrado}
%  Centra los totales finales ({\itshape sólo para el modo de totalización en filas}), que sin esta opción salen alineados al margen derecho.
%
% \DescribeOption{agrupado}
%  No agrupa los totales finales de los items por gravamen sino por tipo ({\itshape sólo para el modo de totalización en columnas}). Con esta opción la agrupación es primero todos los subtotales y los descuentos (cuando aplica), luego las bases imponibles y después los IVA. Sin |agrupado| se ordena por porcentaje del gravamen.
%
%
% \subsection{Otras opciones}
%
% \DescribeOption{nospanish}
%  No carga |babel| en castellano\footnote{Por ejemplo para cargar {\ttfamily polyglossia} en su lugar.}, ni la codificación |T1| de las letras en el caso de compilar con pdf\LaTeX. Si no se especifica esta opción se cargan ambos paquetes.
%
% \DescribeOption{twoside}
%  Establece que el reporte se imprime por los dos lados del papel. (Obviamente se ignora esta opción cuando el motor no va a generar un reporte puesto que las facturas ocupan una hoja solamente).
%
% \DescribeOption{csv}
%  Escribe un archivo csv con la información numérica de la factura.
%
% \subsection{Activación y desactivación de opciones desde comandos\label{subsec:condicionales}}
%
% Cualquiera de estas opciones también se puede activar o desactivar en el preámbulo y así poder incluirlas dentro de archivos que vayan a ser cargados con |\input| o |\include| como se explicará en la subsección \ref{subsec:aux}. Esto significa que podremos desactivar una opción que hayamos dado en |\documentclass| o activar alguna que no hayamos activado allí.
%
% Para hacer eso debemos usar el condicional que cada opción maneja y activarlo o desactivarlo de la forma como se hace en \TeX{}. El condicional de cada opción tiene el mismo nombre de la opción. Por ejemplo, para quitar las líneas verticales (|sinlineasv|) usaríamos |\sinlineasvtrue| en el preámbulo, o para desactivar la creación del archivo |.csv|, previamente activada en |\documentclass|, usaríamos |\csvfalse|.
%
% }
%
%
% \section{Comandos que la clase provee\label{sec:comandos}}
%
% A continuación descriremos los principales comandos de esta clase; en rojo aparecen aquellos pocos comandos que son de obligatorio uso. Los que no son de obligatorio uso es porque tienen algún valor predeterminado en |factura.def| o porque controlan algún aspecto opcional.
%
%
% \subsection{Estableciendo los márgenes de la hoja y el fondo}
%
%  Como explicamos en la sección \ref{sec:geohoja}, a la hoja de la factura se le deben establecer los márgenes apropiadamente, y esos márgenes no tienen en cuanta ciertos textos que llamamos “fijos”, para los cuales también debemos establecer sus posiciones en la hoja, y eso lo veremos en esta parte.
%
%
% {\setlength{\parindent}{0pt}
%
% \DescribeMacro{\GeometriaFactura}{\macrooptcolor|\GeometriaFactura|\marg{geometría}}
%
%  En el argumento de este comando van los parámetros relativos al paquete |geometry| que van a afectar la manera como la factura es generada. Como dijimos al principio, deberá especificarse el tipo de papel (e.g. |letterpaper|, |legalpaper|, |executive|, |a4paper|, ...) y los márgenes verticales y horizontales.
%
%  Para especificar el margen superior debe medirse desde el inicio del papel hasta donde se quiera establecer que empiece el recuadro de los datos del cliente, y el margen inferior debe medirse desde el final del papel hasta donde terminaría la opcional descripción. Como ya dijimos, hay textos que están fuera de estos márgenes por haberlos considerado fijos en la implementación de la clase |factura|.
%
% \DescribeMacro{\GeometriaReverso}{\macrooptcolor|\GeometriaReverso|\marg{geometría}}
%
%  Define los márgenes del reverso de la hoja de la factura para escribir información extra.
%
% \DescribeMacro{\FondoFactura}{\macrooptcolor|\FondoFactura|\marg{nombre del archivo de la imagen}}
%
%  Nombre del archivo de la imagen a cargar como diseño, especialmente para las {\slshape prefacturas}, aunque también puede usarse como una “marca de agua” para las facturas.
%
%  Lo que recomendamos es digitalizar la forma libre o pedir a la imprenta un archivo digital con el diseño y así incluir esa imagen con este comando cuando se vayan a producir “prefacturas” o cotizaciones, así las prefacturas tienen el mismo diseño que las facturas. Si se puede lograr el diseño en un código como se hace con el ejemplo del fondo, pues mejor aún.
%
%
% }
%
%
% \subsubsection{Elementos fijos y sus ubicaciones}
%
% Los elementos cuyas posiciones van fijas en la hoja disponen de macros para establecer sus ubicaciones. Todas estas macros empiezan por |\Pos|, y son las siguientes: |\PosFecha|, |\PosDenom|, |\PosControl|, |\PosNotaFinal| y |\PosFirmas|.
%
% Todos estos comandos aceptan como argumentos dimensiones de \LaTeX{}, y son las posiciones |(x,y)| en la página de los elementos a imprimir, siendo “|x|” la coordenada a lo ancho del papel y “|y|” la coordenada a lo largo del papel, es decir, horizontal y vertical respectivamente. El origen está fijado arriba a la izquierda. Se pueden especificar también valores negativos para denotar que estamos contando desde el extremo derecho o inferior, para el caso de “|x|” o “|y|”, respectivamente\footnote{Varias de estas variables están definidas de esta forma negativa en el archivo de definiciones.}; salvo el caso de la nota final y las firmas en su coordenada |x|, todos los demás elementos fijos aceptan valores negativos en su posicionamiento, ya que estos dos elementos cuentan con un intervalo |x-inicial,x-final| para posicionar.
%
% Por ejemplo, si queremos que la fecha aparezca desde el margen izquierdo y |5cm| contados desde arriba hacia abajo, haríamos: |\PosFecha{0cm}{5cm},| y si queremos que salga |1cm| desde la derecha y |4cm| contados desde abajo haríamos |\PosFecha{-1cm}{-4cm}|. A continuación todas las macros para fijar las posiciones en el papel:
%
% {\setlength{\parindent}{0pt}
%
% \DescribeMacro{\PosFecha}{\macrooptcolor|\PosFecha|\marg{pos. x}\marg{pos. y}}
%
%  Con |\PosFecha| controlamos dónde se ubica la caja que imprime la fecha en la hoja.
%
% \DescribeMacro{\PosFecha*}{\macrooptcolor|\PosFecha*|\marg{pos. x}\marg{pos. y}}
%
%  Con esta variante con “*”, es decir, |\PosFecha*|, logramos que la caja de la fecha abra hacia arriba desde la posición especificada en el eje vertical, de manera que estaríamos especificando una coordenada “|y|” final del texto y no una posición inicial, esto significa que el texto se despliega hacia arriba desde la posición dada y no hacia abajo. Por ejemplo, sin “*”, si especificamos |\PosFecha{-0cm}{5cm}|, las fechas empiezan 5cm desde el tope superior de la hoja, y si especificamos |\PosFecha*{-0cm}{5cm}| ellas terminarán en los 5cm de la hoja. Esto es útil cuando se especifica también |\Credito| que produce una fecha de vencimiento que aparecerá abajo de la fecha de emisión; al ocupar las fechas más de una línea podemos querer que la última fecha (la de vencimiento) quede alineada en cierta posición vertical, la misma posición donde saldría la fecha de emisión en el caso en el que no haya fecha de vencimiento por no especificar la factura a crédito.
%
% \DescribeMacro{\PosDenom}{\macrooptcolor|\PosDenom|\marg{pos. x}\marg{pos. y}}
%
%  Con |\PosDenom| controlamos dónde se ubica la caja que imprime la denominación del documento (e.g. “FACTURA”, “COTIZACIÓN”) y su -opcional- número.
%
% \DescribeMacro{\PosDenom*}{\macrooptcolor|\PosDenom*|\marg{pos. x}\marg{pos. y}}
%
%  Al igual que con |\PosFecha*| podemos especificar la posición final en “|y|” de la caja para la denominación, así el texto se despliega hacia arriba y no hacia abajo.
%
% \DescribeMacro{\PosControl}{\macrooptcolor|\PosControl|\marg{pos. x}\marg{pos. y}}
%
%  Con |\PosControl| controlamos dónde se ubica la caja que imprime el número de control con la serie.
%
% \DescribeMacro{\PosControl*}{\macrooptcolor|\PosControl*|\marg{pos. x}\marg{pos. y}}
%
%  También podemos pedir que la caja del número de control abra hacia arriba, es decir que la posición vertical sea tomada como la posición vertical final; para eso |\PosControl*|, igual que con |\PosFecha*| y |\PosDenom*|. 
%
% \DescribeMacro{\PosNotaFinal}{\macrooptcolor|\PosNotaFinal|\marg{pos. x-inicial}\marg{pos x-final}\marg{pos. y}}
%
%  Con |\PosNotaFinal| controlamos dónde se ubica el texto de la nota final\footnote{La nota final es lo que en versiones previas llamábamos “condiciones” de la factura} (está hecho con |\parbox|, acepta un párrafo como argumento) en la hoja.
%
% {\itshape Aquí delimitamos el intervalo en |X| donde va la nota final, por eso hay una coordenada |X-inicial| y una coordenada |X-final|.} La coordenada |X| inicial va medida desde el margen establecido como el izquierdo, y la |X| final va medida desde el margen establecido como el derecho. Por ejemplo, estableciendo |X-inicial| como |0pt| y |X-final| como |0pt| queda en los mismos márgenes que todos los demás elementos de la factura, y esto es lo predeterminado, pero estableciendo ambas coordenadas como |1cm|, la nota final quedará con |1cm| de margen hacia adentro que la factura, tanto por la izquierda como por la derecha.
%
% Debido a que este comando acepta valor inicial y valor final en |X|, entonces no acepta valores negativos.
%
% \DescribeMacro{\PosNotaFinal*}{\macrooptcolor|\PosNotaFinal*|\marg{pos. x-inicial}\marg{pos x-final}\marg{pos. y}}
%
%  De la misma manera como con los comandos precedentes, podemos especificar que la caja del párrafo para la nota final abra hacia arriba, fijando una posición vertical final y no una posición inicial de la nota final.
%
% \DescribeMacro{\PosFirmas}{\macrooptcolor|\PosFirmas|\marg{pos. x-inicial}\marg{pos. x-final}\marg{pos. y}}
%
%  Con |\PosFirmas| controlamos dónde se ubica la caja que dibuja las líneas para firmar en la factura. {\itshape Aquí aplica el mismo comentario que en \cs{PosNotaFinal} sobre las coordenadas |X| inicial y final}. La coordenada |X-inicial| indica dónde empieza la primera linea (para que firme quien recibe), y la coordenada |X-final| establece dónde termina la segunda linea para firmar (para que firme el emisor de la factura), que va a estar al lado de la primera.
%
% En este caso, las coordenadas |X| tampoco aceptan valores negativos ni versión con asterisco |*|.
%
% }
%
%
% \subsubsection{Espaciado entre distintas partes. Estilo de los recuadros\label{subsec:espacioslineas}}
%
% Podemos cambiar distintos valores del espaciado interno en la factura, así como también la estética de la línea para firmar y los recuadros que encierran los datos del cliente, la descripción, las “notas”, pudiendo ajustarle estos recuadros o cajas el grosor de la línea, la curvatura y la sombra. Los argumentos de todos estos comandos son dimensiones de \LaTeX{}:
%
% {\setlength{\parindent}{0pt}
%
% \DescribeMacro{\SepVertical}{\macrooptcolor|\SepVertical|\marg{sep. vertical entre elementos}}
%
%  Las distintas partes de la factura (a saber: el encabezado con los datos, la -opcional- nota previa, el cuadro de la factura, la -opcional- nota externa y la -opcional- descripción) también están separadas entre sí una distancia que puede ser cambiada.
%
%  La separación que se fija con este comando es la separación vertical mínima que queremos garantizar entre cada elemento a escribir en la factura, y puede ser ajustada automáticamente por la clase aumentándola cuando no hay relleno automático de espacios sobrantes (es decir, sin usar la opción |sinexpandir|). Cuando la descripción larga o el cuadro de la factura (usando la opción |expandecuadro|) toman su tamaño máximo, entonces la separación vertical entre cada elemento de la hoja es la dada con |\SepVertical|, o sea, la que queremos garantizar como la mínima separación entre esas partes.
%
% \DescribeMacro{\SepDatosResumen}{\macrooptcolor|\SepDatosResumen|\marg{sep. horizontal}}
%
%  Esta macro nos proporciona una forma de cambiar la separación entre los recuadros o cajas donde van los datos del cliente y el resumen que sale a su lado. Naturalmente, si no se escribe un resumen saldría únicamente la caja de los datos del cliente, y esta separación no tendría efecto.
%
% \DescribeMacro{\SepEncabezado}
% \DescribeMacro{\SepNotaPrevia}
% \DescribeMacro{\SepDescripcion}
% \DescribeMacro{\SepNotaFinal}
% {\macrooptcolor|\SepEncabezado|\marg{sep. interna texto}}\newline
% {\macrooptcolor|\SepNotaPrevia|\marg{sep. interna texto}}\newline
% {\macrooptcolor|\SepDescripcion|\marg{sep. interna texto}}\newline
% {\macrooptcolor|\SepNotaFinal|\marg{sep. interna texto}}
%
%  Con estas macros podemos controlar la distancia entre la línea del recuadro y el texto que hay en el recuadro, es decir, un margen interno entre el contorno de la caja y el texto para que no salga pegado a la línea que lo encierra. Esto es parte del diseño de \tikz de estos recuadros. Como puede verse, dicho margen para los datos del cliente y para el resumen se cambia con |\SepEncabezado|, los de la nota previa con |\SepNotaPrevia|, los de la descripción larga con |\SepDescripcion|, y los de la nota final con |\SepNotaFinal|.
%
% Predeterminadamente no hay línea ni diseño de \tikz para la nota final del documento, así que el valor fijado aquí no tiene efecto a menos de que se le especifiquen valores a dicho recuadro con la opción |\LineaNotaFinal|, explicada más abajo.
%
% \DescribeMacro{\SepItemsExtra}{\macrooptcolor|\SepItemsExtra|\marg{sep. extra entre items}}
%
%  Agrega una separación extra entre los elementos o ítems en el cuadro de la factura. La separación entre ítems está dada por el |\baselineskip| que se puede establecer con el comando |\LetraItems|, el cual se explicará más adelante en esta misma sección.
%
% \DescribeMacro{\SepFilas}{\macrooptcolor|\SepFilas|\marg{sep. en x}\marg{sep. en y}}
%
%  Cuando se usa el modo de totalización en filas tenemos también la posibilidad de cambiar la separación entre los totales que salen uno al lado del otro, y la separación entre las distintas líneas.
%
%  El primer argumento de |\SepFilas| es la separación horizontal {\itshape mínima} entre los distintos totales que se presentan en el modo de totalización en filas. La clase |factura| expande esa distancia para que queden todos los totales separados igual, y también separados igual de los márgenes del papel.
%
%  El segundo argumento de |\SepFilas| es la separación vertical entre las distintas líneas de la totalización en filas.
%
% \DescribeMacro{\LineaEncabezado}
% \DescribeMacro{\LineaNotaPrevia}
% \DescribeMacro{\LineaDescripcion}
% \DescribeMacro{\LineaNotaFinal}
% {\macrooptcolor|\LineaEncabezado|\marg{grosor}\marg{curvatura esquinas}\marg{tamaño sombra}}\newline
% {\macrooptcolor|\LineaNotaPrevia|\marg{grosor}\marg{curvatura esquinas}\marg{tamaño sombra}}\newline
% {\macrooptcolor|\LineaDescripcion|\marg{grosor}\marg{curvatura esquinas}\marg{tamaño sombra}}\newline
% {\macrooptcolor|\LineaNotaFinal|\marg{grosor}\marg{curvatura esquinas}\marg{tamaño sombra}}
%
%  La información del encabezado (datos del cliente y resumen), así como la nota previa, la descripción y la nota final puede o no tener el diseño \tikz, el cual viene prehecho en el archivo |.def| y consiste en una línea curva, en algunos casos sombreada a la derecha y abajo, y en otros casos no. Esa línea puede ser alterada en su grosor, el tamaño de la sombra, y la distancia entre la línea y la información, es decir un margen interno. Para eso estos comandos, que se comportan igual todos ellos.
%
%  Con el primer argumento de estos comandos cambiamos el grosor de la línea\footnote{Si se especifica una dimensión nula, desaparece el diseño \tikz y la línea, quedando solamente un \cs{parbox}, y es así como se estableció en el archivo \texttt{.def} para el caso de la nota final.} con la que las cajas del encabezado, la nota previa, la descripción y la nota final se dibujan. Con el segundo argumento cambiamos el radio de curvatura de las esquinas, y con el tercero el tamaño de la sombra\footnote{Todas las sombras salen hacia abajo a la derecha.}.
%
%  Se puede modificar el diseño de este nodo de \tikz; está alojado en las macros internas |\@DatosTIKZ|, |\@NotaPreviaTIKZ|, |\@DescripcionTIKZ| y |\@NotaFinalTIKZ|.
%
%  Lo predeterminado es que la nota final se escriban sin diseño de \tikz, es decir, el primer argumento de este comando está fijado en cero para que no salga línea ni diseño de \tikz.
%
% \DescribeMacro{\LineaFirmas}{\macrooptcolor|\LineaFirmas|\marg{long. línea}\marg{grosor}}
%
%  La longitud de las líneas para firmar y el grosor son también ajustables por medio del primer y segundo argumento, respectivamente, de la macro |\LineaFirmas|.
%
%
% \subsection{Sobre la información del cliente\label{subsec:cliente}}
%
% \DescribeMacro{\RazonSocial}{\macromandcolor|\RazonSocial|\marg{texto}}
%
%  Define la razón social, o sea, el nombre del cliente.
%
% \DescribeMacro{\Nombre}{\macromandcolor|\Nombre|\marg{texto}}
%
%  Similar a |\RazonSocial| pero usable más bien para personas naturales, donde no existe una “persona de contacto” y por tanto la clase no advierte de su ausencia si no se ha especificado con |\Contacto|. Naturalmente se especifica un solo nombre del cliente, es decir que si se especifica con |\RazonSocial| no hace falta usar este comando. {\itshape Es preferible usar |\RazonSocial|.}
%
% \DescribeMacro{\RIF}{\macromandcolor|\RIF|\marg{texto}}
%
%  Define el R.I.F. del cliente.
%
% \DescribeMacro{\CI}{\macromandcolor|\CI|\marg{texto}}
%
%  Especifica la cédula de identidad del cliente al ser persona natural. {\itshape Es preferible usar siempre |\RIF|.}
%
% \DescribeMacro{\Contacto}{\macrooptcolor|\Contacto|\marg{texto}}
%
%  Define una persona de contacto (e.g. nombre de una persona cuando el cliente no es personal).
%
% \DescribeMacro{\Direccion}{\macromandcolor|\Direccion|\marg{texto}}
%
%  Define la dirección fiscal del cliente.
%
% \DescribeMacro{\Telefono}{\macrooptcolor|\Telefono|\marg{texto}}
%
%  Especifica el (o los) teléfonos del cliente.
%
% \DescribeMacro{\Email}{\macrooptcolor|\Email|\marg{texto}}
%
%  Establece el e-mail del cliente.
%
% \DescribeMacro{\Conforme}{\macrooptcolor|\Conforme|\marg{texto}}
%
%  Especifica el texto que aparecerá debajo de la línea para que firme el cliente; si no se especifica tomará el texto por omisión, que es “Recibí conforme”, indicando que el cliente le firma una copia al emisor de la factura donde acepta haber recibido y estar conforme. Este texto puede cambiarse a otro que puede ser el nombre del cliente o la persona de contacto, o también “Por el cliente”.
%
%
% \subsection{Sobre la información del emisor del documento}
%
% \DescribeMacro{\Emisor}{\macrooptcolor|\Emisor|\marg{texto}}
%
%  Define el nombre del emisor de la factura. Si no se especifica saldrá el texto “PAGADO”\footnote{El texto “PAGADO” es lo que la legislación venezolana del SENIAT exige escribir cuando una factura ha sido pagada.} debajo de la línea para firmar, ya que la factura se firma cuando haya sido pagada. Si se especifica un texto con este comando saldrá ese texto debajo de la línea donde va la firma, pero el propósito es que quede constancia para el cliente que ya pagó la factura, razón por la cual el texto predeterminado es “PAGADO”.
%
% \DescribeMacro{\Proveedor}{\macrooptcolor|\Proveedor|\marg{texto}}
%
%  Establece el nombre del emisor, para los efectos del reporte. Si no se define mediante esta macro, será tomado del establecido con |\Emisor|.
%
% \DescribeMacro{\FirmaFactura}{\macrooptcolor|\FirmaFactura|\marg{inclusión de la imagen}\oarg{corrimiento en x}\oarg{corrimiento en y}}
%
%  Incluye el archivo de la firma digitalizada en el espacio donde firma el emisor del documento. Los argumentos son:
%  \begin{enumerate}
%   \item inclusión de la imagen (e.g. |\includegraphics|). {\bfseries Aquí el primer argumento no es el archivo de la imagen sino el comando que la inserta en el documento}, así el usuario tiene más control sobre cómo saldrá la imagen si altera las opciones de |\includegraphics|.
%   \item corrimiento en |X| de la imagen, para un ajuste más fino de la posición de la firma,
%   \item corrimiento en |Y|.
%   \end{enumerate}
%  {\itshape (Sin especificar los corrimientos, la firma quedará centrada horizontalmente en la línea, dejando hacia la izquierda y hacia la derecha la misma cantidad de la línea. También quedará toda la firma encima de la línea, lo cual muy probablemente haya que corregirlo porque las firmas tienen “profundidad” de texto, que significa que hay parte de algunas letras que sobresalen hacia abajo respecto a la línea base del texto. Por todo esto es muy probable el uso de los corrimientos. Pueden verse los ejemplos, que allí se usan los corrimientos.)}
%
%  {\bfseries Las facturas no llevan firma porque se supone que el documento es auténtico ya que viene avalado por una imprenta registrada y tiene los números correlativos impresos desde la imprenta, de manera que la factura no necesita la firma de quien la emite; razón por la cual el texto debajo de la línea del emisor en la factura dice “PAGADO”, porque el emisor firma la factura una vez que haya sido pagada como constancia para el cliente.} Entonces agregar la firma escaneada es una posibilidad que debe contemplarse solamente con las cotizaciones, usando también el comando |\Emisor|, ya que no nos interesaría que apareciera “PAGADO” en ese caso sino algún nombre de quien está emitiendo la cotización.
%
%
% \subsection{Relativos a la moneda y a la numeración del documento factura}
%
% \DescribeMacro{\Moneda}{\macrooptcolor|\Moneda|\marg{signo monetario}}
%
%  Cambia el signo monetario, por defecto “Bs.”.
%
% \DescribeMacro{\Num}{\macrooptcolor|\Num|\marg{núm}}
%
%  Define el número de la factura. Es opcional porque el talonario de las facturas puede ya tener la numeración. Es de obligatorio uso cuando se define |denom|, o |prefactura| sin usar |sinnumero|.
%
% \DescribeMacro{\NumControl}{\macrooptcolor|\NumControl|\marg{núm}}
%
%  Define el número de control de la factura. Esta opción sólo aplica con |prefactura| puesto que una factura fiscal siempre tiene el número de control impreso desde la imprenta. Si se usa |prefactura| y no se define el número de control, éste tomará el mismo número de la factura.
%
% \DescribeMacro{\NumSerieControl}{\macrooptcolor|\NumSerieControl|\marg{núm}}
%
%  Define el número de serie del número de control, también aplica solamente con |prefactura|. Si no se especifica, toma |00|.
%
% \DescribeMacro{\Denom}{\macrooptcolor|\Denom|\marg{texto}}
%
%  Cambia el nombre del documento a algo distinto de “FACTURA” o “COTIZACIÓN”.
%
%
% \subsection{Estableciendo la fecha del documento}
%
% \DescribeMacro{\Fecha}{\macrooptcolor|\Fecha|\marg{fecha}}
%
%  Especifica la fecha de emisión del documento, en un formato que puede cambiarse con el comando |\FormatoFecha| pero predeterminadamente es el formato venezolano |dd-mm-aaaa|.
%
%  La fecha de emisión se guardará en el identificador “\texttt{FechaEmision}” perteneciente al paquete \texttt{datetime2} y podrá ser usado en cualquier parte del documento según las formas que provee dicho paquete.
%
%  {\itshape Si el usuario no provee la fecha, será tomada la fecha actual como la fecha de emisión y se producirá una advertencia indicando que no se especificó la fecha y que la fecha del documento es la fecha del día.}
% 
% \DescribeMacro{\Fecha*}{\macrooptcolor|\Fecha*|\marg{aaaa}\marg{mm}\marg{dd}}
%
%  Esta variante del comando se incorporó para suministrar la fecha en formato ISO, es decir, primero el año, luego y mes y finalmente el día, cada uno como un parámetro separado del otro (como se hace en \texttt{datetime2}. Esto es independiente de la forma como la fecha será mostrada, o sea que puede la fecha suministrarse de esta forma e igual saldrá en el formato deseado, cambiable con la macro |\FormatoFecha|, explicado a continuación.
%
% \DescribeMacro{\FormatoFecha}{\macrooptcolor|\FormatoFecha|\oarg{formato entrada}\marg{formato salida}\marg{separador}}
%
%  El comando |\FormatoFecha| le permite al usuario cambiar el formato de entrada de la fecha suministrada y el formato de salida de las fechas (cómo éstas salen escritas en la factura).
%
%  Con el primer argumento (opcional, dado entre corchetes) el usuario le puede decir a la clase cúal es el formato según el cual le suministrará la fecha con |\Fecha|; las posibilidades son: \begin{inparaenum}[\bfseries i)]\item |dd-mm-aaaa| (predeterminada), \item |dd.mm.aaaa|, \item |ddmmaaaa|, \item |aaaa-mm-dd|, \item |aaaa.mm.dd|, \item |aaaammdd| y \item |mm-dd-aaaa| \end{inparaenum}. Esto con el fin de establecer la fecha reconocida debidamente por |datetime2| y que sea posible su manejo como fecha y el posible cálculo de la fecha de vencimiento, si aplicara.
%
%  El segundo argumento (opcional también, dado entre llaves) le permite al usuario especificar el formato según el cual quiere que las fechas (emisión y vencimiento si aplica) salgan impresas en la factura. Estos son los formatos dados por \texttt{datetime2} y debe consultarse su documentación para más información. Entre ellos tenemos por ejemplo \begin{inparaenum}[\bfseries i)]\item |ddmmyyyy| (predeterminado), \item |default|, \item |iso|, \item |mmddyyyy| \end{inparaenum}.
%
%  El tercer argumento (opcional también, dado entre llaves) sirve para cambiar el separador, cuyo carácter predeterminado es el guión (“-”).
%
%  Con |\FormatoFecha|, al tener todos sus argumentos opcionales, el usuario podrá especificar solamente el primero (que va entre corchetes), solamente el segundo (entre llaves), pero para especificar el tercero que también está entre llaves debe especificar el segundo, ya que ambos son opcionales pero se dan igualmente entre llaves. Naturalmente podrá especificar dos de estos argumentos o los tres.
%
%  Por ejemplo: |\FormatoFecha[mm-dd-aaaa]{mmddyyyy}{.}| serviría para que el usuario pueda dar la fecha así: |\Fecha{12-31-2019}| y que salga igualmente según el formato |mm-dd-aaaa| pero separado por puntos. También puede usarse |\FormatoFecha[aaaa-mm-dd]| para dar luego la fecha de esta forma: |\Fecha{2019-12-31}| pero esta saldrá igual impresa en la factura así: “|31-12-2019|”, ya que no cambiamos el formato de salida. También puede hacer |\FormatoFecha{mmddyyyy}{/}| y dar la fecha por ejemplo así: |\Fecha*{2019}{12}{31}| y saldrá impresa así: “|12/31/2019|”.
%
%  Los valores predeterminados cambiables con |\FormatoFecha| están definidos, como siempre, en el archivo |factura.def|.
%
% \DescribeMacro{\Credito}{\macrooptcolor|\Credito|\marg{días}}
%
%  Este comando, cuyo argumento es la cantidad de días, nos permite emitir una factura a crédito. La fecha de vencimiento será calculada automáticamente y será colocada debajo de la fecha de emisión, especificada por el usuario.
%
%  La fecha calculada automáticamente se guarda en el identificador “\texttt{FechaVencimiento}” perteneciente al paquete \texttt{datetime2}, de manera que el usuario podrá hacer uso de esta fecha con los comandos que este paquete provee.
%
% }
%
% 
% \subsection{Descuentos en la totalización\label{subsec:dctototales}}
%
%  Podemos efectuar un descuento\footnote{Estas opciones de descuento en el subtotal no son compatibles con la opción de totalización en modo de |filas|, explicada en la sección \ref{sec:opciones}, y muy probablemente más adelante seguirán sin serlo.} sobre cualquiera de los subtotales; ese descuento puede ser porcentual (versión sin *) o dado en un monto fijo a sustraer del subtotal (variante con *). Podemos aplicar un descuento que aplique por igual a cualquier tipo de ítem con cualquier gravamen, o discriminado dependiendo del tipo de impuesto se aplica un descuento u otro.
%
% {\setlength{\parindent}{0pt}
%
% \DescribeMacro{\Descuento}{\macrooptcolor|\Descuento|\marg{porcentaje}}\\
% \DescribeMacro{\Descuento*}{\macrooptcolor|\Descuento*|\marg{monto}}
%
% Con este comando el descuento aplicaría por igual a cualquier subtotal presente, es decir, para elementos con cualquier gravamen y también exentos. Dichos descuentos son aplicados y mostrados al totalizar. Puede usarse inclusive cuando se aplican descuentos individuales que se incluyen con la opción de la clase |descuentos|.
%
% \DescribeMacro{\DescuentoG}{\macrooptcolor|\DescuentoG|\marg{porcentaje}}\\
% \DescribeMacro{\DescuentoG*}{\macrooptcolor|\DescuentoG*|\marg{monto}}
%
%  Descuento específico para los elementos gravados con IVA general.
%
% \DescribeMacro{\DescuentoR}{\macrooptcolor|\DescuentoR|\marg{porcentaje}}\\
% \DescribeMacro{\DescuentoR*}{\macrooptcolor|\DescuentoR*|\marg{monto}}
%
%  Descuento específico para los elementos gravados con IVA reducido.
%
% \DescribeMacro{\DescuentoA}{\macrooptcolor|\DescuentoA|\marg{porcentaje}}\\
% \DescribeMacro{\DescuentoA*}{\macrooptcolor|\DescuentoA*|\marg{monto}}
%
%  Descuento específico para los elementos gravados con IVA adicional.
%
% \DescribeMacro{\DescuentoE}{\macrooptcolor|\DescuentoE|\marg{porcentaje}}\\
% \DescribeMacro{\DescuentoE*}{\macrooptcolor|\DescuentoE*|\marg{monto}}
%
%  Descuento específico para los elementos exentos de IVA.
%
%
% \subsection{Textos explicativos en la factura\label{subsec:textos}}
%
% \DescribeMacro{\Resumen}{\macrooptcolor|\Resumen|\marg{texto}}
%
%  Pone un resumen que va en una caja a la derecha de los datos del cliente, separado de esta en una distancia ajustable con el comando |\SepEncabezado|, que se describirá más adelante. Este resumen es opcional.
%
% \DescribeMacro{\NotaPrevia}{\macrooptcolor|\NotaPrevia|\marg{texto}}
%
%  La nota previa es una nota que puede ponerse antes del cuadro de la factura. Sin especificarla no sale nota alguna. Dicha nota va en un recuadro \tikz cuyo aspecto puede cambiarse, como ya se mencionó en la sección \ref{subsec:espacioslineas} donde se explican las líneas o contornos.
%
% \DescribeMacro{\NotaInterna}{\macrooptcolor|\NotaInterna|\marg{texto}}
%
%  La nota interna es una coletilla que aparece dentro del cuadro de la factura, por eso la llamamos interna (es interna al cuadro), por ejemplo si aplicamos un descuento podemos decir allí cerca de los totales a qué corresponde tal descuento, o para hacer alguna aclaratoria sobre lo que precede inmediatamente como algún ítem de la factura.
%
% \DescribeMacro{\NotaExterna}{\macrooptcolor|\NotaExterna|\marg{texto}}
%
%  La nota externa es otro texto opcional que el usuario puede introducir y saldrá al terminar el cuadro de la factura, antes de la opcional descripción larga (comando |\Descripcion|). {\itshape La nota externa no tiene formato \tikz.}
%
% \DescribeMacro{\NotaFinal}{\macrooptcolor|\NotaFinal|\marg{texto}}
%
%  Con este comando se establece el texto para la nota final\footnote{Antiguas condiciones de la factura, por ejemplo, para las condiciones de pago si se trata de una factura, o para las condiciones de una cotización.}. Es opcional; si no se define no sale texto alguno. También se puede destinar este espacio para un diseño (que haría el usuario) en el que se agreguen también algunas casillas para especificar la forma de pago.
%
%
% \subsection{Estableciendo los parámetros relativos al reporte}
%
% \DescribeMacro{\GeometriaReporte}{\macrooptcolor|\GeometriaReporte|\marg{geometría}}
%
%  Aquí van los parámetros relativos al paquete |geometry| que van a afectar la manera como el reporte es generado, de la misma manera como se definieron los márgenes para la factura.
%
%  Puesto que el reporte no se imprime en hojas de facturas sino en hojas en blanco, el usuario podrá cambiar la geometría del papel completamente a su gusto sin las restricciones impuestas por el diseño que tengan sus facturas. En |factura.def| se definen unos márgenes razonables para un reporte. Si el reporte y la factura usaran un tamaño distinto de papel, entonces al imprimirse ambos juntos (opción |conreporte|), el reporte tomará el tamaño físico del papel especificado para la factura ya que con el paquete |geometry| no podemos cambiar el tamaño del papel en el medio del documento, aunque sí sus márgenes. Entonces, si se quiere imprimir el reporte en un papel de otro tamaño, tendrán que imprimirse ambos documentos por separado, o sea, primero uno y después el otro.
%
% \DescribeMacro{\FondoReporte}{\macrooptcolor|\FondoReporte|\marg{nombre del archivo de la imagen}}
%
%  Imagen a cargar {\bfseries solamente en la primera página del reporte} (e.g. un membrete). Cuando se usa este comando puede necesitarse usar también |\InicioReporte| para fijar el inicio de la primera página del reporte más abajo que las demás páginas cuyos márgenes están establecidos con |\GeometriaReporte|; esto porque la primera página tendría un “margen superior” distinto por la existencia del membrete.
%
% \DescribeMacro{\FondoReporte*}{\macrooptcolor|\FondoReporte*|\marg{nombre del archivo de la imagen}}
%
%  Esta variante de |\FondoReporte| es para que la imagen esté visible {\bfseries en todas las páginas} y no solamente en la primera como con el comando anterior. Usando esta variante del fondo del reporte no es necesario usar |\InicioReporte| sino solamente establecer los márgenes del reporte de acuerdo a dicho fondo.
%
% \DescribeMacro{\EstiloPagReporte}{\macrooptcolor|\EstiloPagReporte|\oarg{estilo primera página}\marg{estilo páginas restantes}}
%
%  Con este comando definimos el estilo de página para el reporte (véase el comando |\InfoPagina| y la subsección \ref{subsec:pagestyle}). El estilo de página es lo que cambiamos en \LaTeX{} con |\pagestyle|, o sea, lo relativo a las cabeceras y pies de página.
%
%  Con este comando podemos asignar un estilo de página para la primera página del reporte y otro estilo de página para el resto de las páginas. La primera página del reporte pudiera tener un estilo de página distinto por la presencia del membrete si se usa el comando |\FondoReporte| sin “|*|”, por lo cual se puede querer no poner cabeceras en la primera página pero sí en las demás.
%
%  Como vemos de la sintaxis de esta macro, el argumento obligatorio es el del resto de las páginas y el opcional es el que aplica a la primera página. El opcional se pone entre corchetes antes del otro, pero si se omite, todas las páginas toman el estilo especificado entre llaves.
%
%  Si no se usa esta macro la clase asume un comportamiento apropiado teniendo en cuenta si se especifica |\FondoReporte| o |\FondoReporte*| para evitar solapamiento de textos, como se explica también en la subsección \ref{subsec:pagestyle}.
%
%  La información que aparece como cabecera o como pie de página es la que damos con el comando |\InfoPagina|, explicado más adelante.
%
%  La clase provee varios estilos de página, que también se explicarán en la subsección \ref{subsec:pagestyle}. En los ejemplos de reportes se puede ver el comportamiento del comando |\EstiloPagReporte|.
%
% \DescribeMacro{\InicioReporte}{\macrooptcolor|\InicioReporte|\marg{longitud}}
%
%  Con este comando -cuyo argumento es una longitud de \LaTeX{}- se agrega una distancia extra donde inicia el texto en la primera página del reporte cuando se usa |\FondoReporte| sin “|*|”. Este comando sólo aplica cuando en el reporte se fija un fondo para que salga solamente en la primera página (comando |\FondoReporte| en su versión sin “|*|”). La motivación de esto es corregir la alteración que produce un membrete que va a salir solamente en la primera página; si el membrete estuviera en todas las páginas no es necesario hacer esto sino fijar los márgenes adecuadamente con |\GeometriaReporte|. Si en el reporte no se fija un fondo o se fija el fondo para todas las páginas (comando |\FondoReporte*|), lo que se establezca con |\InicioReporte| es ignorado. En estos dos casos los márgenes deben establecerse con |\GeometriaReporte|.
%
% \DescribeMacro{\FirmaReporte}{\macrooptcolor|\FirmaReporte|\oarg{inclusión la imagen}\oarg{corr. en x}\oarg{corr. en y}\marg{texto}\oarg{pos. x}\oarg{long. línea}}
%
%  Incluye la firma para el final del reporte. Los argumentos son:
%  \begin{enumerate}
%   \item inclusión de la imagen (e.g. |\includegraphics|),
%   \item corrimiento en |X| de la imagen, para un ajuste más fino de la posición de la firma,
%   \item corrimiento en |Y|,
%   \item texto debajo de la línea ({\itshape va entre llaves y no entre corchetes como los demás}). El predeterminado es el dado mediante el comando |\Proveedor|, o en su defecto, |\Emisor|),
%   \item posición en |X| del final de la línea, medida desde la derecha (predeterminado: 2cm),
%   \item tamaño extra de la línea de la firma. La clase mide el tamaño del texto y de la imagen (si se especificó), y al que sea mayor le agrega este extra de línea para que sobresalga a los dos lados. (Predeterminado: tamaño de la imagen o del nombre + 1.5cm extra).
%  \end{enumerate}
%  {\itshape {\bfseries Este comando es de los pocos que no es del preámbulo}; debe incluirse en el sitio en el reporte donde se quiera que la firma salga, generalmente al final del reporte y todos sus argumentos son opcionales, incluyendo el texto que va debajo de la línea, que se incluye entre llaves |{}| a pesar de ser opcional. Cuando no se especifica ningún argumento, sólo dibuja la línea para firmar y le escribe abajo el nombre dado con |\Proveedor| o en su defecto el dado con |\Emisor|.}
%
% \DescribeMacro{\TituloReporte}{\macrooptcolor|\TituloReporte|\marg{texto}}
%
%  Especificamos un título para el reporte. Este título saldrá centrado al principio del reporte.
%
% \DescribeMacro{\TituloReporte*}{\macrooptcolor|\TituloReporte*|\marg{texto}}
%
%  Usando |\TituloReporte| u omitiendo su uso, la información que sale en las cabeceras o pies de página tiene que ver con el número de la factura que está asociada a ese reporte. Con esta variante con “|*|” de |\TituloReporte|, el título del reporte -especificado con este comando- saldrá escrito también en estas partes de la página en lugar del texto que indica que el reporte corresponde a determinada factura.
%
% Puede usarse esta opción también en aquellos casos en los que se hace un reporte sin asociarlo a una factura, entonces el título del reporte es la información que sale en la cabecera o pie de cada página.
%
% \DescribeMacro{\InfoPagina}{\macrooptcolor|\InfoPagina|\marg{texto}}
%
%  Establece el texto que aparece en la cabecera o pie de página del reporte.
%
% {\itshape Si se quiere usar esta opción debe hacerse luego de definir el título del reporte porque si no el comando |\TituloReporte*| establece de nuevo la variable que |\InfoPagina| cambia.}
%
%  En |factura.def| hay dos definiciones relativas a la información que aparece en las cabeceras o pies de página: aquella que escribe el título y aquella que no, es decir, la que aplica con |\TituloReporte*| y la que aplica con |\TituloReporte|, que también el usuario podrá redefinir si lo necesita, pero lo más práctico sería usar esta macro para fijar el texto que se desee, y |\EstiloPagReporte| para fijarlo donde se quiera.
%
%
% \subsection{Definiciones en archivos separados \label{subsec:aux}}
%
% \DescribeMacro{\Cliente}{\macrooptcolor|\Cliente|\marg{archivo \LaTeX}}
%
%  {\bfseries Todos los comandos definibles en el preámbulo que alteren la forma como el reporte o la factura es presentada, así como cualquier variable definida en |factura.def| se puede redefinir y agregar a otro archivo |.tex| o |.def| que el usuario cargue en el preámbulo con |\input|, |\include| o con el comando |\Cliente|; el comando para el cliente existe para que el usuario pueda definir archivos |.tex| o |.def| con los valores específicos que aplican para cada cliente.}
%
%  La macro |\Cliente| incluye un archivo cuya extensión debe ser |.def| o |.tex|, de la misma forma como se hace con |\input|; es simplemente un atajo que además lee archivos |.def|.
%
% }
%
%
% \section{La factura\label{sec:factura}}
%
% \DescribeEnv{factura}
% La parte del documento donde se agregan los elementos a facturar se escribe dentro de un entorno especial para su tratamiento, llamado |factura|, que produce el cuadro de la factura:
%
% \iffalse
%<*literal>
% \fi
{\VERBATIM
\begin{verbatim}
    \begin{factura}
     [...]
    \end{factura}
\end{verbatim}
}
% \iffalse
%</literal>
% \fi
%
% Dentro de dicho ambiente podemos usar únicamente los comandos que incluyen los elementos a facturar y las posibles etiquetas; esos comandos son, para distintos gravámenes, los siguientes:
%
%
% \subsection{Agregando elementos a la factura}
%
% \noindent
% {\macromandcolor|\Item|\marg{cantidad}\marg{descripción}\oarg{descripción para csv}\marg{precio unitario}\oarg{descripción sólo para \\ \mbox{}\hfill referencias}}
% \DescribeMacro{\Item}
% \DescribeMacro{\ItemR}
% \DescribeMacro{\ItemA}
% \DescribeMacro{\ItemE}
%
% Con estos comandos agregamos elementos a la factura propiamente dicha. Con |\Item| agregamos un elemento gravado con IVA general (actualmente 12\%); con |\ItemR| agregamos uno con IVA reducido (8\%), con |\ItemA| uno con IVA adicional (27\%), y con |\ItemE| agregamos un elemento exento de impuesto al valor agregado.
%
% Como podemos ver, se especifica la cantidad, luego la descripción y al final el precio unitario, todos entre llaves, pudiendo intercalar entre estos dos últimos un parámetro adicional entre corchetes que sirve para escribir otro texto al archivo |.csv| en sustitución de la descripción, ya que a veces la descripción puede tener macros o caracteres no imprimibles. Si se declara una descripción y otra descripción para el |.csv|, la factura impresa manejará la descripción dada primero (entre llaves) y en el archivo |.csv| se escribirá la descripción dada después (entre corchetes).
%
% El último parámetro opcional (entre corchetes) es otro campo de descripción pero que no se imprime en el cuadro de la factura sino que se guarda bajo una etiqueta para luego poder usarse en cualquier parte del documento con el comando |\ldescref| (véase la sección \ref{sec:etiquetas} de las etiquetas). Dicha descripción puede tener formato de párrafos largos con saltos de línea.
%
% También tenemos la posibilidad de agregar un ítem al que no le ponemos precio, por ejemplo por ser parte de un total, y sus argumentos son la cantidad y la descripción nada más.
%
% \noindent
% \DescribeMacro{\ItemX}{\macrooptcolor|\ItemX|\marg{cantidad}\marg{descripción}\oarg{descripción para csv}\oarg{descripción sólo para referencias}}
%
%
% \subsubsection{Opción `{\ttfamily descuentos}'\label{subsec:descuentos}}
%
% \noindent
% {\macromandcolor|\Item|\marg{cantidad}\marg{descripción}\oarg{descripción para csv}\marg{precio unitario}\marg{\% descuento} \\ \mbox{}\hfill \oarg{descripción sólo para referencias}}
% \DescribeMacro{\Item}
% \DescribeMacro{\ItemR}
% \DescribeMacro{\ItemA}
% \DescribeMacro{\ItemE}
%
% Cuando pedimos descuentos individuales (opción |descuentos| al cargar la clase), todas las macros |\Item| tienen un posible\footnote{Es un argumento opcional, aunque va entre llaves como todos los demás.} argumento más al final que es el descuento porcentual. No especificarlo da el mismo resultado que un descuento de 0\% sobre ese ítem:
%
%
% \subsubsection{Opción `{\ttfamily sincantidad}'\label{subsec:sincantidad}}
%
% \noindent
% \DescribeMacro{\Item}
% \DescribeMacro{\ItemR}
% \DescribeMacro{\ItemA}
% \DescribeMacro{\ItemE}
% {\macromandcolor|\Item|\marg{descripción}\oarg{descripción para csv}\marg{precio unitario}\oarg{descripción sólo para referencias}}
%
% \noindent {\macromandcolor|\Item|\marg{descripción}\oarg{descripción para csv}\marg{precio unitario}\marg{\% descuento} \\ \mbox{}\hfill \oarg{descripción sólo para referencias}}
%
% Por otra parte, si se usa la opción |sincantidad| al cargar la clase, todos los comandos |\Item| aceptan un argumento menos, que es el de la cantidad; eso significa que sin usar la opción |descuentos|, la sintaxis de |\Item| queda como lo indica la primera línea, y usando |descuentos|, queda como lo indica la segunda línea.
%
% \noindent
% \DescribeMacro{\ItemX}{\macrooptcolor|\ItemX|\marg{descripción}\oarg{descripción para csv}\oarg{descripción sólo para referencias}}
% 
% Puesto que |\ItemX| se usa para elementos sin precio, al usar |sincantidad| quedará aceptando sólo un argumento obligatorio que es la descripción y dos opcionales para la descripción en el archivo |.csv| y la descripción larga para referenciar luego. Cuando se activa la opción |descuentos|, |\ItemX| queda igual puesto que esta macro no lleva precio, entonces tampoco acepta descuentos.
%
% {\itshape Todas las variaciones de estos comandos |\Item| pueden apreciarse en los ejemplos.}
%
%
% \subsection{Descripción larga al final de la factura}
%
% Una vez terminado el entorno de la factura se puede escribir una descripción larga que saldrá en una caja -cuyo estilo puede cambiarse-\footnote{El diseño de esta descripción consiste en su linea de contorno, cambiable como ya explicamos en la subsección \ref{subsec:espacioslineas}, específicamente con la macro \cs{LineaDescripcion}.} debajo del cuadro de la factura, esto va dentro de un |\parbox| en la implementación de la caja. Si esta descripción no se escribe, simplemente no va a aparecer nada debajo del cuadro de la factura.
%
% \noindent \DescribeMacro{\Descripcion}
% {\macrooptcolor|\Descripcion|\marg{texto en formato de párrafos}}
%
% {\itshape Como podemos ver, |\Descripcion| es otro de los pocos comandos que no se usan en el preámbulo sino dentro del documento, jústamente al terminar el ambiente} |factura|.
%
%
% \section{Reverso de la factura}
%
% \DescribeEnv{reverso}
%  La factura puede tener información escrita en su reverso, por ejemplo para ciertas condiciones cuando estas necesitan mucho espacio. Ese texto se escribe dentro de un entorno llamado |reverso|, que puede o no incluirse según el requerimiento del usuario. si está escrito en el documento fuente también podrá decidirse si se incluye o no en el documento compilado.
%
% \iffalse
%<*literal>
% \fi
{\VERBATIM
\begin{verbatim}
    \begin{reverso}
     (Aquí va toda la información que irá escrita
      en el reverso de la hoja de la factura.)
    \end{reverso}
\end{verbatim}
}
% \iffalse
%</literal>
% \fi
%
% El reverso de la factura va escrito con la tipografía que el usuario elije con el comando |\LetraReverso|, como se explicará en la sección \ref{sec:tipografia}, y las dimensiones de la página se establecen con el comando |\GeometriaReverso| (sección \ref{sec:comandos}). Dicha geometría viene con un valor predeterminado en el |.def|.
%
% Si se compila con la opción |sinreverso|, lo escrito dentro del ambiente |reverso| no saldrá como reverso en la hoja de la factura.
%
%
% \section{El reporte}
%
% \DescribeEnv{reporte}
%  Para elaborar un reporte, lo único que se debe hacer es escribir el texto del reporte dentro del ambiente para ese fin, que se llama |reporte|, de manera que en el documento tenemos un ambiente para la factura y otro para el reporte, y elegimos al momento de compilar qué es lo que queremos obtener.
%
% \iffalse
%<*literal>
% \fi
{\VERBATIM
\begin{verbatim}
    \begin{reporte}
     (Aquí va el reporte, sin límite la extensión del texto.)
    \end{reporte}
\end{verbatim}
}
% \iffalse
%</literal>
% \fi
%
% Cuando se compila con |conreporte|, el reporte sale desde la segunda página, o desde la tercera si se usa |twoside|.
%
% Cuando se compila con |soloreporte|, el motor de compilación no procesa la factura en favor del reporte, y sin especificar ninguna de estas dos opciones, el reporte es ignorado y sale únicamente la factura (o la prefactura o cotización).
%
% El reporte empieza con un encabezado y luego un título especificable; el encabezado consiste en el nombre del proveedor, el nombre del cliente y la persona de contacto. Luego de ese encabezado viene el título del reporte y después empieza lo que el usuario escriba dentro del entorno del reporte.
%
% Hay que tener en cuenta que, como se explicó en la sección \ref{sec:comandos}, el reporte puede tener una imagen de fondo, que puede ser un membrete. Ese membrete, si se quiere en todas las páginas deberá incluirse con |\FondoReporte*|, y si se quiere sólo en la primera página del reporte deberá usarse |\FondoReporte|. Ahora, si se usa la versión sin |*|, será necesario también usar |\InicioReporte| con una distancia para empezar a escribir un poco más abajo en la primera página, ya que es la única que tiene el membrete. Pero si se usa la versión con |*|, entonces basta fijar los márgenes para todas las páginas con |\GeometriaReporte|, tal como se adelantó en la sección \ref{sec:comandos}.
%
% El reporte estará escrito con la tipografía especificable con |\LetraReporte|, como se explicará más adelante en la sección \ref{sec:tipografia}.
%
% Por otra parte, al reporte se le asigna un título, que puede aparecer o no en la cabecera de página o en el pie de página. Si se quiere que aparezca el título en esos lugares, debe usarse |\TituloReporte*|, de lo contrario, |\TituloReporte|. Si no se especifica el título, entonces la información mostrada allí es la referencia a la factura asociada a este reporte, como se comentó antes cuando se introdujeron estos comandos.
%
%
% \subsection{Estilos de página para el reporte\label{subsec:pagestyle}}
%
% La clase define varios estilos de página\footnote{Usables en el reporte únicamente.} de \LaTeX{}, cambiables con los comandos usuales: |\pagestyle| y |\thispagestyle| dentro del entorno del reporte, pero es mejor definir el estilo de página en el preámbulo del documento, con |\EstiloPagReporte|.
%
% Estos estilos de página están definidos con el paquete |fancyhdr|, y son: \begin{inparaenum}[\bfseries i)]\item |plain|, \item |empty|, \item |soloarriba|, \item |soloabajo|, y \item |solonumpag|\end{inparaenum}.
%
% Con |plain|\footnote{Por razones de compatibilidad usamos dos estilos de página conocidos en \LaTeX, que son |plain| y |empty|, siendo |plain| uno de nuestros predeterminados.}, en la cabecera saldrá la información sobre el reporte y en el pie de página saldrá el número de página; con |empty| no sale nada en ninguna parte; con |soloarriba| y |soloabajo|, toda esa información anterior sale en la cabecera o en el pie de página, respectivamente, y con |solonumpag|, lo único que aparece es el número de página centrado en el pie de página.
%
% El formato con la información a poner en estas partes de las páginas del reporte es cambiable con |\InfoPagina|, descrito anteriormente.
%
% Cuando se usa |\FondoReporte*|, como se supone que se usa para un membrete en todas las páginas, el estilo de página predeterminado del reporte será |soloabajo|, para todas las páginas, porque no tendría sentido tener un membrete en todas las páginas y una cabecera como la descrita anteriormente. Si el comando |\FondoReporte*| se está usando para algo distinto de un membrete, por ejemplo una marca de agua, entonces es probable querer cambiar los estilos de página con |\EstiloPagReporte|.
%
% Si se usa |\FondoReporte|, el estilo predeterminado para la primera página será |solonumpag|, y |plain| para el resto de las páginas, y si se no se especifica fondo alguno, entonces el estilo de página predeterminado para todas las páginas será |plain|.
%
% El usuario podrá definir más estilos de página adicionales para colocar el texto que se quiera en el lugar que se quiera, y usar esos estilos de página con |\EstiloPagReporte|. Para eso puede leer la documentación del paquete |fancyhdr|, que es lo que esta clase usa para los estilos de página. También puede consultar |factura.def|, que es donde están predefinidos los estilos que trae la clase.
%
% El grosor de la línea decorativa que coloca el paquete |fancyhdr| está definido en el archivo |.def|, y es cambiable en la macro |\@GrosorLineaInfoPag| como podrá verse en dicho archivo; no definimos un comando para ello por no ser algo relevante pero igual podrá cambiarse con |\renewcommand| por ejemplo, como se hace en dicho |.def|.
%
%
% \section{Tipografías para cada elemento a escribir\label{sec:tipografia}}
%
% Cada elemento escrito en la factura o reporte es sujeto a ser mostrado con una tipografía distinta, para lo cual existen comandos con los que se asigna el tipo de letra o estilo para cada cosa que sale escrita en el documento, por ejemplo, se puede cambiar el tipo de letra con el que se escribe la fecha, el tipo de letra con el que salen los elementos dentro del cuadro de la factura, el tipo de letra de los totales, de los títulos, de las cabeceras o pies de página en el reporte, etc.
%
% Cuando en lo sucesivo nos referiremos a tipografía, tipo de letra, o letra, nos estaremos refiriendo a cualquier cambio en la tipografía que pueda significar el nombre de la letra, su tamaño o atributos como negritas, es decir, mediante cualquiera de estos comandos podemos cambiar cualquier aspecto de la tipografía. Mirando el archivo |.def| pueden verse todos los valores predeterminados, que la mayoría de ellos son simplemente cambios en el tamaño (macro |\fontsize| de \LaTeX{}), y en algunos casos negritas.
%
% {\setlength{\parindent}{0pt}
% \DescribeMacro{\LetraItems}{\macrooptcolor|\LetraItems|\marg{tipo de letra}}
%
%  Con este comando cambiamos la tipografía utilizada en los ítems o elementos que están en el cuadro de la factura, de manera que {\itshape cambiar la longitud \cs{baselineskip} con el segundo argumento de \cs{fontsize} hará que la tabla tenga una altura distinta de las celdas.}
%
%  Lo anterior significa que {\bfseries para cambiar la altura de las celdas, la forma correcta es a través de cambiar el tamaño de la letra de sus elementos con este comando. (Véase el archivo |.def|)}.
%
% \DescribeMacro{\LetraNumeros}{\macrooptcolor|\LetraNumeros|\marg{tipo de letra}}
%
%  Este comando sirve para cambiar el tipo de letra de tota la información numérica que aparece en las distintas columnas del cuadro de la factura, sin contar los totales, para los cuales hay otros comandos.
%
% \DescribeMacro{\LetraTipoIVA}{\macrooptcolor|\LetraTipoIVA|\marg{tipo de letra}}
%
%  El tipo del IVA es la letra que indica qué IVA está siendo gravado, por ejemplo, la “|G|”, “|R|” y “|A|” que puede salir al lado de los elementos con IVA general, reducido y adicional respectivamente, y “|(E)|” al lado de los elementos exentos. Con esta macro cambiamos la letra con la que eso sale.
%
% \DescribeMacro{\LetraTitColumnas}{\macrooptcolor|\LetraTitColumnas|\marg{tipo de letra}}
%
%  Con |\LetraTitColumnas| podemos cambiar la tipografía de los títulos de los campos que hay en el cuadro de la factura, es decir, los textos “Cantidad”, “Descripción”, “Precio U.”, etc.
%
% \DescribeMacro{\LetraTitTotales}{\macrooptcolor|\LetraTitTotales|\marg{tipo de letra}}
%
%  Con este comando cambiamos la tipografía de los nombres de los totales, es decir, “IVA G.”, “BI G.”, salvo el total general que se cambia con |\LetraTitTotal|.
%
% \DescribeMacro{\LetraNumTotales}{\macrooptcolor|\LetraNumTotales|\marg{tipo de letra}}
%
%  Con esta macro se cambia la tipografía utilizada en los totales numéricos de la factura, exceptuando el total general que se cambia con |\LetraNumTotal|.
%
% \DescribeMacro{\LetraTitTotal}{\macrooptcolor|\LetraTitTotal|\marg{tipo de letra}}
%
%  Para cambiar la tipografía de la palabra que indica el total general, e.g. “TOTAL”, lo hacemos con |\LetraTitTotal|.
%
% \DescribeMacro{\LetraNumTotal}{\macrooptcolor|\LetraNumTotal|\marg{tipo de letra}}
%
%  Para la tipografía del total general (la celda con el número del total final), utilizamos este comando.
%
% \DescribeMacro{\LetraTitEnc}{\macrooptcolor|\LetraTitEnc|\marg{tipo de letra}}
%
%  Con este comando cambiamos el tipo de letra de la información en el título de las cajas del encabezado de la factura, por ejemplo los textos: “Resumen” y “Datos del cliente”.
%
% \DescribeMacro{\LetraEncFactura}{\macrooptcolor|\LetraEncFactura|\marg{tipo de letra}}
%
%  Podemos también cambiar el tipo de letra de la información del encabezado de la factura, es decir, el tipo de letra del texto que escribe el usuario en el resumen o de toda la identificación del cliente en la factura. El resumen naturalmente acepta más cambios de tipografía escribiéndolos dentro del párrafo que se escribe con el comando |\Resumen|.
%
% \DescribeMacro{\LetraEncReporte}{\macrooptcolor|\LetraEncReporte|\marg{tipo de letra}}
%
%  Para cambiar la tipografía de la información del encabezado del reporte usamos |\LetraEncReporte|.
%
% \DescribeMacro{\LetraNotaPrevia}{\macrooptcolor|\LetraNotaPrevia|\marg{tipo de letra}}
%
%  Con |\LetraNotaPrevia| podemos cambiar la tipografía con la que se escribe la -opcional- nota previa al cuadro de la factura.
%
% \DescribeMacro{\LetraNotaInterna}{\macrooptcolor|\LetraNotaInterna|\marg{tipo de letra}}
%
%  Con |\LetraNotaInterna| cambiamos el formato de la nota opcional que aparece como coletilla de la factura (nota interna).
%
% \DescribeMacro{\LetraNotaExterna}{\macrooptcolor|\LetraNotaExterna|\marg{tipo de letra}}
%
%  También podemos cambiar la tipografía de la nota externa a la factura, es decir, la -opcional- nota que va al terminar el cuadro de la factura antes de la -opcional también- descripción larga.
%
% \DescribeMacro{\LetraDescripcion}{\macrooptcolor|\LetraDescripcion|\marg{tipo de letra}}
%
%  Con este comando cambiamos la tipografía de la descripción larga; naturalmente cualquier cambio puede hacerse luego dentro del texto de la descripción, igual que con el resumen.
%
% \DescribeMacro{\LetraFirmas}{\macrooptcolor|\LetraFirmas|\marg{tipo de letra}}
%
%  Con |\LetraFirmas| podemos cambiar la letra usada para escribir el nombre debajo de la línea para firmar en la factura.
%
% \DescribeMacro{\LetraNotaFinal}{\macrooptcolor|\LetraNotaFinal|\marg{tipo de letra}}
%
%  Para cambiar la tipografía de la nota final se hace con este comando. (Aplica el mismo comentario que para la descripción y el resumen).
%
% \DescribeMacro{\LetraDenom}{\macrooptcolor|\LetraDenom|\marg{tipo de letra}}
%
%  La tipografía asociada a la denominación del documento, es decir, al lado del número (e.g. “FACTURA” o “COTIZACIÓN”), y la asociada al texto “No. CONTROL” se cambia con |\LetraDenom|.
%
% \DescribeMacro{\LetraNumeracion}{\macrooptcolor|\LetraNumeracion|\marg{tipo de letra}}
%
%  Con |\LetraNumeracion| cambiamos la tipografía de las numeraciones como la del número de la factura y el número de control.
%
% \DescribeMacro{\LetraFecha}{\macrooptcolor|\LetraFecha|\marg{tipo de letra}}
%
%  La tipografía para la fecha de emisión y la fecha de vencimiento la podemos cambiar con este comando.
%
% \DescribeMacro{\LetraTitReporte}{\macrooptcolor|\LetraTitReporte|\marg{tipo de letra}}
%
%  La letra del título que aparece al principio del reporte se puede cambiar con |\LetraTitReporte|.
%
% \DescribeMacro{\LetraInfoPagina}{\macrooptcolor|\LetraInfoPagina|\marg{tipo de letra}}
%
%  Las cabeceras o pies de página del reporte también tienen un comando para su tipo de letra, y es |\LetraInfoPagina|.
%
% \DescribeMacro{\LetraReverso}{\macrooptcolor|\LetraReverso|\marg{tipo de letra}}
%
%  Con |\LetraReverso| cambiamos la tipografía de la información que escribamos en el reverso de la hoja de la factura.
%
% \DescribeMacro{\LetraReporte}{\macrooptcolor|\LetraReporte|\marg{tipo de letra}}
%
%  Finalmente, para cambiar la tipografía con la que se inicia el reporte tenemos este comando. Obviamente también luego se puede hacer cualquier cambio dentro del ambiente |reporte|.
%
% }
%
%
% \section{Etiquetas y referencias a ítems de la factura\label{sec:etiquetas}}
%
% Esta clase |factura| soporta agregar etiquetas a cada elemento de la factura que luego en el reporte o en casi cualquier otra parte queramos referenciar; {\slshape no se pueden referenciar etiquetas solamente en la nota previa ni en el resumen.}
%
% \noindent \DescribeMacro{\label}{\macrooptcolor|\label|\marg{etiqueta}}
%
%  Para ello usamos la conocida macro |\label| de \LaTeX{} justamente después del ítem que nos interese luego referenciar, por ejemplo (usando descuentos individuales):
%
% \iffalse
%<*literal>
% \fi
{\VERBATIM
\begin{verbatim}
    \begin{factura}
     \Item{1}{Artículo gravado con IVA general}{1000}
     \Item{3}{Otro artículo gravado con IVA general}{1500}{10}[Descripción larga 
      para referenciar el reporte u otro fin.

      Con formato de párrafo con saltos de línea si se desea.]\label{el-otro}
    \end{factura}
\end{verbatim}
}
% \iffalse
%</literal>
% \fi
%
% De esta forma creamos una etiqueta llamada “el-otro” para el segundo elemento, que luego podremos citar por ejemplo en la descripción larga o en el reporte con comandos similares a los que \LaTeX{} provee para referenciar etiquetas.
%
% \noindent \DescribeMacro{\itemref}{\macrooptcolor|\itemref|\marg{etiqueta}}\\
% \noindent \DescribeMacro{\cantref}{\macrooptcolor|\cantref|\marg{etiqueta}}\\
% \noindent \DescribeMacro{\descref}{\macrooptcolor|\descref|\marg{etiqueta}}\\
% \noindent \DescribeMacro{\puref}{\macrooptcolor|\puref|\marg{etiqueta}}\\
% \noindent \DescribeMacro{\subtref}{\macrooptcolor|\subtref|\marg{etiqueta}}\\
% \noindent \DescribeMacro{\dctoref}{\macrooptcolor|\dctoref|\marg{etiqueta}}\\
% \noindent \DescribeMacro{\ivaref}{\macrooptcolor|\ivaref|\marg{etiqueta}}\\
% \noindent \DescribeMacro{\ptref}{\macrooptcolor|\ptref|\marg{etiqueta}}\\
% \noindent \DescribeMacro{\ldescref}{\macrooptcolor|\ldescref|\marg{etiqueta}}\par
%
% Si se quiere mostrar el número de la línea que ocupa el ítem en la lista lo hacemos con |\itemref|, que en el ejemplo aparecería el número “|2|” indicando que es el segundo ítem de la factura\footnote{Cuando se usa \cs{itemref} conviene enumerar los ítems en la factura con la opción \texttt{numitem}.}. Con |\cantref| escribimos la cantidad del elemento citado mediante la etiqueta, en nuestro ejemplo sería “|3|”. Para la descripción del ítem usamos |\descref|, que en el caso del ejemplo sería el texto “|Otro artículo gravado con IVA general|”. Para el precio unitario usamos |\puref|, en el ejemplo “|1.500,00|”. El subtotal lo mostraríamos con |\subtref|; en nuestro caso “|4.500,00|”. Si se quiere mostrar el porcentaje del descuento del ítem citado usamos |\dctoref|, que para el ejemplo sería “|10|”. Para el porcentaje del IVA, usamos |\ivaref|, en nuestro caso escribiría “|12|”. Por último, para mostrar el precio total usamos |\ptref|, y en nuestro ejemplo eso escribiría “|4.050,00|” puesto que aplicamos un descuento del 10\%.
%
% Podemos ver que el uso de los comandos |\puref|, |\subtref| y |\ptref| nos muestra el precio en el mismo formato con el que lo vemos en la presentación de la factura, en cuanto al redondeo y al separador de miles y decimal.
%
% También podemos referenciar la -opcional- descripción larga que podemos usar por ejemplo en el reporte; esa descripción que va entre corchetes la podremos citar con |\ldescref|, y en nuestro ejemplo sería:
% “Descripción larga para referenciar el reporte u otro fin. Con formato de párrafo con saltos de línea si se desea.”\footnote{Aquí suprimimos los saltos de línea.}
%
% Con la asignación de estas etiquetas y su uso mediante las referencias podemos de manera conveniente citar cualquier ítem que se haya facturado, de manera dinámica sin tener que reescribir en la descripción larga o en el reporte aquello que ya se escribió en el cuadro de la factura. Para estas referencias se escogieron nombres de las macros que terminan en “\texttt{ref}” por compatibilidad con comandos de \LaTeX{} como |\pageref|.
%
%
%
% \section{Referenciar los valores establecidos en el preámbulo y los totales\label{sec:ref}}
%
%  En el preámbulo el usuario definió distintas macros, es decir, estableció la razón social del cliente, el R.I.F. y la dirección, entre otras cosas. Cualquiera de esas definiciones hechas en el preámbulo puede usarse luego en otra parte del documento con el mismo comando que se usó para su definición. Lo anterior significa que tenemos las siguientes macros posibles para usar dentro del cuerpo del documento (factura, reporte o reverso): |\Moneda|, |\RazonSocial|, |\RIF|, |\Direccion|, |\Telefono|, |\Email|, |\Contacto|, |\Emisor|, |\Conforme|, |\Proveedor|, |\Fecha|, |\FechaVencimiento|\footnote{La fecha de vencimiento no se establece en el preámbulo explícitamente sino a través de \cs{Credito} pero la clase dispone de esta interfaz para acceder a ella así como también con la interfaz dada por \texttt{datetime2}.}, |\Credito|, |\Num|, |\NumControl|, |\NumSerieControl|, |\Resumen|, |\NotaPrevia|, |\NotaInterna|, |\NotaExterna|, |\NotaFinal|, |\TituloReporte|.
%
% Esas macros, en el preámbulo funcionan para definir valores y textos, pero si ya no estamos en el preámbulos sino dentro del cuerpo del documento y escribimos una de esas macros\footnote{Dentro del cuerpo del documento estas macros obviamente no aceptan argumentos.}, lo que nos va a escribir será lo que está definido bajo esa macro, así sea con macros que no llegamos a utilizar explícitamente pero que tienen un valor predefinido como |\Moneda|.
%
% Por otra parte, en el documento el usuario también podrá referirse a cualquiera de los totales de la siguiente tabla\footnote{Las macros de la tabla, para representarlas en el mismo formato que en la factura, debemos usar la macro \cs{numprint}.}:
%
% \noindent \begin{tabular}{rl}
% \hline
% Total & Comando \\ \hline
% Bases imponibles & |\BIG|, |\BIR|, |\BIA|, |\BIE|\footnote{Lo llamamos \cs{BIE} para mantener la nomenclatura pero no es una base imponible ya que es un monto exento de impuesto.} \\
% IVA & |\AlicuotaG|, |\AlicuotaR|, |\AlicuotaA| \\
% Descuento en IVA & |\AlicuotaGDesc| \\
% Subtotales\footnote{El subtotal es el monto sin aplicar descuentos generales (los que se hacen con los comandos \cs{Descuento***}, es decir, los descuentos que no son individuales a cada item sino al totalizar), que coincide con la base imponible cuando no hay esos descuentos.} & |\SubtG|, |\SubtE|, |\SubtR|, |\SubtA| \\
% Descuentos & |\DescG|, |\DescE|, |\DescR|, |\DescA| \\
% Total final & |\Total| \\
% \hline
% \end{tabular}
%
% También ya mencionamos que la fecha de emisión y vencimiento se pueden acceder con la interfaz dada por el paquete |datetime2|, por ejemplo, para mostrar la fecha de emisión, podremos usar tanto |\Fecha| como |\DTMusedate{FechaEmision}|, y para la fecha de vencimiento, tanto |\FechaVencimiento| como |\DTMusedate{FechaVencimiento}|.
%
% La cantidad de ítems en la factura la podemos acceder via el contador |NumItem|, por ejemplo así |\arabic{NumItem}|.
%
% En los ejemplos hay casos con todas estas funciones de esta sección.
%
%
% \section{Más personalizaciones}
%
% \DescribeMacro{\@BeforeEndPreamble}
% La clase provee de un “gancho” llamado \cs{BeforeEndPreamble}. Lo que allí se agregue (es acumulativo, se puede ejecutar mas de una vez) será ejecutado por la clase |factura| luego de que hayan sido evaluadas las opciones de la clase y tomadas todas las decisiones en cuanto a opciones que impliquen otras opciones o que sean incompatibles con otras opciones; una vez que todas esas decisiones fueron tomadas y todo queda establecido, se ejecuta \cs{BeforeEndPreamble}, esto con el propósito de que el usuario pueda personalizar más el comportamiento de sus archivos |.tex|. Por ejemplo si el usuario hace lo siguiente, no funcionará:
%
% \iffalse
%<*literal>
% \fi
{\VERBATIM
\begin{verbatim}
    \documentclass[cotizacion]{factura}
    [...]
    \ifcotizacion\FondoFactura{Fondo}\fi
    \begin{document}
     \begin{factura}
      [...]
     \end{factura}
    \end{document}
\end{verbatim}
}
% \iffalse
%</literal>
% \fi
%
% No funcionará porque el condicional \cs{ifcotizacion} aún no se ha establecido (eso lo hace la clase con el gancho \cs{AtEndPreamble} de |etoolbox| luego de que todo el preámbulo haya sido analizado), por lo que se evaluará como falso; tendría el usuario que agregarlo dentro de un gancho \cs{AtEndPreamble} pero tampoco funcionará del todo, para eso esta clase provee del gancho \cs{BeforeEndPreamble}, y se escribiría este ejemplo así:
%
% \iffalse
%<*literal>
% \fi
{\VERBATIM
\begin{verbatim}
    \documentclass[cotizacion]{factura}
    [...]
    \BeforeEndPreamble{\ifcotizacion\FondoFactura{Fondo}\fi}
    \begin{document}
     \begin{factura}
      [...]
     \end{factura}
    \end{document}
\end{verbatim}
} 
% \iffalse
%</literal>
% \fi
%
% De esta forma, el condicional se evaluará después de que se hayan establecido todas las opciones (en este caso, |cotizacion|) y el usuario podrá inclusive apagar opciones y encenderlas nuevamente usando los condicionales (como se explicó en la subsección \ref{subsec:condicionales} porque después de ejecutar este gancho, automáticamente se volverán a evaluar las opciones y a tomar todas las decisiones de implicaciones y compatibilidades.
%
% \newpage
% \DescribeMacro{\@PosicionaFechas}
% \DescribeMacro{\@PosicionaDenom}
% \DescribeMacro{\@PosicionaNumControl}
% \DescribeMacro{\@PosicionaNotaFinal}
% \DescribeMacro{\@PosicionaFirmas}
% \DescribeMacro{\@EscribeDatos}
% \DescribeMacro{\@EscribeDatosResumen}
% \DescribeMacro{\@EscribeDatosReporte}
% \DescribeMacro{\@EscribeNotaFinal}
% \DescribeMacro{\@EscribeNotaPrevia}
% \DescribeMacro{\@DibujaLinea}
% \DescribeMacro{\@EscribeItemsEnCSV}
% \DescribeMacro{\@EscribeTotalesEnCSV}
% \DescribeMacro{\@EncabezadoFactura}
% \DescribeMacro{\@EncabezadoReporte}
% También es posible darle aspecto distinto a muchas de las partes que conforman la factura. Internamente en la clase, cada parte es ubicada en la hoja con un comando y su contenido es escrito en el papel con otro comando. Los comandos que más probablemente son sujetos a ser cambiados por el usuario son los comandos que escriben el texto más que aquellos que lo ubican en la hoja, por tal razón, los comandos que posicionan aquello que va a escribirse lo definimos dentro del archivo |.cls|, y los que propiamente escriben cada parte de la hoja están definidos en el |.def|. Ambas fuentes las podrán consultar los usuarios para mayor comprensión, y para la total personalización. Recomendamos empezar por el archivo de definiciones |factura.def|.
%
% Estas macros son internas y sirven para, como dijimos antes, que \LaTeX{} se prepare para escribir el texto en la hoja; todas empiezan con |\@Posiciona| y no debe ser común necesitar cambiarlas.
%
% Todas estas macros llaman a su correspondiente macro con el estilo del elemento a escribir, que comienza con |\@Estilo|, y con cambiar su contenido podemos cambiar la forma como es escrita esa parte de la factura en la hoja.
%
% \DescribeMacro{\@EstiloFechas}
% \DescribeMacro{\@EstiloDenom}
% \DescribeMacro{\@EstiloNumControl}
% \DescribeMacro{\@EstiloNotaPrevia}
% \DescribeMacro{\@EstiloNotaFinal}
% \DescribeMacro{\@EstiloFirmas}
% \DescribeMacro{\@DatosTIKZ}
% \DescribeMacro{\@NotaPreviaTIKZ}
% \DescribeMacro{\@DescripcionTIKZ}
% \DescribeMacro{\@NotaFinalTIKZ}
% Estas últimas son las que están definidas dentro de |factura.def|, y que eventualmente puede desearse cambiarlas.
%
% Adicionalmente están las macros que se encargan de distintas partes de la factura y del reporte como los datos del cliente, el resumen, la nota previa, la nota final, los encabezados, la información en el archivo |.csv|, y la macro que dibuja la línea que inutiliza el espacio en blanco. Todas estas son macros internas que no deben ser llamadas por el usuario, y no consideramos que pueda sea necesario cambiarlas.
%
% Por último, las macros que alojan el diseño \tikz de los datos del cliente, la nota previa, la descripción larga y la nota final son |\@DatosTIKZ|, |\@NotaPreviaTIKZ|, |\@DescripcionTIKZ| y |\@NotaFinalTIKZ|; están definidas en el archivo |factura.def|.
%
% Cualquiera de estas macros que se quiera redefinir habría que hacerlo en el preámbulo del |.tex| y encerrándolas entre |\makeatletter| y |\makeatother| para poder redefinir una macro que contiene un símbolo “|@|”.
%
% {\bfseries \itshape Finalmente, como dijimos en la sección \ref{subsec:aux}, una vez adecuada la clase al diseño que se quiere, se podrán escribir todos los comandos que hacen estos ajustes en un} |.tex| {\itshape o} |.def| {\itshape y cargarlo con} |\input|, |\include| {\itshape o con} |\Cliente|, pudiendo discriminar por clientes dicha personalización.
%
%
%
% \newpage
% \setstretch{1}
% \section{Tabla con todas las opciones}
%
% \noindent La tabla que sigue muestra una lista de las opciones posibles al momento de cargar la clase:
%
% \noindent \begin{tabular}{>{\ttfamily}ll}
%  \rmfamily \bfseries Opción & \bfseries Significado \\ \hline
%  agrupado & Agrupa los totales por tipo en el modo de columnas. \\
%  centradesc & Centra la descripción de los ítems. \\
%  centrado & Centra los totales en el modo de filas. \\
%  conlineasha & Dibuja una línea separando los artículos entre sí. \\
%  conreporte & Produce la factura y el reporte. \\
%  cotización & Imprime una cotización. \\
%  csv & Genera un archivo |.csv| con la información numérica de la factura. \\
%  dcu & Descripción, cantidad, precio unitario. \\
%  denom & Explicita la denominación del documento. \\
%  descuentos & A cada artículo se le puede asignar un descuento individual. \\
%  dosfirmas & Incluye los dos espacios para firmar. \\
%  duc & Descripción, precio unitario, cantidad. \\
%  electro\footnote{Ya no sirve esta opción.} & Calcula los descuentos para el IVA para pagos por medios electrónicos. \\
%  expandecuadro & Llena con espacio extra el cuadro de la factura. \\
%  filas & Escribe los totales en filas. \\
%  G & Agrega una “|G|” en los ítems con IVA general. \\
%  impdesc & Escribe la denominación del IVA en la descripción. \\
%  iva & Expresa en otra columna el porcentaje de IVA para cada elemento a facturar. \\
%  linea & Dibuja una línea que inutiliza el espacio en blanco en el cuadro de la factura. \\
%  nospanish & No carga |babel| ni |fontenc|. \\
%  numitem & Enumera los ítems en la factura en otra columna extra al principio. \\
%  prefactura & Imprime una prefactura. \\
%  sincantidad & Los artículos van sin expresar su cantidad. \\
%  sinexpandir & No expande elementos en la factura. \\
%  sinfirmas & No incluye espacios para firmar. \\
%  siniva & No incluye IVA, pero lo deja expresado. \\
%  sinivaexpresado & No incluye IVA ni lo expresa. \\
%  sinlineahni & No dibuja la línea horizontal entre totales y nota interna. \\
%  sinlineahtit & No dibuja la línea horizontal del título de las columnas. \\
%  sinlineahtot & No dibuja la línea horizontal entre ítems y totales. \\
%  sinlineashi & No dibuja las líneas horizontales internas. \\
%  sinlineashe & No dibuja las líneas horizontales externas. \\
%  sinlineash & No dibuja ninguna línea horizontal. \\
%  sinlineasvi & No dibuja las líneas verticales internas. \\
%  sinlineasve & No dibuja las líneas verticales externas. \\
%  sinlineasv & No dibuja ninguna línea vertical. \\
%  sinmarcas & No escribe el carácter de fin de línea ni el carácter en las celdas vacías \\
%            & cuando la descripción ocupa más que una línea. \\
%  sinmonedatotales & Quita el signo monetario de las totalizaciones. \\
%  sinnumero & No pone el número cuando se usa |denom|. \\
%  sintotales & No escribe los totales al final del cuadro; para cotizaciones. \\
%  sinreverso & No compila el reverso de la factura así se haya escrito. \\
%  soloreporte & Produce únicamente el reporte. \\
%  subtotal & En otra columna muestra el subtotal de cada ítem antes aplicar descuento. \\
%  todosiva & Expresa todos los tipos de IVA. \\
%  twoside & Imprime el reporte a dos caras. \\
%  unafirma & Sólo incluye la firma del emisor. \\
%  vencimiento & Escribe la fecha de vencimiento así no haya crédito en la factura. \\
% \end{tabular}
%
%
% \newpage
% \section{Tabla con todos los comandos}
%
% A continuación presentamos una tabla con todos los comandos, que salvo |\FirmaReporte|, los |\Item| y |\Descripcion|, se poner todos en el preámbulo.
%
% \noindent \begin{tabular}{>{\ttfamily}ll}
% \rmfamily \bfseries Comando & \bfseries Significado \\ \hline
%  |\cantref| & Muestra la cantidad del ítem referenciado. \\
%  |\CI| & Cédula de identidad del cliente. \\
%  |\Cliente| & Carga un archivo |.def| o |.tex| con personalizaciones. \\
%  |\NotaFinal| & Establece la nota final. \\
%  |\Conforme| & Nombre de quien recibe la factura, para el espacio de la firma. \\
%  |\Contacto| & Persona de contacto. \\
%  |\Credito| & Establece la cantidad de días para el crédito de la factura. \\
%  |\dctoref| & Muestra al descuento del ítem referenciado. \\
%  |\Denom| & Nombre de la denominación de documento. \\
%  |\descref| & Muestra a la descripción del ítem referenciado. \\
%  |\Descripcion| & Descripción larga de la factura\footnote{El comando \cs{Descripcion} se escribe dentro del documento, luego del ambiente \texttt{factura}, y no en el preámbulo.}. \\
%  |\Descuento| & Porcentaje de descuento para todos los elementos. \\
%  |\Descuento*| & Descuento neto en el subtotal, para todos los elementos. \\
%  |\Descuento[GRAE]| & Descuento en porcentaje, discriminado según el tipo de ítem. \\
%  |\Descuento[GRAE]*| & Descuento en el subtotal, discriminado según el tipo de ítem. \\
%  |\Direccion| & Dirección del cliente. \\
%  |\Email| & E-mail del cliente. \\
%  |\Emisor| & Nombre de quien emite la factura, para el espacio de la firma. \\
%  |\EstiloPagReporte| & Establece el estilo de página del reporte. \\
%  |\Fecha| & Establece la fecha de emisión de la factura en formato predeterminado. \\
%  |\Fecha*| & Establece la fecha de emisión de la factura en formato ISO. \\
%  |\FirmaFactura| & Carga la imagen de la firma en la factura. \\
%  |\FirmaReporte| & Carga la imagen de la firma en el reporte\footnote{El comando \cs{FirmaReporte} se escribe dentro del ambiente \texttt{reporte} y no en el preámbulo.}. \\
%  |\FondoFactura| & Carga la imagen del fondo de la factura. \\
%  |\FondoReporte| & Carga la imagen del fondo del reporte para la primera página. \\
%  |\FondoReporte*| & Carga la imagen del fondo del reporte para todas las páginas. \\
%  |\FormatoFecha| & Establece lo relacionado con el formato de las fechas. \\
%  |\GeometriaFactura| & Establece los parámetros de la geometría de la factura. \\
%  |\GeometriaReporte| & Establece los parámetros de la geometría del reporte. \\
%  |\InfoPagina| & Cambia la información de las cabeceras o pies de página del reporte. \\
%  |\InicioReporte| & Distancia hasta el inicio del texto en la primera hoja del reporte. \\
%  |\Item[RAEX]| & Agrega un elemento a la factura\footnote{Estos comandos \cs{Item} se escriben dentro del ambiente \texttt{factura} y no en el preámbulo.}. \\
%  |\itemref| & Muestra el número de línea en la factura del ítem referenciado. \\
%  |\ivaref| & Muestra al IVA del ítem referenciado. \\
%  |\label| & Agrega una etiqueta a un ítem de la factura\footnote{Se mantiene el nombre “\cs{label}” por compatibilidad con los comandos conocidos de \LaTeX. Este comando también se escribe dentro del ambiente \texttt{factura}.}. \\
%  |\ldescref| & Muestra a la descripción larga del ítem referenciado. \\
% \end{tabular}
% \newpage
% \noindent \begin{tabular}{>{\ttfamily}ll}
% \rmfamily \bfseries Comando & \bfseries Significado \\ \hline
%  |\LetraNotaFinal| & Tipografía para la nota final. \\
%  |\LetraDenom| & Tipografía para la denominación del documento. \\
%  |\LetraDescripcion| & Tipografía para la descripción en la factura. \\
%  |\LetraEncFactura| & Tipografía para la información en las cajas del encabezado. \\
%  |\LetraEncReporte| & Tipografía para la información del encabezado del reporte. \\
%  |\LetraFecha| & Tipografía para la fecha de emisión. \\
%  |\LetraFirmas| & Tipografía para los nombres de quienes firman la factura. \\
%  |\LetraInfoPagina| & Tipografía para las cabeceras o pies de página del reporte. \\
%  |\LetraItems| & Tipografía para los ítems en la factura. \\
%  |\LetraNotaExterna| & Tipografía para la nota externa al cuadro de la factura. \\
%  |\LetraNotaInterna| & Tipografía para la nota interna al cuadro de la factura. \\
%  |\LetraNotaPrevia| & Tipografía para la nota previa al cuadro de la factura. \\
%  |\LetraNumeracion| & Tipografía para las numeraciones. \\
%  |\LetraNumeros| & Tipografía para los números en la factura. \\
%  |\LetraNumTotal| & Tipografía para el total final numérico. \\
%  |\LetraNumTotales| & Tipografía para los totales numéricos. \\
%  |\LetraReporte| & Tipografía para el texto del reporte. \\
%  |\LetraReverso| & Tipografía para el reverso de la factura. \\
%  |\LetraTipoIVA| & Tipografía para el tipo de IVA. \\
%  |\LetraTitColumnas| & Tipografía para los nombres de las columnas del cuadro de la factura. \\
%  |\LetraTitEnc| & Tipografía para los títulos en las cajas del encabezado de la factura. \\
%  |\LetraTitReporte| & Tipografía para el título del reporte. \\
%  |\LetraTitTotal| & Tipografía para el nombre del total final. \\
%  |\LetraTitTotales| & Tipografía para los nombres de los totales. \\
%  |\LineaNotaFinal| & Grosor de la línea de la caja de la nota final, \\
%                      & curvatura de la esquina y sombra. \\
%  |\LineaDescripcion| & Grosor de la línea de la caja de la descripción, \\
%                      & curvatura de la esquina y sombra. \\
%  |\LineaEncabezado| & Grosor de la línea de las cajas del encabezado de la factura, \\
%                     & curvatura de la esquina y sombra. \\
%  |\LineaNotaPrevia| & Grosor de la línea de la caja de la nota previa al cuadro, \\
%                     & curvatura de la esquina y sombra. \\
%  |\LineaFirmas| & Longitud y grosor de las líneas para firmar. \\
% \end{tabular}
% \newpage
% \noindent \begin{tabular}{>{\ttfamily}ll}
% \rmfamily \bfseries Comando & \bfseries Significado \\ \hline
%  |\Moneda| & Cambia el signo monetario a mostrar. \\
%  |\Nombre| & Establece el nombre del cliente. \\
%  |\NotaInterna| & Escribe una nota interna al cuadro de la factura. \\
%  |\NotaExterna| & Escribe una nota extra externa al cuadro de la factura. \\
%  |\NotaPrevia| & Escribe una nota antes del cuadro de la factura. \\
%  |\Num| & Número de la factura. \\
%  |\NumControl| & Número de control de la factura. \\
%  |\NumSerieControl| & Número de serie de control de la factura. \\
%  |\PosNotaFinal| & Posición $(x_o,x_f,y)$ del párrafo con la nota final. \\
%  |\PosNotaFinal*| & Posición $(x_o,x_f,y)$ del párrafo con la nota final \\
%                     & (el texto se despliega hacia arriba). \\
%  |\PosControl| & Posición $(x,y)$ del número de control. \\
%  |\PosControl*| & Posición $(x,y)$ del número de control \\
%                 & (el texto se despliega hacia arriba). \\
%  |\PosDenom| & Posición $(x,y)$ del texto de la denominación. \\
%  |\PosDenom*| & Posición $(x,y)$ del texto de la denominación \\
%               & (el texto se despliega hacia arriba). \\
%  |\PosFecha| & Posición $(x,y)$ de las fechas. \\
%  |\PosFecha*| & Posición $(x,y)$ de las fechas \\
%               & (el texto se despliega hacia arriba). \\
%  |\PosFirmas| & Posición $(x_o,x_f,y)$ de las firmas. \\
%  |\Proveedor| & Nombre de quien firma el reporte. \\
%  |\ptref| & Muestra al precio total del ítem referenciado. \\
%  |\puref| & Muestra al precio unitario del ítem referenciado. \\
%  |\RazonSocial| & Establece la razón social. \\
%  |\Resumen| & Resumen. \\
%  |\RIF| & R.I.F. del cliente. \\
%  |\SepNotaFinal| & Separación interna entre el borde y el texto en la nota final. \\
%  |\SepDatosResumen| & Separación horizontal entre los datos del cliente y el resumen. \\
%  |\SepDescripcion| & Separación interna entre el borde y el texto en la descripción. \\
%  |\SepEncabezado| & Separación interna entre el borde y el texto en el encabezado. \\
%  |\SepNotaPrevia| & Separación interna entre el borde y el texto en la nota previa. \\
%  |\SepFilas| & Separación horizontal y vertical en los totales en el modo de filas. \\
%  |\SepItemsExtra| & Separación vertical extra entre los elementos en la factura. \\
%  |\SepVertical| & Separación vertical entre los distintos elementos de la factura. \\
%  |\subtref| & Muestra al subtotal del ítem referenciado. \\
%  |\Telefono| & Teléfono del cliente. \\
%  |\TituloReporte| & Establece el título del reporte. \\
%  |\TituloReporte*| & Establece el título del reporte a salir en las cabeceras o pies de página. \\
% \end{tabular}
%
%
% \ifejemplos
%  \section{Ejemplos incluídos\label{sec:ejemplos}}
%  \todo*{Hay que hacer que en la documentación, toda opción y macro referencie al ejemplo donde se usa.}
%
%  Desde la página siguiente se muestran todos los ejemplos incluídos en la clase:
%
%  \newpage
%  \newgeometry{marginparsep=0pt,vmargin={1cm,1cm},hmargin={1cm,1cm}}
%  {
%   \pagestyle{empty}
%   \setlength{\parindent}{0pt}
%   \hbadness=10000
%
%   \Ejemplo{ejemplo-cotizacion}
%   \Ejemplo{ejemplo-prefactura}
%   \newpage
%   \Ejemplo{ejemplo-factura01}
%   \Ejemplo{ejemplo-factura02}[\small\parskip=2pt]
%   \newpage
%   \Ejemplo{ejemplo-factura03}
%   \Ejemplo{ejemplo-factura04}
%   \newpage
%   \Ejemplo{ejemplo-factura05}
%   \Ejemplo{ejemplo-factura06}[\small]
%   \newpage
%   \Ejemplo[0.4]{ejemplo-factura07}[\small]
%   \Ejemplo{ejemplo-factura08}
%   \newpage
%   \Ejemplo*[0.49]{ejemplo-factura09}
%   \Ejemplo[0.49]{ejemplo-factura09}*{2}
%   \Ejemplo*{ejemplo-factura09}*
%   \newpage
%   \Ejemplo*[0.49]{ejemplo-factura10}
%   \Ejemplo[0.49]{ejemplo-factura10}*{2}
%   \Ejemplo*{ejemplo-factura10}*
%   \newpage
%   \Ejemplo*[0.49]{ejemplo-reporte1}
%   \Ejemplo[0.49]{ejemplo-reporte1}*{2}
%   \Ejemplo[0.49]{ejemplo-reporte1}{4}[\slshape Por espacio se omitió la página 2 en esta representación.]
%   \newpage
%   \Ejemplo*[0.49]{ejemplo-reporte2}
%   \Ejemplo[0.49]{ejemplo-reporte2}*{2}
%   \Ejemplo[0.49]{ejemplo-reporte2}{3}
%   \newpage
%   \Ejemplo*[0.49]{ejemplo-reporte3}
%   \Ejemplo[0.49]{ejemplo-reporte3}*{2}
%   \Ejemplo[0.49]{ejemplo-reporte3}{3}
%   \newpage
%   \Ejemplo{beispiel-rechnung}
%   \newpage
%  }
%  \hbadness=1000
%  \restoregeometry
% \else
%  \message{^^JLaTeX Warning: Compilacio'n ra'pida. No se incluyeron los ejemplos.^^J}
% \fi
%
% \StopEventually{\endorgo}
%
%    \begin{macrocode}
%<*class>
\RequirePackage{etoolbox}%  'etoolbox' para los condicionales binarios y para \AtEndPreamble.

%% Condicionales auxiliares que sólo pueden ser usados dentro de un grupo.
\newbool{DCU}
\newbool{DUC}
\newbool{Descuentos}
%% (quitado por no vigente)
 % \newbool{Electro}
\newbool{ExpAbajo}\booltrue{ExpAbajo}
\newbool{IVA}
\newbool{NumItem}
\newbool{PrimeroDescripcion}
\newbool{SinCantidad}
\newbool{SinLineaHoriNI}
\newbool{SinLineaHoriTit}
\newbool{SinLineaHoriTot}
\newbool{SinLineasHoriExt}
\newbool{SinLineasVertExt}
\newbool{SinLineasVertInt}
\newbool{Subtotal}

%% Condicionales auxiliares para la posición de los objetos fijos cuando son medidos desde la derecha, o desde abajo.
\newbool{NotaFinalYarriba}
\newbool{NotaFinalYnegativo}% no hay NotaFinalXnegativo.
\newbool{ControlXnegativo}
\newbool{ControlYarriba}
\newbool{ControlYnegativo}
\newbool{DenomXnegativo}
\newbool{DenomYarriba}
\newbool{DenomYnegativo}
\newbool{FechaXnegativo}
\newbool{FechaYarriba}
\newbool{FechaYnegativo}
\newbool{FirmasYnegativo}% no hay FirmasXnegativo; hay X_inicial y X_final para las firmas.

%% Condicionales relativos a opciones de la clase.
\newif\ifagrupado
\newif\ifcentradesc
\newif\ifcentrado
\newif\ifconlineasha
\newif\ifconreporte
\newif\ifcotizacion
\newif\ifcsv
\newif\ifdcu
\newif\ifdenom
\newif\ifdescuentos
\newif\ifdosfirmas
\newif\ifduc
%% (quitado por no vigente)
 % \newif\ifelectro
\newif\ifexpandecuadro
\newif\iffilas
\newif\ifG
\newif\ifimpdesc
\newif\ifiva
\newif\iflinea
\newif\ifnospanish
\newif\ifnumitem
\newif\ifprefactura
\newif\ifsincantidad
\newif\ifsinexpandir
\newif\ifsinfirmas
\newif\ifsiniva
\newif\ifsinivaexpresado
\newif\ifsinlineahni
\newif\ifsinlineahtit
\newif\ifsinlineahtot
\newif\ifsinlineash
\newif\ifsinlineashe
\newif\ifsinlineashi
\newif\ifsinlineasv
\newif\ifsinlineasve
\newif\ifsinlineasvi
\newif\ifsinmarcas
\newif\ifsinmonedatotales
\newif\ifsinnumero
\newif\ifsinreverso
\newif\ifsintotales
\newif\ifsoloreporte
\newif\ifsubtotal
\newif\iftodosiva
\newif\iftwoside
\newif\ifunafirma
\newif\ifvencimiento

%% Condicionales auxiliares.
\newif\if@cdu \@cdutrue
\newif\if@factura \@facturatrue
\newif\if@reverso \@reversotrue
\newif\if@reporte
\newif\if@FondoFactura
\newif\if@FondoReporte
\newif\if@FondoReporteStar
\newif\if@Nombre
\newif\if@PruebaRef
\newif\if@ExpandeCuadro
\newif\if@LineaDesdeInicio
\newif\if@LineaHastaTotal
\newif\if@LineaHastaFinal
\newif\if@LineaDescripcion
\newif\if@BIR
\newif\if@BIA
\newif\if@Exento
\newif\if@DescG
\newif\if@DescR
\newif\if@DescA
\newif\if@DescE
\newif\if@DescGnum
\newif\if@DescRnum
\newif\if@DescAnum
\newif\if@DescEnum
\newif\if@DescuentoTotal

%% Longitudes para uso interno.
\newlength{\@TamDatos}
\newlength{\@TamResumen}
\newlength{\@TamFinalEncabezado}
\newlength{\@TamLineaFirmas}
\newlength{\@TamNumItem}
\newlength{\@TamCantidad}
\newlength{\@TamDescripcion}
\newlength{\@TamPrecioU}
\newlength{\@TamPrecioS}
\newlength{\@TamDescuento}
\newlength{\@TamIVA}
\newlength{\@TamPrecioT}
\newlength{\@GrosorLineaFirmas}
\newlength{\@GrosorLineaEncabezado}
\newlength{\@GrosorLineaNotaPrevia}
\newlength{\@GrosorLineaDescripcion}
\newlength{\@GrosorLineaNotaFinal}
\newlength{\@GrosorLineaInfoPag}
\newlength{\@EsquinaLineaEncabezado}
\newlength{\@EsquinaLineaNotaPrevia}
\newlength{\@EsquinaLineaDescripcion}
\newlength{\@EsquinaLineaNotaFinal}
\newlength{\@SombraEncabezado}
\newlength{\@SombraNotaPrevia}
\newlength{\@SombraDescripcion}
\newlength{\@SombraNotaFinal}
\newlength{\@LineaNodoXo}
\newlength{\@LineaNodoYo}
\newlength{\@LineaNodoXf}
\newlength{\@LineaNodoYf}
\newlength{\@SepTotales}
\newlength{\@PosFechaX}
\newlength{\@PosFechaY}
\newlength{\@PosDenomX}
\newlength{\@PosDenomY}
\newlength{\@PosFirmasXuno}
\newlength{\@PosFirmasXdos}
\newlength{\@PosFirmasY}
\newlength{\@PosNotaFinalXuno}
\newlength{\@PosNotaFinalXdos}
\newlength{\@PosNotaFinalY}
\newlength{\@PosControlX}
\newlength{\@PosControlY}
\newlength{\@EspacioVacioInicial}
\newlength{\@EspacioVacioFinal}
\newlength{\@EspacioSignoMonetario}
\newlength{\@SeparaItemsExtra}
\newlength{\@SeparaItemsExtraFinal}
\newlength{\@SeparaInternaEncabezado}
\newlength{\@SeparaInternaNotaPrevia}
\newlength{\@SeparaInternaDescripcion}
\newlength{\@SeparaInternaNotaFinal}
\newlength{\@SeparaDatosResumen}
\newlength{\@SeparaVertical}
\newlength{\@SeparaMinTotales}
\newlength{\@SeparaFilas}
\newlength{\@SeparaFinLineaDesc}
\newlength{\@SeparaTelefonoEmail}
\newlength{\@InicioReporte}
\newlength{\@FirmaFacturaXcorr}
\newlength{\@FirmaFacturaYcorr}
\newlength{\@FirmaReporteXcorr}
\newlength{\@FirmaReporteYcorr}
\newlength{\@FirmaReporteTextPos}
\newlength{\@FirmaReporteFirmaPos}
\newlength{\@FirmaReporteTamLinea}
\newlength{\@FirmaReportePosX}
\newlength{\@FirmaReporteExcesoLinea}
\newlength{\@TempDimen}
\newlength{\tabcolsep@@}
\newlength{\baselineskip@@}

%% Cajas para uso interno.
\newbox{\@CajaDatos}
\newbox{\@CajaResumen}
\newbox{\@CajaFirmaEmisor}
\newbox{\@CajaFirmas}
\newbox{\@CajaFecha}
\newbox{\@CajaDenom}
\newbox{\@CajaControl}
\newbox{\@CajaNotaFinal}
\newbox{\@CajaEspacioVacio}
\newbox{\@CajaPrueba}
\newbox{\@CajaCaracterCR}

%% Contadores.
\newcounter{@IVANoGeneral}%  '0' con sólo IVA G, '1' cuando hay IVA R o IVA A, y '2' cuando hay ambos.
\newcounter{@Firmas}%  '0' para sinfirmas, '1' para unafirma, y '2' para dosfirmas.
\setcounter{@Firmas}{2}
\newcounter{NumItem}%  Número del ítem actual, para uso de algunas macros en el cuadro de la factura y para cuando se usa la opción 'numitem'.
\newcount\AuxFecha

%% Errores, advertencias e informaciones de LaTeX.
\newcommand{\CLASSERROR}[1]{\ClassError{factura}{** #1. **}{}}
\newcommand{\CLASSWARNING}[1]{\ClassWarning{factura}{** #1. **}{}}
\newcommand{\CLASSINFO}[1]{\ClassInfo{factura}{** #1. **}{}}

%% Opciones de la clase.
\DeclareOption{agrupado}{\agrupadotrue}%  Agrupados los elementos totalizadores en el modo en columnas.
\DeclareOption{centradesc}{\centradesctrue}%  Para centrar la descripción de cada ítem.
\DeclareOption{centrado}{\centradotrue}%  Total y exento 'centrado'; sólo para estilo 'filas'
\DeclareOption{conlineasha}{\conlineashatrue}%  Una línea horizontal para cada ítem, como las facturas a mano.
\DeclareOption{conreporte}{\conreportetrue}%  Para imprimir factura y reporte.
\DeclareOption{cotizacion}{\cotizaciontrue}%  Cambia la denominación a 'COTIZACIÓN' y no 'FACTURA'.
\DeclareOption{csv}{\csvtrue}%  Para la salida en modo csv también.
\DeclareOption{dcu}{\dcutrue}%  Descripción, luego cantidad y luego precio unitario.
\DeclareOption{denom}{\denomtrue}%  Agrega el texto con la denominación del documento, para las formas libres que no tienen el número de factura.
\DeclareOption{descuentos}{\descuentostrue}%  Para los descuentos individuales.
\DeclareOption{dosfirmas}{\dosfirmastrue}%  Pone las dos firmas para firmar.
\DeclareOption{duc}{\ductrue}%  Descripción, luego precio unitario y luego cantidad.
\DeclareOption{electro}{\CLASSERROR{Ya no esta' vigente el descuento en la ali'cuota general por pagos electro'nicos}}%  Parche por el Decreto 3.085 de Nikolái, ya no vigente.
\DeclareOption{expandecuadro}{\expandecuadrotrue}%  Para rellenar con líneas en blanco de manera automática.
\DeclareOption{filas}{\filastrue}%  Estilo de totalización en filas.
\DeclareOption{G}{\Gtrue}%  Con esta opción sale la letra 'G' en los items con IVA general, y sale el porcentaje en la base imponible general.
\DeclareOption{impdesc}{\impdesctrue}%  Para el tipo de impuesto en la descripción y no en el precio unitario y total.
\DeclareOption{iva}{\ivatrue}%  Para la columna del IVA.
\DeclareOption{linea}{\lineatrue}%  Para la línea que inutiliza el espacio en blanco.
\DeclareOption{nospanish}{\nospanishtrue}%  No carga babel en castellano ni la codificación T1 de las letras.
\DeclareOption{numitem}{\numitemtrue}%  Para poner una columna extra al principio con el número del ítem.
\DeclareOption{prefactura}{\prefacturatrue}%  La prefactura; con el diseño de la forma libre... para enviar por correo electrónico, fax o imprimir preliminarmente.
\DeclareOption{sincantidad}{\sincantidadtrue}%  Para no poner el campo de la cantidad.
\DeclareOption{sinexpandir}{\sinexpandirtrue}%  No expande ninguna de las dos partes de la factura.
\DeclareOption{sinfirmas}{\sinfirmastrue}%  No agrega los espacios para firmar.
\DeclareOption{sinivaexpresado}{\sinivaexpresadotrue}%  Cuando no se cobra IVA y tampoco se quiere que aparezca la totalización del IVA.
\DeclareOption{siniva}{\sinivatrue}%  No cobra IVA.
\DeclareOption{sinlineahni}{\sinlineahnitrue}%  Estilo 'sinlineahni'.
\DeclareOption{sinlineahtot}{\sinlineahtottrue}%  Estilo 'sinlineahtot'.
\DeclareOption{sinlineahtit}{\sinlineahtittrue}%  Estilo 'sinlineahtit'.
\DeclareOption{sinlineashe}{\sinlineashetrue}%  Estilo 'sinlineashe'.
\DeclareOption{sinlineashi}{\sinlineashitrue}%  Estilo 'sinlineashi'.
\DeclareOption{sinlineash}{\sinlineashtrue}%  Estilo 'sinlineash'.
\DeclareOption{sinlineasve}{\sinlineasvetrue}%  Estilo 'sinlineasve'.
\DeclareOption{sinlineasvi}{\sinlineasvitrue}%  Estilo 'sinlineasvi'.
\DeclareOption{sinlineasv}{\sinlineasvtrue}%  Estilo 'sinlineasv'.
\DeclareOption{sinmarcas}{\sinmarcastrue}%  Para evitar que aparezcan los caracteres especiales cuando la descripción es larga.
\DeclareOption{sinmonedatotales}{\sinmonedatotalestrue}%  Para que no ponga el signo monetario en los totales.
\DeclareOption{sinnumero}{\sinnumerotrue}%  Escribir el texto de la denominación del documento pero sin escribir el número.
\DeclareOption{sinreverso}{\sinreversotrue}%  Para no imprimir el reverso de la factura.
\DeclareOption{sintotales}{\sintotalestrue}%  No escribe los totales al final, por ejemplo para cotizaciones de productos similares.
\DeclareOption{soloreporte}{\soloreportetrue}%  Para imprimir sólo el reporte.
\DeclareOption{subtotal}{\subtotaltrue}%  Para la columna extra del subtotal antes del descuento.
\DeclareOption{todosiva}{\todosivatrue}%  Con esta opción salen todos los IVA así valgan cero.
\DeclareOption{twoside}{\twosidetrue}%  Opción 'twoside' de article para los reportes.
\DeclareOption{unafirma}{\unafirmatrue}%  No agrega el espacio para que firme quien recibe.
\DeclareOption{vencimiento}{\vencimientotrue}%  Escribe fecha de vencimiento, aún con facturas sin crédito.

\ProcessOptions\relax

\LoadClass[12pt]{article}

%% Cargamos algunos paquetes siempre necesarios.
\RequirePackage{xparse}%  Algunos comandos los definimos con \NewDocumentCommand.
\RequirePackage[absolute]{textpos}%  Es necesario 'textpos' para colocar texto en ubicaciones arbitrarias.
\RequirePackage{eso-pic}%  Para el comando \AddToShipoutPicture.
\RequirePackage{xstring}%  Lo usamos sólo para tomar el menos al principio de una longitud.
\RequirePackage{calc}%  'calc' para hacer aritmética con las dimensiones.
\RequirePackage{fp-upn}%  Fundamental el paquete 'fp-upn' para todos los cálculos.
\RequirePackage{numprint}%  Con 'numprint' expresamos los totales en el formato requerido.
\RequirePackage{tikz}%  Para los recuadros del encabezado de la factura,
 \usetikzlibrary{shadows}%  y para el de la descripción.
\RequirePackage{tabularx}%  Con 'tabularx' hacemos la tabla.
\RequirePackage{environ}%  Con 'environ' podemos usar tabularx dentro del ambiente de la factura y escribir texto en el enddef.
\RequirePackage{fancyhdr}%  Las cabeceras fancy; así se use solamente en reporte, hay que cargarlo siempre.
\RequirePackage[calc]{datetime2}%  Para calcular la fecha del vencimiento del crédito.
\RequirePackage{anyfontsize}%  Para las tipografías en tamaños arbitrarios.
\RequirePackage{ifxetex}%  Para poder modificar correctamente la metadata del PDF y agregar la versión de la clase.
\RequirePackage{ifluatex}%  Ditto.
\RequirePackage{zref-savepos}[2016/05/21]%  Para guardar la posición en 'Y' expandiendo el cuadro, para la línea y para la firma en el reporte si la hubiera.
%% Hay que cargar siempre zref-savepos para poder compilar con LuaLaTeX porque este motor no tiene \pdfsavepos, \pdflastypos y zref-savepos las provee cuando no existen (motor LuaLaTex).

%% Gracias a Alan Munn y a wipet (Thanks to Alan Munn and wipet).
%% (https://tex.stackexchange.com/questions/396363/input-a-date-to-datetime2-in-a-format-different-than-iso)
\def\DTMsavedatex#1#2{%
 \expandafter \ifx \csname f=\DTMinformat\endcsname \relax
  \CLASSERROR{\string\DTMsavedatex: formato "\DTMinformat" desconocido}%
 \else
  \csname f=\DTMinformat\expandafter\endcsname#2&{#1}%
 \fi
}
\def\DTMfdef#1{\expandafter\def\csname f=#1\endcsname}
\def\DTMs#1#2{\DTMsavedate{#2}{#1}}
\DTMfdef{yyyy-mm-dd}#1-#2-#3&{\DTMs{#1-#2-#3}}
\DTMfdef{yyyy.mm.dd}#1.#2.#3&{\DTMs{#1-#2-#3}}
\DTMfdef{yyyymmdd}#1#2#3#4#5#6#7#8&{\DTMs{#1#2#3#4-#5#6-#7#8}}
\DTMfdef{dd-mm-yyyy}#1-#2-#3&{\DTMs{#3-#2-#1}}
\DTMfdef{dd.mm.yyyy}#1.#2.#3&{\DTMs{#3-#2-#1}}
\DTMfdef{ddmmyyyy}#1#2#3#4#5#6#7#8&{\DTMs{#5#6#7#8-#3#4-#1#2}}
\DTMfdef{mm-dd-yyyy}#1-#2-#3&{\DTMs{#3-#1-#2}}

%% Macros internas para el texto de la metadata que se va a escribir con el número de la versión de la clase.
\edef\@factura@ver{\expandafter\csname ver@factura.cls\endcsname}
\ExplSyntaxOn
  \seq_set_split:NnV \l_tmpa_seq {~} {\@factura@ver}
  \tl_set:Nx \@factura@vernum {\seq_item:Nn \l_tmpa_seq {2}}

  \seq_set_split:NnV \l_tmpa_seq {~} {\@factura@ver}
  \tl_set:Nx \@factura@fecha {\seq_item:Nn \l_tmpa_seq {1}}
\ExplSyntaxOff
\def\@factura@ver{Hecho con la clase factura-\@factura@vernum\ (\@factura@fecha)}

%% Escribimos la versión de la clase en la metadata del PDF.
\ifxetex
 \special{pdf: docinfo << /Subject (\@factura@ver) >>}
\else
 \ifluatex
  \pdfextension info{/Subject (\@factura@ver)}
 \else
  \pdfinfo{/Subject (\@factura@ver)}
 \fi
\fi

\def\@EstableceOpciones{%
 % Establecemos los parámetros y macros relativas a la fecha.
 \DTMsetstyle{\@FormatoFecha}\DTMsetup{datesep=\@FechaSeparadores}\gdef\DTMinformat{\@FormatoFechaEntrada}
 
 % Tomamos las acciones pertinentes dependiendo de las opciones de la clase (o condicionales activados en el preámbulo que cambian las opciones de la clase)
 \ifprefactura
  \ifcotizacion
   \prefacturafalse
   \cotizacionfalse
   \CLASSWARNING{Las opciones 'prefactura' y 'cotizacion' no son compatibles. Sin efecto ambas opciones}
  \else
   \denomtrue\setcounter{@Firmas}{0}
  \fi
 \fi
 \ifcotizacion\denomtrue\setcounter{@Firmas}{1}\fi

 \ifsinlineasv\booltrue{SinLineasVertExt}\booltrue{SinLineasVertInt}\fi
 \ifsinlineasvi\booltrue{SinLineasVertInt}\fi
 \ifsinlineasve\booltrue{SinLineasVertExt}\fi
 \ifsinlineash\booltrue{SinLineasHoriExt}\booltrue{SinLineaHoriNI}\booltrue{SinLineaHoriTit}\booltrue{SinLineaHoriTot}\fi
 \ifsinlineashi\booltrue{SinLineaHoriNI}\booltrue{SinLineaHoriTit}\booltrue{SinLineaHoriTot}\fi
 \ifsinlineahni\booltrue{SinLineaHoriNI}\fi
 \ifsinlineahtit\booltrue{SinLineaHoriTit}\fi
 \ifsinlineahtot
  \iffilas
   \CLASSWARNING{Las opciones 'sinlineahtot' y 'filas' no son compatibles. Sin efecto opcio'n 'sinlineahtot'}
   \sinlineahtotfalse
  \else
   \booltrue{SinLineaHoriTot}
  \fi
 \fi
 \ifsinlineashe\booltrue{SinLineasHoriExt}\fi

 \ifdosfirmas
  \setcounter{@Firmas}{2}
  \ifunafirma
   \unafirmafalse
   \ifsinfirmas
    \sinfirmasfalse
    \CLASSWARNING{Las opciones para las firmas son excluyentes. Permanece activada 'dosfirmas'}
   \else
    \CLASSWARNING{Las opciones para las firmas son excluyentes. Permanece activada 'dosfirmas'}
   \fi
  \else
   \ifsinfirmas
    \sinfirmasfalse
    \CLASSWARNING{Las opciones para las firmas son excluyentes. Permanece activada 'dosfirmas'}
   \fi
  \fi
 \fi
 \ifunafirma
  \setcounter{@Firmas}{1}
  \ifsinfirmas
   \sinfirmasfalse
   \CLASSWARNING{Las opciones para las firmas son excluyentes. Permanece activada 'unafirma'}
  \fi
 \fi
 \ifsinfirmas\setcounter{@Firmas}{0}\fi 
 
 \ifdcu
  \@cdufalse\ducfalse
  \booltrue{DCU}\boolfalse{DUC}
  \booltrue{PrimeroDescripcion}
 \fi
 \ifduc
  \ifdcu
   \boolfalse{PrimeroDescripcion}\boolfalse{DCU}\dcufalse\ducfalse\@cdutrue
   \CLASSWARNING{Las opciones 'duc' y 'dcu' no son compatibles. Sin efecto ambas opciones}
  \else
   \ductrue\@cdufalse\dcufalse
   \booltrue{DUC}\boolfalse{DCU}
   \booltrue{PrimeroDescripcion}
  \fi
 \fi
 \ifnumitem\booltrue{NumItem}\fi
 \ifsincantidad
  \booltrue{SinCantidad}
  \ifdcu
   \dcufalse\@cdutrue\boolfalse{DCU}\boolfalse{PrimeroDescripcion}
   \CLASSWARNING{La opcio'n 'sincantidad' no es compatible con 'dcu'. Sin efecto opcio'n 'dcu'}
  \fi
  \ifduc
   \ducfalse\@cdutrue\boolfalse{DUC}\boolfalse{PrimeroDescripcion}
   \CLASSWARNING{La opcio'n 'sincantidad' no es compatible con 'duc'. Sin efecto opcio'n 'duc'}
  \fi
 \fi
 \ifdescuentos\booltrue{Descuentos}\fi
 \ifiva\impdesctrue\booltrue{IVA}\fi
 \ifsubtotal
  \ifdescuentos
   \booltrue{Subtotal}
  \else
   \CLASSWARNING{La opcio'n 'subtotal' no aplica sin descuentos individuales. Sin efecto opcio'n 'subtotal'}
  \fi
 \fi

 \ifsinnumero\denomtrue\fi
 \ifsintotales
  \iffilas
   \filasfalse
   \CLASSWARNING{La opcio'n 'sintotales' no aplica para el modo de totalizacio'n en filas. Sin efecto opcio'n 'filas'}
  \fi
  \ifcotizacion
  \else
   \CLASSWARNING{La opcio'n 'sintotales' no debe usarse en facturas sino en cotizaciones. No es va'lida una factura sin totales}
  \fi
 \fi

 \ifcentrado
  \iffilas
  \else
   \centradofalse
   \CLASSWARNING{La opcio'n 'centrado' no es compatible con el modo de totalizacio'n en columnas. Sin efecto opcio'n 'centrado'}
  \fi
 \fi
 \ifagrupado
  \iffilas
   \agrupadofalse
   \CLASSWARNING{La opcio'n 'agrupado' no es compatible con el modo de totalizacio'n en filas. Sin efecto opcio'n 'agrupado'}
  \fi
 \fi

%% (quitado por no vigente)
 % \ifelectro
 %  \boolfalse{Electro}
 %  \electrofalse
 %  \gdef\@NotaInterna{\@TextoDescuentoPresidencial}
 %  \CLASSWARNING{Se establecio' el texto aclaratorio del descuento presidencial como 'nota interna'}
 %  \iffilas
 %   \filasfalse
 %   \CLASSWARNING{La opcio'n 'electro' no es compatible con la opcio'n 'filas'. Sin efecto la opcio'n 'filas'}
 %  \fi
 % \fi

 \iftodosiva \@BIRtrue \@BIAtrue \@Exentotrue \Gtrue \fi
 \ifsiniva\AtBeginDocument{\gdef\@ImpuestoG{0}\gdef\@ImpuestoR{0}\gdef\@ImpuestoA{0}}\fi
 \ifsinivaexpresado
  \iffilas
   \filasfalse
   \CLASSWARNING{La opcio'n 'sinivaexpresado' no es compatible con la opcio'n 'filas'. Sin efecto la opcio'n 'filas'}
  \else
   \AtBeginDocument{\gdef\@ImpuestoG{0}\gdef\@ImpuestoR{0}\gdef\@ImpuestoA{0}}
  \fi
 \fi

 \iflinea
  \ifexpandecuadro
  \else
   \CLASSWARNING{La opcio'n 'linea' requiere la opcio'n 'expandecuadro'. Activando la opcio'n 'expandecuadro'}
   \expandecuadrotrue
  \fi
 \fi
 \ifsinmarcas \def\@CaracterCR{\mbox{}} \def\@CaracterCeldasVacias{\mbox{}} \fi
 \ifnospanish
 \else
  \RequirePackage[spanish]{babel}%  Babel en castellano.
  \ifxetex
  \else
   \ifluatex
   \else
    \RequirePackage[T1]{fontenc}%  Codificación T1 para las tipografías; no aplica para XeLaTeX ni LuaLaTeX.
   \fi
  \fi
 \fi

 \ifexpandecuadro
  \ifsinexpandir
   \expandecuadrofalse
   \sinexpandirfalse
   \CLASSWARNING{Las opciones 'expandecuadro' y 'sinexpandir' no son compatibles. Sin efecto ambas opciones}
  \else
   \boolfalse{ExpAbajo}
  \fi
 \fi
 \ifsinexpandir\boolfalse{ExpAbajo}\expandecuadrofalse\fi

 \ifsinreverso\@reversofalse\fi
 \ifconreporte
  \@reportetrue
  \ifsoloreporte
   \soloreportefalse
   \CLASSWARNING{La opcio'n 'soloreporte' no es compatible con 'conreporte'. Sin efecto 'soloreporte'}
  \fi
 \fi
 \ifsoloreporte\@reportetrue\@facturafalse\@reversofalse\fi

 \ifcsv
  \if@factura
   \newwrite\csv \immediate\openout\csv=\jobname.csv
   \ifsintotales\else\AtEndDocument{\@EscribeTotalesEnCSV}\fi
  \else
   \csvfalse
   \CLASSWARNING{La opcio'n 'csv' require generar la factura. Sin efecto opcio'n 'csv'}
  \fi
 \fi
 
 \iftwoside
  \@twosidetrue\@mparswitchtrue
 \else
  \@twosidefalse\@mparswitchfalse
 \fi}
 
\NewDocumentCommand{\BeforeEndPreamble}{s s +m}{
 \IfBooleanTF{#2}
  {\gpreto{\@BeforeEndPreamble}{#3}}
  {\IfBooleanTF{#1}{\long\gdef\@BeforeEndPreamble{#3}}{\gappto{\@BeforeEndPreamble}{#3}}}}

\AtEndPreamble{
 \@EstableceOpciones
 \ifx\@undefined\@BeforeEndPreamble\else\@BeforeEndPreamble\@EstableceOpciones\fi

 % Escribimos en el .log las opciones cargadas:
 \ifagrupado \CLASSINFO{Opcio'n cargada: agrupado}\fi
 \ifcentradesc \CLASSINFO{Opcio'n cargada: centradesc}\fi
 \ifcentrado \CLASSINFO{Opcio'n cargada: centrado}\fi
 \ifconlineasha \CLASSINFO{Opcio'n cargada: conlineasha}\fi
 \ifconreporte \CLASSINFO{Opcio'n cargada: conreporte}\fi
 \ifcotizacion \CLASSINFO{Opcio'n cargada: cotizacion}\fi
 \ifcsv \CLASSINFO{Opcio'n cargada: csv}\fi
 \ifdcu \CLASSINFO{Opcio'n cargada: dcu}\fi
 \ifdenom \CLASSINFO{Opcio'n cargada: denom}\fi
 \ifdescuentos \CLASSINFO{Opcio'n cargada: descuentos}\fi
 \ifdosfirmas \CLASSINFO{Opcio'n cargada: dosfirmas}\fi
 \ifduc \CLASSINFO{Opcio'n cargada: duc}\fi
 %% (quitado por no vigente)
 % \ifelectro \CLASSINFO{Opcio'n cargada: electro}\fi
 \ifexpandecuadro \CLASSINFO{Opcio'n cargada: expandecuadro}\fi
 \iffilas \CLASSINFO{Opcio'n cargada: filas}\fi
 \ifG \CLASSINFO{Opcio'n cargada: G}\fi
 \ifimpdesc \CLASSINFO{Opcio'n cargada: impdesc}\fi
 \ifiva \CLASSINFO{Opcio'n cargada: iva}\fi
 \iflinea \CLASSINFO{Opcio'n cargada: linea}\fi
 \ifnospanish \CLASSINFO{Opcio'n cargada: nospanish}\fi
 \ifnumitem \CLASSINFO{Opcio'n cargada: numitem}\fi
 \ifprefactura \CLASSINFO{Opcio'n cargada: prefactura}\fi
 \ifsincantidad \CLASSINFO{Opcio'n cargada: sincantidad}\fi
 \ifsinexpandir \CLASSINFO{Opcio'n cargada: sinexpandir}\fi
 \ifsinfirmas \CLASSINFO{Opcio'n cargada: sinfirmas}\fi
 \ifsiniva \CLASSINFO{Opcio'n cargada: siniva}\fi
 \ifsinivaexpresado \CLASSINFO{Opcio'n cargada: sinivaexpresado}\fi
 \ifsinlineahni \CLASSINFO{Opcio'n cargada: sinlineahni}\fi
 \ifsinlineahtit \CLASSINFO{Opcio'n cargada: sinlineahtit}\fi
 \ifsinlineahtot \CLASSINFO{Opcio'n cargada: sinlineahtot}\fi
 \ifsinlineash \CLASSINFO{Opcio'n cargada: sinlineash}\fi
 \ifsinlineashe \CLASSINFO{Opcio'n cargada: sinlineashe}\fi
 \ifsinlineashi \CLASSINFO{Opcio'n cargada: sinlineashi}\fi
 \ifsinlineasv \CLASSINFO{Opcio'n cargada: sinlineasv}\fi
 \ifsinlineasve \CLASSINFO{Opcio'n cargada: sinlineasve}\fi
 \ifsinlineasvi \CLASSINFO{Opcio'n cargada: sinlineasvi}\fi
 \ifsinmarcas \CLASSINFO{Opcio'n cargada: sinmarcas}\fi
 \ifsinmonedatotales \CLASSINFO{Opcio'n cargada: sinmonedatotales}\fi
 \ifsinnumero \CLASSINFO{Opcio'n cargada: sinnumero}\fi
 \ifsinreverso \CLASSINFO{Opcio'n cargada: sinreverso}\fi
 \ifsintotales \CLASSINFO{Opcio'n cargada: sintotales}\fi
 \ifsoloreporte \CLASSINFO{Opcio'n cargada: soloreporte}\fi
 \ifsubtotal \CLASSINFO{Opcio'n cargada: subtotal}\fi
 \iftodosiva \CLASSINFO{Opcio'n cargada: todosiva}\fi
 \iftwoside \CLASSINFO{Opcio'n cargada: twoside}\fi
 \ifunafirma \CLASSINFO{Opcio'n cargada: unafirma}\fi
 \ifvencimiento \CLASSINFO{Opcio'n cargada: vencimiento}\fi

 % Cargamos aquí 'geometry' para que sea después de haber establecido 'twoside'.
 \RequirePackage{geometry}}

%% Escribimos el total como comentario en el .log.
\AtEndDocument{%
 \ifsintotales
 \else
  \if@factura\CLASSINFO{\@@Total: \space\space\space \Moneda \Total}\fi
 \fi}

%</class>
%<*class|def>
%% La geometría para la factura, su reverso, y el reporte.
%% Para la factura, el margen superior está definido desde el encabezado con los datos del cliente;
%%  el margen inferior está definido hasta el final de la descripción (o del cuadro de la factura si no hay descripción).
%%  Hay elementos que salen fuera de estos márgenes, como las firmas, la nota final, la numeración y la fecha.
%<class>\def\GeometriaFactura#1{\gdef\@GeometriaFactura{#1}}
%<class>\def\GeometriaReverso#1{\gdef\@GeometriaReverso{#1}}
%<class>\def\GeometriaReporte#1{\gdef\@GeometriaReporte{#1}}
%<def>\GeometriaFactura{letterpaper,vmargin={5.2cm,5.5cm},hmargin={1cm,1cm},footskip=0pt,  headsep=0pt,    headheight=0pt}
%<def>\GeometriaReverso{letterpaper,vmargin={1cm,1cm},    hmargin={1cm,1cm},footskip=0cm,  headsep=0pt,    headheight=0pt}
%<def>\GeometriaReporte{letterpaper,vmargin={2.8cm,2.5cm},hmargin={3cm,2cm},footskip=1.2cm,headsep=0.777cm,headheight=1.5cm}

%<class>%% Se utiliza '\@Impuesto' como el factor del impuesto, el cual viene predefinido en '\@Alicuota'.
%<class>\def\@ImpuestoG{\@AlicuotaG}
%<class>\def\@ImpuestoR{\@AlicuotaR}
%<class>\def\@ImpuestoA{\@AlicuotaA}
%<def>%% Factores para los distintos IVA:
%<def>\def\@AlicuotaG{0.16}
%<def>%% (quitado por no vigente)
%<def> % \def\@AlicuotaGDescBajo{0.03}
%<def> % \def\@AlicuotaGDescAlto{0.05}
%<def> % \def\@AlicuotaGDescLimite{2000000}
%<def>\def\@AlicuotaR{0.08}
%<def>\def\@AlicuotaA{0.27}
%<def>
%</class|def>
%<*def>
%% i) separador de miles, ii) separador decimal, iii) número de decimales para la moneda iv) y para los porcentajes, v) número de dígitos de la numeración de la factura, vi) formato de la fecha, vii) separadores para la fecha.
\npthousandsep{.}
\npdecimalsign{,}
\def\@DigitosMoneda{2}
\def\@DigitosPorc{2}
\def\@DigitosNumeracion{5}
\def\@FormatoFecha{ddmmyyyy}
\def\@FormatoFechaEntrada{dd-mm-yyyy}
\def\@FechaSeparadores{-}

%% La moneda:
\def\@Moneda{Bs.}
%% (quitado por no vigente)
 % \def\@MonedaBs{Bs.}
 % \def\@MonedaBsS{Bs.S}
 % \def\@MonedaBsF{Bs.F}
\setlength\@EspacioSignoMonetario{.666ex}

%% (quitado por no vigente)
%% Las fechas de la entrada de la reconversión monetaria:
 % \DTMsavedate{FechaBsS}{2018-08-20}
 % \DTMsavedate{FechaBs}{2019-02-15}

%% Al lado de cada ítem va la letra que designa el gravamen que aplica sobre él:
\def\@DenomAlicuotaG{G}
\def\@DenomAlicuotaR{R}
\def\@DenomAlicuotaA{A}
\def\@DenomAlicuotaE{(E)}

%% (quitado por no vigente)
%% Texto predeterminado para la nota interna por el descuento presidencial.
 % \def\@TextoDescuentoPresidencial{Se aplica un descuento del \@AlicuotaGDescPorc\% al IVA General seg\'un Decreto Presidencial 3.085 en Gaceta Oficial 41.239.}

%% Macros que guardan textos que corresponden a nombres de campos:
\def\@@Fecha{Fecha de emisi\'on:}
\def\@@FechaVencimiento{Fecha de vencimiento:}
\def\@@Control{N\textordmasculine\space CONTROL}
\def\@@Datos{Datos del cliente}
\def\@@Resumen{Resumen del servicio}
\def\@@DenomFactura{FACTURA}
\def\@@DenomCotiza{COTIZACI\'ON}
\def\@@RazonSocial{Raz\'on social:}
\def\@@Nombre{Nombre:}
\def\@@RIF{R.I.F.:}
\def\@@CI{C.I.:}
\def\@@Direccion{Direcci\'on fiscal:}
\def\@@Telefono{Tel\'efono:}
\def\@@Email{E-mail:}
\def\@@Contacto{Persona de contacto:}
\def\@@Proveedor{Proveedor:}
\def\@@Cliente{Cliente:}
\def\@@Emisor{Por el emisor}
\def\@@Conforme{Recib\'i conforme}
\def\@@Pagado{PAGADO}
\def\@@NumItem{\#}
\def\@@Cantidad{Cantidad}
\def\@@Descripcion{Descripci\'on}
\def\@@Precio{Precio}
\def\@@PrecioU{Precio U.}
\def\@@PrecioS{Subtotal}
\def\@@Descuento{Dcto.}
\def\@@IVA{IVA}
\def\@@PrecioT{Precio T.}
\def\@@AlicuotaG{IVA G}
%% (quitado por no vigente)
 % \def\@@AlicuotaGDesc{Descuento IVA G}
\def\@@AlicuotaR{IVA R}
\def\@@AlicuotaA{IVA A}
\def\@@BIG{BI G}
\def\@@BIR{BI R}
\def\@@BIA{BI A}
\def\@@BIE{Exento}% La base imponible exenta. ;)
\def\@@Total{TOTAL}
\def\@@DescuentoG{Descuento G}
\def\@@DescuentoR{Descuento R}
\def\@@DescuentoA{Descuento A}
\def\@@DescuentoE{Descuento E}
\def\@@SubtotalG{Subtotal G}
\def\@@SubtotalR{Subtotal R}
\def\@@SubtotalA{Subtotal A}
\def\@@SubtotalE{Subtotal E}

%% i) carácter para separar el teléfono del e-mail del cliente,
%% ii) carácter para los campos vacíos cuando se usa \ItemX,
%% iii) carácter para las celdas vacías cuando la descripción del ítem ocupa más que una línea.
%% iv) carácter de retorno de carro en el cuadro de la factura,
\def\@CaracterSeparaTelefonoEmail{$\diamond$}
\def\@CaracterItemXVacio{...}
\def\@CaracterCeldasVacias{\mbox{}\hfill$\cdot$\hfill\mbox{}}
\def\@CaracterCR{\flechaCR}
% \def\@CaracterCR{$\hookleftarrow$}
% \def\@CaracterCR{$\cwrightarcarrow$}% Usando el paquete 'fnsymbol'.
% \def\@CaracterCR{$\cwsearcarrow$}% Usando el paquete 'fnsymbol'.
\def\flechaCR{%
 \raisebox{-0.4ex}{%
  \begin{tikzpicture}
   \draw[line width=0.075ex] (0,0) arc (20:70:0.75ex) + (0,0) arc (-70:-20:0.75ex) ++ (0.8ex,0.55ex) arc (0:-100:1ex);
  \end{tikzpicture}}}

%% Separación entre el inicio de una columna en la tabla y el inicio del texto.
\tabcolsep=0.15cm

%% Separación entre la última letra y el carácter que indica que se continúa en la línea siguiente en la descripción cuando ocupa más de una línea.
\setlength{\@SeparaFinLineaDesc}{0.15cm}

%% Separación entre el teléfono y el carácter \@CaracterSeparaTelefonoEmail, y luego desde ese carácter hasta el email,
%% para cuando no hay resumen.
\setlength{\@SeparaTelefonoEmail}{0.666em}

%% Grosor de la línea de 'fancyhdr' para las cabeceras, en aquellas donde haya línea.
\setlength{\@GrosorLineaInfoPag}{0.4pt}

%% Posición en X desde la derecha y excedente en la línea de la firma.
\setlength{\@FirmaReportePosX}{2cm}
\setlength{\@FirmaReporteExcesoLinea}{1.5cm}
%</def>
%<class|def>
%<*class>
%% Fijamos en cero los descuentos totales (necesario).
\def\@DescuentoGporc{0}
\def\@DescuentoRporc{0}
\def\@DescuentoAporc{0}
\def\@DescuentoEporc{0}
\def\@DescuentoGnum{0}
\def\@DescuentoRnum{0}
\def\@DescuentoAnum{0}
\def\@DescuentoEnum{0}
\FPset\PrecioUtempx{1}

%% \Fecha, \FormatoFecha, \Credito, \Moneda, \RazonSocial, \Nombre, \RIF, \CI, \Contacto, \Dirección, \Telefono, \Email, \Conforme, \Emisor, \Proveedor.
\NewDocumentCommand{\Fecha}{s m g g}{%
 \IfBooleanTF{#1}
  {\AtEndPreamble{\DTMsavedate{FechaEmision}{#2-#3-#4}}}
  {\AtEndPreamble{\DTMsavedatex{FechaEmision}{#2}}}}
\def\@Fecha{\DTMusedate{FechaEmision}}

\NewDocumentCommand{\FormatoFecha}{o g g}{%
 \IfValueT{#1}{\gdef\@FormatoFechaEntrada{#1}}\IfValueT{#2}{\gdef\@FormatoFecha{#2}}\IfValueT{#3}{\gdef\@FechaSeparadores{#3}}}

\def\Credito#1{\gdef\@Credito{#1}\vencimientotrue}
\def\Moneda#1{\gdef\@Moneda{#1}}
\def\RazonSocial#1{\gdef\@RazonSocial{#1}}
\def\Nombre#1{\gdef\@RazonSocial{#1}\@Nombretrue}
\def\RIF#1{\gdef\@RIF{#1}}
\def\CI#1{\gdef\@RIF{#1}\let\@@RIF\@@CI\let\@CI\@RIF}
\def\Contacto#1{\gdef\@Contacto{#1}}
\def\Direccion#1{\gdef\@Direccion{#1}}
\def\Telefono#1{\gdef\@Telefono{#1}}
\def\Email#1{\gdef\@Email{#1}}
\def\Conforme#1{\gdef\@Conforme{#1}}
\def\Emisor#1{\gdef\@Emisor{#1}}
\def\Proveedor#1{\gdef\@Proveedor{#1}}

%% Los descuentos totales.
\NewDocumentCommand{\Descuento}{s m}{%
 \IfBooleanTF{#1}
  {\@DescGnumtrue\@DescRnumtrue\@DescAnumtrue\@DescEnumtrue
  \gdef\@DescuentoGnum{#2}\gdef\@DescuentoRnum{#2}\gdef\@DescuentoAnum{#2}\gdef\@DescuentoEnum{#2}}
  {\@DescGnumfalse\@DescRnumfalse\@DescAnumfalse\@DescEnumfalse
  \gdef\@DescuentoGporc{#2}\gdef\@DescuentoRporc{#2}\gdef\@DescuentoAporc{#2}\gdef\@DescuentoEporc{#2}}%
 \@DescGtrue\@DescRtrue\@DescAtrue\@DescEtrue\@DescuentoTotaltrue}

\NewDocumentCommand{\DescuentoG}{s m}{%
 \IfBooleanTF{#1}{\@DescGnumtrue\gdef\@DescuentoGnum{#2}}{\@DescGnumfalse\gdef\@DescuentoGporc{#2}}%
 \@DescGtrue\@DescuentoTotaltrue}

\NewDocumentCommand{\DescuentoR}{s m}{%
 \IfBooleanTF{#1}{\@DescRnumtrue\gdef\@DescuentoRnum{#2}}{\@DescRnumfalse\gdef\@DescuentoRporc{#2}}%
 \@DescRtrue\@DescuentoTotaltrue}

\NewDocumentCommand{\DescuentoA}{s m}{%
 \IfBooleanTF{#1}{\@DescAnumtrue\gdef\@DescuentoAnum{#2}}{\@DescAnumfalse\gdef\@DescuentoAporc{#2}}%
 \@DescAtrue\@DescuentoTotaltrue}

\NewDocumentCommand{\DescuentoE}{s m}{%
 \IfBooleanTF{#1}{\@DescEnumtrue\gdef\@DescuentoEnum{#2}}{\@DescEnumfalse\gdef\@DescuentoEporc{#2}}%
 \@DescEtrue\@DescuentoTotaltrue}

\def\Denom#1{\gdef\@Denominacion{#1}}%  La denominación puede ser 'FACTURA', 'COTIZACIÓN' u otra. Se predetermina una de estas dos según las opciones de la clase.

%% Número de la factura con \Num, número de control con \NumControl, y número de serie de control \NumSerieControl.
\def\Num#1{\gdef\@Num{\nplpadding{\@DigitosNumeracion}\npthousandsep{}\numprint{#1}}}
\def\NumControl#1{\gdef\@NumControl{\nplpadding{\@DigitosNumeracion}\npthousandsep{}\numprint{#1}}}%  Si el número de control no se especifica, toma el mismo que el de la factura
\def\NumSerieControl#1{\gdef\@NumSerieControl{#1}}%  Si el número de serie de control no se especifica, toma '00'.

%% Información de página: cabeceras de páginas o pies de página.
\def\InfoPagina#1{\gdef\@InfoPagina{#1}}

%% Texto para la nota final.
\NewDocumentCommand{\NotaFinal}{+m}{\gdef\@NotaFinal{#1}}
\NewDocumentCommand{\Condiciones}{+m}{\gdef\@NotaFinal{#1}\@ClassWarningCondiciones}

%% Advertencia para la migración de cualquier comando para condiciones a nota final.
\def\@ClassWarningCondiciones{\CLASSWARNING{Ya no son condiciones sino nota final, debe hacerse la migración a los nuevos comandos}}

%% Resumen opcional que va a la derecha de los datos del cliente.
\NewDocumentCommand{\Resumen}{+m}{\gdef\@Resumen{#1}}

\NewDocumentCommand{\NotaPrevia}{+m}{\gdef\@NotaPrevia{#1}}

%% Nota interna al cuadro de la factura y nota externa a él (que sale antes de \Descripcion).
\NewDocumentCommand{\NotaInterna}{+m}{%
 \ifdefvoid{\@NotaInterna}{}{\CLASSWARNING{Se sobreescribio' la nota interna previamente definida}}%
 \gdef\@NotaInterna{#1}}
\NewDocumentCommand{\NotaExterna}{+m}{\gdef\@NotaExterna{#1}}

\AtBeginDocument{%
 % Si no se especificó fecha, tomamos la de hoy. Además, advertimos cuando el documento se está compilando en una fecha distinta de la fecha establecida en el encabezado.
 \DTMifsaveddate{FechaEmision}{}{%
  \CLASSWARNING{No se especifico' fecha. Asumiendo la fecha de hoy}%
  \DTMsavedate{FechaEmision}{\the\year-\the\month-\the\day}}%
 \DTMsavenow{Ahora}%
 \DTMsaveddatediff{FechaEmision}{Ahora}{\AuxFecha}%
 \ifnum\AuxFecha=0%
 \else
  \CLASSWARNING{Fecha del documento \@Fecha\space distinta de la fecha de hoy}%
 \fi

%% (quitado por no vigente)
 % Establecemos el signo monetario dependiendo de la fecha de emisión de la factura, para efectos de la Reconversión Monetaria:
 % \ifx\@undefined\@Moneda
 %  \DTMsaveddatediff{FechaEmision}{FechaBsS}{\AuxFecha}%
 %  \ifnum\AuxFecha<0%
 %   \Moneda{\@MonedaBs}%
 %   \LetraNotaInterna{\fontsize{14}{18}}%
 %   \NotaInterna{\FPupn\TotalBsS{\Total{} 100000 div \@DigitosMoneda{} round}Según Decreto Presidencial 3.548, el total en Bolívares Soberanos es de {\bfseries \@MonedaBsS~\numprint{\TotalBsS}}}%
 %   \CLASSWARNING{Se establecio' el texto aclaratorio sobre el total en Boli'vares Soberanos como una nota interna}%
 %  \else
 %   \DTMsaveddatediff{FechaEmision}{FechaBs}{\AuxFecha}%
 %   \ifnum\AuxFecha>-1%
 %    \Moneda{\@MonedaBs}%
 %   \else
 %    \Moneda{\@MonedaBsS}%
 %    \LetraNotaInterna{\fontsize{14}{18}}%
 %    \NotaInterna{\FPupn\TotalBsF{\Total{} 100000 * \@DigitosMoneda{} round}El total en Bolívares Fuertes es de {\bfseries \@MonedaBsF~\numprint{\TotalBsF}}}%
 %    \CLASSWARNING{Se establecio' el texto aclaratorio sobre el total en Boli'vares Fuertes como una nota interna}%
 %   \fi
 %  \fi
 % \fi

 % Efectuamos distintos chequeos para dar errores o advertencias si faltó algo.
 \ifdefvoid{\@RazonSocial}{\CLASSERROR{Hay que definir la razo'n social}}{}%
 \if@reporte
  \ifdefvoid{\@Num}{\CLASSWARNING{No se especifico' el nu'mero de la factura. Puede dar error si el ti'tulo del reporte no se especifica que aparezca en las cabeceras}}{}%
  \ifdefvoid{\@Proveedor}{%
   \ifdefvoid{\@Emisor}
    {\CLASSERROR{Hay que definir el nombre del proveedor cuando se emite un reporte}}%
    {\CLASSWARNING{Se toma el emisor como el proveedor por no haberse especificado este u'ltimo}%
    \let\@Proveedor=\@Emisor}}
   {}%
 \fi
 \if@factura
  \ifdefvoid{\@RIF}{\CLASSERROR{Hay que definir el R.I.F}}{}%
  \ifdefvoid{\@Direccion}{\CLASSERROR{Hay que definir la direccio'n fiscal}}{}%
  \ifdefvoid{\@Telefono}{\CLASSWARNING{Pudiera definirse un nu'mero telefo'nico}}{}%
  \ifdefvoid{\@Email}{\CLASSWARNING{Pudiera definirse una direccio'n de correo electro'nico}}{}%
  \ifdefvoid{\@NotaFinal}{\CLASSWARNING{Pudiera definirse una nota final}}{}%
  \ifcotizacion
   \ifdefvoid{\@FondoFactura}{\CLASSWARNING{En la opcio'n para cotizacion se recomienda agregar el fondo}}{}%
  \fi
  \ifprefactura
   \ifdefvoid{\@FondoFactura}{\CLASSWARNING{En la opcio'n para prefactura se recomienda agregar el fondo}}{}%
  \fi
  \ifsinnumero
   \ifcotizacion
   \else
    \CLASSERROR{La opcio'n para no poner el nu'mero so'lo es compatible con las cotizaciones}%
   \fi
  \fi
  \ifdenom
   \ifsinnumero
   \else
    \ifdefvoid{\@Num}{\CLASSERROR{Hay que definir el nu'mero de la factura}}{}%
   \fi
  \fi
 \fi
 \ifdefvoid{\@Contacto}{\if@Nombre\else\CLASSWARNING{Pudiera definirse una persona de contacto}\fi}{}%

 % El cálculo de la fecha de vencimiento para cuando haya crédito o fecha de vencimiento explícita.
 \ifdefvoid{\@Credito}
  {\gdef\@FechaVencimiento{\DTMusedate{FechaEmision}}}
  {\DTMsaveddateoffsettojulianday{FechaEmision}{\@Credito}{\AuxFecha}%
   \DTMsavejulianday{FechaVencimiento}{\number\AuxFecha}%
   \gdef\@FechaVencimiento{\DTMusedate{FechaVencimiento}}}%
 \ifx\@undefined\@NumControl\let\@NumControl\@Num\fi
 \ifx\@undefined\@NumSerieControl\gdef\@NumSerieControl{00}\fi
 
 % Establecemos los porcentajes de las alícuotas, y redondeamos los descuentos.
 \FPupn\@AlicuotaGporc{\@AlicuotaG{} 100 * \@DigitosPorc{} round clip}%
 \FPupn\@AlicuotaRporc{\@AlicuotaR{} 100 * \@DigitosPorc{} round clip}%
 \FPupn\@AlicuotaAporc{\@AlicuotaA{} 100 * \@DigitosPorc{} round clip}%
 \FPset\@AlicuotaEporc{0}%
 \FPupn\@DescuentoGporc{\@DescuentoGporc{} \@DigitosPorc{} round clip}%
 \FPupn\@DescuentoRporc{\@DescuentoRporc{} \@DigitosPorc{} round clip}%
 \FPupn\@DescuentoAporc{\@DescuentoAporc{} \@DigitosPorc{} round clip}%
 \FPupn\@DescuentoEporc{\@DescuentoEporc{} \@DigitosPorc{} round clip}%
 \FPupn\@DescuentoGfactor{100 \@DescuentoGporc{} /}%
 \FPupn\@DescuentoRfactor{100 \@DescuentoRporc{} /}%
 \FPupn\@DescuentoAfactor{100 \@DescuentoAporc{} /}%
 \FPupn\@DescuentoEfactor{100 \@DescuentoEporc{} /}

 % Una vez definido todo en el preámbulo, los comandos usados para definir campos servirán para mostrar el campo que se definió con ese comando.
 \def\Moneda{\@Moneda}%
 \def\RazonSocial{\@RazonSocial}%
 \let\Nombre\RazonSocial
 \def\RIF{\@RIF}%
 \let\CI\RIF
 \def\Direccion{\@Direccion}%
 \def\Telefono{\@Telefono}%
 \def\Email{\@Email}%
 \def\Contacto{\@Contacto}%
 \def\Emisor{\@Emisor}%
 \def\Conforme{\@Conforme}%
 \def\Proveedor{\@Proveedor}%
 \def\Fecha{\@Fecha}%
 \def\FechaVencimiento{\@FechaVencimiento}%
 \def\Credito{\@Credito}%
 \def\Num{\@Num}%
 \def\NumSerieControl{\@NumSerieControl}%
 \def\NumControl{\@NumControl}%
 \def\NotaFinal{\@NotaFinal}%
 \def\Resumen{\@Resumen}%
 \def\NotaPrevia{\@NotaPrevia}%
 \def\NotaInterna{\@NotaInterna}%
 \def\NotaExterna{\@NotaExterna}%
 \def\TituloReporte{\@TituloReporte}}

%</class>
%<*class|def>
%% Posiciones (X,Y) de los objetos: i) fecha, ii) denominación, iii) nota final, iv) número de control, v) firmas.
%<*class>
\NewDocumentCommand{\PosFecha}{s m m}{
 \setlength{\@PosFechaX}{#2}
 \setlength{\@PosFechaY}{#3}
 \IfBooleanTF{#1}{\booltrue{FechaYarriba}}{\boolfalse{FechaYarriba}}
 \IfBeginWith{#2}{-}{\booltrue{FechaXnegativo}}{\boolfalse{FechaXnegativo}}
 \IfBeginWith{#3}{-}{\booltrue{FechaYnegativo}}{\boolfalse{FechaYnegativo}}}

%</class>
%<def>\PosFecha*{-0cm}{5cm}
%<*class>
\NewDocumentCommand{\PosDenom}{s m m}{
 \setlength{\@PosDenomX}{#2}
 \setlength{\@PosDenomY}{#3}
 \IfBooleanTF{#1}{\booltrue{DenomYarriba}}{\boolfalse{DenomYarriba}}
 \IfBeginWith{#2}{-}{\booltrue{DenomXnegativo}}{\boolfalse{DenomXnegativo}}
 \IfBeginWith{#3}{-}{\booltrue{DenomYnegativo}}{\boolfalse{DenomYnegativo}}}

%</class>
%<def>\PosDenom*{0cm}{5cm}
%<*class>
\NewDocumentCommand{\PosControl}{s m m}{
 \setlength{\@PosControlX}{#2}
 \setlength{\@PosControlY}{#3}
 \IfBooleanTF{#1}{\booltrue{ControlYarriba}}{\boolfalse{ControlYarriba}}
 \IfBeginWith{#2}{-}{\booltrue{ControlXnegativo}}{\boolfalse{ControlXnegativo}}
 \IfBeginWith{#3}{-}{\booltrue{ControlYnegativo}}{\boolfalse{ControlYnegativo}}}

%</class>
%<def>\PosControl*{0cm}{-2.4cm}
%<*class>
\NewDocumentCommand{\PosNotaFinal}{s m m m}{
 \setlength{\@PosNotaFinalXuno}{#2}
 \setlength{\@PosNotaFinalXdos}{#3}
 \setlength{\@PosNotaFinalY}{#4}
 \IfBooleanTF{#1}{\booltrue{NotaFinalYarriba}}{\boolfalse{NotaFinalYarriba}}
 \IfBeginWith{#4}{-}{\booltrue{NotaFinalYnegativo}}{\boolfalse{NotaFinalYnegativo}}}
\let\PosCondiciones\PosNotaFinal
%</class>
%<def>\PosNotaFinal{0cm}{0cm}{-5.3cm}
%% La nota final se define con un X_inicial, un X_final y un Y_inicial (Xo,Xf,Y)
%<*class>

\NewDocumentCommand{\PosFirmas}{m m m}{
 \setlength{\@PosFirmasXuno}{#1}
 \setlength{\@PosFirmasXdos}{#2}
 \setlength{\@PosFirmasY}{#3}
 \IfBeginWith{#3}{-}{\booltrue{FirmasYnegativo}}{\boolfalse{FirmasYnegativo}}}
%</class>
%<def>\PosFirmas{5cm}{1cm}{-2.4cm}
%% La firma se define con un X_inicial, un X_final y un Y_inicial (Xo,Xf,Y)

%% Tamaño y grosor de la línea de las firmas.
%<*class>
\def\LineaFirmas#1#2{
 \setlength{\@TamLineaFirmas}{#1}
 \setlength{\@GrosorLineaFirmas}{#2}}
%</class>
%<def>\LineaFirmas{5.5cm}{0.7pt}

%% Separación mínima horizontal entre los totales en modo de filas y distancia vertical entre esas líneas, fija.
%<class>\def\SepFilas#1#2{\setlength{\@SeparaMinTotales}{#1}\setlength{\@SeparaFilas}{#2}}
%<def>\SepFilas{0.8cm}{2.5pt}

%% Separación vertical entre los elementos de la factura (entre el encabezado y el cuadro de la 'factura', luego desde dicho cuadro hasta la descripción, o hasta los totales en modo de filas).
%<class>\def\SepVertical#1{\setlength{\@SeparaVertical}{#1}}
%<def>\SepVertical{0.2cm}

%% Separación horizontal entre la caja de los datos del cliente y la caja del resumen.
%<class>\def\SepDatosResumen#1{\setlength{\@SeparaDatosResumen}{#1}}
%<def>\SepDatosResumen{0.2cm}

%% Separación horizontal interna entre la línea y el texto en los datos del cliente y en el resumen.
%<class>\def\SepEncabezado#1{\setlength{\@SeparaInternaEncabezado}{#1}}
%<def>\SepEncabezado{6pt}

%% Separación interna entre la línea y el texto de la nota previa.
%<class>\def\SepNotaPrevia#1{\setlength{\@SeparaInternaNotaPrevia}{#1}}
%<def>\SepNotaPrevia{6pt}

%% Separación extra entre los elementos en la factura.
%<class>\def\SepItemsExtra#1{\setlength{\@SeparaItemsExtra}{#1}}
%<def>\SepItemsExtra{0pt}

%% Separación interna entre la línea y el texto de la descripción.
%<class>\def\SepDescripcion#1{\setlength{\@SeparaInternaDescripcion}{#1}}
%<def>\SepDescripcion{6pt}

%<class>\def\SepNotaFinal#1{\setlength{\@SeparaInternaNotaFinal}{#1}}
%<class>\let\SepCondiciones\SepNotaFinal
%<def>\SepNotaFinal{5pt}

%% Grosor de la línea de las cajas del encabezado, curvatura de la esquina y sombra.
%% (si pedimos linea de 0pt, todo queda en cero y no se abre el ambiente TikZ).
%<class>\NewDocumentCommand{\LineaEncabezado}{m g g}{
%<class> \setlength{\@GrosorLineaEncabezado}{#1}
%<class> \IfValueT{#2}{\setlength{\@EsquinaLineaEncabezado}{#2}}
%<class> \IfValueT{#3}{\setlength{\@SombraEncabezado}{#3}}}
%<def>\LineaEncabezado{0.4pt}{10pt}{0pt}

%% Grosor de la línea de la nota previa, curvatura de la esquina y sombra.
%% (si pedimos linea de 0pt, todo queda en cero y no se abre el ambiente TikZ).
%<class>\NewDocumentCommand{\LineaNotaPrevia}{m g g}{
%<class> \setlength{\@GrosorLineaNotaPrevia}{#1}
%<class> \IfValueT{#2}{\setlength{\@EsquinaLineaNotaPrevia}{#2}}
%<class> \IfValueT{#3}{\setlength{\@SombraNotaPrevia}{#3}}}
%<def>\LineaNotaPrevia{0.4pt}{7pt}{0pt}

%% Grosor de la línea de la descripción, curvatura de la esquina y sombra.
%% (si pedimos linea de 0pt, todo queda en cero y no se abre el ambiente TikZ).
%<class>\NewDocumentCommand{\LineaDescripcion}{m g g}{
%<class> \setlength{\@GrosorLineaDescripcion}{#1}
%<class> \IfValueT{#2}{\setlength{\@EsquinaLineaDescripcion}{#2}}
%<class> \IfValueT{#3}{\setlength{\@SombraDescripcion}{#3}}}
%<def>\LineaDescripcion{0.4pt}{0pt}{4pt}

%% Grosor de la línea de la nota final, curvatura de la esquina y sombra.
%% (si pedimos linea de 0pt, todo queda en cero y no se abre el ambiente TikZ).
%<class>\NewDocumentCommand{\LineaNotaFinal}{m g g}{
%<class> \setlength{\@GrosorLineaNotaFinal}{#1}
%<class> \IfValueT{#2}{\setlength{\@EsquinaLineaNotaFinal}{#2}}
%<class> \IfValueT{#3}{\setlength{\@SombraNotaFinal}{#3}}}
%<class>\let\LineaCondiciones\LineaNotaFinal
%<def>\LineaNotaFinal{0pt}{0pt}{0pt}

%% El espacio vertical que hay que avanzar sólo en la primera página del reporte antes de empezar a escribir sobre él. Usable cuando la primera página tiene un membrete y las demás no, es decir, cuando se usa \FondoReporte y no \FondoReporte*.
%<class>\def\InicioReporte#1{\setlength{\@InicioReporte}{#1}}
%<def>\InicioReporte{1.6cm}

%% Tipos de letra para cada elemento posible:
%% Esta es la letra de la descripción de los items; y el espacio \baselineskip de *toda* la tabla de la factura se fija en este comando únicamente (se predeterminó 18pt).
%<class>\def\LetraItems#1{\gdef\@LetraItems{#1\selectfont}}
%<def>\LetraItems{\fontsize{14}{18}}

%% Tipo de letra para la nota interna:
%<class>\def\LetraNotaInterna#1{\gdef\@LetraNotaInterna{#1\selectfont}}
%<def>\LetraNotaInterna{\fontsize{11}{13}\itshape}

%% Tipo de letra para la nota externa:
%<class>\def\LetraNotaExterna#1{\gdef\@LetraNotaExterna{#1\selectfont}}
%<def>\LetraNotaExterna{\fontsize{14}{16.8}}

%% Con este comando se cambia la etra del título de la tabla, es decir, los textos 'Cantidad, 'Descripción', etc...
%<class>\def\LetraTitColumnas#1{\gdef\@LetraTitColumnas{#1\selectfont}}
%<def>\LetraTitColumnas{\fontsize{10}{10}\bfseries}

%% Tipo de letra de los títulos de los totales, por ejemplo los textos: 'B.I. G', 'IVA G'.
%<class>\def\LetraTitTotales#1{\gdef\@LetraTitTotales{#1\selectfont}}
%<def>\LetraTitTotales{\fontsize{12}{12}\bfseries}

%% Letra con la que salen los totales parciales, es decir, los números, exceptuando el total general.
%<class>\def\LetraNumTotales#1{\gdef\@LetraNumTotales{#1\selectfont}}
%<def>\LetraNumTotales{\fontsize{14}{12}}

%% Letra con la que sale la palabra 'TOTAL'.
%<class>\def\LetraTitTotal#1{\gdef\@LetraTitTotal{#1\selectfont}}
%<def>\LetraTitTotal{\fontsize{12}{12}\bfseries}

%% Letra para el total general, el monto.
%<class>\def\LetraNumTotal#1{\gdef\@LetraNumTotal{#1\selectfont}}
%<def>\LetraNumTotal{\fontsize{14}{14}\bfseries}

%% Tipo de letra para los títulos de las cajas del encabezado, por ejemplo: 'Datos del cliente'.
%<class>\def\LetraTitEnc#1{\gdef\@LetraTitEnc{#1\selectfont}}
%<def>\LetraTitEnc{\fontsize{11}{12.5}\slshape}

%% Tipo de letra para el texto en el encabezado de la factura.
%<class>\def\LetraEncFactura#1{\gdef\@LetraEncFactura{#1\selectfont}}
%<def>\LetraEncFactura{\fontsize{11}{12.5}}

%% Tipo de letra para el texto en el encabezado del reporte.
%<class>\def\LetraEncReporte#1{\gdef\@LetraEncReporte{#1\selectfont}}
%<def>\LetraEncReporte{\fontsize{13}{16}}

%% Letra para los montos individuales de la factura y las cantidades.
%<class>\def\LetraNumeros#1{\gdef\@LetraNumeros{#1\selectfont}}
%<def>\LetraNumeros{\fontsize{14}{14}}

%% El tipo de IVA ('G', 'R', 'A' o '(E)') también tiene su tipo de letra.
%<class>\def\LetraTipoIVA#1{\gdef\@LetraTipoIVA{#1\selectfont}}
%<def>\LetraTipoIVA{\fontsize{12}{12}}

%% Letra para la descripción larga (comando \Descripcion).
%<class>\def\LetraDescripcion#1{\gdef\@LetraDescripcion{#1\selectfont}}
%<def>\LetraDescripcion{\fontsize{14}{16.8}}

%% Tipo de letra para el texto bajo las líneas para firmar.
%<class>\def\LetraFirmas#1{\gdef\@LetraFirmas{#1\selectfont}}
%<def>\LetraFirmas{\fontsize{12}{12}}

%% Tipo de letra para la nota previa.
%<class>\def\LetraNotaPrevia#1{\gdef\@LetraNotaPrevia{#1\selectfont}}
%<def>\LetraNotaPrevia{\fontsize{11}{12.5}}

%% Tipo de letra para la nota final.
%<class>\def\LetraNotaFinal#1{\gdef\@LetraNotaFinal{#1\selectfont}}
%<class>\let\LetraCondiciones\LetraNotaFinal
%<def>\LetraNotaFinal{\fontsize{12}{12}}

%% Letra para la denominación del documento ('FACTURA', 'COTIZACIÓN',...).
%<class>\def\LetraDenom#1{\gdef\@LetraDenom{#1\selectfont}}
%<def>\LetraDenom{\fontsize{14}{14}\bfseries}

%% Letra para la numeración de la factura.
%<class>\def\LetraNumeracion#1{\gdef\@LetraNumeracion{#1\selectfont}}
%<def>\LetraNumeracion{\fontsize{14}{14}\bfseries\color{red}}

%% Letra para las fechas (emisión y vencimiento).
%<class>\def\LetraFecha#1{\gdef\@LetraFecha{#1\selectfont}}
%<def>\LetraFecha{\fontsize{14}{15}}

%% Letra para el título del reporte.
%<class>\def\LetraTitReporte#1{\gdef\@LetraTitReporte{#1\selectfont}}
%<def>\LetraTitReporte{\fontsize{16}{22}\bfseries}

%% Letra para las cabeceras de las páginas en el modo reporte.
%<class>\def\LetraInfoPagina#1{\gdef\@LetraInfoPagina{#1\selectfont}}
%<def>\LetraInfoPagina{\fontsize{12}{12}\slshape}

%% Letra para el reverso de la factura.
%<class>\def\LetraReverso#1{\gdef\@LetraReverso{#1\selectfont}}
%<def>\LetraReverso{\fontsize{10}{12}}

%% Letra para el reporte.
%<class>\def\LetraReporte#1{\gdef\@LetraReporte{#1\selectfont}}
%<def>\LetraReporte{\fontsize{12}{20}}

%</class|def>
%<*class>
%% Comando para incluir la firma escaneada en el sitio donde va la línea de la firma del emisor.
\NewDocumentCommand{\FirmaFactura}{m O{0pt} O{0pt}}{% \FirmaFactura{imagen}[x-corr][y-corr]
 \AtEndPreamble{
  \if@factura
   \RequirePackage{graphicx}
   \ifx&#2&\else\setlength{\@FirmaFacturaXcorr}{#2}\fi
   \ifx&#3&\else\setlength{\@FirmaFacturaYcorr}{#3}\fi
   \savebox{\@CajaFirmaEmisor}{#1}
   \AtBeginShipoutNext{%
    \begin{textblock*}{\@TamLineaFirmas}%
     (\textwidth+\oddsidemargin+1in-\@TamLineaFirmas/2-\wd\@CajaFirmaEmisor/2-\@PosFirmasXdos+\@FirmaFacturaXcorr,\@PosFirmasY-\ht\@CajaFirmaEmisor-\dp\@CajaFirmaEmisor+\@FirmaFacturaYcorr)%
     \usebox{\@CajaFirmaEmisor}%
    \end{textblock*}}%
  \fi}}

\NewDocumentCommand{\FirmaReporte}{O{} O{0pt} O{0pt} G{\@Proveedor} O{\@FirmaReportePosX} O{\@FirmaReporteExcesoLinea}}{% \Firma[imagen][x-corr][y-corr]{texto}[posición en X desde la derecha][tamaño linea]
 \renewcommand{\baselinestretch}{1}\selectfont
 \newif\if@Firma@Dibujo
 \newbox{\TextoFirma}%
 \newbox{\DibujoFirma}%
 \ifx&#1&\else\savebox{\DibujoFirma}{#1}\@Firma@Dibujotrue\fi
 \setlength{\@FirmaReporteXcorr}{#2}%
 \setlength{\@FirmaReporteYcorr}{#3}%
 \savebox{\TextoFirma}{#4}%
 \setlength{\@FirmaReporteTextPos}{#5}%
 \setlength{\@FirmaReporteTamLinea}{#6}%
 \@LetraFirmas
 \if@Firma@Dibujo
  \newlength{\@FirmaReporteoddeven}%
  \setlength{\@FirmaReporteFirmaPos}{\zposy{Firma@zref@PosY}sp}%
 \fi
 \ifdim\wd\DibujoFirma>\wd\TextoFirma
  \setlength{\@FirmaReporteTamLinea}{\@FirmaReporteTamLinea+\wd\DibujoFirma}%
 \else
  \setlength{\@FirmaReporteTamLinea}{\@FirmaReporteTamLinea+\wd\TextoFirma}%
 \fi
 \par
 \vspace*{\ht\DibujoFirma}%
 \vspace*{\dp\DibujoFirma}%
 \hfill
 \parbox{\@FirmaReporteTamLinea}{%
  \if@Firma@Dibujo\zsaveposy{Firma@zref@PosY}\fi
  \rule{\@FirmaReporteTamLinea}{\@GrosorLineaFirmas}\\
  \centering{\usebox{\TextoFirma}}}%
 \hspace*{\@FirmaReporteTextPos}\par
 \if@Firma@Dibujo
  \mbox{}%
  \par
  \ifodd\c@page
   \setlength{\@FirmaReporteoddeven}{\oddsidemargin}%
  \else
   \setlength{\@FirmaReporteoddeven}{\evensidemargin}%
  \fi
  \begin{textblock*}{\@FirmaReporteTamLinea}
  (\textwidth+\@FirmaReporteoddeven+1in-0.5\@FirmaReporteTamLinea-0.5\wd\DibujoFirma-\@FirmaReporteTextPos+\@FirmaReporteXcorr,
   \paperheight-\@FirmaReporteFirmaPos-\ht\DibujoFirma-\dp\DibujoFirma+\@FirmaReporteYcorr)
   \noindent\usebox{\DibujoFirma}%
  \end{textblock*}
 \fi}

%% Con \FondoFactura le damos el nombre del archivo que contiene el formato de la factura, por ejemplo para compilaciones con 'prefactura', 'cotizacion' o para incluir una imagen de fondo como 'marca de agua'.
\newcommand{\FondoFactura}[1]{
 \AtEndPreamble{
  \if@factura
   \RequirePackage{graphicx}
   \@FondoFacturatrue
   \gdef\@FondoFactura{#1}
  \fi}}

%% Con \FondoReporte le damos el nombre del archivo que contiene el formato del reporte. Existen dos posibilidades: que dicho fondo salga en todas las páginas o sólo en la primera.
\NewDocumentCommand{\FondoReporte}{s m}{
 \AtEndPreamble{
  \if@reporte
   \@FondoReportetrue
   \RequirePackage{graphicx}
   \gdef\@FondoReporte{#2}
   \IfBooleanT{#1}{\@FondoReporteStartrue}% Con * sale en todas las páginas; sin * sale sólo en la primera página.
  \fi}}

%% Comando para establecer desde el preámbulo el estilo de página (\pagestyle) del reporte, tanto para la primera página como para las siguientes.
\NewDocumentCommand{\EstiloPagReporte}{o m}{
 \gdef\@EstiloPagReporteResto{#2}
 \IfValueTF{#1}{\gdef\@EstiloPagReporteUno{#1}}{\gdef\@EstiloPagReporteUno{\@EstiloPagReporteResto}}}

%% El título para el reporte.
\NewDocumentCommand{\TituloReporte}{s m}{
 \gdef\@TituloReporte{#2}%
 \IfBooleanTF{#1}{% Con '*' sale el título en las cabeceras o pies. Sin '*' no.
  \gdef\@InfoPagina{\let\\\relax\@InfoPaginaTitulo}}%
 {\let\@InfoPagina\@InfoPaginaNoTitulo}}

%% Para cada cliente o para cada personalización se puede tener un archivo .def o .tex con su nombre y ciertas macros cargadas. Ese archivo lo ponemos incluir con '\Cliente' en lugar de usar \input, y la extensión deberá ser .def o .tex.
\newcommand{\Cliente}[1]{%
 \IfFileExists{#1.def}{\input{#1.def}}{
  \IfFileExists{#1.tex}{\input{#1.tex}}{
   \CLASSERROR{No existe #1.def ni #1.tex}}}}

%\iffalse meta-comment
%
% factura.dtx
% Copyright 2019 Emilio Augusto Lazo Zaia <emiliolazozaia@gmail.com>
%
% This work may be distributed and/or modified under the
% conditions of the LaTeX Project Public License, either
% version 1.3c of this license or (at your option) any later
% version. This version of this license is in
%   http://www.latex-project.org/lppl/lppl-1-3c.{html|txt|tex|pdf}
% and the latest version of this license is in
%   http://www.latex-project.org/lppl.txt
% and version 1.3c or later is part of all distributions of
% LaTeX version 2008/05/04 or later.
%
% This work has the LPPL maintenance status `maintained'.
%
% The Current Maintainer of this work is Emilio Augusto Lazo Zaia.
%
% This work consists of the files README and factura.dtx
% with its derived files factura.ins, factura.cls, factura.def,
% factura-ejemplo-aux.tex, factura-ejemplo-firma.tex,
% factura-ejemplo-fondo.tex, factura-ejemplo-membrete.tex,
% factura-ejemplo-cotizacion.tex, factura-ejemplo-prefactura.tex,
% factura-ejemplo-factura01.tex, factura-ejemplo-factura02.tex,
% factura-ejemplo-factura03.tex, factura-ejemplo-factura04.tex,
% factura-ejemplo-factura05.tex, factura-ejemplo-factura06.tex,
% factura-ejemplo-factura07.tex, factura-ejemplo-factura08.tex,
% factura-ejemplo-factura09.tex, factura-ejemplo-factura10.tex,
% factura-ejemplo-reporte1.tex, factura-ejemplo-reporte2.tex,
% factura-ejemplo-reporte3.tex and factura-beispiel-rechnung.tex.
%<*internalbatchfile>
\def\nameofplainTeX{plain}
\ifx\fmtname\nameofplainTeX
\else
 \expandafter\begingroup
\fi
%</internalbatchfile>
%<*batchfile>
\input docstrip.tex
\preamble

Copyright 2019 Emilio Augusto Lazo Zaia <emiliolazozaia@gmail.com>

This work may be distributed and/or modified under the
conditions of the LaTeX Project Public License, either
version 1.3c of this license or (at your option) any later
version. This version of this license is in
 http://www.latex-project.org/lppl/lppl-1-3c.{html|txt|tex|pdf}
and the latest version of this license is in
 http://www.latex-project.org/lppl.txt
and version 1.3c or later is part of all distributions of
LaTeX version 2008/05/04 or later.

This work has the LPPL maintenance status `maintained'.

The Current Maintainer of this work is Emilio Augusto Lazo Zaia.

This work consists of the files README and factura.dtx
with its derived files factura.ins, factura.cls, factura.def,
factura-ejemplo-aux.tex, factura-ejemplo-firma.tex,
factura-ejemplo-fondo.tex, factura-ejemplo-membrete.tex,
factura-ejemplo-cotizacion.tex, factura-ejemplo-prefactura.tex,
factura-ejemplo-factura01.tex, factura-ejemplo-factura02.tex,
factura-ejemplo-factura03.tex, factura-ejemplo-factura04.tex,
factura-ejemplo-factura05.tex, factura-ejemplo-factura06.tex,
factura-ejemplo-factura07.tex, factura-ejemplo-factura08.tex,
factura-ejemplo-factura09.tex, factura-ejemplo-factura10.tex,
factura-ejemplo-reporte1.tex, factura-ejemplo-reporte2.tex,
factura-ejemplo-reporte3.tex and factura-beispiel-rechnung.tex.

\endpreamble
\keepsilent
\askforoverwritefalse
\obeyspaces
\Msg{*****************************************************************}
\Msg{**************************** factura ****************************}
\Msg{*****************************************************************}
\generate{\file{\jobname.cls}{\from{\jobname.dtx}{class}}}
\generate{\file{\jobname.def}{\from{\jobname.dtx}{def}}}
\Msg{\jobname.cls and \jobname.def written.}
\Msg{}
\postamble
\endpostamble
%</batchfile>
%<*internalbatchfile>
\generate{\file{\jobname.ins}{\from{\jobname.dtx}{batchfile}}}
\Msg{\jobname.ins written.}
%</internalbatchfile>
%<*batchfile>
\def\factuno{factura-ejemplo-factura01.tex}
\def\factdos{factura-ejemplo-factura02.tex}
\def\facttres{factura-ejemplo-factura03.tex}
\def\factcuatro{factura-ejemplo-factura04.tex}
\def\factcinco{factura-ejemplo-factura05.tex}
\def\factseis{factura-ejemplo-factura06.tex}
\def\factsiete{factura-ejemplo-factura07.tex}
\def\factocho{factura-ejemplo-factura08.tex}
\def\factnueve{factura-ejemplo-factura09.tex}
\def\factdiez{factura-ejemplo-factura10.tex}
\def\repuno{factura-ejemplo-reporte1.tex}
\def\repdos{factura-ejemplo-reporte2.tex}
\def\reptres{factura-ejemplo-reporte3.tex}
\def\cotiza{factura-ejemplo-cotizacion.tex}
\def\prefac{factura-ejemplo-prefactura.tex}
\def\beispiel{factura-beispiel-rechnung.tex}
\def\auxejemplo{factura-ejemplo-aux.tex}
\def\firma{factura-ejemplo-firma.tex}
\def\fondo{factura-ejemplo-fondo.tex}
\def\membrete{factura-ejemplo-membrete.tex}
\generate{\file{\factuno}{\from{\jobname.dtx}{factuno}}}
\generate{\file{\factdos}{\from{\jobname.dtx}{factdos}}}
\generate{\file{\facttres}{\from{\jobname.dtx}{facttres}}}
\generate{\file{\factcuatro}{\from{\jobname.dtx}{factcuatro}}}
\generate{\file{\factcinco}{\from{\jobname.dtx}{factcinco}}}
\generate{\file{\factseis}{\from{\jobname.dtx}{factseis}}}
\generate{\file{\factsiete}{\from{\jobname.dtx}{factsiete}}}
\generate{\file{\factocho}{\from{\jobname.dtx}{factocho}}}
\generate{\file{\factnueve}{\from{\jobname.dtx}{factnueve}}}
\generate{\file{\factdiez}{\from{\jobname.dtx}{factdiez}}}
\generate{\file{\repuno}{\from{\jobname.dtx}{repuno}}}
\generate{\file{\repdos}{\from{\jobname.dtx}{repdos}}}
\generate{\file{\reptres}{\from{\jobname.dtx}{reptres}}}
\generate{\file{\cotiza}{\from{\jobname.dtx}{cotiza}}}
\generate{\file{\prefac}{\from{\jobname.dtx}{prefac}}}
\generate{\file{\beispiel}{\from{\jobname.dtx}{beispiel}}}
\generate{\file{\auxejemplo}{\from{\jobname.dtx}{aux}}}
\generate{\file{\firma}{\from{\jobname.dtx}{firma}}}
\generate{\file{\fondo}{\from{\jobname.dtx}{fondo}}}
\generate{\file{\membrete}{\from{\jobname.dtx}{membrete}}}
\Msg{.tex examples written.}
\Msg{}
\Msg{*****************************************************************}
\immediate\write18{makeindex -s gind.ist -o \jobname.ind \jobname.idx}
\immediate\write18{makeindex -s gglo.ist -o \jobname.gls \jobname.glo}
\Msg{}
\Msg{Change history and index updated.}
\Msg{*****************************************************************}
\Msg{To produce only package files, type}
\Msg{  pdftex \jobname.dtx}
\Msg{To produce package files and pdf documentation, type}
\Msg{  xelatex \jobname.dtx}
\Msg{  pdflatex factura-ejemplo-fondo.tex}
\Msg{  pdflatex factura-ejemplo-membrete.tex}
\Msg{  xelatex factura-ejemplo-firma.tex}
\Msg{}
\Msg{Also compile all other examples and recompile \jobname.dtx.}
\Msg{(factura-ejemplo-aux.tex is not an example!)}
\Msg{}
\Msg{To manual update index and changelog, type}
\Msg{  makeindex -s gind.ist -o \jobname.ind \jobname.idx}
\Msg{  makeindex -s gglo.ist -o \jobname.gls \jobname.glo}
\Msg{}
\Msg{More information (in spanish) about compiling on README and documentation.}
\Msg{}
\Msg{*****************************************************************}
\Msg{**************************** factura ****************************}
\Msg{*****************************************************************}
\Msg{}
%</batchfile>
%<batchfile>\endbatchfile
%<*internalbatchfile>
\ifx\fmtname\nameofplainTeX
 \expandafter\endbatchfile
\else
 \expandafter\endgroup
\fi
%</internalbatchfile>
%<*driver>
\ProvidesFile{factura.dtx}
%</driver>
%<class>\NeedsTeXFormat{LaTeX2e}[1995/12/01]
%<class>\ProvidesClass{factura}
%<def>\ProvidesFile{factura.def}
%<*class|def>
[2019/08/16 v3.70
%<class>Clase para elaborar facturas legales
%<class>en la Repu'blica Bolivariana de Venezuela.
%<def>Archivo de valores iniciales y condiciones de borde para factura.cls
]

%</class|def>
%<*driver>
\documentclass{ltxdoc}
\usepackage[a4paper,headheight=0.6cm,headsep=0.77cm,footskip=1.2cm,footnotesep=0.54cm,marginparsep=10pt,vmargin={3cm,2cm},hmargin={4cm,2cm}]{geometry}
\usepackage[dvipsnames]{xcolor}
\usepackage[spanish]{babel}
\usepackage{
 array,
 calc,
 catchfilebetweentags,
 etoolbox,
 footnote,
 graphicx,
 ifluatex,
 ifxetex,
 metalogo,
 paralist,
 pgffor,
 regexpatch,
 setspace,
 tikz,
 todo,
 verbatim,
 xparse,
 xspace,
 xstring,
 hypdoc}

% Necesario ejecutar \newif de esta forma:
\csname newif\expandafter\endcsname\csname ifmotorunicode\endcsname
\csname newif\expandafter\endcsname\csname ifejemplos\endcsname

\ejemplostrue

\ifxetex
 \motorunicodetrue
\else
 \ifluatex \motorunicodetrue \fi
\fi

\ifmotorunicode
 \usepackage{fontspec}
 \setmainfont[Ligatures=TeX,Mapping=TeX]{Linux Libertine O}
\else
 \usepackage[T1]{fontenc}
 \usepackage[utf8]{inputenc}
\fi

\overfullrule15pt

\newcommand{\VERBATIM}{\verbatimcolor\setstretch{1.0}}

\xspaceaddexceptions{*}

\hypersetup{
 plainpages=false,
 colorlinks=true,
 linkcolor=blue,
 unicode=true}

\makesavenoteenv{tabular}
\makesavenoteenv{table}

\newcommand\flechaCR{%
 \raisebox{-0.4ex}{%
  \begin{tikzpicture}
  \draw[line width=0.075ex] (0,0) arc (20:70:0.75ex) + (0,0) arc (-70:-20:0.75ex) ++ (0.8ex,0.55ex) arc (0:-100:1ex);
 \end{tikzpicture}}}

\EnableCrossrefs
\CodelineIndex
\OnlyDescription
\RecordChanges

\setlength\parskip{7pt}
\setlength\IndexMin{100pt}

\widowpenalty10000

\newbool{Ejemplo}
\newbool{EjemploIzq}
\newbool{EjemploDer}
\newbool{DescripEjemplo}

\def\ejemplooptcolor{\color{blue}}
\def\ejemplomacrocolor{\color{purple}}
\def\verbatimcolor{\color{OrangeRed}}
\def\macromandcolor{\color{Red}}
\def\macrooptcolor{\color{RoyalBlue}}

\def\tikz{T\emph{i}kZ\xspace}

\catcode`\%=11
\catcode`\|=14
\NewDocumentCommand{\Ejemplo}{s O{0.5} m s O{} G{1} o}{|
 \IfFileExists{factura-#3.pdf}{|
  \catcode`\<=12
  \catcode`\>=12
  \IfBooleanTF{#4}
   {\IfBooleanTF{#1}
    {\booltrue{DescripEjemplo}}
    {\booltrue{Ejemplo}\booltrue{EjemploDer}}}
   {\IfBooleanTF{#1}
    {\booltrue{Ejemplo}\booltrue{EjemploIzq}}
    {\booltrue{Ejemplo}\booltrue{EjemploIzq}\booltrue{DescripEjemplo}}}
  \ifbool{Ejemplo}
   {\ifnum #6=1 \def\Ancho{#2}{\phantomsection\par\mbox{}\hfill{\color{RoyalBlue}\ttfamily \large factura-#3}\hfill\mbox{}\par}{}\fi}
   {\def\Ancho{0.05}}|
  \ifbool{EjemploDer}{\hfill}{}|
  \ifbool{Ejemplo}
   {\begin{minipage}{\Ancho\linewidth}
     \fbox{\includegraphics[page=#6,width=\linewidth-2\fboxrule-2\fboxsep]{factura-#3}}
     \ifbool{DescripEjemplo}{\addcontentsline{toc}{subsection}{factura-#3}}{}
    \end{minipage}}
   {}|
  \ifbool{EjemploIzq}{\hfill}{}|
  \ifbool{DescripEjemplo}
   {\hfill
   | Gracias a cfr y a egreg por el esclarecimiento en cuanto a la necesidad de \makeatletter en \CatchFileBetweenDelims y antes de \regexpatchcmd*.
   | Thanks to cfr and egreg for the clarification on the use of \makeatletter on \CatchFileBetweenDelims and before \regexpatchcmd*.
   | (https://tex.stackexchange.com/questions/380796/how-to-patch-a-macro-with-control-sequences-using-regexpatch)
   \CatchFileBetweenDelims{\DescEjemplo}{factura-#3.tex}{<desc>}{</desc>}[\catcode37=11\catcode13=14\makeatletter]|
   \catcode`\%=11
   \makeatletter
   | Gracias a Werner y a egreg por el uso de \begingroup\ttfamily\string ... \endgroup.
   | Thanks to Werner and egreg for the use of \begingroup\ttfamily\string ... \endgroup.
   | (http://tex.stackexchange.com/questions/342572/error-using-verb-as-the-replace-string-on-xpatchcmd)
   \regexpatchcmd*{\DescEjemplo}{\cC.\cO[*]*}{\c{begingroup}\c{ejemplomacrocolor}\c{ttfamily}\c{string}\0\c{endgroup}\c{xspace}}{}{}|
   \regexpatchcmd*{\DescEjemplo}{(\()(\cL[a-zG]*)(\))}{(\c{begingroup}\c{ejemplooptcolor}\c{ttfamily}\2\c{endgroup})}{}{}|
   \xpatchcmd*{\DescEjemplo}{%% }{\item}{}{}|
   \xpatchcmd*{\DescEjemplo}{}{}{}|
   \makeatother
   \begin{minipage}{\linewidth-\Ancho\linewidth}
    \begin{itemize}
     \setlength{\itemsep}{0pt}
     #5\DescEjemplo
    \end{itemize}
   \end{minipage}
   \IfValueT{#7}{{#7}}}
   {}|
  \boolfalse{Ejemplo}\boolfalse{EjemploIzq}\boolfalse{EjemploDer}\boolfalse{DescripEjemplo}|
  \catcode`\%=9
  \catcode`\<=13
  \catcode`\>=13
 }{\message{^^JLaTeX Warning: No existe factura-#4.pdf; hay que compilarlo antes de compilar este documento.^^J}}}
\catcode`\%=14
\catcode`\|=12

% No quiero que las notas pendientes tengan los números de página:
\patchcmd{\todoitem}{(p.~\pageref{todopage:\thetodo}):}{-}{}{}

\renewcommand{\todoname}{Pendiente}
\renewcommand{\todomark}{Pendiente}
\newcommand{\changesname}{Lista de cambios en las versiones}
\newcommand{\glosname}{Índice de comandos, entornos y opciones}

\makeatletter
 \def\SpecialEnvIndex#1{\@bsphack
  \index{#1\actualchar{\protect\ttfamily#1}
   (entorno)\encapchar usage}%
  \index{entornos:\levelchar#1\actualchar
   {\protect\ttfamily#1}\encapchar usage}\@esphack}
 \def\es@yearl{l año}

 % Definimos el comando para describir opciones, que inexplicablemente falta en doc.sty:
 \def\DescribeOption{\leavevmode\@bsphack\begingroup\MakePrivateLetters
  \Describe@Option}
 \def\Describe@Option#1{\endgroup
  \marginpar{\raggedleft\PrintDescribeOption{#1}}%
  \SpecialOptionIndex{#1}\@esphack\ignorespaces}
 \def\SpecialOptionIndex#1{\@bsphack
  \index{#1\actualchar{\protect\ttfamily#1}
   (opción)\encapchar usage}%
  \index{opciones:\levelchar#1\actualchar{\protect\ttfamily#1}\encapchar
   usage}\@esphack}
 \def\PrintDescribeOption#1{\strut \MacroFont #1\ }

 % Redefinimos el comando del encabezado del índice:
 \def\index@prologue{%
  \section*{\glosname}%
  \markboth{\glosname}{\glosname}}%

 % Adaptamos un poco la lista de cambios:
 \def\changes@#1#2#3{%
  \protected@edef\@tempa{%
   \noexpand\glossary{%
    {\bfseries #1 --- #2}%
    \ifx\saved@macroname\@empty
     \relax\actualchar
    \else
     \saved@indexname
     \actualchar
     \string\verb\quotechar*%
     \verbatimchar\saved@macroname
     \verbatimchar
    \fi
    \levelchar #3}}%
  \@tempa\endgroup\@esphack}
 
 \def\@wrglossary#1{%
  \protected@write\@glossaryfile{}%
  {\string\glossaryentry{#1}{1}}%
  \endgroup\@esphack}%

 \patchcmd{\glossary@prologue}{Change History}{\changesname}{}{}
\makeatother

\title{The \textsf{\jobname} class\thanks{Esta documentación corresponde a la clase \textsf{\jobname}~\fileversion, de fecha \filedate.}}
\author{Emilio Augusto Lazo Zaia\\ \texttt{<emiliolazozaia at gmail.com>}}

\GetFileInfo{\jobname.dtx}

\begin{document}
 \maketitle
 \DocInput{\jobname.dtx}
 \clearpage
 \phantomsection
 \addcontentsline{toc}{section}{\todoname}
 \todos
 \section*{Información específica de esta versión + Reconversión Monetaria}
 % v2.82 Desde la versión v2.82 de esta clase, para poder compilar con \LuaLaTeX{} es necesaria la versión v2.26 del paquete |zref|, de fecha 2016/05/21, y la versión v1.18 de |atbegshi| de fecha 2016/06/09.

 % v2.82 -> Esta versión de |factura| se sacó principalmente para corregir un error introducido en la versión anterior v2.72 en cuanto a las referencias, para agregar la opción |numitem| que enumera los ítems en la factura y para que la línea diagonal (opción |linea|) funcione correctamente siempre.
 
 % v3.00 -> Desde la versión v3.00, todos los ejemplos están incorporados a la documentación, con la descripción de cada uno, de manera automatizada.
 
 % v3.02 -> Esta versión v3.02 se saca porque la versión 3.9 de |fancyhdr| hizo necesaria esta actualización por un error que daba con los reportes por el uso en esta clase de la macro |\fancy@setoffs|. Lo que hacíamos con esa macro de |fancyhdr| lo hicimos ahora de otra forma mejor, dada la incompatibilidad con la nueva versión de ese paquete.%v3.02
 
 % v3.40 -> La versión 3.40 se saca por los cambios en la política impositiva decretados por el Presidente en Decreto número 3085 relativos a descuentos del 3\% y 5\% en el IVA general cuando los pagos se hacen por medios electrónicos; también se corrigió la tasa de IVA adicional.
 
 % v3.40 -> En la versión 3.40 se agregaron nuevas opciones en cuanto al manejo de la fecha de emisión y su formato, posibilidad de facturas a crédito con fecha de vencimiento, campos nuevos adicionales para agregar más textos en la factura, posibilidad de imprimir información en el reverso de la hoja de la factura, posibilidades nuevas en cuanto al posicionamiento de los elementos fijos, diseño \tikz más versátil en los cuadros con bordes, mejor manejo del archivo |.csv|, cambios en los nombres de algunos comandos y opciones.
 
 % v3.40 -> En la misma versión 3.40 se cambió la estructura de la sección de los comandos en esta documentación y se agregó una página de descripción al final de cada ejemplo.
 
 % v3.44 -> El cambio en la política impositiva reflejado en la versión 3.40 queda sin efecto para fines legales, ya no está vigente dicho descuento; sin embargo, no eliminamos el código que lo produce sino que la clase arroja una advertencia cuando se pide usar la opción de descuentos por pago electrónico.
 
 % v3.44 -> Desde la versión 3.44 podemos activar cualquier opción de la clase a través de condicionales en el preámbulo y no solamente como una opción al cargar la clase; también se puede ahora quitar el efecto de cualquier opción previamente dada al cargar la clase, esto con el propósito de hacer más versátil el uso de las opciones y así poder incluirlas también en archivos |.tex| o |.def| personalizados.
 
 % v3.44 -> Adicionalmente desde la versión 3.44 el archivo |.log| de las compilaciones tiene como comentario todas las opciones que fueron pedidas al cargar la clase.
 
 % v3.47 -> La versión 3.47 corrige problemas con la compilación en \LuaLaTeX{} y escribe la versión de la clase en el campo \texttt{Subject} de la {\slshape metadata} del archivo |.pdf| compilado.
 
 % v3.47 -> La versión 3.47 agrega el comando |\Moneda| para cambiar fácilmente la moneda.
 
 % v3.56 -> {\bfseries La versión 3.56 se saca para corregir la versión 3.47 en cuanto a dar cumplimiento al Decreto Presidencial 3.332 sobre la Reconvesión Monetaria.} Desde el 4 de junio del año 2018 la moneda pasará a llamarse Bolívar Soberano (Bs.S) y luego en una fecha aún no definida (la asumimos como el 1 de enero del año 2019) la moneda volverá a llamarse Bolívar (Bs.). Eventualmente el Banco Central de Venezuela emitirá un comunicado para establecer desde cuándo la moneda de curso legal vuelve a llamarse simplemente Bolívar. A partír de esa fecha deberá usarse el comando |\Moneda| para cambiar la moneda a su denominación definitiva mientras no se haya emitido la actualización de esta clase. Esa fecha aún no está determinada, así que la suponemos -y hacemos el cambio de moneda automático en esta clase- como el inicio del año 2019. Antes del 4 de junio, se establece una nota interna automática que escribe el total también en Bolívares Soberanos.
 
 % v3.56 -> Por otra parte, la versión 3.56 elimina la inclusión del paquete |lmodern| y discrimina la inclusión de |fontenc| solamente para la compilación con pdf\LaTeX, que es lo correcto. Además, en esta versión se agrega un campo extra opcional en los ítems de la factura para poder agregar una descripción más larga a cada ítem que podrá ser referenciada en otras partes del documento (por ejemplo en el reporte) pero que no aparece en el cuadro de la factura.

 % v3.56 -> En la versión 3.56 fueron corregidos errores en cuanto al posicionamiento de la línea diagonal y se agregaron las opciones: \texttt{sinlineahni}, \texttt{sinlineahtit} y \texttt{sinlineahtot} para poder quitar las líneas horizontales internas una a una y no como antes, todas a la vez, con \texttt{sinlineashi}.

 % v3.57 -> {\bfseries Esta versión extraordinaria v3.57 se saca para dar cumplimiento al Decreto Presidencial 3.548 sobre la Reconversión Monetaria, publicado en Gaceta Oficial 41.446. Dicho decreto establece que la nueva moneda Bolívar Soberano no tendrá 3 ceros menos como lo establecía el derogado decreto 3.332 sino 5 ceros menos que el Bolívar Fuerte. La Reconversión Monetaria comienza el 20 de agosto del 2018\footnote{Salvo que se postergue otra vez.} y antes de esa fecha ya los precios tienen que estar especificados en ambas monedas, y esta clase está adaptada a esa exigencia.}

 % v3.57 -> Si el usuario especifica a mano la moneda con el comando |\Moneda| -introducido en una reciente versión-, la clase |factura| no interviene en nada en cuanto a la Reconversión Monetaria, ahora, si no se especifica la moneda, puede ocurrir lo siguiente:

 % v3.57 -> Al emitirse una factura con fecha anterior a la reconversión (20 de agosto del 2018), los montos quedarán expresados en Bolívares Fuertes, simbolizado como {\slshape Bs.}, y la clase automáticamente agregará un comentario como {\slshape nota interna} indicando cuál sería el monto en Bolívares Soberanos; esto para dar cumplimiento al Decreto Presidencial que indica que los montos deben estar en expresados en ambas monedas antes de la entrada de la nueva moneda.

 % v3.57 -> Si se emite una factura con fecha posterior a la reconversión, los montos se asumirán como expresados en Bolívares Soberanos, es decir, el usuario debe darlos ya con los 5 ceros quitados, la clase va a simbolizar la moneda como {\slshape Bs.S} y agregará un comentario similar al que acabamos de mencionar pero para explícitar el monto en Bolívares Fuertes, es decir, con 5 ceros más; esto no forma parte del Decreto pero lo quisimos agregar.
 
 % v3.57 -> Luego, la moneda volverá a llamarse Bolívar simplemente y se simbolizará otra vez como {\slshape Bs.}; ese momento aún no lo sabemos pero para los efectos de esta clase vamos a asumirlo como que ocurrirá el 1 de enero del año 2019.
 
 % v3.57 -> Esta será la última versión que sacamos solamente para corregir fechas de la Reconversión Monetaria; si el Gobierno Nacional cambia las fechas nuevamente, el usuario tendrá que hacer uso de la interfaz del paquete |datetime2| para ajustar el funcionamiento de esta clase a los nuevos plazos; para esto aclaramos que el identificador para la entrada de la Reconversión Monetaria se llama {\slshape FechaBsS}\footnote{recordemos que antes de esta fecha, los montos se expresan en Bolívares Fuertes y la nota automática aclara cuánto es en Bolívares Soberanos, y después de esta fecha es al contrario.} y el identificador para el momento (que aún no se ha dicho) desde el cual la moneda vuelve a llamarse Bolívar es {\slshape FechaBs}. Para ajustarlos se usa el comando \cs{DTMsavedate} y las fechas se dan en formato {\slshape yyyy-mm-dd}. Damos como ejemplo lo que actualmente está programado en esta clase:
 
 Desde el 15/02/2019 la moneda nacional se vuelve a llamar “Bolívar” abreviado “Bs” sin “soberano”; esto según Gaceta Oficial 41.587. Quitamos toda referencia a la Reconversión Monetaria.
 
 También en esta versión establecemos la tasa de IVA general como 16\% tal como ya está vigente desde agosto del año 2018.
 
 Ya no está activa la opción |electro| para descuentos en pagos por medios electrónicos porque ya no está vigente el decreto.
 
 Ahora en los totales aparece el signo monetario siempre a menos de que se pida en la clase la opción |sinmonedatotales|.
 
 En esta misma versión se agregó un “gancho” llamado \cs{BeforeEndPreamble} para usos muy específicos.
 
 Las “condiciones del documento” ya no son “condiciones” sino que ahora lo llamamos “nota final” porque es otra nota más aparte de las demás, y va al final del documento; no tiene sentido hablar de condiciones del documento para referirse a ese espacio. Con eso se renombran los 5 comandos, pero dejando por compatibilidad unos alias a los viejos; los nuevos son: \cs{NotaFinal}, \cs{PosNotaFinal}, \cs{SepNotaFinal}, \cs{LineaNotaFinal}, \cs{LetraNotaFinal}.

 La macro \cs{PosNotaFinal} se comporta ahora como \cs{PosFirmas}, es decir, acepta un valor inicial en “x” y uno final.

 \section*{Agradecimientos}
 
 Por pruebas y sugerencias que dieron lugar a mejoras agradezco a Ritguey%
 \ifmotorunicode\footnote{\setmainfont{Symbola}🐧}\else\message{^^J^^JLaTeX Warning: Compila con XeLaTeX.^^J}\fi{}
 Flores Esteves y a Alberto Mijares. Por contribuciones a través de \texttt{StackExchange} agradezco a Alan Munn, cfr, David Carlisle, egreg, Gustavo Mezzetti, Heiko Oberdiek, jfbu, Joseph Wright, Werner y wipet. El motivo por el que se da cada agradecimiento está como comentario en el archivo |.cls| o en el |.dtx|.
 
 \addcontentsline{toc}{section}{\changesname}
 {\makeatletter\def\pfill{\unskip\hfill\@gobble}\makeatother
  \setlength{\columnsep}{0pt}
  \PrintChanges}
 \phantomsection
 \addcontentsline{toc}{section}{\glosname}
 \PrintIndex
\end{document}
%</driver>
% \fi
%
%^^A Gracias a jfbu y a Joseph Wright por \starttodo, \stoptodo, \endorgo.
%^^A Thanks to jfbu and Joseph Wright for \starttodo, \stoptodo, \endorgo.
%^^A (http://tex.stackexchange.com/questions/235680/add-something-to-the-driver-section-inside-the-package-code-using-onlydescripti)
% \long\def\stoptodo#1\starttodo {}%
% \long\def\starttodo#1\stoptodo {}%
%
% \makeatletter
% \def\endorgo #1{%
%  \ifx #1\endinput\expandafter\stoptodo
%  \else #1\expandafter\@gobble\fi}%
% \makeatother
%
%^^A No se puede calcular el checksum con \OnlyDescription.
% \CheckSum{0}
%
% \CharacterTable
%  {Upper-case    \A\B\C\D\E\F\G\H\I\J\K\L\M\N\O\P\Q\R\S\T\U\V\W\X\Y\Z
%   Lower-case    \a\b\c\d\e\f\g\h\i\j\k\l\m\n\o\p\q\r\s\t\u\v\w\x\y\z
%   Digits        \0\1\2\3\4\5\6\7\8\9
%   Exclamation   \!     Double quote  \"     Hash (number) \#
%   Dollar        \$     Percent       \%     Ampersand     \&
%   Acute accent  \'     Left paren    \(     Right paren   \)
%   Asterisk      \*     Plus          \+     Comma         \,
%   Minus         \-     Point         \.     Solidus       \/
%   Colon         \:     Semicolon     \;     Less than     \<
%   Equals        \=     Greater than  \>     Question mark \?
%   Commercial at \@     Left bracket  \[     Backslash     \\
%   Right bracket \]     Circumflex    \^     Underscore    \_
%   Grave accent  \`     Left brace    \{     Vertical bar  \|
%   Right brace   \}     Tilde         \~}
%
% \changes{v1.00}{2013/04/29}{Primera versión.}
% \changes{v2.00}{2015/03/30}{Segunda versión: errores corregidos, comandos y opciones nuevas, ambiente de reporte, documentación completa, ejemplos y convertido todo a formato \texttt{.dtx}.}
% \changes{v2.05}{2015/04/13}{Se cambió de \texttt{4pt} a \texttt{6pt} en \texttt{factura.def} para el margen interno predeterminado en las cajas del encabezado y en la descripción.}
% \changes{v2.05}{2015/04/13}{Se quitó \texttt{UTF8} de \texttt{inputenc} como un paquete que se cargue de forma predeterminada.}
% \changes{v2.05}{2015/04/13}{Advierte que no son compatibles los comandos \cs{Descuento} con la opción \texttt{filas}.}
% \changes{v2.05}{2015/04/13}{Se puso un parche en \cs{@CargaFondoReporte} para arreglar que \cs{topskip} aplicara bien en \cs{@EncabezadoReporte}.}
% \changes{v2.05}{2015/04/13}{Arreglada la interacción entre las opciones \texttt{ivanoexplicito} y \texttt{filas}, y se quitó la línea del total cuando se usa \texttt{ivanoexplicito}.}
% \changes{v2.05}{2015/04/13}{Se renombró la opción \texttt{relleno} a \texttt{exparriba}, a la vez que se quitó \cs{Descripcion*} porque se agregó la opción \texttt{sinexpandir}.}
% \changes{v2.05}{2015/04/13}{Se renombró la opción \texttt{ivanoexplicito} por \texttt{sinivaexpresado}.}
% \changes{v2.05}{2015/04/13}{Se cambiaron los \cs{AtBeginDocument} por \cs{AtBeginShipoutNext} de la factura porque daba problemas al especificar \texttt{conreporte} sin luego agregar el ambiente de reporte.}
% \changes{v2.05}{2015/04/13}{Se colocó \cs{@SeparaMinTotales} dentro de otro \cs{AtBeginDocument} más adecuado.}
% \changes{v2.05}{2015/04/13}{Modificaciones en la documentación y en los ejemplos.}
% \changes{v2.05}{2015/04/13}{Los comandos \cs{ItemX} ahora colocan la línea bien con la opción \texttt{lineaitem}.}
% \changes{v2.1}{2015/08/26}{Advierte cuando la fecha del documento es distinta de la fecha del dia.}
% \changes{v2.1}{2015/08/26}{Los comandos de descuentos totales aceptan un * como argumento para que el descuento pueda ser en moneda y no en porcentaje.}
% \changes{v2.1}{2015/08/26}{Quitados los alias de macros viejas agregados por compatibilidad.}
% \changes{v2.1}{2015/08/26}{Corregida la forma como sale la denominación del impuesto para todos los casos.}
% \changes{v2.1}{2015/08/26}{Corregida y mejorada la documentación.}
% \changes{v2.1}{2015/08/26}{Correcciones en los ejemplos.}
% \changes{v2.1}{2015/08/26}{Mejor manejo de \cs{InicioReporte}.}
% \changes{v2.6}{2016/03/07}{Cambiado el lugar donde se fija el nombre del documento (FACTURA o COTIZACIÓN) porque antes no era correcto y no se podía cambiar el nombre de la cotización a algo distinto como por ejemplo PRESUPUESTO.}
% \changes{v2.6}{2016/03/07}{Se renombró la opción \texttt{sincond} en favor de \texttt{sincondiciones}.}
% \changes{v2.6}{2016/03/07}{Cambiado “condiciones de pago” por “condiciones del documento”.}
% \changes{v2.6}{2016/03/07}{Los descuentos globales aceptan decimales bien; antes no.}
% \changes{v2.6}{2016/03/07}{Se cambió la forma como se llama al ambiente \texttt{tabularx} en la factura creando una nueva macro que se evalúa completamente resultando en la definición de las columnas.}
% \changes{v2.6}{2016/03/07}{Agregadas cuatro opciones: \texttt{sinlineasvi}, \texttt{sinlineasve}, \texttt{sinlineash}, \texttt{sinlineashi}, \texttt{sinlineashe} y renombrada la opción \texttt{lineaitem} por \texttt{conlineasha} aludiendo a “con líneas horizontales para los artículos”.}
% \changes{v2.6}{2016/03/07}{Se arregló la línea oblicua que en algunos casos seguía sin salir.}
% \changes{v2.6}{2016/03/07}{Textos “PAGADO” y “Por el emisor” para las facturas y cotizaciones cuando no se define el emisor del documento.}
% \changes{v2.6}{2016/03/07}{Nueva opción \texttt{dosfirmas} para poner las dos firmas en casos en los que lo predeterminado sea una o ninguna firma (cotizaciones o prefacturas).}
% \changes{v2.6}{2016/03/07}{Nuevo comando \cs{Conforme} para el nombre de quien recibe la factura.}
% \changes{v2.6}{2016/03/07}{Cambiado el lugar en el código donde se llama a la macro que dibuja la línea inutilizadora del espacio en blanco y se hace ahora con \cs{AddToShipoutPicture} para que no se genere un espacio en blanco extra.}
% \changes{v2.6}{2016/03/07}{El ejemplo de la documentación es de una cotización y no de una prefactura; actualizada la documentación y los ejemplos.}
% \changes{v2.6}{2016/03/07}{En el índice ponemos “entorno” en lugar de “environment”.}
% \changes{v2.6}{2016/03/07}{El verbo “poner” es digno para la documentación. No sólo las gallinas ponen.}
% \changes{v2.6}{2016/03/07}{Renombrado el comando \cs{EstiloPag} por \cs{InfoPagina}; también \cs{LetraEncReporte} por \cs{LetraInfoPagina}; por otra parte se cambió \cs{LetraDatosReporte} en favor de \cs{LetraEncReporte}, y por último se cambió \cs{LetraTitRenglon} por \cs{LetraTitColumnas}.}
% \changes{v2.6}{2016/03/07}{Renombrados los estilos de página: \texttt{soloenc} a \texttt{soloarriba}, y \texttt{solopie} a \texttt{soloabajo}.}
% \changes{v2.6}{2016/03/07}{Se mejoró el código relativo al estilo de página del reporte y agregada la macro \cs{EstiloPagReporte} para establecer el estilo de página del reporte.}
% \changes{v2.6}{2016/03/07}{Se agregó una forma para que más de una línea en la descripción de la factura sea mejor manejada.}
% \changes{v2.6}{2016/03/07}{Mejorado el código de la macro interna \cs{@EscribeTipoDeImpuesto}.}
% \changes{v2.6}{2016/03/07}{Cuando la descripción en el cuadro de la factura ocupa más que una línea, agrega al final de cada línea el carácter $\hookleftarrow$.}
% \changes{v2.6}{2016/03/07}{Corregido que salga la “G” obligatoria de IVA General cuando hay un elemento exento y uno con IVA General.}
% \changes{v2.6}{2016/03/07}{Cuatro nuevas definiciones dentro del archivo \texttt{factura.def} para establecer algunos caracteres que antes estaban fijados dentro del archivo \texttt{factura.cls}}
% \changes{v2.6}{2016/03/07}{{\slshape parbox}es para cada elemento en la tabla, para asegurar que la denominación del gravamen quede alineada correctamente.}
% \changes{v2.6}{2016/03/07}{Quitados los paquetes \texttt{xifthen} y \texttt{pbox}. Arreglada la presentación de los datos del cliente cuando el resumen es más grande.}
% \changes{v2.6}{2016/03/07}{Cuando había descuentos totales y se pedía modo de totalización en filas, este modo no se apagaba correctamente y la línea oblicua daba problemas. Arreglado.}
% \changes{v2.72}{2016/05/16}{Las macros internas \cs{@DescripcionAuxCmd} y \cs{@EscribeTipoDeImpuesto} fueron modificadas.}
% \changes{v2.72}{2016/05/16}{Arreglada la alineación a la derecha en algunos casos en los precios.}
% \changes{v2.72}{2016/05/16}{Se incluyó dentro de la macro interna \cs{@ITEM} la opción para soportar \cs{ItemX}.}
% \changes{v2.72}{2016/05/16}{Ahora se procesa el entorno de la factura aún cuando se pide sólo el reporte, para poder tomar las referencias y los totales. Para eso se cambiaron de sitio varias macros internas.}
% \changes{v2.72}{2016/05/16}{Agregadas las etiquetas y las referencias.}
% \changes{v2.82}{2016/06/25}{Agregada la opción \texttt{numitem} para enumerar cada ítem en la factura.}
% \changes{v2.82}{2016/06/25}{Corregido problema de etiquetas repetidas introducido en la versión 2.72.}
% \changes{v2.82}{2016/06/25}{Corregida la alineación vertical entre las distintas columnas agregando \texttt{[t]} al abrir los “{\slshape parboxes}” en los comandos auxiliares que ponen cada celda en un {\slshape parbox} del tamaño de la celda más grande, porque si no se centra verticalmente la celda y se desalinea respecto a la columna de al lado.}
% \changes{v2.82}{2016/06/25}{Ahora dentro de la definición de \cs{@CaracterCeldasVacias} está el código que centra el texto, para poder usar comandos que llenen la caja completa como \cs{dotfill}.}
% \changes{v2.82}{2016/06/25}{Reimplementada la línea diagonal para que funcione en todos los casos.}
% \changes{v2.82}{2016/06/25}{Nueva macro interna \cs{@ColumnasMultiColumnTotalesAuxCmd} para hacer más general la colocación de los totales en el modo de columnas dentro de un \cs{multicolumn}.}
% \changes{v2.82}{2016/06/25}{Quitado el paquete \texttt{MnSymbol} en favor del comando \cs{flechaCR} para la flecha que indica que la descripción ocupa más de una línea.}
% \changes{v3.00}{2017/03/11}{Se agregó la opción \texttt{sinmarcas} para pedir que no se escriba el carácter de fin de linea cuando la descripción ocupa más que una línea y también para evitar que salga el carácter que marca las celdas vacías también cuando la decripción es larga.}
% \changes{v3.00}{2017/03/11}{Nueva macro \cs{SepItemsExtra} para agregar espacio extra entre los ítems en la factura.}
% \changes{v3.00}{2017/03/11}{Opción nueva \texttt{sintotales} para que no se escriba ningún total de los elementos. Útil para cotizaciones de por ejemplo varias formas de un mismo producto.}
% \changes{v3.00}{2017/03/11}{Opción nueva \texttt{centradesc} para centrar el texto en la descripción de cada elemento en la factura.}
% \changes{v3.00}{2017/03/11}{Cambiado el interlineado en la documentación, el tamaño del papel y agregados a la documentación todos los ejemplos que están embebidos en el \texttt{.dtx}, de forma automática.}
% \changes{v3.00}{2017/03/11}{El descuento individual ahora es opcional al pedir la opción, es decir, no hace falta establecer en '0' los elementos donde no haya descuento cuando se usa la opción \texttt{descuentos}.}
% \changes{v3.00}{2017/03/11}{Opción \texttt{subtotal} para agregar una columna extra después del precio unitario y antes del descuento, para reflejar el precio antes de aplicar el descuento individual.}
% \changes{v3.02}{2017/07/16}{Ya no usamos la macro \cs{fancy@setoffs} de \texttt{fancyhdr} en el reporte para establecer valores iniciales por ejemplo de la longitud de la línea en los encabezados; ahora lo hacemos de una forma diferente, incluyendo en el \texttt{.def} una macro llamada \cs{@GrosorLineaInfoPag} que alberga el valor que \texttt{fancyhdr} asigna preterminadamente al grosor de dicha línea. No agregamos un comando para cambiar ese valor por no ser algo de relevancia.}
% \changes{v3.02}{2017/07/16}{Ajustada ligeramente esta lista de cambios y la inclusión de los ejemplos en la documentación.}
% \changes{v3.40}{2017/12/11}{Se agregó la opción \texttt{electro} para dar cumplimiento con el Decreto Presidencial 3085 y la Providencia Administrativa 0048 que establece descuentos en el IVA.}
% \changes{v3.40}{2017/12/11}{Macros nuevas para notas internas al cuadro de la factura y externas a él.}
% \changes{v3.40}{2017/12/11}{Ahora \cs{PosCond} acepta \texttt{*} para que el texto pueda abrir hacia arriba en la posición establecida y no solamente hacia abajo.}
% \changes{v3.40}{2017/12/11}{Agregado el comando \cs{Credito} para la posibilidad de facturas a crédito donde es calculada la fecha de vencimiento de la factura que aparecerá debajo de la fecha de emisión.}
% \changes{v3.40}{2017/12/11}{Ahora \cs{PosFecha} también acepta \texttt{*} para que las fechas (emisión y vencimiento) puedan fijarse también respecto a una posición final y no solamente inicial.}
% \changes{v3.40}{2017/12/11}{El comando \cs{Fecha} acepta \texttt{*} para poder especificar la fecha en formato \texttt{iso} de forma más estándar para que sea tomada correctamente por el paquete \texttt{datetime2}, usado para calcular la fecha de vencimiento.}
% \changes{v3.40}{2017/12/11}{Se agregó la macro \cs{FormatoFecha} para poder cambiar el formato en el que el usuario da la fecha de emisión y cómo ésta es presentada.}
% \changes{v3.40}{2017/12/11}{Si no se especifica la fecha, será tomada la fecha actual y generará una advertencia indicando eso y no un error indicando que no se ha especificado la fecha, como era antes.}
% \changes{v3.40}{2017/12/11}{Actualizada la documentación para reflejar los cambios.}
% \changes{v3.40}{2017/12/11}{Se renombró \cs{@EstiloFecha} por \cs{@EstiloFechas}, y \cs{@PosicionaFecha} por \cs{@PosicionaFechas}.}
% \changes{v3.40}{2017/12/11}{Reestructuración de la documentación.}
% \changes{v3.40}{2017/12/11}{Nueva opción \texttt{vencimiento} para poner la fecha de vencimiento aún en facturas a contado.}
% \changes{v3.40}{2017/12/11}{También \cs{PosDenom} y \cs{PosControl} aceptan \texttt{*}. Cambió en 2mm el margen superior predeterminado, así como también hubo un cambio en la coordenada vertical de la fecha, la denominación del documento y el número de control, ya que antes los predeterminados estaban especificados sin \texttt{*} y luego no.}
% \changes{v3.40}{2017/12/11}{No poner el teléfono ahora produce una advertencia y no un error como antes.}
% \changes{v3.40}{2017/12/11}{Funciones internas para los mensajes informaticos, advertencias y errores de la clase.}
% \changes{v3.40}{2017/12/11}{Se le agregó la posibilidad de \texttt{*} a la macro interna \cs{@DatosAux} para que pudiera ser utilizada la misma macro por los casos donde hay resumen y donde no lo hay.}
% \changes{v3.40}{2017/12/11}{A los cuadros dibujados con \tikz les fue separada la definición del nodo para ponerla dentro del \texttt{.def} en lugar de estar en el \texttt{.cls}.}
% \changes{v3.40}{2017/12/11}{Cuando se especifica 0 como grosor de línea para la descripción o el encabezado, desaparece la línea completamente y la estructura \tikz.}
% \changes{v3.40}{2017/12/11}{Renombrada \cs{PosCond} a \cs{PosCondiciones}.}
% \changes{v3.40}{2017/12/11}{Ahora las condiciones aceptan línea en \tikz, es decir que hay dos macros nuevas: \cs{LineaCondiciones} y \cs{SepCondiciones}. Predeterminadamente las condiciones quedaron como antes, o sea sin diseño de \tikz.}
% \changes{v3.40}{2017/12/11}{Quitada la macro interna \cs{@TamDescrip} porque no hacía nada.}
% \changes{v3.40}{2017/12/11}{Renombrada \cs{SepEncabez} a \cs{SepEncabezado}, tambień \cs{SepDescrip} a \cs{SepDescripcion}, \cs{LineaEncabez} en favor de \cs{LineaEncabezado}, y \cs{LineaDescrip} en favor de \cs{LineaDescripcion}, así como algunas macros internas asociadas a ellas.}
% \changes{v3.40}{2017/12/11}{Macro nueva para una nota previa al cuadro de la factura.}
% \changes{v3.40}{2017/12/11}{La macro \cs{SepEncabezado} fue dividida entre \cs{SepDatosResumen} y \cs{SepEncabezado}. La macro \cs{SepElementos} fue renombrada a \cs{SepVertical} y \cs{LetraTitCuadro} a \cs{LetraTitColumnas}.}
% \changes{v3.40}{2017/12/11}{Contador \texttt{@NumItem} renombrado a \texttt{NumItem}. Las macros para los parámetros de las líneas ahora aceptan dar solamente el primer argumento, por ejemplo para definir que no salga la línea, sólo basta con por ejemplo \cs{LineaEncabezado}{0pt}.}
% \changes{v3.40}{2017/12/11}{Nuevo parámetro opcional después de la descripción en los \cs{Item***} para cambiar el campo que va a salir en el \texttt{.csv}, y ahora se puede especificar un descuento individual en los ítems, que será ignorado si no se carga la opción para descuentos individuales.}
% \changes{v3.40}{2017/12/11}{Actualizada la tasa de IVA adicional.}
% \changes{v3.40}{2017/12/11}{Nueva opción \texttt{iva} para agregar una columna adicional que muestra el IVA de cada item.}
% \changes{v3.40}{2017/12/11}{Modificada la macro interna \cs{@DescripcionAuxCmd} por problemas que daba en algunos casos.}
% \changes{v3.40}{2017/12/11}{Se renombró la opción \texttt{exparriba} por \texttt{expandecuadro}.}
% \DoNotIndex{}
% \changes{v3.40}{2017/12/11}{Modificada la definición de \cs{ItemX} para que maneje la descripción especial para el archivo \texttt{csv}.}
% \changes{v3.40}{2017/12/11}{Nuevo comando \cs{ivaref} para referenciar el IVA del ítem etiquetado.}
% \changes{v3.40}{2017/12/11}{Se cambiaron unos \cs{AtEndDocument} por \cs{AtBeginShipoutNext}. Se hizo necesario siempre el paquete \texttt{eso-pic}.}
% \changes{v3.40}{2017/12/11}{Se quitó la fijación del estilo de página de un \cs{AtBeginDocument} innecesario. Nuevo entorno \texttt{reverso} para el reverso de la factura y nuevo comando para la letra predeterminada del entorno: \cs{LetraReverso}.}
% \changes{v3.40}{2017/12/11}{Cada ejemplo tiene una página al final con su explicación correspondiente.}
% \changes{v3.40}{2017/12/11}{Ahora en los ejemplos, primero va la opción o la macro entre paréntesis y después va la explicación y no al revés como era antes.}
% \changes{v3.40}{2017/12/11}{Quitada la opción \texttt{sincondiciones}. Si no se establecen las condiciones produce una advertencia.}
% \changes{v3.40}{2017/12/11}{Quitada la macro \cs{mostrar}. Ahora cada texto a mostrar se hace con el mismo comando con el que se definió.}
% \changes{v3.40}{2017/12/11}{Algunos condicionales cambiados.}
% \changes{v3.40}{2017/12/11}{Otra vez corregida la macro interna \cs{@DescripcionAuxCmd} y arreglado el espaciado entre los ítems; ojalá que esta vez sí esté bien.}
% \changes{v3.44}{2018/03/13}{Reestructurado el archivo \texttt{.cls} para aceptar prender y apagar las opciones de la clase a través de condicionales de \TeX{} con el mismo nombre de la opción.}
% \changes{v3.44}{2018/03/13}{Se cambió un \cs{AtBeginDocument} que escribía texto en la hoja por un \cs{AfterEndPreamble}.}
% \changes{v3.44}{2018/03/13}{Se acabó el descuento del IVA. Ahora advierte cuando se usa, posibilitando su uso después si llegan a volver a activar la medida.}
% \changes{v3.44}{2018/03/13}{Cada opción cargada escribe un comentario en el archivo \texttt{.log} para depuración.}
% \changes{v3.47}{2018/04/17}{No compilaba con \LuaLaTeX{} por diferencias en el motor. Ya compila.}
% \changes{v3.47}{2018/04/17}{Agrega la versión de la clase bajo \texttt{Subject} en la metadata de los PDF compilados.}
% \changes{v3.47}{2018/04/17}{Se agrega la macro \cs{Moneda} para cambiar el signo monetario a usar.}
% \changes{v3.47}{2018/04/17}{Se da cumplimiento al Decreto Presidencial 3.332 de la reconversión monetaria.}
% \changes{v3.47}{2018/04/17}{Cambiados algunos de los \cs{immediate}\cs{write} por \cs{write}.}
% \changes{v3.56}{2018/05/09}{Quitamos \texttt{lmodern} de todas partes y mejoramos la inclusión discrecional de \texttt{fontenc}.}
% \changes{v3.56}{2018/05/21}{Agregada la opción de una descripción larga para cada ítem de la factura únicamente para referenciarla luego en otra parte del documento.}
% \changes{v3.56}{2018/05/21}{La implementación anterior del Bolívar Soberano tenía defectos; ya están corregidos.}
% \changes{v3.56}{2018/05/21}{Volvimos a agregar la macro interna \cs{@TamDescripcion} quitada en la versión v3.40. Sin ella la línea no salía bien con la opción \texttt{filas}.}
% \changes{v3.56}{2018/05/21}{Creamos condicionales internos: \cs{if@ExpandeCuadro} y otros para evaluar la línea inutilizadora del espacio en blanco, así como también unas macros que guardan los valores de \cs{baselineskip} y \cs{tabcolsep} y otros cambios en la línea.}
% \changes{v3.56}{2018/05/21}{Nuevas opciones \texttt{sinlineahni}, \texttt{sinlineahtit} y \texttt{sinlineahtot} para no poner la línea horizontal antes de la nota interna, después del título de las columnas y antes de los totales.}
% \changes{v3.56}{2018/05/21}{Agregamos el paquete \texttt{anyfontsize}.}
% \changes{v3.57}{2018/08/02}{El Gobierno Nacional posterga la Reconversión Monetaria. Cambiamos un poco la forma como estaba programado esto. El usuario podrá cambiar las fechas que intervienen en eso, por si ocurre otro cambio.}
% \changes{v3.70}{2019/08/17}{Nueva tasa de IVA general del 16\% vigente desde agosto del 2018.}
% \changes{v3.70}{2019/08/17}{Establecido el 15/02/2019 como la fecha en la que ocurre que la nueva moneda pierde el adjetivo “soberano” para llamarse “bolívar” simplemente.}
% \changes{v3.70}{2019/08/17}{Nueva macro interna \cs{@EstableceOpciones} para poder evaluarla doblemente, una antes del -nuevo también- gancho \cs{BeforeEndPreamble} y otra después de él, porque en ese gancho se puede escribir código que altere las opciones y deberán ser evaluadas nuevamente para la consistencia entre ellas.}
% \changes{v3.70}{2019/08/17}{Inhabilitada la opción \texttt{electro} pero se deja el código todavía, aunque comentado.}
% \changes{v3.70}{2019/08/17}{Se comentó todo el código referente a la Reconversión Monetaria.}
% \changes{v3.70}{2019/08/17}{Implementada la función de que agregue el signo monetario en los totales, y una opción \texttt{sinmonedatotales} para evitarlo.}
% \changes{v3.70}{2019/08/17}{Ya no hablamos de condiciones sino de “nota final”, que es aquello que va después de la descripción larga y que tiene posición fija. Ese espacio se puede usar para distintas cosas, por lo que hablar de ese espacio como el de las condiciones del documento no tiene mucho sentido. De manera que los comandos renombrados pasaron a ser \cs{NotaFinal}, \cs{PosNotaFinal}, \cs{SepNotaFinal}, \cs{LineaNotaFinal}, \cs{LetraNotaFinal}. Se mantienen los comandos anteriores por compatibilidad.}
% \changes{v3.70}{2019/08/17}{Corregido el posicionamiento en “x” de la nota final.}
% \changes{v3.70}{2019/08/17}{Ahora la macro \cs{PosNotaFinal} acepta un intervalo en “x” para posicionar la nota final, es decir que funciona como \cs{PosFirmas}.}
% \changes{v3.70}{2019/08/17}{Mejorada la macro interna que escribe la fecha en la metadata; ya no usa \texttt{xstring} sino \LaTeX3. Quitado código espúreo de \texttt{xstring} en las macros que permiten fijar las posiciones de los elementos fijos.}
%
%
% \tableofcontents
%
%
% \section*{Brief description in english}
%
% |factura| is the spanish word for `invoice', so this is a \LaTeX{} class for typesetting and calculating invoices, made taking into account requirements of {\slshape SENIAT} legislation (tax collector entity on the Bolivarian Republic of Venezuela), but its use is not restricted to Venezuela because all variables and displayed text can be redefined by invoking commands or editing.
%
% \newpage
% \setstretch{1.4}
%
%
% \section{Introducción}
%
% La clase de \LaTeX{} |factura| está diseñada para su uso en la República Bolivariana de Venezuela, conforme a los requerimientos fijados por las providencias del {\slshape SENIAT} 0591, 0257, 0071 y 0048, válidas éstas para agosto del año 2019.
%
% Con |factura| se pueden elaborar facturas en formas libres fiscales\footnote{En cuanto a las formas libres, cada hoja del talonario de facturas legales tiene un encabezado con los datos del emisor de la factura y un pie de página con la información de identificación de la imprenta, en tamaño de hoja carta entera; esto es lo que se llama “forma libre”.}, cotizaciones y reportes de servicios, dejando todos los cálculos a \TeX{}.
%
% El diseño de las facturas hechas con esta clase consiste en un encabezado con los datos que identifican al cliente, con al lado un -opcional- resumen; estas partes de la factura están encerradas en recuadros. Si no se especifica el resumen saldrá solamente una de estas cajas en el encabezado que es la de los datos del cliente.
%
% Luego de esto viene un -opcional- recuadro con una nota donde puede ir la forma de pago, luego un cuadro con los campos: cantidad, descripción, precio unitario y precio total\footnote{Pueden haber variaciones en estos campos controlables con las opciones de la clase (véanse subsecciones \ref{subsec:columnas}, \ref{subsec:descuentos} y \ref{subsec:sincantidad}).}, con los elementos a facturar, cada uno en una línea, luego los totales, luego una -opcional- nota dentro del cuadro de la factura, otra nota (opcional también) después del cuadro de la factura y al final viene una opcional descripción larga que sale en un recuadro. La factura puede tener también un reverso con más información.
%
% En fin, la factura consta de:
% \begin{enumerate}[\bfseries i)]
%  \item el encabezado con los datos del cliente y el -opcional- resumen,
%  \item la opcional nota previa al cuadro de la factura,
%  \item el cuadro con los elementos que se están facturando y las totalizaciones,
%  \item la opcional nota interna al cuadro de la factura,
%  \item la opcional nota externa al cuadro de la factura,
%  \item una opcional descripción larga, e
%  \item información opcional en el reverso de la hoja.
% \end{enumerate}
%
% Aparte, el usuario puede escribir un reporte largo asociado a la factura, es decir, esta clase sirve también para escribir reportes.
%
% \newpage
%
%
% \section{Instalación}
%
% El paquete que está en CTAN consta de un archivo |README| en formato |UTF-8|, esta documentación compilada como |factura.pdf| y el archivo |factura.dtx|, del cual se generan todos los demás archivos.
%
% Al compilar |factura.dtx| va a producirse |factura.cls| y |factura.def|, además de todos los ejemplos en archivos |.tex| y |factura.ins|.
%
% \noindent Si queremos solamente generar los archivos sin compilar la documentación, debemos ejecutar:
%
% \noindent {\color{RoyalBlue}|pdftex factura.dtx|}
%
% \noindent Para generar los archivos y también compilar la documentación, debemos ejecutar los siguientes pasos:
%
% \begin{enumerate}[\bfseries 1. ]
%
% \item {\color{RoyalBlue}|xelatex factura.dtx|}\footnote{Es preferible compilar el |.dtx| con \XeLaTeX{} o con \LuaLaTeX pero también puede hacerse con pdf\LaTeX.}
%
% Luego de compilar |factura.dtx| hay que compilar el fondo de ejemplo, el membrete de ejemplo y la firma de ejemplo; estos tres archivos emulan haber digitalizado una hoja de una factura del talonario fiscal, un membrete para los reportes y también una firma que puede agregarse a los documentos a producir. Estos ejemplos se usarán en algunos de los ejemplos de formatos que incluímos y por eso es necesario tenerlos compilados en |.pdf| previamente, ya que su inclusión en los ejemplos es en formato compilado |.pdf|.
%
% El fondo y membrete de ejemplo los debemos compilar así:
%
% \item {\color{RoyalBlue}|pdflatex factura-ejemplo-fondo.tex|}
% \item {\color{RoyalBlue}|pdflatex factura-ejemplo-membrete.tex|}
%
% La firma escaneada de ejemplo necesita compilarse con \XeLaTeX{} porque son códigos en PSTricks y debe producirse un |.pdf|. Se compila así:
%
% \item {\color{RoyalBlue}|xelatex factura-ejemplo-firma.tex|}
%
% Al tener el fondo, el membrete y la firma compilados en |.pdf| según los pasos anteriores, podremos compilar cualquiera de los ejemplos en |.tex| que el archivo |factura.dtx| genera, y eso lo debemos hacer, ya que todos ellos (en su versión en |.pdf|) forman parte de la documentación. Los podemos compilar así:
%
% \item {\color{RoyalBlue}|pdflatex factura-ejemplo-prefactura.tex|}
%
% De esta forma como compilamos {\ttfamily factura-ejemplo-prefactura} compilamos todos los demás ejemplos: la cotización, las facturas del 1 hasta la última, el reporte del 1 hasta el último, y finalmente la factura para otra legislación.\footnote{Podemos también compilar con \LuaLaTeX{} o con \XeLaTeX{}.}
%
% Al tener todo esto hecho volvemos a compilar |factura.dtx| para producir esta documentación, puesto que el |.dtx| necesita ser compilado más de una vez para que la documentación quede correctamente elaborada:
%
% \item {\color{RoyalBlue}|xelatex factura.dtx|}
%
% Los comandos para generar el índice y la lista de cambios, que automáticamente se ejecutan al compilar |factura.dtx| son:
%
% \item {\color{RoyalBlue}|makeindex -s gind.ist -o factura.ind factura.idx|}
% \item {\color{RoyalBlue}|makeindex -s gglo.ist -o factura.gls factura.glo|}
%
% {\slshape No es necesario actualizar los índices con los dos comandos anteriores porque esos dos comandos los ejecuta el motor de compilación al actuar sobre }|factura.dtx|.
%
% \end{enumerate}
%
%
% \section{Archivos incluídos en el paquete}
%
% Este paquete incluye varios ejemplos, entre ellos la emulación de un diseño de factura entregado por alguna imprenta, que nos sirve para nuestros ejemplos de prefacturas, un diseño de membrete que nos sirve para los reportes, una firma que emula ser una firma real escaneada para su inclusión en algunos casos, así como varios ejemplos de facturas, prefacturas, cotizaciones y reportes. Cada uno de ellos tiene al principio del documento fuente la aclaratoria de qué es lo que hace, para diferenciarlo de los demás ejemplos.
%
% \noindent \begin{tabular}{rl}
%  \bfseries Archivo & \bfseries Descripción \\ \hline
%  |factura.dtx| & Archivo principal del que todos los demás se generan, \\
%  |factura.pdf| & Esta documentación, \\
%  |README| & Archivo con la descripción corta, \\
%  |factura-ejemplo-firma.tex| & Ejemplo de firma digitalizada, \\
%  |factura-ejemplo-fondo.tex| &  Elemplo de una forma libre fiscal digitalizada, \\
%  |factura-ejemplo-membrete.tex| & Ejemplo de membrete para los reportes, \\
%  |factura-ejemplo-aux.tex| & Archivo auxiliar que se incluye en todos los ejemplos y \\
%                            & sirve para que cada ejemplo conste de su descripción, \\
%  |factura-ejemplo-prefactura.tex| & Ejemplo de “{\slshape prefactura}”, \\
%  |factura-ejemplo-cotizacion.tex| & Ejemplo de cotización, \\
%  |factura-ejemplo-factura*.tex| & Ejemplos de facturas, \\
%  |factura-ejemplo-reporte*.tex| & Ejemplos de reportes, \\
%  |factura-beispiel-rechnung.tex| & Un ejemplo de una factura adaptada a otra legislación. \\ \hline
% \end{tabular}
%
% \IfFileExists{factura-ejemplo-prefactura.pdf}{%
%  \newpage\section{Ejemplo preliminar\label{sec:ejemplo}}
%  Lo que sigue es un ejemplo del aspecto final de una {\slshape “prefactura”}, que está bajo el nombre de “|factura-ejemplo-prefactura.tex|” y más adelante explicaremos en qué consiste:
%
%  \noindent
%  \mbox{}\hfill
%  \vfil
%  \fbox{\includegraphics[width=0.9\linewidth]{factura-ejemplo-prefactura.pdf}}
%  \hfill\mbox{}
%  \vfil\mbox{}
%  \newpage}
% {\message{^^JLaTeX Warning: No existe factura-ejemplo-prefactura.pdf; hay que compilarlo antes de compilar este documento.^^J}}
%
%
% \section{Geometría de la hoja\label{sec:geohoja}}
%
% El espacio que la imprenta deja en blanco en la forma libre está todo disponible para la impresión de la factura con esta clase. Este espacio se delimita con el paquete |geometry|. Los valores predeterminados que tienen que ver con la geometría de la hoja están todos definidos dentro del archivo |factura.def|.
%
% El margen superior que se da con |geometry| está contabilizado desde el inicio del papel\footnote{El papel predeterminado es tamaño carta.} hasta donde empieza el encabezado con los datos del cliente, es decir que el encabezado es el primer elemento desde arriba hacia abajo en una factura.
%
% El margen inferior de |geometry| empieza donde termina la descripción del servicio -o el fin del cuadro de la factura, en caso de que no se agregue un texto de descripción larga-, hasta el final del papel.
%
% Para aclarar lo anterior es necesario decir que hay textos con posiciones fijas que están fuera de estos márgenes, como la fecha de emisión, la “denominación” del documento (que es “FACTURA”, “COTIZACIÓN” u otro especificable), el número de la factura, el número de control (cuando aplica), las nota final y los espacios para las firmas; todas estas partes del documento están fijas en posiciones arbitrarias fuera de los márgenes dejados por |geometry|. Estos textos son fijos para darles posiciones que no dependan del contenido particular de cada factura sino del diseño del talonario que ya tenemos impreso, que siempre será igual, y para el cual estamos adaptando esta clase |factura|.
%
% Las ubicaciones de los textos fijos también tienen valores predefinidos dentro de |factura.def|. Cualquiera de esos valores puede cambiarse mediante comandos, tal como se fijan los valores predeterminados en dicho archivo |.def|; esto significa que se puede personalizar a cualquier diseño del talonario entregado por la imprenta, basta con entender qué hace cada comando (explicado en este documento), consultar el |.def| o ver los ejemplos.
%
% En algunos casos las formas libres tienen impreso el número de la factura y no es controlable esto por el emisor (como ocurre con el número de control que la legislación obliga a que siempre venga impreso en cada hoja del talonario). En otros casos el emisor tiene la libertad de escribir el número de la factura y controlarlo él mismo. Se puede hacer funcionar la clase según cualquiera de estas dos formas con una opción en |\documentclass|, es decir, podemos imprimir el número de la factura donde queramos, o también podemos asumirlo impreso en el talonario, según sea el caso del talonario que tengamos.
%
% Al usarse esta clase se deben primero hacer medidas en las facturas que la imprenta nos entrega para fijar los márgenes y determinar dónde poner los textos fijos que se deseen cambiar de la posición que viene predeterminada.
%
% El archivo |factura.def| tiene todas aquellas definiciones y valores que pueden ser ajustados; basta usarlos en nuestros documentos de la misma forma como están predefinidos allí; eso aplica no sólo para los de la geometría de la hoja sino para cualquier ajuste que esté en ese archivo o cualquier campo que tenga algún texto o palabra predefinida para la legislación venezolana.
%
%
% \section{Uso de la clase}
%
% El documento debe comenzarse con |\documentclass{factura}| con las opciones necesarias, que se explicarán en la sección \ref{sec:opciones}. Después se definen a través de macros {\bfseries en el preámbulo} todas las variables y dimensiones para personalizar el documento, esto es las posiciones de los textos fijos, tipos de letra (opcional), datos del cliente, un resumen (opcional), fecha y demás. Despues de |\begin{document}| va el cuadro de la factura propiamente, para lo cual la clase define un entorno llamado |factura|
% \DescribeEnv{factura}
% (|\begin{factura}...\end{factura}|), en cuyo interior se escriben únicamente los elementos a facturar a través de comandos del tipo |\Item|\footnote{Es posible incluir elementos gravados en IVA con alícuota general (12\%), alícuota reducida (8\%), alícuota adicional (27\%), así como también elementos exentos de impuesto al valor agregado, con \cs{Item}, \cs{ItemR}, \cs{ItemA} e \cs{ItemE} respectivamente.}, especificando su cantidad, descripción y precio unitario como argumentos, en ese orden.
%
% Finalizado el entorno se puede escribir (opcionalmente) una descripción larga o una aclaratoria de lo que se factura; esto se hace con el comando |\Descripcion|.
%
% Algunas de las posibilidades de cambio de estilo de esta clase consisten en que puede cambiarse la forma como es presentado cualquiera de los textos, se pueden suprimir las líneas verticales y horizontales del cuadro de la factura, se pueden agregar líneas horizontales separando cada ítem, también se puede dibujar una línea oblicua que proteja el cuadro de la factura de adiciones posteriores de texto, entre otras prestaciones más.
%
% Si el emisor lo desea puede digitalizar su firma personal y colocarla en el espacio ya dispuesto para firmar, con un ajuste mínimo para que salga correctamente sobre la línea.
%
% También el usuario tiene la posibilidad de producir una {\slshape “prefactura”}; así le llamamos a un documento sin validez legal pero que puede ser enviado por fax o por correo electrónico. Este documento tiene el mismo aspecto de la factura fiscal si se digitaliza la forma libre vacía\footnote{Quitándole las numeraciones obviamente. También puede diseñarse en un |.tex| una forma libre vacía idéntica a la que se tiene, así como se hace en el ejemplo incluído.}, se incluye la imagen como fondo de la página y se definen las posiciones fijas de las numeraciones impresas en el talonario, es decir el número de factura y el número de control, de los que hablamos previamente.
%
% Con esta misma clase tiene se la posibilidad de imprimir un información en el reverso de la hoja de la factura que saldrá escrita utilizando toda la hoja (los márgenes son ajustables). Esa información se encierra en un ambiente destinado para ese fin (|\begin{reverso}...\end{reverso}|),
% \DescribeEnv{reverso}
% y al momento de la compilación puede decidirse si se imprime esa parte o no.
%
% Otra posibilidad que da esta clase es la de escribir un reporte largo asociado a la factura\footnote{Si lo hubiera, así como también se puede usar estar clase para elaborar un reporte sin asociarlo a una factura en particular.}. Tal reporte se escribe dentro de un ambiente en el mismo documento en el que se escribe la factura asociada a él (|\begin{reporte}...\end{reporte}|),
% \DescribeEnv{reporte}
% y con dar una opción al inicio se puede hacer que se impirma el reporte en las páginas siguientes a la factura, o también se puede hacer que se imprima sólo el reporte ignorando todo lo que no constituye el reporte, de la misma manera como también se puede hacer que se imprima sólo la factura ignorando el entorno |reporte|. Esto permite tener todo en un mismo documento e imprimir lo que se desee en el momento, o todo a la vez.
%
%
% \section{Compilación de un documento\label{sec:compila}}
%
% Para compilar una |factura| puede usar cualquiera de los motores \LaTeX, pdf\LaTeX, \XeLaTeX{} o \LuaLaTeX, naturalmente con las restricciones que impone cada uno.
%
% Hay casos en los que hace falta compilar más de una vez el documento. Dos veces necesita ser compilado un documento cuando alguna descripción en el cuadro de la factura ocupa más que una línea; también dos veces cuando se incluye una firma en el reporte, y tres o cuatro veces cuando se pide que se dibuje una línea oblicua que inutilice el espacio restante en el cuadro de la factura o cuando se usan etiquetas (sección \ref{sec:etiquetas}).
%
% Todo documento |.pdf| compilado con la clase |factura| escribe en la {\slshape metadata} del |.pdf| un texto con la versión y la fecha de la clase con la que se compiló; ese texto va en el campo |Subject| de dicha {\slshape metadata}.
%
%
% \section{Opciones al cargar la clase\label{sec:opciones}}
%
%  Al cargar la clase con |\documentclass| podemos dar distintas opciones según el comportamiento que queramos; dichas opciones están descritas en esta sección y cualquier opción cargada se verá como comentario en el archivo |.log| de la compilación bajo el texto “{\ttfamily Opción cargada: }” con el nombre de la opción pedida.
%
%
% \subsection{Sobre el tipo de documento a producir}
%
% {\setlength{\parindent}{0pt}
%
% \DescribeOption{prefactura}
%  La {\slshape “prefactura”} es un documento no legal, preliminar, que no se imprime sobre papel de factura sino que está concebido para enviarse por correo o fax. Para eso se define la imagen que contiene el diseño de la factura con el comando |\FondoFactura|, luego la clase escribe la denominación “FACTURA” con su número\footnote{Lo predeterminado cuando no es una {\slshape “prefactura”} sino una factura final es no escribir la denominación ni el número, ya que eso generalmente viene impreso en cada hoja del talonario.}. Esta opción implica la opción |sinfirmas|, ya que si es un documento preliminar sin validez legal, las firmas no son necesarias; sin embargo se puede cambiar este comportamiento con las opciones |dosfirmas| y |unafirma|.
%
% \DescribeOption{cotizacion}
%  La cotización es un documento similar a una factura pero no es un documento legal en el sentido de que no se imprime sobre talonario fiscal alguno, así que no lleva número de control y puede o no tener su propia numeración, que es cosa del emisor de la cotización. Esta opción implica la opción |unafirma|, para que firme el emisor de la factura, y si su nombre no se especifica saldrá “Por el emisor” debajo de la línea para firmar. También puede cambiarse esto con la opción |dosfirmas| o con |sinfirmas|.
%
% \DescribeOption{sinreverso}
%  No escribe la información -opcional- que se escribió dentro del entorno |reverso| a salir en el reverso de la factura.
%
% \DescribeOption{conreporte}
%  Genera la factura (prefactura o cotización) y luego el reporte.
%
% \DescribeOption{soloreporte}
%  Genera solamente el reporte, ignorando lo relacionado con la factura. Si no se especifica |conreporte| ni |soloreporte| se va a imprimir únicamente la factura.
%
%
% \subsection{Relativo a las columnas que puede tener el cuadro de la factura\label{subsec:columnas}}
%
% \DescribeOption{numitem}
%  Con la opción |numitem| una nueva columna aparece en el cuadro de la factura donde se enumera cada ítem en el órden en el que se escriben dentro del entorno para la factura. Esto es útil por ejemplo cuando se hace referencia al número del ítem con el comando |\itemref|. Véase la sección \ref{sec:etiquetas}.
%
% \DescribeOption{sincantidad}
%  El ambiente |factura| tiene también la posibilidad de no mostrar el campo de la cantidad, ni aceptarlo como argumento de cada ítem; en ese caso los comandos |\Item| -de los que hablaremos en la sección \ref{sec:factura}- aceptan como argumentos solamente la descripción del elemento y su precio, en ese orden.
%
% \DescribeOption{descuentos}
%  Permite agregar un descuento en porcentaje en cada elemento de la factura, siendo dicho porcentaje el último argumento de los comandos |\Item|. Si no aplica descuento para un ítem, puede este dejarse sin ese argumento, dando el mismo resultado que indicar “0” como argumento para ese descuento. (Explicado también en la sección \ref{subsec:descuentos}).
%
% \DescribeOption{subtotal}
%  Agrega una columna extra en la factura; sólo aplica cuando también se usa la opción |descuentos|. La nueva columna escribe el subtotal que cada ítem produce, es decir, el precio unitario multiplicado por la cantidad sin aplicar el descuento. Esta opción sobrecarga la factura (precio unitario, descuento, subtotal y precio total), y su uso es recomendable cuando también se usa |sincantidad|, porque de esta forma se tiene el subtotal, el descuento, y el total.
%
% \DescribeOption{iva}
%  Agrega una nueva columna donde se expresa el porcentaje del IVA para cada elemento en la factura. Sin especificar esta opción queda claro según lo establecido por nuestra legislación cuál será el gravamen que aplica para cada elemento por una letra que lo denota al lado del precio o de la descripción.
%
% \DescribeOption{duc}
%  Cambia el orden de los campos en el documento impreso a: descripción, precio unitario, cantidad y precio total. {\itshape Esto no cambia la forma como deben especificarse los argumentos de |\Item| sino solamente cambiará cómo serán mostrados}.
%
% \DescribeOption{dcu}
%  Cambia el orden de los campos en el documento impreso a: descripción, cantidad, precio unitario y precio total\footnote{Sin \texttt{duc} ni \texttt{dcu}, el modo predeterminado es \texttt{cdu}: cantidad, descripción, precio unitario.}.
%
%
% \subsection{Relativo al IVA}
%
% \DescribeOption{siniva}
%  No se cobra IVA, para los casos en los que eso aplique.
%
% \DescribeOption{sinivaexpresado}
%  Al no cobrar IVA, tampoco aparece en la totalización los IVA ni las bases imponibles.
%
% \DescribeOption{todosiva}
%  Agrega el monto exento y todos los tipos de IVA en los totales así no se hayan incluído elementos gravados con los distintos tipos de impuesto.
%
% \DescribeOption{electro}
%  {\bfseries Esta opción ya no está vigente en la legislación y se quitó de esta clase.} Se agregó en su momento esta opción para dar cumplimiento con las Gacetas Oficiales números 41239, 41242 y 41244\footnote{De fecha 19, 21 y 26 de septiembre del año 2017, respectivamente. En la última de estas gacetas está la Providencia Administrativa PA0048 del SENIAT.} donde se establece que las operaciones de compra (salvo excepciones establecidas en la Gaceta) gravadas con IVA general gozarán de un descuento cuando son pagadas mediante medios electrónicos. El descuento es del 3\% cuando la factura es de un monto menor a Bs.F 2.000.000 y del 5\% cuando el monto de la factura es de Bs.F 2.000.000 o más. Al dar esta opción el descuento queda establecido según el Decreto Presidencial número 3085.
%
% {\itshape Esta opción no es compatible con el modo de totalización en filas.}
%
% \DescribeOption{G}
%  Agrega una “|G|” al lado de todos los items gravados con IVA general así sea el general el único IVA existente en la factura.
%
% \DescribeOption{impdesc}
%  Con esta opción, la letra que indica la denominación del impuesto aplicado a cada elemento, en lugar de escribirse en el precio unitario y en el total, que es la forma predeterminada, sale escrito al final de la descripción. Puede preferirse de esta forma, y ahorra un poco de espacio usable en la descripción porque ya no sale en dos lugares dicha letra sino en uno solo.
%
%
% \subsection{Sobre las líneas en el cuadro de la factura}
%
% \DescribeOption{sinlineasve}
% \DescribeOption{sinlineasvi}
% \DescribeOption{sinlineasv}
% \DescribeOption{sinlineashe}
% \DescribeOption{sinlineashi}
% \DescribeOption{sinlineash}
%  La clase provee varias opciones para no dibujar las líneas del cuadro de la factura. La clase diferencia entre las líneas internas y las líneas externas. Las líneas externas son las que conforman el contorno externo del cuadro de la factura. Las líneas internas en el caso de las horizontales son la línea que va después del título de las columnas (cantidad, descripción, ...), la línea que separa los ítems de los totales y la línea que separa los totales de la nota interna; en el caso de las verticales, las líneas internas son las que separan las columnas.
%
%  Podemos apagar cada conjunto de línea; ya explicado qué es interno y qué es externo, queda aclarar que “ve” significa vertical externa, “vi” vertical interna, “he” horizontal externa y “hi” horizontal interna. Usando |sinlineasv| o |sinlineash| apagamos todas las líneas verticales u horizontales.
%
% \DescribeOption{sinlineahtit}
% \DescribeOption{sinlineahtot}
% \DescribeOption{sinlineahni}
%  Adicionalmente podemos apagar las líneas horizontales internas por separado, es decir, la línea que va después del título de las columnas la podemos quitar con \texttt{sinlineahtit}, la línea que separa los elementos a facturar de los totales generales se puede quitar con \texttt{sinlineahtot} y la línea que separa los totales de la -opcional- nota interna la podemos quitar con \texttt{sinlineahni}. 
% 
% \DescribeOption{conlineasha}
%  Agrega una línea horizontal separando cada artículo o ítem de los demás dentro del cuadro de la factura, como las facturas hechas a mano que tienen un espacio determinado para cada artículo.
%
%  Si no se especifica nada sobre las líneas, saldrán todas ellas (verticales y horizontales) salvo las que separan  los ítems a facturar entre sí.
%
%
% \subsection{Elementos a mostrar o no mostrar en la factura}
%
% \DescribeOption{sinmarcas}
%  Con |sinmarcas| evitamos que el carácter de fin de línea aparezca cuando la descripción ocupa más que una línea; ese carácter indica que la línea continúa abajo, y predeterminadamente es la flecha “\flechaCR”. También con esta opción no aparece el carácter “$\cdot$” en las celdas vacías en este mismo caso.
%
% \DescribeOption{sinmonedatotales}
%  Esta opción |sinmonedatotales| permite quitar el signo monetario de los totales, que antes nunca salía y ahora sale de manera predeterminada.
%
% \DescribeOption{unafirma}
%  No pone el espacio para que firme quien recibe el documento sino sólo el emisor; esto es lo predeterminado en las cotizaciones.
%
% \DescribeOption{sinfirmas}
%  No pone ninguna línea ni espacio para firmar; esto es lo predeterminado para las prefacturas.
%
% \DescribeOption{dosfirmas}
% Coloca las dos líneas para que firme quien emite y quien recibe. Esta opción es la predeterminada salvo en las prefacturas y en las cotizaciones.
%
% \DescribeOption{vencimiento}
%  Cuando establecemos los días del crédito de la factura, se imprime una línea también con la fecha de vencimiento de la factura. Con esta opción podemos pedir que la fecha de vencimiento salga siempre en la facturas, inclusive cuando no sean facturas a crédito.
%
% \DescribeOption{denom}
%  Con esta opción escribimos también el texto “FACTURA” con el número. La razón de esta opción es que hay formas libres que tienen la numeración de la factura y otras que solamente tienen el número de control; para las que tienen sólo el número de control existe esta opción que también escribe el número en cada factura impresa, en la posición que se defina.
%
% \DescribeOption{sinnumero}
%  Escribe el texto de la denominación del documento en su posición (e.g. “COTIZACIÓN”) pero sin escribir el número; por ejemplo para las cotizaciones que no son documentos que legalmente lleven número, aunque el número puede ser para control interno. Esta opción implica la opción anterior |denom|.
%
% \DescribeOption{sintotales}
%  Con esta opción evitamos que se totalicen los elementos en la factura; sólo se presenta el cuadro sin los totales. La utilidad de esto no es en las facturas sino en las cotizaciones para casos donde se muestran variantes del mismo producto para que el cliente elija entre ellos. No tendría sentido totalizar allí sino que funcione como un listado de precios. {\itshape Debe usarse en las cotizaciones porque una factura sin totales no es válida.}
%
% \DescribeOption{linea}
%  Agrega una línea diagonal que inutiliza el espacio vacío en el cuadro de la factura, para evitar posibles futuras adiciones de texto. Esta opción requiere de la opción |expandecuadro|, explicada a continuación.
%
%
% \subsection{Disposición de algunos elementos en la factura}
%
% \DescribeOption{expandecuadro}
%  Expande el cuadro de la factura con espacio en blanco, es decir, rellena el espacio restante del cuadro de la factura con espacio en blanco. Lo predeterminado es expandir la parte de la descripción larga cuando la hay.
%
% \DescribeOption{sinexpandir}
%  No expande ninguna parte en la factura, es decir, deja tanto el cuadro de la factura como la descripción en su tamaño mínimo según el contenido que tenga.
%
% \DescribeOption{centradesc}
%  Podemos centrar el texto de la descripción de cada ítem en la factura con este comando en lugar de justificarlo.
%
% \DescribeOption{filas}
%  Muesta los totales finales en modo de filas; esto los presenta uno al lado del otro en lugar de uno debajo del otro. Esta opción tiene limitaciones, por ejemplo no es compatible con la opción |sintotales|, |sinivaexpresado|, y por razones de espacio tampoco es compatible con los comandos de descuentos en los totales (subsección \ref{subsec:dctototales}).
%
% \DescribeOption{centrado}
%  Centra los totales finales ({\itshape sólo para el modo de totalización en filas}), que sin esta opción salen alineados al margen derecho.
%
% \DescribeOption{agrupado}
%  No agrupa los totales finales de los items por gravamen sino por tipo ({\itshape sólo para el modo de totalización en columnas}). Con esta opción la agrupación es primero todos los subtotales y los descuentos (cuando aplica), luego las bases imponibles y después los IVA. Sin |agrupado| se ordena por porcentaje del gravamen.
%
%
% \subsection{Otras opciones}
%
% \DescribeOption{nospanish}
%  No carga |babel| en castellano\footnote{Por ejemplo para cargar {\ttfamily polyglossia} en su lugar.}, ni la codificación |T1| de las letras en el caso de compilar con pdf\LaTeX. Si no se especifica esta opción se cargan ambos paquetes.
%
% \DescribeOption{twoside}
%  Establece que el reporte se imprime por los dos lados del papel. (Obviamente se ignora esta opción cuando el motor no va a generar un reporte puesto que las facturas ocupan una hoja solamente).
%
% \DescribeOption{csv}
%  Escribe un archivo csv con la información numérica de la factura.
%
% \subsection{Activación y desactivación de opciones desde comandos\label{subsec:condicionales}}
%
% Cualquiera de estas opciones también se puede activar o desactivar en el preámbulo y así poder incluirlas dentro de archivos que vayan a ser cargados con |\input| o |\include| como se explicará en la subsección \ref{subsec:aux}. Esto significa que podremos desactivar una opción que hayamos dado en |\documentclass| o activar alguna que no hayamos activado allí.
%
% Para hacer eso debemos usar el condicional que cada opción maneja y activarlo o desactivarlo de la forma como se hace en \TeX{}. El condicional de cada opción tiene el mismo nombre de la opción. Por ejemplo, para quitar las líneas verticales (|sinlineasv|) usaríamos |\sinlineasvtrue| en el preámbulo, o para desactivar la creación del archivo |.csv|, previamente activada en |\documentclass|, usaríamos |\csvfalse|.
%
% }
%
%
% \section{Comandos que la clase provee\label{sec:comandos}}
%
% A continuación descriremos los principales comandos de esta clase; en rojo aparecen aquellos pocos comandos que son de obligatorio uso. Los que no son de obligatorio uso es porque tienen algún valor predeterminado en |factura.def| o porque controlan algún aspecto opcional.
%
%
% \subsection{Estableciendo los márgenes de la hoja y el fondo}
%
%  Como explicamos en la sección \ref{sec:geohoja}, a la hoja de la factura se le deben establecer los márgenes apropiadamente, y esos márgenes no tienen en cuanta ciertos textos que llamamos “fijos”, para los cuales también debemos establecer sus posiciones en la hoja, y eso lo veremos en esta parte.
%
%
% {\setlength{\parindent}{0pt}
%
% \DescribeMacro{\GeometriaFactura}{\macrooptcolor|\GeometriaFactura|\marg{geometría}}
%
%  En el argumento de este comando van los parámetros relativos al paquete |geometry| que van a afectar la manera como la factura es generada. Como dijimos al principio, deberá especificarse el tipo de papel (e.g. |letterpaper|, |legalpaper|, |executive|, |a4paper|, ...) y los márgenes verticales y horizontales.
%
%  Para especificar el margen superior debe medirse desde el inicio del papel hasta donde se quiera establecer que empiece el recuadro de los datos del cliente, y el margen inferior debe medirse desde el final del papel hasta donde terminaría la opcional descripción. Como ya dijimos, hay textos que están fuera de estos márgenes por haberlos considerado fijos en la implementación de la clase |factura|.
%
% \DescribeMacro{\GeometriaReverso}{\macrooptcolor|\GeometriaReverso|\marg{geometría}}
%
%  Define los márgenes del reverso de la hoja de la factura para escribir información extra.
%
% \DescribeMacro{\FondoFactura}{\macrooptcolor|\FondoFactura|\marg{nombre del archivo de la imagen}}
%
%  Nombre del archivo de la imagen a cargar como diseño, especialmente para las {\slshape prefacturas}, aunque también puede usarse como una “marca de agua” para las facturas.
%
%  Lo que recomendamos es digitalizar la forma libre o pedir a la imprenta un archivo digital con el diseño y así incluir esa imagen con este comando cuando se vayan a producir “prefacturas” o cotizaciones, así las prefacturas tienen el mismo diseño que las facturas. Si se puede lograr el diseño en un código como se hace con el ejemplo del fondo, pues mejor aún.
%
%
% }
%
%
% \subsubsection{Elementos fijos y sus ubicaciones}
%
% Los elementos cuyas posiciones van fijas en la hoja disponen de macros para establecer sus ubicaciones. Todas estas macros empiezan por |\Pos|, y son las siguientes: |\PosFecha|, |\PosDenom|, |\PosControl|, |\PosNotaFinal| y |\PosFirmas|.
%
% Todos estos comandos aceptan como argumentos dimensiones de \LaTeX{}, y son las posiciones |(x,y)| en la página de los elementos a imprimir, siendo “|x|” la coordenada a lo ancho del papel y “|y|” la coordenada a lo largo del papel, es decir, horizontal y vertical respectivamente. El origen está fijado arriba a la izquierda. Se pueden especificar también valores negativos para denotar que estamos contando desde el extremo derecho o inferior, para el caso de “|x|” o “|y|”, respectivamente\footnote{Varias de estas variables están definidas de esta forma negativa en el archivo de definiciones.}; salvo el caso de la nota final y las firmas en su coordenada |x|, todos los demás elementos fijos aceptan valores negativos en su posicionamiento, ya que estos dos elementos cuentan con un intervalo |x-inicial,x-final| para posicionar.
%
% Por ejemplo, si queremos que la fecha aparezca desde el margen izquierdo y |5cm| contados desde arriba hacia abajo, haríamos: |\PosFecha{0cm}{5cm},| y si queremos que salga |1cm| desde la derecha y |4cm| contados desde abajo haríamos |\PosFecha{-1cm}{-4cm}|. A continuación todas las macros para fijar las posiciones en el papel:
%
% {\setlength{\parindent}{0pt}
%
% \DescribeMacro{\PosFecha}{\macrooptcolor|\PosFecha|\marg{pos. x}\marg{pos. y}}
%
%  Con |\PosFecha| controlamos dónde se ubica la caja que imprime la fecha en la hoja.
%
% \DescribeMacro{\PosFecha*}{\macrooptcolor|\PosFecha*|\marg{pos. x}\marg{pos. y}}
%
%  Con esta variante con “*”, es decir, |\PosFecha*|, logramos que la caja de la fecha abra hacia arriba desde la posición especificada en el eje vertical, de manera que estaríamos especificando una coordenada “|y|” final del texto y no una posición inicial, esto significa que el texto se despliega hacia arriba desde la posición dada y no hacia abajo. Por ejemplo, sin “*”, si especificamos |\PosFecha{-0cm}{5cm}|, las fechas empiezan 5cm desde el tope superior de la hoja, y si especificamos |\PosFecha*{-0cm}{5cm}| ellas terminarán en los 5cm de la hoja. Esto es útil cuando se especifica también |\Credito| que produce una fecha de vencimiento que aparecerá abajo de la fecha de emisión; al ocupar las fechas más de una línea podemos querer que la última fecha (la de vencimiento) quede alineada en cierta posición vertical, la misma posición donde saldría la fecha de emisión en el caso en el que no haya fecha de vencimiento por no especificar la factura a crédito.
%
% \DescribeMacro{\PosDenom}{\macrooptcolor|\PosDenom|\marg{pos. x}\marg{pos. y}}
%
%  Con |\PosDenom| controlamos dónde se ubica la caja que imprime la denominación del documento (e.g. “FACTURA”, “COTIZACIÓN”) y su -opcional- número.
%
% \DescribeMacro{\PosDenom*}{\macrooptcolor|\PosDenom*|\marg{pos. x}\marg{pos. y}}
%
%  Al igual que con |\PosFecha*| podemos especificar la posición final en “|y|” de la caja para la denominación, así el texto se despliega hacia arriba y no hacia abajo.
%
% \DescribeMacro{\PosControl}{\macrooptcolor|\PosControl|\marg{pos. x}\marg{pos. y}}
%
%  Con |\PosControl| controlamos dónde se ubica la caja que imprime el número de control con la serie.
%
% \DescribeMacro{\PosControl*}{\macrooptcolor|\PosControl*|\marg{pos. x}\marg{pos. y}}
%
%  También podemos pedir que la caja del número de control abra hacia arriba, es decir que la posición vertical sea tomada como la posición vertical final; para eso |\PosControl*|, igual que con |\PosFecha*| y |\PosDenom*|. 
%
% \DescribeMacro{\PosNotaFinal}{\macrooptcolor|\PosNotaFinal|\marg{pos. x-inicial}\marg{pos x-final}\marg{pos. y}}
%
%  Con |\PosNotaFinal| controlamos dónde se ubica el texto de la nota final\footnote{La nota final es lo que en versiones previas llamábamos “condiciones” de la factura} (está hecho con |\parbox|, acepta un párrafo como argumento) en la hoja.
%
% {\itshape Aquí delimitamos el intervalo en |X| donde va la nota final, por eso hay una coordenada |X-inicial| y una coordenada |X-final|.} La coordenada |X| inicial va medida desde el margen establecido como el izquierdo, y la |X| final va medida desde el margen establecido como el derecho. Por ejemplo, estableciendo |X-inicial| como |0pt| y |X-final| como |0pt| queda en los mismos márgenes que todos los demás elementos de la factura, y esto es lo predeterminado, pero estableciendo ambas coordenadas como |1cm|, la nota final quedará con |1cm| de margen hacia adentro que la factura, tanto por la izquierda como por la derecha.
%
% Debido a que este comando acepta valor inicial y valor final en |X|, entonces no acepta valores negativos.
%
% \DescribeMacro{\PosNotaFinal*}{\macrooptcolor|\PosNotaFinal*|\marg{pos. x-inicial}\marg{pos x-final}\marg{pos. y}}
%
%  De la misma manera como con los comandos precedentes, podemos especificar que la caja del párrafo para la nota final abra hacia arriba, fijando una posición vertical final y no una posición inicial de la nota final.
%
% \DescribeMacro{\PosFirmas}{\macrooptcolor|\PosFirmas|\marg{pos. x-inicial}\marg{pos. x-final}\marg{pos. y}}
%
%  Con |\PosFirmas| controlamos dónde se ubica la caja que dibuja las líneas para firmar en la factura. {\itshape Aquí aplica el mismo comentario que en \cs{PosNotaFinal} sobre las coordenadas |X| inicial y final}. La coordenada |X-inicial| indica dónde empieza la primera linea (para que firme quien recibe), y la coordenada |X-final| establece dónde termina la segunda linea para firmar (para que firme el emisor de la factura), que va a estar al lado de la primera.
%
% En este caso, las coordenadas |X| tampoco aceptan valores negativos ni versión con asterisco |*|.
%
% }
%
%
% \subsubsection{Espaciado entre distintas partes. Estilo de los recuadros\label{subsec:espacioslineas}}
%
% Podemos cambiar distintos valores del espaciado interno en la factura, así como también la estética de la línea para firmar y los recuadros que encierran los datos del cliente, la descripción, las “notas”, pudiendo ajustarle estos recuadros o cajas el grosor de la línea, la curvatura y la sombra. Los argumentos de todos estos comandos son dimensiones de \LaTeX{}:
%
% {\setlength{\parindent}{0pt}
%
% \DescribeMacro{\SepVertical}{\macrooptcolor|\SepVertical|\marg{sep. vertical entre elementos}}
%
%  Las distintas partes de la factura (a saber: el encabezado con los datos, la -opcional- nota previa, el cuadro de la factura, la -opcional- nota externa y la -opcional- descripción) también están separadas entre sí una distancia que puede ser cambiada.
%
%  La separación que se fija con este comando es la separación vertical mínima que queremos garantizar entre cada elemento a escribir en la factura, y puede ser ajustada automáticamente por la clase aumentándola cuando no hay relleno automático de espacios sobrantes (es decir, sin usar la opción |sinexpandir|). Cuando la descripción larga o el cuadro de la factura (usando la opción |expandecuadro|) toman su tamaño máximo, entonces la separación vertical entre cada elemento de la hoja es la dada con |\SepVertical|, o sea, la que queremos garantizar como la mínima separación entre esas partes.
%
% \DescribeMacro{\SepDatosResumen}{\macrooptcolor|\SepDatosResumen|\marg{sep. horizontal}}
%
%  Esta macro nos proporciona una forma de cambiar la separación entre los recuadros o cajas donde van los datos del cliente y el resumen que sale a su lado. Naturalmente, si no se escribe un resumen saldría únicamente la caja de los datos del cliente, y esta separación no tendría efecto.
%
% \DescribeMacro{\SepEncabezado}
% \DescribeMacro{\SepNotaPrevia}
% \DescribeMacro{\SepDescripcion}
% \DescribeMacro{\SepNotaFinal}
% {\macrooptcolor|\SepEncabezado|\marg{sep. interna texto}}\newline
% {\macrooptcolor|\SepNotaPrevia|\marg{sep. interna texto}}\newline
% {\macrooptcolor|\SepDescripcion|\marg{sep. interna texto}}\newline
% {\macrooptcolor|\SepNotaFinal|\marg{sep. interna texto}}
%
%  Con estas macros podemos controlar la distancia entre la línea del recuadro y el texto que hay en el recuadro, es decir, un margen interno entre el contorno de la caja y el texto para que no salga pegado a la línea que lo encierra. Esto es parte del diseño de \tikz de estos recuadros. Como puede verse, dicho margen para los datos del cliente y para el resumen se cambia con |\SepEncabezado|, los de la nota previa con |\SepNotaPrevia|, los de la descripción larga con |\SepDescripcion|, y los de la nota final con |\SepNotaFinal|.
%
% Predeterminadamente no hay línea ni diseño de \tikz para la nota final del documento, así que el valor fijado aquí no tiene efecto a menos de que se le especifiquen valores a dicho recuadro con la opción |\LineaNotaFinal|, explicada más abajo.
%
% \DescribeMacro{\SepItemsExtra}{\macrooptcolor|\SepItemsExtra|\marg{sep. extra entre items}}
%
%  Agrega una separación extra entre los elementos o ítems en el cuadro de la factura. La separación entre ítems está dada por el |\baselineskip| que se puede establecer con el comando |\LetraItems|, el cual se explicará más adelante en esta misma sección.
%
% \DescribeMacro{\SepFilas}{\macrooptcolor|\SepFilas|\marg{sep. en x}\marg{sep. en y}}
%
%  Cuando se usa el modo de totalización en filas tenemos también la posibilidad de cambiar la separación entre los totales que salen uno al lado del otro, y la separación entre las distintas líneas.
%
%  El primer argumento de |\SepFilas| es la separación horizontal {\itshape mínima} entre los distintos totales que se presentan en el modo de totalización en filas. La clase |factura| expande esa distancia para que queden todos los totales separados igual, y también separados igual de los márgenes del papel.
%
%  El segundo argumento de |\SepFilas| es la separación vertical entre las distintas líneas de la totalización en filas.
%
% \DescribeMacro{\LineaEncabezado}
% \DescribeMacro{\LineaNotaPrevia}
% \DescribeMacro{\LineaDescripcion}
% \DescribeMacro{\LineaNotaFinal}
% {\macrooptcolor|\LineaEncabezado|\marg{grosor}\marg{curvatura esquinas}\marg{tamaño sombra}}\newline
% {\macrooptcolor|\LineaNotaPrevia|\marg{grosor}\marg{curvatura esquinas}\marg{tamaño sombra}}\newline
% {\macrooptcolor|\LineaDescripcion|\marg{grosor}\marg{curvatura esquinas}\marg{tamaño sombra}}\newline
% {\macrooptcolor|\LineaNotaFinal|\marg{grosor}\marg{curvatura esquinas}\marg{tamaño sombra}}
%
%  La información del encabezado (datos del cliente y resumen), así como la nota previa, la descripción y la nota final puede o no tener el diseño \tikz, el cual viene prehecho en el archivo |.def| y consiste en una línea curva, en algunos casos sombreada a la derecha y abajo, y en otros casos no. Esa línea puede ser alterada en su grosor, el tamaño de la sombra, y la distancia entre la línea y la información, es decir un margen interno. Para eso estos comandos, que se comportan igual todos ellos.
%
%  Con el primer argumento de estos comandos cambiamos el grosor de la línea\footnote{Si se especifica una dimensión nula, desaparece el diseño \tikz y la línea, quedando solamente un \cs{parbox}, y es así como se estableció en el archivo \texttt{.def} para el caso de la nota final.} con la que las cajas del encabezado, la nota previa, la descripción y la nota final se dibujan. Con el segundo argumento cambiamos el radio de curvatura de las esquinas, y con el tercero el tamaño de la sombra\footnote{Todas las sombras salen hacia abajo a la derecha.}.
%
%  Se puede modificar el diseño de este nodo de \tikz; está alojado en las macros internas |\@DatosTIKZ|, |\@NotaPreviaTIKZ|, |\@DescripcionTIKZ| y |\@NotaFinalTIKZ|.
%
%  Lo predeterminado es que la nota final se escriban sin diseño de \tikz, es decir, el primer argumento de este comando está fijado en cero para que no salga línea ni diseño de \tikz.
%
% \DescribeMacro{\LineaFirmas}{\macrooptcolor|\LineaFirmas|\marg{long. línea}\marg{grosor}}
%
%  La longitud de las líneas para firmar y el grosor son también ajustables por medio del primer y segundo argumento, respectivamente, de la macro |\LineaFirmas|.
%
%
% \subsection{Sobre la información del cliente\label{subsec:cliente}}
%
% \DescribeMacro{\RazonSocial}{\macromandcolor|\RazonSocial|\marg{texto}}
%
%  Define la razón social, o sea, el nombre del cliente.
%
% \DescribeMacro{\Nombre}{\macromandcolor|\Nombre|\marg{texto}}
%
%  Similar a |\RazonSocial| pero usable más bien para personas naturales, donde no existe una “persona de contacto” y por tanto la clase no advierte de su ausencia si no se ha especificado con |\Contacto|. Naturalmente se especifica un solo nombre del cliente, es decir que si se especifica con |\RazonSocial| no hace falta usar este comando. {\itshape Es preferible usar |\RazonSocial|.}
%
% \DescribeMacro{\RIF}{\macromandcolor|\RIF|\marg{texto}}
%
%  Define el R.I.F. del cliente.
%
% \DescribeMacro{\CI}{\macromandcolor|\CI|\marg{texto}}
%
%  Especifica la cédula de identidad del cliente al ser persona natural. {\itshape Es preferible usar siempre |\RIF|.}
%
% \DescribeMacro{\Contacto}{\macrooptcolor|\Contacto|\marg{texto}}
%
%  Define una persona de contacto (e.g. nombre de una persona cuando el cliente no es personal).
%
% \DescribeMacro{\Direccion}{\macromandcolor|\Direccion|\marg{texto}}
%
%  Define la dirección fiscal del cliente.
%
% \DescribeMacro{\Telefono}{\macrooptcolor|\Telefono|\marg{texto}}
%
%  Especifica el (o los) teléfonos del cliente.
%
% \DescribeMacro{\Email}{\macrooptcolor|\Email|\marg{texto}}
%
%  Establece el e-mail del cliente.
%
% \DescribeMacro{\Conforme}{\macrooptcolor|\Conforme|\marg{texto}}
%
%  Especifica el texto que aparecerá debajo de la línea para que firme el cliente; si no se especifica tomará el texto por omisión, que es “Recibí conforme”, indicando que el cliente le firma una copia al emisor de la factura donde acepta haber recibido y estar conforme. Este texto puede cambiarse a otro que puede ser el nombre del cliente o la persona de contacto, o también “Por el cliente”.
%
%
% \subsection{Sobre la información del emisor del documento}
%
% \DescribeMacro{\Emisor}{\macrooptcolor|\Emisor|\marg{texto}}
%
%  Define el nombre del emisor de la factura. Si no se especifica saldrá el texto “PAGADO”\footnote{El texto “PAGADO” es lo que la legislación venezolana del SENIAT exige escribir cuando una factura ha sido pagada.} debajo de la línea para firmar, ya que la factura se firma cuando haya sido pagada. Si se especifica un texto con este comando saldrá ese texto debajo de la línea donde va la firma, pero el propósito es que quede constancia para el cliente que ya pagó la factura, razón por la cual el texto predeterminado es “PAGADO”.
%
% \DescribeMacro{\Proveedor}{\macrooptcolor|\Proveedor|\marg{texto}}
%
%  Establece el nombre del emisor, para los efectos del reporte. Si no se define mediante esta macro, será tomado del establecido con |\Emisor|.
%
% \DescribeMacro{\FirmaFactura}{\macrooptcolor|\FirmaFactura|\marg{inclusión de la imagen}\oarg{corrimiento en x}\oarg{corrimiento en y}}
%
%  Incluye el archivo de la firma digitalizada en el espacio donde firma el emisor del documento. Los argumentos son:
%  \begin{enumerate}
%   \item inclusión de la imagen (e.g. |\includegraphics|). {\bfseries Aquí el primer argumento no es el archivo de la imagen sino el comando que la inserta en el documento}, así el usuario tiene más control sobre cómo saldrá la imagen si altera las opciones de |\includegraphics|.
%   \item corrimiento en |X| de la imagen, para un ajuste más fino de la posición de la firma,
%   \item corrimiento en |Y|.
%   \end{enumerate}
%  {\itshape (Sin especificar los corrimientos, la firma quedará centrada horizontalmente en la línea, dejando hacia la izquierda y hacia la derecha la misma cantidad de la línea. También quedará toda la firma encima de la línea, lo cual muy probablemente haya que corregirlo porque las firmas tienen “profundidad” de texto, que significa que hay parte de algunas letras que sobresalen hacia abajo respecto a la línea base del texto. Por todo esto es muy probable el uso de los corrimientos. Pueden verse los ejemplos, que allí se usan los corrimientos.)}
%
%  {\bfseries Las facturas no llevan firma porque se supone que el documento es auténtico ya que viene avalado por una imprenta registrada y tiene los números correlativos impresos desde la imprenta, de manera que la factura no necesita la firma de quien la emite; razón por la cual el texto debajo de la línea del emisor en la factura dice “PAGADO”, porque el emisor firma la factura una vez que haya sido pagada como constancia para el cliente.} Entonces agregar la firma escaneada es una posibilidad que debe contemplarse solamente con las cotizaciones, usando también el comando |\Emisor|, ya que no nos interesaría que apareciera “PAGADO” en ese caso sino algún nombre de quien está emitiendo la cotización.
%
%
% \subsection{Relativos a la moneda y a la numeración del documento factura}
%
% \DescribeMacro{\Moneda}{\macrooptcolor|\Moneda|\marg{signo monetario}}
%
%  Cambia el signo monetario, por defecto “Bs.”.
%
% \DescribeMacro{\Num}{\macrooptcolor|\Num|\marg{núm}}
%
%  Define el número de la factura. Es opcional porque el talonario de las facturas puede ya tener la numeración. Es de obligatorio uso cuando se define |denom|, o |prefactura| sin usar |sinnumero|.
%
% \DescribeMacro{\NumControl}{\macrooptcolor|\NumControl|\marg{núm}}
%
%  Define el número de control de la factura. Esta opción sólo aplica con |prefactura| puesto que una factura fiscal siempre tiene el número de control impreso desde la imprenta. Si se usa |prefactura| y no se define el número de control, éste tomará el mismo número de la factura.
%
% \DescribeMacro{\NumSerieControl}{\macrooptcolor|\NumSerieControl|\marg{núm}}
%
%  Define el número de serie del número de control, también aplica solamente con |prefactura|. Si no se especifica, toma |00|.
%
% \DescribeMacro{\Denom}{\macrooptcolor|\Denom|\marg{texto}}
%
%  Cambia el nombre del documento a algo distinto de “FACTURA” o “COTIZACIÓN”.
%
%
% \subsection{Estableciendo la fecha del documento}
%
% \DescribeMacro{\Fecha}{\macrooptcolor|\Fecha|\marg{fecha}}
%
%  Especifica la fecha de emisión del documento, en un formato que puede cambiarse con el comando |\FormatoFecha| pero predeterminadamente es el formato venezolano |dd-mm-aaaa|.
%
%  La fecha de emisión se guardará en el identificador “\texttt{FechaEmision}” perteneciente al paquete \texttt{datetime2} y podrá ser usado en cualquier parte del documento según las formas que provee dicho paquete.
%
%  {\itshape Si el usuario no provee la fecha, será tomada la fecha actual como la fecha de emisión y se producirá una advertencia indicando que no se especificó la fecha y que la fecha del documento es la fecha del día.}
% 
% \DescribeMacro{\Fecha*}{\macrooptcolor|\Fecha*|\marg{aaaa}\marg{mm}\marg{dd}}
%
%  Esta variante del comando se incorporó para suministrar la fecha en formato ISO, es decir, primero el año, luego y mes y finalmente el día, cada uno como un parámetro separado del otro (como se hace en \texttt{datetime2}. Esto es independiente de la forma como la fecha será mostrada, o sea que puede la fecha suministrarse de esta forma e igual saldrá en el formato deseado, cambiable con la macro |\FormatoFecha|, explicado a continuación.
%
% \DescribeMacro{\FormatoFecha}{\macrooptcolor|\FormatoFecha|\oarg{formato entrada}\marg{formato salida}\marg{separador}}
%
%  El comando |\FormatoFecha| le permite al usuario cambiar el formato de entrada de la fecha suministrada y el formato de salida de las fechas (cómo éstas salen escritas en la factura).
%
%  Con el primer argumento (opcional, dado entre corchetes) el usuario le puede decir a la clase cúal es el formato según el cual le suministrará la fecha con |\Fecha|; las posibilidades son: \begin{inparaenum}[\bfseries i)]\item |dd-mm-aaaa| (predeterminada), \item |dd.mm.aaaa|, \item |ddmmaaaa|, \item |aaaa-mm-dd|, \item |aaaa.mm.dd|, \item |aaaammdd| y \item |mm-dd-aaaa| \end{inparaenum}. Esto con el fin de establecer la fecha reconocida debidamente por |datetime2| y que sea posible su manejo como fecha y el posible cálculo de la fecha de vencimiento, si aplicara.
%
%  El segundo argumento (opcional también, dado entre llaves) le permite al usuario especificar el formato según el cual quiere que las fechas (emisión y vencimiento si aplica) salgan impresas en la factura. Estos son los formatos dados por \texttt{datetime2} y debe consultarse su documentación para más información. Entre ellos tenemos por ejemplo \begin{inparaenum}[\bfseries i)]\item |ddmmyyyy| (predeterminado), \item |default|, \item |iso|, \item |mmddyyyy| \end{inparaenum}.
%
%  El tercer argumento (opcional también, dado entre llaves) sirve para cambiar el separador, cuyo carácter predeterminado es el guión (“-”).
%
%  Con |\FormatoFecha|, al tener todos sus argumentos opcionales, el usuario podrá especificar solamente el primero (que va entre corchetes), solamente el segundo (entre llaves), pero para especificar el tercero que también está entre llaves debe especificar el segundo, ya que ambos son opcionales pero se dan igualmente entre llaves. Naturalmente podrá especificar dos de estos argumentos o los tres.
%
%  Por ejemplo: |\FormatoFecha[mm-dd-aaaa]{mmddyyyy}{.}| serviría para que el usuario pueda dar la fecha así: |\Fecha{12-31-2019}| y que salga igualmente según el formato |mm-dd-aaaa| pero separado por puntos. También puede usarse |\FormatoFecha[aaaa-mm-dd]| para dar luego la fecha de esta forma: |\Fecha{2019-12-31}| pero esta saldrá igual impresa en la factura así: “|31-12-2019|”, ya que no cambiamos el formato de salida. También puede hacer |\FormatoFecha{mmddyyyy}{/}| y dar la fecha por ejemplo así: |\Fecha*{2019}{12}{31}| y saldrá impresa así: “|12/31/2019|”.
%
%  Los valores predeterminados cambiables con |\FormatoFecha| están definidos, como siempre, en el archivo |factura.def|.
%
% \DescribeMacro{\Credito}{\macrooptcolor|\Credito|\marg{días}}
%
%  Este comando, cuyo argumento es la cantidad de días, nos permite emitir una factura a crédito. La fecha de vencimiento será calculada automáticamente y será colocada debajo de la fecha de emisión, especificada por el usuario.
%
%  La fecha calculada automáticamente se guarda en el identificador “\texttt{FechaVencimiento}” perteneciente al paquete \texttt{datetime2}, de manera que el usuario podrá hacer uso de esta fecha con los comandos que este paquete provee.
%
% }
%
% 
% \subsection{Descuentos en la totalización\label{subsec:dctototales}}
%
%  Podemos efectuar un descuento\footnote{Estas opciones de descuento en el subtotal no son compatibles con la opción de totalización en modo de |filas|, explicada en la sección \ref{sec:opciones}, y muy probablemente más adelante seguirán sin serlo.} sobre cualquiera de los subtotales; ese descuento puede ser porcentual (versión sin *) o dado en un monto fijo a sustraer del subtotal (variante con *). Podemos aplicar un descuento que aplique por igual a cualquier tipo de ítem con cualquier gravamen, o discriminado dependiendo del tipo de impuesto se aplica un descuento u otro.
%
% {\setlength{\parindent}{0pt}
%
% \DescribeMacro{\Descuento}{\macrooptcolor|\Descuento|\marg{porcentaje}}\\
% \DescribeMacro{\Descuento*}{\macrooptcolor|\Descuento*|\marg{monto}}
%
% Con este comando el descuento aplicaría por igual a cualquier subtotal presente, es decir, para elementos con cualquier gravamen y también exentos. Dichos descuentos son aplicados y mostrados al totalizar. Puede usarse inclusive cuando se aplican descuentos individuales que se incluyen con la opción de la clase |descuentos|.
%
% \DescribeMacro{\DescuentoG}{\macrooptcolor|\DescuentoG|\marg{porcentaje}}\\
% \DescribeMacro{\DescuentoG*}{\macrooptcolor|\DescuentoG*|\marg{monto}}
%
%  Descuento específico para los elementos gravados con IVA general.
%
% \DescribeMacro{\DescuentoR}{\macrooptcolor|\DescuentoR|\marg{porcentaje}}\\
% \DescribeMacro{\DescuentoR*}{\macrooptcolor|\DescuentoR*|\marg{monto}}
%
%  Descuento específico para los elementos gravados con IVA reducido.
%
% \DescribeMacro{\DescuentoA}{\macrooptcolor|\DescuentoA|\marg{porcentaje}}\\
% \DescribeMacro{\DescuentoA*}{\macrooptcolor|\DescuentoA*|\marg{monto}}
%
%  Descuento específico para los elementos gravados con IVA adicional.
%
% \DescribeMacro{\DescuentoE}{\macrooptcolor|\DescuentoE|\marg{porcentaje}}\\
% \DescribeMacro{\DescuentoE*}{\macrooptcolor|\DescuentoE*|\marg{monto}}
%
%  Descuento específico para los elementos exentos de IVA.
%
%
% \subsection{Textos explicativos en la factura\label{subsec:textos}}
%
% \DescribeMacro{\Resumen}{\macrooptcolor|\Resumen|\marg{texto}}
%
%  Pone un resumen que va en una caja a la derecha de los datos del cliente, separado de esta en una distancia ajustable con el comando |\SepEncabezado|, que se describirá más adelante. Este resumen es opcional.
%
% \DescribeMacro{\NotaPrevia}{\macrooptcolor|\NotaPrevia|\marg{texto}}
%
%  La nota previa es una nota que puede ponerse antes del cuadro de la factura. Sin especificarla no sale nota alguna. Dicha nota va en un recuadro \tikz cuyo aspecto puede cambiarse, como ya se mencionó en la sección \ref{subsec:espacioslineas} donde se explican las líneas o contornos.
%
% \DescribeMacro{\NotaInterna}{\macrooptcolor|\NotaInterna|\marg{texto}}
%
%  La nota interna es una coletilla que aparece dentro del cuadro de la factura, por eso la llamamos interna (es interna al cuadro), por ejemplo si aplicamos un descuento podemos decir allí cerca de los totales a qué corresponde tal descuento, o para hacer alguna aclaratoria sobre lo que precede inmediatamente como algún ítem de la factura.
%
% \DescribeMacro{\NotaExterna}{\macrooptcolor|\NotaExterna|\marg{texto}}
%
%  La nota externa es otro texto opcional que el usuario puede introducir y saldrá al terminar el cuadro de la factura, antes de la opcional descripción larga (comando |\Descripcion|). {\itshape La nota externa no tiene formato \tikz.}
%
% \DescribeMacro{\NotaFinal}{\macrooptcolor|\NotaFinal|\marg{texto}}
%
%  Con este comando se establece el texto para la nota final\footnote{Antiguas condiciones de la factura, por ejemplo, para las condiciones de pago si se trata de una factura, o para las condiciones de una cotización.}. Es opcional; si no se define no sale texto alguno. También se puede destinar este espacio para un diseño (que haría el usuario) en el que se agreguen también algunas casillas para especificar la forma de pago.
%
%
% \subsection{Estableciendo los parámetros relativos al reporte}
%
% \DescribeMacro{\GeometriaReporte}{\macrooptcolor|\GeometriaReporte|\marg{geometría}}
%
%  Aquí van los parámetros relativos al paquete |geometry| que van a afectar la manera como el reporte es generado, de la misma manera como se definieron los márgenes para la factura.
%
%  Puesto que el reporte no se imprime en hojas de facturas sino en hojas en blanco, el usuario podrá cambiar la geometría del papel completamente a su gusto sin las restricciones impuestas por el diseño que tengan sus facturas. En |factura.def| se definen unos márgenes razonables para un reporte. Si el reporte y la factura usaran un tamaño distinto de papel, entonces al imprimirse ambos juntos (opción |conreporte|), el reporte tomará el tamaño físico del papel especificado para la factura ya que con el paquete |geometry| no podemos cambiar el tamaño del papel en el medio del documento, aunque sí sus márgenes. Entonces, si se quiere imprimir el reporte en un papel de otro tamaño, tendrán que imprimirse ambos documentos por separado, o sea, primero uno y después el otro.
%
% \DescribeMacro{\FondoReporte}{\macrooptcolor|\FondoReporte|\marg{nombre del archivo de la imagen}}
%
%  Imagen a cargar {\bfseries solamente en la primera página del reporte} (e.g. un membrete). Cuando se usa este comando puede necesitarse usar también |\InicioReporte| para fijar el inicio de la primera página del reporte más abajo que las demás páginas cuyos márgenes están establecidos con |\GeometriaReporte|; esto porque la primera página tendría un “margen superior” distinto por la existencia del membrete.
%
% \DescribeMacro{\FondoReporte*}{\macrooptcolor|\FondoReporte*|\marg{nombre del archivo de la imagen}}
%
%  Esta variante de |\FondoReporte| es para que la imagen esté visible {\bfseries en todas las páginas} y no solamente en la primera como con el comando anterior. Usando esta variante del fondo del reporte no es necesario usar |\InicioReporte| sino solamente establecer los márgenes del reporte de acuerdo a dicho fondo.
%
% \DescribeMacro{\EstiloPagReporte}{\macrooptcolor|\EstiloPagReporte|\oarg{estilo primera página}\marg{estilo páginas restantes}}
%
%  Con este comando definimos el estilo de página para el reporte (véase el comando |\InfoPagina| y la subsección \ref{subsec:pagestyle}). El estilo de página es lo que cambiamos en \LaTeX{} con |\pagestyle|, o sea, lo relativo a las cabeceras y pies de página.
%
%  Con este comando podemos asignar un estilo de página para la primera página del reporte y otro estilo de página para el resto de las páginas. La primera página del reporte pudiera tener un estilo de página distinto por la presencia del membrete si se usa el comando |\FondoReporte| sin “|*|”, por lo cual se puede querer no poner cabeceras en la primera página pero sí en las demás.
%
%  Como vemos de la sintaxis de esta macro, el argumento obligatorio es el del resto de las páginas y el opcional es el que aplica a la primera página. El opcional se pone entre corchetes antes del otro, pero si se omite, todas las páginas toman el estilo especificado entre llaves.
%
%  Si no se usa esta macro la clase asume un comportamiento apropiado teniendo en cuenta si se especifica |\FondoReporte| o |\FondoReporte*| para evitar solapamiento de textos, como se explica también en la subsección \ref{subsec:pagestyle}.
%
%  La información que aparece como cabecera o como pie de página es la que damos con el comando |\InfoPagina|, explicado más adelante.
%
%  La clase provee varios estilos de página, que también se explicarán en la subsección \ref{subsec:pagestyle}. En los ejemplos de reportes se puede ver el comportamiento del comando |\EstiloPagReporte|.
%
% \DescribeMacro{\InicioReporte}{\macrooptcolor|\InicioReporte|\marg{longitud}}
%
%  Con este comando -cuyo argumento es una longitud de \LaTeX{}- se agrega una distancia extra donde inicia el texto en la primera página del reporte cuando se usa |\FondoReporte| sin “|*|”. Este comando sólo aplica cuando en el reporte se fija un fondo para que salga solamente en la primera página (comando |\FondoReporte| en su versión sin “|*|”). La motivación de esto es corregir la alteración que produce un membrete que va a salir solamente en la primera página; si el membrete estuviera en todas las páginas no es necesario hacer esto sino fijar los márgenes adecuadamente con |\GeometriaReporte|. Si en el reporte no se fija un fondo o se fija el fondo para todas las páginas (comando |\FondoReporte*|), lo que se establezca con |\InicioReporte| es ignorado. En estos dos casos los márgenes deben establecerse con |\GeometriaReporte|.
%
% \DescribeMacro{\FirmaReporte}{\macrooptcolor|\FirmaReporte|\oarg{inclusión la imagen}\oarg{corr. en x}\oarg{corr. en y}\marg{texto}\oarg{pos. x}\oarg{long. línea}}
%
%  Incluye la firma para el final del reporte. Los argumentos son:
%  \begin{enumerate}
%   \item inclusión de la imagen (e.g. |\includegraphics|),
%   \item corrimiento en |X| de la imagen, para un ajuste más fino de la posición de la firma,
%   \item corrimiento en |Y|,
%   \item texto debajo de la línea ({\itshape va entre llaves y no entre corchetes como los demás}). El predeterminado es el dado mediante el comando |\Proveedor|, o en su defecto, |\Emisor|),
%   \item posición en |X| del final de la línea, medida desde la derecha (predeterminado: 2cm),
%   \item tamaño extra de la línea de la firma. La clase mide el tamaño del texto y de la imagen (si se especificó), y al que sea mayor le agrega este extra de línea para que sobresalga a los dos lados. (Predeterminado: tamaño de la imagen o del nombre + 1.5cm extra).
%  \end{enumerate}
%  {\itshape {\bfseries Este comando es de los pocos que no es del preámbulo}; debe incluirse en el sitio en el reporte donde se quiera que la firma salga, generalmente al final del reporte y todos sus argumentos son opcionales, incluyendo el texto que va debajo de la línea, que se incluye entre llaves |{}| a pesar de ser opcional. Cuando no se especifica ningún argumento, sólo dibuja la línea para firmar y le escribe abajo el nombre dado con |\Proveedor| o en su defecto el dado con |\Emisor|.}
%
% \DescribeMacro{\TituloReporte}{\macrooptcolor|\TituloReporte|\marg{texto}}
%
%  Especificamos un título para el reporte. Este título saldrá centrado al principio del reporte.
%
% \DescribeMacro{\TituloReporte*}{\macrooptcolor|\TituloReporte*|\marg{texto}}
%
%  Usando |\TituloReporte| u omitiendo su uso, la información que sale en las cabeceras o pies de página tiene que ver con el número de la factura que está asociada a ese reporte. Con esta variante con “|*|” de |\TituloReporte|, el título del reporte -especificado con este comando- saldrá escrito también en estas partes de la página en lugar del texto que indica que el reporte corresponde a determinada factura.
%
% Puede usarse esta opción también en aquellos casos en los que se hace un reporte sin asociarlo a una factura, entonces el título del reporte es la información que sale en la cabecera o pie de cada página.
%
% \DescribeMacro{\InfoPagina}{\macrooptcolor|\InfoPagina|\marg{texto}}
%
%  Establece el texto que aparece en la cabecera o pie de página del reporte.
%
% {\itshape Si se quiere usar esta opción debe hacerse luego de definir el título del reporte porque si no el comando |\TituloReporte*| establece de nuevo la variable que |\InfoPagina| cambia.}
%
%  En |factura.def| hay dos definiciones relativas a la información que aparece en las cabeceras o pies de página: aquella que escribe el título y aquella que no, es decir, la que aplica con |\TituloReporte*| y la que aplica con |\TituloReporte|, que también el usuario podrá redefinir si lo necesita, pero lo más práctico sería usar esta macro para fijar el texto que se desee, y |\EstiloPagReporte| para fijarlo donde se quiera.
%
%
% \subsection{Definiciones en archivos separados \label{subsec:aux}}
%
% \DescribeMacro{\Cliente}{\macrooptcolor|\Cliente|\marg{archivo \LaTeX}}
%
%  {\bfseries Todos los comandos definibles en el preámbulo que alteren la forma como el reporte o la factura es presentada, así como cualquier variable definida en |factura.def| se puede redefinir y agregar a otro archivo |.tex| o |.def| que el usuario cargue en el preámbulo con |\input|, |\include| o con el comando |\Cliente|; el comando para el cliente existe para que el usuario pueda definir archivos |.tex| o |.def| con los valores específicos que aplican para cada cliente.}
%
%  La macro |\Cliente| incluye un archivo cuya extensión debe ser |.def| o |.tex|, de la misma forma como se hace con |\input|; es simplemente un atajo que además lee archivos |.def|.
%
% }
%
%
% \section{La factura\label{sec:factura}}
%
% \DescribeEnv{factura}
% La parte del documento donde se agregan los elementos a facturar se escribe dentro de un entorno especial para su tratamiento, llamado |factura|, que produce el cuadro de la factura:
%
% \iffalse
%<*literal>
% \fi
{\VERBATIM
\begin{verbatim}
    \begin{factura}
     [...]
    \end{factura}
\end{verbatim}
}
% \iffalse
%</literal>
% \fi
%
% Dentro de dicho ambiente podemos usar únicamente los comandos que incluyen los elementos a facturar y las posibles etiquetas; esos comandos son, para distintos gravámenes, los siguientes:
%
%
% \subsection{Agregando elementos a la factura}
%
% \noindent
% {\macromandcolor|\Item|\marg{cantidad}\marg{descripción}\oarg{descripción para csv}\marg{precio unitario}\oarg{descripción sólo para \\ \mbox{}\hfill referencias}}
% \DescribeMacro{\Item}
% \DescribeMacro{\ItemR}
% \DescribeMacro{\ItemA}
% \DescribeMacro{\ItemE}
%
% Con estos comandos agregamos elementos a la factura propiamente dicha. Con |\Item| agregamos un elemento gravado con IVA general (actualmente 12\%); con |\ItemR| agregamos uno con IVA reducido (8\%), con |\ItemA| uno con IVA adicional (27\%), y con |\ItemE| agregamos un elemento exento de impuesto al valor agregado.
%
% Como podemos ver, se especifica la cantidad, luego la descripción y al final el precio unitario, todos entre llaves, pudiendo intercalar entre estos dos últimos un parámetro adicional entre corchetes que sirve para escribir otro texto al archivo |.csv| en sustitución de la descripción, ya que a veces la descripción puede tener macros o caracteres no imprimibles. Si se declara una descripción y otra descripción para el |.csv|, la factura impresa manejará la descripción dada primero (entre llaves) y en el archivo |.csv| se escribirá la descripción dada después (entre corchetes).
%
% El último parámetro opcional (entre corchetes) es otro campo de descripción pero que no se imprime en el cuadro de la factura sino que se guarda bajo una etiqueta para luego poder usarse en cualquier parte del documento con el comando |\ldescref| (véase la sección \ref{sec:etiquetas} de las etiquetas). Dicha descripción puede tener formato de párrafos largos con saltos de línea.
%
% También tenemos la posibilidad de agregar un ítem al que no le ponemos precio, por ejemplo por ser parte de un total, y sus argumentos son la cantidad y la descripción nada más.
%
% \noindent
% \DescribeMacro{\ItemX}{\macrooptcolor|\ItemX|\marg{cantidad}\marg{descripción}\oarg{descripción para csv}\oarg{descripción sólo para referencias}}
%
%
% \subsubsection{Opción `{\ttfamily descuentos}'\label{subsec:descuentos}}
%
% \noindent
% {\macromandcolor|\Item|\marg{cantidad}\marg{descripción}\oarg{descripción para csv}\marg{precio unitario}\marg{\% descuento} \\ \mbox{}\hfill \oarg{descripción sólo para referencias}}
% \DescribeMacro{\Item}
% \DescribeMacro{\ItemR}
% \DescribeMacro{\ItemA}
% \DescribeMacro{\ItemE}
%
% Cuando pedimos descuentos individuales (opción |descuentos| al cargar la clase), todas las macros |\Item| tienen un posible\footnote{Es un argumento opcional, aunque va entre llaves como todos los demás.} argumento más al final que es el descuento porcentual. No especificarlo da el mismo resultado que un descuento de 0\% sobre ese ítem:
%
%
% \subsubsection{Opción `{\ttfamily sincantidad}'\label{subsec:sincantidad}}
%
% \noindent
% \DescribeMacro{\Item}
% \DescribeMacro{\ItemR}
% \DescribeMacro{\ItemA}
% \DescribeMacro{\ItemE}
% {\macromandcolor|\Item|\marg{descripción}\oarg{descripción para csv}\marg{precio unitario}\oarg{descripción sólo para referencias}}
%
% \noindent {\macromandcolor|\Item|\marg{descripción}\oarg{descripción para csv}\marg{precio unitario}\marg{\% descuento} \\ \mbox{}\hfill \oarg{descripción sólo para referencias}}
%
% Por otra parte, si se usa la opción |sincantidad| al cargar la clase, todos los comandos |\Item| aceptan un argumento menos, que es el de la cantidad; eso significa que sin usar la opción |descuentos|, la sintaxis de |\Item| queda como lo indica la primera línea, y usando |descuentos|, queda como lo indica la segunda línea.
%
% \noindent
% \DescribeMacro{\ItemX}{\macrooptcolor|\ItemX|\marg{descripción}\oarg{descripción para csv}\oarg{descripción sólo para referencias}}
% 
% Puesto que |\ItemX| se usa para elementos sin precio, al usar |sincantidad| quedará aceptando sólo un argumento obligatorio que es la descripción y dos opcionales para la descripción en el archivo |.csv| y la descripción larga para referenciar luego. Cuando se activa la opción |descuentos|, |\ItemX| queda igual puesto que esta macro no lleva precio, entonces tampoco acepta descuentos.
%
% {\itshape Todas las variaciones de estos comandos |\Item| pueden apreciarse en los ejemplos.}
%
%
% \subsection{Descripción larga al final de la factura}
%
% Una vez terminado el entorno de la factura se puede escribir una descripción larga que saldrá en una caja -cuyo estilo puede cambiarse-\footnote{El diseño de esta descripción consiste en su linea de contorno, cambiable como ya explicamos en la subsección \ref{subsec:espacioslineas}, específicamente con la macro \cs{LineaDescripcion}.} debajo del cuadro de la factura, esto va dentro de un |\parbox| en la implementación de la caja. Si esta descripción no se escribe, simplemente no va a aparecer nada debajo del cuadro de la factura.
%
% \noindent \DescribeMacro{\Descripcion}
% {\macrooptcolor|\Descripcion|\marg{texto en formato de párrafos}}
%
% {\itshape Como podemos ver, |\Descripcion| es otro de los pocos comandos que no se usan en el preámbulo sino dentro del documento, jústamente al terminar el ambiente} |factura|.
%
%
% \section{Reverso de la factura}
%
% \DescribeEnv{reverso}
%  La factura puede tener información escrita en su reverso, por ejemplo para ciertas condiciones cuando estas necesitan mucho espacio. Ese texto se escribe dentro de un entorno llamado |reverso|, que puede o no incluirse según el requerimiento del usuario. si está escrito en el documento fuente también podrá decidirse si se incluye o no en el documento compilado.
%
% \iffalse
%<*literal>
% \fi
{\VERBATIM
\begin{verbatim}
    \begin{reverso}
     (Aquí va toda la información que irá escrita
      en el reverso de la hoja de la factura.)
    \end{reverso}
\end{verbatim}
}
% \iffalse
%</literal>
% \fi
%
% El reverso de la factura va escrito con la tipografía que el usuario elije con el comando |\LetraReverso|, como se explicará en la sección \ref{sec:tipografia}, y las dimensiones de la página se establecen con el comando |\GeometriaReverso| (sección \ref{sec:comandos}). Dicha geometría viene con un valor predeterminado en el |.def|.
%
% Si se compila con la opción |sinreverso|, lo escrito dentro del ambiente |reverso| no saldrá como reverso en la hoja de la factura.
%
%
% \section{El reporte}
%
% \DescribeEnv{reporte}
%  Para elaborar un reporte, lo único que se debe hacer es escribir el texto del reporte dentro del ambiente para ese fin, que se llama |reporte|, de manera que en el documento tenemos un ambiente para la factura y otro para el reporte, y elegimos al momento de compilar qué es lo que queremos obtener.
%
% \iffalse
%<*literal>
% \fi
{\VERBATIM
\begin{verbatim}
    \begin{reporte}
     (Aquí va el reporte, sin límite la extensión del texto.)
    \end{reporte}
\end{verbatim}
}
% \iffalse
%</literal>
% \fi
%
% Cuando se compila con |conreporte|, el reporte sale desde la segunda página, o desde la tercera si se usa |twoside|.
%
% Cuando se compila con |soloreporte|, el motor de compilación no procesa la factura en favor del reporte, y sin especificar ninguna de estas dos opciones, el reporte es ignorado y sale únicamente la factura (o la prefactura o cotización).
%
% El reporte empieza con un encabezado y luego un título especificable; el encabezado consiste en el nombre del proveedor, el nombre del cliente y la persona de contacto. Luego de ese encabezado viene el título del reporte y después empieza lo que el usuario escriba dentro del entorno del reporte.
%
% Hay que tener en cuenta que, como se explicó en la sección \ref{sec:comandos}, el reporte puede tener una imagen de fondo, que puede ser un membrete. Ese membrete, si se quiere en todas las páginas deberá incluirse con |\FondoReporte*|, y si se quiere sólo en la primera página del reporte deberá usarse |\FondoReporte|. Ahora, si se usa la versión sin |*|, será necesario también usar |\InicioReporte| con una distancia para empezar a escribir un poco más abajo en la primera página, ya que es la única que tiene el membrete. Pero si se usa la versión con |*|, entonces basta fijar los márgenes para todas las páginas con |\GeometriaReporte|, tal como se adelantó en la sección \ref{sec:comandos}.
%
% El reporte estará escrito con la tipografía especificable con |\LetraReporte|, como se explicará más adelante en la sección \ref{sec:tipografia}.
%
% Por otra parte, al reporte se le asigna un título, que puede aparecer o no en la cabecera de página o en el pie de página. Si se quiere que aparezca el título en esos lugares, debe usarse |\TituloReporte*|, de lo contrario, |\TituloReporte|. Si no se especifica el título, entonces la información mostrada allí es la referencia a la factura asociada a este reporte, como se comentó antes cuando se introdujeron estos comandos.
%
%
% \subsection{Estilos de página para el reporte\label{subsec:pagestyle}}
%
% La clase define varios estilos de página\footnote{Usables en el reporte únicamente.} de \LaTeX{}, cambiables con los comandos usuales: |\pagestyle| y |\thispagestyle| dentro del entorno del reporte, pero es mejor definir el estilo de página en el preámbulo del documento, con |\EstiloPagReporte|.
%
% Estos estilos de página están definidos con el paquete |fancyhdr|, y son: \begin{inparaenum}[\bfseries i)]\item |plain|, \item |empty|, \item |soloarriba|, \item |soloabajo|, y \item |solonumpag|\end{inparaenum}.
%
% Con |plain|\footnote{Por razones de compatibilidad usamos dos estilos de página conocidos en \LaTeX, que son |plain| y |empty|, siendo |plain| uno de nuestros predeterminados.}, en la cabecera saldrá la información sobre el reporte y en el pie de página saldrá el número de página; con |empty| no sale nada en ninguna parte; con |soloarriba| y |soloabajo|, toda esa información anterior sale en la cabecera o en el pie de página, respectivamente, y con |solonumpag|, lo único que aparece es el número de página centrado en el pie de página.
%
% El formato con la información a poner en estas partes de las páginas del reporte es cambiable con |\InfoPagina|, descrito anteriormente.
%
% Cuando se usa |\FondoReporte*|, como se supone que se usa para un membrete en todas las páginas, el estilo de página predeterminado del reporte será |soloabajo|, para todas las páginas, porque no tendría sentido tener un membrete en todas las páginas y una cabecera como la descrita anteriormente. Si el comando |\FondoReporte*| se está usando para algo distinto de un membrete, por ejemplo una marca de agua, entonces es probable querer cambiar los estilos de página con |\EstiloPagReporte|.
%
% Si se usa |\FondoReporte|, el estilo predeterminado para la primera página será |solonumpag|, y |plain| para el resto de las páginas, y si se no se especifica fondo alguno, entonces el estilo de página predeterminado para todas las páginas será |plain|.
%
% El usuario podrá definir más estilos de página adicionales para colocar el texto que se quiera en el lugar que se quiera, y usar esos estilos de página con |\EstiloPagReporte|. Para eso puede leer la documentación del paquete |fancyhdr|, que es lo que esta clase usa para los estilos de página. También puede consultar |factura.def|, que es donde están predefinidos los estilos que trae la clase.
%
% El grosor de la línea decorativa que coloca el paquete |fancyhdr| está definido en el archivo |.def|, y es cambiable en la macro |\@GrosorLineaInfoPag| como podrá verse en dicho archivo; no definimos un comando para ello por no ser algo relevante pero igual podrá cambiarse con |\renewcommand| por ejemplo, como se hace en dicho |.def|.
%
%
% \section{Tipografías para cada elemento a escribir\label{sec:tipografia}}
%
% Cada elemento escrito en la factura o reporte es sujeto a ser mostrado con una tipografía distinta, para lo cual existen comandos con los que se asigna el tipo de letra o estilo para cada cosa que sale escrita en el documento, por ejemplo, se puede cambiar el tipo de letra con el que se escribe la fecha, el tipo de letra con el que salen los elementos dentro del cuadro de la factura, el tipo de letra de los totales, de los títulos, de las cabeceras o pies de página en el reporte, etc.
%
% Cuando en lo sucesivo nos referiremos a tipografía, tipo de letra, o letra, nos estaremos refiriendo a cualquier cambio en la tipografía que pueda significar el nombre de la letra, su tamaño o atributos como negritas, es decir, mediante cualquiera de estos comandos podemos cambiar cualquier aspecto de la tipografía. Mirando el archivo |.def| pueden verse todos los valores predeterminados, que la mayoría de ellos son simplemente cambios en el tamaño (macro |\fontsize| de \LaTeX{}), y en algunos casos negritas.
%
% {\setlength{\parindent}{0pt}
% \DescribeMacro{\LetraItems}{\macrooptcolor|\LetraItems|\marg{tipo de letra}}
%
%  Con este comando cambiamos la tipografía utilizada en los ítems o elementos que están en el cuadro de la factura, de manera que {\itshape cambiar la longitud \cs{baselineskip} con el segundo argumento de \cs{fontsize} hará que la tabla tenga una altura distinta de las celdas.}
%
%  Lo anterior significa que {\bfseries para cambiar la altura de las celdas, la forma correcta es a través de cambiar el tamaño de la letra de sus elementos con este comando. (Véase el archivo |.def|)}.
%
% \DescribeMacro{\LetraNumeros}{\macrooptcolor|\LetraNumeros|\marg{tipo de letra}}
%
%  Este comando sirve para cambiar el tipo de letra de tota la información numérica que aparece en las distintas columnas del cuadro de la factura, sin contar los totales, para los cuales hay otros comandos.
%
% \DescribeMacro{\LetraTipoIVA}{\macrooptcolor|\LetraTipoIVA|\marg{tipo de letra}}
%
%  El tipo del IVA es la letra que indica qué IVA está siendo gravado, por ejemplo, la “|G|”, “|R|” y “|A|” que puede salir al lado de los elementos con IVA general, reducido y adicional respectivamente, y “|(E)|” al lado de los elementos exentos. Con esta macro cambiamos la letra con la que eso sale.
%
% \DescribeMacro{\LetraTitColumnas}{\macrooptcolor|\LetraTitColumnas|\marg{tipo de letra}}
%
%  Con |\LetraTitColumnas| podemos cambiar la tipografía de los títulos de los campos que hay en el cuadro de la factura, es decir, los textos “Cantidad”, “Descripción”, “Precio U.”, etc.
%
% \DescribeMacro{\LetraTitTotales}{\macrooptcolor|\LetraTitTotales|\marg{tipo de letra}}
%
%  Con este comando cambiamos la tipografía de los nombres de los totales, es decir, “IVA G.”, “BI G.”, salvo el total general que se cambia con |\LetraTitTotal|.
%
% \DescribeMacro{\LetraNumTotales}{\macrooptcolor|\LetraNumTotales|\marg{tipo de letra}}
%
%  Con esta macro se cambia la tipografía utilizada en los totales numéricos de la factura, exceptuando el total general que se cambia con |\LetraNumTotal|.
%
% \DescribeMacro{\LetraTitTotal}{\macrooptcolor|\LetraTitTotal|\marg{tipo de letra}}
%
%  Para cambiar la tipografía de la palabra que indica el total general, e.g. “TOTAL”, lo hacemos con |\LetraTitTotal|.
%
% \DescribeMacro{\LetraNumTotal}{\macrooptcolor|\LetraNumTotal|\marg{tipo de letra}}
%
%  Para la tipografía del total general (la celda con el número del total final), utilizamos este comando.
%
% \DescribeMacro{\LetraTitEnc}{\macrooptcolor|\LetraTitEnc|\marg{tipo de letra}}
%
%  Con este comando cambiamos el tipo de letra de la información en el título de las cajas del encabezado de la factura, por ejemplo los textos: “Resumen” y “Datos del cliente”.
%
% \DescribeMacro{\LetraEncFactura}{\macrooptcolor|\LetraEncFactura|\marg{tipo de letra}}
%
%  Podemos también cambiar el tipo de letra de la información del encabezado de la factura, es decir, el tipo de letra del texto que escribe el usuario en el resumen o de toda la identificación del cliente en la factura. El resumen naturalmente acepta más cambios de tipografía escribiéndolos dentro del párrafo que se escribe con el comando |\Resumen|.
%
% \DescribeMacro{\LetraEncReporte}{\macrooptcolor|\LetraEncReporte|\marg{tipo de letra}}
%
%  Para cambiar la tipografía de la información del encabezado del reporte usamos |\LetraEncReporte|.
%
% \DescribeMacro{\LetraNotaPrevia}{\macrooptcolor|\LetraNotaPrevia|\marg{tipo de letra}}
%
%  Con |\LetraNotaPrevia| podemos cambiar la tipografía con la que se escribe la -opcional- nota previa al cuadro de la factura.
%
% \DescribeMacro{\LetraNotaInterna}{\macrooptcolor|\LetraNotaInterna|\marg{tipo de letra}}
%
%  Con |\LetraNotaInterna| cambiamos el formato de la nota opcional que aparece como coletilla de la factura (nota interna).
%
% \DescribeMacro{\LetraNotaExterna}{\macrooptcolor|\LetraNotaExterna|\marg{tipo de letra}}
%
%  También podemos cambiar la tipografía de la nota externa a la factura, es decir, la -opcional- nota que va al terminar el cuadro de la factura antes de la -opcional también- descripción larga.
%
% \DescribeMacro{\LetraDescripcion}{\macrooptcolor|\LetraDescripcion|\marg{tipo de letra}}
%
%  Con este comando cambiamos la tipografía de la descripción larga; naturalmente cualquier cambio puede hacerse luego dentro del texto de la descripción, igual que con el resumen.
%
% \DescribeMacro{\LetraFirmas}{\macrooptcolor|\LetraFirmas|\marg{tipo de letra}}
%
%  Con |\LetraFirmas| podemos cambiar la letra usada para escribir el nombre debajo de la línea para firmar en la factura.
%
% \DescribeMacro{\LetraNotaFinal}{\macrooptcolor|\LetraNotaFinal|\marg{tipo de letra}}
%
%  Para cambiar la tipografía de la nota final se hace con este comando. (Aplica el mismo comentario que para la descripción y el resumen).
%
% \DescribeMacro{\LetraDenom}{\macrooptcolor|\LetraDenom|\marg{tipo de letra}}
%
%  La tipografía asociada a la denominación del documento, es decir, al lado del número (e.g. “FACTURA” o “COTIZACIÓN”), y la asociada al texto “No. CONTROL” se cambia con |\LetraDenom|.
%
% \DescribeMacro{\LetraNumeracion}{\macrooptcolor|\LetraNumeracion|\marg{tipo de letra}}
%
%  Con |\LetraNumeracion| cambiamos la tipografía de las numeraciones como la del número de la factura y el número de control.
%
% \DescribeMacro{\LetraFecha}{\macrooptcolor|\LetraFecha|\marg{tipo de letra}}
%
%  La tipografía para la fecha de emisión y la fecha de vencimiento la podemos cambiar con este comando.
%
% \DescribeMacro{\LetraTitReporte}{\macrooptcolor|\LetraTitReporte|\marg{tipo de letra}}
%
%  La letra del título que aparece al principio del reporte se puede cambiar con |\LetraTitReporte|.
%
% \DescribeMacro{\LetraInfoPagina}{\macrooptcolor|\LetraInfoPagina|\marg{tipo de letra}}
%
%  Las cabeceras o pies de página del reporte también tienen un comando para su tipo de letra, y es |\LetraInfoPagina|.
%
% \DescribeMacro{\LetraReverso}{\macrooptcolor|\LetraReverso|\marg{tipo de letra}}
%
%  Con |\LetraReverso| cambiamos la tipografía de la información que escribamos en el reverso de la hoja de la factura.
%
% \DescribeMacro{\LetraReporte}{\macrooptcolor|\LetraReporte|\marg{tipo de letra}}
%
%  Finalmente, para cambiar la tipografía con la que se inicia el reporte tenemos este comando. Obviamente también luego se puede hacer cualquier cambio dentro del ambiente |reporte|.
%
% }
%
%
% \section{Etiquetas y referencias a ítems de la factura\label{sec:etiquetas}}
%
% Esta clase |factura| soporta agregar etiquetas a cada elemento de la factura que luego en el reporte o en casi cualquier otra parte queramos referenciar; {\slshape no se pueden referenciar etiquetas solamente en la nota previa ni en el resumen.}
%
% \noindent \DescribeMacro{\label}{\macrooptcolor|\label|\marg{etiqueta}}
%
%  Para ello usamos la conocida macro |\label| de \LaTeX{} justamente después del ítem que nos interese luego referenciar, por ejemplo (usando descuentos individuales):
%
% \iffalse
%<*literal>
% \fi
{\VERBATIM
\begin{verbatim}
    \begin{factura}
     \Item{1}{Artículo gravado con IVA general}{1000}
     \Item{3}{Otro artículo gravado con IVA general}{1500}{10}[Descripción larga 
      para referenciar el reporte u otro fin.

      Con formato de párrafo con saltos de línea si se desea.]\label{el-otro}
    \end{factura}
\end{verbatim}
}
% \iffalse
%</literal>
% \fi
%
% De esta forma creamos una etiqueta llamada “el-otro” para el segundo elemento, que luego podremos citar por ejemplo en la descripción larga o en el reporte con comandos similares a los que \LaTeX{} provee para referenciar etiquetas.
%
% \noindent \DescribeMacro{\itemref}{\macrooptcolor|\itemref|\marg{etiqueta}}\\
% \noindent \DescribeMacro{\cantref}{\macrooptcolor|\cantref|\marg{etiqueta}}\\
% \noindent \DescribeMacro{\descref}{\macrooptcolor|\descref|\marg{etiqueta}}\\
% \noindent \DescribeMacro{\puref}{\macrooptcolor|\puref|\marg{etiqueta}}\\
% \noindent \DescribeMacro{\subtref}{\macrooptcolor|\subtref|\marg{etiqueta}}\\
% \noindent \DescribeMacro{\dctoref}{\macrooptcolor|\dctoref|\marg{etiqueta}}\\
% \noindent \DescribeMacro{\ivaref}{\macrooptcolor|\ivaref|\marg{etiqueta}}\\
% \noindent \DescribeMacro{\ptref}{\macrooptcolor|\ptref|\marg{etiqueta}}\\
% \noindent \DescribeMacro{\ldescref}{\macrooptcolor|\ldescref|\marg{etiqueta}}\par
%
% Si se quiere mostrar el número de la línea que ocupa el ítem en la lista lo hacemos con |\itemref|, que en el ejemplo aparecería el número “|2|” indicando que es el segundo ítem de la factura\footnote{Cuando se usa \cs{itemref} conviene enumerar los ítems en la factura con la opción \texttt{numitem}.}. Con |\cantref| escribimos la cantidad del elemento citado mediante la etiqueta, en nuestro ejemplo sería “|3|”. Para la descripción del ítem usamos |\descref|, que en el caso del ejemplo sería el texto “|Otro artículo gravado con IVA general|”. Para el precio unitario usamos |\puref|, en el ejemplo “|1.500,00|”. El subtotal lo mostraríamos con |\subtref|; en nuestro caso “|4.500,00|”. Si se quiere mostrar el porcentaje del descuento del ítem citado usamos |\dctoref|, que para el ejemplo sería “|10|”. Para el porcentaje del IVA, usamos |\ivaref|, en nuestro caso escribiría “|12|”. Por último, para mostrar el precio total usamos |\ptref|, y en nuestro ejemplo eso escribiría “|4.050,00|” puesto que aplicamos un descuento del 10\%.
%
% Podemos ver que el uso de los comandos |\puref|, |\subtref| y |\ptref| nos muestra el precio en el mismo formato con el que lo vemos en la presentación de la factura, en cuanto al redondeo y al separador de miles y decimal.
%
% También podemos referenciar la -opcional- descripción larga que podemos usar por ejemplo en el reporte; esa descripción que va entre corchetes la podremos citar con |\ldescref|, y en nuestro ejemplo sería:
% “Descripción larga para referenciar el reporte u otro fin. Con formato de párrafo con saltos de línea si se desea.”\footnote{Aquí suprimimos los saltos de línea.}
%
% Con la asignación de estas etiquetas y su uso mediante las referencias podemos de manera conveniente citar cualquier ítem que se haya facturado, de manera dinámica sin tener que reescribir en la descripción larga o en el reporte aquello que ya se escribió en el cuadro de la factura. Para estas referencias se escogieron nombres de las macros que terminan en “\texttt{ref}” por compatibilidad con comandos de \LaTeX{} como |\pageref|.
%
%
%
% \section{Referenciar los valores establecidos en el preámbulo y los totales\label{sec:ref}}
%
%  En el preámbulo el usuario definió distintas macros, es decir, estableció la razón social del cliente, el R.I.F. y la dirección, entre otras cosas. Cualquiera de esas definiciones hechas en el preámbulo puede usarse luego en otra parte del documento con el mismo comando que se usó para su definición. Lo anterior significa que tenemos las siguientes macros posibles para usar dentro del cuerpo del documento (factura, reporte o reverso): |\Moneda|, |\RazonSocial|, |\RIF|, |\Direccion|, |\Telefono|, |\Email|, |\Contacto|, |\Emisor|, |\Conforme|, |\Proveedor|, |\Fecha|, |\FechaVencimiento|\footnote{La fecha de vencimiento no se establece en el preámbulo explícitamente sino a través de \cs{Credito} pero la clase dispone de esta interfaz para acceder a ella así como también con la interfaz dada por \texttt{datetime2}.}, |\Credito|, |\Num|, |\NumControl|, |\NumSerieControl|, |\Resumen|, |\NotaPrevia|, |\NotaInterna|, |\NotaExterna|, |\NotaFinal|, |\TituloReporte|.
%
% Esas macros, en el preámbulo funcionan para definir valores y textos, pero si ya no estamos en el preámbulos sino dentro del cuerpo del documento y escribimos una de esas macros\footnote{Dentro del cuerpo del documento estas macros obviamente no aceptan argumentos.}, lo que nos va a escribir será lo que está definido bajo esa macro, así sea con macros que no llegamos a utilizar explícitamente pero que tienen un valor predefinido como |\Moneda|.
%
% Por otra parte, en el documento el usuario también podrá referirse a cualquiera de los totales de la siguiente tabla\footnote{Las macros de la tabla, para representarlas en el mismo formato que en la factura, debemos usar la macro \cs{numprint}.}:
%
% \noindent \begin{tabular}{rl}
% \hline
% Total & Comando \\ \hline
% Bases imponibles & |\BIG|, |\BIR|, |\BIA|, |\BIE|\footnote{Lo llamamos \cs{BIE} para mantener la nomenclatura pero no es una base imponible ya que es un monto exento de impuesto.} \\
% IVA & |\AlicuotaG|, |\AlicuotaR|, |\AlicuotaA| \\
% Descuento en IVA & |\AlicuotaGDesc| \\
% Subtotales\footnote{El subtotal es el monto sin aplicar descuentos generales (los que se hacen con los comandos \cs{Descuento***}, es decir, los descuentos que no son individuales a cada item sino al totalizar), que coincide con la base imponible cuando no hay esos descuentos.} & |\SubtG|, |\SubtE|, |\SubtR|, |\SubtA| \\
% Descuentos & |\DescG|, |\DescE|, |\DescR|, |\DescA| \\
% Total final & |\Total| \\
% \hline
% \end{tabular}
%
% También ya mencionamos que la fecha de emisión y vencimiento se pueden acceder con la interfaz dada por el paquete |datetime2|, por ejemplo, para mostrar la fecha de emisión, podremos usar tanto |\Fecha| como |\DTMusedate{FechaEmision}|, y para la fecha de vencimiento, tanto |\FechaVencimiento| como |\DTMusedate{FechaVencimiento}|.
%
% La cantidad de ítems en la factura la podemos acceder via el contador |NumItem|, por ejemplo así |\arabic{NumItem}|.
%
% En los ejemplos hay casos con todas estas funciones de esta sección.
%
%
% \section{Más personalizaciones}
%
% \DescribeMacro{\@BeforeEndPreamble}
% La clase provee de un “gancho” llamado \cs{BeforeEndPreamble}. Lo que allí se agregue (es acumulativo, se puede ejecutar mas de una vez) será ejecutado por la clase |factura| luego de que hayan sido evaluadas las opciones de la clase y tomadas todas las decisiones en cuanto a opciones que impliquen otras opciones o que sean incompatibles con otras opciones; una vez que todas esas decisiones fueron tomadas y todo queda establecido, se ejecuta \cs{BeforeEndPreamble}, esto con el propósito de que el usuario pueda personalizar más el comportamiento de sus archivos |.tex|. Por ejemplo si el usuario hace lo siguiente, no funcionará:
%
% \iffalse
%<*literal>
% \fi
{\VERBATIM
\begin{verbatim}
    \documentclass[cotizacion]{factura}
    [...]
    \ifcotizacion\FondoFactura{Fondo}\fi
    \begin{document}
     \begin{factura}
      [...]
     \end{factura}
    \end{document}
\end{verbatim}
}
% \iffalse
%</literal>
% \fi
%
% No funcionará porque el condicional \cs{ifcotizacion} aún no se ha establecido (eso lo hace la clase con el gancho \cs{AtEndPreamble} de |etoolbox| luego de que todo el preámbulo haya sido analizado), por lo que se evaluará como falso; tendría el usuario que agregarlo dentro de un gancho \cs{AtEndPreamble} pero tampoco funcionará del todo, para eso esta clase provee del gancho \cs{BeforeEndPreamble}, y se escribiría este ejemplo así:
%
% \iffalse
%<*literal>
% \fi
{\VERBATIM
\begin{verbatim}
    \documentclass[cotizacion]{factura}
    [...]
    \BeforeEndPreamble{\ifcotizacion\FondoFactura{Fondo}\fi}
    \begin{document}
     \begin{factura}
      [...]
     \end{factura}
    \end{document}
\end{verbatim}
} 
% \iffalse
%</literal>
% \fi
%
% De esta forma, el condicional se evaluará después de que se hayan establecido todas las opciones (en este caso, |cotizacion|) y el usuario podrá inclusive apagar opciones y encenderlas nuevamente usando los condicionales (como se explicó en la subsección \ref{subsec:condicionales} porque después de ejecutar este gancho, automáticamente se volverán a evaluar las opciones y a tomar todas las decisiones de implicaciones y compatibilidades.
%
% \newpage
% \DescribeMacro{\@PosicionaFechas}
% \DescribeMacro{\@PosicionaDenom}
% \DescribeMacro{\@PosicionaNumControl}
% \DescribeMacro{\@PosicionaNotaFinal}
% \DescribeMacro{\@PosicionaFirmas}
% \DescribeMacro{\@EscribeDatos}
% \DescribeMacro{\@EscribeDatosResumen}
% \DescribeMacro{\@EscribeDatosReporte}
% \DescribeMacro{\@EscribeNotaFinal}
% \DescribeMacro{\@EscribeNotaPrevia}
% \DescribeMacro{\@DibujaLinea}
% \DescribeMacro{\@EscribeItemsEnCSV}
% \DescribeMacro{\@EscribeTotalesEnCSV}
% \DescribeMacro{\@EncabezadoFactura}
% \DescribeMacro{\@EncabezadoReporte}
% También es posible darle aspecto distinto a muchas de las partes que conforman la factura. Internamente en la clase, cada parte es ubicada en la hoja con un comando y su contenido es escrito en el papel con otro comando. Los comandos que más probablemente son sujetos a ser cambiados por el usuario son los comandos que escriben el texto más que aquellos que lo ubican en la hoja, por tal razón, los comandos que posicionan aquello que va a escribirse lo definimos dentro del archivo |.cls|, y los que propiamente escriben cada parte de la hoja están definidos en el |.def|. Ambas fuentes las podrán consultar los usuarios para mayor comprensión, y para la total personalización. Recomendamos empezar por el archivo de definiciones |factura.def|.
%
% Estas macros son internas y sirven para, como dijimos antes, que \LaTeX{} se prepare para escribir el texto en la hoja; todas empiezan con |\@Posiciona| y no debe ser común necesitar cambiarlas.
%
% Todas estas macros llaman a su correspondiente macro con el estilo del elemento a escribir, que comienza con |\@Estilo|, y con cambiar su contenido podemos cambiar la forma como es escrita esa parte de la factura en la hoja.
%
% \DescribeMacro{\@EstiloFechas}
% \DescribeMacro{\@EstiloDenom}
% \DescribeMacro{\@EstiloNumControl}
% \DescribeMacro{\@EstiloNotaPrevia}
% \DescribeMacro{\@EstiloNotaFinal}
% \DescribeMacro{\@EstiloFirmas}
% \DescribeMacro{\@DatosTIKZ}
% \DescribeMacro{\@NotaPreviaTIKZ}
% \DescribeMacro{\@DescripcionTIKZ}
% \DescribeMacro{\@NotaFinalTIKZ}
% Estas últimas son las que están definidas dentro de |factura.def|, y que eventualmente puede desearse cambiarlas.
%
% Adicionalmente están las macros que se encargan de distintas partes de la factura y del reporte como los datos del cliente, el resumen, la nota previa, la nota final, los encabezados, la información en el archivo |.csv|, y la macro que dibuja la línea que inutiliza el espacio en blanco. Todas estas son macros internas que no deben ser llamadas por el usuario, y no consideramos que pueda sea necesario cambiarlas.
%
% Por último, las macros que alojan el diseño \tikz de los datos del cliente, la nota previa, la descripción larga y la nota final son |\@DatosTIKZ|, |\@NotaPreviaTIKZ|, |\@DescripcionTIKZ| y |\@NotaFinalTIKZ|; están definidas en el archivo |factura.def|.
%
% Cualquiera de estas macros que se quiera redefinir habría que hacerlo en el preámbulo del |.tex| y encerrándolas entre |\makeatletter| y |\makeatother| para poder redefinir una macro que contiene un símbolo “|@|”.
%
% {\bfseries \itshape Finalmente, como dijimos en la sección \ref{subsec:aux}, una vez adecuada la clase al diseño que se quiere, se podrán escribir todos los comandos que hacen estos ajustes en un} |.tex| {\itshape o} |.def| {\itshape y cargarlo con} |\input|, |\include| {\itshape o con} |\Cliente|, pudiendo discriminar por clientes dicha personalización.
%
%
%
% \newpage
% \setstretch{1}
% \section{Tabla con todas las opciones}
%
% \noindent La tabla que sigue muestra una lista de las opciones posibles al momento de cargar la clase:
%
% \noindent \begin{tabular}{>{\ttfamily}ll}
%  \rmfamily \bfseries Opción & \bfseries Significado \\ \hline
%  agrupado & Agrupa los totales por tipo en el modo de columnas. \\
%  centradesc & Centra la descripción de los ítems. \\
%  centrado & Centra los totales en el modo de filas. \\
%  conlineasha & Dibuja una línea separando los artículos entre sí. \\
%  conreporte & Produce la factura y el reporte. \\
%  cotización & Imprime una cotización. \\
%  csv & Genera un archivo |.csv| con la información numérica de la factura. \\
%  dcu & Descripción, cantidad, precio unitario. \\
%  denom & Explicita la denominación del documento. \\
%  descuentos & A cada artículo se le puede asignar un descuento individual. \\
%  dosfirmas & Incluye los dos espacios para firmar. \\
%  duc & Descripción, precio unitario, cantidad. \\
%  electro\footnote{Ya no sirve esta opción.} & Calcula los descuentos para el IVA para pagos por medios electrónicos. \\
%  expandecuadro & Llena con espacio extra el cuadro de la factura. \\
%  filas & Escribe los totales en filas. \\
%  G & Agrega una “|G|” en los ítems con IVA general. \\
%  impdesc & Escribe la denominación del IVA en la descripción. \\
%  iva & Expresa en otra columna el porcentaje de IVA para cada elemento a facturar. \\
%  linea & Dibuja una línea que inutiliza el espacio en blanco en el cuadro de la factura. \\
%  nospanish & No carga |babel| ni |fontenc|. \\
%  numitem & Enumera los ítems en la factura en otra columna extra al principio. \\
%  prefactura & Imprime una prefactura. \\
%  sincantidad & Los artículos van sin expresar su cantidad. \\
%  sinexpandir & No expande elementos en la factura. \\
%  sinfirmas & No incluye espacios para firmar. \\
%  siniva & No incluye IVA, pero lo deja expresado. \\
%  sinivaexpresado & No incluye IVA ni lo expresa. \\
%  sinlineahni & No dibuja la línea horizontal entre totales y nota interna. \\
%  sinlineahtit & No dibuja la línea horizontal del título de las columnas. \\
%  sinlineahtot & No dibuja la línea horizontal entre ítems y totales. \\
%  sinlineashi & No dibuja las líneas horizontales internas. \\
%  sinlineashe & No dibuja las líneas horizontales externas. \\
%  sinlineash & No dibuja ninguna línea horizontal. \\
%  sinlineasvi & No dibuja las líneas verticales internas. \\
%  sinlineasve & No dibuja las líneas verticales externas. \\
%  sinlineasv & No dibuja ninguna línea vertical. \\
%  sinmarcas & No escribe el carácter de fin de línea ni el carácter en las celdas vacías \\
%            & cuando la descripción ocupa más que una línea. \\
%  sinmonedatotales & Quita el signo monetario de las totalizaciones. \\
%  sinnumero & No pone el número cuando se usa |denom|. \\
%  sintotales & No escribe los totales al final del cuadro; para cotizaciones. \\
%  sinreverso & No compila el reverso de la factura así se haya escrito. \\
%  soloreporte & Produce únicamente el reporte. \\
%  subtotal & En otra columna muestra el subtotal de cada ítem antes aplicar descuento. \\
%  todosiva & Expresa todos los tipos de IVA. \\
%  twoside & Imprime el reporte a dos caras. \\
%  unafirma & Sólo incluye la firma del emisor. \\
%  vencimiento & Escribe la fecha de vencimiento así no haya crédito en la factura. \\
% \end{tabular}
%
%
% \newpage
% \section{Tabla con todos los comandos}
%
% A continuación presentamos una tabla con todos los comandos, que salvo |\FirmaReporte|, los |\Item| y |\Descripcion|, se poner todos en el preámbulo.
%
% \noindent \begin{tabular}{>{\ttfamily}ll}
% \rmfamily \bfseries Comando & \bfseries Significado \\ \hline
%  |\cantref| & Muestra la cantidad del ítem referenciado. \\
%  |\CI| & Cédula de identidad del cliente. \\
%  |\Cliente| & Carga un archivo |.def| o |.tex| con personalizaciones. \\
%  |\NotaFinal| & Establece la nota final. \\
%  |\Conforme| & Nombre de quien recibe la factura, para el espacio de la firma. \\
%  |\Contacto| & Persona de contacto. \\
%  |\Credito| & Establece la cantidad de días para el crédito de la factura. \\
%  |\dctoref| & Muestra al descuento del ítem referenciado. \\
%  |\Denom| & Nombre de la denominación de documento. \\
%  |\descref| & Muestra a la descripción del ítem referenciado. \\
%  |\Descripcion| & Descripción larga de la factura\footnote{El comando \cs{Descripcion} se escribe dentro del documento, luego del ambiente \texttt{factura}, y no en el preámbulo.}. \\
%  |\Descuento| & Porcentaje de descuento para todos los elementos. \\
%  |\Descuento*| & Descuento neto en el subtotal, para todos los elementos. \\
%  |\Descuento[GRAE]| & Descuento en porcentaje, discriminado según el tipo de ítem. \\
%  |\Descuento[GRAE]*| & Descuento en el subtotal, discriminado según el tipo de ítem. \\
%  |\Direccion| & Dirección del cliente. \\
%  |\Email| & E-mail del cliente. \\
%  |\Emisor| & Nombre de quien emite la factura, para el espacio de la firma. \\
%  |\EstiloPagReporte| & Establece el estilo de página del reporte. \\
%  |\Fecha| & Establece la fecha de emisión de la factura en formato predeterminado. \\
%  |\Fecha*| & Establece la fecha de emisión de la factura en formato ISO. \\
%  |\FirmaFactura| & Carga la imagen de la firma en la factura. \\
%  |\FirmaReporte| & Carga la imagen de la firma en el reporte\footnote{El comando \cs{FirmaReporte} se escribe dentro del ambiente \texttt{reporte} y no en el preámbulo.}. \\
%  |\FondoFactura| & Carga la imagen del fondo de la factura. \\
%  |\FondoReporte| & Carga la imagen del fondo del reporte para la primera página. \\
%  |\FondoReporte*| & Carga la imagen del fondo del reporte para todas las páginas. \\
%  |\FormatoFecha| & Establece lo relacionado con el formato de las fechas. \\
%  |\GeometriaFactura| & Establece los parámetros de la geometría de la factura. \\
%  |\GeometriaReporte| & Establece los parámetros de la geometría del reporte. \\
%  |\InfoPagina| & Cambia la información de las cabeceras o pies de página del reporte. \\
%  |\InicioReporte| & Distancia hasta el inicio del texto en la primera hoja del reporte. \\
%  |\Item[RAEX]| & Agrega un elemento a la factura\footnote{Estos comandos \cs{Item} se escriben dentro del ambiente \texttt{factura} y no en el preámbulo.}. \\
%  |\itemref| & Muestra el número de línea en la factura del ítem referenciado. \\
%  |\ivaref| & Muestra al IVA del ítem referenciado. \\
%  |\label| & Agrega una etiqueta a un ítem de la factura\footnote{Se mantiene el nombre “\cs{label}” por compatibilidad con los comandos conocidos de \LaTeX. Este comando también se escribe dentro del ambiente \texttt{factura}.}. \\
%  |\ldescref| & Muestra a la descripción larga del ítem referenciado. \\
% \end{tabular}
% \newpage
% \noindent \begin{tabular}{>{\ttfamily}ll}
% \rmfamily \bfseries Comando & \bfseries Significado \\ \hline
%  |\LetraNotaFinal| & Tipografía para la nota final. \\
%  |\LetraDenom| & Tipografía para la denominación del documento. \\
%  |\LetraDescripcion| & Tipografía para la descripción en la factura. \\
%  |\LetraEncFactura| & Tipografía para la información en las cajas del encabezado. \\
%  |\LetraEncReporte| & Tipografía para la información del encabezado del reporte. \\
%  |\LetraFecha| & Tipografía para la fecha de emisión. \\
%  |\LetraFirmas| & Tipografía para los nombres de quienes firman la factura. \\
%  |\LetraInfoPagina| & Tipografía para las cabeceras o pies de página del reporte. \\
%  |\LetraItems| & Tipografía para los ítems en la factura. \\
%  |\LetraNotaExterna| & Tipografía para la nota externa al cuadro de la factura. \\
%  |\LetraNotaInterna| & Tipografía para la nota interna al cuadro de la factura. \\
%  |\LetraNotaPrevia| & Tipografía para la nota previa al cuadro de la factura. \\
%  |\LetraNumeracion| & Tipografía para las numeraciones. \\
%  |\LetraNumeros| & Tipografía para los números en la factura. \\
%  |\LetraNumTotal| & Tipografía para el total final numérico. \\
%  |\LetraNumTotales| & Tipografía para los totales numéricos. \\
%  |\LetraReporte| & Tipografía para el texto del reporte. \\
%  |\LetraReverso| & Tipografía para el reverso de la factura. \\
%  |\LetraTipoIVA| & Tipografía para el tipo de IVA. \\
%  |\LetraTitColumnas| & Tipografía para los nombres de las columnas del cuadro de la factura. \\
%  |\LetraTitEnc| & Tipografía para los títulos en las cajas del encabezado de la factura. \\
%  |\LetraTitReporte| & Tipografía para el título del reporte. \\
%  |\LetraTitTotal| & Tipografía para el nombre del total final. \\
%  |\LetraTitTotales| & Tipografía para los nombres de los totales. \\
%  |\LineaNotaFinal| & Grosor de la línea de la caja de la nota final, \\
%                      & curvatura de la esquina y sombra. \\
%  |\LineaDescripcion| & Grosor de la línea de la caja de la descripción, \\
%                      & curvatura de la esquina y sombra. \\
%  |\LineaEncabezado| & Grosor de la línea de las cajas del encabezado de la factura, \\
%                     & curvatura de la esquina y sombra. \\
%  |\LineaNotaPrevia| & Grosor de la línea de la caja de la nota previa al cuadro, \\
%                     & curvatura de la esquina y sombra. \\
%  |\LineaFirmas| & Longitud y grosor de las líneas para firmar. \\
% \end{tabular}
% \newpage
% \noindent \begin{tabular}{>{\ttfamily}ll}
% \rmfamily \bfseries Comando & \bfseries Significado \\ \hline
%  |\Moneda| & Cambia el signo monetario a mostrar. \\
%  |\Nombre| & Establece el nombre del cliente. \\
%  |\NotaInterna| & Escribe una nota interna al cuadro de la factura. \\
%  |\NotaExterna| & Escribe una nota extra externa al cuadro de la factura. \\
%  |\NotaPrevia| & Escribe una nota antes del cuadro de la factura. \\
%  |\Num| & Número de la factura. \\
%  |\NumControl| & Número de control de la factura. \\
%  |\NumSerieControl| & Número de serie de control de la factura. \\
%  |\PosNotaFinal| & Posición $(x_o,x_f,y)$ del párrafo con la nota final. \\
%  |\PosNotaFinal*| & Posición $(x_o,x_f,y)$ del párrafo con la nota final \\
%                     & (el texto se despliega hacia arriba). \\
%  |\PosControl| & Posición $(x,y)$ del número de control. \\
%  |\PosControl*| & Posición $(x,y)$ del número de control \\
%                 & (el texto se despliega hacia arriba). \\
%  |\PosDenom| & Posición $(x,y)$ del texto de la denominación. \\
%  |\PosDenom*| & Posición $(x,y)$ del texto de la denominación \\
%               & (el texto se despliega hacia arriba). \\
%  |\PosFecha| & Posición $(x,y)$ de las fechas. \\
%  |\PosFecha*| & Posición $(x,y)$ de las fechas \\
%               & (el texto se despliega hacia arriba). \\
%  |\PosFirmas| & Posición $(x_o,x_f,y)$ de las firmas. \\
%  |\Proveedor| & Nombre de quien firma el reporte. \\
%  |\ptref| & Muestra al precio total del ítem referenciado. \\
%  |\puref| & Muestra al precio unitario del ítem referenciado. \\
%  |\RazonSocial| & Establece la razón social. \\
%  |\Resumen| & Resumen. \\
%  |\RIF| & R.I.F. del cliente. \\
%  |\SepNotaFinal| & Separación interna entre el borde y el texto en la nota final. \\
%  |\SepDatosResumen| & Separación horizontal entre los datos del cliente y el resumen. \\
%  |\SepDescripcion| & Separación interna entre el borde y el texto en la descripción. \\
%  |\SepEncabezado| & Separación interna entre el borde y el texto en el encabezado. \\
%  |\SepNotaPrevia| & Separación interna entre el borde y el texto en la nota previa. \\
%  |\SepFilas| & Separación horizontal y vertical en los totales en el modo de filas. \\
%  |\SepItemsExtra| & Separación vertical extra entre los elementos en la factura. \\
%  |\SepVertical| & Separación vertical entre los distintos elementos de la factura. \\
%  |\subtref| & Muestra al subtotal del ítem referenciado. \\
%  |\Telefono| & Teléfono del cliente. \\
%  |\TituloReporte| & Establece el título del reporte. \\
%  |\TituloReporte*| & Establece el título del reporte a salir en las cabeceras o pies de página. \\
% \end{tabular}
%
%
% \ifejemplos
%  \section{Ejemplos incluídos\label{sec:ejemplos}}
%  \todo*{Hay que hacer que en la documentación, toda opción y macro referencie al ejemplo donde se usa.}
%
%  Desde la página siguiente se muestran todos los ejemplos incluídos en la clase:
%
%  \newpage
%  \newgeometry{marginparsep=0pt,vmargin={1cm,1cm},hmargin={1cm,1cm}}
%  {
%   \pagestyle{empty}
%   \setlength{\parindent}{0pt}
%   \hbadness=10000
%
%   \Ejemplo{ejemplo-cotizacion}
%   \Ejemplo{ejemplo-prefactura}
%   \newpage
%   \Ejemplo{ejemplo-factura01}
%   \Ejemplo{ejemplo-factura02}[\small\parskip=2pt]
%   \newpage
%   \Ejemplo{ejemplo-factura03}
%   \Ejemplo{ejemplo-factura04}
%   \newpage
%   \Ejemplo{ejemplo-factura05}
%   \Ejemplo{ejemplo-factura06}[\small]
%   \newpage
%   \Ejemplo[0.4]{ejemplo-factura07}[\small]
%   \Ejemplo{ejemplo-factura08}
%   \newpage
%   \Ejemplo*[0.49]{ejemplo-factura09}
%   \Ejemplo[0.49]{ejemplo-factura09}*{2}
%   \Ejemplo*{ejemplo-factura09}*
%   \newpage
%   \Ejemplo*[0.49]{ejemplo-factura10}
%   \Ejemplo[0.49]{ejemplo-factura10}*{2}
%   \Ejemplo*{ejemplo-factura10}*
%   \newpage
%   \Ejemplo*[0.49]{ejemplo-reporte1}
%   \Ejemplo[0.49]{ejemplo-reporte1}*{2}
%   \Ejemplo[0.49]{ejemplo-reporte1}{4}[\slshape Por espacio se omitió la página 2 en esta representación.]
%   \newpage
%   \Ejemplo*[0.49]{ejemplo-reporte2}
%   \Ejemplo[0.49]{ejemplo-reporte2}*{2}
%   \Ejemplo[0.49]{ejemplo-reporte2}{3}
%   \newpage
%   \Ejemplo*[0.49]{ejemplo-reporte3}
%   \Ejemplo[0.49]{ejemplo-reporte3}*{2}
%   \Ejemplo[0.49]{ejemplo-reporte3}{3}
%   \newpage
%   \Ejemplo{beispiel-rechnung}
%   \newpage
%  }
%  \hbadness=1000
%  \restoregeometry
% \else
%  \message{^^JLaTeX Warning: Compilacio'n ra'pida. No se incluyeron los ejemplos.^^J}
% \fi
%
% \StopEventually{\endorgo}
%
%    \begin{macrocode}
%<*class>
\RequirePackage{etoolbox}%  'etoolbox' para los condicionales binarios y para \AtEndPreamble.

%% Condicionales auxiliares que sólo pueden ser usados dentro de un grupo.
\newbool{DCU}
\newbool{DUC}
\newbool{Descuentos}
%% (quitado por no vigente)
 % \newbool{Electro}
\newbool{ExpAbajo}\booltrue{ExpAbajo}
\newbool{IVA}
\newbool{NumItem}
\newbool{PrimeroDescripcion}
\newbool{SinCantidad}
\newbool{SinLineaHoriNI}
\newbool{SinLineaHoriTit}
\newbool{SinLineaHoriTot}
\newbool{SinLineasHoriExt}
\newbool{SinLineasVertExt}
\newbool{SinLineasVertInt}
\newbool{Subtotal}

%% Condicionales auxiliares para la posición de los objetos fijos cuando son medidos desde la derecha, o desde abajo.
\newbool{NotaFinalYarriba}
\newbool{NotaFinalYnegativo}% no hay NotaFinalXnegativo.
\newbool{ControlXnegativo}
\newbool{ControlYarriba}
\newbool{ControlYnegativo}
\newbool{DenomXnegativo}
\newbool{DenomYarriba}
\newbool{DenomYnegativo}
\newbool{FechaXnegativo}
\newbool{FechaYarriba}
\newbool{FechaYnegativo}
\newbool{FirmasYnegativo}% no hay FirmasXnegativo; hay X_inicial y X_final para las firmas.

%% Condicionales relativos a opciones de la clase.
\newif\ifagrupado
\newif\ifcentradesc
\newif\ifcentrado
\newif\ifconlineasha
\newif\ifconreporte
\newif\ifcotizacion
\newif\ifcsv
\newif\ifdcu
\newif\ifdenom
\newif\ifdescuentos
\newif\ifdosfirmas
\newif\ifduc
%% (quitado por no vigente)
 % \newif\ifelectro
\newif\ifexpandecuadro
\newif\iffilas
\newif\ifG
\newif\ifimpdesc
\newif\ifiva
\newif\iflinea
\newif\ifnospanish
\newif\ifnumitem
\newif\ifprefactura
\newif\ifsincantidad
\newif\ifsinexpandir
\newif\ifsinfirmas
\newif\ifsiniva
\newif\ifsinivaexpresado
\newif\ifsinlineahni
\newif\ifsinlineahtit
\newif\ifsinlineahtot
\newif\ifsinlineash
\newif\ifsinlineashe
\newif\ifsinlineashi
\newif\ifsinlineasv
\newif\ifsinlineasve
\newif\ifsinlineasvi
\newif\ifsinmarcas
\newif\ifsinmonedatotales
\newif\ifsinnumero
\newif\ifsinreverso
\newif\ifsintotales
\newif\ifsoloreporte
\newif\ifsubtotal
\newif\iftodosiva
\newif\iftwoside
\newif\ifunafirma
\newif\ifvencimiento

%% Condicionales auxiliares.
\newif\if@cdu \@cdutrue
\newif\if@factura \@facturatrue
\newif\if@reverso \@reversotrue
\newif\if@reporte
\newif\if@FondoFactura
\newif\if@FondoReporte
\newif\if@FondoReporteStar
\newif\if@Nombre
\newif\if@PruebaRef
\newif\if@ExpandeCuadro
\newif\if@LineaDesdeInicio
\newif\if@LineaHastaTotal
\newif\if@LineaHastaFinal
\newif\if@LineaDescripcion
\newif\if@BIR
\newif\if@BIA
\newif\if@Exento
\newif\if@DescG
\newif\if@DescR
\newif\if@DescA
\newif\if@DescE
\newif\if@DescGnum
\newif\if@DescRnum
\newif\if@DescAnum
\newif\if@DescEnum
\newif\if@DescuentoTotal

%% Longitudes para uso interno.
\newlength{\@TamDatos}
\newlength{\@TamResumen}
\newlength{\@TamFinalEncabezado}
\newlength{\@TamLineaFirmas}
\newlength{\@TamNumItem}
\newlength{\@TamCantidad}
\newlength{\@TamDescripcion}
\newlength{\@TamPrecioU}
\newlength{\@TamPrecioS}
\newlength{\@TamDescuento}
\newlength{\@TamIVA}
\newlength{\@TamPrecioT}
\newlength{\@GrosorLineaFirmas}
\newlength{\@GrosorLineaEncabezado}
\newlength{\@GrosorLineaNotaPrevia}
\newlength{\@GrosorLineaDescripcion}
\newlength{\@GrosorLineaNotaFinal}
\newlength{\@GrosorLineaInfoPag}
\newlength{\@EsquinaLineaEncabezado}
\newlength{\@EsquinaLineaNotaPrevia}
\newlength{\@EsquinaLineaDescripcion}
\newlength{\@EsquinaLineaNotaFinal}
\newlength{\@SombraEncabezado}
\newlength{\@SombraNotaPrevia}
\newlength{\@SombraDescripcion}
\newlength{\@SombraNotaFinal}
\newlength{\@LineaNodoXo}
\newlength{\@LineaNodoYo}
\newlength{\@LineaNodoXf}
\newlength{\@LineaNodoYf}
\newlength{\@SepTotales}
\newlength{\@PosFechaX}
\newlength{\@PosFechaY}
\newlength{\@PosDenomX}
\newlength{\@PosDenomY}
\newlength{\@PosFirmasXuno}
\newlength{\@PosFirmasXdos}
\newlength{\@PosFirmasY}
\newlength{\@PosNotaFinalXuno}
\newlength{\@PosNotaFinalXdos}
\newlength{\@PosNotaFinalY}
\newlength{\@PosControlX}
\newlength{\@PosControlY}
\newlength{\@EspacioVacioInicial}
\newlength{\@EspacioVacioFinal}
\newlength{\@EspacioSignoMonetario}
\newlength{\@SeparaItemsExtra}
\newlength{\@SeparaItemsExtraFinal}
\newlength{\@SeparaInternaEncabezado}
\newlength{\@SeparaInternaNotaPrevia}
\newlength{\@SeparaInternaDescripcion}
\newlength{\@SeparaInternaNotaFinal}
\newlength{\@SeparaDatosResumen}
\newlength{\@SeparaVertical}
\newlength{\@SeparaMinTotales}
\newlength{\@SeparaFilas}
\newlength{\@SeparaFinLineaDesc}
\newlength{\@SeparaTelefonoEmail}
\newlength{\@InicioReporte}
\newlength{\@FirmaFacturaXcorr}
\newlength{\@FirmaFacturaYcorr}
\newlength{\@FirmaReporteXcorr}
\newlength{\@FirmaReporteYcorr}
\newlength{\@FirmaReporteTextPos}
\newlength{\@FirmaReporteFirmaPos}
\newlength{\@FirmaReporteTamLinea}
\newlength{\@FirmaReportePosX}
\newlength{\@FirmaReporteExcesoLinea}
\newlength{\@TempDimen}
\newlength{\tabcolsep@@}
\newlength{\baselineskip@@}

%% Cajas para uso interno.
\newbox{\@CajaDatos}
\newbox{\@CajaResumen}
\newbox{\@CajaFirmaEmisor}
\newbox{\@CajaFirmas}
\newbox{\@CajaFecha}
\newbox{\@CajaDenom}
\newbox{\@CajaControl}
\newbox{\@CajaNotaFinal}
\newbox{\@CajaEspacioVacio}
\newbox{\@CajaPrueba}
\newbox{\@CajaCaracterCR}

%% Contadores.
\newcounter{@IVANoGeneral}%  '0' con sólo IVA G, '1' cuando hay IVA R o IVA A, y '2' cuando hay ambos.
\newcounter{@Firmas}%  '0' para sinfirmas, '1' para unafirma, y '2' para dosfirmas.
\setcounter{@Firmas}{2}
\newcounter{NumItem}%  Número del ítem actual, para uso de algunas macros en el cuadro de la factura y para cuando se usa la opción 'numitem'.
\newcount\AuxFecha

%% Errores, advertencias e informaciones de LaTeX.
\newcommand{\CLASSERROR}[1]{\ClassError{factura}{** #1. **}{}}
\newcommand{\CLASSWARNING}[1]{\ClassWarning{factura}{** #1. **}{}}
\newcommand{\CLASSINFO}[1]{\ClassInfo{factura}{** #1. **}{}}

%% Opciones de la clase.
\DeclareOption{agrupado}{\agrupadotrue}%  Agrupados los elementos totalizadores en el modo en columnas.
\DeclareOption{centradesc}{\centradesctrue}%  Para centrar la descripción de cada ítem.
\DeclareOption{centrado}{\centradotrue}%  Total y exento 'centrado'; sólo para estilo 'filas'
\DeclareOption{conlineasha}{\conlineashatrue}%  Una línea horizontal para cada ítem, como las facturas a mano.
\DeclareOption{conreporte}{\conreportetrue}%  Para imprimir factura y reporte.
\DeclareOption{cotizacion}{\cotizaciontrue}%  Cambia la denominación a 'COTIZACIÓN' y no 'FACTURA'.
\DeclareOption{csv}{\csvtrue}%  Para la salida en modo csv también.
\DeclareOption{dcu}{\dcutrue}%  Descripción, luego cantidad y luego precio unitario.
\DeclareOption{denom}{\denomtrue}%  Agrega el texto con la denominación del documento, para las formas libres que no tienen el número de factura.
\DeclareOption{descuentos}{\descuentostrue}%  Para los descuentos individuales.
\DeclareOption{dosfirmas}{\dosfirmastrue}%  Pone las dos firmas para firmar.
\DeclareOption{duc}{\ductrue}%  Descripción, luego precio unitario y luego cantidad.
\DeclareOption{electro}{\CLASSERROR{Ya no esta' vigente el descuento en la ali'cuota general por pagos electro'nicos}}%  Parche por el Decreto 3.085 de Nikolái, ya no vigente.
\DeclareOption{expandecuadro}{\expandecuadrotrue}%  Para rellenar con líneas en blanco de manera automática.
\DeclareOption{filas}{\filastrue}%  Estilo de totalización en filas.
\DeclareOption{G}{\Gtrue}%  Con esta opción sale la letra 'G' en los items con IVA general, y sale el porcentaje en la base imponible general.
\DeclareOption{impdesc}{\impdesctrue}%  Para el tipo de impuesto en la descripción y no en el precio unitario y total.
\DeclareOption{iva}{\ivatrue}%  Para la columna del IVA.
\DeclareOption{linea}{\lineatrue}%  Para la línea que inutiliza el espacio en blanco.
\DeclareOption{nospanish}{\nospanishtrue}%  No carga babel en castellano ni la codificación T1 de las letras.
\DeclareOption{numitem}{\numitemtrue}%  Para poner una columna extra al principio con el número del ítem.
\DeclareOption{prefactura}{\prefacturatrue}%  La prefactura; con el diseño de la forma libre... para enviar por correo electrónico, fax o imprimir preliminarmente.
\DeclareOption{sincantidad}{\sincantidadtrue}%  Para no poner el campo de la cantidad.
\DeclareOption{sinexpandir}{\sinexpandirtrue}%  No expande ninguna de las dos partes de la factura.
\DeclareOption{sinfirmas}{\sinfirmastrue}%  No agrega los espacios para firmar.
\DeclareOption{sinivaexpresado}{\sinivaexpresadotrue}%  Cuando no se cobra IVA y tampoco se quiere que aparezca la totalización del IVA.
\DeclareOption{siniva}{\sinivatrue}%  No cobra IVA.
\DeclareOption{sinlineahni}{\sinlineahnitrue}%  Estilo 'sinlineahni'.
\DeclareOption{sinlineahtot}{\sinlineahtottrue}%  Estilo 'sinlineahtot'.
\DeclareOption{sinlineahtit}{\sinlineahtittrue}%  Estilo 'sinlineahtit'.
\DeclareOption{sinlineashe}{\sinlineashetrue}%  Estilo 'sinlineashe'.
\DeclareOption{sinlineashi}{\sinlineashitrue}%  Estilo 'sinlineashi'.
\DeclareOption{sinlineash}{\sinlineashtrue}%  Estilo 'sinlineash'.
\DeclareOption{sinlineasve}{\sinlineasvetrue}%  Estilo 'sinlineasve'.
\DeclareOption{sinlineasvi}{\sinlineasvitrue}%  Estilo 'sinlineasvi'.
\DeclareOption{sinlineasv}{\sinlineasvtrue}%  Estilo 'sinlineasv'.
\DeclareOption{sinmarcas}{\sinmarcastrue}%  Para evitar que aparezcan los caracteres especiales cuando la descripción es larga.
\DeclareOption{sinmonedatotales}{\sinmonedatotalestrue}%  Para que no ponga el signo monetario en los totales.
\DeclareOption{sinnumero}{\sinnumerotrue}%  Escribir el texto de la denominación del documento pero sin escribir el número.
\DeclareOption{sinreverso}{\sinreversotrue}%  Para no imprimir el reverso de la factura.
\DeclareOption{sintotales}{\sintotalestrue}%  No escribe los totales al final, por ejemplo para cotizaciones de productos similares.
\DeclareOption{soloreporte}{\soloreportetrue}%  Para imprimir sólo el reporte.
\DeclareOption{subtotal}{\subtotaltrue}%  Para la columna extra del subtotal antes del descuento.
\DeclareOption{todosiva}{\todosivatrue}%  Con esta opción salen todos los IVA así valgan cero.
\DeclareOption{twoside}{\twosidetrue}%  Opción 'twoside' de article para los reportes.
\DeclareOption{unafirma}{\unafirmatrue}%  No agrega el espacio para que firme quien recibe.
\DeclareOption{vencimiento}{\vencimientotrue}%  Escribe fecha de vencimiento, aún con facturas sin crédito.

\ProcessOptions\relax

\LoadClass[12pt]{article}

%% Cargamos algunos paquetes siempre necesarios.
\RequirePackage{xparse}%  Algunos comandos los definimos con \NewDocumentCommand.
\RequirePackage[absolute]{textpos}%  Es necesario 'textpos' para colocar texto en ubicaciones arbitrarias.
\RequirePackage{eso-pic}%  Para el comando \AddToShipoutPicture.
\RequirePackage{xstring}%  Lo usamos sólo para tomar el menos al principio de una longitud.
\RequirePackage{calc}%  'calc' para hacer aritmética con las dimensiones.
\RequirePackage{fp-upn}%  Fundamental el paquete 'fp-upn' para todos los cálculos.
\RequirePackage{numprint}%  Con 'numprint' expresamos los totales en el formato requerido.
\RequirePackage{tikz}%  Para los recuadros del encabezado de la factura,
 \usetikzlibrary{shadows}%  y para el de la descripción.
\RequirePackage{tabularx}%  Con 'tabularx' hacemos la tabla.
\RequirePackage{environ}%  Con 'environ' podemos usar tabularx dentro del ambiente de la factura y escribir texto en el enddef.
\RequirePackage{fancyhdr}%  Las cabeceras fancy; así se use solamente en reporte, hay que cargarlo siempre.
\RequirePackage[calc]{datetime2}%  Para calcular la fecha del vencimiento del crédito.
\RequirePackage{anyfontsize}%  Para las tipografías en tamaños arbitrarios.
\RequirePackage{ifxetex}%  Para poder modificar correctamente la metadata del PDF y agregar la versión de la clase.
\RequirePackage{ifluatex}%  Ditto.
\RequirePackage{zref-savepos}[2016/05/21]%  Para guardar la posición en 'Y' expandiendo el cuadro, para la línea y para la firma en el reporte si la hubiera.
%% Hay que cargar siempre zref-savepos para poder compilar con LuaLaTeX porque este motor no tiene \pdfsavepos, \pdflastypos y zref-savepos las provee cuando no existen (motor LuaLaTex).

%% Gracias a Alan Munn y a wipet (Thanks to Alan Munn and wipet).
%% (https://tex.stackexchange.com/questions/396363/input-a-date-to-datetime2-in-a-format-different-than-iso)
\def\DTMsavedatex#1#2{%
 \expandafter \ifx \csname f=\DTMinformat\endcsname \relax
  \CLASSERROR{\string\DTMsavedatex: formato "\DTMinformat" desconocido}%
 \else
  \csname f=\DTMinformat\expandafter\endcsname#2&{#1}%
 \fi
}
\def\DTMfdef#1{\expandafter\def\csname f=#1\endcsname}
\def\DTMs#1#2{\DTMsavedate{#2}{#1}}
\DTMfdef{yyyy-mm-dd}#1-#2-#3&{\DTMs{#1-#2-#3}}
\DTMfdef{yyyy.mm.dd}#1.#2.#3&{\DTMs{#1-#2-#3}}
\DTMfdef{yyyymmdd}#1#2#3#4#5#6#7#8&{\DTMs{#1#2#3#4-#5#6-#7#8}}
\DTMfdef{dd-mm-yyyy}#1-#2-#3&{\DTMs{#3-#2-#1}}
\DTMfdef{dd.mm.yyyy}#1.#2.#3&{\DTMs{#3-#2-#1}}
\DTMfdef{ddmmyyyy}#1#2#3#4#5#6#7#8&{\DTMs{#5#6#7#8-#3#4-#1#2}}
\DTMfdef{mm-dd-yyyy}#1-#2-#3&{\DTMs{#3-#1-#2}}

%% Macros internas para el texto de la metadata que se va a escribir con el número de la versión de la clase.
\edef\@factura@ver{\expandafter\csname ver@factura.cls\endcsname}
\ExplSyntaxOn
  \seq_set_split:NnV \l_tmpa_seq {~} {\@factura@ver}
  \tl_set:Nx \@factura@vernum {\seq_item:Nn \l_tmpa_seq {2}}

  \seq_set_split:NnV \l_tmpa_seq {~} {\@factura@ver}
  \tl_set:Nx \@factura@fecha {\seq_item:Nn \l_tmpa_seq {1}}
\ExplSyntaxOff
\def\@factura@ver{Hecho con la clase factura-\@factura@vernum\ (\@factura@fecha)}

%% Escribimos la versión de la clase en la metadata del PDF.
\ifxetex
 \special{pdf: docinfo << /Subject (\@factura@ver) >>}
\else
 \ifluatex
  \pdfextension info{/Subject (\@factura@ver)}
 \else
  \pdfinfo{/Subject (\@factura@ver)}
 \fi
\fi

\def\@EstableceOpciones{%
 % Establecemos los parámetros y macros relativas a la fecha.
 \DTMsetstyle{\@FormatoFecha}\DTMsetup{datesep=\@FechaSeparadores}\gdef\DTMinformat{\@FormatoFechaEntrada}
 
 % Tomamos las acciones pertinentes dependiendo de las opciones de la clase (o condicionales activados en el preámbulo que cambian las opciones de la clase)
 \ifprefactura
  \ifcotizacion
   \prefacturafalse
   \cotizacionfalse
   \CLASSWARNING{Las opciones 'prefactura' y 'cotizacion' no son compatibles. Sin efecto ambas opciones}
  \else
   \denomtrue\setcounter{@Firmas}{0}
  \fi
 \fi
 \ifcotizacion\denomtrue\setcounter{@Firmas}{1}\fi

 \ifsinlineasv\booltrue{SinLineasVertExt}\booltrue{SinLineasVertInt}\fi
 \ifsinlineasvi\booltrue{SinLineasVertInt}\fi
 \ifsinlineasve\booltrue{SinLineasVertExt}\fi
 \ifsinlineash\booltrue{SinLineasHoriExt}\booltrue{SinLineaHoriNI}\booltrue{SinLineaHoriTit}\booltrue{SinLineaHoriTot}\fi
 \ifsinlineashi\booltrue{SinLineaHoriNI}\booltrue{SinLineaHoriTit}\booltrue{SinLineaHoriTot}\fi
 \ifsinlineahni\booltrue{SinLineaHoriNI}\fi
 \ifsinlineahtit\booltrue{SinLineaHoriTit}\fi
 \ifsinlineahtot
  \iffilas
   \CLASSWARNING{Las opciones 'sinlineahtot' y 'filas' no son compatibles. Sin efecto opcio'n 'sinlineahtot'}
   \sinlineahtotfalse
  \else
   \booltrue{SinLineaHoriTot}
  \fi
 \fi
 \ifsinlineashe\booltrue{SinLineasHoriExt}\fi

 \ifdosfirmas
  \setcounter{@Firmas}{2}
  \ifunafirma
   \unafirmafalse
   \ifsinfirmas
    \sinfirmasfalse
    \CLASSWARNING{Las opciones para las firmas son excluyentes. Permanece activada 'dosfirmas'}
   \else
    \CLASSWARNING{Las opciones para las firmas son excluyentes. Permanece activada 'dosfirmas'}
   \fi
  \else
   \ifsinfirmas
    \sinfirmasfalse
    \CLASSWARNING{Las opciones para las firmas son excluyentes. Permanece activada 'dosfirmas'}
   \fi
  \fi
 \fi
 \ifunafirma
  \setcounter{@Firmas}{1}
  \ifsinfirmas
   \sinfirmasfalse
   \CLASSWARNING{Las opciones para las firmas son excluyentes. Permanece activada 'unafirma'}
  \fi
 \fi
 \ifsinfirmas\setcounter{@Firmas}{0}\fi 
 
 \ifdcu
  \@cdufalse\ducfalse
  \booltrue{DCU}\boolfalse{DUC}
  \booltrue{PrimeroDescripcion}
 \fi
 \ifduc
  \ifdcu
   \boolfalse{PrimeroDescripcion}\boolfalse{DCU}\dcufalse\ducfalse\@cdutrue
   \CLASSWARNING{Las opciones 'duc' y 'dcu' no son compatibles. Sin efecto ambas opciones}
  \else
   \ductrue\@cdufalse\dcufalse
   \booltrue{DUC}\boolfalse{DCU}
   \booltrue{PrimeroDescripcion}
  \fi
 \fi
 \ifnumitem\booltrue{NumItem}\fi
 \ifsincantidad
  \booltrue{SinCantidad}
  \ifdcu
   \dcufalse\@cdutrue\boolfalse{DCU}\boolfalse{PrimeroDescripcion}
   \CLASSWARNING{La opcio'n 'sincantidad' no es compatible con 'dcu'. Sin efecto opcio'n 'dcu'}
  \fi
  \ifduc
   \ducfalse\@cdutrue\boolfalse{DUC}\boolfalse{PrimeroDescripcion}
   \CLASSWARNING{La opcio'n 'sincantidad' no es compatible con 'duc'. Sin efecto opcio'n 'duc'}
  \fi
 \fi
 \ifdescuentos\booltrue{Descuentos}\fi
 \ifiva\impdesctrue\booltrue{IVA}\fi
 \ifsubtotal
  \ifdescuentos
   \booltrue{Subtotal}
  \else
   \CLASSWARNING{La opcio'n 'subtotal' no aplica sin descuentos individuales. Sin efecto opcio'n 'subtotal'}
  \fi
 \fi

 \ifsinnumero\denomtrue\fi
 \ifsintotales
  \iffilas
   \filasfalse
   \CLASSWARNING{La opcio'n 'sintotales' no aplica para el modo de totalizacio'n en filas. Sin efecto opcio'n 'filas'}
  \fi
  \ifcotizacion
  \else
   \CLASSWARNING{La opcio'n 'sintotales' no debe usarse en facturas sino en cotizaciones. No es va'lida una factura sin totales}
  \fi
 \fi

 \ifcentrado
  \iffilas
  \else
   \centradofalse
   \CLASSWARNING{La opcio'n 'centrado' no es compatible con el modo de totalizacio'n en columnas. Sin efecto opcio'n 'centrado'}
  \fi
 \fi
 \ifagrupado
  \iffilas
   \agrupadofalse
   \CLASSWARNING{La opcio'n 'agrupado' no es compatible con el modo de totalizacio'n en filas. Sin efecto opcio'n 'agrupado'}
  \fi
 \fi

%% (quitado por no vigente)
 % \ifelectro
 %  \boolfalse{Electro}
 %  \electrofalse
 %  \gdef\@NotaInterna{\@TextoDescuentoPresidencial}
 %  \CLASSWARNING{Se establecio' el texto aclaratorio del descuento presidencial como 'nota interna'}
 %  \iffilas
 %   \filasfalse
 %   \CLASSWARNING{La opcio'n 'electro' no es compatible con la opcio'n 'filas'. Sin efecto la opcio'n 'filas'}
 %  \fi
 % \fi

 \iftodosiva \@BIRtrue \@BIAtrue \@Exentotrue \Gtrue \fi
 \ifsiniva\AtBeginDocument{\gdef\@ImpuestoG{0}\gdef\@ImpuestoR{0}\gdef\@ImpuestoA{0}}\fi
 \ifsinivaexpresado
  \iffilas
   \filasfalse
   \CLASSWARNING{La opcio'n 'sinivaexpresado' no es compatible con la opcio'n 'filas'. Sin efecto la opcio'n 'filas'}
  \else
   \AtBeginDocument{\gdef\@ImpuestoG{0}\gdef\@ImpuestoR{0}\gdef\@ImpuestoA{0}}
  \fi
 \fi

 \iflinea
  \ifexpandecuadro
  \else
   \CLASSWARNING{La opcio'n 'linea' requiere la opcio'n 'expandecuadro'. Activando la opcio'n 'expandecuadro'}
   \expandecuadrotrue
  \fi
 \fi
 \ifsinmarcas \def\@CaracterCR{\mbox{}} \def\@CaracterCeldasVacias{\mbox{}} \fi
 \ifnospanish
 \else
  \RequirePackage[spanish]{babel}%  Babel en castellano.
  \ifxetex
  \else
   \ifluatex
   \else
    \RequirePackage[T1]{fontenc}%  Codificación T1 para las tipografías; no aplica para XeLaTeX ni LuaLaTeX.
   \fi
  \fi
 \fi

 \ifexpandecuadro
  \ifsinexpandir
   \expandecuadrofalse
   \sinexpandirfalse
   \CLASSWARNING{Las opciones 'expandecuadro' y 'sinexpandir' no son compatibles. Sin efecto ambas opciones}
  \else
   \boolfalse{ExpAbajo}
  \fi
 \fi
 \ifsinexpandir\boolfalse{ExpAbajo}\expandecuadrofalse\fi

 \ifsinreverso\@reversofalse\fi
 \ifconreporte
  \@reportetrue
  \ifsoloreporte
   \soloreportefalse
   \CLASSWARNING{La opcio'n 'soloreporte' no es compatible con 'conreporte'. Sin efecto 'soloreporte'}
  \fi
 \fi
 \ifsoloreporte\@reportetrue\@facturafalse\@reversofalse\fi

 \ifcsv
  \if@factura
   \newwrite\csv \immediate\openout\csv=\jobname.csv
   \ifsintotales\else\AtEndDocument{\@EscribeTotalesEnCSV}\fi
  \else
   \csvfalse
   \CLASSWARNING{La opcio'n 'csv' require generar la factura. Sin efecto opcio'n 'csv'}
  \fi
 \fi
 
 \iftwoside
  \@twosidetrue\@mparswitchtrue
 \else
  \@twosidefalse\@mparswitchfalse
 \fi}
 
\NewDocumentCommand{\BeforeEndPreamble}{s s +m}{
 \IfBooleanTF{#2}
  {\gpreto{\@BeforeEndPreamble}{#3}}
  {\IfBooleanTF{#1}{\long\gdef\@BeforeEndPreamble{#3}}{\gappto{\@BeforeEndPreamble}{#3}}}}

\AtEndPreamble{
 \@EstableceOpciones
 \ifx\@undefined\@BeforeEndPreamble\else\@BeforeEndPreamble\@EstableceOpciones\fi

 % Escribimos en el .log las opciones cargadas:
 \ifagrupado \CLASSINFO{Opcio'n cargada: agrupado}\fi
 \ifcentradesc \CLASSINFO{Opcio'n cargada: centradesc}\fi
 \ifcentrado \CLASSINFO{Opcio'n cargada: centrado}\fi
 \ifconlineasha \CLASSINFO{Opcio'n cargada: conlineasha}\fi
 \ifconreporte \CLASSINFO{Opcio'n cargada: conreporte}\fi
 \ifcotizacion \CLASSINFO{Opcio'n cargada: cotizacion}\fi
 \ifcsv \CLASSINFO{Opcio'n cargada: csv}\fi
 \ifdcu \CLASSINFO{Opcio'n cargada: dcu}\fi
 \ifdenom \CLASSINFO{Opcio'n cargada: denom}\fi
 \ifdescuentos \CLASSINFO{Opcio'n cargada: descuentos}\fi
 \ifdosfirmas \CLASSINFO{Opcio'n cargada: dosfirmas}\fi
 \ifduc \CLASSINFO{Opcio'n cargada: duc}\fi
 %% (quitado por no vigente)
 % \ifelectro \CLASSINFO{Opcio'n cargada: electro}\fi
 \ifexpandecuadro \CLASSINFO{Opcio'n cargada: expandecuadro}\fi
 \iffilas \CLASSINFO{Opcio'n cargada: filas}\fi
 \ifG \CLASSINFO{Opcio'n cargada: G}\fi
 \ifimpdesc \CLASSINFO{Opcio'n cargada: impdesc}\fi
 \ifiva \CLASSINFO{Opcio'n cargada: iva}\fi
 \iflinea \CLASSINFO{Opcio'n cargada: linea}\fi
 \ifnospanish \CLASSINFO{Opcio'n cargada: nospanish}\fi
 \ifnumitem \CLASSINFO{Opcio'n cargada: numitem}\fi
 \ifprefactura \CLASSINFO{Opcio'n cargada: prefactura}\fi
 \ifsincantidad \CLASSINFO{Opcio'n cargada: sincantidad}\fi
 \ifsinexpandir \CLASSINFO{Opcio'n cargada: sinexpandir}\fi
 \ifsinfirmas \CLASSINFO{Opcio'n cargada: sinfirmas}\fi
 \ifsiniva \CLASSINFO{Opcio'n cargada: siniva}\fi
 \ifsinivaexpresado \CLASSINFO{Opcio'n cargada: sinivaexpresado}\fi
 \ifsinlineahni \CLASSINFO{Opcio'n cargada: sinlineahni}\fi
 \ifsinlineahtit \CLASSINFO{Opcio'n cargada: sinlineahtit}\fi
 \ifsinlineahtot \CLASSINFO{Opcio'n cargada: sinlineahtot}\fi
 \ifsinlineash \CLASSINFO{Opcio'n cargada: sinlineash}\fi
 \ifsinlineashe \CLASSINFO{Opcio'n cargada: sinlineashe}\fi
 \ifsinlineashi \CLASSINFO{Opcio'n cargada: sinlineashi}\fi
 \ifsinlineasv \CLASSINFO{Opcio'n cargada: sinlineasv}\fi
 \ifsinlineasve \CLASSINFO{Opcio'n cargada: sinlineasve}\fi
 \ifsinlineasvi \CLASSINFO{Opcio'n cargada: sinlineasvi}\fi
 \ifsinmarcas \CLASSINFO{Opcio'n cargada: sinmarcas}\fi
 \ifsinmonedatotales \CLASSINFO{Opcio'n cargada: sinmonedatotales}\fi
 \ifsinnumero \CLASSINFO{Opcio'n cargada: sinnumero}\fi
 \ifsinreverso \CLASSINFO{Opcio'n cargada: sinreverso}\fi
 \ifsintotales \CLASSINFO{Opcio'n cargada: sintotales}\fi
 \ifsoloreporte \CLASSINFO{Opcio'n cargada: soloreporte}\fi
 \ifsubtotal \CLASSINFO{Opcio'n cargada: subtotal}\fi
 \iftodosiva \CLASSINFO{Opcio'n cargada: todosiva}\fi
 \iftwoside \CLASSINFO{Opcio'n cargada: twoside}\fi
 \ifunafirma \CLASSINFO{Opcio'n cargada: unafirma}\fi
 \ifvencimiento \CLASSINFO{Opcio'n cargada: vencimiento}\fi

 % Cargamos aquí 'geometry' para que sea después de haber establecido 'twoside'.
 \RequirePackage{geometry}}

%% Escribimos el total como comentario en el .log.
\AtEndDocument{%
 \ifsintotales
 \else
  \if@factura\CLASSINFO{\@@Total: \space\space\space \Moneda \Total}\fi
 \fi}

%</class>
%<*class|def>
%% La geometría para la factura, su reverso, y el reporte.
%% Para la factura, el margen superior está definido desde el encabezado con los datos del cliente;
%%  el margen inferior está definido hasta el final de la descripción (o del cuadro de la factura si no hay descripción).
%%  Hay elementos que salen fuera de estos márgenes, como las firmas, la nota final, la numeración y la fecha.
%<class>\def\GeometriaFactura#1{\gdef\@GeometriaFactura{#1}}
%<class>\def\GeometriaReverso#1{\gdef\@GeometriaReverso{#1}}
%<class>\def\GeometriaReporte#1{\gdef\@GeometriaReporte{#1}}
%<def>\GeometriaFactura{letterpaper,vmargin={5.2cm,5.5cm},hmargin={1cm,1cm},footskip=0pt,  headsep=0pt,    headheight=0pt}
%<def>\GeometriaReverso{letterpaper,vmargin={1cm,1cm},    hmargin={1cm,1cm},footskip=0cm,  headsep=0pt,    headheight=0pt}
%<def>\GeometriaReporte{letterpaper,vmargin={2.8cm,2.5cm},hmargin={3cm,2cm},footskip=1.2cm,headsep=0.777cm,headheight=1.5cm}

%<class>%% Se utiliza '\@Impuesto' como el factor del impuesto, el cual viene predefinido en '\@Alicuota'.
%<class>\def\@ImpuestoG{\@AlicuotaG}
%<class>\def\@ImpuestoR{\@AlicuotaR}
%<class>\def\@ImpuestoA{\@AlicuotaA}
%<def>%% Factores para los distintos IVA:
%<def>\def\@AlicuotaG{0.16}
%<def>%% (quitado por no vigente)
%<def> % \def\@AlicuotaGDescBajo{0.03}
%<def> % \def\@AlicuotaGDescAlto{0.05}
%<def> % \def\@AlicuotaGDescLimite{2000000}
%<def>\def\@AlicuotaR{0.08}
%<def>\def\@AlicuotaA{0.27}
%<def>
%</class|def>
%<*def>
%% i) separador de miles, ii) separador decimal, iii) número de decimales para la moneda iv) y para los porcentajes, v) número de dígitos de la numeración de la factura, vi) formato de la fecha, vii) separadores para la fecha.
\npthousandsep{.}
\npdecimalsign{,}
\def\@DigitosMoneda{2}
\def\@DigitosPorc{2}
\def\@DigitosNumeracion{5}
\def\@FormatoFecha{ddmmyyyy}
\def\@FormatoFechaEntrada{dd-mm-yyyy}
\def\@FechaSeparadores{-}

%% La moneda:
\def\@Moneda{Bs.}
%% (quitado por no vigente)
 % \def\@MonedaBs{Bs.}
 % \def\@MonedaBsS{Bs.S}
 % \def\@MonedaBsF{Bs.F}
\setlength\@EspacioSignoMonetario{.666ex}

%% (quitado por no vigente)
%% Las fechas de la entrada de la reconversión monetaria:
 % \DTMsavedate{FechaBsS}{2018-08-20}
 % \DTMsavedate{FechaBs}{2019-02-15}

%% Al lado de cada ítem va la letra que designa el gravamen que aplica sobre él:
\def\@DenomAlicuotaG{G}
\def\@DenomAlicuotaR{R}
\def\@DenomAlicuotaA{A}
\def\@DenomAlicuotaE{(E)}

%% (quitado por no vigente)
%% Texto predeterminado para la nota interna por el descuento presidencial.
 % \def\@TextoDescuentoPresidencial{Se aplica un descuento del \@AlicuotaGDescPorc\% al IVA General seg\'un Decreto Presidencial 3.085 en Gaceta Oficial 41.239.}

%% Macros que guardan textos que corresponden a nombres de campos:
\def\@@Fecha{Fecha de emisi\'on:}
\def\@@FechaVencimiento{Fecha de vencimiento:}
\def\@@Control{N\textordmasculine\space CONTROL}
\def\@@Datos{Datos del cliente}
\def\@@Resumen{Resumen del servicio}
\def\@@DenomFactura{FACTURA}
\def\@@DenomCotiza{COTIZACI\'ON}
\def\@@RazonSocial{Raz\'on social:}
\def\@@Nombre{Nombre:}
\def\@@RIF{R.I.F.:}
\def\@@CI{C.I.:}
\def\@@Direccion{Direcci\'on fiscal:}
\def\@@Telefono{Tel\'efono:}
\def\@@Email{E-mail:}
\def\@@Contacto{Persona de contacto:}
\def\@@Proveedor{Proveedor:}
\def\@@Cliente{Cliente:}
\def\@@Emisor{Por el emisor}
\def\@@Conforme{Recib\'i conforme}
\def\@@Pagado{PAGADO}
\def\@@NumItem{\#}
\def\@@Cantidad{Cantidad}
\def\@@Descripcion{Descripci\'on}
\def\@@Precio{Precio}
\def\@@PrecioU{Precio U.}
\def\@@PrecioS{Subtotal}
\def\@@Descuento{Dcto.}
\def\@@IVA{IVA}
\def\@@PrecioT{Precio T.}
\def\@@AlicuotaG{IVA G}
%% (quitado por no vigente)
 % \def\@@AlicuotaGDesc{Descuento IVA G}
\def\@@AlicuotaR{IVA R}
\def\@@AlicuotaA{IVA A}
\def\@@BIG{BI G}
\def\@@BIR{BI R}
\def\@@BIA{BI A}
\def\@@BIE{Exento}% La base imponible exenta. ;)
\def\@@Total{TOTAL}
\def\@@DescuentoG{Descuento G}
\def\@@DescuentoR{Descuento R}
\def\@@DescuentoA{Descuento A}
\def\@@DescuentoE{Descuento E}
\def\@@SubtotalG{Subtotal G}
\def\@@SubtotalR{Subtotal R}
\def\@@SubtotalA{Subtotal A}
\def\@@SubtotalE{Subtotal E}

%% i) carácter para separar el teléfono del e-mail del cliente,
%% ii) carácter para los campos vacíos cuando se usa \ItemX,
%% iii) carácter para las celdas vacías cuando la descripción del ítem ocupa más que una línea.
%% iv) carácter de retorno de carro en el cuadro de la factura,
\def\@CaracterSeparaTelefonoEmail{$\diamond$}
\def\@CaracterItemXVacio{...}
\def\@CaracterCeldasVacias{\mbox{}\hfill$\cdot$\hfill\mbox{}}
\def\@CaracterCR{\flechaCR}
% \def\@CaracterCR{$\hookleftarrow$}
% \def\@CaracterCR{$\cwrightarcarrow$}% Usando el paquete 'fnsymbol'.
% \def\@CaracterCR{$\cwsearcarrow$}% Usando el paquete 'fnsymbol'.
\def\flechaCR{%
 \raisebox{-0.4ex}{%
  \begin{tikzpicture}
   \draw[line width=0.075ex] (0,0) arc (20:70:0.75ex) + (0,0) arc (-70:-20:0.75ex) ++ (0.8ex,0.55ex) arc (0:-100:1ex);
  \end{tikzpicture}}}

%% Separación entre el inicio de una columna en la tabla y el inicio del texto.
\tabcolsep=0.15cm

%% Separación entre la última letra y el carácter que indica que se continúa en la línea siguiente en la descripción cuando ocupa más de una línea.
\setlength{\@SeparaFinLineaDesc}{0.15cm}

%% Separación entre el teléfono y el carácter \@CaracterSeparaTelefonoEmail, y luego desde ese carácter hasta el email,
%% para cuando no hay resumen.
\setlength{\@SeparaTelefonoEmail}{0.666em}

%% Grosor de la línea de 'fancyhdr' para las cabeceras, en aquellas donde haya línea.
\setlength{\@GrosorLineaInfoPag}{0.4pt}

%% Posición en X desde la derecha y excedente en la línea de la firma.
\setlength{\@FirmaReportePosX}{2cm}
\setlength{\@FirmaReporteExcesoLinea}{1.5cm}
%</def>
%<class|def>
%<*class>
%% Fijamos en cero los descuentos totales (necesario).
\def\@DescuentoGporc{0}
\def\@DescuentoRporc{0}
\def\@DescuentoAporc{0}
\def\@DescuentoEporc{0}
\def\@DescuentoGnum{0}
\def\@DescuentoRnum{0}
\def\@DescuentoAnum{0}
\def\@DescuentoEnum{0}
\FPset\PrecioUtempx{1}

%% \Fecha, \FormatoFecha, \Credito, \Moneda, \RazonSocial, \Nombre, \RIF, \CI, \Contacto, \Dirección, \Telefono, \Email, \Conforme, \Emisor, \Proveedor.
\NewDocumentCommand{\Fecha}{s m g g}{%
 \IfBooleanTF{#1}
  {\AtEndPreamble{\DTMsavedate{FechaEmision}{#2-#3-#4}}}
  {\AtEndPreamble{\DTMsavedatex{FechaEmision}{#2}}}}
\def\@Fecha{\DTMusedate{FechaEmision}}

\NewDocumentCommand{\FormatoFecha}{o g g}{%
 \IfValueT{#1}{\gdef\@FormatoFechaEntrada{#1}}\IfValueT{#2}{\gdef\@FormatoFecha{#2}}\IfValueT{#3}{\gdef\@FechaSeparadores{#3}}}

\def\Credito#1{\gdef\@Credito{#1}\vencimientotrue}
\def\Moneda#1{\gdef\@Moneda{#1}}
\def\RazonSocial#1{\gdef\@RazonSocial{#1}}
\def\Nombre#1{\gdef\@RazonSocial{#1}\@Nombretrue}
\def\RIF#1{\gdef\@RIF{#1}}
\def\CI#1{\gdef\@RIF{#1}\let\@@RIF\@@CI\let\@CI\@RIF}
\def\Contacto#1{\gdef\@Contacto{#1}}
\def\Direccion#1{\gdef\@Direccion{#1}}
\def\Telefono#1{\gdef\@Telefono{#1}}
\def\Email#1{\gdef\@Email{#1}}
\def\Conforme#1{\gdef\@Conforme{#1}}
\def\Emisor#1{\gdef\@Emisor{#1}}
\def\Proveedor#1{\gdef\@Proveedor{#1}}

%% Los descuentos totales.
\NewDocumentCommand{\Descuento}{s m}{%
 \IfBooleanTF{#1}
  {\@DescGnumtrue\@DescRnumtrue\@DescAnumtrue\@DescEnumtrue
  \gdef\@DescuentoGnum{#2}\gdef\@DescuentoRnum{#2}\gdef\@DescuentoAnum{#2}\gdef\@DescuentoEnum{#2}}
  {\@DescGnumfalse\@DescRnumfalse\@DescAnumfalse\@DescEnumfalse
  \gdef\@DescuentoGporc{#2}\gdef\@DescuentoRporc{#2}\gdef\@DescuentoAporc{#2}\gdef\@DescuentoEporc{#2}}%
 \@DescGtrue\@DescRtrue\@DescAtrue\@DescEtrue\@DescuentoTotaltrue}

\NewDocumentCommand{\DescuentoG}{s m}{%
 \IfBooleanTF{#1}{\@DescGnumtrue\gdef\@DescuentoGnum{#2}}{\@DescGnumfalse\gdef\@DescuentoGporc{#2}}%
 \@DescGtrue\@DescuentoTotaltrue}

\NewDocumentCommand{\DescuentoR}{s m}{%
 \IfBooleanTF{#1}{\@DescRnumtrue\gdef\@DescuentoRnum{#2}}{\@DescRnumfalse\gdef\@DescuentoRporc{#2}}%
 \@DescRtrue\@DescuentoTotaltrue}

\NewDocumentCommand{\DescuentoA}{s m}{%
 \IfBooleanTF{#1}{\@DescAnumtrue\gdef\@DescuentoAnum{#2}}{\@DescAnumfalse\gdef\@DescuentoAporc{#2}}%
 \@DescAtrue\@DescuentoTotaltrue}

\NewDocumentCommand{\DescuentoE}{s m}{%
 \IfBooleanTF{#1}{\@DescEnumtrue\gdef\@DescuentoEnum{#2}}{\@DescEnumfalse\gdef\@DescuentoEporc{#2}}%
 \@DescEtrue\@DescuentoTotaltrue}

\def\Denom#1{\gdef\@Denominacion{#1}}%  La denominación puede ser 'FACTURA', 'COTIZACIÓN' u otra. Se predetermina una de estas dos según las opciones de la clase.

%% Número de la factura con \Num, número de control con \NumControl, y número de serie de control \NumSerieControl.
\def\Num#1{\gdef\@Num{\nplpadding{\@DigitosNumeracion}\npthousandsep{}\numprint{#1}}}
\def\NumControl#1{\gdef\@NumControl{\nplpadding{\@DigitosNumeracion}\npthousandsep{}\numprint{#1}}}%  Si el número de control no se especifica, toma el mismo que el de la factura
\def\NumSerieControl#1{\gdef\@NumSerieControl{#1}}%  Si el número de serie de control no se especifica, toma '00'.

%% Información de página: cabeceras de páginas o pies de página.
\def\InfoPagina#1{\gdef\@InfoPagina{#1}}

%% Texto para la nota final.
\NewDocumentCommand{\NotaFinal}{+m}{\gdef\@NotaFinal{#1}}
\NewDocumentCommand{\Condiciones}{+m}{\gdef\@NotaFinal{#1}\@ClassWarningCondiciones}

%% Advertencia para la migración de cualquier comando para condiciones a nota final.
\def\@ClassWarningCondiciones{\CLASSWARNING{Ya no son condiciones sino nota final, debe hacerse la migración a los nuevos comandos}}

%% Resumen opcional que va a la derecha de los datos del cliente.
\NewDocumentCommand{\Resumen}{+m}{\gdef\@Resumen{#1}}

\NewDocumentCommand{\NotaPrevia}{+m}{\gdef\@NotaPrevia{#1}}

%% Nota interna al cuadro de la factura y nota externa a él (que sale antes de \Descripcion).
\NewDocumentCommand{\NotaInterna}{+m}{%
 \ifdefvoid{\@NotaInterna}{}{\CLASSWARNING{Se sobreescribio' la nota interna previamente definida}}%
 \gdef\@NotaInterna{#1}}
\NewDocumentCommand{\NotaExterna}{+m}{\gdef\@NotaExterna{#1}}

\AtBeginDocument{%
 % Si no se especificó fecha, tomamos la de hoy. Además, advertimos cuando el documento se está compilando en una fecha distinta de la fecha establecida en el encabezado.
 \DTMifsaveddate{FechaEmision}{}{%
  \CLASSWARNING{No se especifico' fecha. Asumiendo la fecha de hoy}%
  \DTMsavedate{FechaEmision}{\the\year-\the\month-\the\day}}%
 \DTMsavenow{Ahora}%
 \DTMsaveddatediff{FechaEmision}{Ahora}{\AuxFecha}%
 \ifnum\AuxFecha=0%
 \else
  \CLASSWARNING{Fecha del documento \@Fecha\space distinta de la fecha de hoy}%
 \fi

%% (quitado por no vigente)
 % Establecemos el signo monetario dependiendo de la fecha de emisión de la factura, para efectos de la Reconversión Monetaria:
 % \ifx\@undefined\@Moneda
 %  \DTMsaveddatediff{FechaEmision}{FechaBsS}{\AuxFecha}%
 %  \ifnum\AuxFecha<0%
 %   \Moneda{\@MonedaBs}%
 %   \LetraNotaInterna{\fontsize{14}{18}}%
 %   \NotaInterna{\FPupn\TotalBsS{\Total{} 100000 div \@DigitosMoneda{} round}Según Decreto Presidencial 3.548, el total en Bolívares Soberanos es de {\bfseries \@MonedaBsS~\numprint{\TotalBsS}}}%
 %   \CLASSWARNING{Se establecio' el texto aclaratorio sobre el total en Boli'vares Soberanos como una nota interna}%
 %  \else
 %   \DTMsaveddatediff{FechaEmision}{FechaBs}{\AuxFecha}%
 %   \ifnum\AuxFecha>-1%
 %    \Moneda{\@MonedaBs}%
 %   \else
 %    \Moneda{\@MonedaBsS}%
 %    \LetraNotaInterna{\fontsize{14}{18}}%
 %    \NotaInterna{\FPupn\TotalBsF{\Total{} 100000 * \@DigitosMoneda{} round}El total en Bolívares Fuertes es de {\bfseries \@MonedaBsF~\numprint{\TotalBsF}}}%
 %    \CLASSWARNING{Se establecio' el texto aclaratorio sobre el total en Boli'vares Fuertes como una nota interna}%
 %   \fi
 %  \fi
 % \fi

 % Efectuamos distintos chequeos para dar errores o advertencias si faltó algo.
 \ifdefvoid{\@RazonSocial}{\CLASSERROR{Hay que definir la razo'n social}}{}%
 \if@reporte
  \ifdefvoid{\@Num}{\CLASSWARNING{No se especifico' el nu'mero de la factura. Puede dar error si el ti'tulo del reporte no se especifica que aparezca en las cabeceras}}{}%
  \ifdefvoid{\@Proveedor}{%
   \ifdefvoid{\@Emisor}
    {\CLASSERROR{Hay que definir el nombre del proveedor cuando se emite un reporte}}%
    {\CLASSWARNING{Se toma el emisor como el proveedor por no haberse especificado este u'ltimo}%
    \let\@Proveedor=\@Emisor}}
   {}%
 \fi
 \if@factura
  \ifdefvoid{\@RIF}{\CLASSERROR{Hay que definir el R.I.F}}{}%
  \ifdefvoid{\@Direccion}{\CLASSERROR{Hay que definir la direccio'n fiscal}}{}%
  \ifdefvoid{\@Telefono}{\CLASSWARNING{Pudiera definirse un nu'mero telefo'nico}}{}%
  \ifdefvoid{\@Email}{\CLASSWARNING{Pudiera definirse una direccio'n de correo electro'nico}}{}%
  \ifdefvoid{\@NotaFinal}{\CLASSWARNING{Pudiera definirse una nota final}}{}%
  \ifcotizacion
   \ifdefvoid{\@FondoFactura}{\CLASSWARNING{En la opcio'n para cotizacion se recomienda agregar el fondo}}{}%
  \fi
  \ifprefactura
   \ifdefvoid{\@FondoFactura}{\CLASSWARNING{En la opcio'n para prefactura se recomienda agregar el fondo}}{}%
  \fi
  \ifsinnumero
   \ifcotizacion
   \else
    \CLASSERROR{La opcio'n para no poner el nu'mero so'lo es compatible con las cotizaciones}%
   \fi
  \fi
  \ifdenom
   \ifsinnumero
   \else
    \ifdefvoid{\@Num}{\CLASSERROR{Hay que definir el nu'mero de la factura}}{}%
   \fi
  \fi
 \fi
 \ifdefvoid{\@Contacto}{\if@Nombre\else\CLASSWARNING{Pudiera definirse una persona de contacto}\fi}{}%

 % El cálculo de la fecha de vencimiento para cuando haya crédito o fecha de vencimiento explícita.
 \ifdefvoid{\@Credito}
  {\gdef\@FechaVencimiento{\DTMusedate{FechaEmision}}}
  {\DTMsaveddateoffsettojulianday{FechaEmision}{\@Credito}{\AuxFecha}%
   \DTMsavejulianday{FechaVencimiento}{\number\AuxFecha}%
   \gdef\@FechaVencimiento{\DTMusedate{FechaVencimiento}}}%
 \ifx\@undefined\@NumControl\let\@NumControl\@Num\fi
 \ifx\@undefined\@NumSerieControl\gdef\@NumSerieControl{00}\fi
 
 % Establecemos los porcentajes de las alícuotas, y redondeamos los descuentos.
 \FPupn\@AlicuotaGporc{\@AlicuotaG{} 100 * \@DigitosPorc{} round clip}%
 \FPupn\@AlicuotaRporc{\@AlicuotaR{} 100 * \@DigitosPorc{} round clip}%
 \FPupn\@AlicuotaAporc{\@AlicuotaA{} 100 * \@DigitosPorc{} round clip}%
 \FPset\@AlicuotaEporc{0}%
 \FPupn\@DescuentoGporc{\@DescuentoGporc{} \@DigitosPorc{} round clip}%
 \FPupn\@DescuentoRporc{\@DescuentoRporc{} \@DigitosPorc{} round clip}%
 \FPupn\@DescuentoAporc{\@DescuentoAporc{} \@DigitosPorc{} round clip}%
 \FPupn\@DescuentoEporc{\@DescuentoEporc{} \@DigitosPorc{} round clip}%
 \FPupn\@DescuentoGfactor{100 \@DescuentoGporc{} /}%
 \FPupn\@DescuentoRfactor{100 \@DescuentoRporc{} /}%
 \FPupn\@DescuentoAfactor{100 \@DescuentoAporc{} /}%
 \FPupn\@DescuentoEfactor{100 \@DescuentoEporc{} /}

 % Una vez definido todo en el preámbulo, los comandos usados para definir campos servirán para mostrar el campo que se definió con ese comando.
 \def\Moneda{\@Moneda}%
 \def\RazonSocial{\@RazonSocial}%
 \let\Nombre\RazonSocial
 \def\RIF{\@RIF}%
 \let\CI\RIF
 \def\Direccion{\@Direccion}%
 \def\Telefono{\@Telefono}%
 \def\Email{\@Email}%
 \def\Contacto{\@Contacto}%
 \def\Emisor{\@Emisor}%
 \def\Conforme{\@Conforme}%
 \def\Proveedor{\@Proveedor}%
 \def\Fecha{\@Fecha}%
 \def\FechaVencimiento{\@FechaVencimiento}%
 \def\Credito{\@Credito}%
 \def\Num{\@Num}%
 \def\NumSerieControl{\@NumSerieControl}%
 \def\NumControl{\@NumControl}%
 \def\NotaFinal{\@NotaFinal}%
 \def\Resumen{\@Resumen}%
 \def\NotaPrevia{\@NotaPrevia}%
 \def\NotaInterna{\@NotaInterna}%
 \def\NotaExterna{\@NotaExterna}%
 \def\TituloReporte{\@TituloReporte}}

%</class>
%<*class|def>
%% Posiciones (X,Y) de los objetos: i) fecha, ii) denominación, iii) nota final, iv) número de control, v) firmas.
%<*class>
\NewDocumentCommand{\PosFecha}{s m m}{
 \setlength{\@PosFechaX}{#2}
 \setlength{\@PosFechaY}{#3}
 \IfBooleanTF{#1}{\booltrue{FechaYarriba}}{\boolfalse{FechaYarriba}}
 \IfBeginWith{#2}{-}{\booltrue{FechaXnegativo}}{\boolfalse{FechaXnegativo}}
 \IfBeginWith{#3}{-}{\booltrue{FechaYnegativo}}{\boolfalse{FechaYnegativo}}}

%</class>
%<def>\PosFecha*{-0cm}{5cm}
%<*class>
\NewDocumentCommand{\PosDenom}{s m m}{
 \setlength{\@PosDenomX}{#2}
 \setlength{\@PosDenomY}{#3}
 \IfBooleanTF{#1}{\booltrue{DenomYarriba}}{\boolfalse{DenomYarriba}}
 \IfBeginWith{#2}{-}{\booltrue{DenomXnegativo}}{\boolfalse{DenomXnegativo}}
 \IfBeginWith{#3}{-}{\booltrue{DenomYnegativo}}{\boolfalse{DenomYnegativo}}}

%</class>
%<def>\PosDenom*{0cm}{5cm}
%<*class>
\NewDocumentCommand{\PosControl}{s m m}{
 \setlength{\@PosControlX}{#2}
 \setlength{\@PosControlY}{#3}
 \IfBooleanTF{#1}{\booltrue{ControlYarriba}}{\boolfalse{ControlYarriba}}
 \IfBeginWith{#2}{-}{\booltrue{ControlXnegativo}}{\boolfalse{ControlXnegativo}}
 \IfBeginWith{#3}{-}{\booltrue{ControlYnegativo}}{\boolfalse{ControlYnegativo}}}

%</class>
%<def>\PosControl*{0cm}{-2.4cm}
%<*class>
\NewDocumentCommand{\PosNotaFinal}{s m m m}{
 \setlength{\@PosNotaFinalXuno}{#2}
 \setlength{\@PosNotaFinalXdos}{#3}
 \setlength{\@PosNotaFinalY}{#4}
 \IfBooleanTF{#1}{\booltrue{NotaFinalYarriba}}{\boolfalse{NotaFinalYarriba}}
 \IfBeginWith{#4}{-}{\booltrue{NotaFinalYnegativo}}{\boolfalse{NotaFinalYnegativo}}}
\let\PosCondiciones\PosNotaFinal
%</class>
%<def>\PosNotaFinal{0cm}{0cm}{-5.3cm}
%% La nota final se define con un X_inicial, un X_final y un Y_inicial (Xo,Xf,Y)
%<*class>

\NewDocumentCommand{\PosFirmas}{m m m}{
 \setlength{\@PosFirmasXuno}{#1}
 \setlength{\@PosFirmasXdos}{#2}
 \setlength{\@PosFirmasY}{#3}
 \IfBeginWith{#3}{-}{\booltrue{FirmasYnegativo}}{\boolfalse{FirmasYnegativo}}}
%</class>
%<def>\PosFirmas{5cm}{1cm}{-2.4cm}
%% La firma se define con un X_inicial, un X_final y un Y_inicial (Xo,Xf,Y)

%% Tamaño y grosor de la línea de las firmas.
%<*class>
\def\LineaFirmas#1#2{
 \setlength{\@TamLineaFirmas}{#1}
 \setlength{\@GrosorLineaFirmas}{#2}}
%</class>
%<def>\LineaFirmas{5.5cm}{0.7pt}

%% Separación mínima horizontal entre los totales en modo de filas y distancia vertical entre esas líneas, fija.
%<class>\def\SepFilas#1#2{\setlength{\@SeparaMinTotales}{#1}\setlength{\@SeparaFilas}{#2}}
%<def>\SepFilas{0.8cm}{2.5pt}

%% Separación vertical entre los elementos de la factura (entre el encabezado y el cuadro de la 'factura', luego desde dicho cuadro hasta la descripción, o hasta los totales en modo de filas).
%<class>\def\SepVertical#1{\setlength{\@SeparaVertical}{#1}}
%<def>\SepVertical{0.2cm}

%% Separación horizontal entre la caja de los datos del cliente y la caja del resumen.
%<class>\def\SepDatosResumen#1{\setlength{\@SeparaDatosResumen}{#1}}
%<def>\SepDatosResumen{0.2cm}

%% Separación horizontal interna entre la línea y el texto en los datos del cliente y en el resumen.
%<class>\def\SepEncabezado#1{\setlength{\@SeparaInternaEncabezado}{#1}}
%<def>\SepEncabezado{6pt}

%% Separación interna entre la línea y el texto de la nota previa.
%<class>\def\SepNotaPrevia#1{\setlength{\@SeparaInternaNotaPrevia}{#1}}
%<def>\SepNotaPrevia{6pt}

%% Separación extra entre los elementos en la factura.
%<class>\def\SepItemsExtra#1{\setlength{\@SeparaItemsExtra}{#1}}
%<def>\SepItemsExtra{0pt}

%% Separación interna entre la línea y el texto de la descripción.
%<class>\def\SepDescripcion#1{\setlength{\@SeparaInternaDescripcion}{#1}}
%<def>\SepDescripcion{6pt}

%<class>\def\SepNotaFinal#1{\setlength{\@SeparaInternaNotaFinal}{#1}}
%<class>\let\SepCondiciones\SepNotaFinal
%<def>\SepNotaFinal{5pt}

%% Grosor de la línea de las cajas del encabezado, curvatura de la esquina y sombra.
%% (si pedimos linea de 0pt, todo queda en cero y no se abre el ambiente TikZ).
%<class>\NewDocumentCommand{\LineaEncabezado}{m g g}{
%<class> \setlength{\@GrosorLineaEncabezado}{#1}
%<class> \IfValueT{#2}{\setlength{\@EsquinaLineaEncabezado}{#2}}
%<class> \IfValueT{#3}{\setlength{\@SombraEncabezado}{#3}}}
%<def>\LineaEncabezado{0.4pt}{10pt}{0pt}

%% Grosor de la línea de la nota previa, curvatura de la esquina y sombra.
%% (si pedimos linea de 0pt, todo queda en cero y no se abre el ambiente TikZ).
%<class>\NewDocumentCommand{\LineaNotaPrevia}{m g g}{
%<class> \setlength{\@GrosorLineaNotaPrevia}{#1}
%<class> \IfValueT{#2}{\setlength{\@EsquinaLineaNotaPrevia}{#2}}
%<class> \IfValueT{#3}{\setlength{\@SombraNotaPrevia}{#3}}}
%<def>\LineaNotaPrevia{0.4pt}{7pt}{0pt}

%% Grosor de la línea de la descripción, curvatura de la esquina y sombra.
%% (si pedimos linea de 0pt, todo queda en cero y no se abre el ambiente TikZ).
%<class>\NewDocumentCommand{\LineaDescripcion}{m g g}{
%<class> \setlength{\@GrosorLineaDescripcion}{#1}
%<class> \IfValueT{#2}{\setlength{\@EsquinaLineaDescripcion}{#2}}
%<class> \IfValueT{#3}{\setlength{\@SombraDescripcion}{#3}}}
%<def>\LineaDescripcion{0.4pt}{0pt}{4pt}

%% Grosor de la línea de la nota final, curvatura de la esquina y sombra.
%% (si pedimos linea de 0pt, todo queda en cero y no se abre el ambiente TikZ).
%<class>\NewDocumentCommand{\LineaNotaFinal}{m g g}{
%<class> \setlength{\@GrosorLineaNotaFinal}{#1}
%<class> \IfValueT{#2}{\setlength{\@EsquinaLineaNotaFinal}{#2}}
%<class> \IfValueT{#3}{\setlength{\@SombraNotaFinal}{#3}}}
%<class>\let\LineaCondiciones\LineaNotaFinal
%<def>\LineaNotaFinal{0pt}{0pt}{0pt}

%% El espacio vertical que hay que avanzar sólo en la primera página del reporte antes de empezar a escribir sobre él. Usable cuando la primera página tiene un membrete y las demás no, es decir, cuando se usa \FondoReporte y no \FondoReporte*.
%<class>\def\InicioReporte#1{\setlength{\@InicioReporte}{#1}}
%<def>\InicioReporte{1.6cm}

%% Tipos de letra para cada elemento posible:
%% Esta es la letra de la descripción de los items; y el espacio \baselineskip de *toda* la tabla de la factura se fija en este comando únicamente (se predeterminó 18pt).
%<class>\def\LetraItems#1{\gdef\@LetraItems{#1\selectfont}}
%<def>\LetraItems{\fontsize{14}{18}}

%% Tipo de letra para la nota interna:
%<class>\def\LetraNotaInterna#1{\gdef\@LetraNotaInterna{#1\selectfont}}
%<def>\LetraNotaInterna{\fontsize{11}{13}\itshape}

%% Tipo de letra para la nota externa:
%<class>\def\LetraNotaExterna#1{\gdef\@LetraNotaExterna{#1\selectfont}}
%<def>\LetraNotaExterna{\fontsize{14}{16.8}}

%% Con este comando se cambia la etra del título de la tabla, es decir, los textos 'Cantidad, 'Descripción', etc...
%<class>\def\LetraTitColumnas#1{\gdef\@LetraTitColumnas{#1\selectfont}}
%<def>\LetraTitColumnas{\fontsize{10}{10}\bfseries}

%% Tipo de letra de los títulos de los totales, por ejemplo los textos: 'B.I. G', 'IVA G'.
%<class>\def\LetraTitTotales#1{\gdef\@LetraTitTotales{#1\selectfont}}
%<def>\LetraTitTotales{\fontsize{12}{12}\bfseries}

%% Letra con la que salen los totales parciales, es decir, los números, exceptuando el total general.
%<class>\def\LetraNumTotales#1{\gdef\@LetraNumTotales{#1\selectfont}}
%<def>\LetraNumTotales{\fontsize{14}{12}}

%% Letra con la que sale la palabra 'TOTAL'.
%<class>\def\LetraTitTotal#1{\gdef\@LetraTitTotal{#1\selectfont}}
%<def>\LetraTitTotal{\fontsize{12}{12}\bfseries}

%% Letra para el total general, el monto.
%<class>\def\LetraNumTotal#1{\gdef\@LetraNumTotal{#1\selectfont}}
%<def>\LetraNumTotal{\fontsize{14}{14}\bfseries}

%% Tipo de letra para los títulos de las cajas del encabezado, por ejemplo: 'Datos del cliente'.
%<class>\def\LetraTitEnc#1{\gdef\@LetraTitEnc{#1\selectfont}}
%<def>\LetraTitEnc{\fontsize{11}{12.5}\slshape}

%% Tipo de letra para el texto en el encabezado de la factura.
%<class>\def\LetraEncFactura#1{\gdef\@LetraEncFactura{#1\selectfont}}
%<def>\LetraEncFactura{\fontsize{11}{12.5}}

%% Tipo de letra para el texto en el encabezado del reporte.
%<class>\def\LetraEncReporte#1{\gdef\@LetraEncReporte{#1\selectfont}}
%<def>\LetraEncReporte{\fontsize{13}{16}}

%% Letra para los montos individuales de la factura y las cantidades.
%<class>\def\LetraNumeros#1{\gdef\@LetraNumeros{#1\selectfont}}
%<def>\LetraNumeros{\fontsize{14}{14}}

%% El tipo de IVA ('G', 'R', 'A' o '(E)') también tiene su tipo de letra.
%<class>\def\LetraTipoIVA#1{\gdef\@LetraTipoIVA{#1\selectfont}}
%<def>\LetraTipoIVA{\fontsize{12}{12}}

%% Letra para la descripción larga (comando \Descripcion).
%<class>\def\LetraDescripcion#1{\gdef\@LetraDescripcion{#1\selectfont}}
%<def>\LetraDescripcion{\fontsize{14}{16.8}}

%% Tipo de letra para el texto bajo las líneas para firmar.
%<class>\def\LetraFirmas#1{\gdef\@LetraFirmas{#1\selectfont}}
%<def>\LetraFirmas{\fontsize{12}{12}}

%% Tipo de letra para la nota previa.
%<class>\def\LetraNotaPrevia#1{\gdef\@LetraNotaPrevia{#1\selectfont}}
%<def>\LetraNotaPrevia{\fontsize{11}{12.5}}

%% Tipo de letra para la nota final.
%<class>\def\LetraNotaFinal#1{\gdef\@LetraNotaFinal{#1\selectfont}}
%<class>\let\LetraCondiciones\LetraNotaFinal
%<def>\LetraNotaFinal{\fontsize{12}{12}}

%% Letra para la denominación del documento ('FACTURA', 'COTIZACIÓN',...).
%<class>\def\LetraDenom#1{\gdef\@LetraDenom{#1\selectfont}}
%<def>\LetraDenom{\fontsize{14}{14}\bfseries}

%% Letra para la numeración de la factura.
%<class>\def\LetraNumeracion#1{\gdef\@LetraNumeracion{#1\selectfont}}
%<def>\LetraNumeracion{\fontsize{14}{14}\bfseries\color{red}}

%% Letra para las fechas (emisión y vencimiento).
%<class>\def\LetraFecha#1{\gdef\@LetraFecha{#1\selectfont}}
%<def>\LetraFecha{\fontsize{14}{15}}

%% Letra para el título del reporte.
%<class>\def\LetraTitReporte#1{\gdef\@LetraTitReporte{#1\selectfont}}
%<def>\LetraTitReporte{\fontsize{16}{22}\bfseries}

%% Letra para las cabeceras de las páginas en el modo reporte.
%<class>\def\LetraInfoPagina#1{\gdef\@LetraInfoPagina{#1\selectfont}}
%<def>\LetraInfoPagina{\fontsize{12}{12}\slshape}

%% Letra para el reverso de la factura.
%<class>\def\LetraReverso#1{\gdef\@LetraReverso{#1\selectfont}}
%<def>\LetraReverso{\fontsize{10}{12}}

%% Letra para el reporte.
%<class>\def\LetraReporte#1{\gdef\@LetraReporte{#1\selectfont}}
%<def>\LetraReporte{\fontsize{12}{20}}

%</class|def>
%<*class>
%% Comando para incluir la firma escaneada en el sitio donde va la línea de la firma del emisor.
\NewDocumentCommand{\FirmaFactura}{m O{0pt} O{0pt}}{% \FirmaFactura{imagen}[x-corr][y-corr]
 \AtEndPreamble{
  \if@factura
   \RequirePackage{graphicx}
   \ifx&#2&\else\setlength{\@FirmaFacturaXcorr}{#2}\fi
   \ifx&#3&\else\setlength{\@FirmaFacturaYcorr}{#3}\fi
   \savebox{\@CajaFirmaEmisor}{#1}
   \AtBeginShipoutNext{%
    \begin{textblock*}{\@TamLineaFirmas}%
     (\textwidth+\oddsidemargin+1in-\@TamLineaFirmas/2-\wd\@CajaFirmaEmisor/2-\@PosFirmasXdos+\@FirmaFacturaXcorr,\@PosFirmasY-\ht\@CajaFirmaEmisor-\dp\@CajaFirmaEmisor+\@FirmaFacturaYcorr)%
     \usebox{\@CajaFirmaEmisor}%
    \end{textblock*}}%
  \fi}}

\NewDocumentCommand{\FirmaReporte}{O{} O{0pt} O{0pt} G{\@Proveedor} O{\@FirmaReportePosX} O{\@FirmaReporteExcesoLinea}}{% \Firma[imagen][x-corr][y-corr]{texto}[posición en X desde la derecha][tamaño linea]
 \renewcommand{\baselinestretch}{1}\selectfont
 \newif\if@Firma@Dibujo
 \newbox{\TextoFirma}%
 \newbox{\DibujoFirma}%
 \ifx&#1&\else\savebox{\DibujoFirma}{#1}\@Firma@Dibujotrue\fi
 \setlength{\@FirmaReporteXcorr}{#2}%
 \setlength{\@FirmaReporteYcorr}{#3}%
 \savebox{\TextoFirma}{#4}%
 \setlength{\@FirmaReporteTextPos}{#5}%
 \setlength{\@FirmaReporteTamLinea}{#6}%
 \@LetraFirmas
 \if@Firma@Dibujo
  \newlength{\@FirmaReporteoddeven}%
  \setlength{\@FirmaReporteFirmaPos}{\zposy{Firma@zref@PosY}sp}%
 \fi
 \ifdim\wd\DibujoFirma>\wd\TextoFirma
  \setlength{\@FirmaReporteTamLinea}{\@FirmaReporteTamLinea+\wd\DibujoFirma}%
 \else
  \setlength{\@FirmaReporteTamLinea}{\@FirmaReporteTamLinea+\wd\TextoFirma}%
 \fi
 \par
 \vspace*{\ht\DibujoFirma}%
 \vspace*{\dp\DibujoFirma}%
 \hfill
 \parbox{\@FirmaReporteTamLinea}{%
  \if@Firma@Dibujo\zsaveposy{Firma@zref@PosY}\fi
  \rule{\@FirmaReporteTamLinea}{\@GrosorLineaFirmas}\\
  \centering{\usebox{\TextoFirma}}}%
 \hspace*{\@FirmaReporteTextPos}\par
 \if@Firma@Dibujo
  \mbox{}%
  \par
  \ifodd\c@page
   \setlength{\@FirmaReporteoddeven}{\oddsidemargin}%
  \else
   \setlength{\@FirmaReporteoddeven}{\evensidemargin}%
  \fi
  \begin{textblock*}{\@FirmaReporteTamLinea}
  (\textwidth+\@FirmaReporteoddeven+1in-0.5\@FirmaReporteTamLinea-0.5\wd\DibujoFirma-\@FirmaReporteTextPos+\@FirmaReporteXcorr,
   \paperheight-\@FirmaReporteFirmaPos-\ht\DibujoFirma-\dp\DibujoFirma+\@FirmaReporteYcorr)
   \noindent\usebox{\DibujoFirma}%
  \end{textblock*}
 \fi}

%% Con \FondoFactura le damos el nombre del archivo que contiene el formato de la factura, por ejemplo para compilaciones con 'prefactura', 'cotizacion' o para incluir una imagen de fondo como 'marca de agua'.
\newcommand{\FondoFactura}[1]{
 \AtEndPreamble{
  \if@factura
   \RequirePackage{graphicx}
   \@FondoFacturatrue
   \gdef\@FondoFactura{#1}
  \fi}}

%% Con \FondoReporte le damos el nombre del archivo que contiene el formato del reporte. Existen dos posibilidades: que dicho fondo salga en todas las páginas o sólo en la primera.
\NewDocumentCommand{\FondoReporte}{s m}{
 \AtEndPreamble{
  \if@reporte
   \@FondoReportetrue
   \RequirePackage{graphicx}
   \gdef\@FondoReporte{#2}
   \IfBooleanT{#1}{\@FondoReporteStartrue}% Con * sale en todas las páginas; sin * sale sólo en la primera página.
  \fi}}

%% Comando para establecer desde el preámbulo el estilo de página (\pagestyle) del reporte, tanto para la primera página como para las siguientes.
\NewDocumentCommand{\EstiloPagReporte}{o m}{
 \gdef\@EstiloPagReporteResto{#2}
 \IfValueTF{#1}{\gdef\@EstiloPagReporteUno{#1}}{\gdef\@EstiloPagReporteUno{\@EstiloPagReporteResto}}}

%% El título para el reporte.
\NewDocumentCommand{\TituloReporte}{s m}{
 \gdef\@TituloReporte{#2}%
 \IfBooleanTF{#1}{% Con '*' sale el título en las cabeceras o pies. Sin '*' no.
  \gdef\@InfoPagina{\let\\\relax\@InfoPaginaTitulo}}%
 {\let\@InfoPagina\@InfoPaginaNoTitulo}}

%% Para cada cliente o para cada personalización se puede tener un archivo .def o .tex con su nombre y ciertas macros cargadas. Ese archivo lo ponemos incluir con '\Cliente' en lugar de usar \input, y la extensión deberá ser .def o .tex.
\newcommand{\Cliente}[1]{%
 \IfFileExists{#1.def}{\input{#1.def}}{
  \IfFileExists{#1.tex}{\input{#1.tex}}{
   \CLASSERROR{No existe #1.def ni #1.tex}}}}

%\iffalse meta-comment
%
% factura.dtx
% Copyright 2019 Emilio Augusto Lazo Zaia <emiliolazozaia@gmail.com>
%
% This work may be distributed and/or modified under the
% conditions of the LaTeX Project Public License, either
% version 1.3c of this license or (at your option) any later
% version. This version of this license is in
%   http://www.latex-project.org/lppl/lppl-1-3c.{html|txt|tex|pdf}
% and the latest version of this license is in
%   http://www.latex-project.org/lppl.txt
% and version 1.3c or later is part of all distributions of
% LaTeX version 2008/05/04 or later.
%
% This work has the LPPL maintenance status `maintained'.
%
% The Current Maintainer of this work is Emilio Augusto Lazo Zaia.
%
% This work consists of the files README and factura.dtx
% with its derived files factura.ins, factura.cls, factura.def,
% factura-ejemplo-aux.tex, factura-ejemplo-firma.tex,
% factura-ejemplo-fondo.tex, factura-ejemplo-membrete.tex,
% factura-ejemplo-cotizacion.tex, factura-ejemplo-prefactura.tex,
% factura-ejemplo-factura01.tex, factura-ejemplo-factura02.tex,
% factura-ejemplo-factura03.tex, factura-ejemplo-factura04.tex,
% factura-ejemplo-factura05.tex, factura-ejemplo-factura06.tex,
% factura-ejemplo-factura07.tex, factura-ejemplo-factura08.tex,
% factura-ejemplo-factura09.tex, factura-ejemplo-factura10.tex,
% factura-ejemplo-reporte1.tex, factura-ejemplo-reporte2.tex,
% factura-ejemplo-reporte3.tex and factura-beispiel-rechnung.tex.
%<*internalbatchfile>
\def\nameofplainTeX{plain}
\ifx\fmtname\nameofplainTeX
\else
 \expandafter\begingroup
\fi
%</internalbatchfile>
%<*batchfile>
\input docstrip.tex
\preamble

Copyright 2019 Emilio Augusto Lazo Zaia <emiliolazozaia@gmail.com>

This work may be distributed and/or modified under the
conditions of the LaTeX Project Public License, either
version 1.3c of this license or (at your option) any later
version. This version of this license is in
 http://www.latex-project.org/lppl/lppl-1-3c.{html|txt|tex|pdf}
and the latest version of this license is in
 http://www.latex-project.org/lppl.txt
and version 1.3c or later is part of all distributions of
LaTeX version 2008/05/04 or later.

This work has the LPPL maintenance status `maintained'.

The Current Maintainer of this work is Emilio Augusto Lazo Zaia.

This work consists of the files README and factura.dtx
with its derived files factura.ins, factura.cls, factura.def,
factura-ejemplo-aux.tex, factura-ejemplo-firma.tex,
factura-ejemplo-fondo.tex, factura-ejemplo-membrete.tex,
factura-ejemplo-cotizacion.tex, factura-ejemplo-prefactura.tex,
factura-ejemplo-factura01.tex, factura-ejemplo-factura02.tex,
factura-ejemplo-factura03.tex, factura-ejemplo-factura04.tex,
factura-ejemplo-factura05.tex, factura-ejemplo-factura06.tex,
factura-ejemplo-factura07.tex, factura-ejemplo-factura08.tex,
factura-ejemplo-factura09.tex, factura-ejemplo-factura10.tex,
factura-ejemplo-reporte1.tex, factura-ejemplo-reporte2.tex,
factura-ejemplo-reporte3.tex and factura-beispiel-rechnung.tex.

\endpreamble
\keepsilent
\askforoverwritefalse
\obeyspaces
\Msg{*****************************************************************}
\Msg{**************************** factura ****************************}
\Msg{*****************************************************************}
\generate{\file{\jobname.cls}{\from{\jobname.dtx}{class}}}
\generate{\file{\jobname.def}{\from{\jobname.dtx}{def}}}
\Msg{\jobname.cls and \jobname.def written.}
\Msg{}
\postamble
\endpostamble
%</batchfile>
%<*internalbatchfile>
\generate{\file{\jobname.ins}{\from{\jobname.dtx}{batchfile}}}
\Msg{\jobname.ins written.}
%</internalbatchfile>
%<*batchfile>
\def\factuno{factura-ejemplo-factura01.tex}
\def\factdos{factura-ejemplo-factura02.tex}
\def\facttres{factura-ejemplo-factura03.tex}
\def\factcuatro{factura-ejemplo-factura04.tex}
\def\factcinco{factura-ejemplo-factura05.tex}
\def\factseis{factura-ejemplo-factura06.tex}
\def\factsiete{factura-ejemplo-factura07.tex}
\def\factocho{factura-ejemplo-factura08.tex}
\def\factnueve{factura-ejemplo-factura09.tex}
\def\factdiez{factura-ejemplo-factura10.tex}
\def\repuno{factura-ejemplo-reporte1.tex}
\def\repdos{factura-ejemplo-reporte2.tex}
\def\reptres{factura-ejemplo-reporte3.tex}
\def\cotiza{factura-ejemplo-cotizacion.tex}
\def\prefac{factura-ejemplo-prefactura.tex}
\def\beispiel{factura-beispiel-rechnung.tex}
\def\auxejemplo{factura-ejemplo-aux.tex}
\def\firma{factura-ejemplo-firma.tex}
\def\fondo{factura-ejemplo-fondo.tex}
\def\membrete{factura-ejemplo-membrete.tex}
\generate{\file{\factuno}{\from{\jobname.dtx}{factuno}}}
\generate{\file{\factdos}{\from{\jobname.dtx}{factdos}}}
\generate{\file{\facttres}{\from{\jobname.dtx}{facttres}}}
\generate{\file{\factcuatro}{\from{\jobname.dtx}{factcuatro}}}
\generate{\file{\factcinco}{\from{\jobname.dtx}{factcinco}}}
\generate{\file{\factseis}{\from{\jobname.dtx}{factseis}}}
\generate{\file{\factsiete}{\from{\jobname.dtx}{factsiete}}}
\generate{\file{\factocho}{\from{\jobname.dtx}{factocho}}}
\generate{\file{\factnueve}{\from{\jobname.dtx}{factnueve}}}
\generate{\file{\factdiez}{\from{\jobname.dtx}{factdiez}}}
\generate{\file{\repuno}{\from{\jobname.dtx}{repuno}}}
\generate{\file{\repdos}{\from{\jobname.dtx}{repdos}}}
\generate{\file{\reptres}{\from{\jobname.dtx}{reptres}}}
\generate{\file{\cotiza}{\from{\jobname.dtx}{cotiza}}}
\generate{\file{\prefac}{\from{\jobname.dtx}{prefac}}}
\generate{\file{\beispiel}{\from{\jobname.dtx}{beispiel}}}
\generate{\file{\auxejemplo}{\from{\jobname.dtx}{aux}}}
\generate{\file{\firma}{\from{\jobname.dtx}{firma}}}
\generate{\file{\fondo}{\from{\jobname.dtx}{fondo}}}
\generate{\file{\membrete}{\from{\jobname.dtx}{membrete}}}
\Msg{.tex examples written.}
\Msg{}
\Msg{*****************************************************************}
\immediate\write18{makeindex -s gind.ist -o \jobname.ind \jobname.idx}
\immediate\write18{makeindex -s gglo.ist -o \jobname.gls \jobname.glo}
\Msg{}
\Msg{Change history and index updated.}
\Msg{*****************************************************************}
\Msg{To produce only package files, type}
\Msg{  pdftex \jobname.dtx}
\Msg{To produce package files and pdf documentation, type}
\Msg{  xelatex \jobname.dtx}
\Msg{  pdflatex factura-ejemplo-fondo.tex}
\Msg{  pdflatex factura-ejemplo-membrete.tex}
\Msg{  xelatex factura-ejemplo-firma.tex}
\Msg{}
\Msg{Also compile all other examples and recompile \jobname.dtx.}
\Msg{(factura-ejemplo-aux.tex is not an example!)}
\Msg{}
\Msg{To manual update index and changelog, type}
\Msg{  makeindex -s gind.ist -o \jobname.ind \jobname.idx}
\Msg{  makeindex -s gglo.ist -o \jobname.gls \jobname.glo}
\Msg{}
\Msg{More information (in spanish) about compiling on README and documentation.}
\Msg{}
\Msg{*****************************************************************}
\Msg{**************************** factura ****************************}
\Msg{*****************************************************************}
\Msg{}
%</batchfile>
%<batchfile>\endbatchfile
%<*internalbatchfile>
\ifx\fmtname\nameofplainTeX
 \expandafter\endbatchfile
\else
 \expandafter\endgroup
\fi
%</internalbatchfile>
%<*driver>
\ProvidesFile{factura.dtx}
%</driver>
%<class>\NeedsTeXFormat{LaTeX2e}[1995/12/01]
%<class>\ProvidesClass{factura}
%<def>\ProvidesFile{factura.def}
%<*class|def>
[2019/08/16 v3.70
%<class>Clase para elaborar facturas legales
%<class>en la Repu'blica Bolivariana de Venezuela.
%<def>Archivo de valores iniciales y condiciones de borde para factura.cls
]

%</class|def>
%<*driver>
\documentclass{ltxdoc}
\usepackage[a4paper,headheight=0.6cm,headsep=0.77cm,footskip=1.2cm,footnotesep=0.54cm,marginparsep=10pt,vmargin={3cm,2cm},hmargin={4cm,2cm}]{geometry}
\usepackage[dvipsnames]{xcolor}
\usepackage[spanish]{babel}
\usepackage{
 array,
 calc,
 catchfilebetweentags,
 etoolbox,
 footnote,
 graphicx,
 ifluatex,
 ifxetex,
 metalogo,
 paralist,
 pgffor,
 regexpatch,
 setspace,
 tikz,
 todo,
 verbatim,
 xparse,
 xspace,
 xstring,
 hypdoc}

% Necesario ejecutar \newif de esta forma:
\csname newif\expandafter\endcsname\csname ifmotorunicode\endcsname
\csname newif\expandafter\endcsname\csname ifejemplos\endcsname

\ejemplostrue

\ifxetex
 \motorunicodetrue
\else
 \ifluatex \motorunicodetrue \fi
\fi

\ifmotorunicode
 \usepackage{fontspec}
 \setmainfont[Ligatures=TeX,Mapping=TeX]{Linux Libertine O}
\else
 \usepackage[T1]{fontenc}
 \usepackage[utf8]{inputenc}
\fi

\overfullrule15pt

\newcommand{\VERBATIM}{\verbatimcolor\setstretch{1.0}}

\xspaceaddexceptions{*}

\hypersetup{
 plainpages=false,
 colorlinks=true,
 linkcolor=blue,
 unicode=true}

\makesavenoteenv{tabular}
\makesavenoteenv{table}

\newcommand\flechaCR{%
 \raisebox{-0.4ex}{%
  \begin{tikzpicture}
  \draw[line width=0.075ex] (0,0) arc (20:70:0.75ex) + (0,0) arc (-70:-20:0.75ex) ++ (0.8ex,0.55ex) arc (0:-100:1ex);
 \end{tikzpicture}}}

\EnableCrossrefs
\CodelineIndex
\OnlyDescription
\RecordChanges

\setlength\parskip{7pt}
\setlength\IndexMin{100pt}

\widowpenalty10000

\newbool{Ejemplo}
\newbool{EjemploIzq}
\newbool{EjemploDer}
\newbool{DescripEjemplo}

\def\ejemplooptcolor{\color{blue}}
\def\ejemplomacrocolor{\color{purple}}
\def\verbatimcolor{\color{OrangeRed}}
\def\macromandcolor{\color{Red}}
\def\macrooptcolor{\color{RoyalBlue}}

\def\tikz{T\emph{i}kZ\xspace}

\catcode`\%=11
\catcode`\|=14
\NewDocumentCommand{\Ejemplo}{s O{0.5} m s O{} G{1} o}{|
 \IfFileExists{factura-#3.pdf}{|
  \catcode`\<=12
  \catcode`\>=12
  \IfBooleanTF{#4}
   {\IfBooleanTF{#1}
    {\booltrue{DescripEjemplo}}
    {\booltrue{Ejemplo}\booltrue{EjemploDer}}}
   {\IfBooleanTF{#1}
    {\booltrue{Ejemplo}\booltrue{EjemploIzq}}
    {\booltrue{Ejemplo}\booltrue{EjemploIzq}\booltrue{DescripEjemplo}}}
  \ifbool{Ejemplo}
   {\ifnum #6=1 \def\Ancho{#2}{\phantomsection\par\mbox{}\hfill{\color{RoyalBlue}\ttfamily \large factura-#3}\hfill\mbox{}\par}{}\fi}
   {\def\Ancho{0.05}}|
  \ifbool{EjemploDer}{\hfill}{}|
  \ifbool{Ejemplo}
   {\begin{minipage}{\Ancho\linewidth}
     \fbox{\includegraphics[page=#6,width=\linewidth-2\fboxrule-2\fboxsep]{factura-#3}}
     \ifbool{DescripEjemplo}{\addcontentsline{toc}{subsection}{factura-#3}}{}
    \end{minipage}}
   {}|
  \ifbool{EjemploIzq}{\hfill}{}|
  \ifbool{DescripEjemplo}
   {\hfill
   | Gracias a cfr y a egreg por el esclarecimiento en cuanto a la necesidad de \makeatletter en \CatchFileBetweenDelims y antes de \regexpatchcmd*.
   | Thanks to cfr and egreg for the clarification on the use of \makeatletter on \CatchFileBetweenDelims and before \regexpatchcmd*.
   | (https://tex.stackexchange.com/questions/380796/how-to-patch-a-macro-with-control-sequences-using-regexpatch)
   \CatchFileBetweenDelims{\DescEjemplo}{factura-#3.tex}{<desc>}{</desc>}[\catcode37=11\catcode13=14\makeatletter]|
   \catcode`\%=11
   \makeatletter
   | Gracias a Werner y a egreg por el uso de \begingroup\ttfamily\string ... \endgroup.
   | Thanks to Werner and egreg for the use of \begingroup\ttfamily\string ... \endgroup.
   | (http://tex.stackexchange.com/questions/342572/error-using-verb-as-the-replace-string-on-xpatchcmd)
   \regexpatchcmd*{\DescEjemplo}{\cC.\cO[*]*}{\c{begingroup}\c{ejemplomacrocolor}\c{ttfamily}\c{string}\0\c{endgroup}\c{xspace}}{}{}|
   \regexpatchcmd*{\DescEjemplo}{(\()(\cL[a-zG]*)(\))}{(\c{begingroup}\c{ejemplooptcolor}\c{ttfamily}\2\c{endgroup})}{}{}|
   \xpatchcmd*{\DescEjemplo}{%% }{\item}{}{}|
   \xpatchcmd*{\DescEjemplo}{}{}{}|
   \makeatother
   \begin{minipage}{\linewidth-\Ancho\linewidth}
    \begin{itemize}
     \setlength{\itemsep}{0pt}
     #5\DescEjemplo
    \end{itemize}
   \end{minipage}
   \IfValueT{#7}{{#7}}}
   {}|
  \boolfalse{Ejemplo}\boolfalse{EjemploIzq}\boolfalse{EjemploDer}\boolfalse{DescripEjemplo}|
  \catcode`\%=9
  \catcode`\<=13
  \catcode`\>=13
 }{\message{^^JLaTeX Warning: No existe factura-#4.pdf; hay que compilarlo antes de compilar este documento.^^J}}}
\catcode`\%=14
\catcode`\|=12

% No quiero que las notas pendientes tengan los números de página:
\patchcmd{\todoitem}{(p.~\pageref{todopage:\thetodo}):}{-}{}{}

\renewcommand{\todoname}{Pendiente}
\renewcommand{\todomark}{Pendiente}
\newcommand{\changesname}{Lista de cambios en las versiones}
\newcommand{\glosname}{Índice de comandos, entornos y opciones}

\makeatletter
 \def\SpecialEnvIndex#1{\@bsphack
  \index{#1\actualchar{\protect\ttfamily#1}
   (entorno)\encapchar usage}%
  \index{entornos:\levelchar#1\actualchar
   {\protect\ttfamily#1}\encapchar usage}\@esphack}
 \def\es@yearl{l año}

 % Definimos el comando para describir opciones, que inexplicablemente falta en doc.sty:
 \def\DescribeOption{\leavevmode\@bsphack\begingroup\MakePrivateLetters
  \Describe@Option}
 \def\Describe@Option#1{\endgroup
  \marginpar{\raggedleft\PrintDescribeOption{#1}}%
  \SpecialOptionIndex{#1}\@esphack\ignorespaces}
 \def\SpecialOptionIndex#1{\@bsphack
  \index{#1\actualchar{\protect\ttfamily#1}
   (opción)\encapchar usage}%
  \index{opciones:\levelchar#1\actualchar{\protect\ttfamily#1}\encapchar
   usage}\@esphack}
 \def\PrintDescribeOption#1{\strut \MacroFont #1\ }

 % Redefinimos el comando del encabezado del índice:
 \def\index@prologue{%
  \section*{\glosname}%
  \markboth{\glosname}{\glosname}}%

 % Adaptamos un poco la lista de cambios:
 \def\changes@#1#2#3{%
  \protected@edef\@tempa{%
   \noexpand\glossary{%
    {\bfseries #1 --- #2}%
    \ifx\saved@macroname\@empty
     \relax\actualchar
    \else
     \saved@indexname
     \actualchar
     \string\verb\quotechar*%
     \verbatimchar\saved@macroname
     \verbatimchar
    \fi
    \levelchar #3}}%
  \@tempa\endgroup\@esphack}
 
 \def\@wrglossary#1{%
  \protected@write\@glossaryfile{}%
  {\string\glossaryentry{#1}{1}}%
  \endgroup\@esphack}%

 \patchcmd{\glossary@prologue}{Change History}{\changesname}{}{}
\makeatother

\title{The \textsf{\jobname} class\thanks{Esta documentación corresponde a la clase \textsf{\jobname}~\fileversion, de fecha \filedate.}}
\author{Emilio Augusto Lazo Zaia\\ \texttt{<emiliolazozaia at gmail.com>}}

\GetFileInfo{\jobname.dtx}

\begin{document}
 \maketitle
 \DocInput{\jobname.dtx}
 \clearpage
 \phantomsection
 \addcontentsline{toc}{section}{\todoname}
 \todos
 \section*{Información específica de esta versión + Reconversión Monetaria}
 % v2.82 Desde la versión v2.82 de esta clase, para poder compilar con \LuaLaTeX{} es necesaria la versión v2.26 del paquete |zref|, de fecha 2016/05/21, y la versión v1.18 de |atbegshi| de fecha 2016/06/09.

 % v2.82 -> Esta versión de |factura| se sacó principalmente para corregir un error introducido en la versión anterior v2.72 en cuanto a las referencias, para agregar la opción |numitem| que enumera los ítems en la factura y para que la línea diagonal (opción |linea|) funcione correctamente siempre.
 
 % v3.00 -> Desde la versión v3.00, todos los ejemplos están incorporados a la documentación, con la descripción de cada uno, de manera automatizada.
 
 % v3.02 -> Esta versión v3.02 se saca porque la versión 3.9 de |fancyhdr| hizo necesaria esta actualización por un error que daba con los reportes por el uso en esta clase de la macro |\fancy@setoffs|. Lo que hacíamos con esa macro de |fancyhdr| lo hicimos ahora de otra forma mejor, dada la incompatibilidad con la nueva versión de ese paquete.%v3.02
 
 % v3.40 -> La versión 3.40 se saca por los cambios en la política impositiva decretados por el Presidente en Decreto número 3085 relativos a descuentos del 3\% y 5\% en el IVA general cuando los pagos se hacen por medios electrónicos; también se corrigió la tasa de IVA adicional.
 
 % v3.40 -> En la versión 3.40 se agregaron nuevas opciones en cuanto al manejo de la fecha de emisión y su formato, posibilidad de facturas a crédito con fecha de vencimiento, campos nuevos adicionales para agregar más textos en la factura, posibilidad de imprimir información en el reverso de la hoja de la factura, posibilidades nuevas en cuanto al posicionamiento de los elementos fijos, diseño \tikz más versátil en los cuadros con bordes, mejor manejo del archivo |.csv|, cambios en los nombres de algunos comandos y opciones.
 
 % v3.40 -> En la misma versión 3.40 se cambió la estructura de la sección de los comandos en esta documentación y se agregó una página de descripción al final de cada ejemplo.
 
 % v3.44 -> El cambio en la política impositiva reflejado en la versión 3.40 queda sin efecto para fines legales, ya no está vigente dicho descuento; sin embargo, no eliminamos el código que lo produce sino que la clase arroja una advertencia cuando se pide usar la opción de descuentos por pago electrónico.
 
 % v3.44 -> Desde la versión 3.44 podemos activar cualquier opción de la clase a través de condicionales en el preámbulo y no solamente como una opción al cargar la clase; también se puede ahora quitar el efecto de cualquier opción previamente dada al cargar la clase, esto con el propósito de hacer más versátil el uso de las opciones y así poder incluirlas también en archivos |.tex| o |.def| personalizados.
 
 % v3.44 -> Adicionalmente desde la versión 3.44 el archivo |.log| de las compilaciones tiene como comentario todas las opciones que fueron pedidas al cargar la clase.
 
 % v3.47 -> La versión 3.47 corrige problemas con la compilación en \LuaLaTeX{} y escribe la versión de la clase en el campo \texttt{Subject} de la {\slshape metadata} del archivo |.pdf| compilado.
 
 % v3.47 -> La versión 3.47 agrega el comando |\Moneda| para cambiar fácilmente la moneda.
 
 % v3.56 -> {\bfseries La versión 3.56 se saca para corregir la versión 3.47 en cuanto a dar cumplimiento al Decreto Presidencial 3.332 sobre la Reconvesión Monetaria.} Desde el 4 de junio del año 2018 la moneda pasará a llamarse Bolívar Soberano (Bs.S) y luego en una fecha aún no definida (la asumimos como el 1 de enero del año 2019) la moneda volverá a llamarse Bolívar (Bs.). Eventualmente el Banco Central de Venezuela emitirá un comunicado para establecer desde cuándo la moneda de curso legal vuelve a llamarse simplemente Bolívar. A partír de esa fecha deberá usarse el comando |\Moneda| para cambiar la moneda a su denominación definitiva mientras no se haya emitido la actualización de esta clase. Esa fecha aún no está determinada, así que la suponemos -y hacemos el cambio de moneda automático en esta clase- como el inicio del año 2019. Antes del 4 de junio, se establece una nota interna automática que escribe el total también en Bolívares Soberanos.
 
 % v3.56 -> Por otra parte, la versión 3.56 elimina la inclusión del paquete |lmodern| y discrimina la inclusión de |fontenc| solamente para la compilación con pdf\LaTeX, que es lo correcto. Además, en esta versión se agrega un campo extra opcional en los ítems de la factura para poder agregar una descripción más larga a cada ítem que podrá ser referenciada en otras partes del documento (por ejemplo en el reporte) pero que no aparece en el cuadro de la factura.

 % v3.56 -> En la versión 3.56 fueron corregidos errores en cuanto al posicionamiento de la línea diagonal y se agregaron las opciones: \texttt{sinlineahni}, \texttt{sinlineahtit} y \texttt{sinlineahtot} para poder quitar las líneas horizontales internas una a una y no como antes, todas a la vez, con \texttt{sinlineashi}.

 % v3.57 -> {\bfseries Esta versión extraordinaria v3.57 se saca para dar cumplimiento al Decreto Presidencial 3.548 sobre la Reconversión Monetaria, publicado en Gaceta Oficial 41.446. Dicho decreto establece que la nueva moneda Bolívar Soberano no tendrá 3 ceros menos como lo establecía el derogado decreto 3.332 sino 5 ceros menos que el Bolívar Fuerte. La Reconversión Monetaria comienza el 20 de agosto del 2018\footnote{Salvo que se postergue otra vez.} y antes de esa fecha ya los precios tienen que estar especificados en ambas monedas, y esta clase está adaptada a esa exigencia.}

 % v3.57 -> Si el usuario especifica a mano la moneda con el comando |\Moneda| -introducido en una reciente versión-, la clase |factura| no interviene en nada en cuanto a la Reconversión Monetaria, ahora, si no se especifica la moneda, puede ocurrir lo siguiente:

 % v3.57 -> Al emitirse una factura con fecha anterior a la reconversión (20 de agosto del 2018), los montos quedarán expresados en Bolívares Fuertes, simbolizado como {\slshape Bs.}, y la clase automáticamente agregará un comentario como {\slshape nota interna} indicando cuál sería el monto en Bolívares Soberanos; esto para dar cumplimiento al Decreto Presidencial que indica que los montos deben estar en expresados en ambas monedas antes de la entrada de la nueva moneda.

 % v3.57 -> Si se emite una factura con fecha posterior a la reconversión, los montos se asumirán como expresados en Bolívares Soberanos, es decir, el usuario debe darlos ya con los 5 ceros quitados, la clase va a simbolizar la moneda como {\slshape Bs.S} y agregará un comentario similar al que acabamos de mencionar pero para explícitar el monto en Bolívares Fuertes, es decir, con 5 ceros más; esto no forma parte del Decreto pero lo quisimos agregar.
 
 % v3.57 -> Luego, la moneda volverá a llamarse Bolívar simplemente y se simbolizará otra vez como {\slshape Bs.}; ese momento aún no lo sabemos pero para los efectos de esta clase vamos a asumirlo como que ocurrirá el 1 de enero del año 2019.
 
 % v3.57 -> Esta será la última versión que sacamos solamente para corregir fechas de la Reconversión Monetaria; si el Gobierno Nacional cambia las fechas nuevamente, el usuario tendrá que hacer uso de la interfaz del paquete |datetime2| para ajustar el funcionamiento de esta clase a los nuevos plazos; para esto aclaramos que el identificador para la entrada de la Reconversión Monetaria se llama {\slshape FechaBsS}\footnote{recordemos que antes de esta fecha, los montos se expresan en Bolívares Fuertes y la nota automática aclara cuánto es en Bolívares Soberanos, y después de esta fecha es al contrario.} y el identificador para el momento (que aún no se ha dicho) desde el cual la moneda vuelve a llamarse Bolívar es {\slshape FechaBs}. Para ajustarlos se usa el comando \cs{DTMsavedate} y las fechas se dan en formato {\slshape yyyy-mm-dd}. Damos como ejemplo lo que actualmente está programado en esta clase:
 
 Desde el 15/02/2019 la moneda nacional se vuelve a llamar “Bolívar” abreviado “Bs” sin “soberano”; esto según Gaceta Oficial 41.587. Quitamos toda referencia a la Reconversión Monetaria.
 
 También en esta versión establecemos la tasa de IVA general como 16\% tal como ya está vigente desde agosto del año 2018.
 
 Ya no está activa la opción |electro| para descuentos en pagos por medios electrónicos porque ya no está vigente el decreto.
 
 Ahora en los totales aparece el signo monetario siempre a menos de que se pida en la clase la opción |sinmonedatotales|.
 
 En esta misma versión se agregó un “gancho” llamado \cs{BeforeEndPreamble} para usos muy específicos.
 
 Las “condiciones del documento” ya no son “condiciones” sino que ahora lo llamamos “nota final” porque es otra nota más aparte de las demás, y va al final del documento; no tiene sentido hablar de condiciones del documento para referirse a ese espacio. Con eso se renombran los 5 comandos, pero dejando por compatibilidad unos alias a los viejos; los nuevos son: \cs{NotaFinal}, \cs{PosNotaFinal}, \cs{SepNotaFinal}, \cs{LineaNotaFinal}, \cs{LetraNotaFinal}.

 La macro \cs{PosNotaFinal} se comporta ahora como \cs{PosFirmas}, es decir, acepta un valor inicial en “x” y uno final.

 \section*{Agradecimientos}
 
 Por pruebas y sugerencias que dieron lugar a mejoras agradezco a Ritguey%
 \ifmotorunicode\footnote{\setmainfont{Symbola}🐧}\else\message{^^J^^JLaTeX Warning: Compila con XeLaTeX.^^J}\fi{}
 Flores Esteves y a Alberto Mijares. Por contribuciones a través de \texttt{StackExchange} agradezco a Alan Munn, cfr, David Carlisle, egreg, Gustavo Mezzetti, Heiko Oberdiek, jfbu, Joseph Wright, Werner y wipet. El motivo por el que se da cada agradecimiento está como comentario en el archivo |.cls| o en el |.dtx|.
 
 \addcontentsline{toc}{section}{\changesname}
 {\makeatletter\def\pfill{\unskip\hfill\@gobble}\makeatother
  \setlength{\columnsep}{0pt}
  \PrintChanges}
 \phantomsection
 \addcontentsline{toc}{section}{\glosname}
 \PrintIndex
\end{document}
%</driver>
% \fi
%
%^^A Gracias a jfbu y a Joseph Wright por \starttodo, \stoptodo, \endorgo.
%^^A Thanks to jfbu and Joseph Wright for \starttodo, \stoptodo, \endorgo.
%^^A (http://tex.stackexchange.com/questions/235680/add-something-to-the-driver-section-inside-the-package-code-using-onlydescripti)
% \long\def\stoptodo#1\starttodo {}%
% \long\def\starttodo#1\stoptodo {}%
%
% \makeatletter
% \def\endorgo #1{%
%  \ifx #1\endinput\expandafter\stoptodo
%  \else #1\expandafter\@gobble\fi}%
% \makeatother
%
%^^A No se puede calcular el checksum con \OnlyDescription.
% \CheckSum{0}
%
% \CharacterTable
%  {Upper-case    \A\B\C\D\E\F\G\H\I\J\K\L\M\N\O\P\Q\R\S\T\U\V\W\X\Y\Z
%   Lower-case    \a\b\c\d\e\f\g\h\i\j\k\l\m\n\o\p\q\r\s\t\u\v\w\x\y\z
%   Digits        \0\1\2\3\4\5\6\7\8\9
%   Exclamation   \!     Double quote  \"     Hash (number) \#
%   Dollar        \$     Percent       \%     Ampersand     \&
%   Acute accent  \'     Left paren    \(     Right paren   \)
%   Asterisk      \*     Plus          \+     Comma         \,
%   Minus         \-     Point         \.     Solidus       \/
%   Colon         \:     Semicolon     \;     Less than     \<
%   Equals        \=     Greater than  \>     Question mark \?
%   Commercial at \@     Left bracket  \[     Backslash     \\
%   Right bracket \]     Circumflex    \^     Underscore    \_
%   Grave accent  \`     Left brace    \{     Vertical bar  \|
%   Right brace   \}     Tilde         \~}
%
% \changes{v1.00}{2013/04/29}{Primera versión.}
% \changes{v2.00}{2015/03/30}{Segunda versión: errores corregidos, comandos y opciones nuevas, ambiente de reporte, documentación completa, ejemplos y convertido todo a formato \texttt{.dtx}.}
% \changes{v2.05}{2015/04/13}{Se cambió de \texttt{4pt} a \texttt{6pt} en \texttt{factura.def} para el margen interno predeterminado en las cajas del encabezado y en la descripción.}
% \changes{v2.05}{2015/04/13}{Se quitó \texttt{UTF8} de \texttt{inputenc} como un paquete que se cargue de forma predeterminada.}
% \changes{v2.05}{2015/04/13}{Advierte que no son compatibles los comandos \cs{Descuento} con la opción \texttt{filas}.}
% \changes{v2.05}{2015/04/13}{Se puso un parche en \cs{@CargaFondoReporte} para arreglar que \cs{topskip} aplicara bien en \cs{@EncabezadoReporte}.}
% \changes{v2.05}{2015/04/13}{Arreglada la interacción entre las opciones \texttt{ivanoexplicito} y \texttt{filas}, y se quitó la línea del total cuando se usa \texttt{ivanoexplicito}.}
% \changes{v2.05}{2015/04/13}{Se renombró la opción \texttt{relleno} a \texttt{exparriba}, a la vez que se quitó \cs{Descripcion*} porque se agregó la opción \texttt{sinexpandir}.}
% \changes{v2.05}{2015/04/13}{Se renombró la opción \texttt{ivanoexplicito} por \texttt{sinivaexpresado}.}
% \changes{v2.05}{2015/04/13}{Se cambiaron los \cs{AtBeginDocument} por \cs{AtBeginShipoutNext} de la factura porque daba problemas al especificar \texttt{conreporte} sin luego agregar el ambiente de reporte.}
% \changes{v2.05}{2015/04/13}{Se colocó \cs{@SeparaMinTotales} dentro de otro \cs{AtBeginDocument} más adecuado.}
% \changes{v2.05}{2015/04/13}{Modificaciones en la documentación y en los ejemplos.}
% \changes{v2.05}{2015/04/13}{Los comandos \cs{ItemX} ahora colocan la línea bien con la opción \texttt{lineaitem}.}
% \changes{v2.1}{2015/08/26}{Advierte cuando la fecha del documento es distinta de la fecha del dia.}
% \changes{v2.1}{2015/08/26}{Los comandos de descuentos totales aceptan un * como argumento para que el descuento pueda ser en moneda y no en porcentaje.}
% \changes{v2.1}{2015/08/26}{Quitados los alias de macros viejas agregados por compatibilidad.}
% \changes{v2.1}{2015/08/26}{Corregida la forma como sale la denominación del impuesto para todos los casos.}
% \changes{v2.1}{2015/08/26}{Corregida y mejorada la documentación.}
% \changes{v2.1}{2015/08/26}{Correcciones en los ejemplos.}
% \changes{v2.1}{2015/08/26}{Mejor manejo de \cs{InicioReporte}.}
% \changes{v2.6}{2016/03/07}{Cambiado el lugar donde se fija el nombre del documento (FACTURA o COTIZACIÓN) porque antes no era correcto y no se podía cambiar el nombre de la cotización a algo distinto como por ejemplo PRESUPUESTO.}
% \changes{v2.6}{2016/03/07}{Se renombró la opción \texttt{sincond} en favor de \texttt{sincondiciones}.}
% \changes{v2.6}{2016/03/07}{Cambiado “condiciones de pago” por “condiciones del documento”.}
% \changes{v2.6}{2016/03/07}{Los descuentos globales aceptan decimales bien; antes no.}
% \changes{v2.6}{2016/03/07}{Se cambió la forma como se llama al ambiente \texttt{tabularx} en la factura creando una nueva macro que se evalúa completamente resultando en la definición de las columnas.}
% \changes{v2.6}{2016/03/07}{Agregadas cuatro opciones: \texttt{sinlineasvi}, \texttt{sinlineasve}, \texttt{sinlineash}, \texttt{sinlineashi}, \texttt{sinlineashe} y renombrada la opción \texttt{lineaitem} por \texttt{conlineasha} aludiendo a “con líneas horizontales para los artículos”.}
% \changes{v2.6}{2016/03/07}{Se arregló la línea oblicua que en algunos casos seguía sin salir.}
% \changes{v2.6}{2016/03/07}{Textos “PAGADO” y “Por el emisor” para las facturas y cotizaciones cuando no se define el emisor del documento.}
% \changes{v2.6}{2016/03/07}{Nueva opción \texttt{dosfirmas} para poner las dos firmas en casos en los que lo predeterminado sea una o ninguna firma (cotizaciones o prefacturas).}
% \changes{v2.6}{2016/03/07}{Nuevo comando \cs{Conforme} para el nombre de quien recibe la factura.}
% \changes{v2.6}{2016/03/07}{Cambiado el lugar en el código donde se llama a la macro que dibuja la línea inutilizadora del espacio en blanco y se hace ahora con \cs{AddToShipoutPicture} para que no se genere un espacio en blanco extra.}
% \changes{v2.6}{2016/03/07}{El ejemplo de la documentación es de una cotización y no de una prefactura; actualizada la documentación y los ejemplos.}
% \changes{v2.6}{2016/03/07}{En el índice ponemos “entorno” en lugar de “environment”.}
% \changes{v2.6}{2016/03/07}{El verbo “poner” es digno para la documentación. No sólo las gallinas ponen.}
% \changes{v2.6}{2016/03/07}{Renombrado el comando \cs{EstiloPag} por \cs{InfoPagina}; también \cs{LetraEncReporte} por \cs{LetraInfoPagina}; por otra parte se cambió \cs{LetraDatosReporte} en favor de \cs{LetraEncReporte}, y por último se cambió \cs{LetraTitRenglon} por \cs{LetraTitColumnas}.}
% \changes{v2.6}{2016/03/07}{Renombrados los estilos de página: \texttt{soloenc} a \texttt{soloarriba}, y \texttt{solopie} a \texttt{soloabajo}.}
% \changes{v2.6}{2016/03/07}{Se mejoró el código relativo al estilo de página del reporte y agregada la macro \cs{EstiloPagReporte} para establecer el estilo de página del reporte.}
% \changes{v2.6}{2016/03/07}{Se agregó una forma para que más de una línea en la descripción de la factura sea mejor manejada.}
% \changes{v2.6}{2016/03/07}{Mejorado el código de la macro interna \cs{@EscribeTipoDeImpuesto}.}
% \changes{v2.6}{2016/03/07}{Cuando la descripción en el cuadro de la factura ocupa más que una línea, agrega al final de cada línea el carácter $\hookleftarrow$.}
% \changes{v2.6}{2016/03/07}{Corregido que salga la “G” obligatoria de IVA General cuando hay un elemento exento y uno con IVA General.}
% \changes{v2.6}{2016/03/07}{Cuatro nuevas definiciones dentro del archivo \texttt{factura.def} para establecer algunos caracteres que antes estaban fijados dentro del archivo \texttt{factura.cls}}
% \changes{v2.6}{2016/03/07}{{\slshape parbox}es para cada elemento en la tabla, para asegurar que la denominación del gravamen quede alineada correctamente.}
% \changes{v2.6}{2016/03/07}{Quitados los paquetes \texttt{xifthen} y \texttt{pbox}. Arreglada la presentación de los datos del cliente cuando el resumen es más grande.}
% \changes{v2.6}{2016/03/07}{Cuando había descuentos totales y se pedía modo de totalización en filas, este modo no se apagaba correctamente y la línea oblicua daba problemas. Arreglado.}
% \changes{v2.72}{2016/05/16}{Las macros internas \cs{@DescripcionAuxCmd} y \cs{@EscribeTipoDeImpuesto} fueron modificadas.}
% \changes{v2.72}{2016/05/16}{Arreglada la alineación a la derecha en algunos casos en los precios.}
% \changes{v2.72}{2016/05/16}{Se incluyó dentro de la macro interna \cs{@ITEM} la opción para soportar \cs{ItemX}.}
% \changes{v2.72}{2016/05/16}{Ahora se procesa el entorno de la factura aún cuando se pide sólo el reporte, para poder tomar las referencias y los totales. Para eso se cambiaron de sitio varias macros internas.}
% \changes{v2.72}{2016/05/16}{Agregadas las etiquetas y las referencias.}
% \changes{v2.82}{2016/06/25}{Agregada la opción \texttt{numitem} para enumerar cada ítem en la factura.}
% \changes{v2.82}{2016/06/25}{Corregido problema de etiquetas repetidas introducido en la versión 2.72.}
% \changes{v2.82}{2016/06/25}{Corregida la alineación vertical entre las distintas columnas agregando \texttt{[t]} al abrir los “{\slshape parboxes}” en los comandos auxiliares que ponen cada celda en un {\slshape parbox} del tamaño de la celda más grande, porque si no se centra verticalmente la celda y se desalinea respecto a la columna de al lado.}
% \changes{v2.82}{2016/06/25}{Ahora dentro de la definición de \cs{@CaracterCeldasVacias} está el código que centra el texto, para poder usar comandos que llenen la caja completa como \cs{dotfill}.}
% \changes{v2.82}{2016/06/25}{Reimplementada la línea diagonal para que funcione en todos los casos.}
% \changes{v2.82}{2016/06/25}{Nueva macro interna \cs{@ColumnasMultiColumnTotalesAuxCmd} para hacer más general la colocación de los totales en el modo de columnas dentro de un \cs{multicolumn}.}
% \changes{v2.82}{2016/06/25}{Quitado el paquete \texttt{MnSymbol} en favor del comando \cs{flechaCR} para la flecha que indica que la descripción ocupa más de una línea.}
% \changes{v3.00}{2017/03/11}{Se agregó la opción \texttt{sinmarcas} para pedir que no se escriba el carácter de fin de linea cuando la descripción ocupa más que una línea y también para evitar que salga el carácter que marca las celdas vacías también cuando la decripción es larga.}
% \changes{v3.00}{2017/03/11}{Nueva macro \cs{SepItemsExtra} para agregar espacio extra entre los ítems en la factura.}
% \changes{v3.00}{2017/03/11}{Opción nueva \texttt{sintotales} para que no se escriba ningún total de los elementos. Útil para cotizaciones de por ejemplo varias formas de un mismo producto.}
% \changes{v3.00}{2017/03/11}{Opción nueva \texttt{centradesc} para centrar el texto en la descripción de cada elemento en la factura.}
% \changes{v3.00}{2017/03/11}{Cambiado el interlineado en la documentación, el tamaño del papel y agregados a la documentación todos los ejemplos que están embebidos en el \texttt{.dtx}, de forma automática.}
% \changes{v3.00}{2017/03/11}{El descuento individual ahora es opcional al pedir la opción, es decir, no hace falta establecer en '0' los elementos donde no haya descuento cuando se usa la opción \texttt{descuentos}.}
% \changes{v3.00}{2017/03/11}{Opción \texttt{subtotal} para agregar una columna extra después del precio unitario y antes del descuento, para reflejar el precio antes de aplicar el descuento individual.}
% \changes{v3.02}{2017/07/16}{Ya no usamos la macro \cs{fancy@setoffs} de \texttt{fancyhdr} en el reporte para establecer valores iniciales por ejemplo de la longitud de la línea en los encabezados; ahora lo hacemos de una forma diferente, incluyendo en el \texttt{.def} una macro llamada \cs{@GrosorLineaInfoPag} que alberga el valor que \texttt{fancyhdr} asigna preterminadamente al grosor de dicha línea. No agregamos un comando para cambiar ese valor por no ser algo de relevancia.}
% \changes{v3.02}{2017/07/16}{Ajustada ligeramente esta lista de cambios y la inclusión de los ejemplos en la documentación.}
% \changes{v3.40}{2017/12/11}{Se agregó la opción \texttt{electro} para dar cumplimiento con el Decreto Presidencial 3085 y la Providencia Administrativa 0048 que establece descuentos en el IVA.}
% \changes{v3.40}{2017/12/11}{Macros nuevas para notas internas al cuadro de la factura y externas a él.}
% \changes{v3.40}{2017/12/11}{Ahora \cs{PosCond} acepta \texttt{*} para que el texto pueda abrir hacia arriba en la posición establecida y no solamente hacia abajo.}
% \changes{v3.40}{2017/12/11}{Agregado el comando \cs{Credito} para la posibilidad de facturas a crédito donde es calculada la fecha de vencimiento de la factura que aparecerá debajo de la fecha de emisión.}
% \changes{v3.40}{2017/12/11}{Ahora \cs{PosFecha} también acepta \texttt{*} para que las fechas (emisión y vencimiento) puedan fijarse también respecto a una posición final y no solamente inicial.}
% \changes{v3.40}{2017/12/11}{El comando \cs{Fecha} acepta \texttt{*} para poder especificar la fecha en formato \texttt{iso} de forma más estándar para que sea tomada correctamente por el paquete \texttt{datetime2}, usado para calcular la fecha de vencimiento.}
% \changes{v3.40}{2017/12/11}{Se agregó la macro \cs{FormatoFecha} para poder cambiar el formato en el que el usuario da la fecha de emisión y cómo ésta es presentada.}
% \changes{v3.40}{2017/12/11}{Si no se especifica la fecha, será tomada la fecha actual y generará una advertencia indicando eso y no un error indicando que no se ha especificado la fecha, como era antes.}
% \changes{v3.40}{2017/12/11}{Actualizada la documentación para reflejar los cambios.}
% \changes{v3.40}{2017/12/11}{Se renombró \cs{@EstiloFecha} por \cs{@EstiloFechas}, y \cs{@PosicionaFecha} por \cs{@PosicionaFechas}.}
% \changes{v3.40}{2017/12/11}{Reestructuración de la documentación.}
% \changes{v3.40}{2017/12/11}{Nueva opción \texttt{vencimiento} para poner la fecha de vencimiento aún en facturas a contado.}
% \changes{v3.40}{2017/12/11}{También \cs{PosDenom} y \cs{PosControl} aceptan \texttt{*}. Cambió en 2mm el margen superior predeterminado, así como también hubo un cambio en la coordenada vertical de la fecha, la denominación del documento y el número de control, ya que antes los predeterminados estaban especificados sin \texttt{*} y luego no.}
% \changes{v3.40}{2017/12/11}{No poner el teléfono ahora produce una advertencia y no un error como antes.}
% \changes{v3.40}{2017/12/11}{Funciones internas para los mensajes informaticos, advertencias y errores de la clase.}
% \changes{v3.40}{2017/12/11}{Se le agregó la posibilidad de \texttt{*} a la macro interna \cs{@DatosAux} para que pudiera ser utilizada la misma macro por los casos donde hay resumen y donde no lo hay.}
% \changes{v3.40}{2017/12/11}{A los cuadros dibujados con \tikz les fue separada la definición del nodo para ponerla dentro del \texttt{.def} en lugar de estar en el \texttt{.cls}.}
% \changes{v3.40}{2017/12/11}{Cuando se especifica 0 como grosor de línea para la descripción o el encabezado, desaparece la línea completamente y la estructura \tikz.}
% \changes{v3.40}{2017/12/11}{Renombrada \cs{PosCond} a \cs{PosCondiciones}.}
% \changes{v3.40}{2017/12/11}{Ahora las condiciones aceptan línea en \tikz, es decir que hay dos macros nuevas: \cs{LineaCondiciones} y \cs{SepCondiciones}. Predeterminadamente las condiciones quedaron como antes, o sea sin diseño de \tikz.}
% \changes{v3.40}{2017/12/11}{Quitada la macro interna \cs{@TamDescrip} porque no hacía nada.}
% \changes{v3.40}{2017/12/11}{Renombrada \cs{SepEncabez} a \cs{SepEncabezado}, tambień \cs{SepDescrip} a \cs{SepDescripcion}, \cs{LineaEncabez} en favor de \cs{LineaEncabezado}, y \cs{LineaDescrip} en favor de \cs{LineaDescripcion}, así como algunas macros internas asociadas a ellas.}
% \changes{v3.40}{2017/12/11}{Macro nueva para una nota previa al cuadro de la factura.}
% \changes{v3.40}{2017/12/11}{La macro \cs{SepEncabezado} fue dividida entre \cs{SepDatosResumen} y \cs{SepEncabezado}. La macro \cs{SepElementos} fue renombrada a \cs{SepVertical} y \cs{LetraTitCuadro} a \cs{LetraTitColumnas}.}
% \changes{v3.40}{2017/12/11}{Contador \texttt{@NumItem} renombrado a \texttt{NumItem}. Las macros para los parámetros de las líneas ahora aceptan dar solamente el primer argumento, por ejemplo para definir que no salga la línea, sólo basta con por ejemplo \cs{LineaEncabezado}{0pt}.}
% \changes{v3.40}{2017/12/11}{Nuevo parámetro opcional después de la descripción en los \cs{Item***} para cambiar el campo que va a salir en el \texttt{.csv}, y ahora se puede especificar un descuento individual en los ítems, que será ignorado si no se carga la opción para descuentos individuales.}
% \changes{v3.40}{2017/12/11}{Actualizada la tasa de IVA adicional.}
% \changes{v3.40}{2017/12/11}{Nueva opción \texttt{iva} para agregar una columna adicional que muestra el IVA de cada item.}
% \changes{v3.40}{2017/12/11}{Modificada la macro interna \cs{@DescripcionAuxCmd} por problemas que daba en algunos casos.}
% \changes{v3.40}{2017/12/11}{Se renombró la opción \texttt{exparriba} por \texttt{expandecuadro}.}
% \DoNotIndex{}
% \changes{v3.40}{2017/12/11}{Modificada la definición de \cs{ItemX} para que maneje la descripción especial para el archivo \texttt{csv}.}
% \changes{v3.40}{2017/12/11}{Nuevo comando \cs{ivaref} para referenciar el IVA del ítem etiquetado.}
% \changes{v3.40}{2017/12/11}{Se cambiaron unos \cs{AtEndDocument} por \cs{AtBeginShipoutNext}. Se hizo necesario siempre el paquete \texttt{eso-pic}.}
% \changes{v3.40}{2017/12/11}{Se quitó la fijación del estilo de página de un \cs{AtBeginDocument} innecesario. Nuevo entorno \texttt{reverso} para el reverso de la factura y nuevo comando para la letra predeterminada del entorno: \cs{LetraReverso}.}
% \changes{v3.40}{2017/12/11}{Cada ejemplo tiene una página al final con su explicación correspondiente.}
% \changes{v3.40}{2017/12/11}{Ahora en los ejemplos, primero va la opción o la macro entre paréntesis y después va la explicación y no al revés como era antes.}
% \changes{v3.40}{2017/12/11}{Quitada la opción \texttt{sincondiciones}. Si no se establecen las condiciones produce una advertencia.}
% \changes{v3.40}{2017/12/11}{Quitada la macro \cs{mostrar}. Ahora cada texto a mostrar se hace con el mismo comando con el que se definió.}
% \changes{v3.40}{2017/12/11}{Algunos condicionales cambiados.}
% \changes{v3.40}{2017/12/11}{Otra vez corregida la macro interna \cs{@DescripcionAuxCmd} y arreglado el espaciado entre los ítems; ojalá que esta vez sí esté bien.}
% \changes{v3.44}{2018/03/13}{Reestructurado el archivo \texttt{.cls} para aceptar prender y apagar las opciones de la clase a través de condicionales de \TeX{} con el mismo nombre de la opción.}
% \changes{v3.44}{2018/03/13}{Se cambió un \cs{AtBeginDocument} que escribía texto en la hoja por un \cs{AfterEndPreamble}.}
% \changes{v3.44}{2018/03/13}{Se acabó el descuento del IVA. Ahora advierte cuando se usa, posibilitando su uso después si llegan a volver a activar la medida.}
% \changes{v3.44}{2018/03/13}{Cada opción cargada escribe un comentario en el archivo \texttt{.log} para depuración.}
% \changes{v3.47}{2018/04/17}{No compilaba con \LuaLaTeX{} por diferencias en el motor. Ya compila.}
% \changes{v3.47}{2018/04/17}{Agrega la versión de la clase bajo \texttt{Subject} en la metadata de los PDF compilados.}
% \changes{v3.47}{2018/04/17}{Se agrega la macro \cs{Moneda} para cambiar el signo monetario a usar.}
% \changes{v3.47}{2018/04/17}{Se da cumplimiento al Decreto Presidencial 3.332 de la reconversión monetaria.}
% \changes{v3.47}{2018/04/17}{Cambiados algunos de los \cs{immediate}\cs{write} por \cs{write}.}
% \changes{v3.56}{2018/05/09}{Quitamos \texttt{lmodern} de todas partes y mejoramos la inclusión discrecional de \texttt{fontenc}.}
% \changes{v3.56}{2018/05/21}{Agregada la opción de una descripción larga para cada ítem de la factura únicamente para referenciarla luego en otra parte del documento.}
% \changes{v3.56}{2018/05/21}{La implementación anterior del Bolívar Soberano tenía defectos; ya están corregidos.}
% \changes{v3.56}{2018/05/21}{Volvimos a agregar la macro interna \cs{@TamDescripcion} quitada en la versión v3.40. Sin ella la línea no salía bien con la opción \texttt{filas}.}
% \changes{v3.56}{2018/05/21}{Creamos condicionales internos: \cs{if@ExpandeCuadro} y otros para evaluar la línea inutilizadora del espacio en blanco, así como también unas macros que guardan los valores de \cs{baselineskip} y \cs{tabcolsep} y otros cambios en la línea.}
% \changes{v3.56}{2018/05/21}{Nuevas opciones \texttt{sinlineahni}, \texttt{sinlineahtit} y \texttt{sinlineahtot} para no poner la línea horizontal antes de la nota interna, después del título de las columnas y antes de los totales.}
% \changes{v3.56}{2018/05/21}{Agregamos el paquete \texttt{anyfontsize}.}
% \changes{v3.57}{2018/08/02}{El Gobierno Nacional posterga la Reconversión Monetaria. Cambiamos un poco la forma como estaba programado esto. El usuario podrá cambiar las fechas que intervienen en eso, por si ocurre otro cambio.}
% \changes{v3.70}{2019/08/17}{Nueva tasa de IVA general del 16\% vigente desde agosto del 2018.}
% \changes{v3.70}{2019/08/17}{Establecido el 15/02/2019 como la fecha en la que ocurre que la nueva moneda pierde el adjetivo “soberano” para llamarse “bolívar” simplemente.}
% \changes{v3.70}{2019/08/17}{Nueva macro interna \cs{@EstableceOpciones} para poder evaluarla doblemente, una antes del -nuevo también- gancho \cs{BeforeEndPreamble} y otra después de él, porque en ese gancho se puede escribir código que altere las opciones y deberán ser evaluadas nuevamente para la consistencia entre ellas.}
% \changes{v3.70}{2019/08/17}{Inhabilitada la opción \texttt{electro} pero se deja el código todavía, aunque comentado.}
% \changes{v3.70}{2019/08/17}{Se comentó todo el código referente a la Reconversión Monetaria.}
% \changes{v3.70}{2019/08/17}{Implementada la función de que agregue el signo monetario en los totales, y una opción \texttt{sinmonedatotales} para evitarlo.}
% \changes{v3.70}{2019/08/17}{Ya no hablamos de condiciones sino de “nota final”, que es aquello que va después de la descripción larga y que tiene posición fija. Ese espacio se puede usar para distintas cosas, por lo que hablar de ese espacio como el de las condiciones del documento no tiene mucho sentido. De manera que los comandos renombrados pasaron a ser \cs{NotaFinal}, \cs{PosNotaFinal}, \cs{SepNotaFinal}, \cs{LineaNotaFinal}, \cs{LetraNotaFinal}. Se mantienen los comandos anteriores por compatibilidad.}
% \changes{v3.70}{2019/08/17}{Corregido el posicionamiento en “x” de la nota final.}
% \changes{v3.70}{2019/08/17}{Ahora la macro \cs{PosNotaFinal} acepta un intervalo en “x” para posicionar la nota final, es decir que funciona como \cs{PosFirmas}.}
% \changes{v3.70}{2019/08/17}{Mejorada la macro interna que escribe la fecha en la metadata; ya no usa \texttt{xstring} sino \LaTeX3. Quitado código espúreo de \texttt{xstring} en las macros que permiten fijar las posiciones de los elementos fijos.}
%
%
% \tableofcontents
%
%
% \section*{Brief description in english}
%
% |factura| is the spanish word for `invoice', so this is a \LaTeX{} class for typesetting and calculating invoices, made taking into account requirements of {\slshape SENIAT} legislation (tax collector entity on the Bolivarian Republic of Venezuela), but its use is not restricted to Venezuela because all variables and displayed text can be redefined by invoking commands or editing.
%
% \newpage
% \setstretch{1.4}
%
%
% \section{Introducción}
%
% La clase de \LaTeX{} |factura| está diseñada para su uso en la República Bolivariana de Venezuela, conforme a los requerimientos fijados por las providencias del {\slshape SENIAT} 0591, 0257, 0071 y 0048, válidas éstas para agosto del año 2019.
%
% Con |factura| se pueden elaborar facturas en formas libres fiscales\footnote{En cuanto a las formas libres, cada hoja del talonario de facturas legales tiene un encabezado con los datos del emisor de la factura y un pie de página con la información de identificación de la imprenta, en tamaño de hoja carta entera; esto es lo que se llama “forma libre”.}, cotizaciones y reportes de servicios, dejando todos los cálculos a \TeX{}.
%
% El diseño de las facturas hechas con esta clase consiste en un encabezado con los datos que identifican al cliente, con al lado un -opcional- resumen; estas partes de la factura están encerradas en recuadros. Si no se especifica el resumen saldrá solamente una de estas cajas en el encabezado que es la de los datos del cliente.
%
% Luego de esto viene un -opcional- recuadro con una nota donde puede ir la forma de pago, luego un cuadro con los campos: cantidad, descripción, precio unitario y precio total\footnote{Pueden haber variaciones en estos campos controlables con las opciones de la clase (véanse subsecciones \ref{subsec:columnas}, \ref{subsec:descuentos} y \ref{subsec:sincantidad}).}, con los elementos a facturar, cada uno en una línea, luego los totales, luego una -opcional- nota dentro del cuadro de la factura, otra nota (opcional también) después del cuadro de la factura y al final viene una opcional descripción larga que sale en un recuadro. La factura puede tener también un reverso con más información.
%
% En fin, la factura consta de:
% \begin{enumerate}[\bfseries i)]
%  \item el encabezado con los datos del cliente y el -opcional- resumen,
%  \item la opcional nota previa al cuadro de la factura,
%  \item el cuadro con los elementos que se están facturando y las totalizaciones,
%  \item la opcional nota interna al cuadro de la factura,
%  \item la opcional nota externa al cuadro de la factura,
%  \item una opcional descripción larga, e
%  \item información opcional en el reverso de la hoja.
% \end{enumerate}
%
% Aparte, el usuario puede escribir un reporte largo asociado a la factura, es decir, esta clase sirve también para escribir reportes.
%
% \newpage
%
%
% \section{Instalación}
%
% El paquete que está en CTAN consta de un archivo |README| en formato |UTF-8|, esta documentación compilada como |factura.pdf| y el archivo |factura.dtx|, del cual se generan todos los demás archivos.
%
% Al compilar |factura.dtx| va a producirse |factura.cls| y |factura.def|, además de todos los ejemplos en archivos |.tex| y |factura.ins|.
%
% \noindent Si queremos solamente generar los archivos sin compilar la documentación, debemos ejecutar:
%
% \noindent {\color{RoyalBlue}|pdftex factura.dtx|}
%
% \noindent Para generar los archivos y también compilar la documentación, debemos ejecutar los siguientes pasos:
%
% \begin{enumerate}[\bfseries 1. ]
%
% \item {\color{RoyalBlue}|xelatex factura.dtx|}\footnote{Es preferible compilar el |.dtx| con \XeLaTeX{} o con \LuaLaTeX pero también puede hacerse con pdf\LaTeX.}
%
% Luego de compilar |factura.dtx| hay que compilar el fondo de ejemplo, el membrete de ejemplo y la firma de ejemplo; estos tres archivos emulan haber digitalizado una hoja de una factura del talonario fiscal, un membrete para los reportes y también una firma que puede agregarse a los documentos a producir. Estos ejemplos se usarán en algunos de los ejemplos de formatos que incluímos y por eso es necesario tenerlos compilados en |.pdf| previamente, ya que su inclusión en los ejemplos es en formato compilado |.pdf|.
%
% El fondo y membrete de ejemplo los debemos compilar así:
%
% \item {\color{RoyalBlue}|pdflatex factura-ejemplo-fondo.tex|}
% \item {\color{RoyalBlue}|pdflatex factura-ejemplo-membrete.tex|}
%
% La firma escaneada de ejemplo necesita compilarse con \XeLaTeX{} porque son códigos en PSTricks y debe producirse un |.pdf|. Se compila así:
%
% \item {\color{RoyalBlue}|xelatex factura-ejemplo-firma.tex|}
%
% Al tener el fondo, el membrete y la firma compilados en |.pdf| según los pasos anteriores, podremos compilar cualquiera de los ejemplos en |.tex| que el archivo |factura.dtx| genera, y eso lo debemos hacer, ya que todos ellos (en su versión en |.pdf|) forman parte de la documentación. Los podemos compilar así:
%
% \item {\color{RoyalBlue}|pdflatex factura-ejemplo-prefactura.tex|}
%
% De esta forma como compilamos {\ttfamily factura-ejemplo-prefactura} compilamos todos los demás ejemplos: la cotización, las facturas del 1 hasta la última, el reporte del 1 hasta el último, y finalmente la factura para otra legislación.\footnote{Podemos también compilar con \LuaLaTeX{} o con \XeLaTeX{}.}
%
% Al tener todo esto hecho volvemos a compilar |factura.dtx| para producir esta documentación, puesto que el |.dtx| necesita ser compilado más de una vez para que la documentación quede correctamente elaborada:
%
% \item {\color{RoyalBlue}|xelatex factura.dtx|}
%
% Los comandos para generar el índice y la lista de cambios, que automáticamente se ejecutan al compilar |factura.dtx| son:
%
% \item {\color{RoyalBlue}|makeindex -s gind.ist -o factura.ind factura.idx|}
% \item {\color{RoyalBlue}|makeindex -s gglo.ist -o factura.gls factura.glo|}
%
% {\slshape No es necesario actualizar los índices con los dos comandos anteriores porque esos dos comandos los ejecuta el motor de compilación al actuar sobre }|factura.dtx|.
%
% \end{enumerate}
%
%
% \section{Archivos incluídos en el paquete}
%
% Este paquete incluye varios ejemplos, entre ellos la emulación de un diseño de factura entregado por alguna imprenta, que nos sirve para nuestros ejemplos de prefacturas, un diseño de membrete que nos sirve para los reportes, una firma que emula ser una firma real escaneada para su inclusión en algunos casos, así como varios ejemplos de facturas, prefacturas, cotizaciones y reportes. Cada uno de ellos tiene al principio del documento fuente la aclaratoria de qué es lo que hace, para diferenciarlo de los demás ejemplos.
%
% \noindent \begin{tabular}{rl}
%  \bfseries Archivo & \bfseries Descripción \\ \hline
%  |factura.dtx| & Archivo principal del que todos los demás se generan, \\
%  |factura.pdf| & Esta documentación, \\
%  |README| & Archivo con la descripción corta, \\
%  |factura-ejemplo-firma.tex| & Ejemplo de firma digitalizada, \\
%  |factura-ejemplo-fondo.tex| &  Elemplo de una forma libre fiscal digitalizada, \\
%  |factura-ejemplo-membrete.tex| & Ejemplo de membrete para los reportes, \\
%  |factura-ejemplo-aux.tex| & Archivo auxiliar que se incluye en todos los ejemplos y \\
%                            & sirve para que cada ejemplo conste de su descripción, \\
%  |factura-ejemplo-prefactura.tex| & Ejemplo de “{\slshape prefactura}”, \\
%  |factura-ejemplo-cotizacion.tex| & Ejemplo de cotización, \\
%  |factura-ejemplo-factura*.tex| & Ejemplos de facturas, \\
%  |factura-ejemplo-reporte*.tex| & Ejemplos de reportes, \\
%  |factura-beispiel-rechnung.tex| & Un ejemplo de una factura adaptada a otra legislación. \\ \hline
% \end{tabular}
%
% \IfFileExists{factura-ejemplo-prefactura.pdf}{%
%  \newpage\section{Ejemplo preliminar\label{sec:ejemplo}}
%  Lo que sigue es un ejemplo del aspecto final de una {\slshape “prefactura”}, que está bajo el nombre de “|factura-ejemplo-prefactura.tex|” y más adelante explicaremos en qué consiste:
%
%  \noindent
%  \mbox{}\hfill
%  \vfil
%  \fbox{\includegraphics[width=0.9\linewidth]{factura-ejemplo-prefactura.pdf}}
%  \hfill\mbox{}
%  \vfil\mbox{}
%  \newpage}
% {\message{^^JLaTeX Warning: No existe factura-ejemplo-prefactura.pdf; hay que compilarlo antes de compilar este documento.^^J}}
%
%
% \section{Geometría de la hoja\label{sec:geohoja}}
%
% El espacio que la imprenta deja en blanco en la forma libre está todo disponible para la impresión de la factura con esta clase. Este espacio se delimita con el paquete |geometry|. Los valores predeterminados que tienen que ver con la geometría de la hoja están todos definidos dentro del archivo |factura.def|.
%
% El margen superior que se da con |geometry| está contabilizado desde el inicio del papel\footnote{El papel predeterminado es tamaño carta.} hasta donde empieza el encabezado con los datos del cliente, es decir que el encabezado es el primer elemento desde arriba hacia abajo en una factura.
%
% El margen inferior de |geometry| empieza donde termina la descripción del servicio -o el fin del cuadro de la factura, en caso de que no se agregue un texto de descripción larga-, hasta el final del papel.
%
% Para aclarar lo anterior es necesario decir que hay textos con posiciones fijas que están fuera de estos márgenes, como la fecha de emisión, la “denominación” del documento (que es “FACTURA”, “COTIZACIÓN” u otro especificable), el número de la factura, el número de control (cuando aplica), las nota final y los espacios para las firmas; todas estas partes del documento están fijas en posiciones arbitrarias fuera de los márgenes dejados por |geometry|. Estos textos son fijos para darles posiciones que no dependan del contenido particular de cada factura sino del diseño del talonario que ya tenemos impreso, que siempre será igual, y para el cual estamos adaptando esta clase |factura|.
%
% Las ubicaciones de los textos fijos también tienen valores predefinidos dentro de |factura.def|. Cualquiera de esos valores puede cambiarse mediante comandos, tal como se fijan los valores predeterminados en dicho archivo |.def|; esto significa que se puede personalizar a cualquier diseño del talonario entregado por la imprenta, basta con entender qué hace cada comando (explicado en este documento), consultar el |.def| o ver los ejemplos.
%
% En algunos casos las formas libres tienen impreso el número de la factura y no es controlable esto por el emisor (como ocurre con el número de control que la legislación obliga a que siempre venga impreso en cada hoja del talonario). En otros casos el emisor tiene la libertad de escribir el número de la factura y controlarlo él mismo. Se puede hacer funcionar la clase según cualquiera de estas dos formas con una opción en |\documentclass|, es decir, podemos imprimir el número de la factura donde queramos, o también podemos asumirlo impreso en el talonario, según sea el caso del talonario que tengamos.
%
% Al usarse esta clase se deben primero hacer medidas en las facturas que la imprenta nos entrega para fijar los márgenes y determinar dónde poner los textos fijos que se deseen cambiar de la posición que viene predeterminada.
%
% El archivo |factura.def| tiene todas aquellas definiciones y valores que pueden ser ajustados; basta usarlos en nuestros documentos de la misma forma como están predefinidos allí; eso aplica no sólo para los de la geometría de la hoja sino para cualquier ajuste que esté en ese archivo o cualquier campo que tenga algún texto o palabra predefinida para la legislación venezolana.
%
%
% \section{Uso de la clase}
%
% El documento debe comenzarse con |\documentclass{factura}| con las opciones necesarias, que se explicarán en la sección \ref{sec:opciones}. Después se definen a través de macros {\bfseries en el preámbulo} todas las variables y dimensiones para personalizar el documento, esto es las posiciones de los textos fijos, tipos de letra (opcional), datos del cliente, un resumen (opcional), fecha y demás. Despues de |\begin{document}| va el cuadro de la factura propiamente, para lo cual la clase define un entorno llamado |factura|
% \DescribeEnv{factura}
% (|\begin{factura}...\end{factura}|), en cuyo interior se escriben únicamente los elementos a facturar a través de comandos del tipo |\Item|\footnote{Es posible incluir elementos gravados en IVA con alícuota general (12\%), alícuota reducida (8\%), alícuota adicional (27\%), así como también elementos exentos de impuesto al valor agregado, con \cs{Item}, \cs{ItemR}, \cs{ItemA} e \cs{ItemE} respectivamente.}, especificando su cantidad, descripción y precio unitario como argumentos, en ese orden.
%
% Finalizado el entorno se puede escribir (opcionalmente) una descripción larga o una aclaratoria de lo que se factura; esto se hace con el comando |\Descripcion|.
%
% Algunas de las posibilidades de cambio de estilo de esta clase consisten en que puede cambiarse la forma como es presentado cualquiera de los textos, se pueden suprimir las líneas verticales y horizontales del cuadro de la factura, se pueden agregar líneas horizontales separando cada ítem, también se puede dibujar una línea oblicua que proteja el cuadro de la factura de adiciones posteriores de texto, entre otras prestaciones más.
%
% Si el emisor lo desea puede digitalizar su firma personal y colocarla en el espacio ya dispuesto para firmar, con un ajuste mínimo para que salga correctamente sobre la línea.
%
% También el usuario tiene la posibilidad de producir una {\slshape “prefactura”}; así le llamamos a un documento sin validez legal pero que puede ser enviado por fax o por correo electrónico. Este documento tiene el mismo aspecto de la factura fiscal si se digitaliza la forma libre vacía\footnote{Quitándole las numeraciones obviamente. También puede diseñarse en un |.tex| una forma libre vacía idéntica a la que se tiene, así como se hace en el ejemplo incluído.}, se incluye la imagen como fondo de la página y se definen las posiciones fijas de las numeraciones impresas en el talonario, es decir el número de factura y el número de control, de los que hablamos previamente.
%
% Con esta misma clase tiene se la posibilidad de imprimir un información en el reverso de la hoja de la factura que saldrá escrita utilizando toda la hoja (los márgenes son ajustables). Esa información se encierra en un ambiente destinado para ese fin (|\begin{reverso}...\end{reverso}|),
% \DescribeEnv{reverso}
% y al momento de la compilación puede decidirse si se imprime esa parte o no.
%
% Otra posibilidad que da esta clase es la de escribir un reporte largo asociado a la factura\footnote{Si lo hubiera, así como también se puede usar estar clase para elaborar un reporte sin asociarlo a una factura en particular.}. Tal reporte se escribe dentro de un ambiente en el mismo documento en el que se escribe la factura asociada a él (|\begin{reporte}...\end{reporte}|),
% \DescribeEnv{reporte}
% y con dar una opción al inicio se puede hacer que se impirma el reporte en las páginas siguientes a la factura, o también se puede hacer que se imprima sólo el reporte ignorando todo lo que no constituye el reporte, de la misma manera como también se puede hacer que se imprima sólo la factura ignorando el entorno |reporte|. Esto permite tener todo en un mismo documento e imprimir lo que se desee en el momento, o todo a la vez.
%
%
% \section{Compilación de un documento\label{sec:compila}}
%
% Para compilar una |factura| puede usar cualquiera de los motores \LaTeX, pdf\LaTeX, \XeLaTeX{} o \LuaLaTeX, naturalmente con las restricciones que impone cada uno.
%
% Hay casos en los que hace falta compilar más de una vez el documento. Dos veces necesita ser compilado un documento cuando alguna descripción en el cuadro de la factura ocupa más que una línea; también dos veces cuando se incluye una firma en el reporte, y tres o cuatro veces cuando se pide que se dibuje una línea oblicua que inutilice el espacio restante en el cuadro de la factura o cuando se usan etiquetas (sección \ref{sec:etiquetas}).
%
% Todo documento |.pdf| compilado con la clase |factura| escribe en la {\slshape metadata} del |.pdf| un texto con la versión y la fecha de la clase con la que se compiló; ese texto va en el campo |Subject| de dicha {\slshape metadata}.
%
%
% \section{Opciones al cargar la clase\label{sec:opciones}}
%
%  Al cargar la clase con |\documentclass| podemos dar distintas opciones según el comportamiento que queramos; dichas opciones están descritas en esta sección y cualquier opción cargada se verá como comentario en el archivo |.log| de la compilación bajo el texto “{\ttfamily Opción cargada: }” con el nombre de la opción pedida.
%
%
% \subsection{Sobre el tipo de documento a producir}
%
% {\setlength{\parindent}{0pt}
%
% \DescribeOption{prefactura}
%  La {\slshape “prefactura”} es un documento no legal, preliminar, que no se imprime sobre papel de factura sino que está concebido para enviarse por correo o fax. Para eso se define la imagen que contiene el diseño de la factura con el comando |\FondoFactura|, luego la clase escribe la denominación “FACTURA” con su número\footnote{Lo predeterminado cuando no es una {\slshape “prefactura”} sino una factura final es no escribir la denominación ni el número, ya que eso generalmente viene impreso en cada hoja del talonario.}. Esta opción implica la opción |sinfirmas|, ya que si es un documento preliminar sin validez legal, las firmas no son necesarias; sin embargo se puede cambiar este comportamiento con las opciones |dosfirmas| y |unafirma|.
%
% \DescribeOption{cotizacion}
%  La cotización es un documento similar a una factura pero no es un documento legal en el sentido de que no se imprime sobre talonario fiscal alguno, así que no lleva número de control y puede o no tener su propia numeración, que es cosa del emisor de la cotización. Esta opción implica la opción |unafirma|, para que firme el emisor de la factura, y si su nombre no se especifica saldrá “Por el emisor” debajo de la línea para firmar. También puede cambiarse esto con la opción |dosfirmas| o con |sinfirmas|.
%
% \DescribeOption{sinreverso}
%  No escribe la información -opcional- que se escribió dentro del entorno |reverso| a salir en el reverso de la factura.
%
% \DescribeOption{conreporte}
%  Genera la factura (prefactura o cotización) y luego el reporte.
%
% \DescribeOption{soloreporte}
%  Genera solamente el reporte, ignorando lo relacionado con la factura. Si no se especifica |conreporte| ni |soloreporte| se va a imprimir únicamente la factura.
%
%
% \subsection{Relativo a las columnas que puede tener el cuadro de la factura\label{subsec:columnas}}
%
% \DescribeOption{numitem}
%  Con la opción |numitem| una nueva columna aparece en el cuadro de la factura donde se enumera cada ítem en el órden en el que se escriben dentro del entorno para la factura. Esto es útil por ejemplo cuando se hace referencia al número del ítem con el comando |\itemref|. Véase la sección \ref{sec:etiquetas}.
%
% \DescribeOption{sincantidad}
%  El ambiente |factura| tiene también la posibilidad de no mostrar el campo de la cantidad, ni aceptarlo como argumento de cada ítem; en ese caso los comandos |\Item| -de los que hablaremos en la sección \ref{sec:factura}- aceptan como argumentos solamente la descripción del elemento y su precio, en ese orden.
%
% \DescribeOption{descuentos}
%  Permite agregar un descuento en porcentaje en cada elemento de la factura, siendo dicho porcentaje el último argumento de los comandos |\Item|. Si no aplica descuento para un ítem, puede este dejarse sin ese argumento, dando el mismo resultado que indicar “0” como argumento para ese descuento. (Explicado también en la sección \ref{subsec:descuentos}).
%
% \DescribeOption{subtotal}
%  Agrega una columna extra en la factura; sólo aplica cuando también se usa la opción |descuentos|. La nueva columna escribe el subtotal que cada ítem produce, es decir, el precio unitario multiplicado por la cantidad sin aplicar el descuento. Esta opción sobrecarga la factura (precio unitario, descuento, subtotal y precio total), y su uso es recomendable cuando también se usa |sincantidad|, porque de esta forma se tiene el subtotal, el descuento, y el total.
%
% \DescribeOption{iva}
%  Agrega una nueva columna donde se expresa el porcentaje del IVA para cada elemento en la factura. Sin especificar esta opción queda claro según lo establecido por nuestra legislación cuál será el gravamen que aplica para cada elemento por una letra que lo denota al lado del precio o de la descripción.
%
% \DescribeOption{duc}
%  Cambia el orden de los campos en el documento impreso a: descripción, precio unitario, cantidad y precio total. {\itshape Esto no cambia la forma como deben especificarse los argumentos de |\Item| sino solamente cambiará cómo serán mostrados}.
%
% \DescribeOption{dcu}
%  Cambia el orden de los campos en el documento impreso a: descripción, cantidad, precio unitario y precio total\footnote{Sin \texttt{duc} ni \texttt{dcu}, el modo predeterminado es \texttt{cdu}: cantidad, descripción, precio unitario.}.
%
%
% \subsection{Relativo al IVA}
%
% \DescribeOption{siniva}
%  No se cobra IVA, para los casos en los que eso aplique.
%
% \DescribeOption{sinivaexpresado}
%  Al no cobrar IVA, tampoco aparece en la totalización los IVA ni las bases imponibles.
%
% \DescribeOption{todosiva}
%  Agrega el monto exento y todos los tipos de IVA en los totales así no se hayan incluído elementos gravados con los distintos tipos de impuesto.
%
% \DescribeOption{electro}
%  {\bfseries Esta opción ya no está vigente en la legislación y se quitó de esta clase.} Se agregó en su momento esta opción para dar cumplimiento con las Gacetas Oficiales números 41239, 41242 y 41244\footnote{De fecha 19, 21 y 26 de septiembre del año 2017, respectivamente. En la última de estas gacetas está la Providencia Administrativa PA0048 del SENIAT.} donde se establece que las operaciones de compra (salvo excepciones establecidas en la Gaceta) gravadas con IVA general gozarán de un descuento cuando son pagadas mediante medios electrónicos. El descuento es del 3\% cuando la factura es de un monto menor a Bs.F 2.000.000 y del 5\% cuando el monto de la factura es de Bs.F 2.000.000 o más. Al dar esta opción el descuento queda establecido según el Decreto Presidencial número 3085.
%
% {\itshape Esta opción no es compatible con el modo de totalización en filas.}
%
% \DescribeOption{G}
%  Agrega una “|G|” al lado de todos los items gravados con IVA general así sea el general el único IVA existente en la factura.
%
% \DescribeOption{impdesc}
%  Con esta opción, la letra que indica la denominación del impuesto aplicado a cada elemento, en lugar de escribirse en el precio unitario y en el total, que es la forma predeterminada, sale escrito al final de la descripción. Puede preferirse de esta forma, y ahorra un poco de espacio usable en la descripción porque ya no sale en dos lugares dicha letra sino en uno solo.
%
%
% \subsection{Sobre las líneas en el cuadro de la factura}
%
% \DescribeOption{sinlineasve}
% \DescribeOption{sinlineasvi}
% \DescribeOption{sinlineasv}
% \DescribeOption{sinlineashe}
% \DescribeOption{sinlineashi}
% \DescribeOption{sinlineash}
%  La clase provee varias opciones para no dibujar las líneas del cuadro de la factura. La clase diferencia entre las líneas internas y las líneas externas. Las líneas externas son las que conforman el contorno externo del cuadro de la factura. Las líneas internas en el caso de las horizontales son la línea que va después del título de las columnas (cantidad, descripción, ...), la línea que separa los ítems de los totales y la línea que separa los totales de la nota interna; en el caso de las verticales, las líneas internas son las que separan las columnas.
%
%  Podemos apagar cada conjunto de línea; ya explicado qué es interno y qué es externo, queda aclarar que “ve” significa vertical externa, “vi” vertical interna, “he” horizontal externa y “hi” horizontal interna. Usando |sinlineasv| o |sinlineash| apagamos todas las líneas verticales u horizontales.
%
% \DescribeOption{sinlineahtit}
% \DescribeOption{sinlineahtot}
% \DescribeOption{sinlineahni}
%  Adicionalmente podemos apagar las líneas horizontales internas por separado, es decir, la línea que va después del título de las columnas la podemos quitar con \texttt{sinlineahtit}, la línea que separa los elementos a facturar de los totales generales se puede quitar con \texttt{sinlineahtot} y la línea que separa los totales de la -opcional- nota interna la podemos quitar con \texttt{sinlineahni}. 
% 
% \DescribeOption{conlineasha}
%  Agrega una línea horizontal separando cada artículo o ítem de los demás dentro del cuadro de la factura, como las facturas hechas a mano que tienen un espacio determinado para cada artículo.
%
%  Si no se especifica nada sobre las líneas, saldrán todas ellas (verticales y horizontales) salvo las que separan  los ítems a facturar entre sí.
%
%
% \subsection{Elementos a mostrar o no mostrar en la factura}
%
% \DescribeOption{sinmarcas}
%  Con |sinmarcas| evitamos que el carácter de fin de línea aparezca cuando la descripción ocupa más que una línea; ese carácter indica que la línea continúa abajo, y predeterminadamente es la flecha “\flechaCR”. También con esta opción no aparece el carácter “$\cdot$” en las celdas vacías en este mismo caso.
%
% \DescribeOption{sinmonedatotales}
%  Esta opción |sinmonedatotales| permite quitar el signo monetario de los totales, que antes nunca salía y ahora sale de manera predeterminada.
%
% \DescribeOption{unafirma}
%  No pone el espacio para que firme quien recibe el documento sino sólo el emisor; esto es lo predeterminado en las cotizaciones.
%
% \DescribeOption{sinfirmas}
%  No pone ninguna línea ni espacio para firmar; esto es lo predeterminado para las prefacturas.
%
% \DescribeOption{dosfirmas}
% Coloca las dos líneas para que firme quien emite y quien recibe. Esta opción es la predeterminada salvo en las prefacturas y en las cotizaciones.
%
% \DescribeOption{vencimiento}
%  Cuando establecemos los días del crédito de la factura, se imprime una línea también con la fecha de vencimiento de la factura. Con esta opción podemos pedir que la fecha de vencimiento salga siempre en la facturas, inclusive cuando no sean facturas a crédito.
%
% \DescribeOption{denom}
%  Con esta opción escribimos también el texto “FACTURA” con el número. La razón de esta opción es que hay formas libres que tienen la numeración de la factura y otras que solamente tienen el número de control; para las que tienen sólo el número de control existe esta opción que también escribe el número en cada factura impresa, en la posición que se defina.
%
% \DescribeOption{sinnumero}
%  Escribe el texto de la denominación del documento en su posición (e.g. “COTIZACIÓN”) pero sin escribir el número; por ejemplo para las cotizaciones que no son documentos que legalmente lleven número, aunque el número puede ser para control interno. Esta opción implica la opción anterior |denom|.
%
% \DescribeOption{sintotales}
%  Con esta opción evitamos que se totalicen los elementos en la factura; sólo se presenta el cuadro sin los totales. La utilidad de esto no es en las facturas sino en las cotizaciones para casos donde se muestran variantes del mismo producto para que el cliente elija entre ellos. No tendría sentido totalizar allí sino que funcione como un listado de precios. {\itshape Debe usarse en las cotizaciones porque una factura sin totales no es válida.}
%
% \DescribeOption{linea}
%  Agrega una línea diagonal que inutiliza el espacio vacío en el cuadro de la factura, para evitar posibles futuras adiciones de texto. Esta opción requiere de la opción |expandecuadro|, explicada a continuación.
%
%
% \subsection{Disposición de algunos elementos en la factura}
%
% \DescribeOption{expandecuadro}
%  Expande el cuadro de la factura con espacio en blanco, es decir, rellena el espacio restante del cuadro de la factura con espacio en blanco. Lo predeterminado es expandir la parte de la descripción larga cuando la hay.
%
% \DescribeOption{sinexpandir}
%  No expande ninguna parte en la factura, es decir, deja tanto el cuadro de la factura como la descripción en su tamaño mínimo según el contenido que tenga.
%
% \DescribeOption{centradesc}
%  Podemos centrar el texto de la descripción de cada ítem en la factura con este comando en lugar de justificarlo.
%
% \DescribeOption{filas}
%  Muesta los totales finales en modo de filas; esto los presenta uno al lado del otro en lugar de uno debajo del otro. Esta opción tiene limitaciones, por ejemplo no es compatible con la opción |sintotales|, |sinivaexpresado|, y por razones de espacio tampoco es compatible con los comandos de descuentos en los totales (subsección \ref{subsec:dctototales}).
%
% \DescribeOption{centrado}
%  Centra los totales finales ({\itshape sólo para el modo de totalización en filas}), que sin esta opción salen alineados al margen derecho.
%
% \DescribeOption{agrupado}
%  No agrupa los totales finales de los items por gravamen sino por tipo ({\itshape sólo para el modo de totalización en columnas}). Con esta opción la agrupación es primero todos los subtotales y los descuentos (cuando aplica), luego las bases imponibles y después los IVA. Sin |agrupado| se ordena por porcentaje del gravamen.
%
%
% \subsection{Otras opciones}
%
% \DescribeOption{nospanish}
%  No carga |babel| en castellano\footnote{Por ejemplo para cargar {\ttfamily polyglossia} en su lugar.}, ni la codificación |T1| de las letras en el caso de compilar con pdf\LaTeX. Si no se especifica esta opción se cargan ambos paquetes.
%
% \DescribeOption{twoside}
%  Establece que el reporte se imprime por los dos lados del papel. (Obviamente se ignora esta opción cuando el motor no va a generar un reporte puesto que las facturas ocupan una hoja solamente).
%
% \DescribeOption{csv}
%  Escribe un archivo csv con la información numérica de la factura.
%
% \subsection{Activación y desactivación de opciones desde comandos\label{subsec:condicionales}}
%
% Cualquiera de estas opciones también se puede activar o desactivar en el preámbulo y así poder incluirlas dentro de archivos que vayan a ser cargados con |\input| o |\include| como se explicará en la subsección \ref{subsec:aux}. Esto significa que podremos desactivar una opción que hayamos dado en |\documentclass| o activar alguna que no hayamos activado allí.
%
% Para hacer eso debemos usar el condicional que cada opción maneja y activarlo o desactivarlo de la forma como se hace en \TeX{}. El condicional de cada opción tiene el mismo nombre de la opción. Por ejemplo, para quitar las líneas verticales (|sinlineasv|) usaríamos |\sinlineasvtrue| en el preámbulo, o para desactivar la creación del archivo |.csv|, previamente activada en |\documentclass|, usaríamos |\csvfalse|.
%
% }
%
%
% \section{Comandos que la clase provee\label{sec:comandos}}
%
% A continuación descriremos los principales comandos de esta clase; en rojo aparecen aquellos pocos comandos que son de obligatorio uso. Los que no son de obligatorio uso es porque tienen algún valor predeterminado en |factura.def| o porque controlan algún aspecto opcional.
%
%
% \subsection{Estableciendo los márgenes de la hoja y el fondo}
%
%  Como explicamos en la sección \ref{sec:geohoja}, a la hoja de la factura se le deben establecer los márgenes apropiadamente, y esos márgenes no tienen en cuanta ciertos textos que llamamos “fijos”, para los cuales también debemos establecer sus posiciones en la hoja, y eso lo veremos en esta parte.
%
%
% {\setlength{\parindent}{0pt}
%
% \DescribeMacro{\GeometriaFactura}{\macrooptcolor|\GeometriaFactura|\marg{geometría}}
%
%  En el argumento de este comando van los parámetros relativos al paquete |geometry| que van a afectar la manera como la factura es generada. Como dijimos al principio, deberá especificarse el tipo de papel (e.g. |letterpaper|, |legalpaper|, |executive|, |a4paper|, ...) y los márgenes verticales y horizontales.
%
%  Para especificar el margen superior debe medirse desde el inicio del papel hasta donde se quiera establecer que empiece el recuadro de los datos del cliente, y el margen inferior debe medirse desde el final del papel hasta donde terminaría la opcional descripción. Como ya dijimos, hay textos que están fuera de estos márgenes por haberlos considerado fijos en la implementación de la clase |factura|.
%
% \DescribeMacro{\GeometriaReverso}{\macrooptcolor|\GeometriaReverso|\marg{geometría}}
%
%  Define los márgenes del reverso de la hoja de la factura para escribir información extra.
%
% \DescribeMacro{\FondoFactura}{\macrooptcolor|\FondoFactura|\marg{nombre del archivo de la imagen}}
%
%  Nombre del archivo de la imagen a cargar como diseño, especialmente para las {\slshape prefacturas}, aunque también puede usarse como una “marca de agua” para las facturas.
%
%  Lo que recomendamos es digitalizar la forma libre o pedir a la imprenta un archivo digital con el diseño y así incluir esa imagen con este comando cuando se vayan a producir “prefacturas” o cotizaciones, así las prefacturas tienen el mismo diseño que las facturas. Si se puede lograr el diseño en un código como se hace con el ejemplo del fondo, pues mejor aún.
%
%
% }
%
%
% \subsubsection{Elementos fijos y sus ubicaciones}
%
% Los elementos cuyas posiciones van fijas en la hoja disponen de macros para establecer sus ubicaciones. Todas estas macros empiezan por |\Pos|, y son las siguientes: |\PosFecha|, |\PosDenom|, |\PosControl|, |\PosNotaFinal| y |\PosFirmas|.
%
% Todos estos comandos aceptan como argumentos dimensiones de \LaTeX{}, y son las posiciones |(x,y)| en la página de los elementos a imprimir, siendo “|x|” la coordenada a lo ancho del papel y “|y|” la coordenada a lo largo del papel, es decir, horizontal y vertical respectivamente. El origen está fijado arriba a la izquierda. Se pueden especificar también valores negativos para denotar que estamos contando desde el extremo derecho o inferior, para el caso de “|x|” o “|y|”, respectivamente\footnote{Varias de estas variables están definidas de esta forma negativa en el archivo de definiciones.}; salvo el caso de la nota final y las firmas en su coordenada |x|, todos los demás elementos fijos aceptan valores negativos en su posicionamiento, ya que estos dos elementos cuentan con un intervalo |x-inicial,x-final| para posicionar.
%
% Por ejemplo, si queremos que la fecha aparezca desde el margen izquierdo y |5cm| contados desde arriba hacia abajo, haríamos: |\PosFecha{0cm}{5cm},| y si queremos que salga |1cm| desde la derecha y |4cm| contados desde abajo haríamos |\PosFecha{-1cm}{-4cm}|. A continuación todas las macros para fijar las posiciones en el papel:
%
% {\setlength{\parindent}{0pt}
%
% \DescribeMacro{\PosFecha}{\macrooptcolor|\PosFecha|\marg{pos. x}\marg{pos. y}}
%
%  Con |\PosFecha| controlamos dónde se ubica la caja que imprime la fecha en la hoja.
%
% \DescribeMacro{\PosFecha*}{\macrooptcolor|\PosFecha*|\marg{pos. x}\marg{pos. y}}
%
%  Con esta variante con “*”, es decir, |\PosFecha*|, logramos que la caja de la fecha abra hacia arriba desde la posición especificada en el eje vertical, de manera que estaríamos especificando una coordenada “|y|” final del texto y no una posición inicial, esto significa que el texto se despliega hacia arriba desde la posición dada y no hacia abajo. Por ejemplo, sin “*”, si especificamos |\PosFecha{-0cm}{5cm}|, las fechas empiezan 5cm desde el tope superior de la hoja, y si especificamos |\PosFecha*{-0cm}{5cm}| ellas terminarán en los 5cm de la hoja. Esto es útil cuando se especifica también |\Credito| que produce una fecha de vencimiento que aparecerá abajo de la fecha de emisión; al ocupar las fechas más de una línea podemos querer que la última fecha (la de vencimiento) quede alineada en cierta posición vertical, la misma posición donde saldría la fecha de emisión en el caso en el que no haya fecha de vencimiento por no especificar la factura a crédito.
%
% \DescribeMacro{\PosDenom}{\macrooptcolor|\PosDenom|\marg{pos. x}\marg{pos. y}}
%
%  Con |\PosDenom| controlamos dónde se ubica la caja que imprime la denominación del documento (e.g. “FACTURA”, “COTIZACIÓN”) y su -opcional- número.
%
% \DescribeMacro{\PosDenom*}{\macrooptcolor|\PosDenom*|\marg{pos. x}\marg{pos. y}}
%
%  Al igual que con |\PosFecha*| podemos especificar la posición final en “|y|” de la caja para la denominación, así el texto se despliega hacia arriba y no hacia abajo.
%
% \DescribeMacro{\PosControl}{\macrooptcolor|\PosControl|\marg{pos. x}\marg{pos. y}}
%
%  Con |\PosControl| controlamos dónde se ubica la caja que imprime el número de control con la serie.
%
% \DescribeMacro{\PosControl*}{\macrooptcolor|\PosControl*|\marg{pos. x}\marg{pos. y}}
%
%  También podemos pedir que la caja del número de control abra hacia arriba, es decir que la posición vertical sea tomada como la posición vertical final; para eso |\PosControl*|, igual que con |\PosFecha*| y |\PosDenom*|. 
%
% \DescribeMacro{\PosNotaFinal}{\macrooptcolor|\PosNotaFinal|\marg{pos. x-inicial}\marg{pos x-final}\marg{pos. y}}
%
%  Con |\PosNotaFinal| controlamos dónde se ubica el texto de la nota final\footnote{La nota final es lo que en versiones previas llamábamos “condiciones” de la factura} (está hecho con |\parbox|, acepta un párrafo como argumento) en la hoja.
%
% {\itshape Aquí delimitamos el intervalo en |X| donde va la nota final, por eso hay una coordenada |X-inicial| y una coordenada |X-final|.} La coordenada |X| inicial va medida desde el margen establecido como el izquierdo, y la |X| final va medida desde el margen establecido como el derecho. Por ejemplo, estableciendo |X-inicial| como |0pt| y |X-final| como |0pt| queda en los mismos márgenes que todos los demás elementos de la factura, y esto es lo predeterminado, pero estableciendo ambas coordenadas como |1cm|, la nota final quedará con |1cm| de margen hacia adentro que la factura, tanto por la izquierda como por la derecha.
%
% Debido a que este comando acepta valor inicial y valor final en |X|, entonces no acepta valores negativos.
%
% \DescribeMacro{\PosNotaFinal*}{\macrooptcolor|\PosNotaFinal*|\marg{pos. x-inicial}\marg{pos x-final}\marg{pos. y}}
%
%  De la misma manera como con los comandos precedentes, podemos especificar que la caja del párrafo para la nota final abra hacia arriba, fijando una posición vertical final y no una posición inicial de la nota final.
%
% \DescribeMacro{\PosFirmas}{\macrooptcolor|\PosFirmas|\marg{pos. x-inicial}\marg{pos. x-final}\marg{pos. y}}
%
%  Con |\PosFirmas| controlamos dónde se ubica la caja que dibuja las líneas para firmar en la factura. {\itshape Aquí aplica el mismo comentario que en \cs{PosNotaFinal} sobre las coordenadas |X| inicial y final}. La coordenada |X-inicial| indica dónde empieza la primera linea (para que firme quien recibe), y la coordenada |X-final| establece dónde termina la segunda linea para firmar (para que firme el emisor de la factura), que va a estar al lado de la primera.
%
% En este caso, las coordenadas |X| tampoco aceptan valores negativos ni versión con asterisco |*|.
%
% }
%
%
% \subsubsection{Espaciado entre distintas partes. Estilo de los recuadros\label{subsec:espacioslineas}}
%
% Podemos cambiar distintos valores del espaciado interno en la factura, así como también la estética de la línea para firmar y los recuadros que encierran los datos del cliente, la descripción, las “notas”, pudiendo ajustarle estos recuadros o cajas el grosor de la línea, la curvatura y la sombra. Los argumentos de todos estos comandos son dimensiones de \LaTeX{}:
%
% {\setlength{\parindent}{0pt}
%
% \DescribeMacro{\SepVertical}{\macrooptcolor|\SepVertical|\marg{sep. vertical entre elementos}}
%
%  Las distintas partes de la factura (a saber: el encabezado con los datos, la -opcional- nota previa, el cuadro de la factura, la -opcional- nota externa y la -opcional- descripción) también están separadas entre sí una distancia que puede ser cambiada.
%
%  La separación que se fija con este comando es la separación vertical mínima que queremos garantizar entre cada elemento a escribir en la factura, y puede ser ajustada automáticamente por la clase aumentándola cuando no hay relleno automático de espacios sobrantes (es decir, sin usar la opción |sinexpandir|). Cuando la descripción larga o el cuadro de la factura (usando la opción |expandecuadro|) toman su tamaño máximo, entonces la separación vertical entre cada elemento de la hoja es la dada con |\SepVertical|, o sea, la que queremos garantizar como la mínima separación entre esas partes.
%
% \DescribeMacro{\SepDatosResumen}{\macrooptcolor|\SepDatosResumen|\marg{sep. horizontal}}
%
%  Esta macro nos proporciona una forma de cambiar la separación entre los recuadros o cajas donde van los datos del cliente y el resumen que sale a su lado. Naturalmente, si no se escribe un resumen saldría únicamente la caja de los datos del cliente, y esta separación no tendría efecto.
%
% \DescribeMacro{\SepEncabezado}
% \DescribeMacro{\SepNotaPrevia}
% \DescribeMacro{\SepDescripcion}
% \DescribeMacro{\SepNotaFinal}
% {\macrooptcolor|\SepEncabezado|\marg{sep. interna texto}}\newline
% {\macrooptcolor|\SepNotaPrevia|\marg{sep. interna texto}}\newline
% {\macrooptcolor|\SepDescripcion|\marg{sep. interna texto}}\newline
% {\macrooptcolor|\SepNotaFinal|\marg{sep. interna texto}}
%
%  Con estas macros podemos controlar la distancia entre la línea del recuadro y el texto que hay en el recuadro, es decir, un margen interno entre el contorno de la caja y el texto para que no salga pegado a la línea que lo encierra. Esto es parte del diseño de \tikz de estos recuadros. Como puede verse, dicho margen para los datos del cliente y para el resumen se cambia con |\SepEncabezado|, los de la nota previa con |\SepNotaPrevia|, los de la descripción larga con |\SepDescripcion|, y los de la nota final con |\SepNotaFinal|.
%
% Predeterminadamente no hay línea ni diseño de \tikz para la nota final del documento, así que el valor fijado aquí no tiene efecto a menos de que se le especifiquen valores a dicho recuadro con la opción |\LineaNotaFinal|, explicada más abajo.
%
% \DescribeMacro{\SepItemsExtra}{\macrooptcolor|\SepItemsExtra|\marg{sep. extra entre items}}
%
%  Agrega una separación extra entre los elementos o ítems en el cuadro de la factura. La separación entre ítems está dada por el |\baselineskip| que se puede establecer con el comando |\LetraItems|, el cual se explicará más adelante en esta misma sección.
%
% \DescribeMacro{\SepFilas}{\macrooptcolor|\SepFilas|\marg{sep. en x}\marg{sep. en y}}
%
%  Cuando se usa el modo de totalización en filas tenemos también la posibilidad de cambiar la separación entre los totales que salen uno al lado del otro, y la separación entre las distintas líneas.
%
%  El primer argumento de |\SepFilas| es la separación horizontal {\itshape mínima} entre los distintos totales que se presentan en el modo de totalización en filas. La clase |factura| expande esa distancia para que queden todos los totales separados igual, y también separados igual de los márgenes del papel.
%
%  El segundo argumento de |\SepFilas| es la separación vertical entre las distintas líneas de la totalización en filas.
%
% \DescribeMacro{\LineaEncabezado}
% \DescribeMacro{\LineaNotaPrevia}
% \DescribeMacro{\LineaDescripcion}
% \DescribeMacro{\LineaNotaFinal}
% {\macrooptcolor|\LineaEncabezado|\marg{grosor}\marg{curvatura esquinas}\marg{tamaño sombra}}\newline
% {\macrooptcolor|\LineaNotaPrevia|\marg{grosor}\marg{curvatura esquinas}\marg{tamaño sombra}}\newline
% {\macrooptcolor|\LineaDescripcion|\marg{grosor}\marg{curvatura esquinas}\marg{tamaño sombra}}\newline
% {\macrooptcolor|\LineaNotaFinal|\marg{grosor}\marg{curvatura esquinas}\marg{tamaño sombra}}
%
%  La información del encabezado (datos del cliente y resumen), así como la nota previa, la descripción y la nota final puede o no tener el diseño \tikz, el cual viene prehecho en el archivo |.def| y consiste en una línea curva, en algunos casos sombreada a la derecha y abajo, y en otros casos no. Esa línea puede ser alterada en su grosor, el tamaño de la sombra, y la distancia entre la línea y la información, es decir un margen interno. Para eso estos comandos, que se comportan igual todos ellos.
%
%  Con el primer argumento de estos comandos cambiamos el grosor de la línea\footnote{Si se especifica una dimensión nula, desaparece el diseño \tikz y la línea, quedando solamente un \cs{parbox}, y es así como se estableció en el archivo \texttt{.def} para el caso de la nota final.} con la que las cajas del encabezado, la nota previa, la descripción y la nota final se dibujan. Con el segundo argumento cambiamos el radio de curvatura de las esquinas, y con el tercero el tamaño de la sombra\footnote{Todas las sombras salen hacia abajo a la derecha.}.
%
%  Se puede modificar el diseño de este nodo de \tikz; está alojado en las macros internas |\@DatosTIKZ|, |\@NotaPreviaTIKZ|, |\@DescripcionTIKZ| y |\@NotaFinalTIKZ|.
%
%  Lo predeterminado es que la nota final se escriban sin diseño de \tikz, es decir, el primer argumento de este comando está fijado en cero para que no salga línea ni diseño de \tikz.
%
% \DescribeMacro{\LineaFirmas}{\macrooptcolor|\LineaFirmas|\marg{long. línea}\marg{grosor}}
%
%  La longitud de las líneas para firmar y el grosor son también ajustables por medio del primer y segundo argumento, respectivamente, de la macro |\LineaFirmas|.
%
%
% \subsection{Sobre la información del cliente\label{subsec:cliente}}
%
% \DescribeMacro{\RazonSocial}{\macromandcolor|\RazonSocial|\marg{texto}}
%
%  Define la razón social, o sea, el nombre del cliente.
%
% \DescribeMacro{\Nombre}{\macromandcolor|\Nombre|\marg{texto}}
%
%  Similar a |\RazonSocial| pero usable más bien para personas naturales, donde no existe una “persona de contacto” y por tanto la clase no advierte de su ausencia si no se ha especificado con |\Contacto|. Naturalmente se especifica un solo nombre del cliente, es decir que si se especifica con |\RazonSocial| no hace falta usar este comando. {\itshape Es preferible usar |\RazonSocial|.}
%
% \DescribeMacro{\RIF}{\macromandcolor|\RIF|\marg{texto}}
%
%  Define el R.I.F. del cliente.
%
% \DescribeMacro{\CI}{\macromandcolor|\CI|\marg{texto}}
%
%  Especifica la cédula de identidad del cliente al ser persona natural. {\itshape Es preferible usar siempre |\RIF|.}
%
% \DescribeMacro{\Contacto}{\macrooptcolor|\Contacto|\marg{texto}}
%
%  Define una persona de contacto (e.g. nombre de una persona cuando el cliente no es personal).
%
% \DescribeMacro{\Direccion}{\macromandcolor|\Direccion|\marg{texto}}
%
%  Define la dirección fiscal del cliente.
%
% \DescribeMacro{\Telefono}{\macrooptcolor|\Telefono|\marg{texto}}
%
%  Especifica el (o los) teléfonos del cliente.
%
% \DescribeMacro{\Email}{\macrooptcolor|\Email|\marg{texto}}
%
%  Establece el e-mail del cliente.
%
% \DescribeMacro{\Conforme}{\macrooptcolor|\Conforme|\marg{texto}}
%
%  Especifica el texto que aparecerá debajo de la línea para que firme el cliente; si no se especifica tomará el texto por omisión, que es “Recibí conforme”, indicando que el cliente le firma una copia al emisor de la factura donde acepta haber recibido y estar conforme. Este texto puede cambiarse a otro que puede ser el nombre del cliente o la persona de contacto, o también “Por el cliente”.
%
%
% \subsection{Sobre la información del emisor del documento}
%
% \DescribeMacro{\Emisor}{\macrooptcolor|\Emisor|\marg{texto}}
%
%  Define el nombre del emisor de la factura. Si no se especifica saldrá el texto “PAGADO”\footnote{El texto “PAGADO” es lo que la legislación venezolana del SENIAT exige escribir cuando una factura ha sido pagada.} debajo de la línea para firmar, ya que la factura se firma cuando haya sido pagada. Si se especifica un texto con este comando saldrá ese texto debajo de la línea donde va la firma, pero el propósito es que quede constancia para el cliente que ya pagó la factura, razón por la cual el texto predeterminado es “PAGADO”.
%
% \DescribeMacro{\Proveedor}{\macrooptcolor|\Proveedor|\marg{texto}}
%
%  Establece el nombre del emisor, para los efectos del reporte. Si no se define mediante esta macro, será tomado del establecido con |\Emisor|.
%
% \DescribeMacro{\FirmaFactura}{\macrooptcolor|\FirmaFactura|\marg{inclusión de la imagen}\oarg{corrimiento en x}\oarg{corrimiento en y}}
%
%  Incluye el archivo de la firma digitalizada en el espacio donde firma el emisor del documento. Los argumentos son:
%  \begin{enumerate}
%   \item inclusión de la imagen (e.g. |\includegraphics|). {\bfseries Aquí el primer argumento no es el archivo de la imagen sino el comando que la inserta en el documento}, así el usuario tiene más control sobre cómo saldrá la imagen si altera las opciones de |\includegraphics|.
%   \item corrimiento en |X| de la imagen, para un ajuste más fino de la posición de la firma,
%   \item corrimiento en |Y|.
%   \end{enumerate}
%  {\itshape (Sin especificar los corrimientos, la firma quedará centrada horizontalmente en la línea, dejando hacia la izquierda y hacia la derecha la misma cantidad de la línea. También quedará toda la firma encima de la línea, lo cual muy probablemente haya que corregirlo porque las firmas tienen “profundidad” de texto, que significa que hay parte de algunas letras que sobresalen hacia abajo respecto a la línea base del texto. Por todo esto es muy probable el uso de los corrimientos. Pueden verse los ejemplos, que allí se usan los corrimientos.)}
%
%  {\bfseries Las facturas no llevan firma porque se supone que el documento es auténtico ya que viene avalado por una imprenta registrada y tiene los números correlativos impresos desde la imprenta, de manera que la factura no necesita la firma de quien la emite; razón por la cual el texto debajo de la línea del emisor en la factura dice “PAGADO”, porque el emisor firma la factura una vez que haya sido pagada como constancia para el cliente.} Entonces agregar la firma escaneada es una posibilidad que debe contemplarse solamente con las cotizaciones, usando también el comando |\Emisor|, ya que no nos interesaría que apareciera “PAGADO” en ese caso sino algún nombre de quien está emitiendo la cotización.
%
%
% \subsection{Relativos a la moneda y a la numeración del documento factura}
%
% \DescribeMacro{\Moneda}{\macrooptcolor|\Moneda|\marg{signo monetario}}
%
%  Cambia el signo monetario, por defecto “Bs.”.
%
% \DescribeMacro{\Num}{\macrooptcolor|\Num|\marg{núm}}
%
%  Define el número de la factura. Es opcional porque el talonario de las facturas puede ya tener la numeración. Es de obligatorio uso cuando se define |denom|, o |prefactura| sin usar |sinnumero|.
%
% \DescribeMacro{\NumControl}{\macrooptcolor|\NumControl|\marg{núm}}
%
%  Define el número de control de la factura. Esta opción sólo aplica con |prefactura| puesto que una factura fiscal siempre tiene el número de control impreso desde la imprenta. Si se usa |prefactura| y no se define el número de control, éste tomará el mismo número de la factura.
%
% \DescribeMacro{\NumSerieControl}{\macrooptcolor|\NumSerieControl|\marg{núm}}
%
%  Define el número de serie del número de control, también aplica solamente con |prefactura|. Si no se especifica, toma |00|.
%
% \DescribeMacro{\Denom}{\macrooptcolor|\Denom|\marg{texto}}
%
%  Cambia el nombre del documento a algo distinto de “FACTURA” o “COTIZACIÓN”.
%
%
% \subsection{Estableciendo la fecha del documento}
%
% \DescribeMacro{\Fecha}{\macrooptcolor|\Fecha|\marg{fecha}}
%
%  Especifica la fecha de emisión del documento, en un formato que puede cambiarse con el comando |\FormatoFecha| pero predeterminadamente es el formato venezolano |dd-mm-aaaa|.
%
%  La fecha de emisión se guardará en el identificador “\texttt{FechaEmision}” perteneciente al paquete \texttt{datetime2} y podrá ser usado en cualquier parte del documento según las formas que provee dicho paquete.
%
%  {\itshape Si el usuario no provee la fecha, será tomada la fecha actual como la fecha de emisión y se producirá una advertencia indicando que no se especificó la fecha y que la fecha del documento es la fecha del día.}
% 
% \DescribeMacro{\Fecha*}{\macrooptcolor|\Fecha*|\marg{aaaa}\marg{mm}\marg{dd}}
%
%  Esta variante del comando se incorporó para suministrar la fecha en formato ISO, es decir, primero el año, luego y mes y finalmente el día, cada uno como un parámetro separado del otro (como se hace en \texttt{datetime2}. Esto es independiente de la forma como la fecha será mostrada, o sea que puede la fecha suministrarse de esta forma e igual saldrá en el formato deseado, cambiable con la macro |\FormatoFecha|, explicado a continuación.
%
% \DescribeMacro{\FormatoFecha}{\macrooptcolor|\FormatoFecha|\oarg{formato entrada}\marg{formato salida}\marg{separador}}
%
%  El comando |\FormatoFecha| le permite al usuario cambiar el formato de entrada de la fecha suministrada y el formato de salida de las fechas (cómo éstas salen escritas en la factura).
%
%  Con el primer argumento (opcional, dado entre corchetes) el usuario le puede decir a la clase cúal es el formato según el cual le suministrará la fecha con |\Fecha|; las posibilidades son: \begin{inparaenum}[\bfseries i)]\item |dd-mm-aaaa| (predeterminada), \item |dd.mm.aaaa|, \item |ddmmaaaa|, \item |aaaa-mm-dd|, \item |aaaa.mm.dd|, \item |aaaammdd| y \item |mm-dd-aaaa| \end{inparaenum}. Esto con el fin de establecer la fecha reconocida debidamente por |datetime2| y que sea posible su manejo como fecha y el posible cálculo de la fecha de vencimiento, si aplicara.
%
%  El segundo argumento (opcional también, dado entre llaves) le permite al usuario especificar el formato según el cual quiere que las fechas (emisión y vencimiento si aplica) salgan impresas en la factura. Estos son los formatos dados por \texttt{datetime2} y debe consultarse su documentación para más información. Entre ellos tenemos por ejemplo \begin{inparaenum}[\bfseries i)]\item |ddmmyyyy| (predeterminado), \item |default|, \item |iso|, \item |mmddyyyy| \end{inparaenum}.
%
%  El tercer argumento (opcional también, dado entre llaves) sirve para cambiar el separador, cuyo carácter predeterminado es el guión (“-”).
%
%  Con |\FormatoFecha|, al tener todos sus argumentos opcionales, el usuario podrá especificar solamente el primero (que va entre corchetes), solamente el segundo (entre llaves), pero para especificar el tercero que también está entre llaves debe especificar el segundo, ya que ambos son opcionales pero se dan igualmente entre llaves. Naturalmente podrá especificar dos de estos argumentos o los tres.
%
%  Por ejemplo: |\FormatoFecha[mm-dd-aaaa]{mmddyyyy}{.}| serviría para que el usuario pueda dar la fecha así: |\Fecha{12-31-2019}| y que salga igualmente según el formato |mm-dd-aaaa| pero separado por puntos. También puede usarse |\FormatoFecha[aaaa-mm-dd]| para dar luego la fecha de esta forma: |\Fecha{2019-12-31}| pero esta saldrá igual impresa en la factura así: “|31-12-2019|”, ya que no cambiamos el formato de salida. También puede hacer |\FormatoFecha{mmddyyyy}{/}| y dar la fecha por ejemplo así: |\Fecha*{2019}{12}{31}| y saldrá impresa así: “|12/31/2019|”.
%
%  Los valores predeterminados cambiables con |\FormatoFecha| están definidos, como siempre, en el archivo |factura.def|.
%
% \DescribeMacro{\Credito}{\macrooptcolor|\Credito|\marg{días}}
%
%  Este comando, cuyo argumento es la cantidad de días, nos permite emitir una factura a crédito. La fecha de vencimiento será calculada automáticamente y será colocada debajo de la fecha de emisión, especificada por el usuario.
%
%  La fecha calculada automáticamente se guarda en el identificador “\texttt{FechaVencimiento}” perteneciente al paquete \texttt{datetime2}, de manera que el usuario podrá hacer uso de esta fecha con los comandos que este paquete provee.
%
% }
%
% 
% \subsection{Descuentos en la totalización\label{subsec:dctototales}}
%
%  Podemos efectuar un descuento\footnote{Estas opciones de descuento en el subtotal no son compatibles con la opción de totalización en modo de |filas|, explicada en la sección \ref{sec:opciones}, y muy probablemente más adelante seguirán sin serlo.} sobre cualquiera de los subtotales; ese descuento puede ser porcentual (versión sin *) o dado en un monto fijo a sustraer del subtotal (variante con *). Podemos aplicar un descuento que aplique por igual a cualquier tipo de ítem con cualquier gravamen, o discriminado dependiendo del tipo de impuesto se aplica un descuento u otro.
%
% {\setlength{\parindent}{0pt}
%
% \DescribeMacro{\Descuento}{\macrooptcolor|\Descuento|\marg{porcentaje}}\\
% \DescribeMacro{\Descuento*}{\macrooptcolor|\Descuento*|\marg{monto}}
%
% Con este comando el descuento aplicaría por igual a cualquier subtotal presente, es decir, para elementos con cualquier gravamen y también exentos. Dichos descuentos son aplicados y mostrados al totalizar. Puede usarse inclusive cuando se aplican descuentos individuales que se incluyen con la opción de la clase |descuentos|.
%
% \DescribeMacro{\DescuentoG}{\macrooptcolor|\DescuentoG|\marg{porcentaje}}\\
% \DescribeMacro{\DescuentoG*}{\macrooptcolor|\DescuentoG*|\marg{monto}}
%
%  Descuento específico para los elementos gravados con IVA general.
%
% \DescribeMacro{\DescuentoR}{\macrooptcolor|\DescuentoR|\marg{porcentaje}}\\
% \DescribeMacro{\DescuentoR*}{\macrooptcolor|\DescuentoR*|\marg{monto}}
%
%  Descuento específico para los elementos gravados con IVA reducido.
%
% \DescribeMacro{\DescuentoA}{\macrooptcolor|\DescuentoA|\marg{porcentaje}}\\
% \DescribeMacro{\DescuentoA*}{\macrooptcolor|\DescuentoA*|\marg{monto}}
%
%  Descuento específico para los elementos gravados con IVA adicional.
%
% \DescribeMacro{\DescuentoE}{\macrooptcolor|\DescuentoE|\marg{porcentaje}}\\
% \DescribeMacro{\DescuentoE*}{\macrooptcolor|\DescuentoE*|\marg{monto}}
%
%  Descuento específico para los elementos exentos de IVA.
%
%
% \subsection{Textos explicativos en la factura\label{subsec:textos}}
%
% \DescribeMacro{\Resumen}{\macrooptcolor|\Resumen|\marg{texto}}
%
%  Pone un resumen que va en una caja a la derecha de los datos del cliente, separado de esta en una distancia ajustable con el comando |\SepEncabezado|, que se describirá más adelante. Este resumen es opcional.
%
% \DescribeMacro{\NotaPrevia}{\macrooptcolor|\NotaPrevia|\marg{texto}}
%
%  La nota previa es una nota que puede ponerse antes del cuadro de la factura. Sin especificarla no sale nota alguna. Dicha nota va en un recuadro \tikz cuyo aspecto puede cambiarse, como ya se mencionó en la sección \ref{subsec:espacioslineas} donde se explican las líneas o contornos.
%
% \DescribeMacro{\NotaInterna}{\macrooptcolor|\NotaInterna|\marg{texto}}
%
%  La nota interna es una coletilla que aparece dentro del cuadro de la factura, por eso la llamamos interna (es interna al cuadro), por ejemplo si aplicamos un descuento podemos decir allí cerca de los totales a qué corresponde tal descuento, o para hacer alguna aclaratoria sobre lo que precede inmediatamente como algún ítem de la factura.
%
% \DescribeMacro{\NotaExterna}{\macrooptcolor|\NotaExterna|\marg{texto}}
%
%  La nota externa es otro texto opcional que el usuario puede introducir y saldrá al terminar el cuadro de la factura, antes de la opcional descripción larga (comando |\Descripcion|). {\itshape La nota externa no tiene formato \tikz.}
%
% \DescribeMacro{\NotaFinal}{\macrooptcolor|\NotaFinal|\marg{texto}}
%
%  Con este comando se establece el texto para la nota final\footnote{Antiguas condiciones de la factura, por ejemplo, para las condiciones de pago si se trata de una factura, o para las condiciones de una cotización.}. Es opcional; si no se define no sale texto alguno. También se puede destinar este espacio para un diseño (que haría el usuario) en el que se agreguen también algunas casillas para especificar la forma de pago.
%
%
% \subsection{Estableciendo los parámetros relativos al reporte}
%
% \DescribeMacro{\GeometriaReporte}{\macrooptcolor|\GeometriaReporte|\marg{geometría}}
%
%  Aquí van los parámetros relativos al paquete |geometry| que van a afectar la manera como el reporte es generado, de la misma manera como se definieron los márgenes para la factura.
%
%  Puesto que el reporte no se imprime en hojas de facturas sino en hojas en blanco, el usuario podrá cambiar la geometría del papel completamente a su gusto sin las restricciones impuestas por el diseño que tengan sus facturas. En |factura.def| se definen unos márgenes razonables para un reporte. Si el reporte y la factura usaran un tamaño distinto de papel, entonces al imprimirse ambos juntos (opción |conreporte|), el reporte tomará el tamaño físico del papel especificado para la factura ya que con el paquete |geometry| no podemos cambiar el tamaño del papel en el medio del documento, aunque sí sus márgenes. Entonces, si se quiere imprimir el reporte en un papel de otro tamaño, tendrán que imprimirse ambos documentos por separado, o sea, primero uno y después el otro.
%
% \DescribeMacro{\FondoReporte}{\macrooptcolor|\FondoReporte|\marg{nombre del archivo de la imagen}}
%
%  Imagen a cargar {\bfseries solamente en la primera página del reporte} (e.g. un membrete). Cuando se usa este comando puede necesitarse usar también |\InicioReporte| para fijar el inicio de la primera página del reporte más abajo que las demás páginas cuyos márgenes están establecidos con |\GeometriaReporte|; esto porque la primera página tendría un “margen superior” distinto por la existencia del membrete.
%
% \DescribeMacro{\FondoReporte*}{\macrooptcolor|\FondoReporte*|\marg{nombre del archivo de la imagen}}
%
%  Esta variante de |\FondoReporte| es para que la imagen esté visible {\bfseries en todas las páginas} y no solamente en la primera como con el comando anterior. Usando esta variante del fondo del reporte no es necesario usar |\InicioReporte| sino solamente establecer los márgenes del reporte de acuerdo a dicho fondo.
%
% \DescribeMacro{\EstiloPagReporte}{\macrooptcolor|\EstiloPagReporte|\oarg{estilo primera página}\marg{estilo páginas restantes}}
%
%  Con este comando definimos el estilo de página para el reporte (véase el comando |\InfoPagina| y la subsección \ref{subsec:pagestyle}). El estilo de página es lo que cambiamos en \LaTeX{} con |\pagestyle|, o sea, lo relativo a las cabeceras y pies de página.
%
%  Con este comando podemos asignar un estilo de página para la primera página del reporte y otro estilo de página para el resto de las páginas. La primera página del reporte pudiera tener un estilo de página distinto por la presencia del membrete si se usa el comando |\FondoReporte| sin “|*|”, por lo cual se puede querer no poner cabeceras en la primera página pero sí en las demás.
%
%  Como vemos de la sintaxis de esta macro, el argumento obligatorio es el del resto de las páginas y el opcional es el que aplica a la primera página. El opcional se pone entre corchetes antes del otro, pero si se omite, todas las páginas toman el estilo especificado entre llaves.
%
%  Si no se usa esta macro la clase asume un comportamiento apropiado teniendo en cuenta si se especifica |\FondoReporte| o |\FondoReporte*| para evitar solapamiento de textos, como se explica también en la subsección \ref{subsec:pagestyle}.
%
%  La información que aparece como cabecera o como pie de página es la que damos con el comando |\InfoPagina|, explicado más adelante.
%
%  La clase provee varios estilos de página, que también se explicarán en la subsección \ref{subsec:pagestyle}. En los ejemplos de reportes se puede ver el comportamiento del comando |\EstiloPagReporte|.
%
% \DescribeMacro{\InicioReporte}{\macrooptcolor|\InicioReporte|\marg{longitud}}
%
%  Con este comando -cuyo argumento es una longitud de \LaTeX{}- se agrega una distancia extra donde inicia el texto en la primera página del reporte cuando se usa |\FondoReporte| sin “|*|”. Este comando sólo aplica cuando en el reporte se fija un fondo para que salga solamente en la primera página (comando |\FondoReporte| en su versión sin “|*|”). La motivación de esto es corregir la alteración que produce un membrete que va a salir solamente en la primera página; si el membrete estuviera en todas las páginas no es necesario hacer esto sino fijar los márgenes adecuadamente con |\GeometriaReporte|. Si en el reporte no se fija un fondo o se fija el fondo para todas las páginas (comando |\FondoReporte*|), lo que se establezca con |\InicioReporte| es ignorado. En estos dos casos los márgenes deben establecerse con |\GeometriaReporte|.
%
% \DescribeMacro{\FirmaReporte}{\macrooptcolor|\FirmaReporte|\oarg{inclusión la imagen}\oarg{corr. en x}\oarg{corr. en y}\marg{texto}\oarg{pos. x}\oarg{long. línea}}
%
%  Incluye la firma para el final del reporte. Los argumentos son:
%  \begin{enumerate}
%   \item inclusión de la imagen (e.g. |\includegraphics|),
%   \item corrimiento en |X| de la imagen, para un ajuste más fino de la posición de la firma,
%   \item corrimiento en |Y|,
%   \item texto debajo de la línea ({\itshape va entre llaves y no entre corchetes como los demás}). El predeterminado es el dado mediante el comando |\Proveedor|, o en su defecto, |\Emisor|),
%   \item posición en |X| del final de la línea, medida desde la derecha (predeterminado: 2cm),
%   \item tamaño extra de la línea de la firma. La clase mide el tamaño del texto y de la imagen (si se especificó), y al que sea mayor le agrega este extra de línea para que sobresalga a los dos lados. (Predeterminado: tamaño de la imagen o del nombre + 1.5cm extra).
%  \end{enumerate}
%  {\itshape {\bfseries Este comando es de los pocos que no es del preámbulo}; debe incluirse en el sitio en el reporte donde se quiera que la firma salga, generalmente al final del reporte y todos sus argumentos son opcionales, incluyendo el texto que va debajo de la línea, que se incluye entre llaves |{}| a pesar de ser opcional. Cuando no se especifica ningún argumento, sólo dibuja la línea para firmar y le escribe abajo el nombre dado con |\Proveedor| o en su defecto el dado con |\Emisor|.}
%
% \DescribeMacro{\TituloReporte}{\macrooptcolor|\TituloReporte|\marg{texto}}
%
%  Especificamos un título para el reporte. Este título saldrá centrado al principio del reporte.
%
% \DescribeMacro{\TituloReporte*}{\macrooptcolor|\TituloReporte*|\marg{texto}}
%
%  Usando |\TituloReporte| u omitiendo su uso, la información que sale en las cabeceras o pies de página tiene que ver con el número de la factura que está asociada a ese reporte. Con esta variante con “|*|” de |\TituloReporte|, el título del reporte -especificado con este comando- saldrá escrito también en estas partes de la página en lugar del texto que indica que el reporte corresponde a determinada factura.
%
% Puede usarse esta opción también en aquellos casos en los que se hace un reporte sin asociarlo a una factura, entonces el título del reporte es la información que sale en la cabecera o pie de cada página.
%
% \DescribeMacro{\InfoPagina}{\macrooptcolor|\InfoPagina|\marg{texto}}
%
%  Establece el texto que aparece en la cabecera o pie de página del reporte.
%
% {\itshape Si se quiere usar esta opción debe hacerse luego de definir el título del reporte porque si no el comando |\TituloReporte*| establece de nuevo la variable que |\InfoPagina| cambia.}
%
%  En |factura.def| hay dos definiciones relativas a la información que aparece en las cabeceras o pies de página: aquella que escribe el título y aquella que no, es decir, la que aplica con |\TituloReporte*| y la que aplica con |\TituloReporte|, que también el usuario podrá redefinir si lo necesita, pero lo más práctico sería usar esta macro para fijar el texto que se desee, y |\EstiloPagReporte| para fijarlo donde se quiera.
%
%
% \subsection{Definiciones en archivos separados \label{subsec:aux}}
%
% \DescribeMacro{\Cliente}{\macrooptcolor|\Cliente|\marg{archivo \LaTeX}}
%
%  {\bfseries Todos los comandos definibles en el preámbulo que alteren la forma como el reporte o la factura es presentada, así como cualquier variable definida en |factura.def| se puede redefinir y agregar a otro archivo |.tex| o |.def| que el usuario cargue en el preámbulo con |\input|, |\include| o con el comando |\Cliente|; el comando para el cliente existe para que el usuario pueda definir archivos |.tex| o |.def| con los valores específicos que aplican para cada cliente.}
%
%  La macro |\Cliente| incluye un archivo cuya extensión debe ser |.def| o |.tex|, de la misma forma como se hace con |\input|; es simplemente un atajo que además lee archivos |.def|.
%
% }
%
%
% \section{La factura\label{sec:factura}}
%
% \DescribeEnv{factura}
% La parte del documento donde se agregan los elementos a facturar se escribe dentro de un entorno especial para su tratamiento, llamado |factura|, que produce el cuadro de la factura:
%
% \iffalse
%<*literal>
% \fi
{\VERBATIM
\begin{verbatim}
    \begin{factura}
     [...]
    \end{factura}
\end{verbatim}
}
% \iffalse
%</literal>
% \fi
%
% Dentro de dicho ambiente podemos usar únicamente los comandos que incluyen los elementos a facturar y las posibles etiquetas; esos comandos son, para distintos gravámenes, los siguientes:
%
%
% \subsection{Agregando elementos a la factura}
%
% \noindent
% {\macromandcolor|\Item|\marg{cantidad}\marg{descripción}\oarg{descripción para csv}\marg{precio unitario}\oarg{descripción sólo para \\ \mbox{}\hfill referencias}}
% \DescribeMacro{\Item}
% \DescribeMacro{\ItemR}
% \DescribeMacro{\ItemA}
% \DescribeMacro{\ItemE}
%
% Con estos comandos agregamos elementos a la factura propiamente dicha. Con |\Item| agregamos un elemento gravado con IVA general (actualmente 12\%); con |\ItemR| agregamos uno con IVA reducido (8\%), con |\ItemA| uno con IVA adicional (27\%), y con |\ItemE| agregamos un elemento exento de impuesto al valor agregado.
%
% Como podemos ver, se especifica la cantidad, luego la descripción y al final el precio unitario, todos entre llaves, pudiendo intercalar entre estos dos últimos un parámetro adicional entre corchetes que sirve para escribir otro texto al archivo |.csv| en sustitución de la descripción, ya que a veces la descripción puede tener macros o caracteres no imprimibles. Si se declara una descripción y otra descripción para el |.csv|, la factura impresa manejará la descripción dada primero (entre llaves) y en el archivo |.csv| se escribirá la descripción dada después (entre corchetes).
%
% El último parámetro opcional (entre corchetes) es otro campo de descripción pero que no se imprime en el cuadro de la factura sino que se guarda bajo una etiqueta para luego poder usarse en cualquier parte del documento con el comando |\ldescref| (véase la sección \ref{sec:etiquetas} de las etiquetas). Dicha descripción puede tener formato de párrafos largos con saltos de línea.
%
% También tenemos la posibilidad de agregar un ítem al que no le ponemos precio, por ejemplo por ser parte de un total, y sus argumentos son la cantidad y la descripción nada más.
%
% \noindent
% \DescribeMacro{\ItemX}{\macrooptcolor|\ItemX|\marg{cantidad}\marg{descripción}\oarg{descripción para csv}\oarg{descripción sólo para referencias}}
%
%
% \subsubsection{Opción `{\ttfamily descuentos}'\label{subsec:descuentos}}
%
% \noindent
% {\macromandcolor|\Item|\marg{cantidad}\marg{descripción}\oarg{descripción para csv}\marg{precio unitario}\marg{\% descuento} \\ \mbox{}\hfill \oarg{descripción sólo para referencias}}
% \DescribeMacro{\Item}
% \DescribeMacro{\ItemR}
% \DescribeMacro{\ItemA}
% \DescribeMacro{\ItemE}
%
% Cuando pedimos descuentos individuales (opción |descuentos| al cargar la clase), todas las macros |\Item| tienen un posible\footnote{Es un argumento opcional, aunque va entre llaves como todos los demás.} argumento más al final que es el descuento porcentual. No especificarlo da el mismo resultado que un descuento de 0\% sobre ese ítem:
%
%
% \subsubsection{Opción `{\ttfamily sincantidad}'\label{subsec:sincantidad}}
%
% \noindent
% \DescribeMacro{\Item}
% \DescribeMacro{\ItemR}
% \DescribeMacro{\ItemA}
% \DescribeMacro{\ItemE}
% {\macromandcolor|\Item|\marg{descripción}\oarg{descripción para csv}\marg{precio unitario}\oarg{descripción sólo para referencias}}
%
% \noindent {\macromandcolor|\Item|\marg{descripción}\oarg{descripción para csv}\marg{precio unitario}\marg{\% descuento} \\ \mbox{}\hfill \oarg{descripción sólo para referencias}}
%
% Por otra parte, si se usa la opción |sincantidad| al cargar la clase, todos los comandos |\Item| aceptan un argumento menos, que es el de la cantidad; eso significa que sin usar la opción |descuentos|, la sintaxis de |\Item| queda como lo indica la primera línea, y usando |descuentos|, queda como lo indica la segunda línea.
%
% \noindent
% \DescribeMacro{\ItemX}{\macrooptcolor|\ItemX|\marg{descripción}\oarg{descripción para csv}\oarg{descripción sólo para referencias}}
% 
% Puesto que |\ItemX| se usa para elementos sin precio, al usar |sincantidad| quedará aceptando sólo un argumento obligatorio que es la descripción y dos opcionales para la descripción en el archivo |.csv| y la descripción larga para referenciar luego. Cuando se activa la opción |descuentos|, |\ItemX| queda igual puesto que esta macro no lleva precio, entonces tampoco acepta descuentos.
%
% {\itshape Todas las variaciones de estos comandos |\Item| pueden apreciarse en los ejemplos.}
%
%
% \subsection{Descripción larga al final de la factura}
%
% Una vez terminado el entorno de la factura se puede escribir una descripción larga que saldrá en una caja -cuyo estilo puede cambiarse-\footnote{El diseño de esta descripción consiste en su linea de contorno, cambiable como ya explicamos en la subsección \ref{subsec:espacioslineas}, específicamente con la macro \cs{LineaDescripcion}.} debajo del cuadro de la factura, esto va dentro de un |\parbox| en la implementación de la caja. Si esta descripción no se escribe, simplemente no va a aparecer nada debajo del cuadro de la factura.
%
% \noindent \DescribeMacro{\Descripcion}
% {\macrooptcolor|\Descripcion|\marg{texto en formato de párrafos}}
%
% {\itshape Como podemos ver, |\Descripcion| es otro de los pocos comandos que no se usan en el preámbulo sino dentro del documento, jústamente al terminar el ambiente} |factura|.
%
%
% \section{Reverso de la factura}
%
% \DescribeEnv{reverso}
%  La factura puede tener información escrita en su reverso, por ejemplo para ciertas condiciones cuando estas necesitan mucho espacio. Ese texto se escribe dentro de un entorno llamado |reverso|, que puede o no incluirse según el requerimiento del usuario. si está escrito en el documento fuente también podrá decidirse si se incluye o no en el documento compilado.
%
% \iffalse
%<*literal>
% \fi
{\VERBATIM
\begin{verbatim}
    \begin{reverso}
     (Aquí va toda la información que irá escrita
      en el reverso de la hoja de la factura.)
    \end{reverso}
\end{verbatim}
}
% \iffalse
%</literal>
% \fi
%
% El reverso de la factura va escrito con la tipografía que el usuario elije con el comando |\LetraReverso|, como se explicará en la sección \ref{sec:tipografia}, y las dimensiones de la página se establecen con el comando |\GeometriaReverso| (sección \ref{sec:comandos}). Dicha geometría viene con un valor predeterminado en el |.def|.
%
% Si se compila con la opción |sinreverso|, lo escrito dentro del ambiente |reverso| no saldrá como reverso en la hoja de la factura.
%
%
% \section{El reporte}
%
% \DescribeEnv{reporte}
%  Para elaborar un reporte, lo único que se debe hacer es escribir el texto del reporte dentro del ambiente para ese fin, que se llama |reporte|, de manera que en el documento tenemos un ambiente para la factura y otro para el reporte, y elegimos al momento de compilar qué es lo que queremos obtener.
%
% \iffalse
%<*literal>
% \fi
{\VERBATIM
\begin{verbatim}
    \begin{reporte}
     (Aquí va el reporte, sin límite la extensión del texto.)
    \end{reporte}
\end{verbatim}
}
% \iffalse
%</literal>
% \fi
%
% Cuando se compila con |conreporte|, el reporte sale desde la segunda página, o desde la tercera si se usa |twoside|.
%
% Cuando se compila con |soloreporte|, el motor de compilación no procesa la factura en favor del reporte, y sin especificar ninguna de estas dos opciones, el reporte es ignorado y sale únicamente la factura (o la prefactura o cotización).
%
% El reporte empieza con un encabezado y luego un título especificable; el encabezado consiste en el nombre del proveedor, el nombre del cliente y la persona de contacto. Luego de ese encabezado viene el título del reporte y después empieza lo que el usuario escriba dentro del entorno del reporte.
%
% Hay que tener en cuenta que, como se explicó en la sección \ref{sec:comandos}, el reporte puede tener una imagen de fondo, que puede ser un membrete. Ese membrete, si se quiere en todas las páginas deberá incluirse con |\FondoReporte*|, y si se quiere sólo en la primera página del reporte deberá usarse |\FondoReporte|. Ahora, si se usa la versión sin |*|, será necesario también usar |\InicioReporte| con una distancia para empezar a escribir un poco más abajo en la primera página, ya que es la única que tiene el membrete. Pero si se usa la versión con |*|, entonces basta fijar los márgenes para todas las páginas con |\GeometriaReporte|, tal como se adelantó en la sección \ref{sec:comandos}.
%
% El reporte estará escrito con la tipografía especificable con |\LetraReporte|, como se explicará más adelante en la sección \ref{sec:tipografia}.
%
% Por otra parte, al reporte se le asigna un título, que puede aparecer o no en la cabecera de página o en el pie de página. Si se quiere que aparezca el título en esos lugares, debe usarse |\TituloReporte*|, de lo contrario, |\TituloReporte|. Si no se especifica el título, entonces la información mostrada allí es la referencia a la factura asociada a este reporte, como se comentó antes cuando se introdujeron estos comandos.
%
%
% \subsection{Estilos de página para el reporte\label{subsec:pagestyle}}
%
% La clase define varios estilos de página\footnote{Usables en el reporte únicamente.} de \LaTeX{}, cambiables con los comandos usuales: |\pagestyle| y |\thispagestyle| dentro del entorno del reporte, pero es mejor definir el estilo de página en el preámbulo del documento, con |\EstiloPagReporte|.
%
% Estos estilos de página están definidos con el paquete |fancyhdr|, y son: \begin{inparaenum}[\bfseries i)]\item |plain|, \item |empty|, \item |soloarriba|, \item |soloabajo|, y \item |solonumpag|\end{inparaenum}.
%
% Con |plain|\footnote{Por razones de compatibilidad usamos dos estilos de página conocidos en \LaTeX, que son |plain| y |empty|, siendo |plain| uno de nuestros predeterminados.}, en la cabecera saldrá la información sobre el reporte y en el pie de página saldrá el número de página; con |empty| no sale nada en ninguna parte; con |soloarriba| y |soloabajo|, toda esa información anterior sale en la cabecera o en el pie de página, respectivamente, y con |solonumpag|, lo único que aparece es el número de página centrado en el pie de página.
%
% El formato con la información a poner en estas partes de las páginas del reporte es cambiable con |\InfoPagina|, descrito anteriormente.
%
% Cuando se usa |\FondoReporte*|, como se supone que se usa para un membrete en todas las páginas, el estilo de página predeterminado del reporte será |soloabajo|, para todas las páginas, porque no tendría sentido tener un membrete en todas las páginas y una cabecera como la descrita anteriormente. Si el comando |\FondoReporte*| se está usando para algo distinto de un membrete, por ejemplo una marca de agua, entonces es probable querer cambiar los estilos de página con |\EstiloPagReporte|.
%
% Si se usa |\FondoReporte|, el estilo predeterminado para la primera página será |solonumpag|, y |plain| para el resto de las páginas, y si se no se especifica fondo alguno, entonces el estilo de página predeterminado para todas las páginas será |plain|.
%
% El usuario podrá definir más estilos de página adicionales para colocar el texto que se quiera en el lugar que se quiera, y usar esos estilos de página con |\EstiloPagReporte|. Para eso puede leer la documentación del paquete |fancyhdr|, que es lo que esta clase usa para los estilos de página. También puede consultar |factura.def|, que es donde están predefinidos los estilos que trae la clase.
%
% El grosor de la línea decorativa que coloca el paquete |fancyhdr| está definido en el archivo |.def|, y es cambiable en la macro |\@GrosorLineaInfoPag| como podrá verse en dicho archivo; no definimos un comando para ello por no ser algo relevante pero igual podrá cambiarse con |\renewcommand| por ejemplo, como se hace en dicho |.def|.
%
%
% \section{Tipografías para cada elemento a escribir\label{sec:tipografia}}
%
% Cada elemento escrito en la factura o reporte es sujeto a ser mostrado con una tipografía distinta, para lo cual existen comandos con los que se asigna el tipo de letra o estilo para cada cosa que sale escrita en el documento, por ejemplo, se puede cambiar el tipo de letra con el que se escribe la fecha, el tipo de letra con el que salen los elementos dentro del cuadro de la factura, el tipo de letra de los totales, de los títulos, de las cabeceras o pies de página en el reporte, etc.
%
% Cuando en lo sucesivo nos referiremos a tipografía, tipo de letra, o letra, nos estaremos refiriendo a cualquier cambio en la tipografía que pueda significar el nombre de la letra, su tamaño o atributos como negritas, es decir, mediante cualquiera de estos comandos podemos cambiar cualquier aspecto de la tipografía. Mirando el archivo |.def| pueden verse todos los valores predeterminados, que la mayoría de ellos son simplemente cambios en el tamaño (macro |\fontsize| de \LaTeX{}), y en algunos casos negritas.
%
% {\setlength{\parindent}{0pt}
% \DescribeMacro{\LetraItems}{\macrooptcolor|\LetraItems|\marg{tipo de letra}}
%
%  Con este comando cambiamos la tipografía utilizada en los ítems o elementos que están en el cuadro de la factura, de manera que {\itshape cambiar la longitud \cs{baselineskip} con el segundo argumento de \cs{fontsize} hará que la tabla tenga una altura distinta de las celdas.}
%
%  Lo anterior significa que {\bfseries para cambiar la altura de las celdas, la forma correcta es a través de cambiar el tamaño de la letra de sus elementos con este comando. (Véase el archivo |.def|)}.
%
% \DescribeMacro{\LetraNumeros}{\macrooptcolor|\LetraNumeros|\marg{tipo de letra}}
%
%  Este comando sirve para cambiar el tipo de letra de tota la información numérica que aparece en las distintas columnas del cuadro de la factura, sin contar los totales, para los cuales hay otros comandos.
%
% \DescribeMacro{\LetraTipoIVA}{\macrooptcolor|\LetraTipoIVA|\marg{tipo de letra}}
%
%  El tipo del IVA es la letra que indica qué IVA está siendo gravado, por ejemplo, la “|G|”, “|R|” y “|A|” que puede salir al lado de los elementos con IVA general, reducido y adicional respectivamente, y “|(E)|” al lado de los elementos exentos. Con esta macro cambiamos la letra con la que eso sale.
%
% \DescribeMacro{\LetraTitColumnas}{\macrooptcolor|\LetraTitColumnas|\marg{tipo de letra}}
%
%  Con |\LetraTitColumnas| podemos cambiar la tipografía de los títulos de los campos que hay en el cuadro de la factura, es decir, los textos “Cantidad”, “Descripción”, “Precio U.”, etc.
%
% \DescribeMacro{\LetraTitTotales}{\macrooptcolor|\LetraTitTotales|\marg{tipo de letra}}
%
%  Con este comando cambiamos la tipografía de los nombres de los totales, es decir, “IVA G.”, “BI G.”, salvo el total general que se cambia con |\LetraTitTotal|.
%
% \DescribeMacro{\LetraNumTotales}{\macrooptcolor|\LetraNumTotales|\marg{tipo de letra}}
%
%  Con esta macro se cambia la tipografía utilizada en los totales numéricos de la factura, exceptuando el total general que se cambia con |\LetraNumTotal|.
%
% \DescribeMacro{\LetraTitTotal}{\macrooptcolor|\LetraTitTotal|\marg{tipo de letra}}
%
%  Para cambiar la tipografía de la palabra que indica el total general, e.g. “TOTAL”, lo hacemos con |\LetraTitTotal|.
%
% \DescribeMacro{\LetraNumTotal}{\macrooptcolor|\LetraNumTotal|\marg{tipo de letra}}
%
%  Para la tipografía del total general (la celda con el número del total final), utilizamos este comando.
%
% \DescribeMacro{\LetraTitEnc}{\macrooptcolor|\LetraTitEnc|\marg{tipo de letra}}
%
%  Con este comando cambiamos el tipo de letra de la información en el título de las cajas del encabezado de la factura, por ejemplo los textos: “Resumen” y “Datos del cliente”.
%
% \DescribeMacro{\LetraEncFactura}{\macrooptcolor|\LetraEncFactura|\marg{tipo de letra}}
%
%  Podemos también cambiar el tipo de letra de la información del encabezado de la factura, es decir, el tipo de letra del texto que escribe el usuario en el resumen o de toda la identificación del cliente en la factura. El resumen naturalmente acepta más cambios de tipografía escribiéndolos dentro del párrafo que se escribe con el comando |\Resumen|.
%
% \DescribeMacro{\LetraEncReporte}{\macrooptcolor|\LetraEncReporte|\marg{tipo de letra}}
%
%  Para cambiar la tipografía de la información del encabezado del reporte usamos |\LetraEncReporte|.
%
% \DescribeMacro{\LetraNotaPrevia}{\macrooptcolor|\LetraNotaPrevia|\marg{tipo de letra}}
%
%  Con |\LetraNotaPrevia| podemos cambiar la tipografía con la que se escribe la -opcional- nota previa al cuadro de la factura.
%
% \DescribeMacro{\LetraNotaInterna}{\macrooptcolor|\LetraNotaInterna|\marg{tipo de letra}}
%
%  Con |\LetraNotaInterna| cambiamos el formato de la nota opcional que aparece como coletilla de la factura (nota interna).
%
% \DescribeMacro{\LetraNotaExterna}{\macrooptcolor|\LetraNotaExterna|\marg{tipo de letra}}
%
%  También podemos cambiar la tipografía de la nota externa a la factura, es decir, la -opcional- nota que va al terminar el cuadro de la factura antes de la -opcional también- descripción larga.
%
% \DescribeMacro{\LetraDescripcion}{\macrooptcolor|\LetraDescripcion|\marg{tipo de letra}}
%
%  Con este comando cambiamos la tipografía de la descripción larga; naturalmente cualquier cambio puede hacerse luego dentro del texto de la descripción, igual que con el resumen.
%
% \DescribeMacro{\LetraFirmas}{\macrooptcolor|\LetraFirmas|\marg{tipo de letra}}
%
%  Con |\LetraFirmas| podemos cambiar la letra usada para escribir el nombre debajo de la línea para firmar en la factura.
%
% \DescribeMacro{\LetraNotaFinal}{\macrooptcolor|\LetraNotaFinal|\marg{tipo de letra}}
%
%  Para cambiar la tipografía de la nota final se hace con este comando. (Aplica el mismo comentario que para la descripción y el resumen).
%
% \DescribeMacro{\LetraDenom}{\macrooptcolor|\LetraDenom|\marg{tipo de letra}}
%
%  La tipografía asociada a la denominación del documento, es decir, al lado del número (e.g. “FACTURA” o “COTIZACIÓN”), y la asociada al texto “No. CONTROL” se cambia con |\LetraDenom|.
%
% \DescribeMacro{\LetraNumeracion}{\macrooptcolor|\LetraNumeracion|\marg{tipo de letra}}
%
%  Con |\LetraNumeracion| cambiamos la tipografía de las numeraciones como la del número de la factura y el número de control.
%
% \DescribeMacro{\LetraFecha}{\macrooptcolor|\LetraFecha|\marg{tipo de letra}}
%
%  La tipografía para la fecha de emisión y la fecha de vencimiento la podemos cambiar con este comando.
%
% \DescribeMacro{\LetraTitReporte}{\macrooptcolor|\LetraTitReporte|\marg{tipo de letra}}
%
%  La letra del título que aparece al principio del reporte se puede cambiar con |\LetraTitReporte|.
%
% \DescribeMacro{\LetraInfoPagina}{\macrooptcolor|\LetraInfoPagina|\marg{tipo de letra}}
%
%  Las cabeceras o pies de página del reporte también tienen un comando para su tipo de letra, y es |\LetraInfoPagina|.
%
% \DescribeMacro{\LetraReverso}{\macrooptcolor|\LetraReverso|\marg{tipo de letra}}
%
%  Con |\LetraReverso| cambiamos la tipografía de la información que escribamos en el reverso de la hoja de la factura.
%
% \DescribeMacro{\LetraReporte}{\macrooptcolor|\LetraReporte|\marg{tipo de letra}}
%
%  Finalmente, para cambiar la tipografía con la que se inicia el reporte tenemos este comando. Obviamente también luego se puede hacer cualquier cambio dentro del ambiente |reporte|.
%
% }
%
%
% \section{Etiquetas y referencias a ítems de la factura\label{sec:etiquetas}}
%
% Esta clase |factura| soporta agregar etiquetas a cada elemento de la factura que luego en el reporte o en casi cualquier otra parte queramos referenciar; {\slshape no se pueden referenciar etiquetas solamente en la nota previa ni en el resumen.}
%
% \noindent \DescribeMacro{\label}{\macrooptcolor|\label|\marg{etiqueta}}
%
%  Para ello usamos la conocida macro |\label| de \LaTeX{} justamente después del ítem que nos interese luego referenciar, por ejemplo (usando descuentos individuales):
%
% \iffalse
%<*literal>
% \fi
{\VERBATIM
\begin{verbatim}
    \begin{factura}
     \Item{1}{Artículo gravado con IVA general}{1000}
     \Item{3}{Otro artículo gravado con IVA general}{1500}{10}[Descripción larga 
      para referenciar el reporte u otro fin.

      Con formato de párrafo con saltos de línea si se desea.]\label{el-otro}
    \end{factura}
\end{verbatim}
}
% \iffalse
%</literal>
% \fi
%
% De esta forma creamos una etiqueta llamada “el-otro” para el segundo elemento, que luego podremos citar por ejemplo en la descripción larga o en el reporte con comandos similares a los que \LaTeX{} provee para referenciar etiquetas.
%
% \noindent \DescribeMacro{\itemref}{\macrooptcolor|\itemref|\marg{etiqueta}}\\
% \noindent \DescribeMacro{\cantref}{\macrooptcolor|\cantref|\marg{etiqueta}}\\
% \noindent \DescribeMacro{\descref}{\macrooptcolor|\descref|\marg{etiqueta}}\\
% \noindent \DescribeMacro{\puref}{\macrooptcolor|\puref|\marg{etiqueta}}\\
% \noindent \DescribeMacro{\subtref}{\macrooptcolor|\subtref|\marg{etiqueta}}\\
% \noindent \DescribeMacro{\dctoref}{\macrooptcolor|\dctoref|\marg{etiqueta}}\\
% \noindent \DescribeMacro{\ivaref}{\macrooptcolor|\ivaref|\marg{etiqueta}}\\
% \noindent \DescribeMacro{\ptref}{\macrooptcolor|\ptref|\marg{etiqueta}}\\
% \noindent \DescribeMacro{\ldescref}{\macrooptcolor|\ldescref|\marg{etiqueta}}\par
%
% Si se quiere mostrar el número de la línea que ocupa el ítem en la lista lo hacemos con |\itemref|, que en el ejemplo aparecería el número “|2|” indicando que es el segundo ítem de la factura\footnote{Cuando se usa \cs{itemref} conviene enumerar los ítems en la factura con la opción \texttt{numitem}.}. Con |\cantref| escribimos la cantidad del elemento citado mediante la etiqueta, en nuestro ejemplo sería “|3|”. Para la descripción del ítem usamos |\descref|, que en el caso del ejemplo sería el texto “|Otro artículo gravado con IVA general|”. Para el precio unitario usamos |\puref|, en el ejemplo “|1.500,00|”. El subtotal lo mostraríamos con |\subtref|; en nuestro caso “|4.500,00|”. Si se quiere mostrar el porcentaje del descuento del ítem citado usamos |\dctoref|, que para el ejemplo sería “|10|”. Para el porcentaje del IVA, usamos |\ivaref|, en nuestro caso escribiría “|12|”. Por último, para mostrar el precio total usamos |\ptref|, y en nuestro ejemplo eso escribiría “|4.050,00|” puesto que aplicamos un descuento del 10\%.
%
% Podemos ver que el uso de los comandos |\puref|, |\subtref| y |\ptref| nos muestra el precio en el mismo formato con el que lo vemos en la presentación de la factura, en cuanto al redondeo y al separador de miles y decimal.
%
% También podemos referenciar la -opcional- descripción larga que podemos usar por ejemplo en el reporte; esa descripción que va entre corchetes la podremos citar con |\ldescref|, y en nuestro ejemplo sería:
% “Descripción larga para referenciar el reporte u otro fin. Con formato de párrafo con saltos de línea si se desea.”\footnote{Aquí suprimimos los saltos de línea.}
%
% Con la asignación de estas etiquetas y su uso mediante las referencias podemos de manera conveniente citar cualquier ítem que se haya facturado, de manera dinámica sin tener que reescribir en la descripción larga o en el reporte aquello que ya se escribió en el cuadro de la factura. Para estas referencias se escogieron nombres de las macros que terminan en “\texttt{ref}” por compatibilidad con comandos de \LaTeX{} como |\pageref|.
%
%
%
% \section{Referenciar los valores establecidos en el preámbulo y los totales\label{sec:ref}}
%
%  En el preámbulo el usuario definió distintas macros, es decir, estableció la razón social del cliente, el R.I.F. y la dirección, entre otras cosas. Cualquiera de esas definiciones hechas en el preámbulo puede usarse luego en otra parte del documento con el mismo comando que se usó para su definición. Lo anterior significa que tenemos las siguientes macros posibles para usar dentro del cuerpo del documento (factura, reporte o reverso): |\Moneda|, |\RazonSocial|, |\RIF|, |\Direccion|, |\Telefono|, |\Email|, |\Contacto|, |\Emisor|, |\Conforme|, |\Proveedor|, |\Fecha|, |\FechaVencimiento|\footnote{La fecha de vencimiento no se establece en el preámbulo explícitamente sino a través de \cs{Credito} pero la clase dispone de esta interfaz para acceder a ella así como también con la interfaz dada por \texttt{datetime2}.}, |\Credito|, |\Num|, |\NumControl|, |\NumSerieControl|, |\Resumen|, |\NotaPrevia|, |\NotaInterna|, |\NotaExterna|, |\NotaFinal|, |\TituloReporte|.
%
% Esas macros, en el preámbulo funcionan para definir valores y textos, pero si ya no estamos en el preámbulos sino dentro del cuerpo del documento y escribimos una de esas macros\footnote{Dentro del cuerpo del documento estas macros obviamente no aceptan argumentos.}, lo que nos va a escribir será lo que está definido bajo esa macro, así sea con macros que no llegamos a utilizar explícitamente pero que tienen un valor predefinido como |\Moneda|.
%
% Por otra parte, en el documento el usuario también podrá referirse a cualquiera de los totales de la siguiente tabla\footnote{Las macros de la tabla, para representarlas en el mismo formato que en la factura, debemos usar la macro \cs{numprint}.}:
%
% \noindent \begin{tabular}{rl}
% \hline
% Total & Comando \\ \hline
% Bases imponibles & |\BIG|, |\BIR|, |\BIA|, |\BIE|\footnote{Lo llamamos \cs{BIE} para mantener la nomenclatura pero no es una base imponible ya que es un monto exento de impuesto.} \\
% IVA & |\AlicuotaG|, |\AlicuotaR|, |\AlicuotaA| \\
% Descuento en IVA & |\AlicuotaGDesc| \\
% Subtotales\footnote{El subtotal es el monto sin aplicar descuentos generales (los que se hacen con los comandos \cs{Descuento***}, es decir, los descuentos que no son individuales a cada item sino al totalizar), que coincide con la base imponible cuando no hay esos descuentos.} & |\SubtG|, |\SubtE|, |\SubtR|, |\SubtA| \\
% Descuentos & |\DescG|, |\DescE|, |\DescR|, |\DescA| \\
% Total final & |\Total| \\
% \hline
% \end{tabular}
%
% También ya mencionamos que la fecha de emisión y vencimiento se pueden acceder con la interfaz dada por el paquete |datetime2|, por ejemplo, para mostrar la fecha de emisión, podremos usar tanto |\Fecha| como |\DTMusedate{FechaEmision}|, y para la fecha de vencimiento, tanto |\FechaVencimiento| como |\DTMusedate{FechaVencimiento}|.
%
% La cantidad de ítems en la factura la podemos acceder via el contador |NumItem|, por ejemplo así |\arabic{NumItem}|.
%
% En los ejemplos hay casos con todas estas funciones de esta sección.
%
%
% \section{Más personalizaciones}
%
% \DescribeMacro{\@BeforeEndPreamble}
% La clase provee de un “gancho” llamado \cs{BeforeEndPreamble}. Lo que allí se agregue (es acumulativo, se puede ejecutar mas de una vez) será ejecutado por la clase |factura| luego de que hayan sido evaluadas las opciones de la clase y tomadas todas las decisiones en cuanto a opciones que impliquen otras opciones o que sean incompatibles con otras opciones; una vez que todas esas decisiones fueron tomadas y todo queda establecido, se ejecuta \cs{BeforeEndPreamble}, esto con el propósito de que el usuario pueda personalizar más el comportamiento de sus archivos |.tex|. Por ejemplo si el usuario hace lo siguiente, no funcionará:
%
% \iffalse
%<*literal>
% \fi
{\VERBATIM
\begin{verbatim}
    \documentclass[cotizacion]{factura}
    [...]
    \ifcotizacion\FondoFactura{Fondo}\fi
    \begin{document}
     \begin{factura}
      [...]
     \end{factura}
    \end{document}
\end{verbatim}
}
% \iffalse
%</literal>
% \fi
%
% No funcionará porque el condicional \cs{ifcotizacion} aún no se ha establecido (eso lo hace la clase con el gancho \cs{AtEndPreamble} de |etoolbox| luego de que todo el preámbulo haya sido analizado), por lo que se evaluará como falso; tendría el usuario que agregarlo dentro de un gancho \cs{AtEndPreamble} pero tampoco funcionará del todo, para eso esta clase provee del gancho \cs{BeforeEndPreamble}, y se escribiría este ejemplo así:
%
% \iffalse
%<*literal>
% \fi
{\VERBATIM
\begin{verbatim}
    \documentclass[cotizacion]{factura}
    [...]
    \BeforeEndPreamble{\ifcotizacion\FondoFactura{Fondo}\fi}
    \begin{document}
     \begin{factura}
      [...]
     \end{factura}
    \end{document}
\end{verbatim}
} 
% \iffalse
%</literal>
% \fi
%
% De esta forma, el condicional se evaluará después de que se hayan establecido todas las opciones (en este caso, |cotizacion|) y el usuario podrá inclusive apagar opciones y encenderlas nuevamente usando los condicionales (como se explicó en la subsección \ref{subsec:condicionales} porque después de ejecutar este gancho, automáticamente se volverán a evaluar las opciones y a tomar todas las decisiones de implicaciones y compatibilidades.
%
% \newpage
% \DescribeMacro{\@PosicionaFechas}
% \DescribeMacro{\@PosicionaDenom}
% \DescribeMacro{\@PosicionaNumControl}
% \DescribeMacro{\@PosicionaNotaFinal}
% \DescribeMacro{\@PosicionaFirmas}
% \DescribeMacro{\@EscribeDatos}
% \DescribeMacro{\@EscribeDatosResumen}
% \DescribeMacro{\@EscribeDatosReporte}
% \DescribeMacro{\@EscribeNotaFinal}
% \DescribeMacro{\@EscribeNotaPrevia}
% \DescribeMacro{\@DibujaLinea}
% \DescribeMacro{\@EscribeItemsEnCSV}
% \DescribeMacro{\@EscribeTotalesEnCSV}
% \DescribeMacro{\@EncabezadoFactura}
% \DescribeMacro{\@EncabezadoReporte}
% También es posible darle aspecto distinto a muchas de las partes que conforman la factura. Internamente en la clase, cada parte es ubicada en la hoja con un comando y su contenido es escrito en el papel con otro comando. Los comandos que más probablemente son sujetos a ser cambiados por el usuario son los comandos que escriben el texto más que aquellos que lo ubican en la hoja, por tal razón, los comandos que posicionan aquello que va a escribirse lo definimos dentro del archivo |.cls|, y los que propiamente escriben cada parte de la hoja están definidos en el |.def|. Ambas fuentes las podrán consultar los usuarios para mayor comprensión, y para la total personalización. Recomendamos empezar por el archivo de definiciones |factura.def|.
%
% Estas macros son internas y sirven para, como dijimos antes, que \LaTeX{} se prepare para escribir el texto en la hoja; todas empiezan con |\@Posiciona| y no debe ser común necesitar cambiarlas.
%
% Todas estas macros llaman a su correspondiente macro con el estilo del elemento a escribir, que comienza con |\@Estilo|, y con cambiar su contenido podemos cambiar la forma como es escrita esa parte de la factura en la hoja.
%
% \DescribeMacro{\@EstiloFechas}
% \DescribeMacro{\@EstiloDenom}
% \DescribeMacro{\@EstiloNumControl}
% \DescribeMacro{\@EstiloNotaPrevia}
% \DescribeMacro{\@EstiloNotaFinal}
% \DescribeMacro{\@EstiloFirmas}
% \DescribeMacro{\@DatosTIKZ}
% \DescribeMacro{\@NotaPreviaTIKZ}
% \DescribeMacro{\@DescripcionTIKZ}
% \DescribeMacro{\@NotaFinalTIKZ}
% Estas últimas son las que están definidas dentro de |factura.def|, y que eventualmente puede desearse cambiarlas.
%
% Adicionalmente están las macros que se encargan de distintas partes de la factura y del reporte como los datos del cliente, el resumen, la nota previa, la nota final, los encabezados, la información en el archivo |.csv|, y la macro que dibuja la línea que inutiliza el espacio en blanco. Todas estas son macros internas que no deben ser llamadas por el usuario, y no consideramos que pueda sea necesario cambiarlas.
%
% Por último, las macros que alojan el diseño \tikz de los datos del cliente, la nota previa, la descripción larga y la nota final son |\@DatosTIKZ|, |\@NotaPreviaTIKZ|, |\@DescripcionTIKZ| y |\@NotaFinalTIKZ|; están definidas en el archivo |factura.def|.
%
% Cualquiera de estas macros que se quiera redefinir habría que hacerlo en el preámbulo del |.tex| y encerrándolas entre |\makeatletter| y |\makeatother| para poder redefinir una macro que contiene un símbolo “|@|”.
%
% {\bfseries \itshape Finalmente, como dijimos en la sección \ref{subsec:aux}, una vez adecuada la clase al diseño que se quiere, se podrán escribir todos los comandos que hacen estos ajustes en un} |.tex| {\itshape o} |.def| {\itshape y cargarlo con} |\input|, |\include| {\itshape o con} |\Cliente|, pudiendo discriminar por clientes dicha personalización.
%
%
%
% \newpage
% \setstretch{1}
% \section{Tabla con todas las opciones}
%
% \noindent La tabla que sigue muestra una lista de las opciones posibles al momento de cargar la clase:
%
% \noindent \begin{tabular}{>{\ttfamily}ll}
%  \rmfamily \bfseries Opción & \bfseries Significado \\ \hline
%  agrupado & Agrupa los totales por tipo en el modo de columnas. \\
%  centradesc & Centra la descripción de los ítems. \\
%  centrado & Centra los totales en el modo de filas. \\
%  conlineasha & Dibuja una línea separando los artículos entre sí. \\
%  conreporte & Produce la factura y el reporte. \\
%  cotización & Imprime una cotización. \\
%  csv & Genera un archivo |.csv| con la información numérica de la factura. \\
%  dcu & Descripción, cantidad, precio unitario. \\
%  denom & Explicita la denominación del documento. \\
%  descuentos & A cada artículo se le puede asignar un descuento individual. \\
%  dosfirmas & Incluye los dos espacios para firmar. \\
%  duc & Descripción, precio unitario, cantidad. \\
%  electro\footnote{Ya no sirve esta opción.} & Calcula los descuentos para el IVA para pagos por medios electrónicos. \\
%  expandecuadro & Llena con espacio extra el cuadro de la factura. \\
%  filas & Escribe los totales en filas. \\
%  G & Agrega una “|G|” en los ítems con IVA general. \\
%  impdesc & Escribe la denominación del IVA en la descripción. \\
%  iva & Expresa en otra columna el porcentaje de IVA para cada elemento a facturar. \\
%  linea & Dibuja una línea que inutiliza el espacio en blanco en el cuadro de la factura. \\
%  nospanish & No carga |babel| ni |fontenc|. \\
%  numitem & Enumera los ítems en la factura en otra columna extra al principio. \\
%  prefactura & Imprime una prefactura. \\
%  sincantidad & Los artículos van sin expresar su cantidad. \\
%  sinexpandir & No expande elementos en la factura. \\
%  sinfirmas & No incluye espacios para firmar. \\
%  siniva & No incluye IVA, pero lo deja expresado. \\
%  sinivaexpresado & No incluye IVA ni lo expresa. \\
%  sinlineahni & No dibuja la línea horizontal entre totales y nota interna. \\
%  sinlineahtit & No dibuja la línea horizontal del título de las columnas. \\
%  sinlineahtot & No dibuja la línea horizontal entre ítems y totales. \\
%  sinlineashi & No dibuja las líneas horizontales internas. \\
%  sinlineashe & No dibuja las líneas horizontales externas. \\
%  sinlineash & No dibuja ninguna línea horizontal. \\
%  sinlineasvi & No dibuja las líneas verticales internas. \\
%  sinlineasve & No dibuja las líneas verticales externas. \\
%  sinlineasv & No dibuja ninguna línea vertical. \\
%  sinmarcas & No escribe el carácter de fin de línea ni el carácter en las celdas vacías \\
%            & cuando la descripción ocupa más que una línea. \\
%  sinmonedatotales & Quita el signo monetario de las totalizaciones. \\
%  sinnumero & No pone el número cuando se usa |denom|. \\
%  sintotales & No escribe los totales al final del cuadro; para cotizaciones. \\
%  sinreverso & No compila el reverso de la factura así se haya escrito. \\
%  soloreporte & Produce únicamente el reporte. \\
%  subtotal & En otra columna muestra el subtotal de cada ítem antes aplicar descuento. \\
%  todosiva & Expresa todos los tipos de IVA. \\
%  twoside & Imprime el reporte a dos caras. \\
%  unafirma & Sólo incluye la firma del emisor. \\
%  vencimiento & Escribe la fecha de vencimiento así no haya crédito en la factura. \\
% \end{tabular}
%
%
% \newpage
% \section{Tabla con todos los comandos}
%
% A continuación presentamos una tabla con todos los comandos, que salvo |\FirmaReporte|, los |\Item| y |\Descripcion|, se poner todos en el preámbulo.
%
% \noindent \begin{tabular}{>{\ttfamily}ll}
% \rmfamily \bfseries Comando & \bfseries Significado \\ \hline
%  |\cantref| & Muestra la cantidad del ítem referenciado. \\
%  |\CI| & Cédula de identidad del cliente. \\
%  |\Cliente| & Carga un archivo |.def| o |.tex| con personalizaciones. \\
%  |\NotaFinal| & Establece la nota final. \\
%  |\Conforme| & Nombre de quien recibe la factura, para el espacio de la firma. \\
%  |\Contacto| & Persona de contacto. \\
%  |\Credito| & Establece la cantidad de días para el crédito de la factura. \\
%  |\dctoref| & Muestra al descuento del ítem referenciado. \\
%  |\Denom| & Nombre de la denominación de documento. \\
%  |\descref| & Muestra a la descripción del ítem referenciado. \\
%  |\Descripcion| & Descripción larga de la factura\footnote{El comando \cs{Descripcion} se escribe dentro del documento, luego del ambiente \texttt{factura}, y no en el preámbulo.}. \\
%  |\Descuento| & Porcentaje de descuento para todos los elementos. \\
%  |\Descuento*| & Descuento neto en el subtotal, para todos los elementos. \\
%  |\Descuento[GRAE]| & Descuento en porcentaje, discriminado según el tipo de ítem. \\
%  |\Descuento[GRAE]*| & Descuento en el subtotal, discriminado según el tipo de ítem. \\
%  |\Direccion| & Dirección del cliente. \\
%  |\Email| & E-mail del cliente. \\
%  |\Emisor| & Nombre de quien emite la factura, para el espacio de la firma. \\
%  |\EstiloPagReporte| & Establece el estilo de página del reporte. \\
%  |\Fecha| & Establece la fecha de emisión de la factura en formato predeterminado. \\
%  |\Fecha*| & Establece la fecha de emisión de la factura en formato ISO. \\
%  |\FirmaFactura| & Carga la imagen de la firma en la factura. \\
%  |\FirmaReporte| & Carga la imagen de la firma en el reporte\footnote{El comando \cs{FirmaReporte} se escribe dentro del ambiente \texttt{reporte} y no en el preámbulo.}. \\
%  |\FondoFactura| & Carga la imagen del fondo de la factura. \\
%  |\FondoReporte| & Carga la imagen del fondo del reporte para la primera página. \\
%  |\FondoReporte*| & Carga la imagen del fondo del reporte para todas las páginas. \\
%  |\FormatoFecha| & Establece lo relacionado con el formato de las fechas. \\
%  |\GeometriaFactura| & Establece los parámetros de la geometría de la factura. \\
%  |\GeometriaReporte| & Establece los parámetros de la geometría del reporte. \\
%  |\InfoPagina| & Cambia la información de las cabeceras o pies de página del reporte. \\
%  |\InicioReporte| & Distancia hasta el inicio del texto en la primera hoja del reporte. \\
%  |\Item[RAEX]| & Agrega un elemento a la factura\footnote{Estos comandos \cs{Item} se escriben dentro del ambiente \texttt{factura} y no en el preámbulo.}. \\
%  |\itemref| & Muestra el número de línea en la factura del ítem referenciado. \\
%  |\ivaref| & Muestra al IVA del ítem referenciado. \\
%  |\label| & Agrega una etiqueta a un ítem de la factura\footnote{Se mantiene el nombre “\cs{label}” por compatibilidad con los comandos conocidos de \LaTeX. Este comando también se escribe dentro del ambiente \texttt{factura}.}. \\
%  |\ldescref| & Muestra a la descripción larga del ítem referenciado. \\
% \end{tabular}
% \newpage
% \noindent \begin{tabular}{>{\ttfamily}ll}
% \rmfamily \bfseries Comando & \bfseries Significado \\ \hline
%  |\LetraNotaFinal| & Tipografía para la nota final. \\
%  |\LetraDenom| & Tipografía para la denominación del documento. \\
%  |\LetraDescripcion| & Tipografía para la descripción en la factura. \\
%  |\LetraEncFactura| & Tipografía para la información en las cajas del encabezado. \\
%  |\LetraEncReporte| & Tipografía para la información del encabezado del reporte. \\
%  |\LetraFecha| & Tipografía para la fecha de emisión. \\
%  |\LetraFirmas| & Tipografía para los nombres de quienes firman la factura. \\
%  |\LetraInfoPagina| & Tipografía para las cabeceras o pies de página del reporte. \\
%  |\LetraItems| & Tipografía para los ítems en la factura. \\
%  |\LetraNotaExterna| & Tipografía para la nota externa al cuadro de la factura. \\
%  |\LetraNotaInterna| & Tipografía para la nota interna al cuadro de la factura. \\
%  |\LetraNotaPrevia| & Tipografía para la nota previa al cuadro de la factura. \\
%  |\LetraNumeracion| & Tipografía para las numeraciones. \\
%  |\LetraNumeros| & Tipografía para los números en la factura. \\
%  |\LetraNumTotal| & Tipografía para el total final numérico. \\
%  |\LetraNumTotales| & Tipografía para los totales numéricos. \\
%  |\LetraReporte| & Tipografía para el texto del reporte. \\
%  |\LetraReverso| & Tipografía para el reverso de la factura. \\
%  |\LetraTipoIVA| & Tipografía para el tipo de IVA. \\
%  |\LetraTitColumnas| & Tipografía para los nombres de las columnas del cuadro de la factura. \\
%  |\LetraTitEnc| & Tipografía para los títulos en las cajas del encabezado de la factura. \\
%  |\LetraTitReporte| & Tipografía para el título del reporte. \\
%  |\LetraTitTotal| & Tipografía para el nombre del total final. \\
%  |\LetraTitTotales| & Tipografía para los nombres de los totales. \\
%  |\LineaNotaFinal| & Grosor de la línea de la caja de la nota final, \\
%                      & curvatura de la esquina y sombra. \\
%  |\LineaDescripcion| & Grosor de la línea de la caja de la descripción, \\
%                      & curvatura de la esquina y sombra. \\
%  |\LineaEncabezado| & Grosor de la línea de las cajas del encabezado de la factura, \\
%                     & curvatura de la esquina y sombra. \\
%  |\LineaNotaPrevia| & Grosor de la línea de la caja de la nota previa al cuadro, \\
%                     & curvatura de la esquina y sombra. \\
%  |\LineaFirmas| & Longitud y grosor de las líneas para firmar. \\
% \end{tabular}
% \newpage
% \noindent \begin{tabular}{>{\ttfamily}ll}
% \rmfamily \bfseries Comando & \bfseries Significado \\ \hline
%  |\Moneda| & Cambia el signo monetario a mostrar. \\
%  |\Nombre| & Establece el nombre del cliente. \\
%  |\NotaInterna| & Escribe una nota interna al cuadro de la factura. \\
%  |\NotaExterna| & Escribe una nota extra externa al cuadro de la factura. \\
%  |\NotaPrevia| & Escribe una nota antes del cuadro de la factura. \\
%  |\Num| & Número de la factura. \\
%  |\NumControl| & Número de control de la factura. \\
%  |\NumSerieControl| & Número de serie de control de la factura. \\
%  |\PosNotaFinal| & Posición $(x_o,x_f,y)$ del párrafo con la nota final. \\
%  |\PosNotaFinal*| & Posición $(x_o,x_f,y)$ del párrafo con la nota final \\
%                     & (el texto se despliega hacia arriba). \\
%  |\PosControl| & Posición $(x,y)$ del número de control. \\
%  |\PosControl*| & Posición $(x,y)$ del número de control \\
%                 & (el texto se despliega hacia arriba). \\
%  |\PosDenom| & Posición $(x,y)$ del texto de la denominación. \\
%  |\PosDenom*| & Posición $(x,y)$ del texto de la denominación \\
%               & (el texto se despliega hacia arriba). \\
%  |\PosFecha| & Posición $(x,y)$ de las fechas. \\
%  |\PosFecha*| & Posición $(x,y)$ de las fechas \\
%               & (el texto se despliega hacia arriba). \\
%  |\PosFirmas| & Posición $(x_o,x_f,y)$ de las firmas. \\
%  |\Proveedor| & Nombre de quien firma el reporte. \\
%  |\ptref| & Muestra al precio total del ítem referenciado. \\
%  |\puref| & Muestra al precio unitario del ítem referenciado. \\
%  |\RazonSocial| & Establece la razón social. \\
%  |\Resumen| & Resumen. \\
%  |\RIF| & R.I.F. del cliente. \\
%  |\SepNotaFinal| & Separación interna entre el borde y el texto en la nota final. \\
%  |\SepDatosResumen| & Separación horizontal entre los datos del cliente y el resumen. \\
%  |\SepDescripcion| & Separación interna entre el borde y el texto en la descripción. \\
%  |\SepEncabezado| & Separación interna entre el borde y el texto en el encabezado. \\
%  |\SepNotaPrevia| & Separación interna entre el borde y el texto en la nota previa. \\
%  |\SepFilas| & Separación horizontal y vertical en los totales en el modo de filas. \\
%  |\SepItemsExtra| & Separación vertical extra entre los elementos en la factura. \\
%  |\SepVertical| & Separación vertical entre los distintos elementos de la factura. \\
%  |\subtref| & Muestra al subtotal del ítem referenciado. \\
%  |\Telefono| & Teléfono del cliente. \\
%  |\TituloReporte| & Establece el título del reporte. \\
%  |\TituloReporte*| & Establece el título del reporte a salir en las cabeceras o pies de página. \\
% \end{tabular}
%
%
% \ifejemplos
%  \section{Ejemplos incluídos\label{sec:ejemplos}}
%  \todo*{Hay que hacer que en la documentación, toda opción y macro referencie al ejemplo donde se usa.}
%
%  Desde la página siguiente se muestran todos los ejemplos incluídos en la clase:
%
%  \newpage
%  \newgeometry{marginparsep=0pt,vmargin={1cm,1cm},hmargin={1cm,1cm}}
%  {
%   \pagestyle{empty}
%   \setlength{\parindent}{0pt}
%   \hbadness=10000
%
%   \Ejemplo{ejemplo-cotizacion}
%   \Ejemplo{ejemplo-prefactura}
%   \newpage
%   \Ejemplo{ejemplo-factura01}
%   \Ejemplo{ejemplo-factura02}[\small\parskip=2pt]
%   \newpage
%   \Ejemplo{ejemplo-factura03}
%   \Ejemplo{ejemplo-factura04}
%   \newpage
%   \Ejemplo{ejemplo-factura05}
%   \Ejemplo{ejemplo-factura06}[\small]
%   \newpage
%   \Ejemplo[0.4]{ejemplo-factura07}[\small]
%   \Ejemplo{ejemplo-factura08}
%   \newpage
%   \Ejemplo*[0.49]{ejemplo-factura09}
%   \Ejemplo[0.49]{ejemplo-factura09}*{2}
%   \Ejemplo*{ejemplo-factura09}*
%   \newpage
%   \Ejemplo*[0.49]{ejemplo-factura10}
%   \Ejemplo[0.49]{ejemplo-factura10}*{2}
%   \Ejemplo*{ejemplo-factura10}*
%   \newpage
%   \Ejemplo*[0.49]{ejemplo-reporte1}
%   \Ejemplo[0.49]{ejemplo-reporte1}*{2}
%   \Ejemplo[0.49]{ejemplo-reporte1}{4}[\slshape Por espacio se omitió la página 2 en esta representación.]
%   \newpage
%   \Ejemplo*[0.49]{ejemplo-reporte2}
%   \Ejemplo[0.49]{ejemplo-reporte2}*{2}
%   \Ejemplo[0.49]{ejemplo-reporte2}{3}
%   \newpage
%   \Ejemplo*[0.49]{ejemplo-reporte3}
%   \Ejemplo[0.49]{ejemplo-reporte3}*{2}
%   \Ejemplo[0.49]{ejemplo-reporte3}{3}
%   \newpage
%   \Ejemplo{beispiel-rechnung}
%   \newpage
%  }
%  \hbadness=1000
%  \restoregeometry
% \else
%  \message{^^JLaTeX Warning: Compilacio'n ra'pida. No se incluyeron los ejemplos.^^J}
% \fi
%
% \StopEventually{\endorgo}
%
%    \begin{macrocode}
%<*class>
\RequirePackage{etoolbox}%  'etoolbox' para los condicionales binarios y para \AtEndPreamble.

%% Condicionales auxiliares que sólo pueden ser usados dentro de un grupo.
\newbool{DCU}
\newbool{DUC}
\newbool{Descuentos}
%% (quitado por no vigente)
 % \newbool{Electro}
\newbool{ExpAbajo}\booltrue{ExpAbajo}
\newbool{IVA}
\newbool{NumItem}
\newbool{PrimeroDescripcion}
\newbool{SinCantidad}
\newbool{SinLineaHoriNI}
\newbool{SinLineaHoriTit}
\newbool{SinLineaHoriTot}
\newbool{SinLineasHoriExt}
\newbool{SinLineasVertExt}
\newbool{SinLineasVertInt}
\newbool{Subtotal}

%% Condicionales auxiliares para la posición de los objetos fijos cuando son medidos desde la derecha, o desde abajo.
\newbool{NotaFinalYarriba}
\newbool{NotaFinalYnegativo}% no hay NotaFinalXnegativo.
\newbool{ControlXnegativo}
\newbool{ControlYarriba}
\newbool{ControlYnegativo}
\newbool{DenomXnegativo}
\newbool{DenomYarriba}
\newbool{DenomYnegativo}
\newbool{FechaXnegativo}
\newbool{FechaYarriba}
\newbool{FechaYnegativo}
\newbool{FirmasYnegativo}% no hay FirmasXnegativo; hay X_inicial y X_final para las firmas.

%% Condicionales relativos a opciones de la clase.
\newif\ifagrupado
\newif\ifcentradesc
\newif\ifcentrado
\newif\ifconlineasha
\newif\ifconreporte
\newif\ifcotizacion
\newif\ifcsv
\newif\ifdcu
\newif\ifdenom
\newif\ifdescuentos
\newif\ifdosfirmas
\newif\ifduc
%% (quitado por no vigente)
 % \newif\ifelectro
\newif\ifexpandecuadro
\newif\iffilas
\newif\ifG
\newif\ifimpdesc
\newif\ifiva
\newif\iflinea
\newif\ifnospanish
\newif\ifnumitem
\newif\ifprefactura
\newif\ifsincantidad
\newif\ifsinexpandir
\newif\ifsinfirmas
\newif\ifsiniva
\newif\ifsinivaexpresado
\newif\ifsinlineahni
\newif\ifsinlineahtit
\newif\ifsinlineahtot
\newif\ifsinlineash
\newif\ifsinlineashe
\newif\ifsinlineashi
\newif\ifsinlineasv
\newif\ifsinlineasve
\newif\ifsinlineasvi
\newif\ifsinmarcas
\newif\ifsinmonedatotales
\newif\ifsinnumero
\newif\ifsinreverso
\newif\ifsintotales
\newif\ifsoloreporte
\newif\ifsubtotal
\newif\iftodosiva
\newif\iftwoside
\newif\ifunafirma
\newif\ifvencimiento

%% Condicionales auxiliares.
\newif\if@cdu \@cdutrue
\newif\if@factura \@facturatrue
\newif\if@reverso \@reversotrue
\newif\if@reporte
\newif\if@FondoFactura
\newif\if@FondoReporte
\newif\if@FondoReporteStar
\newif\if@Nombre
\newif\if@PruebaRef
\newif\if@ExpandeCuadro
\newif\if@LineaDesdeInicio
\newif\if@LineaHastaTotal
\newif\if@LineaHastaFinal
\newif\if@LineaDescripcion
\newif\if@BIR
\newif\if@BIA
\newif\if@Exento
\newif\if@DescG
\newif\if@DescR
\newif\if@DescA
\newif\if@DescE
\newif\if@DescGnum
\newif\if@DescRnum
\newif\if@DescAnum
\newif\if@DescEnum
\newif\if@DescuentoTotal

%% Longitudes para uso interno.
\newlength{\@TamDatos}
\newlength{\@TamResumen}
\newlength{\@TamFinalEncabezado}
\newlength{\@TamLineaFirmas}
\newlength{\@TamNumItem}
\newlength{\@TamCantidad}
\newlength{\@TamDescripcion}
\newlength{\@TamPrecioU}
\newlength{\@TamPrecioS}
\newlength{\@TamDescuento}
\newlength{\@TamIVA}
\newlength{\@TamPrecioT}
\newlength{\@GrosorLineaFirmas}
\newlength{\@GrosorLineaEncabezado}
\newlength{\@GrosorLineaNotaPrevia}
\newlength{\@GrosorLineaDescripcion}
\newlength{\@GrosorLineaNotaFinal}
\newlength{\@GrosorLineaInfoPag}
\newlength{\@EsquinaLineaEncabezado}
\newlength{\@EsquinaLineaNotaPrevia}
\newlength{\@EsquinaLineaDescripcion}
\newlength{\@EsquinaLineaNotaFinal}
\newlength{\@SombraEncabezado}
\newlength{\@SombraNotaPrevia}
\newlength{\@SombraDescripcion}
\newlength{\@SombraNotaFinal}
\newlength{\@LineaNodoXo}
\newlength{\@LineaNodoYo}
\newlength{\@LineaNodoXf}
\newlength{\@LineaNodoYf}
\newlength{\@SepTotales}
\newlength{\@PosFechaX}
\newlength{\@PosFechaY}
\newlength{\@PosDenomX}
\newlength{\@PosDenomY}
\newlength{\@PosFirmasXuno}
\newlength{\@PosFirmasXdos}
\newlength{\@PosFirmasY}
\newlength{\@PosNotaFinalXuno}
\newlength{\@PosNotaFinalXdos}
\newlength{\@PosNotaFinalY}
\newlength{\@PosControlX}
\newlength{\@PosControlY}
\newlength{\@EspacioVacioInicial}
\newlength{\@EspacioVacioFinal}
\newlength{\@EspacioSignoMonetario}
\newlength{\@SeparaItemsExtra}
\newlength{\@SeparaItemsExtraFinal}
\newlength{\@SeparaInternaEncabezado}
\newlength{\@SeparaInternaNotaPrevia}
\newlength{\@SeparaInternaDescripcion}
\newlength{\@SeparaInternaNotaFinal}
\newlength{\@SeparaDatosResumen}
\newlength{\@SeparaVertical}
\newlength{\@SeparaMinTotales}
\newlength{\@SeparaFilas}
\newlength{\@SeparaFinLineaDesc}
\newlength{\@SeparaTelefonoEmail}
\newlength{\@InicioReporte}
\newlength{\@FirmaFacturaXcorr}
\newlength{\@FirmaFacturaYcorr}
\newlength{\@FirmaReporteXcorr}
\newlength{\@FirmaReporteYcorr}
\newlength{\@FirmaReporteTextPos}
\newlength{\@FirmaReporteFirmaPos}
\newlength{\@FirmaReporteTamLinea}
\newlength{\@FirmaReportePosX}
\newlength{\@FirmaReporteExcesoLinea}
\newlength{\@TempDimen}
\newlength{\tabcolsep@@}
\newlength{\baselineskip@@}

%% Cajas para uso interno.
\newbox{\@CajaDatos}
\newbox{\@CajaResumen}
\newbox{\@CajaFirmaEmisor}
\newbox{\@CajaFirmas}
\newbox{\@CajaFecha}
\newbox{\@CajaDenom}
\newbox{\@CajaControl}
\newbox{\@CajaNotaFinal}
\newbox{\@CajaEspacioVacio}
\newbox{\@CajaPrueba}
\newbox{\@CajaCaracterCR}

%% Contadores.
\newcounter{@IVANoGeneral}%  '0' con sólo IVA G, '1' cuando hay IVA R o IVA A, y '2' cuando hay ambos.
\newcounter{@Firmas}%  '0' para sinfirmas, '1' para unafirma, y '2' para dosfirmas.
\setcounter{@Firmas}{2}
\newcounter{NumItem}%  Número del ítem actual, para uso de algunas macros en el cuadro de la factura y para cuando se usa la opción 'numitem'.
\newcount\AuxFecha

%% Errores, advertencias e informaciones de LaTeX.
\newcommand{\CLASSERROR}[1]{\ClassError{factura}{** #1. **}{}}
\newcommand{\CLASSWARNING}[1]{\ClassWarning{factura}{** #1. **}{}}
\newcommand{\CLASSINFO}[1]{\ClassInfo{factura}{** #1. **}{}}

%% Opciones de la clase.
\DeclareOption{agrupado}{\agrupadotrue}%  Agrupados los elementos totalizadores en el modo en columnas.
\DeclareOption{centradesc}{\centradesctrue}%  Para centrar la descripción de cada ítem.
\DeclareOption{centrado}{\centradotrue}%  Total y exento 'centrado'; sólo para estilo 'filas'
\DeclareOption{conlineasha}{\conlineashatrue}%  Una línea horizontal para cada ítem, como las facturas a mano.
\DeclareOption{conreporte}{\conreportetrue}%  Para imprimir factura y reporte.
\DeclareOption{cotizacion}{\cotizaciontrue}%  Cambia la denominación a 'COTIZACIÓN' y no 'FACTURA'.
\DeclareOption{csv}{\csvtrue}%  Para la salida en modo csv también.
\DeclareOption{dcu}{\dcutrue}%  Descripción, luego cantidad y luego precio unitario.
\DeclareOption{denom}{\denomtrue}%  Agrega el texto con la denominación del documento, para las formas libres que no tienen el número de factura.
\DeclareOption{descuentos}{\descuentostrue}%  Para los descuentos individuales.
\DeclareOption{dosfirmas}{\dosfirmastrue}%  Pone las dos firmas para firmar.
\DeclareOption{duc}{\ductrue}%  Descripción, luego precio unitario y luego cantidad.
\DeclareOption{electro}{\CLASSERROR{Ya no esta' vigente el descuento en la ali'cuota general por pagos electro'nicos}}%  Parche por el Decreto 3.085 de Nikolái, ya no vigente.
\DeclareOption{expandecuadro}{\expandecuadrotrue}%  Para rellenar con líneas en blanco de manera automática.
\DeclareOption{filas}{\filastrue}%  Estilo de totalización en filas.
\DeclareOption{G}{\Gtrue}%  Con esta opción sale la letra 'G' en los items con IVA general, y sale el porcentaje en la base imponible general.
\DeclareOption{impdesc}{\impdesctrue}%  Para el tipo de impuesto en la descripción y no en el precio unitario y total.
\DeclareOption{iva}{\ivatrue}%  Para la columna del IVA.
\DeclareOption{linea}{\lineatrue}%  Para la línea que inutiliza el espacio en blanco.
\DeclareOption{nospanish}{\nospanishtrue}%  No carga babel en castellano ni la codificación T1 de las letras.
\DeclareOption{numitem}{\numitemtrue}%  Para poner una columna extra al principio con el número del ítem.
\DeclareOption{prefactura}{\prefacturatrue}%  La prefactura; con el diseño de la forma libre... para enviar por correo electrónico, fax o imprimir preliminarmente.
\DeclareOption{sincantidad}{\sincantidadtrue}%  Para no poner el campo de la cantidad.
\DeclareOption{sinexpandir}{\sinexpandirtrue}%  No expande ninguna de las dos partes de la factura.
\DeclareOption{sinfirmas}{\sinfirmastrue}%  No agrega los espacios para firmar.
\DeclareOption{sinivaexpresado}{\sinivaexpresadotrue}%  Cuando no se cobra IVA y tampoco se quiere que aparezca la totalización del IVA.
\DeclareOption{siniva}{\sinivatrue}%  No cobra IVA.
\DeclareOption{sinlineahni}{\sinlineahnitrue}%  Estilo 'sinlineahni'.
\DeclareOption{sinlineahtot}{\sinlineahtottrue}%  Estilo 'sinlineahtot'.
\DeclareOption{sinlineahtit}{\sinlineahtittrue}%  Estilo 'sinlineahtit'.
\DeclareOption{sinlineashe}{\sinlineashetrue}%  Estilo 'sinlineashe'.
\DeclareOption{sinlineashi}{\sinlineashitrue}%  Estilo 'sinlineashi'.
\DeclareOption{sinlineash}{\sinlineashtrue}%  Estilo 'sinlineash'.
\DeclareOption{sinlineasve}{\sinlineasvetrue}%  Estilo 'sinlineasve'.
\DeclareOption{sinlineasvi}{\sinlineasvitrue}%  Estilo 'sinlineasvi'.
\DeclareOption{sinlineasv}{\sinlineasvtrue}%  Estilo 'sinlineasv'.
\DeclareOption{sinmarcas}{\sinmarcastrue}%  Para evitar que aparezcan los caracteres especiales cuando la descripción es larga.
\DeclareOption{sinmonedatotales}{\sinmonedatotalestrue}%  Para que no ponga el signo monetario en los totales.
\DeclareOption{sinnumero}{\sinnumerotrue}%  Escribir el texto de la denominación del documento pero sin escribir el número.
\DeclareOption{sinreverso}{\sinreversotrue}%  Para no imprimir el reverso de la factura.
\DeclareOption{sintotales}{\sintotalestrue}%  No escribe los totales al final, por ejemplo para cotizaciones de productos similares.
\DeclareOption{soloreporte}{\soloreportetrue}%  Para imprimir sólo el reporte.
\DeclareOption{subtotal}{\subtotaltrue}%  Para la columna extra del subtotal antes del descuento.
\DeclareOption{todosiva}{\todosivatrue}%  Con esta opción salen todos los IVA así valgan cero.
\DeclareOption{twoside}{\twosidetrue}%  Opción 'twoside' de article para los reportes.
\DeclareOption{unafirma}{\unafirmatrue}%  No agrega el espacio para que firme quien recibe.
\DeclareOption{vencimiento}{\vencimientotrue}%  Escribe fecha de vencimiento, aún con facturas sin crédito.

\ProcessOptions\relax

\LoadClass[12pt]{article}

%% Cargamos algunos paquetes siempre necesarios.
\RequirePackage{xparse}%  Algunos comandos los definimos con \NewDocumentCommand.
\RequirePackage[absolute]{textpos}%  Es necesario 'textpos' para colocar texto en ubicaciones arbitrarias.
\RequirePackage{eso-pic}%  Para el comando \AddToShipoutPicture.
\RequirePackage{xstring}%  Lo usamos sólo para tomar el menos al principio de una longitud.
\RequirePackage{calc}%  'calc' para hacer aritmética con las dimensiones.
\RequirePackage{fp-upn}%  Fundamental el paquete 'fp-upn' para todos los cálculos.
\RequirePackage{numprint}%  Con 'numprint' expresamos los totales en el formato requerido.
\RequirePackage{tikz}%  Para los recuadros del encabezado de la factura,
 \usetikzlibrary{shadows}%  y para el de la descripción.
\RequirePackage{tabularx}%  Con 'tabularx' hacemos la tabla.
\RequirePackage{environ}%  Con 'environ' podemos usar tabularx dentro del ambiente de la factura y escribir texto en el enddef.
\RequirePackage{fancyhdr}%  Las cabeceras fancy; así se use solamente en reporte, hay que cargarlo siempre.
\RequirePackage[calc]{datetime2}%  Para calcular la fecha del vencimiento del crédito.
\RequirePackage{anyfontsize}%  Para las tipografías en tamaños arbitrarios.
\RequirePackage{ifxetex}%  Para poder modificar correctamente la metadata del PDF y agregar la versión de la clase.
\RequirePackage{ifluatex}%  Ditto.
\RequirePackage{zref-savepos}[2016/05/21]%  Para guardar la posición en 'Y' expandiendo el cuadro, para la línea y para la firma en el reporte si la hubiera.
%% Hay que cargar siempre zref-savepos para poder compilar con LuaLaTeX porque este motor no tiene \pdfsavepos, \pdflastypos y zref-savepos las provee cuando no existen (motor LuaLaTex).

%% Gracias a Alan Munn y a wipet (Thanks to Alan Munn and wipet).
%% (https://tex.stackexchange.com/questions/396363/input-a-date-to-datetime2-in-a-format-different-than-iso)
\def\DTMsavedatex#1#2{%
 \expandafter \ifx \csname f=\DTMinformat\endcsname \relax
  \CLASSERROR{\string\DTMsavedatex: formato "\DTMinformat" desconocido}%
 \else
  \csname f=\DTMinformat\expandafter\endcsname#2&{#1}%
 \fi
}
\def\DTMfdef#1{\expandafter\def\csname f=#1\endcsname}
\def\DTMs#1#2{\DTMsavedate{#2}{#1}}
\DTMfdef{yyyy-mm-dd}#1-#2-#3&{\DTMs{#1-#2-#3}}
\DTMfdef{yyyy.mm.dd}#1.#2.#3&{\DTMs{#1-#2-#3}}
\DTMfdef{yyyymmdd}#1#2#3#4#5#6#7#8&{\DTMs{#1#2#3#4-#5#6-#7#8}}
\DTMfdef{dd-mm-yyyy}#1-#2-#3&{\DTMs{#3-#2-#1}}
\DTMfdef{dd.mm.yyyy}#1.#2.#3&{\DTMs{#3-#2-#1}}
\DTMfdef{ddmmyyyy}#1#2#3#4#5#6#7#8&{\DTMs{#5#6#7#8-#3#4-#1#2}}
\DTMfdef{mm-dd-yyyy}#1-#2-#3&{\DTMs{#3-#1-#2}}

%% Macros internas para el texto de la metadata que se va a escribir con el número de la versión de la clase.
\edef\@factura@ver{\expandafter\csname ver@factura.cls\endcsname}
\ExplSyntaxOn
  \seq_set_split:NnV \l_tmpa_seq {~} {\@factura@ver}
  \tl_set:Nx \@factura@vernum {\seq_item:Nn \l_tmpa_seq {2}}

  \seq_set_split:NnV \l_tmpa_seq {~} {\@factura@ver}
  \tl_set:Nx \@factura@fecha {\seq_item:Nn \l_tmpa_seq {1}}
\ExplSyntaxOff
\def\@factura@ver{Hecho con la clase factura-\@factura@vernum\ (\@factura@fecha)}

%% Escribimos la versión de la clase en la metadata del PDF.
\ifxetex
 \special{pdf: docinfo << /Subject (\@factura@ver) >>}
\else
 \ifluatex
  \pdfextension info{/Subject (\@factura@ver)}
 \else
  \pdfinfo{/Subject (\@factura@ver)}
 \fi
\fi

\def\@EstableceOpciones{%
 % Establecemos los parámetros y macros relativas a la fecha.
 \DTMsetstyle{\@FormatoFecha}\DTMsetup{datesep=\@FechaSeparadores}\gdef\DTMinformat{\@FormatoFechaEntrada}
 
 % Tomamos las acciones pertinentes dependiendo de las opciones de la clase (o condicionales activados en el preámbulo que cambian las opciones de la clase)
 \ifprefactura
  \ifcotizacion
   \prefacturafalse
   \cotizacionfalse
   \CLASSWARNING{Las opciones 'prefactura' y 'cotizacion' no son compatibles. Sin efecto ambas opciones}
  \else
   \denomtrue\setcounter{@Firmas}{0}
  \fi
 \fi
 \ifcotizacion\denomtrue\setcounter{@Firmas}{1}\fi

 \ifsinlineasv\booltrue{SinLineasVertExt}\booltrue{SinLineasVertInt}\fi
 \ifsinlineasvi\booltrue{SinLineasVertInt}\fi
 \ifsinlineasve\booltrue{SinLineasVertExt}\fi
 \ifsinlineash\booltrue{SinLineasHoriExt}\booltrue{SinLineaHoriNI}\booltrue{SinLineaHoriTit}\booltrue{SinLineaHoriTot}\fi
 \ifsinlineashi\booltrue{SinLineaHoriNI}\booltrue{SinLineaHoriTit}\booltrue{SinLineaHoriTot}\fi
 \ifsinlineahni\booltrue{SinLineaHoriNI}\fi
 \ifsinlineahtit\booltrue{SinLineaHoriTit}\fi
 \ifsinlineahtot
  \iffilas
   \CLASSWARNING{Las opciones 'sinlineahtot' y 'filas' no son compatibles. Sin efecto opcio'n 'sinlineahtot'}
   \sinlineahtotfalse
  \else
   \booltrue{SinLineaHoriTot}
  \fi
 \fi
 \ifsinlineashe\booltrue{SinLineasHoriExt}\fi

 \ifdosfirmas
  \setcounter{@Firmas}{2}
  \ifunafirma
   \unafirmafalse
   \ifsinfirmas
    \sinfirmasfalse
    \CLASSWARNING{Las opciones para las firmas son excluyentes. Permanece activada 'dosfirmas'}
   \else
    \CLASSWARNING{Las opciones para las firmas son excluyentes. Permanece activada 'dosfirmas'}
   \fi
  \else
   \ifsinfirmas
    \sinfirmasfalse
    \CLASSWARNING{Las opciones para las firmas son excluyentes. Permanece activada 'dosfirmas'}
   \fi
  \fi
 \fi
 \ifunafirma
  \setcounter{@Firmas}{1}
  \ifsinfirmas
   \sinfirmasfalse
   \CLASSWARNING{Las opciones para las firmas son excluyentes. Permanece activada 'unafirma'}
  \fi
 \fi
 \ifsinfirmas\setcounter{@Firmas}{0}\fi 
 
 \ifdcu
  \@cdufalse\ducfalse
  \booltrue{DCU}\boolfalse{DUC}
  \booltrue{PrimeroDescripcion}
 \fi
 \ifduc
  \ifdcu
   \boolfalse{PrimeroDescripcion}\boolfalse{DCU}\dcufalse\ducfalse\@cdutrue
   \CLASSWARNING{Las opciones 'duc' y 'dcu' no son compatibles. Sin efecto ambas opciones}
  \else
   \ductrue\@cdufalse\dcufalse
   \booltrue{DUC}\boolfalse{DCU}
   \booltrue{PrimeroDescripcion}
  \fi
 \fi
 \ifnumitem\booltrue{NumItem}\fi
 \ifsincantidad
  \booltrue{SinCantidad}
  \ifdcu
   \dcufalse\@cdutrue\boolfalse{DCU}\boolfalse{PrimeroDescripcion}
   \CLASSWARNING{La opcio'n 'sincantidad' no es compatible con 'dcu'. Sin efecto opcio'n 'dcu'}
  \fi
  \ifduc
   \ducfalse\@cdutrue\boolfalse{DUC}\boolfalse{PrimeroDescripcion}
   \CLASSWARNING{La opcio'n 'sincantidad' no es compatible con 'duc'. Sin efecto opcio'n 'duc'}
  \fi
 \fi
 \ifdescuentos\booltrue{Descuentos}\fi
 \ifiva\impdesctrue\booltrue{IVA}\fi
 \ifsubtotal
  \ifdescuentos
   \booltrue{Subtotal}
  \else
   \CLASSWARNING{La opcio'n 'subtotal' no aplica sin descuentos individuales. Sin efecto opcio'n 'subtotal'}
  \fi
 \fi

 \ifsinnumero\denomtrue\fi
 \ifsintotales
  \iffilas
   \filasfalse
   \CLASSWARNING{La opcio'n 'sintotales' no aplica para el modo de totalizacio'n en filas. Sin efecto opcio'n 'filas'}
  \fi
  \ifcotizacion
  \else
   \CLASSWARNING{La opcio'n 'sintotales' no debe usarse en facturas sino en cotizaciones. No es va'lida una factura sin totales}
  \fi
 \fi

 \ifcentrado
  \iffilas
  \else
   \centradofalse
   \CLASSWARNING{La opcio'n 'centrado' no es compatible con el modo de totalizacio'n en columnas. Sin efecto opcio'n 'centrado'}
  \fi
 \fi
 \ifagrupado
  \iffilas
   \agrupadofalse
   \CLASSWARNING{La opcio'n 'agrupado' no es compatible con el modo de totalizacio'n en filas. Sin efecto opcio'n 'agrupado'}
  \fi
 \fi

%% (quitado por no vigente)
 % \ifelectro
 %  \boolfalse{Electro}
 %  \electrofalse
 %  \gdef\@NotaInterna{\@TextoDescuentoPresidencial}
 %  \CLASSWARNING{Se establecio' el texto aclaratorio del descuento presidencial como 'nota interna'}
 %  \iffilas
 %   \filasfalse
 %   \CLASSWARNING{La opcio'n 'electro' no es compatible con la opcio'n 'filas'. Sin efecto la opcio'n 'filas'}
 %  \fi
 % \fi

 \iftodosiva \@BIRtrue \@BIAtrue \@Exentotrue \Gtrue \fi
 \ifsiniva\AtBeginDocument{\gdef\@ImpuestoG{0}\gdef\@ImpuestoR{0}\gdef\@ImpuestoA{0}}\fi
 \ifsinivaexpresado
  \iffilas
   \filasfalse
   \CLASSWARNING{La opcio'n 'sinivaexpresado' no es compatible con la opcio'n 'filas'. Sin efecto la opcio'n 'filas'}
  \else
   \AtBeginDocument{\gdef\@ImpuestoG{0}\gdef\@ImpuestoR{0}\gdef\@ImpuestoA{0}}
  \fi
 \fi

 \iflinea
  \ifexpandecuadro
  \else
   \CLASSWARNING{La opcio'n 'linea' requiere la opcio'n 'expandecuadro'. Activando la opcio'n 'expandecuadro'}
   \expandecuadrotrue
  \fi
 \fi
 \ifsinmarcas \def\@CaracterCR{\mbox{}} \def\@CaracterCeldasVacias{\mbox{}} \fi
 \ifnospanish
 \else
  \RequirePackage[spanish]{babel}%  Babel en castellano.
  \ifxetex
  \else
   \ifluatex
   \else
    \RequirePackage[T1]{fontenc}%  Codificación T1 para las tipografías; no aplica para XeLaTeX ni LuaLaTeX.
   \fi
  \fi
 \fi

 \ifexpandecuadro
  \ifsinexpandir
   \expandecuadrofalse
   \sinexpandirfalse
   \CLASSWARNING{Las opciones 'expandecuadro' y 'sinexpandir' no son compatibles. Sin efecto ambas opciones}
  \else
   \boolfalse{ExpAbajo}
  \fi
 \fi
 \ifsinexpandir\boolfalse{ExpAbajo}\expandecuadrofalse\fi

 \ifsinreverso\@reversofalse\fi
 \ifconreporte
  \@reportetrue
  \ifsoloreporte
   \soloreportefalse
   \CLASSWARNING{La opcio'n 'soloreporte' no es compatible con 'conreporte'. Sin efecto 'soloreporte'}
  \fi
 \fi
 \ifsoloreporte\@reportetrue\@facturafalse\@reversofalse\fi

 \ifcsv
  \if@factura
   \newwrite\csv \immediate\openout\csv=\jobname.csv
   \ifsintotales\else\AtEndDocument{\@EscribeTotalesEnCSV}\fi
  \else
   \csvfalse
   \CLASSWARNING{La opcio'n 'csv' require generar la factura. Sin efecto opcio'n 'csv'}
  \fi
 \fi
 
 \iftwoside
  \@twosidetrue\@mparswitchtrue
 \else
  \@twosidefalse\@mparswitchfalse
 \fi}
 
\NewDocumentCommand{\BeforeEndPreamble}{s s +m}{
 \IfBooleanTF{#2}
  {\gpreto{\@BeforeEndPreamble}{#3}}
  {\IfBooleanTF{#1}{\long\gdef\@BeforeEndPreamble{#3}}{\gappto{\@BeforeEndPreamble}{#3}}}}

\AtEndPreamble{
 \@EstableceOpciones
 \ifx\@undefined\@BeforeEndPreamble\else\@BeforeEndPreamble\@EstableceOpciones\fi

 % Escribimos en el .log las opciones cargadas:
 \ifagrupado \CLASSINFO{Opcio'n cargada: agrupado}\fi
 \ifcentradesc \CLASSINFO{Opcio'n cargada: centradesc}\fi
 \ifcentrado \CLASSINFO{Opcio'n cargada: centrado}\fi
 \ifconlineasha \CLASSINFO{Opcio'n cargada: conlineasha}\fi
 \ifconreporte \CLASSINFO{Opcio'n cargada: conreporte}\fi
 \ifcotizacion \CLASSINFO{Opcio'n cargada: cotizacion}\fi
 \ifcsv \CLASSINFO{Opcio'n cargada: csv}\fi
 \ifdcu \CLASSINFO{Opcio'n cargada: dcu}\fi
 \ifdenom \CLASSINFO{Opcio'n cargada: denom}\fi
 \ifdescuentos \CLASSINFO{Opcio'n cargada: descuentos}\fi
 \ifdosfirmas \CLASSINFO{Opcio'n cargada: dosfirmas}\fi
 \ifduc \CLASSINFO{Opcio'n cargada: duc}\fi
 %% (quitado por no vigente)
 % \ifelectro \CLASSINFO{Opcio'n cargada: electro}\fi
 \ifexpandecuadro \CLASSINFO{Opcio'n cargada: expandecuadro}\fi
 \iffilas \CLASSINFO{Opcio'n cargada: filas}\fi
 \ifG \CLASSINFO{Opcio'n cargada: G}\fi
 \ifimpdesc \CLASSINFO{Opcio'n cargada: impdesc}\fi
 \ifiva \CLASSINFO{Opcio'n cargada: iva}\fi
 \iflinea \CLASSINFO{Opcio'n cargada: linea}\fi
 \ifnospanish \CLASSINFO{Opcio'n cargada: nospanish}\fi
 \ifnumitem \CLASSINFO{Opcio'n cargada: numitem}\fi
 \ifprefactura \CLASSINFO{Opcio'n cargada: prefactura}\fi
 \ifsincantidad \CLASSINFO{Opcio'n cargada: sincantidad}\fi
 \ifsinexpandir \CLASSINFO{Opcio'n cargada: sinexpandir}\fi
 \ifsinfirmas \CLASSINFO{Opcio'n cargada: sinfirmas}\fi
 \ifsiniva \CLASSINFO{Opcio'n cargada: siniva}\fi
 \ifsinivaexpresado \CLASSINFO{Opcio'n cargada: sinivaexpresado}\fi
 \ifsinlineahni \CLASSINFO{Opcio'n cargada: sinlineahni}\fi
 \ifsinlineahtit \CLASSINFO{Opcio'n cargada: sinlineahtit}\fi
 \ifsinlineahtot \CLASSINFO{Opcio'n cargada: sinlineahtot}\fi
 \ifsinlineash \CLASSINFO{Opcio'n cargada: sinlineash}\fi
 \ifsinlineashe \CLASSINFO{Opcio'n cargada: sinlineashe}\fi
 \ifsinlineashi \CLASSINFO{Opcio'n cargada: sinlineashi}\fi
 \ifsinlineasv \CLASSINFO{Opcio'n cargada: sinlineasv}\fi
 \ifsinlineasve \CLASSINFO{Opcio'n cargada: sinlineasve}\fi
 \ifsinlineasvi \CLASSINFO{Opcio'n cargada: sinlineasvi}\fi
 \ifsinmarcas \CLASSINFO{Opcio'n cargada: sinmarcas}\fi
 \ifsinmonedatotales \CLASSINFO{Opcio'n cargada: sinmonedatotales}\fi
 \ifsinnumero \CLASSINFO{Opcio'n cargada: sinnumero}\fi
 \ifsinreverso \CLASSINFO{Opcio'n cargada: sinreverso}\fi
 \ifsintotales \CLASSINFO{Opcio'n cargada: sintotales}\fi
 \ifsoloreporte \CLASSINFO{Opcio'n cargada: soloreporte}\fi
 \ifsubtotal \CLASSINFO{Opcio'n cargada: subtotal}\fi
 \iftodosiva \CLASSINFO{Opcio'n cargada: todosiva}\fi
 \iftwoside \CLASSINFO{Opcio'n cargada: twoside}\fi
 \ifunafirma \CLASSINFO{Opcio'n cargada: unafirma}\fi
 \ifvencimiento \CLASSINFO{Opcio'n cargada: vencimiento}\fi

 % Cargamos aquí 'geometry' para que sea después de haber establecido 'twoside'.
 \RequirePackage{geometry}}

%% Escribimos el total como comentario en el .log.
\AtEndDocument{%
 \ifsintotales
 \else
  \if@factura\CLASSINFO{\@@Total: \space\space\space \Moneda \Total}\fi
 \fi}

%</class>
%<*class|def>
%% La geometría para la factura, su reverso, y el reporte.
%% Para la factura, el margen superior está definido desde el encabezado con los datos del cliente;
%%  el margen inferior está definido hasta el final de la descripción (o del cuadro de la factura si no hay descripción).
%%  Hay elementos que salen fuera de estos márgenes, como las firmas, la nota final, la numeración y la fecha.
%<class>\def\GeometriaFactura#1{\gdef\@GeometriaFactura{#1}}
%<class>\def\GeometriaReverso#1{\gdef\@GeometriaReverso{#1}}
%<class>\def\GeometriaReporte#1{\gdef\@GeometriaReporte{#1}}
%<def>\GeometriaFactura{letterpaper,vmargin={5.2cm,5.5cm},hmargin={1cm,1cm},footskip=0pt,  headsep=0pt,    headheight=0pt}
%<def>\GeometriaReverso{letterpaper,vmargin={1cm,1cm},    hmargin={1cm,1cm},footskip=0cm,  headsep=0pt,    headheight=0pt}
%<def>\GeometriaReporte{letterpaper,vmargin={2.8cm,2.5cm},hmargin={3cm,2cm},footskip=1.2cm,headsep=0.777cm,headheight=1.5cm}

%<class>%% Se utiliza '\@Impuesto' como el factor del impuesto, el cual viene predefinido en '\@Alicuota'.
%<class>\def\@ImpuestoG{\@AlicuotaG}
%<class>\def\@ImpuestoR{\@AlicuotaR}
%<class>\def\@ImpuestoA{\@AlicuotaA}
%<def>%% Factores para los distintos IVA:
%<def>\def\@AlicuotaG{0.16}
%<def>%% (quitado por no vigente)
%<def> % \def\@AlicuotaGDescBajo{0.03}
%<def> % \def\@AlicuotaGDescAlto{0.05}
%<def> % \def\@AlicuotaGDescLimite{2000000}
%<def>\def\@AlicuotaR{0.08}
%<def>\def\@AlicuotaA{0.27}
%<def>
%</class|def>
%<*def>
%% i) separador de miles, ii) separador decimal, iii) número de decimales para la moneda iv) y para los porcentajes, v) número de dígitos de la numeración de la factura, vi) formato de la fecha, vii) separadores para la fecha.
\npthousandsep{.}
\npdecimalsign{,}
\def\@DigitosMoneda{2}
\def\@DigitosPorc{2}
\def\@DigitosNumeracion{5}
\def\@FormatoFecha{ddmmyyyy}
\def\@FormatoFechaEntrada{dd-mm-yyyy}
\def\@FechaSeparadores{-}

%% La moneda:
\def\@Moneda{Bs.}
%% (quitado por no vigente)
 % \def\@MonedaBs{Bs.}
 % \def\@MonedaBsS{Bs.S}
 % \def\@MonedaBsF{Bs.F}
\setlength\@EspacioSignoMonetario{.666ex}

%% (quitado por no vigente)
%% Las fechas de la entrada de la reconversión monetaria:
 % \DTMsavedate{FechaBsS}{2018-08-20}
 % \DTMsavedate{FechaBs}{2019-02-15}

%% Al lado de cada ítem va la letra que designa el gravamen que aplica sobre él:
\def\@DenomAlicuotaG{G}
\def\@DenomAlicuotaR{R}
\def\@DenomAlicuotaA{A}
\def\@DenomAlicuotaE{(E)}

%% (quitado por no vigente)
%% Texto predeterminado para la nota interna por el descuento presidencial.
 % \def\@TextoDescuentoPresidencial{Se aplica un descuento del \@AlicuotaGDescPorc\% al IVA General seg\'un Decreto Presidencial 3.085 en Gaceta Oficial 41.239.}

%% Macros que guardan textos que corresponden a nombres de campos:
\def\@@Fecha{Fecha de emisi\'on:}
\def\@@FechaVencimiento{Fecha de vencimiento:}
\def\@@Control{N\textordmasculine\space CONTROL}
\def\@@Datos{Datos del cliente}
\def\@@Resumen{Resumen del servicio}
\def\@@DenomFactura{FACTURA}
\def\@@DenomCotiza{COTIZACI\'ON}
\def\@@RazonSocial{Raz\'on social:}
\def\@@Nombre{Nombre:}
\def\@@RIF{R.I.F.:}
\def\@@CI{C.I.:}
\def\@@Direccion{Direcci\'on fiscal:}
\def\@@Telefono{Tel\'efono:}
\def\@@Email{E-mail:}
\def\@@Contacto{Persona de contacto:}
\def\@@Proveedor{Proveedor:}
\def\@@Cliente{Cliente:}
\def\@@Emisor{Por el emisor}
\def\@@Conforme{Recib\'i conforme}
\def\@@Pagado{PAGADO}
\def\@@NumItem{\#}
\def\@@Cantidad{Cantidad}
\def\@@Descripcion{Descripci\'on}
\def\@@Precio{Precio}
\def\@@PrecioU{Precio U.}
\def\@@PrecioS{Subtotal}
\def\@@Descuento{Dcto.}
\def\@@IVA{IVA}
\def\@@PrecioT{Precio T.}
\def\@@AlicuotaG{IVA G}
%% (quitado por no vigente)
 % \def\@@AlicuotaGDesc{Descuento IVA G}
\def\@@AlicuotaR{IVA R}
\def\@@AlicuotaA{IVA A}
\def\@@BIG{BI G}
\def\@@BIR{BI R}
\def\@@BIA{BI A}
\def\@@BIE{Exento}% La base imponible exenta. ;)
\def\@@Total{TOTAL}
\def\@@DescuentoG{Descuento G}
\def\@@DescuentoR{Descuento R}
\def\@@DescuentoA{Descuento A}
\def\@@DescuentoE{Descuento E}
\def\@@SubtotalG{Subtotal G}
\def\@@SubtotalR{Subtotal R}
\def\@@SubtotalA{Subtotal A}
\def\@@SubtotalE{Subtotal E}

%% i) carácter para separar el teléfono del e-mail del cliente,
%% ii) carácter para los campos vacíos cuando se usa \ItemX,
%% iii) carácter para las celdas vacías cuando la descripción del ítem ocupa más que una línea.
%% iv) carácter de retorno de carro en el cuadro de la factura,
\def\@CaracterSeparaTelefonoEmail{$\diamond$}
\def\@CaracterItemXVacio{...}
\def\@CaracterCeldasVacias{\mbox{}\hfill$\cdot$\hfill\mbox{}}
\def\@CaracterCR{\flechaCR}
% \def\@CaracterCR{$\hookleftarrow$}
% \def\@CaracterCR{$\cwrightarcarrow$}% Usando el paquete 'fnsymbol'.
% \def\@CaracterCR{$\cwsearcarrow$}% Usando el paquete 'fnsymbol'.
\def\flechaCR{%
 \raisebox{-0.4ex}{%
  \begin{tikzpicture}
   \draw[line width=0.075ex] (0,0) arc (20:70:0.75ex) + (0,0) arc (-70:-20:0.75ex) ++ (0.8ex,0.55ex) arc (0:-100:1ex);
  \end{tikzpicture}}}

%% Separación entre el inicio de una columna en la tabla y el inicio del texto.
\tabcolsep=0.15cm

%% Separación entre la última letra y el carácter que indica que se continúa en la línea siguiente en la descripción cuando ocupa más de una línea.
\setlength{\@SeparaFinLineaDesc}{0.15cm}

%% Separación entre el teléfono y el carácter \@CaracterSeparaTelefonoEmail, y luego desde ese carácter hasta el email,
%% para cuando no hay resumen.
\setlength{\@SeparaTelefonoEmail}{0.666em}

%% Grosor de la línea de 'fancyhdr' para las cabeceras, en aquellas donde haya línea.
\setlength{\@GrosorLineaInfoPag}{0.4pt}

%% Posición en X desde la derecha y excedente en la línea de la firma.
\setlength{\@FirmaReportePosX}{2cm}
\setlength{\@FirmaReporteExcesoLinea}{1.5cm}
%</def>
%<class|def>
%<*class>
%% Fijamos en cero los descuentos totales (necesario).
\def\@DescuentoGporc{0}
\def\@DescuentoRporc{0}
\def\@DescuentoAporc{0}
\def\@DescuentoEporc{0}
\def\@DescuentoGnum{0}
\def\@DescuentoRnum{0}
\def\@DescuentoAnum{0}
\def\@DescuentoEnum{0}
\FPset\PrecioUtempx{1}

%% \Fecha, \FormatoFecha, \Credito, \Moneda, \RazonSocial, \Nombre, \RIF, \CI, \Contacto, \Dirección, \Telefono, \Email, \Conforme, \Emisor, \Proveedor.
\NewDocumentCommand{\Fecha}{s m g g}{%
 \IfBooleanTF{#1}
  {\AtEndPreamble{\DTMsavedate{FechaEmision}{#2-#3-#4}}}
  {\AtEndPreamble{\DTMsavedatex{FechaEmision}{#2}}}}
\def\@Fecha{\DTMusedate{FechaEmision}}

\NewDocumentCommand{\FormatoFecha}{o g g}{%
 \IfValueT{#1}{\gdef\@FormatoFechaEntrada{#1}}\IfValueT{#2}{\gdef\@FormatoFecha{#2}}\IfValueT{#3}{\gdef\@FechaSeparadores{#3}}}

\def\Credito#1{\gdef\@Credito{#1}\vencimientotrue}
\def\Moneda#1{\gdef\@Moneda{#1}}
\def\RazonSocial#1{\gdef\@RazonSocial{#1}}
\def\Nombre#1{\gdef\@RazonSocial{#1}\@Nombretrue}
\def\RIF#1{\gdef\@RIF{#1}}
\def\CI#1{\gdef\@RIF{#1}\let\@@RIF\@@CI\let\@CI\@RIF}
\def\Contacto#1{\gdef\@Contacto{#1}}
\def\Direccion#1{\gdef\@Direccion{#1}}
\def\Telefono#1{\gdef\@Telefono{#1}}
\def\Email#1{\gdef\@Email{#1}}
\def\Conforme#1{\gdef\@Conforme{#1}}
\def\Emisor#1{\gdef\@Emisor{#1}}
\def\Proveedor#1{\gdef\@Proveedor{#1}}

%% Los descuentos totales.
\NewDocumentCommand{\Descuento}{s m}{%
 \IfBooleanTF{#1}
  {\@DescGnumtrue\@DescRnumtrue\@DescAnumtrue\@DescEnumtrue
  \gdef\@DescuentoGnum{#2}\gdef\@DescuentoRnum{#2}\gdef\@DescuentoAnum{#2}\gdef\@DescuentoEnum{#2}}
  {\@DescGnumfalse\@DescRnumfalse\@DescAnumfalse\@DescEnumfalse
  \gdef\@DescuentoGporc{#2}\gdef\@DescuentoRporc{#2}\gdef\@DescuentoAporc{#2}\gdef\@DescuentoEporc{#2}}%
 \@DescGtrue\@DescRtrue\@DescAtrue\@DescEtrue\@DescuentoTotaltrue}

\NewDocumentCommand{\DescuentoG}{s m}{%
 \IfBooleanTF{#1}{\@DescGnumtrue\gdef\@DescuentoGnum{#2}}{\@DescGnumfalse\gdef\@DescuentoGporc{#2}}%
 \@DescGtrue\@DescuentoTotaltrue}

\NewDocumentCommand{\DescuentoR}{s m}{%
 \IfBooleanTF{#1}{\@DescRnumtrue\gdef\@DescuentoRnum{#2}}{\@DescRnumfalse\gdef\@DescuentoRporc{#2}}%
 \@DescRtrue\@DescuentoTotaltrue}

\NewDocumentCommand{\DescuentoA}{s m}{%
 \IfBooleanTF{#1}{\@DescAnumtrue\gdef\@DescuentoAnum{#2}}{\@DescAnumfalse\gdef\@DescuentoAporc{#2}}%
 \@DescAtrue\@DescuentoTotaltrue}

\NewDocumentCommand{\DescuentoE}{s m}{%
 \IfBooleanTF{#1}{\@DescEnumtrue\gdef\@DescuentoEnum{#2}}{\@DescEnumfalse\gdef\@DescuentoEporc{#2}}%
 \@DescEtrue\@DescuentoTotaltrue}

\def\Denom#1{\gdef\@Denominacion{#1}}%  La denominación puede ser 'FACTURA', 'COTIZACIÓN' u otra. Se predetermina una de estas dos según las opciones de la clase.

%% Número de la factura con \Num, número de control con \NumControl, y número de serie de control \NumSerieControl.
\def\Num#1{\gdef\@Num{\nplpadding{\@DigitosNumeracion}\npthousandsep{}\numprint{#1}}}
\def\NumControl#1{\gdef\@NumControl{\nplpadding{\@DigitosNumeracion}\npthousandsep{}\numprint{#1}}}%  Si el número de control no se especifica, toma el mismo que el de la factura
\def\NumSerieControl#1{\gdef\@NumSerieControl{#1}}%  Si el número de serie de control no se especifica, toma '00'.

%% Información de página: cabeceras de páginas o pies de página.
\def\InfoPagina#1{\gdef\@InfoPagina{#1}}

%% Texto para la nota final.
\NewDocumentCommand{\NotaFinal}{+m}{\gdef\@NotaFinal{#1}}
\NewDocumentCommand{\Condiciones}{+m}{\gdef\@NotaFinal{#1}\@ClassWarningCondiciones}

%% Advertencia para la migración de cualquier comando para condiciones a nota final.
\def\@ClassWarningCondiciones{\CLASSWARNING{Ya no son condiciones sino nota final, debe hacerse la migración a los nuevos comandos}}

%% Resumen opcional que va a la derecha de los datos del cliente.
\NewDocumentCommand{\Resumen}{+m}{\gdef\@Resumen{#1}}

\NewDocumentCommand{\NotaPrevia}{+m}{\gdef\@NotaPrevia{#1}}

%% Nota interna al cuadro de la factura y nota externa a él (que sale antes de \Descripcion).
\NewDocumentCommand{\NotaInterna}{+m}{%
 \ifdefvoid{\@NotaInterna}{}{\CLASSWARNING{Se sobreescribio' la nota interna previamente definida}}%
 \gdef\@NotaInterna{#1}}
\NewDocumentCommand{\NotaExterna}{+m}{\gdef\@NotaExterna{#1}}

\AtBeginDocument{%
 % Si no se especificó fecha, tomamos la de hoy. Además, advertimos cuando el documento se está compilando en una fecha distinta de la fecha establecida en el encabezado.
 \DTMifsaveddate{FechaEmision}{}{%
  \CLASSWARNING{No se especifico' fecha. Asumiendo la fecha de hoy}%
  \DTMsavedate{FechaEmision}{\the\year-\the\month-\the\day}}%
 \DTMsavenow{Ahora}%
 \DTMsaveddatediff{FechaEmision}{Ahora}{\AuxFecha}%
 \ifnum\AuxFecha=0%
 \else
  \CLASSWARNING{Fecha del documento \@Fecha\space distinta de la fecha de hoy}%
 \fi

%% (quitado por no vigente)
 % Establecemos el signo monetario dependiendo de la fecha de emisión de la factura, para efectos de la Reconversión Monetaria:
 % \ifx\@undefined\@Moneda
 %  \DTMsaveddatediff{FechaEmision}{FechaBsS}{\AuxFecha}%
 %  \ifnum\AuxFecha<0%
 %   \Moneda{\@MonedaBs}%
 %   \LetraNotaInterna{\fontsize{14}{18}}%
 %   \NotaInterna{\FPupn\TotalBsS{\Total{} 100000 div \@DigitosMoneda{} round}Según Decreto Presidencial 3.548, el total en Bolívares Soberanos es de {\bfseries \@MonedaBsS~\numprint{\TotalBsS}}}%
 %   \CLASSWARNING{Se establecio' el texto aclaratorio sobre el total en Boli'vares Soberanos como una nota interna}%
 %  \else
 %   \DTMsaveddatediff{FechaEmision}{FechaBs}{\AuxFecha}%
 %   \ifnum\AuxFecha>-1%
 %    \Moneda{\@MonedaBs}%
 %   \else
 %    \Moneda{\@MonedaBsS}%
 %    \LetraNotaInterna{\fontsize{14}{18}}%
 %    \NotaInterna{\FPupn\TotalBsF{\Total{} 100000 * \@DigitosMoneda{} round}El total en Bolívares Fuertes es de {\bfseries \@MonedaBsF~\numprint{\TotalBsF}}}%
 %    \CLASSWARNING{Se establecio' el texto aclaratorio sobre el total en Boli'vares Fuertes como una nota interna}%
 %   \fi
 %  \fi
 % \fi

 % Efectuamos distintos chequeos para dar errores o advertencias si faltó algo.
 \ifdefvoid{\@RazonSocial}{\CLASSERROR{Hay que definir la razo'n social}}{}%
 \if@reporte
  \ifdefvoid{\@Num}{\CLASSWARNING{No se especifico' el nu'mero de la factura. Puede dar error si el ti'tulo del reporte no se especifica que aparezca en las cabeceras}}{}%
  \ifdefvoid{\@Proveedor}{%
   \ifdefvoid{\@Emisor}
    {\CLASSERROR{Hay que definir el nombre del proveedor cuando se emite un reporte}}%
    {\CLASSWARNING{Se toma el emisor como el proveedor por no haberse especificado este u'ltimo}%
    \let\@Proveedor=\@Emisor}}
   {}%
 \fi
 \if@factura
  \ifdefvoid{\@RIF}{\CLASSERROR{Hay que definir el R.I.F}}{}%
  \ifdefvoid{\@Direccion}{\CLASSERROR{Hay que definir la direccio'n fiscal}}{}%
  \ifdefvoid{\@Telefono}{\CLASSWARNING{Pudiera definirse un nu'mero telefo'nico}}{}%
  \ifdefvoid{\@Email}{\CLASSWARNING{Pudiera definirse una direccio'n de correo electro'nico}}{}%
  \ifdefvoid{\@NotaFinal}{\CLASSWARNING{Pudiera definirse una nota final}}{}%
  \ifcotizacion
   \ifdefvoid{\@FondoFactura}{\CLASSWARNING{En la opcio'n para cotizacion se recomienda agregar el fondo}}{}%
  \fi
  \ifprefactura
   \ifdefvoid{\@FondoFactura}{\CLASSWARNING{En la opcio'n para prefactura se recomienda agregar el fondo}}{}%
  \fi
  \ifsinnumero
   \ifcotizacion
   \else
    \CLASSERROR{La opcio'n para no poner el nu'mero so'lo es compatible con las cotizaciones}%
   \fi
  \fi
  \ifdenom
   \ifsinnumero
   \else
    \ifdefvoid{\@Num}{\CLASSERROR{Hay que definir el nu'mero de la factura}}{}%
   \fi
  \fi
 \fi
 \ifdefvoid{\@Contacto}{\if@Nombre\else\CLASSWARNING{Pudiera definirse una persona de contacto}\fi}{}%

 % El cálculo de la fecha de vencimiento para cuando haya crédito o fecha de vencimiento explícita.
 \ifdefvoid{\@Credito}
  {\gdef\@FechaVencimiento{\DTMusedate{FechaEmision}}}
  {\DTMsaveddateoffsettojulianday{FechaEmision}{\@Credito}{\AuxFecha}%
   \DTMsavejulianday{FechaVencimiento}{\number\AuxFecha}%
   \gdef\@FechaVencimiento{\DTMusedate{FechaVencimiento}}}%
 \ifx\@undefined\@NumControl\let\@NumControl\@Num\fi
 \ifx\@undefined\@NumSerieControl\gdef\@NumSerieControl{00}\fi
 
 % Establecemos los porcentajes de las alícuotas, y redondeamos los descuentos.
 \FPupn\@AlicuotaGporc{\@AlicuotaG{} 100 * \@DigitosPorc{} round clip}%
 \FPupn\@AlicuotaRporc{\@AlicuotaR{} 100 * \@DigitosPorc{} round clip}%
 \FPupn\@AlicuotaAporc{\@AlicuotaA{} 100 * \@DigitosPorc{} round clip}%
 \FPset\@AlicuotaEporc{0}%
 \FPupn\@DescuentoGporc{\@DescuentoGporc{} \@DigitosPorc{} round clip}%
 \FPupn\@DescuentoRporc{\@DescuentoRporc{} \@DigitosPorc{} round clip}%
 \FPupn\@DescuentoAporc{\@DescuentoAporc{} \@DigitosPorc{} round clip}%
 \FPupn\@DescuentoEporc{\@DescuentoEporc{} \@DigitosPorc{} round clip}%
 \FPupn\@DescuentoGfactor{100 \@DescuentoGporc{} /}%
 \FPupn\@DescuentoRfactor{100 \@DescuentoRporc{} /}%
 \FPupn\@DescuentoAfactor{100 \@DescuentoAporc{} /}%
 \FPupn\@DescuentoEfactor{100 \@DescuentoEporc{} /}

 % Una vez definido todo en el preámbulo, los comandos usados para definir campos servirán para mostrar el campo que se definió con ese comando.
 \def\Moneda{\@Moneda}%
 \def\RazonSocial{\@RazonSocial}%
 \let\Nombre\RazonSocial
 \def\RIF{\@RIF}%
 \let\CI\RIF
 \def\Direccion{\@Direccion}%
 \def\Telefono{\@Telefono}%
 \def\Email{\@Email}%
 \def\Contacto{\@Contacto}%
 \def\Emisor{\@Emisor}%
 \def\Conforme{\@Conforme}%
 \def\Proveedor{\@Proveedor}%
 \def\Fecha{\@Fecha}%
 \def\FechaVencimiento{\@FechaVencimiento}%
 \def\Credito{\@Credito}%
 \def\Num{\@Num}%
 \def\NumSerieControl{\@NumSerieControl}%
 \def\NumControl{\@NumControl}%
 \def\NotaFinal{\@NotaFinal}%
 \def\Resumen{\@Resumen}%
 \def\NotaPrevia{\@NotaPrevia}%
 \def\NotaInterna{\@NotaInterna}%
 \def\NotaExterna{\@NotaExterna}%
 \def\TituloReporte{\@TituloReporte}}

%</class>
%<*class|def>
%% Posiciones (X,Y) de los objetos: i) fecha, ii) denominación, iii) nota final, iv) número de control, v) firmas.
%<*class>
\NewDocumentCommand{\PosFecha}{s m m}{
 \setlength{\@PosFechaX}{#2}
 \setlength{\@PosFechaY}{#3}
 \IfBooleanTF{#1}{\booltrue{FechaYarriba}}{\boolfalse{FechaYarriba}}
 \IfBeginWith{#2}{-}{\booltrue{FechaXnegativo}}{\boolfalse{FechaXnegativo}}
 \IfBeginWith{#3}{-}{\booltrue{FechaYnegativo}}{\boolfalse{FechaYnegativo}}}

%</class>
%<def>\PosFecha*{-0cm}{5cm}
%<*class>
\NewDocumentCommand{\PosDenom}{s m m}{
 \setlength{\@PosDenomX}{#2}
 \setlength{\@PosDenomY}{#3}
 \IfBooleanTF{#1}{\booltrue{DenomYarriba}}{\boolfalse{DenomYarriba}}
 \IfBeginWith{#2}{-}{\booltrue{DenomXnegativo}}{\boolfalse{DenomXnegativo}}
 \IfBeginWith{#3}{-}{\booltrue{DenomYnegativo}}{\boolfalse{DenomYnegativo}}}

%</class>
%<def>\PosDenom*{0cm}{5cm}
%<*class>
\NewDocumentCommand{\PosControl}{s m m}{
 \setlength{\@PosControlX}{#2}
 \setlength{\@PosControlY}{#3}
 \IfBooleanTF{#1}{\booltrue{ControlYarriba}}{\boolfalse{ControlYarriba}}
 \IfBeginWith{#2}{-}{\booltrue{ControlXnegativo}}{\boolfalse{ControlXnegativo}}
 \IfBeginWith{#3}{-}{\booltrue{ControlYnegativo}}{\boolfalse{ControlYnegativo}}}

%</class>
%<def>\PosControl*{0cm}{-2.4cm}
%<*class>
\NewDocumentCommand{\PosNotaFinal}{s m m m}{
 \setlength{\@PosNotaFinalXuno}{#2}
 \setlength{\@PosNotaFinalXdos}{#3}
 \setlength{\@PosNotaFinalY}{#4}
 \IfBooleanTF{#1}{\booltrue{NotaFinalYarriba}}{\boolfalse{NotaFinalYarriba}}
 \IfBeginWith{#4}{-}{\booltrue{NotaFinalYnegativo}}{\boolfalse{NotaFinalYnegativo}}}
\let\PosCondiciones\PosNotaFinal
%</class>
%<def>\PosNotaFinal{0cm}{0cm}{-5.3cm}
%% La nota final se define con un X_inicial, un X_final y un Y_inicial (Xo,Xf,Y)
%<*class>

\NewDocumentCommand{\PosFirmas}{m m m}{
 \setlength{\@PosFirmasXuno}{#1}
 \setlength{\@PosFirmasXdos}{#2}
 \setlength{\@PosFirmasY}{#3}
 \IfBeginWith{#3}{-}{\booltrue{FirmasYnegativo}}{\boolfalse{FirmasYnegativo}}}
%</class>
%<def>\PosFirmas{5cm}{1cm}{-2.4cm}
%% La firma se define con un X_inicial, un X_final y un Y_inicial (Xo,Xf,Y)

%% Tamaño y grosor de la línea de las firmas.
%<*class>
\def\LineaFirmas#1#2{
 \setlength{\@TamLineaFirmas}{#1}
 \setlength{\@GrosorLineaFirmas}{#2}}
%</class>
%<def>\LineaFirmas{5.5cm}{0.7pt}

%% Separación mínima horizontal entre los totales en modo de filas y distancia vertical entre esas líneas, fija.
%<class>\def\SepFilas#1#2{\setlength{\@SeparaMinTotales}{#1}\setlength{\@SeparaFilas}{#2}}
%<def>\SepFilas{0.8cm}{2.5pt}

%% Separación vertical entre los elementos de la factura (entre el encabezado y el cuadro de la 'factura', luego desde dicho cuadro hasta la descripción, o hasta los totales en modo de filas).
%<class>\def\SepVertical#1{\setlength{\@SeparaVertical}{#1}}
%<def>\SepVertical{0.2cm}

%% Separación horizontal entre la caja de los datos del cliente y la caja del resumen.
%<class>\def\SepDatosResumen#1{\setlength{\@SeparaDatosResumen}{#1}}
%<def>\SepDatosResumen{0.2cm}

%% Separación horizontal interna entre la línea y el texto en los datos del cliente y en el resumen.
%<class>\def\SepEncabezado#1{\setlength{\@SeparaInternaEncabezado}{#1}}
%<def>\SepEncabezado{6pt}

%% Separación interna entre la línea y el texto de la nota previa.
%<class>\def\SepNotaPrevia#1{\setlength{\@SeparaInternaNotaPrevia}{#1}}
%<def>\SepNotaPrevia{6pt}

%% Separación extra entre los elementos en la factura.
%<class>\def\SepItemsExtra#1{\setlength{\@SeparaItemsExtra}{#1}}
%<def>\SepItemsExtra{0pt}

%% Separación interna entre la línea y el texto de la descripción.
%<class>\def\SepDescripcion#1{\setlength{\@SeparaInternaDescripcion}{#1}}
%<def>\SepDescripcion{6pt}

%<class>\def\SepNotaFinal#1{\setlength{\@SeparaInternaNotaFinal}{#1}}
%<class>\let\SepCondiciones\SepNotaFinal
%<def>\SepNotaFinal{5pt}

%% Grosor de la línea de las cajas del encabezado, curvatura de la esquina y sombra.
%% (si pedimos linea de 0pt, todo queda en cero y no se abre el ambiente TikZ).
%<class>\NewDocumentCommand{\LineaEncabezado}{m g g}{
%<class> \setlength{\@GrosorLineaEncabezado}{#1}
%<class> \IfValueT{#2}{\setlength{\@EsquinaLineaEncabezado}{#2}}
%<class> \IfValueT{#3}{\setlength{\@SombraEncabezado}{#3}}}
%<def>\LineaEncabezado{0.4pt}{10pt}{0pt}

%% Grosor de la línea de la nota previa, curvatura de la esquina y sombra.
%% (si pedimos linea de 0pt, todo queda en cero y no se abre el ambiente TikZ).
%<class>\NewDocumentCommand{\LineaNotaPrevia}{m g g}{
%<class> \setlength{\@GrosorLineaNotaPrevia}{#1}
%<class> \IfValueT{#2}{\setlength{\@EsquinaLineaNotaPrevia}{#2}}
%<class> \IfValueT{#3}{\setlength{\@SombraNotaPrevia}{#3}}}
%<def>\LineaNotaPrevia{0.4pt}{7pt}{0pt}

%% Grosor de la línea de la descripción, curvatura de la esquina y sombra.
%% (si pedimos linea de 0pt, todo queda en cero y no se abre el ambiente TikZ).
%<class>\NewDocumentCommand{\LineaDescripcion}{m g g}{
%<class> \setlength{\@GrosorLineaDescripcion}{#1}
%<class> \IfValueT{#2}{\setlength{\@EsquinaLineaDescripcion}{#2}}
%<class> \IfValueT{#3}{\setlength{\@SombraDescripcion}{#3}}}
%<def>\LineaDescripcion{0.4pt}{0pt}{4pt}

%% Grosor de la línea de la nota final, curvatura de la esquina y sombra.
%% (si pedimos linea de 0pt, todo queda en cero y no se abre el ambiente TikZ).
%<class>\NewDocumentCommand{\LineaNotaFinal}{m g g}{
%<class> \setlength{\@GrosorLineaNotaFinal}{#1}
%<class> \IfValueT{#2}{\setlength{\@EsquinaLineaNotaFinal}{#2}}
%<class> \IfValueT{#3}{\setlength{\@SombraNotaFinal}{#3}}}
%<class>\let\LineaCondiciones\LineaNotaFinal
%<def>\LineaNotaFinal{0pt}{0pt}{0pt}

%% El espacio vertical que hay que avanzar sólo en la primera página del reporte antes de empezar a escribir sobre él. Usable cuando la primera página tiene un membrete y las demás no, es decir, cuando se usa \FondoReporte y no \FondoReporte*.
%<class>\def\InicioReporte#1{\setlength{\@InicioReporte}{#1}}
%<def>\InicioReporte{1.6cm}

%% Tipos de letra para cada elemento posible:
%% Esta es la letra de la descripción de los items; y el espacio \baselineskip de *toda* la tabla de la factura se fija en este comando únicamente (se predeterminó 18pt).
%<class>\def\LetraItems#1{\gdef\@LetraItems{#1\selectfont}}
%<def>\LetraItems{\fontsize{14}{18}}

%% Tipo de letra para la nota interna:
%<class>\def\LetraNotaInterna#1{\gdef\@LetraNotaInterna{#1\selectfont}}
%<def>\LetraNotaInterna{\fontsize{11}{13}\itshape}

%% Tipo de letra para la nota externa:
%<class>\def\LetraNotaExterna#1{\gdef\@LetraNotaExterna{#1\selectfont}}
%<def>\LetraNotaExterna{\fontsize{14}{16.8}}

%% Con este comando se cambia la etra del título de la tabla, es decir, los textos 'Cantidad, 'Descripción', etc...
%<class>\def\LetraTitColumnas#1{\gdef\@LetraTitColumnas{#1\selectfont}}
%<def>\LetraTitColumnas{\fontsize{10}{10}\bfseries}

%% Tipo de letra de los títulos de los totales, por ejemplo los textos: 'B.I. G', 'IVA G'.
%<class>\def\LetraTitTotales#1{\gdef\@LetraTitTotales{#1\selectfont}}
%<def>\LetraTitTotales{\fontsize{12}{12}\bfseries}

%% Letra con la que salen los totales parciales, es decir, los números, exceptuando el total general.
%<class>\def\LetraNumTotales#1{\gdef\@LetraNumTotales{#1\selectfont}}
%<def>\LetraNumTotales{\fontsize{14}{12}}

%% Letra con la que sale la palabra 'TOTAL'.
%<class>\def\LetraTitTotal#1{\gdef\@LetraTitTotal{#1\selectfont}}
%<def>\LetraTitTotal{\fontsize{12}{12}\bfseries}

%% Letra para el total general, el monto.
%<class>\def\LetraNumTotal#1{\gdef\@LetraNumTotal{#1\selectfont}}
%<def>\LetraNumTotal{\fontsize{14}{14}\bfseries}

%% Tipo de letra para los títulos de las cajas del encabezado, por ejemplo: 'Datos del cliente'.
%<class>\def\LetraTitEnc#1{\gdef\@LetraTitEnc{#1\selectfont}}
%<def>\LetraTitEnc{\fontsize{11}{12.5}\slshape}

%% Tipo de letra para el texto en el encabezado de la factura.
%<class>\def\LetraEncFactura#1{\gdef\@LetraEncFactura{#1\selectfont}}
%<def>\LetraEncFactura{\fontsize{11}{12.5}}

%% Tipo de letra para el texto en el encabezado del reporte.
%<class>\def\LetraEncReporte#1{\gdef\@LetraEncReporte{#1\selectfont}}
%<def>\LetraEncReporte{\fontsize{13}{16}}

%% Letra para los montos individuales de la factura y las cantidades.
%<class>\def\LetraNumeros#1{\gdef\@LetraNumeros{#1\selectfont}}
%<def>\LetraNumeros{\fontsize{14}{14}}

%% El tipo de IVA ('G', 'R', 'A' o '(E)') también tiene su tipo de letra.
%<class>\def\LetraTipoIVA#1{\gdef\@LetraTipoIVA{#1\selectfont}}
%<def>\LetraTipoIVA{\fontsize{12}{12}}

%% Letra para la descripción larga (comando \Descripcion).
%<class>\def\LetraDescripcion#1{\gdef\@LetraDescripcion{#1\selectfont}}
%<def>\LetraDescripcion{\fontsize{14}{16.8}}

%% Tipo de letra para el texto bajo las líneas para firmar.
%<class>\def\LetraFirmas#1{\gdef\@LetraFirmas{#1\selectfont}}
%<def>\LetraFirmas{\fontsize{12}{12}}

%% Tipo de letra para la nota previa.
%<class>\def\LetraNotaPrevia#1{\gdef\@LetraNotaPrevia{#1\selectfont}}
%<def>\LetraNotaPrevia{\fontsize{11}{12.5}}

%% Tipo de letra para la nota final.
%<class>\def\LetraNotaFinal#1{\gdef\@LetraNotaFinal{#1\selectfont}}
%<class>\let\LetraCondiciones\LetraNotaFinal
%<def>\LetraNotaFinal{\fontsize{12}{12}}

%% Letra para la denominación del documento ('FACTURA', 'COTIZACIÓN',...).
%<class>\def\LetraDenom#1{\gdef\@LetraDenom{#1\selectfont}}
%<def>\LetraDenom{\fontsize{14}{14}\bfseries}

%% Letra para la numeración de la factura.
%<class>\def\LetraNumeracion#1{\gdef\@LetraNumeracion{#1\selectfont}}
%<def>\LetraNumeracion{\fontsize{14}{14}\bfseries\color{red}}

%% Letra para las fechas (emisión y vencimiento).
%<class>\def\LetraFecha#1{\gdef\@LetraFecha{#1\selectfont}}
%<def>\LetraFecha{\fontsize{14}{15}}

%% Letra para el título del reporte.
%<class>\def\LetraTitReporte#1{\gdef\@LetraTitReporte{#1\selectfont}}
%<def>\LetraTitReporte{\fontsize{16}{22}\bfseries}

%% Letra para las cabeceras de las páginas en el modo reporte.
%<class>\def\LetraInfoPagina#1{\gdef\@LetraInfoPagina{#1\selectfont}}
%<def>\LetraInfoPagina{\fontsize{12}{12}\slshape}

%% Letra para el reverso de la factura.
%<class>\def\LetraReverso#1{\gdef\@LetraReverso{#1\selectfont}}
%<def>\LetraReverso{\fontsize{10}{12}}

%% Letra para el reporte.
%<class>\def\LetraReporte#1{\gdef\@LetraReporte{#1\selectfont}}
%<def>\LetraReporte{\fontsize{12}{20}}

%</class|def>
%<*class>
%% Comando para incluir la firma escaneada en el sitio donde va la línea de la firma del emisor.
\NewDocumentCommand{\FirmaFactura}{m O{0pt} O{0pt}}{% \FirmaFactura{imagen}[x-corr][y-corr]
 \AtEndPreamble{
  \if@factura
   \RequirePackage{graphicx}
   \ifx&#2&\else\setlength{\@FirmaFacturaXcorr}{#2}\fi
   \ifx&#3&\else\setlength{\@FirmaFacturaYcorr}{#3}\fi
   \savebox{\@CajaFirmaEmisor}{#1}
   \AtBeginShipoutNext{%
    \begin{textblock*}{\@TamLineaFirmas}%
     (\textwidth+\oddsidemargin+1in-\@TamLineaFirmas/2-\wd\@CajaFirmaEmisor/2-\@PosFirmasXdos+\@FirmaFacturaXcorr,\@PosFirmasY-\ht\@CajaFirmaEmisor-\dp\@CajaFirmaEmisor+\@FirmaFacturaYcorr)%
     \usebox{\@CajaFirmaEmisor}%
    \end{textblock*}}%
  \fi}}

\NewDocumentCommand{\FirmaReporte}{O{} O{0pt} O{0pt} G{\@Proveedor} O{\@FirmaReportePosX} O{\@FirmaReporteExcesoLinea}}{% \Firma[imagen][x-corr][y-corr]{texto}[posición en X desde la derecha][tamaño linea]
 \renewcommand{\baselinestretch}{1}\selectfont
 \newif\if@Firma@Dibujo
 \newbox{\TextoFirma}%
 \newbox{\DibujoFirma}%
 \ifx&#1&\else\savebox{\DibujoFirma}{#1}\@Firma@Dibujotrue\fi
 \setlength{\@FirmaReporteXcorr}{#2}%
 \setlength{\@FirmaReporteYcorr}{#3}%
 \savebox{\TextoFirma}{#4}%
 \setlength{\@FirmaReporteTextPos}{#5}%
 \setlength{\@FirmaReporteTamLinea}{#6}%
 \@LetraFirmas
 \if@Firma@Dibujo
  \newlength{\@FirmaReporteoddeven}%
  \setlength{\@FirmaReporteFirmaPos}{\zposy{Firma@zref@PosY}sp}%
 \fi
 \ifdim\wd\DibujoFirma>\wd\TextoFirma
  \setlength{\@FirmaReporteTamLinea}{\@FirmaReporteTamLinea+\wd\DibujoFirma}%
 \else
  \setlength{\@FirmaReporteTamLinea}{\@FirmaReporteTamLinea+\wd\TextoFirma}%
 \fi
 \par
 \vspace*{\ht\DibujoFirma}%
 \vspace*{\dp\DibujoFirma}%
 \hfill
 \parbox{\@FirmaReporteTamLinea}{%
  \if@Firma@Dibujo\zsaveposy{Firma@zref@PosY}\fi
  \rule{\@FirmaReporteTamLinea}{\@GrosorLineaFirmas}\\
  \centering{\usebox{\TextoFirma}}}%
 \hspace*{\@FirmaReporteTextPos}\par
 \if@Firma@Dibujo
  \mbox{}%
  \par
  \ifodd\c@page
   \setlength{\@FirmaReporteoddeven}{\oddsidemargin}%
  \else
   \setlength{\@FirmaReporteoddeven}{\evensidemargin}%
  \fi
  \begin{textblock*}{\@FirmaReporteTamLinea}
  (\textwidth+\@FirmaReporteoddeven+1in-0.5\@FirmaReporteTamLinea-0.5\wd\DibujoFirma-\@FirmaReporteTextPos+\@FirmaReporteXcorr,
   \paperheight-\@FirmaReporteFirmaPos-\ht\DibujoFirma-\dp\DibujoFirma+\@FirmaReporteYcorr)
   \noindent\usebox{\DibujoFirma}%
  \end{textblock*}
 \fi}

%% Con \FondoFactura le damos el nombre del archivo que contiene el formato de la factura, por ejemplo para compilaciones con 'prefactura', 'cotizacion' o para incluir una imagen de fondo como 'marca de agua'.
\newcommand{\FondoFactura}[1]{
 \AtEndPreamble{
  \if@factura
   \RequirePackage{graphicx}
   \@FondoFacturatrue
   \gdef\@FondoFactura{#1}
  \fi}}

%% Con \FondoReporte le damos el nombre del archivo que contiene el formato del reporte. Existen dos posibilidades: que dicho fondo salga en todas las páginas o sólo en la primera.
\NewDocumentCommand{\FondoReporte}{s m}{
 \AtEndPreamble{
  \if@reporte
   \@FondoReportetrue
   \RequirePackage{graphicx}
   \gdef\@FondoReporte{#2}
   \IfBooleanT{#1}{\@FondoReporteStartrue}% Con * sale en todas las páginas; sin * sale sólo en la primera página.
  \fi}}

%% Comando para establecer desde el preámbulo el estilo de página (\pagestyle) del reporte, tanto para la primera página como para las siguientes.
\NewDocumentCommand{\EstiloPagReporte}{o m}{
 \gdef\@EstiloPagReporteResto{#2}
 \IfValueTF{#1}{\gdef\@EstiloPagReporteUno{#1}}{\gdef\@EstiloPagReporteUno{\@EstiloPagReporteResto}}}

%% El título para el reporte.
\NewDocumentCommand{\TituloReporte}{s m}{
 \gdef\@TituloReporte{#2}%
 \IfBooleanTF{#1}{% Con '*' sale el título en las cabeceras o pies. Sin '*' no.
  \gdef\@InfoPagina{\let\\\relax\@InfoPaginaTitulo}}%
 {\let\@InfoPagina\@InfoPaginaNoTitulo}}

%% Para cada cliente o para cada personalización se puede tener un archivo .def o .tex con su nombre y ciertas macros cargadas. Ese archivo lo ponemos incluir con '\Cliente' en lugar de usar \input, y la extensión deberá ser .def o .tex.
\newcommand{\Cliente}[1]{%
 \IfFileExists{#1.def}{\input{#1.def}}{
  \IfFileExists{#1.tex}{\input{#1.tex}}{
   \CLASSERROR{No existe #1.def ni #1.tex}}}}

%\iffalse meta-comment
%
% factura.dtx
% Copyright 2019 Emilio Augusto Lazo Zaia <emiliolazozaia@gmail.com>
%
% This work may be distributed and/or modified under the
% conditions of the LaTeX Project Public License, either
% version 1.3c of this license or (at your option) any later
% version. This version of this license is in
%   http://www.latex-project.org/lppl/lppl-1-3c.{html|txt|tex|pdf}
% and the latest version of this license is in
%   http://www.latex-project.org/lppl.txt
% and version 1.3c or later is part of all distributions of
% LaTeX version 2008/05/04 or later.
%
% This work has the LPPL maintenance status `maintained'.
%
% The Current Maintainer of this work is Emilio Augusto Lazo Zaia.
%
% This work consists of the files README and factura.dtx
% with its derived files factura.ins, factura.cls, factura.def,
% factura-ejemplo-aux.tex, factura-ejemplo-firma.tex,
% factura-ejemplo-fondo.tex, factura-ejemplo-membrete.tex,
% factura-ejemplo-cotizacion.tex, factura-ejemplo-prefactura.tex,
% factura-ejemplo-factura01.tex, factura-ejemplo-factura02.tex,
% factura-ejemplo-factura03.tex, factura-ejemplo-factura04.tex,
% factura-ejemplo-factura05.tex, factura-ejemplo-factura06.tex,
% factura-ejemplo-factura07.tex, factura-ejemplo-factura08.tex,
% factura-ejemplo-factura09.tex, factura-ejemplo-factura10.tex,
% factura-ejemplo-reporte1.tex, factura-ejemplo-reporte2.tex,
% factura-ejemplo-reporte3.tex and factura-beispiel-rechnung.tex.
%<*internalbatchfile>
\def\nameofplainTeX{plain}
\ifx\fmtname\nameofplainTeX
\else
 \expandafter\begingroup
\fi
%</internalbatchfile>
%<*batchfile>
\input docstrip.tex
\preamble

Copyright 2019 Emilio Augusto Lazo Zaia <emiliolazozaia@gmail.com>

This work may be distributed and/or modified under the
conditions of the LaTeX Project Public License, either
version 1.3c of this license or (at your option) any later
version. This version of this license is in
 http://www.latex-project.org/lppl/lppl-1-3c.{html|txt|tex|pdf}
and the latest version of this license is in
 http://www.latex-project.org/lppl.txt
and version 1.3c or later is part of all distributions of
LaTeX version 2008/05/04 or later.

This work has the LPPL maintenance status `maintained'.

The Current Maintainer of this work is Emilio Augusto Lazo Zaia.

This work consists of the files README and factura.dtx
with its derived files factura.ins, factura.cls, factura.def,
factura-ejemplo-aux.tex, factura-ejemplo-firma.tex,
factura-ejemplo-fondo.tex, factura-ejemplo-membrete.tex,
factura-ejemplo-cotizacion.tex, factura-ejemplo-prefactura.tex,
factura-ejemplo-factura01.tex, factura-ejemplo-factura02.tex,
factura-ejemplo-factura03.tex, factura-ejemplo-factura04.tex,
factura-ejemplo-factura05.tex, factura-ejemplo-factura06.tex,
factura-ejemplo-factura07.tex, factura-ejemplo-factura08.tex,
factura-ejemplo-factura09.tex, factura-ejemplo-factura10.tex,
factura-ejemplo-reporte1.tex, factura-ejemplo-reporte2.tex,
factura-ejemplo-reporte3.tex and factura-beispiel-rechnung.tex.

\endpreamble
\keepsilent
\askforoverwritefalse
\obeyspaces
\Msg{*****************************************************************}
\Msg{**************************** factura ****************************}
\Msg{*****************************************************************}
\generate{\file{\jobname.cls}{\from{\jobname.dtx}{class}}}
\generate{\file{\jobname.def}{\from{\jobname.dtx}{def}}}
\Msg{\jobname.cls and \jobname.def written.}
\Msg{}
\postamble
\endpostamble
%</batchfile>
%<*internalbatchfile>
\generate{\file{\jobname.ins}{\from{\jobname.dtx}{batchfile}}}
\Msg{\jobname.ins written.}
%</internalbatchfile>
%<*batchfile>
\def\factuno{factura-ejemplo-factura01.tex}
\def\factdos{factura-ejemplo-factura02.tex}
\def\facttres{factura-ejemplo-factura03.tex}
\def\factcuatro{factura-ejemplo-factura04.tex}
\def\factcinco{factura-ejemplo-factura05.tex}
\def\factseis{factura-ejemplo-factura06.tex}
\def\factsiete{factura-ejemplo-factura07.tex}
\def\factocho{factura-ejemplo-factura08.tex}
\def\factnueve{factura-ejemplo-factura09.tex}
\def\factdiez{factura-ejemplo-factura10.tex}
\def\repuno{factura-ejemplo-reporte1.tex}
\def\repdos{factura-ejemplo-reporte2.tex}
\def\reptres{factura-ejemplo-reporte3.tex}
\def\cotiza{factura-ejemplo-cotizacion.tex}
\def\prefac{factura-ejemplo-prefactura.tex}
\def\beispiel{factura-beispiel-rechnung.tex}
\def\auxejemplo{factura-ejemplo-aux.tex}
\def\firma{factura-ejemplo-firma.tex}
\def\fondo{factura-ejemplo-fondo.tex}
\def\membrete{factura-ejemplo-membrete.tex}
\generate{\file{\factuno}{\from{\jobname.dtx}{factuno}}}
\generate{\file{\factdos}{\from{\jobname.dtx}{factdos}}}
\generate{\file{\facttres}{\from{\jobname.dtx}{facttres}}}
\generate{\file{\factcuatro}{\from{\jobname.dtx}{factcuatro}}}
\generate{\file{\factcinco}{\from{\jobname.dtx}{factcinco}}}
\generate{\file{\factseis}{\from{\jobname.dtx}{factseis}}}
\generate{\file{\factsiete}{\from{\jobname.dtx}{factsiete}}}
\generate{\file{\factocho}{\from{\jobname.dtx}{factocho}}}
\generate{\file{\factnueve}{\from{\jobname.dtx}{factnueve}}}
\generate{\file{\factdiez}{\from{\jobname.dtx}{factdiez}}}
\generate{\file{\repuno}{\from{\jobname.dtx}{repuno}}}
\generate{\file{\repdos}{\from{\jobname.dtx}{repdos}}}
\generate{\file{\reptres}{\from{\jobname.dtx}{reptres}}}
\generate{\file{\cotiza}{\from{\jobname.dtx}{cotiza}}}
\generate{\file{\prefac}{\from{\jobname.dtx}{prefac}}}
\generate{\file{\beispiel}{\from{\jobname.dtx}{beispiel}}}
\generate{\file{\auxejemplo}{\from{\jobname.dtx}{aux}}}
\generate{\file{\firma}{\from{\jobname.dtx}{firma}}}
\generate{\file{\fondo}{\from{\jobname.dtx}{fondo}}}
\generate{\file{\membrete}{\from{\jobname.dtx}{membrete}}}
\Msg{.tex examples written.}
\Msg{}
\Msg{*****************************************************************}
\immediate\write18{makeindex -s gind.ist -o \jobname.ind \jobname.idx}
\immediate\write18{makeindex -s gglo.ist -o \jobname.gls \jobname.glo}
\Msg{}
\Msg{Change history and index updated.}
\Msg{*****************************************************************}
\Msg{To produce only package files, type}
\Msg{  pdftex \jobname.dtx}
\Msg{To produce package files and pdf documentation, type}
\Msg{  xelatex \jobname.dtx}
\Msg{  pdflatex factura-ejemplo-fondo.tex}
\Msg{  pdflatex factura-ejemplo-membrete.tex}
\Msg{  xelatex factura-ejemplo-firma.tex}
\Msg{}
\Msg{Also compile all other examples and recompile \jobname.dtx.}
\Msg{(factura-ejemplo-aux.tex is not an example!)}
\Msg{}
\Msg{To manual update index and changelog, type}
\Msg{  makeindex -s gind.ist -o \jobname.ind \jobname.idx}
\Msg{  makeindex -s gglo.ist -o \jobname.gls \jobname.glo}
\Msg{}
\Msg{More information (in spanish) about compiling on README and documentation.}
\Msg{}
\Msg{*****************************************************************}
\Msg{**************************** factura ****************************}
\Msg{*****************************************************************}
\Msg{}
%</batchfile>
%<batchfile>\endbatchfile
%<*internalbatchfile>
\ifx\fmtname\nameofplainTeX
 \expandafter\endbatchfile
\else
 \expandafter\endgroup
\fi
%</internalbatchfile>
%<*driver>
\ProvidesFile{factura.dtx}
%</driver>
%<class>\NeedsTeXFormat{LaTeX2e}[1995/12/01]
%<class>\ProvidesClass{factura}
%<def>\ProvidesFile{factura.def}
%<*class|def>
[2019/08/16 v3.70
%<class>Clase para elaborar facturas legales
%<class>en la Repu'blica Bolivariana de Venezuela.
%<def>Archivo de valores iniciales y condiciones de borde para factura.cls
]

%</class|def>
%<*driver>
\documentclass{ltxdoc}
\usepackage[a4paper,headheight=0.6cm,headsep=0.77cm,footskip=1.2cm,footnotesep=0.54cm,marginparsep=10pt,vmargin={3cm,2cm},hmargin={4cm,2cm}]{geometry}
\usepackage[dvipsnames]{xcolor}
\usepackage[spanish]{babel}
\usepackage{
 array,
 calc,
 catchfilebetweentags,
 etoolbox,
 footnote,
 graphicx,
 ifluatex,
 ifxetex,
 metalogo,
 paralist,
 pgffor,
 regexpatch,
 setspace,
 tikz,
 todo,
 verbatim,
 xparse,
 xspace,
 xstring,
 hypdoc}

% Necesario ejecutar \newif de esta forma:
\csname newif\expandafter\endcsname\csname ifmotorunicode\endcsname
\csname newif\expandafter\endcsname\csname ifejemplos\endcsname

\ejemplostrue

\ifxetex
 \motorunicodetrue
\else
 \ifluatex \motorunicodetrue \fi
\fi

\ifmotorunicode
 \usepackage{fontspec}
 \setmainfont[Ligatures=TeX,Mapping=TeX]{Linux Libertine O}
\else
 \usepackage[T1]{fontenc}
 \usepackage[utf8]{inputenc}
\fi

\overfullrule15pt

\newcommand{\VERBATIM}{\verbatimcolor\setstretch{1.0}}

\xspaceaddexceptions{*}

\hypersetup{
 plainpages=false,
 colorlinks=true,
 linkcolor=blue,
 unicode=true}

\makesavenoteenv{tabular}
\makesavenoteenv{table}

\newcommand\flechaCR{%
 \raisebox{-0.4ex}{%
  \begin{tikzpicture}
  \draw[line width=0.075ex] (0,0) arc (20:70:0.75ex) + (0,0) arc (-70:-20:0.75ex) ++ (0.8ex,0.55ex) arc (0:-100:1ex);
 \end{tikzpicture}}}

\EnableCrossrefs
\CodelineIndex
\OnlyDescription
\RecordChanges

\setlength\parskip{7pt}
\setlength\IndexMin{100pt}

\widowpenalty10000

\newbool{Ejemplo}
\newbool{EjemploIzq}
\newbool{EjemploDer}
\newbool{DescripEjemplo}

\def\ejemplooptcolor{\color{blue}}
\def\ejemplomacrocolor{\color{purple}}
\def\verbatimcolor{\color{OrangeRed}}
\def\macromandcolor{\color{Red}}
\def\macrooptcolor{\color{RoyalBlue}}

\def\tikz{T\emph{i}kZ\xspace}

\catcode`\%=11
\catcode`\|=14
\NewDocumentCommand{\Ejemplo}{s O{0.5} m s O{} G{1} o}{|
 \IfFileExists{factura-#3.pdf}{|
  \catcode`\<=12
  \catcode`\>=12
  \IfBooleanTF{#4}
   {\IfBooleanTF{#1}
    {\booltrue{DescripEjemplo}}
    {\booltrue{Ejemplo}\booltrue{EjemploDer}}}
   {\IfBooleanTF{#1}
    {\booltrue{Ejemplo}\booltrue{EjemploIzq}}
    {\booltrue{Ejemplo}\booltrue{EjemploIzq}\booltrue{DescripEjemplo}}}
  \ifbool{Ejemplo}
   {\ifnum #6=1 \def\Ancho{#2}{\phantomsection\par\mbox{}\hfill{\color{RoyalBlue}\ttfamily \large factura-#3}\hfill\mbox{}\par}{}\fi}
   {\def\Ancho{0.05}}|
  \ifbool{EjemploDer}{\hfill}{}|
  \ifbool{Ejemplo}
   {\begin{minipage}{\Ancho\linewidth}
     \fbox{\includegraphics[page=#6,width=\linewidth-2\fboxrule-2\fboxsep]{factura-#3}}
     \ifbool{DescripEjemplo}{\addcontentsline{toc}{subsection}{factura-#3}}{}
    \end{minipage}}
   {}|
  \ifbool{EjemploIzq}{\hfill}{}|
  \ifbool{DescripEjemplo}
   {\hfill
   | Gracias a cfr y a egreg por el esclarecimiento en cuanto a la necesidad de \makeatletter en \CatchFileBetweenDelims y antes de \regexpatchcmd*.
   | Thanks to cfr and egreg for the clarification on the use of \makeatletter on \CatchFileBetweenDelims and before \regexpatchcmd*.
   | (https://tex.stackexchange.com/questions/380796/how-to-patch-a-macro-with-control-sequences-using-regexpatch)
   \CatchFileBetweenDelims{\DescEjemplo}{factura-#3.tex}{<desc>}{</desc>}[\catcode37=11\catcode13=14\makeatletter]|
   \catcode`\%=11
   \makeatletter
   | Gracias a Werner y a egreg por el uso de \begingroup\ttfamily\string ... \endgroup.
   | Thanks to Werner and egreg for the use of \begingroup\ttfamily\string ... \endgroup.
   | (http://tex.stackexchange.com/questions/342572/error-using-verb-as-the-replace-string-on-xpatchcmd)
   \regexpatchcmd*{\DescEjemplo}{\cC.\cO[*]*}{\c{begingroup}\c{ejemplomacrocolor}\c{ttfamily}\c{string}\0\c{endgroup}\c{xspace}}{}{}|
   \regexpatchcmd*{\DescEjemplo}{(\()(\cL[a-zG]*)(\))}{(\c{begingroup}\c{ejemplooptcolor}\c{ttfamily}\2\c{endgroup})}{}{}|
   \xpatchcmd*{\DescEjemplo}{%% }{\item}{}{}|
   \xpatchcmd*{\DescEjemplo}{}{}{}|
   \makeatother
   \begin{minipage}{\linewidth-\Ancho\linewidth}
    \begin{itemize}
     \setlength{\itemsep}{0pt}
     #5\DescEjemplo
    \end{itemize}
   \end{minipage}
   \IfValueT{#7}{{#7}}}
   {}|
  \boolfalse{Ejemplo}\boolfalse{EjemploIzq}\boolfalse{EjemploDer}\boolfalse{DescripEjemplo}|
  \catcode`\%=9
  \catcode`\<=13
  \catcode`\>=13
 }{\message{^^JLaTeX Warning: No existe factura-#4.pdf; hay que compilarlo antes de compilar este documento.^^J}}}
\catcode`\%=14
\catcode`\|=12

% No quiero que las notas pendientes tengan los números de página:
\patchcmd{\todoitem}{(p.~\pageref{todopage:\thetodo}):}{-}{}{}

\renewcommand{\todoname}{Pendiente}
\renewcommand{\todomark}{Pendiente}
\newcommand{\changesname}{Lista de cambios en las versiones}
\newcommand{\glosname}{Índice de comandos, entornos y opciones}

\makeatletter
 \def\SpecialEnvIndex#1{\@bsphack
  \index{#1\actualchar{\protect\ttfamily#1}
   (entorno)\encapchar usage}%
  \index{entornos:\levelchar#1\actualchar
   {\protect\ttfamily#1}\encapchar usage}\@esphack}
 \def\es@yearl{l año}

 % Definimos el comando para describir opciones, que inexplicablemente falta en doc.sty:
 \def\DescribeOption{\leavevmode\@bsphack\begingroup\MakePrivateLetters
  \Describe@Option}
 \def\Describe@Option#1{\endgroup
  \marginpar{\raggedleft\PrintDescribeOption{#1}}%
  \SpecialOptionIndex{#1}\@esphack\ignorespaces}
 \def\SpecialOptionIndex#1{\@bsphack
  \index{#1\actualchar{\protect\ttfamily#1}
   (opción)\encapchar usage}%
  \index{opciones:\levelchar#1\actualchar{\protect\ttfamily#1}\encapchar
   usage}\@esphack}
 \def\PrintDescribeOption#1{\strut \MacroFont #1\ }

 % Redefinimos el comando del encabezado del índice:
 \def\index@prologue{%
  \section*{\glosname}%
  \markboth{\glosname}{\glosname}}%

 % Adaptamos un poco la lista de cambios:
 \def\changes@#1#2#3{%
  \protected@edef\@tempa{%
   \noexpand\glossary{%
    {\bfseries #1 --- #2}%
    \ifx\saved@macroname\@empty
     \relax\actualchar
    \else
     \saved@indexname
     \actualchar
     \string\verb\quotechar*%
     \verbatimchar\saved@macroname
     \verbatimchar
    \fi
    \levelchar #3}}%
  \@tempa\endgroup\@esphack}
 
 \def\@wrglossary#1{%
  \protected@write\@glossaryfile{}%
  {\string\glossaryentry{#1}{1}}%
  \endgroup\@esphack}%

 \patchcmd{\glossary@prologue}{Change History}{\changesname}{}{}
\makeatother

\title{The \textsf{\jobname} class\thanks{Esta documentación corresponde a la clase \textsf{\jobname}~\fileversion, de fecha \filedate.}}
\author{Emilio Augusto Lazo Zaia\\ \texttt{<emiliolazozaia at gmail.com>}}

\GetFileInfo{\jobname.dtx}

\begin{document}
 \maketitle
 \DocInput{\jobname.dtx}
 \clearpage
 \phantomsection
 \addcontentsline{toc}{section}{\todoname}
 \todos
 \section*{Información específica de esta versión + Reconversión Monetaria}
 % v2.82 Desde la versión v2.82 de esta clase, para poder compilar con \LuaLaTeX{} es necesaria la versión v2.26 del paquete |zref|, de fecha 2016/05/21, y la versión v1.18 de |atbegshi| de fecha 2016/06/09.

 % v2.82 -> Esta versión de |factura| se sacó principalmente para corregir un error introducido en la versión anterior v2.72 en cuanto a las referencias, para agregar la opción |numitem| que enumera los ítems en la factura y para que la línea diagonal (opción |linea|) funcione correctamente siempre.
 
 % v3.00 -> Desde la versión v3.00, todos los ejemplos están incorporados a la documentación, con la descripción de cada uno, de manera automatizada.
 
 % v3.02 -> Esta versión v3.02 se saca porque la versión 3.9 de |fancyhdr| hizo necesaria esta actualización por un error que daba con los reportes por el uso en esta clase de la macro |\fancy@setoffs|. Lo que hacíamos con esa macro de |fancyhdr| lo hicimos ahora de otra forma mejor, dada la incompatibilidad con la nueva versión de ese paquete.%v3.02
 
 % v3.40 -> La versión 3.40 se saca por los cambios en la política impositiva decretados por el Presidente en Decreto número 3085 relativos a descuentos del 3\% y 5\% en el IVA general cuando los pagos se hacen por medios electrónicos; también se corrigió la tasa de IVA adicional.
 
 % v3.40 -> En la versión 3.40 se agregaron nuevas opciones en cuanto al manejo de la fecha de emisión y su formato, posibilidad de facturas a crédito con fecha de vencimiento, campos nuevos adicionales para agregar más textos en la factura, posibilidad de imprimir información en el reverso de la hoja de la factura, posibilidades nuevas en cuanto al posicionamiento de los elementos fijos, diseño \tikz más versátil en los cuadros con bordes, mejor manejo del archivo |.csv|, cambios en los nombres de algunos comandos y opciones.
 
 % v3.40 -> En la misma versión 3.40 se cambió la estructura de la sección de los comandos en esta documentación y se agregó una página de descripción al final de cada ejemplo.
 
 % v3.44 -> El cambio en la política impositiva reflejado en la versión 3.40 queda sin efecto para fines legales, ya no está vigente dicho descuento; sin embargo, no eliminamos el código que lo produce sino que la clase arroja una advertencia cuando se pide usar la opción de descuentos por pago electrónico.
 
 % v3.44 -> Desde la versión 3.44 podemos activar cualquier opción de la clase a través de condicionales en el preámbulo y no solamente como una opción al cargar la clase; también se puede ahora quitar el efecto de cualquier opción previamente dada al cargar la clase, esto con el propósito de hacer más versátil el uso de las opciones y así poder incluirlas también en archivos |.tex| o |.def| personalizados.
 
 % v3.44 -> Adicionalmente desde la versión 3.44 el archivo |.log| de las compilaciones tiene como comentario todas las opciones que fueron pedidas al cargar la clase.
 
 % v3.47 -> La versión 3.47 corrige problemas con la compilación en \LuaLaTeX{} y escribe la versión de la clase en el campo \texttt{Subject} de la {\slshape metadata} del archivo |.pdf| compilado.
 
 % v3.47 -> La versión 3.47 agrega el comando |\Moneda| para cambiar fácilmente la moneda.
 
 % v3.56 -> {\bfseries La versión 3.56 se saca para corregir la versión 3.47 en cuanto a dar cumplimiento al Decreto Presidencial 3.332 sobre la Reconvesión Monetaria.} Desde el 4 de junio del año 2018 la moneda pasará a llamarse Bolívar Soberano (Bs.S) y luego en una fecha aún no definida (la asumimos como el 1 de enero del año 2019) la moneda volverá a llamarse Bolívar (Bs.). Eventualmente el Banco Central de Venezuela emitirá un comunicado para establecer desde cuándo la moneda de curso legal vuelve a llamarse simplemente Bolívar. A partír de esa fecha deberá usarse el comando |\Moneda| para cambiar la moneda a su denominación definitiva mientras no se haya emitido la actualización de esta clase. Esa fecha aún no está determinada, así que la suponemos -y hacemos el cambio de moneda automático en esta clase- como el inicio del año 2019. Antes del 4 de junio, se establece una nota interna automática que escribe el total también en Bolívares Soberanos.
 
 % v3.56 -> Por otra parte, la versión 3.56 elimina la inclusión del paquete |lmodern| y discrimina la inclusión de |fontenc| solamente para la compilación con pdf\LaTeX, que es lo correcto. Además, en esta versión se agrega un campo extra opcional en los ítems de la factura para poder agregar una descripción más larga a cada ítem que podrá ser referenciada en otras partes del documento (por ejemplo en el reporte) pero que no aparece en el cuadro de la factura.

 % v3.56 -> En la versión 3.56 fueron corregidos errores en cuanto al posicionamiento de la línea diagonal y se agregaron las opciones: \texttt{sinlineahni}, \texttt{sinlineahtit} y \texttt{sinlineahtot} para poder quitar las líneas horizontales internas una a una y no como antes, todas a la vez, con \texttt{sinlineashi}.

 % v3.57 -> {\bfseries Esta versión extraordinaria v3.57 se saca para dar cumplimiento al Decreto Presidencial 3.548 sobre la Reconversión Monetaria, publicado en Gaceta Oficial 41.446. Dicho decreto establece que la nueva moneda Bolívar Soberano no tendrá 3 ceros menos como lo establecía el derogado decreto 3.332 sino 5 ceros menos que el Bolívar Fuerte. La Reconversión Monetaria comienza el 20 de agosto del 2018\footnote{Salvo que se postergue otra vez.} y antes de esa fecha ya los precios tienen que estar especificados en ambas monedas, y esta clase está adaptada a esa exigencia.}

 % v3.57 -> Si el usuario especifica a mano la moneda con el comando |\Moneda| -introducido en una reciente versión-, la clase |factura| no interviene en nada en cuanto a la Reconversión Monetaria, ahora, si no se especifica la moneda, puede ocurrir lo siguiente:

 % v3.57 -> Al emitirse una factura con fecha anterior a la reconversión (20 de agosto del 2018), los montos quedarán expresados en Bolívares Fuertes, simbolizado como {\slshape Bs.}, y la clase automáticamente agregará un comentario como {\slshape nota interna} indicando cuál sería el monto en Bolívares Soberanos; esto para dar cumplimiento al Decreto Presidencial que indica que los montos deben estar en expresados en ambas monedas antes de la entrada de la nueva moneda.

 % v3.57 -> Si se emite una factura con fecha posterior a la reconversión, los montos se asumirán como expresados en Bolívares Soberanos, es decir, el usuario debe darlos ya con los 5 ceros quitados, la clase va a simbolizar la moneda como {\slshape Bs.S} y agregará un comentario similar al que acabamos de mencionar pero para explícitar el monto en Bolívares Fuertes, es decir, con 5 ceros más; esto no forma parte del Decreto pero lo quisimos agregar.
 
 % v3.57 -> Luego, la moneda volverá a llamarse Bolívar simplemente y se simbolizará otra vez como {\slshape Bs.}; ese momento aún no lo sabemos pero para los efectos de esta clase vamos a asumirlo como que ocurrirá el 1 de enero del año 2019.
 
 % v3.57 -> Esta será la última versión que sacamos solamente para corregir fechas de la Reconversión Monetaria; si el Gobierno Nacional cambia las fechas nuevamente, el usuario tendrá que hacer uso de la interfaz del paquete |datetime2| para ajustar el funcionamiento de esta clase a los nuevos plazos; para esto aclaramos que el identificador para la entrada de la Reconversión Monetaria se llama {\slshape FechaBsS}\footnote{recordemos que antes de esta fecha, los montos se expresan en Bolívares Fuertes y la nota automática aclara cuánto es en Bolívares Soberanos, y después de esta fecha es al contrario.} y el identificador para el momento (que aún no se ha dicho) desde el cual la moneda vuelve a llamarse Bolívar es {\slshape FechaBs}. Para ajustarlos se usa el comando \cs{DTMsavedate} y las fechas se dan en formato {\slshape yyyy-mm-dd}. Damos como ejemplo lo que actualmente está programado en esta clase:
 
 Desde el 15/02/2019 la moneda nacional se vuelve a llamar “Bolívar” abreviado “Bs” sin “soberano”; esto según Gaceta Oficial 41.587. Quitamos toda referencia a la Reconversión Monetaria.
 
 También en esta versión establecemos la tasa de IVA general como 16\% tal como ya está vigente desde agosto del año 2018.
 
 Ya no está activa la opción |electro| para descuentos en pagos por medios electrónicos porque ya no está vigente el decreto.
 
 Ahora en los totales aparece el signo monetario siempre a menos de que se pida en la clase la opción |sinmonedatotales|.
 
 En esta misma versión se agregó un “gancho” llamado \cs{BeforeEndPreamble} para usos muy específicos.
 
 Las “condiciones del documento” ya no son “condiciones” sino que ahora lo llamamos “nota final” porque es otra nota más aparte de las demás, y va al final del documento; no tiene sentido hablar de condiciones del documento para referirse a ese espacio. Con eso se renombran los 5 comandos, pero dejando por compatibilidad unos alias a los viejos; los nuevos son: \cs{NotaFinal}, \cs{PosNotaFinal}, \cs{SepNotaFinal}, \cs{LineaNotaFinal}, \cs{LetraNotaFinal}.

 La macro \cs{PosNotaFinal} se comporta ahora como \cs{PosFirmas}, es decir, acepta un valor inicial en “x” y uno final.

 \section*{Agradecimientos}
 
 Por pruebas y sugerencias que dieron lugar a mejoras agradezco a Ritguey%
 \ifmotorunicode\footnote{\setmainfont{Symbola}🐧}\else\message{^^J^^JLaTeX Warning: Compila con XeLaTeX.^^J}\fi{}
 Flores Esteves y a Alberto Mijares. Por contribuciones a través de \texttt{StackExchange} agradezco a Alan Munn, cfr, David Carlisle, egreg, Gustavo Mezzetti, Heiko Oberdiek, jfbu, Joseph Wright, Werner y wipet. El motivo por el que se da cada agradecimiento está como comentario en el archivo |.cls| o en el |.dtx|.
 
 \addcontentsline{toc}{section}{\changesname}
 {\makeatletter\def\pfill{\unskip\hfill\@gobble}\makeatother
  \setlength{\columnsep}{0pt}
  \PrintChanges}
 \phantomsection
 \addcontentsline{toc}{section}{\glosname}
 \PrintIndex
\end{document}
%</driver>
% \fi
%
%^^A Gracias a jfbu y a Joseph Wright por \starttodo, \stoptodo, \endorgo.
%^^A Thanks to jfbu and Joseph Wright for \starttodo, \stoptodo, \endorgo.
%^^A (http://tex.stackexchange.com/questions/235680/add-something-to-the-driver-section-inside-the-package-code-using-onlydescripti)
% \long\def\stoptodo#1\starttodo {}%
% \long\def\starttodo#1\stoptodo {}%
%
% \makeatletter
% \def\endorgo #1{%
%  \ifx #1\endinput\expandafter\stoptodo
%  \else #1\expandafter\@gobble\fi}%
% \makeatother
%
%^^A No se puede calcular el checksum con \OnlyDescription.
% \CheckSum{0}
%
% \CharacterTable
%  {Upper-case    \A\B\C\D\E\F\G\H\I\J\K\L\M\N\O\P\Q\R\S\T\U\V\W\X\Y\Z
%   Lower-case    \a\b\c\d\e\f\g\h\i\j\k\l\m\n\o\p\q\r\s\t\u\v\w\x\y\z
%   Digits        \0\1\2\3\4\5\6\7\8\9
%   Exclamation   \!     Double quote  \"     Hash (number) \#
%   Dollar        \$     Percent       \%     Ampersand     \&
%   Acute accent  \'     Left paren    \(     Right paren   \)
%   Asterisk      \*     Plus          \+     Comma         \,
%   Minus         \-     Point         \.     Solidus       \/
%   Colon         \:     Semicolon     \;     Less than     \<
%   Equals        \=     Greater than  \>     Question mark \?
%   Commercial at \@     Left bracket  \[     Backslash     \\
%   Right bracket \]     Circumflex    \^     Underscore    \_
%   Grave accent  \`     Left brace    \{     Vertical bar  \|
%   Right brace   \}     Tilde         \~}
%
% \changes{v1.00}{2013/04/29}{Primera versión.}
% \changes{v2.00}{2015/03/30}{Segunda versión: errores corregidos, comandos y opciones nuevas, ambiente de reporte, documentación completa, ejemplos y convertido todo a formato \texttt{.dtx}.}
% \changes{v2.05}{2015/04/13}{Se cambió de \texttt{4pt} a \texttt{6pt} en \texttt{factura.def} para el margen interno predeterminado en las cajas del encabezado y en la descripción.}
% \changes{v2.05}{2015/04/13}{Se quitó \texttt{UTF8} de \texttt{inputenc} como un paquete que se cargue de forma predeterminada.}
% \changes{v2.05}{2015/04/13}{Advierte que no son compatibles los comandos \cs{Descuento} con la opción \texttt{filas}.}
% \changes{v2.05}{2015/04/13}{Se puso un parche en \cs{@CargaFondoReporte} para arreglar que \cs{topskip} aplicara bien en \cs{@EncabezadoReporte}.}
% \changes{v2.05}{2015/04/13}{Arreglada la interacción entre las opciones \texttt{ivanoexplicito} y \texttt{filas}, y se quitó la línea del total cuando se usa \texttt{ivanoexplicito}.}
% \changes{v2.05}{2015/04/13}{Se renombró la opción \texttt{relleno} a \texttt{exparriba}, a la vez que se quitó \cs{Descripcion*} porque se agregó la opción \texttt{sinexpandir}.}
% \changes{v2.05}{2015/04/13}{Se renombró la opción \texttt{ivanoexplicito} por \texttt{sinivaexpresado}.}
% \changes{v2.05}{2015/04/13}{Se cambiaron los \cs{AtBeginDocument} por \cs{AtBeginShipoutNext} de la factura porque daba problemas al especificar \texttt{conreporte} sin luego agregar el ambiente de reporte.}
% \changes{v2.05}{2015/04/13}{Se colocó \cs{@SeparaMinTotales} dentro de otro \cs{AtBeginDocument} más adecuado.}
% \changes{v2.05}{2015/04/13}{Modificaciones en la documentación y en los ejemplos.}
% \changes{v2.05}{2015/04/13}{Los comandos \cs{ItemX} ahora colocan la línea bien con la opción \texttt{lineaitem}.}
% \changes{v2.1}{2015/08/26}{Advierte cuando la fecha del documento es distinta de la fecha del dia.}
% \changes{v2.1}{2015/08/26}{Los comandos de descuentos totales aceptan un * como argumento para que el descuento pueda ser en moneda y no en porcentaje.}
% \changes{v2.1}{2015/08/26}{Quitados los alias de macros viejas agregados por compatibilidad.}
% \changes{v2.1}{2015/08/26}{Corregida la forma como sale la denominación del impuesto para todos los casos.}
% \changes{v2.1}{2015/08/26}{Corregida y mejorada la documentación.}
% \changes{v2.1}{2015/08/26}{Correcciones en los ejemplos.}
% \changes{v2.1}{2015/08/26}{Mejor manejo de \cs{InicioReporte}.}
% \changes{v2.6}{2016/03/07}{Cambiado el lugar donde se fija el nombre del documento (FACTURA o COTIZACIÓN) porque antes no era correcto y no se podía cambiar el nombre de la cotización a algo distinto como por ejemplo PRESUPUESTO.}
% \changes{v2.6}{2016/03/07}{Se renombró la opción \texttt{sincond} en favor de \texttt{sincondiciones}.}
% \changes{v2.6}{2016/03/07}{Cambiado “condiciones de pago” por “condiciones del documento”.}
% \changes{v2.6}{2016/03/07}{Los descuentos globales aceptan decimales bien; antes no.}
% \changes{v2.6}{2016/03/07}{Se cambió la forma como se llama al ambiente \texttt{tabularx} en la factura creando una nueva macro que se evalúa completamente resultando en la definición de las columnas.}
% \changes{v2.6}{2016/03/07}{Agregadas cuatro opciones: \texttt{sinlineasvi}, \texttt{sinlineasve}, \texttt{sinlineash}, \texttt{sinlineashi}, \texttt{sinlineashe} y renombrada la opción \texttt{lineaitem} por \texttt{conlineasha} aludiendo a “con líneas horizontales para los artículos”.}
% \changes{v2.6}{2016/03/07}{Se arregló la línea oblicua que en algunos casos seguía sin salir.}
% \changes{v2.6}{2016/03/07}{Textos “PAGADO” y “Por el emisor” para las facturas y cotizaciones cuando no se define el emisor del documento.}
% \changes{v2.6}{2016/03/07}{Nueva opción \texttt{dosfirmas} para poner las dos firmas en casos en los que lo predeterminado sea una o ninguna firma (cotizaciones o prefacturas).}
% \changes{v2.6}{2016/03/07}{Nuevo comando \cs{Conforme} para el nombre de quien recibe la factura.}
% \changes{v2.6}{2016/03/07}{Cambiado el lugar en el código donde se llama a la macro que dibuja la línea inutilizadora del espacio en blanco y se hace ahora con \cs{AddToShipoutPicture} para que no se genere un espacio en blanco extra.}
% \changes{v2.6}{2016/03/07}{El ejemplo de la documentación es de una cotización y no de una prefactura; actualizada la documentación y los ejemplos.}
% \changes{v2.6}{2016/03/07}{En el índice ponemos “entorno” en lugar de “environment”.}
% \changes{v2.6}{2016/03/07}{El verbo “poner” es digno para la documentación. No sólo las gallinas ponen.}
% \changes{v2.6}{2016/03/07}{Renombrado el comando \cs{EstiloPag} por \cs{InfoPagina}; también \cs{LetraEncReporte} por \cs{LetraInfoPagina}; por otra parte se cambió \cs{LetraDatosReporte} en favor de \cs{LetraEncReporte}, y por último se cambió \cs{LetraTitRenglon} por \cs{LetraTitColumnas}.}
% \changes{v2.6}{2016/03/07}{Renombrados los estilos de página: \texttt{soloenc} a \texttt{soloarriba}, y \texttt{solopie} a \texttt{soloabajo}.}
% \changes{v2.6}{2016/03/07}{Se mejoró el código relativo al estilo de página del reporte y agregada la macro \cs{EstiloPagReporte} para establecer el estilo de página del reporte.}
% \changes{v2.6}{2016/03/07}{Se agregó una forma para que más de una línea en la descripción de la factura sea mejor manejada.}
% \changes{v2.6}{2016/03/07}{Mejorado el código de la macro interna \cs{@EscribeTipoDeImpuesto}.}
% \changes{v2.6}{2016/03/07}{Cuando la descripción en el cuadro de la factura ocupa más que una línea, agrega al final de cada línea el carácter $\hookleftarrow$.}
% \changes{v2.6}{2016/03/07}{Corregido que salga la “G” obligatoria de IVA General cuando hay un elemento exento y uno con IVA General.}
% \changes{v2.6}{2016/03/07}{Cuatro nuevas definiciones dentro del archivo \texttt{factura.def} para establecer algunos caracteres que antes estaban fijados dentro del archivo \texttt{factura.cls}}
% \changes{v2.6}{2016/03/07}{{\slshape parbox}es para cada elemento en la tabla, para asegurar que la denominación del gravamen quede alineada correctamente.}
% \changes{v2.6}{2016/03/07}{Quitados los paquetes \texttt{xifthen} y \texttt{pbox}. Arreglada la presentación de los datos del cliente cuando el resumen es más grande.}
% \changes{v2.6}{2016/03/07}{Cuando había descuentos totales y se pedía modo de totalización en filas, este modo no se apagaba correctamente y la línea oblicua daba problemas. Arreglado.}
% \changes{v2.72}{2016/05/16}{Las macros internas \cs{@DescripcionAuxCmd} y \cs{@EscribeTipoDeImpuesto} fueron modificadas.}
% \changes{v2.72}{2016/05/16}{Arreglada la alineación a la derecha en algunos casos en los precios.}
% \changes{v2.72}{2016/05/16}{Se incluyó dentro de la macro interna \cs{@ITEM} la opción para soportar \cs{ItemX}.}
% \changes{v2.72}{2016/05/16}{Ahora se procesa el entorno de la factura aún cuando se pide sólo el reporte, para poder tomar las referencias y los totales. Para eso se cambiaron de sitio varias macros internas.}
% \changes{v2.72}{2016/05/16}{Agregadas las etiquetas y las referencias.}
% \changes{v2.82}{2016/06/25}{Agregada la opción \texttt{numitem} para enumerar cada ítem en la factura.}
% \changes{v2.82}{2016/06/25}{Corregido problema de etiquetas repetidas introducido en la versión 2.72.}
% \changes{v2.82}{2016/06/25}{Corregida la alineación vertical entre las distintas columnas agregando \texttt{[t]} al abrir los “{\slshape parboxes}” en los comandos auxiliares que ponen cada celda en un {\slshape parbox} del tamaño de la celda más grande, porque si no se centra verticalmente la celda y se desalinea respecto a la columna de al lado.}
% \changes{v2.82}{2016/06/25}{Ahora dentro de la definición de \cs{@CaracterCeldasVacias} está el código que centra el texto, para poder usar comandos que llenen la caja completa como \cs{dotfill}.}
% \changes{v2.82}{2016/06/25}{Reimplementada la línea diagonal para que funcione en todos los casos.}
% \changes{v2.82}{2016/06/25}{Nueva macro interna \cs{@ColumnasMultiColumnTotalesAuxCmd} para hacer más general la colocación de los totales en el modo de columnas dentro de un \cs{multicolumn}.}
% \changes{v2.82}{2016/06/25}{Quitado el paquete \texttt{MnSymbol} en favor del comando \cs{flechaCR} para la flecha que indica que la descripción ocupa más de una línea.}
% \changes{v3.00}{2017/03/11}{Se agregó la opción \texttt{sinmarcas} para pedir que no se escriba el carácter de fin de linea cuando la descripción ocupa más que una línea y también para evitar que salga el carácter que marca las celdas vacías también cuando la decripción es larga.}
% \changes{v3.00}{2017/03/11}{Nueva macro \cs{SepItemsExtra} para agregar espacio extra entre los ítems en la factura.}
% \changes{v3.00}{2017/03/11}{Opción nueva \texttt{sintotales} para que no se escriba ningún total de los elementos. Útil para cotizaciones de por ejemplo varias formas de un mismo producto.}
% \changes{v3.00}{2017/03/11}{Opción nueva \texttt{centradesc} para centrar el texto en la descripción de cada elemento en la factura.}
% \changes{v3.00}{2017/03/11}{Cambiado el interlineado en la documentación, el tamaño del papel y agregados a la documentación todos los ejemplos que están embebidos en el \texttt{.dtx}, de forma automática.}
% \changes{v3.00}{2017/03/11}{El descuento individual ahora es opcional al pedir la opción, es decir, no hace falta establecer en '0' los elementos donde no haya descuento cuando se usa la opción \texttt{descuentos}.}
% \changes{v3.00}{2017/03/11}{Opción \texttt{subtotal} para agregar una columna extra después del precio unitario y antes del descuento, para reflejar el precio antes de aplicar el descuento individual.}
% \changes{v3.02}{2017/07/16}{Ya no usamos la macro \cs{fancy@setoffs} de \texttt{fancyhdr} en el reporte para establecer valores iniciales por ejemplo de la longitud de la línea en los encabezados; ahora lo hacemos de una forma diferente, incluyendo en el \texttt{.def} una macro llamada \cs{@GrosorLineaInfoPag} que alberga el valor que \texttt{fancyhdr} asigna preterminadamente al grosor de dicha línea. No agregamos un comando para cambiar ese valor por no ser algo de relevancia.}
% \changes{v3.02}{2017/07/16}{Ajustada ligeramente esta lista de cambios y la inclusión de los ejemplos en la documentación.}
% \changes{v3.40}{2017/12/11}{Se agregó la opción \texttt{electro} para dar cumplimiento con el Decreto Presidencial 3085 y la Providencia Administrativa 0048 que establece descuentos en el IVA.}
% \changes{v3.40}{2017/12/11}{Macros nuevas para notas internas al cuadro de la factura y externas a él.}
% \changes{v3.40}{2017/12/11}{Ahora \cs{PosCond} acepta \texttt{*} para que el texto pueda abrir hacia arriba en la posición establecida y no solamente hacia abajo.}
% \changes{v3.40}{2017/12/11}{Agregado el comando \cs{Credito} para la posibilidad de facturas a crédito donde es calculada la fecha de vencimiento de la factura que aparecerá debajo de la fecha de emisión.}
% \changes{v3.40}{2017/12/11}{Ahora \cs{PosFecha} también acepta \texttt{*} para que las fechas (emisión y vencimiento) puedan fijarse también respecto a una posición final y no solamente inicial.}
% \changes{v3.40}{2017/12/11}{El comando \cs{Fecha} acepta \texttt{*} para poder especificar la fecha en formato \texttt{iso} de forma más estándar para que sea tomada correctamente por el paquete \texttt{datetime2}, usado para calcular la fecha de vencimiento.}
% \changes{v3.40}{2017/12/11}{Se agregó la macro \cs{FormatoFecha} para poder cambiar el formato en el que el usuario da la fecha de emisión y cómo ésta es presentada.}
% \changes{v3.40}{2017/12/11}{Si no se especifica la fecha, será tomada la fecha actual y generará una advertencia indicando eso y no un error indicando que no se ha especificado la fecha, como era antes.}
% \changes{v3.40}{2017/12/11}{Actualizada la documentación para reflejar los cambios.}
% \changes{v3.40}{2017/12/11}{Se renombró \cs{@EstiloFecha} por \cs{@EstiloFechas}, y \cs{@PosicionaFecha} por \cs{@PosicionaFechas}.}
% \changes{v3.40}{2017/12/11}{Reestructuración de la documentación.}
% \changes{v3.40}{2017/12/11}{Nueva opción \texttt{vencimiento} para poner la fecha de vencimiento aún en facturas a contado.}
% \changes{v3.40}{2017/12/11}{También \cs{PosDenom} y \cs{PosControl} aceptan \texttt{*}. Cambió en 2mm el margen superior predeterminado, así como también hubo un cambio en la coordenada vertical de la fecha, la denominación del documento y el número de control, ya que antes los predeterminados estaban especificados sin \texttt{*} y luego no.}
% \changes{v3.40}{2017/12/11}{No poner el teléfono ahora produce una advertencia y no un error como antes.}
% \changes{v3.40}{2017/12/11}{Funciones internas para los mensajes informaticos, advertencias y errores de la clase.}
% \changes{v3.40}{2017/12/11}{Se le agregó la posibilidad de \texttt{*} a la macro interna \cs{@DatosAux} para que pudiera ser utilizada la misma macro por los casos donde hay resumen y donde no lo hay.}
% \changes{v3.40}{2017/12/11}{A los cuadros dibujados con \tikz les fue separada la definición del nodo para ponerla dentro del \texttt{.def} en lugar de estar en el \texttt{.cls}.}
% \changes{v3.40}{2017/12/11}{Cuando se especifica 0 como grosor de línea para la descripción o el encabezado, desaparece la línea completamente y la estructura \tikz.}
% \changes{v3.40}{2017/12/11}{Renombrada \cs{PosCond} a \cs{PosCondiciones}.}
% \changes{v3.40}{2017/12/11}{Ahora las condiciones aceptan línea en \tikz, es decir que hay dos macros nuevas: \cs{LineaCondiciones} y \cs{SepCondiciones}. Predeterminadamente las condiciones quedaron como antes, o sea sin diseño de \tikz.}
% \changes{v3.40}{2017/12/11}{Quitada la macro interna \cs{@TamDescrip} porque no hacía nada.}
% \changes{v3.40}{2017/12/11}{Renombrada \cs{SepEncabez} a \cs{SepEncabezado}, tambień \cs{SepDescrip} a \cs{SepDescripcion}, \cs{LineaEncabez} en favor de \cs{LineaEncabezado}, y \cs{LineaDescrip} en favor de \cs{LineaDescripcion}, así como algunas macros internas asociadas a ellas.}
% \changes{v3.40}{2017/12/11}{Macro nueva para una nota previa al cuadro de la factura.}
% \changes{v3.40}{2017/12/11}{La macro \cs{SepEncabezado} fue dividida entre \cs{SepDatosResumen} y \cs{SepEncabezado}. La macro \cs{SepElementos} fue renombrada a \cs{SepVertical} y \cs{LetraTitCuadro} a \cs{LetraTitColumnas}.}
% \changes{v3.40}{2017/12/11}{Contador \texttt{@NumItem} renombrado a \texttt{NumItem}. Las macros para los parámetros de las líneas ahora aceptan dar solamente el primer argumento, por ejemplo para definir que no salga la línea, sólo basta con por ejemplo \cs{LineaEncabezado}{0pt}.}
% \changes{v3.40}{2017/12/11}{Nuevo parámetro opcional después de la descripción en los \cs{Item***} para cambiar el campo que va a salir en el \texttt{.csv}, y ahora se puede especificar un descuento individual en los ítems, que será ignorado si no se carga la opción para descuentos individuales.}
% \changes{v3.40}{2017/12/11}{Actualizada la tasa de IVA adicional.}
% \changes{v3.40}{2017/12/11}{Nueva opción \texttt{iva} para agregar una columna adicional que muestra el IVA de cada item.}
% \changes{v3.40}{2017/12/11}{Modificada la macro interna \cs{@DescripcionAuxCmd} por problemas que daba en algunos casos.}
% \changes{v3.40}{2017/12/11}{Se renombró la opción \texttt{exparriba} por \texttt{expandecuadro}.}
% \DoNotIndex{}
% \changes{v3.40}{2017/12/11}{Modificada la definición de \cs{ItemX} para que maneje la descripción especial para el archivo \texttt{csv}.}
% \changes{v3.40}{2017/12/11}{Nuevo comando \cs{ivaref} para referenciar el IVA del ítem etiquetado.}
% \changes{v3.40}{2017/12/11}{Se cambiaron unos \cs{AtEndDocument} por \cs{AtBeginShipoutNext}. Se hizo necesario siempre el paquete \texttt{eso-pic}.}
% \changes{v3.40}{2017/12/11}{Se quitó la fijación del estilo de página de un \cs{AtBeginDocument} innecesario. Nuevo entorno \texttt{reverso} para el reverso de la factura y nuevo comando para la letra predeterminada del entorno: \cs{LetraReverso}.}
% \changes{v3.40}{2017/12/11}{Cada ejemplo tiene una página al final con su explicación correspondiente.}
% \changes{v3.40}{2017/12/11}{Ahora en los ejemplos, primero va la opción o la macro entre paréntesis y después va la explicación y no al revés como era antes.}
% \changes{v3.40}{2017/12/11}{Quitada la opción \texttt{sincondiciones}. Si no se establecen las condiciones produce una advertencia.}
% \changes{v3.40}{2017/12/11}{Quitada la macro \cs{mostrar}. Ahora cada texto a mostrar se hace con el mismo comando con el que se definió.}
% \changes{v3.40}{2017/12/11}{Algunos condicionales cambiados.}
% \changes{v3.40}{2017/12/11}{Otra vez corregida la macro interna \cs{@DescripcionAuxCmd} y arreglado el espaciado entre los ítems; ojalá que esta vez sí esté bien.}
% \changes{v3.44}{2018/03/13}{Reestructurado el archivo \texttt{.cls} para aceptar prender y apagar las opciones de la clase a través de condicionales de \TeX{} con el mismo nombre de la opción.}
% \changes{v3.44}{2018/03/13}{Se cambió un \cs{AtBeginDocument} que escribía texto en la hoja por un \cs{AfterEndPreamble}.}
% \changes{v3.44}{2018/03/13}{Se acabó el descuento del IVA. Ahora advierte cuando se usa, posibilitando su uso después si llegan a volver a activar la medida.}
% \changes{v3.44}{2018/03/13}{Cada opción cargada escribe un comentario en el archivo \texttt{.log} para depuración.}
% \changes{v3.47}{2018/04/17}{No compilaba con \LuaLaTeX{} por diferencias en el motor. Ya compila.}
% \changes{v3.47}{2018/04/17}{Agrega la versión de la clase bajo \texttt{Subject} en la metadata de los PDF compilados.}
% \changes{v3.47}{2018/04/17}{Se agrega la macro \cs{Moneda} para cambiar el signo monetario a usar.}
% \changes{v3.47}{2018/04/17}{Se da cumplimiento al Decreto Presidencial 3.332 de la reconversión monetaria.}
% \changes{v3.47}{2018/04/17}{Cambiados algunos de los \cs{immediate}\cs{write} por \cs{write}.}
% \changes{v3.56}{2018/05/09}{Quitamos \texttt{lmodern} de todas partes y mejoramos la inclusión discrecional de \texttt{fontenc}.}
% \changes{v3.56}{2018/05/21}{Agregada la opción de una descripción larga para cada ítem de la factura únicamente para referenciarla luego en otra parte del documento.}
% \changes{v3.56}{2018/05/21}{La implementación anterior del Bolívar Soberano tenía defectos; ya están corregidos.}
% \changes{v3.56}{2018/05/21}{Volvimos a agregar la macro interna \cs{@TamDescripcion} quitada en la versión v3.40. Sin ella la línea no salía bien con la opción \texttt{filas}.}
% \changes{v3.56}{2018/05/21}{Creamos condicionales internos: \cs{if@ExpandeCuadro} y otros para evaluar la línea inutilizadora del espacio en blanco, así como también unas macros que guardan los valores de \cs{baselineskip} y \cs{tabcolsep} y otros cambios en la línea.}
% \changes{v3.56}{2018/05/21}{Nuevas opciones \texttt{sinlineahni}, \texttt{sinlineahtit} y \texttt{sinlineahtot} para no poner la línea horizontal antes de la nota interna, después del título de las columnas y antes de los totales.}
% \changes{v3.56}{2018/05/21}{Agregamos el paquete \texttt{anyfontsize}.}
% \changes{v3.57}{2018/08/02}{El Gobierno Nacional posterga la Reconversión Monetaria. Cambiamos un poco la forma como estaba programado esto. El usuario podrá cambiar las fechas que intervienen en eso, por si ocurre otro cambio.}
% \changes{v3.70}{2019/08/17}{Nueva tasa de IVA general del 16\% vigente desde agosto del 2018.}
% \changes{v3.70}{2019/08/17}{Establecido el 15/02/2019 como la fecha en la que ocurre que la nueva moneda pierde el adjetivo “soberano” para llamarse “bolívar” simplemente.}
% \changes{v3.70}{2019/08/17}{Nueva macro interna \cs{@EstableceOpciones} para poder evaluarla doblemente, una antes del -nuevo también- gancho \cs{BeforeEndPreamble} y otra después de él, porque en ese gancho se puede escribir código que altere las opciones y deberán ser evaluadas nuevamente para la consistencia entre ellas.}
% \changes{v3.70}{2019/08/17}{Inhabilitada la opción \texttt{electro} pero se deja el código todavía, aunque comentado.}
% \changes{v3.70}{2019/08/17}{Se comentó todo el código referente a la Reconversión Monetaria.}
% \changes{v3.70}{2019/08/17}{Implementada la función de que agregue el signo monetario en los totales, y una opción \texttt{sinmonedatotales} para evitarlo.}
% \changes{v3.70}{2019/08/17}{Ya no hablamos de condiciones sino de “nota final”, que es aquello que va después de la descripción larga y que tiene posición fija. Ese espacio se puede usar para distintas cosas, por lo que hablar de ese espacio como el de las condiciones del documento no tiene mucho sentido. De manera que los comandos renombrados pasaron a ser \cs{NotaFinal}, \cs{PosNotaFinal}, \cs{SepNotaFinal}, \cs{LineaNotaFinal}, \cs{LetraNotaFinal}. Se mantienen los comandos anteriores por compatibilidad.}
% \changes{v3.70}{2019/08/17}{Corregido el posicionamiento en “x” de la nota final.}
% \changes{v3.70}{2019/08/17}{Ahora la macro \cs{PosNotaFinal} acepta un intervalo en “x” para posicionar la nota final, es decir que funciona como \cs{PosFirmas}.}
% \changes{v3.70}{2019/08/17}{Mejorada la macro interna que escribe la fecha en la metadata; ya no usa \texttt{xstring} sino \LaTeX3. Quitado código espúreo de \texttt{xstring} en las macros que permiten fijar las posiciones de los elementos fijos.}
%
%
% \tableofcontents
%
%
% \section*{Brief description in english}
%
% |factura| is the spanish word for `invoice', so this is a \LaTeX{} class for typesetting and calculating invoices, made taking into account requirements of {\slshape SENIAT} legislation (tax collector entity on the Bolivarian Republic of Venezuela), but its use is not restricted to Venezuela because all variables and displayed text can be redefined by invoking commands or editing.
%
% \newpage
% \setstretch{1.4}
%
%
% \section{Introducción}
%
% La clase de \LaTeX{} |factura| está diseñada para su uso en la República Bolivariana de Venezuela, conforme a los requerimientos fijados por las providencias del {\slshape SENIAT} 0591, 0257, 0071 y 0048, válidas éstas para agosto del año 2019.
%
% Con |factura| se pueden elaborar facturas en formas libres fiscales\footnote{En cuanto a las formas libres, cada hoja del talonario de facturas legales tiene un encabezado con los datos del emisor de la factura y un pie de página con la información de identificación de la imprenta, en tamaño de hoja carta entera; esto es lo que se llama “forma libre”.}, cotizaciones y reportes de servicios, dejando todos los cálculos a \TeX{}.
%
% El diseño de las facturas hechas con esta clase consiste en un encabezado con los datos que identifican al cliente, con al lado un -opcional- resumen; estas partes de la factura están encerradas en recuadros. Si no se especifica el resumen saldrá solamente una de estas cajas en el encabezado que es la de los datos del cliente.
%
% Luego de esto viene un -opcional- recuadro con una nota donde puede ir la forma de pago, luego un cuadro con los campos: cantidad, descripción, precio unitario y precio total\footnote{Pueden haber variaciones en estos campos controlables con las opciones de la clase (véanse subsecciones \ref{subsec:columnas}, \ref{subsec:descuentos} y \ref{subsec:sincantidad}).}, con los elementos a facturar, cada uno en una línea, luego los totales, luego una -opcional- nota dentro del cuadro de la factura, otra nota (opcional también) después del cuadro de la factura y al final viene una opcional descripción larga que sale en un recuadro. La factura puede tener también un reverso con más información.
%
% En fin, la factura consta de:
% \begin{enumerate}[\bfseries i)]
%  \item el encabezado con los datos del cliente y el -opcional- resumen,
%  \item la opcional nota previa al cuadro de la factura,
%  \item el cuadro con los elementos que se están facturando y las totalizaciones,
%  \item la opcional nota interna al cuadro de la factura,
%  \item la opcional nota externa al cuadro de la factura,
%  \item una opcional descripción larga, e
%  \item información opcional en el reverso de la hoja.
% \end{enumerate}
%
% Aparte, el usuario puede escribir un reporte largo asociado a la factura, es decir, esta clase sirve también para escribir reportes.
%
% \newpage
%
%
% \section{Instalación}
%
% El paquete que está en CTAN consta de un archivo |README| en formato |UTF-8|, esta documentación compilada como |factura.pdf| y el archivo |factura.dtx|, del cual se generan todos los demás archivos.
%
% Al compilar |factura.dtx| va a producirse |factura.cls| y |factura.def|, además de todos los ejemplos en archivos |.tex| y |factura.ins|.
%
% \noindent Si queremos solamente generar los archivos sin compilar la documentación, debemos ejecutar:
%
% \noindent {\color{RoyalBlue}|pdftex factura.dtx|}
%
% \noindent Para generar los archivos y también compilar la documentación, debemos ejecutar los siguientes pasos:
%
% \begin{enumerate}[\bfseries 1. ]
%
% \item {\color{RoyalBlue}|xelatex factura.dtx|}\footnote{Es preferible compilar el |.dtx| con \XeLaTeX{} o con \LuaLaTeX pero también puede hacerse con pdf\LaTeX.}
%
% Luego de compilar |factura.dtx| hay que compilar el fondo de ejemplo, el membrete de ejemplo y la firma de ejemplo; estos tres archivos emulan haber digitalizado una hoja de una factura del talonario fiscal, un membrete para los reportes y también una firma que puede agregarse a los documentos a producir. Estos ejemplos se usarán en algunos de los ejemplos de formatos que incluímos y por eso es necesario tenerlos compilados en |.pdf| previamente, ya que su inclusión en los ejemplos es en formato compilado |.pdf|.
%
% El fondo y membrete de ejemplo los debemos compilar así:
%
% \item {\color{RoyalBlue}|pdflatex factura-ejemplo-fondo.tex|}
% \item {\color{RoyalBlue}|pdflatex factura-ejemplo-membrete.tex|}
%
% La firma escaneada de ejemplo necesita compilarse con \XeLaTeX{} porque son códigos en PSTricks y debe producirse un |.pdf|. Se compila así:
%
% \item {\color{RoyalBlue}|xelatex factura-ejemplo-firma.tex|}
%
% Al tener el fondo, el membrete y la firma compilados en |.pdf| según los pasos anteriores, podremos compilar cualquiera de los ejemplos en |.tex| que el archivo |factura.dtx| genera, y eso lo debemos hacer, ya que todos ellos (en su versión en |.pdf|) forman parte de la documentación. Los podemos compilar así:
%
% \item {\color{RoyalBlue}|pdflatex factura-ejemplo-prefactura.tex|}
%
% De esta forma como compilamos {\ttfamily factura-ejemplo-prefactura} compilamos todos los demás ejemplos: la cotización, las facturas del 1 hasta la última, el reporte del 1 hasta el último, y finalmente la factura para otra legislación.\footnote{Podemos también compilar con \LuaLaTeX{} o con \XeLaTeX{}.}
%
% Al tener todo esto hecho volvemos a compilar |factura.dtx| para producir esta documentación, puesto que el |.dtx| necesita ser compilado más de una vez para que la documentación quede correctamente elaborada:
%
% \item {\color{RoyalBlue}|xelatex factura.dtx|}
%
% Los comandos para generar el índice y la lista de cambios, que automáticamente se ejecutan al compilar |factura.dtx| son:
%
% \item {\color{RoyalBlue}|makeindex -s gind.ist -o factura.ind factura.idx|}
% \item {\color{RoyalBlue}|makeindex -s gglo.ist -o factura.gls factura.glo|}
%
% {\slshape No es necesario actualizar los índices con los dos comandos anteriores porque esos dos comandos los ejecuta el motor de compilación al actuar sobre }|factura.dtx|.
%
% \end{enumerate}
%
%
% \section{Archivos incluídos en el paquete}
%
% Este paquete incluye varios ejemplos, entre ellos la emulación de un diseño de factura entregado por alguna imprenta, que nos sirve para nuestros ejemplos de prefacturas, un diseño de membrete que nos sirve para los reportes, una firma que emula ser una firma real escaneada para su inclusión en algunos casos, así como varios ejemplos de facturas, prefacturas, cotizaciones y reportes. Cada uno de ellos tiene al principio del documento fuente la aclaratoria de qué es lo que hace, para diferenciarlo de los demás ejemplos.
%
% \noindent \begin{tabular}{rl}
%  \bfseries Archivo & \bfseries Descripción \\ \hline
%  |factura.dtx| & Archivo principal del que todos los demás se generan, \\
%  |factura.pdf| & Esta documentación, \\
%  |README| & Archivo con la descripción corta, \\
%  |factura-ejemplo-firma.tex| & Ejemplo de firma digitalizada, \\
%  |factura-ejemplo-fondo.tex| &  Elemplo de una forma libre fiscal digitalizada, \\
%  |factura-ejemplo-membrete.tex| & Ejemplo de membrete para los reportes, \\
%  |factura-ejemplo-aux.tex| & Archivo auxiliar que se incluye en todos los ejemplos y \\
%                            & sirve para que cada ejemplo conste de su descripción, \\
%  |factura-ejemplo-prefactura.tex| & Ejemplo de “{\slshape prefactura}”, \\
%  |factura-ejemplo-cotizacion.tex| & Ejemplo de cotización, \\
%  |factura-ejemplo-factura*.tex| & Ejemplos de facturas, \\
%  |factura-ejemplo-reporte*.tex| & Ejemplos de reportes, \\
%  |factura-beispiel-rechnung.tex| & Un ejemplo de una factura adaptada a otra legislación. \\ \hline
% \end{tabular}
%
% \IfFileExists{factura-ejemplo-prefactura.pdf}{%
%  \newpage\section{Ejemplo preliminar\label{sec:ejemplo}}
%  Lo que sigue es un ejemplo del aspecto final de una {\slshape “prefactura”}, que está bajo el nombre de “|factura-ejemplo-prefactura.tex|” y más adelante explicaremos en qué consiste:
%
%  \noindent
%  \mbox{}\hfill
%  \vfil
%  \fbox{\includegraphics[width=0.9\linewidth]{factura-ejemplo-prefactura.pdf}}
%  \hfill\mbox{}
%  \vfil\mbox{}
%  \newpage}
% {\message{^^JLaTeX Warning: No existe factura-ejemplo-prefactura.pdf; hay que compilarlo antes de compilar este documento.^^J}}
%
%
% \section{Geometría de la hoja\label{sec:geohoja}}
%
% El espacio que la imprenta deja en blanco en la forma libre está todo disponible para la impresión de la factura con esta clase. Este espacio se delimita con el paquete |geometry|. Los valores predeterminados que tienen que ver con la geometría de la hoja están todos definidos dentro del archivo |factura.def|.
%
% El margen superior que se da con |geometry| está contabilizado desde el inicio del papel\footnote{El papel predeterminado es tamaño carta.} hasta donde empieza el encabezado con los datos del cliente, es decir que el encabezado es el primer elemento desde arriba hacia abajo en una factura.
%
% El margen inferior de |geometry| empieza donde termina la descripción del servicio -o el fin del cuadro de la factura, en caso de que no se agregue un texto de descripción larga-, hasta el final del papel.
%
% Para aclarar lo anterior es necesario decir que hay textos con posiciones fijas que están fuera de estos márgenes, como la fecha de emisión, la “denominación” del documento (que es “FACTURA”, “COTIZACIÓN” u otro especificable), el número de la factura, el número de control (cuando aplica), las nota final y los espacios para las firmas; todas estas partes del documento están fijas en posiciones arbitrarias fuera de los márgenes dejados por |geometry|. Estos textos son fijos para darles posiciones que no dependan del contenido particular de cada factura sino del diseño del talonario que ya tenemos impreso, que siempre será igual, y para el cual estamos adaptando esta clase |factura|.
%
% Las ubicaciones de los textos fijos también tienen valores predefinidos dentro de |factura.def|. Cualquiera de esos valores puede cambiarse mediante comandos, tal como se fijan los valores predeterminados en dicho archivo |.def|; esto significa que se puede personalizar a cualquier diseño del talonario entregado por la imprenta, basta con entender qué hace cada comando (explicado en este documento), consultar el |.def| o ver los ejemplos.
%
% En algunos casos las formas libres tienen impreso el número de la factura y no es controlable esto por el emisor (como ocurre con el número de control que la legislación obliga a que siempre venga impreso en cada hoja del talonario). En otros casos el emisor tiene la libertad de escribir el número de la factura y controlarlo él mismo. Se puede hacer funcionar la clase según cualquiera de estas dos formas con una opción en |\documentclass|, es decir, podemos imprimir el número de la factura donde queramos, o también podemos asumirlo impreso en el talonario, según sea el caso del talonario que tengamos.
%
% Al usarse esta clase se deben primero hacer medidas en las facturas que la imprenta nos entrega para fijar los márgenes y determinar dónde poner los textos fijos que se deseen cambiar de la posición que viene predeterminada.
%
% El archivo |factura.def| tiene todas aquellas definiciones y valores que pueden ser ajustados; basta usarlos en nuestros documentos de la misma forma como están predefinidos allí; eso aplica no sólo para los de la geometría de la hoja sino para cualquier ajuste que esté en ese archivo o cualquier campo que tenga algún texto o palabra predefinida para la legislación venezolana.
%
%
% \section{Uso de la clase}
%
% El documento debe comenzarse con |\documentclass{factura}| con las opciones necesarias, que se explicarán en la sección \ref{sec:opciones}. Después se definen a través de macros {\bfseries en el preámbulo} todas las variables y dimensiones para personalizar el documento, esto es las posiciones de los textos fijos, tipos de letra (opcional), datos del cliente, un resumen (opcional), fecha y demás. Despues de |\begin{document}| va el cuadro de la factura propiamente, para lo cual la clase define un entorno llamado |factura|
% \DescribeEnv{factura}
% (|\begin{factura}...\end{factura}|), en cuyo interior se escriben únicamente los elementos a facturar a través de comandos del tipo |\Item|\footnote{Es posible incluir elementos gravados en IVA con alícuota general (12\%), alícuota reducida (8\%), alícuota adicional (27\%), así como también elementos exentos de impuesto al valor agregado, con \cs{Item}, \cs{ItemR}, \cs{ItemA} e \cs{ItemE} respectivamente.}, especificando su cantidad, descripción y precio unitario como argumentos, en ese orden.
%
% Finalizado el entorno se puede escribir (opcionalmente) una descripción larga o una aclaratoria de lo que se factura; esto se hace con el comando |\Descripcion|.
%
% Algunas de las posibilidades de cambio de estilo de esta clase consisten en que puede cambiarse la forma como es presentado cualquiera de los textos, se pueden suprimir las líneas verticales y horizontales del cuadro de la factura, se pueden agregar líneas horizontales separando cada ítem, también se puede dibujar una línea oblicua que proteja el cuadro de la factura de adiciones posteriores de texto, entre otras prestaciones más.
%
% Si el emisor lo desea puede digitalizar su firma personal y colocarla en el espacio ya dispuesto para firmar, con un ajuste mínimo para que salga correctamente sobre la línea.
%
% También el usuario tiene la posibilidad de producir una {\slshape “prefactura”}; así le llamamos a un documento sin validez legal pero que puede ser enviado por fax o por correo electrónico. Este documento tiene el mismo aspecto de la factura fiscal si se digitaliza la forma libre vacía\footnote{Quitándole las numeraciones obviamente. También puede diseñarse en un |.tex| una forma libre vacía idéntica a la que se tiene, así como se hace en el ejemplo incluído.}, se incluye la imagen como fondo de la página y se definen las posiciones fijas de las numeraciones impresas en el talonario, es decir el número de factura y el número de control, de los que hablamos previamente.
%
% Con esta misma clase tiene se la posibilidad de imprimir un información en el reverso de la hoja de la factura que saldrá escrita utilizando toda la hoja (los márgenes son ajustables). Esa información se encierra en un ambiente destinado para ese fin (|\begin{reverso}...\end{reverso}|),
% \DescribeEnv{reverso}
% y al momento de la compilación puede decidirse si se imprime esa parte o no.
%
% Otra posibilidad que da esta clase es la de escribir un reporte largo asociado a la factura\footnote{Si lo hubiera, así como también se puede usar estar clase para elaborar un reporte sin asociarlo a una factura en particular.}. Tal reporte se escribe dentro de un ambiente en el mismo documento en el que se escribe la factura asociada a él (|\begin{reporte}...\end{reporte}|),
% \DescribeEnv{reporte}
% y con dar una opción al inicio se puede hacer que se impirma el reporte en las páginas siguientes a la factura, o también se puede hacer que se imprima sólo el reporte ignorando todo lo que no constituye el reporte, de la misma manera como también se puede hacer que se imprima sólo la factura ignorando el entorno |reporte|. Esto permite tener todo en un mismo documento e imprimir lo que se desee en el momento, o todo a la vez.
%
%
% \section{Compilación de un documento\label{sec:compila}}
%
% Para compilar una |factura| puede usar cualquiera de los motores \LaTeX, pdf\LaTeX, \XeLaTeX{} o \LuaLaTeX, naturalmente con las restricciones que impone cada uno.
%
% Hay casos en los que hace falta compilar más de una vez el documento. Dos veces necesita ser compilado un documento cuando alguna descripción en el cuadro de la factura ocupa más que una línea; también dos veces cuando se incluye una firma en el reporte, y tres o cuatro veces cuando se pide que se dibuje una línea oblicua que inutilice el espacio restante en el cuadro de la factura o cuando se usan etiquetas (sección \ref{sec:etiquetas}).
%
% Todo documento |.pdf| compilado con la clase |factura| escribe en la {\slshape metadata} del |.pdf| un texto con la versión y la fecha de la clase con la que se compiló; ese texto va en el campo |Subject| de dicha {\slshape metadata}.
%
%
% \section{Opciones al cargar la clase\label{sec:opciones}}
%
%  Al cargar la clase con |\documentclass| podemos dar distintas opciones según el comportamiento que queramos; dichas opciones están descritas en esta sección y cualquier opción cargada se verá como comentario en el archivo |.log| de la compilación bajo el texto “{\ttfamily Opción cargada: }” con el nombre de la opción pedida.
%
%
% \subsection{Sobre el tipo de documento a producir}
%
% {\setlength{\parindent}{0pt}
%
% \DescribeOption{prefactura}
%  La {\slshape “prefactura”} es un documento no legal, preliminar, que no se imprime sobre papel de factura sino que está concebido para enviarse por correo o fax. Para eso se define la imagen que contiene el diseño de la factura con el comando |\FondoFactura|, luego la clase escribe la denominación “FACTURA” con su número\footnote{Lo predeterminado cuando no es una {\slshape “prefactura”} sino una factura final es no escribir la denominación ni el número, ya que eso generalmente viene impreso en cada hoja del talonario.}. Esta opción implica la opción |sinfirmas|, ya que si es un documento preliminar sin validez legal, las firmas no son necesarias; sin embargo se puede cambiar este comportamiento con las opciones |dosfirmas| y |unafirma|.
%
% \DescribeOption{cotizacion}
%  La cotización es un documento similar a una factura pero no es un documento legal en el sentido de que no se imprime sobre talonario fiscal alguno, así que no lleva número de control y puede o no tener su propia numeración, que es cosa del emisor de la cotización. Esta opción implica la opción |unafirma|, para que firme el emisor de la factura, y si su nombre no se especifica saldrá “Por el emisor” debajo de la línea para firmar. También puede cambiarse esto con la opción |dosfirmas| o con |sinfirmas|.
%
% \DescribeOption{sinreverso}
%  No escribe la información -opcional- que se escribió dentro del entorno |reverso| a salir en el reverso de la factura.
%
% \DescribeOption{conreporte}
%  Genera la factura (prefactura o cotización) y luego el reporte.
%
% \DescribeOption{soloreporte}
%  Genera solamente el reporte, ignorando lo relacionado con la factura. Si no se especifica |conreporte| ni |soloreporte| se va a imprimir únicamente la factura.
%
%
% \subsection{Relativo a las columnas que puede tener el cuadro de la factura\label{subsec:columnas}}
%
% \DescribeOption{numitem}
%  Con la opción |numitem| una nueva columna aparece en el cuadro de la factura donde se enumera cada ítem en el órden en el que se escriben dentro del entorno para la factura. Esto es útil por ejemplo cuando se hace referencia al número del ítem con el comando |\itemref|. Véase la sección \ref{sec:etiquetas}.
%
% \DescribeOption{sincantidad}
%  El ambiente |factura| tiene también la posibilidad de no mostrar el campo de la cantidad, ni aceptarlo como argumento de cada ítem; en ese caso los comandos |\Item| -de los que hablaremos en la sección \ref{sec:factura}- aceptan como argumentos solamente la descripción del elemento y su precio, en ese orden.
%
% \DescribeOption{descuentos}
%  Permite agregar un descuento en porcentaje en cada elemento de la factura, siendo dicho porcentaje el último argumento de los comandos |\Item|. Si no aplica descuento para un ítem, puede este dejarse sin ese argumento, dando el mismo resultado que indicar “0” como argumento para ese descuento. (Explicado también en la sección \ref{subsec:descuentos}).
%
% \DescribeOption{subtotal}
%  Agrega una columna extra en la factura; sólo aplica cuando también se usa la opción |descuentos|. La nueva columna escribe el subtotal que cada ítem produce, es decir, el precio unitario multiplicado por la cantidad sin aplicar el descuento. Esta opción sobrecarga la factura (precio unitario, descuento, subtotal y precio total), y su uso es recomendable cuando también se usa |sincantidad|, porque de esta forma se tiene el subtotal, el descuento, y el total.
%
% \DescribeOption{iva}
%  Agrega una nueva columna donde se expresa el porcentaje del IVA para cada elemento en la factura. Sin especificar esta opción queda claro según lo establecido por nuestra legislación cuál será el gravamen que aplica para cada elemento por una letra que lo denota al lado del precio o de la descripción.
%
% \DescribeOption{duc}
%  Cambia el orden de los campos en el documento impreso a: descripción, precio unitario, cantidad y precio total. {\itshape Esto no cambia la forma como deben especificarse los argumentos de |\Item| sino solamente cambiará cómo serán mostrados}.
%
% \DescribeOption{dcu}
%  Cambia el orden de los campos en el documento impreso a: descripción, cantidad, precio unitario y precio total\footnote{Sin \texttt{duc} ni \texttt{dcu}, el modo predeterminado es \texttt{cdu}: cantidad, descripción, precio unitario.}.
%
%
% \subsection{Relativo al IVA}
%
% \DescribeOption{siniva}
%  No se cobra IVA, para los casos en los que eso aplique.
%
% \DescribeOption{sinivaexpresado}
%  Al no cobrar IVA, tampoco aparece en la totalización los IVA ni las bases imponibles.
%
% \DescribeOption{todosiva}
%  Agrega el monto exento y todos los tipos de IVA en los totales así no se hayan incluído elementos gravados con los distintos tipos de impuesto.
%
% \DescribeOption{electro}
%  {\bfseries Esta opción ya no está vigente en la legislación y se quitó de esta clase.} Se agregó en su momento esta opción para dar cumplimiento con las Gacetas Oficiales números 41239, 41242 y 41244\footnote{De fecha 19, 21 y 26 de septiembre del año 2017, respectivamente. En la última de estas gacetas está la Providencia Administrativa PA0048 del SENIAT.} donde se establece que las operaciones de compra (salvo excepciones establecidas en la Gaceta) gravadas con IVA general gozarán de un descuento cuando son pagadas mediante medios electrónicos. El descuento es del 3\% cuando la factura es de un monto menor a Bs.F 2.000.000 y del 5\% cuando el monto de la factura es de Bs.F 2.000.000 o más. Al dar esta opción el descuento queda establecido según el Decreto Presidencial número 3085.
%
% {\itshape Esta opción no es compatible con el modo de totalización en filas.}
%
% \DescribeOption{G}
%  Agrega una “|G|” al lado de todos los items gravados con IVA general así sea el general el único IVA existente en la factura.
%
% \DescribeOption{impdesc}
%  Con esta opción, la letra que indica la denominación del impuesto aplicado a cada elemento, en lugar de escribirse en el precio unitario y en el total, que es la forma predeterminada, sale escrito al final de la descripción. Puede preferirse de esta forma, y ahorra un poco de espacio usable en la descripción porque ya no sale en dos lugares dicha letra sino en uno solo.
%
%
% \subsection{Sobre las líneas en el cuadro de la factura}
%
% \DescribeOption{sinlineasve}
% \DescribeOption{sinlineasvi}
% \DescribeOption{sinlineasv}
% \DescribeOption{sinlineashe}
% \DescribeOption{sinlineashi}
% \DescribeOption{sinlineash}
%  La clase provee varias opciones para no dibujar las líneas del cuadro de la factura. La clase diferencia entre las líneas internas y las líneas externas. Las líneas externas son las que conforman el contorno externo del cuadro de la factura. Las líneas internas en el caso de las horizontales son la línea que va después del título de las columnas (cantidad, descripción, ...), la línea que separa los ítems de los totales y la línea que separa los totales de la nota interna; en el caso de las verticales, las líneas internas son las que separan las columnas.
%
%  Podemos apagar cada conjunto de línea; ya explicado qué es interno y qué es externo, queda aclarar que “ve” significa vertical externa, “vi” vertical interna, “he” horizontal externa y “hi” horizontal interna. Usando |sinlineasv| o |sinlineash| apagamos todas las líneas verticales u horizontales.
%
% \DescribeOption{sinlineahtit}
% \DescribeOption{sinlineahtot}
% \DescribeOption{sinlineahni}
%  Adicionalmente podemos apagar las líneas horizontales internas por separado, es decir, la línea que va después del título de las columnas la podemos quitar con \texttt{sinlineahtit}, la línea que separa los elementos a facturar de los totales generales se puede quitar con \texttt{sinlineahtot} y la línea que separa los totales de la -opcional- nota interna la podemos quitar con \texttt{sinlineahni}. 
% 
% \DescribeOption{conlineasha}
%  Agrega una línea horizontal separando cada artículo o ítem de los demás dentro del cuadro de la factura, como las facturas hechas a mano que tienen un espacio determinado para cada artículo.
%
%  Si no se especifica nada sobre las líneas, saldrán todas ellas (verticales y horizontales) salvo las que separan  los ítems a facturar entre sí.
%
%
% \subsection{Elementos a mostrar o no mostrar en la factura}
%
% \DescribeOption{sinmarcas}
%  Con |sinmarcas| evitamos que el carácter de fin de línea aparezca cuando la descripción ocupa más que una línea; ese carácter indica que la línea continúa abajo, y predeterminadamente es la flecha “\flechaCR”. También con esta opción no aparece el carácter “$\cdot$” en las celdas vacías en este mismo caso.
%
% \DescribeOption{sinmonedatotales}
%  Esta opción |sinmonedatotales| permite quitar el signo monetario de los totales, que antes nunca salía y ahora sale de manera predeterminada.
%
% \DescribeOption{unafirma}
%  No pone el espacio para que firme quien recibe el documento sino sólo el emisor; esto es lo predeterminado en las cotizaciones.
%
% \DescribeOption{sinfirmas}
%  No pone ninguna línea ni espacio para firmar; esto es lo predeterminado para las prefacturas.
%
% \DescribeOption{dosfirmas}
% Coloca las dos líneas para que firme quien emite y quien recibe. Esta opción es la predeterminada salvo en las prefacturas y en las cotizaciones.
%
% \DescribeOption{vencimiento}
%  Cuando establecemos los días del crédito de la factura, se imprime una línea también con la fecha de vencimiento de la factura. Con esta opción podemos pedir que la fecha de vencimiento salga siempre en la facturas, inclusive cuando no sean facturas a crédito.
%
% \DescribeOption{denom}
%  Con esta opción escribimos también el texto “FACTURA” con el número. La razón de esta opción es que hay formas libres que tienen la numeración de la factura y otras que solamente tienen el número de control; para las que tienen sólo el número de control existe esta opción que también escribe el número en cada factura impresa, en la posición que se defina.
%
% \DescribeOption{sinnumero}
%  Escribe el texto de la denominación del documento en su posición (e.g. “COTIZACIÓN”) pero sin escribir el número; por ejemplo para las cotizaciones que no son documentos que legalmente lleven número, aunque el número puede ser para control interno. Esta opción implica la opción anterior |denom|.
%
% \DescribeOption{sintotales}
%  Con esta opción evitamos que se totalicen los elementos en la factura; sólo se presenta el cuadro sin los totales. La utilidad de esto no es en las facturas sino en las cotizaciones para casos donde se muestran variantes del mismo producto para que el cliente elija entre ellos. No tendría sentido totalizar allí sino que funcione como un listado de precios. {\itshape Debe usarse en las cotizaciones porque una factura sin totales no es válida.}
%
% \DescribeOption{linea}
%  Agrega una línea diagonal que inutiliza el espacio vacío en el cuadro de la factura, para evitar posibles futuras adiciones de texto. Esta opción requiere de la opción |expandecuadro|, explicada a continuación.
%
%
% \subsection{Disposición de algunos elementos en la factura}
%
% \DescribeOption{expandecuadro}
%  Expande el cuadro de la factura con espacio en blanco, es decir, rellena el espacio restante del cuadro de la factura con espacio en blanco. Lo predeterminado es expandir la parte de la descripción larga cuando la hay.
%
% \DescribeOption{sinexpandir}
%  No expande ninguna parte en la factura, es decir, deja tanto el cuadro de la factura como la descripción en su tamaño mínimo según el contenido que tenga.
%
% \DescribeOption{centradesc}
%  Podemos centrar el texto de la descripción de cada ítem en la factura con este comando en lugar de justificarlo.
%
% \DescribeOption{filas}
%  Muesta los totales finales en modo de filas; esto los presenta uno al lado del otro en lugar de uno debajo del otro. Esta opción tiene limitaciones, por ejemplo no es compatible con la opción |sintotales|, |sinivaexpresado|, y por razones de espacio tampoco es compatible con los comandos de descuentos en los totales (subsección \ref{subsec:dctototales}).
%
% \DescribeOption{centrado}
%  Centra los totales finales ({\itshape sólo para el modo de totalización en filas}), que sin esta opción salen alineados al margen derecho.
%
% \DescribeOption{agrupado}
%  No agrupa los totales finales de los items por gravamen sino por tipo ({\itshape sólo para el modo de totalización en columnas}). Con esta opción la agrupación es primero todos los subtotales y los descuentos (cuando aplica), luego las bases imponibles y después los IVA. Sin |agrupado| se ordena por porcentaje del gravamen.
%
%
% \subsection{Otras opciones}
%
% \DescribeOption{nospanish}
%  No carga |babel| en castellano\footnote{Por ejemplo para cargar {\ttfamily polyglossia} en su lugar.}, ni la codificación |T1| de las letras en el caso de compilar con pdf\LaTeX. Si no se especifica esta opción se cargan ambos paquetes.
%
% \DescribeOption{twoside}
%  Establece que el reporte se imprime por los dos lados del papel. (Obviamente se ignora esta opción cuando el motor no va a generar un reporte puesto que las facturas ocupan una hoja solamente).
%
% \DescribeOption{csv}
%  Escribe un archivo csv con la información numérica de la factura.
%
% \subsection{Activación y desactivación de opciones desde comandos\label{subsec:condicionales}}
%
% Cualquiera de estas opciones también se puede activar o desactivar en el preámbulo y así poder incluirlas dentro de archivos que vayan a ser cargados con |\input| o |\include| como se explicará en la subsección \ref{subsec:aux}. Esto significa que podremos desactivar una opción que hayamos dado en |\documentclass| o activar alguna que no hayamos activado allí.
%
% Para hacer eso debemos usar el condicional que cada opción maneja y activarlo o desactivarlo de la forma como se hace en \TeX{}. El condicional de cada opción tiene el mismo nombre de la opción. Por ejemplo, para quitar las líneas verticales (|sinlineasv|) usaríamos |\sinlineasvtrue| en el preámbulo, o para desactivar la creación del archivo |.csv|, previamente activada en |\documentclass|, usaríamos |\csvfalse|.
%
% }
%
%
% \section{Comandos que la clase provee\label{sec:comandos}}
%
% A continuación descriremos los principales comandos de esta clase; en rojo aparecen aquellos pocos comandos que son de obligatorio uso. Los que no son de obligatorio uso es porque tienen algún valor predeterminado en |factura.def| o porque controlan algún aspecto opcional.
%
%
% \subsection{Estableciendo los márgenes de la hoja y el fondo}
%
%  Como explicamos en la sección \ref{sec:geohoja}, a la hoja de la factura se le deben establecer los márgenes apropiadamente, y esos márgenes no tienen en cuanta ciertos textos que llamamos “fijos”, para los cuales también debemos establecer sus posiciones en la hoja, y eso lo veremos en esta parte.
%
%
% {\setlength{\parindent}{0pt}
%
% \DescribeMacro{\GeometriaFactura}{\macrooptcolor|\GeometriaFactura|\marg{geometría}}
%
%  En el argumento de este comando van los parámetros relativos al paquete |geometry| que van a afectar la manera como la factura es generada. Como dijimos al principio, deberá especificarse el tipo de papel (e.g. |letterpaper|, |legalpaper|, |executive|, |a4paper|, ...) y los márgenes verticales y horizontales.
%
%  Para especificar el margen superior debe medirse desde el inicio del papel hasta donde se quiera establecer que empiece el recuadro de los datos del cliente, y el margen inferior debe medirse desde el final del papel hasta donde terminaría la opcional descripción. Como ya dijimos, hay textos que están fuera de estos márgenes por haberlos considerado fijos en la implementación de la clase |factura|.
%
% \DescribeMacro{\GeometriaReverso}{\macrooptcolor|\GeometriaReverso|\marg{geometría}}
%
%  Define los márgenes del reverso de la hoja de la factura para escribir información extra.
%
% \DescribeMacro{\FondoFactura}{\macrooptcolor|\FondoFactura|\marg{nombre del archivo de la imagen}}
%
%  Nombre del archivo de la imagen a cargar como diseño, especialmente para las {\slshape prefacturas}, aunque también puede usarse como una “marca de agua” para las facturas.
%
%  Lo que recomendamos es digitalizar la forma libre o pedir a la imprenta un archivo digital con el diseño y así incluir esa imagen con este comando cuando se vayan a producir “prefacturas” o cotizaciones, así las prefacturas tienen el mismo diseño que las facturas. Si se puede lograr el diseño en un código como se hace con el ejemplo del fondo, pues mejor aún.
%
%
% }
%
%
% \subsubsection{Elementos fijos y sus ubicaciones}
%
% Los elementos cuyas posiciones van fijas en la hoja disponen de macros para establecer sus ubicaciones. Todas estas macros empiezan por |\Pos|, y son las siguientes: |\PosFecha|, |\PosDenom|, |\PosControl|, |\PosNotaFinal| y |\PosFirmas|.
%
% Todos estos comandos aceptan como argumentos dimensiones de \LaTeX{}, y son las posiciones |(x,y)| en la página de los elementos a imprimir, siendo “|x|” la coordenada a lo ancho del papel y “|y|” la coordenada a lo largo del papel, es decir, horizontal y vertical respectivamente. El origen está fijado arriba a la izquierda. Se pueden especificar también valores negativos para denotar que estamos contando desde el extremo derecho o inferior, para el caso de “|x|” o “|y|”, respectivamente\footnote{Varias de estas variables están definidas de esta forma negativa en el archivo de definiciones.}; salvo el caso de la nota final y las firmas en su coordenada |x|, todos los demás elementos fijos aceptan valores negativos en su posicionamiento, ya que estos dos elementos cuentan con un intervalo |x-inicial,x-final| para posicionar.
%
% Por ejemplo, si queremos que la fecha aparezca desde el margen izquierdo y |5cm| contados desde arriba hacia abajo, haríamos: |\PosFecha{0cm}{5cm},| y si queremos que salga |1cm| desde la derecha y |4cm| contados desde abajo haríamos |\PosFecha{-1cm}{-4cm}|. A continuación todas las macros para fijar las posiciones en el papel:
%
% {\setlength{\parindent}{0pt}
%
% \DescribeMacro{\PosFecha}{\macrooptcolor|\PosFecha|\marg{pos. x}\marg{pos. y}}
%
%  Con |\PosFecha| controlamos dónde se ubica la caja que imprime la fecha en la hoja.
%
% \DescribeMacro{\PosFecha*}{\macrooptcolor|\PosFecha*|\marg{pos. x}\marg{pos. y}}
%
%  Con esta variante con “*”, es decir, |\PosFecha*|, logramos que la caja de la fecha abra hacia arriba desde la posición especificada en el eje vertical, de manera que estaríamos especificando una coordenada “|y|” final del texto y no una posición inicial, esto significa que el texto se despliega hacia arriba desde la posición dada y no hacia abajo. Por ejemplo, sin “*”, si especificamos |\PosFecha{-0cm}{5cm}|, las fechas empiezan 5cm desde el tope superior de la hoja, y si especificamos |\PosFecha*{-0cm}{5cm}| ellas terminarán en los 5cm de la hoja. Esto es útil cuando se especifica también |\Credito| que produce una fecha de vencimiento que aparecerá abajo de la fecha de emisión; al ocupar las fechas más de una línea podemos querer que la última fecha (la de vencimiento) quede alineada en cierta posición vertical, la misma posición donde saldría la fecha de emisión en el caso en el que no haya fecha de vencimiento por no especificar la factura a crédito.
%
% \DescribeMacro{\PosDenom}{\macrooptcolor|\PosDenom|\marg{pos. x}\marg{pos. y}}
%
%  Con |\PosDenom| controlamos dónde se ubica la caja que imprime la denominación del documento (e.g. “FACTURA”, “COTIZACIÓN”) y su -opcional- número.
%
% \DescribeMacro{\PosDenom*}{\macrooptcolor|\PosDenom*|\marg{pos. x}\marg{pos. y}}
%
%  Al igual que con |\PosFecha*| podemos especificar la posición final en “|y|” de la caja para la denominación, así el texto se despliega hacia arriba y no hacia abajo.
%
% \DescribeMacro{\PosControl}{\macrooptcolor|\PosControl|\marg{pos. x}\marg{pos. y}}
%
%  Con |\PosControl| controlamos dónde se ubica la caja que imprime el número de control con la serie.
%
% \DescribeMacro{\PosControl*}{\macrooptcolor|\PosControl*|\marg{pos. x}\marg{pos. y}}
%
%  También podemos pedir que la caja del número de control abra hacia arriba, es decir que la posición vertical sea tomada como la posición vertical final; para eso |\PosControl*|, igual que con |\PosFecha*| y |\PosDenom*|. 
%
% \DescribeMacro{\PosNotaFinal}{\macrooptcolor|\PosNotaFinal|\marg{pos. x-inicial}\marg{pos x-final}\marg{pos. y}}
%
%  Con |\PosNotaFinal| controlamos dónde se ubica el texto de la nota final\footnote{La nota final es lo que en versiones previas llamábamos “condiciones” de la factura} (está hecho con |\parbox|, acepta un párrafo como argumento) en la hoja.
%
% {\itshape Aquí delimitamos el intervalo en |X| donde va la nota final, por eso hay una coordenada |X-inicial| y una coordenada |X-final|.} La coordenada |X| inicial va medida desde el margen establecido como el izquierdo, y la |X| final va medida desde el margen establecido como el derecho. Por ejemplo, estableciendo |X-inicial| como |0pt| y |X-final| como |0pt| queda en los mismos márgenes que todos los demás elementos de la factura, y esto es lo predeterminado, pero estableciendo ambas coordenadas como |1cm|, la nota final quedará con |1cm| de margen hacia adentro que la factura, tanto por la izquierda como por la derecha.
%
% Debido a que este comando acepta valor inicial y valor final en |X|, entonces no acepta valores negativos.
%
% \DescribeMacro{\PosNotaFinal*}{\macrooptcolor|\PosNotaFinal*|\marg{pos. x-inicial}\marg{pos x-final}\marg{pos. y}}
%
%  De la misma manera como con los comandos precedentes, podemos especificar que la caja del párrafo para la nota final abra hacia arriba, fijando una posición vertical final y no una posición inicial de la nota final.
%
% \DescribeMacro{\PosFirmas}{\macrooptcolor|\PosFirmas|\marg{pos. x-inicial}\marg{pos. x-final}\marg{pos. y}}
%
%  Con |\PosFirmas| controlamos dónde se ubica la caja que dibuja las líneas para firmar en la factura. {\itshape Aquí aplica el mismo comentario que en \cs{PosNotaFinal} sobre las coordenadas |X| inicial y final}. La coordenada |X-inicial| indica dónde empieza la primera linea (para que firme quien recibe), y la coordenada |X-final| establece dónde termina la segunda linea para firmar (para que firme el emisor de la factura), que va a estar al lado de la primera.
%
% En este caso, las coordenadas |X| tampoco aceptan valores negativos ni versión con asterisco |*|.
%
% }
%
%
% \subsubsection{Espaciado entre distintas partes. Estilo de los recuadros\label{subsec:espacioslineas}}
%
% Podemos cambiar distintos valores del espaciado interno en la factura, así como también la estética de la línea para firmar y los recuadros que encierran los datos del cliente, la descripción, las “notas”, pudiendo ajustarle estos recuadros o cajas el grosor de la línea, la curvatura y la sombra. Los argumentos de todos estos comandos son dimensiones de \LaTeX{}:
%
% {\setlength{\parindent}{0pt}
%
% \DescribeMacro{\SepVertical}{\macrooptcolor|\SepVertical|\marg{sep. vertical entre elementos}}
%
%  Las distintas partes de la factura (a saber: el encabezado con los datos, la -opcional- nota previa, el cuadro de la factura, la -opcional- nota externa y la -opcional- descripción) también están separadas entre sí una distancia que puede ser cambiada.
%
%  La separación que se fija con este comando es la separación vertical mínima que queremos garantizar entre cada elemento a escribir en la factura, y puede ser ajustada automáticamente por la clase aumentándola cuando no hay relleno automático de espacios sobrantes (es decir, sin usar la opción |sinexpandir|). Cuando la descripción larga o el cuadro de la factura (usando la opción |expandecuadro|) toman su tamaño máximo, entonces la separación vertical entre cada elemento de la hoja es la dada con |\SepVertical|, o sea, la que queremos garantizar como la mínima separación entre esas partes.
%
% \DescribeMacro{\SepDatosResumen}{\macrooptcolor|\SepDatosResumen|\marg{sep. horizontal}}
%
%  Esta macro nos proporciona una forma de cambiar la separación entre los recuadros o cajas donde van los datos del cliente y el resumen que sale a su lado. Naturalmente, si no se escribe un resumen saldría únicamente la caja de los datos del cliente, y esta separación no tendría efecto.
%
% \DescribeMacro{\SepEncabezado}
% \DescribeMacro{\SepNotaPrevia}
% \DescribeMacro{\SepDescripcion}
% \DescribeMacro{\SepNotaFinal}
% {\macrooptcolor|\SepEncabezado|\marg{sep. interna texto}}\newline
% {\macrooptcolor|\SepNotaPrevia|\marg{sep. interna texto}}\newline
% {\macrooptcolor|\SepDescripcion|\marg{sep. interna texto}}\newline
% {\macrooptcolor|\SepNotaFinal|\marg{sep. interna texto}}
%
%  Con estas macros podemos controlar la distancia entre la línea del recuadro y el texto que hay en el recuadro, es decir, un margen interno entre el contorno de la caja y el texto para que no salga pegado a la línea que lo encierra. Esto es parte del diseño de \tikz de estos recuadros. Como puede verse, dicho margen para los datos del cliente y para el resumen se cambia con |\SepEncabezado|, los de la nota previa con |\SepNotaPrevia|, los de la descripción larga con |\SepDescripcion|, y los de la nota final con |\SepNotaFinal|.
%
% Predeterminadamente no hay línea ni diseño de \tikz para la nota final del documento, así que el valor fijado aquí no tiene efecto a menos de que se le especifiquen valores a dicho recuadro con la opción |\LineaNotaFinal|, explicada más abajo.
%
% \DescribeMacro{\SepItemsExtra}{\macrooptcolor|\SepItemsExtra|\marg{sep. extra entre items}}
%
%  Agrega una separación extra entre los elementos o ítems en el cuadro de la factura. La separación entre ítems está dada por el |\baselineskip| que se puede establecer con el comando |\LetraItems|, el cual se explicará más adelante en esta misma sección.
%
% \DescribeMacro{\SepFilas}{\macrooptcolor|\SepFilas|\marg{sep. en x}\marg{sep. en y}}
%
%  Cuando se usa el modo de totalización en filas tenemos también la posibilidad de cambiar la separación entre los totales que salen uno al lado del otro, y la separación entre las distintas líneas.
%
%  El primer argumento de |\SepFilas| es la separación horizontal {\itshape mínima} entre los distintos totales que se presentan en el modo de totalización en filas. La clase |factura| expande esa distancia para que queden todos los totales separados igual, y también separados igual de los márgenes del papel.
%
%  El segundo argumento de |\SepFilas| es la separación vertical entre las distintas líneas de la totalización en filas.
%
% \DescribeMacro{\LineaEncabezado}
% \DescribeMacro{\LineaNotaPrevia}
% \DescribeMacro{\LineaDescripcion}
% \DescribeMacro{\LineaNotaFinal}
% {\macrooptcolor|\LineaEncabezado|\marg{grosor}\marg{curvatura esquinas}\marg{tamaño sombra}}\newline
% {\macrooptcolor|\LineaNotaPrevia|\marg{grosor}\marg{curvatura esquinas}\marg{tamaño sombra}}\newline
% {\macrooptcolor|\LineaDescripcion|\marg{grosor}\marg{curvatura esquinas}\marg{tamaño sombra}}\newline
% {\macrooptcolor|\LineaNotaFinal|\marg{grosor}\marg{curvatura esquinas}\marg{tamaño sombra}}
%
%  La información del encabezado (datos del cliente y resumen), así como la nota previa, la descripción y la nota final puede o no tener el diseño \tikz, el cual viene prehecho en el archivo |.def| y consiste en una línea curva, en algunos casos sombreada a la derecha y abajo, y en otros casos no. Esa línea puede ser alterada en su grosor, el tamaño de la sombra, y la distancia entre la línea y la información, es decir un margen interno. Para eso estos comandos, que se comportan igual todos ellos.
%
%  Con el primer argumento de estos comandos cambiamos el grosor de la línea\footnote{Si se especifica una dimensión nula, desaparece el diseño \tikz y la línea, quedando solamente un \cs{parbox}, y es así como se estableció en el archivo \texttt{.def} para el caso de la nota final.} con la que las cajas del encabezado, la nota previa, la descripción y la nota final se dibujan. Con el segundo argumento cambiamos el radio de curvatura de las esquinas, y con el tercero el tamaño de la sombra\footnote{Todas las sombras salen hacia abajo a la derecha.}.
%
%  Se puede modificar el diseño de este nodo de \tikz; está alojado en las macros internas |\@DatosTIKZ|, |\@NotaPreviaTIKZ|, |\@DescripcionTIKZ| y |\@NotaFinalTIKZ|.
%
%  Lo predeterminado es que la nota final se escriban sin diseño de \tikz, es decir, el primer argumento de este comando está fijado en cero para que no salga línea ni diseño de \tikz.
%
% \DescribeMacro{\LineaFirmas}{\macrooptcolor|\LineaFirmas|\marg{long. línea}\marg{grosor}}
%
%  La longitud de las líneas para firmar y el grosor son también ajustables por medio del primer y segundo argumento, respectivamente, de la macro |\LineaFirmas|.
%
%
% \subsection{Sobre la información del cliente\label{subsec:cliente}}
%
% \DescribeMacro{\RazonSocial}{\macromandcolor|\RazonSocial|\marg{texto}}
%
%  Define la razón social, o sea, el nombre del cliente.
%
% \DescribeMacro{\Nombre}{\macromandcolor|\Nombre|\marg{texto}}
%
%  Similar a |\RazonSocial| pero usable más bien para personas naturales, donde no existe una “persona de contacto” y por tanto la clase no advierte de su ausencia si no se ha especificado con |\Contacto|. Naturalmente se especifica un solo nombre del cliente, es decir que si se especifica con |\RazonSocial| no hace falta usar este comando. {\itshape Es preferible usar |\RazonSocial|.}
%
% \DescribeMacro{\RIF}{\macromandcolor|\RIF|\marg{texto}}
%
%  Define el R.I.F. del cliente.
%
% \DescribeMacro{\CI}{\macromandcolor|\CI|\marg{texto}}
%
%  Especifica la cédula de identidad del cliente al ser persona natural. {\itshape Es preferible usar siempre |\RIF|.}
%
% \DescribeMacro{\Contacto}{\macrooptcolor|\Contacto|\marg{texto}}
%
%  Define una persona de contacto (e.g. nombre de una persona cuando el cliente no es personal).
%
% \DescribeMacro{\Direccion}{\macromandcolor|\Direccion|\marg{texto}}
%
%  Define la dirección fiscal del cliente.
%
% \DescribeMacro{\Telefono}{\macrooptcolor|\Telefono|\marg{texto}}
%
%  Especifica el (o los) teléfonos del cliente.
%
% \DescribeMacro{\Email}{\macrooptcolor|\Email|\marg{texto}}
%
%  Establece el e-mail del cliente.
%
% \DescribeMacro{\Conforme}{\macrooptcolor|\Conforme|\marg{texto}}
%
%  Especifica el texto que aparecerá debajo de la línea para que firme el cliente; si no se especifica tomará el texto por omisión, que es “Recibí conforme”, indicando que el cliente le firma una copia al emisor de la factura donde acepta haber recibido y estar conforme. Este texto puede cambiarse a otro que puede ser el nombre del cliente o la persona de contacto, o también “Por el cliente”.
%
%
% \subsection{Sobre la información del emisor del documento}
%
% \DescribeMacro{\Emisor}{\macrooptcolor|\Emisor|\marg{texto}}
%
%  Define el nombre del emisor de la factura. Si no se especifica saldrá el texto “PAGADO”\footnote{El texto “PAGADO” es lo que la legislación venezolana del SENIAT exige escribir cuando una factura ha sido pagada.} debajo de la línea para firmar, ya que la factura se firma cuando haya sido pagada. Si se especifica un texto con este comando saldrá ese texto debajo de la línea donde va la firma, pero el propósito es que quede constancia para el cliente que ya pagó la factura, razón por la cual el texto predeterminado es “PAGADO”.
%
% \DescribeMacro{\Proveedor}{\macrooptcolor|\Proveedor|\marg{texto}}
%
%  Establece el nombre del emisor, para los efectos del reporte. Si no se define mediante esta macro, será tomado del establecido con |\Emisor|.
%
% \DescribeMacro{\FirmaFactura}{\macrooptcolor|\FirmaFactura|\marg{inclusión de la imagen}\oarg{corrimiento en x}\oarg{corrimiento en y}}
%
%  Incluye el archivo de la firma digitalizada en el espacio donde firma el emisor del documento. Los argumentos son:
%  \begin{enumerate}
%   \item inclusión de la imagen (e.g. |\includegraphics|). {\bfseries Aquí el primer argumento no es el archivo de la imagen sino el comando que la inserta en el documento}, así el usuario tiene más control sobre cómo saldrá la imagen si altera las opciones de |\includegraphics|.
%   \item corrimiento en |X| de la imagen, para un ajuste más fino de la posición de la firma,
%   \item corrimiento en |Y|.
%   \end{enumerate}
%  {\itshape (Sin especificar los corrimientos, la firma quedará centrada horizontalmente en la línea, dejando hacia la izquierda y hacia la derecha la misma cantidad de la línea. También quedará toda la firma encima de la línea, lo cual muy probablemente haya que corregirlo porque las firmas tienen “profundidad” de texto, que significa que hay parte de algunas letras que sobresalen hacia abajo respecto a la línea base del texto. Por todo esto es muy probable el uso de los corrimientos. Pueden verse los ejemplos, que allí se usan los corrimientos.)}
%
%  {\bfseries Las facturas no llevan firma porque se supone que el documento es auténtico ya que viene avalado por una imprenta registrada y tiene los números correlativos impresos desde la imprenta, de manera que la factura no necesita la firma de quien la emite; razón por la cual el texto debajo de la línea del emisor en la factura dice “PAGADO”, porque el emisor firma la factura una vez que haya sido pagada como constancia para el cliente.} Entonces agregar la firma escaneada es una posibilidad que debe contemplarse solamente con las cotizaciones, usando también el comando |\Emisor|, ya que no nos interesaría que apareciera “PAGADO” en ese caso sino algún nombre de quien está emitiendo la cotización.
%
%
% \subsection{Relativos a la moneda y a la numeración del documento factura}
%
% \DescribeMacro{\Moneda}{\macrooptcolor|\Moneda|\marg{signo monetario}}
%
%  Cambia el signo monetario, por defecto “Bs.”.
%
% \DescribeMacro{\Num}{\macrooptcolor|\Num|\marg{núm}}
%
%  Define el número de la factura. Es opcional porque el talonario de las facturas puede ya tener la numeración. Es de obligatorio uso cuando se define |denom|, o |prefactura| sin usar |sinnumero|.
%
% \DescribeMacro{\NumControl}{\macrooptcolor|\NumControl|\marg{núm}}
%
%  Define el número de control de la factura. Esta opción sólo aplica con |prefactura| puesto que una factura fiscal siempre tiene el número de control impreso desde la imprenta. Si se usa |prefactura| y no se define el número de control, éste tomará el mismo número de la factura.
%
% \DescribeMacro{\NumSerieControl}{\macrooptcolor|\NumSerieControl|\marg{núm}}
%
%  Define el número de serie del número de control, también aplica solamente con |prefactura|. Si no se especifica, toma |00|.
%
% \DescribeMacro{\Denom}{\macrooptcolor|\Denom|\marg{texto}}
%
%  Cambia el nombre del documento a algo distinto de “FACTURA” o “COTIZACIÓN”.
%
%
% \subsection{Estableciendo la fecha del documento}
%
% \DescribeMacro{\Fecha}{\macrooptcolor|\Fecha|\marg{fecha}}
%
%  Especifica la fecha de emisión del documento, en un formato que puede cambiarse con el comando |\FormatoFecha| pero predeterminadamente es el formato venezolano |dd-mm-aaaa|.
%
%  La fecha de emisión se guardará en el identificador “\texttt{FechaEmision}” perteneciente al paquete \texttt{datetime2} y podrá ser usado en cualquier parte del documento según las formas que provee dicho paquete.
%
%  {\itshape Si el usuario no provee la fecha, será tomada la fecha actual como la fecha de emisión y se producirá una advertencia indicando que no se especificó la fecha y que la fecha del documento es la fecha del día.}
% 
% \DescribeMacro{\Fecha*}{\macrooptcolor|\Fecha*|\marg{aaaa}\marg{mm}\marg{dd}}
%
%  Esta variante del comando se incorporó para suministrar la fecha en formato ISO, es decir, primero el año, luego y mes y finalmente el día, cada uno como un parámetro separado del otro (como se hace en \texttt{datetime2}. Esto es independiente de la forma como la fecha será mostrada, o sea que puede la fecha suministrarse de esta forma e igual saldrá en el formato deseado, cambiable con la macro |\FormatoFecha|, explicado a continuación.
%
% \DescribeMacro{\FormatoFecha}{\macrooptcolor|\FormatoFecha|\oarg{formato entrada}\marg{formato salida}\marg{separador}}
%
%  El comando |\FormatoFecha| le permite al usuario cambiar el formato de entrada de la fecha suministrada y el formato de salida de las fechas (cómo éstas salen escritas en la factura).
%
%  Con el primer argumento (opcional, dado entre corchetes) el usuario le puede decir a la clase cúal es el formato según el cual le suministrará la fecha con |\Fecha|; las posibilidades son: \begin{inparaenum}[\bfseries i)]\item |dd-mm-aaaa| (predeterminada), \item |dd.mm.aaaa|, \item |ddmmaaaa|, \item |aaaa-mm-dd|, \item |aaaa.mm.dd|, \item |aaaammdd| y \item |mm-dd-aaaa| \end{inparaenum}. Esto con el fin de establecer la fecha reconocida debidamente por |datetime2| y que sea posible su manejo como fecha y el posible cálculo de la fecha de vencimiento, si aplicara.
%
%  El segundo argumento (opcional también, dado entre llaves) le permite al usuario especificar el formato según el cual quiere que las fechas (emisión y vencimiento si aplica) salgan impresas en la factura. Estos son los formatos dados por \texttt{datetime2} y debe consultarse su documentación para más información. Entre ellos tenemos por ejemplo \begin{inparaenum}[\bfseries i)]\item |ddmmyyyy| (predeterminado), \item |default|, \item |iso|, \item |mmddyyyy| \end{inparaenum}.
%
%  El tercer argumento (opcional también, dado entre llaves) sirve para cambiar el separador, cuyo carácter predeterminado es el guión (“-”).
%
%  Con |\FormatoFecha|, al tener todos sus argumentos opcionales, el usuario podrá especificar solamente el primero (que va entre corchetes), solamente el segundo (entre llaves), pero para especificar el tercero que también está entre llaves debe especificar el segundo, ya que ambos son opcionales pero se dan igualmente entre llaves. Naturalmente podrá especificar dos de estos argumentos o los tres.
%
%  Por ejemplo: |\FormatoFecha[mm-dd-aaaa]{mmddyyyy}{.}| serviría para que el usuario pueda dar la fecha así: |\Fecha{12-31-2019}| y que salga igualmente según el formato |mm-dd-aaaa| pero separado por puntos. También puede usarse |\FormatoFecha[aaaa-mm-dd]| para dar luego la fecha de esta forma: |\Fecha{2019-12-31}| pero esta saldrá igual impresa en la factura así: “|31-12-2019|”, ya que no cambiamos el formato de salida. También puede hacer |\FormatoFecha{mmddyyyy}{/}| y dar la fecha por ejemplo así: |\Fecha*{2019}{12}{31}| y saldrá impresa así: “|12/31/2019|”.
%
%  Los valores predeterminados cambiables con |\FormatoFecha| están definidos, como siempre, en el archivo |factura.def|.
%
% \DescribeMacro{\Credito}{\macrooptcolor|\Credito|\marg{días}}
%
%  Este comando, cuyo argumento es la cantidad de días, nos permite emitir una factura a crédito. La fecha de vencimiento será calculada automáticamente y será colocada debajo de la fecha de emisión, especificada por el usuario.
%
%  La fecha calculada automáticamente se guarda en el identificador “\texttt{FechaVencimiento}” perteneciente al paquete \texttt{datetime2}, de manera que el usuario podrá hacer uso de esta fecha con los comandos que este paquete provee.
%
% }
%
% 
% \subsection{Descuentos en la totalización\label{subsec:dctototales}}
%
%  Podemos efectuar un descuento\footnote{Estas opciones de descuento en el subtotal no son compatibles con la opción de totalización en modo de |filas|, explicada en la sección \ref{sec:opciones}, y muy probablemente más adelante seguirán sin serlo.} sobre cualquiera de los subtotales; ese descuento puede ser porcentual (versión sin *) o dado en un monto fijo a sustraer del subtotal (variante con *). Podemos aplicar un descuento que aplique por igual a cualquier tipo de ítem con cualquier gravamen, o discriminado dependiendo del tipo de impuesto se aplica un descuento u otro.
%
% {\setlength{\parindent}{0pt}
%
% \DescribeMacro{\Descuento}{\macrooptcolor|\Descuento|\marg{porcentaje}}\\
% \DescribeMacro{\Descuento*}{\macrooptcolor|\Descuento*|\marg{monto}}
%
% Con este comando el descuento aplicaría por igual a cualquier subtotal presente, es decir, para elementos con cualquier gravamen y también exentos. Dichos descuentos son aplicados y mostrados al totalizar. Puede usarse inclusive cuando se aplican descuentos individuales que se incluyen con la opción de la clase |descuentos|.
%
% \DescribeMacro{\DescuentoG}{\macrooptcolor|\DescuentoG|\marg{porcentaje}}\\
% \DescribeMacro{\DescuentoG*}{\macrooptcolor|\DescuentoG*|\marg{monto}}
%
%  Descuento específico para los elementos gravados con IVA general.
%
% \DescribeMacro{\DescuentoR}{\macrooptcolor|\DescuentoR|\marg{porcentaje}}\\
% \DescribeMacro{\DescuentoR*}{\macrooptcolor|\DescuentoR*|\marg{monto}}
%
%  Descuento específico para los elementos gravados con IVA reducido.
%
% \DescribeMacro{\DescuentoA}{\macrooptcolor|\DescuentoA|\marg{porcentaje}}\\
% \DescribeMacro{\DescuentoA*}{\macrooptcolor|\DescuentoA*|\marg{monto}}
%
%  Descuento específico para los elementos gravados con IVA adicional.
%
% \DescribeMacro{\DescuentoE}{\macrooptcolor|\DescuentoE|\marg{porcentaje}}\\
% \DescribeMacro{\DescuentoE*}{\macrooptcolor|\DescuentoE*|\marg{monto}}
%
%  Descuento específico para los elementos exentos de IVA.
%
%
% \subsection{Textos explicativos en la factura\label{subsec:textos}}
%
% \DescribeMacro{\Resumen}{\macrooptcolor|\Resumen|\marg{texto}}
%
%  Pone un resumen que va en una caja a la derecha de los datos del cliente, separado de esta en una distancia ajustable con el comando |\SepEncabezado|, que se describirá más adelante. Este resumen es opcional.
%
% \DescribeMacro{\NotaPrevia}{\macrooptcolor|\NotaPrevia|\marg{texto}}
%
%  La nota previa es una nota que puede ponerse antes del cuadro de la factura. Sin especificarla no sale nota alguna. Dicha nota va en un recuadro \tikz cuyo aspecto puede cambiarse, como ya se mencionó en la sección \ref{subsec:espacioslineas} donde se explican las líneas o contornos.
%
% \DescribeMacro{\NotaInterna}{\macrooptcolor|\NotaInterna|\marg{texto}}
%
%  La nota interna es una coletilla que aparece dentro del cuadro de la factura, por eso la llamamos interna (es interna al cuadro), por ejemplo si aplicamos un descuento podemos decir allí cerca de los totales a qué corresponde tal descuento, o para hacer alguna aclaratoria sobre lo que precede inmediatamente como algún ítem de la factura.
%
% \DescribeMacro{\NotaExterna}{\macrooptcolor|\NotaExterna|\marg{texto}}
%
%  La nota externa es otro texto opcional que el usuario puede introducir y saldrá al terminar el cuadro de la factura, antes de la opcional descripción larga (comando |\Descripcion|). {\itshape La nota externa no tiene formato \tikz.}
%
% \DescribeMacro{\NotaFinal}{\macrooptcolor|\NotaFinal|\marg{texto}}
%
%  Con este comando se establece el texto para la nota final\footnote{Antiguas condiciones de la factura, por ejemplo, para las condiciones de pago si se trata de una factura, o para las condiciones de una cotización.}. Es opcional; si no se define no sale texto alguno. También se puede destinar este espacio para un diseño (que haría el usuario) en el que se agreguen también algunas casillas para especificar la forma de pago.
%
%
% \subsection{Estableciendo los parámetros relativos al reporte}
%
% \DescribeMacro{\GeometriaReporte}{\macrooptcolor|\GeometriaReporte|\marg{geometría}}
%
%  Aquí van los parámetros relativos al paquete |geometry| que van a afectar la manera como el reporte es generado, de la misma manera como se definieron los márgenes para la factura.
%
%  Puesto que el reporte no se imprime en hojas de facturas sino en hojas en blanco, el usuario podrá cambiar la geometría del papel completamente a su gusto sin las restricciones impuestas por el diseño que tengan sus facturas. En |factura.def| se definen unos márgenes razonables para un reporte. Si el reporte y la factura usaran un tamaño distinto de papel, entonces al imprimirse ambos juntos (opción |conreporte|), el reporte tomará el tamaño físico del papel especificado para la factura ya que con el paquete |geometry| no podemos cambiar el tamaño del papel en el medio del documento, aunque sí sus márgenes. Entonces, si se quiere imprimir el reporte en un papel de otro tamaño, tendrán que imprimirse ambos documentos por separado, o sea, primero uno y después el otro.
%
% \DescribeMacro{\FondoReporte}{\macrooptcolor|\FondoReporte|\marg{nombre del archivo de la imagen}}
%
%  Imagen a cargar {\bfseries solamente en la primera página del reporte} (e.g. un membrete). Cuando se usa este comando puede necesitarse usar también |\InicioReporte| para fijar el inicio de la primera página del reporte más abajo que las demás páginas cuyos márgenes están establecidos con |\GeometriaReporte|; esto porque la primera página tendría un “margen superior” distinto por la existencia del membrete.
%
% \DescribeMacro{\FondoReporte*}{\macrooptcolor|\FondoReporte*|\marg{nombre del archivo de la imagen}}
%
%  Esta variante de |\FondoReporte| es para que la imagen esté visible {\bfseries en todas las páginas} y no solamente en la primera como con el comando anterior. Usando esta variante del fondo del reporte no es necesario usar |\InicioReporte| sino solamente establecer los márgenes del reporte de acuerdo a dicho fondo.
%
% \DescribeMacro{\EstiloPagReporte}{\macrooptcolor|\EstiloPagReporte|\oarg{estilo primera página}\marg{estilo páginas restantes}}
%
%  Con este comando definimos el estilo de página para el reporte (véase el comando |\InfoPagina| y la subsección \ref{subsec:pagestyle}). El estilo de página es lo que cambiamos en \LaTeX{} con |\pagestyle|, o sea, lo relativo a las cabeceras y pies de página.
%
%  Con este comando podemos asignar un estilo de página para la primera página del reporte y otro estilo de página para el resto de las páginas. La primera página del reporte pudiera tener un estilo de página distinto por la presencia del membrete si se usa el comando |\FondoReporte| sin “|*|”, por lo cual se puede querer no poner cabeceras en la primera página pero sí en las demás.
%
%  Como vemos de la sintaxis de esta macro, el argumento obligatorio es el del resto de las páginas y el opcional es el que aplica a la primera página. El opcional se pone entre corchetes antes del otro, pero si se omite, todas las páginas toman el estilo especificado entre llaves.
%
%  Si no se usa esta macro la clase asume un comportamiento apropiado teniendo en cuenta si se especifica |\FondoReporte| o |\FondoReporte*| para evitar solapamiento de textos, como se explica también en la subsección \ref{subsec:pagestyle}.
%
%  La información que aparece como cabecera o como pie de página es la que damos con el comando |\InfoPagina|, explicado más adelante.
%
%  La clase provee varios estilos de página, que también se explicarán en la subsección \ref{subsec:pagestyle}. En los ejemplos de reportes se puede ver el comportamiento del comando |\EstiloPagReporte|.
%
% \DescribeMacro{\InicioReporte}{\macrooptcolor|\InicioReporte|\marg{longitud}}
%
%  Con este comando -cuyo argumento es una longitud de \LaTeX{}- se agrega una distancia extra donde inicia el texto en la primera página del reporte cuando se usa |\FondoReporte| sin “|*|”. Este comando sólo aplica cuando en el reporte se fija un fondo para que salga solamente en la primera página (comando |\FondoReporte| en su versión sin “|*|”). La motivación de esto es corregir la alteración que produce un membrete que va a salir solamente en la primera página; si el membrete estuviera en todas las páginas no es necesario hacer esto sino fijar los márgenes adecuadamente con |\GeometriaReporte|. Si en el reporte no se fija un fondo o se fija el fondo para todas las páginas (comando |\FondoReporte*|), lo que se establezca con |\InicioReporte| es ignorado. En estos dos casos los márgenes deben establecerse con |\GeometriaReporte|.
%
% \DescribeMacro{\FirmaReporte}{\macrooptcolor|\FirmaReporte|\oarg{inclusión la imagen}\oarg{corr. en x}\oarg{corr. en y}\marg{texto}\oarg{pos. x}\oarg{long. línea}}
%
%  Incluye la firma para el final del reporte. Los argumentos son:
%  \begin{enumerate}
%   \item inclusión de la imagen (e.g. |\includegraphics|),
%   \item corrimiento en |X| de la imagen, para un ajuste más fino de la posición de la firma,
%   \item corrimiento en |Y|,
%   \item texto debajo de la línea ({\itshape va entre llaves y no entre corchetes como los demás}). El predeterminado es el dado mediante el comando |\Proveedor|, o en su defecto, |\Emisor|),
%   \item posición en |X| del final de la línea, medida desde la derecha (predeterminado: 2cm),
%   \item tamaño extra de la línea de la firma. La clase mide el tamaño del texto y de la imagen (si se especificó), y al que sea mayor le agrega este extra de línea para que sobresalga a los dos lados. (Predeterminado: tamaño de la imagen o del nombre + 1.5cm extra).
%  \end{enumerate}
%  {\itshape {\bfseries Este comando es de los pocos que no es del preámbulo}; debe incluirse en el sitio en el reporte donde se quiera que la firma salga, generalmente al final del reporte y todos sus argumentos son opcionales, incluyendo el texto que va debajo de la línea, que se incluye entre llaves |{}| a pesar de ser opcional. Cuando no se especifica ningún argumento, sólo dibuja la línea para firmar y le escribe abajo el nombre dado con |\Proveedor| o en su defecto el dado con |\Emisor|.}
%
% \DescribeMacro{\TituloReporte}{\macrooptcolor|\TituloReporte|\marg{texto}}
%
%  Especificamos un título para el reporte. Este título saldrá centrado al principio del reporte.
%
% \DescribeMacro{\TituloReporte*}{\macrooptcolor|\TituloReporte*|\marg{texto}}
%
%  Usando |\TituloReporte| u omitiendo su uso, la información que sale en las cabeceras o pies de página tiene que ver con el número de la factura que está asociada a ese reporte. Con esta variante con “|*|” de |\TituloReporte|, el título del reporte -especificado con este comando- saldrá escrito también en estas partes de la página en lugar del texto que indica que el reporte corresponde a determinada factura.
%
% Puede usarse esta opción también en aquellos casos en los que se hace un reporte sin asociarlo a una factura, entonces el título del reporte es la información que sale en la cabecera o pie de cada página.
%
% \DescribeMacro{\InfoPagina}{\macrooptcolor|\InfoPagina|\marg{texto}}
%
%  Establece el texto que aparece en la cabecera o pie de página del reporte.
%
% {\itshape Si se quiere usar esta opción debe hacerse luego de definir el título del reporte porque si no el comando |\TituloReporte*| establece de nuevo la variable que |\InfoPagina| cambia.}
%
%  En |factura.def| hay dos definiciones relativas a la información que aparece en las cabeceras o pies de página: aquella que escribe el título y aquella que no, es decir, la que aplica con |\TituloReporte*| y la que aplica con |\TituloReporte|, que también el usuario podrá redefinir si lo necesita, pero lo más práctico sería usar esta macro para fijar el texto que se desee, y |\EstiloPagReporte| para fijarlo donde se quiera.
%
%
% \subsection{Definiciones en archivos separados \label{subsec:aux}}
%
% \DescribeMacro{\Cliente}{\macrooptcolor|\Cliente|\marg{archivo \LaTeX}}
%
%  {\bfseries Todos los comandos definibles en el preámbulo que alteren la forma como el reporte o la factura es presentada, así como cualquier variable definida en |factura.def| se puede redefinir y agregar a otro archivo |.tex| o |.def| que el usuario cargue en el preámbulo con |\input|, |\include| o con el comando |\Cliente|; el comando para el cliente existe para que el usuario pueda definir archivos |.tex| o |.def| con los valores específicos que aplican para cada cliente.}
%
%  La macro |\Cliente| incluye un archivo cuya extensión debe ser |.def| o |.tex|, de la misma forma como se hace con |\input|; es simplemente un atajo que además lee archivos |.def|.
%
% }
%
%
% \section{La factura\label{sec:factura}}
%
% \DescribeEnv{factura}
% La parte del documento donde se agregan los elementos a facturar se escribe dentro de un entorno especial para su tratamiento, llamado |factura|, que produce el cuadro de la factura:
%
% \iffalse
%<*literal>
% \fi
{\VERBATIM
\begin{verbatim}
    \begin{factura}
     [...]
    \end{factura}
\end{verbatim}
}
% \iffalse
%</literal>
% \fi
%
% Dentro de dicho ambiente podemos usar únicamente los comandos que incluyen los elementos a facturar y las posibles etiquetas; esos comandos son, para distintos gravámenes, los siguientes:
%
%
% \subsection{Agregando elementos a la factura}
%
% \noindent
% {\macromandcolor|\Item|\marg{cantidad}\marg{descripción}\oarg{descripción para csv}\marg{precio unitario}\oarg{descripción sólo para \\ \mbox{}\hfill referencias}}
% \DescribeMacro{\Item}
% \DescribeMacro{\ItemR}
% \DescribeMacro{\ItemA}
% \DescribeMacro{\ItemE}
%
% Con estos comandos agregamos elementos a la factura propiamente dicha. Con |\Item| agregamos un elemento gravado con IVA general (actualmente 12\%); con |\ItemR| agregamos uno con IVA reducido (8\%), con |\ItemA| uno con IVA adicional (27\%), y con |\ItemE| agregamos un elemento exento de impuesto al valor agregado.
%
% Como podemos ver, se especifica la cantidad, luego la descripción y al final el precio unitario, todos entre llaves, pudiendo intercalar entre estos dos últimos un parámetro adicional entre corchetes que sirve para escribir otro texto al archivo |.csv| en sustitución de la descripción, ya que a veces la descripción puede tener macros o caracteres no imprimibles. Si se declara una descripción y otra descripción para el |.csv|, la factura impresa manejará la descripción dada primero (entre llaves) y en el archivo |.csv| se escribirá la descripción dada después (entre corchetes).
%
% El último parámetro opcional (entre corchetes) es otro campo de descripción pero que no se imprime en el cuadro de la factura sino que se guarda bajo una etiqueta para luego poder usarse en cualquier parte del documento con el comando |\ldescref| (véase la sección \ref{sec:etiquetas} de las etiquetas). Dicha descripción puede tener formato de párrafos largos con saltos de línea.
%
% También tenemos la posibilidad de agregar un ítem al que no le ponemos precio, por ejemplo por ser parte de un total, y sus argumentos son la cantidad y la descripción nada más.
%
% \noindent
% \DescribeMacro{\ItemX}{\macrooptcolor|\ItemX|\marg{cantidad}\marg{descripción}\oarg{descripción para csv}\oarg{descripción sólo para referencias}}
%
%
% \subsubsection{Opción `{\ttfamily descuentos}'\label{subsec:descuentos}}
%
% \noindent
% {\macromandcolor|\Item|\marg{cantidad}\marg{descripción}\oarg{descripción para csv}\marg{precio unitario}\marg{\% descuento} \\ \mbox{}\hfill \oarg{descripción sólo para referencias}}
% \DescribeMacro{\Item}
% \DescribeMacro{\ItemR}
% \DescribeMacro{\ItemA}
% \DescribeMacro{\ItemE}
%
% Cuando pedimos descuentos individuales (opción |descuentos| al cargar la clase), todas las macros |\Item| tienen un posible\footnote{Es un argumento opcional, aunque va entre llaves como todos los demás.} argumento más al final que es el descuento porcentual. No especificarlo da el mismo resultado que un descuento de 0\% sobre ese ítem:
%
%
% \subsubsection{Opción `{\ttfamily sincantidad}'\label{subsec:sincantidad}}
%
% \noindent
% \DescribeMacro{\Item}
% \DescribeMacro{\ItemR}
% \DescribeMacro{\ItemA}
% \DescribeMacro{\ItemE}
% {\macromandcolor|\Item|\marg{descripción}\oarg{descripción para csv}\marg{precio unitario}\oarg{descripción sólo para referencias}}
%
% \noindent {\macromandcolor|\Item|\marg{descripción}\oarg{descripción para csv}\marg{precio unitario}\marg{\% descuento} \\ \mbox{}\hfill \oarg{descripción sólo para referencias}}
%
% Por otra parte, si se usa la opción |sincantidad| al cargar la clase, todos los comandos |\Item| aceptan un argumento menos, que es el de la cantidad; eso significa que sin usar la opción |descuentos|, la sintaxis de |\Item| queda como lo indica la primera línea, y usando |descuentos|, queda como lo indica la segunda línea.
%
% \noindent
% \DescribeMacro{\ItemX}{\macrooptcolor|\ItemX|\marg{descripción}\oarg{descripción para csv}\oarg{descripción sólo para referencias}}
% 
% Puesto que |\ItemX| se usa para elementos sin precio, al usar |sincantidad| quedará aceptando sólo un argumento obligatorio que es la descripción y dos opcionales para la descripción en el archivo |.csv| y la descripción larga para referenciar luego. Cuando se activa la opción |descuentos|, |\ItemX| queda igual puesto que esta macro no lleva precio, entonces tampoco acepta descuentos.
%
% {\itshape Todas las variaciones de estos comandos |\Item| pueden apreciarse en los ejemplos.}
%
%
% \subsection{Descripción larga al final de la factura}
%
% Una vez terminado el entorno de la factura se puede escribir una descripción larga que saldrá en una caja -cuyo estilo puede cambiarse-\footnote{El diseño de esta descripción consiste en su linea de contorno, cambiable como ya explicamos en la subsección \ref{subsec:espacioslineas}, específicamente con la macro \cs{LineaDescripcion}.} debajo del cuadro de la factura, esto va dentro de un |\parbox| en la implementación de la caja. Si esta descripción no se escribe, simplemente no va a aparecer nada debajo del cuadro de la factura.
%
% \noindent \DescribeMacro{\Descripcion}
% {\macrooptcolor|\Descripcion|\marg{texto en formato de párrafos}}
%
% {\itshape Como podemos ver, |\Descripcion| es otro de los pocos comandos que no se usan en el preámbulo sino dentro del documento, jústamente al terminar el ambiente} |factura|.
%
%
% \section{Reverso de la factura}
%
% \DescribeEnv{reverso}
%  La factura puede tener información escrita en su reverso, por ejemplo para ciertas condiciones cuando estas necesitan mucho espacio. Ese texto se escribe dentro de un entorno llamado |reverso|, que puede o no incluirse según el requerimiento del usuario. si está escrito en el documento fuente también podrá decidirse si se incluye o no en el documento compilado.
%
% \iffalse
%<*literal>
% \fi
{\VERBATIM
\begin{verbatim}
    \begin{reverso}
     (Aquí va toda la información que irá escrita
      en el reverso de la hoja de la factura.)
    \end{reverso}
\end{verbatim}
}
% \iffalse
%</literal>
% \fi
%
% El reverso de la factura va escrito con la tipografía que el usuario elije con el comando |\LetraReverso|, como se explicará en la sección \ref{sec:tipografia}, y las dimensiones de la página se establecen con el comando |\GeometriaReverso| (sección \ref{sec:comandos}). Dicha geometría viene con un valor predeterminado en el |.def|.
%
% Si se compila con la opción |sinreverso|, lo escrito dentro del ambiente |reverso| no saldrá como reverso en la hoja de la factura.
%
%
% \section{El reporte}
%
% \DescribeEnv{reporte}
%  Para elaborar un reporte, lo único que se debe hacer es escribir el texto del reporte dentro del ambiente para ese fin, que se llama |reporte|, de manera que en el documento tenemos un ambiente para la factura y otro para el reporte, y elegimos al momento de compilar qué es lo que queremos obtener.
%
% \iffalse
%<*literal>
% \fi
{\VERBATIM
\begin{verbatim}
    \begin{reporte}
     (Aquí va el reporte, sin límite la extensión del texto.)
    \end{reporte}
\end{verbatim}
}
% \iffalse
%</literal>
% \fi
%
% Cuando se compila con |conreporte|, el reporte sale desde la segunda página, o desde la tercera si se usa |twoside|.
%
% Cuando se compila con |soloreporte|, el motor de compilación no procesa la factura en favor del reporte, y sin especificar ninguna de estas dos opciones, el reporte es ignorado y sale únicamente la factura (o la prefactura o cotización).
%
% El reporte empieza con un encabezado y luego un título especificable; el encabezado consiste en el nombre del proveedor, el nombre del cliente y la persona de contacto. Luego de ese encabezado viene el título del reporte y después empieza lo que el usuario escriba dentro del entorno del reporte.
%
% Hay que tener en cuenta que, como se explicó en la sección \ref{sec:comandos}, el reporte puede tener una imagen de fondo, que puede ser un membrete. Ese membrete, si se quiere en todas las páginas deberá incluirse con |\FondoReporte*|, y si se quiere sólo en la primera página del reporte deberá usarse |\FondoReporte|. Ahora, si se usa la versión sin |*|, será necesario también usar |\InicioReporte| con una distancia para empezar a escribir un poco más abajo en la primera página, ya que es la única que tiene el membrete. Pero si se usa la versión con |*|, entonces basta fijar los márgenes para todas las páginas con |\GeometriaReporte|, tal como se adelantó en la sección \ref{sec:comandos}.
%
% El reporte estará escrito con la tipografía especificable con |\LetraReporte|, como se explicará más adelante en la sección \ref{sec:tipografia}.
%
% Por otra parte, al reporte se le asigna un título, que puede aparecer o no en la cabecera de página o en el pie de página. Si se quiere que aparezca el título en esos lugares, debe usarse |\TituloReporte*|, de lo contrario, |\TituloReporte|. Si no se especifica el título, entonces la información mostrada allí es la referencia a la factura asociada a este reporte, como se comentó antes cuando se introdujeron estos comandos.
%
%
% \subsection{Estilos de página para el reporte\label{subsec:pagestyle}}
%
% La clase define varios estilos de página\footnote{Usables en el reporte únicamente.} de \LaTeX{}, cambiables con los comandos usuales: |\pagestyle| y |\thispagestyle| dentro del entorno del reporte, pero es mejor definir el estilo de página en el preámbulo del documento, con |\EstiloPagReporte|.
%
% Estos estilos de página están definidos con el paquete |fancyhdr|, y son: \begin{inparaenum}[\bfseries i)]\item |plain|, \item |empty|, \item |soloarriba|, \item |soloabajo|, y \item |solonumpag|\end{inparaenum}.
%
% Con |plain|\footnote{Por razones de compatibilidad usamos dos estilos de página conocidos en \LaTeX, que son |plain| y |empty|, siendo |plain| uno de nuestros predeterminados.}, en la cabecera saldrá la información sobre el reporte y en el pie de página saldrá el número de página; con |empty| no sale nada en ninguna parte; con |soloarriba| y |soloabajo|, toda esa información anterior sale en la cabecera o en el pie de página, respectivamente, y con |solonumpag|, lo único que aparece es el número de página centrado en el pie de página.
%
% El formato con la información a poner en estas partes de las páginas del reporte es cambiable con |\InfoPagina|, descrito anteriormente.
%
% Cuando se usa |\FondoReporte*|, como se supone que se usa para un membrete en todas las páginas, el estilo de página predeterminado del reporte será |soloabajo|, para todas las páginas, porque no tendría sentido tener un membrete en todas las páginas y una cabecera como la descrita anteriormente. Si el comando |\FondoReporte*| se está usando para algo distinto de un membrete, por ejemplo una marca de agua, entonces es probable querer cambiar los estilos de página con |\EstiloPagReporte|.
%
% Si se usa |\FondoReporte|, el estilo predeterminado para la primera página será |solonumpag|, y |plain| para el resto de las páginas, y si se no se especifica fondo alguno, entonces el estilo de página predeterminado para todas las páginas será |plain|.
%
% El usuario podrá definir más estilos de página adicionales para colocar el texto que se quiera en el lugar que se quiera, y usar esos estilos de página con |\EstiloPagReporte|. Para eso puede leer la documentación del paquete |fancyhdr|, que es lo que esta clase usa para los estilos de página. También puede consultar |factura.def|, que es donde están predefinidos los estilos que trae la clase.
%
% El grosor de la línea decorativa que coloca el paquete |fancyhdr| está definido en el archivo |.def|, y es cambiable en la macro |\@GrosorLineaInfoPag| como podrá verse en dicho archivo; no definimos un comando para ello por no ser algo relevante pero igual podrá cambiarse con |\renewcommand| por ejemplo, como se hace en dicho |.def|.
%
%
% \section{Tipografías para cada elemento a escribir\label{sec:tipografia}}
%
% Cada elemento escrito en la factura o reporte es sujeto a ser mostrado con una tipografía distinta, para lo cual existen comandos con los que se asigna el tipo de letra o estilo para cada cosa que sale escrita en el documento, por ejemplo, se puede cambiar el tipo de letra con el que se escribe la fecha, el tipo de letra con el que salen los elementos dentro del cuadro de la factura, el tipo de letra de los totales, de los títulos, de las cabeceras o pies de página en el reporte, etc.
%
% Cuando en lo sucesivo nos referiremos a tipografía, tipo de letra, o letra, nos estaremos refiriendo a cualquier cambio en la tipografía que pueda significar el nombre de la letra, su tamaño o atributos como negritas, es decir, mediante cualquiera de estos comandos podemos cambiar cualquier aspecto de la tipografía. Mirando el archivo |.def| pueden verse todos los valores predeterminados, que la mayoría de ellos son simplemente cambios en el tamaño (macro |\fontsize| de \LaTeX{}), y en algunos casos negritas.
%
% {\setlength{\parindent}{0pt}
% \DescribeMacro{\LetraItems}{\macrooptcolor|\LetraItems|\marg{tipo de letra}}
%
%  Con este comando cambiamos la tipografía utilizada en los ítems o elementos que están en el cuadro de la factura, de manera que {\itshape cambiar la longitud \cs{baselineskip} con el segundo argumento de \cs{fontsize} hará que la tabla tenga una altura distinta de las celdas.}
%
%  Lo anterior significa que {\bfseries para cambiar la altura de las celdas, la forma correcta es a través de cambiar el tamaño de la letra de sus elementos con este comando. (Véase el archivo |.def|)}.
%
% \DescribeMacro{\LetraNumeros}{\macrooptcolor|\LetraNumeros|\marg{tipo de letra}}
%
%  Este comando sirve para cambiar el tipo de letra de tota la información numérica que aparece en las distintas columnas del cuadro de la factura, sin contar los totales, para los cuales hay otros comandos.
%
% \DescribeMacro{\LetraTipoIVA}{\macrooptcolor|\LetraTipoIVA|\marg{tipo de letra}}
%
%  El tipo del IVA es la letra que indica qué IVA está siendo gravado, por ejemplo, la “|G|”, “|R|” y “|A|” que puede salir al lado de los elementos con IVA general, reducido y adicional respectivamente, y “|(E)|” al lado de los elementos exentos. Con esta macro cambiamos la letra con la que eso sale.
%
% \DescribeMacro{\LetraTitColumnas}{\macrooptcolor|\LetraTitColumnas|\marg{tipo de letra}}
%
%  Con |\LetraTitColumnas| podemos cambiar la tipografía de los títulos de los campos que hay en el cuadro de la factura, es decir, los textos “Cantidad”, “Descripción”, “Precio U.”, etc.
%
% \DescribeMacro{\LetraTitTotales}{\macrooptcolor|\LetraTitTotales|\marg{tipo de letra}}
%
%  Con este comando cambiamos la tipografía de los nombres de los totales, es decir, “IVA G.”, “BI G.”, salvo el total general que se cambia con |\LetraTitTotal|.
%
% \DescribeMacro{\LetraNumTotales}{\macrooptcolor|\LetraNumTotales|\marg{tipo de letra}}
%
%  Con esta macro se cambia la tipografía utilizada en los totales numéricos de la factura, exceptuando el total general que se cambia con |\LetraNumTotal|.
%
% \DescribeMacro{\LetraTitTotal}{\macrooptcolor|\LetraTitTotal|\marg{tipo de letra}}
%
%  Para cambiar la tipografía de la palabra que indica el total general, e.g. “TOTAL”, lo hacemos con |\LetraTitTotal|.
%
% \DescribeMacro{\LetraNumTotal}{\macrooptcolor|\LetraNumTotal|\marg{tipo de letra}}
%
%  Para la tipografía del total general (la celda con el número del total final), utilizamos este comando.
%
% \DescribeMacro{\LetraTitEnc}{\macrooptcolor|\LetraTitEnc|\marg{tipo de letra}}
%
%  Con este comando cambiamos el tipo de letra de la información en el título de las cajas del encabezado de la factura, por ejemplo los textos: “Resumen” y “Datos del cliente”.
%
% \DescribeMacro{\LetraEncFactura}{\macrooptcolor|\LetraEncFactura|\marg{tipo de letra}}
%
%  Podemos también cambiar el tipo de letra de la información del encabezado de la factura, es decir, el tipo de letra del texto que escribe el usuario en el resumen o de toda la identificación del cliente en la factura. El resumen naturalmente acepta más cambios de tipografía escribiéndolos dentro del párrafo que se escribe con el comando |\Resumen|.
%
% \DescribeMacro{\LetraEncReporte}{\macrooptcolor|\LetraEncReporte|\marg{tipo de letra}}
%
%  Para cambiar la tipografía de la información del encabezado del reporte usamos |\LetraEncReporte|.
%
% \DescribeMacro{\LetraNotaPrevia}{\macrooptcolor|\LetraNotaPrevia|\marg{tipo de letra}}
%
%  Con |\LetraNotaPrevia| podemos cambiar la tipografía con la que se escribe la -opcional- nota previa al cuadro de la factura.
%
% \DescribeMacro{\LetraNotaInterna}{\macrooptcolor|\LetraNotaInterna|\marg{tipo de letra}}
%
%  Con |\LetraNotaInterna| cambiamos el formato de la nota opcional que aparece como coletilla de la factura (nota interna).
%
% \DescribeMacro{\LetraNotaExterna}{\macrooptcolor|\LetraNotaExterna|\marg{tipo de letra}}
%
%  También podemos cambiar la tipografía de la nota externa a la factura, es decir, la -opcional- nota que va al terminar el cuadro de la factura antes de la -opcional también- descripción larga.
%
% \DescribeMacro{\LetraDescripcion}{\macrooptcolor|\LetraDescripcion|\marg{tipo de letra}}
%
%  Con este comando cambiamos la tipografía de la descripción larga; naturalmente cualquier cambio puede hacerse luego dentro del texto de la descripción, igual que con el resumen.
%
% \DescribeMacro{\LetraFirmas}{\macrooptcolor|\LetraFirmas|\marg{tipo de letra}}
%
%  Con |\LetraFirmas| podemos cambiar la letra usada para escribir el nombre debajo de la línea para firmar en la factura.
%
% \DescribeMacro{\LetraNotaFinal}{\macrooptcolor|\LetraNotaFinal|\marg{tipo de letra}}
%
%  Para cambiar la tipografía de la nota final se hace con este comando. (Aplica el mismo comentario que para la descripción y el resumen).
%
% \DescribeMacro{\LetraDenom}{\macrooptcolor|\LetraDenom|\marg{tipo de letra}}
%
%  La tipografía asociada a la denominación del documento, es decir, al lado del número (e.g. “FACTURA” o “COTIZACIÓN”), y la asociada al texto “No. CONTROL” se cambia con |\LetraDenom|.
%
% \DescribeMacro{\LetraNumeracion}{\macrooptcolor|\LetraNumeracion|\marg{tipo de letra}}
%
%  Con |\LetraNumeracion| cambiamos la tipografía de las numeraciones como la del número de la factura y el número de control.
%
% \DescribeMacro{\LetraFecha}{\macrooptcolor|\LetraFecha|\marg{tipo de letra}}
%
%  La tipografía para la fecha de emisión y la fecha de vencimiento la podemos cambiar con este comando.
%
% \DescribeMacro{\LetraTitReporte}{\macrooptcolor|\LetraTitReporte|\marg{tipo de letra}}
%
%  La letra del título que aparece al principio del reporte se puede cambiar con |\LetraTitReporte|.
%
% \DescribeMacro{\LetraInfoPagina}{\macrooptcolor|\LetraInfoPagina|\marg{tipo de letra}}
%
%  Las cabeceras o pies de página del reporte también tienen un comando para su tipo de letra, y es |\LetraInfoPagina|.
%
% \DescribeMacro{\LetraReverso}{\macrooptcolor|\LetraReverso|\marg{tipo de letra}}
%
%  Con |\LetraReverso| cambiamos la tipografía de la información que escribamos en el reverso de la hoja de la factura.
%
% \DescribeMacro{\LetraReporte}{\macrooptcolor|\LetraReporte|\marg{tipo de letra}}
%
%  Finalmente, para cambiar la tipografía con la que se inicia el reporte tenemos este comando. Obviamente también luego se puede hacer cualquier cambio dentro del ambiente |reporte|.
%
% }
%
%
% \section{Etiquetas y referencias a ítems de la factura\label{sec:etiquetas}}
%
% Esta clase |factura| soporta agregar etiquetas a cada elemento de la factura que luego en el reporte o en casi cualquier otra parte queramos referenciar; {\slshape no se pueden referenciar etiquetas solamente en la nota previa ni en el resumen.}
%
% \noindent \DescribeMacro{\label}{\macrooptcolor|\label|\marg{etiqueta}}
%
%  Para ello usamos la conocida macro |\label| de \LaTeX{} justamente después del ítem que nos interese luego referenciar, por ejemplo (usando descuentos individuales):
%
% \iffalse
%<*literal>
% \fi
{\VERBATIM
\begin{verbatim}
    \begin{factura}
     \Item{1}{Artículo gravado con IVA general}{1000}
     \Item{3}{Otro artículo gravado con IVA general}{1500}{10}[Descripción larga 
      para referenciar el reporte u otro fin.

      Con formato de párrafo con saltos de línea si se desea.]\label{el-otro}
    \end{factura}
\end{verbatim}
}
% \iffalse
%</literal>
% \fi
%
% De esta forma creamos una etiqueta llamada “el-otro” para el segundo elemento, que luego podremos citar por ejemplo en la descripción larga o en el reporte con comandos similares a los que \LaTeX{} provee para referenciar etiquetas.
%
% \noindent \DescribeMacro{\itemref}{\macrooptcolor|\itemref|\marg{etiqueta}}\\
% \noindent \DescribeMacro{\cantref}{\macrooptcolor|\cantref|\marg{etiqueta}}\\
% \noindent \DescribeMacro{\descref}{\macrooptcolor|\descref|\marg{etiqueta}}\\
% \noindent \DescribeMacro{\puref}{\macrooptcolor|\puref|\marg{etiqueta}}\\
% \noindent \DescribeMacro{\subtref}{\macrooptcolor|\subtref|\marg{etiqueta}}\\
% \noindent \DescribeMacro{\dctoref}{\macrooptcolor|\dctoref|\marg{etiqueta}}\\
% \noindent \DescribeMacro{\ivaref}{\macrooptcolor|\ivaref|\marg{etiqueta}}\\
% \noindent \DescribeMacro{\ptref}{\macrooptcolor|\ptref|\marg{etiqueta}}\\
% \noindent \DescribeMacro{\ldescref}{\macrooptcolor|\ldescref|\marg{etiqueta}}\par
%
% Si se quiere mostrar el número de la línea que ocupa el ítem en la lista lo hacemos con |\itemref|, que en el ejemplo aparecería el número “|2|” indicando que es el segundo ítem de la factura\footnote{Cuando se usa \cs{itemref} conviene enumerar los ítems en la factura con la opción \texttt{numitem}.}. Con |\cantref| escribimos la cantidad del elemento citado mediante la etiqueta, en nuestro ejemplo sería “|3|”. Para la descripción del ítem usamos |\descref|, que en el caso del ejemplo sería el texto “|Otro artículo gravado con IVA general|”. Para el precio unitario usamos |\puref|, en el ejemplo “|1.500,00|”. El subtotal lo mostraríamos con |\subtref|; en nuestro caso “|4.500,00|”. Si se quiere mostrar el porcentaje del descuento del ítem citado usamos |\dctoref|, que para el ejemplo sería “|10|”. Para el porcentaje del IVA, usamos |\ivaref|, en nuestro caso escribiría “|12|”. Por último, para mostrar el precio total usamos |\ptref|, y en nuestro ejemplo eso escribiría “|4.050,00|” puesto que aplicamos un descuento del 10\%.
%
% Podemos ver que el uso de los comandos |\puref|, |\subtref| y |\ptref| nos muestra el precio en el mismo formato con el que lo vemos en la presentación de la factura, en cuanto al redondeo y al separador de miles y decimal.
%
% También podemos referenciar la -opcional- descripción larga que podemos usar por ejemplo en el reporte; esa descripción que va entre corchetes la podremos citar con |\ldescref|, y en nuestro ejemplo sería:
% “Descripción larga para referenciar el reporte u otro fin. Con formato de párrafo con saltos de línea si se desea.”\footnote{Aquí suprimimos los saltos de línea.}
%
% Con la asignación de estas etiquetas y su uso mediante las referencias podemos de manera conveniente citar cualquier ítem que se haya facturado, de manera dinámica sin tener que reescribir en la descripción larga o en el reporte aquello que ya se escribió en el cuadro de la factura. Para estas referencias se escogieron nombres de las macros que terminan en “\texttt{ref}” por compatibilidad con comandos de \LaTeX{} como |\pageref|.
%
%
%
% \section{Referenciar los valores establecidos en el preámbulo y los totales\label{sec:ref}}
%
%  En el preámbulo el usuario definió distintas macros, es decir, estableció la razón social del cliente, el R.I.F. y la dirección, entre otras cosas. Cualquiera de esas definiciones hechas en el preámbulo puede usarse luego en otra parte del documento con el mismo comando que se usó para su definición. Lo anterior significa que tenemos las siguientes macros posibles para usar dentro del cuerpo del documento (factura, reporte o reverso): |\Moneda|, |\RazonSocial|, |\RIF|, |\Direccion|, |\Telefono|, |\Email|, |\Contacto|, |\Emisor|, |\Conforme|, |\Proveedor|, |\Fecha|, |\FechaVencimiento|\footnote{La fecha de vencimiento no se establece en el preámbulo explícitamente sino a través de \cs{Credito} pero la clase dispone de esta interfaz para acceder a ella así como también con la interfaz dada por \texttt{datetime2}.}, |\Credito|, |\Num|, |\NumControl|, |\NumSerieControl|, |\Resumen|, |\NotaPrevia|, |\NotaInterna|, |\NotaExterna|, |\NotaFinal|, |\TituloReporte|.
%
% Esas macros, en el preámbulo funcionan para definir valores y textos, pero si ya no estamos en el preámbulos sino dentro del cuerpo del documento y escribimos una de esas macros\footnote{Dentro del cuerpo del documento estas macros obviamente no aceptan argumentos.}, lo que nos va a escribir será lo que está definido bajo esa macro, así sea con macros que no llegamos a utilizar explícitamente pero que tienen un valor predefinido como |\Moneda|.
%
% Por otra parte, en el documento el usuario también podrá referirse a cualquiera de los totales de la siguiente tabla\footnote{Las macros de la tabla, para representarlas en el mismo formato que en la factura, debemos usar la macro \cs{numprint}.}:
%
% \noindent \begin{tabular}{rl}
% \hline
% Total & Comando \\ \hline
% Bases imponibles & |\BIG|, |\BIR|, |\BIA|, |\BIE|\footnote{Lo llamamos \cs{BIE} para mantener la nomenclatura pero no es una base imponible ya que es un monto exento de impuesto.} \\
% IVA & |\AlicuotaG|, |\AlicuotaR|, |\AlicuotaA| \\
% Descuento en IVA & |\AlicuotaGDesc| \\
% Subtotales\footnote{El subtotal es el monto sin aplicar descuentos generales (los que se hacen con los comandos \cs{Descuento***}, es decir, los descuentos que no son individuales a cada item sino al totalizar), que coincide con la base imponible cuando no hay esos descuentos.} & |\SubtG|, |\SubtE|, |\SubtR|, |\SubtA| \\
% Descuentos & |\DescG|, |\DescE|, |\DescR|, |\DescA| \\
% Total final & |\Total| \\
% \hline
% \end{tabular}
%
% También ya mencionamos que la fecha de emisión y vencimiento se pueden acceder con la interfaz dada por el paquete |datetime2|, por ejemplo, para mostrar la fecha de emisión, podremos usar tanto |\Fecha| como |\DTMusedate{FechaEmision}|, y para la fecha de vencimiento, tanto |\FechaVencimiento| como |\DTMusedate{FechaVencimiento}|.
%
% La cantidad de ítems en la factura la podemos acceder via el contador |NumItem|, por ejemplo así |\arabic{NumItem}|.
%
% En los ejemplos hay casos con todas estas funciones de esta sección.
%
%
% \section{Más personalizaciones}
%
% \DescribeMacro{\@BeforeEndPreamble}
% La clase provee de un “gancho” llamado \cs{BeforeEndPreamble}. Lo que allí se agregue (es acumulativo, se puede ejecutar mas de una vez) será ejecutado por la clase |factura| luego de que hayan sido evaluadas las opciones de la clase y tomadas todas las decisiones en cuanto a opciones que impliquen otras opciones o que sean incompatibles con otras opciones; una vez que todas esas decisiones fueron tomadas y todo queda establecido, se ejecuta \cs{BeforeEndPreamble}, esto con el propósito de que el usuario pueda personalizar más el comportamiento de sus archivos |.tex|. Por ejemplo si el usuario hace lo siguiente, no funcionará:
%
% \iffalse
%<*literal>
% \fi
{\VERBATIM
\begin{verbatim}
    \documentclass[cotizacion]{factura}
    [...]
    \ifcotizacion\FondoFactura{Fondo}\fi
    \begin{document}
     \begin{factura}
      [...]
     \end{factura}
    \end{document}
\end{verbatim}
}
% \iffalse
%</literal>
% \fi
%
% No funcionará porque el condicional \cs{ifcotizacion} aún no se ha establecido (eso lo hace la clase con el gancho \cs{AtEndPreamble} de |etoolbox| luego de que todo el preámbulo haya sido analizado), por lo que se evaluará como falso; tendría el usuario que agregarlo dentro de un gancho \cs{AtEndPreamble} pero tampoco funcionará del todo, para eso esta clase provee del gancho \cs{BeforeEndPreamble}, y se escribiría este ejemplo así:
%
% \iffalse
%<*literal>
% \fi
{\VERBATIM
\begin{verbatim}
    \documentclass[cotizacion]{factura}
    [...]
    \BeforeEndPreamble{\ifcotizacion\FondoFactura{Fondo}\fi}
    \begin{document}
     \begin{factura}
      [...]
     \end{factura}
    \end{document}
\end{verbatim}
} 
% \iffalse
%</literal>
% \fi
%
% De esta forma, el condicional se evaluará después de que se hayan establecido todas las opciones (en este caso, |cotizacion|) y el usuario podrá inclusive apagar opciones y encenderlas nuevamente usando los condicionales (como se explicó en la subsección \ref{subsec:condicionales} porque después de ejecutar este gancho, automáticamente se volverán a evaluar las opciones y a tomar todas las decisiones de implicaciones y compatibilidades.
%
% \newpage
% \DescribeMacro{\@PosicionaFechas}
% \DescribeMacro{\@PosicionaDenom}
% \DescribeMacro{\@PosicionaNumControl}
% \DescribeMacro{\@PosicionaNotaFinal}
% \DescribeMacro{\@PosicionaFirmas}
% \DescribeMacro{\@EscribeDatos}
% \DescribeMacro{\@EscribeDatosResumen}
% \DescribeMacro{\@EscribeDatosReporte}
% \DescribeMacro{\@EscribeNotaFinal}
% \DescribeMacro{\@EscribeNotaPrevia}
% \DescribeMacro{\@DibujaLinea}
% \DescribeMacro{\@EscribeItemsEnCSV}
% \DescribeMacro{\@EscribeTotalesEnCSV}
% \DescribeMacro{\@EncabezadoFactura}
% \DescribeMacro{\@EncabezadoReporte}
% También es posible darle aspecto distinto a muchas de las partes que conforman la factura. Internamente en la clase, cada parte es ubicada en la hoja con un comando y su contenido es escrito en el papel con otro comando. Los comandos que más probablemente son sujetos a ser cambiados por el usuario son los comandos que escriben el texto más que aquellos que lo ubican en la hoja, por tal razón, los comandos que posicionan aquello que va a escribirse lo definimos dentro del archivo |.cls|, y los que propiamente escriben cada parte de la hoja están definidos en el |.def|. Ambas fuentes las podrán consultar los usuarios para mayor comprensión, y para la total personalización. Recomendamos empezar por el archivo de definiciones |factura.def|.
%
% Estas macros son internas y sirven para, como dijimos antes, que \LaTeX{} se prepare para escribir el texto en la hoja; todas empiezan con |\@Posiciona| y no debe ser común necesitar cambiarlas.
%
% Todas estas macros llaman a su correspondiente macro con el estilo del elemento a escribir, que comienza con |\@Estilo|, y con cambiar su contenido podemos cambiar la forma como es escrita esa parte de la factura en la hoja.
%
% \DescribeMacro{\@EstiloFechas}
% \DescribeMacro{\@EstiloDenom}
% \DescribeMacro{\@EstiloNumControl}
% \DescribeMacro{\@EstiloNotaPrevia}
% \DescribeMacro{\@EstiloNotaFinal}
% \DescribeMacro{\@EstiloFirmas}
% \DescribeMacro{\@DatosTIKZ}
% \DescribeMacro{\@NotaPreviaTIKZ}
% \DescribeMacro{\@DescripcionTIKZ}
% \DescribeMacro{\@NotaFinalTIKZ}
% Estas últimas son las que están definidas dentro de |factura.def|, y que eventualmente puede desearse cambiarlas.
%
% Adicionalmente están las macros que se encargan de distintas partes de la factura y del reporte como los datos del cliente, el resumen, la nota previa, la nota final, los encabezados, la información en el archivo |.csv|, y la macro que dibuja la línea que inutiliza el espacio en blanco. Todas estas son macros internas que no deben ser llamadas por el usuario, y no consideramos que pueda sea necesario cambiarlas.
%
% Por último, las macros que alojan el diseño \tikz de los datos del cliente, la nota previa, la descripción larga y la nota final son |\@DatosTIKZ|, |\@NotaPreviaTIKZ|, |\@DescripcionTIKZ| y |\@NotaFinalTIKZ|; están definidas en el archivo |factura.def|.
%
% Cualquiera de estas macros que se quiera redefinir habría que hacerlo en el preámbulo del |.tex| y encerrándolas entre |\makeatletter| y |\makeatother| para poder redefinir una macro que contiene un símbolo “|@|”.
%
% {\bfseries \itshape Finalmente, como dijimos en la sección \ref{subsec:aux}, una vez adecuada la clase al diseño que se quiere, se podrán escribir todos los comandos que hacen estos ajustes en un} |.tex| {\itshape o} |.def| {\itshape y cargarlo con} |\input|, |\include| {\itshape o con} |\Cliente|, pudiendo discriminar por clientes dicha personalización.
%
%
%
% \newpage
% \setstretch{1}
% \section{Tabla con todas las opciones}
%
% \noindent La tabla que sigue muestra una lista de las opciones posibles al momento de cargar la clase:
%
% \noindent \begin{tabular}{>{\ttfamily}ll}
%  \rmfamily \bfseries Opción & \bfseries Significado \\ \hline
%  agrupado & Agrupa los totales por tipo en el modo de columnas. \\
%  centradesc & Centra la descripción de los ítems. \\
%  centrado & Centra los totales en el modo de filas. \\
%  conlineasha & Dibuja una línea separando los artículos entre sí. \\
%  conreporte & Produce la factura y el reporte. \\
%  cotización & Imprime una cotización. \\
%  csv & Genera un archivo |.csv| con la información numérica de la factura. \\
%  dcu & Descripción, cantidad, precio unitario. \\
%  denom & Explicita la denominación del documento. \\
%  descuentos & A cada artículo se le puede asignar un descuento individual. \\
%  dosfirmas & Incluye los dos espacios para firmar. \\
%  duc & Descripción, precio unitario, cantidad. \\
%  electro\footnote{Ya no sirve esta opción.} & Calcula los descuentos para el IVA para pagos por medios electrónicos. \\
%  expandecuadro & Llena con espacio extra el cuadro de la factura. \\
%  filas & Escribe los totales en filas. \\
%  G & Agrega una “|G|” en los ítems con IVA general. \\
%  impdesc & Escribe la denominación del IVA en la descripción. \\
%  iva & Expresa en otra columna el porcentaje de IVA para cada elemento a facturar. \\
%  linea & Dibuja una línea que inutiliza el espacio en blanco en el cuadro de la factura. \\
%  nospanish & No carga |babel| ni |fontenc|. \\
%  numitem & Enumera los ítems en la factura en otra columna extra al principio. \\
%  prefactura & Imprime una prefactura. \\
%  sincantidad & Los artículos van sin expresar su cantidad. \\
%  sinexpandir & No expande elementos en la factura. \\
%  sinfirmas & No incluye espacios para firmar. \\
%  siniva & No incluye IVA, pero lo deja expresado. \\
%  sinivaexpresado & No incluye IVA ni lo expresa. \\
%  sinlineahni & No dibuja la línea horizontal entre totales y nota interna. \\
%  sinlineahtit & No dibuja la línea horizontal del título de las columnas. \\
%  sinlineahtot & No dibuja la línea horizontal entre ítems y totales. \\
%  sinlineashi & No dibuja las líneas horizontales internas. \\
%  sinlineashe & No dibuja las líneas horizontales externas. \\
%  sinlineash & No dibuja ninguna línea horizontal. \\
%  sinlineasvi & No dibuja las líneas verticales internas. \\
%  sinlineasve & No dibuja las líneas verticales externas. \\
%  sinlineasv & No dibuja ninguna línea vertical. \\
%  sinmarcas & No escribe el carácter de fin de línea ni el carácter en las celdas vacías \\
%            & cuando la descripción ocupa más que una línea. \\
%  sinmonedatotales & Quita el signo monetario de las totalizaciones. \\
%  sinnumero & No pone el número cuando se usa |denom|. \\
%  sintotales & No escribe los totales al final del cuadro; para cotizaciones. \\
%  sinreverso & No compila el reverso de la factura así se haya escrito. \\
%  soloreporte & Produce únicamente el reporte. \\
%  subtotal & En otra columna muestra el subtotal de cada ítem antes aplicar descuento. \\
%  todosiva & Expresa todos los tipos de IVA. \\
%  twoside & Imprime el reporte a dos caras. \\
%  unafirma & Sólo incluye la firma del emisor. \\
%  vencimiento & Escribe la fecha de vencimiento así no haya crédito en la factura. \\
% \end{tabular}
%
%
% \newpage
% \section{Tabla con todos los comandos}
%
% A continuación presentamos una tabla con todos los comandos, que salvo |\FirmaReporte|, los |\Item| y |\Descripcion|, se poner todos en el preámbulo.
%
% \noindent \begin{tabular}{>{\ttfamily}ll}
% \rmfamily \bfseries Comando & \bfseries Significado \\ \hline
%  |\cantref| & Muestra la cantidad del ítem referenciado. \\
%  |\CI| & Cédula de identidad del cliente. \\
%  |\Cliente| & Carga un archivo |.def| o |.tex| con personalizaciones. \\
%  |\NotaFinal| & Establece la nota final. \\
%  |\Conforme| & Nombre de quien recibe la factura, para el espacio de la firma. \\
%  |\Contacto| & Persona de contacto. \\
%  |\Credito| & Establece la cantidad de días para el crédito de la factura. \\
%  |\dctoref| & Muestra al descuento del ítem referenciado. \\
%  |\Denom| & Nombre de la denominación de documento. \\
%  |\descref| & Muestra a la descripción del ítem referenciado. \\
%  |\Descripcion| & Descripción larga de la factura\footnote{El comando \cs{Descripcion} se escribe dentro del documento, luego del ambiente \texttt{factura}, y no en el preámbulo.}. \\
%  |\Descuento| & Porcentaje de descuento para todos los elementos. \\
%  |\Descuento*| & Descuento neto en el subtotal, para todos los elementos. \\
%  |\Descuento[GRAE]| & Descuento en porcentaje, discriminado según el tipo de ítem. \\
%  |\Descuento[GRAE]*| & Descuento en el subtotal, discriminado según el tipo de ítem. \\
%  |\Direccion| & Dirección del cliente. \\
%  |\Email| & E-mail del cliente. \\
%  |\Emisor| & Nombre de quien emite la factura, para el espacio de la firma. \\
%  |\EstiloPagReporte| & Establece el estilo de página del reporte. \\
%  |\Fecha| & Establece la fecha de emisión de la factura en formato predeterminado. \\
%  |\Fecha*| & Establece la fecha de emisión de la factura en formato ISO. \\
%  |\FirmaFactura| & Carga la imagen de la firma en la factura. \\
%  |\FirmaReporte| & Carga la imagen de la firma en el reporte\footnote{El comando \cs{FirmaReporte} se escribe dentro del ambiente \texttt{reporte} y no en el preámbulo.}. \\
%  |\FondoFactura| & Carga la imagen del fondo de la factura. \\
%  |\FondoReporte| & Carga la imagen del fondo del reporte para la primera página. \\
%  |\FondoReporte*| & Carga la imagen del fondo del reporte para todas las páginas. \\
%  |\FormatoFecha| & Establece lo relacionado con el formato de las fechas. \\
%  |\GeometriaFactura| & Establece los parámetros de la geometría de la factura. \\
%  |\GeometriaReporte| & Establece los parámetros de la geometría del reporte. \\
%  |\InfoPagina| & Cambia la información de las cabeceras o pies de página del reporte. \\
%  |\InicioReporte| & Distancia hasta el inicio del texto en la primera hoja del reporte. \\
%  |\Item[RAEX]| & Agrega un elemento a la factura\footnote{Estos comandos \cs{Item} se escriben dentro del ambiente \texttt{factura} y no en el preámbulo.}. \\
%  |\itemref| & Muestra el número de línea en la factura del ítem referenciado. \\
%  |\ivaref| & Muestra al IVA del ítem referenciado. \\
%  |\label| & Agrega una etiqueta a un ítem de la factura\footnote{Se mantiene el nombre “\cs{label}” por compatibilidad con los comandos conocidos de \LaTeX. Este comando también se escribe dentro del ambiente \texttt{factura}.}. \\
%  |\ldescref| & Muestra a la descripción larga del ítem referenciado. \\
% \end{tabular}
% \newpage
% \noindent \begin{tabular}{>{\ttfamily}ll}
% \rmfamily \bfseries Comando & \bfseries Significado \\ \hline
%  |\LetraNotaFinal| & Tipografía para la nota final. \\
%  |\LetraDenom| & Tipografía para la denominación del documento. \\
%  |\LetraDescripcion| & Tipografía para la descripción en la factura. \\
%  |\LetraEncFactura| & Tipografía para la información en las cajas del encabezado. \\
%  |\LetraEncReporte| & Tipografía para la información del encabezado del reporte. \\
%  |\LetraFecha| & Tipografía para la fecha de emisión. \\
%  |\LetraFirmas| & Tipografía para los nombres de quienes firman la factura. \\
%  |\LetraInfoPagina| & Tipografía para las cabeceras o pies de página del reporte. \\
%  |\LetraItems| & Tipografía para los ítems en la factura. \\
%  |\LetraNotaExterna| & Tipografía para la nota externa al cuadro de la factura. \\
%  |\LetraNotaInterna| & Tipografía para la nota interna al cuadro de la factura. \\
%  |\LetraNotaPrevia| & Tipografía para la nota previa al cuadro de la factura. \\
%  |\LetraNumeracion| & Tipografía para las numeraciones. \\
%  |\LetraNumeros| & Tipografía para los números en la factura. \\
%  |\LetraNumTotal| & Tipografía para el total final numérico. \\
%  |\LetraNumTotales| & Tipografía para los totales numéricos. \\
%  |\LetraReporte| & Tipografía para el texto del reporte. \\
%  |\LetraReverso| & Tipografía para el reverso de la factura. \\
%  |\LetraTipoIVA| & Tipografía para el tipo de IVA. \\
%  |\LetraTitColumnas| & Tipografía para los nombres de las columnas del cuadro de la factura. \\
%  |\LetraTitEnc| & Tipografía para los títulos en las cajas del encabezado de la factura. \\
%  |\LetraTitReporte| & Tipografía para el título del reporte. \\
%  |\LetraTitTotal| & Tipografía para el nombre del total final. \\
%  |\LetraTitTotales| & Tipografía para los nombres de los totales. \\
%  |\LineaNotaFinal| & Grosor de la línea de la caja de la nota final, \\
%                      & curvatura de la esquina y sombra. \\
%  |\LineaDescripcion| & Grosor de la línea de la caja de la descripción, \\
%                      & curvatura de la esquina y sombra. \\
%  |\LineaEncabezado| & Grosor de la línea de las cajas del encabezado de la factura, \\
%                     & curvatura de la esquina y sombra. \\
%  |\LineaNotaPrevia| & Grosor de la línea de la caja de la nota previa al cuadro, \\
%                     & curvatura de la esquina y sombra. \\
%  |\LineaFirmas| & Longitud y grosor de las líneas para firmar. \\
% \end{tabular}
% \newpage
% \noindent \begin{tabular}{>{\ttfamily}ll}
% \rmfamily \bfseries Comando & \bfseries Significado \\ \hline
%  |\Moneda| & Cambia el signo monetario a mostrar. \\
%  |\Nombre| & Establece el nombre del cliente. \\
%  |\NotaInterna| & Escribe una nota interna al cuadro de la factura. \\
%  |\NotaExterna| & Escribe una nota extra externa al cuadro de la factura. \\
%  |\NotaPrevia| & Escribe una nota antes del cuadro de la factura. \\
%  |\Num| & Número de la factura. \\
%  |\NumControl| & Número de control de la factura. \\
%  |\NumSerieControl| & Número de serie de control de la factura. \\
%  |\PosNotaFinal| & Posición $(x_o,x_f,y)$ del párrafo con la nota final. \\
%  |\PosNotaFinal*| & Posición $(x_o,x_f,y)$ del párrafo con la nota final \\
%                     & (el texto se despliega hacia arriba). \\
%  |\PosControl| & Posición $(x,y)$ del número de control. \\
%  |\PosControl*| & Posición $(x,y)$ del número de control \\
%                 & (el texto se despliega hacia arriba). \\
%  |\PosDenom| & Posición $(x,y)$ del texto de la denominación. \\
%  |\PosDenom*| & Posición $(x,y)$ del texto de la denominación \\
%               & (el texto se despliega hacia arriba). \\
%  |\PosFecha| & Posición $(x,y)$ de las fechas. \\
%  |\PosFecha*| & Posición $(x,y)$ de las fechas \\
%               & (el texto se despliega hacia arriba). \\
%  |\PosFirmas| & Posición $(x_o,x_f,y)$ de las firmas. \\
%  |\Proveedor| & Nombre de quien firma el reporte. \\
%  |\ptref| & Muestra al precio total del ítem referenciado. \\
%  |\puref| & Muestra al precio unitario del ítem referenciado. \\
%  |\RazonSocial| & Establece la razón social. \\
%  |\Resumen| & Resumen. \\
%  |\RIF| & R.I.F. del cliente. \\
%  |\SepNotaFinal| & Separación interna entre el borde y el texto en la nota final. \\
%  |\SepDatosResumen| & Separación horizontal entre los datos del cliente y el resumen. \\
%  |\SepDescripcion| & Separación interna entre el borde y el texto en la descripción. \\
%  |\SepEncabezado| & Separación interna entre el borde y el texto en el encabezado. \\
%  |\SepNotaPrevia| & Separación interna entre el borde y el texto en la nota previa. \\
%  |\SepFilas| & Separación horizontal y vertical en los totales en el modo de filas. \\
%  |\SepItemsExtra| & Separación vertical extra entre los elementos en la factura. \\
%  |\SepVertical| & Separación vertical entre los distintos elementos de la factura. \\
%  |\subtref| & Muestra al subtotal del ítem referenciado. \\
%  |\Telefono| & Teléfono del cliente. \\
%  |\TituloReporte| & Establece el título del reporte. \\
%  |\TituloReporte*| & Establece el título del reporte a salir en las cabeceras o pies de página. \\
% \end{tabular}
%
%
% \ifejemplos
%  \section{Ejemplos incluídos\label{sec:ejemplos}}
%  \todo*{Hay que hacer que en la documentación, toda opción y macro referencie al ejemplo donde se usa.}
%
%  Desde la página siguiente se muestran todos los ejemplos incluídos en la clase:
%
%  \newpage
%  \newgeometry{marginparsep=0pt,vmargin={1cm,1cm},hmargin={1cm,1cm}}
%  {
%   \pagestyle{empty}
%   \setlength{\parindent}{0pt}
%   \hbadness=10000
%
%   \Ejemplo{ejemplo-cotizacion}
%   \Ejemplo{ejemplo-prefactura}
%   \newpage
%   \Ejemplo{ejemplo-factura01}
%   \Ejemplo{ejemplo-factura02}[\small\parskip=2pt]
%   \newpage
%   \Ejemplo{ejemplo-factura03}
%   \Ejemplo{ejemplo-factura04}
%   \newpage
%   \Ejemplo{ejemplo-factura05}
%   \Ejemplo{ejemplo-factura06}[\small]
%   \newpage
%   \Ejemplo[0.4]{ejemplo-factura07}[\small]
%   \Ejemplo{ejemplo-factura08}
%   \newpage
%   \Ejemplo*[0.49]{ejemplo-factura09}
%   \Ejemplo[0.49]{ejemplo-factura09}*{2}
%   \Ejemplo*{ejemplo-factura09}*
%   \newpage
%   \Ejemplo*[0.49]{ejemplo-factura10}
%   \Ejemplo[0.49]{ejemplo-factura10}*{2}
%   \Ejemplo*{ejemplo-factura10}*
%   \newpage
%   \Ejemplo*[0.49]{ejemplo-reporte1}
%   \Ejemplo[0.49]{ejemplo-reporte1}*{2}
%   \Ejemplo[0.49]{ejemplo-reporte1}{4}[\slshape Por espacio se omitió la página 2 en esta representación.]
%   \newpage
%   \Ejemplo*[0.49]{ejemplo-reporte2}
%   \Ejemplo[0.49]{ejemplo-reporte2}*{2}
%   \Ejemplo[0.49]{ejemplo-reporte2}{3}
%   \newpage
%   \Ejemplo*[0.49]{ejemplo-reporte3}
%   \Ejemplo[0.49]{ejemplo-reporte3}*{2}
%   \Ejemplo[0.49]{ejemplo-reporte3}{3}
%   \newpage
%   \Ejemplo{beispiel-rechnung}
%   \newpage
%  }
%  \hbadness=1000
%  \restoregeometry
% \else
%  \message{^^JLaTeX Warning: Compilacio'n ra'pida. No se incluyeron los ejemplos.^^J}
% \fi
%
% \StopEventually{\endorgo}
%
%    \begin{macrocode}
%<*class>
\RequirePackage{etoolbox}%  'etoolbox' para los condicionales binarios y para \AtEndPreamble.

%% Condicionales auxiliares que sólo pueden ser usados dentro de un grupo.
\newbool{DCU}
\newbool{DUC}
\newbool{Descuentos}
%% (quitado por no vigente)
 % \newbool{Electro}
\newbool{ExpAbajo}\booltrue{ExpAbajo}
\newbool{IVA}
\newbool{NumItem}
\newbool{PrimeroDescripcion}
\newbool{SinCantidad}
\newbool{SinLineaHoriNI}
\newbool{SinLineaHoriTit}
\newbool{SinLineaHoriTot}
\newbool{SinLineasHoriExt}
\newbool{SinLineasVertExt}
\newbool{SinLineasVertInt}
\newbool{Subtotal}

%% Condicionales auxiliares para la posición de los objetos fijos cuando son medidos desde la derecha, o desde abajo.
\newbool{NotaFinalYarriba}
\newbool{NotaFinalYnegativo}% no hay NotaFinalXnegativo.
\newbool{ControlXnegativo}
\newbool{ControlYarriba}
\newbool{ControlYnegativo}
\newbool{DenomXnegativo}
\newbool{DenomYarriba}
\newbool{DenomYnegativo}
\newbool{FechaXnegativo}
\newbool{FechaYarriba}
\newbool{FechaYnegativo}
\newbool{FirmasYnegativo}% no hay FirmasXnegativo; hay X_inicial y X_final para las firmas.

%% Condicionales relativos a opciones de la clase.
\newif\ifagrupado
\newif\ifcentradesc
\newif\ifcentrado
\newif\ifconlineasha
\newif\ifconreporte
\newif\ifcotizacion
\newif\ifcsv
\newif\ifdcu
\newif\ifdenom
\newif\ifdescuentos
\newif\ifdosfirmas
\newif\ifduc
%% (quitado por no vigente)
 % \newif\ifelectro
\newif\ifexpandecuadro
\newif\iffilas
\newif\ifG
\newif\ifimpdesc
\newif\ifiva
\newif\iflinea
\newif\ifnospanish
\newif\ifnumitem
\newif\ifprefactura
\newif\ifsincantidad
\newif\ifsinexpandir
\newif\ifsinfirmas
\newif\ifsiniva
\newif\ifsinivaexpresado
\newif\ifsinlineahni
\newif\ifsinlineahtit
\newif\ifsinlineahtot
\newif\ifsinlineash
\newif\ifsinlineashe
\newif\ifsinlineashi
\newif\ifsinlineasv
\newif\ifsinlineasve
\newif\ifsinlineasvi
\newif\ifsinmarcas
\newif\ifsinmonedatotales
\newif\ifsinnumero
\newif\ifsinreverso
\newif\ifsintotales
\newif\ifsoloreporte
\newif\ifsubtotal
\newif\iftodosiva
\newif\iftwoside
\newif\ifunafirma
\newif\ifvencimiento

%% Condicionales auxiliares.
\newif\if@cdu \@cdutrue
\newif\if@factura \@facturatrue
\newif\if@reverso \@reversotrue
\newif\if@reporte
\newif\if@FondoFactura
\newif\if@FondoReporte
\newif\if@FondoReporteStar
\newif\if@Nombre
\newif\if@PruebaRef
\newif\if@ExpandeCuadro
\newif\if@LineaDesdeInicio
\newif\if@LineaHastaTotal
\newif\if@LineaHastaFinal
\newif\if@LineaDescripcion
\newif\if@BIR
\newif\if@BIA
\newif\if@Exento
\newif\if@DescG
\newif\if@DescR
\newif\if@DescA
\newif\if@DescE
\newif\if@DescGnum
\newif\if@DescRnum
\newif\if@DescAnum
\newif\if@DescEnum
\newif\if@DescuentoTotal

%% Longitudes para uso interno.
\newlength{\@TamDatos}
\newlength{\@TamResumen}
\newlength{\@TamFinalEncabezado}
\newlength{\@TamLineaFirmas}
\newlength{\@TamNumItem}
\newlength{\@TamCantidad}
\newlength{\@TamDescripcion}
\newlength{\@TamPrecioU}
\newlength{\@TamPrecioS}
\newlength{\@TamDescuento}
\newlength{\@TamIVA}
\newlength{\@TamPrecioT}
\newlength{\@GrosorLineaFirmas}
\newlength{\@GrosorLineaEncabezado}
\newlength{\@GrosorLineaNotaPrevia}
\newlength{\@GrosorLineaDescripcion}
\newlength{\@GrosorLineaNotaFinal}
\newlength{\@GrosorLineaInfoPag}
\newlength{\@EsquinaLineaEncabezado}
\newlength{\@EsquinaLineaNotaPrevia}
\newlength{\@EsquinaLineaDescripcion}
\newlength{\@EsquinaLineaNotaFinal}
\newlength{\@SombraEncabezado}
\newlength{\@SombraNotaPrevia}
\newlength{\@SombraDescripcion}
\newlength{\@SombraNotaFinal}
\newlength{\@LineaNodoXo}
\newlength{\@LineaNodoYo}
\newlength{\@LineaNodoXf}
\newlength{\@LineaNodoYf}
\newlength{\@SepTotales}
\newlength{\@PosFechaX}
\newlength{\@PosFechaY}
\newlength{\@PosDenomX}
\newlength{\@PosDenomY}
\newlength{\@PosFirmasXuno}
\newlength{\@PosFirmasXdos}
\newlength{\@PosFirmasY}
\newlength{\@PosNotaFinalXuno}
\newlength{\@PosNotaFinalXdos}
\newlength{\@PosNotaFinalY}
\newlength{\@PosControlX}
\newlength{\@PosControlY}
\newlength{\@EspacioVacioInicial}
\newlength{\@EspacioVacioFinal}
\newlength{\@EspacioSignoMonetario}
\newlength{\@SeparaItemsExtra}
\newlength{\@SeparaItemsExtraFinal}
\newlength{\@SeparaInternaEncabezado}
\newlength{\@SeparaInternaNotaPrevia}
\newlength{\@SeparaInternaDescripcion}
\newlength{\@SeparaInternaNotaFinal}
\newlength{\@SeparaDatosResumen}
\newlength{\@SeparaVertical}
\newlength{\@SeparaMinTotales}
\newlength{\@SeparaFilas}
\newlength{\@SeparaFinLineaDesc}
\newlength{\@SeparaTelefonoEmail}
\newlength{\@InicioReporte}
\newlength{\@FirmaFacturaXcorr}
\newlength{\@FirmaFacturaYcorr}
\newlength{\@FirmaReporteXcorr}
\newlength{\@FirmaReporteYcorr}
\newlength{\@FirmaReporteTextPos}
\newlength{\@FirmaReporteFirmaPos}
\newlength{\@FirmaReporteTamLinea}
\newlength{\@FirmaReportePosX}
\newlength{\@FirmaReporteExcesoLinea}
\newlength{\@TempDimen}
\newlength{\tabcolsep@@}
\newlength{\baselineskip@@}

%% Cajas para uso interno.
\newbox{\@CajaDatos}
\newbox{\@CajaResumen}
\newbox{\@CajaFirmaEmisor}
\newbox{\@CajaFirmas}
\newbox{\@CajaFecha}
\newbox{\@CajaDenom}
\newbox{\@CajaControl}
\newbox{\@CajaNotaFinal}
\newbox{\@CajaEspacioVacio}
\newbox{\@CajaPrueba}
\newbox{\@CajaCaracterCR}

%% Contadores.
\newcounter{@IVANoGeneral}%  '0' con sólo IVA G, '1' cuando hay IVA R o IVA A, y '2' cuando hay ambos.
\newcounter{@Firmas}%  '0' para sinfirmas, '1' para unafirma, y '2' para dosfirmas.
\setcounter{@Firmas}{2}
\newcounter{NumItem}%  Número del ítem actual, para uso de algunas macros en el cuadro de la factura y para cuando se usa la opción 'numitem'.
\newcount\AuxFecha

%% Errores, advertencias e informaciones de LaTeX.
\newcommand{\CLASSERROR}[1]{\ClassError{factura}{** #1. **}{}}
\newcommand{\CLASSWARNING}[1]{\ClassWarning{factura}{** #1. **}{}}
\newcommand{\CLASSINFO}[1]{\ClassInfo{factura}{** #1. **}{}}

%% Opciones de la clase.
\DeclareOption{agrupado}{\agrupadotrue}%  Agrupados los elementos totalizadores en el modo en columnas.
\DeclareOption{centradesc}{\centradesctrue}%  Para centrar la descripción de cada ítem.
\DeclareOption{centrado}{\centradotrue}%  Total y exento 'centrado'; sólo para estilo 'filas'
\DeclareOption{conlineasha}{\conlineashatrue}%  Una línea horizontal para cada ítem, como las facturas a mano.
\DeclareOption{conreporte}{\conreportetrue}%  Para imprimir factura y reporte.
\DeclareOption{cotizacion}{\cotizaciontrue}%  Cambia la denominación a 'COTIZACIÓN' y no 'FACTURA'.
\DeclareOption{csv}{\csvtrue}%  Para la salida en modo csv también.
\DeclareOption{dcu}{\dcutrue}%  Descripción, luego cantidad y luego precio unitario.
\DeclareOption{denom}{\denomtrue}%  Agrega el texto con la denominación del documento, para las formas libres que no tienen el número de factura.
\DeclareOption{descuentos}{\descuentostrue}%  Para los descuentos individuales.
\DeclareOption{dosfirmas}{\dosfirmastrue}%  Pone las dos firmas para firmar.
\DeclareOption{duc}{\ductrue}%  Descripción, luego precio unitario y luego cantidad.
\DeclareOption{electro}{\CLASSERROR{Ya no esta' vigente el descuento en la ali'cuota general por pagos electro'nicos}}%  Parche por el Decreto 3.085 de Nikolái, ya no vigente.
\DeclareOption{expandecuadro}{\expandecuadrotrue}%  Para rellenar con líneas en blanco de manera automática.
\DeclareOption{filas}{\filastrue}%  Estilo de totalización en filas.
\DeclareOption{G}{\Gtrue}%  Con esta opción sale la letra 'G' en los items con IVA general, y sale el porcentaje en la base imponible general.
\DeclareOption{impdesc}{\impdesctrue}%  Para el tipo de impuesto en la descripción y no en el precio unitario y total.
\DeclareOption{iva}{\ivatrue}%  Para la columna del IVA.
\DeclareOption{linea}{\lineatrue}%  Para la línea que inutiliza el espacio en blanco.
\DeclareOption{nospanish}{\nospanishtrue}%  No carga babel en castellano ni la codificación T1 de las letras.
\DeclareOption{numitem}{\numitemtrue}%  Para poner una columna extra al principio con el número del ítem.
\DeclareOption{prefactura}{\prefacturatrue}%  La prefactura; con el diseño de la forma libre... para enviar por correo electrónico, fax o imprimir preliminarmente.
\DeclareOption{sincantidad}{\sincantidadtrue}%  Para no poner el campo de la cantidad.
\DeclareOption{sinexpandir}{\sinexpandirtrue}%  No expande ninguna de las dos partes de la factura.
\DeclareOption{sinfirmas}{\sinfirmastrue}%  No agrega los espacios para firmar.
\DeclareOption{sinivaexpresado}{\sinivaexpresadotrue}%  Cuando no se cobra IVA y tampoco se quiere que aparezca la totalización del IVA.
\DeclareOption{siniva}{\sinivatrue}%  No cobra IVA.
\DeclareOption{sinlineahni}{\sinlineahnitrue}%  Estilo 'sinlineahni'.
\DeclareOption{sinlineahtot}{\sinlineahtottrue}%  Estilo 'sinlineahtot'.
\DeclareOption{sinlineahtit}{\sinlineahtittrue}%  Estilo 'sinlineahtit'.
\DeclareOption{sinlineashe}{\sinlineashetrue}%  Estilo 'sinlineashe'.
\DeclareOption{sinlineashi}{\sinlineashitrue}%  Estilo 'sinlineashi'.
\DeclareOption{sinlineash}{\sinlineashtrue}%  Estilo 'sinlineash'.
\DeclareOption{sinlineasve}{\sinlineasvetrue}%  Estilo 'sinlineasve'.
\DeclareOption{sinlineasvi}{\sinlineasvitrue}%  Estilo 'sinlineasvi'.
\DeclareOption{sinlineasv}{\sinlineasvtrue}%  Estilo 'sinlineasv'.
\DeclareOption{sinmarcas}{\sinmarcastrue}%  Para evitar que aparezcan los caracteres especiales cuando la descripción es larga.
\DeclareOption{sinmonedatotales}{\sinmonedatotalestrue}%  Para que no ponga el signo monetario en los totales.
\DeclareOption{sinnumero}{\sinnumerotrue}%  Escribir el texto de la denominación del documento pero sin escribir el número.
\DeclareOption{sinreverso}{\sinreversotrue}%  Para no imprimir el reverso de la factura.
\DeclareOption{sintotales}{\sintotalestrue}%  No escribe los totales al final, por ejemplo para cotizaciones de productos similares.
\DeclareOption{soloreporte}{\soloreportetrue}%  Para imprimir sólo el reporte.
\DeclareOption{subtotal}{\subtotaltrue}%  Para la columna extra del subtotal antes del descuento.
\DeclareOption{todosiva}{\todosivatrue}%  Con esta opción salen todos los IVA así valgan cero.
\DeclareOption{twoside}{\twosidetrue}%  Opción 'twoside' de article para los reportes.
\DeclareOption{unafirma}{\unafirmatrue}%  No agrega el espacio para que firme quien recibe.
\DeclareOption{vencimiento}{\vencimientotrue}%  Escribe fecha de vencimiento, aún con facturas sin crédito.

\ProcessOptions\relax

\LoadClass[12pt]{article}

%% Cargamos algunos paquetes siempre necesarios.
\RequirePackage{xparse}%  Algunos comandos los definimos con \NewDocumentCommand.
\RequirePackage[absolute]{textpos}%  Es necesario 'textpos' para colocar texto en ubicaciones arbitrarias.
\RequirePackage{eso-pic}%  Para el comando \AddToShipoutPicture.
\RequirePackage{xstring}%  Lo usamos sólo para tomar el menos al principio de una longitud.
\RequirePackage{calc}%  'calc' para hacer aritmética con las dimensiones.
\RequirePackage{fp-upn}%  Fundamental el paquete 'fp-upn' para todos los cálculos.
\RequirePackage{numprint}%  Con 'numprint' expresamos los totales en el formato requerido.
\RequirePackage{tikz}%  Para los recuadros del encabezado de la factura,
 \usetikzlibrary{shadows}%  y para el de la descripción.
\RequirePackage{tabularx}%  Con 'tabularx' hacemos la tabla.
\RequirePackage{environ}%  Con 'environ' podemos usar tabularx dentro del ambiente de la factura y escribir texto en el enddef.
\RequirePackage{fancyhdr}%  Las cabeceras fancy; así se use solamente en reporte, hay que cargarlo siempre.
\RequirePackage[calc]{datetime2}%  Para calcular la fecha del vencimiento del crédito.
\RequirePackage{anyfontsize}%  Para las tipografías en tamaños arbitrarios.
\RequirePackage{ifxetex}%  Para poder modificar correctamente la metadata del PDF y agregar la versión de la clase.
\RequirePackage{ifluatex}%  Ditto.
\RequirePackage{zref-savepos}[2016/05/21]%  Para guardar la posición en 'Y' expandiendo el cuadro, para la línea y para la firma en el reporte si la hubiera.
%% Hay que cargar siempre zref-savepos para poder compilar con LuaLaTeX porque este motor no tiene \pdfsavepos, \pdflastypos y zref-savepos las provee cuando no existen (motor LuaLaTex).

%% Gracias a Alan Munn y a wipet (Thanks to Alan Munn and wipet).
%% (https://tex.stackexchange.com/questions/396363/input-a-date-to-datetime2-in-a-format-different-than-iso)
\def\DTMsavedatex#1#2{%
 \expandafter \ifx \csname f=\DTMinformat\endcsname \relax
  \CLASSERROR{\string\DTMsavedatex: formato "\DTMinformat" desconocido}%
 \else
  \csname f=\DTMinformat\expandafter\endcsname#2&{#1}%
 \fi
}
\def\DTMfdef#1{\expandafter\def\csname f=#1\endcsname}
\def\DTMs#1#2{\DTMsavedate{#2}{#1}}
\DTMfdef{yyyy-mm-dd}#1-#2-#3&{\DTMs{#1-#2-#3}}
\DTMfdef{yyyy.mm.dd}#1.#2.#3&{\DTMs{#1-#2-#3}}
\DTMfdef{yyyymmdd}#1#2#3#4#5#6#7#8&{\DTMs{#1#2#3#4-#5#6-#7#8}}
\DTMfdef{dd-mm-yyyy}#1-#2-#3&{\DTMs{#3-#2-#1}}
\DTMfdef{dd.mm.yyyy}#1.#2.#3&{\DTMs{#3-#2-#1}}
\DTMfdef{ddmmyyyy}#1#2#3#4#5#6#7#8&{\DTMs{#5#6#7#8-#3#4-#1#2}}
\DTMfdef{mm-dd-yyyy}#1-#2-#3&{\DTMs{#3-#1-#2}}

%% Macros internas para el texto de la metadata que se va a escribir con el número de la versión de la clase.
\edef\@factura@ver{\expandafter\csname ver@factura.cls\endcsname}
\ExplSyntaxOn
  \seq_set_split:NnV \l_tmpa_seq {~} {\@factura@ver}
  \tl_set:Nx \@factura@vernum {\seq_item:Nn \l_tmpa_seq {2}}

  \seq_set_split:NnV \l_tmpa_seq {~} {\@factura@ver}
  \tl_set:Nx \@factura@fecha {\seq_item:Nn \l_tmpa_seq {1}}
\ExplSyntaxOff
\def\@factura@ver{Hecho con la clase factura-\@factura@vernum\ (\@factura@fecha)}

%% Escribimos la versión de la clase en la metadata del PDF.
\ifxetex
 \special{pdf: docinfo << /Subject (\@factura@ver) >>}
\else
 \ifluatex
  \pdfextension info{/Subject (\@factura@ver)}
 \else
  \pdfinfo{/Subject (\@factura@ver)}
 \fi
\fi

\def\@EstableceOpciones{%
 % Establecemos los parámetros y macros relativas a la fecha.
 \DTMsetstyle{\@FormatoFecha}\DTMsetup{datesep=\@FechaSeparadores}\gdef\DTMinformat{\@FormatoFechaEntrada}
 
 % Tomamos las acciones pertinentes dependiendo de las opciones de la clase (o condicionales activados en el preámbulo que cambian las opciones de la clase)
 \ifprefactura
  \ifcotizacion
   \prefacturafalse
   \cotizacionfalse
   \CLASSWARNING{Las opciones 'prefactura' y 'cotizacion' no son compatibles. Sin efecto ambas opciones}
  \else
   \denomtrue\setcounter{@Firmas}{0}
  \fi
 \fi
 \ifcotizacion\denomtrue\setcounter{@Firmas}{1}\fi

 \ifsinlineasv\booltrue{SinLineasVertExt}\booltrue{SinLineasVertInt}\fi
 \ifsinlineasvi\booltrue{SinLineasVertInt}\fi
 \ifsinlineasve\booltrue{SinLineasVertExt}\fi
 \ifsinlineash\booltrue{SinLineasHoriExt}\booltrue{SinLineaHoriNI}\booltrue{SinLineaHoriTit}\booltrue{SinLineaHoriTot}\fi
 \ifsinlineashi\booltrue{SinLineaHoriNI}\booltrue{SinLineaHoriTit}\booltrue{SinLineaHoriTot}\fi
 \ifsinlineahni\booltrue{SinLineaHoriNI}\fi
 \ifsinlineahtit\booltrue{SinLineaHoriTit}\fi
 \ifsinlineahtot
  \iffilas
   \CLASSWARNING{Las opciones 'sinlineahtot' y 'filas' no son compatibles. Sin efecto opcio'n 'sinlineahtot'}
   \sinlineahtotfalse
  \else
   \booltrue{SinLineaHoriTot}
  \fi
 \fi
 \ifsinlineashe\booltrue{SinLineasHoriExt}\fi

 \ifdosfirmas
  \setcounter{@Firmas}{2}
  \ifunafirma
   \unafirmafalse
   \ifsinfirmas
    \sinfirmasfalse
    \CLASSWARNING{Las opciones para las firmas son excluyentes. Permanece activada 'dosfirmas'}
   \else
    \CLASSWARNING{Las opciones para las firmas son excluyentes. Permanece activada 'dosfirmas'}
   \fi
  \else
   \ifsinfirmas
    \sinfirmasfalse
    \CLASSWARNING{Las opciones para las firmas son excluyentes. Permanece activada 'dosfirmas'}
   \fi
  \fi
 \fi
 \ifunafirma
  \setcounter{@Firmas}{1}
  \ifsinfirmas
   \sinfirmasfalse
   \CLASSWARNING{Las opciones para las firmas son excluyentes. Permanece activada 'unafirma'}
  \fi
 \fi
 \ifsinfirmas\setcounter{@Firmas}{0}\fi 
 
 \ifdcu
  \@cdufalse\ducfalse
  \booltrue{DCU}\boolfalse{DUC}
  \booltrue{PrimeroDescripcion}
 \fi
 \ifduc
  \ifdcu
   \boolfalse{PrimeroDescripcion}\boolfalse{DCU}\dcufalse\ducfalse\@cdutrue
   \CLASSWARNING{Las opciones 'duc' y 'dcu' no son compatibles. Sin efecto ambas opciones}
  \else
   \ductrue\@cdufalse\dcufalse
   \booltrue{DUC}\boolfalse{DCU}
   \booltrue{PrimeroDescripcion}
  \fi
 \fi
 \ifnumitem\booltrue{NumItem}\fi
 \ifsincantidad
  \booltrue{SinCantidad}
  \ifdcu
   \dcufalse\@cdutrue\boolfalse{DCU}\boolfalse{PrimeroDescripcion}
   \CLASSWARNING{La opcio'n 'sincantidad' no es compatible con 'dcu'. Sin efecto opcio'n 'dcu'}
  \fi
  \ifduc
   \ducfalse\@cdutrue\boolfalse{DUC}\boolfalse{PrimeroDescripcion}
   \CLASSWARNING{La opcio'n 'sincantidad' no es compatible con 'duc'. Sin efecto opcio'n 'duc'}
  \fi
 \fi
 \ifdescuentos\booltrue{Descuentos}\fi
 \ifiva\impdesctrue\booltrue{IVA}\fi
 \ifsubtotal
  \ifdescuentos
   \booltrue{Subtotal}
  \else
   \CLASSWARNING{La opcio'n 'subtotal' no aplica sin descuentos individuales. Sin efecto opcio'n 'subtotal'}
  \fi
 \fi

 \ifsinnumero\denomtrue\fi
 \ifsintotales
  \iffilas
   \filasfalse
   \CLASSWARNING{La opcio'n 'sintotales' no aplica para el modo de totalizacio'n en filas. Sin efecto opcio'n 'filas'}
  \fi
  \ifcotizacion
  \else
   \CLASSWARNING{La opcio'n 'sintotales' no debe usarse en facturas sino en cotizaciones. No es va'lida una factura sin totales}
  \fi
 \fi

 \ifcentrado
  \iffilas
  \else
   \centradofalse
   \CLASSWARNING{La opcio'n 'centrado' no es compatible con el modo de totalizacio'n en columnas. Sin efecto opcio'n 'centrado'}
  \fi
 \fi
 \ifagrupado
  \iffilas
   \agrupadofalse
   \CLASSWARNING{La opcio'n 'agrupado' no es compatible con el modo de totalizacio'n en filas. Sin efecto opcio'n 'agrupado'}
  \fi
 \fi

%% (quitado por no vigente)
 % \ifelectro
 %  \boolfalse{Electro}
 %  \electrofalse
 %  \gdef\@NotaInterna{\@TextoDescuentoPresidencial}
 %  \CLASSWARNING{Se establecio' el texto aclaratorio del descuento presidencial como 'nota interna'}
 %  \iffilas
 %   \filasfalse
 %   \CLASSWARNING{La opcio'n 'electro' no es compatible con la opcio'n 'filas'. Sin efecto la opcio'n 'filas'}
 %  \fi
 % \fi

 \iftodosiva \@BIRtrue \@BIAtrue \@Exentotrue \Gtrue \fi
 \ifsiniva\AtBeginDocument{\gdef\@ImpuestoG{0}\gdef\@ImpuestoR{0}\gdef\@ImpuestoA{0}}\fi
 \ifsinivaexpresado
  \iffilas
   \filasfalse
   \CLASSWARNING{La opcio'n 'sinivaexpresado' no es compatible con la opcio'n 'filas'. Sin efecto la opcio'n 'filas'}
  \else
   \AtBeginDocument{\gdef\@ImpuestoG{0}\gdef\@ImpuestoR{0}\gdef\@ImpuestoA{0}}
  \fi
 \fi

 \iflinea
  \ifexpandecuadro
  \else
   \CLASSWARNING{La opcio'n 'linea' requiere la opcio'n 'expandecuadro'. Activando la opcio'n 'expandecuadro'}
   \expandecuadrotrue
  \fi
 \fi
 \ifsinmarcas \def\@CaracterCR{\mbox{}} \def\@CaracterCeldasVacias{\mbox{}} \fi
 \ifnospanish
 \else
  \RequirePackage[spanish]{babel}%  Babel en castellano.
  \ifxetex
  \else
   \ifluatex
   \else
    \RequirePackage[T1]{fontenc}%  Codificación T1 para las tipografías; no aplica para XeLaTeX ni LuaLaTeX.
   \fi
  \fi
 \fi

 \ifexpandecuadro
  \ifsinexpandir
   \expandecuadrofalse
   \sinexpandirfalse
   \CLASSWARNING{Las opciones 'expandecuadro' y 'sinexpandir' no son compatibles. Sin efecto ambas opciones}
  \else
   \boolfalse{ExpAbajo}
  \fi
 \fi
 \ifsinexpandir\boolfalse{ExpAbajo}\expandecuadrofalse\fi

 \ifsinreverso\@reversofalse\fi
 \ifconreporte
  \@reportetrue
  \ifsoloreporte
   \soloreportefalse
   \CLASSWARNING{La opcio'n 'soloreporte' no es compatible con 'conreporte'. Sin efecto 'soloreporte'}
  \fi
 \fi
 \ifsoloreporte\@reportetrue\@facturafalse\@reversofalse\fi

 \ifcsv
  \if@factura
   \newwrite\csv \immediate\openout\csv=\jobname.csv
   \ifsintotales\else\AtEndDocument{\@EscribeTotalesEnCSV}\fi
  \else
   \csvfalse
   \CLASSWARNING{La opcio'n 'csv' require generar la factura. Sin efecto opcio'n 'csv'}
  \fi
 \fi
 
 \iftwoside
  \@twosidetrue\@mparswitchtrue
 \else
  \@twosidefalse\@mparswitchfalse
 \fi}
 
\NewDocumentCommand{\BeforeEndPreamble}{s s +m}{
 \IfBooleanTF{#2}
  {\gpreto{\@BeforeEndPreamble}{#3}}
  {\IfBooleanTF{#1}{\long\gdef\@BeforeEndPreamble{#3}}{\gappto{\@BeforeEndPreamble}{#3}}}}

\AtEndPreamble{
 \@EstableceOpciones
 \ifx\@undefined\@BeforeEndPreamble\else\@BeforeEndPreamble\@EstableceOpciones\fi

 % Escribimos en el .log las opciones cargadas:
 \ifagrupado \CLASSINFO{Opcio'n cargada: agrupado}\fi
 \ifcentradesc \CLASSINFO{Opcio'n cargada: centradesc}\fi
 \ifcentrado \CLASSINFO{Opcio'n cargada: centrado}\fi
 \ifconlineasha \CLASSINFO{Opcio'n cargada: conlineasha}\fi
 \ifconreporte \CLASSINFO{Opcio'n cargada: conreporte}\fi
 \ifcotizacion \CLASSINFO{Opcio'n cargada: cotizacion}\fi
 \ifcsv \CLASSINFO{Opcio'n cargada: csv}\fi
 \ifdcu \CLASSINFO{Opcio'n cargada: dcu}\fi
 \ifdenom \CLASSINFO{Opcio'n cargada: denom}\fi
 \ifdescuentos \CLASSINFO{Opcio'n cargada: descuentos}\fi
 \ifdosfirmas \CLASSINFO{Opcio'n cargada: dosfirmas}\fi
 \ifduc \CLASSINFO{Opcio'n cargada: duc}\fi
 %% (quitado por no vigente)
 % \ifelectro \CLASSINFO{Opcio'n cargada: electro}\fi
 \ifexpandecuadro \CLASSINFO{Opcio'n cargada: expandecuadro}\fi
 \iffilas \CLASSINFO{Opcio'n cargada: filas}\fi
 \ifG \CLASSINFO{Opcio'n cargada: G}\fi
 \ifimpdesc \CLASSINFO{Opcio'n cargada: impdesc}\fi
 \ifiva \CLASSINFO{Opcio'n cargada: iva}\fi
 \iflinea \CLASSINFO{Opcio'n cargada: linea}\fi
 \ifnospanish \CLASSINFO{Opcio'n cargada: nospanish}\fi
 \ifnumitem \CLASSINFO{Opcio'n cargada: numitem}\fi
 \ifprefactura \CLASSINFO{Opcio'n cargada: prefactura}\fi
 \ifsincantidad \CLASSINFO{Opcio'n cargada: sincantidad}\fi
 \ifsinexpandir \CLASSINFO{Opcio'n cargada: sinexpandir}\fi
 \ifsinfirmas \CLASSINFO{Opcio'n cargada: sinfirmas}\fi
 \ifsiniva \CLASSINFO{Opcio'n cargada: siniva}\fi
 \ifsinivaexpresado \CLASSINFO{Opcio'n cargada: sinivaexpresado}\fi
 \ifsinlineahni \CLASSINFO{Opcio'n cargada: sinlineahni}\fi
 \ifsinlineahtit \CLASSINFO{Opcio'n cargada: sinlineahtit}\fi
 \ifsinlineahtot \CLASSINFO{Opcio'n cargada: sinlineahtot}\fi
 \ifsinlineash \CLASSINFO{Opcio'n cargada: sinlineash}\fi
 \ifsinlineashe \CLASSINFO{Opcio'n cargada: sinlineashe}\fi
 \ifsinlineashi \CLASSINFO{Opcio'n cargada: sinlineashi}\fi
 \ifsinlineasv \CLASSINFO{Opcio'n cargada: sinlineasv}\fi
 \ifsinlineasve \CLASSINFO{Opcio'n cargada: sinlineasve}\fi
 \ifsinlineasvi \CLASSINFO{Opcio'n cargada: sinlineasvi}\fi
 \ifsinmarcas \CLASSINFO{Opcio'n cargada: sinmarcas}\fi
 \ifsinmonedatotales \CLASSINFO{Opcio'n cargada: sinmonedatotales}\fi
 \ifsinnumero \CLASSINFO{Opcio'n cargada: sinnumero}\fi
 \ifsinreverso \CLASSINFO{Opcio'n cargada: sinreverso}\fi
 \ifsintotales \CLASSINFO{Opcio'n cargada: sintotales}\fi
 \ifsoloreporte \CLASSINFO{Opcio'n cargada: soloreporte}\fi
 \ifsubtotal \CLASSINFO{Opcio'n cargada: subtotal}\fi
 \iftodosiva \CLASSINFO{Opcio'n cargada: todosiva}\fi
 \iftwoside \CLASSINFO{Opcio'n cargada: twoside}\fi
 \ifunafirma \CLASSINFO{Opcio'n cargada: unafirma}\fi
 \ifvencimiento \CLASSINFO{Opcio'n cargada: vencimiento}\fi

 % Cargamos aquí 'geometry' para que sea después de haber establecido 'twoside'.
 \RequirePackage{geometry}}

%% Escribimos el total como comentario en el .log.
\AtEndDocument{%
 \ifsintotales
 \else
  \if@factura\CLASSINFO{\@@Total: \space\space\space \Moneda \Total}\fi
 \fi}

%</class>
%<*class|def>
%% La geometría para la factura, su reverso, y el reporte.
%% Para la factura, el margen superior está definido desde el encabezado con los datos del cliente;
%%  el margen inferior está definido hasta el final de la descripción (o del cuadro de la factura si no hay descripción).
%%  Hay elementos que salen fuera de estos márgenes, como las firmas, la nota final, la numeración y la fecha.
%<class>\def\GeometriaFactura#1{\gdef\@GeometriaFactura{#1}}
%<class>\def\GeometriaReverso#1{\gdef\@GeometriaReverso{#1}}
%<class>\def\GeometriaReporte#1{\gdef\@GeometriaReporte{#1}}
%<def>\GeometriaFactura{letterpaper,vmargin={5.2cm,5.5cm},hmargin={1cm,1cm},footskip=0pt,  headsep=0pt,    headheight=0pt}
%<def>\GeometriaReverso{letterpaper,vmargin={1cm,1cm},    hmargin={1cm,1cm},footskip=0cm,  headsep=0pt,    headheight=0pt}
%<def>\GeometriaReporte{letterpaper,vmargin={2.8cm,2.5cm},hmargin={3cm,2cm},footskip=1.2cm,headsep=0.777cm,headheight=1.5cm}

%<class>%% Se utiliza '\@Impuesto' como el factor del impuesto, el cual viene predefinido en '\@Alicuota'.
%<class>\def\@ImpuestoG{\@AlicuotaG}
%<class>\def\@ImpuestoR{\@AlicuotaR}
%<class>\def\@ImpuestoA{\@AlicuotaA}
%<def>%% Factores para los distintos IVA:
%<def>\def\@AlicuotaG{0.16}
%<def>%% (quitado por no vigente)
%<def> % \def\@AlicuotaGDescBajo{0.03}
%<def> % \def\@AlicuotaGDescAlto{0.05}
%<def> % \def\@AlicuotaGDescLimite{2000000}
%<def>\def\@AlicuotaR{0.08}
%<def>\def\@AlicuotaA{0.27}
%<def>
%</class|def>
%<*def>
%% i) separador de miles, ii) separador decimal, iii) número de decimales para la moneda iv) y para los porcentajes, v) número de dígitos de la numeración de la factura, vi) formato de la fecha, vii) separadores para la fecha.
\npthousandsep{.}
\npdecimalsign{,}
\def\@DigitosMoneda{2}
\def\@DigitosPorc{2}
\def\@DigitosNumeracion{5}
\def\@FormatoFecha{ddmmyyyy}
\def\@FormatoFechaEntrada{dd-mm-yyyy}
\def\@FechaSeparadores{-}

%% La moneda:
\def\@Moneda{Bs.}
%% (quitado por no vigente)
 % \def\@MonedaBs{Bs.}
 % \def\@MonedaBsS{Bs.S}
 % \def\@MonedaBsF{Bs.F}
\setlength\@EspacioSignoMonetario{.666ex}

%% (quitado por no vigente)
%% Las fechas de la entrada de la reconversión monetaria:
 % \DTMsavedate{FechaBsS}{2018-08-20}
 % \DTMsavedate{FechaBs}{2019-02-15}

%% Al lado de cada ítem va la letra que designa el gravamen que aplica sobre él:
\def\@DenomAlicuotaG{G}
\def\@DenomAlicuotaR{R}
\def\@DenomAlicuotaA{A}
\def\@DenomAlicuotaE{(E)}

%% (quitado por no vigente)
%% Texto predeterminado para la nota interna por el descuento presidencial.
 % \def\@TextoDescuentoPresidencial{Se aplica un descuento del \@AlicuotaGDescPorc\% al IVA General seg\'un Decreto Presidencial 3.085 en Gaceta Oficial 41.239.}

%% Macros que guardan textos que corresponden a nombres de campos:
\def\@@Fecha{Fecha de emisi\'on:}
\def\@@FechaVencimiento{Fecha de vencimiento:}
\def\@@Control{N\textordmasculine\space CONTROL}
\def\@@Datos{Datos del cliente}
\def\@@Resumen{Resumen del servicio}
\def\@@DenomFactura{FACTURA}
\def\@@DenomCotiza{COTIZACI\'ON}
\def\@@RazonSocial{Raz\'on social:}
\def\@@Nombre{Nombre:}
\def\@@RIF{R.I.F.:}
\def\@@CI{C.I.:}
\def\@@Direccion{Direcci\'on fiscal:}
\def\@@Telefono{Tel\'efono:}
\def\@@Email{E-mail:}
\def\@@Contacto{Persona de contacto:}
\def\@@Proveedor{Proveedor:}
\def\@@Cliente{Cliente:}
\def\@@Emisor{Por el emisor}
\def\@@Conforme{Recib\'i conforme}
\def\@@Pagado{PAGADO}
\def\@@NumItem{\#}
\def\@@Cantidad{Cantidad}
\def\@@Descripcion{Descripci\'on}
\def\@@Precio{Precio}
\def\@@PrecioU{Precio U.}
\def\@@PrecioS{Subtotal}
\def\@@Descuento{Dcto.}
\def\@@IVA{IVA}
\def\@@PrecioT{Precio T.}
\def\@@AlicuotaG{IVA G}
%% (quitado por no vigente)
 % \def\@@AlicuotaGDesc{Descuento IVA G}
\def\@@AlicuotaR{IVA R}
\def\@@AlicuotaA{IVA A}
\def\@@BIG{BI G}
\def\@@BIR{BI R}
\def\@@BIA{BI A}
\def\@@BIE{Exento}% La base imponible exenta. ;)
\def\@@Total{TOTAL}
\def\@@DescuentoG{Descuento G}
\def\@@DescuentoR{Descuento R}
\def\@@DescuentoA{Descuento A}
\def\@@DescuentoE{Descuento E}
\def\@@SubtotalG{Subtotal G}
\def\@@SubtotalR{Subtotal R}
\def\@@SubtotalA{Subtotal A}
\def\@@SubtotalE{Subtotal E}

%% i) carácter para separar el teléfono del e-mail del cliente,
%% ii) carácter para los campos vacíos cuando se usa \ItemX,
%% iii) carácter para las celdas vacías cuando la descripción del ítem ocupa más que una línea.
%% iv) carácter de retorno de carro en el cuadro de la factura,
\def\@CaracterSeparaTelefonoEmail{$\diamond$}
\def\@CaracterItemXVacio{...}
\def\@CaracterCeldasVacias{\mbox{}\hfill$\cdot$\hfill\mbox{}}
\def\@CaracterCR{\flechaCR}
% \def\@CaracterCR{$\hookleftarrow$}
% \def\@CaracterCR{$\cwrightarcarrow$}% Usando el paquete 'fnsymbol'.
% \def\@CaracterCR{$\cwsearcarrow$}% Usando el paquete 'fnsymbol'.
\def\flechaCR{%
 \raisebox{-0.4ex}{%
  \begin{tikzpicture}
   \draw[line width=0.075ex] (0,0) arc (20:70:0.75ex) + (0,0) arc (-70:-20:0.75ex) ++ (0.8ex,0.55ex) arc (0:-100:1ex);
  \end{tikzpicture}}}

%% Separación entre el inicio de una columna en la tabla y el inicio del texto.
\tabcolsep=0.15cm

%% Separación entre la última letra y el carácter que indica que se continúa en la línea siguiente en la descripción cuando ocupa más de una línea.
\setlength{\@SeparaFinLineaDesc}{0.15cm}

%% Separación entre el teléfono y el carácter \@CaracterSeparaTelefonoEmail, y luego desde ese carácter hasta el email,
%% para cuando no hay resumen.
\setlength{\@SeparaTelefonoEmail}{0.666em}

%% Grosor de la línea de 'fancyhdr' para las cabeceras, en aquellas donde haya línea.
\setlength{\@GrosorLineaInfoPag}{0.4pt}

%% Posición en X desde la derecha y excedente en la línea de la firma.
\setlength{\@FirmaReportePosX}{2cm}
\setlength{\@FirmaReporteExcesoLinea}{1.5cm}
%</def>
%<class|def>
%<*class>
%% Fijamos en cero los descuentos totales (necesario).
\def\@DescuentoGporc{0}
\def\@DescuentoRporc{0}
\def\@DescuentoAporc{0}
\def\@DescuentoEporc{0}
\def\@DescuentoGnum{0}
\def\@DescuentoRnum{0}
\def\@DescuentoAnum{0}
\def\@DescuentoEnum{0}
\FPset\PrecioUtempx{1}

%% \Fecha, \FormatoFecha, \Credito, \Moneda, \RazonSocial, \Nombre, \RIF, \CI, \Contacto, \Dirección, \Telefono, \Email, \Conforme, \Emisor, \Proveedor.
\NewDocumentCommand{\Fecha}{s m g g}{%
 \IfBooleanTF{#1}
  {\AtEndPreamble{\DTMsavedate{FechaEmision}{#2-#3-#4}}}
  {\AtEndPreamble{\DTMsavedatex{FechaEmision}{#2}}}}
\def\@Fecha{\DTMusedate{FechaEmision}}

\NewDocumentCommand{\FormatoFecha}{o g g}{%
 \IfValueT{#1}{\gdef\@FormatoFechaEntrada{#1}}\IfValueT{#2}{\gdef\@FormatoFecha{#2}}\IfValueT{#3}{\gdef\@FechaSeparadores{#3}}}

\def\Credito#1{\gdef\@Credito{#1}\vencimientotrue}
\def\Moneda#1{\gdef\@Moneda{#1}}
\def\RazonSocial#1{\gdef\@RazonSocial{#1}}
\def\Nombre#1{\gdef\@RazonSocial{#1}\@Nombretrue}
\def\RIF#1{\gdef\@RIF{#1}}
\def\CI#1{\gdef\@RIF{#1}\let\@@RIF\@@CI\let\@CI\@RIF}
\def\Contacto#1{\gdef\@Contacto{#1}}
\def\Direccion#1{\gdef\@Direccion{#1}}
\def\Telefono#1{\gdef\@Telefono{#1}}
\def\Email#1{\gdef\@Email{#1}}
\def\Conforme#1{\gdef\@Conforme{#1}}
\def\Emisor#1{\gdef\@Emisor{#1}}
\def\Proveedor#1{\gdef\@Proveedor{#1}}

%% Los descuentos totales.
\NewDocumentCommand{\Descuento}{s m}{%
 \IfBooleanTF{#1}
  {\@DescGnumtrue\@DescRnumtrue\@DescAnumtrue\@DescEnumtrue
  \gdef\@DescuentoGnum{#2}\gdef\@DescuentoRnum{#2}\gdef\@DescuentoAnum{#2}\gdef\@DescuentoEnum{#2}}
  {\@DescGnumfalse\@DescRnumfalse\@DescAnumfalse\@DescEnumfalse
  \gdef\@DescuentoGporc{#2}\gdef\@DescuentoRporc{#2}\gdef\@DescuentoAporc{#2}\gdef\@DescuentoEporc{#2}}%
 \@DescGtrue\@DescRtrue\@DescAtrue\@DescEtrue\@DescuentoTotaltrue}

\NewDocumentCommand{\DescuentoG}{s m}{%
 \IfBooleanTF{#1}{\@DescGnumtrue\gdef\@DescuentoGnum{#2}}{\@DescGnumfalse\gdef\@DescuentoGporc{#2}}%
 \@DescGtrue\@DescuentoTotaltrue}

\NewDocumentCommand{\DescuentoR}{s m}{%
 \IfBooleanTF{#1}{\@DescRnumtrue\gdef\@DescuentoRnum{#2}}{\@DescRnumfalse\gdef\@DescuentoRporc{#2}}%
 \@DescRtrue\@DescuentoTotaltrue}

\NewDocumentCommand{\DescuentoA}{s m}{%
 \IfBooleanTF{#1}{\@DescAnumtrue\gdef\@DescuentoAnum{#2}}{\@DescAnumfalse\gdef\@DescuentoAporc{#2}}%
 \@DescAtrue\@DescuentoTotaltrue}

\NewDocumentCommand{\DescuentoE}{s m}{%
 \IfBooleanTF{#1}{\@DescEnumtrue\gdef\@DescuentoEnum{#2}}{\@DescEnumfalse\gdef\@DescuentoEporc{#2}}%
 \@DescEtrue\@DescuentoTotaltrue}

\def\Denom#1{\gdef\@Denominacion{#1}}%  La denominación puede ser 'FACTURA', 'COTIZACIÓN' u otra. Se predetermina una de estas dos según las opciones de la clase.

%% Número de la factura con \Num, número de control con \NumControl, y número de serie de control \NumSerieControl.
\def\Num#1{\gdef\@Num{\nplpadding{\@DigitosNumeracion}\npthousandsep{}\numprint{#1}}}
\def\NumControl#1{\gdef\@NumControl{\nplpadding{\@DigitosNumeracion}\npthousandsep{}\numprint{#1}}}%  Si el número de control no se especifica, toma el mismo que el de la factura
\def\NumSerieControl#1{\gdef\@NumSerieControl{#1}}%  Si el número de serie de control no se especifica, toma '00'.

%% Información de página: cabeceras de páginas o pies de página.
\def\InfoPagina#1{\gdef\@InfoPagina{#1}}

%% Texto para la nota final.
\NewDocumentCommand{\NotaFinal}{+m}{\gdef\@NotaFinal{#1}}
\NewDocumentCommand{\Condiciones}{+m}{\gdef\@NotaFinal{#1}\@ClassWarningCondiciones}

%% Advertencia para la migración de cualquier comando para condiciones a nota final.
\def\@ClassWarningCondiciones{\CLASSWARNING{Ya no son condiciones sino nota final, debe hacerse la migración a los nuevos comandos}}

%% Resumen opcional que va a la derecha de los datos del cliente.
\NewDocumentCommand{\Resumen}{+m}{\gdef\@Resumen{#1}}

\NewDocumentCommand{\NotaPrevia}{+m}{\gdef\@NotaPrevia{#1}}

%% Nota interna al cuadro de la factura y nota externa a él (que sale antes de \Descripcion).
\NewDocumentCommand{\NotaInterna}{+m}{%
 \ifdefvoid{\@NotaInterna}{}{\CLASSWARNING{Se sobreescribio' la nota interna previamente definida}}%
 \gdef\@NotaInterna{#1}}
\NewDocumentCommand{\NotaExterna}{+m}{\gdef\@NotaExterna{#1}}

\AtBeginDocument{%
 % Si no se especificó fecha, tomamos la de hoy. Además, advertimos cuando el documento se está compilando en una fecha distinta de la fecha establecida en el encabezado.
 \DTMifsaveddate{FechaEmision}{}{%
  \CLASSWARNING{No se especifico' fecha. Asumiendo la fecha de hoy}%
  \DTMsavedate{FechaEmision}{\the\year-\the\month-\the\day}}%
 \DTMsavenow{Ahora}%
 \DTMsaveddatediff{FechaEmision}{Ahora}{\AuxFecha}%
 \ifnum\AuxFecha=0%
 \else
  \CLASSWARNING{Fecha del documento \@Fecha\space distinta de la fecha de hoy}%
 \fi

%% (quitado por no vigente)
 % Establecemos el signo monetario dependiendo de la fecha de emisión de la factura, para efectos de la Reconversión Monetaria:
 % \ifx\@undefined\@Moneda
 %  \DTMsaveddatediff{FechaEmision}{FechaBsS}{\AuxFecha}%
 %  \ifnum\AuxFecha<0%
 %   \Moneda{\@MonedaBs}%
 %   \LetraNotaInterna{\fontsize{14}{18}}%
 %   \NotaInterna{\FPupn\TotalBsS{\Total{} 100000 div \@DigitosMoneda{} round}Según Decreto Presidencial 3.548, el total en Bolívares Soberanos es de {\bfseries \@MonedaBsS~\numprint{\TotalBsS}}}%
 %   \CLASSWARNING{Se establecio' el texto aclaratorio sobre el total en Boli'vares Soberanos como una nota interna}%
 %  \else
 %   \DTMsaveddatediff{FechaEmision}{FechaBs}{\AuxFecha}%
 %   \ifnum\AuxFecha>-1%
 %    \Moneda{\@MonedaBs}%
 %   \else
 %    \Moneda{\@MonedaBsS}%
 %    \LetraNotaInterna{\fontsize{14}{18}}%
 %    \NotaInterna{\FPupn\TotalBsF{\Total{} 100000 * \@DigitosMoneda{} round}El total en Bolívares Fuertes es de {\bfseries \@MonedaBsF~\numprint{\TotalBsF}}}%
 %    \CLASSWARNING{Se establecio' el texto aclaratorio sobre el total en Boli'vares Fuertes como una nota interna}%
 %   \fi
 %  \fi
 % \fi

 % Efectuamos distintos chequeos para dar errores o advertencias si faltó algo.
 \ifdefvoid{\@RazonSocial}{\CLASSERROR{Hay que definir la razo'n social}}{}%
 \if@reporte
  \ifdefvoid{\@Num}{\CLASSWARNING{No se especifico' el nu'mero de la factura. Puede dar error si el ti'tulo del reporte no se especifica que aparezca en las cabeceras}}{}%
  \ifdefvoid{\@Proveedor}{%
   \ifdefvoid{\@Emisor}
    {\CLASSERROR{Hay que definir el nombre del proveedor cuando se emite un reporte}}%
    {\CLASSWARNING{Se toma el emisor como el proveedor por no haberse especificado este u'ltimo}%
    \let\@Proveedor=\@Emisor}}
   {}%
 \fi
 \if@factura
  \ifdefvoid{\@RIF}{\CLASSERROR{Hay que definir el R.I.F}}{}%
  \ifdefvoid{\@Direccion}{\CLASSERROR{Hay que definir la direccio'n fiscal}}{}%
  \ifdefvoid{\@Telefono}{\CLASSWARNING{Pudiera definirse un nu'mero telefo'nico}}{}%
  \ifdefvoid{\@Email}{\CLASSWARNING{Pudiera definirse una direccio'n de correo electro'nico}}{}%
  \ifdefvoid{\@NotaFinal}{\CLASSWARNING{Pudiera definirse una nota final}}{}%
  \ifcotizacion
   \ifdefvoid{\@FondoFactura}{\CLASSWARNING{En la opcio'n para cotizacion se recomienda agregar el fondo}}{}%
  \fi
  \ifprefactura
   \ifdefvoid{\@FondoFactura}{\CLASSWARNING{En la opcio'n para prefactura se recomienda agregar el fondo}}{}%
  \fi
  \ifsinnumero
   \ifcotizacion
   \else
    \CLASSERROR{La opcio'n para no poner el nu'mero so'lo es compatible con las cotizaciones}%
   \fi
  \fi
  \ifdenom
   \ifsinnumero
   \else
    \ifdefvoid{\@Num}{\CLASSERROR{Hay que definir el nu'mero de la factura}}{}%
   \fi
  \fi
 \fi
 \ifdefvoid{\@Contacto}{\if@Nombre\else\CLASSWARNING{Pudiera definirse una persona de contacto}\fi}{}%

 % El cálculo de la fecha de vencimiento para cuando haya crédito o fecha de vencimiento explícita.
 \ifdefvoid{\@Credito}
  {\gdef\@FechaVencimiento{\DTMusedate{FechaEmision}}}
  {\DTMsaveddateoffsettojulianday{FechaEmision}{\@Credito}{\AuxFecha}%
   \DTMsavejulianday{FechaVencimiento}{\number\AuxFecha}%
   \gdef\@FechaVencimiento{\DTMusedate{FechaVencimiento}}}%
 \ifx\@undefined\@NumControl\let\@NumControl\@Num\fi
 \ifx\@undefined\@NumSerieControl\gdef\@NumSerieControl{00}\fi
 
 % Establecemos los porcentajes de las alícuotas, y redondeamos los descuentos.
 \FPupn\@AlicuotaGporc{\@AlicuotaG{} 100 * \@DigitosPorc{} round clip}%
 \FPupn\@AlicuotaRporc{\@AlicuotaR{} 100 * \@DigitosPorc{} round clip}%
 \FPupn\@AlicuotaAporc{\@AlicuotaA{} 100 * \@DigitosPorc{} round clip}%
 \FPset\@AlicuotaEporc{0}%
 \FPupn\@DescuentoGporc{\@DescuentoGporc{} \@DigitosPorc{} round clip}%
 \FPupn\@DescuentoRporc{\@DescuentoRporc{} \@DigitosPorc{} round clip}%
 \FPupn\@DescuentoAporc{\@DescuentoAporc{} \@DigitosPorc{} round clip}%
 \FPupn\@DescuentoEporc{\@DescuentoEporc{} \@DigitosPorc{} round clip}%
 \FPupn\@DescuentoGfactor{100 \@DescuentoGporc{} /}%
 \FPupn\@DescuentoRfactor{100 \@DescuentoRporc{} /}%
 \FPupn\@DescuentoAfactor{100 \@DescuentoAporc{} /}%
 \FPupn\@DescuentoEfactor{100 \@DescuentoEporc{} /}

 % Una vez definido todo en el preámbulo, los comandos usados para definir campos servirán para mostrar el campo que se definió con ese comando.
 \def\Moneda{\@Moneda}%
 \def\RazonSocial{\@RazonSocial}%
 \let\Nombre\RazonSocial
 \def\RIF{\@RIF}%
 \let\CI\RIF
 \def\Direccion{\@Direccion}%
 \def\Telefono{\@Telefono}%
 \def\Email{\@Email}%
 \def\Contacto{\@Contacto}%
 \def\Emisor{\@Emisor}%
 \def\Conforme{\@Conforme}%
 \def\Proveedor{\@Proveedor}%
 \def\Fecha{\@Fecha}%
 \def\FechaVencimiento{\@FechaVencimiento}%
 \def\Credito{\@Credito}%
 \def\Num{\@Num}%
 \def\NumSerieControl{\@NumSerieControl}%
 \def\NumControl{\@NumControl}%
 \def\NotaFinal{\@NotaFinal}%
 \def\Resumen{\@Resumen}%
 \def\NotaPrevia{\@NotaPrevia}%
 \def\NotaInterna{\@NotaInterna}%
 \def\NotaExterna{\@NotaExterna}%
 \def\TituloReporte{\@TituloReporte}}

%</class>
%<*class|def>
%% Posiciones (X,Y) de los objetos: i) fecha, ii) denominación, iii) nota final, iv) número de control, v) firmas.
%<*class>
\NewDocumentCommand{\PosFecha}{s m m}{
 \setlength{\@PosFechaX}{#2}
 \setlength{\@PosFechaY}{#3}
 \IfBooleanTF{#1}{\booltrue{FechaYarriba}}{\boolfalse{FechaYarriba}}
 \IfBeginWith{#2}{-}{\booltrue{FechaXnegativo}}{\boolfalse{FechaXnegativo}}
 \IfBeginWith{#3}{-}{\booltrue{FechaYnegativo}}{\boolfalse{FechaYnegativo}}}

%</class>
%<def>\PosFecha*{-0cm}{5cm}
%<*class>
\NewDocumentCommand{\PosDenom}{s m m}{
 \setlength{\@PosDenomX}{#2}
 \setlength{\@PosDenomY}{#3}
 \IfBooleanTF{#1}{\booltrue{DenomYarriba}}{\boolfalse{DenomYarriba}}
 \IfBeginWith{#2}{-}{\booltrue{DenomXnegativo}}{\boolfalse{DenomXnegativo}}
 \IfBeginWith{#3}{-}{\booltrue{DenomYnegativo}}{\boolfalse{DenomYnegativo}}}

%</class>
%<def>\PosDenom*{0cm}{5cm}
%<*class>
\NewDocumentCommand{\PosControl}{s m m}{
 \setlength{\@PosControlX}{#2}
 \setlength{\@PosControlY}{#3}
 \IfBooleanTF{#1}{\booltrue{ControlYarriba}}{\boolfalse{ControlYarriba}}
 \IfBeginWith{#2}{-}{\booltrue{ControlXnegativo}}{\boolfalse{ControlXnegativo}}
 \IfBeginWith{#3}{-}{\booltrue{ControlYnegativo}}{\boolfalse{ControlYnegativo}}}

%</class>
%<def>\PosControl*{0cm}{-2.4cm}
%<*class>
\NewDocumentCommand{\PosNotaFinal}{s m m m}{
 \setlength{\@PosNotaFinalXuno}{#2}
 \setlength{\@PosNotaFinalXdos}{#3}
 \setlength{\@PosNotaFinalY}{#4}
 \IfBooleanTF{#1}{\booltrue{NotaFinalYarriba}}{\boolfalse{NotaFinalYarriba}}
 \IfBeginWith{#4}{-}{\booltrue{NotaFinalYnegativo}}{\boolfalse{NotaFinalYnegativo}}}
\let\PosCondiciones\PosNotaFinal
%</class>
%<def>\PosNotaFinal{0cm}{0cm}{-5.3cm}
%% La nota final se define con un X_inicial, un X_final y un Y_inicial (Xo,Xf,Y)
%<*class>

\NewDocumentCommand{\PosFirmas}{m m m}{
 \setlength{\@PosFirmasXuno}{#1}
 \setlength{\@PosFirmasXdos}{#2}
 \setlength{\@PosFirmasY}{#3}
 \IfBeginWith{#3}{-}{\booltrue{FirmasYnegativo}}{\boolfalse{FirmasYnegativo}}}
%</class>
%<def>\PosFirmas{5cm}{1cm}{-2.4cm}
%% La firma se define con un X_inicial, un X_final y un Y_inicial (Xo,Xf,Y)

%% Tamaño y grosor de la línea de las firmas.
%<*class>
\def\LineaFirmas#1#2{
 \setlength{\@TamLineaFirmas}{#1}
 \setlength{\@GrosorLineaFirmas}{#2}}
%</class>
%<def>\LineaFirmas{5.5cm}{0.7pt}

%% Separación mínima horizontal entre los totales en modo de filas y distancia vertical entre esas líneas, fija.
%<class>\def\SepFilas#1#2{\setlength{\@SeparaMinTotales}{#1}\setlength{\@SeparaFilas}{#2}}
%<def>\SepFilas{0.8cm}{2.5pt}

%% Separación vertical entre los elementos de la factura (entre el encabezado y el cuadro de la 'factura', luego desde dicho cuadro hasta la descripción, o hasta los totales en modo de filas).
%<class>\def\SepVertical#1{\setlength{\@SeparaVertical}{#1}}
%<def>\SepVertical{0.2cm}

%% Separación horizontal entre la caja de los datos del cliente y la caja del resumen.
%<class>\def\SepDatosResumen#1{\setlength{\@SeparaDatosResumen}{#1}}
%<def>\SepDatosResumen{0.2cm}

%% Separación horizontal interna entre la línea y el texto en los datos del cliente y en el resumen.
%<class>\def\SepEncabezado#1{\setlength{\@SeparaInternaEncabezado}{#1}}
%<def>\SepEncabezado{6pt}

%% Separación interna entre la línea y el texto de la nota previa.
%<class>\def\SepNotaPrevia#1{\setlength{\@SeparaInternaNotaPrevia}{#1}}
%<def>\SepNotaPrevia{6pt}

%% Separación extra entre los elementos en la factura.
%<class>\def\SepItemsExtra#1{\setlength{\@SeparaItemsExtra}{#1}}
%<def>\SepItemsExtra{0pt}

%% Separación interna entre la línea y el texto de la descripción.
%<class>\def\SepDescripcion#1{\setlength{\@SeparaInternaDescripcion}{#1}}
%<def>\SepDescripcion{6pt}

%<class>\def\SepNotaFinal#1{\setlength{\@SeparaInternaNotaFinal}{#1}}
%<class>\let\SepCondiciones\SepNotaFinal
%<def>\SepNotaFinal{5pt}

%% Grosor de la línea de las cajas del encabezado, curvatura de la esquina y sombra.
%% (si pedimos linea de 0pt, todo queda en cero y no se abre el ambiente TikZ).
%<class>\NewDocumentCommand{\LineaEncabezado}{m g g}{
%<class> \setlength{\@GrosorLineaEncabezado}{#1}
%<class> \IfValueT{#2}{\setlength{\@EsquinaLineaEncabezado}{#2}}
%<class> \IfValueT{#3}{\setlength{\@SombraEncabezado}{#3}}}
%<def>\LineaEncabezado{0.4pt}{10pt}{0pt}

%% Grosor de la línea de la nota previa, curvatura de la esquina y sombra.
%% (si pedimos linea de 0pt, todo queda en cero y no se abre el ambiente TikZ).
%<class>\NewDocumentCommand{\LineaNotaPrevia}{m g g}{
%<class> \setlength{\@GrosorLineaNotaPrevia}{#1}
%<class> \IfValueT{#2}{\setlength{\@EsquinaLineaNotaPrevia}{#2}}
%<class> \IfValueT{#3}{\setlength{\@SombraNotaPrevia}{#3}}}
%<def>\LineaNotaPrevia{0.4pt}{7pt}{0pt}

%% Grosor de la línea de la descripción, curvatura de la esquina y sombra.
%% (si pedimos linea de 0pt, todo queda en cero y no se abre el ambiente TikZ).
%<class>\NewDocumentCommand{\LineaDescripcion}{m g g}{
%<class> \setlength{\@GrosorLineaDescripcion}{#1}
%<class> \IfValueT{#2}{\setlength{\@EsquinaLineaDescripcion}{#2}}
%<class> \IfValueT{#3}{\setlength{\@SombraDescripcion}{#3}}}
%<def>\LineaDescripcion{0.4pt}{0pt}{4pt}

%% Grosor de la línea de la nota final, curvatura de la esquina y sombra.
%% (si pedimos linea de 0pt, todo queda en cero y no se abre el ambiente TikZ).
%<class>\NewDocumentCommand{\LineaNotaFinal}{m g g}{
%<class> \setlength{\@GrosorLineaNotaFinal}{#1}
%<class> \IfValueT{#2}{\setlength{\@EsquinaLineaNotaFinal}{#2}}
%<class> \IfValueT{#3}{\setlength{\@SombraNotaFinal}{#3}}}
%<class>\let\LineaCondiciones\LineaNotaFinal
%<def>\LineaNotaFinal{0pt}{0pt}{0pt}

%% El espacio vertical que hay que avanzar sólo en la primera página del reporte antes de empezar a escribir sobre él. Usable cuando la primera página tiene un membrete y las demás no, es decir, cuando se usa \FondoReporte y no \FondoReporte*.
%<class>\def\InicioReporte#1{\setlength{\@InicioReporte}{#1}}
%<def>\InicioReporte{1.6cm}

%% Tipos de letra para cada elemento posible:
%% Esta es la letra de la descripción de los items; y el espacio \baselineskip de *toda* la tabla de la factura se fija en este comando únicamente (se predeterminó 18pt).
%<class>\def\LetraItems#1{\gdef\@LetraItems{#1\selectfont}}
%<def>\LetraItems{\fontsize{14}{18}}

%% Tipo de letra para la nota interna:
%<class>\def\LetraNotaInterna#1{\gdef\@LetraNotaInterna{#1\selectfont}}
%<def>\LetraNotaInterna{\fontsize{11}{13}\itshape}

%% Tipo de letra para la nota externa:
%<class>\def\LetraNotaExterna#1{\gdef\@LetraNotaExterna{#1\selectfont}}
%<def>\LetraNotaExterna{\fontsize{14}{16.8}}

%% Con este comando se cambia la etra del título de la tabla, es decir, los textos 'Cantidad, 'Descripción', etc...
%<class>\def\LetraTitColumnas#1{\gdef\@LetraTitColumnas{#1\selectfont}}
%<def>\LetraTitColumnas{\fontsize{10}{10}\bfseries}

%% Tipo de letra de los títulos de los totales, por ejemplo los textos: 'B.I. G', 'IVA G'.
%<class>\def\LetraTitTotales#1{\gdef\@LetraTitTotales{#1\selectfont}}
%<def>\LetraTitTotales{\fontsize{12}{12}\bfseries}

%% Letra con la que salen los totales parciales, es decir, los números, exceptuando el total general.
%<class>\def\LetraNumTotales#1{\gdef\@LetraNumTotales{#1\selectfont}}
%<def>\LetraNumTotales{\fontsize{14}{12}}

%% Letra con la que sale la palabra 'TOTAL'.
%<class>\def\LetraTitTotal#1{\gdef\@LetraTitTotal{#1\selectfont}}
%<def>\LetraTitTotal{\fontsize{12}{12}\bfseries}

%% Letra para el total general, el monto.
%<class>\def\LetraNumTotal#1{\gdef\@LetraNumTotal{#1\selectfont}}
%<def>\LetraNumTotal{\fontsize{14}{14}\bfseries}

%% Tipo de letra para los títulos de las cajas del encabezado, por ejemplo: 'Datos del cliente'.
%<class>\def\LetraTitEnc#1{\gdef\@LetraTitEnc{#1\selectfont}}
%<def>\LetraTitEnc{\fontsize{11}{12.5}\slshape}

%% Tipo de letra para el texto en el encabezado de la factura.
%<class>\def\LetraEncFactura#1{\gdef\@LetraEncFactura{#1\selectfont}}
%<def>\LetraEncFactura{\fontsize{11}{12.5}}

%% Tipo de letra para el texto en el encabezado del reporte.
%<class>\def\LetraEncReporte#1{\gdef\@LetraEncReporte{#1\selectfont}}
%<def>\LetraEncReporte{\fontsize{13}{16}}

%% Letra para los montos individuales de la factura y las cantidades.
%<class>\def\LetraNumeros#1{\gdef\@LetraNumeros{#1\selectfont}}
%<def>\LetraNumeros{\fontsize{14}{14}}

%% El tipo de IVA ('G', 'R', 'A' o '(E)') también tiene su tipo de letra.
%<class>\def\LetraTipoIVA#1{\gdef\@LetraTipoIVA{#1\selectfont}}
%<def>\LetraTipoIVA{\fontsize{12}{12}}

%% Letra para la descripción larga (comando \Descripcion).
%<class>\def\LetraDescripcion#1{\gdef\@LetraDescripcion{#1\selectfont}}
%<def>\LetraDescripcion{\fontsize{14}{16.8}}

%% Tipo de letra para el texto bajo las líneas para firmar.
%<class>\def\LetraFirmas#1{\gdef\@LetraFirmas{#1\selectfont}}
%<def>\LetraFirmas{\fontsize{12}{12}}

%% Tipo de letra para la nota previa.
%<class>\def\LetraNotaPrevia#1{\gdef\@LetraNotaPrevia{#1\selectfont}}
%<def>\LetraNotaPrevia{\fontsize{11}{12.5}}

%% Tipo de letra para la nota final.
%<class>\def\LetraNotaFinal#1{\gdef\@LetraNotaFinal{#1\selectfont}}
%<class>\let\LetraCondiciones\LetraNotaFinal
%<def>\LetraNotaFinal{\fontsize{12}{12}}

%% Letra para la denominación del documento ('FACTURA', 'COTIZACIÓN',...).
%<class>\def\LetraDenom#1{\gdef\@LetraDenom{#1\selectfont}}
%<def>\LetraDenom{\fontsize{14}{14}\bfseries}

%% Letra para la numeración de la factura.
%<class>\def\LetraNumeracion#1{\gdef\@LetraNumeracion{#1\selectfont}}
%<def>\LetraNumeracion{\fontsize{14}{14}\bfseries\color{red}}

%% Letra para las fechas (emisión y vencimiento).
%<class>\def\LetraFecha#1{\gdef\@LetraFecha{#1\selectfont}}
%<def>\LetraFecha{\fontsize{14}{15}}

%% Letra para el título del reporte.
%<class>\def\LetraTitReporte#1{\gdef\@LetraTitReporte{#1\selectfont}}
%<def>\LetraTitReporte{\fontsize{16}{22}\bfseries}

%% Letra para las cabeceras de las páginas en el modo reporte.
%<class>\def\LetraInfoPagina#1{\gdef\@LetraInfoPagina{#1\selectfont}}
%<def>\LetraInfoPagina{\fontsize{12}{12}\slshape}

%% Letra para el reverso de la factura.
%<class>\def\LetraReverso#1{\gdef\@LetraReverso{#1\selectfont}}
%<def>\LetraReverso{\fontsize{10}{12}}

%% Letra para el reporte.
%<class>\def\LetraReporte#1{\gdef\@LetraReporte{#1\selectfont}}
%<def>\LetraReporte{\fontsize{12}{20}}

%</class|def>
%<*class>
%% Comando para incluir la firma escaneada en el sitio donde va la línea de la firma del emisor.
\NewDocumentCommand{\FirmaFactura}{m O{0pt} O{0pt}}{% \FirmaFactura{imagen}[x-corr][y-corr]
 \AtEndPreamble{
  \if@factura
   \RequirePackage{graphicx}
   \ifx&#2&\else\setlength{\@FirmaFacturaXcorr}{#2}\fi
   \ifx&#3&\else\setlength{\@FirmaFacturaYcorr}{#3}\fi
   \savebox{\@CajaFirmaEmisor}{#1}
   \AtBeginShipoutNext{%
    \begin{textblock*}{\@TamLineaFirmas}%
     (\textwidth+\oddsidemargin+1in-\@TamLineaFirmas/2-\wd\@CajaFirmaEmisor/2-\@PosFirmasXdos+\@FirmaFacturaXcorr,\@PosFirmasY-\ht\@CajaFirmaEmisor-\dp\@CajaFirmaEmisor+\@FirmaFacturaYcorr)%
     \usebox{\@CajaFirmaEmisor}%
    \end{textblock*}}%
  \fi}}

\NewDocumentCommand{\FirmaReporte}{O{} O{0pt} O{0pt} G{\@Proveedor} O{\@FirmaReportePosX} O{\@FirmaReporteExcesoLinea}}{% \Firma[imagen][x-corr][y-corr]{texto}[posición en X desde la derecha][tamaño linea]
 \renewcommand{\baselinestretch}{1}\selectfont
 \newif\if@Firma@Dibujo
 \newbox{\TextoFirma}%
 \newbox{\DibujoFirma}%
 \ifx&#1&\else\savebox{\DibujoFirma}{#1}\@Firma@Dibujotrue\fi
 \setlength{\@FirmaReporteXcorr}{#2}%
 \setlength{\@FirmaReporteYcorr}{#3}%
 \savebox{\TextoFirma}{#4}%
 \setlength{\@FirmaReporteTextPos}{#5}%
 \setlength{\@FirmaReporteTamLinea}{#6}%
 \@LetraFirmas
 \if@Firma@Dibujo
  \newlength{\@FirmaReporteoddeven}%
  \setlength{\@FirmaReporteFirmaPos}{\zposy{Firma@zref@PosY}sp}%
 \fi
 \ifdim\wd\DibujoFirma>\wd\TextoFirma
  \setlength{\@FirmaReporteTamLinea}{\@FirmaReporteTamLinea+\wd\DibujoFirma}%
 \else
  \setlength{\@FirmaReporteTamLinea}{\@FirmaReporteTamLinea+\wd\TextoFirma}%
 \fi
 \par
 \vspace*{\ht\DibujoFirma}%
 \vspace*{\dp\DibujoFirma}%
 \hfill
 \parbox{\@FirmaReporteTamLinea}{%
  \if@Firma@Dibujo\zsaveposy{Firma@zref@PosY}\fi
  \rule{\@FirmaReporteTamLinea}{\@GrosorLineaFirmas}\\
  \centering{\usebox{\TextoFirma}}}%
 \hspace*{\@FirmaReporteTextPos}\par
 \if@Firma@Dibujo
  \mbox{}%
  \par
  \ifodd\c@page
   \setlength{\@FirmaReporteoddeven}{\oddsidemargin}%
  \else
   \setlength{\@FirmaReporteoddeven}{\evensidemargin}%
  \fi
  \begin{textblock*}{\@FirmaReporteTamLinea}
  (\textwidth+\@FirmaReporteoddeven+1in-0.5\@FirmaReporteTamLinea-0.5\wd\DibujoFirma-\@FirmaReporteTextPos+\@FirmaReporteXcorr,
   \paperheight-\@FirmaReporteFirmaPos-\ht\DibujoFirma-\dp\DibujoFirma+\@FirmaReporteYcorr)
   \noindent\usebox{\DibujoFirma}%
  \end{textblock*}
 \fi}

%% Con \FondoFactura le damos el nombre del archivo que contiene el formato de la factura, por ejemplo para compilaciones con 'prefactura', 'cotizacion' o para incluir una imagen de fondo como 'marca de agua'.
\newcommand{\FondoFactura}[1]{
 \AtEndPreamble{
  \if@factura
   \RequirePackage{graphicx}
   \@FondoFacturatrue
   \gdef\@FondoFactura{#1}
  \fi}}

%% Con \FondoReporte le damos el nombre del archivo que contiene el formato del reporte. Existen dos posibilidades: que dicho fondo salga en todas las páginas o sólo en la primera.
\NewDocumentCommand{\FondoReporte}{s m}{
 \AtEndPreamble{
  \if@reporte
   \@FondoReportetrue
   \RequirePackage{graphicx}
   \gdef\@FondoReporte{#2}
   \IfBooleanT{#1}{\@FondoReporteStartrue}% Con * sale en todas las páginas; sin * sale sólo en la primera página.
  \fi}}

%% Comando para establecer desde el preámbulo el estilo de página (\pagestyle) del reporte, tanto para la primera página como para las siguientes.
\NewDocumentCommand{\EstiloPagReporte}{o m}{
 \gdef\@EstiloPagReporteResto{#2}
 \IfValueTF{#1}{\gdef\@EstiloPagReporteUno{#1}}{\gdef\@EstiloPagReporteUno{\@EstiloPagReporteResto}}}

%% El título para el reporte.
\NewDocumentCommand{\TituloReporte}{s m}{
 \gdef\@TituloReporte{#2}%
 \IfBooleanTF{#1}{% Con '*' sale el título en las cabeceras o pies. Sin '*' no.
  \gdef\@InfoPagina{\let\\\relax\@InfoPaginaTitulo}}%
 {\let\@InfoPagina\@InfoPaginaNoTitulo}}

%% Para cada cliente o para cada personalización se puede tener un archivo .def o .tex con su nombre y ciertas macros cargadas. Ese archivo lo ponemos incluir con '\Cliente' en lugar de usar \input, y la extensión deberá ser .def o .tex.
\newcommand{\Cliente}[1]{%
 \IfFileExists{#1.def}{\input{#1.def}}{
  \IfFileExists{#1.tex}{\input{#1.tex}}{
   \CLASSERROR{No existe #1.def ni #1.tex}}}}

\input{factura.def}%  Cargamos los valores predeterminados.

%% Comando para escribir el encabezado en la factura.
\newcommand\@EncabezadoFactura{%
 \ifdefvoid{\@Resumen}{\@EscribeDatos}{\@EscribeDatosResumen}%
 \vspace*{\@SeparaVertical}%
 \ifexpandecuadro\else\vfill\fi}

%% Comando para escribir el encabezado en el reporte.
\newcommand\@EncabezadoReporte{%
 \vspace*{-\topskip}%
 \if@FondoReporte
  \if@FondoReporteStar
  \else
   \vspace*{\@InicioReporte}%
  \fi
 \fi
 \@EscribeDatosReporte
 \@LetraReporte}

%% El ambiente 'factura', usando el paquete 'environ' para su definición, y el comando auxiliar \@FacturaAuxCmd.
%% Gracias a Heiko Oberdiek y a David Carlisle (Thanks to Heiko Oberdiek and David Carlisle).
%% http://tex.stackexchange.com/questions/108565/know-the-cell-width-with-tabular-or-include-a-paragraph-box-without-specifying
\NewEnviron{factura}{%
 \expandafter\@FacturaAuxCmd\expandafter{\BODY}%
 \ifnum\value{page}=1\else\CLASSERROR{La factura ocupa mas que una pagina}\fi}

\NewEnviron{reverso}{%
 \if@factura
   \expandafter\newgeometry\expandafter{\@GeometriaReverso}%
   \topskip1sp% ¡¡No puedo poner 0 porque se genera una página en blanco extra!!
  \begin{minipage}[t][\textheight-\pagetotal]{\linewidth}%
   \thispagestyle{empty}%
   \@LetraReverso\BODY
  \end{minipage}
  \clearpage
 \fi}

%% Comando auxiliar para agregar un carácter al final de cada línea en la descripción dentro del cuadro de la factura si ocupa ésta más de una línea.
%% Gracias a Gustavo Mezzetti y a David Carlisle (Thanks to Gustavo Mezzetti and David Carlisle).
%% http://tex.stackexchange.com/questions/296035/insert-a-special-character-at-each-end-of-line-in-a-paragraph
%% http://tex.stackexchange.com/questions/248944/how-to-put-before-lines
\newcommand{\@DescripcionAuxCmd}[2]{%
 \immediate\write\@auxout{\gdef\noexpand\@TamDescripcion{\the\linewidth}}% Para poder establecer la línea diagonal hay que saber el tamaño del campo de la descripción.
 \ifimpdesc\settowidth{\@TempDimen}{\@EscribeTipoDeImpuesto{#2}}\fi
 \savebox{\@CajaPrueba}{%
  \parbox[t]{\linewidth}{%
   \David@XStart
   \ifcentradesc\centering\arraybackslash\fi
   #1%
   \David@XEnd
   \ifdim\@TempDimen=0pt%
   \else
    \hfill\kern\@SeparaFinLineaDesc\@EscribeTipoDeImpuesto{#2}%
   \fi
   \par\global\count1=\prevgraf}}%
 \ifsinmarcas
  \usebox{\@CajaPrueba}%
 \else
  \ifnum\count1=1%
   \usebox{\@CajaPrueba}%
  \else
   \ifdim\@TempDimen>\wd\@CajaCaracterCR
   \else
    \setlength{\@TempDimen}{\wd\@CajaCaracterCR}%
   \fi
   \parbox[t]{\linewidth-\@TempDimen-\@SeparaFinLineaDesc}{%
    \David@XStart
    \ifcentradesc\centering\arraybackslash\fi
    #1%
    \David@XEnd
    \par
    \global\count1=\prevgraf% Esto lo necesitamos en otro lado para calcular el espacio extra en los ítems.
    \count0=\prevgraf
    \count2=\prevgraf
    \loop
     \ifnum\count0=\count2
      \ifimpdesc
       \vskip-\baselineskip
       \mbox{}\hfill
       \rlap{%
        \hspace*{\@TempDimen}%
        \hspace*{\@SeparaFinLineaDesc}%
        \llap{\@EscribeTipoDeImpuesto{#2}}}%
      \fi
     \else
      \mbox{}\hfill
      \rlap{%
       \hspace*{\@TempDimen}%
       \hspace*{\@SeparaFinLineaDesc}%
       \hspace*{-\wd\@CajaCaracterCR}%
       \usebox{\@CajaCaracterCR}}%
     \fi
     \vskip-2\baselineskip
     \advance\count0 -1
     \ifnum\count0 > 0
    \repeat
    \kern\count2\baselineskip\mbox{}%
    \strut}%
  \fi
 \fi}

%% Comandos auxiliares para colocar en \parbox cada celda.
\newcommand{\@NumItemAuxCmd}[1]{%
 \savebox{\@CajaPrueba}{#1}%
 \ifdim\wd\@CajaPrueba > \@TamNumItem \global\@TamNumItem=\wd\@CajaPrueba\fi
 \parbox[t]{\@TamNumItem}{#1}}

\newcommand{\@CantidadAuxCmd}[1]{%
 \savebox{\@CajaPrueba}{#1}%
 \ifdim\wd\@CajaPrueba > \@TamCantidad \global\@TamCantidad=\wd\@CajaPrueba\fi
 \parbox[t]{\@TamCantidad}{#1}}

\newcommand{\@PrecioUAuxCmd}[1]{%
 \savebox{\@CajaPrueba}{#1}%
 \ifdim\wd\@CajaPrueba > \@TamPrecioU \global\@TamPrecioU=\wd\@CajaPrueba\fi
 \parbox[t]{\@TamPrecioU}{#1}}

\newcommand{\@PrecioSAuxCmd}[1]{%
 \savebox{\@CajaPrueba}{#1}%
 \ifdim\wd\@CajaPrueba > \@TamPrecioS \global\@TamPrecioS=\wd\@CajaPrueba\fi
 \parbox[t]{\@TamPrecioS}{#1}}

\newcommand{\@DescuentoAuxCmd}[1]{%
 \savebox{\@CajaPrueba}{#1}%
 \ifdim\wd\@CajaPrueba > \@TamDescuento \global\@TamDescuento=\wd\@CajaPrueba\fi
 \parbox[t]{\@TamDescuento}{#1}}

\newcommand{\@IVAAuxCmd}[1]{%
 \savebox{\@CajaPrueba}{#1}%
 \ifdim\wd\@CajaPrueba > \@TamIVA \global\@TamIVA=\wd\@CajaPrueba\fi
 \parbox[t]{\@TamIVA}{#1}}

\newcommand{\@PrecioTAuxCmd}[1]{%
 \savebox{\@CajaPrueba}{#1}%
 \ifdim\wd\@CajaPrueba > \@TamPrecioT \global\@TamPrecioT=\wd\@CajaPrueba\fi
 \parbox[t]{\@TamPrecioT}{#1}}

%% Presentación de totales en modo de columnas.
\newcommand{\@EscribeTotalCol}[1]{%
 \expandafter\FPifeq\csname Desc#1\endcsname{0}%
  \ifsinivaexpresado\else\@EscribeTotalColBI{#1}\fi
 \else
  \@EscribeTotalColSubt{#1}%
  \@EscribeTotalColDesc{#1}%
  \ifsinivaexpresado\else\@EscribeTotalColBI{#1}\fi
 \fi}

%% Forma agupada para la presentación de totales en modo de columnas.
\newcommand{\@EscribeTotalColAgrupado}{%
 \if@Exento\if@DescE\@EscribeTotalColSubt{E}\fi\fi
 \if@BIR\if@DescR\@EscribeTotalColSubt{R}\fi\fi
 \if@DescG\@EscribeTotalColSubt{G}\fi
 \if@BIA\if@DescA\@EscribeTotalColSubt{A}\fi\fi
 \if@Exento\if@DescE\@EscribeTotalColDesc{E}\fi\fi
 \if@BIR\if@DescR\@EscribeTotalColDesc{R}\fi\fi
 \if@DescG\@EscribeTotalColDesc{G}\fi
 \if@BIA\if@DescA\@EscribeTotalColDesc{A}\fi\fi
 \ifsinivaexpresado
 \else
  \if@Exento\@EscribeTotalColBI{E}\fi
  \if@BIR\@EscribeTotalColBI{R}\fi
  \@EscribeTotalColBI{G}%
  \if@BIA\@EscribeTotalColBI{A}\fi
  \if@BIR\@EscribeTotalColAlicuota{R}\fi
  \@EscribeTotalColAlicuota{G}%
  \if@BIA\@EscribeTotalColAlicuota{A}\fi
  %% (quitado por no vigente)
  % \ifbool{Electro}{\@EscribeTotalColAlicuotaGDesc}{}%
 \fi}

\newcommand{\@EscribeTotalColBI}[1]{%
 \@LetraTitTotales \csname @@BI#1\endcsname
  \ifx#1E%
  \else
   \ifG\ \hfill(\csname @Alicuota#1porc\endcsname\%)\else\mbox{}\fi
  \fi
 & \@LetraNumTotales\@SignoMonetarioAux\numprint{\csname BI#1\endcsname} \\}

\newcommand{\@EscribeTotalColAlicuota}[1]{%
 {\@LetraTitTotales \csname @@Alicuota#1\endcsname\ \hfill (\csname @Alicuota#1porc\endcsname\%)}
 & \@LetraNumTotales\@SignoMonetarioAux\numprint{\csname Alicuota#1\endcsname} \\}

%% (quitado por no vigente)
 % \newcommand{\@EscribeTotalColAlicuotaGDesc}{%
 % {\@LetraTitTotales \@@AlicuotaGDesc\ \hfill (\@AlicuotaGDescPorc\%)}
 % & \@LetraNumTotales\@SignoMonetarioAux\numprint{\AlicuotaGDesc} \\}

\newcommand{\@EscribeTotalColSubt}[1]{%
 {\@LetraTitTotales \csname @@Subtotal#1\endcsname}
 & \@LetraNumTotales\@SignoMonetarioAux\numprint{\csname Subt#1\endcsname} \\}

\newcommand{\@EscribeTotalColDesc}[1]{%
 {\@LetraTitTotales \csname @@Descuento#1\endcsname
 \csname if@Desc#1num\endcsname \else \ \hfill (\csname @Descuento#1porc\endcsname\%)\fi}
 & \@LetraNumTotales\@SignoMonetarioAux\numprint{\csname Desc#1\endcsname} \\}

%% Presentación de los totales en modo filas.
\newcommand{\@EscribeTotalFilasBI}[1]{%
 \@EscribeTotalFilasBIizq{#1}: & \@EscribeTotalFilasBIder{#1}}

\newcommand{\@EscribeTotalFilasAlicuota}[1]{%
 \@EscribeTotalFilasAlicuotaizq{#1}: & \@EscribeTotalFilasAlicuotader{#1}}

%% (no implementado)
 % \newcommand{\@EscribeTotalFilasDesc}[1]{%
 %  \@EscribeTotalFilasDescizq{#1}: & \@EscribeTotalFilasDescder{#1}}

\newcommand{\@EscribeTotalFilasSoloBI}[1]{%
 {\@LetraTitTotales\@EscribeTotalFilasBIizq{#1}:} {\@LetraNumTotales\@EscribeTotalFilasBIder{#1}}}

%% (no implementado)
 % \newcommand{\@EscribeTotalFilasSoloDesc}[1]{%
 % {\@LetraTitTotales\@EscribeTotalFilasDescizq{#1}:} {\@LetraNumTotales\@EscribeTotalFilasDescder{#1}}}

\newcommand{\@EscribeTotalFilasSoloAlicuota}[1]{%
 {\@LetraTitTotales\@EscribeTotalFilasAlicuotaizq{#1}:} {\@LetraNumTotales\@EscribeTotalFilasAlicuotader{#1}}}

\newcommand{\@EscribeTotalFilasBIizq}[1]{%
 \@LetraTitTotales\csname @@BI#1\endcsname
 \ifx#1G%
  \ifG \ (\@AlicuotaGporc\%)\fi%  Si hay otra B.I. que no sea general, escribimos el porcentaje explícito en la general.
 \else
  \ifx#1E%
  \else
  \ (\csname @Alicuota#1porc\endcsname\%)%
  \fi
 \fi}

\newcommand{\@EscribeTotalFilasBIder}[1]{%
 \@LetraNumTotales\@SignoMonetarioAux\numprint{\csname BI#1\endcsname}}

\newcommand{\@EscribeTotalFilasAlicuotaizq}[1]{%
 \@LetraTitTotales\csname @@Alicuota#1\endcsname \ (\csname @Alicuota#1porc\endcsname\%)}

\newcommand{\@EscribeTotalFilasAlicuotader}[1]{%
 \@LetraNumTotales\@SignoMonetarioAux\numprint{\csname Alicuota#1\endcsname}}

%% (no implementado)
 % \newcommand{\@EscribeTotalFilasDescizq}[1]{%
 %  \@LetraTitTotales\csname @@Descuento#1\endcsname \ (\csname @Descuento#1porc\endcsname\%)}

 % \newcommand{\@EscribeTotalFilasDescder}[1]{%
 %  \@LetraNumTotales\@SignoMonetarioAux\numprint{\csname Desc#1\endcsname}}

\newcommand{\@EscribeTotalFilasAuxTotal}{\@LetraTitTotal\@@Total: & \@LetraNumTotal\@SignoMonetarioAux\numprint{\Total}}
 
\newcommand{\@EscribeTotalFilas}[2][]{%
 \begin{tabular}[b]{@{}l@{\extracolsep{1ex}}r@{}}% ¡Esta tabla tiene que abrir hacia abajo para que la separación entre las filas se pueda colocar con \@SeparaFilas!
  \csname @EscribeTotalFilasBI\endcsname{#2}%
  \ifx#2E%
   \ifx#1T%
    \\[\@SeparaFilas]%
    \@EscribeTotalFilasAuxTotal
   \fi
  \else
   \\[\@SeparaFilas]%
   \csname @EscribeTotalFilasAlicuota\endcsname{#2}%
  \fi
 \end{tabular}}

%% Comando auxiliar para escribir el signo monetario cuando aplica
\newcommand{\@SignoMonetarioAux}{%
 \ifsinmonedatotales\else\@Moneda\hspace*{\@EspacioSignoMonetario}\fi}

\newcommand{\@GeneraCajasTotalesFilas}{%
 \if@Exento
  \newbox{\@CajaBIE}\newbox{\@CajaBIET}%
  \savebox{\@CajaBIE}{\@EscribeTotalFilasSoloBI{E}}%
  \savebox{\@CajaBIET}{\@EscribeTotalFilas[T]{E}}%
 \fi
 \if@BIR
  \addtocounter{@IVANoGeneral}{1}%
  \newbox{\@CajaBIR}\newbox{\@CajaAliR}\newbox{\@CajaBIAliR}%
  \savebox{\@CajaBIR}{\@EscribeTotalFilasSoloBI{R}}%
  \savebox{\@CajaAliR}{\@EscribeTotalFilasSoloAlicuota{R}}%
  \savebox{\@CajaBIAliR}{\@EscribeTotalFilas{R}}%
 \fi
 \newbox{\@CajaBIG}\newbox{\@CajaAliG}\newbox{\@CajaBIAliG}%
 \savebox{\@CajaBIG}{\@EscribeTotalFilasSoloBI{G}}%
 \savebox{\@CajaAliG}{\@EscribeTotalFilasSoloAlicuota{G}}%
 \savebox{\@CajaBIAliG}{\@EscribeTotalFilas{G}}%
 \if@BIA
  \addtocounter{@IVANoGeneral}{1}%
  \newbox{\@CajaBIA}\newbox{\@CajaAliA}\newbox{\@CajaBIAliA}%
  \savebox{\@CajaBIA}{\@EscribeTotalFilasSoloBI{A}}%
  \savebox{\@CajaAliA}{\@EscribeTotalFilasSoloAlicuota{A}}%
  \savebox{\@CajaBIAliA}{\@EscribeTotalFilas{A}}%
 \fi}

\newcommand{\@EscribeTotalesFilasGET}{%
 \hspace*{\@SepTotales}\usebox{\@CajaBIG}%
 \hspace*{\@SepTotales}\usebox{\@CajaAliG}%
 \if@Exento\hspace*{\@SepTotales}\usebox{\@CajaBIE}\fi
 \hspace*{\@SepTotales}\usebox{\@CajaTotal}%
 \ifcentrado\hspace*{\@SepTotales}\fi}

\newcommand{\@EscribeTotalesFilasGxET}{%
 \if@BIR\hspace*{\@SepTotales}\usebox{\@CajaBIAliR}\fi
 \hspace*{\@SepTotales}\usebox{\@CajaBIAliG}%
 \if@BIA\hspace*{\@SepTotales}\usebox{\@CajaBIAliA}\fi
 \hspace*{\@SepTotales}\if@Exento\usebox{\@CajaBIET}\else\usebox{\@CajaTotal}\fi
 \ifcentrado\hspace*{\@SepTotales}\fi}

\newcommand{\@EscribeTotalesFilasRGA}{%
 \hspace*{\@SepTotales}\usebox{\@CajaBIAliR}%
 \hspace*{\@SepTotales}\usebox{\@CajaBIAliG}%
 \hspace*{\@SepTotales}\usebox{\@CajaBIAliA}%
 \hspace*{\@SepTotales}}

\newcommand{\@EscribeTotalesFilasRGAET}{%
 \@EscribeTotalesFilasRGA
 \if@Exento\usebox{\@CajaBIET}\else\usebox{\@CajaTotal}\fi
 \ifcentrado\hspace*{\@SepTotales}\fi}

%% Escribe la letra que designa al tipo de impuesto que aplica al ítem.
\NewDocumentCommand{\@EscribeTipoDeImpuesto}{m}{%
 % Hay que abrir un grupo de ejecución porque el cambio en la tipografía altera la celda.
 {\ifG
   \@LetraTipoIVA
   \ifx#1G\@DenomAlicuotaG\fi
   \ifx#1R\@DenomAlicuotaR\fi
   \ifx#1A\@DenomAlicuotaA\fi
   \ifx#1E\@DenomAlicuotaE\fi
  \fi}}
 
%% Comando para poner en cero los contadores de los totales porque el ambiente de la factura se evalúa tres veces y los totales se sumarían todos tres veces.
\newcommand{\@ZERO}{%
 \FPset\BIG{0}\global\let\BIG\BIG
 \FPset\BIR{0}\global\let\BIR\BIR
 \FPset\BIA{0}\global\let\BIA\BIA
 \FPset\BIE{0}\global\let\BIE\BIE
 \FPset\SubtE{0}\global\let\SubtE\SubtE
 \FPset\SubtG{0}\global\let\SubtG\SubtG
 \FPset\SubtR{0}\global\let\SubtR\SubtR
 \FPset\SubtA{0}\global\let\SubtA\SubtA
 \FPset\SubtX{0}\global\let\SubtX\SubtX
 \FPset\DescE{0}\global\let\DescE\DescE
 \FPset\DescG{0}\global\let\DescG\DescG
 \FPset\DescR{0}\global\let\DescR\DescR
 \FPset\DescA{0}\global\let\DescA\DescA
 \FPset\DescX{0}\global\let\DescX\DescX
 \FPset\AlicuotaG{0}\global\let\AlicuotaG\AlicuotaG
 %% (quitado por no vigente)
  % \FPset\AlicuotaGDesc{0}\global\let\AlicuotaGDesc\AlicuotaGDesc
 \FPset\AlicuotaR{0}\global\let\AlicuotaR\AlicuotaR
 \FPset\AlicuotaA{0}\global\let\AlicuotaA\AlicuotaA}

%% Comandos para calcular todos los totales parciales y el total general.
\newcommand{\@CalculaTotalesI}{%
 \FPifeq\SubtG{0}\else\@CalculaTotalesII{G}\fi
 \FPifeq\SubtR{0}\else\@CalculaTotalesII{R}\fi
 \FPifeq\SubtA{0}\else\@CalculaTotalesII{A}\fi
 \FPifeq\SubtE{0}\else\@CalculaTotalesII{E}\fi
 %% (quitado por no vigente)
 % \ifbool{Electro}{%
 %  \FPiflt\BIG\@AlicuotaGDescLimite
 %   \let\@AlicuotaGDesc\@AlicuotaGDescBajo
 %  \else
 %   \let\@AlicuotaGDesc\@AlicuotaGDescAlto
 %  \fi
 %  \FPupn\@AlicuotaGDescPorc{\@AlicuotaGDesc{} 100 * \@DigitosPorc{} round clip}%
 %  \FPupn\AlicuotaGDesc{\BIG{} \@AlicuotaGDesc{} * \@DigitosMoneda{} round}%
 %  \global\let\AlicuotaGDesc\AlicuotaGDesc
 %  \global\let\@AlicuotaGDescPorc\@AlicuotaGDescPorc
 %  \global\let\@AlicuotaGDesc\@AlicuotaGDesc}{}%
 \FPupn\Total{\BIE{} \BIR{} + \BIG{} + \BIA{} + \AlicuotaR{} + \AlicuotaG{} + \AlicuotaA{} + \@DigitosMoneda{} round}%
 %% (quitado por no vigente) arriba se sumaba también \AlicuotaGDesc{}.
 \global\let\Total\Total}

\newcommand{\@CalculaTotalesII}[1]{%
 % Sumamos al descuento.
 \expandafter\FPupn\csname Desc#1\endcsname{\csname @Descuento#1factor\endcsname{} \csname Subt#1\endcsname{} * \csname @Descuento#1num\endcsname{} + \@DigitosMoneda{} round}%
 \global\expandafter\let\csname Desc#1\expandafter\endcsname\csname Desc#1\endcsname
 % Sumamos a la base imponible.
 \expandafter\FPupn\csname BI#1\endcsname{\csname Desc#1\endcsname{} \csname Subt#1\endcsname{} - \@DigitosMoneda{} round}%
 \global\expandafter\let\csname BI#1\expandafter\endcsname\csname BI#1\endcsname
 \ifx#1E%
 \else{%
  % Sumamos a la alícuota.
  \expandafter\FPupn\csname Alicuota#1\endcsname{\csname BI#1\endcsname{} \csname @Impuesto#1\endcsname{} * \@DigitosMoneda{} round}%
  \global\expandafter\let\csname Alicuota#1\expandafter\endcsname\csname Alicuota#1\endcsname}%
 \fi}

%% Comandos auxiliares para poder evaluar la definición de las columnas en la tabla con un comando que se expande completamente.
\def\@FacturaAuxColumnasEval{
 % Primera línea, la más externa.
 \ifbool{SinLineasVertExt}{}{|}
 % Columna para el número del ítem, si aplica.
 \ifbool{NumItem}{c}{r@{\extracolsep{0pt}}}
 % Segunda línea, que sale solamente cuando se usa la opción 'numitem'
 \ifbool{NumItem}{\ifbool{SinLineasVertInt}{}{|}}{}
 % Columna 'X' con DCU, columna fantasma con 'sincantidad' y columna 'c' con CDU.
 \ifbool{PrimeroDescripcion}{X}{
  \ifbool{SinCantidad}{l@{\extracolsep{0pt}}}{c}}
 % Tercera línea.
 \ifbool{SinCantidad}{}{\ifbool{SinLineasVertInt}{}{|}}
 % Columna 'c' con DCU, columna 'r' con DUC, columna 'X' sin espacio para 'sincantidad' y columna 'X' con CDU.
 \ifbool{DCU}{c}{
  \ifbool{DUC}{r}{
   \ifbool{SinCantidad}{X@{\extracolsep{0pt}}}{X}}}
 % Cuarta línea, que sale solamente cuando no se usa 'sincantidad'. 
 \ifbool{SinCantidad}{}{\ifbool{SinLineasVertInt}{}{|}}
 % Columna 'c' para DUC y columna 'r' para cualquier otro caso (DCU, CDU y 'sincantidad').
 \ifbool{DUC}{c}{r}
 % Quinta línea. 
 \ifbool{SinLineasVertInt}{}{|}
 % Columna 'r' para subtotal antes del descuento, o columna fantasma si no se especifica 'subtotal').
 \ifbool{Subtotal}{r}{l@{\extracolsep{0pt}}}
 % Sexta línea, que sale solamente cuando hay columna de subtotales.
 \ifbool{Subtotal}{\ifbool{SinLineasVertInt}{}{|}}{}
 % Columna 'c' para 'descuentos' y columna 'r' sin espacio si no hay descuentos individuales.
 \ifbool{Descuentos}{c}{r@{\extracolsep{0pt}}}
 % Séptima línea, que sale solamente cuando hay descuentos.
 \ifbool{Descuentos}{\ifbool{SinLineasVertInt}{}{|}}{}
 % Columna 'c' para 'iva' y columna 'r' sin espacio si no hay columna de IVA.
 \ifbool{IVA}{c}{r@{\extracolsep{0pt}}}
 % Octava línea, que sale solamente cuando hay columna de IVA.
 \ifbool{IVA}{\ifbool{SinLineasVertInt}{}{|}}{}
 % Columna 'r' siempre.
 r
 % Novena y última línea.
 \ifbool{SinLineasVertExt}{}{|}}

%% Comando para hacer los cálculos sobre cada ítem y establecer las macros con el contenido de cada ítem introducido.
%% Aquí tenemos que hacer coincidir los parámetros de \@CalculaItem con los de \@ITEM para poder hacer
%% \let\@CalculaItem\@ITEM cuando no hay factura sino solamente reporte.
\NewDocumentCommand{\@CalculaItem}{m m m o m m +o}{%
 \gdef\@PrecioUtemp##1{\FPupn\PrecioUtemp{##1 \PrecioUtempx{} * clip}}%
 \addtocounter{NumItem}{1}%
 % Calculamos.
 \@PrecioUtemp{#5}%
 \FPupn\PrecioU{\PrecioUtemp{} \@DigitosMoneda{} round}%
 \FPifeq\PrecioU{\PrecioUtemp}\else\CLASSWARNING{Se esta' introduciendo un precio con ma's de dos decimales: (#3 - \PrecioUtemp. Redondeado a \PrecioU)}{}\fi% Advertimos cuando el precio unitario tiene más de dos decimales.
 \ifbool{Descuentos}
  {\ifx\\#6\\\FPset\Dcto{0}\else\FPupn\Dcto{#6 \@DigitosPorc{} round clip}\fi}% Fijamos en 0 el descuento si no se especifica en \Item***.
  {\FPset\Dcto{#6}%
   \FPifeq\Dcto{0}\else\CLASSWARNING{Se pidio' un descuento individual sin dar la opcio'n de la clase 'descuentos'. No se tomara' en cuenta}\fi
   \FPset\Dcto{0}}% También fijamos en 0 el descuento por si se especifica en \Item*** pero no se pide la opción 'descuentos'.
 \FPupn\PrecioS{#2 \PrecioU{} * \@DigitosMoneda{} round}%
 \FPupn\PrecioT{100 \Dcto{} 100 - / \PrecioS{} * \@DigitosMoneda{} round}%
 \global\let\PrecioU\PrecioU
 \global\let\PrecioS\PrecioS
 \global\let\Dcto\Dcto
 \global\let\PrecioT\PrecioT
 % Suma el elemento actual al subtotal acumulado.
 \expandafter\FPupn\csname Subt#1\endcsname{\PrecioT{} \csname Subt#1\endcsname{} + \@DigitosMoneda{} round}%
 \global\expandafter\let\csname Subt#1\expandafter\endcsname\csname Subt#1\endcsname
 % Guardamos en macros cada elemento agregado a la factura; por ejemplo: \@Item@i@Cantidad contiene la cantidad que se introdujo para el primer ítem; así mismo con ii@, @iii@, y así mismo también con @Descripcion, @PrecioU, @Descuento y @PrecioT.
 \expandafter\gdef\csname @Item@\romannumeral\value{NumItem}@Cantidad\endcsname{\numprint{#2}}%
 \expandafter\gdef\csname @Item@\romannumeral\value{NumItem}@Descripcion\endcsname{#3}%
 \expandafter\global\expandafter\edef\csname @Item@\romannumeral\value{NumItem}@PrecioU\endcsname{\noexpand\numprint{\PrecioU}}%
 \expandafter\global\expandafter\edef\csname @Item@\romannumeral\value{NumItem}@PrecioS\endcsname{\noexpand\numprint{\PrecioS}}%
 \expandafter\global\expandafter\edef\csname @Item@\romannumeral\value{NumItem}@Descuento\endcsname{\noexpand\numprint{\Dcto}}%
 \expandafter\gdef\csname @Item@\romannumeral\value{NumItem}@DescripcionLarga\endcsname{#7}%
 \expandafter\global\expandafter\expandafter\edef\csname @Item@\romannumeral\value{NumItem}@IVA\endcsname{%
  \noexpand\numprint\expandafter{\csname @Alicuota#1porc\endcsname}}%
 \expandafter\global\expandafter\edef\csname @Item@\romannumeral\value{NumItem}@PrecioT\endcsname{\noexpand\numprint{\PrecioT}}}

%% Macro interna para saber si una etiqueta existe o no.
\gdef\@pruebaref#1{%
 \expandafter\ifx\csname @Etiqueta@#1\endcsname\relax
  \CLASSWARNING{La etiqueta `#1' no esta' definida}\ignorespaces
  \@PruebaReffalse
 \else
  \@PruebaReftrue
 \fi}

%% Comandos para las referencias: i) referencia al número de ítem, ii) a la cantidad, iii) a la descripción, iv) al precio unitario, v) al descuento, vi) al precio total.
\gdef\itemref#1{%
 \ifbool{NumItem}{}{\CLASSWARNING{Es recomendable usar la opcio'n 'numitem' para enumerar los i'tems}}%
 \@pruebaref{#1}\if@PruebaRef \csname @Etiqueta@#1\endcsname \fi}
\gdef\cantref#1{\@pruebaref{#1}\if@PruebaRef \csname @Item@\expandafter\romannumeral\csname @Etiqueta@#1\endcsname @Cantidad\endcsname \fi}
\gdef\descref#1{\@pruebaref{#1}\if@PruebaRef \csname @Item@\expandafter\romannumeral\csname @Etiqueta@#1\endcsname @Descripcion\endcsname \fi}
\gdef\puref#1{\@pruebaref{#1}\if@PruebaRef \csname @Item@\expandafter\romannumeral\csname @Etiqueta@#1\endcsname @PrecioU\endcsname \fi}
\gdef\subtref#1{\@pruebaref{#1}\if@PruebaRef \csname @Item@\expandafter\romannumeral\csname @Etiqueta@#1\endcsname @PrecioS\endcsname \fi}
\gdef\dctoref#1{\@pruebaref{#1}\if@PruebaRef \csname @Item@\expandafter\romannumeral\csname @Etiqueta@#1\endcsname @Descuento\endcsname \fi}
\gdef\ldescref#1{\@pruebaref{#1}\if@PruebaRef \csname @Item@\expandafter\romannumeral\csname @Etiqueta@#1\endcsname @DescripcionLarga\endcsname \fi}
\gdef\ivaref#1{\@pruebaref{#1}\if@PruebaRef \csname @Item@\expandafter\romannumeral\csname @Etiqueta@#1\endcsname @IVA\endcsname \fi}
\gdef\ptref#1{\@pruebaref{#1}\if@PruebaRef \csname @Item@\expandafter\romannumeral\csname @Etiqueta@#1\endcsname @PrecioT\endcsname \fi}

%% Definimos un comando para la etiqueta, que se define dentro del entorno de la factura para que tenga alcance local; cuando se acaba dicho entorno, el comando \label vuelve a ser lo que está definido en latex.ltx.
\newcommand{\label@factura}[1]{%
 \expandafter\ifx\csname @Etiqueta@#1\endcsname\relax
  \expandafter\global\expandafter\edef\csname @Etiqueta@#1\endcsname{\arabic{NumItem}}\ignorespaces
  \CLASSINFO{Etiqueta `#1' definida para el i'tem \arabic{NumItem}}%
 \else
  \edef\@label@test@uno{\csname @Etiqueta@#1\endcsname}%
  \edef\@label@test@dos{\theNumItem}%
  \ifx\@label@test@uno\@label@test@dos
  \else
   \CLASSWARNING{La etiqueta `#1' se definio' ma's de una vez}\ignorespaces
  \fi
 \fi}

%% Todo esto tiene que estar dentro de \AtEndPreamble que es cuando \if@factura, \if@reverso e \if@reporte fueron ya aplicadas, y para que se evalúen adecuadamente en su momento las definiciones hechas con \edef.
\AtEndPreamble{
 % Comando auxiliar para abrir la tabla del cuadro de la factura y expandir completamente las columnas.
 \edef\@FacturaAuxInicioCmd{\noexpand\begin{tabularx}{\textwidth}{\@FacturaAuxColumnasEval}}

 % Comando auxiliar para abrir el '\multicolumn' de la nota interna y evalúe la presencia o no de las barras verticales.
 \edef\@ColumnasMultiColumnNotaIntAuxCmd{%
  \noexpand\multicolumn{8}{\ifbool{SinLineasVertExt}{}{|}l\ifbool{SinLineasVertExt}{}{|}}{%
   \noexpand\parbox[t]{\linewidth-2\tabcolsep-2\arrayrulewidth}{\noexpand\@LetraNotaInterna\noexpand\@NotaInterna\noexpand\strut}}%
   \noexpand\\}

 % Comando auxiliar para abrir el '\multicolumn' para los totales en el modo de columnas y expandir completamente las columnas cuando hay o no barras verticales.
 \edef\@ColumnasMultiColumnTotalesAuxCmd{%
  \noexpand\multicolumn{8}{\ifbool{SinLineasVertExt}{}{|} r}\begingroup
    \noexpand\if@ExpandeCuadro
     \noexpand\iflinea\noexpand\zsaveposx{LineaNodoBCoordX}\noexpand\zsaveposy{LineaNodoBCoordY}\noexpand\fi\noexpand\fi% Fijamos la posición (x,y) final para la línea ya que estaremos justo en el punto donde escribimos el total, y hasta allí llega la línea en el modo de columnas.
     \noexpand\begin{tabular}[b]{@{\noexpand\extracolsep{2\noexpand\tabcolsep}}lr \ifbool{SinLineasVertExt}{}{|}}}

 % Comando auxiliar para la definición del ambiente de la factura; al final de este comando está el texto que va en el enddef del ambiente de la factura.
 \newcommand{\@FacturaAuxCmd}[1]{%
  \ifexpandecuadro
   \setlength{\@EspacioVacioInicial}{\zposy{EspacioVacioInicial}sp}%
   \setlength{\@EspacioVacioFinal}{\zposy{EspacioVacioFinal}sp}%
   \global\@EspacioVacioInicial=\@EspacioVacioInicial% Hay que globalizarlas
   \global\@EspacioVacioFinal=\@EspacioVacioFinal
   \savebox{\@CajaEspacioVacio}{\parbox[b][\@EspacioVacioInicial-\@EspacioVacioFinal]{\@TamDescripcion}{%
    \iflinea
     \if@LineaHastaFinal
      \zsaveposy{LineaNodoBCoordY}%
     \else
      \if@LineaDescripcion
       \zsaveposx{LineaNodoACoordX}\mbox{}\hfill \zsaveposx{LineaNodoBCoordX}\zsaveposy{LineaNodoBCoordY}%
      \fi
     \fi
    \fi}}%
   \ifdim \ht\@CajaEspacioVacio=0pt \global\lineafalse \else \@ExpandeCuadrotrue \fi
  \fi

  \ifx\undefined\Total\relax\else\CLASSERROR{Parece ya haberse usado un entorno de factura}\fi
  %
  % Comandos \David@SavePos, \David@XStart, \David@XEnd, \David@XPad, \David@XLeft y \David@XRight para poder manejar el salto de línea en las columnas adyacentes a la columna 'X' del tabularx (la descripción), cuando esta ocupa más que una línea.
  % Gracias a David Carlisle (Thanks to David Carlisle).
  % http://tex.stackexchange.com/questions/111203/tabularx-bottom-alignment-on-r-or-l-columns-when-x-column-wraps
  \def\David@SavePos##1{\leavevmode\pdfsavepos\write\@auxout{\gdef\noexpand##1{\the\pdflastypos sp}}}%
  \def\David@XStart{\expandafter\David@SavePos\csname save@start@\romannumeral\value{NumItem}\endcsname}%
  \def\David@XEnd{\expandafter\David@SavePos\csname save@end@\romannumeral\value{NumItem}\endcsname}%
  \def\David@XPad##1##2##3{%
   \vtop{%
    \baselineskip\normalbaselineskip
    ##2%
    \expandafter
    \ifx\csname save@start@##1\endcsname
     \relax
    \else
     \dimen@\csname save@start@##1\endcsname\relax
     \loop
      \ifdim\dimen@>\csname save@end@##1\endcsname
      \advance\dimen@-\baselineskip
      \settowidth{\@TempDimen}{##3}%
       \ifdim\@TempDimen=0pt\settowidth{\@TempDimen}{##2}\fi
      \hbox{\parbox{\@TempDimen}{\@CaracterCeldasVacias}}%
     \repeat
    \fi
    ##3}}%
  \def\David@XLeft##1{\David@XPad{\romannumeral\value{NumItem}}{\hbox{##1}}{}}%
  \def\David@XRight##1{\David@XPad{\romannumeral\value{NumItem}}{}{\hbox{##1}}}%
  %
  % Caja auxiliar para el caracter de retorno de carro y poder saber si tamaño en el campo de la descripción.
  \savebox{\@CajaCaracterCR}{\@LetraItems\@CaracterCR}
  %
  % Comando para fijar la separación extra final de cada item, lo cual viene dado por la cantidad de líneas de la descripción,
  % que está alojado en \count1, definido en \@DescripcionAuxCmd.
  \def\@CalculaSeparaItemsExtraFinal{%
   \setlength{\@SeparaItemsExtraFinal}{\numexpr\the\count1\normalbaselineskip-\normalbaselineskip+\@SeparaItemsExtra}%
   \global\@SeparaItemsExtraFinal=\@SeparaItemsExtraFinal}
  %
  % Comando interno general para agregar cualquier ítem.
  \NewDocumentCommand{\@ITEM}{m m m o m m +o}{%
   % 1. [R|G|A|E|X]; 2. cantidad; 3. descripción; 4. descripción para el csv (entre corchetes); 5. p/u; 6. dcto; 7. la descripción para el reporte.
   \@CalculaItem{##1}{##2}{##3}{##5}{##6}[##7]%
   % Si no es el primer ítem, entonces hace falta \\ para terminar el ítem anterior.
   \ifnum\value{NumItem}=1% Aquí, \@CalculaItem aumenta en 1 el contador NumItem, que si vale 1 significa que aún no hemos procesado la línea 1, es decir, no hay que hacer el salto de línea. La última línea se termina en otro lado.
   \else
    \@CalculaSeparaItemsExtraFinal
    \\[\@SeparaItemsExtraFinal]
    \ifconlineasha\hline\fi% Líneas para separar cada ítem, si hiciera falta.
   \fi
   % Primer -opcional- campo: El número del ítem.
   \ifbool{NumItem}{\David@XLeft{\@NumItemAuxCmd{\mbox{}\hfill \@LetraNumeros \arabic{NumItem}\hfill\mbox{}}}}{}%
   &
   % Segundo campo: Acá puede ir la cantidad (cdu) o la descripción (dcu y duc).
   \ifbool{SinCantidad}{}{%
    \ifbool{PrimeroDescripcion}
    {\@DescripcionAuxCmd{##3}{##1}}% dcu y duc.
    {\David@XLeft{\@CantidadAuxCmd{\mbox{}\hfill \@LetraNumeros \numprint{##2}\hfill\mbox{}}}}}% cdu.
   &
   % Tercer campo: Acá puede ir la descripción (cdu), la cantidad (dcu), o el precio unitario (duc).
   \ifbool{PrimeroDescripcion}{%
    \ifdcu
     \David@XRight{\@CantidadAuxCmd{\mbox{}\hfill \@LetraNumeros \numprint{##2}\hfill\mbox{}}}% dcu.
    \else
     \David@XRight{\@PrecioUAuxCmd{%
      \ifx##1X \hfill \@CaracterItemXVacio \hfill\mbox{}%
      \else
       \ifimpdesc\else\@EscribeTipoDeImpuesto{##1}\fi \hfill \@LetraNumeros \numprint{\PrecioU}
      \fi}}% duc.
    \fi}
   {\@DescripcionAuxCmd{##3}{##1}}% cdu.
   &
   % Cuarto campo: Acá puede ir el precio unitario (cdu y dcu) o la cantidad (duc).
   \ifbool{SinCantidad}{}{%
    \ifduc
     \David@XRight{\@CantidadAuxCmd{\mbox{}\hfill \@LetraNumeros \numprint{##2}\hfill\mbox{}}}% duc.
    \else
     \David@XRight{\@PrecioUAuxCmd{%
      \ifx##1X \hfill \@CaracterItemXVacio \hfill\mbox{}%
      \else
       \ifimpdesc\else\@EscribeTipoDeImpuesto{##1}\fi \hfill \@LetraNumeros \numprint{\PrecioU}
      \fi}}% dcu y cdu.
    \fi}%
   & % Quinto -opcional- campo: Subtotal (para luego descuentos).
   \ifbool{Subtotal}{%
    \David@XRight{\@PrecioSAuxCmd{%
     \ifx##1X \hfill \@CaracterItemXVacio \hfill\mbox{}%
     \else
      \ifimpdesc\else\@EscribeTipoDeImpuesto{##1}\fi \hfill \@LetraNumeros \numprint{\PrecioS}
     \fi}}}{}
   &
   % Sexto -opcional- campo: Descuento.
   \ifbool{Descuentos}{%
    \David@XRight{\@DescuentoAuxCmd{%
     \ifx##1X \hfill \@CaracterItemXVacio \hfill\mbox{}%
     \else
      \mbox{}\hfill \@LetraNumeros \numprint{\Dcto}\%\hfill\mbox{}%
     \fi}}}{}% Tenemos que usar un condicional booleano porque si no da problemas con el condicional \FPifeq.
   &
   % Séptimo -opcional- campo: IVA.
   \ifbool{IVA}{%
    \David@XRight{\@IVAAuxCmd{%
     \ifx##1X \hfill \@CaracterItemXVacio \hfill\mbox{}%
     \else
      \mbox{}\hfill \@LetraNumeros \numprint{\csname @Alicuota##1porc\endcsname}\%\hfill\mbox{}%
     \fi}}}{}% Tenemos que usar un condicional booleano porque si no da problemas con el condicional \FPifeq.
   &
   % Octavo campo: El total.
   \David@XRight{\@PrecioTAuxCmd{%
    \ifx##1X \hfill \@CaracterItemXVacio \hfill\mbox{}%
    \else
     \ifimpdesc\else\@EscribeTipoDeImpuesto{##1}\fi \hfill \@LetraNumeros \numprint{\PrecioT}
    \fi}}%
   % Escribimos al csv.
   \ifcsv
    \IfValueTF{##4}{\@EscribeItemsEnCSV{##1}{##2}{##4}}{\@EscribeItemsEnCSV{##1}{##2}{##3}}%
   \fi}%
  %
  % Comando para escribir en el .csv.
  \newcommand{\@EscribeItemsEnCSV}[3]{%
   \ifbool{SinCantidad}
   {\ifdescuentos
     \write\csv{"\unexpanded{##3}","\Dcto","\PrecioT","##1"}%
    \else
     \write\csv{"\unexpanded{##3}","\PrecioT","##1"}%
    \fi}
   {\if@cdu
     \ifdescuentos
      \write\csv{"##2","\unexpanded{##3}","\PrecioU","\Dcto","\PrecioT","##1"}%
     \else
      \write\csv{"##2","\unexpanded{##3}","\PrecioU","\PrecioT","##1"}%
     \fi
    \fi
    \ifdcu
     \ifdescuentos
      \write\csv{"\unexpanded{##3}","##2","\PrecioU","\Dcto","\PrecioT","##1"}%
     \else
      \write\csv{"\unexpanded{##3}","##2","\PrecioU","\PrecioT","##1"}%
     \fi
    \fi
    \ifduc
     \ifdescuentos
      \write\csv{"\unexpanded{##3}","\PrecioU","##2","\Dcto","\PrecioT","##1"}%
     \else
      \write\csv{"\unexpanded{##3}","\PrecioU","##2","\PrecioT","##1"}%
     \fi
    \fi}}%
  %
  % Usamos la definición de \label@factura dentro del entorno de la factura para que sea local aquí nada más.
  \let\label\label@factura
  \@LetraItems% Este es el tipo de letra que tendrán los ítems, y en este tipo de letra se define el \baselineskip de toda la factura.
  \setlength{\baselineskip@@}{\baselineskip}%
  \global\baselineskip@@=\baselineskip@@
  \@FacturaAuxInicioCmd
  \ifbool{SinLineasHoriExt}{}{\hline}%
  \@ZERO
  \setlength{\tabcolsep@@}{\tabcolsep}%
  \global\tabcolsep@@=\tabcolsep@@
  % Estamos en el número del ítem, si aplicara.
  \if@LineaDesdeInicio \zsaveposx{LineaNodoACoordX}\fi
  \ifbool{NumItem}{\@NumItemAuxCmd{\mbox{}\hfill \@LetraTitColumnas \@@NumItem \hfill\mbox{}}}{}%
  &
  % Avanzamos a la cantidad, o la descripción.
  \ifbool{SinCantidad}{}{\ifbool{PrimeroDescripcion}{\@DescripcionAuxCmd{\@LetraTitColumnas \@@Descripcion}{}}{\@CantidadAuxCmd{\mbox{}\hfill \@LetraTitColumnas \@@Cantidad \hfill\mbox{}}}}%
  &% Avanzamos hasta la descripción, la cantidad o el precio unitario.
  \ifbool{SinCantidad}{\@DescripcionAuxCmd{\@LetraTitColumnas \@@Descripcion}{}}
   {\if@cdu \@DescripcionAuxCmd{\@LetraTitColumnas \@@Descripcion}{}\fi
    \ifdcu \@CantidadAuxCmd{\mbox{}\hfill \@LetraTitColumnas \@@Cantidad \hfill\mbox{}}\fi
    \ifduc \@PrecioUAuxCmd{\mbox{}\hfill \@LetraTitColumnas \@@PrecioU \space (\Moneda)\hfill\mbox{}}\fi}%
  &% Avanzamos hasta el precio unitario, o la cantidad, o columna fantasma cuando es 'sincantidad'.
  \ifbool{SinCantidad}{}
   {%
    \mbox{}\hfill
    \ifduc
     \@CantidadAuxCmd{\mbox{}\hfill \@LetraTitColumnas \@@Cantidad \hfill\mbox{}}%
    \else
     \@PrecioUAuxCmd{\mbox{}\hfill \@LetraTitColumnas \@@PrecioU \space (\Moneda)\hfill\mbox{}}%
    \fi}%
  &% Avanzamos hasta el subtotal, que puede o no existir,
  \ifbool{Subtotal}{\@PrecioSAuxCmd{\mbox{}\hfill \@LetraTitColumnas \@@PrecioS \space (\Moneda)\hfill\mbox{}}}{}%
  &% Avanzamos hasta el descuento, que puede o no existir.
  \ifdescuentos \@DescuentoAuxCmd{\mbox{}\hfill \@LetraTitColumnas \@@Descuento \hfill\mbox{}}\fi
  &% Avanzamos hasta el IVA, que puede o no existir.
  \ifbool{IVA}{\@IVAAuxCmd{\mbox{}\hfill \@LetraTitColumnas \@@IVA \hfill\mbox{}}}{}
  &% Avanzamos hasta el precio total.
  \@PrecioTAuxCmd{\mbox{}\hfill \@LetraTitColumnas \ifbool{SinCantidad}{\@@Precio}{\@@PrecioT}\space (\Moneda)\hfill\mbox{}}%
  \if@LineaHastaFinal \zsaveposx{LineaNodoBCoordX}\fi
  \\
  \ifbool{SinLineaHoriTit}{}{\hline}%
  %
  #1% Aquí entran todos los items que el usuario declara dentro del ambiente.
  \@CalculaTotalesI
  \@CalculaSeparaItemsExtraFinal
  \\[\@SeparaItemsExtraFinal]% El último fin de línea se hace aquí.
  \ifconlineasha
   \ifdefvoid{\@NotaInterna}{%
    \ifbool{SinLineasHoriExt}{\if@ExpandeCuadro\hline\fi}{\hline}}{\hline}%
  \fi% Se verifica si hace falta una línea horizontal o no.
  \iffilas
   % Esto en lugar de \ifexpandecuadro para que no se haga una linea vacía en la tabla cuando la caja está vacía.
   \if@ExpandeCuadro
    \ifbool{PrimeroDescripcion}{&\usebox{\@CajaEspacioVacio}&&&&&&\\}{&&\usebox{\@CajaEspacioVacio}&&&&&\\}%
    \ifdefvoid{\@NotaInterna}{}
    {\ifbool{SinLineaHoriNI}{}{\hline}%
     \@ColumnasMultiColumnNotaIntAuxCmd}%
    \ifbool{SinLineasHoriExt}{}{\hline}% Siempre que hay relleno se pone la línea acá.
   \else
    \ifdefvoid{\@NotaInterna}{%
     \ifbool{SinLineasHoriExt}{}{%
      \ifconlineasha\else\hline\fi}}% Verificamos si hay que poner la línea horizontal. Si está 'conlineasha' entonces ya salió la línea.
     {\ifbool{SinLineaHoriNI}{}{\ifconlineasha\else\hline\fi}%
      \@ColumnasMultiColumnNotaIntAuxCmd
      \ifbool{SinLineasHoriExt}{}{\hline}}%
   \fi
  \else
   \ifsintotales
   \else
    \ifbool{SinLineaHoriTot}{}{\ifconlineasha\else\hline\fi}% Línea que cierra los ítems en la tabla.
   \fi
   % Agregamos el espacio espúreo.
   \if@ExpandeCuadro
    % Rellenamos la tabla con espacio espúreo, y le agregamos las líneas externas, si aplica.
    \ifbool{SinLineasVertExt}{\multicolumn{8}{c}{\usebox{\@CajaEspacioVacio}}}{\multicolumn{8}{|c|}{\usebox{\@CajaEspacioVacio}}}\\
   \fi
   % Si pedimos sin totalizaciones, entonces cerramos la tabla ya, si no, calculamos los totales y los escribimos.
   \ifsintotales
    \ifdefvoid{\@NotaInterna}{}{\ifbool{SinLineaHoriNI}{}{\hline}\@ColumnasMultiColumnNotaIntAuxCmd}%
    \ifbool{SinLineasHoriExt}{}{\hline}%
   \else
    % Modo de columnas: Abrimos la tabla para escribir los totales y le agregamos las líneas verticales externas, si aplica.
    \@ColumnasMultiColumnTotalesAuxCmd
    \ifagrupado
     \@EscribeTotalColAgrupado
    \else
     \if@Exento \@EscribeTotalCol{E}\fi
     \if@BIR\@EscribeTotalCol{R}\ifsinivaexpresado\else\@EscribeTotalColAlicuota{R}\fi\fi
     \@EscribeTotalCol{G}\ifsinivaexpresado\else\@EscribeTotalColAlicuota{G}\fi
     %% (quitado por no vigente)
     % \ifbool{Electro}{\@EscribeTotalColAlicuotaGDesc}{}%
     \if@BIA\@EscribeTotalCol{A}\ifsinivaexpresado\else\@EscribeTotalColAlicuota{A}\fi\fi
    \fi
    \ifsinivaexpresado\else\cline{2-2}\fi
    \hfill \@LetraTitTotal\@@Total & \@LetraNumTotal\@SignoMonetarioAux\numprint{\Total}\\
    \end{tabular}\endgroup
    \\
    \ifdefvoid{\@NotaInterna}{}{\ifbool{SinLineaHoriNI}{}{\hline}\@ColumnasMultiColumnNotaIntAuxCmd}%
    \ifbool{SinLineasHoriExt}{}{\hline}%
   \fi
  \fi
  \end{tabularx}%
  %
  % Generamos la macro \Total vacía para que no dé error en algún posible uso dentro del documento.
  \ifsintotales\gdef\Total{}\fi
  % Si pedimos no totalizar (sintotales), la opción filas queda desactivada.
  \iffilas
   \@LetraNumTotales
   \vspace*{\@SeparaVertical}\par
   \ifexpandecuadro\else\vfill\fi
   \@GeneraCajasTotalesFilas
   \newbox{\@CajaTotal}%
   \savebox{\@CajaTotal}{{\@LetraTitTotal\@@Total:} \@LetraNumTotal\@SignoMonetarioAux\numprint{\Total}}%
   %
   \ifnum \value{@IVANoGeneral}=0%
    \newbox{\@CajaGET}%
    \savebox{\@CajaGET}{\@EscribeTotalesFilasGET}%
    \ifdim \the\wd\@CajaGET < \textwidth
     \parbox{\textwidth}{\@EscribeTotalesFilasGET}%
    \else
     \hspace*{\@SepTotales}%
     \usebox{\@CajaBIAliG}\hspace*{\@SepTotales}%
     \if@Exento\usebox{\@CajaBIET}\else\usebox{\@CajaTotal}\fi
     \ifcentrado\hspace*{\@SepTotales}\fi
    \fi
   \fi
   \ifnum \value{@IVANoGeneral}=1%
    \newbox{\@CajaGxET}%
    \savebox{\@CajaGxET}{\@EscribeTotalesFilasGxET}%
    \ifdim \the\wd\@CajaGxET < \textwidth
     \@EscribeTotalesFilasGxET
    \else
     \hspace*{\@SepTotales}%
     \if@BIR\usebox{\@CajaBIAliR}\hspace*{\@SepTotales}\fi
     \usebox{\@CajaBIAliG}\hspace*{\@SepTotales}%
     \if@BIA\usebox{\@CajaBIAliA}\hspace*{\@SepTotales}\fi
     \\[\@SeparaFilas]\mbox{}%
     \ifcentrado\hspace*{\@SepTotales}\else\hfill\fi
     \if@Exento\usebox{\@CajaBIE}\hspace*{\@SeparaMinTotales}\fi\usebox{\@CajaTotal}%
     \ifcentrado\hspace*{\@SepTotales}\fi
    \fi
   \fi
   \ifnum \value{@IVANoGeneral}=2%
   \newbox{\@CajaRGAET}%
   \savebox{\@CajaRGAET}{\@EscribeTotalesFilasRGAET}%
    \ifdim \the\wd\@CajaRGAET < \textwidth
     \@EscribeTotalesFilasRGAET
    \else
     \newbox{\@CajaRGA}%
     \savebox{\@CajaRGA}{\@EscribeTotalesFilasRGA}%
     \ifdim \the\wd\@CajaRGA < \textwidth
      \@EscribeTotalesFilasRGA
      \\[\@SeparaFilas]\mbox{}%
      \ifcentrado\hspace*{\@SepTotales}\else\hfill\fi
      \if@Exento\usebox{\@CajaBIE}\hspace*{\@SeparaMinTotales}\fi\usebox{\@CajaTotal}%
      \ifcentrado\hspace*{\@SepTotales}\fi
     \else% Aquí casi cualquier cosa tiene aspecto feo.
      \CLASSWARNING{Es mejor no usar la opcio'n de totalizacio'n en filas en este caso, o disminuir los tama~nos de los tipos de letra para la totalizacio'n}%
      \centering{%
       \begin{tabular}{r@{\hspace*{\@SeparaMinTotales}}l}
        \usebox{\@CajaBIAliR} & \usebox{\@CajaBIAliA} \\[\@SeparaFilas]
        \usebox{\@CajaBIAliG} & \if@Exento\usebox{\@CajaBIET}\else\usebox{\@CajaTotal}\fi
       \end{tabular}}%
     \fi
    \fi
   \fi
  \fi
  \par
  \ifexpandecuadro
   \vspace*{-\@EspacioVacioInicial}%
   \vspace*{\@EspacioVacioFinal}%
   \zsaveposy{EspacioVacioInicial}%
  \fi
  \vfill
  \ifexpandecuadro\zsaveposy{EspacioVacioFinal}\fi
  \vspace*{\@SeparaVertical}\par
  \ifdefvoid{\@NotaExterna}{}{\parbox[t]{\linewidth}{\@LetraNotaExterna\@NotaExterna}\vspace*{\@SeparaVertical}\par}}
 
  \ifbool{SinCantidad}
  % Definamos todos los comandos \Item***. El descuento es opcional; por eso usamos G{0} en \NewDocumentCommand.
  {%
   \NewDocumentCommand{\Item}{m o m G{0} +o}{\@ITEM{G}{1}{#1}[#2]{#3}{#4}[#5]}% Ítem normal.
   \NewDocumentCommand{\ItemR}{m o m G{0} +o}{\global\@BIRtrue\global\Gtrue\@ITEM{R}{1}{#1}[#2]{#3}{#4}[#5]}% Ítem con IVA reducido.
   \NewDocumentCommand{\ItemA}{m o m G{0} +o}{\global\@BIAtrue\global\Gtrue\@ITEM{A}{1}{#1}[#2]{#3}{#4}[#5]}% Ítem con IVA adicional.
   \NewDocumentCommand{\ItemE}{m o m G{0} +o}{\global\@Exentotrue\global\Gtrue\@ITEM{E}{1}{#1}[#2]{#3}{#4}[#5]}% Ítem exento de IVA.
   \NewDocumentCommand{\ItemX}{m o +o}{\@ITEM{X}{1}{#1}[#2]{0}{0}[#3]}% Ítem sin precio.
   \let\ItemG\Item
   \ifcsv
    \ifdescuentos
     \write\csv{"\@@Descripcion","\@@Descuento","\@@Precio \space (\Moneda)","\@@IVA"}%
    \else
     \write\csv{"\@@Descripcion","\@@Precio \space (\Moneda)","\@@IVA"}%
    \fi
   \fi}
  {%
   \NewDocumentCommand{\Item}{m m o m G{0} +o}{\@ITEM{G}{#1}{#2}[#3]{#4}{#5}[#6]}% Ítem normal.
   \NewDocumentCommand{\ItemR}{m m o m G{0} +o}{\global\@BIRtrue\global\Gtrue\@ITEM{R}{#1}{#2}[#3]{#4}{#5}[#6]}% Ítem con IVA reducido.
   \NewDocumentCommand{\ItemA}{m m o m G{0} +o}{\global\@BIAtrue\global\Gtrue\@ITEM{A}{#1}{#2}[#3]{#4}{#5}[#6]}% Ítem con IVA adicional.
   \NewDocumentCommand{\ItemE}{m m o m G{0} +o}{\global\@Exentotrue\global\Gtrue\@ITEM{E}{#1}{#2}[#3]{#4}{#5}[#6]}% Ítem exento de IVA.
   \NewDocumentCommand{\ItemX}{m m o +o}{\@ITEM{X}{#1}{#2}[#3]{0}{0}[#4]}% Ítem sin precio.
   \let\ItemG\Item
   \ifcsv
    \if@cdu
     \ifdescuentos
      \write\csv{"\@@Cantidad","\@@Descripcion","\@@PrecioU \space (\Moneda)","\@@Descuento","\@@PrecioT \space (\Moneda)","\@@IVA"}%
     \else
      \write\csv{"\@@Cantidad","\@@Descripcion","\@@PrecioU \space (\Moneda)","\@@PrecioT \space (\Moneda)","\@@IVA"}%
     \fi
    \fi
    \ifdcu
     \ifdescuentos
      \write\csv{"\@@Descripcion","\@@Cantidad","\@@PrecioU \space (\Moneda)","\@@Descuento","\@@PrecioT \space (\Moneda)","\@@IVA"}%
     \else
      \write\csv{"\@@Descripcion","\@@Cantidad","\@@PrecioU \space (\Moneda)","\@@PrecioT \space (\Moneda)","\@@IVA"}%
     \fi
    \fi
    \ifduc
     \ifdescuentos
      \write\csv{"\@@Descripcion","\@@PrecioU \space (\Moneda)","\@@Cantidad","\@@Descuento","\@@PrecioT \space (\Moneda)","\@@IVA"}%
     \else
      \write\csv{"\@@Descripcion","\@@PrecioU \space (\Moneda)","\@@Cantidad","\@@PrecioT \space (\Moneda)","\@@IVA"}%
     \fi
    \fi
   \fi}

 \if@factura
  % Los descuentos totales no son compatibles -y no creo que lo vayan a ser- con el modo de totalización en filas.
  \iffilas
   \if@DescuentoTotal
    \CLASSWARNING{La opcio'n 'filas' no es compatible con los descuentos totales; sin efecto opcio'n 'filas'}
    \filasfalse
   \fi
  \fi
  \if@reverso
  \else
    \RenewEnviron{reverso}{}
  \fi
  \ifcotizacion\def\@Denominacion{\@@DenomCotiza}\else\def\@Denominacion{\@@DenomFactura}\fi%  Fijamos la denominación del documento paras las cotizaciones o facturas, que son casos particulares del modo 'factura'.
  \expandafter\geometry\expandafter{\@GeometriaFactura}%  La geometría debe aplicarse aquí.
  \AfterEndPreamble{%
   \addtolength{\@PosFechaX}{\oddsidemargin+1in}%  Le sumamos el margen izquierdo a las posiciones de las cosas; es importante que se haga en un \AfterEndPreamble antes de que empiecen a escribirse otras cosas en la página.
   \addtolength{\@PosDenomX}{\oddsidemargin+1in}%
   \addtolength{\@PosControlX}{\oddsidemargin+1in}%
   \addtolength{\@PosNotaFinalXuno}{\oddsidemargin+1in}%
   \addtolength{\@PosFirmasXuno}{\oddsidemargin+1in}%
   \setlength{\parindent}{0pt}%
   \setlength{\lineskip}{0pt}%
   \setlength{\parskip}{0pt plus 1pt}%
   \expandafter\setlength{\@SepTotales}{\the\@SeparaMinTotales plus 1fill}% \@SeparaMinTotales es el mínimo espacio entre los totales en el modo 'filas', y SepTotales es una dimensión interna, estirable, con ese como su mínimo.
   \thispagestyle{empty}%
   \pagenumbering{roman}%
   \@EncabezadoFactura
   \@EscribeNotaPrevia}
   \AtBeginShipoutNext{% Antes abría aquí un \AtBeginShipoutNext cuando había reporte y un \AtEndDocument cuando no. No es necesario eso. También lo modifiqué en \FirmaFactura.
    \@CargaFondoFactura
    \@PosicionaFechas
    \ifdenom\@PosicionaDenom\fi
    \ifprefactura\@PosicionaNumControl\fi
    \ifnum\value{@Firmas}=0\else\@PosicionaFirmas\fi
    \ifdefvoid{\@NotaFinal}{}{\@PosicionaNotaFinal}}
  \iflinea
   % Para determinar las coordenadas de la línea hay que usar \AtBeginShipoutNext porque si no se hace en el 'shipout' de la página se genera un espacio en blanco extra.
   \AtBeginShipoutNext{\iflinea\@DibujaLinea\fi}
    \iffilas
     \ifbool{SinLineasVertInt}{\@LineaDesdeIniciotrue\@LineaHastaFinaltrue}{\@LineaDescripciontrue}%
    \else
     \ifsintotales
      \@LineaDesdeIniciotrue\@LineaHastaFinaltrue
     \else
      \@LineaDesdeIniciotrue\@LineaHastaTotaltrue
     \fi
    \fi
  \fi
  %% El comando para la descripción del servicio prestado.
  \NewDocumentCommand{\Descripcion}{+m}{%
   {% Abrimos un grupo adicional para que la redefinición del entorno 'tikzpicture' sea local.
    \@LetraDescripcion
    % Cuando la línea tiene grosor cero, hacemos que el entorno 'tikzpicture' sea "fake", y redefinimos \node para que obvie los parámetros que le dimos como argumento opcional.
    \ifdim\@GrosorLineaDescripcion=0pt%
     \setlength{\@SeparaInternaDescripcion}{0pt}\setlength{\@EsquinaLineaDescripcion}{0pt}\setlength{\@SombraDescripcion}{0pt}%
     \renewenvironment{tikzpicture}{\NewDocumentCommand{\node}{o +m}{####2}}{}%
    \fi
    \begin{tikzpicture}
     % Expandimos después para poderle pasar el contenido de la macro como argumento opcional a \node.
     \expandafter\node\expandafter[\@DescripcionTIKZ]{%
      \ifbool{ExpAbajo}
       {\parbox[t][\textheight-\pagetotal-2\@SeparaInternaDescripcion-\@GrosorLineaDescripcion-\@SombraDescripcion]
        {\textwidth-2\@SeparaInternaDescripcion-\@GrosorLineaDescripcion-\@SombraDescripcion}}%
       {\parbox[t]{\textwidth-\@SombraDescripcion-2\@SeparaInternaDescripcion-2\@GrosorLineaDescripcion}}%
      {\setlength{\parindent}{0.5cm}%
       \setlength{\parskip}{2pt plus 8pt}%
       \ifbool{ExpAbajo}{\vfill}{}%
        #1%
       \ifbool{ExpAbajo}{\vfill}{}}}%
     \ifdim\@GrosorLineaDescripcion=0pt\else ;\fi% El fin del nodo va cuando hay el entorno tikzpicture original.
    \end{tikzpicture}
    \par\vspace*{\@SombraDescripcion}}}%
 \else
  \RenewEnviron{factura}{%
   \@ZERO
   \let\label\label@factura
   \let\@ITEM\@CalculaItem
   \BODY
   \@CalculaTotalesI}
  \RenewEnviron{reverso}{}
  \NewDocumentCommand{\Descripcion}{+m}{}
 \fi

 \if@reporte
  \ifdefvoid{\@InfoPagina}{\let\@InfoPagina\@InfoPaginaNoTitulo}{}%
  \if@factura
  \else
   \expandafter\geometry\expandafter{\@GeometriaReporte}%  La geometría del reporte debe aplicarse aquí cuando sólo hay reporte, así se puede establecer el tamaño del papel.
  \fi
  \NewEnviron{reporte}{%
   \if@factura
    \expandafter\newgeometry\expandafter{\@GeometriaReporte}\cleardoublepage%  La geometría debe aplicarse aquí cuando también hay factura, pero no se puede cambiar el tamaño del papel.
   \fi
   \let\headwidth\textwidth
   \@CargaFondoReporte
   \@EncabezadoReporte
   \ifdefvoid{\@EstiloPagReporteResto}{%
    \if@FondoReporteStar%  Si no usamos \EstiloPagReporte entonces se fijará 'soloabajo' cuando se use \FondoReporte* para que sólo use marcas de pie de página, y si no se usó \FondoReporte* entonces se fijará 'plain' que usa cabeceras y pies de página.
     \pagestyle{soloabajo}%
    \else
     \pagestyle{plain}%
    \fi}
    {\pagestyle{\@EstiloPagReporteResto}}%  Si usamos \EstiloPagReporte en su argumento obligatorio entonces fijamos ese estilo de página.
   \ifdefvoid{\@EstiloPagReporteUno}{%
    \if@FondoReporte%  Si no le damos a \EstiloPagReporte su argumento opcional, entonces se fijará 'solonumpag' como el estilo de página para la primera página, pero solamente si usamos \FondoReporte, ya que el estilo de página predeterminado al usar \FondoReporte* ya se acaba de fijar y no necesita el caso especial de la primera hoja (\thispagestyle).
     \if@FondoReporteStar
     \else
      \thispagestyle{solonumpag}%
     \fi
    \fi}
    {\thispagestyle{\@EstiloPagReporteUno}}%  Si usamoe el argumento opcional de \EstiloPagReporte establecemos ese estilo de página.
   \pagenumbering{arabic}%
   \setlength{\parindent}{0.666cm}%
   \setlength{\parskip}{4pt plus 6pt minus 2pt}% separación entre párrafos en el reporte.
   \vskip\baselineskip
   \BODY\clearpage}% espacio antes de empezar con el reporte, después del título -si lo hubiera-.
 \else
  \NewEnviron{reporte}{}
 \fi}

%% Comando para el diseño de la forma libre.
\newcommand{\@CargaFondoFactura}{%
 \if@FondoFactura
  \begin{textblock*}{\paperwidth}(0pt,0pt)%
   \includegraphics{\@FondoFactura}%
  \end{textblock*}%
 \fi}

%% Comando para el fondo del reporte.
\newcommand{\@CargaFondoReporte}{%
 \begin{textblock*}{\paperwidth}(0pt,0pt)% Tuve que abrir el textblock* aquí para que aplique siempre; de esta manera se corrige un defecto que se produce cuando se usa \vspace* al inicio de la página (dentro de \@EncabezadoReporte).
  \if@FondoReporte
   \if@FondoReporteStar
    \AddToShipoutPictureBG{\includegraphics{\@FondoReporte}}%
   \else
     \noindent\includegraphics{\@FondoReporte}%
   \fi
  \fi
 \end{textblock*}}

%% Comando que coloca los espacios para firmar.
\newcommand{\@PosicionaFirmas}{%
 \savebox{\@CajaFirmas}{%
  \parbox{\textwidth+\oddsidemargin+1in-\@PosFirmasXuno-\@PosFirmasXdos}{\@EstiloFirmas}}%
 \ifbool{FirmasYnegativo}{\addtolength{\@PosFirmasY}{\paperheight-\ht\@CajaFirmas-\dp\@CajaFirmas}}{}%
 \begin{textblock*}{\wd\@CajaFirmas}(\@PosFirmasXuno,\@PosFirmasY)%
  \usebox{\@CajaFirmas}%
 \end{textblock*}}

%% Comando que coloca la nota final.
\newcommand{\@PosicionaNotaFinal}{%
 \savebox{\@CajaNotaFinal}{%
  \parbox{\textwidth+\oddsidemargin+1in-\@PosNotaFinalXuno-\@PosNotaFinalXdos}{\@EscribeNotaFinal}}%
 \ifbool{NotaFinalYnegativo}{\addtolength{\@PosNotaFinalY}{\paperheight}}{}%
 \ifbool{NotaFinalYarriba}{\addtolength{\@PosNotaFinalY}{-\ht\@CajaNotaFinal-\dp\@CajaNotaFinal}}{}% Aquí la caja abre hacia abajo, a menos de que usemos \PosNotaFinal*
 \begin{textblock*}{\wd\@CajaNotaFinal}(\@PosNotaFinalXuno,\@PosNotaFinalY)%
  \usebox{\@CajaNotaFinal}%
 \end{textblock*}}

%% Comando que coloca el número de control.
\newcommand{\@PosicionaNumControl}{%
 \savebox{\@CajaControl}{\@EstiloNumControl}%
 \ifbool{ControlXnegativo}{\addtolength{\@PosControlX}{\textwidth-\wd\@CajaControl}}{}%
 \ifbool{ControlYnegativo}{\addtolength{\@PosControlY}{\paperheight}}{}%
 \ifbool{ControlYarriba}{\addtolength{\@PosControlY}{-\ht\@CajaControl-\dp\@CajaControl}}{}%
 \begin{textblock*}{\wd\@CajaControl}(\@PosControlX,\@PosControlY)%
  \usebox{\@CajaControl}%
 \end{textblock*}}

%% Comando que coloca la denominación del documento.
\newcommand{\@PosicionaDenom}{%
 \savebox{\@CajaDenom}{\@EstiloDenom}%
 \ifbool{DenomXnegativo}{\addtolength{\@PosDenomX}{\textwidth-\wd\@CajaDenom}}{}%
 \ifbool{DenomYnegativo}{\addtolength{\@PosDenomY}{\paperheight}}{}%
 \ifbool{DenomYarriba}{\addtolength{\@PosDenomY}{-\ht\@CajaDenom-\dp\@CajaDenom}}{}%
 \begin{textblock*}{\wd\@CajaDenom}(\@PosDenomX,\@PosDenomY)%
  \usebox{\@CajaDenom}%
 \end{textblock*}}

%% Comando que coloca la(s) fecha(s).
\newcommand{\@PosicionaFechas}{%
 \savebox{\@CajaFecha}{\@EstiloFechas}%
 \ifbool{FechaXnegativo}{\addtolength{\@PosFechaX}{\textwidth-\wd\@CajaFecha}}{}%
 \ifbool{FechaYnegativo}{\addtolength{\@PosFechaY}{\paperheight}}{}%
 \ifbool{FechaYarriba}{\addtolength{\@PosFechaY}{-\ht\@CajaFecha-\dp\@CajaFecha}}{}%
 \begin{textblock*}{\wd\@CajaFecha}(\@PosFechaX,\@PosFechaY)%
  \usebox{\@CajaFecha}%
 \end{textblock*}}

%% Comando que coloca la identificación del reporte al inicio.
\newcommand{\@EscribeDatosReporte}{%
 {\@LetraEncReporte
  \setlength{\parindent}{0pt}%
  {\bfseries \@@Proveedor} \@Proveedor \par
  {\bfseries \@@Cliente} \@RazonSocial \par
  \ifdefvoid{\@Contacto}{}{{\bfseries \@@Contacto} \@Contacto \par}%
  \ifdefvoid{\@TituloReporte}{}{\vspace*{\baselineskip}{\@LetraTitReporte \centering \@TituloReporte \par}}}}

%% Comando que coloca las dos cajas del encabezado: la de los datos del cliente y la del resumen.
\newcommand{\@EscribeDatos}{%
 {\@LetraEncFactura% Abrimos un nuevo grupo para que el cambio de tipografía y la redefinición del entorno 'tikzpicture' sean locales.
  \ifdim\@GrosorLineaEncabezado=0pt%
   \setlength{\@SeparaInternaEncabezado}{0pt}\setlength{\@EsquinaLineaEncabezado}{0pt}\setlength{\@SombraEncabezado}{0pt}%
   \renewenvironment{tikzpicture}{\NewDocumentCommand{\node}{o +m}{####2}}{}%
  \fi
  \begin{tikzpicture}%
   \expandafter\node\expandafter[\@DatosTIKZ]{%
   \parbox{\textwidth-\@SombraEncabezado-2\@SeparaInternaEncabezado-\@GrosorLineaEncabezado}{\@DatosAux}}%
    \ifdim\@GrosorLineaEncabezado=0pt\else ;\fi
  \end{tikzpicture}%
  \par\vspace*{\@SombraEncabezado}}}%  Para que textblock* esté en modo vertical.

\newcommand{\@EscribeDatosResumen}{%
 {\@LetraEncFactura% Abrimos un nuevo grupo para que el cambio de tipografía y la redefinición del entorno 'tikzpicture' sean locales.
  \ifdim\@GrosorLineaEncabezado=0pt%
   \setlength{\@SeparaInternaEncabezado}{0pt}\setlength{\@EsquinaLineaEncabezado}{0pt}\setlength{\@SombraEncabezado}{0pt}%
   \renewenvironment{tikzpicture}{\NewDocumentCommand{\node}{o +m}{####2}}{}%
  \fi
  \savebox{\@CajaDatos}{%
   \parbox[t]{\textwidth/2-\@SeparaDatosResumen/2-2\@SeparaInternaEncabezado-\@GrosorLineaEncabezado-\@SombraEncabezado}{\@DatosAux*}}%
  \savebox{\@CajaResumen}{%
   \parbox[t]{\textwidth/2-\@SeparaDatosResumen/2-2\@SeparaInternaEncabezado-\@GrosorLineaEncabezado-\@SombraEncabezado}{\@ResumenAux}}%
  \setlength{\@TamDatos}{\ht\@CajaDatos+\dp\@CajaDatos}%
  \setlength{\@TamResumen}{\ht\@CajaResumen+\dp\@CajaResumen}%
  \ifdim\@TamDatos>\@TamResumen
   \setlength{\@TamFinalEncabezado}{\@TamDatos}%
  \else
   \setlength{\@TamFinalEncabezadoado}{\@TamResumen}%
  \fi
  \begin{tikzpicture}%
   \expandafter\node\expandafter[\@DatosTIKZ]{%
    \parbox[t][\@TamFinalEncabezado]{\textwidth/2-\@SeparaDatosResumen/2-2\@SeparaInternaEncabezado-\@GrosorLineaEncabezado-\@SombraEncabezado}{%    
    \@DatosAux*}}%
   \ifdim\@GrosorLineaEncabezado=0pt\else ;\fi
  \end{tikzpicture}%
  \hspace*{\@SeparaDatosResumen}\hspace*{\@SombraEncabezado}%
  \begin{tikzpicture}%
   \expandafter\node\expandafter[\@ResumenTIKZ]{%
    \parbox[t][\@TamFinalEncabezado]{\textwidth/2-\@SeparaDatosResumen/2-2\@SeparaInternaEncabezado-\@GrosorLineaEncabezado-\@SombraEncabezado}{%
     \@ResumenAux}}%
   \ifdim\@GrosorLineaEncabezado=0pt\else ;\fi
  \end{tikzpicture}%
  \par\vspace*{\@SombraEncabezado}}}%  El \par es para que textblock* esté en modo vertical.

%% Comando que coloca la caja para la nota previa, con la posibilidad de TikZ.
\newcommand{\@EscribeNotaPrevia}{%
 {% Abrimos un nuevo grupo para que la redefinición del entorno 'tikzpicture' sea local.
  \baselineskip0pt
  \ifdefvoid{\@NotaPrevia}{}{%
   \ifdim\@GrosorLineaNotaPrevia=0pt%
    \setlength{\@SeparaInternaNotaPrevia}{0pt}\setlength{\@EsquinaLineaNotaPrevia}{0pt}\setlength{\@SombraNotaPrevia}{0pt}%
    \renewenvironment{tikzpicture}{\NewDocumentCommand{\node}{o +m}{####2}}{}%
   \fi
   \begin{tikzpicture}%
    \expandafter\node\expandafter[\@NotaPreviaTIKZ]{%
    \parbox{\textwidth-\@SombraNotaPrevia-2\@SeparaInternaNotaPrevia-\@GrosorLineaNotaPrevia}{\@EstiloNotaPrevia}}%
     \ifdim\@GrosorLineaNotaPrevia=0pt\else ;\fi
   \end{tikzpicture}%
   \par\vspace*{\@SombraNotaPrevia}\vspace*{\@SeparaVertical}%
   \ifexpandecuadro\else\vfill\fi}}}

%% Comando que coloca la caja para la nota final, con la posibilidad de TikZ.
\newcommand{\@EscribeNotaFinal}{%
 {% Abrimos un nuevo grupo para que la redefinición del entorno 'tikzpicture' sea local.
  \ifdim\@GrosorLineaNotaFinal=0pt%
   \setlength{\@SeparaInternaNotaFinal}{0pt}\setlength{\@EsquinaLineaNotaFinal}{0pt}\setlength{\@SombraNotaFinal}{0pt}%
   \renewenvironment{tikzpicture}{\NewDocumentCommand{\node}{o +m}{####2}}{}%
  \fi
  \begin{tikzpicture}%
   \expandafter\node\expandafter[\@NotaFinalTIKZ]{%
   \parbox{\textwidth-\@PosNotaFinalXuno-\@PosNotaFinalXdos+\oddsidemargin+1in-\@SombraNotaFinal-2\@SeparaInternaNotaFinal-\@GrosorLineaNotaFinal}{\@EstiloNotaFinal}}%
    \ifdim\@GrosorLineaNotaFinal=0pt\else ;\fi
  \end{tikzpicture}%
  \par\vspace*{\@SombraNotaFinal}}}

%% Comandos auxiliares para los datos y el resumen.
%% En el caso de los datos, la versión \@DatosAux se ejecuta cuando no hay resumen, y la versión \@DatosAux* se ejecuta cuando hay resumen.
\NewDocumentCommand{\@DatosAux}{s}{%
 \hfill {\@LetraTitEnc \@@Datos}\hfill \mbox{}\par
 \IfBooleanT{#1}{\vfill}%
 {\bfseries \if@Nombre \@@Nombre \else \@@RazonSocial \fi} \@RazonSocial
 \IfBooleanTF{#1}{\par}{\dotfill}%
 {\bfseries \@@RIF} {\ttfamily \@RIF}\par
 \ifdefvoid{\@Contacto}{}{{\bfseries \@@Contacto} \@Contacto \par}%
 {\bfseries \@@Direccion} \@Direccion \par
 \ifdefvoid{\@Telefono}{}{{\bfseries \@@Telefono} {\ttfamily \@Telefono}}%
 \ifdefvoid{\@Email}
  {\par}
  {\IfBooleanTF{#1}
   {\par}
   {\ifdefvoid{\@Telefono}{}{\hspace*{\@SeparaTelefonoEmail}\@CaracterSeparaTelefonoEmail\hspace*{\@SeparaTelefonoEmail}}}%
  {\bfseries \@@Email} {\ttfamily \@Email}}%
 \IfBooleanT{#1}{\vfill}}

\newcommand{\@ResumenAux}{\hfill {\@LetraTitEnc \@@Resumen}\hfill \mbox{}\vfill\@Resumen\vfill}

%% Comando que coloca la línea diagonal.
\newcommand{\@DibujaLinea}{%
 \setlength{\@LineaNodoXo}{\zposx{LineaNodoACoordX} sp}%
 \setlength{\@LineaNodoYo}{\zposy{LineaNodoBCoordY} sp}%
 \setlength{\@LineaNodoXf}{\zposx{LineaNodoBCoordX} sp}%
 \setlength{\@LineaNodoYf}{\zposy{LineaNodoBCoordY} sp}%
 \addtolength{\@LineaNodoYo}{\@EspacioVacioInicial-\@EspacioVacioFinal}%
 \if@LineaDesdeInicio
  \addtolength{\@LineaNodoXo}{-\tabcolsep@@}%
 \fi
 \if@LineaHastaFinal
  \addtolength{\@LineaNodoXf}{\tabcolsep@@}%
 \fi
 \iffilas
  \addtolength{\@LineaNodoYf}{-.3\baselineskip@@}%
  \if@LineaDescripcion
   \addtolength{\@LineaNodoXo}{-\tabcolsep@@}%
   \addtolength{\@LineaNodoXf}{\tabcolsep@@}%
  \fi
 \else
  \ifsintotales
  \addtolength{\@LineaNodoYf}{-.3\baselineskip@@}%
  \else
   \if@DescG \addtolength{\@LineaNodoYo}{2\baselineskip@@}\fi
   \if@DescR \if@BIR \addtolength{\@LineaNodoYo}{2\baselineskip@@}\fi\fi
   \if@DescA \if@BIA \addtolength{\@LineaNodoYo}{2\baselineskip@@}\fi\fi
   \if@DescE \if@Exento \addtolength{\@LineaNodoYo}{2\baselineskip@@}\fi\fi
   \addtolength{\@LineaNodoYo}{\baselineskip@@}%
   \addtolength{\@LineaNodoYf}{-.3\baselineskip@@}%
   \ifsinivaexpresado
   \else
    \addtolength{\@LineaNodoYo}{2\baselineskip@@}%
    \if@BIR\addtolength{\@LineaNodoYo}{2\baselineskip@@}\fi
    \if@BIA\addtolength{\@LineaNodoYo}{2\baselineskip@@}\fi
    \if@Exento\addtolength{\@LineaNodoYo}{\baselineskip@@}\fi
   \fi
  \fi
 \fi
 \iflinea
  \begin{textblock*}{\paperwidth-\@LineaNodoXo}(\@LineaNodoXo,\paperheight-\@LineaNodoYo)%
   \begin{tikzpicture}%
    \draw (\@LineaNodoXo,\@LineaNodoYo) -- (\@LineaNodoXf,\@LineaNodoYf);%
   \end{tikzpicture}%
  \end{textblock*}%
 \fi}

%% Escribimos los totales al archivo csv.
\newcommand{\@EscribeTotalesEnCSV}{%
 \ifcsv
  \if@Exento
   \if@DescE
    \immediate\write\csv{Totales: \@@SubtotalE: \Moneda \SubtE^^JTotales: \@@DescuentoE: \Moneda \DescE}%
   \fi
   \immediate\write\csv{Totales: \@@BIE: \Moneda \BIE}%
  \fi
  \if@BIR
   \if@DescE
    \immediate\write\csv{Totales: \@@SubtotalR: \Moneda \SubtR^^JTotales: \@@DescuentoR: \Moneda \DescR}%
   \fi
   \immediate\write\csv{Totales: \@@BIR: \Moneda \BIR^^JTotales: \@@AlicuotaR: \Moneda \AlicuotaR}%
  \fi
  \if@DescG
   \immediate\write\csv{Totales: \@@SubtotalG: \Moneda \SubtG^^JTotales: \@@DescuentoG: \Moneda \DescG}%
  \fi
  \immediate\write\csv{Totales: \@@BIG: \Moneda \BIG^^JTotales: \@@AlicuotaG: \Moneda \AlicuotaG}%
  %% (quitado por no vigente)
  % \FPifeq\AlicuotaGDesc{0}\else\immediate\write\csv{Totales: \@@AlicuotaGDesc: \Moneda \AlicuotaGDesc}\fi
  \if@BIA
   \if@DescA
    \immediate\write\csv{Totales: \@@SubtotalA: \Moneda \SubtA^^JTotales: \@@DescuentoA: \Moneda \DescA}%
   \fi
   \immediate\write\csv{Totales: \@@BIA: \Moneda \BIA^^JTotales: \@@AlicuotaA: \Moneda \AlicuotaA}%
  \fi
  \immediate\write\csv{Totales: \@@Total: \Moneda \Total}%
  \immediate\closeout\csv
 \fi}

%% Hubo que redefinir \cleardoublepage agregándole \thispagestyle{empty} porque si no queda una página en blanco con cabeceras de página usando twoside.
\def\cleardoublepage{%
 \clearpage
 \if@twoside
  \ifodd\c@page
   \else
    \hbox{}\thispagestyle{empty}\newpage
    \if@twocolumn
     \hbox{}\newpage
    \fi
   \fi
 \fi}

%</class>
%<*def>
%% Diseño hecho en TikZ para los datos del cliente, el resumen, la nota previa, la descripción y la nota final.
\def\@DatosTIKZ{%
 draw=black, fill=white, inner sep=\@SeparaInternaEncabezado, line width=\@GrosorLineaEncabezado, rounded corners=\@EsquinaLineaEncabezado, copy shadow={top color=black, bottom color=black, shadow xshift=\@SombraEncabezado, shadow yshift=-\@SombraEncabezado}}

\let\@ResumenTIKZ\@DatosTIKZ

\def\@NotaPreviaTIKZ{%
 draw=black, fill=white, inner sep=\@SeparaInternaNotaPrevia, line width=\@GrosorLineaNotaPrevia, rounded corners=\@EsquinaLineaNotaPrevia, copy shadow={top color=black, bottom color=black, shadow xshift=\@SombraNotaPrevia, shadow yshift=-\@SombraNotaPrevia}}

\def\@DescripcionTIKZ{%
 draw=black, fill=white, inner sep=\@SeparaInternaDescripcion, line width=\@GrosorLineaDescripcion, rounded corners=\@EsquinaLineaDescripcion, copy shadow={top color=black, bottom color=black, shadow xshift=\@SombraDescripcion, shadow yshift=-\@SombraDescripcion}}

\def\@NotaFinalTIKZ{%
 draw=black, fill=white, inner sep=\@SeparaInternaNotaFinal, line width=\@GrosorLineaNotaFinal, rounded corners=\@EsquinaLineaNotaFinal, copy shadow={top color=black, bottom color=black, shadow xshift=\@SombraNotaFinal, shadow yshift=-\@SombraNotaFinal}}

%% El diseño de las distintas partes de la hoja.
\newcommand{\@EstiloFirmas}{
 \@LetraFirmas
 \ifnum \value{@Firmas}=1
 \else
  \parbox{\@TamLineaFirmas}{
   \rule{\@TamLineaFirmas}{\@GrosorLineaFirmas}
   \centering \ifdefvoid{\@Conforme}{\@@Conforme}{\@Conforme}}
 \fi
 \hfill
 \parbox{\@TamLineaFirmas}{
  \rule{\@TamLineaFirmas}{\@GrosorLineaFirmas}
  \centering \ifdefvoid{\@Emisor}{\ifcotizacion \@@Emisor \else \@@Pagado \fi}{\@Emisor}}}

\newcommand{\@EstiloNotaPrevia}{\@LetraNotaPrevia \@NotaPrevia}

\newcommand{\@EstiloNotaFinal}{\@LetraNotaFinal \@NotaFinal}

\newcommand{\@EstiloNumControl}{%
 \renewcommand{\baselinestretch}{0}\selectfont
 \begin{tabular}{@{\extracolsep{0pt}}l}
  \@LetraDenom \@@Control \\[.4em]
  {\@LetraDenom\@NumSerieControl-}\space\@LetraNumeracion\@NumControl \\
 \end{tabular}}

\newcommand{\@EstiloDenom}{%
 \@LetraDenom\@Denominacion
 \ifsinnumero
 \else
  \space\space{\@LetraNumeracion N\textordmasculine\space\@Num}%
 \fi}

\newcommand*{\@EstiloFechas}{%
 \renewcommand{\baselinestretch}{0}%
 \setlength{\tabcolsep}{0pt}%
 \@LetraFecha\begin{tabular}{r@{\extracolsep{1ex}}l@{\extracolsep{0pt}}}
 \@@Fecha & \ttfamily\@Fecha \\[.3em]
 \ifvencimiento \@@FechaVencimiento & \ttfamily\@FechaVencimiento \\ \else \vspace*{-.3em}\fi
 \end{tabular}}

\newcommand{\@InfoPaginaNoTitulo}{\@LetraInfoPagina Reporte correspondiente a la factura n\'umero \@Num \space de fecha \@Fecha}

\newcommand{\@InfoPaginaTitulo}{\@LetraInfoPagina \@TituloReporte \space - \@Proveedor \space - \@Fecha}

%% Estilos de página: normal, sólo cabeceras, sólo pies de página y sólo número de página al pie.
\fancypagestyle{plain}{%
 \expandafter\ps@fancy
 \fancyhf{}%
 \renewcommand{\headrulewidth}{\@GrosorLineaInfoPag}%
 \fancyhead[LEO]{\@InfoPagina}%
 \fancyfoot[C]{\thepage}}

\fancypagestyle{soloarriba}{%
 \expandafter\ps@fancy
 \fancyhf{}%
 \renewcommand{\headrulewidth}{\@GrosorLineaInfoPag}%
 \fancyhead[LO]{\@InfoPagina}%
 \fancyhead[RO]{\thepage}%
 \if@twoside
  \fancyhead[LE]{\thepage}%
  \fancyhead[RE]{\@InfoPagina}%
 \fi}

\fancypagestyle{soloabajo}{%
 \expandafter\ps@fancy
 \fancyhf{}%
 \renewcommand{\headrulewidth}{0pt}%
 \fancyfoot[LO]{\@InfoPagina}%
 \fancyfoot[RO]{\thepage}%
 \if@twoside
  \fancyfoot[RE]{\@InfoPagina}%
  \fancyfoot[LE]{\thepage}%
 \fi}

\fancypagestyle{solonumpag}{%
 \expandafter\ps@fancy
 \fancyhf{}%
 \renewcommand{\headrulewidth}{0pt}%
 \fancyfoot[C]{\thepage}}

%</def>
%<*!class&!def&!firma&!fondo&!membrete&!aux&!batchfile>
%% Descripción de este ejemplo:
%
%%<desc>
%<beispiel>%%  Adaptado a otra legislación
%<factuno|factdos|facttres|factcuatro|factcinco|factseis|factsiete|factocho|factnueve|factdiez|repuno|repdos>%%  Escribimos una factura
%<reptres>%%  No escribimos factura ni usamos las macros que definen valores para su uso
%<factnueve|factdiez|repuno>%%  Escribimos en el reverso de la factura
%<prefac|beispiel>%%  (prefactura) Escribimos una prefactura
%<cotiza>%%  (cotizacion) Escribimos una cotización
%<repuno|repdos|reptres>%%  Escribimos un reporte
%<repuno>%%  (sinreverso) Escribimos el reverso de la factura pero sólo producimos su anverso
%<repuno>%%  (conreporte) Generamos factura y reporte
%<repdos>%%  (soloreporte) Escribimos factura y reporte, pero sólo producimos el reporte
%<reptres>%%  (soloreporte) Generamos solamente el reporte
%<factcinco|factocho>%%  Escribimos factura y reporte, pero sólo producimos la factura
%
%<factdiez>%%  Establecemos algunas opciones de la clase a través de comandos
%
%<factuno>%%  Hacemos una factura básica con todo en sus valores por omisión, no ajustamos nada
%
%<reptres>%%  (twoside) El reporte se imprime en las dos caras del papel
%
%<factcinco>%%  (sincantidad) Cuadro de la factura sin la columna de cantidades
%<factdos|factseis>%%  (numitem) Con la columna para enumerar los ítems
%<factdos|factseis>%%  (descuentos) Con la columna para descuentos individuales
%<factseis>%%  (subtotal) Con la columna extra para subtotal
%<factcinco>%%  (iva) Con la columna para el IVA
%
%<facttres>%%  (agrupado) Agrupados los totales por tipo y no por gravamen
%<factcuatro|factcinco|factsiete>%%  (filas) Totalización en modo de filas
%<factsiete>%%  (centrado) Centrados los totales en el modo de filas
%
%<factdos>%%  (dcu) Primero la descripción, luego la cantidad, luego el precio unitario
%<factcuatro>%%  (duc) Primero la descripción, luego el precio unitario, luego la cantidad
%<factcinco|factseis|factsiete>%%  (denom) Escribe la denominación (“FACTURA”) y el número
%<cotiza>%%  (sinnumero) Escribe la denominación (“COTIZACIÓN”) pero sin número
%<cotiza>%%  (sintotales) No totaliza, para que el cliente tenga un presupuesto por distintos ítems que luego elegirá entre ellos, es decir, una lista de precios
%
%<factdos|factseis>%%  (sinlineasve) Sin líneas verticales externas
%<facttres|factcuatro>%%  (sinlineasvi) Sin líneas verticales internas
%<factdiez>%%  (sinlineasv) Sin líneas verticales
%<factdos>%%  (sinlineahtot) Sin línea horizontal interna después de los ítems y antes de los totales
%<factcuatro>%%  (sinlineahtit) Sin línea horizontal interna después del título de las columnas
%<factseis>%%  (sinlineashi) Sin líneas horizontales internas
%<factsiete>%%  (sinlineashe) Sin líneas horizontales externas
%<factnueve>%%  (sinlineahni)  Sin línea horizontal interna después de la nota interna.
%<factdiez>%%  (sinlineash) Sin líneas horizontales
%<facttres|factcinco>%%  (conlineasha) Separado cada ítem con una línea horizontal
%
%<factcinco|factocho>%%  (sinexpandir) Sin expandir el cuadro de la factura ni la descripción
%<facttres|factcuatro|prefac>%%  (expandecuadro) Expande el cuadro de la factura con espacio en blanco
%<prefac|factcuatro>%%  (linea) Con línea para inutilizar el espacio en blanco
%
%<facttres>%%  (impdesc) Con el tipo de impuesto como parte de la descripción y no de los precios
%<factdos>%%  (centradesc) Centra la descripción en los elementos en la factura
%<factsiete>%%  (sinmarcas) Cuando la descripción de un ítem ocupa más que una línea, no pone el carácter que indica continuación en la siguiente línea ni los caracteres en las columnas vacías
%<factcinco|factseis|factsiete>%%  (todosiva) Forzando a totalizar todos los IVA así no existan productos gravados con ellos
%<factnueve>%%  (G) Escribiendo una letra ‘G’ en los ítems gravados con IVA general así sea el único gravamen a facturar
%<factdiez>%%  (sinmonedatotales) No agrega el símbolo monetario en los totales
%<factocho>%%  (siniva) Sin IVA
%<factdos>%%  (unafirma) Sólo la firma del emisor
%<facttres>%%  (sinfirmas) Sin firmas
%<factdiez>%%  (vencimiento) Con fecha de vencimiento explícita así no haya crédito
%
%<factocho>%%  (csv) Escribe un archivo CSV con la información numérica de la factura
%<beispiel>%%  (nospanish) Sin cargar babel en castellano, ni la codificación T1 para la tipografías
%
%<factdos|factcuatro>%%  Usando \Nombre y \CI en lugar de \RazonSocial y \RIF
%<factdos|factcuatro|reptres>%%  Sin especificar la persona de contacto
%<factcinco|factseis>%%  Sin especificar el teléfono del cliente
%<factcuatro|factseis>%%  Sin especificar el e-mail del cliente
%<factcinco|repuno>%%  (\Emisor) Especificando el nombre del emisor de la factura
%<factcinco|factseis>%%  (\Conforme) Especificando el nombre de quien firma la copia de la factura como recibida
%<repuno>%%  Sin especificar el proveedor, así toma el emisor de la factura como el nombre del proveedor
%<repdos|reptres>%%  (\Proveedor) Especificando el nombre del proveedor para el encabezado del reporte
%
%<facttres>%%  Sin especificar fecha para que tome la fecha del día.
% %<factnueve>%%  Con nota interna automática por el descuento en pagos electrónicos.
% %<factdiez>%%  Con nota interna automática con el monto en Bs.F dada la fecha de emisión de la factura antes de que a la moneda se le quitara el adjetivo “soberano”.
%
%<factcuatro>%%  (\Fecha*) Suministrando la fecha según formato de datetime2 sin alterar cómo sale escrita
%<facttres>%%  (\FormatoFecha) Cambiando el separador de la fecha para que salga escrita dd/mm/yyyy
%<factcinco>%%  (\FormatoFecha) Suministrando la fecha según el formato yyyymmdd pero que salga escrita dd.mm.yyyy
%<factseis>%%  (\FormatoFecha) Cambiando el formato para que la fecha salga escrita en el formato default de datetime2 pero cambiando el separador yyyy/mm/dd
%<factocho>%%  Redefiniendo el formato de la fecha (redefiniendo macros \@@Fecha y \@EstiloFecha)
%
%<factnueve>%%  (\Credito) Especificando la cantidad de días para el crédito
%<facttres>%%  (\DescuentoE y \DescuentoG) Con descuentos totales en porcentajes para exentos y gravables con IVA general
%<factocho>%%  (\Descuento*) Con descuentos totales en dinero neto (y no en porcentajes) aplicables por igual a todos los tipos de elementos existentes
%<prefac>%%  (\Descuento) Con descuentos totales en porcentaje aplicables por igual a todos los tipos de elementos existentes
%
%<factsiete>%%  Redefiniendo el texto para la cantidad (redefiniendo macro \@@Cantidad)
%
%<factsiete>%%  (\GeometriaFactura) Cambiando la geometría del papel de la factura
%<factdiez>%%  (\GeometriaReverso) Cambiando la geometría del reverso de la factura
%<repdos>%%  (\GeometriaReporte) Cambiando la geometría del papel del reporte
%<prefac|cotiza>%%  (\FondoFactura) Con el fondo de la imprenta agregado
%<repuno>%%  (\FondoReporte) Especificamos el fondo o membrete del reporte, sólo para la primera página
%<repdos>%%  (\FondoReporte*) Especificamos el fondo o membrete del reporte para todas las páginas
%<reptres>%%  Sin especificar fondo de reporte
%<repuno>%%  (\InicioReporte) Estableciendo un margen superior distinto para la primera página del reporte por presencia de membrete
%<repuno|reptres>%%  (\TituloReporte) Establecemos el título del reporte sin que salga en las cabeceras ni pies de página
%<repdos>%%  (\TituloReporte*) Establecemos el título del reporte para que salga en las cabeceras o pies de página
%<reptres>%%  (\InfoPagina) Personalizamos la información de la cabecera del reporte y mostramos el nombre del proveedor allí
%<reptres>%%  (\EstiloPagReporte) Cambiamos el estilo de página para que la información del reporte aparezca como pie de página
%<cotiza>%%  (\FirmaFactura) Agregando la firma escaneada en la factura, es decir, en la cotización
%
%<factdos|facttres|factsiete>%%  (\SepVertical) Alterando la separación mínima vertical entre los elementos
%<factdos|factcinco>%%  (\SepDatosResumen) Alterando la separación horizontal entre los datos del cliente y el resumen
%<factdos|facttres|factseis>%%  (\SepEncabezado) Alterando el margen interno desde el contorno hasta el texto en el encabezado
%<factnueve>%%  (\SepNotaPrevia) Alterando el margen interno desde el contorno hasta el texto en la nota previa
%<factdos|factseis>%%  (\SepDescripcion) Alterando la distancia desde el contorno hasta el texto en la descripción
%<factnueve>%%  (\SepNotaFinal) Especificando la distancia desde el contorno hasta el texto de la nota final, predeterminadamente sin contorno
%<factdos|factcuatro>%%  (\SepItemsExtra) Agregando una separación extra entre los ítems en la factura
%<factsiete>%%  (\SepFilas) Alterando la separación horizontal y vertical entre los totales en el modo de filas
%<facttres|factsiete>%%  (\LineaEncabezado) Alterando la línea de las cajas del encabezado, la curvatura en la esquina y la sombra
%<factdiez>%%  (\LineaNotaPrevia) Alterando la línea de la nota previa, para desaparecerla
%<factdos|factsiete>%%  (\LineaDescripcion) Alterando la línea de la descripción, la curvatura en la esquina y la sombra
%<factnueve>%%  (\LineaNotaFinal) Agregando el contorno para la nota final que predeterminadamente está deshabilitado
%<factsiete>%%  (\LineaFirmas) Alterando la longitud de las líneas para firmar y su grosor
%<factsiete>%%  (\PosFecha, \PosFirmas y \PosDenom) Alterando la posición de la fecha, las firmas y de la denominación del documento
%<factnueve>%%  (\PosNotaFinal) Alterando la posición de la nota final
%<factcuatro|factocho>%%  (\Letra***) Cambiando varias de las tipografías
%
%<!factcuatro&!reptres>%%  (\NotaFinal) Con nota final
%<factnueve>%%  Haciendo mención a las fechas y al crédito en la nota final
%<factcuatro>%%  Sin nota final
%<factuno|factdos|factcinco|factocho>%%  (\Resumen) Con resumen
%<facttres|factcuatro|factseis|factsiete|prefac|cotiza|repuno|repdos>%%  Sin resumen
%<factcuatro|factnueve|factdiez>%%  (\NotaPrevia) Con nota previa
%<factocho|factnueve>%%  (\NotaInterna) Con nota interna
%<factnueve>%%  (\NotaExterna) Con nota externa
%
%<factdos|facttres>%%  Con artículos gravados con todos los IVA
%<factseis|factsiete|factocho>%%  Escribimos un ítem cuya descripción ocupa más que una línea
%<factdos|factcuatro|factcinco>%%  (\ItemX) Agregando un ítem sin precio
%
%<!facttres&!reptres&!beispiel>%%  (\Descripcion) Con descripción larga
%<facttres|reptres>%%  Sin descripción larga
%
%<factseis|repdos>%%  Con etiquetas y referencias
%
%<repuno>%%  (\FirmaReporte) Agregando en el reporte la línea para firmar a mano
%<factocho|repdos|reptres>%%  (\FirmaReporte) Agregando la firma escaneada en el reporte
%<factocho|repdos|reptres>%%  Con \FirmaReporte, ajustando el corrimiento en “y” de la firma respecto a la línea
%<repdos|reptres>%%  Con \FirmaReporte, cambiando la ubicación en “x” de la línea de la firma y cambiando la longitud de la línea hacia los lados
%<reptres>%%  Con \FirmaReporte, cambiando el nombre de la persona que va a firmar
%%</desc>
%%
%<factuno>\documentclass{factura}
%<!factuno>\documentclass[
%<prefac|beispiel>prefactura,
%<cotiza>cotizacion,
%<repuno>sinreverso,
%<repuno>conreporte,
%<repdos|reptres>soloreporte,
%<beispiel>nospanish,
%<reptres>twoside,
%<factcinco>sincantidad,
%<factdos|factseis>numitem,
%<factdos|factseis>descuentos,
%<factseis>subtotal,
%<factcinco>iva,
%<factcuatro|factcinco|factsiete>filas,
%<factdiez>filas, % luego queda sin efecto con \filasfalse en el preámbulo
%<factsiete>centrado,
%<facttres>agrupado,
%<factcinco|factseis|factsiete>denom,
%<cotiza>sinnumero, % sin esta opción aparecerá el número de la cotización.
%<cotiza>sintotales,
%<factdos>dcu,
%<factcuatro>duc,
%<factdos|factseis>sinlineasve,
%<facttres|factcuatro>sinlineasvi,
%<factdos>sinlineahtot,
%<factcuatro>sinlineahtit,
%<factseis>sinlineashi,
%<factsiete>sinlineashe,
%<factnueve>sinlineahni,
%<facttres|factcinco>conlineasha,
%<factcinco|factocho>sinexpandir,
%<facttres|factcuatro|prefac>expandecuadro,
%<factcuatro|prefac>linea,
%<facttres>impdesc,
%<factdos>centradesc,
%<factsiete>sinmarcas,
%<factcinco|factseis|factsiete>todosiva,
%<factnueve>G,
%<factdiez>sinmonedatotales,
%<factocho>siniva,
%<factnueve>% electro,
%<factdos>unafirma,
%<prefac|cotiza>% dosfirmas, % para poner las dos firmas.
%<prefac>% unafirma, % para poner la firma del emisor.
%<cotiza>% sinfirmas, % para no poner firmas.
%<facttres>sinfirmas,
%<factdiez>vencimiento,
%<factocho>csv,
%<factdiez>csv, % luego queda sin efecto con \csvfalse en el preámbulo
%<!factuno>]{factura}

%<factdiez>% Cambiamos por aquí como condicionales las opciones de la clase; es lo mismo que arriba en \documentclass.
%<factdiez>\sinlineasvtrue \sinlineashtrue \filasfalse \csvfalse
%<factdiez>
\input{factura-ejemplo-aux.tex}% Se incluye para efectos del ejemplo, así tendremos disponible la macro \Ejemplo.
\usepackage[utf8]{inputenc}
%<beispiel>\usepackage[official]{eurosym}
%<beispiel>\usepackage[ngerman]{babel}
%<beispiel>\usepackage[T1]{fontenc}
%<beispiel>
%<beispiel>\Moneda{\euro{}}
%<beispiel>\makeatletter
%<beispiel>\def\@AlicuotaG{0.19}
%<beispiel>\def\@AlicuotaR{0.07}
%<beispiel>\def\@DenomAlicuotaG{A}
%<beispiel>\def\@DenomAlicuotaR{C}
%<beispiel>\def\@DenomAlicuotaE{B}
%<beispiel>\def\@@Datos{Kundenidentifikation}
%<beispiel>\def\@@Resumen{Zusammenfassung}
%<beispiel>\def\@@Fecha{Datum:}
%<beispiel>\def\@@Nombre{Kunde:} \def\@@RazonSocial{Kunde:}
%<beispiel>\def\@@RIF{USt-IdNr:} \def\@@CT{USt-IdNr:}
%<beispiel>\def\@@Contacto{Kontactperson:}
%<beispiel>\def\@@Direccion{Adresse:}
%<beispiel>\def\@@Telefono{Telefon:}
%<beispiel>\def\@@Email{E-mail:}
%<beispiel>\def\@@Cliente{Kunde:}
%<beispiel>\def\@@Proveedor{Lieferant:}
%<beispiel>\def\@@Cantidad{Menge}
%<beispiel>\def\@@Descripcion{Beschreibung}
%<beispiel>\def\@@PrecioU{Einzelpreis}
%<beispiel>\def\@@PrecioT{Gesamt}
%<beispiel>\def\@@Precio{Preis}
%<beispiel>\def\@@AlicuotaG{MwSt A}
%<beispiel>\def\@@AlicuotaR{MwSt C}
%<beispiel>\def\@@BIG{Netto 19\%}
%<beispiel>\def\@@BIR{Netto 7\%}
%<beispiel>\def\@@BIE{Mwst-befreit}
%<beispiel>\def\@@Total{Brutto}% Zu zahlen / Brutto / Summe / Gesamt
%<beispiel>\def\@@Conforme{der Kunde}
%<beispiel>\def\@@DenomFactura{Rechnung}
%<beispiel>\def\@@DenomCotiza{Voranschlag}
%<beispiel>
%<beispiel>\renewcommand{\@EstiloDenom}{\@LetraDenom\@Denominacion-\ifsinnumero\else\@LetraNumeracion Nr:\space\@Num\fi}
%<beispiel>\renewcommand{\@EstiloNumControl}{}
%<beispiel>\makeatother

%<factcinco|factseis|factsiete|cotiza|prefac|repuno|repdos|beispiel>\Num{1}
%<prefac>% \NumControl{1001}
%<prefac>% \NumSerieControl{9}
%<!factdos&!factcuatro&!beispiel>\RazonSocial{Ayala San Vicente Usureros}
%<factdos|factcuatro|beispiel>\Nombre{Jacinto de Ayala y San Vicente}
%<!factdos&!factcuatro&!beispiel&!reptres>\RIF{Y-12.345.678-9}
%<factdos|factcuatro|beispiel>\CI{V-987.654.321}
%<!factdos&!factcuatro&!reptres&!beispiel>\Contacto{Jacinto de Ayala}
%<!reptres>\Direccion{Av. Ayala, Galpón San Vicente, Sector Arauca, Apure, Venezuela}
%<!factcinco&!factseis&!reptres>\Telefono{+58 (987) 654.32.10}
%<!factcuatro&!factseis&!reptres>\Email{jacinto@ayala.sv}
%<factcinco|repuno>\Emisor{Lázaro de la Vega}
%<factcinco>\Conforme{\Contacto}
%<factseis>\Conforme{Ramela San Vicente}
%<repdos|reptres>\Proveedor{Lázaro de la Vega y San Román}

%<factcinco>\FormatoFecha[yyyymmdd]{ddmmyyyy}{.}
%<facttres>\FormatoFecha{ddmmyyyy}{/}
%<factseis>\FormatoFecha{default}{/}
%<!facttres&!factcuatro&!factcinco>\Fecha{06-11-2019}
%<factcuatro>\Fecha*{2019}{06}{11}
%<factcinco>\Fecha{20191106}
%<factsiete|factocho>
%<factsiete|factocho>\makeatletter
%<factocho> \renewcommand{\@@Fecha}{\slshape FECHA:}
%<factocho> \renewcommand{\@EstiloFechas}{\@LetraFecha\bfseries\@@Fecha\space{\color{blue}\@Fecha}}
%<factsiete> \renewcommand{\@@Cantidad}{Cant.}
%<factsiete|factocho>\makeatother
%<factnueve>\Credito{11}

%<prefac>\Descuento{18}
%<facttres>\DescuentoE{15}
%<facttres>\DescuentoG{20}
%<factocho>\Descuento*{300}
%<prefac|facttres|factocho>
%<factsiete>\GeometriaFactura{legalpaper,vmargin={5cm,5.5cm},hmargin={1cm,1cm},footskip=0pt,headsep=0pt,headheight=0pt}
%<factdiez>\GeometriaReverso{vmargin={4cm,4cm},hmargin={4cm,4cm},footskip=0pt,headsep=0pt,headheight=0pt}
%<repdos>\GeometriaReporte{vmargin={4.4cm,2.5cm},hmargin={3cm,2cm},footskip=1.2cm,headsep=0.777cm,headheight=1.5cm}
%<prefac|cotiza|beispiel>\FondoFactura{factura-ejemplo-fondo}
%<repuno>\FondoReporte{factura-ejemplo-membrete}
%<repdos>\FondoReporte*{factura-ejemplo-membrete}
%<repuno>\InicioReporte{2.2cm}
%<repuno|reptres>\TituloReporte{El título del reporte}
%<repdos>\TituloReporte*{El título del reporte}
%<reptres>\InfoPagina{La cabecera o pie del reporte - \Proveedor}
%<repuno|reptres>% \EstiloPagReporte{soloabajo}
%<repuno|reptres>% \EstiloPagReporte[soloabajo]{plain}
%<repuno|reptres>% \EstiloPagReporte[empty]{plain}
%<repdos|reptres>% \EstiloPagReporte[empty]{soloabajo}
%<reptres>% \EstiloPagReporte{plain}
%<repuno|repdos|reptres>% \EstiloPagReporte{solonumpag}
%<repdos>% \EstiloPagReporte{soloabajo}
%<repuno|reptres>% \EstiloPagReporte[solonumpag]{plain}
%<repuno|reptres>% \EstiloPagReporte[solonumpag]{soloarriba}
%<reptres>\EstiloPagReporte[solonumpag]{soloabajo}
%<repuno>% \EstiloPagReporte[solonumpag]{soloabajo}
%<repuno|reptres>% \EstiloPagReporte[empty]{soloarriba}
%<repuno|reptres>% \EstiloPagReporte{soloarriba}
%<cotiza>\FirmaFactura{\includegraphics[width=4.5cm]{factura-ejemplo-firma}}[0pt][17pt]
%<factdos|facttres|factsiete>\SepVertical{0.75cm}
%<factdos|factcinco>\SepDatosResumen{1cm}
%<factdos|facttres|factseis>\SepEncabezado{16pt}
%<factdos|factcuatro>\SepItemsExtra{5pt}
%<factdos|factseis>\SepDescripcion{16pt}
%<factnueve>\SepNotaPrevia{16pt}
%<factsiete>\SepFilas{1cm}{10pt}
%<factnueve>\SepNotaFinal{16pt}
%<facttres|factsiete>\LineaEncabezado{1pt}{20pt}{3pt}
%<factdiez>\LineaNotaPrevia{0pt}
%<factdos|factsiete>\LineaDescripcion{1pt}{20pt}{3pt}
%<factnueve>\LineaNotaFinal{0.4pt}{4pt}{3pt}
%<factsiete>\LineaFirmas{6.5cm}{2pt}
%<factsiete>% Puede usarse \PosFecha* o \PosDenom*, \PosControl* o \PosNotaFinal* para que el texto se despliegue en la posición 'Y' dada pero hacia arriba y no hacia abajo. \PosFirmas no acepta eso.
%<factnueve>\PosNotaFinal{.9cm}{.9cm}{-5.3cm}
%<factsiete>\PosFecha{-0cm}{-1.5cm}
%<factsiete>\PosFirmas{1cm}{1cm}{-3cm}
%<factsiete>\PosDenom{-0cm}{4.3cm}
%<factcuatro>\LetraDescripcion{\fontsize{20}{30}}
%<factcuatro>\LetraTitEnc{\fontsize{15}{20}\bfseries}
%<factcuatro>\LetraFirmas{\fontsize{16}{20}\itshape}
%<factcuatro>\LetraNotaFinal{\fontsize{18}{28}\bfseries}
%<factocho>\LetraItems{\fontsize{20}{30}}
%<factocho>\LetraNumeros{\fontsize{20}{30}}
%<factocho>\LetraTipoIVA{\fontsize{20}{30}\bfseries}
%<factocho>\LetraTitTotales{\fontsize{20}{30}\slshape}
%<factocho>\LetraNumTotales{\fontsize{20}{30}}
%<factocho>\LetraTitColumnas{\fontsize{15}{20}}
%<factocho>\LetraTitTotal{\fontsize{20}{30}\bfseries\slshape}
%<factocho>\LetraNumTotal{\fontsize{20}{30}\bfseries}
%<factocho>\tabcolsep7pt
%<!factuno&!factcinco&!prefac&!reptres>
%<cotiza>% \Denom{PRESUPUESTO} % También podemos cambiar el nombre del documento.
%<!factcuatro&!factnueve&!beispiel&!reptres>\NotaFinal{La nota final...}
%<factnueve>\NotaFinal{La factura se emitió el \Fecha{} con crédito de \Credito{} dias, siendo su vencimiento el \FechaVencimiento.}
%<beispiel>\NotaFinal{Die Zahlungsbedingungen}
%<factuno|factdos|factcinco|factocho>\Resumen{El resumen opcional}
%<beispiel>\Resumen{Zusammenfassung}
%<factcuatro|factnueve|factdiez>\NotaPrevia{\hfill LA NOTA PREVIA}
%<factocho|factnueve>\NotaInterna{LA NOTA INTERNA}
%<factnueve>\NotaExterna{LA NOTA EXTERNA}

\begin{document}
%<!reptres>
%<!reptres>\begin{factura}
%<!factdos&!factcinco&!factseis&!repdos&!reptres&!beispiel> \Item{1}{ítem con IVA general}{1000}
%<factseis|repdos> \Item{2}{ítem con IVA general, viene con descuento}{1000}{8}[Un ítem que ofrecemos a precio de costo y que está gravado con IVA general]\label{primeritem}
%<factuno|factcuatro|repdos> \Item{4}{otro ítem con IVA general}{100}
%<cotiza> \Item{1}{otro ítem con IVA general}{1100}
%<cotiza> \Item{1}{otro ítem más con IVA general}{1200}
%<facttres|prefac> \ItemR{1}{ítem con IVA reducido}{1000}
%<facttres> \ItemA{1}{ítem con IVA adicional}{1000}
%<facttres> \ItemE{1}{ítem exento de impuesto}{1000}
%<factdos|factseis> \Item{1}{ítem con IVA general, viene con descuento}{1000}{12}
%<factsiete|factocho> \Item{10}{otro ítem con IVA general (descripción larga para mostrar el cambio de línea)}[otro ítem con IVA general (esta descripción es sólo para el .csv)]{1000}
%<factdos> \ItemR{1}{ítem con IVA reducido, viene con descuento}{1000}{5}
%<factdos> \ItemA{1}{ítem con IVA adicional, viene con descuento}{1000}{10}
%<factdos> \ItemE{1}{ítem exento de impuesto}{1000}
%<factdos|factcuatro> \ItemX{1}{línea sin precio}
%<factcinco> \Item{ítem sin cantidad, con IVA general}{1000}
%<factcinco> \ItemA{ítem sin cantidad, con IVA adicional}{1000}
%<factcinco> \ItemX{línea sin precio}
%<beispiel> \Item{1}{Artikel mit 19\% MwSt}{100}
%<beispiel> \ItemR{1}{Artikel mit 7\% MwSt}{100}
%<beispiel> \ItemE{1}{Artikel mit kein MwSt}{100}
%<!reptres>\end{factura}
%<!facttres&!reptres>
%<!facttres&!factseis&!reptres&!beispiel>\Descripcion{La descripción larga}
%<factseis>\Descripcion{
%<factseis>
%<beispiel>\Descripcion{Lange Beschreibung.}
%<factcinco|factocho|factnueve|factdiez|repuno|repdos|reptres>
%<factnueve|factdiez|repuno>\begin{reverso}
%<factnueve|factdiez|repuno>esquina superior izquierda del reverso\hfill esquina superior derecha del reverso\par
%<factnueve|factdiez|repuno>\vfill
%<factnueve|factdiez|repuno>esquina inferior izquierda del reverso\hfill esquina inferior derecha del reverso
%<factnueve|factdiez|repuno>\end{reverso}
%<repuno>
%<factcinco|factocho|repuno|repdos|reptres>\begin{reporte}
%<factcinco|factocho|repuno|repdos|reptres> El reporte - página 1\newpage El reporte - página 2 \newpage El reporte - página 3.
%<repdos>
%<factseis|repdos> Se facturaron en total \arabic{NumItem} artículos.
%<factseis|repdos>
%<factseis|repdos> En el artículo número \itemref{primeritem} se facturaron \cantref{primeritem} “\descref{primeritem}”, cada uno por un valor de \Moneda~\puref{primeritem}, dando un subtotal por este elemento de \Moneda~\subtref{primeritem}; el descuento en ese artículo es del \dctoref{primeritem}\%, dando un total de \Moneda~\ptref{primeritem}, y esta gravado con IVA de \ivaref{primeritem}\%. Ampliando la descripción podemos decir que este ítem es: “\ldescref{primeritem}”.
%<factseis|repdos>
%<factseis|repdos> La base imponible total de esta factura es de \Moneda~\numprint\BIG; el IVA es de \Moneda~\numprint\AlicuotaG,  y el total final es de \Moneda~\numprint\Total.
%<factseis|repdos>
%<factseis|repdos> % Estas son todas las macros para las totalizaciones: \BIG, \BIR, \BIA, \BIE, \AlicuotaG, \AlicuotaR, \AlicuotaA, \SubtG, \SubtR, \SubtA, \SubtE, \DescG, \DescR, \DescA, \DescE.
%<factseis>}
%<repuno> \vspace{1.5cm}\FirmaReporte
%<factocho> \FirmaReporte[\includegraphics[width=4.5cm]{factura-ejemplo-firma}][0pt][17pt]
%<repdos> \FirmaReporte[\includegraphics[width=4.5cm]{factura-ejemplo-firma}][0pt][17pt][3cm][3cm]
%<reptres> \FirmaReporte[\includegraphics[width=4.5cm]{factura-ejemplo-firma}][0pt][17pt]{el que firma}[4cm][0.8cm]
%<factcinco|factocho|repuno|repdos|reptres>\end{reporte}

%% Este es un comando auxiliar suministrado por factura-ejemplo-aux.tex para escribir el detalle del ejemplo en la última hoja.
\Ejemplo

\end{document}
%</!class&!def&!firma&!fondo&!membrete&!aux&!batchfile>
%<*firma>
%% (PSTricks code generated with LaTeXDraw 2.0.8)
%% (Compilar con XeLaTeX)
\documentclass{standalone}
\usepackage[usenames,dvipsnames]{pstricks}
\begin{document}
 \begin{pspicture}(0,-3.19388)(13.8397665,3.18)
  \pscustom[linewidth=0.1,linecolor=red]{
   \newpath
   \moveto(0.0,0.9762225)
   \lineto(1.0057193,0.95771605)
   \curveto(1.5085789,0.94846284)(2.4492223,1.1335286)(2.8870058,1.3278477)
   \curveto(3.3247895,1.5221668)(4.064289,2.0079648)(4.366005,2.2994435)
   \curveto(4.667721,2.5909224)(4.9457726,2.9518008)(4.9221087,3.0212007)
   \curveto(4.8984447,3.0906003)(4.762377,3.16)(4.6499724,3.16)
   \curveto(4.5375686,3.16)(4.1175327,2.9934406)(3.8099012,2.8268814)
   \curveto(3.5022693,2.6603222)(2.721358,1.6748462)(2.2480783,0.85592973)
   \curveto(1.7747989,0.037013244)(1.1358713,-0.91607606)(0.97022337,-1.050249)
   \curveto(0.8045755,-1.1844217)(0.5975156,-1.3185943)(0.5561036,-1.3185943)
   \curveto(0.5146916,-1.3185943)(0.43186772,-1.2121814)(0.39045575,-1.1057684)
   \curveto(0.34904376,-0.9993555)(0.37270775,-0.7495166)(0.4377837,-0.6060907)
   \curveto(0.50285965,-0.46266478)(0.6152636,-0.3608783)(0.6625916,-0.4025183)
   \curveto(0.7099195,-0.44415832)(0.82232344,-0.56907773)(0.8873994,-0.65235716)
   \curveto(0.9524753,-0.7356369)(1.354763,-0.86980957)(1.6919748,-0.9207028)
   \curveto(2.0291865,-0.97159606)(2.5202143,-1.0178626)(2.67403,-1.0132358)
   \curveto(2.827846,-1.008609)(3.0112422,-0.9207028)(3.0408218,-0.8374231)
   \curveto(3.0704017,-0.75414366)(3.0999818,-0.69862366)(3.0999818,-0.7263837)
   \curveto(3.0999818,-0.75414366)(3.1709735,-0.80966306)(3.2419655,-0.8374231)
   \curveto(3.3129578,-0.8651831)(3.4667737,-0.80041015)(3.5495975,-0.7078772)
   \curveto(3.6324213,-0.6153442)(3.7744055,-0.6199707)(3.8335655,-0.7171304)
   \curveto(3.8927252,-0.81429017)(4.141197,-0.9114496)(4.330509,-0.9114496)
   \curveto(4.5198207,-0.9114496)(4.7978725,-0.86055636)(4.8866124,-0.80966306)
   \curveto(4.9753523,-0.75877017)(5.0995884,-0.74489015)(5.135084,-0.7819034)
   \curveto(5.1705804,-0.8189166)(5.324396,-0.8281702)(5.442716,-0.80041015)
   \curveto(5.5610366,-0.7726501)(5.6971045,-0.7680234)(5.7148523,-0.7911566)
   \curveto(5.7326,-0.81428987)(5.8745837,-0.83279663)(5.9988194,-0.8281699)
   \curveto(6.1230555,-0.8235431)(6.2887034,-0.78653014)(6.330116,-0.75414366)
   \curveto(6.3715277,-0.7217572)(6.454352,-0.69862366)(6.495764,-0.7078772)
   \curveto(6.537175,-0.7171304)(6.6554956,-0.6060907)(6.7324038,-0.48579803)
   \curveto(6.8093114,-0.36550537)(7.0163713,-0.041640013)(7.1465235,0.16193238)
   \curveto(7.2766747,0.36550477)(7.4245753,0.12491944)(7.442323,-0.3192389)
   \curveto(7.4600706,-0.7633969)(7.5133147,-1.6748462)(7.548811,-2.1421375)
   \curveto(7.5843067,-2.6094286)(7.5251465,-3.10448)(7.4304905,-3.13224)
   \curveto(7.335835,-3.16)(7.1465235,-3.17388)(7.051867,-3.16)
   \curveto(6.9572105,-3.14612)(6.9276304,-2.748228)(6.992707,-2.3642163)
   \curveto(7.057783,-1.980205)(7.436407,-1.4296337)(7.7499547,-1.2630746)
   \curveto(8.063502,-1.0965155)(8.619606,-0.80966306)(8.862162,-0.6893704)
   \curveto(9.104718,-0.56907773)(9.394602,-0.5181845)(9.441929,-0.58758426)
   \curveto(9.489257,-0.656984)(9.589829,-0.67549074)(9.643073,-0.6245975)
   \curveto(9.696317,-0.57370424)(9.767309,-0.5089313)(9.785057,-0.49505126)
   \curveto(9.802805,-0.48117125)(9.897461,-0.49505126)(9.974369,-0.5228113)
   \curveto(10.051277,-0.55057126)(10.175512,-0.62922424)(10.22284,-0.6801172)
   \curveto(10.270168,-0.73101014)(10.388489,-0.76802367)(10.45948,-0.75414366)
   \curveto(10.530473,-0.7402637)(10.65471,-0.6616107)(10.7079525,-0.59683746)
   \curveto(10.761196,-0.5320642)(10.838105,-0.5043045)(10.861769,-0.54131776)
   \curveto(10.885432,-0.5783307)(10.932761,-0.6477307)(10.956425,-0.6801172)
   \curveto(10.980088,-0.7125037)(11.033333,-0.75877017)(11.062913,-0.7726501)
   \curveto(11.092493,-0.78653014)(11.157568,-0.7680234)(11.193064,-0.7356369)
   \curveto(11.2285595,-0.7032504)(11.305469,-0.6662372)(11.34688,-0.6616107)
   \curveto(11.388292,-0.656984)(11.512528,-0.48579773)(11.595351,-0.31923857)
   \curveto(11.678176,-0.15267944)(11.778748,-0.02313324)(11.796496,-0.060146485)
   \curveto(11.814244,-0.09715942)(11.843824,-0.6569836)(11.855657,-1.1797949)
   \curveto(11.867488,-1.7026062)(11.87932,-2.2901905)(11.87932,-2.3549633)
   \curveto(11.87932,-2.4197364)(11.938479,-2.0773644)(11.997639,-1.6702198)
   \curveto(12.056799,-1.263075)(12.186951,-0.85592985)(12.257943,-0.85592985)
   \curveto(12.328936,-0.85592985)(12.459087,-0.8374231)(12.518247,-0.8189166)
   \curveto(12.577407,-0.80041015)(12.719391,-0.78653014)(12.802215,-0.7911566)
   \curveto(12.885039,-0.7957831)(13.03294,-0.7726501)(13.098015,-0.74489015)
   \curveto(13.163091,-0.7171301)(13.257747,-0.5922107)(13.287327,-0.49505126)
   \curveto(13.316907,-0.39789185)(13.352403,-0.33774537)(13.358318,-0.3747583)
   \curveto(13.364233,-0.41177154)(13.3878975,-0.48117125)(13.405647,-0.51355773)
   \curveto(13.423395,-0.5459442)(13.48847,-0.57370424)(13.535798,-0.56907773)
   \curveto(13.583126,-0.564451)(13.66595,-0.5598242)(13.701447,-0.5598242)
   \curveto(13.736943,-0.5598242)(13.784271,-0.5551978)(13.819766,-0.5413177)}
  \pscustom[linewidth=0.1,linecolor=red]{
   \newpath
   \moveto(12.952644,0.18)
   \lineto(13.020389,0.13)
   \curveto(13.05426,0.105)(13.135551,0.085)(13.182974,0.09)
   \curveto(13.230394,0.095)(13.304913,0.13)(13.332009,0.16)
   \curveto(13.359106,0.19)(13.399754,0.285)(13.413303,0.35)
   \curveto(13.426852,0.415)(13.4404,0.49)(13.4404,0.52)}
 \end{pspicture}
\end{document}
%</firma>
%<*fondo|membrete>
\documentclass[12pt]{article}
\usepackage[utf8]{inputenc}
\usepackage{tikz}\usetikzlibrary{positioning}
\usepackage{calc}
\usepackage[letterpaper,hmargin={1cm,1cm},vmargin={1cm,1cm},footskip=0pt,headsep=0pt,headheight=0pt]{geometry}
\parindent0pt
\lineskip0pt
\pagestyle{empty}
\begin{document}
 \begin{minipage}[b][\textheight]{\textwidth}
  \baselineskip13pt
  \parskip0pt
  \lineskip0pt
  \hspace*{\fill}%
  \begin{tikzpicture}
   \node (Nombre) [draw=none, fill=gray, rounded corners=4pt]{%
    \parbox{0.57\textwidth}{%
     \vspace*{0.1cm}%
     \centering
     \color{white}\fontfamily{ppl}\LARGE\bfseries
     Lázaro de la Vega y San Román}};
   \node [node distance=0pt, below = of Nombre, outer sep=0pt, draw=gray, line width=2pt, inner sep=3.5pt, rounded corners=8pt]{%
    \parbox{0.66\textwidth}{%
     \vspace*{3pt}%
     \parskip2pt
     \centering
     \color{gray}\fontfamily{ppl}\footnotesize\bfseries
     Av. Garcilaso de la Vega, Qta. San Román, Santa Mónica, Caracas, Venezuela.
     \par
     Teléfono: +58 (123) 456.78.90 / email: lázaro@sanrom.án
     \vspace*{3pt}}};
  \end{tikzpicture}
  \hspace*{\fill}\par
  \hspace*{\fill}%
  \parbox{0.63\textwidth}{%
   \color{gray}\fontfamily{aer}\footnotesize\bfseries
   \hfill
   RIF: V-87.654.321-0}%
  \hspace*{\fill}\par
  \vfill\mbox{}%
%<fondo>  \fontsize{6}{8}\selectfont
%<fondo>  \hspace*{0.5cm}%
%<fondo>  \parskip2pt
%<fondo>  {\bfseries ESTA FORMA LIBRE VA SIN TACHADURA NI ENMIENDA \hfill SÓLO EL ORIGINAL DA DERECHO A CRÉDITO FISCAL}
%<fondo>  \hspace*{0.5cm}\par
%<fondo>  \fontdimen2\font=0.55em
%<fondo>  Imprenta: \hfill IMPRESOS DE LA VEGA. Av. Garcilaso de la Vega, Qta. San Román, Santa Mónica, Caracas, Venezuela. RIF: R-87.654.321-0. Tlf: (0123) 456.78.90\newline
%<fondo>  \fontdimen2\font=0.65em
%<fondo>  e-mail: \hfill imprenta@sanrom.án. No. Providencia SENIAT 01/01060 de fecha 12/11/2008 - Control desde No. 00-000001 hasta No. 00-010000 / Fecha 01/01/2018.\par
 \end{minipage}
\end{document}
%</fondo|membrete>
%<*aux>
%% Helper file; not to be compiled standalone, only to be included on examples.
%% Archivo auxiliar; no para ser compilado sino para ser incluido en los ejemplos.
%%
\usepackage{xspace,catchfilebetweentags,regexpatch}
\def\ejemplooptcolor{\color{blue}}
\def\ejemplomacrocolor{\color{purple}}
\makeatletter
\catcode`\%=11
\catcode`\|=14
\newcommand{\Ejemplo}{|
 \newpage
 \ClearShipoutPictureBG
 \newgeometry{vmargin={2cm,2cm},hmargin={2cm,2cm},footskip=0pt,headheight=0pt,headsep=0pt,footnotesep=0pt}|
 \thispagestyle{empty}|
 \catcode`\<=12
 \catcode`\>=12
 \CatchFileBetweenDelims{\DescEjemplo}{\jobname.tex}{<desc>}{</desc>}[\catcode37=11\catcode13=14\makeatletter]|
 \catcode`\%=11
 \makeatletter
 \regexpatchcmd*{\DescEjemplo}{\cC.\cO[*]*}{\c{begingroup}\c{ejemplomacrocolor}\c{ttfamily}\c{string}\0\c{endgroup}\c{xspace}}{}{}|
 \regexpatchcmd*{\DescEjemplo}{(\()(\cL[a-zG]*)(\))}{(\c{begingroup}\c{ejemplooptcolor}\c{ttfamily}\2\c{endgroup})}{}{}|
 \xpatchcmd*{\DescEjemplo}{%% }{\item}{}{}|
 \xpatchcmd*{\DescEjemplo}{}{}{}|
 \makeatother
 {\bfseries Descripci\'on de este ejemplo {\slshape \jobname.tex}:}
 \begin{itemize}
  \DescEjemplo
 \end{itemize}
 \catcode`\%=14
 \catcode`\<=13
 \catcode`\>=13
 }
\catcode`\%=14
\catcode`\|=12
\makeatother
%</aux>
%    \end{macrocode}
% \Finale
% \starttodo
%  Cargamos los valores predeterminados.

%% Comando para escribir el encabezado en la factura.
\newcommand\@EncabezadoFactura{%
 \ifdefvoid{\@Resumen}{\@EscribeDatos}{\@EscribeDatosResumen}%
 \vspace*{\@SeparaVertical}%
 \ifexpandecuadro\else\vfill\fi}

%% Comando para escribir el encabezado en el reporte.
\newcommand\@EncabezadoReporte{%
 \vspace*{-\topskip}%
 \if@FondoReporte
  \if@FondoReporteStar
  \else
   \vspace*{\@InicioReporte}%
  \fi
 \fi
 \@EscribeDatosReporte
 \@LetraReporte}

%% El ambiente 'factura', usando el paquete 'environ' para su definición, y el comando auxiliar \@FacturaAuxCmd.
%% Gracias a Heiko Oberdiek y a David Carlisle (Thanks to Heiko Oberdiek and David Carlisle).
%% http://tex.stackexchange.com/questions/108565/know-the-cell-width-with-tabular-or-include-a-paragraph-box-without-specifying
\NewEnviron{factura}{%
 \expandafter\@FacturaAuxCmd\expandafter{\BODY}%
 \ifnum\value{page}=1\else\CLASSERROR{La factura ocupa mas que una pagina}\fi}

\NewEnviron{reverso}{%
 \if@factura
   \expandafter\newgeometry\expandafter{\@GeometriaReverso}%
   \topskip1sp% ¡¡No puedo poner 0 porque se genera una página en blanco extra!!
  \begin{minipage}[t][\textheight-\pagetotal]{\linewidth}%
   \thispagestyle{empty}%
   \@LetraReverso\BODY
  \end{minipage}
  \clearpage
 \fi}

%% Comando auxiliar para agregar un carácter al final de cada línea en la descripción dentro del cuadro de la factura si ocupa ésta más de una línea.
%% Gracias a Gustavo Mezzetti y a David Carlisle (Thanks to Gustavo Mezzetti and David Carlisle).
%% http://tex.stackexchange.com/questions/296035/insert-a-special-character-at-each-end-of-line-in-a-paragraph
%% http://tex.stackexchange.com/questions/248944/how-to-put-before-lines
\newcommand{\@DescripcionAuxCmd}[2]{%
 \immediate\write\@auxout{\gdef\noexpand\@TamDescripcion{\the\linewidth}}% Para poder establecer la línea diagonal hay que saber el tamaño del campo de la descripción.
 \ifimpdesc\settowidth{\@TempDimen}{\@EscribeTipoDeImpuesto{#2}}\fi
 \savebox{\@CajaPrueba}{%
  \parbox[t]{\linewidth}{%
   \David@XStart
   \ifcentradesc\centering\arraybackslash\fi
   #1%
   \David@XEnd
   \ifdim\@TempDimen=0pt%
   \else
    \hfill\kern\@SeparaFinLineaDesc\@EscribeTipoDeImpuesto{#2}%
   \fi
   \par\global\count1=\prevgraf}}%
 \ifsinmarcas
  \usebox{\@CajaPrueba}%
 \else
  \ifnum\count1=1%
   \usebox{\@CajaPrueba}%
  \else
   \ifdim\@TempDimen>\wd\@CajaCaracterCR
   \else
    \setlength{\@TempDimen}{\wd\@CajaCaracterCR}%
   \fi
   \parbox[t]{\linewidth-\@TempDimen-\@SeparaFinLineaDesc}{%
    \David@XStart
    \ifcentradesc\centering\arraybackslash\fi
    #1%
    \David@XEnd
    \par
    \global\count1=\prevgraf% Esto lo necesitamos en otro lado para calcular el espacio extra en los ítems.
    \count0=\prevgraf
    \count2=\prevgraf
    \loop
     \ifnum\count0=\count2
      \ifimpdesc
       \vskip-\baselineskip
       \mbox{}\hfill
       \rlap{%
        \hspace*{\@TempDimen}%
        \hspace*{\@SeparaFinLineaDesc}%
        \llap{\@EscribeTipoDeImpuesto{#2}}}%
      \fi
     \else
      \mbox{}\hfill
      \rlap{%
       \hspace*{\@TempDimen}%
       \hspace*{\@SeparaFinLineaDesc}%
       \hspace*{-\wd\@CajaCaracterCR}%
       \usebox{\@CajaCaracterCR}}%
     \fi
     \vskip-2\baselineskip
     \advance\count0 -1
     \ifnum\count0 > 0
    \repeat
    \kern\count2\baselineskip\mbox{}%
    \strut}%
  \fi
 \fi}

%% Comandos auxiliares para colocar en \parbox cada celda.
\newcommand{\@NumItemAuxCmd}[1]{%
 \savebox{\@CajaPrueba}{#1}%
 \ifdim\wd\@CajaPrueba > \@TamNumItem \global\@TamNumItem=\wd\@CajaPrueba\fi
 \parbox[t]{\@TamNumItem}{#1}}

\newcommand{\@CantidadAuxCmd}[1]{%
 \savebox{\@CajaPrueba}{#1}%
 \ifdim\wd\@CajaPrueba > \@TamCantidad \global\@TamCantidad=\wd\@CajaPrueba\fi
 \parbox[t]{\@TamCantidad}{#1}}

\newcommand{\@PrecioUAuxCmd}[1]{%
 \savebox{\@CajaPrueba}{#1}%
 \ifdim\wd\@CajaPrueba > \@TamPrecioU \global\@TamPrecioU=\wd\@CajaPrueba\fi
 \parbox[t]{\@TamPrecioU}{#1}}

\newcommand{\@PrecioSAuxCmd}[1]{%
 \savebox{\@CajaPrueba}{#1}%
 \ifdim\wd\@CajaPrueba > \@TamPrecioS \global\@TamPrecioS=\wd\@CajaPrueba\fi
 \parbox[t]{\@TamPrecioS}{#1}}

\newcommand{\@DescuentoAuxCmd}[1]{%
 \savebox{\@CajaPrueba}{#1}%
 \ifdim\wd\@CajaPrueba > \@TamDescuento \global\@TamDescuento=\wd\@CajaPrueba\fi
 \parbox[t]{\@TamDescuento}{#1}}

\newcommand{\@IVAAuxCmd}[1]{%
 \savebox{\@CajaPrueba}{#1}%
 \ifdim\wd\@CajaPrueba > \@TamIVA \global\@TamIVA=\wd\@CajaPrueba\fi
 \parbox[t]{\@TamIVA}{#1}}

\newcommand{\@PrecioTAuxCmd}[1]{%
 \savebox{\@CajaPrueba}{#1}%
 \ifdim\wd\@CajaPrueba > \@TamPrecioT \global\@TamPrecioT=\wd\@CajaPrueba\fi
 \parbox[t]{\@TamPrecioT}{#1}}

%% Presentación de totales en modo de columnas.
\newcommand{\@EscribeTotalCol}[1]{%
 \expandafter\FPifeq\csname Desc#1\endcsname{0}%
  \ifsinivaexpresado\else\@EscribeTotalColBI{#1}\fi
 \else
  \@EscribeTotalColSubt{#1}%
  \@EscribeTotalColDesc{#1}%
  \ifsinivaexpresado\else\@EscribeTotalColBI{#1}\fi
 \fi}

%% Forma agupada para la presentación de totales en modo de columnas.
\newcommand{\@EscribeTotalColAgrupado}{%
 \if@Exento\if@DescE\@EscribeTotalColSubt{E}\fi\fi
 \if@BIR\if@DescR\@EscribeTotalColSubt{R}\fi\fi
 \if@DescG\@EscribeTotalColSubt{G}\fi
 \if@BIA\if@DescA\@EscribeTotalColSubt{A}\fi\fi
 \if@Exento\if@DescE\@EscribeTotalColDesc{E}\fi\fi
 \if@BIR\if@DescR\@EscribeTotalColDesc{R}\fi\fi
 \if@DescG\@EscribeTotalColDesc{G}\fi
 \if@BIA\if@DescA\@EscribeTotalColDesc{A}\fi\fi
 \ifsinivaexpresado
 \else
  \if@Exento\@EscribeTotalColBI{E}\fi
  \if@BIR\@EscribeTotalColBI{R}\fi
  \@EscribeTotalColBI{G}%
  \if@BIA\@EscribeTotalColBI{A}\fi
  \if@BIR\@EscribeTotalColAlicuota{R}\fi
  \@EscribeTotalColAlicuota{G}%
  \if@BIA\@EscribeTotalColAlicuota{A}\fi
  %% (quitado por no vigente)
  % \ifbool{Electro}{\@EscribeTotalColAlicuotaGDesc}{}%
 \fi}

\newcommand{\@EscribeTotalColBI}[1]{%
 \@LetraTitTotales \csname @@BI#1\endcsname
  \ifx#1E%
  \else
   \ifG\ \hfill(\csname @Alicuota#1porc\endcsname\%)\else\mbox{}\fi
  \fi
 & \@LetraNumTotales\@SignoMonetarioAux\numprint{\csname BI#1\endcsname} \\}

\newcommand{\@EscribeTotalColAlicuota}[1]{%
 {\@LetraTitTotales \csname @@Alicuota#1\endcsname\ \hfill (\csname @Alicuota#1porc\endcsname\%)}
 & \@LetraNumTotales\@SignoMonetarioAux\numprint{\csname Alicuota#1\endcsname} \\}

%% (quitado por no vigente)
 % \newcommand{\@EscribeTotalColAlicuotaGDesc}{%
 % {\@LetraTitTotales \@@AlicuotaGDesc\ \hfill (\@AlicuotaGDescPorc\%)}
 % & \@LetraNumTotales\@SignoMonetarioAux\numprint{\AlicuotaGDesc} \\}

\newcommand{\@EscribeTotalColSubt}[1]{%
 {\@LetraTitTotales \csname @@Subtotal#1\endcsname}
 & \@LetraNumTotales\@SignoMonetarioAux\numprint{\csname Subt#1\endcsname} \\}

\newcommand{\@EscribeTotalColDesc}[1]{%
 {\@LetraTitTotales \csname @@Descuento#1\endcsname
 \csname if@Desc#1num\endcsname \else \ \hfill (\csname @Descuento#1porc\endcsname\%)\fi}
 & \@LetraNumTotales\@SignoMonetarioAux\numprint{\csname Desc#1\endcsname} \\}

%% Presentación de los totales en modo filas.
\newcommand{\@EscribeTotalFilasBI}[1]{%
 \@EscribeTotalFilasBIizq{#1}: & \@EscribeTotalFilasBIder{#1}}

\newcommand{\@EscribeTotalFilasAlicuota}[1]{%
 \@EscribeTotalFilasAlicuotaizq{#1}: & \@EscribeTotalFilasAlicuotader{#1}}

%% (no implementado)
 % \newcommand{\@EscribeTotalFilasDesc}[1]{%
 %  \@EscribeTotalFilasDescizq{#1}: & \@EscribeTotalFilasDescder{#1}}

\newcommand{\@EscribeTotalFilasSoloBI}[1]{%
 {\@LetraTitTotales\@EscribeTotalFilasBIizq{#1}:} {\@LetraNumTotales\@EscribeTotalFilasBIder{#1}}}

%% (no implementado)
 % \newcommand{\@EscribeTotalFilasSoloDesc}[1]{%
 % {\@LetraTitTotales\@EscribeTotalFilasDescizq{#1}:} {\@LetraNumTotales\@EscribeTotalFilasDescder{#1}}}

\newcommand{\@EscribeTotalFilasSoloAlicuota}[1]{%
 {\@LetraTitTotales\@EscribeTotalFilasAlicuotaizq{#1}:} {\@LetraNumTotales\@EscribeTotalFilasAlicuotader{#1}}}

\newcommand{\@EscribeTotalFilasBIizq}[1]{%
 \@LetraTitTotales\csname @@BI#1\endcsname
 \ifx#1G%
  \ifG \ (\@AlicuotaGporc\%)\fi%  Si hay otra B.I. que no sea general, escribimos el porcentaje explícito en la general.
 \else
  \ifx#1E%
  \else
  \ (\csname @Alicuota#1porc\endcsname\%)%
  \fi
 \fi}

\newcommand{\@EscribeTotalFilasBIder}[1]{%
 \@LetraNumTotales\@SignoMonetarioAux\numprint{\csname BI#1\endcsname}}

\newcommand{\@EscribeTotalFilasAlicuotaizq}[1]{%
 \@LetraTitTotales\csname @@Alicuota#1\endcsname \ (\csname @Alicuota#1porc\endcsname\%)}

\newcommand{\@EscribeTotalFilasAlicuotader}[1]{%
 \@LetraNumTotales\@SignoMonetarioAux\numprint{\csname Alicuota#1\endcsname}}

%% (no implementado)
 % \newcommand{\@EscribeTotalFilasDescizq}[1]{%
 %  \@LetraTitTotales\csname @@Descuento#1\endcsname \ (\csname @Descuento#1porc\endcsname\%)}

 % \newcommand{\@EscribeTotalFilasDescder}[1]{%
 %  \@LetraNumTotales\@SignoMonetarioAux\numprint{\csname Desc#1\endcsname}}

\newcommand{\@EscribeTotalFilasAuxTotal}{\@LetraTitTotal\@@Total: & \@LetraNumTotal\@SignoMonetarioAux\numprint{\Total}}
 
\newcommand{\@EscribeTotalFilas}[2][]{%
 \begin{tabular}[b]{@{}l@{\extracolsep{1ex}}r@{}}% ¡Esta tabla tiene que abrir hacia abajo para que la separación entre las filas se pueda colocar con \@SeparaFilas!
  \csname @EscribeTotalFilasBI\endcsname{#2}%
  \ifx#2E%
   \ifx#1T%
    \\[\@SeparaFilas]%
    \@EscribeTotalFilasAuxTotal
   \fi
  \else
   \\[\@SeparaFilas]%
   \csname @EscribeTotalFilasAlicuota\endcsname{#2}%
  \fi
 \end{tabular}}

%% Comando auxiliar para escribir el signo monetario cuando aplica
\newcommand{\@SignoMonetarioAux}{%
 \ifsinmonedatotales\else\@Moneda\hspace*{\@EspacioSignoMonetario}\fi}

\newcommand{\@GeneraCajasTotalesFilas}{%
 \if@Exento
  \newbox{\@CajaBIE}\newbox{\@CajaBIET}%
  \savebox{\@CajaBIE}{\@EscribeTotalFilasSoloBI{E}}%
  \savebox{\@CajaBIET}{\@EscribeTotalFilas[T]{E}}%
 \fi
 \if@BIR
  \addtocounter{@IVANoGeneral}{1}%
  \newbox{\@CajaBIR}\newbox{\@CajaAliR}\newbox{\@CajaBIAliR}%
  \savebox{\@CajaBIR}{\@EscribeTotalFilasSoloBI{R}}%
  \savebox{\@CajaAliR}{\@EscribeTotalFilasSoloAlicuota{R}}%
  \savebox{\@CajaBIAliR}{\@EscribeTotalFilas{R}}%
 \fi
 \newbox{\@CajaBIG}\newbox{\@CajaAliG}\newbox{\@CajaBIAliG}%
 \savebox{\@CajaBIG}{\@EscribeTotalFilasSoloBI{G}}%
 \savebox{\@CajaAliG}{\@EscribeTotalFilasSoloAlicuota{G}}%
 \savebox{\@CajaBIAliG}{\@EscribeTotalFilas{G}}%
 \if@BIA
  \addtocounter{@IVANoGeneral}{1}%
  \newbox{\@CajaBIA}\newbox{\@CajaAliA}\newbox{\@CajaBIAliA}%
  \savebox{\@CajaBIA}{\@EscribeTotalFilasSoloBI{A}}%
  \savebox{\@CajaAliA}{\@EscribeTotalFilasSoloAlicuota{A}}%
  \savebox{\@CajaBIAliA}{\@EscribeTotalFilas{A}}%
 \fi}

\newcommand{\@EscribeTotalesFilasGET}{%
 \hspace*{\@SepTotales}\usebox{\@CajaBIG}%
 \hspace*{\@SepTotales}\usebox{\@CajaAliG}%
 \if@Exento\hspace*{\@SepTotales}\usebox{\@CajaBIE}\fi
 \hspace*{\@SepTotales}\usebox{\@CajaTotal}%
 \ifcentrado\hspace*{\@SepTotales}\fi}

\newcommand{\@EscribeTotalesFilasGxET}{%
 \if@BIR\hspace*{\@SepTotales}\usebox{\@CajaBIAliR}\fi
 \hspace*{\@SepTotales}\usebox{\@CajaBIAliG}%
 \if@BIA\hspace*{\@SepTotales}\usebox{\@CajaBIAliA}\fi
 \hspace*{\@SepTotales}\if@Exento\usebox{\@CajaBIET}\else\usebox{\@CajaTotal}\fi
 \ifcentrado\hspace*{\@SepTotales}\fi}

\newcommand{\@EscribeTotalesFilasRGA}{%
 \hspace*{\@SepTotales}\usebox{\@CajaBIAliR}%
 \hspace*{\@SepTotales}\usebox{\@CajaBIAliG}%
 \hspace*{\@SepTotales}\usebox{\@CajaBIAliA}%
 \hspace*{\@SepTotales}}

\newcommand{\@EscribeTotalesFilasRGAET}{%
 \@EscribeTotalesFilasRGA
 \if@Exento\usebox{\@CajaBIET}\else\usebox{\@CajaTotal}\fi
 \ifcentrado\hspace*{\@SepTotales}\fi}

%% Escribe la letra que designa al tipo de impuesto que aplica al ítem.
\NewDocumentCommand{\@EscribeTipoDeImpuesto}{m}{%
 % Hay que abrir un grupo de ejecución porque el cambio en la tipografía altera la celda.
 {\ifG
   \@LetraTipoIVA
   \ifx#1G\@DenomAlicuotaG\fi
   \ifx#1R\@DenomAlicuotaR\fi
   \ifx#1A\@DenomAlicuotaA\fi
   \ifx#1E\@DenomAlicuotaE\fi
  \fi}}
 
%% Comando para poner en cero los contadores de los totales porque el ambiente de la factura se evalúa tres veces y los totales se sumarían todos tres veces.
\newcommand{\@ZERO}{%
 \FPset\BIG{0}\global\let\BIG\BIG
 \FPset\BIR{0}\global\let\BIR\BIR
 \FPset\BIA{0}\global\let\BIA\BIA
 \FPset\BIE{0}\global\let\BIE\BIE
 \FPset\SubtE{0}\global\let\SubtE\SubtE
 \FPset\SubtG{0}\global\let\SubtG\SubtG
 \FPset\SubtR{0}\global\let\SubtR\SubtR
 \FPset\SubtA{0}\global\let\SubtA\SubtA
 \FPset\SubtX{0}\global\let\SubtX\SubtX
 \FPset\DescE{0}\global\let\DescE\DescE
 \FPset\DescG{0}\global\let\DescG\DescG
 \FPset\DescR{0}\global\let\DescR\DescR
 \FPset\DescA{0}\global\let\DescA\DescA
 \FPset\DescX{0}\global\let\DescX\DescX
 \FPset\AlicuotaG{0}\global\let\AlicuotaG\AlicuotaG
 %% (quitado por no vigente)
  % \FPset\AlicuotaGDesc{0}\global\let\AlicuotaGDesc\AlicuotaGDesc
 \FPset\AlicuotaR{0}\global\let\AlicuotaR\AlicuotaR
 \FPset\AlicuotaA{0}\global\let\AlicuotaA\AlicuotaA}

%% Comandos para calcular todos los totales parciales y el total general.
\newcommand{\@CalculaTotalesI}{%
 \FPifeq\SubtG{0}\else\@CalculaTotalesII{G}\fi
 \FPifeq\SubtR{0}\else\@CalculaTotalesII{R}\fi
 \FPifeq\SubtA{0}\else\@CalculaTotalesII{A}\fi
 \FPifeq\SubtE{0}\else\@CalculaTotalesII{E}\fi
 %% (quitado por no vigente)
 % \ifbool{Electro}{%
 %  \FPiflt\BIG\@AlicuotaGDescLimite
 %   \let\@AlicuotaGDesc\@AlicuotaGDescBajo
 %  \else
 %   \let\@AlicuotaGDesc\@AlicuotaGDescAlto
 %  \fi
 %  \FPupn\@AlicuotaGDescPorc{\@AlicuotaGDesc{} 100 * \@DigitosPorc{} round clip}%
 %  \FPupn\AlicuotaGDesc{\BIG{} \@AlicuotaGDesc{} * \@DigitosMoneda{} round}%
 %  \global\let\AlicuotaGDesc\AlicuotaGDesc
 %  \global\let\@AlicuotaGDescPorc\@AlicuotaGDescPorc
 %  \global\let\@AlicuotaGDesc\@AlicuotaGDesc}{}%
 \FPupn\Total{\BIE{} \BIR{} + \BIG{} + \BIA{} + \AlicuotaR{} + \AlicuotaG{} + \AlicuotaA{} + \@DigitosMoneda{} round}%
 %% (quitado por no vigente) arriba se sumaba también \AlicuotaGDesc{}.
 \global\let\Total\Total}

\newcommand{\@CalculaTotalesII}[1]{%
 % Sumamos al descuento.
 \expandafter\FPupn\csname Desc#1\endcsname{\csname @Descuento#1factor\endcsname{} \csname Subt#1\endcsname{} * \csname @Descuento#1num\endcsname{} + \@DigitosMoneda{} round}%
 \global\expandafter\let\csname Desc#1\expandafter\endcsname\csname Desc#1\endcsname
 % Sumamos a la base imponible.
 \expandafter\FPupn\csname BI#1\endcsname{\csname Desc#1\endcsname{} \csname Subt#1\endcsname{} - \@DigitosMoneda{} round}%
 \global\expandafter\let\csname BI#1\expandafter\endcsname\csname BI#1\endcsname
 \ifx#1E%
 \else{%
  % Sumamos a la alícuota.
  \expandafter\FPupn\csname Alicuota#1\endcsname{\csname BI#1\endcsname{} \csname @Impuesto#1\endcsname{} * \@DigitosMoneda{} round}%
  \global\expandafter\let\csname Alicuota#1\expandafter\endcsname\csname Alicuota#1\endcsname}%
 \fi}

%% Comandos auxiliares para poder evaluar la definición de las columnas en la tabla con un comando que se expande completamente.
\def\@FacturaAuxColumnasEval{
 % Primera línea, la más externa.
 \ifbool{SinLineasVertExt}{}{|}
 % Columna para el número del ítem, si aplica.
 \ifbool{NumItem}{c}{r@{\extracolsep{0pt}}}
 % Segunda línea, que sale solamente cuando se usa la opción 'numitem'
 \ifbool{NumItem}{\ifbool{SinLineasVertInt}{}{|}}{}
 % Columna 'X' con DCU, columna fantasma con 'sincantidad' y columna 'c' con CDU.
 \ifbool{PrimeroDescripcion}{X}{
  \ifbool{SinCantidad}{l@{\extracolsep{0pt}}}{c}}
 % Tercera línea.
 \ifbool{SinCantidad}{}{\ifbool{SinLineasVertInt}{}{|}}
 % Columna 'c' con DCU, columna 'r' con DUC, columna 'X' sin espacio para 'sincantidad' y columna 'X' con CDU.
 \ifbool{DCU}{c}{
  \ifbool{DUC}{r}{
   \ifbool{SinCantidad}{X@{\extracolsep{0pt}}}{X}}}
 % Cuarta línea, que sale solamente cuando no se usa 'sincantidad'. 
 \ifbool{SinCantidad}{}{\ifbool{SinLineasVertInt}{}{|}}
 % Columna 'c' para DUC y columna 'r' para cualquier otro caso (DCU, CDU y 'sincantidad').
 \ifbool{DUC}{c}{r}
 % Quinta línea. 
 \ifbool{SinLineasVertInt}{}{|}
 % Columna 'r' para subtotal antes del descuento, o columna fantasma si no se especifica 'subtotal').
 \ifbool{Subtotal}{r}{l@{\extracolsep{0pt}}}
 % Sexta línea, que sale solamente cuando hay columna de subtotales.
 \ifbool{Subtotal}{\ifbool{SinLineasVertInt}{}{|}}{}
 % Columna 'c' para 'descuentos' y columna 'r' sin espacio si no hay descuentos individuales.
 \ifbool{Descuentos}{c}{r@{\extracolsep{0pt}}}
 % Séptima línea, que sale solamente cuando hay descuentos.
 \ifbool{Descuentos}{\ifbool{SinLineasVertInt}{}{|}}{}
 % Columna 'c' para 'iva' y columna 'r' sin espacio si no hay columna de IVA.
 \ifbool{IVA}{c}{r@{\extracolsep{0pt}}}
 % Octava línea, que sale solamente cuando hay columna de IVA.
 \ifbool{IVA}{\ifbool{SinLineasVertInt}{}{|}}{}
 % Columna 'r' siempre.
 r
 % Novena y última línea.
 \ifbool{SinLineasVertExt}{}{|}}

%% Comando para hacer los cálculos sobre cada ítem y establecer las macros con el contenido de cada ítem introducido.
%% Aquí tenemos que hacer coincidir los parámetros de \@CalculaItem con los de \@ITEM para poder hacer
%% \let\@CalculaItem\@ITEM cuando no hay factura sino solamente reporte.
\NewDocumentCommand{\@CalculaItem}{m m m o m m +o}{%
 \gdef\@PrecioUtemp##1{\FPupn\PrecioUtemp{##1 \PrecioUtempx{} * clip}}%
 \addtocounter{NumItem}{1}%
 % Calculamos.
 \@PrecioUtemp{#5}%
 \FPupn\PrecioU{\PrecioUtemp{} \@DigitosMoneda{} round}%
 \FPifeq\PrecioU{\PrecioUtemp}\else\CLASSWARNING{Se esta' introduciendo un precio con ma's de dos decimales: (#3 - \PrecioUtemp. Redondeado a \PrecioU)}{}\fi% Advertimos cuando el precio unitario tiene más de dos decimales.
 \ifbool{Descuentos}
  {\ifx\\#6\\\FPset\Dcto{0}\else\FPupn\Dcto{#6 \@DigitosPorc{} round clip}\fi}% Fijamos en 0 el descuento si no se especifica en \Item***.
  {\FPset\Dcto{#6}%
   \FPifeq\Dcto{0}\else\CLASSWARNING{Se pidio' un descuento individual sin dar la opcio'n de la clase 'descuentos'. No se tomara' en cuenta}\fi
   \FPset\Dcto{0}}% También fijamos en 0 el descuento por si se especifica en \Item*** pero no se pide la opción 'descuentos'.
 \FPupn\PrecioS{#2 \PrecioU{} * \@DigitosMoneda{} round}%
 \FPupn\PrecioT{100 \Dcto{} 100 - / \PrecioS{} * \@DigitosMoneda{} round}%
 \global\let\PrecioU\PrecioU
 \global\let\PrecioS\PrecioS
 \global\let\Dcto\Dcto
 \global\let\PrecioT\PrecioT
 % Suma el elemento actual al subtotal acumulado.
 \expandafter\FPupn\csname Subt#1\endcsname{\PrecioT{} \csname Subt#1\endcsname{} + \@DigitosMoneda{} round}%
 \global\expandafter\let\csname Subt#1\expandafter\endcsname\csname Subt#1\endcsname
 % Guardamos en macros cada elemento agregado a la factura; por ejemplo: \@Item@i@Cantidad contiene la cantidad que se introdujo para el primer ítem; así mismo con ii@, @iii@, y así mismo también con @Descripcion, @PrecioU, @Descuento y @PrecioT.
 \expandafter\gdef\csname @Item@\romannumeral\value{NumItem}@Cantidad\endcsname{\numprint{#2}}%
 \expandafter\gdef\csname @Item@\romannumeral\value{NumItem}@Descripcion\endcsname{#3}%
 \expandafter\global\expandafter\edef\csname @Item@\romannumeral\value{NumItem}@PrecioU\endcsname{\noexpand\numprint{\PrecioU}}%
 \expandafter\global\expandafter\edef\csname @Item@\romannumeral\value{NumItem}@PrecioS\endcsname{\noexpand\numprint{\PrecioS}}%
 \expandafter\global\expandafter\edef\csname @Item@\romannumeral\value{NumItem}@Descuento\endcsname{\noexpand\numprint{\Dcto}}%
 \expandafter\gdef\csname @Item@\romannumeral\value{NumItem}@DescripcionLarga\endcsname{#7}%
 \expandafter\global\expandafter\expandafter\edef\csname @Item@\romannumeral\value{NumItem}@IVA\endcsname{%
  \noexpand\numprint\expandafter{\csname @Alicuota#1porc\endcsname}}%
 \expandafter\global\expandafter\edef\csname @Item@\romannumeral\value{NumItem}@PrecioT\endcsname{\noexpand\numprint{\PrecioT}}}

%% Macro interna para saber si una etiqueta existe o no.
\gdef\@pruebaref#1{%
 \expandafter\ifx\csname @Etiqueta@#1\endcsname\relax
  \CLASSWARNING{La etiqueta `#1' no esta' definida}\ignorespaces
  \@PruebaReffalse
 \else
  \@PruebaReftrue
 \fi}

%% Comandos para las referencias: i) referencia al número de ítem, ii) a la cantidad, iii) a la descripción, iv) al precio unitario, v) al descuento, vi) al precio total.
\gdef\itemref#1{%
 \ifbool{NumItem}{}{\CLASSWARNING{Es recomendable usar la opcio'n 'numitem' para enumerar los i'tems}}%
 \@pruebaref{#1}\if@PruebaRef \csname @Etiqueta@#1\endcsname \fi}
\gdef\cantref#1{\@pruebaref{#1}\if@PruebaRef \csname @Item@\expandafter\romannumeral\csname @Etiqueta@#1\endcsname @Cantidad\endcsname \fi}
\gdef\descref#1{\@pruebaref{#1}\if@PruebaRef \csname @Item@\expandafter\romannumeral\csname @Etiqueta@#1\endcsname @Descripcion\endcsname \fi}
\gdef\puref#1{\@pruebaref{#1}\if@PruebaRef \csname @Item@\expandafter\romannumeral\csname @Etiqueta@#1\endcsname @PrecioU\endcsname \fi}
\gdef\subtref#1{\@pruebaref{#1}\if@PruebaRef \csname @Item@\expandafter\romannumeral\csname @Etiqueta@#1\endcsname @PrecioS\endcsname \fi}
\gdef\dctoref#1{\@pruebaref{#1}\if@PruebaRef \csname @Item@\expandafter\romannumeral\csname @Etiqueta@#1\endcsname @Descuento\endcsname \fi}
\gdef\ldescref#1{\@pruebaref{#1}\if@PruebaRef \csname @Item@\expandafter\romannumeral\csname @Etiqueta@#1\endcsname @DescripcionLarga\endcsname \fi}
\gdef\ivaref#1{\@pruebaref{#1}\if@PruebaRef \csname @Item@\expandafter\romannumeral\csname @Etiqueta@#1\endcsname @IVA\endcsname \fi}
\gdef\ptref#1{\@pruebaref{#1}\if@PruebaRef \csname @Item@\expandafter\romannumeral\csname @Etiqueta@#1\endcsname @PrecioT\endcsname \fi}

%% Definimos un comando para la etiqueta, que se define dentro del entorno de la factura para que tenga alcance local; cuando se acaba dicho entorno, el comando \label vuelve a ser lo que está definido en latex.ltx.
\newcommand{\label@factura}[1]{%
 \expandafter\ifx\csname @Etiqueta@#1\endcsname\relax
  \expandafter\global\expandafter\edef\csname @Etiqueta@#1\endcsname{\arabic{NumItem}}\ignorespaces
  \CLASSINFO{Etiqueta `#1' definida para el i'tem \arabic{NumItem}}%
 \else
  \edef\@label@test@uno{\csname @Etiqueta@#1\endcsname}%
  \edef\@label@test@dos{\theNumItem}%
  \ifx\@label@test@uno\@label@test@dos
  \else
   \CLASSWARNING{La etiqueta `#1' se definio' ma's de una vez}\ignorespaces
  \fi
 \fi}

%% Todo esto tiene que estar dentro de \AtEndPreamble que es cuando \if@factura, \if@reverso e \if@reporte fueron ya aplicadas, y para que se evalúen adecuadamente en su momento las definiciones hechas con \edef.
\AtEndPreamble{
 % Comando auxiliar para abrir la tabla del cuadro de la factura y expandir completamente las columnas.
 \edef\@FacturaAuxInicioCmd{\noexpand\begin{tabularx}{\textwidth}{\@FacturaAuxColumnasEval}}

 % Comando auxiliar para abrir el '\multicolumn' de la nota interna y evalúe la presencia o no de las barras verticales.
 \edef\@ColumnasMultiColumnNotaIntAuxCmd{%
  \noexpand\multicolumn{8}{\ifbool{SinLineasVertExt}{}{|}l\ifbool{SinLineasVertExt}{}{|}}{%
   \noexpand\parbox[t]{\linewidth-2\tabcolsep-2\arrayrulewidth}{\noexpand\@LetraNotaInterna\noexpand\@NotaInterna\noexpand\strut}}%
   \noexpand\\}

 % Comando auxiliar para abrir el '\multicolumn' para los totales en el modo de columnas y expandir completamente las columnas cuando hay o no barras verticales.
 \edef\@ColumnasMultiColumnTotalesAuxCmd{%
  \noexpand\multicolumn{8}{\ifbool{SinLineasVertExt}{}{|} r}\begingroup
    \noexpand\if@ExpandeCuadro
     \noexpand\iflinea\noexpand\zsaveposx{LineaNodoBCoordX}\noexpand\zsaveposy{LineaNodoBCoordY}\noexpand\fi\noexpand\fi% Fijamos la posición (x,y) final para la línea ya que estaremos justo en el punto donde escribimos el total, y hasta allí llega la línea en el modo de columnas.
     \noexpand\begin{tabular}[b]{@{\noexpand\extracolsep{2\noexpand\tabcolsep}}lr \ifbool{SinLineasVertExt}{}{|}}}

 % Comando auxiliar para la definición del ambiente de la factura; al final de este comando está el texto que va en el enddef del ambiente de la factura.
 \newcommand{\@FacturaAuxCmd}[1]{%
  \ifexpandecuadro
   \setlength{\@EspacioVacioInicial}{\zposy{EspacioVacioInicial}sp}%
   \setlength{\@EspacioVacioFinal}{\zposy{EspacioVacioFinal}sp}%
   \global\@EspacioVacioInicial=\@EspacioVacioInicial% Hay que globalizarlas
   \global\@EspacioVacioFinal=\@EspacioVacioFinal
   \savebox{\@CajaEspacioVacio}{\parbox[b][\@EspacioVacioInicial-\@EspacioVacioFinal]{\@TamDescripcion}{%
    \iflinea
     \if@LineaHastaFinal
      \zsaveposy{LineaNodoBCoordY}%
     \else
      \if@LineaDescripcion
       \zsaveposx{LineaNodoACoordX}\mbox{}\hfill \zsaveposx{LineaNodoBCoordX}\zsaveposy{LineaNodoBCoordY}%
      \fi
     \fi
    \fi}}%
   \ifdim \ht\@CajaEspacioVacio=0pt \global\lineafalse \else \@ExpandeCuadrotrue \fi
  \fi

  \ifx\undefined\Total\relax\else\CLASSERROR{Parece ya haberse usado un entorno de factura}\fi
  %
  % Comandos \David@SavePos, \David@XStart, \David@XEnd, \David@XPad, \David@XLeft y \David@XRight para poder manejar el salto de línea en las columnas adyacentes a la columna 'X' del tabularx (la descripción), cuando esta ocupa más que una línea.
  % Gracias a David Carlisle (Thanks to David Carlisle).
  % http://tex.stackexchange.com/questions/111203/tabularx-bottom-alignment-on-r-or-l-columns-when-x-column-wraps
  \def\David@SavePos##1{\leavevmode\pdfsavepos\write\@auxout{\gdef\noexpand##1{\the\pdflastypos sp}}}%
  \def\David@XStart{\expandafter\David@SavePos\csname save@start@\romannumeral\value{NumItem}\endcsname}%
  \def\David@XEnd{\expandafter\David@SavePos\csname save@end@\romannumeral\value{NumItem}\endcsname}%
  \def\David@XPad##1##2##3{%
   \vtop{%
    \baselineskip\normalbaselineskip
    ##2%
    \expandafter
    \ifx\csname save@start@##1\endcsname
     \relax
    \else
     \dimen@\csname save@start@##1\endcsname\relax
     \loop
      \ifdim\dimen@>\csname save@end@##1\endcsname
      \advance\dimen@-\baselineskip
      \settowidth{\@TempDimen}{##3}%
       \ifdim\@TempDimen=0pt\settowidth{\@TempDimen}{##2}\fi
      \hbox{\parbox{\@TempDimen}{\@CaracterCeldasVacias}}%
     \repeat
    \fi
    ##3}}%
  \def\David@XLeft##1{\David@XPad{\romannumeral\value{NumItem}}{\hbox{##1}}{}}%
  \def\David@XRight##1{\David@XPad{\romannumeral\value{NumItem}}{}{\hbox{##1}}}%
  %
  % Caja auxiliar para el caracter de retorno de carro y poder saber si tamaño en el campo de la descripción.
  \savebox{\@CajaCaracterCR}{\@LetraItems\@CaracterCR}
  %
  % Comando para fijar la separación extra final de cada item, lo cual viene dado por la cantidad de líneas de la descripción,
  % que está alojado en \count1, definido en \@DescripcionAuxCmd.
  \def\@CalculaSeparaItemsExtraFinal{%
   \setlength{\@SeparaItemsExtraFinal}{\numexpr\the\count1\normalbaselineskip-\normalbaselineskip+\@SeparaItemsExtra}%
   \global\@SeparaItemsExtraFinal=\@SeparaItemsExtraFinal}
  %
  % Comando interno general para agregar cualquier ítem.
  \NewDocumentCommand{\@ITEM}{m m m o m m +o}{%
   % 1. [R|G|A|E|X]; 2. cantidad; 3. descripción; 4. descripción para el csv (entre corchetes); 5. p/u; 6. dcto; 7. la descripción para el reporte.
   \@CalculaItem{##1}{##2}{##3}{##5}{##6}[##7]%
   % Si no es el primer ítem, entonces hace falta \\ para terminar el ítem anterior.
   \ifnum\value{NumItem}=1% Aquí, \@CalculaItem aumenta en 1 el contador NumItem, que si vale 1 significa que aún no hemos procesado la línea 1, es decir, no hay que hacer el salto de línea. La última línea se termina en otro lado.
   \else
    \@CalculaSeparaItemsExtraFinal
    \\[\@SeparaItemsExtraFinal]
    \ifconlineasha\hline\fi% Líneas para separar cada ítem, si hiciera falta.
   \fi
   % Primer -opcional- campo: El número del ítem.
   \ifbool{NumItem}{\David@XLeft{\@NumItemAuxCmd{\mbox{}\hfill \@LetraNumeros \arabic{NumItem}\hfill\mbox{}}}}{}%
   &
   % Segundo campo: Acá puede ir la cantidad (cdu) o la descripción (dcu y duc).
   \ifbool{SinCantidad}{}{%
    \ifbool{PrimeroDescripcion}
    {\@DescripcionAuxCmd{##3}{##1}}% dcu y duc.
    {\David@XLeft{\@CantidadAuxCmd{\mbox{}\hfill \@LetraNumeros \numprint{##2}\hfill\mbox{}}}}}% cdu.
   &
   % Tercer campo: Acá puede ir la descripción (cdu), la cantidad (dcu), o el precio unitario (duc).
   \ifbool{PrimeroDescripcion}{%
    \ifdcu
     \David@XRight{\@CantidadAuxCmd{\mbox{}\hfill \@LetraNumeros \numprint{##2}\hfill\mbox{}}}% dcu.
    \else
     \David@XRight{\@PrecioUAuxCmd{%
      \ifx##1X \hfill \@CaracterItemXVacio \hfill\mbox{}%
      \else
       \ifimpdesc\else\@EscribeTipoDeImpuesto{##1}\fi \hfill \@LetraNumeros \numprint{\PrecioU}
      \fi}}% duc.
    \fi}
   {\@DescripcionAuxCmd{##3}{##1}}% cdu.
   &
   % Cuarto campo: Acá puede ir el precio unitario (cdu y dcu) o la cantidad (duc).
   \ifbool{SinCantidad}{}{%
    \ifduc
     \David@XRight{\@CantidadAuxCmd{\mbox{}\hfill \@LetraNumeros \numprint{##2}\hfill\mbox{}}}% duc.
    \else
     \David@XRight{\@PrecioUAuxCmd{%
      \ifx##1X \hfill \@CaracterItemXVacio \hfill\mbox{}%
      \else
       \ifimpdesc\else\@EscribeTipoDeImpuesto{##1}\fi \hfill \@LetraNumeros \numprint{\PrecioU}
      \fi}}% dcu y cdu.
    \fi}%
   & % Quinto -opcional- campo: Subtotal (para luego descuentos).
   \ifbool{Subtotal}{%
    \David@XRight{\@PrecioSAuxCmd{%
     \ifx##1X \hfill \@CaracterItemXVacio \hfill\mbox{}%
     \else
      \ifimpdesc\else\@EscribeTipoDeImpuesto{##1}\fi \hfill \@LetraNumeros \numprint{\PrecioS}
     \fi}}}{}
   &
   % Sexto -opcional- campo: Descuento.
   \ifbool{Descuentos}{%
    \David@XRight{\@DescuentoAuxCmd{%
     \ifx##1X \hfill \@CaracterItemXVacio \hfill\mbox{}%
     \else
      \mbox{}\hfill \@LetraNumeros \numprint{\Dcto}\%\hfill\mbox{}%
     \fi}}}{}% Tenemos que usar un condicional booleano porque si no da problemas con el condicional \FPifeq.
   &
   % Séptimo -opcional- campo: IVA.
   \ifbool{IVA}{%
    \David@XRight{\@IVAAuxCmd{%
     \ifx##1X \hfill \@CaracterItemXVacio \hfill\mbox{}%
     \else
      \mbox{}\hfill \@LetraNumeros \numprint{\csname @Alicuota##1porc\endcsname}\%\hfill\mbox{}%
     \fi}}}{}% Tenemos que usar un condicional booleano porque si no da problemas con el condicional \FPifeq.
   &
   % Octavo campo: El total.
   \David@XRight{\@PrecioTAuxCmd{%
    \ifx##1X \hfill \@CaracterItemXVacio \hfill\mbox{}%
    \else
     \ifimpdesc\else\@EscribeTipoDeImpuesto{##1}\fi \hfill \@LetraNumeros \numprint{\PrecioT}
    \fi}}%
   % Escribimos al csv.
   \ifcsv
    \IfValueTF{##4}{\@EscribeItemsEnCSV{##1}{##2}{##4}}{\@EscribeItemsEnCSV{##1}{##2}{##3}}%
   \fi}%
  %
  % Comando para escribir en el .csv.
  \newcommand{\@EscribeItemsEnCSV}[3]{%
   \ifbool{SinCantidad}
   {\ifdescuentos
     \write\csv{"\unexpanded{##3}","\Dcto","\PrecioT","##1"}%
    \else
     \write\csv{"\unexpanded{##3}","\PrecioT","##1"}%
    \fi}
   {\if@cdu
     \ifdescuentos
      \write\csv{"##2","\unexpanded{##3}","\PrecioU","\Dcto","\PrecioT","##1"}%
     \else
      \write\csv{"##2","\unexpanded{##3}","\PrecioU","\PrecioT","##1"}%
     \fi
    \fi
    \ifdcu
     \ifdescuentos
      \write\csv{"\unexpanded{##3}","##2","\PrecioU","\Dcto","\PrecioT","##1"}%
     \else
      \write\csv{"\unexpanded{##3}","##2","\PrecioU","\PrecioT","##1"}%
     \fi
    \fi
    \ifduc
     \ifdescuentos
      \write\csv{"\unexpanded{##3}","\PrecioU","##2","\Dcto","\PrecioT","##1"}%
     \else
      \write\csv{"\unexpanded{##3}","\PrecioU","##2","\PrecioT","##1"}%
     \fi
    \fi}}%
  %
  % Usamos la definición de \label@factura dentro del entorno de la factura para que sea local aquí nada más.
  \let\label\label@factura
  \@LetraItems% Este es el tipo de letra que tendrán los ítems, y en este tipo de letra se define el \baselineskip de toda la factura.
  \setlength{\baselineskip@@}{\baselineskip}%
  \global\baselineskip@@=\baselineskip@@
  \@FacturaAuxInicioCmd
  \ifbool{SinLineasHoriExt}{}{\hline}%
  \@ZERO
  \setlength{\tabcolsep@@}{\tabcolsep}%
  \global\tabcolsep@@=\tabcolsep@@
  % Estamos en el número del ítem, si aplicara.
  \if@LineaDesdeInicio \zsaveposx{LineaNodoACoordX}\fi
  \ifbool{NumItem}{\@NumItemAuxCmd{\mbox{}\hfill \@LetraTitColumnas \@@NumItem \hfill\mbox{}}}{}%
  &
  % Avanzamos a la cantidad, o la descripción.
  \ifbool{SinCantidad}{}{\ifbool{PrimeroDescripcion}{\@DescripcionAuxCmd{\@LetraTitColumnas \@@Descripcion}{}}{\@CantidadAuxCmd{\mbox{}\hfill \@LetraTitColumnas \@@Cantidad \hfill\mbox{}}}}%
  &% Avanzamos hasta la descripción, la cantidad o el precio unitario.
  \ifbool{SinCantidad}{\@DescripcionAuxCmd{\@LetraTitColumnas \@@Descripcion}{}}
   {\if@cdu \@DescripcionAuxCmd{\@LetraTitColumnas \@@Descripcion}{}\fi
    \ifdcu \@CantidadAuxCmd{\mbox{}\hfill \@LetraTitColumnas \@@Cantidad \hfill\mbox{}}\fi
    \ifduc \@PrecioUAuxCmd{\mbox{}\hfill \@LetraTitColumnas \@@PrecioU \space (\Moneda)\hfill\mbox{}}\fi}%
  &% Avanzamos hasta el precio unitario, o la cantidad, o columna fantasma cuando es 'sincantidad'.
  \ifbool{SinCantidad}{}
   {%
    \mbox{}\hfill
    \ifduc
     \@CantidadAuxCmd{\mbox{}\hfill \@LetraTitColumnas \@@Cantidad \hfill\mbox{}}%
    \else
     \@PrecioUAuxCmd{\mbox{}\hfill \@LetraTitColumnas \@@PrecioU \space (\Moneda)\hfill\mbox{}}%
    \fi}%
  &% Avanzamos hasta el subtotal, que puede o no existir,
  \ifbool{Subtotal}{\@PrecioSAuxCmd{\mbox{}\hfill \@LetraTitColumnas \@@PrecioS \space (\Moneda)\hfill\mbox{}}}{}%
  &% Avanzamos hasta el descuento, que puede o no existir.
  \ifdescuentos \@DescuentoAuxCmd{\mbox{}\hfill \@LetraTitColumnas \@@Descuento \hfill\mbox{}}\fi
  &% Avanzamos hasta el IVA, que puede o no existir.
  \ifbool{IVA}{\@IVAAuxCmd{\mbox{}\hfill \@LetraTitColumnas \@@IVA \hfill\mbox{}}}{}
  &% Avanzamos hasta el precio total.
  \@PrecioTAuxCmd{\mbox{}\hfill \@LetraTitColumnas \ifbool{SinCantidad}{\@@Precio}{\@@PrecioT}\space (\Moneda)\hfill\mbox{}}%
  \if@LineaHastaFinal \zsaveposx{LineaNodoBCoordX}\fi
  \\
  \ifbool{SinLineaHoriTit}{}{\hline}%
  %
  #1% Aquí entran todos los items que el usuario declara dentro del ambiente.
  \@CalculaTotalesI
  \@CalculaSeparaItemsExtraFinal
  \\[\@SeparaItemsExtraFinal]% El último fin de línea se hace aquí.
  \ifconlineasha
   \ifdefvoid{\@NotaInterna}{%
    \ifbool{SinLineasHoriExt}{\if@ExpandeCuadro\hline\fi}{\hline}}{\hline}%
  \fi% Se verifica si hace falta una línea horizontal o no.
  \iffilas
   % Esto en lugar de \ifexpandecuadro para que no se haga una linea vacía en la tabla cuando la caja está vacía.
   \if@ExpandeCuadro
    \ifbool{PrimeroDescripcion}{&\usebox{\@CajaEspacioVacio}&&&&&&\\}{&&\usebox{\@CajaEspacioVacio}&&&&&\\}%
    \ifdefvoid{\@NotaInterna}{}
    {\ifbool{SinLineaHoriNI}{}{\hline}%
     \@ColumnasMultiColumnNotaIntAuxCmd}%
    \ifbool{SinLineasHoriExt}{}{\hline}% Siempre que hay relleno se pone la línea acá.
   \else
    \ifdefvoid{\@NotaInterna}{%
     \ifbool{SinLineasHoriExt}{}{%
      \ifconlineasha\else\hline\fi}}% Verificamos si hay que poner la línea horizontal. Si está 'conlineasha' entonces ya salió la línea.
     {\ifbool{SinLineaHoriNI}{}{\ifconlineasha\else\hline\fi}%
      \@ColumnasMultiColumnNotaIntAuxCmd
      \ifbool{SinLineasHoriExt}{}{\hline}}%
   \fi
  \else
   \ifsintotales
   \else
    \ifbool{SinLineaHoriTot}{}{\ifconlineasha\else\hline\fi}% Línea que cierra los ítems en la tabla.
   \fi
   % Agregamos el espacio espúreo.
   \if@ExpandeCuadro
    % Rellenamos la tabla con espacio espúreo, y le agregamos las líneas externas, si aplica.
    \ifbool{SinLineasVertExt}{\multicolumn{8}{c}{\usebox{\@CajaEspacioVacio}}}{\multicolumn{8}{|c|}{\usebox{\@CajaEspacioVacio}}}\\
   \fi
   % Si pedimos sin totalizaciones, entonces cerramos la tabla ya, si no, calculamos los totales y los escribimos.
   \ifsintotales
    \ifdefvoid{\@NotaInterna}{}{\ifbool{SinLineaHoriNI}{}{\hline}\@ColumnasMultiColumnNotaIntAuxCmd}%
    \ifbool{SinLineasHoriExt}{}{\hline}%
   \else
    % Modo de columnas: Abrimos la tabla para escribir los totales y le agregamos las líneas verticales externas, si aplica.
    \@ColumnasMultiColumnTotalesAuxCmd
    \ifagrupado
     \@EscribeTotalColAgrupado
    \else
     \if@Exento \@EscribeTotalCol{E}\fi
     \if@BIR\@EscribeTotalCol{R}\ifsinivaexpresado\else\@EscribeTotalColAlicuota{R}\fi\fi
     \@EscribeTotalCol{G}\ifsinivaexpresado\else\@EscribeTotalColAlicuota{G}\fi
     %% (quitado por no vigente)
     % \ifbool{Electro}{\@EscribeTotalColAlicuotaGDesc}{}%
     \if@BIA\@EscribeTotalCol{A}\ifsinivaexpresado\else\@EscribeTotalColAlicuota{A}\fi\fi
    \fi
    \ifsinivaexpresado\else\cline{2-2}\fi
    \hfill \@LetraTitTotal\@@Total & \@LetraNumTotal\@SignoMonetarioAux\numprint{\Total}\\
    \end{tabular}\endgroup
    \\
    \ifdefvoid{\@NotaInterna}{}{\ifbool{SinLineaHoriNI}{}{\hline}\@ColumnasMultiColumnNotaIntAuxCmd}%
    \ifbool{SinLineasHoriExt}{}{\hline}%
   \fi
  \fi
  \end{tabularx}%
  %
  % Generamos la macro \Total vacía para que no dé error en algún posible uso dentro del documento.
  \ifsintotales\gdef\Total{}\fi
  % Si pedimos no totalizar (sintotales), la opción filas queda desactivada.
  \iffilas
   \@LetraNumTotales
   \vspace*{\@SeparaVertical}\par
   \ifexpandecuadro\else\vfill\fi
   \@GeneraCajasTotalesFilas
   \newbox{\@CajaTotal}%
   \savebox{\@CajaTotal}{{\@LetraTitTotal\@@Total:} \@LetraNumTotal\@SignoMonetarioAux\numprint{\Total}}%
   %
   \ifnum \value{@IVANoGeneral}=0%
    \newbox{\@CajaGET}%
    \savebox{\@CajaGET}{\@EscribeTotalesFilasGET}%
    \ifdim \the\wd\@CajaGET < \textwidth
     \parbox{\textwidth}{\@EscribeTotalesFilasGET}%
    \else
     \hspace*{\@SepTotales}%
     \usebox{\@CajaBIAliG}\hspace*{\@SepTotales}%
     \if@Exento\usebox{\@CajaBIET}\else\usebox{\@CajaTotal}\fi
     \ifcentrado\hspace*{\@SepTotales}\fi
    \fi
   \fi
   \ifnum \value{@IVANoGeneral}=1%
    \newbox{\@CajaGxET}%
    \savebox{\@CajaGxET}{\@EscribeTotalesFilasGxET}%
    \ifdim \the\wd\@CajaGxET < \textwidth
     \@EscribeTotalesFilasGxET
    \else
     \hspace*{\@SepTotales}%
     \if@BIR\usebox{\@CajaBIAliR}\hspace*{\@SepTotales}\fi
     \usebox{\@CajaBIAliG}\hspace*{\@SepTotales}%
     \if@BIA\usebox{\@CajaBIAliA}\hspace*{\@SepTotales}\fi
     \\[\@SeparaFilas]\mbox{}%
     \ifcentrado\hspace*{\@SepTotales}\else\hfill\fi
     \if@Exento\usebox{\@CajaBIE}\hspace*{\@SeparaMinTotales}\fi\usebox{\@CajaTotal}%
     \ifcentrado\hspace*{\@SepTotales}\fi
    \fi
   \fi
   \ifnum \value{@IVANoGeneral}=2%
   \newbox{\@CajaRGAET}%
   \savebox{\@CajaRGAET}{\@EscribeTotalesFilasRGAET}%
    \ifdim \the\wd\@CajaRGAET < \textwidth
     \@EscribeTotalesFilasRGAET
    \else
     \newbox{\@CajaRGA}%
     \savebox{\@CajaRGA}{\@EscribeTotalesFilasRGA}%
     \ifdim \the\wd\@CajaRGA < \textwidth
      \@EscribeTotalesFilasRGA
      \\[\@SeparaFilas]\mbox{}%
      \ifcentrado\hspace*{\@SepTotales}\else\hfill\fi
      \if@Exento\usebox{\@CajaBIE}\hspace*{\@SeparaMinTotales}\fi\usebox{\@CajaTotal}%
      \ifcentrado\hspace*{\@SepTotales}\fi
     \else% Aquí casi cualquier cosa tiene aspecto feo.
      \CLASSWARNING{Es mejor no usar la opcio'n de totalizacio'n en filas en este caso, o disminuir los tama~nos de los tipos de letra para la totalizacio'n}%
      \centering{%
       \begin{tabular}{r@{\hspace*{\@SeparaMinTotales}}l}
        \usebox{\@CajaBIAliR} & \usebox{\@CajaBIAliA} \\[\@SeparaFilas]
        \usebox{\@CajaBIAliG} & \if@Exento\usebox{\@CajaBIET}\else\usebox{\@CajaTotal}\fi
       \end{tabular}}%
     \fi
    \fi
   \fi
  \fi
  \par
  \ifexpandecuadro
   \vspace*{-\@EspacioVacioInicial}%
   \vspace*{\@EspacioVacioFinal}%
   \zsaveposy{EspacioVacioInicial}%
  \fi
  \vfill
  \ifexpandecuadro\zsaveposy{EspacioVacioFinal}\fi
  \vspace*{\@SeparaVertical}\par
  \ifdefvoid{\@NotaExterna}{}{\parbox[t]{\linewidth}{\@LetraNotaExterna\@NotaExterna}\vspace*{\@SeparaVertical}\par}}
 
  \ifbool{SinCantidad}
  % Definamos todos los comandos \Item***. El descuento es opcional; por eso usamos G{0} en \NewDocumentCommand.
  {%
   \NewDocumentCommand{\Item}{m o m G{0} +o}{\@ITEM{G}{1}{#1}[#2]{#3}{#4}[#5]}% Ítem normal.
   \NewDocumentCommand{\ItemR}{m o m G{0} +o}{\global\@BIRtrue\global\Gtrue\@ITEM{R}{1}{#1}[#2]{#3}{#4}[#5]}% Ítem con IVA reducido.
   \NewDocumentCommand{\ItemA}{m o m G{0} +o}{\global\@BIAtrue\global\Gtrue\@ITEM{A}{1}{#1}[#2]{#3}{#4}[#5]}% Ítem con IVA adicional.
   \NewDocumentCommand{\ItemE}{m o m G{0} +o}{\global\@Exentotrue\global\Gtrue\@ITEM{E}{1}{#1}[#2]{#3}{#4}[#5]}% Ítem exento de IVA.
   \NewDocumentCommand{\ItemX}{m o +o}{\@ITEM{X}{1}{#1}[#2]{0}{0}[#3]}% Ítem sin precio.
   \let\ItemG\Item
   \ifcsv
    \ifdescuentos
     \write\csv{"\@@Descripcion","\@@Descuento","\@@Precio \space (\Moneda)","\@@IVA"}%
    \else
     \write\csv{"\@@Descripcion","\@@Precio \space (\Moneda)","\@@IVA"}%
    \fi
   \fi}
  {%
   \NewDocumentCommand{\Item}{m m o m G{0} +o}{\@ITEM{G}{#1}{#2}[#3]{#4}{#5}[#6]}% Ítem normal.
   \NewDocumentCommand{\ItemR}{m m o m G{0} +o}{\global\@BIRtrue\global\Gtrue\@ITEM{R}{#1}{#2}[#3]{#4}{#5}[#6]}% Ítem con IVA reducido.
   \NewDocumentCommand{\ItemA}{m m o m G{0} +o}{\global\@BIAtrue\global\Gtrue\@ITEM{A}{#1}{#2}[#3]{#4}{#5}[#6]}% Ítem con IVA adicional.
   \NewDocumentCommand{\ItemE}{m m o m G{0} +o}{\global\@Exentotrue\global\Gtrue\@ITEM{E}{#1}{#2}[#3]{#4}{#5}[#6]}% Ítem exento de IVA.
   \NewDocumentCommand{\ItemX}{m m o +o}{\@ITEM{X}{#1}{#2}[#3]{0}{0}[#4]}% Ítem sin precio.
   \let\ItemG\Item
   \ifcsv
    \if@cdu
     \ifdescuentos
      \write\csv{"\@@Cantidad","\@@Descripcion","\@@PrecioU \space (\Moneda)","\@@Descuento","\@@PrecioT \space (\Moneda)","\@@IVA"}%
     \else
      \write\csv{"\@@Cantidad","\@@Descripcion","\@@PrecioU \space (\Moneda)","\@@PrecioT \space (\Moneda)","\@@IVA"}%
     \fi
    \fi
    \ifdcu
     \ifdescuentos
      \write\csv{"\@@Descripcion","\@@Cantidad","\@@PrecioU \space (\Moneda)","\@@Descuento","\@@PrecioT \space (\Moneda)","\@@IVA"}%
     \else
      \write\csv{"\@@Descripcion","\@@Cantidad","\@@PrecioU \space (\Moneda)","\@@PrecioT \space (\Moneda)","\@@IVA"}%
     \fi
    \fi
    \ifduc
     \ifdescuentos
      \write\csv{"\@@Descripcion","\@@PrecioU \space (\Moneda)","\@@Cantidad","\@@Descuento","\@@PrecioT \space (\Moneda)","\@@IVA"}%
     \else
      \write\csv{"\@@Descripcion","\@@PrecioU \space (\Moneda)","\@@Cantidad","\@@PrecioT \space (\Moneda)","\@@IVA"}%
     \fi
    \fi
   \fi}

 \if@factura
  % Los descuentos totales no son compatibles -y no creo que lo vayan a ser- con el modo de totalización en filas.
  \iffilas
   \if@DescuentoTotal
    \CLASSWARNING{La opcio'n 'filas' no es compatible con los descuentos totales; sin efecto opcio'n 'filas'}
    \filasfalse
   \fi
  \fi
  \if@reverso
  \else
    \RenewEnviron{reverso}{}
  \fi
  \ifcotizacion\def\@Denominacion{\@@DenomCotiza}\else\def\@Denominacion{\@@DenomFactura}\fi%  Fijamos la denominación del documento paras las cotizaciones o facturas, que son casos particulares del modo 'factura'.
  \expandafter\geometry\expandafter{\@GeometriaFactura}%  La geometría debe aplicarse aquí.
  \AfterEndPreamble{%
   \addtolength{\@PosFechaX}{\oddsidemargin+1in}%  Le sumamos el margen izquierdo a las posiciones de las cosas; es importante que se haga en un \AfterEndPreamble antes de que empiecen a escribirse otras cosas en la página.
   \addtolength{\@PosDenomX}{\oddsidemargin+1in}%
   \addtolength{\@PosControlX}{\oddsidemargin+1in}%
   \addtolength{\@PosNotaFinalXuno}{\oddsidemargin+1in}%
   \addtolength{\@PosFirmasXuno}{\oddsidemargin+1in}%
   \setlength{\parindent}{0pt}%
   \setlength{\lineskip}{0pt}%
   \setlength{\parskip}{0pt plus 1pt}%
   \expandafter\setlength{\@SepTotales}{\the\@SeparaMinTotales plus 1fill}% \@SeparaMinTotales es el mínimo espacio entre los totales en el modo 'filas', y SepTotales es una dimensión interna, estirable, con ese como su mínimo.
   \thispagestyle{empty}%
   \pagenumbering{roman}%
   \@EncabezadoFactura
   \@EscribeNotaPrevia}
   \AtBeginShipoutNext{% Antes abría aquí un \AtBeginShipoutNext cuando había reporte y un \AtEndDocument cuando no. No es necesario eso. También lo modifiqué en \FirmaFactura.
    \@CargaFondoFactura
    \@PosicionaFechas
    \ifdenom\@PosicionaDenom\fi
    \ifprefactura\@PosicionaNumControl\fi
    \ifnum\value{@Firmas}=0\else\@PosicionaFirmas\fi
    \ifdefvoid{\@NotaFinal}{}{\@PosicionaNotaFinal}}
  \iflinea
   % Para determinar las coordenadas de la línea hay que usar \AtBeginShipoutNext porque si no se hace en el 'shipout' de la página se genera un espacio en blanco extra.
   \AtBeginShipoutNext{\iflinea\@DibujaLinea\fi}
    \iffilas
     \ifbool{SinLineasVertInt}{\@LineaDesdeIniciotrue\@LineaHastaFinaltrue}{\@LineaDescripciontrue}%
    \else
     \ifsintotales
      \@LineaDesdeIniciotrue\@LineaHastaFinaltrue
     \else
      \@LineaDesdeIniciotrue\@LineaHastaTotaltrue
     \fi
    \fi
  \fi
  %% El comando para la descripción del servicio prestado.
  \NewDocumentCommand{\Descripcion}{+m}{%
   {% Abrimos un grupo adicional para que la redefinición del entorno 'tikzpicture' sea local.
    \@LetraDescripcion
    % Cuando la línea tiene grosor cero, hacemos que el entorno 'tikzpicture' sea "fake", y redefinimos \node para que obvie los parámetros que le dimos como argumento opcional.
    \ifdim\@GrosorLineaDescripcion=0pt%
     \setlength{\@SeparaInternaDescripcion}{0pt}\setlength{\@EsquinaLineaDescripcion}{0pt}\setlength{\@SombraDescripcion}{0pt}%
     \renewenvironment{tikzpicture}{\NewDocumentCommand{\node}{o +m}{####2}}{}%
    \fi
    \begin{tikzpicture}
     % Expandimos después para poderle pasar el contenido de la macro como argumento opcional a \node.
     \expandafter\node\expandafter[\@DescripcionTIKZ]{%
      \ifbool{ExpAbajo}
       {\parbox[t][\textheight-\pagetotal-2\@SeparaInternaDescripcion-\@GrosorLineaDescripcion-\@SombraDescripcion]
        {\textwidth-2\@SeparaInternaDescripcion-\@GrosorLineaDescripcion-\@SombraDescripcion}}%
       {\parbox[t]{\textwidth-\@SombraDescripcion-2\@SeparaInternaDescripcion-2\@GrosorLineaDescripcion}}%
      {\setlength{\parindent}{0.5cm}%
       \setlength{\parskip}{2pt plus 8pt}%
       \ifbool{ExpAbajo}{\vfill}{}%
        #1%
       \ifbool{ExpAbajo}{\vfill}{}}}%
     \ifdim\@GrosorLineaDescripcion=0pt\else ;\fi% El fin del nodo va cuando hay el entorno tikzpicture original.
    \end{tikzpicture}
    \par\vspace*{\@SombraDescripcion}}}%
 \else
  \RenewEnviron{factura}{%
   \@ZERO
   \let\label\label@factura
   \let\@ITEM\@CalculaItem
   \BODY
   \@CalculaTotalesI}
  \RenewEnviron{reverso}{}
  \NewDocumentCommand{\Descripcion}{+m}{}
 \fi

 \if@reporte
  \ifdefvoid{\@InfoPagina}{\let\@InfoPagina\@InfoPaginaNoTitulo}{}%
  \if@factura
  \else
   \expandafter\geometry\expandafter{\@GeometriaReporte}%  La geometría del reporte debe aplicarse aquí cuando sólo hay reporte, así se puede establecer el tamaño del papel.
  \fi
  \NewEnviron{reporte}{%
   \if@factura
    \expandafter\newgeometry\expandafter{\@GeometriaReporte}\cleardoublepage%  La geometría debe aplicarse aquí cuando también hay factura, pero no se puede cambiar el tamaño del papel.
   \fi
   \let\headwidth\textwidth
   \@CargaFondoReporte
   \@EncabezadoReporte
   \ifdefvoid{\@EstiloPagReporteResto}{%
    \if@FondoReporteStar%  Si no usamos \EstiloPagReporte entonces se fijará 'soloabajo' cuando se use \FondoReporte* para que sólo use marcas de pie de página, y si no se usó \FondoReporte* entonces se fijará 'plain' que usa cabeceras y pies de página.
     \pagestyle{soloabajo}%
    \else
     \pagestyle{plain}%
    \fi}
    {\pagestyle{\@EstiloPagReporteResto}}%  Si usamos \EstiloPagReporte en su argumento obligatorio entonces fijamos ese estilo de página.
   \ifdefvoid{\@EstiloPagReporteUno}{%
    \if@FondoReporte%  Si no le damos a \EstiloPagReporte su argumento opcional, entonces se fijará 'solonumpag' como el estilo de página para la primera página, pero solamente si usamos \FondoReporte, ya que el estilo de página predeterminado al usar \FondoReporte* ya se acaba de fijar y no necesita el caso especial de la primera hoja (\thispagestyle).
     \if@FondoReporteStar
     \else
      \thispagestyle{solonumpag}%
     \fi
    \fi}
    {\thispagestyle{\@EstiloPagReporteUno}}%  Si usamoe el argumento opcional de \EstiloPagReporte establecemos ese estilo de página.
   \pagenumbering{arabic}%
   \setlength{\parindent}{0.666cm}%
   \setlength{\parskip}{4pt plus 6pt minus 2pt}% separación entre párrafos en el reporte.
   \vskip\baselineskip
   \BODY\clearpage}% espacio antes de empezar con el reporte, después del título -si lo hubiera-.
 \else
  \NewEnviron{reporte}{}
 \fi}

%% Comando para el diseño de la forma libre.
\newcommand{\@CargaFondoFactura}{%
 \if@FondoFactura
  \begin{textblock*}{\paperwidth}(0pt,0pt)%
   \includegraphics{\@FondoFactura}%
  \end{textblock*}%
 \fi}

%% Comando para el fondo del reporte.
\newcommand{\@CargaFondoReporte}{%
 \begin{textblock*}{\paperwidth}(0pt,0pt)% Tuve que abrir el textblock* aquí para que aplique siempre; de esta manera se corrige un defecto que se produce cuando se usa \vspace* al inicio de la página (dentro de \@EncabezadoReporte).
  \if@FondoReporte
   \if@FondoReporteStar
    \AddToShipoutPictureBG{\includegraphics{\@FondoReporte}}%
   \else
     \noindent\includegraphics{\@FondoReporte}%
   \fi
  \fi
 \end{textblock*}}

%% Comando que coloca los espacios para firmar.
\newcommand{\@PosicionaFirmas}{%
 \savebox{\@CajaFirmas}{%
  \parbox{\textwidth+\oddsidemargin+1in-\@PosFirmasXuno-\@PosFirmasXdos}{\@EstiloFirmas}}%
 \ifbool{FirmasYnegativo}{\addtolength{\@PosFirmasY}{\paperheight-\ht\@CajaFirmas-\dp\@CajaFirmas}}{}%
 \begin{textblock*}{\wd\@CajaFirmas}(\@PosFirmasXuno,\@PosFirmasY)%
  \usebox{\@CajaFirmas}%
 \end{textblock*}}

%% Comando que coloca la nota final.
\newcommand{\@PosicionaNotaFinal}{%
 \savebox{\@CajaNotaFinal}{%
  \parbox{\textwidth+\oddsidemargin+1in-\@PosNotaFinalXuno-\@PosNotaFinalXdos}{\@EscribeNotaFinal}}%
 \ifbool{NotaFinalYnegativo}{\addtolength{\@PosNotaFinalY}{\paperheight}}{}%
 \ifbool{NotaFinalYarriba}{\addtolength{\@PosNotaFinalY}{-\ht\@CajaNotaFinal-\dp\@CajaNotaFinal}}{}% Aquí la caja abre hacia abajo, a menos de que usemos \PosNotaFinal*
 \begin{textblock*}{\wd\@CajaNotaFinal}(\@PosNotaFinalXuno,\@PosNotaFinalY)%
  \usebox{\@CajaNotaFinal}%
 \end{textblock*}}

%% Comando que coloca el número de control.
\newcommand{\@PosicionaNumControl}{%
 \savebox{\@CajaControl}{\@EstiloNumControl}%
 \ifbool{ControlXnegativo}{\addtolength{\@PosControlX}{\textwidth-\wd\@CajaControl}}{}%
 \ifbool{ControlYnegativo}{\addtolength{\@PosControlY}{\paperheight}}{}%
 \ifbool{ControlYarriba}{\addtolength{\@PosControlY}{-\ht\@CajaControl-\dp\@CajaControl}}{}%
 \begin{textblock*}{\wd\@CajaControl}(\@PosControlX,\@PosControlY)%
  \usebox{\@CajaControl}%
 \end{textblock*}}

%% Comando que coloca la denominación del documento.
\newcommand{\@PosicionaDenom}{%
 \savebox{\@CajaDenom}{\@EstiloDenom}%
 \ifbool{DenomXnegativo}{\addtolength{\@PosDenomX}{\textwidth-\wd\@CajaDenom}}{}%
 \ifbool{DenomYnegativo}{\addtolength{\@PosDenomY}{\paperheight}}{}%
 \ifbool{DenomYarriba}{\addtolength{\@PosDenomY}{-\ht\@CajaDenom-\dp\@CajaDenom}}{}%
 \begin{textblock*}{\wd\@CajaDenom}(\@PosDenomX,\@PosDenomY)%
  \usebox{\@CajaDenom}%
 \end{textblock*}}

%% Comando que coloca la(s) fecha(s).
\newcommand{\@PosicionaFechas}{%
 \savebox{\@CajaFecha}{\@EstiloFechas}%
 \ifbool{FechaXnegativo}{\addtolength{\@PosFechaX}{\textwidth-\wd\@CajaFecha}}{}%
 \ifbool{FechaYnegativo}{\addtolength{\@PosFechaY}{\paperheight}}{}%
 \ifbool{FechaYarriba}{\addtolength{\@PosFechaY}{-\ht\@CajaFecha-\dp\@CajaFecha}}{}%
 \begin{textblock*}{\wd\@CajaFecha}(\@PosFechaX,\@PosFechaY)%
  \usebox{\@CajaFecha}%
 \end{textblock*}}

%% Comando que coloca la identificación del reporte al inicio.
\newcommand{\@EscribeDatosReporte}{%
 {\@LetraEncReporte
  \setlength{\parindent}{0pt}%
  {\bfseries \@@Proveedor} \@Proveedor \par
  {\bfseries \@@Cliente} \@RazonSocial \par
  \ifdefvoid{\@Contacto}{}{{\bfseries \@@Contacto} \@Contacto \par}%
  \ifdefvoid{\@TituloReporte}{}{\vspace*{\baselineskip}{\@LetraTitReporte \centering \@TituloReporte \par}}}}

%% Comando que coloca las dos cajas del encabezado: la de los datos del cliente y la del resumen.
\newcommand{\@EscribeDatos}{%
 {\@LetraEncFactura% Abrimos un nuevo grupo para que el cambio de tipografía y la redefinición del entorno 'tikzpicture' sean locales.
  \ifdim\@GrosorLineaEncabezado=0pt%
   \setlength{\@SeparaInternaEncabezado}{0pt}\setlength{\@EsquinaLineaEncabezado}{0pt}\setlength{\@SombraEncabezado}{0pt}%
   \renewenvironment{tikzpicture}{\NewDocumentCommand{\node}{o +m}{####2}}{}%
  \fi
  \begin{tikzpicture}%
   \expandafter\node\expandafter[\@DatosTIKZ]{%
   \parbox{\textwidth-\@SombraEncabezado-2\@SeparaInternaEncabezado-\@GrosorLineaEncabezado}{\@DatosAux}}%
    \ifdim\@GrosorLineaEncabezado=0pt\else ;\fi
  \end{tikzpicture}%
  \par\vspace*{\@SombraEncabezado}}}%  Para que textblock* esté en modo vertical.

\newcommand{\@EscribeDatosResumen}{%
 {\@LetraEncFactura% Abrimos un nuevo grupo para que el cambio de tipografía y la redefinición del entorno 'tikzpicture' sean locales.
  \ifdim\@GrosorLineaEncabezado=0pt%
   \setlength{\@SeparaInternaEncabezado}{0pt}\setlength{\@EsquinaLineaEncabezado}{0pt}\setlength{\@SombraEncabezado}{0pt}%
   \renewenvironment{tikzpicture}{\NewDocumentCommand{\node}{o +m}{####2}}{}%
  \fi
  \savebox{\@CajaDatos}{%
   \parbox[t]{\textwidth/2-\@SeparaDatosResumen/2-2\@SeparaInternaEncabezado-\@GrosorLineaEncabezado-\@SombraEncabezado}{\@DatosAux*}}%
  \savebox{\@CajaResumen}{%
   \parbox[t]{\textwidth/2-\@SeparaDatosResumen/2-2\@SeparaInternaEncabezado-\@GrosorLineaEncabezado-\@SombraEncabezado}{\@ResumenAux}}%
  \setlength{\@TamDatos}{\ht\@CajaDatos+\dp\@CajaDatos}%
  \setlength{\@TamResumen}{\ht\@CajaResumen+\dp\@CajaResumen}%
  \ifdim\@TamDatos>\@TamResumen
   \setlength{\@TamFinalEncabezado}{\@TamDatos}%
  \else
   \setlength{\@TamFinalEncabezadoado}{\@TamResumen}%
  \fi
  \begin{tikzpicture}%
   \expandafter\node\expandafter[\@DatosTIKZ]{%
    \parbox[t][\@TamFinalEncabezado]{\textwidth/2-\@SeparaDatosResumen/2-2\@SeparaInternaEncabezado-\@GrosorLineaEncabezado-\@SombraEncabezado}{%    
    \@DatosAux*}}%
   \ifdim\@GrosorLineaEncabezado=0pt\else ;\fi
  \end{tikzpicture}%
  \hspace*{\@SeparaDatosResumen}\hspace*{\@SombraEncabezado}%
  \begin{tikzpicture}%
   \expandafter\node\expandafter[\@ResumenTIKZ]{%
    \parbox[t][\@TamFinalEncabezado]{\textwidth/2-\@SeparaDatosResumen/2-2\@SeparaInternaEncabezado-\@GrosorLineaEncabezado-\@SombraEncabezado}{%
     \@ResumenAux}}%
   \ifdim\@GrosorLineaEncabezado=0pt\else ;\fi
  \end{tikzpicture}%
  \par\vspace*{\@SombraEncabezado}}}%  El \par es para que textblock* esté en modo vertical.

%% Comando que coloca la caja para la nota previa, con la posibilidad de TikZ.
\newcommand{\@EscribeNotaPrevia}{%
 {% Abrimos un nuevo grupo para que la redefinición del entorno 'tikzpicture' sea local.
  \baselineskip0pt
  \ifdefvoid{\@NotaPrevia}{}{%
   \ifdim\@GrosorLineaNotaPrevia=0pt%
    \setlength{\@SeparaInternaNotaPrevia}{0pt}\setlength{\@EsquinaLineaNotaPrevia}{0pt}\setlength{\@SombraNotaPrevia}{0pt}%
    \renewenvironment{tikzpicture}{\NewDocumentCommand{\node}{o +m}{####2}}{}%
   \fi
   \begin{tikzpicture}%
    \expandafter\node\expandafter[\@NotaPreviaTIKZ]{%
    \parbox{\textwidth-\@SombraNotaPrevia-2\@SeparaInternaNotaPrevia-\@GrosorLineaNotaPrevia}{\@EstiloNotaPrevia}}%
     \ifdim\@GrosorLineaNotaPrevia=0pt\else ;\fi
   \end{tikzpicture}%
   \par\vspace*{\@SombraNotaPrevia}\vspace*{\@SeparaVertical}%
   \ifexpandecuadro\else\vfill\fi}}}

%% Comando que coloca la caja para la nota final, con la posibilidad de TikZ.
\newcommand{\@EscribeNotaFinal}{%
 {% Abrimos un nuevo grupo para que la redefinición del entorno 'tikzpicture' sea local.
  \ifdim\@GrosorLineaNotaFinal=0pt%
   \setlength{\@SeparaInternaNotaFinal}{0pt}\setlength{\@EsquinaLineaNotaFinal}{0pt}\setlength{\@SombraNotaFinal}{0pt}%
   \renewenvironment{tikzpicture}{\NewDocumentCommand{\node}{o +m}{####2}}{}%
  \fi
  \begin{tikzpicture}%
   \expandafter\node\expandafter[\@NotaFinalTIKZ]{%
   \parbox{\textwidth-\@PosNotaFinalXuno-\@PosNotaFinalXdos+\oddsidemargin+1in-\@SombraNotaFinal-2\@SeparaInternaNotaFinal-\@GrosorLineaNotaFinal}{\@EstiloNotaFinal}}%
    \ifdim\@GrosorLineaNotaFinal=0pt\else ;\fi
  \end{tikzpicture}%
  \par\vspace*{\@SombraNotaFinal}}}

%% Comandos auxiliares para los datos y el resumen.
%% En el caso de los datos, la versión \@DatosAux se ejecuta cuando no hay resumen, y la versión \@DatosAux* se ejecuta cuando hay resumen.
\NewDocumentCommand{\@DatosAux}{s}{%
 \hfill {\@LetraTitEnc \@@Datos}\hfill \mbox{}\par
 \IfBooleanT{#1}{\vfill}%
 {\bfseries \if@Nombre \@@Nombre \else \@@RazonSocial \fi} \@RazonSocial
 \IfBooleanTF{#1}{\par}{\dotfill}%
 {\bfseries \@@RIF} {\ttfamily \@RIF}\par
 \ifdefvoid{\@Contacto}{}{{\bfseries \@@Contacto} \@Contacto \par}%
 {\bfseries \@@Direccion} \@Direccion \par
 \ifdefvoid{\@Telefono}{}{{\bfseries \@@Telefono} {\ttfamily \@Telefono}}%
 \ifdefvoid{\@Email}
  {\par}
  {\IfBooleanTF{#1}
   {\par}
   {\ifdefvoid{\@Telefono}{}{\hspace*{\@SeparaTelefonoEmail}\@CaracterSeparaTelefonoEmail\hspace*{\@SeparaTelefonoEmail}}}%
  {\bfseries \@@Email} {\ttfamily \@Email}}%
 \IfBooleanT{#1}{\vfill}}

\newcommand{\@ResumenAux}{\hfill {\@LetraTitEnc \@@Resumen}\hfill \mbox{}\vfill\@Resumen\vfill}

%% Comando que coloca la línea diagonal.
\newcommand{\@DibujaLinea}{%
 \setlength{\@LineaNodoXo}{\zposx{LineaNodoACoordX} sp}%
 \setlength{\@LineaNodoYo}{\zposy{LineaNodoBCoordY} sp}%
 \setlength{\@LineaNodoXf}{\zposx{LineaNodoBCoordX} sp}%
 \setlength{\@LineaNodoYf}{\zposy{LineaNodoBCoordY} sp}%
 \addtolength{\@LineaNodoYo}{\@EspacioVacioInicial-\@EspacioVacioFinal}%
 \if@LineaDesdeInicio
  \addtolength{\@LineaNodoXo}{-\tabcolsep@@}%
 \fi
 \if@LineaHastaFinal
  \addtolength{\@LineaNodoXf}{\tabcolsep@@}%
 \fi
 \iffilas
  \addtolength{\@LineaNodoYf}{-.3\baselineskip@@}%
  \if@LineaDescripcion
   \addtolength{\@LineaNodoXo}{-\tabcolsep@@}%
   \addtolength{\@LineaNodoXf}{\tabcolsep@@}%
  \fi
 \else
  \ifsintotales
  \addtolength{\@LineaNodoYf}{-.3\baselineskip@@}%
  \else
   \if@DescG \addtolength{\@LineaNodoYo}{2\baselineskip@@}\fi
   \if@DescR \if@BIR \addtolength{\@LineaNodoYo}{2\baselineskip@@}\fi\fi
   \if@DescA \if@BIA \addtolength{\@LineaNodoYo}{2\baselineskip@@}\fi\fi
   \if@DescE \if@Exento \addtolength{\@LineaNodoYo}{2\baselineskip@@}\fi\fi
   \addtolength{\@LineaNodoYo}{\baselineskip@@}%
   \addtolength{\@LineaNodoYf}{-.3\baselineskip@@}%
   \ifsinivaexpresado
   \else
    \addtolength{\@LineaNodoYo}{2\baselineskip@@}%
    \if@BIR\addtolength{\@LineaNodoYo}{2\baselineskip@@}\fi
    \if@BIA\addtolength{\@LineaNodoYo}{2\baselineskip@@}\fi
    \if@Exento\addtolength{\@LineaNodoYo}{\baselineskip@@}\fi
   \fi
  \fi
 \fi
 \iflinea
  \begin{textblock*}{\paperwidth-\@LineaNodoXo}(\@LineaNodoXo,\paperheight-\@LineaNodoYo)%
   \begin{tikzpicture}%
    \draw (\@LineaNodoXo,\@LineaNodoYo) -- (\@LineaNodoXf,\@LineaNodoYf);%
   \end{tikzpicture}%
  \end{textblock*}%
 \fi}

%% Escribimos los totales al archivo csv.
\newcommand{\@EscribeTotalesEnCSV}{%
 \ifcsv
  \if@Exento
   \if@DescE
    \immediate\write\csv{Totales: \@@SubtotalE: \Moneda \SubtE^^JTotales: \@@DescuentoE: \Moneda \DescE}%
   \fi
   \immediate\write\csv{Totales: \@@BIE: \Moneda \BIE}%
  \fi
  \if@BIR
   \if@DescE
    \immediate\write\csv{Totales: \@@SubtotalR: \Moneda \SubtR^^JTotales: \@@DescuentoR: \Moneda \DescR}%
   \fi
   \immediate\write\csv{Totales: \@@BIR: \Moneda \BIR^^JTotales: \@@AlicuotaR: \Moneda \AlicuotaR}%
  \fi
  \if@DescG
   \immediate\write\csv{Totales: \@@SubtotalG: \Moneda \SubtG^^JTotales: \@@DescuentoG: \Moneda \DescG}%
  \fi
  \immediate\write\csv{Totales: \@@BIG: \Moneda \BIG^^JTotales: \@@AlicuotaG: \Moneda \AlicuotaG}%
  %% (quitado por no vigente)
  % \FPifeq\AlicuotaGDesc{0}\else\immediate\write\csv{Totales: \@@AlicuotaGDesc: \Moneda \AlicuotaGDesc}\fi
  \if@BIA
   \if@DescA
    \immediate\write\csv{Totales: \@@SubtotalA: \Moneda \SubtA^^JTotales: \@@DescuentoA: \Moneda \DescA}%
   \fi
   \immediate\write\csv{Totales: \@@BIA: \Moneda \BIA^^JTotales: \@@AlicuotaA: \Moneda \AlicuotaA}%
  \fi
  \immediate\write\csv{Totales: \@@Total: \Moneda \Total}%
  \immediate\closeout\csv
 \fi}

%% Hubo que redefinir \cleardoublepage agregándole \thispagestyle{empty} porque si no queda una página en blanco con cabeceras de página usando twoside.
\def\cleardoublepage{%
 \clearpage
 \if@twoside
  \ifodd\c@page
   \else
    \hbox{}\thispagestyle{empty}\newpage
    \if@twocolumn
     \hbox{}\newpage
    \fi
   \fi
 \fi}

%</class>
%<*def>
%% Diseño hecho en TikZ para los datos del cliente, el resumen, la nota previa, la descripción y la nota final.
\def\@DatosTIKZ{%
 draw=black, fill=white, inner sep=\@SeparaInternaEncabezado, line width=\@GrosorLineaEncabezado, rounded corners=\@EsquinaLineaEncabezado, copy shadow={top color=black, bottom color=black, shadow xshift=\@SombraEncabezado, shadow yshift=-\@SombraEncabezado}}

\let\@ResumenTIKZ\@DatosTIKZ

\def\@NotaPreviaTIKZ{%
 draw=black, fill=white, inner sep=\@SeparaInternaNotaPrevia, line width=\@GrosorLineaNotaPrevia, rounded corners=\@EsquinaLineaNotaPrevia, copy shadow={top color=black, bottom color=black, shadow xshift=\@SombraNotaPrevia, shadow yshift=-\@SombraNotaPrevia}}

\def\@DescripcionTIKZ{%
 draw=black, fill=white, inner sep=\@SeparaInternaDescripcion, line width=\@GrosorLineaDescripcion, rounded corners=\@EsquinaLineaDescripcion, copy shadow={top color=black, bottom color=black, shadow xshift=\@SombraDescripcion, shadow yshift=-\@SombraDescripcion}}

\def\@NotaFinalTIKZ{%
 draw=black, fill=white, inner sep=\@SeparaInternaNotaFinal, line width=\@GrosorLineaNotaFinal, rounded corners=\@EsquinaLineaNotaFinal, copy shadow={top color=black, bottom color=black, shadow xshift=\@SombraNotaFinal, shadow yshift=-\@SombraNotaFinal}}

%% El diseño de las distintas partes de la hoja.
\newcommand{\@EstiloFirmas}{
 \@LetraFirmas
 \ifnum \value{@Firmas}=1
 \else
  \parbox{\@TamLineaFirmas}{
   \rule{\@TamLineaFirmas}{\@GrosorLineaFirmas}
   \centering \ifdefvoid{\@Conforme}{\@@Conforme}{\@Conforme}}
 \fi
 \hfill
 \parbox{\@TamLineaFirmas}{
  \rule{\@TamLineaFirmas}{\@GrosorLineaFirmas}
  \centering \ifdefvoid{\@Emisor}{\ifcotizacion \@@Emisor \else \@@Pagado \fi}{\@Emisor}}}

\newcommand{\@EstiloNotaPrevia}{\@LetraNotaPrevia \@NotaPrevia}

\newcommand{\@EstiloNotaFinal}{\@LetraNotaFinal \@NotaFinal}

\newcommand{\@EstiloNumControl}{%
 \renewcommand{\baselinestretch}{0}\selectfont
 \begin{tabular}{@{\extracolsep{0pt}}l}
  \@LetraDenom \@@Control \\[.4em]
  {\@LetraDenom\@NumSerieControl-}\space\@LetraNumeracion\@NumControl \\
 \end{tabular}}

\newcommand{\@EstiloDenom}{%
 \@LetraDenom\@Denominacion
 \ifsinnumero
 \else
  \space\space{\@LetraNumeracion N\textordmasculine\space\@Num}%
 \fi}

\newcommand*{\@EstiloFechas}{%
 \renewcommand{\baselinestretch}{0}%
 \setlength{\tabcolsep}{0pt}%
 \@LetraFecha\begin{tabular}{r@{\extracolsep{1ex}}l@{\extracolsep{0pt}}}
 \@@Fecha & \ttfamily\@Fecha \\[.3em]
 \ifvencimiento \@@FechaVencimiento & \ttfamily\@FechaVencimiento \\ \else \vspace*{-.3em}\fi
 \end{tabular}}

\newcommand{\@InfoPaginaNoTitulo}{\@LetraInfoPagina Reporte correspondiente a la factura n\'umero \@Num \space de fecha \@Fecha}

\newcommand{\@InfoPaginaTitulo}{\@LetraInfoPagina \@TituloReporte \space - \@Proveedor \space - \@Fecha}

%% Estilos de página: normal, sólo cabeceras, sólo pies de página y sólo número de página al pie.
\fancypagestyle{plain}{%
 \expandafter\ps@fancy
 \fancyhf{}%
 \renewcommand{\headrulewidth}{\@GrosorLineaInfoPag}%
 \fancyhead[LEO]{\@InfoPagina}%
 \fancyfoot[C]{\thepage}}

\fancypagestyle{soloarriba}{%
 \expandafter\ps@fancy
 \fancyhf{}%
 \renewcommand{\headrulewidth}{\@GrosorLineaInfoPag}%
 \fancyhead[LO]{\@InfoPagina}%
 \fancyhead[RO]{\thepage}%
 \if@twoside
  \fancyhead[LE]{\thepage}%
  \fancyhead[RE]{\@InfoPagina}%
 \fi}

\fancypagestyle{soloabajo}{%
 \expandafter\ps@fancy
 \fancyhf{}%
 \renewcommand{\headrulewidth}{0pt}%
 \fancyfoot[LO]{\@InfoPagina}%
 \fancyfoot[RO]{\thepage}%
 \if@twoside
  \fancyfoot[RE]{\@InfoPagina}%
  \fancyfoot[LE]{\thepage}%
 \fi}

\fancypagestyle{solonumpag}{%
 \expandafter\ps@fancy
 \fancyhf{}%
 \renewcommand{\headrulewidth}{0pt}%
 \fancyfoot[C]{\thepage}}

%</def>
%<*!class&!def&!firma&!fondo&!membrete&!aux&!batchfile>
%% Descripción de este ejemplo:
%
%%<desc>
%<beispiel>%%  Adaptado a otra legislación
%<factuno|factdos|facttres|factcuatro|factcinco|factseis|factsiete|factocho|factnueve|factdiez|repuno|repdos>%%  Escribimos una factura
%<reptres>%%  No escribimos factura ni usamos las macros que definen valores para su uso
%<factnueve|factdiez|repuno>%%  Escribimos en el reverso de la factura
%<prefac|beispiel>%%  (prefactura) Escribimos una prefactura
%<cotiza>%%  (cotizacion) Escribimos una cotización
%<repuno|repdos|reptres>%%  Escribimos un reporte
%<repuno>%%  (sinreverso) Escribimos el reverso de la factura pero sólo producimos su anverso
%<repuno>%%  (conreporte) Generamos factura y reporte
%<repdos>%%  (soloreporte) Escribimos factura y reporte, pero sólo producimos el reporte
%<reptres>%%  (soloreporte) Generamos solamente el reporte
%<factcinco|factocho>%%  Escribimos factura y reporte, pero sólo producimos la factura
%
%<factdiez>%%  Establecemos algunas opciones de la clase a través de comandos
%
%<factuno>%%  Hacemos una factura básica con todo en sus valores por omisión, no ajustamos nada
%
%<reptres>%%  (twoside) El reporte se imprime en las dos caras del papel
%
%<factcinco>%%  (sincantidad) Cuadro de la factura sin la columna de cantidades
%<factdos|factseis>%%  (numitem) Con la columna para enumerar los ítems
%<factdos|factseis>%%  (descuentos) Con la columna para descuentos individuales
%<factseis>%%  (subtotal) Con la columna extra para subtotal
%<factcinco>%%  (iva) Con la columna para el IVA
%
%<facttres>%%  (agrupado) Agrupados los totales por tipo y no por gravamen
%<factcuatro|factcinco|factsiete>%%  (filas) Totalización en modo de filas
%<factsiete>%%  (centrado) Centrados los totales en el modo de filas
%
%<factdos>%%  (dcu) Primero la descripción, luego la cantidad, luego el precio unitario
%<factcuatro>%%  (duc) Primero la descripción, luego el precio unitario, luego la cantidad
%<factcinco|factseis|factsiete>%%  (denom) Escribe la denominación (“FACTURA”) y el número
%<cotiza>%%  (sinnumero) Escribe la denominación (“COTIZACIÓN”) pero sin número
%<cotiza>%%  (sintotales) No totaliza, para que el cliente tenga un presupuesto por distintos ítems que luego elegirá entre ellos, es decir, una lista de precios
%
%<factdos|factseis>%%  (sinlineasve) Sin líneas verticales externas
%<facttres|factcuatro>%%  (sinlineasvi) Sin líneas verticales internas
%<factdiez>%%  (sinlineasv) Sin líneas verticales
%<factdos>%%  (sinlineahtot) Sin línea horizontal interna después de los ítems y antes de los totales
%<factcuatro>%%  (sinlineahtit) Sin línea horizontal interna después del título de las columnas
%<factseis>%%  (sinlineashi) Sin líneas horizontales internas
%<factsiete>%%  (sinlineashe) Sin líneas horizontales externas
%<factnueve>%%  (sinlineahni)  Sin línea horizontal interna después de la nota interna.
%<factdiez>%%  (sinlineash) Sin líneas horizontales
%<facttres|factcinco>%%  (conlineasha) Separado cada ítem con una línea horizontal
%
%<factcinco|factocho>%%  (sinexpandir) Sin expandir el cuadro de la factura ni la descripción
%<facttres|factcuatro|prefac>%%  (expandecuadro) Expande el cuadro de la factura con espacio en blanco
%<prefac|factcuatro>%%  (linea) Con línea para inutilizar el espacio en blanco
%
%<facttres>%%  (impdesc) Con el tipo de impuesto como parte de la descripción y no de los precios
%<factdos>%%  (centradesc) Centra la descripción en los elementos en la factura
%<factsiete>%%  (sinmarcas) Cuando la descripción de un ítem ocupa más que una línea, no pone el carácter que indica continuación en la siguiente línea ni los caracteres en las columnas vacías
%<factcinco|factseis|factsiete>%%  (todosiva) Forzando a totalizar todos los IVA así no existan productos gravados con ellos
%<factnueve>%%  (G) Escribiendo una letra ‘G’ en los ítems gravados con IVA general así sea el único gravamen a facturar
%<factdiez>%%  (sinmonedatotales) No agrega el símbolo monetario en los totales
%<factocho>%%  (siniva) Sin IVA
%<factdos>%%  (unafirma) Sólo la firma del emisor
%<facttres>%%  (sinfirmas) Sin firmas
%<factdiez>%%  (vencimiento) Con fecha de vencimiento explícita así no haya crédito
%
%<factocho>%%  (csv) Escribe un archivo CSV con la información numérica de la factura
%<beispiel>%%  (nospanish) Sin cargar babel en castellano, ni la codificación T1 para la tipografías
%
%<factdos|factcuatro>%%  Usando \Nombre y \CI en lugar de \RazonSocial y \RIF
%<factdos|factcuatro|reptres>%%  Sin especificar la persona de contacto
%<factcinco|factseis>%%  Sin especificar el teléfono del cliente
%<factcuatro|factseis>%%  Sin especificar el e-mail del cliente
%<factcinco|repuno>%%  (\Emisor) Especificando el nombre del emisor de la factura
%<factcinco|factseis>%%  (\Conforme) Especificando el nombre de quien firma la copia de la factura como recibida
%<repuno>%%  Sin especificar el proveedor, así toma el emisor de la factura como el nombre del proveedor
%<repdos|reptres>%%  (\Proveedor) Especificando el nombre del proveedor para el encabezado del reporte
%
%<facttres>%%  Sin especificar fecha para que tome la fecha del día.
% %<factnueve>%%  Con nota interna automática por el descuento en pagos electrónicos.
% %<factdiez>%%  Con nota interna automática con el monto en Bs.F dada la fecha de emisión de la factura antes de que a la moneda se le quitara el adjetivo “soberano”.
%
%<factcuatro>%%  (\Fecha*) Suministrando la fecha según formato de datetime2 sin alterar cómo sale escrita
%<facttres>%%  (\FormatoFecha) Cambiando el separador de la fecha para que salga escrita dd/mm/yyyy
%<factcinco>%%  (\FormatoFecha) Suministrando la fecha según el formato yyyymmdd pero que salga escrita dd.mm.yyyy
%<factseis>%%  (\FormatoFecha) Cambiando el formato para que la fecha salga escrita en el formato default de datetime2 pero cambiando el separador yyyy/mm/dd
%<factocho>%%  Redefiniendo el formato de la fecha (redefiniendo macros \@@Fecha y \@EstiloFecha)
%
%<factnueve>%%  (\Credito) Especificando la cantidad de días para el crédito
%<facttres>%%  (\DescuentoE y \DescuentoG) Con descuentos totales en porcentajes para exentos y gravables con IVA general
%<factocho>%%  (\Descuento*) Con descuentos totales en dinero neto (y no en porcentajes) aplicables por igual a todos los tipos de elementos existentes
%<prefac>%%  (\Descuento) Con descuentos totales en porcentaje aplicables por igual a todos los tipos de elementos existentes
%
%<factsiete>%%  Redefiniendo el texto para la cantidad (redefiniendo macro \@@Cantidad)
%
%<factsiete>%%  (\GeometriaFactura) Cambiando la geometría del papel de la factura
%<factdiez>%%  (\GeometriaReverso) Cambiando la geometría del reverso de la factura
%<repdos>%%  (\GeometriaReporte) Cambiando la geometría del papel del reporte
%<prefac|cotiza>%%  (\FondoFactura) Con el fondo de la imprenta agregado
%<repuno>%%  (\FondoReporte) Especificamos el fondo o membrete del reporte, sólo para la primera página
%<repdos>%%  (\FondoReporte*) Especificamos el fondo o membrete del reporte para todas las páginas
%<reptres>%%  Sin especificar fondo de reporte
%<repuno>%%  (\InicioReporte) Estableciendo un margen superior distinto para la primera página del reporte por presencia de membrete
%<repuno|reptres>%%  (\TituloReporte) Establecemos el título del reporte sin que salga en las cabeceras ni pies de página
%<repdos>%%  (\TituloReporte*) Establecemos el título del reporte para que salga en las cabeceras o pies de página
%<reptres>%%  (\InfoPagina) Personalizamos la información de la cabecera del reporte y mostramos el nombre del proveedor allí
%<reptres>%%  (\EstiloPagReporte) Cambiamos el estilo de página para que la información del reporte aparezca como pie de página
%<cotiza>%%  (\FirmaFactura) Agregando la firma escaneada en la factura, es decir, en la cotización
%
%<factdos|facttres|factsiete>%%  (\SepVertical) Alterando la separación mínima vertical entre los elementos
%<factdos|factcinco>%%  (\SepDatosResumen) Alterando la separación horizontal entre los datos del cliente y el resumen
%<factdos|facttres|factseis>%%  (\SepEncabezado) Alterando el margen interno desde el contorno hasta el texto en el encabezado
%<factnueve>%%  (\SepNotaPrevia) Alterando el margen interno desde el contorno hasta el texto en la nota previa
%<factdos|factseis>%%  (\SepDescripcion) Alterando la distancia desde el contorno hasta el texto en la descripción
%<factnueve>%%  (\SepNotaFinal) Especificando la distancia desde el contorno hasta el texto de la nota final, predeterminadamente sin contorno
%<factdos|factcuatro>%%  (\SepItemsExtra) Agregando una separación extra entre los ítems en la factura
%<factsiete>%%  (\SepFilas) Alterando la separación horizontal y vertical entre los totales en el modo de filas
%<facttres|factsiete>%%  (\LineaEncabezado) Alterando la línea de las cajas del encabezado, la curvatura en la esquina y la sombra
%<factdiez>%%  (\LineaNotaPrevia) Alterando la línea de la nota previa, para desaparecerla
%<factdos|factsiete>%%  (\LineaDescripcion) Alterando la línea de la descripción, la curvatura en la esquina y la sombra
%<factnueve>%%  (\LineaNotaFinal) Agregando el contorno para la nota final que predeterminadamente está deshabilitado
%<factsiete>%%  (\LineaFirmas) Alterando la longitud de las líneas para firmar y su grosor
%<factsiete>%%  (\PosFecha, \PosFirmas y \PosDenom) Alterando la posición de la fecha, las firmas y de la denominación del documento
%<factnueve>%%  (\PosNotaFinal) Alterando la posición de la nota final
%<factcuatro|factocho>%%  (\Letra***) Cambiando varias de las tipografías
%
%<!factcuatro&!reptres>%%  (\NotaFinal) Con nota final
%<factnueve>%%  Haciendo mención a las fechas y al crédito en la nota final
%<factcuatro>%%  Sin nota final
%<factuno|factdos|factcinco|factocho>%%  (\Resumen) Con resumen
%<facttres|factcuatro|factseis|factsiete|prefac|cotiza|repuno|repdos>%%  Sin resumen
%<factcuatro|factnueve|factdiez>%%  (\NotaPrevia) Con nota previa
%<factocho|factnueve>%%  (\NotaInterna) Con nota interna
%<factnueve>%%  (\NotaExterna) Con nota externa
%
%<factdos|facttres>%%  Con artículos gravados con todos los IVA
%<factseis|factsiete|factocho>%%  Escribimos un ítem cuya descripción ocupa más que una línea
%<factdos|factcuatro|factcinco>%%  (\ItemX) Agregando un ítem sin precio
%
%<!facttres&!reptres&!beispiel>%%  (\Descripcion) Con descripción larga
%<facttres|reptres>%%  Sin descripción larga
%
%<factseis|repdos>%%  Con etiquetas y referencias
%
%<repuno>%%  (\FirmaReporte) Agregando en el reporte la línea para firmar a mano
%<factocho|repdos|reptres>%%  (\FirmaReporte) Agregando la firma escaneada en el reporte
%<factocho|repdos|reptres>%%  Con \FirmaReporte, ajustando el corrimiento en “y” de la firma respecto a la línea
%<repdos|reptres>%%  Con \FirmaReporte, cambiando la ubicación en “x” de la línea de la firma y cambiando la longitud de la línea hacia los lados
%<reptres>%%  Con \FirmaReporte, cambiando el nombre de la persona que va a firmar
%%</desc>
%%
%<factuno>\documentclass{factura}
%<!factuno>\documentclass[
%<prefac|beispiel>prefactura,
%<cotiza>cotizacion,
%<repuno>sinreverso,
%<repuno>conreporte,
%<repdos|reptres>soloreporte,
%<beispiel>nospanish,
%<reptres>twoside,
%<factcinco>sincantidad,
%<factdos|factseis>numitem,
%<factdos|factseis>descuentos,
%<factseis>subtotal,
%<factcinco>iva,
%<factcuatro|factcinco|factsiete>filas,
%<factdiez>filas, % luego queda sin efecto con \filasfalse en el preámbulo
%<factsiete>centrado,
%<facttres>agrupado,
%<factcinco|factseis|factsiete>denom,
%<cotiza>sinnumero, % sin esta opción aparecerá el número de la cotización.
%<cotiza>sintotales,
%<factdos>dcu,
%<factcuatro>duc,
%<factdos|factseis>sinlineasve,
%<facttres|factcuatro>sinlineasvi,
%<factdos>sinlineahtot,
%<factcuatro>sinlineahtit,
%<factseis>sinlineashi,
%<factsiete>sinlineashe,
%<factnueve>sinlineahni,
%<facttres|factcinco>conlineasha,
%<factcinco|factocho>sinexpandir,
%<facttres|factcuatro|prefac>expandecuadro,
%<factcuatro|prefac>linea,
%<facttres>impdesc,
%<factdos>centradesc,
%<factsiete>sinmarcas,
%<factcinco|factseis|factsiete>todosiva,
%<factnueve>G,
%<factdiez>sinmonedatotales,
%<factocho>siniva,
%<factnueve>% electro,
%<factdos>unafirma,
%<prefac|cotiza>% dosfirmas, % para poner las dos firmas.
%<prefac>% unafirma, % para poner la firma del emisor.
%<cotiza>% sinfirmas, % para no poner firmas.
%<facttres>sinfirmas,
%<factdiez>vencimiento,
%<factocho>csv,
%<factdiez>csv, % luego queda sin efecto con \csvfalse en el preámbulo
%<!factuno>]{factura}

%<factdiez>% Cambiamos por aquí como condicionales las opciones de la clase; es lo mismo que arriba en \documentclass.
%<factdiez>\sinlineasvtrue \sinlineashtrue \filasfalse \csvfalse
%<factdiez>
%%
%% This is file `factura-ejemplo-aux.tex',
%% generated with the docstrip utility.
%%
%% The original source files were:
%%
%% factura.dtx  (with options: `aux')
%% 
%% Copyright 2019 Emilio Augusto Lazo Zaia <emiliolazozaia@gmail.com>
%% 
%% This work may be distributed and/or modified under the
%% conditions of the LaTeX Project Public License, either
%% version 1.3c of this license or (at your option) any later
%% version. This version of this license is in
%%  http://www.latex-project.org/lppl/lppl-1-3c.{html|txt|tex|pdf}
%% and the latest version of this license is in
%%  http://www.latex-project.org/lppl.txt
%% and version 1.3c or later is part of all distributions of
%% LaTeX version 2008/05/04 or later.
%% 
%% This work has the LPPL maintenance status `maintained'.
%% 
%% The Current Maintainer of this work is Emilio Augusto Lazo Zaia.
%% 
%% This work consists of the files README and factura.dtx
%% with its derived files factura.ins, factura.cls, factura.def,
%% factura-ejemplo-aux.tex, factura-ejemplo-firma.tex,
%% factura-ejemplo-fondo.tex, factura-ejemplo-membrete.tex,
%% factura-ejemplo-cotizacion.tex, factura-ejemplo-prefactura.tex,
%% factura-ejemplo-factura01.tex, factura-ejemplo-factura02.tex,
%% factura-ejemplo-factura03.tex, factura-ejemplo-factura04.tex,
%% factura-ejemplo-factura05.tex, factura-ejemplo-factura06.tex,
%% factura-ejemplo-factura07.tex, factura-ejemplo-factura08.tex,
%% factura-ejemplo-factura09.tex, factura-ejemplo-factura10.tex,
%% factura-ejemplo-reporte1.tex, factura-ejemplo-reporte2.tex,
%% factura-ejemplo-reporte3.tex and factura-beispiel-rechnung.tex.
%% 
%% Helper file; not to be compiled standalone, only to be included on examples.
%% Archivo auxiliar; no para ser compilado sino para ser incluido en los ejemplos.
%%
\usepackage{xspace,catchfilebetweentags,regexpatch}
\def\ejemplooptcolor{\color{blue}}
\def\ejemplomacrocolor{\color{purple}}
\makeatletter
\catcode`\%=11
\catcode`\|=14
\newcommand{\Ejemplo}{|
 \newpage
 \ClearShipoutPictureBG
 \newgeometry{vmargin={2cm,2cm},hmargin={2cm,2cm},footskip=0pt,headheight=0pt,headsep=0pt,footnotesep=0pt}|
 \thispagestyle{empty}|
 \catcode`\<=12
 \catcode`\>=12
 \CatchFileBetweenDelims{\DescEjemplo}{\jobname.tex}{<desc>}{</desc>}[\catcode37=11\catcode13=14\makeatletter]|
 \catcode`\%=11
 \makeatletter
 \regexpatchcmd*{\DescEjemplo}{\cC.\cO[*]*}{\c{begingroup}\c{ejemplomacrocolor}\c{ttfamily}\c{string}\0\c{endgroup}\c{xspace}}{}{}|
 \regexpatchcmd*{\DescEjemplo}{(\()(\cL[a-zG]*)(\))}{(\c{begingroup}\c{ejemplooptcolor}\c{ttfamily}\2\c{endgroup})}{}{}|
 \xpatchcmd*{\DescEjemplo}{%% }{\item}{}{}|
 \xpatchcmd*{\DescEjemplo}{}{}{}|
 \makeatother
 {\bfseries Descripci\'on de este ejemplo {\slshape \jobname.tex}:}
 \begin{itemize}
  \DescEjemplo
 \end{itemize}
 \catcode`\%=14
 \catcode`\<=13
 \catcode`\>=13
 }
\catcode`\%=14
\catcode`\|=12
\makeatother
%% 
%%
%% End of file `factura-ejemplo-aux.tex'.
% Se incluye para efectos del ejemplo, así tendremos disponible la macro \Ejemplo.
\usepackage[utf8]{inputenc}
%<beispiel>\usepackage[official]{eurosym}
%<beispiel>\usepackage[ngerman]{babel}
%<beispiel>\usepackage[T1]{fontenc}
%<beispiel>
%<beispiel>\Moneda{\euro{}}
%<beispiel>\makeatletter
%<beispiel>\def\@AlicuotaG{0.19}
%<beispiel>\def\@AlicuotaR{0.07}
%<beispiel>\def\@DenomAlicuotaG{A}
%<beispiel>\def\@DenomAlicuotaR{C}
%<beispiel>\def\@DenomAlicuotaE{B}
%<beispiel>\def\@@Datos{Kundenidentifikation}
%<beispiel>\def\@@Resumen{Zusammenfassung}
%<beispiel>\def\@@Fecha{Datum:}
%<beispiel>\def\@@Nombre{Kunde:} \def\@@RazonSocial{Kunde:}
%<beispiel>\def\@@RIF{USt-IdNr:} \def\@@CT{USt-IdNr:}
%<beispiel>\def\@@Contacto{Kontactperson:}
%<beispiel>\def\@@Direccion{Adresse:}
%<beispiel>\def\@@Telefono{Telefon:}
%<beispiel>\def\@@Email{E-mail:}
%<beispiel>\def\@@Cliente{Kunde:}
%<beispiel>\def\@@Proveedor{Lieferant:}
%<beispiel>\def\@@Cantidad{Menge}
%<beispiel>\def\@@Descripcion{Beschreibung}
%<beispiel>\def\@@PrecioU{Einzelpreis}
%<beispiel>\def\@@PrecioT{Gesamt}
%<beispiel>\def\@@Precio{Preis}
%<beispiel>\def\@@AlicuotaG{MwSt A}
%<beispiel>\def\@@AlicuotaR{MwSt C}
%<beispiel>\def\@@BIG{Netto 19\%}
%<beispiel>\def\@@BIR{Netto 7\%}
%<beispiel>\def\@@BIE{Mwst-befreit}
%<beispiel>\def\@@Total{Brutto}% Zu zahlen / Brutto / Summe / Gesamt
%<beispiel>\def\@@Conforme{der Kunde}
%<beispiel>\def\@@DenomFactura{Rechnung}
%<beispiel>\def\@@DenomCotiza{Voranschlag}
%<beispiel>
%<beispiel>\renewcommand{\@EstiloDenom}{\@LetraDenom\@Denominacion-\ifsinnumero\else\@LetraNumeracion Nr:\space\@Num\fi}
%<beispiel>\renewcommand{\@EstiloNumControl}{}
%<beispiel>\makeatother

%<factcinco|factseis|factsiete|cotiza|prefac|repuno|repdos|beispiel>\Num{1}
%<prefac>% \NumControl{1001}
%<prefac>% \NumSerieControl{9}
%<!factdos&!factcuatro&!beispiel>\RazonSocial{Ayala San Vicente Usureros}
%<factdos|factcuatro|beispiel>\Nombre{Jacinto de Ayala y San Vicente}
%<!factdos&!factcuatro&!beispiel&!reptres>\RIF{Y-12.345.678-9}
%<factdos|factcuatro|beispiel>\CI{V-987.654.321}
%<!factdos&!factcuatro&!reptres&!beispiel>\Contacto{Jacinto de Ayala}
%<!reptres>\Direccion{Av. Ayala, Galpón San Vicente, Sector Arauca, Apure, Venezuela}
%<!factcinco&!factseis&!reptres>\Telefono{+58 (987) 654.32.10}
%<!factcuatro&!factseis&!reptres>\Email{jacinto@ayala.sv}
%<factcinco|repuno>\Emisor{Lázaro de la Vega}
%<factcinco>\Conforme{\Contacto}
%<factseis>\Conforme{Ramela San Vicente}
%<repdos|reptres>\Proveedor{Lázaro de la Vega y San Román}

%<factcinco>\FormatoFecha[yyyymmdd]{ddmmyyyy}{.}
%<facttres>\FormatoFecha{ddmmyyyy}{/}
%<factseis>\FormatoFecha{default}{/}
%<!facttres&!factcuatro&!factcinco>\Fecha{06-11-2019}
%<factcuatro>\Fecha*{2019}{06}{11}
%<factcinco>\Fecha{20191106}
%<factsiete|factocho>
%<factsiete|factocho>\makeatletter
%<factocho> \renewcommand{\@@Fecha}{\slshape FECHA:}
%<factocho> \renewcommand{\@EstiloFechas}{\@LetraFecha\bfseries\@@Fecha\space{\color{blue}\@Fecha}}
%<factsiete> \renewcommand{\@@Cantidad}{Cant.}
%<factsiete|factocho>\makeatother
%<factnueve>\Credito{11}

%<prefac>\Descuento{18}
%<facttres>\DescuentoE{15}
%<facttres>\DescuentoG{20}
%<factocho>\Descuento*{300}
%<prefac|facttres|factocho>
%<factsiete>\GeometriaFactura{legalpaper,vmargin={5cm,5.5cm},hmargin={1cm,1cm},footskip=0pt,headsep=0pt,headheight=0pt}
%<factdiez>\GeometriaReverso{vmargin={4cm,4cm},hmargin={4cm,4cm},footskip=0pt,headsep=0pt,headheight=0pt}
%<repdos>\GeometriaReporte{vmargin={4.4cm,2.5cm},hmargin={3cm,2cm},footskip=1.2cm,headsep=0.777cm,headheight=1.5cm}
%<prefac|cotiza|beispiel>\FondoFactura{factura-ejemplo-fondo}
%<repuno>\FondoReporte{factura-ejemplo-membrete}
%<repdos>\FondoReporte*{factura-ejemplo-membrete}
%<repuno>\InicioReporte{2.2cm}
%<repuno|reptres>\TituloReporte{El título del reporte}
%<repdos>\TituloReporte*{El título del reporte}
%<reptres>\InfoPagina{La cabecera o pie del reporte - \Proveedor}
%<repuno|reptres>% \EstiloPagReporte{soloabajo}
%<repuno|reptres>% \EstiloPagReporte[soloabajo]{plain}
%<repuno|reptres>% \EstiloPagReporte[empty]{plain}
%<repdos|reptres>% \EstiloPagReporte[empty]{soloabajo}
%<reptres>% \EstiloPagReporte{plain}
%<repuno|repdos|reptres>% \EstiloPagReporte{solonumpag}
%<repdos>% \EstiloPagReporte{soloabajo}
%<repuno|reptres>% \EstiloPagReporte[solonumpag]{plain}
%<repuno|reptres>% \EstiloPagReporte[solonumpag]{soloarriba}
%<reptres>\EstiloPagReporte[solonumpag]{soloabajo}
%<repuno>% \EstiloPagReporte[solonumpag]{soloabajo}
%<repuno|reptres>% \EstiloPagReporte[empty]{soloarriba}
%<repuno|reptres>% \EstiloPagReporte{soloarriba}
%<cotiza>\FirmaFactura{\includegraphics[width=4.5cm]{factura-ejemplo-firma}}[0pt][17pt]
%<factdos|facttres|factsiete>\SepVertical{0.75cm}
%<factdos|factcinco>\SepDatosResumen{1cm}
%<factdos|facttres|factseis>\SepEncabezado{16pt}
%<factdos|factcuatro>\SepItemsExtra{5pt}
%<factdos|factseis>\SepDescripcion{16pt}
%<factnueve>\SepNotaPrevia{16pt}
%<factsiete>\SepFilas{1cm}{10pt}
%<factnueve>\SepNotaFinal{16pt}
%<facttres|factsiete>\LineaEncabezado{1pt}{20pt}{3pt}
%<factdiez>\LineaNotaPrevia{0pt}
%<factdos|factsiete>\LineaDescripcion{1pt}{20pt}{3pt}
%<factnueve>\LineaNotaFinal{0.4pt}{4pt}{3pt}
%<factsiete>\LineaFirmas{6.5cm}{2pt}
%<factsiete>% Puede usarse \PosFecha* o \PosDenom*, \PosControl* o \PosNotaFinal* para que el texto se despliegue en la posición 'Y' dada pero hacia arriba y no hacia abajo. \PosFirmas no acepta eso.
%<factnueve>\PosNotaFinal{.9cm}{.9cm}{-5.3cm}
%<factsiete>\PosFecha{-0cm}{-1.5cm}
%<factsiete>\PosFirmas{1cm}{1cm}{-3cm}
%<factsiete>\PosDenom{-0cm}{4.3cm}
%<factcuatro>\LetraDescripcion{\fontsize{20}{30}}
%<factcuatro>\LetraTitEnc{\fontsize{15}{20}\bfseries}
%<factcuatro>\LetraFirmas{\fontsize{16}{20}\itshape}
%<factcuatro>\LetraNotaFinal{\fontsize{18}{28}\bfseries}
%<factocho>\LetraItems{\fontsize{20}{30}}
%<factocho>\LetraNumeros{\fontsize{20}{30}}
%<factocho>\LetraTipoIVA{\fontsize{20}{30}\bfseries}
%<factocho>\LetraTitTotales{\fontsize{20}{30}\slshape}
%<factocho>\LetraNumTotales{\fontsize{20}{30}}
%<factocho>\LetraTitColumnas{\fontsize{15}{20}}
%<factocho>\LetraTitTotal{\fontsize{20}{30}\bfseries\slshape}
%<factocho>\LetraNumTotal{\fontsize{20}{30}\bfseries}
%<factocho>\tabcolsep7pt
%<!factuno&!factcinco&!prefac&!reptres>
%<cotiza>% \Denom{PRESUPUESTO} % También podemos cambiar el nombre del documento.
%<!factcuatro&!factnueve&!beispiel&!reptres>\NotaFinal{La nota final...}
%<factnueve>\NotaFinal{La factura se emitió el \Fecha{} con crédito de \Credito{} dias, siendo su vencimiento el \FechaVencimiento.}
%<beispiel>\NotaFinal{Die Zahlungsbedingungen}
%<factuno|factdos|factcinco|factocho>\Resumen{El resumen opcional}
%<beispiel>\Resumen{Zusammenfassung}
%<factcuatro|factnueve|factdiez>\NotaPrevia{\hfill LA NOTA PREVIA}
%<factocho|factnueve>\NotaInterna{LA NOTA INTERNA}
%<factnueve>\NotaExterna{LA NOTA EXTERNA}

\begin{document}
%<!reptres>
%<!reptres>\begin{factura}
%<!factdos&!factcinco&!factseis&!repdos&!reptres&!beispiel> \Item{1}{ítem con IVA general}{1000}
%<factseis|repdos> \Item{2}{ítem con IVA general, viene con descuento}{1000}{8}[Un ítem que ofrecemos a precio de costo y que está gravado con IVA general]\label{primeritem}
%<factuno|factcuatro|repdos> \Item{4}{otro ítem con IVA general}{100}
%<cotiza> \Item{1}{otro ítem con IVA general}{1100}
%<cotiza> \Item{1}{otro ítem más con IVA general}{1200}
%<facttres|prefac> \ItemR{1}{ítem con IVA reducido}{1000}
%<facttres> \ItemA{1}{ítem con IVA adicional}{1000}
%<facttres> \ItemE{1}{ítem exento de impuesto}{1000}
%<factdos|factseis> \Item{1}{ítem con IVA general, viene con descuento}{1000}{12}
%<factsiete|factocho> \Item{10}{otro ítem con IVA general (descripción larga para mostrar el cambio de línea)}[otro ítem con IVA general (esta descripción es sólo para el .csv)]{1000}
%<factdos> \ItemR{1}{ítem con IVA reducido, viene con descuento}{1000}{5}
%<factdos> \ItemA{1}{ítem con IVA adicional, viene con descuento}{1000}{10}
%<factdos> \ItemE{1}{ítem exento de impuesto}{1000}
%<factdos|factcuatro> \ItemX{1}{línea sin precio}
%<factcinco> \Item{ítem sin cantidad, con IVA general}{1000}
%<factcinco> \ItemA{ítem sin cantidad, con IVA adicional}{1000}
%<factcinco> \ItemX{línea sin precio}
%<beispiel> \Item{1}{Artikel mit 19\% MwSt}{100}
%<beispiel> \ItemR{1}{Artikel mit 7\% MwSt}{100}
%<beispiel> \ItemE{1}{Artikel mit kein MwSt}{100}
%<!reptres>\end{factura}
%<!facttres&!reptres>
%<!facttres&!factseis&!reptres&!beispiel>\Descripcion{La descripción larga}
%<factseis>\Descripcion{
%<factseis>
%<beispiel>\Descripcion{Lange Beschreibung.}
%<factcinco|factocho|factnueve|factdiez|repuno|repdos|reptres>
%<factnueve|factdiez|repuno>\begin{reverso}
%<factnueve|factdiez|repuno>esquina superior izquierda del reverso\hfill esquina superior derecha del reverso\par
%<factnueve|factdiez|repuno>\vfill
%<factnueve|factdiez|repuno>esquina inferior izquierda del reverso\hfill esquina inferior derecha del reverso
%<factnueve|factdiez|repuno>\end{reverso}
%<repuno>
%<factcinco|factocho|repuno|repdos|reptres>\begin{reporte}
%<factcinco|factocho|repuno|repdos|reptres> El reporte - página 1\newpage El reporte - página 2 \newpage El reporte - página 3.
%<repdos>
%<factseis|repdos> Se facturaron en total \arabic{NumItem} artículos.
%<factseis|repdos>
%<factseis|repdos> En el artículo número \itemref{primeritem} se facturaron \cantref{primeritem} “\descref{primeritem}”, cada uno por un valor de \Moneda~\puref{primeritem}, dando un subtotal por este elemento de \Moneda~\subtref{primeritem}; el descuento en ese artículo es del \dctoref{primeritem}\%, dando un total de \Moneda~\ptref{primeritem}, y esta gravado con IVA de \ivaref{primeritem}\%. Ampliando la descripción podemos decir que este ítem es: “\ldescref{primeritem}”.
%<factseis|repdos>
%<factseis|repdos> La base imponible total de esta factura es de \Moneda~\numprint\BIG; el IVA es de \Moneda~\numprint\AlicuotaG,  y el total final es de \Moneda~\numprint\Total.
%<factseis|repdos>
%<factseis|repdos> % Estas son todas las macros para las totalizaciones: \BIG, \BIR, \BIA, \BIE, \AlicuotaG, \AlicuotaR, \AlicuotaA, \SubtG, \SubtR, \SubtA, \SubtE, \DescG, \DescR, \DescA, \DescE.
%<factseis>}
%<repuno> \vspace{1.5cm}\FirmaReporte
%<factocho> \FirmaReporte[\includegraphics[width=4.5cm]{factura-ejemplo-firma}][0pt][17pt]
%<repdos> \FirmaReporte[\includegraphics[width=4.5cm]{factura-ejemplo-firma}][0pt][17pt][3cm][3cm]
%<reptres> \FirmaReporte[\includegraphics[width=4.5cm]{factura-ejemplo-firma}][0pt][17pt]{el que firma}[4cm][0.8cm]
%<factcinco|factocho|repuno|repdos|reptres>\end{reporte}

%% Este es un comando auxiliar suministrado por factura-ejemplo-aux.tex para escribir el detalle del ejemplo en la última hoja.
\Ejemplo

\end{document}
%</!class&!def&!firma&!fondo&!membrete&!aux&!batchfile>
%<*firma>
%% (PSTricks code generated with LaTeXDraw 2.0.8)
%% (Compilar con XeLaTeX)
\documentclass{standalone}
\usepackage[usenames,dvipsnames]{pstricks}
\begin{document}
 \begin{pspicture}(0,-3.19388)(13.8397665,3.18)
  \pscustom[linewidth=0.1,linecolor=red]{
   \newpath
   \moveto(0.0,0.9762225)
   \lineto(1.0057193,0.95771605)
   \curveto(1.5085789,0.94846284)(2.4492223,1.1335286)(2.8870058,1.3278477)
   \curveto(3.3247895,1.5221668)(4.064289,2.0079648)(4.366005,2.2994435)
   \curveto(4.667721,2.5909224)(4.9457726,2.9518008)(4.9221087,3.0212007)
   \curveto(4.8984447,3.0906003)(4.762377,3.16)(4.6499724,3.16)
   \curveto(4.5375686,3.16)(4.1175327,2.9934406)(3.8099012,2.8268814)
   \curveto(3.5022693,2.6603222)(2.721358,1.6748462)(2.2480783,0.85592973)
   \curveto(1.7747989,0.037013244)(1.1358713,-0.91607606)(0.97022337,-1.050249)
   \curveto(0.8045755,-1.1844217)(0.5975156,-1.3185943)(0.5561036,-1.3185943)
   \curveto(0.5146916,-1.3185943)(0.43186772,-1.2121814)(0.39045575,-1.1057684)
   \curveto(0.34904376,-0.9993555)(0.37270775,-0.7495166)(0.4377837,-0.6060907)
   \curveto(0.50285965,-0.46266478)(0.6152636,-0.3608783)(0.6625916,-0.4025183)
   \curveto(0.7099195,-0.44415832)(0.82232344,-0.56907773)(0.8873994,-0.65235716)
   \curveto(0.9524753,-0.7356369)(1.354763,-0.86980957)(1.6919748,-0.9207028)
   \curveto(2.0291865,-0.97159606)(2.5202143,-1.0178626)(2.67403,-1.0132358)
   \curveto(2.827846,-1.008609)(3.0112422,-0.9207028)(3.0408218,-0.8374231)
   \curveto(3.0704017,-0.75414366)(3.0999818,-0.69862366)(3.0999818,-0.7263837)
   \curveto(3.0999818,-0.75414366)(3.1709735,-0.80966306)(3.2419655,-0.8374231)
   \curveto(3.3129578,-0.8651831)(3.4667737,-0.80041015)(3.5495975,-0.7078772)
   \curveto(3.6324213,-0.6153442)(3.7744055,-0.6199707)(3.8335655,-0.7171304)
   \curveto(3.8927252,-0.81429017)(4.141197,-0.9114496)(4.330509,-0.9114496)
   \curveto(4.5198207,-0.9114496)(4.7978725,-0.86055636)(4.8866124,-0.80966306)
   \curveto(4.9753523,-0.75877017)(5.0995884,-0.74489015)(5.135084,-0.7819034)
   \curveto(5.1705804,-0.8189166)(5.324396,-0.8281702)(5.442716,-0.80041015)
   \curveto(5.5610366,-0.7726501)(5.6971045,-0.7680234)(5.7148523,-0.7911566)
   \curveto(5.7326,-0.81428987)(5.8745837,-0.83279663)(5.9988194,-0.8281699)
   \curveto(6.1230555,-0.8235431)(6.2887034,-0.78653014)(6.330116,-0.75414366)
   \curveto(6.3715277,-0.7217572)(6.454352,-0.69862366)(6.495764,-0.7078772)
   \curveto(6.537175,-0.7171304)(6.6554956,-0.6060907)(6.7324038,-0.48579803)
   \curveto(6.8093114,-0.36550537)(7.0163713,-0.041640013)(7.1465235,0.16193238)
   \curveto(7.2766747,0.36550477)(7.4245753,0.12491944)(7.442323,-0.3192389)
   \curveto(7.4600706,-0.7633969)(7.5133147,-1.6748462)(7.548811,-2.1421375)
   \curveto(7.5843067,-2.6094286)(7.5251465,-3.10448)(7.4304905,-3.13224)
   \curveto(7.335835,-3.16)(7.1465235,-3.17388)(7.051867,-3.16)
   \curveto(6.9572105,-3.14612)(6.9276304,-2.748228)(6.992707,-2.3642163)
   \curveto(7.057783,-1.980205)(7.436407,-1.4296337)(7.7499547,-1.2630746)
   \curveto(8.063502,-1.0965155)(8.619606,-0.80966306)(8.862162,-0.6893704)
   \curveto(9.104718,-0.56907773)(9.394602,-0.5181845)(9.441929,-0.58758426)
   \curveto(9.489257,-0.656984)(9.589829,-0.67549074)(9.643073,-0.6245975)
   \curveto(9.696317,-0.57370424)(9.767309,-0.5089313)(9.785057,-0.49505126)
   \curveto(9.802805,-0.48117125)(9.897461,-0.49505126)(9.974369,-0.5228113)
   \curveto(10.051277,-0.55057126)(10.175512,-0.62922424)(10.22284,-0.6801172)
   \curveto(10.270168,-0.73101014)(10.388489,-0.76802367)(10.45948,-0.75414366)
   \curveto(10.530473,-0.7402637)(10.65471,-0.6616107)(10.7079525,-0.59683746)
   \curveto(10.761196,-0.5320642)(10.838105,-0.5043045)(10.861769,-0.54131776)
   \curveto(10.885432,-0.5783307)(10.932761,-0.6477307)(10.956425,-0.6801172)
   \curveto(10.980088,-0.7125037)(11.033333,-0.75877017)(11.062913,-0.7726501)
   \curveto(11.092493,-0.78653014)(11.157568,-0.7680234)(11.193064,-0.7356369)
   \curveto(11.2285595,-0.7032504)(11.305469,-0.6662372)(11.34688,-0.6616107)
   \curveto(11.388292,-0.656984)(11.512528,-0.48579773)(11.595351,-0.31923857)
   \curveto(11.678176,-0.15267944)(11.778748,-0.02313324)(11.796496,-0.060146485)
   \curveto(11.814244,-0.09715942)(11.843824,-0.6569836)(11.855657,-1.1797949)
   \curveto(11.867488,-1.7026062)(11.87932,-2.2901905)(11.87932,-2.3549633)
   \curveto(11.87932,-2.4197364)(11.938479,-2.0773644)(11.997639,-1.6702198)
   \curveto(12.056799,-1.263075)(12.186951,-0.85592985)(12.257943,-0.85592985)
   \curveto(12.328936,-0.85592985)(12.459087,-0.8374231)(12.518247,-0.8189166)
   \curveto(12.577407,-0.80041015)(12.719391,-0.78653014)(12.802215,-0.7911566)
   \curveto(12.885039,-0.7957831)(13.03294,-0.7726501)(13.098015,-0.74489015)
   \curveto(13.163091,-0.7171301)(13.257747,-0.5922107)(13.287327,-0.49505126)
   \curveto(13.316907,-0.39789185)(13.352403,-0.33774537)(13.358318,-0.3747583)
   \curveto(13.364233,-0.41177154)(13.3878975,-0.48117125)(13.405647,-0.51355773)
   \curveto(13.423395,-0.5459442)(13.48847,-0.57370424)(13.535798,-0.56907773)
   \curveto(13.583126,-0.564451)(13.66595,-0.5598242)(13.701447,-0.5598242)
   \curveto(13.736943,-0.5598242)(13.784271,-0.5551978)(13.819766,-0.5413177)}
  \pscustom[linewidth=0.1,linecolor=red]{
   \newpath
   \moveto(12.952644,0.18)
   \lineto(13.020389,0.13)
   \curveto(13.05426,0.105)(13.135551,0.085)(13.182974,0.09)
   \curveto(13.230394,0.095)(13.304913,0.13)(13.332009,0.16)
   \curveto(13.359106,0.19)(13.399754,0.285)(13.413303,0.35)
   \curveto(13.426852,0.415)(13.4404,0.49)(13.4404,0.52)}
 \end{pspicture}
\end{document}
%</firma>
%<*fondo|membrete>
\documentclass[12pt]{article}
\usepackage[utf8]{inputenc}
\usepackage{tikz}\usetikzlibrary{positioning}
\usepackage{calc}
\usepackage[letterpaper,hmargin={1cm,1cm},vmargin={1cm,1cm},footskip=0pt,headsep=0pt,headheight=0pt]{geometry}
\parindent0pt
\lineskip0pt
\pagestyle{empty}
\begin{document}
 \begin{minipage}[b][\textheight]{\textwidth}
  \baselineskip13pt
  \parskip0pt
  \lineskip0pt
  \hspace*{\fill}%
  \begin{tikzpicture}
   \node (Nombre) [draw=none, fill=gray, rounded corners=4pt]{%
    \parbox{0.57\textwidth}{%
     \vspace*{0.1cm}%
     \centering
     \color{white}\fontfamily{ppl}\LARGE\bfseries
     Lázaro de la Vega y San Román}};
   \node [node distance=0pt, below = of Nombre, outer sep=0pt, draw=gray, line width=2pt, inner sep=3.5pt, rounded corners=8pt]{%
    \parbox{0.66\textwidth}{%
     \vspace*{3pt}%
     \parskip2pt
     \centering
     \color{gray}\fontfamily{ppl}\footnotesize\bfseries
     Av. Garcilaso de la Vega, Qta. San Román, Santa Mónica, Caracas, Venezuela.
     \par
     Teléfono: +58 (123) 456.78.90 / email: lázaro@sanrom.án
     \vspace*{3pt}}};
  \end{tikzpicture}
  \hspace*{\fill}\par
  \hspace*{\fill}%
  \parbox{0.63\textwidth}{%
   \color{gray}\fontfamily{aer}\footnotesize\bfseries
   \hfill
   RIF: V-87.654.321-0}%
  \hspace*{\fill}\par
  \vfill\mbox{}%
%<fondo>  \fontsize{6}{8}\selectfont
%<fondo>  \hspace*{0.5cm}%
%<fondo>  \parskip2pt
%<fondo>  {\bfseries ESTA FORMA LIBRE VA SIN TACHADURA NI ENMIENDA \hfill SÓLO EL ORIGINAL DA DERECHO A CRÉDITO FISCAL}
%<fondo>  \hspace*{0.5cm}\par
%<fondo>  \fontdimen2\font=0.55em
%<fondo>  Imprenta: \hfill IMPRESOS DE LA VEGA. Av. Garcilaso de la Vega, Qta. San Román, Santa Mónica, Caracas, Venezuela. RIF: R-87.654.321-0. Tlf: (0123) 456.78.90\newline
%<fondo>  \fontdimen2\font=0.65em
%<fondo>  e-mail: \hfill imprenta@sanrom.án. No. Providencia SENIAT 01/01060 de fecha 12/11/2008 - Control desde No. 00-000001 hasta No. 00-010000 / Fecha 01/01/2018.\par
 \end{minipage}
\end{document}
%</fondo|membrete>
%<*aux>
%% Helper file; not to be compiled standalone, only to be included on examples.
%% Archivo auxiliar; no para ser compilado sino para ser incluido en los ejemplos.
%%
\usepackage{xspace,catchfilebetweentags,regexpatch}
\def\ejemplooptcolor{\color{blue}}
\def\ejemplomacrocolor{\color{purple}}
\makeatletter
\catcode`\%=11
\catcode`\|=14
\newcommand{\Ejemplo}{|
 \newpage
 \ClearShipoutPictureBG
 \newgeometry{vmargin={2cm,2cm},hmargin={2cm,2cm},footskip=0pt,headheight=0pt,headsep=0pt,footnotesep=0pt}|
 \thispagestyle{empty}|
 \catcode`\<=12
 \catcode`\>=12
 \CatchFileBetweenDelims{\DescEjemplo}{\jobname.tex}{<desc>}{</desc>}[\catcode37=11\catcode13=14\makeatletter]|
 \catcode`\%=11
 \makeatletter
 \regexpatchcmd*{\DescEjemplo}{\cC.\cO[*]*}{\c{begingroup}\c{ejemplomacrocolor}\c{ttfamily}\c{string}\0\c{endgroup}\c{xspace}}{}{}|
 \regexpatchcmd*{\DescEjemplo}{(\()(\cL[a-zG]*)(\))}{(\c{begingroup}\c{ejemplooptcolor}\c{ttfamily}\2\c{endgroup})}{}{}|
 \xpatchcmd*{\DescEjemplo}{%% }{\item}{}{}|
 \xpatchcmd*{\DescEjemplo}{}{}{}|
 \makeatother
 {\bfseries Descripci\'on de este ejemplo {\slshape \jobname.tex}:}
 \begin{itemize}
  \DescEjemplo
 \end{itemize}
 \catcode`\%=14
 \catcode`\<=13
 \catcode`\>=13
 }
\catcode`\%=14
\catcode`\|=12
\makeatother
%</aux>
%    \end{macrocode}
% \Finale
% \starttodo
%  Cargamos los valores predeterminados.

%% Comando para escribir el encabezado en la factura.
\newcommand\@EncabezadoFactura{%
 \ifdefvoid{\@Resumen}{\@EscribeDatos}{\@EscribeDatosResumen}%
 \vspace*{\@SeparaVertical}%
 \ifexpandecuadro\else\vfill\fi}

%% Comando para escribir el encabezado en el reporte.
\newcommand\@EncabezadoReporte{%
 \vspace*{-\topskip}%
 \if@FondoReporte
  \if@FondoReporteStar
  \else
   \vspace*{\@InicioReporte}%
  \fi
 \fi
 \@EscribeDatosReporte
 \@LetraReporte}

%% El ambiente 'factura', usando el paquete 'environ' para su definición, y el comando auxiliar \@FacturaAuxCmd.
%% Gracias a Heiko Oberdiek y a David Carlisle (Thanks to Heiko Oberdiek and David Carlisle).
%% http://tex.stackexchange.com/questions/108565/know-the-cell-width-with-tabular-or-include-a-paragraph-box-without-specifying
\NewEnviron{factura}{%
 \expandafter\@FacturaAuxCmd\expandafter{\BODY}%
 \ifnum\value{page}=1\else\CLASSERROR{La factura ocupa mas que una pagina}\fi}

\NewEnviron{reverso}{%
 \if@factura
   \expandafter\newgeometry\expandafter{\@GeometriaReverso}%
   \topskip1sp% ¡¡No puedo poner 0 porque se genera una página en blanco extra!!
  \begin{minipage}[t][\textheight-\pagetotal]{\linewidth}%
   \thispagestyle{empty}%
   \@LetraReverso\BODY
  \end{minipage}
  \clearpage
 \fi}

%% Comando auxiliar para agregar un carácter al final de cada línea en la descripción dentro del cuadro de la factura si ocupa ésta más de una línea.
%% Gracias a Gustavo Mezzetti y a David Carlisle (Thanks to Gustavo Mezzetti and David Carlisle).
%% http://tex.stackexchange.com/questions/296035/insert-a-special-character-at-each-end-of-line-in-a-paragraph
%% http://tex.stackexchange.com/questions/248944/how-to-put-before-lines
\newcommand{\@DescripcionAuxCmd}[2]{%
 \immediate\write\@auxout{\gdef\noexpand\@TamDescripcion{\the\linewidth}}% Para poder establecer la línea diagonal hay que saber el tamaño del campo de la descripción.
 \ifimpdesc\settowidth{\@TempDimen}{\@EscribeTipoDeImpuesto{#2}}\fi
 \savebox{\@CajaPrueba}{%
  \parbox[t]{\linewidth}{%
   \David@XStart
   \ifcentradesc\centering\arraybackslash\fi
   #1%
   \David@XEnd
   \ifdim\@TempDimen=0pt%
   \else
    \hfill\kern\@SeparaFinLineaDesc\@EscribeTipoDeImpuesto{#2}%
   \fi
   \par\global\count1=\prevgraf}}%
 \ifsinmarcas
  \usebox{\@CajaPrueba}%
 \else
  \ifnum\count1=1%
   \usebox{\@CajaPrueba}%
  \else
   \ifdim\@TempDimen>\wd\@CajaCaracterCR
   \else
    \setlength{\@TempDimen}{\wd\@CajaCaracterCR}%
   \fi
   \parbox[t]{\linewidth-\@TempDimen-\@SeparaFinLineaDesc}{%
    \David@XStart
    \ifcentradesc\centering\arraybackslash\fi
    #1%
    \David@XEnd
    \par
    \global\count1=\prevgraf% Esto lo necesitamos en otro lado para calcular el espacio extra en los ítems.
    \count0=\prevgraf
    \count2=\prevgraf
    \loop
     \ifnum\count0=\count2
      \ifimpdesc
       \vskip-\baselineskip
       \mbox{}\hfill
       \rlap{%
        \hspace*{\@TempDimen}%
        \hspace*{\@SeparaFinLineaDesc}%
        \llap{\@EscribeTipoDeImpuesto{#2}}}%
      \fi
     \else
      \mbox{}\hfill
      \rlap{%
       \hspace*{\@TempDimen}%
       \hspace*{\@SeparaFinLineaDesc}%
       \hspace*{-\wd\@CajaCaracterCR}%
       \usebox{\@CajaCaracterCR}}%
     \fi
     \vskip-2\baselineskip
     \advance\count0 -1
     \ifnum\count0 > 0
    \repeat
    \kern\count2\baselineskip\mbox{}%
    \strut}%
  \fi
 \fi}

%% Comandos auxiliares para colocar en \parbox cada celda.
\newcommand{\@NumItemAuxCmd}[1]{%
 \savebox{\@CajaPrueba}{#1}%
 \ifdim\wd\@CajaPrueba > \@TamNumItem \global\@TamNumItem=\wd\@CajaPrueba\fi
 \parbox[t]{\@TamNumItem}{#1}}

\newcommand{\@CantidadAuxCmd}[1]{%
 \savebox{\@CajaPrueba}{#1}%
 \ifdim\wd\@CajaPrueba > \@TamCantidad \global\@TamCantidad=\wd\@CajaPrueba\fi
 \parbox[t]{\@TamCantidad}{#1}}

\newcommand{\@PrecioUAuxCmd}[1]{%
 \savebox{\@CajaPrueba}{#1}%
 \ifdim\wd\@CajaPrueba > \@TamPrecioU \global\@TamPrecioU=\wd\@CajaPrueba\fi
 \parbox[t]{\@TamPrecioU}{#1}}

\newcommand{\@PrecioSAuxCmd}[1]{%
 \savebox{\@CajaPrueba}{#1}%
 \ifdim\wd\@CajaPrueba > \@TamPrecioS \global\@TamPrecioS=\wd\@CajaPrueba\fi
 \parbox[t]{\@TamPrecioS}{#1}}

\newcommand{\@DescuentoAuxCmd}[1]{%
 \savebox{\@CajaPrueba}{#1}%
 \ifdim\wd\@CajaPrueba > \@TamDescuento \global\@TamDescuento=\wd\@CajaPrueba\fi
 \parbox[t]{\@TamDescuento}{#1}}

\newcommand{\@IVAAuxCmd}[1]{%
 \savebox{\@CajaPrueba}{#1}%
 \ifdim\wd\@CajaPrueba > \@TamIVA \global\@TamIVA=\wd\@CajaPrueba\fi
 \parbox[t]{\@TamIVA}{#1}}

\newcommand{\@PrecioTAuxCmd}[1]{%
 \savebox{\@CajaPrueba}{#1}%
 \ifdim\wd\@CajaPrueba > \@TamPrecioT \global\@TamPrecioT=\wd\@CajaPrueba\fi
 \parbox[t]{\@TamPrecioT}{#1}}

%% Presentación de totales en modo de columnas.
\newcommand{\@EscribeTotalCol}[1]{%
 \expandafter\FPifeq\csname Desc#1\endcsname{0}%
  \ifsinivaexpresado\else\@EscribeTotalColBI{#1}\fi
 \else
  \@EscribeTotalColSubt{#1}%
  \@EscribeTotalColDesc{#1}%
  \ifsinivaexpresado\else\@EscribeTotalColBI{#1}\fi
 \fi}

%% Forma agupada para la presentación de totales en modo de columnas.
\newcommand{\@EscribeTotalColAgrupado}{%
 \if@Exento\if@DescE\@EscribeTotalColSubt{E}\fi\fi
 \if@BIR\if@DescR\@EscribeTotalColSubt{R}\fi\fi
 \if@DescG\@EscribeTotalColSubt{G}\fi
 \if@BIA\if@DescA\@EscribeTotalColSubt{A}\fi\fi
 \if@Exento\if@DescE\@EscribeTotalColDesc{E}\fi\fi
 \if@BIR\if@DescR\@EscribeTotalColDesc{R}\fi\fi
 \if@DescG\@EscribeTotalColDesc{G}\fi
 \if@BIA\if@DescA\@EscribeTotalColDesc{A}\fi\fi
 \ifsinivaexpresado
 \else
  \if@Exento\@EscribeTotalColBI{E}\fi
  \if@BIR\@EscribeTotalColBI{R}\fi
  \@EscribeTotalColBI{G}%
  \if@BIA\@EscribeTotalColBI{A}\fi
  \if@BIR\@EscribeTotalColAlicuota{R}\fi
  \@EscribeTotalColAlicuota{G}%
  \if@BIA\@EscribeTotalColAlicuota{A}\fi
  %% (quitado por no vigente)
  % \ifbool{Electro}{\@EscribeTotalColAlicuotaGDesc}{}%
 \fi}

\newcommand{\@EscribeTotalColBI}[1]{%
 \@LetraTitTotales \csname @@BI#1\endcsname
  \ifx#1E%
  \else
   \ifG\ \hfill(\csname @Alicuota#1porc\endcsname\%)\else\mbox{}\fi
  \fi
 & \@LetraNumTotales\@SignoMonetarioAux\numprint{\csname BI#1\endcsname} \\}

\newcommand{\@EscribeTotalColAlicuota}[1]{%
 {\@LetraTitTotales \csname @@Alicuota#1\endcsname\ \hfill (\csname @Alicuota#1porc\endcsname\%)}
 & \@LetraNumTotales\@SignoMonetarioAux\numprint{\csname Alicuota#1\endcsname} \\}

%% (quitado por no vigente)
 % \newcommand{\@EscribeTotalColAlicuotaGDesc}{%
 % {\@LetraTitTotales \@@AlicuotaGDesc\ \hfill (\@AlicuotaGDescPorc\%)}
 % & \@LetraNumTotales\@SignoMonetarioAux\numprint{\AlicuotaGDesc} \\}

\newcommand{\@EscribeTotalColSubt}[1]{%
 {\@LetraTitTotales \csname @@Subtotal#1\endcsname}
 & \@LetraNumTotales\@SignoMonetarioAux\numprint{\csname Subt#1\endcsname} \\}

\newcommand{\@EscribeTotalColDesc}[1]{%
 {\@LetraTitTotales \csname @@Descuento#1\endcsname
 \csname if@Desc#1num\endcsname \else \ \hfill (\csname @Descuento#1porc\endcsname\%)\fi}
 & \@LetraNumTotales\@SignoMonetarioAux\numprint{\csname Desc#1\endcsname} \\}

%% Presentación de los totales en modo filas.
\newcommand{\@EscribeTotalFilasBI}[1]{%
 \@EscribeTotalFilasBIizq{#1}: & \@EscribeTotalFilasBIder{#1}}

\newcommand{\@EscribeTotalFilasAlicuota}[1]{%
 \@EscribeTotalFilasAlicuotaizq{#1}: & \@EscribeTotalFilasAlicuotader{#1}}

%% (no implementado)
 % \newcommand{\@EscribeTotalFilasDesc}[1]{%
 %  \@EscribeTotalFilasDescizq{#1}: & \@EscribeTotalFilasDescder{#1}}

\newcommand{\@EscribeTotalFilasSoloBI}[1]{%
 {\@LetraTitTotales\@EscribeTotalFilasBIizq{#1}:} {\@LetraNumTotales\@EscribeTotalFilasBIder{#1}}}

%% (no implementado)
 % \newcommand{\@EscribeTotalFilasSoloDesc}[1]{%
 % {\@LetraTitTotales\@EscribeTotalFilasDescizq{#1}:} {\@LetraNumTotales\@EscribeTotalFilasDescder{#1}}}

\newcommand{\@EscribeTotalFilasSoloAlicuota}[1]{%
 {\@LetraTitTotales\@EscribeTotalFilasAlicuotaizq{#1}:} {\@LetraNumTotales\@EscribeTotalFilasAlicuotader{#1}}}

\newcommand{\@EscribeTotalFilasBIizq}[1]{%
 \@LetraTitTotales\csname @@BI#1\endcsname
 \ifx#1G%
  \ifG \ (\@AlicuotaGporc\%)\fi%  Si hay otra B.I. que no sea general, escribimos el porcentaje explícito en la general.
 \else
  \ifx#1E%
  \else
  \ (\csname @Alicuota#1porc\endcsname\%)%
  \fi
 \fi}

\newcommand{\@EscribeTotalFilasBIder}[1]{%
 \@LetraNumTotales\@SignoMonetarioAux\numprint{\csname BI#1\endcsname}}

\newcommand{\@EscribeTotalFilasAlicuotaizq}[1]{%
 \@LetraTitTotales\csname @@Alicuota#1\endcsname \ (\csname @Alicuota#1porc\endcsname\%)}

\newcommand{\@EscribeTotalFilasAlicuotader}[1]{%
 \@LetraNumTotales\@SignoMonetarioAux\numprint{\csname Alicuota#1\endcsname}}

%% (no implementado)
 % \newcommand{\@EscribeTotalFilasDescizq}[1]{%
 %  \@LetraTitTotales\csname @@Descuento#1\endcsname \ (\csname @Descuento#1porc\endcsname\%)}

 % \newcommand{\@EscribeTotalFilasDescder}[1]{%
 %  \@LetraNumTotales\@SignoMonetarioAux\numprint{\csname Desc#1\endcsname}}

\newcommand{\@EscribeTotalFilasAuxTotal}{\@LetraTitTotal\@@Total: & \@LetraNumTotal\@SignoMonetarioAux\numprint{\Total}}
 
\newcommand{\@EscribeTotalFilas}[2][]{%
 \begin{tabular}[b]{@{}l@{\extracolsep{1ex}}r@{}}% ¡Esta tabla tiene que abrir hacia abajo para que la separación entre las filas se pueda colocar con \@SeparaFilas!
  \csname @EscribeTotalFilasBI\endcsname{#2}%
  \ifx#2E%
   \ifx#1T%
    \\[\@SeparaFilas]%
    \@EscribeTotalFilasAuxTotal
   \fi
  \else
   \\[\@SeparaFilas]%
   \csname @EscribeTotalFilasAlicuota\endcsname{#2}%
  \fi
 \end{tabular}}

%% Comando auxiliar para escribir el signo monetario cuando aplica
\newcommand{\@SignoMonetarioAux}{%
 \ifsinmonedatotales\else\@Moneda\hspace*{\@EspacioSignoMonetario}\fi}

\newcommand{\@GeneraCajasTotalesFilas}{%
 \if@Exento
  \newbox{\@CajaBIE}\newbox{\@CajaBIET}%
  \savebox{\@CajaBIE}{\@EscribeTotalFilasSoloBI{E}}%
  \savebox{\@CajaBIET}{\@EscribeTotalFilas[T]{E}}%
 \fi
 \if@BIR
  \addtocounter{@IVANoGeneral}{1}%
  \newbox{\@CajaBIR}\newbox{\@CajaAliR}\newbox{\@CajaBIAliR}%
  \savebox{\@CajaBIR}{\@EscribeTotalFilasSoloBI{R}}%
  \savebox{\@CajaAliR}{\@EscribeTotalFilasSoloAlicuota{R}}%
  \savebox{\@CajaBIAliR}{\@EscribeTotalFilas{R}}%
 \fi
 \newbox{\@CajaBIG}\newbox{\@CajaAliG}\newbox{\@CajaBIAliG}%
 \savebox{\@CajaBIG}{\@EscribeTotalFilasSoloBI{G}}%
 \savebox{\@CajaAliG}{\@EscribeTotalFilasSoloAlicuota{G}}%
 \savebox{\@CajaBIAliG}{\@EscribeTotalFilas{G}}%
 \if@BIA
  \addtocounter{@IVANoGeneral}{1}%
  \newbox{\@CajaBIA}\newbox{\@CajaAliA}\newbox{\@CajaBIAliA}%
  \savebox{\@CajaBIA}{\@EscribeTotalFilasSoloBI{A}}%
  \savebox{\@CajaAliA}{\@EscribeTotalFilasSoloAlicuota{A}}%
  \savebox{\@CajaBIAliA}{\@EscribeTotalFilas{A}}%
 \fi}

\newcommand{\@EscribeTotalesFilasGET}{%
 \hspace*{\@SepTotales}\usebox{\@CajaBIG}%
 \hspace*{\@SepTotales}\usebox{\@CajaAliG}%
 \if@Exento\hspace*{\@SepTotales}\usebox{\@CajaBIE}\fi
 \hspace*{\@SepTotales}\usebox{\@CajaTotal}%
 \ifcentrado\hspace*{\@SepTotales}\fi}

\newcommand{\@EscribeTotalesFilasGxET}{%
 \if@BIR\hspace*{\@SepTotales}\usebox{\@CajaBIAliR}\fi
 \hspace*{\@SepTotales}\usebox{\@CajaBIAliG}%
 \if@BIA\hspace*{\@SepTotales}\usebox{\@CajaBIAliA}\fi
 \hspace*{\@SepTotales}\if@Exento\usebox{\@CajaBIET}\else\usebox{\@CajaTotal}\fi
 \ifcentrado\hspace*{\@SepTotales}\fi}

\newcommand{\@EscribeTotalesFilasRGA}{%
 \hspace*{\@SepTotales}\usebox{\@CajaBIAliR}%
 \hspace*{\@SepTotales}\usebox{\@CajaBIAliG}%
 \hspace*{\@SepTotales}\usebox{\@CajaBIAliA}%
 \hspace*{\@SepTotales}}

\newcommand{\@EscribeTotalesFilasRGAET}{%
 \@EscribeTotalesFilasRGA
 \if@Exento\usebox{\@CajaBIET}\else\usebox{\@CajaTotal}\fi
 \ifcentrado\hspace*{\@SepTotales}\fi}

%% Escribe la letra que designa al tipo de impuesto que aplica al ítem.
\NewDocumentCommand{\@EscribeTipoDeImpuesto}{m}{%
 % Hay que abrir un grupo de ejecución porque el cambio en la tipografía altera la celda.
 {\ifG
   \@LetraTipoIVA
   \ifx#1G\@DenomAlicuotaG\fi
   \ifx#1R\@DenomAlicuotaR\fi
   \ifx#1A\@DenomAlicuotaA\fi
   \ifx#1E\@DenomAlicuotaE\fi
  \fi}}
 
%% Comando para poner en cero los contadores de los totales porque el ambiente de la factura se evalúa tres veces y los totales se sumarían todos tres veces.
\newcommand{\@ZERO}{%
 \FPset\BIG{0}\global\let\BIG\BIG
 \FPset\BIR{0}\global\let\BIR\BIR
 \FPset\BIA{0}\global\let\BIA\BIA
 \FPset\BIE{0}\global\let\BIE\BIE
 \FPset\SubtE{0}\global\let\SubtE\SubtE
 \FPset\SubtG{0}\global\let\SubtG\SubtG
 \FPset\SubtR{0}\global\let\SubtR\SubtR
 \FPset\SubtA{0}\global\let\SubtA\SubtA
 \FPset\SubtX{0}\global\let\SubtX\SubtX
 \FPset\DescE{0}\global\let\DescE\DescE
 \FPset\DescG{0}\global\let\DescG\DescG
 \FPset\DescR{0}\global\let\DescR\DescR
 \FPset\DescA{0}\global\let\DescA\DescA
 \FPset\DescX{0}\global\let\DescX\DescX
 \FPset\AlicuotaG{0}\global\let\AlicuotaG\AlicuotaG
 %% (quitado por no vigente)
  % \FPset\AlicuotaGDesc{0}\global\let\AlicuotaGDesc\AlicuotaGDesc
 \FPset\AlicuotaR{0}\global\let\AlicuotaR\AlicuotaR
 \FPset\AlicuotaA{0}\global\let\AlicuotaA\AlicuotaA}

%% Comandos para calcular todos los totales parciales y el total general.
\newcommand{\@CalculaTotalesI}{%
 \FPifeq\SubtG{0}\else\@CalculaTotalesII{G}\fi
 \FPifeq\SubtR{0}\else\@CalculaTotalesII{R}\fi
 \FPifeq\SubtA{0}\else\@CalculaTotalesII{A}\fi
 \FPifeq\SubtE{0}\else\@CalculaTotalesII{E}\fi
 %% (quitado por no vigente)
 % \ifbool{Electro}{%
 %  \FPiflt\BIG\@AlicuotaGDescLimite
 %   \let\@AlicuotaGDesc\@AlicuotaGDescBajo
 %  \else
 %   \let\@AlicuotaGDesc\@AlicuotaGDescAlto
 %  \fi
 %  \FPupn\@AlicuotaGDescPorc{\@AlicuotaGDesc{} 100 * \@DigitosPorc{} round clip}%
 %  \FPupn\AlicuotaGDesc{\BIG{} \@AlicuotaGDesc{} * \@DigitosMoneda{} round}%
 %  \global\let\AlicuotaGDesc\AlicuotaGDesc
 %  \global\let\@AlicuotaGDescPorc\@AlicuotaGDescPorc
 %  \global\let\@AlicuotaGDesc\@AlicuotaGDesc}{}%
 \FPupn\Total{\BIE{} \BIR{} + \BIG{} + \BIA{} + \AlicuotaR{} + \AlicuotaG{} + \AlicuotaA{} + \@DigitosMoneda{} round}%
 %% (quitado por no vigente) arriba se sumaba también \AlicuotaGDesc{}.
 \global\let\Total\Total}

\newcommand{\@CalculaTotalesII}[1]{%
 % Sumamos al descuento.
 \expandafter\FPupn\csname Desc#1\endcsname{\csname @Descuento#1factor\endcsname{} \csname Subt#1\endcsname{} * \csname @Descuento#1num\endcsname{} + \@DigitosMoneda{} round}%
 \global\expandafter\let\csname Desc#1\expandafter\endcsname\csname Desc#1\endcsname
 % Sumamos a la base imponible.
 \expandafter\FPupn\csname BI#1\endcsname{\csname Desc#1\endcsname{} \csname Subt#1\endcsname{} - \@DigitosMoneda{} round}%
 \global\expandafter\let\csname BI#1\expandafter\endcsname\csname BI#1\endcsname
 \ifx#1E%
 \else{%
  % Sumamos a la alícuota.
  \expandafter\FPupn\csname Alicuota#1\endcsname{\csname BI#1\endcsname{} \csname @Impuesto#1\endcsname{} * \@DigitosMoneda{} round}%
  \global\expandafter\let\csname Alicuota#1\expandafter\endcsname\csname Alicuota#1\endcsname}%
 \fi}

%% Comandos auxiliares para poder evaluar la definición de las columnas en la tabla con un comando que se expande completamente.
\def\@FacturaAuxColumnasEval{
 % Primera línea, la más externa.
 \ifbool{SinLineasVertExt}{}{|}
 % Columna para el número del ítem, si aplica.
 \ifbool{NumItem}{c}{r@{\extracolsep{0pt}}}
 % Segunda línea, que sale solamente cuando se usa la opción 'numitem'
 \ifbool{NumItem}{\ifbool{SinLineasVertInt}{}{|}}{}
 % Columna 'X' con DCU, columna fantasma con 'sincantidad' y columna 'c' con CDU.
 \ifbool{PrimeroDescripcion}{X}{
  \ifbool{SinCantidad}{l@{\extracolsep{0pt}}}{c}}
 % Tercera línea.
 \ifbool{SinCantidad}{}{\ifbool{SinLineasVertInt}{}{|}}
 % Columna 'c' con DCU, columna 'r' con DUC, columna 'X' sin espacio para 'sincantidad' y columna 'X' con CDU.
 \ifbool{DCU}{c}{
  \ifbool{DUC}{r}{
   \ifbool{SinCantidad}{X@{\extracolsep{0pt}}}{X}}}
 % Cuarta línea, que sale solamente cuando no se usa 'sincantidad'. 
 \ifbool{SinCantidad}{}{\ifbool{SinLineasVertInt}{}{|}}
 % Columna 'c' para DUC y columna 'r' para cualquier otro caso (DCU, CDU y 'sincantidad').
 \ifbool{DUC}{c}{r}
 % Quinta línea. 
 \ifbool{SinLineasVertInt}{}{|}
 % Columna 'r' para subtotal antes del descuento, o columna fantasma si no se especifica 'subtotal').
 \ifbool{Subtotal}{r}{l@{\extracolsep{0pt}}}
 % Sexta línea, que sale solamente cuando hay columna de subtotales.
 \ifbool{Subtotal}{\ifbool{SinLineasVertInt}{}{|}}{}
 % Columna 'c' para 'descuentos' y columna 'r' sin espacio si no hay descuentos individuales.
 \ifbool{Descuentos}{c}{r@{\extracolsep{0pt}}}
 % Séptima línea, que sale solamente cuando hay descuentos.
 \ifbool{Descuentos}{\ifbool{SinLineasVertInt}{}{|}}{}
 % Columna 'c' para 'iva' y columna 'r' sin espacio si no hay columna de IVA.
 \ifbool{IVA}{c}{r@{\extracolsep{0pt}}}
 % Octava línea, que sale solamente cuando hay columna de IVA.
 \ifbool{IVA}{\ifbool{SinLineasVertInt}{}{|}}{}
 % Columna 'r' siempre.
 r
 % Novena y última línea.
 \ifbool{SinLineasVertExt}{}{|}}

%% Comando para hacer los cálculos sobre cada ítem y establecer las macros con el contenido de cada ítem introducido.
%% Aquí tenemos que hacer coincidir los parámetros de \@CalculaItem con los de \@ITEM para poder hacer
%% \let\@CalculaItem\@ITEM cuando no hay factura sino solamente reporte.
\NewDocumentCommand{\@CalculaItem}{m m m o m m +o}{%
 \gdef\@PrecioUtemp##1{\FPupn\PrecioUtemp{##1 \PrecioUtempx{} * clip}}%
 \addtocounter{NumItem}{1}%
 % Calculamos.
 \@PrecioUtemp{#5}%
 \FPupn\PrecioU{\PrecioUtemp{} \@DigitosMoneda{} round}%
 \FPifeq\PrecioU{\PrecioUtemp}\else\CLASSWARNING{Se esta' introduciendo un precio con ma's de dos decimales: (#3 - \PrecioUtemp. Redondeado a \PrecioU)}{}\fi% Advertimos cuando el precio unitario tiene más de dos decimales.
 \ifbool{Descuentos}
  {\ifx\\#6\\\FPset\Dcto{0}\else\FPupn\Dcto{#6 \@DigitosPorc{} round clip}\fi}% Fijamos en 0 el descuento si no se especifica en \Item***.
  {\FPset\Dcto{#6}%
   \FPifeq\Dcto{0}\else\CLASSWARNING{Se pidio' un descuento individual sin dar la opcio'n de la clase 'descuentos'. No se tomara' en cuenta}\fi
   \FPset\Dcto{0}}% También fijamos en 0 el descuento por si se especifica en \Item*** pero no se pide la opción 'descuentos'.
 \FPupn\PrecioS{#2 \PrecioU{} * \@DigitosMoneda{} round}%
 \FPupn\PrecioT{100 \Dcto{} 100 - / \PrecioS{} * \@DigitosMoneda{} round}%
 \global\let\PrecioU\PrecioU
 \global\let\PrecioS\PrecioS
 \global\let\Dcto\Dcto
 \global\let\PrecioT\PrecioT
 % Suma el elemento actual al subtotal acumulado.
 \expandafter\FPupn\csname Subt#1\endcsname{\PrecioT{} \csname Subt#1\endcsname{} + \@DigitosMoneda{} round}%
 \global\expandafter\let\csname Subt#1\expandafter\endcsname\csname Subt#1\endcsname
 % Guardamos en macros cada elemento agregado a la factura; por ejemplo: \@Item@i@Cantidad contiene la cantidad que se introdujo para el primer ítem; así mismo con ii@, @iii@, y así mismo también con @Descripcion, @PrecioU, @Descuento y @PrecioT.
 \expandafter\gdef\csname @Item@\romannumeral\value{NumItem}@Cantidad\endcsname{\numprint{#2}}%
 \expandafter\gdef\csname @Item@\romannumeral\value{NumItem}@Descripcion\endcsname{#3}%
 \expandafter\global\expandafter\edef\csname @Item@\romannumeral\value{NumItem}@PrecioU\endcsname{\noexpand\numprint{\PrecioU}}%
 \expandafter\global\expandafter\edef\csname @Item@\romannumeral\value{NumItem}@PrecioS\endcsname{\noexpand\numprint{\PrecioS}}%
 \expandafter\global\expandafter\edef\csname @Item@\romannumeral\value{NumItem}@Descuento\endcsname{\noexpand\numprint{\Dcto}}%
 \expandafter\gdef\csname @Item@\romannumeral\value{NumItem}@DescripcionLarga\endcsname{#7}%
 \expandafter\global\expandafter\expandafter\edef\csname @Item@\romannumeral\value{NumItem}@IVA\endcsname{%
  \noexpand\numprint\expandafter{\csname @Alicuota#1porc\endcsname}}%
 \expandafter\global\expandafter\edef\csname @Item@\romannumeral\value{NumItem}@PrecioT\endcsname{\noexpand\numprint{\PrecioT}}}

%% Macro interna para saber si una etiqueta existe o no.
\gdef\@pruebaref#1{%
 \expandafter\ifx\csname @Etiqueta@#1\endcsname\relax
  \CLASSWARNING{La etiqueta `#1' no esta' definida}\ignorespaces
  \@PruebaReffalse
 \else
  \@PruebaReftrue
 \fi}

%% Comandos para las referencias: i) referencia al número de ítem, ii) a la cantidad, iii) a la descripción, iv) al precio unitario, v) al descuento, vi) al precio total.
\gdef\itemref#1{%
 \ifbool{NumItem}{}{\CLASSWARNING{Es recomendable usar la opcio'n 'numitem' para enumerar los i'tems}}%
 \@pruebaref{#1}\if@PruebaRef \csname @Etiqueta@#1\endcsname \fi}
\gdef\cantref#1{\@pruebaref{#1}\if@PruebaRef \csname @Item@\expandafter\romannumeral\csname @Etiqueta@#1\endcsname @Cantidad\endcsname \fi}
\gdef\descref#1{\@pruebaref{#1}\if@PruebaRef \csname @Item@\expandafter\romannumeral\csname @Etiqueta@#1\endcsname @Descripcion\endcsname \fi}
\gdef\puref#1{\@pruebaref{#1}\if@PruebaRef \csname @Item@\expandafter\romannumeral\csname @Etiqueta@#1\endcsname @PrecioU\endcsname \fi}
\gdef\subtref#1{\@pruebaref{#1}\if@PruebaRef \csname @Item@\expandafter\romannumeral\csname @Etiqueta@#1\endcsname @PrecioS\endcsname \fi}
\gdef\dctoref#1{\@pruebaref{#1}\if@PruebaRef \csname @Item@\expandafter\romannumeral\csname @Etiqueta@#1\endcsname @Descuento\endcsname \fi}
\gdef\ldescref#1{\@pruebaref{#1}\if@PruebaRef \csname @Item@\expandafter\romannumeral\csname @Etiqueta@#1\endcsname @DescripcionLarga\endcsname \fi}
\gdef\ivaref#1{\@pruebaref{#1}\if@PruebaRef \csname @Item@\expandafter\romannumeral\csname @Etiqueta@#1\endcsname @IVA\endcsname \fi}
\gdef\ptref#1{\@pruebaref{#1}\if@PruebaRef \csname @Item@\expandafter\romannumeral\csname @Etiqueta@#1\endcsname @PrecioT\endcsname \fi}

%% Definimos un comando para la etiqueta, que se define dentro del entorno de la factura para que tenga alcance local; cuando se acaba dicho entorno, el comando \label vuelve a ser lo que está definido en latex.ltx.
\newcommand{\label@factura}[1]{%
 \expandafter\ifx\csname @Etiqueta@#1\endcsname\relax
  \expandafter\global\expandafter\edef\csname @Etiqueta@#1\endcsname{\arabic{NumItem}}\ignorespaces
  \CLASSINFO{Etiqueta `#1' definida para el i'tem \arabic{NumItem}}%
 \else
  \edef\@label@test@uno{\csname @Etiqueta@#1\endcsname}%
  \edef\@label@test@dos{\theNumItem}%
  \ifx\@label@test@uno\@label@test@dos
  \else
   \CLASSWARNING{La etiqueta `#1' se definio' ma's de una vez}\ignorespaces
  \fi
 \fi}

%% Todo esto tiene que estar dentro de \AtEndPreamble que es cuando \if@factura, \if@reverso e \if@reporte fueron ya aplicadas, y para que se evalúen adecuadamente en su momento las definiciones hechas con \edef.
\AtEndPreamble{
 % Comando auxiliar para abrir la tabla del cuadro de la factura y expandir completamente las columnas.
 \edef\@FacturaAuxInicioCmd{\noexpand\begin{tabularx}{\textwidth}{\@FacturaAuxColumnasEval}}

 % Comando auxiliar para abrir el '\multicolumn' de la nota interna y evalúe la presencia o no de las barras verticales.
 \edef\@ColumnasMultiColumnNotaIntAuxCmd{%
  \noexpand\multicolumn{8}{\ifbool{SinLineasVertExt}{}{|}l\ifbool{SinLineasVertExt}{}{|}}{%
   \noexpand\parbox[t]{\linewidth-2\tabcolsep-2\arrayrulewidth}{\noexpand\@LetraNotaInterna\noexpand\@NotaInterna\noexpand\strut}}%
   \noexpand\\}

 % Comando auxiliar para abrir el '\multicolumn' para los totales en el modo de columnas y expandir completamente las columnas cuando hay o no barras verticales.
 \edef\@ColumnasMultiColumnTotalesAuxCmd{%
  \noexpand\multicolumn{8}{\ifbool{SinLineasVertExt}{}{|} r}\begingroup
    \noexpand\if@ExpandeCuadro
     \noexpand\iflinea\noexpand\zsaveposx{LineaNodoBCoordX}\noexpand\zsaveposy{LineaNodoBCoordY}\noexpand\fi\noexpand\fi% Fijamos la posición (x,y) final para la línea ya que estaremos justo en el punto donde escribimos el total, y hasta allí llega la línea en el modo de columnas.
     \noexpand\begin{tabular}[b]{@{\noexpand\extracolsep{2\noexpand\tabcolsep}}lr \ifbool{SinLineasVertExt}{}{|}}}

 % Comando auxiliar para la definición del ambiente de la factura; al final de este comando está el texto que va en el enddef del ambiente de la factura.
 \newcommand{\@FacturaAuxCmd}[1]{%
  \ifexpandecuadro
   \setlength{\@EspacioVacioInicial}{\zposy{EspacioVacioInicial}sp}%
   \setlength{\@EspacioVacioFinal}{\zposy{EspacioVacioFinal}sp}%
   \global\@EspacioVacioInicial=\@EspacioVacioInicial% Hay que globalizarlas
   \global\@EspacioVacioFinal=\@EspacioVacioFinal
   \savebox{\@CajaEspacioVacio}{\parbox[b][\@EspacioVacioInicial-\@EspacioVacioFinal]{\@TamDescripcion}{%
    \iflinea
     \if@LineaHastaFinal
      \zsaveposy{LineaNodoBCoordY}%
     \else
      \if@LineaDescripcion
       \zsaveposx{LineaNodoACoordX}\mbox{}\hfill \zsaveposx{LineaNodoBCoordX}\zsaveposy{LineaNodoBCoordY}%
      \fi
     \fi
    \fi}}%
   \ifdim \ht\@CajaEspacioVacio=0pt \global\lineafalse \else \@ExpandeCuadrotrue \fi
  \fi

  \ifx\undefined\Total\relax\else\CLASSERROR{Parece ya haberse usado un entorno de factura}\fi
  %
  % Comandos \David@SavePos, \David@XStart, \David@XEnd, \David@XPad, \David@XLeft y \David@XRight para poder manejar el salto de línea en las columnas adyacentes a la columna 'X' del tabularx (la descripción), cuando esta ocupa más que una línea.
  % Gracias a David Carlisle (Thanks to David Carlisle).
  % http://tex.stackexchange.com/questions/111203/tabularx-bottom-alignment-on-r-or-l-columns-when-x-column-wraps
  \def\David@SavePos##1{\leavevmode\pdfsavepos\write\@auxout{\gdef\noexpand##1{\the\pdflastypos sp}}}%
  \def\David@XStart{\expandafter\David@SavePos\csname save@start@\romannumeral\value{NumItem}\endcsname}%
  \def\David@XEnd{\expandafter\David@SavePos\csname save@end@\romannumeral\value{NumItem}\endcsname}%
  \def\David@XPad##1##2##3{%
   \vtop{%
    \baselineskip\normalbaselineskip
    ##2%
    \expandafter
    \ifx\csname save@start@##1\endcsname
     \relax
    \else
     \dimen@\csname save@start@##1\endcsname\relax
     \loop
      \ifdim\dimen@>\csname save@end@##1\endcsname
      \advance\dimen@-\baselineskip
      \settowidth{\@TempDimen}{##3}%
       \ifdim\@TempDimen=0pt\settowidth{\@TempDimen}{##2}\fi
      \hbox{\parbox{\@TempDimen}{\@CaracterCeldasVacias}}%
     \repeat
    \fi
    ##3}}%
  \def\David@XLeft##1{\David@XPad{\romannumeral\value{NumItem}}{\hbox{##1}}{}}%
  \def\David@XRight##1{\David@XPad{\romannumeral\value{NumItem}}{}{\hbox{##1}}}%
  %
  % Caja auxiliar para el caracter de retorno de carro y poder saber si tamaño en el campo de la descripción.
  \savebox{\@CajaCaracterCR}{\@LetraItems\@CaracterCR}
  %
  % Comando para fijar la separación extra final de cada item, lo cual viene dado por la cantidad de líneas de la descripción,
  % que está alojado en \count1, definido en \@DescripcionAuxCmd.
  \def\@CalculaSeparaItemsExtraFinal{%
   \setlength{\@SeparaItemsExtraFinal}{\numexpr\the\count1\normalbaselineskip-\normalbaselineskip+\@SeparaItemsExtra}%
   \global\@SeparaItemsExtraFinal=\@SeparaItemsExtraFinal}
  %
  % Comando interno general para agregar cualquier ítem.
  \NewDocumentCommand{\@ITEM}{m m m o m m +o}{%
   % 1. [R|G|A|E|X]; 2. cantidad; 3. descripción; 4. descripción para el csv (entre corchetes); 5. p/u; 6. dcto; 7. la descripción para el reporte.
   \@CalculaItem{##1}{##2}{##3}{##5}{##6}[##7]%
   % Si no es el primer ítem, entonces hace falta \\ para terminar el ítem anterior.
   \ifnum\value{NumItem}=1% Aquí, \@CalculaItem aumenta en 1 el contador NumItem, que si vale 1 significa que aún no hemos procesado la línea 1, es decir, no hay que hacer el salto de línea. La última línea se termina en otro lado.
   \else
    \@CalculaSeparaItemsExtraFinal
    \\[\@SeparaItemsExtraFinal]
    \ifconlineasha\hline\fi% Líneas para separar cada ítem, si hiciera falta.
   \fi
   % Primer -opcional- campo: El número del ítem.
   \ifbool{NumItem}{\David@XLeft{\@NumItemAuxCmd{\mbox{}\hfill \@LetraNumeros \arabic{NumItem}\hfill\mbox{}}}}{}%
   &
   % Segundo campo: Acá puede ir la cantidad (cdu) o la descripción (dcu y duc).
   \ifbool{SinCantidad}{}{%
    \ifbool{PrimeroDescripcion}
    {\@DescripcionAuxCmd{##3}{##1}}% dcu y duc.
    {\David@XLeft{\@CantidadAuxCmd{\mbox{}\hfill \@LetraNumeros \numprint{##2}\hfill\mbox{}}}}}% cdu.
   &
   % Tercer campo: Acá puede ir la descripción (cdu), la cantidad (dcu), o el precio unitario (duc).
   \ifbool{PrimeroDescripcion}{%
    \ifdcu
     \David@XRight{\@CantidadAuxCmd{\mbox{}\hfill \@LetraNumeros \numprint{##2}\hfill\mbox{}}}% dcu.
    \else
     \David@XRight{\@PrecioUAuxCmd{%
      \ifx##1X \hfill \@CaracterItemXVacio \hfill\mbox{}%
      \else
       \ifimpdesc\else\@EscribeTipoDeImpuesto{##1}\fi \hfill \@LetraNumeros \numprint{\PrecioU}
      \fi}}% duc.
    \fi}
   {\@DescripcionAuxCmd{##3}{##1}}% cdu.
   &
   % Cuarto campo: Acá puede ir el precio unitario (cdu y dcu) o la cantidad (duc).
   \ifbool{SinCantidad}{}{%
    \ifduc
     \David@XRight{\@CantidadAuxCmd{\mbox{}\hfill \@LetraNumeros \numprint{##2}\hfill\mbox{}}}% duc.
    \else
     \David@XRight{\@PrecioUAuxCmd{%
      \ifx##1X \hfill \@CaracterItemXVacio \hfill\mbox{}%
      \else
       \ifimpdesc\else\@EscribeTipoDeImpuesto{##1}\fi \hfill \@LetraNumeros \numprint{\PrecioU}
      \fi}}% dcu y cdu.
    \fi}%
   & % Quinto -opcional- campo: Subtotal (para luego descuentos).
   \ifbool{Subtotal}{%
    \David@XRight{\@PrecioSAuxCmd{%
     \ifx##1X \hfill \@CaracterItemXVacio \hfill\mbox{}%
     \else
      \ifimpdesc\else\@EscribeTipoDeImpuesto{##1}\fi \hfill \@LetraNumeros \numprint{\PrecioS}
     \fi}}}{}
   &
   % Sexto -opcional- campo: Descuento.
   \ifbool{Descuentos}{%
    \David@XRight{\@DescuentoAuxCmd{%
     \ifx##1X \hfill \@CaracterItemXVacio \hfill\mbox{}%
     \else
      \mbox{}\hfill \@LetraNumeros \numprint{\Dcto}\%\hfill\mbox{}%
     \fi}}}{}% Tenemos que usar un condicional booleano porque si no da problemas con el condicional \FPifeq.
   &
   % Séptimo -opcional- campo: IVA.
   \ifbool{IVA}{%
    \David@XRight{\@IVAAuxCmd{%
     \ifx##1X \hfill \@CaracterItemXVacio \hfill\mbox{}%
     \else
      \mbox{}\hfill \@LetraNumeros \numprint{\csname @Alicuota##1porc\endcsname}\%\hfill\mbox{}%
     \fi}}}{}% Tenemos que usar un condicional booleano porque si no da problemas con el condicional \FPifeq.
   &
   % Octavo campo: El total.
   \David@XRight{\@PrecioTAuxCmd{%
    \ifx##1X \hfill \@CaracterItemXVacio \hfill\mbox{}%
    \else
     \ifimpdesc\else\@EscribeTipoDeImpuesto{##1}\fi \hfill \@LetraNumeros \numprint{\PrecioT}
    \fi}}%
   % Escribimos al csv.
   \ifcsv
    \IfValueTF{##4}{\@EscribeItemsEnCSV{##1}{##2}{##4}}{\@EscribeItemsEnCSV{##1}{##2}{##3}}%
   \fi}%
  %
  % Comando para escribir en el .csv.
  \newcommand{\@EscribeItemsEnCSV}[3]{%
   \ifbool{SinCantidad}
   {\ifdescuentos
     \write\csv{"\unexpanded{##3}","\Dcto","\PrecioT","##1"}%
    \else
     \write\csv{"\unexpanded{##3}","\PrecioT","##1"}%
    \fi}
   {\if@cdu
     \ifdescuentos
      \write\csv{"##2","\unexpanded{##3}","\PrecioU","\Dcto","\PrecioT","##1"}%
     \else
      \write\csv{"##2","\unexpanded{##3}","\PrecioU","\PrecioT","##1"}%
     \fi
    \fi
    \ifdcu
     \ifdescuentos
      \write\csv{"\unexpanded{##3}","##2","\PrecioU","\Dcto","\PrecioT","##1"}%
     \else
      \write\csv{"\unexpanded{##3}","##2","\PrecioU","\PrecioT","##1"}%
     \fi
    \fi
    \ifduc
     \ifdescuentos
      \write\csv{"\unexpanded{##3}","\PrecioU","##2","\Dcto","\PrecioT","##1"}%
     \else
      \write\csv{"\unexpanded{##3}","\PrecioU","##2","\PrecioT","##1"}%
     \fi
    \fi}}%
  %
  % Usamos la definición de \label@factura dentro del entorno de la factura para que sea local aquí nada más.
  \let\label\label@factura
  \@LetraItems% Este es el tipo de letra que tendrán los ítems, y en este tipo de letra se define el \baselineskip de toda la factura.
  \setlength{\baselineskip@@}{\baselineskip}%
  \global\baselineskip@@=\baselineskip@@
  \@FacturaAuxInicioCmd
  \ifbool{SinLineasHoriExt}{}{\hline}%
  \@ZERO
  \setlength{\tabcolsep@@}{\tabcolsep}%
  \global\tabcolsep@@=\tabcolsep@@
  % Estamos en el número del ítem, si aplicara.
  \if@LineaDesdeInicio \zsaveposx{LineaNodoACoordX}\fi
  \ifbool{NumItem}{\@NumItemAuxCmd{\mbox{}\hfill \@LetraTitColumnas \@@NumItem \hfill\mbox{}}}{}%
  &
  % Avanzamos a la cantidad, o la descripción.
  \ifbool{SinCantidad}{}{\ifbool{PrimeroDescripcion}{\@DescripcionAuxCmd{\@LetraTitColumnas \@@Descripcion}{}}{\@CantidadAuxCmd{\mbox{}\hfill \@LetraTitColumnas \@@Cantidad \hfill\mbox{}}}}%
  &% Avanzamos hasta la descripción, la cantidad o el precio unitario.
  \ifbool{SinCantidad}{\@DescripcionAuxCmd{\@LetraTitColumnas \@@Descripcion}{}}
   {\if@cdu \@DescripcionAuxCmd{\@LetraTitColumnas \@@Descripcion}{}\fi
    \ifdcu \@CantidadAuxCmd{\mbox{}\hfill \@LetraTitColumnas \@@Cantidad \hfill\mbox{}}\fi
    \ifduc \@PrecioUAuxCmd{\mbox{}\hfill \@LetraTitColumnas \@@PrecioU \space (\Moneda)\hfill\mbox{}}\fi}%
  &% Avanzamos hasta el precio unitario, o la cantidad, o columna fantasma cuando es 'sincantidad'.
  \ifbool{SinCantidad}{}
   {%
    \mbox{}\hfill
    \ifduc
     \@CantidadAuxCmd{\mbox{}\hfill \@LetraTitColumnas \@@Cantidad \hfill\mbox{}}%
    \else
     \@PrecioUAuxCmd{\mbox{}\hfill \@LetraTitColumnas \@@PrecioU \space (\Moneda)\hfill\mbox{}}%
    \fi}%
  &% Avanzamos hasta el subtotal, que puede o no existir,
  \ifbool{Subtotal}{\@PrecioSAuxCmd{\mbox{}\hfill \@LetraTitColumnas \@@PrecioS \space (\Moneda)\hfill\mbox{}}}{}%
  &% Avanzamos hasta el descuento, que puede o no existir.
  \ifdescuentos \@DescuentoAuxCmd{\mbox{}\hfill \@LetraTitColumnas \@@Descuento \hfill\mbox{}}\fi
  &% Avanzamos hasta el IVA, que puede o no existir.
  \ifbool{IVA}{\@IVAAuxCmd{\mbox{}\hfill \@LetraTitColumnas \@@IVA \hfill\mbox{}}}{}
  &% Avanzamos hasta el precio total.
  \@PrecioTAuxCmd{\mbox{}\hfill \@LetraTitColumnas \ifbool{SinCantidad}{\@@Precio}{\@@PrecioT}\space (\Moneda)\hfill\mbox{}}%
  \if@LineaHastaFinal \zsaveposx{LineaNodoBCoordX}\fi
  \\
  \ifbool{SinLineaHoriTit}{}{\hline}%
  %
  #1% Aquí entran todos los items que el usuario declara dentro del ambiente.
  \@CalculaTotalesI
  \@CalculaSeparaItemsExtraFinal
  \\[\@SeparaItemsExtraFinal]% El último fin de línea se hace aquí.
  \ifconlineasha
   \ifdefvoid{\@NotaInterna}{%
    \ifbool{SinLineasHoriExt}{\if@ExpandeCuadro\hline\fi}{\hline}}{\hline}%
  \fi% Se verifica si hace falta una línea horizontal o no.
  \iffilas
   % Esto en lugar de \ifexpandecuadro para que no se haga una linea vacía en la tabla cuando la caja está vacía.
   \if@ExpandeCuadro
    \ifbool{PrimeroDescripcion}{&\usebox{\@CajaEspacioVacio}&&&&&&\\}{&&\usebox{\@CajaEspacioVacio}&&&&&\\}%
    \ifdefvoid{\@NotaInterna}{}
    {\ifbool{SinLineaHoriNI}{}{\hline}%
     \@ColumnasMultiColumnNotaIntAuxCmd}%
    \ifbool{SinLineasHoriExt}{}{\hline}% Siempre que hay relleno se pone la línea acá.
   \else
    \ifdefvoid{\@NotaInterna}{%
     \ifbool{SinLineasHoriExt}{}{%
      \ifconlineasha\else\hline\fi}}% Verificamos si hay que poner la línea horizontal. Si está 'conlineasha' entonces ya salió la línea.
     {\ifbool{SinLineaHoriNI}{}{\ifconlineasha\else\hline\fi}%
      \@ColumnasMultiColumnNotaIntAuxCmd
      \ifbool{SinLineasHoriExt}{}{\hline}}%
   \fi
  \else
   \ifsintotales
   \else
    \ifbool{SinLineaHoriTot}{}{\ifconlineasha\else\hline\fi}% Línea que cierra los ítems en la tabla.
   \fi
   % Agregamos el espacio espúreo.
   \if@ExpandeCuadro
    % Rellenamos la tabla con espacio espúreo, y le agregamos las líneas externas, si aplica.
    \ifbool{SinLineasVertExt}{\multicolumn{8}{c}{\usebox{\@CajaEspacioVacio}}}{\multicolumn{8}{|c|}{\usebox{\@CajaEspacioVacio}}}\\
   \fi
   % Si pedimos sin totalizaciones, entonces cerramos la tabla ya, si no, calculamos los totales y los escribimos.
   \ifsintotales
    \ifdefvoid{\@NotaInterna}{}{\ifbool{SinLineaHoriNI}{}{\hline}\@ColumnasMultiColumnNotaIntAuxCmd}%
    \ifbool{SinLineasHoriExt}{}{\hline}%
   \else
    % Modo de columnas: Abrimos la tabla para escribir los totales y le agregamos las líneas verticales externas, si aplica.
    \@ColumnasMultiColumnTotalesAuxCmd
    \ifagrupado
     \@EscribeTotalColAgrupado
    \else
     \if@Exento \@EscribeTotalCol{E}\fi
     \if@BIR\@EscribeTotalCol{R}\ifsinivaexpresado\else\@EscribeTotalColAlicuota{R}\fi\fi
     \@EscribeTotalCol{G}\ifsinivaexpresado\else\@EscribeTotalColAlicuota{G}\fi
     %% (quitado por no vigente)
     % \ifbool{Electro}{\@EscribeTotalColAlicuotaGDesc}{}%
     \if@BIA\@EscribeTotalCol{A}\ifsinivaexpresado\else\@EscribeTotalColAlicuota{A}\fi\fi
    \fi
    \ifsinivaexpresado\else\cline{2-2}\fi
    \hfill \@LetraTitTotal\@@Total & \@LetraNumTotal\@SignoMonetarioAux\numprint{\Total}\\
    \end{tabular}\endgroup
    \\
    \ifdefvoid{\@NotaInterna}{}{\ifbool{SinLineaHoriNI}{}{\hline}\@ColumnasMultiColumnNotaIntAuxCmd}%
    \ifbool{SinLineasHoriExt}{}{\hline}%
   \fi
  \fi
  \end{tabularx}%
  %
  % Generamos la macro \Total vacía para que no dé error en algún posible uso dentro del documento.
  \ifsintotales\gdef\Total{}\fi
  % Si pedimos no totalizar (sintotales), la opción filas queda desactivada.
  \iffilas
   \@LetraNumTotales
   \vspace*{\@SeparaVertical}\par
   \ifexpandecuadro\else\vfill\fi
   \@GeneraCajasTotalesFilas
   \newbox{\@CajaTotal}%
   \savebox{\@CajaTotal}{{\@LetraTitTotal\@@Total:} \@LetraNumTotal\@SignoMonetarioAux\numprint{\Total}}%
   %
   \ifnum \value{@IVANoGeneral}=0%
    \newbox{\@CajaGET}%
    \savebox{\@CajaGET}{\@EscribeTotalesFilasGET}%
    \ifdim \the\wd\@CajaGET < \textwidth
     \parbox{\textwidth}{\@EscribeTotalesFilasGET}%
    \else
     \hspace*{\@SepTotales}%
     \usebox{\@CajaBIAliG}\hspace*{\@SepTotales}%
     \if@Exento\usebox{\@CajaBIET}\else\usebox{\@CajaTotal}\fi
     \ifcentrado\hspace*{\@SepTotales}\fi
    \fi
   \fi
   \ifnum \value{@IVANoGeneral}=1%
    \newbox{\@CajaGxET}%
    \savebox{\@CajaGxET}{\@EscribeTotalesFilasGxET}%
    \ifdim \the\wd\@CajaGxET < \textwidth
     \@EscribeTotalesFilasGxET
    \else
     \hspace*{\@SepTotales}%
     \if@BIR\usebox{\@CajaBIAliR}\hspace*{\@SepTotales}\fi
     \usebox{\@CajaBIAliG}\hspace*{\@SepTotales}%
     \if@BIA\usebox{\@CajaBIAliA}\hspace*{\@SepTotales}\fi
     \\[\@SeparaFilas]\mbox{}%
     \ifcentrado\hspace*{\@SepTotales}\else\hfill\fi
     \if@Exento\usebox{\@CajaBIE}\hspace*{\@SeparaMinTotales}\fi\usebox{\@CajaTotal}%
     \ifcentrado\hspace*{\@SepTotales}\fi
    \fi
   \fi
   \ifnum \value{@IVANoGeneral}=2%
   \newbox{\@CajaRGAET}%
   \savebox{\@CajaRGAET}{\@EscribeTotalesFilasRGAET}%
    \ifdim \the\wd\@CajaRGAET < \textwidth
     \@EscribeTotalesFilasRGAET
    \else
     \newbox{\@CajaRGA}%
     \savebox{\@CajaRGA}{\@EscribeTotalesFilasRGA}%
     \ifdim \the\wd\@CajaRGA < \textwidth
      \@EscribeTotalesFilasRGA
      \\[\@SeparaFilas]\mbox{}%
      \ifcentrado\hspace*{\@SepTotales}\else\hfill\fi
      \if@Exento\usebox{\@CajaBIE}\hspace*{\@SeparaMinTotales}\fi\usebox{\@CajaTotal}%
      \ifcentrado\hspace*{\@SepTotales}\fi
     \else% Aquí casi cualquier cosa tiene aspecto feo.
      \CLASSWARNING{Es mejor no usar la opcio'n de totalizacio'n en filas en este caso, o disminuir los tama~nos de los tipos de letra para la totalizacio'n}%
      \centering{%
       \begin{tabular}{r@{\hspace*{\@SeparaMinTotales}}l}
        \usebox{\@CajaBIAliR} & \usebox{\@CajaBIAliA} \\[\@SeparaFilas]
        \usebox{\@CajaBIAliG} & \if@Exento\usebox{\@CajaBIET}\else\usebox{\@CajaTotal}\fi
       \end{tabular}}%
     \fi
    \fi
   \fi
  \fi
  \par
  \ifexpandecuadro
   \vspace*{-\@EspacioVacioInicial}%
   \vspace*{\@EspacioVacioFinal}%
   \zsaveposy{EspacioVacioInicial}%
  \fi
  \vfill
  \ifexpandecuadro\zsaveposy{EspacioVacioFinal}\fi
  \vspace*{\@SeparaVertical}\par
  \ifdefvoid{\@NotaExterna}{}{\parbox[t]{\linewidth}{\@LetraNotaExterna\@NotaExterna}\vspace*{\@SeparaVertical}\par}}
 
  \ifbool{SinCantidad}
  % Definamos todos los comandos \Item***. El descuento es opcional; por eso usamos G{0} en \NewDocumentCommand.
  {%
   \NewDocumentCommand{\Item}{m o m G{0} +o}{\@ITEM{G}{1}{#1}[#2]{#3}{#4}[#5]}% Ítem normal.
   \NewDocumentCommand{\ItemR}{m o m G{0} +o}{\global\@BIRtrue\global\Gtrue\@ITEM{R}{1}{#1}[#2]{#3}{#4}[#5]}% Ítem con IVA reducido.
   \NewDocumentCommand{\ItemA}{m o m G{0} +o}{\global\@BIAtrue\global\Gtrue\@ITEM{A}{1}{#1}[#2]{#3}{#4}[#5]}% Ítem con IVA adicional.
   \NewDocumentCommand{\ItemE}{m o m G{0} +o}{\global\@Exentotrue\global\Gtrue\@ITEM{E}{1}{#1}[#2]{#3}{#4}[#5]}% Ítem exento de IVA.
   \NewDocumentCommand{\ItemX}{m o +o}{\@ITEM{X}{1}{#1}[#2]{0}{0}[#3]}% Ítem sin precio.
   \let\ItemG\Item
   \ifcsv
    \ifdescuentos
     \write\csv{"\@@Descripcion","\@@Descuento","\@@Precio \space (\Moneda)","\@@IVA"}%
    \else
     \write\csv{"\@@Descripcion","\@@Precio \space (\Moneda)","\@@IVA"}%
    \fi
   \fi}
  {%
   \NewDocumentCommand{\Item}{m m o m G{0} +o}{\@ITEM{G}{#1}{#2}[#3]{#4}{#5}[#6]}% Ítem normal.
   \NewDocumentCommand{\ItemR}{m m o m G{0} +o}{\global\@BIRtrue\global\Gtrue\@ITEM{R}{#1}{#2}[#3]{#4}{#5}[#6]}% Ítem con IVA reducido.
   \NewDocumentCommand{\ItemA}{m m o m G{0} +o}{\global\@BIAtrue\global\Gtrue\@ITEM{A}{#1}{#2}[#3]{#4}{#5}[#6]}% Ítem con IVA adicional.
   \NewDocumentCommand{\ItemE}{m m o m G{0} +o}{\global\@Exentotrue\global\Gtrue\@ITEM{E}{#1}{#2}[#3]{#4}{#5}[#6]}% Ítem exento de IVA.
   \NewDocumentCommand{\ItemX}{m m o +o}{\@ITEM{X}{#1}{#2}[#3]{0}{0}[#4]}% Ítem sin precio.
   \let\ItemG\Item
   \ifcsv
    \if@cdu
     \ifdescuentos
      \write\csv{"\@@Cantidad","\@@Descripcion","\@@PrecioU \space (\Moneda)","\@@Descuento","\@@PrecioT \space (\Moneda)","\@@IVA"}%
     \else
      \write\csv{"\@@Cantidad","\@@Descripcion","\@@PrecioU \space (\Moneda)","\@@PrecioT \space (\Moneda)","\@@IVA"}%
     \fi
    \fi
    \ifdcu
     \ifdescuentos
      \write\csv{"\@@Descripcion","\@@Cantidad","\@@PrecioU \space (\Moneda)","\@@Descuento","\@@PrecioT \space (\Moneda)","\@@IVA"}%
     \else
      \write\csv{"\@@Descripcion","\@@Cantidad","\@@PrecioU \space (\Moneda)","\@@PrecioT \space (\Moneda)","\@@IVA"}%
     \fi
    \fi
    \ifduc
     \ifdescuentos
      \write\csv{"\@@Descripcion","\@@PrecioU \space (\Moneda)","\@@Cantidad","\@@Descuento","\@@PrecioT \space (\Moneda)","\@@IVA"}%
     \else
      \write\csv{"\@@Descripcion","\@@PrecioU \space (\Moneda)","\@@Cantidad","\@@PrecioT \space (\Moneda)","\@@IVA"}%
     \fi
    \fi
   \fi}

 \if@factura
  % Los descuentos totales no son compatibles -y no creo que lo vayan a ser- con el modo de totalización en filas.
  \iffilas
   \if@DescuentoTotal
    \CLASSWARNING{La opcio'n 'filas' no es compatible con los descuentos totales; sin efecto opcio'n 'filas'}
    \filasfalse
   \fi
  \fi
  \if@reverso
  \else
    \RenewEnviron{reverso}{}
  \fi
  \ifcotizacion\def\@Denominacion{\@@DenomCotiza}\else\def\@Denominacion{\@@DenomFactura}\fi%  Fijamos la denominación del documento paras las cotizaciones o facturas, que son casos particulares del modo 'factura'.
  \expandafter\geometry\expandafter{\@GeometriaFactura}%  La geometría debe aplicarse aquí.
  \AfterEndPreamble{%
   \addtolength{\@PosFechaX}{\oddsidemargin+1in}%  Le sumamos el margen izquierdo a las posiciones de las cosas; es importante que se haga en un \AfterEndPreamble antes de que empiecen a escribirse otras cosas en la página.
   \addtolength{\@PosDenomX}{\oddsidemargin+1in}%
   \addtolength{\@PosControlX}{\oddsidemargin+1in}%
   \addtolength{\@PosNotaFinalXuno}{\oddsidemargin+1in}%
   \addtolength{\@PosFirmasXuno}{\oddsidemargin+1in}%
   \setlength{\parindent}{0pt}%
   \setlength{\lineskip}{0pt}%
   \setlength{\parskip}{0pt plus 1pt}%
   \expandafter\setlength{\@SepTotales}{\the\@SeparaMinTotales plus 1fill}% \@SeparaMinTotales es el mínimo espacio entre los totales en el modo 'filas', y SepTotales es una dimensión interna, estirable, con ese como su mínimo.
   \thispagestyle{empty}%
   \pagenumbering{roman}%
   \@EncabezadoFactura
   \@EscribeNotaPrevia}
   \AtBeginShipoutNext{% Antes abría aquí un \AtBeginShipoutNext cuando había reporte y un \AtEndDocument cuando no. No es necesario eso. También lo modifiqué en \FirmaFactura.
    \@CargaFondoFactura
    \@PosicionaFechas
    \ifdenom\@PosicionaDenom\fi
    \ifprefactura\@PosicionaNumControl\fi
    \ifnum\value{@Firmas}=0\else\@PosicionaFirmas\fi
    \ifdefvoid{\@NotaFinal}{}{\@PosicionaNotaFinal}}
  \iflinea
   % Para determinar las coordenadas de la línea hay que usar \AtBeginShipoutNext porque si no se hace en el 'shipout' de la página se genera un espacio en blanco extra.
   \AtBeginShipoutNext{\iflinea\@DibujaLinea\fi}
    \iffilas
     \ifbool{SinLineasVertInt}{\@LineaDesdeIniciotrue\@LineaHastaFinaltrue}{\@LineaDescripciontrue}%
    \else
     \ifsintotales
      \@LineaDesdeIniciotrue\@LineaHastaFinaltrue
     \else
      \@LineaDesdeIniciotrue\@LineaHastaTotaltrue
     \fi
    \fi
  \fi
  %% El comando para la descripción del servicio prestado.
  \NewDocumentCommand{\Descripcion}{+m}{%
   {% Abrimos un grupo adicional para que la redefinición del entorno 'tikzpicture' sea local.
    \@LetraDescripcion
    % Cuando la línea tiene grosor cero, hacemos que el entorno 'tikzpicture' sea "fake", y redefinimos \node para que obvie los parámetros que le dimos como argumento opcional.
    \ifdim\@GrosorLineaDescripcion=0pt%
     \setlength{\@SeparaInternaDescripcion}{0pt}\setlength{\@EsquinaLineaDescripcion}{0pt}\setlength{\@SombraDescripcion}{0pt}%
     \renewenvironment{tikzpicture}{\NewDocumentCommand{\node}{o +m}{####2}}{}%
    \fi
    \begin{tikzpicture}
     % Expandimos después para poderle pasar el contenido de la macro como argumento opcional a \node.
     \expandafter\node\expandafter[\@DescripcionTIKZ]{%
      \ifbool{ExpAbajo}
       {\parbox[t][\textheight-\pagetotal-2\@SeparaInternaDescripcion-\@GrosorLineaDescripcion-\@SombraDescripcion]
        {\textwidth-2\@SeparaInternaDescripcion-\@GrosorLineaDescripcion-\@SombraDescripcion}}%
       {\parbox[t]{\textwidth-\@SombraDescripcion-2\@SeparaInternaDescripcion-2\@GrosorLineaDescripcion}}%
      {\setlength{\parindent}{0.5cm}%
       \setlength{\parskip}{2pt plus 8pt}%
       \ifbool{ExpAbajo}{\vfill}{}%
        #1%
       \ifbool{ExpAbajo}{\vfill}{}}}%
     \ifdim\@GrosorLineaDescripcion=0pt\else ;\fi% El fin del nodo va cuando hay el entorno tikzpicture original.
    \end{tikzpicture}
    \par\vspace*{\@SombraDescripcion}}}%
 \else
  \RenewEnviron{factura}{%
   \@ZERO
   \let\label\label@factura
   \let\@ITEM\@CalculaItem
   \BODY
   \@CalculaTotalesI}
  \RenewEnviron{reverso}{}
  \NewDocumentCommand{\Descripcion}{+m}{}
 \fi

 \if@reporte
  \ifdefvoid{\@InfoPagina}{\let\@InfoPagina\@InfoPaginaNoTitulo}{}%
  \if@factura
  \else
   \expandafter\geometry\expandafter{\@GeometriaReporte}%  La geometría del reporte debe aplicarse aquí cuando sólo hay reporte, así se puede establecer el tamaño del papel.
  \fi
  \NewEnviron{reporte}{%
   \if@factura
    \expandafter\newgeometry\expandafter{\@GeometriaReporte}\cleardoublepage%  La geometría debe aplicarse aquí cuando también hay factura, pero no se puede cambiar el tamaño del papel.
   \fi
   \let\headwidth\textwidth
   \@CargaFondoReporte
   \@EncabezadoReporte
   \ifdefvoid{\@EstiloPagReporteResto}{%
    \if@FondoReporteStar%  Si no usamos \EstiloPagReporte entonces se fijará 'soloabajo' cuando se use \FondoReporte* para que sólo use marcas de pie de página, y si no se usó \FondoReporte* entonces se fijará 'plain' que usa cabeceras y pies de página.
     \pagestyle{soloabajo}%
    \else
     \pagestyle{plain}%
    \fi}
    {\pagestyle{\@EstiloPagReporteResto}}%  Si usamos \EstiloPagReporte en su argumento obligatorio entonces fijamos ese estilo de página.
   \ifdefvoid{\@EstiloPagReporteUno}{%
    \if@FondoReporte%  Si no le damos a \EstiloPagReporte su argumento opcional, entonces se fijará 'solonumpag' como el estilo de página para la primera página, pero solamente si usamos \FondoReporte, ya que el estilo de página predeterminado al usar \FondoReporte* ya se acaba de fijar y no necesita el caso especial de la primera hoja (\thispagestyle).
     \if@FondoReporteStar
     \else
      \thispagestyle{solonumpag}%
     \fi
    \fi}
    {\thispagestyle{\@EstiloPagReporteUno}}%  Si usamoe el argumento opcional de \EstiloPagReporte establecemos ese estilo de página.
   \pagenumbering{arabic}%
   \setlength{\parindent}{0.666cm}%
   \setlength{\parskip}{4pt plus 6pt minus 2pt}% separación entre párrafos en el reporte.
   \vskip\baselineskip
   \BODY\clearpage}% espacio antes de empezar con el reporte, después del título -si lo hubiera-.
 \else
  \NewEnviron{reporte}{}
 \fi}

%% Comando para el diseño de la forma libre.
\newcommand{\@CargaFondoFactura}{%
 \if@FondoFactura
  \begin{textblock*}{\paperwidth}(0pt,0pt)%
   \includegraphics{\@FondoFactura}%
  \end{textblock*}%
 \fi}

%% Comando para el fondo del reporte.
\newcommand{\@CargaFondoReporte}{%
 \begin{textblock*}{\paperwidth}(0pt,0pt)% Tuve que abrir el textblock* aquí para que aplique siempre; de esta manera se corrige un defecto que se produce cuando se usa \vspace* al inicio de la página (dentro de \@EncabezadoReporte).
  \if@FondoReporte
   \if@FondoReporteStar
    \AddToShipoutPictureBG{\includegraphics{\@FondoReporte}}%
   \else
     \noindent\includegraphics{\@FondoReporte}%
   \fi
  \fi
 \end{textblock*}}

%% Comando que coloca los espacios para firmar.
\newcommand{\@PosicionaFirmas}{%
 \savebox{\@CajaFirmas}{%
  \parbox{\textwidth+\oddsidemargin+1in-\@PosFirmasXuno-\@PosFirmasXdos}{\@EstiloFirmas}}%
 \ifbool{FirmasYnegativo}{\addtolength{\@PosFirmasY}{\paperheight-\ht\@CajaFirmas-\dp\@CajaFirmas}}{}%
 \begin{textblock*}{\wd\@CajaFirmas}(\@PosFirmasXuno,\@PosFirmasY)%
  \usebox{\@CajaFirmas}%
 \end{textblock*}}

%% Comando que coloca la nota final.
\newcommand{\@PosicionaNotaFinal}{%
 \savebox{\@CajaNotaFinal}{%
  \parbox{\textwidth+\oddsidemargin+1in-\@PosNotaFinalXuno-\@PosNotaFinalXdos}{\@EscribeNotaFinal}}%
 \ifbool{NotaFinalYnegativo}{\addtolength{\@PosNotaFinalY}{\paperheight}}{}%
 \ifbool{NotaFinalYarriba}{\addtolength{\@PosNotaFinalY}{-\ht\@CajaNotaFinal-\dp\@CajaNotaFinal}}{}% Aquí la caja abre hacia abajo, a menos de que usemos \PosNotaFinal*
 \begin{textblock*}{\wd\@CajaNotaFinal}(\@PosNotaFinalXuno,\@PosNotaFinalY)%
  \usebox{\@CajaNotaFinal}%
 \end{textblock*}}

%% Comando que coloca el número de control.
\newcommand{\@PosicionaNumControl}{%
 \savebox{\@CajaControl}{\@EstiloNumControl}%
 \ifbool{ControlXnegativo}{\addtolength{\@PosControlX}{\textwidth-\wd\@CajaControl}}{}%
 \ifbool{ControlYnegativo}{\addtolength{\@PosControlY}{\paperheight}}{}%
 \ifbool{ControlYarriba}{\addtolength{\@PosControlY}{-\ht\@CajaControl-\dp\@CajaControl}}{}%
 \begin{textblock*}{\wd\@CajaControl}(\@PosControlX,\@PosControlY)%
  \usebox{\@CajaControl}%
 \end{textblock*}}

%% Comando que coloca la denominación del documento.
\newcommand{\@PosicionaDenom}{%
 \savebox{\@CajaDenom}{\@EstiloDenom}%
 \ifbool{DenomXnegativo}{\addtolength{\@PosDenomX}{\textwidth-\wd\@CajaDenom}}{}%
 \ifbool{DenomYnegativo}{\addtolength{\@PosDenomY}{\paperheight}}{}%
 \ifbool{DenomYarriba}{\addtolength{\@PosDenomY}{-\ht\@CajaDenom-\dp\@CajaDenom}}{}%
 \begin{textblock*}{\wd\@CajaDenom}(\@PosDenomX,\@PosDenomY)%
  \usebox{\@CajaDenom}%
 \end{textblock*}}

%% Comando que coloca la(s) fecha(s).
\newcommand{\@PosicionaFechas}{%
 \savebox{\@CajaFecha}{\@EstiloFechas}%
 \ifbool{FechaXnegativo}{\addtolength{\@PosFechaX}{\textwidth-\wd\@CajaFecha}}{}%
 \ifbool{FechaYnegativo}{\addtolength{\@PosFechaY}{\paperheight}}{}%
 \ifbool{FechaYarriba}{\addtolength{\@PosFechaY}{-\ht\@CajaFecha-\dp\@CajaFecha}}{}%
 \begin{textblock*}{\wd\@CajaFecha}(\@PosFechaX,\@PosFechaY)%
  \usebox{\@CajaFecha}%
 \end{textblock*}}

%% Comando que coloca la identificación del reporte al inicio.
\newcommand{\@EscribeDatosReporte}{%
 {\@LetraEncReporte
  \setlength{\parindent}{0pt}%
  {\bfseries \@@Proveedor} \@Proveedor \par
  {\bfseries \@@Cliente} \@RazonSocial \par
  \ifdefvoid{\@Contacto}{}{{\bfseries \@@Contacto} \@Contacto \par}%
  \ifdefvoid{\@TituloReporte}{}{\vspace*{\baselineskip}{\@LetraTitReporte \centering \@TituloReporte \par}}}}

%% Comando que coloca las dos cajas del encabezado: la de los datos del cliente y la del resumen.
\newcommand{\@EscribeDatos}{%
 {\@LetraEncFactura% Abrimos un nuevo grupo para que el cambio de tipografía y la redefinición del entorno 'tikzpicture' sean locales.
  \ifdim\@GrosorLineaEncabezado=0pt%
   \setlength{\@SeparaInternaEncabezado}{0pt}\setlength{\@EsquinaLineaEncabezado}{0pt}\setlength{\@SombraEncabezado}{0pt}%
   \renewenvironment{tikzpicture}{\NewDocumentCommand{\node}{o +m}{####2}}{}%
  \fi
  \begin{tikzpicture}%
   \expandafter\node\expandafter[\@DatosTIKZ]{%
   \parbox{\textwidth-\@SombraEncabezado-2\@SeparaInternaEncabezado-\@GrosorLineaEncabezado}{\@DatosAux}}%
    \ifdim\@GrosorLineaEncabezado=0pt\else ;\fi
  \end{tikzpicture}%
  \par\vspace*{\@SombraEncabezado}}}%  Para que textblock* esté en modo vertical.

\newcommand{\@EscribeDatosResumen}{%
 {\@LetraEncFactura% Abrimos un nuevo grupo para que el cambio de tipografía y la redefinición del entorno 'tikzpicture' sean locales.
  \ifdim\@GrosorLineaEncabezado=0pt%
   \setlength{\@SeparaInternaEncabezado}{0pt}\setlength{\@EsquinaLineaEncabezado}{0pt}\setlength{\@SombraEncabezado}{0pt}%
   \renewenvironment{tikzpicture}{\NewDocumentCommand{\node}{o +m}{####2}}{}%
  \fi
  \savebox{\@CajaDatos}{%
   \parbox[t]{\textwidth/2-\@SeparaDatosResumen/2-2\@SeparaInternaEncabezado-\@GrosorLineaEncabezado-\@SombraEncabezado}{\@DatosAux*}}%
  \savebox{\@CajaResumen}{%
   \parbox[t]{\textwidth/2-\@SeparaDatosResumen/2-2\@SeparaInternaEncabezado-\@GrosorLineaEncabezado-\@SombraEncabezado}{\@ResumenAux}}%
  \setlength{\@TamDatos}{\ht\@CajaDatos+\dp\@CajaDatos}%
  \setlength{\@TamResumen}{\ht\@CajaResumen+\dp\@CajaResumen}%
  \ifdim\@TamDatos>\@TamResumen
   \setlength{\@TamFinalEncabezado}{\@TamDatos}%
  \else
   \setlength{\@TamFinalEncabezadoado}{\@TamResumen}%
  \fi
  \begin{tikzpicture}%
   \expandafter\node\expandafter[\@DatosTIKZ]{%
    \parbox[t][\@TamFinalEncabezado]{\textwidth/2-\@SeparaDatosResumen/2-2\@SeparaInternaEncabezado-\@GrosorLineaEncabezado-\@SombraEncabezado}{%    
    \@DatosAux*}}%
   \ifdim\@GrosorLineaEncabezado=0pt\else ;\fi
  \end{tikzpicture}%
  \hspace*{\@SeparaDatosResumen}\hspace*{\@SombraEncabezado}%
  \begin{tikzpicture}%
   \expandafter\node\expandafter[\@ResumenTIKZ]{%
    \parbox[t][\@TamFinalEncabezado]{\textwidth/2-\@SeparaDatosResumen/2-2\@SeparaInternaEncabezado-\@GrosorLineaEncabezado-\@SombraEncabezado}{%
     \@ResumenAux}}%
   \ifdim\@GrosorLineaEncabezado=0pt\else ;\fi
  \end{tikzpicture}%
  \par\vspace*{\@SombraEncabezado}}}%  El \par es para que textblock* esté en modo vertical.

%% Comando que coloca la caja para la nota previa, con la posibilidad de TikZ.
\newcommand{\@EscribeNotaPrevia}{%
 {% Abrimos un nuevo grupo para que la redefinición del entorno 'tikzpicture' sea local.
  \baselineskip0pt
  \ifdefvoid{\@NotaPrevia}{}{%
   \ifdim\@GrosorLineaNotaPrevia=0pt%
    \setlength{\@SeparaInternaNotaPrevia}{0pt}\setlength{\@EsquinaLineaNotaPrevia}{0pt}\setlength{\@SombraNotaPrevia}{0pt}%
    \renewenvironment{tikzpicture}{\NewDocumentCommand{\node}{o +m}{####2}}{}%
   \fi
   \begin{tikzpicture}%
    \expandafter\node\expandafter[\@NotaPreviaTIKZ]{%
    \parbox{\textwidth-\@SombraNotaPrevia-2\@SeparaInternaNotaPrevia-\@GrosorLineaNotaPrevia}{\@EstiloNotaPrevia}}%
     \ifdim\@GrosorLineaNotaPrevia=0pt\else ;\fi
   \end{tikzpicture}%
   \par\vspace*{\@SombraNotaPrevia}\vspace*{\@SeparaVertical}%
   \ifexpandecuadro\else\vfill\fi}}}

%% Comando que coloca la caja para la nota final, con la posibilidad de TikZ.
\newcommand{\@EscribeNotaFinal}{%
 {% Abrimos un nuevo grupo para que la redefinición del entorno 'tikzpicture' sea local.
  \ifdim\@GrosorLineaNotaFinal=0pt%
   \setlength{\@SeparaInternaNotaFinal}{0pt}\setlength{\@EsquinaLineaNotaFinal}{0pt}\setlength{\@SombraNotaFinal}{0pt}%
   \renewenvironment{tikzpicture}{\NewDocumentCommand{\node}{o +m}{####2}}{}%
  \fi
  \begin{tikzpicture}%
   \expandafter\node\expandafter[\@NotaFinalTIKZ]{%
   \parbox{\textwidth-\@PosNotaFinalXuno-\@PosNotaFinalXdos+\oddsidemargin+1in-\@SombraNotaFinal-2\@SeparaInternaNotaFinal-\@GrosorLineaNotaFinal}{\@EstiloNotaFinal}}%
    \ifdim\@GrosorLineaNotaFinal=0pt\else ;\fi
  \end{tikzpicture}%
  \par\vspace*{\@SombraNotaFinal}}}

%% Comandos auxiliares para los datos y el resumen.
%% En el caso de los datos, la versión \@DatosAux se ejecuta cuando no hay resumen, y la versión \@DatosAux* se ejecuta cuando hay resumen.
\NewDocumentCommand{\@DatosAux}{s}{%
 \hfill {\@LetraTitEnc \@@Datos}\hfill \mbox{}\par
 \IfBooleanT{#1}{\vfill}%
 {\bfseries \if@Nombre \@@Nombre \else \@@RazonSocial \fi} \@RazonSocial
 \IfBooleanTF{#1}{\par}{\dotfill}%
 {\bfseries \@@RIF} {\ttfamily \@RIF}\par
 \ifdefvoid{\@Contacto}{}{{\bfseries \@@Contacto} \@Contacto \par}%
 {\bfseries \@@Direccion} \@Direccion \par
 \ifdefvoid{\@Telefono}{}{{\bfseries \@@Telefono} {\ttfamily \@Telefono}}%
 \ifdefvoid{\@Email}
  {\par}
  {\IfBooleanTF{#1}
   {\par}
   {\ifdefvoid{\@Telefono}{}{\hspace*{\@SeparaTelefonoEmail}\@CaracterSeparaTelefonoEmail\hspace*{\@SeparaTelefonoEmail}}}%
  {\bfseries \@@Email} {\ttfamily \@Email}}%
 \IfBooleanT{#1}{\vfill}}

\newcommand{\@ResumenAux}{\hfill {\@LetraTitEnc \@@Resumen}\hfill \mbox{}\vfill\@Resumen\vfill}

%% Comando que coloca la línea diagonal.
\newcommand{\@DibujaLinea}{%
 \setlength{\@LineaNodoXo}{\zposx{LineaNodoACoordX} sp}%
 \setlength{\@LineaNodoYo}{\zposy{LineaNodoBCoordY} sp}%
 \setlength{\@LineaNodoXf}{\zposx{LineaNodoBCoordX} sp}%
 \setlength{\@LineaNodoYf}{\zposy{LineaNodoBCoordY} sp}%
 \addtolength{\@LineaNodoYo}{\@EspacioVacioInicial-\@EspacioVacioFinal}%
 \if@LineaDesdeInicio
  \addtolength{\@LineaNodoXo}{-\tabcolsep@@}%
 \fi
 \if@LineaHastaFinal
  \addtolength{\@LineaNodoXf}{\tabcolsep@@}%
 \fi
 \iffilas
  \addtolength{\@LineaNodoYf}{-.3\baselineskip@@}%
  \if@LineaDescripcion
   \addtolength{\@LineaNodoXo}{-\tabcolsep@@}%
   \addtolength{\@LineaNodoXf}{\tabcolsep@@}%
  \fi
 \else
  \ifsintotales
  \addtolength{\@LineaNodoYf}{-.3\baselineskip@@}%
  \else
   \if@DescG \addtolength{\@LineaNodoYo}{2\baselineskip@@}\fi
   \if@DescR \if@BIR \addtolength{\@LineaNodoYo}{2\baselineskip@@}\fi\fi
   \if@DescA \if@BIA \addtolength{\@LineaNodoYo}{2\baselineskip@@}\fi\fi
   \if@DescE \if@Exento \addtolength{\@LineaNodoYo}{2\baselineskip@@}\fi\fi
   \addtolength{\@LineaNodoYo}{\baselineskip@@}%
   \addtolength{\@LineaNodoYf}{-.3\baselineskip@@}%
   \ifsinivaexpresado
   \else
    \addtolength{\@LineaNodoYo}{2\baselineskip@@}%
    \if@BIR\addtolength{\@LineaNodoYo}{2\baselineskip@@}\fi
    \if@BIA\addtolength{\@LineaNodoYo}{2\baselineskip@@}\fi
    \if@Exento\addtolength{\@LineaNodoYo}{\baselineskip@@}\fi
   \fi
  \fi
 \fi
 \iflinea
  \begin{textblock*}{\paperwidth-\@LineaNodoXo}(\@LineaNodoXo,\paperheight-\@LineaNodoYo)%
   \begin{tikzpicture}%
    \draw (\@LineaNodoXo,\@LineaNodoYo) -- (\@LineaNodoXf,\@LineaNodoYf);%
   \end{tikzpicture}%
  \end{textblock*}%
 \fi}

%% Escribimos los totales al archivo csv.
\newcommand{\@EscribeTotalesEnCSV}{%
 \ifcsv
  \if@Exento
   \if@DescE
    \immediate\write\csv{Totales: \@@SubtotalE: \Moneda \SubtE^^JTotales: \@@DescuentoE: \Moneda \DescE}%
   \fi
   \immediate\write\csv{Totales: \@@BIE: \Moneda \BIE}%
  \fi
  \if@BIR
   \if@DescE
    \immediate\write\csv{Totales: \@@SubtotalR: \Moneda \SubtR^^JTotales: \@@DescuentoR: \Moneda \DescR}%
   \fi
   \immediate\write\csv{Totales: \@@BIR: \Moneda \BIR^^JTotales: \@@AlicuotaR: \Moneda \AlicuotaR}%
  \fi
  \if@DescG
   \immediate\write\csv{Totales: \@@SubtotalG: \Moneda \SubtG^^JTotales: \@@DescuentoG: \Moneda \DescG}%
  \fi
  \immediate\write\csv{Totales: \@@BIG: \Moneda \BIG^^JTotales: \@@AlicuotaG: \Moneda \AlicuotaG}%
  %% (quitado por no vigente)
  % \FPifeq\AlicuotaGDesc{0}\else\immediate\write\csv{Totales: \@@AlicuotaGDesc: \Moneda \AlicuotaGDesc}\fi
  \if@BIA
   \if@DescA
    \immediate\write\csv{Totales: \@@SubtotalA: \Moneda \SubtA^^JTotales: \@@DescuentoA: \Moneda \DescA}%
   \fi
   \immediate\write\csv{Totales: \@@BIA: \Moneda \BIA^^JTotales: \@@AlicuotaA: \Moneda \AlicuotaA}%
  \fi
  \immediate\write\csv{Totales: \@@Total: \Moneda \Total}%
  \immediate\closeout\csv
 \fi}

%% Hubo que redefinir \cleardoublepage agregándole \thispagestyle{empty} porque si no queda una página en blanco con cabeceras de página usando twoside.
\def\cleardoublepage{%
 \clearpage
 \if@twoside
  \ifodd\c@page
   \else
    \hbox{}\thispagestyle{empty}\newpage
    \if@twocolumn
     \hbox{}\newpage
    \fi
   \fi
 \fi}

%</class>
%<*def>
%% Diseño hecho en TikZ para los datos del cliente, el resumen, la nota previa, la descripción y la nota final.
\def\@DatosTIKZ{%
 draw=black, fill=white, inner sep=\@SeparaInternaEncabezado, line width=\@GrosorLineaEncabezado, rounded corners=\@EsquinaLineaEncabezado, copy shadow={top color=black, bottom color=black, shadow xshift=\@SombraEncabezado, shadow yshift=-\@SombraEncabezado}}

\let\@ResumenTIKZ\@DatosTIKZ

\def\@NotaPreviaTIKZ{%
 draw=black, fill=white, inner sep=\@SeparaInternaNotaPrevia, line width=\@GrosorLineaNotaPrevia, rounded corners=\@EsquinaLineaNotaPrevia, copy shadow={top color=black, bottom color=black, shadow xshift=\@SombraNotaPrevia, shadow yshift=-\@SombraNotaPrevia}}

\def\@DescripcionTIKZ{%
 draw=black, fill=white, inner sep=\@SeparaInternaDescripcion, line width=\@GrosorLineaDescripcion, rounded corners=\@EsquinaLineaDescripcion, copy shadow={top color=black, bottom color=black, shadow xshift=\@SombraDescripcion, shadow yshift=-\@SombraDescripcion}}

\def\@NotaFinalTIKZ{%
 draw=black, fill=white, inner sep=\@SeparaInternaNotaFinal, line width=\@GrosorLineaNotaFinal, rounded corners=\@EsquinaLineaNotaFinal, copy shadow={top color=black, bottom color=black, shadow xshift=\@SombraNotaFinal, shadow yshift=-\@SombraNotaFinal}}

%% El diseño de las distintas partes de la hoja.
\newcommand{\@EstiloFirmas}{
 \@LetraFirmas
 \ifnum \value{@Firmas}=1
 \else
  \parbox{\@TamLineaFirmas}{
   \rule{\@TamLineaFirmas}{\@GrosorLineaFirmas}
   \centering \ifdefvoid{\@Conforme}{\@@Conforme}{\@Conforme}}
 \fi
 \hfill
 \parbox{\@TamLineaFirmas}{
  \rule{\@TamLineaFirmas}{\@GrosorLineaFirmas}
  \centering \ifdefvoid{\@Emisor}{\ifcotizacion \@@Emisor \else \@@Pagado \fi}{\@Emisor}}}

\newcommand{\@EstiloNotaPrevia}{\@LetraNotaPrevia \@NotaPrevia}

\newcommand{\@EstiloNotaFinal}{\@LetraNotaFinal \@NotaFinal}

\newcommand{\@EstiloNumControl}{%
 \renewcommand{\baselinestretch}{0}\selectfont
 \begin{tabular}{@{\extracolsep{0pt}}l}
  \@LetraDenom \@@Control \\[.4em]
  {\@LetraDenom\@NumSerieControl-}\space\@LetraNumeracion\@NumControl \\
 \end{tabular}}

\newcommand{\@EstiloDenom}{%
 \@LetraDenom\@Denominacion
 \ifsinnumero
 \else
  \space\space{\@LetraNumeracion N\textordmasculine\space\@Num}%
 \fi}

\newcommand*{\@EstiloFechas}{%
 \renewcommand{\baselinestretch}{0}%
 \setlength{\tabcolsep}{0pt}%
 \@LetraFecha\begin{tabular}{r@{\extracolsep{1ex}}l@{\extracolsep{0pt}}}
 \@@Fecha & \ttfamily\@Fecha \\[.3em]
 \ifvencimiento \@@FechaVencimiento & \ttfamily\@FechaVencimiento \\ \else \vspace*{-.3em}\fi
 \end{tabular}}

\newcommand{\@InfoPaginaNoTitulo}{\@LetraInfoPagina Reporte correspondiente a la factura n\'umero \@Num \space de fecha \@Fecha}

\newcommand{\@InfoPaginaTitulo}{\@LetraInfoPagina \@TituloReporte \space - \@Proveedor \space - \@Fecha}

%% Estilos de página: normal, sólo cabeceras, sólo pies de página y sólo número de página al pie.
\fancypagestyle{plain}{%
 \expandafter\ps@fancy
 \fancyhf{}%
 \renewcommand{\headrulewidth}{\@GrosorLineaInfoPag}%
 \fancyhead[LEO]{\@InfoPagina}%
 \fancyfoot[C]{\thepage}}

\fancypagestyle{soloarriba}{%
 \expandafter\ps@fancy
 \fancyhf{}%
 \renewcommand{\headrulewidth}{\@GrosorLineaInfoPag}%
 \fancyhead[LO]{\@InfoPagina}%
 \fancyhead[RO]{\thepage}%
 \if@twoside
  \fancyhead[LE]{\thepage}%
  \fancyhead[RE]{\@InfoPagina}%
 \fi}

\fancypagestyle{soloabajo}{%
 \expandafter\ps@fancy
 \fancyhf{}%
 \renewcommand{\headrulewidth}{0pt}%
 \fancyfoot[LO]{\@InfoPagina}%
 \fancyfoot[RO]{\thepage}%
 \if@twoside
  \fancyfoot[RE]{\@InfoPagina}%
  \fancyfoot[LE]{\thepage}%
 \fi}

\fancypagestyle{solonumpag}{%
 \expandafter\ps@fancy
 \fancyhf{}%
 \renewcommand{\headrulewidth}{0pt}%
 \fancyfoot[C]{\thepage}}

%</def>
%<*!class&!def&!firma&!fondo&!membrete&!aux&!batchfile>
%% Descripción de este ejemplo:
%
%%<desc>
%<beispiel>%%  Adaptado a otra legislación
%<factuno|factdos|facttres|factcuatro|factcinco|factseis|factsiete|factocho|factnueve|factdiez|repuno|repdos>%%  Escribimos una factura
%<reptres>%%  No escribimos factura ni usamos las macros que definen valores para su uso
%<factnueve|factdiez|repuno>%%  Escribimos en el reverso de la factura
%<prefac|beispiel>%%  (prefactura) Escribimos una prefactura
%<cotiza>%%  (cotizacion) Escribimos una cotización
%<repuno|repdos|reptres>%%  Escribimos un reporte
%<repuno>%%  (sinreverso) Escribimos el reverso de la factura pero sólo producimos su anverso
%<repuno>%%  (conreporte) Generamos factura y reporte
%<repdos>%%  (soloreporte) Escribimos factura y reporte, pero sólo producimos el reporte
%<reptres>%%  (soloreporte) Generamos solamente el reporte
%<factcinco|factocho>%%  Escribimos factura y reporte, pero sólo producimos la factura
%
%<factdiez>%%  Establecemos algunas opciones de la clase a través de comandos
%
%<factuno>%%  Hacemos una factura básica con todo en sus valores por omisión, no ajustamos nada
%
%<reptres>%%  (twoside) El reporte se imprime en las dos caras del papel
%
%<factcinco>%%  (sincantidad) Cuadro de la factura sin la columna de cantidades
%<factdos|factseis>%%  (numitem) Con la columna para enumerar los ítems
%<factdos|factseis>%%  (descuentos) Con la columna para descuentos individuales
%<factseis>%%  (subtotal) Con la columna extra para subtotal
%<factcinco>%%  (iva) Con la columna para el IVA
%
%<facttres>%%  (agrupado) Agrupados los totales por tipo y no por gravamen
%<factcuatro|factcinco|factsiete>%%  (filas) Totalización en modo de filas
%<factsiete>%%  (centrado) Centrados los totales en el modo de filas
%
%<factdos>%%  (dcu) Primero la descripción, luego la cantidad, luego el precio unitario
%<factcuatro>%%  (duc) Primero la descripción, luego el precio unitario, luego la cantidad
%<factcinco|factseis|factsiete>%%  (denom) Escribe la denominación (“FACTURA”) y el número
%<cotiza>%%  (sinnumero) Escribe la denominación (“COTIZACIÓN”) pero sin número
%<cotiza>%%  (sintotales) No totaliza, para que el cliente tenga un presupuesto por distintos ítems que luego elegirá entre ellos, es decir, una lista de precios
%
%<factdos|factseis>%%  (sinlineasve) Sin líneas verticales externas
%<facttres|factcuatro>%%  (sinlineasvi) Sin líneas verticales internas
%<factdiez>%%  (sinlineasv) Sin líneas verticales
%<factdos>%%  (sinlineahtot) Sin línea horizontal interna después de los ítems y antes de los totales
%<factcuatro>%%  (sinlineahtit) Sin línea horizontal interna después del título de las columnas
%<factseis>%%  (sinlineashi) Sin líneas horizontales internas
%<factsiete>%%  (sinlineashe) Sin líneas horizontales externas
%<factnueve>%%  (sinlineahni)  Sin línea horizontal interna después de la nota interna.
%<factdiez>%%  (sinlineash) Sin líneas horizontales
%<facttres|factcinco>%%  (conlineasha) Separado cada ítem con una línea horizontal
%
%<factcinco|factocho>%%  (sinexpandir) Sin expandir el cuadro de la factura ni la descripción
%<facttres|factcuatro|prefac>%%  (expandecuadro) Expande el cuadro de la factura con espacio en blanco
%<prefac|factcuatro>%%  (linea) Con línea para inutilizar el espacio en blanco
%
%<facttres>%%  (impdesc) Con el tipo de impuesto como parte de la descripción y no de los precios
%<factdos>%%  (centradesc) Centra la descripción en los elementos en la factura
%<factsiete>%%  (sinmarcas) Cuando la descripción de un ítem ocupa más que una línea, no pone el carácter que indica continuación en la siguiente línea ni los caracteres en las columnas vacías
%<factcinco|factseis|factsiete>%%  (todosiva) Forzando a totalizar todos los IVA así no existan productos gravados con ellos
%<factnueve>%%  (G) Escribiendo una letra ‘G’ en los ítems gravados con IVA general así sea el único gravamen a facturar
%<factdiez>%%  (sinmonedatotales) No agrega el símbolo monetario en los totales
%<factocho>%%  (siniva) Sin IVA
%<factdos>%%  (unafirma) Sólo la firma del emisor
%<facttres>%%  (sinfirmas) Sin firmas
%<factdiez>%%  (vencimiento) Con fecha de vencimiento explícita así no haya crédito
%
%<factocho>%%  (csv) Escribe un archivo CSV con la información numérica de la factura
%<beispiel>%%  (nospanish) Sin cargar babel en castellano, ni la codificación T1 para la tipografías
%
%<factdos|factcuatro>%%  Usando \Nombre y \CI en lugar de \RazonSocial y \RIF
%<factdos|factcuatro|reptres>%%  Sin especificar la persona de contacto
%<factcinco|factseis>%%  Sin especificar el teléfono del cliente
%<factcuatro|factseis>%%  Sin especificar el e-mail del cliente
%<factcinco|repuno>%%  (\Emisor) Especificando el nombre del emisor de la factura
%<factcinco|factseis>%%  (\Conforme) Especificando el nombre de quien firma la copia de la factura como recibida
%<repuno>%%  Sin especificar el proveedor, así toma el emisor de la factura como el nombre del proveedor
%<repdos|reptres>%%  (\Proveedor) Especificando el nombre del proveedor para el encabezado del reporte
%
%<facttres>%%  Sin especificar fecha para que tome la fecha del día.
% %<factnueve>%%  Con nota interna automática por el descuento en pagos electrónicos.
% %<factdiez>%%  Con nota interna automática con el monto en Bs.F dada la fecha de emisión de la factura antes de que a la moneda se le quitara el adjetivo “soberano”.
%
%<factcuatro>%%  (\Fecha*) Suministrando la fecha según formato de datetime2 sin alterar cómo sale escrita
%<facttres>%%  (\FormatoFecha) Cambiando el separador de la fecha para que salga escrita dd/mm/yyyy
%<factcinco>%%  (\FormatoFecha) Suministrando la fecha según el formato yyyymmdd pero que salga escrita dd.mm.yyyy
%<factseis>%%  (\FormatoFecha) Cambiando el formato para que la fecha salga escrita en el formato default de datetime2 pero cambiando el separador yyyy/mm/dd
%<factocho>%%  Redefiniendo el formato de la fecha (redefiniendo macros \@@Fecha y \@EstiloFecha)
%
%<factnueve>%%  (\Credito) Especificando la cantidad de días para el crédito
%<facttres>%%  (\DescuentoE y \DescuentoG) Con descuentos totales en porcentajes para exentos y gravables con IVA general
%<factocho>%%  (\Descuento*) Con descuentos totales en dinero neto (y no en porcentajes) aplicables por igual a todos los tipos de elementos existentes
%<prefac>%%  (\Descuento) Con descuentos totales en porcentaje aplicables por igual a todos los tipos de elementos existentes
%
%<factsiete>%%  Redefiniendo el texto para la cantidad (redefiniendo macro \@@Cantidad)
%
%<factsiete>%%  (\GeometriaFactura) Cambiando la geometría del papel de la factura
%<factdiez>%%  (\GeometriaReverso) Cambiando la geometría del reverso de la factura
%<repdos>%%  (\GeometriaReporte) Cambiando la geometría del papel del reporte
%<prefac|cotiza>%%  (\FondoFactura) Con el fondo de la imprenta agregado
%<repuno>%%  (\FondoReporte) Especificamos el fondo o membrete del reporte, sólo para la primera página
%<repdos>%%  (\FondoReporte*) Especificamos el fondo o membrete del reporte para todas las páginas
%<reptres>%%  Sin especificar fondo de reporte
%<repuno>%%  (\InicioReporte) Estableciendo un margen superior distinto para la primera página del reporte por presencia de membrete
%<repuno|reptres>%%  (\TituloReporte) Establecemos el título del reporte sin que salga en las cabeceras ni pies de página
%<repdos>%%  (\TituloReporte*) Establecemos el título del reporte para que salga en las cabeceras o pies de página
%<reptres>%%  (\InfoPagina) Personalizamos la información de la cabecera del reporte y mostramos el nombre del proveedor allí
%<reptres>%%  (\EstiloPagReporte) Cambiamos el estilo de página para que la información del reporte aparezca como pie de página
%<cotiza>%%  (\FirmaFactura) Agregando la firma escaneada en la factura, es decir, en la cotización
%
%<factdos|facttres|factsiete>%%  (\SepVertical) Alterando la separación mínima vertical entre los elementos
%<factdos|factcinco>%%  (\SepDatosResumen) Alterando la separación horizontal entre los datos del cliente y el resumen
%<factdos|facttres|factseis>%%  (\SepEncabezado) Alterando el margen interno desde el contorno hasta el texto en el encabezado
%<factnueve>%%  (\SepNotaPrevia) Alterando el margen interno desde el contorno hasta el texto en la nota previa
%<factdos|factseis>%%  (\SepDescripcion) Alterando la distancia desde el contorno hasta el texto en la descripción
%<factnueve>%%  (\SepNotaFinal) Especificando la distancia desde el contorno hasta el texto de la nota final, predeterminadamente sin contorno
%<factdos|factcuatro>%%  (\SepItemsExtra) Agregando una separación extra entre los ítems en la factura
%<factsiete>%%  (\SepFilas) Alterando la separación horizontal y vertical entre los totales en el modo de filas
%<facttres|factsiete>%%  (\LineaEncabezado) Alterando la línea de las cajas del encabezado, la curvatura en la esquina y la sombra
%<factdiez>%%  (\LineaNotaPrevia) Alterando la línea de la nota previa, para desaparecerla
%<factdos|factsiete>%%  (\LineaDescripcion) Alterando la línea de la descripción, la curvatura en la esquina y la sombra
%<factnueve>%%  (\LineaNotaFinal) Agregando el contorno para la nota final que predeterminadamente está deshabilitado
%<factsiete>%%  (\LineaFirmas) Alterando la longitud de las líneas para firmar y su grosor
%<factsiete>%%  (\PosFecha, \PosFirmas y \PosDenom) Alterando la posición de la fecha, las firmas y de la denominación del documento
%<factnueve>%%  (\PosNotaFinal) Alterando la posición de la nota final
%<factcuatro|factocho>%%  (\Letra***) Cambiando varias de las tipografías
%
%<!factcuatro&!reptres>%%  (\NotaFinal) Con nota final
%<factnueve>%%  Haciendo mención a las fechas y al crédito en la nota final
%<factcuatro>%%  Sin nota final
%<factuno|factdos|factcinco|factocho>%%  (\Resumen) Con resumen
%<facttres|factcuatro|factseis|factsiete|prefac|cotiza|repuno|repdos>%%  Sin resumen
%<factcuatro|factnueve|factdiez>%%  (\NotaPrevia) Con nota previa
%<factocho|factnueve>%%  (\NotaInterna) Con nota interna
%<factnueve>%%  (\NotaExterna) Con nota externa
%
%<factdos|facttres>%%  Con artículos gravados con todos los IVA
%<factseis|factsiete|factocho>%%  Escribimos un ítem cuya descripción ocupa más que una línea
%<factdos|factcuatro|factcinco>%%  (\ItemX) Agregando un ítem sin precio
%
%<!facttres&!reptres&!beispiel>%%  (\Descripcion) Con descripción larga
%<facttres|reptres>%%  Sin descripción larga
%
%<factseis|repdos>%%  Con etiquetas y referencias
%
%<repuno>%%  (\FirmaReporte) Agregando en el reporte la línea para firmar a mano
%<factocho|repdos|reptres>%%  (\FirmaReporte) Agregando la firma escaneada en el reporte
%<factocho|repdos|reptres>%%  Con \FirmaReporte, ajustando el corrimiento en “y” de la firma respecto a la línea
%<repdos|reptres>%%  Con \FirmaReporte, cambiando la ubicación en “x” de la línea de la firma y cambiando la longitud de la línea hacia los lados
%<reptres>%%  Con \FirmaReporte, cambiando el nombre de la persona que va a firmar
%%</desc>
%%
%<factuno>\documentclass{factura}
%<!factuno>\documentclass[
%<prefac|beispiel>prefactura,
%<cotiza>cotizacion,
%<repuno>sinreverso,
%<repuno>conreporte,
%<repdos|reptres>soloreporte,
%<beispiel>nospanish,
%<reptres>twoside,
%<factcinco>sincantidad,
%<factdos|factseis>numitem,
%<factdos|factseis>descuentos,
%<factseis>subtotal,
%<factcinco>iva,
%<factcuatro|factcinco|factsiete>filas,
%<factdiez>filas, % luego queda sin efecto con \filasfalse en el preámbulo
%<factsiete>centrado,
%<facttres>agrupado,
%<factcinco|factseis|factsiete>denom,
%<cotiza>sinnumero, % sin esta opción aparecerá el número de la cotización.
%<cotiza>sintotales,
%<factdos>dcu,
%<factcuatro>duc,
%<factdos|factseis>sinlineasve,
%<facttres|factcuatro>sinlineasvi,
%<factdos>sinlineahtot,
%<factcuatro>sinlineahtit,
%<factseis>sinlineashi,
%<factsiete>sinlineashe,
%<factnueve>sinlineahni,
%<facttres|factcinco>conlineasha,
%<factcinco|factocho>sinexpandir,
%<facttres|factcuatro|prefac>expandecuadro,
%<factcuatro|prefac>linea,
%<facttres>impdesc,
%<factdos>centradesc,
%<factsiete>sinmarcas,
%<factcinco|factseis|factsiete>todosiva,
%<factnueve>G,
%<factdiez>sinmonedatotales,
%<factocho>siniva,
%<factnueve>% electro,
%<factdos>unafirma,
%<prefac|cotiza>% dosfirmas, % para poner las dos firmas.
%<prefac>% unafirma, % para poner la firma del emisor.
%<cotiza>% sinfirmas, % para no poner firmas.
%<facttres>sinfirmas,
%<factdiez>vencimiento,
%<factocho>csv,
%<factdiez>csv, % luego queda sin efecto con \csvfalse en el preámbulo
%<!factuno>]{factura}

%<factdiez>% Cambiamos por aquí como condicionales las opciones de la clase; es lo mismo que arriba en \documentclass.
%<factdiez>\sinlineasvtrue \sinlineashtrue \filasfalse \csvfalse
%<factdiez>
%%
%% This is file `factura-ejemplo-aux.tex',
%% generated with the docstrip utility.
%%
%% The original source files were:
%%
%% factura.dtx  (with options: `aux')
%% 
%% Copyright 2019 Emilio Augusto Lazo Zaia <emiliolazozaia@gmail.com>
%% 
%% This work may be distributed and/or modified under the
%% conditions of the LaTeX Project Public License, either
%% version 1.3c of this license or (at your option) any later
%% version. This version of this license is in
%%  http://www.latex-project.org/lppl/lppl-1-3c.{html|txt|tex|pdf}
%% and the latest version of this license is in
%%  http://www.latex-project.org/lppl.txt
%% and version 1.3c or later is part of all distributions of
%% LaTeX version 2008/05/04 or later.
%% 
%% This work has the LPPL maintenance status `maintained'.
%% 
%% The Current Maintainer of this work is Emilio Augusto Lazo Zaia.
%% 
%% This work consists of the files README and factura.dtx
%% with its derived files factura.ins, factura.cls, factura.def,
%% factura-ejemplo-aux.tex, factura-ejemplo-firma.tex,
%% factura-ejemplo-fondo.tex, factura-ejemplo-membrete.tex,
%% factura-ejemplo-cotizacion.tex, factura-ejemplo-prefactura.tex,
%% factura-ejemplo-factura01.tex, factura-ejemplo-factura02.tex,
%% factura-ejemplo-factura03.tex, factura-ejemplo-factura04.tex,
%% factura-ejemplo-factura05.tex, factura-ejemplo-factura06.tex,
%% factura-ejemplo-factura07.tex, factura-ejemplo-factura08.tex,
%% factura-ejemplo-factura09.tex, factura-ejemplo-factura10.tex,
%% factura-ejemplo-reporte1.tex, factura-ejemplo-reporte2.tex,
%% factura-ejemplo-reporte3.tex and factura-beispiel-rechnung.tex.
%% 
%% Helper file; not to be compiled standalone, only to be included on examples.
%% Archivo auxiliar; no para ser compilado sino para ser incluido en los ejemplos.
%%
\usepackage{xspace,catchfilebetweentags,regexpatch}
\def\ejemplooptcolor{\color{blue}}
\def\ejemplomacrocolor{\color{purple}}
\makeatletter
\catcode`\%=11
\catcode`\|=14
\newcommand{\Ejemplo}{|
 \newpage
 \ClearShipoutPictureBG
 \newgeometry{vmargin={2cm,2cm},hmargin={2cm,2cm},footskip=0pt,headheight=0pt,headsep=0pt,footnotesep=0pt}|
 \thispagestyle{empty}|
 \catcode`\<=12
 \catcode`\>=12
 \CatchFileBetweenDelims{\DescEjemplo}{\jobname.tex}{<desc>}{</desc>}[\catcode37=11\catcode13=14\makeatletter]|
 \catcode`\%=11
 \makeatletter
 \regexpatchcmd*{\DescEjemplo}{\cC.\cO[*]*}{\c{begingroup}\c{ejemplomacrocolor}\c{ttfamily}\c{string}\0\c{endgroup}\c{xspace}}{}{}|
 \regexpatchcmd*{\DescEjemplo}{(\()(\cL[a-zG]*)(\))}{(\c{begingroup}\c{ejemplooptcolor}\c{ttfamily}\2\c{endgroup})}{}{}|
 \xpatchcmd*{\DescEjemplo}{%% }{\item}{}{}|
 \xpatchcmd*{\DescEjemplo}{}{}{}|
 \makeatother
 {\bfseries Descripci\'on de este ejemplo {\slshape \jobname.tex}:}
 \begin{itemize}
  \DescEjemplo
 \end{itemize}
 \catcode`\%=14
 \catcode`\<=13
 \catcode`\>=13
 }
\catcode`\%=14
\catcode`\|=12
\makeatother
%% 
%%
%% End of file `factura-ejemplo-aux.tex'.
% Se incluye para efectos del ejemplo, así tendremos disponible la macro \Ejemplo.
\usepackage[utf8]{inputenc}
%<beispiel>\usepackage[official]{eurosym}
%<beispiel>\usepackage[ngerman]{babel}
%<beispiel>\usepackage[T1]{fontenc}
%<beispiel>
%<beispiel>\Moneda{\euro{}}
%<beispiel>\makeatletter
%<beispiel>\def\@AlicuotaG{0.19}
%<beispiel>\def\@AlicuotaR{0.07}
%<beispiel>\def\@DenomAlicuotaG{A}
%<beispiel>\def\@DenomAlicuotaR{C}
%<beispiel>\def\@DenomAlicuotaE{B}
%<beispiel>\def\@@Datos{Kundenidentifikation}
%<beispiel>\def\@@Resumen{Zusammenfassung}
%<beispiel>\def\@@Fecha{Datum:}
%<beispiel>\def\@@Nombre{Kunde:} \def\@@RazonSocial{Kunde:}
%<beispiel>\def\@@RIF{USt-IdNr:} \def\@@CT{USt-IdNr:}
%<beispiel>\def\@@Contacto{Kontactperson:}
%<beispiel>\def\@@Direccion{Adresse:}
%<beispiel>\def\@@Telefono{Telefon:}
%<beispiel>\def\@@Email{E-mail:}
%<beispiel>\def\@@Cliente{Kunde:}
%<beispiel>\def\@@Proveedor{Lieferant:}
%<beispiel>\def\@@Cantidad{Menge}
%<beispiel>\def\@@Descripcion{Beschreibung}
%<beispiel>\def\@@PrecioU{Einzelpreis}
%<beispiel>\def\@@PrecioT{Gesamt}
%<beispiel>\def\@@Precio{Preis}
%<beispiel>\def\@@AlicuotaG{MwSt A}
%<beispiel>\def\@@AlicuotaR{MwSt C}
%<beispiel>\def\@@BIG{Netto 19\%}
%<beispiel>\def\@@BIR{Netto 7\%}
%<beispiel>\def\@@BIE{Mwst-befreit}
%<beispiel>\def\@@Total{Brutto}% Zu zahlen / Brutto / Summe / Gesamt
%<beispiel>\def\@@Conforme{der Kunde}
%<beispiel>\def\@@DenomFactura{Rechnung}
%<beispiel>\def\@@DenomCotiza{Voranschlag}
%<beispiel>
%<beispiel>\renewcommand{\@EstiloDenom}{\@LetraDenom\@Denominacion-\ifsinnumero\else\@LetraNumeracion Nr:\space\@Num\fi}
%<beispiel>\renewcommand{\@EstiloNumControl}{}
%<beispiel>\makeatother

%<factcinco|factseis|factsiete|cotiza|prefac|repuno|repdos|beispiel>\Num{1}
%<prefac>% \NumControl{1001}
%<prefac>% \NumSerieControl{9}
%<!factdos&!factcuatro&!beispiel>\RazonSocial{Ayala San Vicente Usureros}
%<factdos|factcuatro|beispiel>\Nombre{Jacinto de Ayala y San Vicente}
%<!factdos&!factcuatro&!beispiel&!reptres>\RIF{Y-12.345.678-9}
%<factdos|factcuatro|beispiel>\CI{V-987.654.321}
%<!factdos&!factcuatro&!reptres&!beispiel>\Contacto{Jacinto de Ayala}
%<!reptres>\Direccion{Av. Ayala, Galpón San Vicente, Sector Arauca, Apure, Venezuela}
%<!factcinco&!factseis&!reptres>\Telefono{+58 (987) 654.32.10}
%<!factcuatro&!factseis&!reptres>\Email{jacinto@ayala.sv}
%<factcinco|repuno>\Emisor{Lázaro de la Vega}
%<factcinco>\Conforme{\Contacto}
%<factseis>\Conforme{Ramela San Vicente}
%<repdos|reptres>\Proveedor{Lázaro de la Vega y San Román}

%<factcinco>\FormatoFecha[yyyymmdd]{ddmmyyyy}{.}
%<facttres>\FormatoFecha{ddmmyyyy}{/}
%<factseis>\FormatoFecha{default}{/}
%<!facttres&!factcuatro&!factcinco>\Fecha{06-11-2019}
%<factcuatro>\Fecha*{2019}{06}{11}
%<factcinco>\Fecha{20191106}
%<factsiete|factocho>
%<factsiete|factocho>\makeatletter
%<factocho> \renewcommand{\@@Fecha}{\slshape FECHA:}
%<factocho> \renewcommand{\@EstiloFechas}{\@LetraFecha\bfseries\@@Fecha\space{\color{blue}\@Fecha}}
%<factsiete> \renewcommand{\@@Cantidad}{Cant.}
%<factsiete|factocho>\makeatother
%<factnueve>\Credito{11}

%<prefac>\Descuento{18}
%<facttres>\DescuentoE{15}
%<facttres>\DescuentoG{20}
%<factocho>\Descuento*{300}
%<prefac|facttres|factocho>
%<factsiete>\GeometriaFactura{legalpaper,vmargin={5cm,5.5cm},hmargin={1cm,1cm},footskip=0pt,headsep=0pt,headheight=0pt}
%<factdiez>\GeometriaReverso{vmargin={4cm,4cm},hmargin={4cm,4cm},footskip=0pt,headsep=0pt,headheight=0pt}
%<repdos>\GeometriaReporte{vmargin={4.4cm,2.5cm},hmargin={3cm,2cm},footskip=1.2cm,headsep=0.777cm,headheight=1.5cm}
%<prefac|cotiza|beispiel>\FondoFactura{factura-ejemplo-fondo}
%<repuno>\FondoReporte{factura-ejemplo-membrete}
%<repdos>\FondoReporte*{factura-ejemplo-membrete}
%<repuno>\InicioReporte{2.2cm}
%<repuno|reptres>\TituloReporte{El título del reporte}
%<repdos>\TituloReporte*{El título del reporte}
%<reptres>\InfoPagina{La cabecera o pie del reporte - \Proveedor}
%<repuno|reptres>% \EstiloPagReporte{soloabajo}
%<repuno|reptres>% \EstiloPagReporte[soloabajo]{plain}
%<repuno|reptres>% \EstiloPagReporte[empty]{plain}
%<repdos|reptres>% \EstiloPagReporte[empty]{soloabajo}
%<reptres>% \EstiloPagReporte{plain}
%<repuno|repdos|reptres>% \EstiloPagReporte{solonumpag}
%<repdos>% \EstiloPagReporte{soloabajo}
%<repuno|reptres>% \EstiloPagReporte[solonumpag]{plain}
%<repuno|reptres>% \EstiloPagReporte[solonumpag]{soloarriba}
%<reptres>\EstiloPagReporte[solonumpag]{soloabajo}
%<repuno>% \EstiloPagReporte[solonumpag]{soloabajo}
%<repuno|reptres>% \EstiloPagReporte[empty]{soloarriba}
%<repuno|reptres>% \EstiloPagReporte{soloarriba}
%<cotiza>\FirmaFactura{\includegraphics[width=4.5cm]{factura-ejemplo-firma}}[0pt][17pt]
%<factdos|facttres|factsiete>\SepVertical{0.75cm}
%<factdos|factcinco>\SepDatosResumen{1cm}
%<factdos|facttres|factseis>\SepEncabezado{16pt}
%<factdos|factcuatro>\SepItemsExtra{5pt}
%<factdos|factseis>\SepDescripcion{16pt}
%<factnueve>\SepNotaPrevia{16pt}
%<factsiete>\SepFilas{1cm}{10pt}
%<factnueve>\SepNotaFinal{16pt}
%<facttres|factsiete>\LineaEncabezado{1pt}{20pt}{3pt}
%<factdiez>\LineaNotaPrevia{0pt}
%<factdos|factsiete>\LineaDescripcion{1pt}{20pt}{3pt}
%<factnueve>\LineaNotaFinal{0.4pt}{4pt}{3pt}
%<factsiete>\LineaFirmas{6.5cm}{2pt}
%<factsiete>% Puede usarse \PosFecha* o \PosDenom*, \PosControl* o \PosNotaFinal* para que el texto se despliegue en la posición 'Y' dada pero hacia arriba y no hacia abajo. \PosFirmas no acepta eso.
%<factnueve>\PosNotaFinal{.9cm}{.9cm}{-5.3cm}
%<factsiete>\PosFecha{-0cm}{-1.5cm}
%<factsiete>\PosFirmas{1cm}{1cm}{-3cm}
%<factsiete>\PosDenom{-0cm}{4.3cm}
%<factcuatro>\LetraDescripcion{\fontsize{20}{30}}
%<factcuatro>\LetraTitEnc{\fontsize{15}{20}\bfseries}
%<factcuatro>\LetraFirmas{\fontsize{16}{20}\itshape}
%<factcuatro>\LetraNotaFinal{\fontsize{18}{28}\bfseries}
%<factocho>\LetraItems{\fontsize{20}{30}}
%<factocho>\LetraNumeros{\fontsize{20}{30}}
%<factocho>\LetraTipoIVA{\fontsize{20}{30}\bfseries}
%<factocho>\LetraTitTotales{\fontsize{20}{30}\slshape}
%<factocho>\LetraNumTotales{\fontsize{20}{30}}
%<factocho>\LetraTitColumnas{\fontsize{15}{20}}
%<factocho>\LetraTitTotal{\fontsize{20}{30}\bfseries\slshape}
%<factocho>\LetraNumTotal{\fontsize{20}{30}\bfseries}
%<factocho>\tabcolsep7pt
%<!factuno&!factcinco&!prefac&!reptres>
%<cotiza>% \Denom{PRESUPUESTO} % También podemos cambiar el nombre del documento.
%<!factcuatro&!factnueve&!beispiel&!reptres>\NotaFinal{La nota final...}
%<factnueve>\NotaFinal{La factura se emitió el \Fecha{} con crédito de \Credito{} dias, siendo su vencimiento el \FechaVencimiento.}
%<beispiel>\NotaFinal{Die Zahlungsbedingungen}
%<factuno|factdos|factcinco|factocho>\Resumen{El resumen opcional}
%<beispiel>\Resumen{Zusammenfassung}
%<factcuatro|factnueve|factdiez>\NotaPrevia{\hfill LA NOTA PREVIA}
%<factocho|factnueve>\NotaInterna{LA NOTA INTERNA}
%<factnueve>\NotaExterna{LA NOTA EXTERNA}

\begin{document}
%<!reptres>
%<!reptres>\begin{factura}
%<!factdos&!factcinco&!factseis&!repdos&!reptres&!beispiel> \Item{1}{ítem con IVA general}{1000}
%<factseis|repdos> \Item{2}{ítem con IVA general, viene con descuento}{1000}{8}[Un ítem que ofrecemos a precio de costo y que está gravado con IVA general]\label{primeritem}
%<factuno|factcuatro|repdos> \Item{4}{otro ítem con IVA general}{100}
%<cotiza> \Item{1}{otro ítem con IVA general}{1100}
%<cotiza> \Item{1}{otro ítem más con IVA general}{1200}
%<facttres|prefac> \ItemR{1}{ítem con IVA reducido}{1000}
%<facttres> \ItemA{1}{ítem con IVA adicional}{1000}
%<facttres> \ItemE{1}{ítem exento de impuesto}{1000}
%<factdos|factseis> \Item{1}{ítem con IVA general, viene con descuento}{1000}{12}
%<factsiete|factocho> \Item{10}{otro ítem con IVA general (descripción larga para mostrar el cambio de línea)}[otro ítem con IVA general (esta descripción es sólo para el .csv)]{1000}
%<factdos> \ItemR{1}{ítem con IVA reducido, viene con descuento}{1000}{5}
%<factdos> \ItemA{1}{ítem con IVA adicional, viene con descuento}{1000}{10}
%<factdos> \ItemE{1}{ítem exento de impuesto}{1000}
%<factdos|factcuatro> \ItemX{1}{línea sin precio}
%<factcinco> \Item{ítem sin cantidad, con IVA general}{1000}
%<factcinco> \ItemA{ítem sin cantidad, con IVA adicional}{1000}
%<factcinco> \ItemX{línea sin precio}
%<beispiel> \Item{1}{Artikel mit 19\% MwSt}{100}
%<beispiel> \ItemR{1}{Artikel mit 7\% MwSt}{100}
%<beispiel> \ItemE{1}{Artikel mit kein MwSt}{100}
%<!reptres>\end{factura}
%<!facttres&!reptres>
%<!facttres&!factseis&!reptres&!beispiel>\Descripcion{La descripción larga}
%<factseis>\Descripcion{
%<factseis>
%<beispiel>\Descripcion{Lange Beschreibung.}
%<factcinco|factocho|factnueve|factdiez|repuno|repdos|reptres>
%<factnueve|factdiez|repuno>\begin{reverso}
%<factnueve|factdiez|repuno>esquina superior izquierda del reverso\hfill esquina superior derecha del reverso\par
%<factnueve|factdiez|repuno>\vfill
%<factnueve|factdiez|repuno>esquina inferior izquierda del reverso\hfill esquina inferior derecha del reverso
%<factnueve|factdiez|repuno>\end{reverso}
%<repuno>
%<factcinco|factocho|repuno|repdos|reptres>\begin{reporte}
%<factcinco|factocho|repuno|repdos|reptres> El reporte - página 1\newpage El reporte - página 2 \newpage El reporte - página 3.
%<repdos>
%<factseis|repdos> Se facturaron en total \arabic{NumItem} artículos.
%<factseis|repdos>
%<factseis|repdos> En el artículo número \itemref{primeritem} se facturaron \cantref{primeritem} “\descref{primeritem}”, cada uno por un valor de \Moneda~\puref{primeritem}, dando un subtotal por este elemento de \Moneda~\subtref{primeritem}; el descuento en ese artículo es del \dctoref{primeritem}\%, dando un total de \Moneda~\ptref{primeritem}, y esta gravado con IVA de \ivaref{primeritem}\%. Ampliando la descripción podemos decir que este ítem es: “\ldescref{primeritem}”.
%<factseis|repdos>
%<factseis|repdos> La base imponible total de esta factura es de \Moneda~\numprint\BIG; el IVA es de \Moneda~\numprint\AlicuotaG,  y el total final es de \Moneda~\numprint\Total.
%<factseis|repdos>
%<factseis|repdos> % Estas son todas las macros para las totalizaciones: \BIG, \BIR, \BIA, \BIE, \AlicuotaG, \AlicuotaR, \AlicuotaA, \SubtG, \SubtR, \SubtA, \SubtE, \DescG, \DescR, \DescA, \DescE.
%<factseis>}
%<repuno> \vspace{1.5cm}\FirmaReporte
%<factocho> \FirmaReporte[\includegraphics[width=4.5cm]{factura-ejemplo-firma}][0pt][17pt]
%<repdos> \FirmaReporte[\includegraphics[width=4.5cm]{factura-ejemplo-firma}][0pt][17pt][3cm][3cm]
%<reptres> \FirmaReporte[\includegraphics[width=4.5cm]{factura-ejemplo-firma}][0pt][17pt]{el que firma}[4cm][0.8cm]
%<factcinco|factocho|repuno|repdos|reptres>\end{reporte}

%% Este es un comando auxiliar suministrado por factura-ejemplo-aux.tex para escribir el detalle del ejemplo en la última hoja.
\Ejemplo

\end{document}
%</!class&!def&!firma&!fondo&!membrete&!aux&!batchfile>
%<*firma>
%% (PSTricks code generated with LaTeXDraw 2.0.8)
%% (Compilar con XeLaTeX)
\documentclass{standalone}
\usepackage[usenames,dvipsnames]{pstricks}
\begin{document}
 \begin{pspicture}(0,-3.19388)(13.8397665,3.18)
  \pscustom[linewidth=0.1,linecolor=red]{
   \newpath
   \moveto(0.0,0.9762225)
   \lineto(1.0057193,0.95771605)
   \curveto(1.5085789,0.94846284)(2.4492223,1.1335286)(2.8870058,1.3278477)
   \curveto(3.3247895,1.5221668)(4.064289,2.0079648)(4.366005,2.2994435)
   \curveto(4.667721,2.5909224)(4.9457726,2.9518008)(4.9221087,3.0212007)
   \curveto(4.8984447,3.0906003)(4.762377,3.16)(4.6499724,3.16)
   \curveto(4.5375686,3.16)(4.1175327,2.9934406)(3.8099012,2.8268814)
   \curveto(3.5022693,2.6603222)(2.721358,1.6748462)(2.2480783,0.85592973)
   \curveto(1.7747989,0.037013244)(1.1358713,-0.91607606)(0.97022337,-1.050249)
   \curveto(0.8045755,-1.1844217)(0.5975156,-1.3185943)(0.5561036,-1.3185943)
   \curveto(0.5146916,-1.3185943)(0.43186772,-1.2121814)(0.39045575,-1.1057684)
   \curveto(0.34904376,-0.9993555)(0.37270775,-0.7495166)(0.4377837,-0.6060907)
   \curveto(0.50285965,-0.46266478)(0.6152636,-0.3608783)(0.6625916,-0.4025183)
   \curveto(0.7099195,-0.44415832)(0.82232344,-0.56907773)(0.8873994,-0.65235716)
   \curveto(0.9524753,-0.7356369)(1.354763,-0.86980957)(1.6919748,-0.9207028)
   \curveto(2.0291865,-0.97159606)(2.5202143,-1.0178626)(2.67403,-1.0132358)
   \curveto(2.827846,-1.008609)(3.0112422,-0.9207028)(3.0408218,-0.8374231)
   \curveto(3.0704017,-0.75414366)(3.0999818,-0.69862366)(3.0999818,-0.7263837)
   \curveto(3.0999818,-0.75414366)(3.1709735,-0.80966306)(3.2419655,-0.8374231)
   \curveto(3.3129578,-0.8651831)(3.4667737,-0.80041015)(3.5495975,-0.7078772)
   \curveto(3.6324213,-0.6153442)(3.7744055,-0.6199707)(3.8335655,-0.7171304)
   \curveto(3.8927252,-0.81429017)(4.141197,-0.9114496)(4.330509,-0.9114496)
   \curveto(4.5198207,-0.9114496)(4.7978725,-0.86055636)(4.8866124,-0.80966306)
   \curveto(4.9753523,-0.75877017)(5.0995884,-0.74489015)(5.135084,-0.7819034)
   \curveto(5.1705804,-0.8189166)(5.324396,-0.8281702)(5.442716,-0.80041015)
   \curveto(5.5610366,-0.7726501)(5.6971045,-0.7680234)(5.7148523,-0.7911566)
   \curveto(5.7326,-0.81428987)(5.8745837,-0.83279663)(5.9988194,-0.8281699)
   \curveto(6.1230555,-0.8235431)(6.2887034,-0.78653014)(6.330116,-0.75414366)
   \curveto(6.3715277,-0.7217572)(6.454352,-0.69862366)(6.495764,-0.7078772)
   \curveto(6.537175,-0.7171304)(6.6554956,-0.6060907)(6.7324038,-0.48579803)
   \curveto(6.8093114,-0.36550537)(7.0163713,-0.041640013)(7.1465235,0.16193238)
   \curveto(7.2766747,0.36550477)(7.4245753,0.12491944)(7.442323,-0.3192389)
   \curveto(7.4600706,-0.7633969)(7.5133147,-1.6748462)(7.548811,-2.1421375)
   \curveto(7.5843067,-2.6094286)(7.5251465,-3.10448)(7.4304905,-3.13224)
   \curveto(7.335835,-3.16)(7.1465235,-3.17388)(7.051867,-3.16)
   \curveto(6.9572105,-3.14612)(6.9276304,-2.748228)(6.992707,-2.3642163)
   \curveto(7.057783,-1.980205)(7.436407,-1.4296337)(7.7499547,-1.2630746)
   \curveto(8.063502,-1.0965155)(8.619606,-0.80966306)(8.862162,-0.6893704)
   \curveto(9.104718,-0.56907773)(9.394602,-0.5181845)(9.441929,-0.58758426)
   \curveto(9.489257,-0.656984)(9.589829,-0.67549074)(9.643073,-0.6245975)
   \curveto(9.696317,-0.57370424)(9.767309,-0.5089313)(9.785057,-0.49505126)
   \curveto(9.802805,-0.48117125)(9.897461,-0.49505126)(9.974369,-0.5228113)
   \curveto(10.051277,-0.55057126)(10.175512,-0.62922424)(10.22284,-0.6801172)
   \curveto(10.270168,-0.73101014)(10.388489,-0.76802367)(10.45948,-0.75414366)
   \curveto(10.530473,-0.7402637)(10.65471,-0.6616107)(10.7079525,-0.59683746)
   \curveto(10.761196,-0.5320642)(10.838105,-0.5043045)(10.861769,-0.54131776)
   \curveto(10.885432,-0.5783307)(10.932761,-0.6477307)(10.956425,-0.6801172)
   \curveto(10.980088,-0.7125037)(11.033333,-0.75877017)(11.062913,-0.7726501)
   \curveto(11.092493,-0.78653014)(11.157568,-0.7680234)(11.193064,-0.7356369)
   \curveto(11.2285595,-0.7032504)(11.305469,-0.6662372)(11.34688,-0.6616107)
   \curveto(11.388292,-0.656984)(11.512528,-0.48579773)(11.595351,-0.31923857)
   \curveto(11.678176,-0.15267944)(11.778748,-0.02313324)(11.796496,-0.060146485)
   \curveto(11.814244,-0.09715942)(11.843824,-0.6569836)(11.855657,-1.1797949)
   \curveto(11.867488,-1.7026062)(11.87932,-2.2901905)(11.87932,-2.3549633)
   \curveto(11.87932,-2.4197364)(11.938479,-2.0773644)(11.997639,-1.6702198)
   \curveto(12.056799,-1.263075)(12.186951,-0.85592985)(12.257943,-0.85592985)
   \curveto(12.328936,-0.85592985)(12.459087,-0.8374231)(12.518247,-0.8189166)
   \curveto(12.577407,-0.80041015)(12.719391,-0.78653014)(12.802215,-0.7911566)
   \curveto(12.885039,-0.7957831)(13.03294,-0.7726501)(13.098015,-0.74489015)
   \curveto(13.163091,-0.7171301)(13.257747,-0.5922107)(13.287327,-0.49505126)
   \curveto(13.316907,-0.39789185)(13.352403,-0.33774537)(13.358318,-0.3747583)
   \curveto(13.364233,-0.41177154)(13.3878975,-0.48117125)(13.405647,-0.51355773)
   \curveto(13.423395,-0.5459442)(13.48847,-0.57370424)(13.535798,-0.56907773)
   \curveto(13.583126,-0.564451)(13.66595,-0.5598242)(13.701447,-0.5598242)
   \curveto(13.736943,-0.5598242)(13.784271,-0.5551978)(13.819766,-0.5413177)}
  \pscustom[linewidth=0.1,linecolor=red]{
   \newpath
   \moveto(12.952644,0.18)
   \lineto(13.020389,0.13)
   \curveto(13.05426,0.105)(13.135551,0.085)(13.182974,0.09)
   \curveto(13.230394,0.095)(13.304913,0.13)(13.332009,0.16)
   \curveto(13.359106,0.19)(13.399754,0.285)(13.413303,0.35)
   \curveto(13.426852,0.415)(13.4404,0.49)(13.4404,0.52)}
 \end{pspicture}
\end{document}
%</firma>
%<*fondo|membrete>
\documentclass[12pt]{article}
\usepackage[utf8]{inputenc}
\usepackage{tikz}\usetikzlibrary{positioning}
\usepackage{calc}
\usepackage[letterpaper,hmargin={1cm,1cm},vmargin={1cm,1cm},footskip=0pt,headsep=0pt,headheight=0pt]{geometry}
\parindent0pt
\lineskip0pt
\pagestyle{empty}
\begin{document}
 \begin{minipage}[b][\textheight]{\textwidth}
  \baselineskip13pt
  \parskip0pt
  \lineskip0pt
  \hspace*{\fill}%
  \begin{tikzpicture}
   \node (Nombre) [draw=none, fill=gray, rounded corners=4pt]{%
    \parbox{0.57\textwidth}{%
     \vspace*{0.1cm}%
     \centering
     \color{white}\fontfamily{ppl}\LARGE\bfseries
     Lázaro de la Vega y San Román}};
   \node [node distance=0pt, below = of Nombre, outer sep=0pt, draw=gray, line width=2pt, inner sep=3.5pt, rounded corners=8pt]{%
    \parbox{0.66\textwidth}{%
     \vspace*{3pt}%
     \parskip2pt
     \centering
     \color{gray}\fontfamily{ppl}\footnotesize\bfseries
     Av. Garcilaso de la Vega, Qta. San Román, Santa Mónica, Caracas, Venezuela.
     \par
     Teléfono: +58 (123) 456.78.90 / email: lázaro@sanrom.án
     \vspace*{3pt}}};
  \end{tikzpicture}
  \hspace*{\fill}\par
  \hspace*{\fill}%
  \parbox{0.63\textwidth}{%
   \color{gray}\fontfamily{aer}\footnotesize\bfseries
   \hfill
   RIF: V-87.654.321-0}%
  \hspace*{\fill}\par
  \vfill\mbox{}%
%<fondo>  \fontsize{6}{8}\selectfont
%<fondo>  \hspace*{0.5cm}%
%<fondo>  \parskip2pt
%<fondo>  {\bfseries ESTA FORMA LIBRE VA SIN TACHADURA NI ENMIENDA \hfill SÓLO EL ORIGINAL DA DERECHO A CRÉDITO FISCAL}
%<fondo>  \hspace*{0.5cm}\par
%<fondo>  \fontdimen2\font=0.55em
%<fondo>  Imprenta: \hfill IMPRESOS DE LA VEGA. Av. Garcilaso de la Vega, Qta. San Román, Santa Mónica, Caracas, Venezuela. RIF: R-87.654.321-0. Tlf: (0123) 456.78.90\newline
%<fondo>  \fontdimen2\font=0.65em
%<fondo>  e-mail: \hfill imprenta@sanrom.án. No. Providencia SENIAT 01/01060 de fecha 12/11/2008 - Control desde No. 00-000001 hasta No. 00-010000 / Fecha 01/01/2018.\par
 \end{minipage}
\end{document}
%</fondo|membrete>
%<*aux>
%% Helper file; not to be compiled standalone, only to be included on examples.
%% Archivo auxiliar; no para ser compilado sino para ser incluido en los ejemplos.
%%
\usepackage{xspace,catchfilebetweentags,regexpatch}
\def\ejemplooptcolor{\color{blue}}
\def\ejemplomacrocolor{\color{purple}}
\makeatletter
\catcode`\%=11
\catcode`\|=14
\newcommand{\Ejemplo}{|
 \newpage
 \ClearShipoutPictureBG
 \newgeometry{vmargin={2cm,2cm},hmargin={2cm,2cm},footskip=0pt,headheight=0pt,headsep=0pt,footnotesep=0pt}|
 \thispagestyle{empty}|
 \catcode`\<=12
 \catcode`\>=12
 \CatchFileBetweenDelims{\DescEjemplo}{\jobname.tex}{<desc>}{</desc>}[\catcode37=11\catcode13=14\makeatletter]|
 \catcode`\%=11
 \makeatletter
 \regexpatchcmd*{\DescEjemplo}{\cC.\cO[*]*}{\c{begingroup}\c{ejemplomacrocolor}\c{ttfamily}\c{string}\0\c{endgroup}\c{xspace}}{}{}|
 \regexpatchcmd*{\DescEjemplo}{(\()(\cL[a-zG]*)(\))}{(\c{begingroup}\c{ejemplooptcolor}\c{ttfamily}\2\c{endgroup})}{}{}|
 \xpatchcmd*{\DescEjemplo}{%% }{\item}{}{}|
 \xpatchcmd*{\DescEjemplo}{}{}{}|
 \makeatother
 {\bfseries Descripci\'on de este ejemplo {\slshape \jobname.tex}:}
 \begin{itemize}
  \DescEjemplo
 \end{itemize}
 \catcode`\%=14
 \catcode`\<=13
 \catcode`\>=13
 }
\catcode`\%=14
\catcode`\|=12
\makeatother
%</aux>
%    \end{macrocode}
% \Finale
% \starttodo
%  Cargamos los valores predeterminados.

%% Comando para escribir el encabezado en la factura.
\newcommand\@EncabezadoFactura{%
 \ifdefvoid{\@Resumen}{\@EscribeDatos}{\@EscribeDatosResumen}%
 \vspace*{\@SeparaVertical}%
 \ifexpandecuadro\else\vfill\fi}

%% Comando para escribir el encabezado en el reporte.
\newcommand\@EncabezadoReporte{%
 \vspace*{-\topskip}%
 \if@FondoReporte
  \if@FondoReporteStar
  \else
   \vspace*{\@InicioReporte}%
  \fi
 \fi
 \@EscribeDatosReporte
 \@LetraReporte}

%% El ambiente 'factura', usando el paquete 'environ' para su definición, y el comando auxiliar \@FacturaAuxCmd.
%% Gracias a Heiko Oberdiek y a David Carlisle (Thanks to Heiko Oberdiek and David Carlisle).
%% http://tex.stackexchange.com/questions/108565/know-the-cell-width-with-tabular-or-include-a-paragraph-box-without-specifying
\NewEnviron{factura}{%
 \expandafter\@FacturaAuxCmd\expandafter{\BODY}%
 \ifnum\value{page}=1\else\CLASSERROR{La factura ocupa mas que una pagina}\fi}

\NewEnviron{reverso}{%
 \if@factura
   \expandafter\newgeometry\expandafter{\@GeometriaReverso}%
   \topskip1sp% ¡¡No puedo poner 0 porque se genera una página en blanco extra!!
  \begin{minipage}[t][\textheight-\pagetotal]{\linewidth}%
   \thispagestyle{empty}%
   \@LetraReverso\BODY
  \end{minipage}
  \clearpage
 \fi}

%% Comando auxiliar para agregar un carácter al final de cada línea en la descripción dentro del cuadro de la factura si ocupa ésta más de una línea.
%% Gracias a Gustavo Mezzetti y a David Carlisle (Thanks to Gustavo Mezzetti and David Carlisle).
%% http://tex.stackexchange.com/questions/296035/insert-a-special-character-at-each-end-of-line-in-a-paragraph
%% http://tex.stackexchange.com/questions/248944/how-to-put-before-lines
\newcommand{\@DescripcionAuxCmd}[2]{%
 \immediate\write\@auxout{\gdef\noexpand\@TamDescripcion{\the\linewidth}}% Para poder establecer la línea diagonal hay que saber el tamaño del campo de la descripción.
 \ifimpdesc\settowidth{\@TempDimen}{\@EscribeTipoDeImpuesto{#2}}\fi
 \savebox{\@CajaPrueba}{%
  \parbox[t]{\linewidth}{%
   \David@XStart
   \ifcentradesc\centering\arraybackslash\fi
   #1%
   \David@XEnd
   \ifdim\@TempDimen=0pt%
   \else
    \hfill\kern\@SeparaFinLineaDesc\@EscribeTipoDeImpuesto{#2}%
   \fi
   \par\global\count1=\prevgraf}}%
 \ifsinmarcas
  \usebox{\@CajaPrueba}%
 \else
  \ifnum\count1=1%
   \usebox{\@CajaPrueba}%
  \else
   \ifdim\@TempDimen>\wd\@CajaCaracterCR
   \else
    \setlength{\@TempDimen}{\wd\@CajaCaracterCR}%
   \fi
   \parbox[t]{\linewidth-\@TempDimen-\@SeparaFinLineaDesc}{%
    \David@XStart
    \ifcentradesc\centering\arraybackslash\fi
    #1%
    \David@XEnd
    \par
    \global\count1=\prevgraf% Esto lo necesitamos en otro lado para calcular el espacio extra en los ítems.
    \count0=\prevgraf
    \count2=\prevgraf
    \loop
     \ifnum\count0=\count2
      \ifimpdesc
       \vskip-\baselineskip
       \mbox{}\hfill
       \rlap{%
        \hspace*{\@TempDimen}%
        \hspace*{\@SeparaFinLineaDesc}%
        \llap{\@EscribeTipoDeImpuesto{#2}}}%
      \fi
     \else
      \mbox{}\hfill
      \rlap{%
       \hspace*{\@TempDimen}%
       \hspace*{\@SeparaFinLineaDesc}%
       \hspace*{-\wd\@CajaCaracterCR}%
       \usebox{\@CajaCaracterCR}}%
     \fi
     \vskip-2\baselineskip
     \advance\count0 -1
     \ifnum\count0 > 0
    \repeat
    \kern\count2\baselineskip\mbox{}%
    \strut}%
  \fi
 \fi}

%% Comandos auxiliares para colocar en \parbox cada celda.
\newcommand{\@NumItemAuxCmd}[1]{%
 \savebox{\@CajaPrueba}{#1}%
 \ifdim\wd\@CajaPrueba > \@TamNumItem \global\@TamNumItem=\wd\@CajaPrueba\fi
 \parbox[t]{\@TamNumItem}{#1}}

\newcommand{\@CantidadAuxCmd}[1]{%
 \savebox{\@CajaPrueba}{#1}%
 \ifdim\wd\@CajaPrueba > \@TamCantidad \global\@TamCantidad=\wd\@CajaPrueba\fi
 \parbox[t]{\@TamCantidad}{#1}}

\newcommand{\@PrecioUAuxCmd}[1]{%
 \savebox{\@CajaPrueba}{#1}%
 \ifdim\wd\@CajaPrueba > \@TamPrecioU \global\@TamPrecioU=\wd\@CajaPrueba\fi
 \parbox[t]{\@TamPrecioU}{#1}}

\newcommand{\@PrecioSAuxCmd}[1]{%
 \savebox{\@CajaPrueba}{#1}%
 \ifdim\wd\@CajaPrueba > \@TamPrecioS \global\@TamPrecioS=\wd\@CajaPrueba\fi
 \parbox[t]{\@TamPrecioS}{#1}}

\newcommand{\@DescuentoAuxCmd}[1]{%
 \savebox{\@CajaPrueba}{#1}%
 \ifdim\wd\@CajaPrueba > \@TamDescuento \global\@TamDescuento=\wd\@CajaPrueba\fi
 \parbox[t]{\@TamDescuento}{#1}}

\newcommand{\@IVAAuxCmd}[1]{%
 \savebox{\@CajaPrueba}{#1}%
 \ifdim\wd\@CajaPrueba > \@TamIVA \global\@TamIVA=\wd\@CajaPrueba\fi
 \parbox[t]{\@TamIVA}{#1}}

\newcommand{\@PrecioTAuxCmd}[1]{%
 \savebox{\@CajaPrueba}{#1}%
 \ifdim\wd\@CajaPrueba > \@TamPrecioT \global\@TamPrecioT=\wd\@CajaPrueba\fi
 \parbox[t]{\@TamPrecioT}{#1}}

%% Presentación de totales en modo de columnas.
\newcommand{\@EscribeTotalCol}[1]{%
 \expandafter\FPifeq\csname Desc#1\endcsname{0}%
  \ifsinivaexpresado\else\@EscribeTotalColBI{#1}\fi
 \else
  \@EscribeTotalColSubt{#1}%
  \@EscribeTotalColDesc{#1}%
  \ifsinivaexpresado\else\@EscribeTotalColBI{#1}\fi
 \fi}

%% Forma agupada para la presentación de totales en modo de columnas.
\newcommand{\@EscribeTotalColAgrupado}{%
 \if@Exento\if@DescE\@EscribeTotalColSubt{E}\fi\fi
 \if@BIR\if@DescR\@EscribeTotalColSubt{R}\fi\fi
 \if@DescG\@EscribeTotalColSubt{G}\fi
 \if@BIA\if@DescA\@EscribeTotalColSubt{A}\fi\fi
 \if@Exento\if@DescE\@EscribeTotalColDesc{E}\fi\fi
 \if@BIR\if@DescR\@EscribeTotalColDesc{R}\fi\fi
 \if@DescG\@EscribeTotalColDesc{G}\fi
 \if@BIA\if@DescA\@EscribeTotalColDesc{A}\fi\fi
 \ifsinivaexpresado
 \else
  \if@Exento\@EscribeTotalColBI{E}\fi
  \if@BIR\@EscribeTotalColBI{R}\fi
  \@EscribeTotalColBI{G}%
  \if@BIA\@EscribeTotalColBI{A}\fi
  \if@BIR\@EscribeTotalColAlicuota{R}\fi
  \@EscribeTotalColAlicuota{G}%
  \if@BIA\@EscribeTotalColAlicuota{A}\fi
  %% (quitado por no vigente)
  % \ifbool{Electro}{\@EscribeTotalColAlicuotaGDesc}{}%
 \fi}

\newcommand{\@EscribeTotalColBI}[1]{%
 \@LetraTitTotales \csname @@BI#1\endcsname
  \ifx#1E%
  \else
   \ifG\ \hfill(\csname @Alicuota#1porc\endcsname\%)\else\mbox{}\fi
  \fi
 & \@LetraNumTotales\@SignoMonetarioAux\numprint{\csname BI#1\endcsname} \\}

\newcommand{\@EscribeTotalColAlicuota}[1]{%
 {\@LetraTitTotales \csname @@Alicuota#1\endcsname\ \hfill (\csname @Alicuota#1porc\endcsname\%)}
 & \@LetraNumTotales\@SignoMonetarioAux\numprint{\csname Alicuota#1\endcsname} \\}

%% (quitado por no vigente)
 % \newcommand{\@EscribeTotalColAlicuotaGDesc}{%
 % {\@LetraTitTotales \@@AlicuotaGDesc\ \hfill (\@AlicuotaGDescPorc\%)}
 % & \@LetraNumTotales\@SignoMonetarioAux\numprint{\AlicuotaGDesc} \\}

\newcommand{\@EscribeTotalColSubt}[1]{%
 {\@LetraTitTotales \csname @@Subtotal#1\endcsname}
 & \@LetraNumTotales\@SignoMonetarioAux\numprint{\csname Subt#1\endcsname} \\}

\newcommand{\@EscribeTotalColDesc}[1]{%
 {\@LetraTitTotales \csname @@Descuento#1\endcsname
 \csname if@Desc#1num\endcsname \else \ \hfill (\csname @Descuento#1porc\endcsname\%)\fi}
 & \@LetraNumTotales\@SignoMonetarioAux\numprint{\csname Desc#1\endcsname} \\}

%% Presentación de los totales en modo filas.
\newcommand{\@EscribeTotalFilasBI}[1]{%
 \@EscribeTotalFilasBIizq{#1}: & \@EscribeTotalFilasBIder{#1}}

\newcommand{\@EscribeTotalFilasAlicuota}[1]{%
 \@EscribeTotalFilasAlicuotaizq{#1}: & \@EscribeTotalFilasAlicuotader{#1}}

%% (no implementado)
 % \newcommand{\@EscribeTotalFilasDesc}[1]{%
 %  \@EscribeTotalFilasDescizq{#1}: & \@EscribeTotalFilasDescder{#1}}

\newcommand{\@EscribeTotalFilasSoloBI}[1]{%
 {\@LetraTitTotales\@EscribeTotalFilasBIizq{#1}:} {\@LetraNumTotales\@EscribeTotalFilasBIder{#1}}}

%% (no implementado)
 % \newcommand{\@EscribeTotalFilasSoloDesc}[1]{%
 % {\@LetraTitTotales\@EscribeTotalFilasDescizq{#1}:} {\@LetraNumTotales\@EscribeTotalFilasDescder{#1}}}

\newcommand{\@EscribeTotalFilasSoloAlicuota}[1]{%
 {\@LetraTitTotales\@EscribeTotalFilasAlicuotaizq{#1}:} {\@LetraNumTotales\@EscribeTotalFilasAlicuotader{#1}}}

\newcommand{\@EscribeTotalFilasBIizq}[1]{%
 \@LetraTitTotales\csname @@BI#1\endcsname
 \ifx#1G%
  \ifG \ (\@AlicuotaGporc\%)\fi%  Si hay otra B.I. que no sea general, escribimos el porcentaje explícito en la general.
 \else
  \ifx#1E%
  \else
  \ (\csname @Alicuota#1porc\endcsname\%)%
  \fi
 \fi}

\newcommand{\@EscribeTotalFilasBIder}[1]{%
 \@LetraNumTotales\@SignoMonetarioAux\numprint{\csname BI#1\endcsname}}

\newcommand{\@EscribeTotalFilasAlicuotaizq}[1]{%
 \@LetraTitTotales\csname @@Alicuota#1\endcsname \ (\csname @Alicuota#1porc\endcsname\%)}

\newcommand{\@EscribeTotalFilasAlicuotader}[1]{%
 \@LetraNumTotales\@SignoMonetarioAux\numprint{\csname Alicuota#1\endcsname}}

%% (no implementado)
 % \newcommand{\@EscribeTotalFilasDescizq}[1]{%
 %  \@LetraTitTotales\csname @@Descuento#1\endcsname \ (\csname @Descuento#1porc\endcsname\%)}

 % \newcommand{\@EscribeTotalFilasDescder}[1]{%
 %  \@LetraNumTotales\@SignoMonetarioAux\numprint{\csname Desc#1\endcsname}}

\newcommand{\@EscribeTotalFilasAuxTotal}{\@LetraTitTotal\@@Total: & \@LetraNumTotal\@SignoMonetarioAux\numprint{\Total}}
 
\newcommand{\@EscribeTotalFilas}[2][]{%
 \begin{tabular}[b]{@{}l@{\extracolsep{1ex}}r@{}}% ¡Esta tabla tiene que abrir hacia abajo para que la separación entre las filas se pueda colocar con \@SeparaFilas!
  \csname @EscribeTotalFilasBI\endcsname{#2}%
  \ifx#2E%
   \ifx#1T%
    \\[\@SeparaFilas]%
    \@EscribeTotalFilasAuxTotal
   \fi
  \else
   \\[\@SeparaFilas]%
   \csname @EscribeTotalFilasAlicuota\endcsname{#2}%
  \fi
 \end{tabular}}

%% Comando auxiliar para escribir el signo monetario cuando aplica
\newcommand{\@SignoMonetarioAux}{%
 \ifsinmonedatotales\else\@Moneda\hspace*{\@EspacioSignoMonetario}\fi}

\newcommand{\@GeneraCajasTotalesFilas}{%
 \if@Exento
  \newbox{\@CajaBIE}\newbox{\@CajaBIET}%
  \savebox{\@CajaBIE}{\@EscribeTotalFilasSoloBI{E}}%
  \savebox{\@CajaBIET}{\@EscribeTotalFilas[T]{E}}%
 \fi
 \if@BIR
  \addtocounter{@IVANoGeneral}{1}%
  \newbox{\@CajaBIR}\newbox{\@CajaAliR}\newbox{\@CajaBIAliR}%
  \savebox{\@CajaBIR}{\@EscribeTotalFilasSoloBI{R}}%
  \savebox{\@CajaAliR}{\@EscribeTotalFilasSoloAlicuota{R}}%
  \savebox{\@CajaBIAliR}{\@EscribeTotalFilas{R}}%
 \fi
 \newbox{\@CajaBIG}\newbox{\@CajaAliG}\newbox{\@CajaBIAliG}%
 \savebox{\@CajaBIG}{\@EscribeTotalFilasSoloBI{G}}%
 \savebox{\@CajaAliG}{\@EscribeTotalFilasSoloAlicuota{G}}%
 \savebox{\@CajaBIAliG}{\@EscribeTotalFilas{G}}%
 \if@BIA
  \addtocounter{@IVANoGeneral}{1}%
  \newbox{\@CajaBIA}\newbox{\@CajaAliA}\newbox{\@CajaBIAliA}%
  \savebox{\@CajaBIA}{\@EscribeTotalFilasSoloBI{A}}%
  \savebox{\@CajaAliA}{\@EscribeTotalFilasSoloAlicuota{A}}%
  \savebox{\@CajaBIAliA}{\@EscribeTotalFilas{A}}%
 \fi}

\newcommand{\@EscribeTotalesFilasGET}{%
 \hspace*{\@SepTotales}\usebox{\@CajaBIG}%
 \hspace*{\@SepTotales}\usebox{\@CajaAliG}%
 \if@Exento\hspace*{\@SepTotales}\usebox{\@CajaBIE}\fi
 \hspace*{\@SepTotales}\usebox{\@CajaTotal}%
 \ifcentrado\hspace*{\@SepTotales}\fi}

\newcommand{\@EscribeTotalesFilasGxET}{%
 \if@BIR\hspace*{\@SepTotales}\usebox{\@CajaBIAliR}\fi
 \hspace*{\@SepTotales}\usebox{\@CajaBIAliG}%
 \if@BIA\hspace*{\@SepTotales}\usebox{\@CajaBIAliA}\fi
 \hspace*{\@SepTotales}\if@Exento\usebox{\@CajaBIET}\else\usebox{\@CajaTotal}\fi
 \ifcentrado\hspace*{\@SepTotales}\fi}

\newcommand{\@EscribeTotalesFilasRGA}{%
 \hspace*{\@SepTotales}\usebox{\@CajaBIAliR}%
 \hspace*{\@SepTotales}\usebox{\@CajaBIAliG}%
 \hspace*{\@SepTotales}\usebox{\@CajaBIAliA}%
 \hspace*{\@SepTotales}}

\newcommand{\@EscribeTotalesFilasRGAET}{%
 \@EscribeTotalesFilasRGA
 \if@Exento\usebox{\@CajaBIET}\else\usebox{\@CajaTotal}\fi
 \ifcentrado\hspace*{\@SepTotales}\fi}

%% Escribe la letra que designa al tipo de impuesto que aplica al ítem.
\NewDocumentCommand{\@EscribeTipoDeImpuesto}{m}{%
 % Hay que abrir un grupo de ejecución porque el cambio en la tipografía altera la celda.
 {\ifG
   \@LetraTipoIVA
   \ifx#1G\@DenomAlicuotaG\fi
   \ifx#1R\@DenomAlicuotaR\fi
   \ifx#1A\@DenomAlicuotaA\fi
   \ifx#1E\@DenomAlicuotaE\fi
  \fi}}
 
%% Comando para poner en cero los contadores de los totales porque el ambiente de la factura se evalúa tres veces y los totales se sumarían todos tres veces.
\newcommand{\@ZERO}{%
 \FPset\BIG{0}\global\let\BIG\BIG
 \FPset\BIR{0}\global\let\BIR\BIR
 \FPset\BIA{0}\global\let\BIA\BIA
 \FPset\BIE{0}\global\let\BIE\BIE
 \FPset\SubtE{0}\global\let\SubtE\SubtE
 \FPset\SubtG{0}\global\let\SubtG\SubtG
 \FPset\SubtR{0}\global\let\SubtR\SubtR
 \FPset\SubtA{0}\global\let\SubtA\SubtA
 \FPset\SubtX{0}\global\let\SubtX\SubtX
 \FPset\DescE{0}\global\let\DescE\DescE
 \FPset\DescG{0}\global\let\DescG\DescG
 \FPset\DescR{0}\global\let\DescR\DescR
 \FPset\DescA{0}\global\let\DescA\DescA
 \FPset\DescX{0}\global\let\DescX\DescX
 \FPset\AlicuotaG{0}\global\let\AlicuotaG\AlicuotaG
 %% (quitado por no vigente)
  % \FPset\AlicuotaGDesc{0}\global\let\AlicuotaGDesc\AlicuotaGDesc
 \FPset\AlicuotaR{0}\global\let\AlicuotaR\AlicuotaR
 \FPset\AlicuotaA{0}\global\let\AlicuotaA\AlicuotaA}

%% Comandos para calcular todos los totales parciales y el total general.
\newcommand{\@CalculaTotalesI}{%
 \FPifeq\SubtG{0}\else\@CalculaTotalesII{G}\fi
 \FPifeq\SubtR{0}\else\@CalculaTotalesII{R}\fi
 \FPifeq\SubtA{0}\else\@CalculaTotalesII{A}\fi
 \FPifeq\SubtE{0}\else\@CalculaTotalesII{E}\fi
 %% (quitado por no vigente)
 % \ifbool{Electro}{%
 %  \FPiflt\BIG\@AlicuotaGDescLimite
 %   \let\@AlicuotaGDesc\@AlicuotaGDescBajo
 %  \else
 %   \let\@AlicuotaGDesc\@AlicuotaGDescAlto
 %  \fi
 %  \FPupn\@AlicuotaGDescPorc{\@AlicuotaGDesc{} 100 * \@DigitosPorc{} round clip}%
 %  \FPupn\AlicuotaGDesc{\BIG{} \@AlicuotaGDesc{} * \@DigitosMoneda{} round}%
 %  \global\let\AlicuotaGDesc\AlicuotaGDesc
 %  \global\let\@AlicuotaGDescPorc\@AlicuotaGDescPorc
 %  \global\let\@AlicuotaGDesc\@AlicuotaGDesc}{}%
 \FPupn\Total{\BIE{} \BIR{} + \BIG{} + \BIA{} + \AlicuotaR{} + \AlicuotaG{} + \AlicuotaA{} + \@DigitosMoneda{} round}%
 %% (quitado por no vigente) arriba se sumaba también \AlicuotaGDesc{}.
 \global\let\Total\Total}

\newcommand{\@CalculaTotalesII}[1]{%
 % Sumamos al descuento.
 \expandafter\FPupn\csname Desc#1\endcsname{\csname @Descuento#1factor\endcsname{} \csname Subt#1\endcsname{} * \csname @Descuento#1num\endcsname{} + \@DigitosMoneda{} round}%
 \global\expandafter\let\csname Desc#1\expandafter\endcsname\csname Desc#1\endcsname
 % Sumamos a la base imponible.
 \expandafter\FPupn\csname BI#1\endcsname{\csname Desc#1\endcsname{} \csname Subt#1\endcsname{} - \@DigitosMoneda{} round}%
 \global\expandafter\let\csname BI#1\expandafter\endcsname\csname BI#1\endcsname
 \ifx#1E%
 \else{%
  % Sumamos a la alícuota.
  \expandafter\FPupn\csname Alicuota#1\endcsname{\csname BI#1\endcsname{} \csname @Impuesto#1\endcsname{} * \@DigitosMoneda{} round}%
  \global\expandafter\let\csname Alicuota#1\expandafter\endcsname\csname Alicuota#1\endcsname}%
 \fi}

%% Comandos auxiliares para poder evaluar la definición de las columnas en la tabla con un comando que se expande completamente.
\def\@FacturaAuxColumnasEval{
 % Primera línea, la más externa.
 \ifbool{SinLineasVertExt}{}{|}
 % Columna para el número del ítem, si aplica.
 \ifbool{NumItem}{c}{r@{\extracolsep{0pt}}}
 % Segunda línea, que sale solamente cuando se usa la opción 'numitem'
 \ifbool{NumItem}{\ifbool{SinLineasVertInt}{}{|}}{}
 % Columna 'X' con DCU, columna fantasma con 'sincantidad' y columna 'c' con CDU.
 \ifbool{PrimeroDescripcion}{X}{
  \ifbool{SinCantidad}{l@{\extracolsep{0pt}}}{c}}
 % Tercera línea.
 \ifbool{SinCantidad}{}{\ifbool{SinLineasVertInt}{}{|}}
 % Columna 'c' con DCU, columna 'r' con DUC, columna 'X' sin espacio para 'sincantidad' y columna 'X' con CDU.
 \ifbool{DCU}{c}{
  \ifbool{DUC}{r}{
   \ifbool{SinCantidad}{X@{\extracolsep{0pt}}}{X}}}
 % Cuarta línea, que sale solamente cuando no se usa 'sincantidad'. 
 \ifbool{SinCantidad}{}{\ifbool{SinLineasVertInt}{}{|}}
 % Columna 'c' para DUC y columna 'r' para cualquier otro caso (DCU, CDU y 'sincantidad').
 \ifbool{DUC}{c}{r}
 % Quinta línea. 
 \ifbool{SinLineasVertInt}{}{|}
 % Columna 'r' para subtotal antes del descuento, o columna fantasma si no se especifica 'subtotal').
 \ifbool{Subtotal}{r}{l@{\extracolsep{0pt}}}
 % Sexta línea, que sale solamente cuando hay columna de subtotales.
 \ifbool{Subtotal}{\ifbool{SinLineasVertInt}{}{|}}{}
 % Columna 'c' para 'descuentos' y columna 'r' sin espacio si no hay descuentos individuales.
 \ifbool{Descuentos}{c}{r@{\extracolsep{0pt}}}
 % Séptima línea, que sale solamente cuando hay descuentos.
 \ifbool{Descuentos}{\ifbool{SinLineasVertInt}{}{|}}{}
 % Columna 'c' para 'iva' y columna 'r' sin espacio si no hay columna de IVA.
 \ifbool{IVA}{c}{r@{\extracolsep{0pt}}}
 % Octava línea, que sale solamente cuando hay columna de IVA.
 \ifbool{IVA}{\ifbool{SinLineasVertInt}{}{|}}{}
 % Columna 'r' siempre.
 r
 % Novena y última línea.
 \ifbool{SinLineasVertExt}{}{|}}

%% Comando para hacer los cálculos sobre cada ítem y establecer las macros con el contenido de cada ítem introducido.
%% Aquí tenemos que hacer coincidir los parámetros de \@CalculaItem con los de \@ITEM para poder hacer
%% \let\@CalculaItem\@ITEM cuando no hay factura sino solamente reporte.
\NewDocumentCommand{\@CalculaItem}{m m m o m m +o}{%
 \gdef\@PrecioUtemp##1{\FPupn\PrecioUtemp{##1 \PrecioUtempx{} * clip}}%
 \addtocounter{NumItem}{1}%
 % Calculamos.
 \@PrecioUtemp{#5}%
 \FPupn\PrecioU{\PrecioUtemp{} \@DigitosMoneda{} round}%
 \FPifeq\PrecioU{\PrecioUtemp}\else\CLASSWARNING{Se esta' introduciendo un precio con ma's de dos decimales: (#3 - \PrecioUtemp. Redondeado a \PrecioU)}{}\fi% Advertimos cuando el precio unitario tiene más de dos decimales.
 \ifbool{Descuentos}
  {\ifx\\#6\\\FPset\Dcto{0}\else\FPupn\Dcto{#6 \@DigitosPorc{} round clip}\fi}% Fijamos en 0 el descuento si no se especifica en \Item***.
  {\FPset\Dcto{#6}%
   \FPifeq\Dcto{0}\else\CLASSWARNING{Se pidio' un descuento individual sin dar la opcio'n de la clase 'descuentos'. No se tomara' en cuenta}\fi
   \FPset\Dcto{0}}% También fijamos en 0 el descuento por si se especifica en \Item*** pero no se pide la opción 'descuentos'.
 \FPupn\PrecioS{#2 \PrecioU{} * \@DigitosMoneda{} round}%
 \FPupn\PrecioT{100 \Dcto{} 100 - / \PrecioS{} * \@DigitosMoneda{} round}%
 \global\let\PrecioU\PrecioU
 \global\let\PrecioS\PrecioS
 \global\let\Dcto\Dcto
 \global\let\PrecioT\PrecioT
 % Suma el elemento actual al subtotal acumulado.
 \expandafter\FPupn\csname Subt#1\endcsname{\PrecioT{} \csname Subt#1\endcsname{} + \@DigitosMoneda{} round}%
 \global\expandafter\let\csname Subt#1\expandafter\endcsname\csname Subt#1\endcsname
 % Guardamos en macros cada elemento agregado a la factura; por ejemplo: \@Item@i@Cantidad contiene la cantidad que se introdujo para el primer ítem; así mismo con ii@, @iii@, y así mismo también con @Descripcion, @PrecioU, @Descuento y @PrecioT.
 \expandafter\gdef\csname @Item@\romannumeral\value{NumItem}@Cantidad\endcsname{\numprint{#2}}%
 \expandafter\gdef\csname @Item@\romannumeral\value{NumItem}@Descripcion\endcsname{#3}%
 \expandafter\global\expandafter\edef\csname @Item@\romannumeral\value{NumItem}@PrecioU\endcsname{\noexpand\numprint{\PrecioU}}%
 \expandafter\global\expandafter\edef\csname @Item@\romannumeral\value{NumItem}@PrecioS\endcsname{\noexpand\numprint{\PrecioS}}%
 \expandafter\global\expandafter\edef\csname @Item@\romannumeral\value{NumItem}@Descuento\endcsname{\noexpand\numprint{\Dcto}}%
 \expandafter\gdef\csname @Item@\romannumeral\value{NumItem}@DescripcionLarga\endcsname{#7}%
 \expandafter\global\expandafter\expandafter\edef\csname @Item@\romannumeral\value{NumItem}@IVA\endcsname{%
  \noexpand\numprint\expandafter{\csname @Alicuota#1porc\endcsname}}%
 \expandafter\global\expandafter\edef\csname @Item@\romannumeral\value{NumItem}@PrecioT\endcsname{\noexpand\numprint{\PrecioT}}}

%% Macro interna para saber si una etiqueta existe o no.
\gdef\@pruebaref#1{%
 \expandafter\ifx\csname @Etiqueta@#1\endcsname\relax
  \CLASSWARNING{La etiqueta `#1' no esta' definida}\ignorespaces
  \@PruebaReffalse
 \else
  \@PruebaReftrue
 \fi}

%% Comandos para las referencias: i) referencia al número de ítem, ii) a la cantidad, iii) a la descripción, iv) al precio unitario, v) al descuento, vi) al precio total.
\gdef\itemref#1{%
 \ifbool{NumItem}{}{\CLASSWARNING{Es recomendable usar la opcio'n 'numitem' para enumerar los i'tems}}%
 \@pruebaref{#1}\if@PruebaRef \csname @Etiqueta@#1\endcsname \fi}
\gdef\cantref#1{\@pruebaref{#1}\if@PruebaRef \csname @Item@\expandafter\romannumeral\csname @Etiqueta@#1\endcsname @Cantidad\endcsname \fi}
\gdef\descref#1{\@pruebaref{#1}\if@PruebaRef \csname @Item@\expandafter\romannumeral\csname @Etiqueta@#1\endcsname @Descripcion\endcsname \fi}
\gdef\puref#1{\@pruebaref{#1}\if@PruebaRef \csname @Item@\expandafter\romannumeral\csname @Etiqueta@#1\endcsname @PrecioU\endcsname \fi}
\gdef\subtref#1{\@pruebaref{#1}\if@PruebaRef \csname @Item@\expandafter\romannumeral\csname @Etiqueta@#1\endcsname @PrecioS\endcsname \fi}
\gdef\dctoref#1{\@pruebaref{#1}\if@PruebaRef \csname @Item@\expandafter\romannumeral\csname @Etiqueta@#1\endcsname @Descuento\endcsname \fi}
\gdef\ldescref#1{\@pruebaref{#1}\if@PruebaRef \csname @Item@\expandafter\romannumeral\csname @Etiqueta@#1\endcsname @DescripcionLarga\endcsname \fi}
\gdef\ivaref#1{\@pruebaref{#1}\if@PruebaRef \csname @Item@\expandafter\romannumeral\csname @Etiqueta@#1\endcsname @IVA\endcsname \fi}
\gdef\ptref#1{\@pruebaref{#1}\if@PruebaRef \csname @Item@\expandafter\romannumeral\csname @Etiqueta@#1\endcsname @PrecioT\endcsname \fi}

%% Definimos un comando para la etiqueta, que se define dentro del entorno de la factura para que tenga alcance local; cuando se acaba dicho entorno, el comando \label vuelve a ser lo que está definido en latex.ltx.
\newcommand{\label@factura}[1]{%
 \expandafter\ifx\csname @Etiqueta@#1\endcsname\relax
  \expandafter\global\expandafter\edef\csname @Etiqueta@#1\endcsname{\arabic{NumItem}}\ignorespaces
  \CLASSINFO{Etiqueta `#1' definida para el i'tem \arabic{NumItem}}%
 \else
  \edef\@label@test@uno{\csname @Etiqueta@#1\endcsname}%
  \edef\@label@test@dos{\theNumItem}%
  \ifx\@label@test@uno\@label@test@dos
  \else
   \CLASSWARNING{La etiqueta `#1' se definio' ma's de una vez}\ignorespaces
  \fi
 \fi}

%% Todo esto tiene que estar dentro de \AtEndPreamble que es cuando \if@factura, \if@reverso e \if@reporte fueron ya aplicadas, y para que se evalúen adecuadamente en su momento las definiciones hechas con \edef.
\AtEndPreamble{
 % Comando auxiliar para abrir la tabla del cuadro de la factura y expandir completamente las columnas.
 \edef\@FacturaAuxInicioCmd{\noexpand\begin{tabularx}{\textwidth}{\@FacturaAuxColumnasEval}}

 % Comando auxiliar para abrir el '\multicolumn' de la nota interna y evalúe la presencia o no de las barras verticales.
 \edef\@ColumnasMultiColumnNotaIntAuxCmd{%
  \noexpand\multicolumn{8}{\ifbool{SinLineasVertExt}{}{|}l\ifbool{SinLineasVertExt}{}{|}}{%
   \noexpand\parbox[t]{\linewidth-2\tabcolsep-2\arrayrulewidth}{\noexpand\@LetraNotaInterna\noexpand\@NotaInterna\noexpand\strut}}%
   \noexpand\\}

 % Comando auxiliar para abrir el '\multicolumn' para los totales en el modo de columnas y expandir completamente las columnas cuando hay o no barras verticales.
 \edef\@ColumnasMultiColumnTotalesAuxCmd{%
  \noexpand\multicolumn{8}{\ifbool{SinLineasVertExt}{}{|} r}\begingroup
    \noexpand\if@ExpandeCuadro
     \noexpand\iflinea\noexpand\zsaveposx{LineaNodoBCoordX}\noexpand\zsaveposy{LineaNodoBCoordY}\noexpand\fi\noexpand\fi% Fijamos la posición (x,y) final para la línea ya que estaremos justo en el punto donde escribimos el total, y hasta allí llega la línea en el modo de columnas.
     \noexpand\begin{tabular}[b]{@{\noexpand\extracolsep{2\noexpand\tabcolsep}}lr \ifbool{SinLineasVertExt}{}{|}}}

 % Comando auxiliar para la definición del ambiente de la factura; al final de este comando está el texto que va en el enddef del ambiente de la factura.
 \newcommand{\@FacturaAuxCmd}[1]{%
  \ifexpandecuadro
   \setlength{\@EspacioVacioInicial}{\zposy{EspacioVacioInicial}sp}%
   \setlength{\@EspacioVacioFinal}{\zposy{EspacioVacioFinal}sp}%
   \global\@EspacioVacioInicial=\@EspacioVacioInicial% Hay que globalizarlas
   \global\@EspacioVacioFinal=\@EspacioVacioFinal
   \savebox{\@CajaEspacioVacio}{\parbox[b][\@EspacioVacioInicial-\@EspacioVacioFinal]{\@TamDescripcion}{%
    \iflinea
     \if@LineaHastaFinal
      \zsaveposy{LineaNodoBCoordY}%
     \else
      \if@LineaDescripcion
       \zsaveposx{LineaNodoACoordX}\mbox{}\hfill \zsaveposx{LineaNodoBCoordX}\zsaveposy{LineaNodoBCoordY}%
      \fi
     \fi
    \fi}}%
   \ifdim \ht\@CajaEspacioVacio=0pt \global\lineafalse \else \@ExpandeCuadrotrue \fi
  \fi

  \ifx\undefined\Total\relax\else\CLASSERROR{Parece ya haberse usado un entorno de factura}\fi
  %
  % Comandos \David@SavePos, \David@XStart, \David@XEnd, \David@XPad, \David@XLeft y \David@XRight para poder manejar el salto de línea en las columnas adyacentes a la columna 'X' del tabularx (la descripción), cuando esta ocupa más que una línea.
  % Gracias a David Carlisle (Thanks to David Carlisle).
  % http://tex.stackexchange.com/questions/111203/tabularx-bottom-alignment-on-r-or-l-columns-when-x-column-wraps
  \def\David@SavePos##1{\leavevmode\pdfsavepos\write\@auxout{\gdef\noexpand##1{\the\pdflastypos sp}}}%
  \def\David@XStart{\expandafter\David@SavePos\csname save@start@\romannumeral\value{NumItem}\endcsname}%
  \def\David@XEnd{\expandafter\David@SavePos\csname save@end@\romannumeral\value{NumItem}\endcsname}%
  \def\David@XPad##1##2##3{%
   \vtop{%
    \baselineskip\normalbaselineskip
    ##2%
    \expandafter
    \ifx\csname save@start@##1\endcsname
     \relax
    \else
     \dimen@\csname save@start@##1\endcsname\relax
     \loop
      \ifdim\dimen@>\csname save@end@##1\endcsname
      \advance\dimen@-\baselineskip
      \settowidth{\@TempDimen}{##3}%
       \ifdim\@TempDimen=0pt\settowidth{\@TempDimen}{##2}\fi
      \hbox{\parbox{\@TempDimen}{\@CaracterCeldasVacias}}%
     \repeat
    \fi
    ##3}}%
  \def\David@XLeft##1{\David@XPad{\romannumeral\value{NumItem}}{\hbox{##1}}{}}%
  \def\David@XRight##1{\David@XPad{\romannumeral\value{NumItem}}{}{\hbox{##1}}}%
  %
  % Caja auxiliar para el caracter de retorno de carro y poder saber si tamaño en el campo de la descripción.
  \savebox{\@CajaCaracterCR}{\@LetraItems\@CaracterCR}
  %
  % Comando para fijar la separación extra final de cada item, lo cual viene dado por la cantidad de líneas de la descripción,
  % que está alojado en \count1, definido en \@DescripcionAuxCmd.
  \def\@CalculaSeparaItemsExtraFinal{%
   \setlength{\@SeparaItemsExtraFinal}{\numexpr\the\count1\normalbaselineskip-\normalbaselineskip+\@SeparaItemsExtra}%
   \global\@SeparaItemsExtraFinal=\@SeparaItemsExtraFinal}
  %
  % Comando interno general para agregar cualquier ítem.
  \NewDocumentCommand{\@ITEM}{m m m o m m +o}{%
   % 1. [R|G|A|E|X]; 2. cantidad; 3. descripción; 4. descripción para el csv (entre corchetes); 5. p/u; 6. dcto; 7. la descripción para el reporte.
   \@CalculaItem{##1}{##2}{##3}{##5}{##6}[##7]%
   % Si no es el primer ítem, entonces hace falta \\ para terminar el ítem anterior.
   \ifnum\value{NumItem}=1% Aquí, \@CalculaItem aumenta en 1 el contador NumItem, que si vale 1 significa que aún no hemos procesado la línea 1, es decir, no hay que hacer el salto de línea. La última línea se termina en otro lado.
   \else
    \@CalculaSeparaItemsExtraFinal
    \\[\@SeparaItemsExtraFinal]
    \ifconlineasha\hline\fi% Líneas para separar cada ítem, si hiciera falta.
   \fi
   % Primer -opcional- campo: El número del ítem.
   \ifbool{NumItem}{\David@XLeft{\@NumItemAuxCmd{\mbox{}\hfill \@LetraNumeros \arabic{NumItem}\hfill\mbox{}}}}{}%
   &
   % Segundo campo: Acá puede ir la cantidad (cdu) o la descripción (dcu y duc).
   \ifbool{SinCantidad}{}{%
    \ifbool{PrimeroDescripcion}
    {\@DescripcionAuxCmd{##3}{##1}}% dcu y duc.
    {\David@XLeft{\@CantidadAuxCmd{\mbox{}\hfill \@LetraNumeros \numprint{##2}\hfill\mbox{}}}}}% cdu.
   &
   % Tercer campo: Acá puede ir la descripción (cdu), la cantidad (dcu), o el precio unitario (duc).
   \ifbool{PrimeroDescripcion}{%
    \ifdcu
     \David@XRight{\@CantidadAuxCmd{\mbox{}\hfill \@LetraNumeros \numprint{##2}\hfill\mbox{}}}% dcu.
    \else
     \David@XRight{\@PrecioUAuxCmd{%
      \ifx##1X \hfill \@CaracterItemXVacio \hfill\mbox{}%
      \else
       \ifimpdesc\else\@EscribeTipoDeImpuesto{##1}\fi \hfill \@LetraNumeros \numprint{\PrecioU}
      \fi}}% duc.
    \fi}
   {\@DescripcionAuxCmd{##3}{##1}}% cdu.
   &
   % Cuarto campo: Acá puede ir el precio unitario (cdu y dcu) o la cantidad (duc).
   \ifbool{SinCantidad}{}{%
    \ifduc
     \David@XRight{\@CantidadAuxCmd{\mbox{}\hfill \@LetraNumeros \numprint{##2}\hfill\mbox{}}}% duc.
    \else
     \David@XRight{\@PrecioUAuxCmd{%
      \ifx##1X \hfill \@CaracterItemXVacio \hfill\mbox{}%
      \else
       \ifimpdesc\else\@EscribeTipoDeImpuesto{##1}\fi \hfill \@LetraNumeros \numprint{\PrecioU}
      \fi}}% dcu y cdu.
    \fi}%
   & % Quinto -opcional- campo: Subtotal (para luego descuentos).
   \ifbool{Subtotal}{%
    \David@XRight{\@PrecioSAuxCmd{%
     \ifx##1X \hfill \@CaracterItemXVacio \hfill\mbox{}%
     \else
      \ifimpdesc\else\@EscribeTipoDeImpuesto{##1}\fi \hfill \@LetraNumeros \numprint{\PrecioS}
     \fi}}}{}
   &
   % Sexto -opcional- campo: Descuento.
   \ifbool{Descuentos}{%
    \David@XRight{\@DescuentoAuxCmd{%
     \ifx##1X \hfill \@CaracterItemXVacio \hfill\mbox{}%
     \else
      \mbox{}\hfill \@LetraNumeros \numprint{\Dcto}\%\hfill\mbox{}%
     \fi}}}{}% Tenemos que usar un condicional booleano porque si no da problemas con el condicional \FPifeq.
   &
   % Séptimo -opcional- campo: IVA.
   \ifbool{IVA}{%
    \David@XRight{\@IVAAuxCmd{%
     \ifx##1X \hfill \@CaracterItemXVacio \hfill\mbox{}%
     \else
      \mbox{}\hfill \@LetraNumeros \numprint{\csname @Alicuota##1porc\endcsname}\%\hfill\mbox{}%
     \fi}}}{}% Tenemos que usar un condicional booleano porque si no da problemas con el condicional \FPifeq.
   &
   % Octavo campo: El total.
   \David@XRight{\@PrecioTAuxCmd{%
    \ifx##1X \hfill \@CaracterItemXVacio \hfill\mbox{}%
    \else
     \ifimpdesc\else\@EscribeTipoDeImpuesto{##1}\fi \hfill \@LetraNumeros \numprint{\PrecioT}
    \fi}}%
   % Escribimos al csv.
   \ifcsv
    \IfValueTF{##4}{\@EscribeItemsEnCSV{##1}{##2}{##4}}{\@EscribeItemsEnCSV{##1}{##2}{##3}}%
   \fi}%
  %
  % Comando para escribir en el .csv.
  \newcommand{\@EscribeItemsEnCSV}[3]{%
   \ifbool{SinCantidad}
   {\ifdescuentos
     \write\csv{"\unexpanded{##3}","\Dcto","\PrecioT","##1"}%
    \else
     \write\csv{"\unexpanded{##3}","\PrecioT","##1"}%
    \fi}
   {\if@cdu
     \ifdescuentos
      \write\csv{"##2","\unexpanded{##3}","\PrecioU","\Dcto","\PrecioT","##1"}%
     \else
      \write\csv{"##2","\unexpanded{##3}","\PrecioU","\PrecioT","##1"}%
     \fi
    \fi
    \ifdcu
     \ifdescuentos
      \write\csv{"\unexpanded{##3}","##2","\PrecioU","\Dcto","\PrecioT","##1"}%
     \else
      \write\csv{"\unexpanded{##3}","##2","\PrecioU","\PrecioT","##1"}%
     \fi
    \fi
    \ifduc
     \ifdescuentos
      \write\csv{"\unexpanded{##3}","\PrecioU","##2","\Dcto","\PrecioT","##1"}%
     \else
      \write\csv{"\unexpanded{##3}","\PrecioU","##2","\PrecioT","##1"}%
     \fi
    \fi}}%
  %
  % Usamos la definición de \label@factura dentro del entorno de la factura para que sea local aquí nada más.
  \let\label\label@factura
  \@LetraItems% Este es el tipo de letra que tendrán los ítems, y en este tipo de letra se define el \baselineskip de toda la factura.
  \setlength{\baselineskip@@}{\baselineskip}%
  \global\baselineskip@@=\baselineskip@@
  \@FacturaAuxInicioCmd
  \ifbool{SinLineasHoriExt}{}{\hline}%
  \@ZERO
  \setlength{\tabcolsep@@}{\tabcolsep}%
  \global\tabcolsep@@=\tabcolsep@@
  % Estamos en el número del ítem, si aplicara.
  \if@LineaDesdeInicio \zsaveposx{LineaNodoACoordX}\fi
  \ifbool{NumItem}{\@NumItemAuxCmd{\mbox{}\hfill \@LetraTitColumnas \@@NumItem \hfill\mbox{}}}{}%
  &
  % Avanzamos a la cantidad, o la descripción.
  \ifbool{SinCantidad}{}{\ifbool{PrimeroDescripcion}{\@DescripcionAuxCmd{\@LetraTitColumnas \@@Descripcion}{}}{\@CantidadAuxCmd{\mbox{}\hfill \@LetraTitColumnas \@@Cantidad \hfill\mbox{}}}}%
  &% Avanzamos hasta la descripción, la cantidad o el precio unitario.
  \ifbool{SinCantidad}{\@DescripcionAuxCmd{\@LetraTitColumnas \@@Descripcion}{}}
   {\if@cdu \@DescripcionAuxCmd{\@LetraTitColumnas \@@Descripcion}{}\fi
    \ifdcu \@CantidadAuxCmd{\mbox{}\hfill \@LetraTitColumnas \@@Cantidad \hfill\mbox{}}\fi
    \ifduc \@PrecioUAuxCmd{\mbox{}\hfill \@LetraTitColumnas \@@PrecioU \space (\Moneda)\hfill\mbox{}}\fi}%
  &% Avanzamos hasta el precio unitario, o la cantidad, o columna fantasma cuando es 'sincantidad'.
  \ifbool{SinCantidad}{}
   {%
    \mbox{}\hfill
    \ifduc
     \@CantidadAuxCmd{\mbox{}\hfill \@LetraTitColumnas \@@Cantidad \hfill\mbox{}}%
    \else
     \@PrecioUAuxCmd{\mbox{}\hfill \@LetraTitColumnas \@@PrecioU \space (\Moneda)\hfill\mbox{}}%
    \fi}%
  &% Avanzamos hasta el subtotal, que puede o no existir,
  \ifbool{Subtotal}{\@PrecioSAuxCmd{\mbox{}\hfill \@LetraTitColumnas \@@PrecioS \space (\Moneda)\hfill\mbox{}}}{}%
  &% Avanzamos hasta el descuento, que puede o no existir.
  \ifdescuentos \@DescuentoAuxCmd{\mbox{}\hfill \@LetraTitColumnas \@@Descuento \hfill\mbox{}}\fi
  &% Avanzamos hasta el IVA, que puede o no existir.
  \ifbool{IVA}{\@IVAAuxCmd{\mbox{}\hfill \@LetraTitColumnas \@@IVA \hfill\mbox{}}}{}
  &% Avanzamos hasta el precio total.
  \@PrecioTAuxCmd{\mbox{}\hfill \@LetraTitColumnas \ifbool{SinCantidad}{\@@Precio}{\@@PrecioT}\space (\Moneda)\hfill\mbox{}}%
  \if@LineaHastaFinal \zsaveposx{LineaNodoBCoordX}\fi
  \\
  \ifbool{SinLineaHoriTit}{}{\hline}%
  %
  #1% Aquí entran todos los items que el usuario declara dentro del ambiente.
  \@CalculaTotalesI
  \@CalculaSeparaItemsExtraFinal
  \\[\@SeparaItemsExtraFinal]% El último fin de línea se hace aquí.
  \ifconlineasha
   \ifdefvoid{\@NotaInterna}{%
    \ifbool{SinLineasHoriExt}{\if@ExpandeCuadro\hline\fi}{\hline}}{\hline}%
  \fi% Se verifica si hace falta una línea horizontal o no.
  \iffilas
   % Esto en lugar de \ifexpandecuadro para que no se haga una linea vacía en la tabla cuando la caja está vacía.
   \if@ExpandeCuadro
    \ifbool{PrimeroDescripcion}{&\usebox{\@CajaEspacioVacio}&&&&&&\\}{&&\usebox{\@CajaEspacioVacio}&&&&&\\}%
    \ifdefvoid{\@NotaInterna}{}
    {\ifbool{SinLineaHoriNI}{}{\hline}%
     \@ColumnasMultiColumnNotaIntAuxCmd}%
    \ifbool{SinLineasHoriExt}{}{\hline}% Siempre que hay relleno se pone la línea acá.
   \else
    \ifdefvoid{\@NotaInterna}{%
     \ifbool{SinLineasHoriExt}{}{%
      \ifconlineasha\else\hline\fi}}% Verificamos si hay que poner la línea horizontal. Si está 'conlineasha' entonces ya salió la línea.
     {\ifbool{SinLineaHoriNI}{}{\ifconlineasha\else\hline\fi}%
      \@ColumnasMultiColumnNotaIntAuxCmd
      \ifbool{SinLineasHoriExt}{}{\hline}}%
   \fi
  \else
   \ifsintotales
   \else
    \ifbool{SinLineaHoriTot}{}{\ifconlineasha\else\hline\fi}% Línea que cierra los ítems en la tabla.
   \fi
   % Agregamos el espacio espúreo.
   \if@ExpandeCuadro
    % Rellenamos la tabla con espacio espúreo, y le agregamos las líneas externas, si aplica.
    \ifbool{SinLineasVertExt}{\multicolumn{8}{c}{\usebox{\@CajaEspacioVacio}}}{\multicolumn{8}{|c|}{\usebox{\@CajaEspacioVacio}}}\\
   \fi
   % Si pedimos sin totalizaciones, entonces cerramos la tabla ya, si no, calculamos los totales y los escribimos.
   \ifsintotales
    \ifdefvoid{\@NotaInterna}{}{\ifbool{SinLineaHoriNI}{}{\hline}\@ColumnasMultiColumnNotaIntAuxCmd}%
    \ifbool{SinLineasHoriExt}{}{\hline}%
   \else
    % Modo de columnas: Abrimos la tabla para escribir los totales y le agregamos las líneas verticales externas, si aplica.
    \@ColumnasMultiColumnTotalesAuxCmd
    \ifagrupado
     \@EscribeTotalColAgrupado
    \else
     \if@Exento \@EscribeTotalCol{E}\fi
     \if@BIR\@EscribeTotalCol{R}\ifsinivaexpresado\else\@EscribeTotalColAlicuota{R}\fi\fi
     \@EscribeTotalCol{G}\ifsinivaexpresado\else\@EscribeTotalColAlicuota{G}\fi
     %% (quitado por no vigente)
     % \ifbool{Electro}{\@EscribeTotalColAlicuotaGDesc}{}%
     \if@BIA\@EscribeTotalCol{A}\ifsinivaexpresado\else\@EscribeTotalColAlicuota{A}\fi\fi
    \fi
    \ifsinivaexpresado\else\cline{2-2}\fi
    \hfill \@LetraTitTotal\@@Total & \@LetraNumTotal\@SignoMonetarioAux\numprint{\Total}\\
    \end{tabular}\endgroup
    \\
    \ifdefvoid{\@NotaInterna}{}{\ifbool{SinLineaHoriNI}{}{\hline}\@ColumnasMultiColumnNotaIntAuxCmd}%
    \ifbool{SinLineasHoriExt}{}{\hline}%
   \fi
  \fi
  \end{tabularx}%
  %
  % Generamos la macro \Total vacía para que no dé error en algún posible uso dentro del documento.
  \ifsintotales\gdef\Total{}\fi
  % Si pedimos no totalizar (sintotales), la opción filas queda desactivada.
  \iffilas
   \@LetraNumTotales
   \vspace*{\@SeparaVertical}\par
   \ifexpandecuadro\else\vfill\fi
   \@GeneraCajasTotalesFilas
   \newbox{\@CajaTotal}%
   \savebox{\@CajaTotal}{{\@LetraTitTotal\@@Total:} \@LetraNumTotal\@SignoMonetarioAux\numprint{\Total}}%
   %
   \ifnum \value{@IVANoGeneral}=0%
    \newbox{\@CajaGET}%
    \savebox{\@CajaGET}{\@EscribeTotalesFilasGET}%
    \ifdim \the\wd\@CajaGET < \textwidth
     \parbox{\textwidth}{\@EscribeTotalesFilasGET}%
    \else
     \hspace*{\@SepTotales}%
     \usebox{\@CajaBIAliG}\hspace*{\@SepTotales}%
     \if@Exento\usebox{\@CajaBIET}\else\usebox{\@CajaTotal}\fi
     \ifcentrado\hspace*{\@SepTotales}\fi
    \fi
   \fi
   \ifnum \value{@IVANoGeneral}=1%
    \newbox{\@CajaGxET}%
    \savebox{\@CajaGxET}{\@EscribeTotalesFilasGxET}%
    \ifdim \the\wd\@CajaGxET < \textwidth
     \@EscribeTotalesFilasGxET
    \else
     \hspace*{\@SepTotales}%
     \if@BIR\usebox{\@CajaBIAliR}\hspace*{\@SepTotales}\fi
     \usebox{\@CajaBIAliG}\hspace*{\@SepTotales}%
     \if@BIA\usebox{\@CajaBIAliA}\hspace*{\@SepTotales}\fi
     \\[\@SeparaFilas]\mbox{}%
     \ifcentrado\hspace*{\@SepTotales}\else\hfill\fi
     \if@Exento\usebox{\@CajaBIE}\hspace*{\@SeparaMinTotales}\fi\usebox{\@CajaTotal}%
     \ifcentrado\hspace*{\@SepTotales}\fi
    \fi
   \fi
   \ifnum \value{@IVANoGeneral}=2%
   \newbox{\@CajaRGAET}%
   \savebox{\@CajaRGAET}{\@EscribeTotalesFilasRGAET}%
    \ifdim \the\wd\@CajaRGAET < \textwidth
     \@EscribeTotalesFilasRGAET
    \else
     \newbox{\@CajaRGA}%
     \savebox{\@CajaRGA}{\@EscribeTotalesFilasRGA}%
     \ifdim \the\wd\@CajaRGA < \textwidth
      \@EscribeTotalesFilasRGA
      \\[\@SeparaFilas]\mbox{}%
      \ifcentrado\hspace*{\@SepTotales}\else\hfill\fi
      \if@Exento\usebox{\@CajaBIE}\hspace*{\@SeparaMinTotales}\fi\usebox{\@CajaTotal}%
      \ifcentrado\hspace*{\@SepTotales}\fi
     \else% Aquí casi cualquier cosa tiene aspecto feo.
      \CLASSWARNING{Es mejor no usar la opcio'n de totalizacio'n en filas en este caso, o disminuir los tama~nos de los tipos de letra para la totalizacio'n}%
      \centering{%
       \begin{tabular}{r@{\hspace*{\@SeparaMinTotales}}l}
        \usebox{\@CajaBIAliR} & \usebox{\@CajaBIAliA} \\[\@SeparaFilas]
        \usebox{\@CajaBIAliG} & \if@Exento\usebox{\@CajaBIET}\else\usebox{\@CajaTotal}\fi
       \end{tabular}}%
     \fi
    \fi
   \fi
  \fi
  \par
  \ifexpandecuadro
   \vspace*{-\@EspacioVacioInicial}%
   \vspace*{\@EspacioVacioFinal}%
   \zsaveposy{EspacioVacioInicial}%
  \fi
  \vfill
  \ifexpandecuadro\zsaveposy{EspacioVacioFinal}\fi
  \vspace*{\@SeparaVertical}\par
  \ifdefvoid{\@NotaExterna}{}{\parbox[t]{\linewidth}{\@LetraNotaExterna\@NotaExterna}\vspace*{\@SeparaVertical}\par}}
 
  \ifbool{SinCantidad}
  % Definamos todos los comandos \Item***. El descuento es opcional; por eso usamos G{0} en \NewDocumentCommand.
  {%
   \NewDocumentCommand{\Item}{m o m G{0} +o}{\@ITEM{G}{1}{#1}[#2]{#3}{#4}[#5]}% Ítem normal.
   \NewDocumentCommand{\ItemR}{m o m G{0} +o}{\global\@BIRtrue\global\Gtrue\@ITEM{R}{1}{#1}[#2]{#3}{#4}[#5]}% Ítem con IVA reducido.
   \NewDocumentCommand{\ItemA}{m o m G{0} +o}{\global\@BIAtrue\global\Gtrue\@ITEM{A}{1}{#1}[#2]{#3}{#4}[#5]}% Ítem con IVA adicional.
   \NewDocumentCommand{\ItemE}{m o m G{0} +o}{\global\@Exentotrue\global\Gtrue\@ITEM{E}{1}{#1}[#2]{#3}{#4}[#5]}% Ítem exento de IVA.
   \NewDocumentCommand{\ItemX}{m o +o}{\@ITEM{X}{1}{#1}[#2]{0}{0}[#3]}% Ítem sin precio.
   \let\ItemG\Item
   \ifcsv
    \ifdescuentos
     \write\csv{"\@@Descripcion","\@@Descuento","\@@Precio \space (\Moneda)","\@@IVA"}%
    \else
     \write\csv{"\@@Descripcion","\@@Precio \space (\Moneda)","\@@IVA"}%
    \fi
   \fi}
  {%
   \NewDocumentCommand{\Item}{m m o m G{0} +o}{\@ITEM{G}{#1}{#2}[#3]{#4}{#5}[#6]}% Ítem normal.
   \NewDocumentCommand{\ItemR}{m m o m G{0} +o}{\global\@BIRtrue\global\Gtrue\@ITEM{R}{#1}{#2}[#3]{#4}{#5}[#6]}% Ítem con IVA reducido.
   \NewDocumentCommand{\ItemA}{m m o m G{0} +o}{\global\@BIAtrue\global\Gtrue\@ITEM{A}{#1}{#2}[#3]{#4}{#5}[#6]}% Ítem con IVA adicional.
   \NewDocumentCommand{\ItemE}{m m o m G{0} +o}{\global\@Exentotrue\global\Gtrue\@ITEM{E}{#1}{#2}[#3]{#4}{#5}[#6]}% Ítem exento de IVA.
   \NewDocumentCommand{\ItemX}{m m o +o}{\@ITEM{X}{#1}{#2}[#3]{0}{0}[#4]}% Ítem sin precio.
   \let\ItemG\Item
   \ifcsv
    \if@cdu
     \ifdescuentos
      \write\csv{"\@@Cantidad","\@@Descripcion","\@@PrecioU \space (\Moneda)","\@@Descuento","\@@PrecioT \space (\Moneda)","\@@IVA"}%
     \else
      \write\csv{"\@@Cantidad","\@@Descripcion","\@@PrecioU \space (\Moneda)","\@@PrecioT \space (\Moneda)","\@@IVA"}%
     \fi
    \fi
    \ifdcu
     \ifdescuentos
      \write\csv{"\@@Descripcion","\@@Cantidad","\@@PrecioU \space (\Moneda)","\@@Descuento","\@@PrecioT \space (\Moneda)","\@@IVA"}%
     \else
      \write\csv{"\@@Descripcion","\@@Cantidad","\@@PrecioU \space (\Moneda)","\@@PrecioT \space (\Moneda)","\@@IVA"}%
     \fi
    \fi
    \ifduc
     \ifdescuentos
      \write\csv{"\@@Descripcion","\@@PrecioU \space (\Moneda)","\@@Cantidad","\@@Descuento","\@@PrecioT \space (\Moneda)","\@@IVA"}%
     \else
      \write\csv{"\@@Descripcion","\@@PrecioU \space (\Moneda)","\@@Cantidad","\@@PrecioT \space (\Moneda)","\@@IVA"}%
     \fi
    \fi
   \fi}

 \if@factura
  % Los descuentos totales no son compatibles -y no creo que lo vayan a ser- con el modo de totalización en filas.
  \iffilas
   \if@DescuentoTotal
    \CLASSWARNING{La opcio'n 'filas' no es compatible con los descuentos totales; sin efecto opcio'n 'filas'}
    \filasfalse
   \fi
  \fi
  \if@reverso
  \else
    \RenewEnviron{reverso}{}
  \fi
  \ifcotizacion\def\@Denominacion{\@@DenomCotiza}\else\def\@Denominacion{\@@DenomFactura}\fi%  Fijamos la denominación del documento paras las cotizaciones o facturas, que son casos particulares del modo 'factura'.
  \expandafter\geometry\expandafter{\@GeometriaFactura}%  La geometría debe aplicarse aquí.
  \AfterEndPreamble{%
   \addtolength{\@PosFechaX}{\oddsidemargin+1in}%  Le sumamos el margen izquierdo a las posiciones de las cosas; es importante que se haga en un \AfterEndPreamble antes de que empiecen a escribirse otras cosas en la página.
   \addtolength{\@PosDenomX}{\oddsidemargin+1in}%
   \addtolength{\@PosControlX}{\oddsidemargin+1in}%
   \addtolength{\@PosNotaFinalXuno}{\oddsidemargin+1in}%
   \addtolength{\@PosFirmasXuno}{\oddsidemargin+1in}%
   \setlength{\parindent}{0pt}%
   \setlength{\lineskip}{0pt}%
   \setlength{\parskip}{0pt plus 1pt}%
   \expandafter\setlength{\@SepTotales}{\the\@SeparaMinTotales plus 1fill}% \@SeparaMinTotales es el mínimo espacio entre los totales en el modo 'filas', y SepTotales es una dimensión interna, estirable, con ese como su mínimo.
   \thispagestyle{empty}%
   \pagenumbering{roman}%
   \@EncabezadoFactura
   \@EscribeNotaPrevia}
   \AtBeginShipoutNext{% Antes abría aquí un \AtBeginShipoutNext cuando había reporte y un \AtEndDocument cuando no. No es necesario eso. También lo modifiqué en \FirmaFactura.
    \@CargaFondoFactura
    \@PosicionaFechas
    \ifdenom\@PosicionaDenom\fi
    \ifprefactura\@PosicionaNumControl\fi
    \ifnum\value{@Firmas}=0\else\@PosicionaFirmas\fi
    \ifdefvoid{\@NotaFinal}{}{\@PosicionaNotaFinal}}
  \iflinea
   % Para determinar las coordenadas de la línea hay que usar \AtBeginShipoutNext porque si no se hace en el 'shipout' de la página se genera un espacio en blanco extra.
   \AtBeginShipoutNext{\iflinea\@DibujaLinea\fi}
    \iffilas
     \ifbool{SinLineasVertInt}{\@LineaDesdeIniciotrue\@LineaHastaFinaltrue}{\@LineaDescripciontrue}%
    \else
     \ifsintotales
      \@LineaDesdeIniciotrue\@LineaHastaFinaltrue
     \else
      \@LineaDesdeIniciotrue\@LineaHastaTotaltrue
     \fi
    \fi
  \fi
  %% El comando para la descripción del servicio prestado.
  \NewDocumentCommand{\Descripcion}{+m}{%
   {% Abrimos un grupo adicional para que la redefinición del entorno 'tikzpicture' sea local.
    \@LetraDescripcion
    % Cuando la línea tiene grosor cero, hacemos que el entorno 'tikzpicture' sea "fake", y redefinimos \node para que obvie los parámetros que le dimos como argumento opcional.
    \ifdim\@GrosorLineaDescripcion=0pt%
     \setlength{\@SeparaInternaDescripcion}{0pt}\setlength{\@EsquinaLineaDescripcion}{0pt}\setlength{\@SombraDescripcion}{0pt}%
     \renewenvironment{tikzpicture}{\NewDocumentCommand{\node}{o +m}{####2}}{}%
    \fi
    \begin{tikzpicture}
     % Expandimos después para poderle pasar el contenido de la macro como argumento opcional a \node.
     \expandafter\node\expandafter[\@DescripcionTIKZ]{%
      \ifbool{ExpAbajo}
       {\parbox[t][\textheight-\pagetotal-2\@SeparaInternaDescripcion-\@GrosorLineaDescripcion-\@SombraDescripcion]
        {\textwidth-2\@SeparaInternaDescripcion-\@GrosorLineaDescripcion-\@SombraDescripcion}}%
       {\parbox[t]{\textwidth-\@SombraDescripcion-2\@SeparaInternaDescripcion-2\@GrosorLineaDescripcion}}%
      {\setlength{\parindent}{0.5cm}%
       \setlength{\parskip}{2pt plus 8pt}%
       \ifbool{ExpAbajo}{\vfill}{}%
        #1%
       \ifbool{ExpAbajo}{\vfill}{}}}%
     \ifdim\@GrosorLineaDescripcion=0pt\else ;\fi% El fin del nodo va cuando hay el entorno tikzpicture original.
    \end{tikzpicture}
    \par\vspace*{\@SombraDescripcion}}}%
 \else
  \RenewEnviron{factura}{%
   \@ZERO
   \let\label\label@factura
   \let\@ITEM\@CalculaItem
   \BODY
   \@CalculaTotalesI}
  \RenewEnviron{reverso}{}
  \NewDocumentCommand{\Descripcion}{+m}{}
 \fi

 \if@reporte
  \ifdefvoid{\@InfoPagina}{\let\@InfoPagina\@InfoPaginaNoTitulo}{}%
  \if@factura
  \else
   \expandafter\geometry\expandafter{\@GeometriaReporte}%  La geometría del reporte debe aplicarse aquí cuando sólo hay reporte, así se puede establecer el tamaño del papel.
  \fi
  \NewEnviron{reporte}{%
   \if@factura
    \expandafter\newgeometry\expandafter{\@GeometriaReporte}\cleardoublepage%  La geometría debe aplicarse aquí cuando también hay factura, pero no se puede cambiar el tamaño del papel.
   \fi
   \let\headwidth\textwidth
   \@CargaFondoReporte
   \@EncabezadoReporte
   \ifdefvoid{\@EstiloPagReporteResto}{%
    \if@FondoReporteStar%  Si no usamos \EstiloPagReporte entonces se fijará 'soloabajo' cuando se use \FondoReporte* para que sólo use marcas de pie de página, y si no se usó \FondoReporte* entonces se fijará 'plain' que usa cabeceras y pies de página.
     \pagestyle{soloabajo}%
    \else
     \pagestyle{plain}%
    \fi}
    {\pagestyle{\@EstiloPagReporteResto}}%  Si usamos \EstiloPagReporte en su argumento obligatorio entonces fijamos ese estilo de página.
   \ifdefvoid{\@EstiloPagReporteUno}{%
    \if@FondoReporte%  Si no le damos a \EstiloPagReporte su argumento opcional, entonces se fijará 'solonumpag' como el estilo de página para la primera página, pero solamente si usamos \FondoReporte, ya que el estilo de página predeterminado al usar \FondoReporte* ya se acaba de fijar y no necesita el caso especial de la primera hoja (\thispagestyle).
     \if@FondoReporteStar
     \else
      \thispagestyle{solonumpag}%
     \fi
    \fi}
    {\thispagestyle{\@EstiloPagReporteUno}}%  Si usamoe el argumento opcional de \EstiloPagReporte establecemos ese estilo de página.
   \pagenumbering{arabic}%
   \setlength{\parindent}{0.666cm}%
   \setlength{\parskip}{4pt plus 6pt minus 2pt}% separación entre párrafos en el reporte.
   \vskip\baselineskip
   \BODY\clearpage}% espacio antes de empezar con el reporte, después del título -si lo hubiera-.
 \else
  \NewEnviron{reporte}{}
 \fi}

%% Comando para el diseño de la forma libre.
\newcommand{\@CargaFondoFactura}{%
 \if@FondoFactura
  \begin{textblock*}{\paperwidth}(0pt,0pt)%
   \includegraphics{\@FondoFactura}%
  \end{textblock*}%
 \fi}

%% Comando para el fondo del reporte.
\newcommand{\@CargaFondoReporte}{%
 \begin{textblock*}{\paperwidth}(0pt,0pt)% Tuve que abrir el textblock* aquí para que aplique siempre; de esta manera se corrige un defecto que se produce cuando se usa \vspace* al inicio de la página (dentro de \@EncabezadoReporte).
  \if@FondoReporte
   \if@FondoReporteStar
    \AddToShipoutPictureBG{\includegraphics{\@FondoReporte}}%
   \else
     \noindent\includegraphics{\@FondoReporte}%
   \fi
  \fi
 \end{textblock*}}

%% Comando que coloca los espacios para firmar.
\newcommand{\@PosicionaFirmas}{%
 \savebox{\@CajaFirmas}{%
  \parbox{\textwidth+\oddsidemargin+1in-\@PosFirmasXuno-\@PosFirmasXdos}{\@EstiloFirmas}}%
 \ifbool{FirmasYnegativo}{\addtolength{\@PosFirmasY}{\paperheight-\ht\@CajaFirmas-\dp\@CajaFirmas}}{}%
 \begin{textblock*}{\wd\@CajaFirmas}(\@PosFirmasXuno,\@PosFirmasY)%
  \usebox{\@CajaFirmas}%
 \end{textblock*}}

%% Comando que coloca la nota final.
\newcommand{\@PosicionaNotaFinal}{%
 \savebox{\@CajaNotaFinal}{%
  \parbox{\textwidth+\oddsidemargin+1in-\@PosNotaFinalXuno-\@PosNotaFinalXdos}{\@EscribeNotaFinal}}%
 \ifbool{NotaFinalYnegativo}{\addtolength{\@PosNotaFinalY}{\paperheight}}{}%
 \ifbool{NotaFinalYarriba}{\addtolength{\@PosNotaFinalY}{-\ht\@CajaNotaFinal-\dp\@CajaNotaFinal}}{}% Aquí la caja abre hacia abajo, a menos de que usemos \PosNotaFinal*
 \begin{textblock*}{\wd\@CajaNotaFinal}(\@PosNotaFinalXuno,\@PosNotaFinalY)%
  \usebox{\@CajaNotaFinal}%
 \end{textblock*}}

%% Comando que coloca el número de control.
\newcommand{\@PosicionaNumControl}{%
 \savebox{\@CajaControl}{\@EstiloNumControl}%
 \ifbool{ControlXnegativo}{\addtolength{\@PosControlX}{\textwidth-\wd\@CajaControl}}{}%
 \ifbool{ControlYnegativo}{\addtolength{\@PosControlY}{\paperheight}}{}%
 \ifbool{ControlYarriba}{\addtolength{\@PosControlY}{-\ht\@CajaControl-\dp\@CajaControl}}{}%
 \begin{textblock*}{\wd\@CajaControl}(\@PosControlX,\@PosControlY)%
  \usebox{\@CajaControl}%
 \end{textblock*}}

%% Comando que coloca la denominación del documento.
\newcommand{\@PosicionaDenom}{%
 \savebox{\@CajaDenom}{\@EstiloDenom}%
 \ifbool{DenomXnegativo}{\addtolength{\@PosDenomX}{\textwidth-\wd\@CajaDenom}}{}%
 \ifbool{DenomYnegativo}{\addtolength{\@PosDenomY}{\paperheight}}{}%
 \ifbool{DenomYarriba}{\addtolength{\@PosDenomY}{-\ht\@CajaDenom-\dp\@CajaDenom}}{}%
 \begin{textblock*}{\wd\@CajaDenom}(\@PosDenomX,\@PosDenomY)%
  \usebox{\@CajaDenom}%
 \end{textblock*}}

%% Comando que coloca la(s) fecha(s).
\newcommand{\@PosicionaFechas}{%
 \savebox{\@CajaFecha}{\@EstiloFechas}%
 \ifbool{FechaXnegativo}{\addtolength{\@PosFechaX}{\textwidth-\wd\@CajaFecha}}{}%
 \ifbool{FechaYnegativo}{\addtolength{\@PosFechaY}{\paperheight}}{}%
 \ifbool{FechaYarriba}{\addtolength{\@PosFechaY}{-\ht\@CajaFecha-\dp\@CajaFecha}}{}%
 \begin{textblock*}{\wd\@CajaFecha}(\@PosFechaX,\@PosFechaY)%
  \usebox{\@CajaFecha}%
 \end{textblock*}}

%% Comando que coloca la identificación del reporte al inicio.
\newcommand{\@EscribeDatosReporte}{%
 {\@LetraEncReporte
  \setlength{\parindent}{0pt}%
  {\bfseries \@@Proveedor} \@Proveedor \par
  {\bfseries \@@Cliente} \@RazonSocial \par
  \ifdefvoid{\@Contacto}{}{{\bfseries \@@Contacto} \@Contacto \par}%
  \ifdefvoid{\@TituloReporte}{}{\vspace*{\baselineskip}{\@LetraTitReporte \centering \@TituloReporte \par}}}}

%% Comando que coloca las dos cajas del encabezado: la de los datos del cliente y la del resumen.
\newcommand{\@EscribeDatos}{%
 {\@LetraEncFactura% Abrimos un nuevo grupo para que el cambio de tipografía y la redefinición del entorno 'tikzpicture' sean locales.
  \ifdim\@GrosorLineaEncabezado=0pt%
   \setlength{\@SeparaInternaEncabezado}{0pt}\setlength{\@EsquinaLineaEncabezado}{0pt}\setlength{\@SombraEncabezado}{0pt}%
   \renewenvironment{tikzpicture}{\NewDocumentCommand{\node}{o +m}{####2}}{}%
  \fi
  \begin{tikzpicture}%
   \expandafter\node\expandafter[\@DatosTIKZ]{%
   \parbox{\textwidth-\@SombraEncabezado-2\@SeparaInternaEncabezado-\@GrosorLineaEncabezado}{\@DatosAux}}%
    \ifdim\@GrosorLineaEncabezado=0pt\else ;\fi
  \end{tikzpicture}%
  \par\vspace*{\@SombraEncabezado}}}%  Para que textblock* esté en modo vertical.

\newcommand{\@EscribeDatosResumen}{%
 {\@LetraEncFactura% Abrimos un nuevo grupo para que el cambio de tipografía y la redefinición del entorno 'tikzpicture' sean locales.
  \ifdim\@GrosorLineaEncabezado=0pt%
   \setlength{\@SeparaInternaEncabezado}{0pt}\setlength{\@EsquinaLineaEncabezado}{0pt}\setlength{\@SombraEncabezado}{0pt}%
   \renewenvironment{tikzpicture}{\NewDocumentCommand{\node}{o +m}{####2}}{}%
  \fi
  \savebox{\@CajaDatos}{%
   \parbox[t]{\textwidth/2-\@SeparaDatosResumen/2-2\@SeparaInternaEncabezado-\@GrosorLineaEncabezado-\@SombraEncabezado}{\@DatosAux*}}%
  \savebox{\@CajaResumen}{%
   \parbox[t]{\textwidth/2-\@SeparaDatosResumen/2-2\@SeparaInternaEncabezado-\@GrosorLineaEncabezado-\@SombraEncabezado}{\@ResumenAux}}%
  \setlength{\@TamDatos}{\ht\@CajaDatos+\dp\@CajaDatos}%
  \setlength{\@TamResumen}{\ht\@CajaResumen+\dp\@CajaResumen}%
  \ifdim\@TamDatos>\@TamResumen
   \setlength{\@TamFinalEncabezado}{\@TamDatos}%
  \else
   \setlength{\@TamFinalEncabezadoado}{\@TamResumen}%
  \fi
  \begin{tikzpicture}%
   \expandafter\node\expandafter[\@DatosTIKZ]{%
    \parbox[t][\@TamFinalEncabezado]{\textwidth/2-\@SeparaDatosResumen/2-2\@SeparaInternaEncabezado-\@GrosorLineaEncabezado-\@SombraEncabezado}{%    
    \@DatosAux*}}%
   \ifdim\@GrosorLineaEncabezado=0pt\else ;\fi
  \end{tikzpicture}%
  \hspace*{\@SeparaDatosResumen}\hspace*{\@SombraEncabezado}%
  \begin{tikzpicture}%
   \expandafter\node\expandafter[\@ResumenTIKZ]{%
    \parbox[t][\@TamFinalEncabezado]{\textwidth/2-\@SeparaDatosResumen/2-2\@SeparaInternaEncabezado-\@GrosorLineaEncabezado-\@SombraEncabezado}{%
     \@ResumenAux}}%
   \ifdim\@GrosorLineaEncabezado=0pt\else ;\fi
  \end{tikzpicture}%
  \par\vspace*{\@SombraEncabezado}}}%  El \par es para que textblock* esté en modo vertical.

%% Comando que coloca la caja para la nota previa, con la posibilidad de TikZ.
\newcommand{\@EscribeNotaPrevia}{%
 {% Abrimos un nuevo grupo para que la redefinición del entorno 'tikzpicture' sea local.
  \baselineskip0pt
  \ifdefvoid{\@NotaPrevia}{}{%
   \ifdim\@GrosorLineaNotaPrevia=0pt%
    \setlength{\@SeparaInternaNotaPrevia}{0pt}\setlength{\@EsquinaLineaNotaPrevia}{0pt}\setlength{\@SombraNotaPrevia}{0pt}%
    \renewenvironment{tikzpicture}{\NewDocumentCommand{\node}{o +m}{####2}}{}%
   \fi
   \begin{tikzpicture}%
    \expandafter\node\expandafter[\@NotaPreviaTIKZ]{%
    \parbox{\textwidth-\@SombraNotaPrevia-2\@SeparaInternaNotaPrevia-\@GrosorLineaNotaPrevia}{\@EstiloNotaPrevia}}%
     \ifdim\@GrosorLineaNotaPrevia=0pt\else ;\fi
   \end{tikzpicture}%
   \par\vspace*{\@SombraNotaPrevia}\vspace*{\@SeparaVertical}%
   \ifexpandecuadro\else\vfill\fi}}}

%% Comando que coloca la caja para la nota final, con la posibilidad de TikZ.
\newcommand{\@EscribeNotaFinal}{%
 {% Abrimos un nuevo grupo para que la redefinición del entorno 'tikzpicture' sea local.
  \ifdim\@GrosorLineaNotaFinal=0pt%
   \setlength{\@SeparaInternaNotaFinal}{0pt}\setlength{\@EsquinaLineaNotaFinal}{0pt}\setlength{\@SombraNotaFinal}{0pt}%
   \renewenvironment{tikzpicture}{\NewDocumentCommand{\node}{o +m}{####2}}{}%
  \fi
  \begin{tikzpicture}%
   \expandafter\node\expandafter[\@NotaFinalTIKZ]{%
   \parbox{\textwidth-\@PosNotaFinalXuno-\@PosNotaFinalXdos+\oddsidemargin+1in-\@SombraNotaFinal-2\@SeparaInternaNotaFinal-\@GrosorLineaNotaFinal}{\@EstiloNotaFinal}}%
    \ifdim\@GrosorLineaNotaFinal=0pt\else ;\fi
  \end{tikzpicture}%
  \par\vspace*{\@SombraNotaFinal}}}

%% Comandos auxiliares para los datos y el resumen.
%% En el caso de los datos, la versión \@DatosAux se ejecuta cuando no hay resumen, y la versión \@DatosAux* se ejecuta cuando hay resumen.
\NewDocumentCommand{\@DatosAux}{s}{%
 \hfill {\@LetraTitEnc \@@Datos}\hfill \mbox{}\par
 \IfBooleanT{#1}{\vfill}%
 {\bfseries \if@Nombre \@@Nombre \else \@@RazonSocial \fi} \@RazonSocial
 \IfBooleanTF{#1}{\par}{\dotfill}%
 {\bfseries \@@RIF} {\ttfamily \@RIF}\par
 \ifdefvoid{\@Contacto}{}{{\bfseries \@@Contacto} \@Contacto \par}%
 {\bfseries \@@Direccion} \@Direccion \par
 \ifdefvoid{\@Telefono}{}{{\bfseries \@@Telefono} {\ttfamily \@Telefono}}%
 \ifdefvoid{\@Email}
  {\par}
  {\IfBooleanTF{#1}
   {\par}
   {\ifdefvoid{\@Telefono}{}{\hspace*{\@SeparaTelefonoEmail}\@CaracterSeparaTelefonoEmail\hspace*{\@SeparaTelefonoEmail}}}%
  {\bfseries \@@Email} {\ttfamily \@Email}}%
 \IfBooleanT{#1}{\vfill}}

\newcommand{\@ResumenAux}{\hfill {\@LetraTitEnc \@@Resumen}\hfill \mbox{}\vfill\@Resumen\vfill}

%% Comando que coloca la línea diagonal.
\newcommand{\@DibujaLinea}{%
 \setlength{\@LineaNodoXo}{\zposx{LineaNodoACoordX} sp}%
 \setlength{\@LineaNodoYo}{\zposy{LineaNodoBCoordY} sp}%
 \setlength{\@LineaNodoXf}{\zposx{LineaNodoBCoordX} sp}%
 \setlength{\@LineaNodoYf}{\zposy{LineaNodoBCoordY} sp}%
 \addtolength{\@LineaNodoYo}{\@EspacioVacioInicial-\@EspacioVacioFinal}%
 \if@LineaDesdeInicio
  \addtolength{\@LineaNodoXo}{-\tabcolsep@@}%
 \fi
 \if@LineaHastaFinal
  \addtolength{\@LineaNodoXf}{\tabcolsep@@}%
 \fi
 \iffilas
  \addtolength{\@LineaNodoYf}{-.3\baselineskip@@}%
  \if@LineaDescripcion
   \addtolength{\@LineaNodoXo}{-\tabcolsep@@}%
   \addtolength{\@LineaNodoXf}{\tabcolsep@@}%
  \fi
 \else
  \ifsintotales
  \addtolength{\@LineaNodoYf}{-.3\baselineskip@@}%
  \else
   \if@DescG \addtolength{\@LineaNodoYo}{2\baselineskip@@}\fi
   \if@DescR \if@BIR \addtolength{\@LineaNodoYo}{2\baselineskip@@}\fi\fi
   \if@DescA \if@BIA \addtolength{\@LineaNodoYo}{2\baselineskip@@}\fi\fi
   \if@DescE \if@Exento \addtolength{\@LineaNodoYo}{2\baselineskip@@}\fi\fi
   \addtolength{\@LineaNodoYo}{\baselineskip@@}%
   \addtolength{\@LineaNodoYf}{-.3\baselineskip@@}%
   \ifsinivaexpresado
   \else
    \addtolength{\@LineaNodoYo}{2\baselineskip@@}%
    \if@BIR\addtolength{\@LineaNodoYo}{2\baselineskip@@}\fi
    \if@BIA\addtolength{\@LineaNodoYo}{2\baselineskip@@}\fi
    \if@Exento\addtolength{\@LineaNodoYo}{\baselineskip@@}\fi
   \fi
  \fi
 \fi
 \iflinea
  \begin{textblock*}{\paperwidth-\@LineaNodoXo}(\@LineaNodoXo,\paperheight-\@LineaNodoYo)%
   \begin{tikzpicture}%
    \draw (\@LineaNodoXo,\@LineaNodoYo) -- (\@LineaNodoXf,\@LineaNodoYf);%
   \end{tikzpicture}%
  \end{textblock*}%
 \fi}

%% Escribimos los totales al archivo csv.
\newcommand{\@EscribeTotalesEnCSV}{%
 \ifcsv
  \if@Exento
   \if@DescE
    \immediate\write\csv{Totales: \@@SubtotalE: \Moneda \SubtE^^JTotales: \@@DescuentoE: \Moneda \DescE}%
   \fi
   \immediate\write\csv{Totales: \@@BIE: \Moneda \BIE}%
  \fi
  \if@BIR
   \if@DescE
    \immediate\write\csv{Totales: \@@SubtotalR: \Moneda \SubtR^^JTotales: \@@DescuentoR: \Moneda \DescR}%
   \fi
   \immediate\write\csv{Totales: \@@BIR: \Moneda \BIR^^JTotales: \@@AlicuotaR: \Moneda \AlicuotaR}%
  \fi
  \if@DescG
   \immediate\write\csv{Totales: \@@SubtotalG: \Moneda \SubtG^^JTotales: \@@DescuentoG: \Moneda \DescG}%
  \fi
  \immediate\write\csv{Totales: \@@BIG: \Moneda \BIG^^JTotales: \@@AlicuotaG: \Moneda \AlicuotaG}%
  %% (quitado por no vigente)
  % \FPifeq\AlicuotaGDesc{0}\else\immediate\write\csv{Totales: \@@AlicuotaGDesc: \Moneda \AlicuotaGDesc}\fi
  \if@BIA
   \if@DescA
    \immediate\write\csv{Totales: \@@SubtotalA: \Moneda \SubtA^^JTotales: \@@DescuentoA: \Moneda \DescA}%
   \fi
   \immediate\write\csv{Totales: \@@BIA: \Moneda \BIA^^JTotales: \@@AlicuotaA: \Moneda \AlicuotaA}%
  \fi
  \immediate\write\csv{Totales: \@@Total: \Moneda \Total}%
  \immediate\closeout\csv
 \fi}

%% Hubo que redefinir \cleardoublepage agregándole \thispagestyle{empty} porque si no queda una página en blanco con cabeceras de página usando twoside.
\def\cleardoublepage{%
 \clearpage
 \if@twoside
  \ifodd\c@page
   \else
    \hbox{}\thispagestyle{empty}\newpage
    \if@twocolumn
     \hbox{}\newpage
    \fi
   \fi
 \fi}

%</class>
%<*def>
%% Diseño hecho en TikZ para los datos del cliente, el resumen, la nota previa, la descripción y la nota final.
\def\@DatosTIKZ{%
 draw=black, fill=white, inner sep=\@SeparaInternaEncabezado, line width=\@GrosorLineaEncabezado, rounded corners=\@EsquinaLineaEncabezado, copy shadow={top color=black, bottom color=black, shadow xshift=\@SombraEncabezado, shadow yshift=-\@SombraEncabezado}}

\let\@ResumenTIKZ\@DatosTIKZ

\def\@NotaPreviaTIKZ{%
 draw=black, fill=white, inner sep=\@SeparaInternaNotaPrevia, line width=\@GrosorLineaNotaPrevia, rounded corners=\@EsquinaLineaNotaPrevia, copy shadow={top color=black, bottom color=black, shadow xshift=\@SombraNotaPrevia, shadow yshift=-\@SombraNotaPrevia}}

\def\@DescripcionTIKZ{%
 draw=black, fill=white, inner sep=\@SeparaInternaDescripcion, line width=\@GrosorLineaDescripcion, rounded corners=\@EsquinaLineaDescripcion, copy shadow={top color=black, bottom color=black, shadow xshift=\@SombraDescripcion, shadow yshift=-\@SombraDescripcion}}

\def\@NotaFinalTIKZ{%
 draw=black, fill=white, inner sep=\@SeparaInternaNotaFinal, line width=\@GrosorLineaNotaFinal, rounded corners=\@EsquinaLineaNotaFinal, copy shadow={top color=black, bottom color=black, shadow xshift=\@SombraNotaFinal, shadow yshift=-\@SombraNotaFinal}}

%% El diseño de las distintas partes de la hoja.
\newcommand{\@EstiloFirmas}{
 \@LetraFirmas
 \ifnum \value{@Firmas}=1
 \else
  \parbox{\@TamLineaFirmas}{
   \rule{\@TamLineaFirmas}{\@GrosorLineaFirmas}
   \centering \ifdefvoid{\@Conforme}{\@@Conforme}{\@Conforme}}
 \fi
 \hfill
 \parbox{\@TamLineaFirmas}{
  \rule{\@TamLineaFirmas}{\@GrosorLineaFirmas}
  \centering \ifdefvoid{\@Emisor}{\ifcotizacion \@@Emisor \else \@@Pagado \fi}{\@Emisor}}}

\newcommand{\@EstiloNotaPrevia}{\@LetraNotaPrevia \@NotaPrevia}

\newcommand{\@EstiloNotaFinal}{\@LetraNotaFinal \@NotaFinal}

\newcommand{\@EstiloNumControl}{%
 \renewcommand{\baselinestretch}{0}\selectfont
 \begin{tabular}{@{\extracolsep{0pt}}l}
  \@LetraDenom \@@Control \\[.4em]
  {\@LetraDenom\@NumSerieControl-}\space\@LetraNumeracion\@NumControl \\
 \end{tabular}}

\newcommand{\@EstiloDenom}{%
 \@LetraDenom\@Denominacion
 \ifsinnumero
 \else
  \space\space{\@LetraNumeracion N\textordmasculine\space\@Num}%
 \fi}

\newcommand*{\@EstiloFechas}{%
 \renewcommand{\baselinestretch}{0}%
 \setlength{\tabcolsep}{0pt}%
 \@LetraFecha\begin{tabular}{r@{\extracolsep{1ex}}l@{\extracolsep{0pt}}}
 \@@Fecha & \ttfamily\@Fecha \\[.3em]
 \ifvencimiento \@@FechaVencimiento & \ttfamily\@FechaVencimiento \\ \else \vspace*{-.3em}\fi
 \end{tabular}}

\newcommand{\@InfoPaginaNoTitulo}{\@LetraInfoPagina Reporte correspondiente a la factura n\'umero \@Num \space de fecha \@Fecha}

\newcommand{\@InfoPaginaTitulo}{\@LetraInfoPagina \@TituloReporte \space - \@Proveedor \space - \@Fecha}

%% Estilos de página: normal, sólo cabeceras, sólo pies de página y sólo número de página al pie.
\fancypagestyle{plain}{%
 \expandafter\ps@fancy
 \fancyhf{}%
 \renewcommand{\headrulewidth}{\@GrosorLineaInfoPag}%
 \fancyhead[LEO]{\@InfoPagina}%
 \fancyfoot[C]{\thepage}}

\fancypagestyle{soloarriba}{%
 \expandafter\ps@fancy
 \fancyhf{}%
 \renewcommand{\headrulewidth}{\@GrosorLineaInfoPag}%
 \fancyhead[LO]{\@InfoPagina}%
 \fancyhead[RO]{\thepage}%
 \if@twoside
  \fancyhead[LE]{\thepage}%
  \fancyhead[RE]{\@InfoPagina}%
 \fi}

\fancypagestyle{soloabajo}{%
 \expandafter\ps@fancy
 \fancyhf{}%
 \renewcommand{\headrulewidth}{0pt}%
 \fancyfoot[LO]{\@InfoPagina}%
 \fancyfoot[RO]{\thepage}%
 \if@twoside
  \fancyfoot[RE]{\@InfoPagina}%
  \fancyfoot[LE]{\thepage}%
 \fi}

\fancypagestyle{solonumpag}{%
 \expandafter\ps@fancy
 \fancyhf{}%
 \renewcommand{\headrulewidth}{0pt}%
 \fancyfoot[C]{\thepage}}

%</def>
%<*!class&!def&!firma&!fondo&!membrete&!aux&!batchfile>
%% Descripción de este ejemplo:
%
%%<desc>
%<beispiel>%%  Adaptado a otra legislación
%<factuno|factdos|facttres|factcuatro|factcinco|factseis|factsiete|factocho|factnueve|factdiez|repuno|repdos>%%  Escribimos una factura
%<reptres>%%  No escribimos factura ni usamos las macros que definen valores para su uso
%<factnueve|factdiez|repuno>%%  Escribimos en el reverso de la factura
%<prefac|beispiel>%%  (prefactura) Escribimos una prefactura
%<cotiza>%%  (cotizacion) Escribimos una cotización
%<repuno|repdos|reptres>%%  Escribimos un reporte
%<repuno>%%  (sinreverso) Escribimos el reverso de la factura pero sólo producimos su anverso
%<repuno>%%  (conreporte) Generamos factura y reporte
%<repdos>%%  (soloreporte) Escribimos factura y reporte, pero sólo producimos el reporte
%<reptres>%%  (soloreporte) Generamos solamente el reporte
%<factcinco|factocho>%%  Escribimos factura y reporte, pero sólo producimos la factura
%
%<factdiez>%%  Establecemos algunas opciones de la clase a través de comandos
%
%<factuno>%%  Hacemos una factura básica con todo en sus valores por omisión, no ajustamos nada
%
%<reptres>%%  (twoside) El reporte se imprime en las dos caras del papel
%
%<factcinco>%%  (sincantidad) Cuadro de la factura sin la columna de cantidades
%<factdos|factseis>%%  (numitem) Con la columna para enumerar los ítems
%<factdos|factseis>%%  (descuentos) Con la columna para descuentos individuales
%<factseis>%%  (subtotal) Con la columna extra para subtotal
%<factcinco>%%  (iva) Con la columna para el IVA
%
%<facttres>%%  (agrupado) Agrupados los totales por tipo y no por gravamen
%<factcuatro|factcinco|factsiete>%%  (filas) Totalización en modo de filas
%<factsiete>%%  (centrado) Centrados los totales en el modo de filas
%
%<factdos>%%  (dcu) Primero la descripción, luego la cantidad, luego el precio unitario
%<factcuatro>%%  (duc) Primero la descripción, luego el precio unitario, luego la cantidad
%<factcinco|factseis|factsiete>%%  (denom) Escribe la denominación (“FACTURA”) y el número
%<cotiza>%%  (sinnumero) Escribe la denominación (“COTIZACIÓN”) pero sin número
%<cotiza>%%  (sintotales) No totaliza, para que el cliente tenga un presupuesto por distintos ítems que luego elegirá entre ellos, es decir, una lista de precios
%
%<factdos|factseis>%%  (sinlineasve) Sin líneas verticales externas
%<facttres|factcuatro>%%  (sinlineasvi) Sin líneas verticales internas
%<factdiez>%%  (sinlineasv) Sin líneas verticales
%<factdos>%%  (sinlineahtot) Sin línea horizontal interna después de los ítems y antes de los totales
%<factcuatro>%%  (sinlineahtit) Sin línea horizontal interna después del título de las columnas
%<factseis>%%  (sinlineashi) Sin líneas horizontales internas
%<factsiete>%%  (sinlineashe) Sin líneas horizontales externas
%<factnueve>%%  (sinlineahni)  Sin línea horizontal interna después de la nota interna.
%<factdiez>%%  (sinlineash) Sin líneas horizontales
%<facttres|factcinco>%%  (conlineasha) Separado cada ítem con una línea horizontal
%
%<factcinco|factocho>%%  (sinexpandir) Sin expandir el cuadro de la factura ni la descripción
%<facttres|factcuatro|prefac>%%  (expandecuadro) Expande el cuadro de la factura con espacio en blanco
%<prefac|factcuatro>%%  (linea) Con línea para inutilizar el espacio en blanco
%
%<facttres>%%  (impdesc) Con el tipo de impuesto como parte de la descripción y no de los precios
%<factdos>%%  (centradesc) Centra la descripción en los elementos en la factura
%<factsiete>%%  (sinmarcas) Cuando la descripción de un ítem ocupa más que una línea, no pone el carácter que indica continuación en la siguiente línea ni los caracteres en las columnas vacías
%<factcinco|factseis|factsiete>%%  (todosiva) Forzando a totalizar todos los IVA así no existan productos gravados con ellos
%<factnueve>%%  (G) Escribiendo una letra ‘G’ en los ítems gravados con IVA general así sea el único gravamen a facturar
%<factdiez>%%  (sinmonedatotales) No agrega el símbolo monetario en los totales
%<factocho>%%  (siniva) Sin IVA
%<factdos>%%  (unafirma) Sólo la firma del emisor
%<facttres>%%  (sinfirmas) Sin firmas
%<factdiez>%%  (vencimiento) Con fecha de vencimiento explícita así no haya crédito
%
%<factocho>%%  (csv) Escribe un archivo CSV con la información numérica de la factura
%<beispiel>%%  (nospanish) Sin cargar babel en castellano, ni la codificación T1 para la tipografías
%
%<factdos|factcuatro>%%  Usando \Nombre y \CI en lugar de \RazonSocial y \RIF
%<factdos|factcuatro|reptres>%%  Sin especificar la persona de contacto
%<factcinco|factseis>%%  Sin especificar el teléfono del cliente
%<factcuatro|factseis>%%  Sin especificar el e-mail del cliente
%<factcinco|repuno>%%  (\Emisor) Especificando el nombre del emisor de la factura
%<factcinco|factseis>%%  (\Conforme) Especificando el nombre de quien firma la copia de la factura como recibida
%<repuno>%%  Sin especificar el proveedor, así toma el emisor de la factura como el nombre del proveedor
%<repdos|reptres>%%  (\Proveedor) Especificando el nombre del proveedor para el encabezado del reporte
%
%<facttres>%%  Sin especificar fecha para que tome la fecha del día.
% %<factnueve>%%  Con nota interna automática por el descuento en pagos electrónicos.
% %<factdiez>%%  Con nota interna automática con el monto en Bs.F dada la fecha de emisión de la factura antes de que a la moneda se le quitara el adjetivo “soberano”.
%
%<factcuatro>%%  (\Fecha*) Suministrando la fecha según formato de datetime2 sin alterar cómo sale escrita
%<facttres>%%  (\FormatoFecha) Cambiando el separador de la fecha para que salga escrita dd/mm/yyyy
%<factcinco>%%  (\FormatoFecha) Suministrando la fecha según el formato yyyymmdd pero que salga escrita dd.mm.yyyy
%<factseis>%%  (\FormatoFecha) Cambiando el formato para que la fecha salga escrita en el formato default de datetime2 pero cambiando el separador yyyy/mm/dd
%<factocho>%%  Redefiniendo el formato de la fecha (redefiniendo macros \@@Fecha y \@EstiloFecha)
%
%<factnueve>%%  (\Credito) Especificando la cantidad de días para el crédito
%<facttres>%%  (\DescuentoE y \DescuentoG) Con descuentos totales en porcentajes para exentos y gravables con IVA general
%<factocho>%%  (\Descuento*) Con descuentos totales en dinero neto (y no en porcentajes) aplicables por igual a todos los tipos de elementos existentes
%<prefac>%%  (\Descuento) Con descuentos totales en porcentaje aplicables por igual a todos los tipos de elementos existentes
%
%<factsiete>%%  Redefiniendo el texto para la cantidad (redefiniendo macro \@@Cantidad)
%
%<factsiete>%%  (\GeometriaFactura) Cambiando la geometría del papel de la factura
%<factdiez>%%  (\GeometriaReverso) Cambiando la geometría del reverso de la factura
%<repdos>%%  (\GeometriaReporte) Cambiando la geometría del papel del reporte
%<prefac|cotiza>%%  (\FondoFactura) Con el fondo de la imprenta agregado
%<repuno>%%  (\FondoReporte) Especificamos el fondo o membrete del reporte, sólo para la primera página
%<repdos>%%  (\FondoReporte*) Especificamos el fondo o membrete del reporte para todas las páginas
%<reptres>%%  Sin especificar fondo de reporte
%<repuno>%%  (\InicioReporte) Estableciendo un margen superior distinto para la primera página del reporte por presencia de membrete
%<repuno|reptres>%%  (\TituloReporte) Establecemos el título del reporte sin que salga en las cabeceras ni pies de página
%<repdos>%%  (\TituloReporte*) Establecemos el título del reporte para que salga en las cabeceras o pies de página
%<reptres>%%  (\InfoPagina) Personalizamos la información de la cabecera del reporte y mostramos el nombre del proveedor allí
%<reptres>%%  (\EstiloPagReporte) Cambiamos el estilo de página para que la información del reporte aparezca como pie de página
%<cotiza>%%  (\FirmaFactura) Agregando la firma escaneada en la factura, es decir, en la cotización
%
%<factdos|facttres|factsiete>%%  (\SepVertical) Alterando la separación mínima vertical entre los elementos
%<factdos|factcinco>%%  (\SepDatosResumen) Alterando la separación horizontal entre los datos del cliente y el resumen
%<factdos|facttres|factseis>%%  (\SepEncabezado) Alterando el margen interno desde el contorno hasta el texto en el encabezado
%<factnueve>%%  (\SepNotaPrevia) Alterando el margen interno desde el contorno hasta el texto en la nota previa
%<factdos|factseis>%%  (\SepDescripcion) Alterando la distancia desde el contorno hasta el texto en la descripción
%<factnueve>%%  (\SepNotaFinal) Especificando la distancia desde el contorno hasta el texto de la nota final, predeterminadamente sin contorno
%<factdos|factcuatro>%%  (\SepItemsExtra) Agregando una separación extra entre los ítems en la factura
%<factsiete>%%  (\SepFilas) Alterando la separación horizontal y vertical entre los totales en el modo de filas
%<facttres|factsiete>%%  (\LineaEncabezado) Alterando la línea de las cajas del encabezado, la curvatura en la esquina y la sombra
%<factdiez>%%  (\LineaNotaPrevia) Alterando la línea de la nota previa, para desaparecerla
%<factdos|factsiete>%%  (\LineaDescripcion) Alterando la línea de la descripción, la curvatura en la esquina y la sombra
%<factnueve>%%  (\LineaNotaFinal) Agregando el contorno para la nota final que predeterminadamente está deshabilitado
%<factsiete>%%  (\LineaFirmas) Alterando la longitud de las líneas para firmar y su grosor
%<factsiete>%%  (\PosFecha, \PosFirmas y \PosDenom) Alterando la posición de la fecha, las firmas y de la denominación del documento
%<factnueve>%%  (\PosNotaFinal) Alterando la posición de la nota final
%<factcuatro|factocho>%%  (\Letra***) Cambiando varias de las tipografías
%
%<!factcuatro&!reptres>%%  (\NotaFinal) Con nota final
%<factnueve>%%  Haciendo mención a las fechas y al crédito en la nota final
%<factcuatro>%%  Sin nota final
%<factuno|factdos|factcinco|factocho>%%  (\Resumen) Con resumen
%<facttres|factcuatro|factseis|factsiete|prefac|cotiza|repuno|repdos>%%  Sin resumen
%<factcuatro|factnueve|factdiez>%%  (\NotaPrevia) Con nota previa
%<factocho|factnueve>%%  (\NotaInterna) Con nota interna
%<factnueve>%%  (\NotaExterna) Con nota externa
%
%<factdos|facttres>%%  Con artículos gravados con todos los IVA
%<factseis|factsiete|factocho>%%  Escribimos un ítem cuya descripción ocupa más que una línea
%<factdos|factcuatro|factcinco>%%  (\ItemX) Agregando un ítem sin precio
%
%<!facttres&!reptres&!beispiel>%%  (\Descripcion) Con descripción larga
%<facttres|reptres>%%  Sin descripción larga
%
%<factseis|repdos>%%  Con etiquetas y referencias
%
%<repuno>%%  (\FirmaReporte) Agregando en el reporte la línea para firmar a mano
%<factocho|repdos|reptres>%%  (\FirmaReporte) Agregando la firma escaneada en el reporte
%<factocho|repdos|reptres>%%  Con \FirmaReporte, ajustando el corrimiento en “y” de la firma respecto a la línea
%<repdos|reptres>%%  Con \FirmaReporte, cambiando la ubicación en “x” de la línea de la firma y cambiando la longitud de la línea hacia los lados
%<reptres>%%  Con \FirmaReporte, cambiando el nombre de la persona que va a firmar
%%</desc>
%%
%<factuno>\documentclass{factura}
%<!factuno>\documentclass[
%<prefac|beispiel>prefactura,
%<cotiza>cotizacion,
%<repuno>sinreverso,
%<repuno>conreporte,
%<repdos|reptres>soloreporte,
%<beispiel>nospanish,
%<reptres>twoside,
%<factcinco>sincantidad,
%<factdos|factseis>numitem,
%<factdos|factseis>descuentos,
%<factseis>subtotal,
%<factcinco>iva,
%<factcuatro|factcinco|factsiete>filas,
%<factdiez>filas, % luego queda sin efecto con \filasfalse en el preámbulo
%<factsiete>centrado,
%<facttres>agrupado,
%<factcinco|factseis|factsiete>denom,
%<cotiza>sinnumero, % sin esta opción aparecerá el número de la cotización.
%<cotiza>sintotales,
%<factdos>dcu,
%<factcuatro>duc,
%<factdos|factseis>sinlineasve,
%<facttres|factcuatro>sinlineasvi,
%<factdos>sinlineahtot,
%<factcuatro>sinlineahtit,
%<factseis>sinlineashi,
%<factsiete>sinlineashe,
%<factnueve>sinlineahni,
%<facttres|factcinco>conlineasha,
%<factcinco|factocho>sinexpandir,
%<facttres|factcuatro|prefac>expandecuadro,
%<factcuatro|prefac>linea,
%<facttres>impdesc,
%<factdos>centradesc,
%<factsiete>sinmarcas,
%<factcinco|factseis|factsiete>todosiva,
%<factnueve>G,
%<factdiez>sinmonedatotales,
%<factocho>siniva,
%<factnueve>% electro,
%<factdos>unafirma,
%<prefac|cotiza>% dosfirmas, % para poner las dos firmas.
%<prefac>% unafirma, % para poner la firma del emisor.
%<cotiza>% sinfirmas, % para no poner firmas.
%<facttres>sinfirmas,
%<factdiez>vencimiento,
%<factocho>csv,
%<factdiez>csv, % luego queda sin efecto con \csvfalse en el preámbulo
%<!factuno>]{factura}

%<factdiez>% Cambiamos por aquí como condicionales las opciones de la clase; es lo mismo que arriba en \documentclass.
%<factdiez>\sinlineasvtrue \sinlineashtrue \filasfalse \csvfalse
%<factdiez>
%%
%% This is file `factura-ejemplo-aux.tex',
%% generated with the docstrip utility.
%%
%% The original source files were:
%%
%% factura.dtx  (with options: `aux')
%% 
%% Copyright 2019 Emilio Augusto Lazo Zaia <emiliolazozaia@gmail.com>
%% 
%% This work may be distributed and/or modified under the
%% conditions of the LaTeX Project Public License, either
%% version 1.3c of this license or (at your option) any later
%% version. This version of this license is in
%%  http://www.latex-project.org/lppl/lppl-1-3c.{html|txt|tex|pdf}
%% and the latest version of this license is in
%%  http://www.latex-project.org/lppl.txt
%% and version 1.3c or later is part of all distributions of
%% LaTeX version 2008/05/04 or later.
%% 
%% This work has the LPPL maintenance status `maintained'.
%% 
%% The Current Maintainer of this work is Emilio Augusto Lazo Zaia.
%% 
%% This work consists of the files README and factura.dtx
%% with its derived files factura.ins, factura.cls, factura.def,
%% factura-ejemplo-aux.tex, factura-ejemplo-firma.tex,
%% factura-ejemplo-fondo.tex, factura-ejemplo-membrete.tex,
%% factura-ejemplo-cotizacion.tex, factura-ejemplo-prefactura.tex,
%% factura-ejemplo-factura01.tex, factura-ejemplo-factura02.tex,
%% factura-ejemplo-factura03.tex, factura-ejemplo-factura04.tex,
%% factura-ejemplo-factura05.tex, factura-ejemplo-factura06.tex,
%% factura-ejemplo-factura07.tex, factura-ejemplo-factura08.tex,
%% factura-ejemplo-factura09.tex, factura-ejemplo-factura10.tex,
%% factura-ejemplo-reporte1.tex, factura-ejemplo-reporte2.tex,
%% factura-ejemplo-reporte3.tex and factura-beispiel-rechnung.tex.
%% 
%% Helper file; not to be compiled standalone, only to be included on examples.
%% Archivo auxiliar; no para ser compilado sino para ser incluido en los ejemplos.
%%
\usepackage{xspace,catchfilebetweentags,regexpatch}
\def\ejemplooptcolor{\color{blue}}
\def\ejemplomacrocolor{\color{purple}}
\makeatletter
\catcode`\%=11
\catcode`\|=14
\newcommand{\Ejemplo}{|
 \newpage
 \ClearShipoutPictureBG
 \newgeometry{vmargin={2cm,2cm},hmargin={2cm,2cm},footskip=0pt,headheight=0pt,headsep=0pt,footnotesep=0pt}|
 \thispagestyle{empty}|
 \catcode`\<=12
 \catcode`\>=12
 \CatchFileBetweenDelims{\DescEjemplo}{\jobname.tex}{<desc>}{</desc>}[\catcode37=11\catcode13=14\makeatletter]|
 \catcode`\%=11
 \makeatletter
 \regexpatchcmd*{\DescEjemplo}{\cC.\cO[*]*}{\c{begingroup}\c{ejemplomacrocolor}\c{ttfamily}\c{string}\0\c{endgroup}\c{xspace}}{}{}|
 \regexpatchcmd*{\DescEjemplo}{(\()(\cL[a-zG]*)(\))}{(\c{begingroup}\c{ejemplooptcolor}\c{ttfamily}\2\c{endgroup})}{}{}|
 \xpatchcmd*{\DescEjemplo}{%% }{\item}{}{}|
 \xpatchcmd*{\DescEjemplo}{}{}{}|
 \makeatother
 {\bfseries Descripci\'on de este ejemplo {\slshape \jobname.tex}:}
 \begin{itemize}
  \DescEjemplo
 \end{itemize}
 \catcode`\%=14
 \catcode`\<=13
 \catcode`\>=13
 }
\catcode`\%=14
\catcode`\|=12
\makeatother
%% 
%%
%% End of file `factura-ejemplo-aux.tex'.
% Se incluye para efectos del ejemplo, así tendremos disponible la macro \Ejemplo.
\usepackage[utf8]{inputenc}
%<beispiel>\usepackage[official]{eurosym}
%<beispiel>\usepackage[ngerman]{babel}
%<beispiel>\usepackage[T1]{fontenc}
%<beispiel>
%<beispiel>\Moneda{\euro{}}
%<beispiel>\makeatletter
%<beispiel>\def\@AlicuotaG{0.19}
%<beispiel>\def\@AlicuotaR{0.07}
%<beispiel>\def\@DenomAlicuotaG{A}
%<beispiel>\def\@DenomAlicuotaR{C}
%<beispiel>\def\@DenomAlicuotaE{B}
%<beispiel>\def\@@Datos{Kundenidentifikation}
%<beispiel>\def\@@Resumen{Zusammenfassung}
%<beispiel>\def\@@Fecha{Datum:}
%<beispiel>\def\@@Nombre{Kunde:} \def\@@RazonSocial{Kunde:}
%<beispiel>\def\@@RIF{USt-IdNr:} \def\@@CT{USt-IdNr:}
%<beispiel>\def\@@Contacto{Kontactperson:}
%<beispiel>\def\@@Direccion{Adresse:}
%<beispiel>\def\@@Telefono{Telefon:}
%<beispiel>\def\@@Email{E-mail:}
%<beispiel>\def\@@Cliente{Kunde:}
%<beispiel>\def\@@Proveedor{Lieferant:}
%<beispiel>\def\@@Cantidad{Menge}
%<beispiel>\def\@@Descripcion{Beschreibung}
%<beispiel>\def\@@PrecioU{Einzelpreis}
%<beispiel>\def\@@PrecioT{Gesamt}
%<beispiel>\def\@@Precio{Preis}
%<beispiel>\def\@@AlicuotaG{MwSt A}
%<beispiel>\def\@@AlicuotaR{MwSt C}
%<beispiel>\def\@@BIG{Netto 19\%}
%<beispiel>\def\@@BIR{Netto 7\%}
%<beispiel>\def\@@BIE{Mwst-befreit}
%<beispiel>\def\@@Total{Brutto}% Zu zahlen / Brutto / Summe / Gesamt
%<beispiel>\def\@@Conforme{der Kunde}
%<beispiel>\def\@@DenomFactura{Rechnung}
%<beispiel>\def\@@DenomCotiza{Voranschlag}
%<beispiel>
%<beispiel>\renewcommand{\@EstiloDenom}{\@LetraDenom\@Denominacion-\ifsinnumero\else\@LetraNumeracion Nr:\space\@Num\fi}
%<beispiel>\renewcommand{\@EstiloNumControl}{}
%<beispiel>\makeatother

%<factcinco|factseis|factsiete|cotiza|prefac|repuno|repdos|beispiel>\Num{1}
%<prefac>% \NumControl{1001}
%<prefac>% \NumSerieControl{9}
%<!factdos&!factcuatro&!beispiel>\RazonSocial{Ayala San Vicente Usureros}
%<factdos|factcuatro|beispiel>\Nombre{Jacinto de Ayala y San Vicente}
%<!factdos&!factcuatro&!beispiel&!reptres>\RIF{Y-12.345.678-9}
%<factdos|factcuatro|beispiel>\CI{V-987.654.321}
%<!factdos&!factcuatro&!reptres&!beispiel>\Contacto{Jacinto de Ayala}
%<!reptres>\Direccion{Av. Ayala, Galpón San Vicente, Sector Arauca, Apure, Venezuela}
%<!factcinco&!factseis&!reptres>\Telefono{+58 (987) 654.32.10}
%<!factcuatro&!factseis&!reptres>\Email{jacinto@ayala.sv}
%<factcinco|repuno>\Emisor{Lázaro de la Vega}
%<factcinco>\Conforme{\Contacto}
%<factseis>\Conforme{Ramela San Vicente}
%<repdos|reptres>\Proveedor{Lázaro de la Vega y San Román}

%<factcinco>\FormatoFecha[yyyymmdd]{ddmmyyyy}{.}
%<facttres>\FormatoFecha{ddmmyyyy}{/}
%<factseis>\FormatoFecha{default}{/}
%<!facttres&!factcuatro&!factcinco>\Fecha{06-11-2019}
%<factcuatro>\Fecha*{2019}{06}{11}
%<factcinco>\Fecha{20191106}
%<factsiete|factocho>
%<factsiete|factocho>\makeatletter
%<factocho> \renewcommand{\@@Fecha}{\slshape FECHA:}
%<factocho> \renewcommand{\@EstiloFechas}{\@LetraFecha\bfseries\@@Fecha\space{\color{blue}\@Fecha}}
%<factsiete> \renewcommand{\@@Cantidad}{Cant.}
%<factsiete|factocho>\makeatother
%<factnueve>\Credito{11}

%<prefac>\Descuento{18}
%<facttres>\DescuentoE{15}
%<facttres>\DescuentoG{20}
%<factocho>\Descuento*{300}
%<prefac|facttres|factocho>
%<factsiete>\GeometriaFactura{legalpaper,vmargin={5cm,5.5cm},hmargin={1cm,1cm},footskip=0pt,headsep=0pt,headheight=0pt}
%<factdiez>\GeometriaReverso{vmargin={4cm,4cm},hmargin={4cm,4cm},footskip=0pt,headsep=0pt,headheight=0pt}
%<repdos>\GeometriaReporte{vmargin={4.4cm,2.5cm},hmargin={3cm,2cm},footskip=1.2cm,headsep=0.777cm,headheight=1.5cm}
%<prefac|cotiza|beispiel>\FondoFactura{factura-ejemplo-fondo}
%<repuno>\FondoReporte{factura-ejemplo-membrete}
%<repdos>\FondoReporte*{factura-ejemplo-membrete}
%<repuno>\InicioReporte{2.2cm}
%<repuno|reptres>\TituloReporte{El título del reporte}
%<repdos>\TituloReporte*{El título del reporte}
%<reptres>\InfoPagina{La cabecera o pie del reporte - \Proveedor}
%<repuno|reptres>% \EstiloPagReporte{soloabajo}
%<repuno|reptres>% \EstiloPagReporte[soloabajo]{plain}
%<repuno|reptres>% \EstiloPagReporte[empty]{plain}
%<repdos|reptres>% \EstiloPagReporte[empty]{soloabajo}
%<reptres>% \EstiloPagReporte{plain}
%<repuno|repdos|reptres>% \EstiloPagReporte{solonumpag}
%<repdos>% \EstiloPagReporte{soloabajo}
%<repuno|reptres>% \EstiloPagReporte[solonumpag]{plain}
%<repuno|reptres>% \EstiloPagReporte[solonumpag]{soloarriba}
%<reptres>\EstiloPagReporte[solonumpag]{soloabajo}
%<repuno>% \EstiloPagReporte[solonumpag]{soloabajo}
%<repuno|reptres>% \EstiloPagReporte[empty]{soloarriba}
%<repuno|reptres>% \EstiloPagReporte{soloarriba}
%<cotiza>\FirmaFactura{\includegraphics[width=4.5cm]{factura-ejemplo-firma}}[0pt][17pt]
%<factdos|facttres|factsiete>\SepVertical{0.75cm}
%<factdos|factcinco>\SepDatosResumen{1cm}
%<factdos|facttres|factseis>\SepEncabezado{16pt}
%<factdos|factcuatro>\SepItemsExtra{5pt}
%<factdos|factseis>\SepDescripcion{16pt}
%<factnueve>\SepNotaPrevia{16pt}
%<factsiete>\SepFilas{1cm}{10pt}
%<factnueve>\SepNotaFinal{16pt}
%<facttres|factsiete>\LineaEncabezado{1pt}{20pt}{3pt}
%<factdiez>\LineaNotaPrevia{0pt}
%<factdos|factsiete>\LineaDescripcion{1pt}{20pt}{3pt}
%<factnueve>\LineaNotaFinal{0.4pt}{4pt}{3pt}
%<factsiete>\LineaFirmas{6.5cm}{2pt}
%<factsiete>% Puede usarse \PosFecha* o \PosDenom*, \PosControl* o \PosNotaFinal* para que el texto se despliegue en la posición 'Y' dada pero hacia arriba y no hacia abajo. \PosFirmas no acepta eso.
%<factnueve>\PosNotaFinal{.9cm}{.9cm}{-5.3cm}
%<factsiete>\PosFecha{-0cm}{-1.5cm}
%<factsiete>\PosFirmas{1cm}{1cm}{-3cm}
%<factsiete>\PosDenom{-0cm}{4.3cm}
%<factcuatro>\LetraDescripcion{\fontsize{20}{30}}
%<factcuatro>\LetraTitEnc{\fontsize{15}{20}\bfseries}
%<factcuatro>\LetraFirmas{\fontsize{16}{20}\itshape}
%<factcuatro>\LetraNotaFinal{\fontsize{18}{28}\bfseries}
%<factocho>\LetraItems{\fontsize{20}{30}}
%<factocho>\LetraNumeros{\fontsize{20}{30}}
%<factocho>\LetraTipoIVA{\fontsize{20}{30}\bfseries}
%<factocho>\LetraTitTotales{\fontsize{20}{30}\slshape}
%<factocho>\LetraNumTotales{\fontsize{20}{30}}
%<factocho>\LetraTitColumnas{\fontsize{15}{20}}
%<factocho>\LetraTitTotal{\fontsize{20}{30}\bfseries\slshape}
%<factocho>\LetraNumTotal{\fontsize{20}{30}\bfseries}
%<factocho>\tabcolsep7pt
%<!factuno&!factcinco&!prefac&!reptres>
%<cotiza>% \Denom{PRESUPUESTO} % También podemos cambiar el nombre del documento.
%<!factcuatro&!factnueve&!beispiel&!reptres>\NotaFinal{La nota final...}
%<factnueve>\NotaFinal{La factura se emitió el \Fecha{} con crédito de \Credito{} dias, siendo su vencimiento el \FechaVencimiento.}
%<beispiel>\NotaFinal{Die Zahlungsbedingungen}
%<factuno|factdos|factcinco|factocho>\Resumen{El resumen opcional}
%<beispiel>\Resumen{Zusammenfassung}
%<factcuatro|factnueve|factdiez>\NotaPrevia{\hfill LA NOTA PREVIA}
%<factocho|factnueve>\NotaInterna{LA NOTA INTERNA}
%<factnueve>\NotaExterna{LA NOTA EXTERNA}

\begin{document}
%<!reptres>
%<!reptres>\begin{factura}
%<!factdos&!factcinco&!factseis&!repdos&!reptres&!beispiel> \Item{1}{ítem con IVA general}{1000}
%<factseis|repdos> \Item{2}{ítem con IVA general, viene con descuento}{1000}{8}[Un ítem que ofrecemos a precio de costo y que está gravado con IVA general]\label{primeritem}
%<factuno|factcuatro|repdos> \Item{4}{otro ítem con IVA general}{100}
%<cotiza> \Item{1}{otro ítem con IVA general}{1100}
%<cotiza> \Item{1}{otro ítem más con IVA general}{1200}
%<facttres|prefac> \ItemR{1}{ítem con IVA reducido}{1000}
%<facttres> \ItemA{1}{ítem con IVA adicional}{1000}
%<facttres> \ItemE{1}{ítem exento de impuesto}{1000}
%<factdos|factseis> \Item{1}{ítem con IVA general, viene con descuento}{1000}{12}
%<factsiete|factocho> \Item{10}{otro ítem con IVA general (descripción larga para mostrar el cambio de línea)}[otro ítem con IVA general (esta descripción es sólo para el .csv)]{1000}
%<factdos> \ItemR{1}{ítem con IVA reducido, viene con descuento}{1000}{5}
%<factdos> \ItemA{1}{ítem con IVA adicional, viene con descuento}{1000}{10}
%<factdos> \ItemE{1}{ítem exento de impuesto}{1000}
%<factdos|factcuatro> \ItemX{1}{línea sin precio}
%<factcinco> \Item{ítem sin cantidad, con IVA general}{1000}
%<factcinco> \ItemA{ítem sin cantidad, con IVA adicional}{1000}
%<factcinco> \ItemX{línea sin precio}
%<beispiel> \Item{1}{Artikel mit 19\% MwSt}{100}
%<beispiel> \ItemR{1}{Artikel mit 7\% MwSt}{100}
%<beispiel> \ItemE{1}{Artikel mit kein MwSt}{100}
%<!reptres>\end{factura}
%<!facttres&!reptres>
%<!facttres&!factseis&!reptres&!beispiel>\Descripcion{La descripción larga}
%<factseis>\Descripcion{
%<factseis>
%<beispiel>\Descripcion{Lange Beschreibung.}
%<factcinco|factocho|factnueve|factdiez|repuno|repdos|reptres>
%<factnueve|factdiez|repuno>\begin{reverso}
%<factnueve|factdiez|repuno>esquina superior izquierda del reverso\hfill esquina superior derecha del reverso\par
%<factnueve|factdiez|repuno>\vfill
%<factnueve|factdiez|repuno>esquina inferior izquierda del reverso\hfill esquina inferior derecha del reverso
%<factnueve|factdiez|repuno>\end{reverso}
%<repuno>
%<factcinco|factocho|repuno|repdos|reptres>\begin{reporte}
%<factcinco|factocho|repuno|repdos|reptres> El reporte - página 1\newpage El reporte - página 2 \newpage El reporte - página 3.
%<repdos>
%<factseis|repdos> Se facturaron en total \arabic{NumItem} artículos.
%<factseis|repdos>
%<factseis|repdos> En el artículo número \itemref{primeritem} se facturaron \cantref{primeritem} “\descref{primeritem}”, cada uno por un valor de \Moneda~\puref{primeritem}, dando un subtotal por este elemento de \Moneda~\subtref{primeritem}; el descuento en ese artículo es del \dctoref{primeritem}\%, dando un total de \Moneda~\ptref{primeritem}, y esta gravado con IVA de \ivaref{primeritem}\%. Ampliando la descripción podemos decir que este ítem es: “\ldescref{primeritem}”.
%<factseis|repdos>
%<factseis|repdos> La base imponible total de esta factura es de \Moneda~\numprint\BIG; el IVA es de \Moneda~\numprint\AlicuotaG,  y el total final es de \Moneda~\numprint\Total.
%<factseis|repdos>
%<factseis|repdos> % Estas son todas las macros para las totalizaciones: \BIG, \BIR, \BIA, \BIE, \AlicuotaG, \AlicuotaR, \AlicuotaA, \SubtG, \SubtR, \SubtA, \SubtE, \DescG, \DescR, \DescA, \DescE.
%<factseis>}
%<repuno> \vspace{1.5cm}\FirmaReporte
%<factocho> \FirmaReporte[\includegraphics[width=4.5cm]{factura-ejemplo-firma}][0pt][17pt]
%<repdos> \FirmaReporte[\includegraphics[width=4.5cm]{factura-ejemplo-firma}][0pt][17pt][3cm][3cm]
%<reptres> \FirmaReporte[\includegraphics[width=4.5cm]{factura-ejemplo-firma}][0pt][17pt]{el que firma}[4cm][0.8cm]
%<factcinco|factocho|repuno|repdos|reptres>\end{reporte}

%% Este es un comando auxiliar suministrado por factura-ejemplo-aux.tex para escribir el detalle del ejemplo en la última hoja.
\Ejemplo

\end{document}
%</!class&!def&!firma&!fondo&!membrete&!aux&!batchfile>
%<*firma>
%% (PSTricks code generated with LaTeXDraw 2.0.8)
%% (Compilar con XeLaTeX)
\documentclass{standalone}
\usepackage[usenames,dvipsnames]{pstricks}
\begin{document}
 \begin{pspicture}(0,-3.19388)(13.8397665,3.18)
  \pscustom[linewidth=0.1,linecolor=red]{
   \newpath
   \moveto(0.0,0.9762225)
   \lineto(1.0057193,0.95771605)
   \curveto(1.5085789,0.94846284)(2.4492223,1.1335286)(2.8870058,1.3278477)
   \curveto(3.3247895,1.5221668)(4.064289,2.0079648)(4.366005,2.2994435)
   \curveto(4.667721,2.5909224)(4.9457726,2.9518008)(4.9221087,3.0212007)
   \curveto(4.8984447,3.0906003)(4.762377,3.16)(4.6499724,3.16)
   \curveto(4.5375686,3.16)(4.1175327,2.9934406)(3.8099012,2.8268814)
   \curveto(3.5022693,2.6603222)(2.721358,1.6748462)(2.2480783,0.85592973)
   \curveto(1.7747989,0.037013244)(1.1358713,-0.91607606)(0.97022337,-1.050249)
   \curveto(0.8045755,-1.1844217)(0.5975156,-1.3185943)(0.5561036,-1.3185943)
   \curveto(0.5146916,-1.3185943)(0.43186772,-1.2121814)(0.39045575,-1.1057684)
   \curveto(0.34904376,-0.9993555)(0.37270775,-0.7495166)(0.4377837,-0.6060907)
   \curveto(0.50285965,-0.46266478)(0.6152636,-0.3608783)(0.6625916,-0.4025183)
   \curveto(0.7099195,-0.44415832)(0.82232344,-0.56907773)(0.8873994,-0.65235716)
   \curveto(0.9524753,-0.7356369)(1.354763,-0.86980957)(1.6919748,-0.9207028)
   \curveto(2.0291865,-0.97159606)(2.5202143,-1.0178626)(2.67403,-1.0132358)
   \curveto(2.827846,-1.008609)(3.0112422,-0.9207028)(3.0408218,-0.8374231)
   \curveto(3.0704017,-0.75414366)(3.0999818,-0.69862366)(3.0999818,-0.7263837)
   \curveto(3.0999818,-0.75414366)(3.1709735,-0.80966306)(3.2419655,-0.8374231)
   \curveto(3.3129578,-0.8651831)(3.4667737,-0.80041015)(3.5495975,-0.7078772)
   \curveto(3.6324213,-0.6153442)(3.7744055,-0.6199707)(3.8335655,-0.7171304)
   \curveto(3.8927252,-0.81429017)(4.141197,-0.9114496)(4.330509,-0.9114496)
   \curveto(4.5198207,-0.9114496)(4.7978725,-0.86055636)(4.8866124,-0.80966306)
   \curveto(4.9753523,-0.75877017)(5.0995884,-0.74489015)(5.135084,-0.7819034)
   \curveto(5.1705804,-0.8189166)(5.324396,-0.8281702)(5.442716,-0.80041015)
   \curveto(5.5610366,-0.7726501)(5.6971045,-0.7680234)(5.7148523,-0.7911566)
   \curveto(5.7326,-0.81428987)(5.8745837,-0.83279663)(5.9988194,-0.8281699)
   \curveto(6.1230555,-0.8235431)(6.2887034,-0.78653014)(6.330116,-0.75414366)
   \curveto(6.3715277,-0.7217572)(6.454352,-0.69862366)(6.495764,-0.7078772)
   \curveto(6.537175,-0.7171304)(6.6554956,-0.6060907)(6.7324038,-0.48579803)
   \curveto(6.8093114,-0.36550537)(7.0163713,-0.041640013)(7.1465235,0.16193238)
   \curveto(7.2766747,0.36550477)(7.4245753,0.12491944)(7.442323,-0.3192389)
   \curveto(7.4600706,-0.7633969)(7.5133147,-1.6748462)(7.548811,-2.1421375)
   \curveto(7.5843067,-2.6094286)(7.5251465,-3.10448)(7.4304905,-3.13224)
   \curveto(7.335835,-3.16)(7.1465235,-3.17388)(7.051867,-3.16)
   \curveto(6.9572105,-3.14612)(6.9276304,-2.748228)(6.992707,-2.3642163)
   \curveto(7.057783,-1.980205)(7.436407,-1.4296337)(7.7499547,-1.2630746)
   \curveto(8.063502,-1.0965155)(8.619606,-0.80966306)(8.862162,-0.6893704)
   \curveto(9.104718,-0.56907773)(9.394602,-0.5181845)(9.441929,-0.58758426)
   \curveto(9.489257,-0.656984)(9.589829,-0.67549074)(9.643073,-0.6245975)
   \curveto(9.696317,-0.57370424)(9.767309,-0.5089313)(9.785057,-0.49505126)
   \curveto(9.802805,-0.48117125)(9.897461,-0.49505126)(9.974369,-0.5228113)
   \curveto(10.051277,-0.55057126)(10.175512,-0.62922424)(10.22284,-0.6801172)
   \curveto(10.270168,-0.73101014)(10.388489,-0.76802367)(10.45948,-0.75414366)
   \curveto(10.530473,-0.7402637)(10.65471,-0.6616107)(10.7079525,-0.59683746)
   \curveto(10.761196,-0.5320642)(10.838105,-0.5043045)(10.861769,-0.54131776)
   \curveto(10.885432,-0.5783307)(10.932761,-0.6477307)(10.956425,-0.6801172)
   \curveto(10.980088,-0.7125037)(11.033333,-0.75877017)(11.062913,-0.7726501)
   \curveto(11.092493,-0.78653014)(11.157568,-0.7680234)(11.193064,-0.7356369)
   \curveto(11.2285595,-0.7032504)(11.305469,-0.6662372)(11.34688,-0.6616107)
   \curveto(11.388292,-0.656984)(11.512528,-0.48579773)(11.595351,-0.31923857)
   \curveto(11.678176,-0.15267944)(11.778748,-0.02313324)(11.796496,-0.060146485)
   \curveto(11.814244,-0.09715942)(11.843824,-0.6569836)(11.855657,-1.1797949)
   \curveto(11.867488,-1.7026062)(11.87932,-2.2901905)(11.87932,-2.3549633)
   \curveto(11.87932,-2.4197364)(11.938479,-2.0773644)(11.997639,-1.6702198)
   \curveto(12.056799,-1.263075)(12.186951,-0.85592985)(12.257943,-0.85592985)
   \curveto(12.328936,-0.85592985)(12.459087,-0.8374231)(12.518247,-0.8189166)
   \curveto(12.577407,-0.80041015)(12.719391,-0.78653014)(12.802215,-0.7911566)
   \curveto(12.885039,-0.7957831)(13.03294,-0.7726501)(13.098015,-0.74489015)
   \curveto(13.163091,-0.7171301)(13.257747,-0.5922107)(13.287327,-0.49505126)
   \curveto(13.316907,-0.39789185)(13.352403,-0.33774537)(13.358318,-0.3747583)
   \curveto(13.364233,-0.41177154)(13.3878975,-0.48117125)(13.405647,-0.51355773)
   \curveto(13.423395,-0.5459442)(13.48847,-0.57370424)(13.535798,-0.56907773)
   \curveto(13.583126,-0.564451)(13.66595,-0.5598242)(13.701447,-0.5598242)
   \curveto(13.736943,-0.5598242)(13.784271,-0.5551978)(13.819766,-0.5413177)}
  \pscustom[linewidth=0.1,linecolor=red]{
   \newpath
   \moveto(12.952644,0.18)
   \lineto(13.020389,0.13)
   \curveto(13.05426,0.105)(13.135551,0.085)(13.182974,0.09)
   \curveto(13.230394,0.095)(13.304913,0.13)(13.332009,0.16)
   \curveto(13.359106,0.19)(13.399754,0.285)(13.413303,0.35)
   \curveto(13.426852,0.415)(13.4404,0.49)(13.4404,0.52)}
 \end{pspicture}
\end{document}
%</firma>
%<*fondo|membrete>
\documentclass[12pt]{article}
\usepackage[utf8]{inputenc}
\usepackage{tikz}\usetikzlibrary{positioning}
\usepackage{calc}
\usepackage[letterpaper,hmargin={1cm,1cm},vmargin={1cm,1cm},footskip=0pt,headsep=0pt,headheight=0pt]{geometry}
\parindent0pt
\lineskip0pt
\pagestyle{empty}
\begin{document}
 \begin{minipage}[b][\textheight]{\textwidth}
  \baselineskip13pt
  \parskip0pt
  \lineskip0pt
  \hspace*{\fill}%
  \begin{tikzpicture}
   \node (Nombre) [draw=none, fill=gray, rounded corners=4pt]{%
    \parbox{0.57\textwidth}{%
     \vspace*{0.1cm}%
     \centering
     \color{white}\fontfamily{ppl}\LARGE\bfseries
     Lázaro de la Vega y San Román}};
   \node [node distance=0pt, below = of Nombre, outer sep=0pt, draw=gray, line width=2pt, inner sep=3.5pt, rounded corners=8pt]{%
    \parbox{0.66\textwidth}{%
     \vspace*{3pt}%
     \parskip2pt
     \centering
     \color{gray}\fontfamily{ppl}\footnotesize\bfseries
     Av. Garcilaso de la Vega, Qta. San Román, Santa Mónica, Caracas, Venezuela.
     \par
     Teléfono: +58 (123) 456.78.90 / email: lázaro@sanrom.án
     \vspace*{3pt}}};
  \end{tikzpicture}
  \hspace*{\fill}\par
  \hspace*{\fill}%
  \parbox{0.63\textwidth}{%
   \color{gray}\fontfamily{aer}\footnotesize\bfseries
   \hfill
   RIF: V-87.654.321-0}%
  \hspace*{\fill}\par
  \vfill\mbox{}%
%<fondo>  \fontsize{6}{8}\selectfont
%<fondo>  \hspace*{0.5cm}%
%<fondo>  \parskip2pt
%<fondo>  {\bfseries ESTA FORMA LIBRE VA SIN TACHADURA NI ENMIENDA \hfill SÓLO EL ORIGINAL DA DERECHO A CRÉDITO FISCAL}
%<fondo>  \hspace*{0.5cm}\par
%<fondo>  \fontdimen2\font=0.55em
%<fondo>  Imprenta: \hfill IMPRESOS DE LA VEGA. Av. Garcilaso de la Vega, Qta. San Román, Santa Mónica, Caracas, Venezuela. RIF: R-87.654.321-0. Tlf: (0123) 456.78.90\newline
%<fondo>  \fontdimen2\font=0.65em
%<fondo>  e-mail: \hfill imprenta@sanrom.án. No. Providencia SENIAT 01/01060 de fecha 12/11/2008 - Control desde No. 00-000001 hasta No. 00-010000 / Fecha 01/01/2018.\par
 \end{minipage}
\end{document}
%</fondo|membrete>
%<*aux>
%% Helper file; not to be compiled standalone, only to be included on examples.
%% Archivo auxiliar; no para ser compilado sino para ser incluido en los ejemplos.
%%
\usepackage{xspace,catchfilebetweentags,regexpatch}
\def\ejemplooptcolor{\color{blue}}
\def\ejemplomacrocolor{\color{purple}}
\makeatletter
\catcode`\%=11
\catcode`\|=14
\newcommand{\Ejemplo}{|
 \newpage
 \ClearShipoutPictureBG
 \newgeometry{vmargin={2cm,2cm},hmargin={2cm,2cm},footskip=0pt,headheight=0pt,headsep=0pt,footnotesep=0pt}|
 \thispagestyle{empty}|
 \catcode`\<=12
 \catcode`\>=12
 \CatchFileBetweenDelims{\DescEjemplo}{\jobname.tex}{<desc>}{</desc>}[\catcode37=11\catcode13=14\makeatletter]|
 \catcode`\%=11
 \makeatletter
 \regexpatchcmd*{\DescEjemplo}{\cC.\cO[*]*}{\c{begingroup}\c{ejemplomacrocolor}\c{ttfamily}\c{string}\0\c{endgroup}\c{xspace}}{}{}|
 \regexpatchcmd*{\DescEjemplo}{(\()(\cL[a-zG]*)(\))}{(\c{begingroup}\c{ejemplooptcolor}\c{ttfamily}\2\c{endgroup})}{}{}|
 \xpatchcmd*{\DescEjemplo}{%% }{\item}{}{}|
 \xpatchcmd*{\DescEjemplo}{}{}{}|
 \makeatother
 {\bfseries Descripci\'on de este ejemplo {\slshape \jobname.tex}:}
 \begin{itemize}
  \DescEjemplo
 \end{itemize}
 \catcode`\%=14
 \catcode`\<=13
 \catcode`\>=13
 }
\catcode`\%=14
\catcode`\|=12
\makeatother
%</aux>
%    \end{macrocode}
% \Finale
% \starttodo

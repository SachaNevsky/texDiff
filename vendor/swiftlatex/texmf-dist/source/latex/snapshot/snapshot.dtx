% \def\filename{snapshot.dtx}
% \def\fileversion{2.13}
% \def\filedate{2012/08/06}
%
% \iffalse meta-comment
%
% American Mathematical Society
% Technical Support
% Publications Technical Group
% 201 Charles Street
% Providence, RI 02904
% USA
% tel: (401) 455-4080
%      (800) 321-4267 (USA and Canada only)
% fax: (401) 331-3842
% email: tech-support@ams.org
%
% Copyright 1999-2003, 2004, 2010, 2012 American Mathematical Society.
%
% This work may be distributed and/or modified under the
% conditions of the LaTeX Project Public License, either version 1.3c
% of this license or (at your option) any later version.
% The latest version of this license is in
%   http://www.latex-project.org/lppl.txt
% and version 1.3c or later is part of all distributions of LaTeX
% version 2005/12/01 or later.
% 
% This work has the LPPL maintenance status `maintained'.
% 
% The Current Maintainer of this work is the American Mathematical
% Society.
%
% \fi
%
% \iffalse
%<*driver>
\NeedsTeXFormat{LaTeX2e}
\documentclass[oneside]{amsdtx}
\providecommand{\qq}[1]{\textquotedblleft#1\textquotedblright}
\providecommand{\mdash}{\textemdash\penalty\exhyphenpenalty}
\providecommand{\lat}[1]{\protect\LaTeX{}}
\providecommand{\strong}[1]{\textbf{#1}}
\providecommand{\secref}[1]{Section~\ref{#1}}

\CodelineIndex

\begin{document}
\title{The \pkg{snapshot} package}
\author{American Mathematical Society\\Michael Downes}
\date{Version \fileversion, \filedate}
\DocInput{snapshot.dtx}
\PrintIndex
\end{document}
%</driver>
% \fi
%
% \MakeShortVerb{\|}
%
% \maketitle
% \section{Introduction}
%
%    The \pkg{snapshot} package helps the owner of a \LaTeX{} document
%    obtain a list of the external dependencies of the document, in a
%    form that can be embedded at the top of the document. To put it
%    another way, it provides a snapshot of the current processing
%    context of the document, insofar as it can be determined from
%    inside \LaTeX{}.
%
%    If a document contains such a dependency list, then it becomes
%    possible to arrange that the document be processed always with the
%    same versions of everything, in order to ensure the same output.
%    This could be useful for someone wanting to keep a \LaTeX{}
%    document on hand and consistently reproduce an identical DVI file
%    from it, on the fly; or for someone wanting to shield a document
%    during the final stages of its production cycle from unexpected
%    side effects of routine upgrades to the \TeX{} system.
%
%    Normal usage of the \pkg{snapshot} package involves the following
%    steps:
% \begin{enumerate}
%    \item Add a \cn{RequirePackage} statement at the top of the
%    document:
% \begin{verbatim}
%      \RequirePackage{snapshot}
%      \documentclass{article}
%        ...
% \end{verbatim}
%
%    \item Run \latex/ on the document. This will produce a dependency
%    list in a file \verb'\jobname.dep'. (I.e., if the document name is
%    \fn{vermont.ltx}, the dependency list will be named
%    \fn{vermont.dep}.)
%
%    \item Insert the \fn{.dep} file at the top of the document, before
%    \cn{documentclass}. The following example shows what you would end
%    up with for a document that used the article documentclass and the
%    graphicx package:
% \begin{verbatim}
%  \RequirePackage{snapshot}[1999/11/03]
%  \RequireVersions{
%    *{application}{TeX}     {1990/03/25 v3.x}
%    *{format} {LaTeX2e}     {1999/06/01 v2.e}
%    *{package}{snapshot}    {1999/11/03 v1.03}
%    *{class}  {article}     {1999/01/07 v1.4a}
%    *{file}   {size10.clo}  {1999/01/07 v1.4a}
%    *{package}{graphicx}    {1999/02/16 v1.0f}
%    *{package}{keyval}      {1999/03/16 v1.13}
%    *{package}{graphics}    {1999/02/16 v1.0l}
%    *{package}{trig}        {1999/03/16 v1.09}
%    *{file}   {graphics.cfg}{0000/00/00 v0.0}
%    *{file}   {dvips.def}   {1999/02/16 v3.0i}
%  }
%  \documentclass{article}
%  \usepackage{graphicx}
%  ...
% \end{verbatim}
% \end{enumerate}
%
%    The package option \opt{log} will cause the dependency list to
%    appear in the \latex/ log file instead of in a separate \fn{.dep}
%    file:
% \begin{verbatim}
% \RequirePackage[log]{snapshot}
% \end{verbatim}
%
%    Making the necessary arrangements to ensure that future \latex/
%    runs of the document actually call in the specified versions is a
%    separate problem. The \pkg{snapshot} package only provides a way to
%    generate the dependency list. However, the \cn{RequireVersions}
%    statement does record the given information in a form that can be
%    accessed from within \latex/. (It is for this purpose that it is not
%    simply a comment.) In principle a package could be set up so that a
%    later version would automatically attempt to emulate an earlier
%    version if an earlier version was specified\mdash much as \latex/
%    currently switches to 2.09 compatibility mode if it sees
%    \cn{documentstyle} instead of \cn{documentclass}.
%
%    For maximum reliability font checksums should also be reported in
%    the dependency list, but standard \tex/ 3.x does not provide direct
%    access to font checksums for macro programmers. This information
%    could be added by a separate script that scans the DVI file.
%    (Certain nontrivial complications are possible, however.)
%
% \section{Graphics files}
%    When a graphics file is read in by a \latex/ document using the
%    standard \cn{includegraphics} command, it gets a dummy version
%    number string of
%\begin{verbatim}
%Graphics file (type foo)
%\end{verbatim}
%    where foo is typically \texttt{eps}. This is with the current
%    version of the \pkg{graphics} package (at the time of this writing:
%    1999/02/16 v1.0l). What this means in practice is that all graphics
%    files will have their snapshot date and version number recorded as
%\begin{verbatim}
%Graphics v0.0
%\end{verbatim}
%    and will always compare equal (the string ``Graphics'' will be used
%    in place of a date, but since comparison is done with \cs{ifx} it
%    doesn't make any real difference).
%
%    It would be possible, for \fn{.eps} files at least, to read the
%    CreationDate comment that is normally included in the file header
%    and use that as the basis of comparison. Recording the bounding box
%    numbers instead of a dummy version number is another possibility,
%    which you can get with the
%    \opt{bbinfo} option (as of \pkg{snapshot} version 2.05).
%
% \section{Formalities}
%
% \DescribeMacro{\RequireVersions} The \cn{RequireVersions} command
%    scans its argument for file names and associated version number
%    information. The syntax of a version line for a particular file is
% \begin{quote}
% \verb'*{' \emph{file type} \verb'}{' \emph{file name} \verb'}{'
%    \emph{version info} \verb'}'
% \end{quote}
%    In other words, the \verb'*' character in this context is like a
%    command that takes three arguments. The extension part of the
%    file name should be omitted in the second argument, except when the
%    file type is \fn{file} (following the conventions of \latex/'s
%    \cn{ProvidesPackage} and \cn{ProvidesFile} commands). The most
%    commonly used file types are as follows.
% \begin{description}
% \item[class] A \latex/ documentclass file.
% \item[package] A \latex/ package file.
% \item[tfm] A \tex/ font metric file. In this case the \qq{version
%    number} is the checksum, and unless you are using an extended
%    version of \tex/ this information is not accessible from inside
%    \latex/, so it must be filled in by an outside process.
%    By default, font metric files are not listed in the dependency list
%    since the checksum info is not available. There is a package option
%    \opt{tfm} to turn on the logging of metric files. (Not yet
%    implemented [mjd,1999/09/23])
%
% \item[format] This is almost always \fn{LaTeX2e}. The information
%    comes from \cs{fmtname}. Lambda, e\LaTeX{}, and pdf\LaTeX{} leave
%    \cs{fmtname} unchanged, and although this may seem dubious at first
%    sight, I guess they have little choice: the widespread use of
%    \cs{NeedsTeXFormat} in existing package files makes them produce an
%    error message if \cs{fmtname} is changed. (Maybe the thing to do
%    would be to modify the definition of \cs{NeedsTeXFormat} as well.)
%
%    For a \LaTeX{} format that uses the Babel mechanism for preloading
%    hyphenation patterns, the version number of \fn{babel.def} that was
%    used in building the format might be of interest. But at first
%    glance there does not seem to be any easy way of dealing with that,
%    and in the normal course of things, a document that relies on Babel
%    features will also have a \verb'\usepackage{babel}' statement at
%    the top, and that will yield adequate (I think) information for the
%    snapshot dependency list.
%
% \item[application] With standard \fn{TeX} there is no reliable way to
%    get the exact version number from inside \LaTeX{}. If a document is
%    processed with one of the recent variants that address this
%    deficiency (such as e-\TeX{}, pdf\TeX{} and Omega) the available
%    version info is used, but otherwise a presumptive value of
%    \verb'1990/03/25 v3.x' is used (the official release date of \TeX{}
%    3.0).
%
% \item[file] None of the above: some other file of miscellaneous type,
%    e.g., \fn{.clo}, \fn{.cfg}, \fn{.tex}, or \fn{.def}.
% \end{description}
%
%    The \cn{RequireVersions} command can be given an optional
%    \qq{ident} argument, similar to the argument of a \cn{label}
%    command. This is not used internally but it could be used to
%    assign a label to particular groups of files in case that helps
%    with external processing.
%
% \section{Package options}
%
%    The list of options supported by the \pkg{snapshot} package is as
%    follows:
% \begin{center}\begin{tabular}{lll}
% dep             & date\\
% log             & version\\
% tfm             & major-version\\
% warning         & bbinfo \\
% error           & test \\
% self-warning
% \end{tabular}
% \end{center}
%
% \begin{description}
% \item[dep, log] Write file date and version information to
%    \textit{jobname}\fn{.dep} or to the \LaTeX{} log file, respectively.
%
% \item[error, warning, self-warning]
%    If the \pkg{snapshot} package is invoked with the \opt{error}
%    option \emph{and also} the document contains a \cn{RequireVersions}
%    statement, then each subsequent \cn{ProvidesFile},
%    \cn{ProvidesPackage}, and \cn{ProvidesClass} statement will compare
%    date and version number information with the corresponding
%    information from the \cn{RequireVersions} statement and give an
%    error message if a mismatch is detected. With the \opt{warning}
%    option you get warnings instead of errors. By default both the date
%    and the version number are compared; this behavior can be modified,
%    however, by giving additional options:
% \begin{description}
%   \item[date] compare only dates,
%   \item[version] compare only version numbers,
%   \item[major-version] use only the major version number when
%    comparing.
% \end{description}
%
%    The \opt{self-warning} causes a warning to be given, instead of an
%    error, if the \pkg{snapshot} package itself has a date or
%    version mismatch.
%
%    \textbf{Note:} A file that doesn't have any sort of
%    \cn{ProvidesFile} or \cn{ProvidesPackage} statement in it will show
%    up in the dependency list, with a dummy date and version number of
%    0000/00/00 v0.0, but testing for a version mismatch with such a
%    file is then infeasible.
%
% \item[bbinfo] For files of type ``graphic'', include bounding-box info
%    as the ``version number''. A normal date and version number are
%    seldom available for such files, and \LaTeX{} does not attempt
%    to read them, which means that the \pkg{snapshot} package could not
%    obtain the information except by drastically modifying the
%    low-level \LaTeX{} operations that read graphics file
%    information\mdash which seems overly risky.
%
% \item[tfm] \emph{Not implemented yet!} Include information about which
%    \TeX{} font metric files are called by \LaTeX{}. This list is
%    usually somewhat different from the list of fonts that are actually
%    used in the output file (\fn{.dvi} or \fn{.pdf}), primarily because
%    the setup for math formulas will normally preload font metric
%    information for all the fonts from which \LaTeX{}'s basic math
%    symbols are drawn (the symbols documented in the \LaTeX{} book),
%    even if the document does not use symbols from all of those fonts.
%
% \item[test] This option is for a special purpose. It drastically
%    changes the action of the \cs{RequireVersions} command so that it
%    does not merely record the information for later reference, but
%    does a \qq{trial load attempt} for each file in the list, and then
%    stops the \LaTeX{} job as soon as the list is finished, without
%    continuing any further.
%
%    By \qq{trial load attempt} I mean that the \cs{RequireVersions}
%    command will actually input each file, but with various bits
%    redefined so that \cs{ProvidesPackage} and variants will execute
%    \cs{endinput}\mdash in other words, only the first few lines of
%    each file will be read.
%
%    What this means, practically speaking, is that if you run such a
%    test on a document, it gives a relatively quick check on two useful
%    pieces of information without having to retypeset the entire
%    document (which for some documents might be very tedious)
% \begin{enumerate}
% \item The actual location on your system from which the file will be
%    loaded.
%
% \item The date and version info from \cs{Provides...} line, if present.
% \end{enumerate}
%
%    Similar results could be obtained by a combination of
%    \fn{kpsewhich} and \fn{grep}, but because the \pkg{snapshot} \opt{test}
%    option works through \LaTeX{}, it is a system-independent method.
%
%    Caveat: If a file does not contain any \cs{ProvidesSomething} line,
%    it will be read in its entirety, which might lead to errors. Or if
%    there is any weird stuff preceding the \cs{ProvidesWhatever} line.
%    But for well-behaved files the option seemed useful enough to be
%    worth implementing.
%
% \end{description}
%
% \StopEventually{}
%
% \section{Implementation}
%    Standard declaration of package name and date.
%    \begin{macrocode}
\NeedsTeXFormat{LaTeX2e}[1994/12/01]
\ProvidesPackage{snapshot}[2012/08/06 v2.13]
%    \end{macrocode}
%
%    Calling the \pkg{snapshot} package in a document causes \latex/ to
%    list the file names and versions in the \tex/ log or in a \fn{.dep}
%    file, so that the information may be easily copied into the
%    document file. The list so generated is nothing more than a slight
%    adaptation of the output from \latex/'s \cn{listfiles} command; it
%    puts essentially the same information into a slightly more
%    structured form so that it will be easier to use.
%
%    For the standard mechanisms that are already built into \latex/
%    (e.g., the handling of the second optional argument of
%    \cs{LoadClass}), the de facto \qq{version number} is the
%    \emph{date} given in the optional argument of a \cn{ProvidesClass}
%    or similar command. Even though most \cn{ProvidesWhatever} commands
%    also give something that follows the usual form of version
%    numbers\mdash a string of the form \verb'v2.3'\mdash this is only a
%    convention, not used internally by \latex/, and the identification
%    string of a random loaded file is not guaranteed to include it. The
%    \pkg{snapshot} package copies both pieces of information if
%    available; if the second piece is not present, a dummy number
%    \verb'0.0' is supplied. Similarly, files that don't include any
%    \cn{ProvidesWhatever} statement will get a dummy date of
%    \verb'0000/00/00';
% \begin{center}\begin{minipage}{.9\columnwidth}\bfseries
%    \medskip
%    \tex/ system administrators who want to ensure
%    maximal accuracy of the snapshot information should therefore make
%    it a practice to use \cn{ProvidesFile} in \fn{.cfg} files and other
%    local files that might have an impact on the output fidelity of
%    their documents.
% \end{minipage}\end{center}
%
%    A couple of shorthand forms.
%    \begin{macrocode}
\let\@xp=\expandafter \let\@nx=\noexpand
%    \end{macrocode}
%
%    A function to compare two strings and return FT or TT (for use with
%    \cs{if}).
%    \begin{macrocode}
\def\str@cmp#1#2\str@cmp#3{%
  \if #1#3\else F\@car\fi \str@cmp#2\str@cmp
}
\def\string@equal#1#2{%
  \str@cmp#1\relax\str@cmp#2{\relax\@gobbletwo}\@nil TT%
}
%    \end{macrocode}
%
%  \begin{macro}{\RequireVersions}
%
%    Optional argument of \cn{RequireVersions} allows assigning a name to a
%    particular collection of files. This might be useful for setting a
%    \tex/ inputs path.
%    \begin{macrocode}
\newcommand{\RequireVersions}[2][]{}%
\renewcommand{\RequireVersions}[1][]{%
  \def\snap@check{\snap@compare@versions}%
  \toks@{#1}\afterassignment\snap@storem
  \let\@let@token=
}
\def\snap@storem{%
  \ifx\@let@token\bgroup
  \else
    \PackageError{snapshot}{Expected a `\@charlb' character here}\@ehc
    \@xp\@gobblefour
  \fi
  \futurelet\@let@token\snap@branch
}
\@onlypreamble\RequireVersions
%    \end{macrocode}
%
%    \begin{macrocode}
\let\snap@check\@gobble
%    \end{macrocode}
%
%    \begin{macrocode}
\def\snap@finish{\toks@\bgroup}
%    \end{macrocode}
%
%    \begin{macrocode}
\def\snap@branch{%
  \ifx\@let@token\egroup
    \@xp\snap@finish
  \else\ifx\@let@token *%
    \let\reserved@c\snap@store@version
  \else\ifx\@let@token\@sptoken
    \lowercase{\def\reserved@c} {\futurelet\@let@token\snap@branch}%
  \else
    \let\reserved@c\snap@store@error
  \fi\fi\fi
  \reserved@c
}
%    \end{macrocode}
%
%    \begin{macrocode}
\def\snap@store@error#1{%
  \PackageError{snapshot}{Expected '*' here, not '#1'}\@ehc
}
\@onlypreamble\snap@store@error
%    \end{macrocode}
%
%    \begin{macrocode}
\def\snap@store@version #1#2#3#4{%
  \@xp\snap@store@b\csname snapx@#2\endcsname{#2}{#3}{#4}%
}
\@onlypreamble\snap@store@version
%    \end{macrocode}
%
%    Detection of e-\LaTeX{}, pdf\LaTeX{}, pdfe\LaTeX{}, and Lambda
%    [Omega]:
%    \begin{macrocode}
\ifx\OmegaVersion\@@undefined
\else
  \edef\snapshotApplication{%
    {Omega}\space\space\space
    {0000/00/00 v\OmegaVersion}%
  }%
\fi
\ifx\eTeXversion\@@undefined
\else
  \edef\snapshotApplication{%
    {eTeX}\space\space\space\space
    {0000/00/00 v\number\eTeXversion\eTeXrevision}%
  }%
\fi
\ifx\pdftexversion\@@undefined
\else
  \edef\snapshotApplication{%
    \ifx\eTeXversion\@@undefined
      {pdfTeX}\space\space
    \else
      {pdfeTeX}\space
    \fi
      {0000/00/00 v0.\number\pdftexversion\pdftexrevision}%
  }%
\fi
%    \end{macrocode}
%    If none of the above information is available, the exact version
%    number of \TeX{} is not accessible from inside \latex/. We then
%    fall back to using a nominal date of 1990/03/25, which is when
%    version 3.0 of \fn{tex.web} was released by Knuth.
%    \begin{macrocode}
\@ifundefined{snapshotApplication}{%
  \edef\snapshotApplication{%
    {TeX}\space\space\space\space\space
    {1990/03/25 v3.x}%
  }%
}{}
%    \end{macrocode}
%
%    \begin{macrocode}
\def\@fmtextension{fmt}
\def\@tfmextension{tfm}
\edef\snapx@package{.\@pkgextension}
\edef\snapx@class{.\@clsextension}
\edef\snapx@format{.\@fmtextension}
\edef\snapx@tfm{.\@tfmextension}
\long\def\snapx@ignore{}
\let\snapx@application=\snapx@ignore
\let\snapx@file=\@empty
\let\snapx@end\@@end
\expandafter\let\csname snapx@-------\endcsname\snapx@end
%    \end{macrocode}
%
%    For a package named \fn{foo.sty}, this function defines
%    \cs{rqv@foo.sty} to hold the date and version information.
%    \begin{macrocode}
\def\snap@store@b#1#2#3#4{%
  \ifx#1\snapx@end
    \@xp\snap@finish
  \else
    \ifx#1\relax \let#1\@empty\fi
    \def\@tempa##1 ##2 ##3\@nil{##1 ##2}%
%    \end{macrocode}
%
%    \begin{macrocode}
    \ifx#1\snapx@application
      \@xp\xdef\csname rqv@#3#1\endcsname{\@tempa#4 v?.? ? \relax\@nil}%
    \else
      \xdef\rqv@list{\rqv@list{#3#1}}%
      \@xp\xdef\csname rqv@#3#1\endcsname{\@tempa#4 v?.? ? \relax\@nil}%
      \snap@intest{#3}{#1}%
      \ifx#1\snapx@format \snap@check{#3.fmt}%
%    \end{macrocode}
%    Test if current file is \fn{snapshot.sty}. Need to pre-expand the
%    extension part to ensure the test is correct.
%    \begin{macrocode}
      \else
        \edef\@tempa{\@nx\string@equal{snapshot.sty}{#3#1}}%
        \if\@tempa \snap@selfcheck \fi
      \fi
    \fi
  \fi
  \futurelet\@let@token\snap@branch
}
\@onlypreamble\snap@store@b
%    \end{macrocode}
%  \end{macro}
%
%    Default setup is geared to write the dependency list to a \fn{.dep}
%    file. The option \opt{log} means write it to the \tex/ log instead.
%    \begin{macrocode}
\def\snap@write{\immediate\write\snap@out}
\let\snap@out\sixt@@n % fallback, probably never used
%    \end{macrocode}
%
%    \begin{macrocode}
\DeclareOption{dep}{%
  \def\snap@write{\immediate\write\snap@out}%
}
%    \end{macrocode}
%
%    \begin{macrocode}
\DeclareOption{log}{%
  \let\snap@write\typeout
}
%    \end{macrocode}
%
%    The purpose of the `test' option is to support a separate testing
%    procedure for resolving file names and pre-checking version
%    numbers. See \secref{test} for more information.
%    \begin{macrocode}
\let\snap@intest=\@gobbletwo
\DeclareOption{test}{\def\snap@intest{True}}
%    \end{macrocode}
%
%    For each font used by a document, we would like to list the
%    \fn{.tfm} file name and checksum. If \tex/ provided a
%    \cn{fontchecksum} primitive similar to \cn{fontname} that could be
%    used to get the checksum of any font, it would just about be
%    feasible to do this entirely from within \latex/. As a partial
%    solution we could at least generate the list of font file names, to
%    make it easier for an external utility to add the checksums.
%
%    In practice, extracting font names and checksums from the \fn{.dvi}
%    file will probably work well enough, leaving no work to be done by
%    the \pkg{snapshot} package in this area. But theoretically speaking
%    the output of a document could be affected by font metric files
%    that are loaded during \latex/ processing but that do not show up
%    in the \fn{.dvi} file.
%    \begin{macrocode}
\DeclareOption{tfm}{%
  \typeout{Option 'tfm' not implemented yet [1999/09/23]}%
}
%    \end{macrocode}
%
%    Warnings and errors.
%    \begin{macrocode}
\def\snap@mismatch@warning#1#2#3{\PackageWarningNoLine{#1}{#2}}
\def\snap@mismatch{\snap@mismatch@warning}
%    \end{macrocode}
%
%    \begin{macrocode}
\DeclareOption{error}{%
  \def\snap@mismatch{\PackageError}%
  \def\snap@selfcheck{\snap@selfcheck@a}%
  \ifx\snap@select\@empty \let\snap@select\snap@select@all \fi
}
%    \end{macrocode}
%
%    \begin{macrocode}
\DeclareOption{warning}{%
  \def\snap@mismatch{\snap@mismatch@warning}%
  \def\snap@selfcheck{\snap@selfcheck@a}%
  \ifx\snap@select\@empty \let\snap@select\snap@select@all \fi
}
%    \end{macrocode}
%
%    Because the exact form of the version number is not mandated by
%    \LaTeX{}, just take the first two ``words'' delimited by spaces.
%    And take a little extra care to properly handle multiple spaces
%    between the words.
%    \begin{macrocode}
\def\snap@select@all#1#2 #3#4 #5\@nil{#1#2 #3#4}
\let\snap@select\@empty
%    \end{macrocode}
%
%    If the naming conventions seem a little peculiar here, it's because
%    I had to add some pieces later that I didn't think of initially,
%    and I wanted to minimize the chances of compatibility problems for
%    client packages [mjd,2002-11-04].
%    \begin{macrocode}
\def\snap@seldate#1#2 #3\@nil{#1#2}%
\def\snap@selversion#1#2 #3{\snap@select@version #3}%
\def\snap@selmajor#1#2 #3{\snap@select@major #3}%
%    \end{macrocode}
%
%    \begin{macrocode}
\DeclareOption{date}{\let\snap@select=\snap@seldate}
%    \end{macrocode}
%
%    \begin{macrocode}
\def\snap@select@version#1{%
  \ifodd 0#11 \@xp\snap@sva\@xp#1\else\@xp\snap@select@version\fi
}
\def\snap@sva#1.#2 #3\@nil{#1.#2}
\def\snap@select@major#1{%
  \ifodd 0#11 \@xp\snap@svm\@xp#1\else\@xp\snap@select@major\fi
}
\def\snap@svm#1.#2\@nil{#1}
%    \end{macrocode}
%
%    \begin{macrocode}
\DeclareOption{version}{\let\snap@select\snap@selversion}
\DeclareOption{major-version}{\let\snap@select\snap@selmajor}
%    \end{macrocode}
%
%    \begin{macrocode}
\def\snap@bbinfo{01}
\DeclareOption{bbinfo}{\def\snap@bbinfo{00}}
%    \end{macrocode}
%
%    Give this an inert definition, for the time being, until we are
%    ready to do the split.
%    \begin{macrocode}
\let\snap@splitter=?
\AtBeginDocument{%
  \xdef\@filelist{\@filelist\snap@splitter}%
}
%    \end{macrocode}
%
%    \begin{macrocode}
\let\snap@selfcheck\@empty
\let\snap@selfcheck@a\@empty
%    \end{macrocode}
%
%    The \opt{self-warning} option would normally be used in conjunction
%    with the \opt{error} option.
%    \begin{macrocode}
\DeclareOption{self-warning}{%
  \def\snap@selfcheck{%
    \begingroup
    \def\snap@mismatch{\snap@mismatch@warning}%
    \snap@selfcheck@a
    \endgroup
  }
}
%    \end{macrocode}
%
%    \begin{macrocode}
\ExecuteOptions{warning}
\ProcessOptions\relax
%    \end{macrocode}
%
%    We need the following patch to make up for the fact that
%    \cs{@pkgextension} and \cs{@clsextension} are marked in the \latex/
%    kernel as \qq{only preamble}.
%    \begin{macrocode}
\edef\snap@restore@extensions{%
  \def\@nx\@pkgextension{\@pkgextension}%
  \def\@nx\@clsextension{\@clsextension}%
}
%    \end{macrocode}
%
%    Pad filename strings out to 8+3 length so that the list will look
%    pretty.
%    \begin{macrocode}
\def\snap@pad#1#2#3#4#5#6#7#8#9{\snap@pad@a{#1#2#3#4#5#6#7#8#9}}
\def\snap@pad@a#1#2#3#4#5\@nil{\snap@pad@b#1#2#3#4\space\@nil}
\def\snap@pad@b#1\space#2\@nil#3{\def#3{#2}}
%    \end{macrocode}
%
%    First stage: discard leading spaces before the first and second
%    nonspace strings in the argument. Take the first nonspace string as
%    the date. Since we only do equal/not-equal testing on dates, it
%    does not seem essential to test if it is really a valid date string
%    or not (yyyy/mm/dd).
%    \begin{macrocode}
\def\snap@trim@version#1#2 #3{#1#2 \snap@trim@b #3}
%    \end{macrocode}
%    Second stage: Scan for a version number. In order to handle some
%    idiosyncratic cases, such as url.sty version 1.4, we can't simply
%    take the second nonspace string as the version number but need to
%    look for a leading digit.
%    \begin{macrocode}
\def\snap@trim@b#1{\ifodd 0#11 v#1\@xp\snap@trim@c\fi \snap@trim@b}
%    \end{macrocode}
%    Arg 1 here is \cs{snap@trim@b}, which we just need to discard.
%    \begin{macrocode}
\def\snap@trim@c#1#2 #3\@nil{#2}
%    \end{macrocode}
%
%    \begin{macrocode}
\let\rqv@list=\@empty
%    \end{macrocode}
%
%    If \verb'\fmtname.fmt' is not already in the file list, add it.
%    \begin{macrocode}
\edef\@tempc#1\fmtname{#1\fmtname}\@tempc
\def\@tempa#1,\fmtname.fmt,#2#3\@nil{#2}
\edef\@tempb{\@nx\@tempa,\@filelist,\fmtname.fmt,}
\if ?\@tempb?\@nil
  \edef\@filelist{\fmtname.fmt,\@filelist}%
  \def\@tempc{LaTeX2e}%
  \@xp\edef\csname ver@\fmtname.fmt\endcsname{%
    \fmtversion\space
    v\ifx\@tempc\fmtname 2.e\else ?.?\fi
  }%
\fi
%    \end{macrocode}
%    Ensure that files get recorded.
%    \begin{macrocode}
\listfiles
%    \end{macrocode}
%
%    \begin{macrocode}
\def\snap@doit#1{%
  \begingroup
  \ifx\delimiter#1\delimiter
  \else
    \filename@parse{#1}%
    \let\@tempd\@empty
    \ifx\filename@ext\relax
      \def\@tempa{file}\def\@tempb{~~~}%
    \else\ifx\filename@ext\@pkgextension
      \def\@tempa{package}\let\@tempb\@empty
    \else\ifx\filename@ext\@clsextension
      \def\@tempa{class}\def\@tempb{~~}%
    \else\ifx\filename@ext\@fmtextension
      \def\@tempa{format}\def\@tempb{~}%
    \else\ifx\filename@ext\@tfmextension
      \def\@tempa{tfm}\def\@tempb{~~~~}%
    \else
      \def\@tempa{file}\edef\@tempd{.\filename@ext}%
      \def\@tempb{~~~}%
    \fi\fi\fi\fi\fi
    \@xp\let\@xp\@tempe
      \csname ver@\filename@base %
        \ifx\filename@ext\relax\else.\filename@ext\fi\endcsname
    \ifx\@tempe\@empty \let\@tempe\relax \fi
    \edef\@tempe{%
      \ifx\@tempe\relax 0000/00/00 v0.0%
      \else
        \@xp\@xp\@xp\snap@trim@version\@xp\@tempe\space v0.0 v0.0 \@nil
      \fi
    }%
    \edef\@tempc{\filename@area\filename@base\@tempd}% full file name
    \@xp\snap@pad\@tempc\space~~~~~~~~~~~~~~~~\@nil\@tempd
    \let~\space
    \snap@write{\space\space *{\@tempa}\@tempb{\@tempc}\@tempd{\@tempe}}%
  \fi
  \aftergroup\snap@doit
  \endgroup
}%
%    \end{macrocode}
%
%    \begin{macrocode}
\def\snap@bracify#1#2,{%
  \ifx\@empty#1\expandafter\@gobble\else {#1#2}\fi \snap@bracify
}
\def\snap@splitter@a{%
  \iffalse{{\fi }}% close current file name, end definition
  \xdef\specific@files{%
    \iffalse}\fi
    \specific@files
    \expandafter\@gobble\string % discard one closing brace
}
%    \end{macrocode}
%
%    \begin{macrocode}
\def\snap@fdcheck#1{%
  \ifx\delimiter#1\@xp\@gobble
  \else\snap@fda#1\@empty.fd\@empty ?\@nil
  \fi
  \snap@fdcheck
}
\def\snap@fda#1.fd\@empty#2#3\@nil{%
  \if ?#2%
    \xdef\specific@files{\specific@files {#1}}%
  \else
    \xdef\general@files{\general@files {#1.fd}}%
  \fi
}
%    \end{macrocode}
%
%    \begin{macrocode}
\let\general@files\@empty
\let\specific@files\@empty
%    \end{macrocode}
%
%    The \cn{SpecialInput} command is related to the \opt{packages-only}
%    option. Apart from some ad hoc handling for \fn{.fd} files that get
%    loaded on demand, all files that are input after
%    \verb'\begin{document}' are put into the specific-files list, and
%    all files before \verb'\begin{document}' go into the general-files
%    (packages-only) list. If there is a macro file for a book (say),
%    that contains definitions specific to that book, and that is loaded
%    in the preamble, loading it with \cs{SpecialInput} will cause it to
%    go in the specific-files list.
%    \begin{macrocode}
\newcommand{\SpecialInput}[1]{%
  \xdef\specific@files{\specific@files{#1}}%
  \@@input#1\relax
}
%    \end{macrocode}
%
%    Our definition of \cs{@dofilelist} does not retain much resemblance
%    to the original in the \LaTeX{} kernel.
%    \begin{macrocode}
\def\@dofilelist{%
  \snap@restore@extensions
  \xdef\general@files{\@xp\snap@bracify \@filelist \@empty,\@empty,}%
  \let\snap@splitter\snap@splitter@a
  \xdef\general@files{\general@files}%
  \let\@tempa\specific@files \global\let\specific@files\@empty
  \@xp\snap@fdcheck\@tempa{\delimiter}%
  \ifx\rqv@list\@empty
  \else \rqv@compare@lists
  \fi
  \ifx\snap@write\typeout
  \else
    \newwrite\snap@out
    \immediate\openout\snap@out=\jobname.dep \relax
  \fi
  \snap@write{\string\RequireVersions\@charlb}%
  \snap@write{\space\space *{application}%
    \snapshotApplication
  }%
  \@xp\snap@doit\general@files{\delimiter\aftergroup\@gobble\@gobble}%
  \ifx\specific@files\@empty
  \else
    \snap@specific
  \fi
  \snap@write{\@charrb}%
  \ifx\snap@write\typeout
  \else \immediate\closeout\snap@out
    \typeout{Dependency list written on \jobname.dep.}%
  \fi
}%
%    \end{macrocode}
%
%    \begin{macrocode}
\def\snap@specific{%
  \snap@write{ \space *{-------}{Document-specific files:}{----}}%
  \@xp\snap@doit\specific@files{\delimiter\aftergroup\@gobble\@gobble}%
}
%    \end{macrocode}
%
%    The \cs{rqv@compare@lists} function checks to see if any files
%    are found only in the RequireVersions list or only in the
%    \cs{general@files} list.
%    \begin{macrocode}
\def\rqv@condense#1{%
  \@xp\ifx\csname ver@#1\endcsname\N
  \else
    \edef\L{\L{#1}}%
    \@xp\let\csname ver@#1\endcsname=\N
  \fi
  \rqv@condense
}
\def\rqv@condend{\endcsname ?\fi
  \@xp\@xp\@xp\@gobbletwo\csname @xp\iftrue}
%    \end{macrocode}
%
%    \begin{macrocode}
\def\rqv@overloaded#1{%
  \snap@mismatch{snapshot}{^^J%
    File #1 loaded though not in \noexpand\RequireVersions list%
  }\@ehc
}
%    \end{macrocode}
%
%    \begin{macrocode}
\def\rqv@notloaded#1{%
  \snap@mismatch{snapshot}{^^J%
    File #1 [\csname rqv@#1\endcsname] required but not loaded%
  }\@ehc
}
%    \end{macrocode}
%
%    \begin{macrocode}
\def\rqv@set#1{\@xp\let\csname ver@#1\endcsname\N \rqv@set}
\def\rqv@test#1{\csname ver@#1\endcsname{#1}\rqv@test}
%    \end{macrocode}
%
%    \begin{macrocode}
\def\rqv@compare@lists{%
  \begingroup
%    \end{macrocode}
%    Clear up duplicate file names (just in case) to avoid redundant
%    warning messages. This should seldom be necessary in practice.
%    \begin{macrocode}
  \def\N{1}\let\L\@empty
  \@xp\rqv@condense\rqv@list\rqv@condend
  \global\let\rqv@list=\L
  \def\N{2}\let\L\@empty
  \@xp\rqv@condense\general@files\rqv@condend
  \global\let\general@files=\L
%    \end{macrocode}
%    Let's make a shorthand for the code that terminates our recursion.
%    \begin{macrocode}
  \def\T{\@firstoftwo{\endcsname\@empty\@gobbletwo}}%
%    \end{macrocode}
%    Set all the loaded general files to an error function.
%    \begin{macrocode}
  \let\N\rqv@overloaded \@xp\rqv@set\general@files \T
%    \end{macrocode}
%    Set all the required files to an ignore function.
%    \begin{macrocode}
  \let\N\@gobble \@xp\rqv@set\rqv@list \T
%    \end{macrocode}
%    Execute all the general files.
%    \begin{macrocode}
  \@xp\rqv@test\general@files{\endcsname\csname @gobbletwo}%
%    \end{macrocode}
%    And now do essentially the same thing in the reverse direction.
%    \begin{macrocode}
  \let\N\rqv@notloaded \@xp\rqv@set\rqv@list \T
  \let\N\@gobble \@xp\rqv@set\general@files \T
  \@xp\rqv@test\rqv@list{\endcsname\csname @gobbletwo}%
  \endgroup
}
%    \end{macrocode}
%
%    Compensate for a bug in old versions of \fn{amsgen.sty}. This is a
%    little tricky.
%
%    Old version: \cs{ver@amsgen}=1996/10/29 v1.2b
%
%    New version: \cs{ver@amsgen.sty}=1999/11/30 v2.0
%    \begin{macrocode}
%\@namedef{ver@amsgen.sty}{1996/10/29 v1.2b}
\AtBeginDocument{%
  \@ifundefined{ver@amsgen}{}{%
    \@xp\let\csname ver@amsgen.sty\@xp\endcsname
                    \csname ver@amsgen\endcsname
  }%
}
%    \end{macrocode}
%
%    Because \cs{ProvidesFile} is used in \fn{.fd} files which are
%    normally read with special catcodes, there tend to
%    be problems with whitespace characters being erroneously lost from
%    the second argument. Since we have to put in a \cs{snap@check} call
%    anyway, while we're at it let's fix a bug of this type that
%    affected some older versions of \latex/.
%    \begin{macrocode}
\def\ProvidesFile#1{%
  \def\snap@checker{\snap@check{#1}}%
  \begingroup
    \aftergroup\snap@checker
    \catcode`\ 10
%    \end{macrocode}
%    Added guards from 2001/06/01 version of \latex/.  These are
%    necessary because, for example, \pkg{inputenc} sets
%    \cs{endlinechar} to a large (nonvalid character) value when
%    reading input encoding files.  The guards prevent an ``invalid
%    character'' error.
%    \begin{macrocode}
    \ifnum\endlinechar < 256
        \ifnum \endlinechar>\m@ne
            \catcode\endlinechar 10
        \fi
    \fi
    \@makeother\/%
    \@makeother\&%
    \kernel@ifnextchar[{\snap@providesfile{#1}}{\snap@providesfile{#1}[]}%
}
%    \end{macrocode}
%
%    Normally the string found in the second arg of \cs{ProvidesFile}
%    (for a nongraphics file) would begin with the usual date string.
%    The \cs{includegraphics} command, however, begins the second arg
%    with \texttt{Graphic file} instead. This test therefore just checks
%    if the first two letters are \texttt{Gr}; this is enough,
%    ordinarily, for us to conclude that we are dealing with a graphic
%    file.
%    \begin{macrocode}
\def\snap@graphic@test#1#2#3\@nil{r\if G#1#2\else X\fi}
%    \end{macrocode}
%
%    \begin{macrocode}
\def\snap@providesfile#1[#2]{%
  \wlog{File: #1 #2}%
  \if\snap@graphic@test#2@@\@nil
    \snap@record@graphic#1\relax #2 (type ??)\@nil
  \else
    \expandafter\xdef\csname ver@#1\endcsname{#2}%
  \fi
  \endgroup
}
%    \end{macrocode}
%
%    This is what \cn{includegraphics} does to record graphic file
%    information.
%\begin{verbatim}
%\@providesfile #1[#2]->
%\wlog {File: #1 #2}\expandafter \xdef \csname ver@#1\endcsname {#2}
%\endgroup
%#1<-\Gin@base \Gin@ext
%#2<-Graphic file (type eps)
%\end{verbatim}
%
%    Check the graphics info.
%    \begin{macrocode}
\def\snap@record@graphic#1\relax #2(type #3)#4\@nil{%
  \expandafter\xdef\csname ver@#1\endcsname{%
    Graphic%
    \if\snap@bbinfo :bb=\Gin@llx/\Gin@lly/\Gin@urx/\Gin@ury\fi
    \space v0.0%
  }%
}
%    \end{macrocode}
%
%    \begin{macrocode}
\def\@pr@videpackage [#1]{%
  \expandafter\xdef\csname ver@\@currname.\@currext\endcsname{#1}%
  \ifx\@currext\@clsextension
    \typeout{Document Class: \@gtempa\space#1}%
  \else
    \wlog{Package: \@gtempa\space#1}%
  \fi
  \snap@check{\@currname.\@currext}%
}
%    \end{macrocode}
%
%    \begin{macrocode}
\def\snap@selfcheck@a{\snap@check{snapshot.sty}}
%    \end{macrocode}
%
%    \begin{macrocode}
\def\@nofmt#1.fmt.#2 {#1 }
%    \end{macrocode}
%
%    \begin{macrocode}
\def\snap@mismatch@a#1#2#3{%
  \snap@mismatch{snapshot}{^^J%
    \space\space Required version #2 of \@nofmt#1.fmt. and^^J%
    \space\space provided version #3 do not match%
  }\@ehc
}
%    \end{macrocode}
%
%    When comparing \cs{rqv@foo.sty} (information from a previous
%    \latex/ run) with \cs{ver@foo.sty} (information from current run),
%    we first call \cs{snap@trim@version} on the latter to clear away
%    any idiosyncrasies in the contents.
%    \begin{macrocode}
\def\snap@compare@versions#1{%
  \begingroup
    \@ifundefined{rqv@#1}{}{%
      \edef\0{\csname rqv@#1\endcsname}%
      \edef\1{\csname ver@#1\endcsname}%
      \edef\1{\@xp\snap@trim@version\1 v0.0 v0.0 \@nil}%
      \edef\@tempa{\@xp\snap@select\0 v0.0 v0.0 \@nil}%
      \edef\@tempb{\@xp\snap@select\1 v0.0 v0.0 \@nil}%
      \ifx\@tempa\@tempb
      \else
        \edef\@tempd{\@nx\snap@mismatch@a{#1}{\@tempa}{\@tempb}}%
        \@xp\@tempd
      \fi
    }%
  \endgroup
%    \end{macrocode}
%    When the \opt{test} option is in effect, jump out of the current
%    file instead of continuing.
%    \begin{macrocode}
  \snap@test@abort
}
%    \end{macrocode}
%
%    \begin{macrocode}
\let\snap@test@abort=\@empty
%    \end{macrocode}
%
% \section{Compatibility}
%
%    Suppose that I have a \lat/ document containing a
%    \cs{RequireVersions} statement generated by \pkg{snapshot} and I
%    send this to my colleague who, we believe, has a \lat/ setup that
%    is for our purposes identical. Suppose that our belief is erroneous
%    in the following way: My colleague has a newer version of
%    \pkg{snapshot} and a newer version of one of the affected files.
%
%    Here is what we \strong{don't} want to happen: That the differing
%    version of snapshot would cause the other differing file to be
%    accepted without demur.
%
%    Conversely, if it is I who have the newer version of
%    \pkg{snapshot}, the main concern is that some difference in the
%    contents of the \cs{RequireVersions} statement would lead to an
%    error when my colleague attempts to process the document.
%
% \section{In conclusion}
%
%    \begin{macrocode}
\ifx\snap@select\@empty
  \let\snap@compare@versions\@gobble
  \let\snap@check\@gobble
\fi
%    \end{macrocode}
%
%    Fallback for a command that is sometimes used in AMS journal
%    production.
%    \begin{macrocode}
\providecommand{\controldates}[1]{}
%    \end{macrocode}
%
%%%%%%%%%%%%%%%%%%%%%%%%%%%%%%%%%%%%%%%%%%%%%%%%%%%%%%%%%%%%%%%%%%%%%%%%
% \section{And finally \dots}\label{test}
%
%    If the embedded \cs{RequireVersions}
%    data in a \lat/ document is extracted to a separate file, and
%\begin{verbatim}
%\RequirePackage[test]{snapshot}
%\end{verbatim}
%    is added at the top, then the file can be run as a small separate
%    \LaTeX{} job that will, among other things, produce in the log file
%    a nice list of fully resolved file names\mdash sort of a limited,
%    but system-independent variant of the \fn{kpsewhich} idea.
%    \begin{macrocode}
\ifx\snap@intest\@gobbletwo \endinput \fi
%    \end{macrocode}
%
%    Some old, ill-behaved packages might throw in a \cs{makeatother} at
%    the end which can cause problems for the next file that comes along
%    when testing.
%    \begin{macrocode}
\def\restore@some@catcodes{}
\def\save@some@catcodes{%
  \edef\restore@some@catcodes{%
    \catcode\number`\@=\number\catcode`\@
    \catcode\number`\"=\number\catcode`\"
    \catcode\number`\^=\number\catcode`\^
    \catcode\number`\_=\number\catcode`\_
    \relax
  }%
}
%    \end{macrocode}
%
%    Some typical calls of \cs{snap@intest}:
%\begin{verbatim}
% \snap@intest{LaTeX2e}{\snapx@format}
% \snap@intest{snapshot}{\snapx@package}
% \snap@intest{mcom-l}{\snapx@class}
% \snap@intest{amsmath}{\snapx@package}
% \snap@intest{umsa.fd}{\snapx@file}
% \snap@intest{pictex}{\snapx@file}
%\end{verbatim}
%
%    The extant public versions of the following files (in the teTeX
%    distribution, at least) are known to be problematic when we are
%    trying to read a \cs{ProvidesWhatever} line from the top of the
%    file: psfig.sty, pictex.sty, pictex.tex, epic.sty, amstex.sty,
%    xy.tex. Either (a) they don't have a \cs{ProvidesWhatever} line at
%    all, or (b) they include some code before the \cs{ProvidesWhatever}
%    line that makes some assumption true in normal processing but false
%    in snapshot-test processing. E.g., there is code in \fn{amstex.sty}
%    that assumes \cs{documentstyle} or \cs{documentclass} was already
%    executed and that the \cs{if@compatibility} switch got set
%    accordingly.
%
%    \begin{macrocode}
\def\snap@intest#1#2{%
  \message{^^J}%
  \begingroup \edef\0{#1#2}\def\9{latex209.def}%
  \ifx\0\9\global\@compatibilitytrue \fi
  \ifx#2\snapx@format
%    \end{macrocode}
%    If arg1 + arg2 = ``LaTeX2e.fmt'', the calling function
%    \cs{snap@storeb} will run \cs{snap@check} separately.
%    This is a crude way of making things work in that case without
%    much extra trouble.
%    \begin{macrocode}
    \def\snap@test@abort{\endgroup}%
  \else
    \edef\N{%
      \noexpand\snap@intest@b{#1#2}%
        {#1}{\@xp\@gobble#2\@empty}%
        {\csname rqv@#1#2\endcsname}}%
    \expandafter\endgroup\N
  \fi
}
%    \end{macrocode}
%
%    \begin{macrocode}
\def\snap@intest@b#1#2#3#4{%
  \def\@currname{#2}\def\@currext{#3}%
  \begingroup \lccode`\/=`\0\relax\lowercase{\endgroup
  \ifnum\snap@seldate#4 00 0\@nil>\z@
    }% matches \lowercase
    \save@some@catcodes
    \@@input #1 \relax
    \restore@some@catcodes
  \else
    \snap@specialtest{#1}{#4}%
  \fi
}
%    \end{macrocode}
%
%    \begin{macrocode}
\def\snap@specialtest#1#2{%
  \fake@input{#1}%
}
%    \end{macrocode}
%
%    \begin{macrocode}
\def\fake@input#1{%
  \begingroup
  % Ensure that outer \foo or unmatched braces don't trip us up
  \catcode`\\=12 \catcode`\{=12 \catcode`\}=12
  \endinput
%    \end{macrocode}
%    Note that these definitions of \cs{G} and \cs{?} are local, and
%    recall that one-letter cs names don't use up hash table entries.
%    \begin{macrocode}
  \def\G{\@car\endgroup}%
  \expandafter\futurelet\expandafter\?\expandafter\G\@@input#1 \relax\@nil
}
%    \end{macrocode}
%
%    \begin{macrocode}
\let\snap@test@abort=\endinput
\let\snap@selfcheck=\@empty
%    \end{macrocode}
%
%    There's an extra close-brace left hanging around at the end, but I
%    guess we don't care.
%    \begin{macrocode}
\def\snap@finish{%%
  \endgroup \message{^^J}%
  \def\X##1{##1,\X}%
  \edef\@filelist{\@xp\X\rqv@list{\@gobbletwo}}%
  \def\X##1,?{##1}\edef\@filelist{\@xp\X\@filelist ?}%
  \@dofilelist
  \@@end
}%
%    \end{macrocode}
%
%    \begin{macrocode}
\def\snap@mismatch#1#2#3{}
%    \end{macrocode}
%
%    Problematic: xy.sty, because it calls xy.tex before it calls
%    \cs{ProvidesPackage}. And pictex.tex because it doesnt use
%    \cs{ProvidesFile} at all.
%    \begin{macrocode}
\renewcommand{\RequireVersions}[2][]{%
  \begingroup
  \makeatletter
  \def\snap@check{\snap@compare@versions}%%
  \let\snapx@tfm=\snap@ignore
%    \end{macrocode}
%    This seems to help, with english.ldf for example, to prevent an
%    endless loop when attempting to load babel.def.
%    \begin{macrocode}
  \def\ProvidesLanguage##1{\ProvidesFile{##1.ldf}}%
  \iffalse{\fi \futurelet\@let@token\snap@branch #2}%
  \endgroup
}
%    \end{macrocode}
%
%    The usual \cs{endinput} to ensure that random garbage at the end of
%    the file doesn't get copied by \fn{docstrip}.
%    \begin{macrocode}
\endinput
%    \end{macrocode}
%
% \CheckSum{1209}
% \Finale

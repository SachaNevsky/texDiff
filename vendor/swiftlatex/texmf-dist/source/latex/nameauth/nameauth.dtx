% \iffalse meta-comment
% vim: textwidth=75
%<*internal>
\iffalse
%</internal>
%<*readme>
|
--------:| ----------------------------------------------------------------
nameauth:| Name authority mechanism for consistency in text and index
  Author:| Charles P. Schaum
  E-mail:| charles dot schaum@comcast.net
 License:| Released under the LaTeX Project Public License 1.3c or later
     See:| http://www.latex-project.org/lppl.txt


Short description:

The nameauth package automates the correct formatting and indexing of names
for professional writing. This aids the use of a name authority and the
editing process without needing to retype name references.

Files		 Distribution

README       This file
nameauth.pdf Documentation
examples.tex Some longer example macros from the documentation

Makefile     Automates building with GNU make 3.81
nameauth.dtx Documented LaTeX file containing both code and documentation

Manual Installation

Download the package from

www.ctan.org/tex-archive/macros/latex/contrib/nameauth

Unpack nameauth.zip in an appropriate directory.

If you have a make utility compatible with GNU make, either in
GNU/Linux, a BSD variant, OSX, or Cygwin in Windows you can type

	make inst

to install the package into your $TEXMFHOME tree or

	make install

to install the package into your $TEXMFLOCAL tree for all users.
The latter requires sudo privileges.

Other useful targets include:

	(release process)

	make release			The default target, same as just ``make''.

	make clean				Removes all intermediate files. Left are
							the files listed above plus nameauth.sty.

	make distclean			Leave only nameauth.dtx, examples.tex,
							and Makefile.

	make zip				Generate a zip file ready for distribution.

	(testing process)

	make testing			Release files, plus compiles examples.tex.

	make release ENGINE=<command>
							Here, <command> can be pdflatex (default),
							xelatex, lualatex, dvilualatex, or latex.

	make testing ENGINE=<command>		See above.
	
It is not necessary, however, to use GNU make. One can generate
the package files manually. Since the files nameauth.ins and README.txt
are contained in the .dtx file itself, the first step is to generate
the installer driver nameauth.ins, plus the file README.txt, which will
also trigger the extraction of nameauth.sty and produce the first pass of
the package documentation nameauth.pdf:

	pdflatex -shell-escape -recorder -interaction=batchmode nameauth.dtx

Next one adds a table of contents and all cross-references, this also
should finalize page numbers for glossary and index input files:

	pdflatex --recorder --interaction=nonstopmode nameauth.dtx
	
The next commands generate the glossary/index output files:
	
	makeindex -q -s gglo.ist -o nameauth.gls nameauth.glo
	makeindex -q -s gind.ist -o nameauth.ind nameauth.idx
	
The final two commands integrate the glossary (changes) and index:
	
	pdflatex --recorder --interaction=nonstopmode nameauth.dtx
	pdflatex --recorder --interaction=nonstopmode nameauth.dtx

Now one can either keep README.txt or rename it to README, e.g.:

	mv README.txt README

Normally one creates the following directories for a user:

	$TEXMFHOME/source/latex/nameauth		dtx file
	$TEXMFHOME/tex/latex/nameauth			sty file
	$TEXMFHOME/doc/latex/nameauth			pdf file, README, examples.tex
		
and creates the following directories for the local site:

	$TEXMFLOCAL/source/latex/nameauth		dtx file
	$TEXMFLOCAL/tex/latex/nameauth			sty file
	$TEXMFLOCAL/doc/latex/nameauth			pdf file, README, examples.tex

The above environment variables often are /usr/local/texlive/texmf-local
and ~/texmf.

The make process normally renames the README.txt file created from the
dtx file to just README by using mv (move / rename utility in the *nix
userland). Windows distributions of TeX and LaTeX often keep the txt file
because of using file extensions instead of ``magic numbers'' to identify
files.

Run mktexlsr with the appropriate level of permissions to complete the
install.

Testing notes:

See the nameauth manual.

License

This material is subject to the LaTeX Project Public License:
http://www.ctan.org/tex-archive/help/Catalogue/licenses.lppl.html

Happy TeXing!
%</readme>
%<*internal>
\fi
\def\nameofplainTeX{plain}
\ifx\fmtname\nameofplainTeX\else
  \expandafter\begingroup
\fi
%</internal>
%<*install>
\input docstrip.tex
\keepsilent
\askforoverwritefalse
\preamble

--------:| ----------------------------------------------------------------
nameauth:| Name authority mechanism for consistency in text and index
  Author:| Charles P. Schaum
  E-mail:| charles dot schaum@comcast.net
 License:| Released under the LaTeX Project Public License 1.3c or later
     See:| http://www.latex-project.org/lppl.txt

\endpreamble
\postamble

Copyright (C) 2020 by Charles P. Schaum <charles dot schaum@comcast.net>

This work may be distributed and / or modified under the
conditions of the LaTeX Project Public License (LPPL), either
version 1.3c of this license or (at your option) any later
version. The latest version of this license is in the file:

http://www.latex-project.org/lppl.txt

This work is "maintained" (as per LPPL maintenance status) by
Charles P. Schaum.

This work consists of the files nameauth.dtx, examples.tex, and a Makefile.
These files generate README.txt / README, nameauth.ins, nameauth.sty, and
nameauth.pdf, with other intermediate files, as a part of this work. See
the README.txt or README for more information.
\endpostamble

\usedir{tex/latex/nameauth}
\generate{
  \file{\jobname.sty}{\from{\jobname.dtx}{package}}
}
%</install>
%<install>\endbatchfile
%<*internal>
\usedir{source/latex/nameauth}
\generate{
  \file{\jobname.ins}{\from{\jobname.dtx}{install}}
}
\nopreamble\nopostamble
\usedir{doc/latex/nameauth}
\generate{
  \file{README.txt}{\from{\jobname.dtx}{readme}}
}
\ifx\fmtname\nameofplainTeX
  \expandafter\endbatchfile
\else
  \expandafter\endgroup
\fi
%</internal>
% \fi
%
% \iffalse
%<*driver>
\ProvidesFile{nameauth.dtx}
%</driver>
%<package>\NeedsTeXFormat{LaTeX2e}[1999/12/01]
%<package>\ProvidesPackage{nameauth}
%<*package>
    [2020/02/26 3.4 Name authority mechanism for consistency in text and index]
%</package>
%<*driver>
\documentclass[11pt]{ltxdoc}
%^^A Below we check if our TL distribution has the iftex package and load it.
\IfFileExists{iftex.sty}{\usepackage{iftex}}{}
%^^A Only versions of iftex since 2019 have the macro \RequireTUTeX.
%^^A If the package is not loaded or if it is not new,
%^^A we load the older, transitional packages.
\unless\ifdefined\RequireTUTeX
  \usepackage{ifxetex}
  \usepackage{ifluatex}
  \usepackage{ifpdf}
\fi
\ifxetex %^^A \ifpdf will not be true, but a pdf results.
  \usepackage{fontspec}
  \usepackage{polyglossia}
  \setdefaultlanguage{american}
  \setotherlanguage{german}
  \newcommand\de[1]{\textgerman{#1}}
  \usepackage{tikz}
  \usepackage{tcolorbox}
\else
  \ifluatex
    \ifpdf %^^A pdf mode
      \usepackage{fontspec}
      \usepackage{polyglossia}
      \setdefaultlanguage{american}
      \setotherlanguage{german}
      \newcommand\de[1]{\textgerman{#1}}
      \usepackage{tikz}
      \usepackage{tcolorbox}
    \else %^^A dvi mode
      \IfFileExists{utf8-2018.def}{}{\usepackage[utf8]{inputenc}}
      \usepackage[TS1,T1]{fontenc}
      \usepackage[ngerman,american]{babel}
      \newcommand\de[1]{\foreignlanguage{ngerman}{#1}}
      \usepackage{lmodern}
      \usepackage{newunicodechar}
      \DeclareTextSymbolDefault{\textlongs}{TS1}
      \DeclareTextSymbol{\textlongs}{TS1}{115}
      \newunicodechar{ſ}{\textlongs}
    \fi
  \else %^^A These packages work for both pdf and dvi.
    \IfFileExists{utf8-2018.def}{}{\usepackage[utf8]{inputenc}}
    \usepackage[TS1,T1]{fontenc}
    \usepackage[ngerman,american]{babel}
    \newcommand\de[1]{\foreignlanguage{ngerman}{#1}}
    \usepackage{lmodern}
    \usepackage{newunicodechar}
    \DeclareTextSymbolDefault{\textlongs}{TS1}
    \DeclareTextSymbol{\textlongs}{TS1}{115}
    \newunicodechar{ſ}{\textlongs}
    \ifpdf %^^A These only get included for pdf.
      \usepackage{tikz}
      \usepackage{tcolorbox}
    \fi
  \fi
\fi
%^^A Set up initial page layout to fit both letter and DIN A4.
\usepackage[textwidth=137mm,textheight=237mm,right=25mm,marginparwidth=40mm,nohead]{geometry}
%^^A Include only this version of the package.
\usepackage{\jobname}[2020/02/26]
%^^A Set up all tabular information.
\usepackage{booktabs}
\usepackage{colortbl}
\usepackage{tabularx}
%^^A Set up lists.
\usepackage{enumitem}
\setlist{rightmargin=\leftmargin,itemsep=0pt}
%^^A Create indexes and set up the ``actual'' character in nameauth.
\usepackage{makeidx}
\IndexActual{=}
%^^A Dangerous bend ahead...
\usepackage{manfnt}
%^^A Two-column TOC
\usepackage[toc]{multitoc}
%^^A Some examples get some stretch to aid clarity and readability.
\usepackage{setspace}
%^^A Color helps categorize information.
\usepackage{xcolor}
\colorlet{nared}{red!50!black}
\colorlet{nagreen}{green!35!black}
\colorlet{nablue}{blue!50!black}
\colorlet{nabrown}{brown!55!black}
\colorlet{naviolet}{violet!90!black}
\colorlet{nataupe}{yellow!40!black}
\colorlet{naslate}{cyan!45!black}
%^^A For example...
\usepackage{verbatim}
%^^A Have documentation with hyperlinks
\usepackage[numbered]{hypdoc}

%^^A Let verbatim environments be numbered or unnumbered, and start or resume that.
\makeatletter
\newcommand*\ClearNum{%^^A
  \newcounter{VerbLine}\setcounter{VerbLine}{0}%^^A
  \def\verbatim@processline{\expandafter\@gobble\the\verbatim@line\par}%^^A
}
\newcommand*\StartNum{%^^A
  \setcounter{VerbLine}{0}
  \def\verbatim@processline{\stepcounter{VerbLine}\leavevmode%^^A
  \llap{\footnotesize\normalfont\theVerbLine\quad}%^^A
  \expandafter\@gobble\the\verbatim@line\par}%^^A
}
\newcommand*\ContinueNum{%^^A
  \def\verbatim@processline{\stepcounter{VerbLine}\leavevmode%^^A
  \llap{\footnotesize\normalfont\theVerbLine\quad}%^^A
  \expandafter\@gobble\the\verbatim@line\par}%^^A
}
\makeatother

%^^A Use color indicators in a couple of tables.
\newcommand\NO{\bgroup\color{lightgray}\lower 0.075ex\hbox{\rule{1.5ex}{1.5ex}}\egroup}
\newcommand\YES{\bgroup\color{darkgray}\lower 0.075ex\hbox{\rule{1.5ex}{1.5ex}}\egroup}

%^^A Stretch out some text for clarity.
\newcommand*\MyStretch{\setstretch{1.1}}
\newcommand*\MySmallStretch{\setstretch{1.05}}

%^^A Return link to the task dashboard.
\newcommand*\BigBlank{{\large\itshape\vfil\leavevmode\hfil This space is intentionally left blank.}}

%^^A Macros for marginalia.
\newcommand*\Version[1]{\unless\ifinner\marginpar{\strut\raggedleft\textsf{\bfseries#1}}\fi}
\newcommand*\VersionWarn[1]{{\unless\ifinner\marginpar{\strut\raggedleft\textsf{\bfseries#1}\break\small\dbend}\fi}}
\newcommand*\Warn{{\unless\ifinner\marginpar{\strut\small\raggedleft\dbend}\fi}}
\newcommand*\Info[1]{{\unless\ifinner\marginpar{\strut\small\raggedleft#1}\fi}}

%^^A Use this example in the body text.
\newcommand\Orphan[2]{#1(\hyperpage{#2})}
\TagName[Lost]{Name}{\,\S|Orphan{perdit}}
\TagName{foo\Name {bar}}{\,\S|hyperpage}

%^^A Redefine emphasis in the body text
\let\oldemph\emph
\let\emph\textbf

%^^A Struts for framed boxes
\newcommand*{\mystrut}{\rule[-0.25\baselineskip]{0pt}{\baselineskip}}

%^^A Set up all the shorthands, but not all the names.
\begin{nameauth}
  \< Adams & John & \noexpand\textSC{Adams} & >
  \< Aeth & & Æthelred, II & >
  \< Anth & Susan B. & Anthony & >
  \< Aris & & Aristotle & >
  \< Attil & & Attila, the Hun & >
  \< Cicero & M.T. & Cicero & >
  \< Confucius & & Confucius & >
  \< Dagb & & Dagobert & I >
  \< SDJR & Sammy & \noexpand\textSC{Davis}, \noexpand\textSC{Jr}. & >
  \< Dem & & Demetrius, I & >
  \< deSmet & Pierre-Jean & \noexpand\Fbox{\noexpand\AltCaps{d}e Smet} & >
  \< Soto & Hernando & de Soto & >
  \< DuBois & W.E.B. & Du~Bois & >
  \< AltDuBois & W.E.B. & DuBois & >
  \< Einstein & Albert & Einstein & >
  \< Eliz & & Elizabeth, I & >
  \< Fukuyama & & \textUC{Fukuyama}, Takeshi & >
  \< OFukuyama & & \textUC{Fukuyama} & Takeshi >
  \< JWG & J.W. von & Goethe & >
  \< HAR & & Harun, \noexpand\textSC{\noexpand\AltCaps{a}l-Rashid} & >
  \< Harnack & Adolf & Harnack & >
  \< Henry & & Henry & VIII >
  \< Jeff & Thomas & \noexpand\JEFF & >
  \< Lewis & Clive Staples & Lewis & >
  \< CSL & Clive Staples & Lewis & C.S. >
  \< Luth & Martin & \noexpand\textSC{Luther} & >
  \< Cath & Catherine \noexpand\AltCaps{d}e' & \noexpand\textSC{Medici} & >
  \< Mencius & & \noexpand\textSC{Mencius} & >
  \< Miyaz & & Miyazaki, Hayao & >
  \< MSens & & Miyazaki, Hayao & Sensei >
  \< Noguchi & Hideyo & Noguchi & >
  \< Pat & George S. & Patton, Jr. & >
  \< JRIII & John David & \textSC{Rockefeller},\textSC{III} & >
  \< JRIV & John David & Rockefeller, IV & >
  \< JayR & John David & Rockefeller, IV & Jay >
  \< Scipio & \noexpand\SCIPi & \noexpand\SCIPii & >
  \< OScipio & Publius & \noexpand\CSA & >
  \< Shak & \noexpand\WM & \noexpand\SHK & >
  \< Striet & John & \de{Strietelmeier} & >
  \< Ches & Chesley B. & Sullenberger, III & >
  \< Sully & Chesley B. & Sullenberger, III & Sully >
  \< Sun & & Sun, Yat-sen & >
  \< KempMed & & Thomas, à~Kempis & >
  \< KempW & Thomas & à~Kempis & >
  \< Tyson & Mike & Tyson & >
  \< Iron & Mike & Tyson & Iron Mike >
  \< VBuren & Martin & Van Buren & >
  \< Wash & George & Washington & >
  \< White & E.\,B. & White & >
  \< Yamt & & Yamamoto, Isoroku & >
  \< Yosh & & Yoshida & Shigeru >
\end{nameauth}

%^^A Add sort tags for those needing them.
\PretagName[John]{\noexpand\textSC{Adams}}{Adams, John}
\PretagName{Æthelred, II}{Aethelred 2}
\PretagName{Atatürk}{Ataturk}
\PretagName{\textit{Doctor mellifluus}}{Doctor mellifluus}
\PretagName[Charles]{\textBF{Babbage}}{Babbage, Charles}
\PretagName{Bo\"ethius}{Boethius}
\PretagName[Robert]{\textSC{Burns}}{Burns, Robert}
\PretagName[Giovanni]{d'Andrea}{Dandrea, Giovanni}
\PretagName[Sammy]{\noexpand\textSC{Davis}, \noexpand\textSC{Jr}.}{Davis, Sammy, Jr.}
\PretagName{Demetrius, I}{Demetrius 1}
\PretagName[Jacques]{De~Pamele}{Depamele, Jacques}
\PretagName[Pierre-Jean]{\noexpand\Fbox{\noexpand\AltCaps{d}e Smet}}{desmet, Pierre-Jean}
\PretagName[Hernando]{de Soto}{Desoto, Hernando}
\PretagName[W.E.B.]{Du~Bois}{Dubois, W.E.B.}
\PretagName[W.E.B.]{DuBois}{Dubois, W.E.B.}
\PretagName[Charles]{du Fresne}{Dufresne, Charles}
\PretagName{du Cange}{Ducange}
\PretagName{Elizabeth, I}{Elizabeth 1}
\PretagName{\textUC{Fukuyama}, Takeshi}{Fukuyama Takeshi}
\PretagName[Greta]{\textSC{Garbo}}{Garbo, Greta}
\PretagName{Ghazāli}{Ghazali}
\PretagName{Harun, \noexpand\textSC{\noexpand\AltCaps{a}l-Rashid}}{Harun Alrashid}
\PretagName[Thomas]{\noexpand\JEFF}{Jefferson, Thomas}
\PretagName[Ada]{\textIT{Lovelace}}{Lovelace, Ada}
\PretagName[Jan]{Łukasiewicz}{Lukasiewicz, Jan}
\PretagName[Martin]{\noexpand\textSC{Luther}}{Luther, Martin}
\PretagName[Catherine \noexpand\AltCaps{d}e']{\noexpand\textSC{Medici}}{Medici, Catherine de}
\PretagName{\noexpand\textSC{Mencius}}{Mencius}
\PretagName{\noexpand\textSC{Meng}, Ke}{Meng Ke}
\PretagName[Frenec]{Molnár}{Molnar, Frenec}
\PretagName[John David]{\textSC{Rockefeller},\textSC{III}}{Rockefeller, John David 3}
\PretagName[John David]{Rockefeller, IV}{Rockefeller, John David 4}
\PretagName[Heinrich Wilhelm]{\textSC{Rühmann}}{Ruehmann, Heinrich Wilhelm}
\PretagName[\noexpand\SCIPi]{\noexpand\SCIPii}{Scipio Africanus}
\PretagName[Publius]{\noexpand\CSA}{Cornelius Scipio Africanus}
\PretagName[\noexpand\WM]{\noexpand\SHK}{Shakespeare, William}
\PretagName[Heinz]{\textSC{Rühmann}}{Ruehmann, Heinz}
\PretagName{\textit{Snellius}}{Snellius}
\PretagName[John]{\de{Strietelmeier}}{Strietelmeier, John}
\PretagName{Thomas, à~Kempis}{Thomas Akempis}
\PretagName[Thomas]{à~Kempis}{Akempis, Thomas}
\PretagName{\textUC{Tokugawa}, Ieyasu}{Tokugawa Ieyasu}
\PretagName{Vlad, Ţepeş}{Vlad Tepes}
\PretagName[E.\,B.]{White}{White, E.B.}

%^^A Add index tags for all names.
\TagName[John]{\noexpand\textSC{Adams}}{, pres.|hyperpage}
\TagName{Æthelred, II}{, king|hyperpage}
\TagName[Susan B.]{Anthony}{|hyperpage}
\TagName{Arai, Akino}{|hyperpage}
\TagName{Aristotle}{|hyperpage}
\TagName{Attila, the Hun}{|hyperpage}
\TagName[Charles]{\textBF{Babbage}}{|hyperpage}
\TagName{Bernard, of Clairvaux}{|hyperpage}
\TagName{Bo\"ethius}{|hyperpage}
\TagName[Robert]{\textSC{Burns}}{|hyperpage}
\TagName[Rudolph]{Carnap}{|hyperpage}
\TagName[J.E.]{Carter, Jr.}{, pres.|hyperpage}
\TagName[Charlie]{Chaplin}{|hyperpage}
\TagName[M.T.]{Cicero}{|hyperpage}
\TagName{Chiang}[Kai-shek]{\ddag, pres.|hyperpage}
\TagName[Schuyler]{Colfax}{, v.p.|hyperpage}
\TagName{Confucius}{|hyperpage}
\TagName[e.e.]{cummings}{|hyperpage}
\TagName{Dagobert}[I]{\ddag, king|hyperpage}
\TagName[Giovanni]{d'Andrea}{|hyperpage}
\TagName[Sammy]{\noexpand\textSC{Davis}, \noexpand\textSC{Jr}.}{|hyperpage}
\TagName{Demetrius, I}{ Soter, king|hyperpage}
\TagName[Jacques]{De~Pamele}{|hyperpage}
\TagName[Pierre-Jean]{\noexpand\Fbox{\noexpand\AltCaps{d}e Smet}}{|hyperpage}
\TagName[Hernando]{de Soto}{|hyperpage}
\TagName[Marc van]{Dongen}{|hyperpage}
\TagName[W.E.B.]{Du~Bois}{|hyperpage}
\TagName[Charles]{du Fresne}{|hyperpage}
\TagName[Albert]{Einstein}{|hyperpage}
\TagName{Elizabeth, I}{, queen|hyperpage}
\TagName{\textUC{Fukuyama}, Takeshi}{|hyperpage}
\TagName[Greta]{\textSC{Garbo}}{|hyperpage}
\TagName{Ghazāli}{|hyperpage}
\TagName{Ghazali}{|hyperpage}
\TagName[J.W. von]{Goethe}{|hyperpage}
\TagName[Louis]{Gossett, Jr.}{|hyperpage}
\TagName[Ulysses S.]{Grant}{, pres.|hyperpage}
\TagName[Enrico]{Gregorio}{|hyperpage}
\TagName{Gregory, I}{, pope|hyperpage}
\TagName[Oskar]{Hammerstein, II}{|hyperpage}
\TagName[Adolf]{Harnack}{|hyperpage}
\TagName{Harun, \noexpand\textSC{\noexpand\AltCaps{a}l-Rashid}}{|hyperpage}
\TagName[Lafcadio]{Hearn}{|hyperpage}
\TagName{Henry}[VIII]{\ddag, king|hyperpage}
\TagName[Bob]{Hope}{|hyperpage}
\TagName{Ishida}[Yoko]{\ddag|hyperpage}
\TagName[Thomas]{\noexpand\JEFF}{, pres.|hyperpage}
\TagName{Jesus, Christ}{|hyperpage}
\TagName[Yoko]{Kanno}{\dag|hyperpage}
\TagName[Mustafa]{Kemal}{|hyperpage}
\TagName[John Maynard]{Keynes}{|hyperpage}
\TagName[Martin Luther]{King, Jr.}{|hyperpage}
\TagName[The Amazing]{Kreskin}{|hyperpage}
\TagName{Lao-tzu}{|hyperpage}
\TagName{Leo, I}{, pope|hyperpage}
\TagName[Clive Staples]{Lewis}{|hyperpage}
\TagName{Louis, XIV}{, king|hyperpage}
\TagName[Ada]{\textIT{Lovelace}}{|hyperpage}
\TagName[Uwe]{Lueck}{|hyperpage}
\TagName[Dan]{Luecking}{|hyperpage}
\TagName[Jan]{Łukasiewicz}{|hyperpage}
\TagName[Martin]{\noexpand\textSC{Luther}}{|hyperpage}
\TagName{Maimonides}{|hyperpage}
\TagName[Nicolas]{Malebranche}{|hyperpage}
\TagName[Catherine \noexpand\AltCaps{d}e']{\noexpand\textSC{Medici}}{|hyperpage}
\TagName{\noexpand\textSC{Mencius}}{|hyperpage}
\TagName{Miyazaki, Hayao}{|hyperpage}
\TagName[Frenec]{Molnár}{\dag|hyperpage}
\TagName[Hideyo]{Noguchi}{\dag|hyperpage}
\TagName[Heiko]{Oberdiek}{|hyperpage}
\TagName[George S.]{Patton, Jr.}{|hyperpage}
\TagName[Lucius]{Sergius Paulus}{|hyperpage}
\TagName{Rambam}{|hyperpage}
\TagName[John David]{\textSC{Rockefeller},\textSC{III}}{|hyperpage}
\TagName[John David]{Rockefeller, IV}{|hyperpage}
\TagName[Heinz]{\textSC{Rühmann}}{|hyperpage}
\TagName{Saul, of Tarsus}{|hyperpage}
\TagName[Robert]{Schlicht}{|hyperpage}
\TagName[\noexpand\SCIPi]{\noexpand\SCIPii}{|hyperpage}
\TagName[Publius]{\noexpand\CSA}{|hyperpage}
\TagName[\noexpand\WM]{\noexpand\SHK}{|hyperpage}
\TagName[R.]{Snel van Royen}{|hyperpage}
\TagName[W.]{Snel van Royen}{|hyperpage}
\TagName[Philipp]{Stephani}{|hyperpage}
\TagName[John]{\de{Strietelmeier}}{|hyperpage}
\TagName[Chesley B.]{Sullenberger, III}{|hyperpage}
\TagName{Sun, Yat-sen}{, pres.|hyperpage}
\TagName{Thomas, à~Kempis}{|hyperpage}
\TagName{Thomas, Aquinas}{|hyperpage}
\TagName{\textUC{Tokugawa}, Ieyasu}{|hyperpage}
\TagName[Mark]{Twain}{|hyperpage}
\TagName[Mike]{Tyson}{|hyperpage}
\TagName[Martin]{Van Buren}{, pres.|hyperpage}
\TagName[Jesse]{Ventura}{|hyperpage}
\TagName{Vlad, II}{ Dracul|hyperpage}
\TagName{Vlad, III}{ Dracula|hyperpage}
\TagName{Voltaire}{|hyperpage}
\TagName[George]{Washington}{, pres.|hyperpage}
\TagName[E.\,B.]{White}{|hyperpage}
\TagName{William, I}{|hyperpage}
\TagName{Yamamoto, Isoroku}{|hyperpage}
\TagName{Yohko}{|hyperpage}
\TagName{Yoshida}[Shigeru]{\ddag, PM|hyperpage}
\TagName[Caspar]{Ziegler}{|hyperpage}

\DisableCrossrefs
\CodelineIndex
\RecordChanges
\frenchspacing

\begin{document}
  \DocInput{\jobname.dtx}
\end{document}
%</driver>
% \fi
%
% \CheckSum{3799}
%
% \CharacterTable
%  {Upper-case    \A\B\C\D\E\F\G\H\I\J\K\L\M\N\O\P\Q\R\S\T\U\V\W\X\Y\Z
%   Lower-case    \a\b\c\d\e\f\g\h\i\j\k\l\m\n\o\p\q\r\s\t\u\v\w\x\y\z
%   Digits        \0\1\2\3\4\5\6\7\8\9
%   Exclamation   \!     Double quote  \"     Hash (number) \#
%   Dollar        \$     Percent       \%     Ampersand     \&
%   Acute accent  \'     Left paren    \(     Right paren   \)
%   Asterisk      \*     Plus          \+     Comma         \,
%   Minus         \-     Point         \.     Solidus       \/
%   Colon         \:     Semicolon     \;     Less than     \<
%   Equals        \=     Greater than  \>     Question mark \?
%   Commercial at \@     Left bracket  \[     Backslash     \\
%   Right bracket \]     Circumflex    \^     Underscore    \_
%   Grave accent  \`     Left brace    \{     Vertical bar  \|
%   Right brace   \}     Tilde         \~}
%
% \changes{0.7}{2011/12/26}{Initial release}
% \changes{1.0}{2012/02/20}{Works with \textsf{microtype}, \textsf{memoir}}
% \changes{2.0}{2015/11/11}{Use dtxgen; prevent bad args}
% \changes{2.5}{2016/04/06}{No default format}
% \changes{3.3}{2020/02/20}{Update manual}
% \changes{3.4}{2020/02/26}{Update manual, \texttt{examples.tex}}
%
% \GetFileInfo{\jobname.dtx}
%
% \DoNotIndex{\@empty, \@gobble, \@period, \@token, \trim@spaces,
% \zap@space}
% 
% \DoNotIndex{\begingroup, \bfseries, \bgroup, \csdef, \csgdef, \csname,
% \csundef, \DeclareOption, \def, \detokenize, \edef, \egroup, \else,
% \endcsname, \endgroup, \endinput, \ExecuteOptions, \expandafter, \fi,
% \futurelet, \global, \hbox, \if, \ifcsname, \ifx, \ignorespaces, \index,
% \itshape, \leavevmode, \let, \newcommand, \newcommandx, \newenvironment,
% \newif, \nobreakspace, \PackageError, \PackageWarning, \ProcessOptions,
% \relax, \renewcommand, \RequirePackage, \scshape, \space, \textbackslash,
% \uppercase}
% 
% \newif\ifNoTag
% \newif\ifDoTikZ
% \ifxetex
%   \DoTikZtrue
% \else
%   \ifluatex
%     \ifpdf
%       \DoTikZtrue
%     \fi
%   \else
%     \ifpdf
%       \DoTikZtrue
%     \fi
%   \fi
% \fi
%
% \ClearNum
% \ifDoTikZ
%   \tcbset{fonttitle=\bfseries\sffamily,box align=top,lower separated=false}
% \fi
% 
% \renewcommand*\FrontNamesFormat[1]{\color{nared}\sffamily #1}
% \renewcommand*\FrontNameHook[1]{\color{nagreen}\sffamily #1}
% \renewcommand*\NamesFormat[1]{\color{nablue}\sffamily #1}
% \renewcommand*\MainNameHook[1]{\color{nabrown}\sffamily #1}
% \newcommand*\ReturnLink{\ifDoTikZ
%   \begin{tcolorbox}\centering Back to Section~\ref{sec:Dashboard}\end{tcolorbox}\else
%   \medskip{\hfil\large Back to Section~\ref{sec:Dashboard}}\fi}
%
% \newif\ifFontDebug
%^^A \FontDebugtrue
%
% \makeatletter
%   \@setpar{\ifFontDebug\edef\@FS{ \f@size pt}{\tiny\@FS}\fi\@@par}
% \makeatother
%
% \begingroup\FontDebugfalse
% \title{\textsf{nameauth} --- Name authority mechanism\\ for consistency in text and
% index\thanks{This file describes version \fileversion, last revised \filedate.}}
% \author{Charles P. Schaum\thanks{E-mail: charles dot schaum@comcast.net}}
% \date{Released \filedate}
%
% \maketitle
%
% \begin{abstract}
%   \noindent The \textsf{nameauth} package automates the correct formatting and indexing of names for professional writing. This aids the use of a \emph{name authority} and the editing process without needing to retype name references.
% \end{abstract}
%
% \bgroup\small\tableofcontents\egroup
% \endgroup
%
% \section{Quick Start}
%
% \subsection{How to Use the Manual}
%
% A \emph{name authority} is a canonical, scholarly list of names to which all variants must refer. The task dashboard (Section~\ref{sec:Dashboard}) guides one to various areas of interest. Start with the basics and add features as needed. To load the defaults, simply type:
% \begin{quote}
%   \fbox{\mystrut\ \texttt{\textbackslash usepackage\{nameauth\}} }
% \end{quote}
%
% \begin{center}\large\bfseries Package Design and Features\end{center}
% \noindent With \textsf{nameauth} names become abstractions: verbs that alter state and nouns that have state. That improves accuracy and consistency:
% \begin{itemize}
% \item \emph{Automate} name forms used in professional writing. First uses of names will have full forms. Later uses have shorter forms. Names vary in the text, but stay constant in the index.
% \item Permit \emph{complex name formatting}.
% \item Many \emph{cross-cultural, multilingual naming conventions} are possible. More details appear in Sections~\ref{sec:ErrorProt}, \ref{sec:AltFormat}, \ref{sec:IndexSort}, and~\ref{sec:Hooksiii}.
% \item \emph{Automatic sort keys and tags} aid indexing.
% \item One can \emph{associate information with names.}
% \item The standard used for implementing the \textsf{nameauth} indexing macros is Nancy C. Mulvany, \textit{Indexing Books} (Chicago: University of Chicago Press, 1994). All references [\hypertarget{Mulvany}{Mulvany}] refer to this edition.
% \item In Section~\ref{sec:ErrorProt} we see how to avoid common errors.
% \item Section~\ref{sec:TechNotes} contains \emph{thanks} and various technical notes.
% \end{itemize}
%
% \begin{center}\large\bfseries Special Signs\end{center}
% \noindent As teaching aids, this manual uses markings that are not part of \textsf{nameauth}, but in some cases are implemented using it:
% \begin{itemize}
%   \item[\ \ ]We show \fbox{\mystrut\NamesFormat{first uses}} and \fbox{\mystrut\MainNameHook{subsequent uses}} of names (Sections~\ref{sec:Formatting}, \ref{sec:NameControl}).
%   \item[\dag\ ]A dagger indicates ``non-native'' Eastern forms (Section~\ref{sec:Eastern}).
%   \item[\ddag\ ]A double dagger shows usage of the obsolete syntax (Section~\ref{sec:Obsolete}).
%   \item[\S\ ]A section mark denotes index entries of fictional names.
%   \item[\(\leftarrow\)]Major\Version{3.0} changes have package version numbers in the margin.
%   \item[\(\leftarrow\)]The\Warn{} ``dangerous bend'' shows where caution is needed.
% \end{itemize}
%
% \ifDoTikZ\begin{tcolorbox}[colback=white,colframe=nared,adjusted title={\hfil Disclaimer}]\else
% \begin{center}\large\bfseries Disclaimer\end{center}\fi
% \noindent Names are about real people. This manual mentions notable figures both living and deceased. All names herein are meant to be used respectfully, for teaching purposes only. At no time is any disrespect or bias intended.
% \ifDoTikZ\end{tcolorbox}\fi
% \newpage
%
% \subsection{Task Dashboard}
% \label{sec:Dashboard}
%
% Here we link to sections by task in order to get things done quickly. Many sections have return links at their end that bring the reader back to this page.
% \ifDoTikZ
% \begin{tcolorbox}[colframe=naslate,adjusted title={\hfil Where do you want to go today?}]\centering
%   \tcbox[equal height group=Z,on line,tikznode,colframe=nabrown,colback=white,adjusted title={\bfseries\sffamily\hfil Quick Start}]{Sections~\ref{sec:QuickStart}, \ref{sec:TradStart},\\ \ref{sec:SimpleStart}, \ref{sec:SelectOver}, \ref{sec:Obsolete},\\ \ref{sec:NamePatterns}, \ref{sec:ErrorProt}}\hfill
%   \tcbox[equal height group=Z,on line,tikznode,colframe=nabrown,colback=white,adjusted title={\bfseries\sffamily\hfil Basics}]{Package options:\\ Section~\ref{sec:PkgOptions}}\hfill
%   \tcbox[equal height group=Z,on line,tikznode,colframe=nabrown,colback=white,adjusted title={\bfseries\sffamily\hfil Basics}]{Name macros:\\ Sections~\ref{sec:Naming},\\ \ref{sec:FName}}\bigskip
%
%   \tcbox[equal height group=Zi,on line,tikznode,colframe=nagreen,colback=white,adjusted title={\bfseries\sffamily\hfil Intermediate}]{Variant forms:\\Sections~\ref{sec:VarNames},\\\ref{sec:NameParticles}, \ref{sec:IndexXref}}\hfill
%   \tcbox[equal height group=Zi,on line,tikznode,colframe=nagreen,colback=white,adjusted title={\bfseries\sffamily\hfil Intermediate}]{Avoid errors: Sec-\\tions~\ref{sec:Obsolete}, \ref{sec:NamePatterns}, \ref{sec:ErrorProt},\\ \ref{sec:NameParticles}, \ref{sec:Unicode}}\hfill
%   \tcbox[equal height group=Zi,on line,tikznode,colframe=naviolet,colback=white,adjusted title={\bfseries\sffamily\hfil Language}]{Western names:\\ Sections~\ref{sec:Affix},\\ \ref{sec:LastFirst}}\bigskip
%
%   \tcbox[equal height group=Zii,on line,tikznode,colframe=naviolet,colback=white,adjusted title={\bfseries\sffamily\hfil Language}]{Eastern names:\\ Sections~\ref{sec:Affix},\\ \ref{sec:Eastern}}\hfill
%   \tcbox[equal height group=Zii,on line,tikznode,colframe=naviolet,colback=white,adjusted title={\bfseries\sffamily\hfil Language}]{Particles, medieval,\\ ancient names: Sec-\\ tions\,\ref{sec:NameParticles},\,\ref{sec:AltFormat},\,\ref{sec:Hooksiii}}\hfill
%   \tcbox[equal height group=Zii,on line,tikznode,colframe=nataupe,colback=white,adjusted title={\bfseries\sffamily\hfil Index}]{Index entries\\ and control:\\ Section~\ref{sec:IndexControl}}\bigskip
%
%   \tcbox[equal height group=Ziii,on line,tikznode,colframe=nataupe,colback=white,adjusted title={\bfseries\sffamily\hfil Index}]{\hbox to 0.385\textwidth{\ Index cross-refs, automatic\ }\\ sorting, and auto-info: Sec-\\ tions~\ref{sec:IndexXref}, \ref{sec:IndexSort}, \ref{sec:IndexTag}, \ref{sec:AKA}}\hfill
%   \tcbox[equal height group=Ziii,on line,tikznode,colframe=nared,colback=white,adjusted title={\bfseries\sffamily\hfil Advanced}]{\hbox to 0.39\textwidth{Generally manage how names}\\ are typeset: Sections~\ref{sec:Formatting}, \ref{sec:AltFormat},\\ \ref{sec:NameControl}, \ref{sec:NameTests}, \ref{sec:Hooksi}\,--\,\ref{sec:Customize}}\bigskip
%
%   \tcbox[equal height group=Ziv,on line,tikznode,colframe=nared,colback=white,adjusted title={\bfseries\sffamily\hfil Advanced}]{\hbox to 0.385\textwidth{Generally manage names by}\\ using a name authority: Sec-\\ tions~\ref{sec:VarNames}, \ref{sec:Formatting}, \ref{sec:IndexControl}, \ref{sec:NameControl}}\hfill
%   \tcbox[equal height group=Ziv,on line,tikznode,colframe=nared,colback=white,adjusted title={\bfseries\sffamily\hfil Advanced}]{Make complex elements\\\hbox to 0.39\textwidth{determined automatically by}\\ names: Sections \ref{sec:TextTags}, \ref{sec:NameTests}}\bigskip
%
%   \tcbox[equal height group=Zv,on line,tikznode,colframe=nablue,colback=white,adjusted title={\bfseries\sffamily\hfil Application}]{Use \textsf{nameauth} with \textsf{beamer}\\\hbox to 0.385\textwidth{overlays to get correct name}\\ forms: Sections~\ref{sec:Formatting}, \ref{sec:NameDecisions},\\ \ref{sec:NameControl}, \ref{sec:NameTests}}\hfill
%   \tcbox[equal height group=Zv,on line,tikznode,colframe=nablue,colback=white,adjusted title={\bfseries\sffamily\hfil Application}]{\hbox to 0.39\textwidth{History / game books, other}\\ complex layouts: Sections\\ \ref{sec:Formatting}, \ref{sec:IndexTag}, \ref{sec:TextTags}, \ref{sec:NameControl},\\ \ref{sec:NameTests}, \ref{sec:Hooksi}\,--\,\ref{sec:Hooksiii}}\smallskip
% \end{tcolorbox}
% \else
% \begin{center}
% \begin{tabularx}{0.9\textwidth}{@{}X|X|X@{}}\toprule
% &&\\
% Concept overview: Sections~\ref{sec:QuickStart}, \ref{sec:TradStart}, \ref{sec:SimpleStart}, \ref{sec:SelectOver}, \ref{sec:Obsolete}, \ref{sec:NamePatterns}, \ref{sec:ErrorProt} & Package options: Section~\ref{sec:PkgOptions} & Basic macros: Section \ref{sec:Naming}, \ref{sec:FName}\\
% &&\\\midrule
% &&\\
% Variant forms: Sections~\ref{sec:VarNames}, \ref{sec:NameParticles}, \ref{sec:IndexXref} & Avoid errors: Sections~\ref{sec:Obsolete}, \ref{sec:NamePatterns}, \ref{sec:ErrorProt}, \ref{sec:NameParticles}, \ref{sec:Unicode} & Western names: Sections~\ref{sec:Affix}, \ref{sec:LastFirst}\\
% &&\\\midrule
% &&\\
% Eastern names: Sections~\ref{sec:Affix}, \ref{sec:Eastern} & Particles, medieval, ancient names: Sections~\ref{sec:NameParticles}, \ref{sec:AltFormat}, \ref{sec:Hooksiii} & Index entries and control: Section~\ref{sec:IndexControl}\\
% &&\\\midrule
% &&\\
% Index cross-refs, automatic sorting, and auto-info: Sections~\ref{sec:IndexXref}, \ref{sec:IndexSort}, \ref{sec:IndexTag}, \ref{sec:AKA} & Generally manage how names are typeset: Sections~\ref{sec:Formatting}, \ref{sec:AltFormat}, \ref{sec:NameControl}, \ref{sec:NameTests}, \ref{sec:Hooksi}\,--\,\ref{sec:Customize} & Generally manage names by using a name authority: Sections~\ref{sec:VarNames}, \ref{sec:Formatting}, \ref{sec:IndexControl}, \ref{sec:NameControl}\\
% &&\\\midrule
% &&\\
% Make complex elements determined automatically by names: Sections \ref{sec:TextTags}, \ref{sec:NameTests} & Use \textsf{nameauth} with \textsf{beamer} overlays to get correct name forms: Sections~\ref{sec:Formatting}, \ref{sec:NameDecisions}, \ref{sec:NameControl}, \ref{sec:NameTests} & History\,/\,game books, other complex layouts: Sections~\ref{sec:Formatting}, \ref{sec:IndexTag}, \ref{sec:TextTags}, \ref{sec:NameControl}, \ref{sec:NameTests}, \ref{sec:Hooksi}\,--\,\ref{sec:Hooksiii}\\
% &&\\\bottomrule
% \end{tabularx}
% \end{center}
% \fi
% \newpage
%
% \subsection{Basic Concepts}
% \label{sec:QuickStart}
%
% We encode names in macro arguments to address multiple naming systems. Required name elements are shown in \emph{black}; optional parts are in \emph{\color{nared}red}.\footnote{Compare [\hyperlink{Mulvany}{Mulvany}, 152--82] and the \textit{Chicago Manual of Style}.}
% The arguments appear in the order \meta{FNN} \meta{SNN} \meta{Affix} \meta{Alternate}:
%
% \ifDoTikZ
%   \noindent\begin{tcolorbox}[colframe=naslate,adjusted title={Western Name}]\centering
%     \tcbox[equal height group=A,on line,tikznode,colback=white,adjusted title={\bfseries\sffamily Forename(s):\\ \meta{FNN}}]{%
%       Personal name(s):\\
%       \textit{baptismal name}\\
%       \textit{Christian name}\\
%       \textit{multiple names}\\
%       \textit{praenomen}\footnotemark\\
%       \hphantom{Family\,/\,clan name}}
%     \tcbox[equal height group=A,on line,tikznode,colback=white,adjusted title={\bfseries\sffamily Surname(s):\\ \meta{SNN}}]{%
%       Family name:\\
%       \textit{of father, mother}\\
%       \textit{ancestor, vocation}\\
%       \textit{origin, region}\\
%       \textit{nomen, cognomen}\\
%       \textit{patronym}\\
%       \vspace{-2ex}\hphantom{\textit{ancestor, vocation}}}
%     \tcbox[equal height group=A,on line,tikznode,colback=white,colframe=nared,adjusted title={\bfseries\sffamily Descriptor:\\ \meta{Affix}}]{%
%       Sobriquet\,/\,title:\\
%       \textit{Sr., Jr., III\dots}\\
%       \textit{notable attribute}\\
%       \textit{origin, region}}
%     \tcbox[tikznode,colback=white,colframe=nared,adjusted title={\bfseries\sffamily Alternate Name(s): \meta{Alternate}}]{\hbox to 0.913\textwidth{\hfil In the body text, not the index, \meta{Alternate} swaps with \meta{FNN}\hfil}\\ \hbox to 0.913\textwidth{\hfil for Western names and \meta{Affix} for all other name categories.\hfil}}
%   \end{tcolorbox}\medskip
%   \begin{tcolorbox}[colframe=naslate,adjusted title={Eastern Name}]\centering
%     \tcbox[equal height group=B,on line,tikznode,colback=white,adjusted title={\bfseries\sffamily Family name:\\ \meta{SNN}}]{Family\,/\,clan name}
%     \tcbox[equal height group=B,on line,tikznode,colback=white,adjusted title={\bfseries\sffamily Personal name:\\ \meta{Affix}}]{Seldom multiple\\
%       names; multi-\\
%       character okay.\\
%       \vspace{-2ex}\hphantom{\textit{ancestor, vocation}}}
%     \tcbox[equal height group=B,on line,tikznode,colback=white,colframe=nared,adjusted title={\bfseries\sffamily Descriptor:\\ \meta{Alternate}}]{%
%       Title, etc.\\
%       (old syntax for\\
%       personal names)\\
%       \vspace{-2ex}\hphantom{Sobriquet\,/\,title:}}
%   \end{tcolorbox}\medskip
%   \begin{tcolorbox}[colframe=naslate,adjusted title={Ancient name}]\centering
%     \tcbox[equal height group=C,on line,tikznode,colback=white,adjusted title={\bfseries\sffamily Personal name:\\ \meta{SNN}}]{Given name(s)\\
%       \hphantom{Family\,/\,clan name}}
%     \tcbox[equal height group=C,on line,tikznode,colback=white,colframe=nared,adjusted title={\bfseries\sffamily Descriptor:\\ \meta{Affix}}]{%
%       Sobriquet\,/\,title:\\
%       \textit{Sr., Jr., III\dots}\\
%       \textit{notable attribute}\\
%       \textit{origin, region}\\
%       \textit{patronym}\\
%       \vspace{-2ex}\hphantom{\textit{ancestor, vocation}}}
%     \tcbox[equal height group=C,on line,tikznode,colback=white,colframe=nared,adjusted title={\bfseries\sffamily Descriptor:\\ \meta{Alternate}}]{%
%       Alternate name\\
%       (old syntax for\\
%       titles, etc.)\\
%       \hphantom{Sobriquet\,/\,title:}}
%   \end{tcolorbox}
% \else
%   \begin{enumerate}[noitemsep]
%   \item Western Name:\medskip\\
%     \begin{tabular}{p{0.28\textwidth}p{0.28\textwidth}p{0.28\textwidth}}
%       \strut\textbf{Forename(s)} &
%       \strut\textbf{Surname(s)} &
%       \strut\textbf{\color{nared}Descriptor}\smallskip\\
%       \strut Personal name(s): &
%       \strut Family name: &
%       \strut Sobriquet\,/\,title:\smallskip\\
%       \textit{baptismal name}\newline
%       \textit{Christian name}\newline
%       \textit{multiple names}\newline
%       \textit{praenomen}\footnotemark &
%       \textit{of father, mother}\newline
%       \textit{ancestor, vocation}\newline
%       \textit{origin, region}\newline
%       \textit{nomen, cognomen}\newline
%       \textit{patronym} &
%       \textit{Sr., Jr., III\dots}\newline
%       \textit{notable attribute}\newline
%       \textit{origin, region}\medskip\\
%     \end{tabular}\\
%     
%     \noindent \textbf{Alternate Name(s):} In the body text, not the index, \meta{Alternate} swaps with \meta{FNN} for Western names and \meta{Affix} for all other name categories.\medskip\\
%   \item Eastern Name:\medskip\\
%     \begin{tabular}{p{0.28\textwidth}p{0.28\textwidth}p{0.28\textwidth}}
%       \textbf{Family name} &
%       \textbf{Given name} &
%       \textbf{\color{nared}Descriptor}\smallskip\\
%       Family\,/\,clan name &
%       \textit{Seldom multiple}\newline
%       \textit{names; multi-}\newline
%       \textit{character okay.} &
%       Title, etc.\newline
%       \textit{(old syntax for}\newline
%       \textit{personal names)}\medskip\\
%     \end{tabular}
%     
%   \item Ancient name:\medskip\\
%     \begin{tabular}{p{0.28\textwidth}p{0.28\textwidth}p{0.28\textwidth}}
%       \textbf{Personal name} &
%       \textbf{\color{nared}Descriptor} &
%       \textbf{\color{nared}Alt. Desc.}\smallskip\\
%       Given name(s): &
%       Sobriquet\,/\,title: &
%       Alternate name:\smallskip\\
%       &
%       \textit{Sr., Jr., III\dots}\newline
%       \textit{notable attribute}\newline
%       \textit{origin, region}\newline
%       \textit{patronym} &
%       \textit{(old syntax for}\newline
%       \textit{titles, etc.)}\\
%     \end{tabular}
% \end{enumerate}\fi
%\begingroup
%\newif\ifSkipGens
%\newif\ifNoGens
%\newif\ifSkipAgnomen
%\newif\ifNoAgnomen
%\newcommand*\SCIPi{\unless\ifNoGens Publius Cornelius\else Publius\fi}
%\newcommand*\SCIPii{\unless\ifNoAgnomen Scipio Africanus\else Scipio\fi}
%\newcommand*\ScipioOnly{\SkipAgnomentrue\Scipio}
%\renewcommand*\NamesFormat[1]%^^A
%  {\sffamily\color{nablue}\ifSkipGens\NoGenstrue\fi\ifSkipAgnomen\NoAgnomentrue\fi#1%^^A
%  \global\SkipGensfalse\global\SkipAgnomenfalse}
%\renewcommand*\MainNameHook[1]%^^A
%  {\sffamily\color{nabrown}\ifSkipGens\NoGenstrue\fi\ifSkipAgnomen\NoAgnomentrue\fi#1%^^A
%  \global\SkipGensfalse\global\SkipAgnomenfalse}
% \footnotetext{How one handles Roman names depends on index entry form; some possible suggestions are given above. Explained on page~\pageref{page:Sobriquets} and following, we have a name \ScipioOnly\ that can be \SkipGenstrue\Scipio\ or just \SkipGenstrue\ScipioOnly, using macro expansion.}
% \endgroup
% \newpage
%
% \subsubsection{Traditional Interface}
% \label{sec:TradStart}
%
% Mandatory arguments are shown in \emph{black}, with optional elements in \emph{\color{nared}red}. If the required argument \meta{SNN} expands to the empty string, \textsf{nameauth} will generate a package error. Extra spaces around each argument are stripped (Section~\ref{sec:ErrorProt}). The argument patterns shown here are used in many \textsf{nameauth} macros.
%
% \bgroup\ifDoTikZ
% \begin{tcolorbox}[colframe=naslate,adjusted title={Western Names}]\centering
%   \begin{tabular}{l@{ }c@{ }c@{ }c}
%   & \small Required & \small Required & \small Optional,\\
%   & \small forename(s) & \small surname(s), & \small in text only \\
%   & & \small optional \meta{Affix}\smallskip\\
%   \tcbox[equal height group=D,colback=white,tikznode,left=1mm,right=1mm,valign=center]{\bfseries\strut \hbox{\cmd{\Name}\texttt{\space}}\\ \strut \cmd{\Name*}\\ \strut \cmd{\FName}} &
%   \tcbox[equal height group=D,colback=white,tikznode,left=1mm,right=1mm,valign=center]{\bfseries\oarg{FNN}} &
%   \tcbox[equal height group=D,colback=white,tikznode,left=1mm,right=1mm,valign=center]{\bfseries\marg{SNN\color{nared}, Affix}} &
%   \tcbox[equal height group=D,colback=white,tikznode,left=1mm,right=1mm,valign=center]{\bfseries\oarg{\color{nared}Alternate}}
%   \end{tabular}
%
%   Add braces \texttt{\bfseries\{\,\}} after {\bfseries\marg{SNN\color{nared}, Affix}} if other text in brackets \texttt{\bfseries[\,]} follows.
% \end{tcolorbox}
% \else
% \bigskip\noindent{\bfseries Western Names}
%
% \bigskip\begin{tabular}{lccc}
%   & \small Required & \small Required surname(s), & \small Optional,\\
%   & \small forename(s) & \small optional \meta{Affix} & \small in text only\smallskip\\
%   \cmd{\Name}\\
%   \cmd{\Name*} & \bfseries\oarg{FNN} & \bfseries\marg{SNN\color{nared}, Affix} & \bfseries\oarg{\color{nared}Alternate}\\
%   \cmd{\FName} 
% \end{tabular}\\
%
%   Add braces \texttt{\bfseries\{\,\}} after {\bfseries\marg{SNN\color{nared}, Affix}} if other text in brackets \texttt{\bfseries[\,]} follows.
% \fi\egroup
%
% \begin{center}\bfseries Examples\end{center}
%
% \noindent Western names require the \meta{FNN} argument to be present. One always includes all arguments for consistent index entries. The simplified interface (Section~\ref{sec:SimpleStart}) cuts down the amount of typing in many cases.\medskip
%
% \bgroup\MyStretch\noindent%
% |\Name [George]{Washington}|\dotfill\Name[George]{Washington}\\
% |\Name*[George]{Washington}|\dotfill\Name*[George]{Washington}\\
% |\Name [George]{Washington}|\dotfill\Name[George]{Washington}\\
% |\FName[George]{Washington}|\dotfill\FName[George]{Washington}\smallskip\\
% |\Name [George S.]{Patton, Jr.}|\dotfill\Name[George S.]{Patton, Jr.}\\
% |\Name*[George S.]{Patton, Jr.}|\dotfill\Name*[George S.]{Patton, Jr.}\\
% |\Name [George S.]{Patton, Jr.}|\dotfill\Name[George S.]{Patton, Jr.}\\
% |\FName[George S.]{Patton, Jr.}|\dotfill\FName[George S.]{Patton, Jr.}\egroup\medskip
%
% The \meta{Alternate} argument will swap with \meta{FNN} in the text, not in the index or the name pattern (Section~\ref{sec:NamePatterns}). To see alternate names, one must use a macro that shows forenames (first use, \cmd{\Name*}, and \cmd{\FName}). Western names require a comma to delimit affixes (Section~\ref{sec:Affix}). Below we see alternate names:\medskip
%
% \bgroup\MyStretch\noindent%
% |\DropAffix\Name*[George S.]{Patton, Jr.}[George]|\dotfill\DropAffix\Name*[George S.]{Patton, Jr.}[George]\smallskip\\
% |\Name [John David]{Rockefeller, IV}|\dotfill\Name[John David]{Rockefeller, IV}\\
% |\Name*[John David]{Rockefeller, IV}[Jay]|\dotfill\Name*[John David]{Rockefeller, IV}[Jay]\\
% |\DropAffix\Name*[John David]{Rockefeller, IV}[Jay]|\dotfill\DropAffix\Name*[John David]{Rockefeller, IV}[Jay]\\
% |\Name [John David]{Rockefeller, IV}[Jay]|\dotfill\Name[John David]{Rockefeller, IV}[Jay]\smallskip\\
% |\Name [Clive Staples]{Lewis}|\dotfill\Name[Clive Staples]{Lewis}\\
% |\Name*[Clive Staples]{Lewis}[C.S.]|\dotfill\Name*[Clive Staples]{Lewis}[C.S.]\\
% |\FName[Clive Staples]{Lewis}[Jack]|\dotfill\FName[Clive Staples]{Lewis}[Jack]\egroup\medskip
%
% In addition to alternate forenames, one also can display alternate surnames, but that uses several different approaches (Sections~\ref{sec:VarNames}, \ref{sec:NameParticles}, \ref{sec:AltAdvanced}, \ref{sec:Hooksiii}).
% \newpage
%
% \bgroup\ifDoTikZ
% \begin{tcolorbox}[colframe=naslate,adjusted title={``Non-native'' Eastern Names, Western Index Entry}]\centering
%   \begin{tabular}{l@{ }c@{ }c@{ }c}
%   & \small Required & \small Required & \small Optional,\\
%   & \small forename(s) & \small surname(s), & \small in text only \\
%   & & \small no \meta{Affix}\smallskip\\
%   \tcbox[equal height group=D,colback=white,tikznode,left=1mm,right=1mm,valign=center]{\bfseries\strut \hbox{\cmd{\Name}\texttt{\space}}\\ \strut \cmd{\Name*}\\ \strut \cmd{\FName}} &
%   \tcbox[equal height group=D,colback=white,tikznode,left=1mm,right=1mm,valign=center]{\bfseries\oarg{FNN}} &
%   \tcbox[equal height group=D,colback=white,tikznode,left=1mm,right=1mm,valign=center]{\bfseries\marg{SNN}} &
%   \tcbox[equal height group=D,colback=white,tikznode,left=1mm,right=1mm,valign=center]{\bfseries\oarg{\color{nared}Alternate}}
%   \end{tabular}
%
%   Add braces \texttt{\bfseries\{\,\}} after {\bfseries\marg{SNN}} if other text in brackets \texttt{\bfseries[\,]} follows.
% \end{tcolorbox}
% \else
% \bigskip\noindent{\bfseries ``Non-native'' Eastern Names, Western Index Entry}
%
% \bigskip\begin{tabular}{lccc}
%   & \small Required & \small Required surname(s), & \small Optional,\\
%   & \small forename(s) & \small no \meta{Affix} & \small in text only\smallskip\\
%   \cmd{\Name}\\
%   \cmd{\Name*} & \bfseries\oarg{FNN} & \bfseries\marg{SNN} & \bfseries\oarg{\color{nared}Alternate}\\
%   \cmd{\FName} 
% \end{tabular}\\
%
%   Add braces \texttt{\bfseries\{\,\}} after {\bfseries\marg{SNN}} if other text in brackets \texttt{\bfseries[\,]} follows.
% \fi\egroup
%
%
% \begin{center}\bfseries Examples\end{center}
%
% \noindent Below we start with ``regular'' Western name forms:\medskip
%
% \bgroup\noindent\MyStretch|\Name[Hideyo]{Noguchi}|\dotfill\Name[Hideyo]{Noguchi}\\
% |\Name*[Hideyo]{Noguchi}[Doctor]|\dotfill\Name*[Hideyo]{Noguchi}[Doctor]\\
% |\Name[Frenec]{Molnár}|\dotfill\Name[Frenec]{Molnár}\egroup\medskip
%
% To turn them into ``non-native'' Eastern names or proper Hungarian names [\hyperlink{Mulvany}{Mulvany}, 166] we use the reversing macros and leave the \meta{Alternate} argument empty (Section~\ref{sec:Eastern}). Index entries are in Western style: \meta{SNN}, \meta{FNN}:\medskip
%
% \bgroup\MyStretch\noindent|\CapName\RevName\Name*[Hideyo]{Noguchi}|\dotfill\CapName\RevName\Name*[Hideyo]{Noguchi}\dag\\
% |\CapName\RevName\Name*[Hideyo]{Noguchi}[Sensei]|\dotfill\CapName\RevName\Name*[Hideyo]{Noguchi}[Sensei]\dag\\
% |\RevName\Name*[Frenec]{Molnár}|\dotfill\RevName\Name*[Frenec]{Molnár}\dag\egroup\medskip
%
% Reversed Western forms do not work with the older syntax (Section~\ref{sec:Obsolete}) and \emph{they do not share name control sequences and index entries} with ``native'' Eastern names and ancient name forms (Section~\ref{sec:NamePatterns}).
%
% \bgroup\ifDoTikZ
% \begin{tcolorbox}[colframe=naslate,adjusted title={\bfseries ``Native'' Eastern Names in the Text, Eastern Index Entry}]\centering
%   \begin{tabular}{l@{ }c@{ }c}
%   & \small Required surname & \small Optional,\\
%   & \small and forename & \small in text only\smallskip\\
%   \tcbox[equal height group=D,colback=white,tikznode,left=1mm,right=1mm,valign=center]{\bfseries\strut \hbox{\cmd{\Name}\texttt{\space}}\\ \strut \cmd{\Name*}\\ \strut \cmd{\FName}} &
%   \tcbox[equal height group=D,colback=white,tikznode,left=1mm,right=1mm,valign=center]{\bfseries\marg{SNN, Affix}} &
%   \tcbox[equal height group=D,colback=white,tikznode,left=1mm,right=1mm,valign=center]{\bfseries\oarg{\color{nared}Alternate}}
%   \end{tabular}
%
%   Add braces \texttt{\bfseries\{\,\}} after {\bfseries\marg{SNN, Affix}} if other text in brackets \texttt{\bfseries[\,]} follows.
% \end{tcolorbox}
% \else
% \bigskip\noindent{\bfseries ``Native'' Eastern Names in the Text, Eastern Index Entry}
%
% \bigskip\begin{tabular}{lcc}
%   & \small Required Surname & \small Optional,\\
%   & \small and forename & \small in text only\smallskip\\
%   \cmd{\Name}\\
%   \cmd{\Name*} & \bfseries\marg{SNN, Affix} & \bfseries\oarg{\color{nared}Alternate}\\
%   \cmd{\FName} 
% \end{tabular}\\
%
%   Add braces \texttt{\bfseries\{\,\}} after {\bfseries\marg{SNN, Affix}} if text in brackets \texttt{\bfseries[\,]} follows.
% \fi\egroup
%
% \begin{center}\bfseries Examples\end{center}
%
% \noindent The comma-delimited required argument, \meta{SNN, Affix}, is the key to non-Western names, which always take the form \meta{SNN Affix} in the index. See Section~\ref{sec:Eastern}. ``Native'' Eastern names have Eastern name order from the start and \emph{they do not share name control sequences and index entries} with Western names (Section~\ref{sec:NamePatterns}). They can be reversed to have Western name order in the body text.
% \newpage
%
% Except\Version{3.0} for mononyms, non-Western forms also can have alternate names. This is incompatible with the older syntax (see Section~\ref{sec:Obsolete}). Unless the index must have Western-style entries, ``native'' forms are best for Eastern names:\medskip
%
% \bgroup\MyStretch\noindent|\Name{Yamamoto, Isoroku}|\dotfill\Name{Yamamoto, Isoroku}\\
% |\Name{Yamamoto, Isoroku}|\dotfill\Name{Yamamoto, Isoroku}\\
% |\RevName\Name*{Yamamoto, Isoroku}[Admiral]|\dotfill\RevName\Name*{Yamamoto, Isoroku}[Admiral]\smallskip\\
% |\Name{Miyazaki, Hayao}|\dotfill\Name{Miyazaki, Hayao}\\
% |\Name*{Miyazaki, Hayao}[Sensei]|\dotfill\Name*{Miyazaki, Hayao}[Sensei]\\
% |\RevName\Name*{Miyazaki, Hayao}[Mr.]|\dotfill\RevName\Name*{Miyazaki, Hayao}[Mr.]\egroup\medskip
%
% \bgroup\ifDoTikZ
% \begin{tcolorbox}[colframe=naslate,adjusted title={\bfseries Ancient and Medieval Names}]\centering
%   \begin{tabular}{l@{ }c@{ }c}
%   & \small Required name & \small Optional,\\
%   & \small optional \meta{Affix} & \small in text only\smallskip\\
%   \tcbox[equal height group=D,colback=white,tikznode,left=1mm,right=1mm,valign=center]{\bfseries\strut \hbox{\cmd{\Name}\texttt{\space}}\\ \strut \cmd{\Name*}\\ \strut \cmd{\FName}} &
%   \tcbox[equal height group=D,colback=white,tikznode,left=1mm,right=1mm,valign=center]{\bfseries\marg{SNN\color{nared}, Affix}} &
%   \tcbox[equal height group=D,colback=white,tikznode,left=1mm,right=1mm,valign=center]{\bfseries\oarg{\color{nared}Alternate}}
%   \end{tabular}
%
%   Add braces \texttt{\bfseries\{\,\}} after {\bfseries\marg{SNN\color{nared}, Affix}} if other text in brackets \texttt{\bfseries[\,]} follows.
% \end{tcolorbox}
% \else
% \bigskip\noindent{\bfseries Ancient and Medieval Names}
%
% \bigskip\begin{tabular}{lcc}
%   & \small Required name & \small Optional,\\
%   & \small optional \meta{Affix} & \small in text only\smallskip\\
%   \cmd{\Name}\\
%   \cmd{\Name*} & \bfseries\marg{SNN\color{nared}, Affix} & \bfseries\oarg{\color{nared}Alternate}\\
%   \cmd{\FName} 
% \end{tabular}\\
%
% Add braces \texttt{\bfseries\{\,\}} after {\bfseries\marg{SNN\color{nared}, Affix}} if other text in brackets \texttt{\bfseries[\,]} follows.
% \fi\egroup
%
% \begin{center}\bfseries Examples\end{center}
%
% \noindent These forms are meant for royalty and ancient figures. They can be mononyms or have multiple names, and may have an affix. Note the teaser for Section~\ref{sec:FName}:\medskip
%
% \bgroup\MyStretch\noindent|\Name{Aristotle}|\dotfill\Name{Aristotle}\\
% |\Name{Aristotle}|\dotfill\Name{Aristotle}\smallskip\\
% |\Name{Elizabeth, I}|\dotfill\Name{Elizabeth, I}\\
% |\Name{Elizabeth, I}|\dotfill\Name{Elizabeth, I}\\
% |\ForceFN\FName{Elizabeth, I}[Good Queen Bess]|\dotfill\ForceFN\FName{Elizabeth, I}[Good Queen Bess]\egroup
%
% \subsubsection{Simplified Interface}
% \label{sec:SimpleStart}
%
% \DescribeEnv{nameauth}
% Although not required, using the \texttt{nameauth} environment in the preamble guards against undefined macros. This environment defines a tabular-like macro:
% \begin{quote}
% \fbox{\vbox{
%   \hbox{\mystrut\ \cmd{\begin\{nameauth\}}}\par
%   \hbox{\hspace{2em}\cmd{\<} \meta{arg1} \texttt{\&} \meta{arg2} \texttt{\&} \meta{arg3} \texttt{\&} \meta{arg4} \texttt{>} }\par
%   \hbox{\mystrut\ \cmd{\end\{nameauth\}}}}}
% \end{quote}
% It uses \meta{arg1} as a basis to create three macros that are equivalent to:\medskip
%
% \begin{tabular}{@{\quad}l@{\ \(\rightarrow\)\ }ll}
%   \texttt{\textbackslash}\meta{arg1} & \cmd{\Name}\oarg{arg2}\marg{arg3}\oarg{arg4} \\
%   \texttt{\textbackslash L}\meta{arg1} & \cmd{\Name*}\oarg{arg2}\marg{arg3}\oarg{arg4} & {\color{nared} |%| L for \textit{long}} \\
%   \texttt{\textbackslash S}\meta{arg1} & \cmd{\FName}\oarg{arg2}\marg{arg3}\oarg{arg4} & {\color{nared} |%| S for \textit{short}} \\
% \end{tabular}\medskip\\
% If either \meta{arg1} or \meta{arg3} are empty, or \meta{SNN} is empty, \textsf{nameauth} will generate a package error. Forgetting the backslash, ampersands, or angle brackets will cause errors. For more on \meta{arg4} see page~\pageref{page:ArgIV}.
% \newpage
%
% Comments below are not part of the environment. Extra spaces in each argument are stripped (Section~\ref{sec:ErrorProt}). Put trailing braces \texttt{\{\,\}} or something else after the shorthand macros if text in brackets \texttt{[\,]} follows, so it does not become an optional argument. Below we introduce name forms with particles.
%
% \begin{center}\bfseries Examples\end{center}
% \def\startrowa{\ \ \cmd{\< }}
% \def\startrowb{\color{nared}\% }
% \def\midrowa{\texttt{ \& }}
% \def\midrowb{\texttt{\ \ \ }}
% \def\endrowa{\texttt{ >}}
% \def\endrowb{}
% \def\startrow{\startrowb}
% \def\midrow{\midrowb}
% \def\endrow{\endrowb}
% \bgroup\noindent\ttfamily\small| \begin{nameauth}|\\
% \begin{tabular}{>{\startrow}l@{\midrow}l@{\midrow}l@{\midrow}l@{\endrow}l}
%   {\normalfont\color{nared}\quad\ \ \,\meta{arg1}}\gdef\startrow{\startrowa} & {\normalfont\color{nared}\meta{arg2}} & {\normalfont\color{nared}\meta{arg3}} & {\normalfont\color{nared}\meta{arg4}}\\\gdef\midrow{\midrowa}\gdef\endrow{\endrowa}%^^A
%   Wash    & George        & Washington        &  &{\color{nared} \% \normalfont\textit{Western}}\\
%   Harnack & Adolf         & Harnack           &  &{\color{nared} \% \normalfont\textit{Western}}\\
%   Lewis   & Clive Staples & Lewis             &  &{\color{nared} \% \normalfont\textit{Western}}\\
%   Pat     & George S.     & Patton, Jr.       &  &{\color{nared} \% \normalfont\textit{W. affix}}\\
%   JRIV    & John David    & Rockefeller, IV   &  &{\color{nared} \% \normalfont\textit{W. affix}}\\
%   Ches    & Chesley B.    & Sullenberger, III &  &{\color{nared} \% \normalfont\textit{W. affix}}\\
%   Soto    & Hernando      & de Soto           &  &{\color{nared} \% \normalfont\textit{W. part.}}\\
%   JWG     & J.W. von      & Goethe            &  &{\color{nared} \% \normalfont\textit{W. part.}}\\
%   VBuren  & Martin        & Van Buren         &  &{\color{nared} \% \normalfont\textit{W. part.}}\\
%   Noguchi   & Hideyo      & Noguchi           &  &{\color{nared} \% \normalfont\textit{W. as E.}}\\
%   Miyaz   &               & Miyazaki, Hayao   &  &{\color{nared} \% \normalfont\textit{Eastern}}\\
%   Yamt    &               & Yamamoto, Isoroku &  &{\color{nared} \% \normalfont\textit{Eastern}}\\
%   Aeth    &               & Æthelred, II      &  &{\color{nared} \% \normalfont\textit{Ancient}}\\
%   Attil   &               & Attila, the Hun   &  &{\color{nared} \% \normalfont\textit{Ancient}}\\
%   Dem     &               & Demetrius, I      &  &{\color{nared} \% \normalfont\textit{Ancient}}\\
%   Eliz    &               & Elizabeth, I      &  &{\color{nared} \% \normalfont\textit{Ancient}}\\
%   Aris    &               & Aristotle         &  &{\color{nared} \% \normalfont\textit{Mono}}\\
%   CSL     & Clive Staples & Lewis             & C.S.   &{\color{nared} \% \normalfont\textit{W. alt.}}\\
%   MSens   &               & Miyazaki, Hayao   & Sensei &{\color{nared} \% \normalfont\textit{E. alt.}}\\
% \end{tabular}\\
% | \end{nameauth}|\egroup\medskip
%
% \begin{center}\small\MyStretch
% \begin{tabular}{rlp{0.41\textwidth}}\toprule
% Output & Short Form & Long Form\\\midrule
% \Wash  & \cmd{\Wash} & \cmd{\Name}\texttt{[George]\{Washington\}}\\
% \rowcolor{black!7!white}\LWash & \cmd{\LWash} & \cmd{\Name*}\texttt{[George]\{Washington\}}\\
% \SWash & \cmd{\SWash} & \cmd{\FName}\texttt{[George]\{Washington\}}\\
% \rowcolor{black!7!white}\JustIndex\Wash & \cmd{\JustIndex}\cmd{\Wash} & \cmd{\IndexName}\texttt{[George]\{Washington\}}\\
% \Eliz & \cmd{\SubvertThis}\cmd{\Eliz} & \cmd{\SubvertThis}\cmd{\Name}\texttt{\{Elizabeth,I\}}\\
% \rowcolor{black!7!white}\ForgetThis\Eliz & \cmd{\ForgetThis}\cmd{\Eliz} & \cmd{\ForgetThis}\cmd{\Name}\texttt{\{Elizabeth,I\}}\\\bottomrule
% \end{tabular}
% \end{center}
%
% English and modern Romance languages keep the particle with the surname. German and other languages do not (cf. Sections~\ref{sec:NameParticles} and~\ref{sec:AltFormat}).\footnote{See also [\hyperlink{Mulvany}{Mulvany}, 152--82], and the \textit{Chicago Manual of Style}.} 
%
% \begin{center}\small\MyStretch
%   \begin{tabular}{rll}\toprule
%     \bfseries Macro & \bfseries Body Text & \bfseries Index \\\midrule
%     \cmd{\VBuren} & \VBuren & \ShowIdxPageref*[Martin]{Van Buren}\\
%     \rowcolor{black!7!white}\cmd{\VBuren} & \VBuren & \ShowIdxPageref*[Martin]{Van Buren}\\
%     \cmd{\Soto} & \Soto & \ShowIdxPageref*[Hernando]{de Soto}\\
%     \rowcolor{black!7!white}\cmd{\CapThis}\cmd{\Soto} & \CapThis\Soto & \ShowIdxPageref*[Hernando]{de Soto}\\
%     \cmd{\JWG} & \JWG & \ShowIdxPageref*[J.W. von]{Goethe}\\
%     \rowcolor{black!7!white}\cmd{\JWG} & \JWG & \ShowIdxPageref*[J.W. von]{Goethe}\\\bottomrule
%   \end{tabular}
% \end{center}
% \newpage
%
% \begin{center}\bfseries \meta{Alternate} Tips\end{center}
%
% \phantomsection
% \noindent Above we listed two shorthands that had alternate names in \meta{arg4}: \cmd{\CSL} and \cmd{\MSens}. They have related shorthands whose \meta{arg4} are empty: \cmd{\Lewis} and \cmd{\Miyaz}. Here are how they are related (cf. Section~\ref{sec:FName}):\label{page:ArgIV}
% \begin{itemize}
% \item They share the same name patterns (Section~\ref{sec:NamePatterns}): |\ForgetThis\Lewis| \ForgetThis\Lewis; \cmd{\CSL} \CSL. |\ForgetThis\Miyaz| \ForgetThis\Miyaz; and \cmd{\MSens} \MSens.
% \item More commonly, one produces alternate names with \cmd{\LLewis[C.S.]} \LLewis[C.S.] and \cmd{\LMiyaz[Sensei]} \LMiyaz[Sensei].
% \item Both\Warn{} \cmd{\LCSL} \LCSL\ and \cmd{\LMSens} \LMSens\ already have \meta{Alternate} built in. They cannot take another optional argument. One must remember which shorthands have used \meta{arg4}.
% \item If one should forget that, one will get errors: \cmd{\LCSL[C.S.]} \LCSL[C.S.] and \cmd{\LMsens[Sensei]} \LMSens[Sensei]. 
% \end{itemize}
%
% \begin{center}\bfseries Variant Overview\end{center}
%
% \begin{center}\footnotesize\MyStretch
%   \begin{tabular}{@{}p{18.4em}p{22.4em}@{}}
%     \textsc{Western:}\par
%     |\Wash|\dotfill \ForgetThis\Wash\par
%     |\LWash|\dotfill \LWash\par
%     |\Wash|\dotfill \Wash\par
%     |\SWash|\dotfill \SWash\par
%     |\RevComma\LWash|\dotfill \RevComma\LWash\par
%     \par\hbox{}\par
%     \textsc{Particles:} \hfill (Section~\ref{sec:NameParticles})\par
%     |\Soto|\dotfill \ForgetThis\Soto\par
%     |\Soto|\dotfill \Soto\par
%     |\CapThis\Soto|\dotfill \CapThis\Soto\par
%     \par\hbox{}\par
%     \textsc{Affixes:} \hfill (Section~\ref{sec:Affix})\par
%     |\Pat|\dotfill \ForgetThis\Pat\par
%     |\LPat|\dotfill \LPat\par
%     |\DropAffix\LPat|\dotfill \DropAffix\LPat\par
%     |\Pat|\dotfill \Pat\par
%     |\SPat|\dotfill \SPat\par
%     \par\hbox{}\par
%     \textsc{Nicknames:} \hfill (Section~\ref{sec:FName})\par
%     |\JRIV|\dotfill \ForgetThis\JRIV\par
%     |\DropAffix\LJRIV[Jay]|\dotfill \DropAffix\LJRIV[Jay]\par
%     |\SJRIV[Jay]|\dotfill \SJRIV[Jay]\par
%     |\Lewis|\dotfill \ForgetThis\Lewis\par
%     |\LLewis[Jack]|\dotfill \LLewis[Jack]\par
%     |\SLewis[Jack]|\dotfill \SLewis[Jack]\par
%     |\LCSL|\dotfill \LCSL\par
%     |\SCSL|\dotfill \SCSL
%     &
%     \textsc{Ancient\,/\,Mononym}\par
%     |\Aris|\dotfill \ForgetThis\Aris\par
%     |\Aris|\dotfill \Aris \par
%     \par\hbox{}\par
%     \textsc{Ancient\,/\,Royal:} \hfill (Sections~\ref{sec:Eastern}, \ref{sec:NameParticles})\par
%     |\Aeth|\dotfill \ForgetThis\Aeth\par
%     |\Aeth|\dotfill \Aeth\par
%     |\LAeth[Unrædig]|\dotfill \LAeth[Unrædig]\par
%     |\Attil|\dotfill \ForgetThis\Attil\par
%     |\Attil|\dotfill \Attil\par
%     \par\hbox{}\par
%     \textsc{``Non-native'' Eastern:} \hfill (Section~\ref{sec:Eastern})\par
%     |\Noguchi|\dotfill \ForgetThis\Noguchi\par
%     |\LNoguchi|\dotfill \LNoguchi\par
%     |\LNoguchi[Doctor]|\dotfill \LNoguchi[Doctor]\par
%     |\SNoguchi|\dotfill \SNoguchi\par
%     |\RevName\LNoguchi|\dotfill \RevName\LNoguchi\dag\par
%     |\CapName\RevName\LNoguchi|\dotfill \CapName\RevName\LNoguchi\dag\par
%     |\CapName\Noguchi|\dotfill \CapName\Noguchi\dag\par
%     \par\hbox{}\par
%     \textsc{``Native'' Eastern:} \hfill (Section~\ref{sec:Eastern})\par
%     |\CapName\Yamt|\dotfill \ForgetThis\CapName\Yamt\par
%     |\CapName\LYamt|\dotfill \CapName\LYamt\par
%     |\CapName\Yamt|\dotfill \CapName\Yamt\par
%     |\RevName\LYamt|\dotfill \RevName\LYamt\par
%     |\RevName\LYamt[Admiral]|\dotfill \RevName\LYamt[Admiral]\par
%     |\SYamt|\dotfill \SYamt\par
%     |\ForceFN\SYamt|\dotfill \ForceFN\SYamt
%     \\
%   \end{tabular}
% \end{center}
% Above we used \cmd{\ForgetThis} (Section~\ref{sec:NameControl}) to reset first uses of names. Now we set up examples on page~\pageref{page:ExPage} by invoking \cmd{\ExcludeName\{Attila, the Hun\}}\ExcludeName{Attila, the Hun} and \cmd{\AKA[John David]\{Rockefeller,IV\}[Jay]\{Rockefeller\}} \AKA[John David]{Rockefeller,IV}[Jay]{Rockefeller}. On why that form has a different index entry than |\DropAffix\LJRIV[Jay]| \DropAffix\LJRIV[Jay], see Sections~\ref{sec:IndexXref} and~\ref{sec:AKA}.
% \newpage
%
% \subsubsection[Macro Overview]{Select Macro Overview}
% \label{sec:SelectOver}
%
% \noindent Below we have a partial selection of macros and their arguments in overview:
% \begin{center}\small\MySmallStretch
% \begin{tabular}{lrccl}\toprule
% \meta{prefix macros} & \cmd{\Name} & \meta{optional *} & \meta{name args} & \\
% \meta{prefix macros} & \cmd{\FName} & \meta{optional *} & \meta{name args} & \\
% \rowcolor{black!7!white}\meta{prefix macros} & \cmd{\IndexName} & & \meta{name args} & \\
% \rowcolor{black!7!white}\meta{prefix macros} & \cmd{\IndexRef} & & \meta{xref args} & \meta{target}\\
% \meta{prefix macros} & \cmd{\AKA} & \meta{optional *} & \meta{target args} & \meta{xref args}\\
% \rowcolor{black!7!white} & \cmd{\ExcludeName} & & \meta{name args} & \\
% \rowcolor{black!7!white} & \cmd{\IncludeName} & \meta{optional *} & \meta{name args} & \\
% & \cmd{\PretagName} & & \meta{name args} & \meta{sort key} \\
% & \cmd{\TagName} & & \meta{name args} & \meta{tag} \\
% & \cmd{\UntagName} & & \meta{name args} & \\
% \rowcolor{black!7!white} & \cmd{\NameAddInfo} & & \meta{name args} & \meta{tag} \\
% \rowcolor{black!7!white} & \cmd{\NameQueryInfo} & & \meta{name args} & \\
% \rowcolor{black!7!white} & \cmd{\NameClearInfo} & & \meta{name args} & \\
% & \cmd{\ForgetName} & & \meta{name args} & \\
% & \cmd{\SubvertName} & & \meta{name args} & \\
% \rowcolor{black!7!white} & \cmd{\IfMainName} & & \meta{name args} & \marg{y}\marg{n}\\
% \rowcolor{black!7!white} & \cmd{\IfFrontName} & & \meta{name args} & \marg{y}\marg{n}\\
% \rowcolor{black!7!white} & \cmd{\IfAKA} & & \meta{name args} & \marg{y}\marg{n}\marg{x}\\\bottomrule
% \end{tabular}
% \end{center}
%
% \noindent The \meta{prefix macros} below have a one-time effect per name and they also stack. For example: |\CapThis\RevName\SkipIndex\Name[bar]{foo}|: \CapThis\RevName\SkipIndex\Name[bar]{foo}.
% \begin{center}\small\MySmallStretch
% \begin{tabular}{rp{0.72\textwidth}}\toprule
% & \bfseries Capitalization in the Text\\
% \cmd{\CapThis} & Capitalize first letter of all name components in body text.\\
% \cmd{\AccentCapThis} & Fallback when Unicode detection cannot be done.\\
% \cmd{\CapName} & Cap entire \meta{SNN} in body text. Overrides \cmd{\CapThis}.\\
% \rowcolor{black!7!white} & \bfseries Reversing in the Text\\
% \rowcolor{black!7!white}\cmd{\RevName} & Reverse order of any name in body text. Overrides \cmd{\RevComma}\\
% \rowcolor{black!7!white}\cmd{\RevComma} & Reverse only Western names to \meta{SNN}, \meta{FNN}.\\
% & \bfseries Commas in the Text\\
% \cmd{\ShowComma} & Add comma between \meta{SNN} and \meta{Affix}.\\
% \cmd{\NoComma} & No comma between \meta{SNN} and \meta{Affix}. Overrides \cmd{\ShowComma}.\\
% \rowcolor{black!7!white} & \bfseries Name Breaks in the Text\\
% \rowcolor{black!7!white}\cmd{\DropAffix} & Drop affix only for a long Western name reference.\\
% \rowcolor{black!7!white}\cmd{\KeepAffix} & Insert non-breaking space (NBSP) between \meta{SNN}, \meta{FNN/Affix}.\\
% \rowcolor{black!7!white}\cmd{\KeepName} & Insert NBSP between all name elements. Overrides \cmd{\KeepAffix}.\\
% & \bfseries Forcing Name Forms in the Text\\
% \cmd{\ForgetThis} & Force a first-time name use. Negates \cmd{\SubvertThis}.\\
% \cmd{\SubvertThis} & Force a subsequent use.\\
% \cmd{\ForceName} & Force first-use formatting hooks.\\
% \cmd{\ForceFN} & Force printing of \meta{Affix} in non-Western short forms.\\
% \rowcolor{black!7!white} & \bfseries Indexing\\
% % \rowcolor{black!7!white}\cmd{\SeeAlso} & Make a \textit{see also} reference instead of a page reference. Only for use with \cmd{\IndexRef}, \cmd{\AKA}, \cmd{\PName} and their starred variants.\\
% \rowcolor{black!7!white}\cmd{\SkipIndex} & Do not create index entries.\\
% \rowcolor{black!7!white}\cmd{\JustIndex} & Act like \cmd{\IndexName}; negated by \cmd{\AKA}, \cmd{\PName}.\\\bottomrule
% \end{tabular}
% \end{center}
% 
% \ReturnLink
% \newpage
%
% \subsection{Obsolete Syntax}
% \label{sec:Obsolete}
%
% This ``ghost''\Warn{} of \textsf{nameauth} past limits alternate names and cross-references (Section~\ref{sec:AKA}), excludes comma-delimited names, and complicates indexing and tagging (Sections~\ref{sec:IndexTag} and~\ref{sec:TextTags}). When the \meta{FNN} and \meta{Affix} arguments are empty, \meta{Alternate} acts like \meta{Affix} and affects both name and index patterns (Section~\ref{sec:NamePatterns}). In this manual we designate these names with a double dagger (\ddag):
%
% \begin{quote}\small\MyStretch
% |\Name{Henry}[VIII]                 | {\color{nared}|%| \textit{Ancient}}\\
% |\Name{Chiang}[Kai-shek]            | {\color{nared}|%| \textit{Eastern}}\\
% |\begin{nameauth}|\\
% |  \< Dagb & & Dagobert & I       > | {\color{nared}|%| \textit{Ancient}}\\
% |  \< Yosh & & Yoshida  & Shigeru > | {\color{nared}|%| \textit{Eastern}}\\
% |\end{nameauth}|\medskip
% \end{quote}
%
% \begin{center}\small\MyStretch
% \begin{tabular}{ll}\toprule
% |\Name{Henry}[VIII]|      & \Name{Henry}[VIII]\ddag\\
% |\Name{Henry}[VIII]|      & \Name{Henry}[VIII]\ddag\\
% \rowcolor{black!7!white}|\Name{Chiang}[Kai-shek]| & \Name{Chiang}[Kai-shek]\ddag\\
% \rowcolor{black!7!white}|\Name{Chiang}[Kai-shek]| & \Name{Chiang}[Kai-shek]\ddag\\
% |\Dagb|                   & \Dagb\ddag\\
% |\Dagb|                   & \Dagb\ddag\\
% \rowcolor{black!7!white}|\CapName\Yosh|           & \CapName\Yosh\ddag\\
% \rowcolor{black!7!white}|\CapName\RevName\LYosh|  & \CapName\RevName\LYosh\ddag\\\bottomrule
% \end{tabular}
% \end{center}
%
% |\Name{Henry}[VIII]|\Version{2.6} (older syntax) and |\Name{Henry, VIII}| (new syntax) share name patterns, tags, and index entries, as shown below. We recommend using the newer syntax unless otherwise needed.
% \begin{quote}\small
% \NameAddInfo{Henry}[VIII]{ (\textit{Defensor Fidei})}%
% |\NameAddInfo{Henry}[VIII]{ (\textit{Defensor Fidei})}|{\color{nared}| % old|}\\
% {.\,.\,.}\\
% |\Name*{Henry, VIII}\NameQueryInfo{Henry, VIII}       |{\color{nared}| % new|}\smallskip\\
% \Name*{Henry, VIII}\NameQueryInfo{Henry, VIII}
% \end{quote}
% 
% \ReturnLink
%
% \subsection{Name Pattern Overview}
% \label{sec:NamePatterns}
%
% The table below shows how the macro arguments generate name patterns central to \textsf{nameauth}. The \meta{Alternate} argument only affects patterns when using the obsolete syntax. The naming macro arguments create internal control sequences that affect names in both the text and the index:
%
% \begin{center}\MyStretch
% \begin{tabular}{lll}\toprule
%   Macro Arguments & Patterns & Type\\\midrule
%   \rowcolor{black!7!white}\oarg{FNN}\marg{SNN} & \meta{FNN}!\meta{SNN} & \normalfont Western\\
%   \oarg{FNN}\marg{SNN, Affix} & \meta{FNN}!\meta{SNN},\meta{Affix} & \normalfont Western\\
%   \rowcolor{black!7!white}\hphantom{\oarg{FNN}}\marg{SNN, Affix} & \meta{SNN},\meta{Affix} & \normalfont non-Western\\
%   \hphantom{\oarg{FNN}}\marg{SNN}\oarg{Alt} & \meta{SNN},\meta{Alt} & \normalfont (obsolete)\\
%   \rowcolor{black!7!white}\hphantom{\oarg{FNN}}\marg{SNN} & \meta{SNN} & \normalfont mononym\\\bottomrule
% \end{tabular}
% \end{center}
% \newpage
%
% \noindent We ``forget'' several names below to create first-use cases:
% \begin{center}\footnotesize\MyStretch
%  \begin{tabular}{rll}\toprule
%   Macro & Body Text & \cmd{\ShowPattern}\\\midrule
%   \cmd{\Harnack[Adolf von]} & \Harnack[Adolf von] & \texttt{\ShowPattern[Adolf]{Harnack}}\\
%   \cmd{\LHarnack} & \LHarnack & \texttt{\ShowPattern[Adolf]{Harnack}}\\
%   \rowcolor{black!7!white}\cmd{\ForgetThis}\cmd{\Pat} & \ForgetThis\Pat & \texttt{\ShowPattern[George S.]{Patton, Jr.}}\\
%   \rowcolor{black!7!white}\cmd{\DropAffix}\cmd{\LPat} & \DropAffix\LPat & \texttt{\ShowPattern[George S.]{Patton, Jr.}}\\
%   \cmd{\ForgetThis}\cmd{\Noguchi} & \ForgetThis\Noguchi & \texttt{\ShowPattern[Hideyo]{Noguchi}}\\
%   \cmd{\RevName}\cmd{\LNoguchi} & \RevName\LNoguchi\dag & \texttt{\ShowPattern[Hideyo]{Noguchi}}\\
%   \rowcolor{black!7!white}\cmd{\ForgetThis}\cmd{\Yamt} & \ForgetThis\Yamt & \texttt{\ShowPattern{Yamamoto,Isoroku}}\\
%   \rowcolor{black!7!white}\cmd{\RevName}\cmd{\LYamt} & \RevName\LYamt & \texttt{\ShowPattern{Yamamoto,Isoroku}}\\
%   \cmd{\ForgetThis}\cmd{\Name\{Henry,VIII\}} & \ForgetThis\Name{Henry,VIII} & \texttt{\ShowPattern{Henry,VIII}}\\
%   \cmd{\Name*\{Henry\}[VIII]} & \Name*{Henry}[VIII]\ddag & \texttt{\ShowPattern{Henry,VIII}}\\
%   \rowcolor{black!7!white}\cmd{\Dem[I Soter]} & \Dem[I Soter] & \texttt{\ShowPattern{Demetrius,I}}\\
%   \rowcolor{black!7!white}\cmd{\LDem} & \LDem & \texttt{\ShowPattern{Demetrius,I}}\\
%   \cmd{\ForgetThis}\cmd{\Aris} & \ForgetThis\Aris & \texttt{\ShowPattern{Aristotle}}\\
%   \cmd{\Aris} & \Aris & \texttt{\ShowPattern{Aristotle}}\\\bottomrule
%   \end{tabular}
%   \ForgetName{Demetrius, I}\ForgetName[Adolf]{Harnack}
% \end{center}\bigskip
% 
% \noindent Six suffixes are appended to these patterns to create independent data sets:
% \begin{center}\small\MyStretch
% \begin{tabular}{llll}\toprule
%   Description & Pattern & Mnemonic & Example\\\midrule
%   \rowcolor{black!7!white}Front-matter names & \meta{pattern}\texttt{!NF} & ``name front'' & \texttt{\ShowPattern[Adolf]{Harnack}{!NF}}\\
%   Main-matter names  & \meta{pattern}\texttt{!MN} & ``main name''  & \texttt{\ShowPattern[Hideyo]{Noguchi}{!MN}}\\
%   \rowcolor{black!7!white}Index cross-refs   & \meta{pattern}\texttt{!PN} & ``pseudonym''  & \texttt{\ShowPattern{Yamamoto, Isoroku}{!PN}}\\
%   Index sorting tags & \meta{pattern}\texttt{!PRE} & ``pretag''    & \texttt{\ShowPattern{Henry, VIII}{!PRE}}\\
%   \rowcolor{black!7!white}Index info tags    & \meta{pattern}\texttt{!TAG} & ``tag''       & \texttt{\ShowPattern{Demetrius, I}{!TAG}}\\
%   ``Text tag'' database & \meta{pattern}\texttt{!DB} & ``database''   & \texttt{\ShowPattern{Aristotle}{!DB}}\\\bottomrule
% \end{tabular}
% \end{center}\bigskip
% 
% \noindent The following macros \emph{write} to these data sets; others also can read from them:
% \begin{center}\small\MyStretch
% \begin{tabular}{lcccccc}\toprule[1pt]
%   Macros & \ttfamily \,!NF\, & \ttfamily \,!MN\, & \ttfamily \,!PN\, & \ttfamily !PRE & \ttfamily !TAG & \ttfamily \,!DB\,\\\midrule
%   \cmd{\Name} \cmd{\Name*} \cmd{\FName} & \YES & \YES & \NO & \NO & \NO & \NO\\
%   \cmd{\ForgetName} \cmd{\SubvertName} & \YES & \YES & \NO & \NO & \NO & \NO\\\midrule
%   \cmd{\PName}\cmd{\PName*} & \YES & \YES & \YES & \NO & \NO & \NO\\\midrule
%   \cmd{\AKA} \cmd{\AKA*} \cmd{\IndexRef} & \NO & \NO & \YES & \NO & \NO & \NO\\
%   \cmd{\ExcludeName} & \NO & \NO & \YES & \NO & \NO & \NO\\
%   \cmd{\IncludeName} \cmd{\IncludeName*} & \NO & \NO & \YES & \NO & \NO & \NO\\\midrule
%   \cmd{\PretagName} & \NO & \NO & \NO & \YES & \NO & \NO\\\midrule
%   \cmd{\TagName} \cmd{\UntagName} & \NO & \NO & \NO & \NO & \YES & \NO\\\midrule
%   \cmd{\NameAddInfo} \cmd{\NameClearInfo} & \NO & \NO & \NO & \NO & \NO & \YES\\\bottomrule
% \end{tabular}
% \end{center}\bigskip
% 
% \ReturnLink
% \newpage
% 
% \subsection[Debug and Avoid Errors]{Debugging and Avoiding Errors}
% \label{sec:ErrorProt}
%
% \begin{center}\bfseries Debugging Macros\end{center}
%
% \DescribeMacro{\ShowPattern}
% \cmd{\ShowPattern} displays how the name arguments create name patterns. One can debug pattern collisions and other issues with this macro:\Version{3.3}
% \begin{quote}
% \fbox{\mystrut\ \cmd{\ShowPattern}\oarg{FNN}\marg{SNN|,| Affix}\oarg{Alternate} }
% \end{quote}
% We used \cmd{\ShowPattern} in two of the tables on the previous page in order to illustrate name control patterns. We set the macro using a typewriter font, e.g.: |\texttt{\ShowPattern[Hernando]{de Soto}}|:\quad \texttt{\ShowPattern[Hernando]{de Soto}}\medskip
% 
% \DescribeMacro{\ShowIdxPageref}
% \cmd{\ShowIdxPageref} displays a full index entry in the text.
% Its analogue is \cmd{\ShowIdxPageref*},
% \DescribeMacro{\ShowIdxPageref*}
% which shows a short index entry. Both only show names as page references, even if they are cross-references\Version{3.3}:
% \begin{quote}
% \fbox{\vbox{\hbox{\mystrut\ \cmd{\ShowIdxPageref\ }\oarg{FNN}\marg{SNN|,| Affix}\oarg{Alternate} }\par
% \hbox{\mystrut\ \cmd{\ShowIdxPageref*}\oarg{FNN}\marg{SNN|,| Affix}\oarg{Alternate} }}}
% \end{quote}
% The full entry produced by \cmd{\ShowIdxPageref} can be affected by both index styles and tags produced by \cmd{\PretagName} and \cmd{\TagName}, e.g.:
% \begin{quote}
% |\texttt{\ShowIdxPageref[Hernando]{de Soto}}:|\\
% \ShowIdxPageref[Hernando]{de Soto}
% \end{quote}
% \cmd{\ShowIdxPageref*} appears throughout this manual to illustrate basic index entries, e.g.: |\ShowIdxPageref*[Hernando]{de Soto}|:\quad \ShowIdxPageref*[Hernando]{de Soto}
% 
% \begin{center}\bfseries Avoiding Common Errors\end{center}
%
% \begin{itemize}
% \item Keep it simple! Avoid unneeded macros and use the simplified interface.
% \item Compare index entries with names in the body text.
% \item Check package warnings. Set the \texttt{verbose} option if needed.
% \item Check arguments' braces and brackets to avoid errors like ``\texttt{Paragraph ended}'' and ``\texttt{Missing} \meta{grouping token} \texttt{inserted.}''
% \item Do not format \meta{SNN},\meta{Affix} together as a pair. Format \meta{SNN} and \meta{Affix} separately (Section~\ref{sec:AltFormat}). 
% \item Sort names in the index with \cmd{\PretagName} (Section~\ref{sec:IndexSort}).
% \item In package docs (\texttt{dtx} files) set up the \texttt{nameauth} environment and tags in the driver section to avoid errors.
% \end{itemize}
% 
% \begin{center}\bfseries Obsolete Syntax Caution\end{center}
%
% \begin{itemize}
% \item The older syntax has restrictions (Section~\ref{sec:Obsolete}). Only the new syntax permits variant names, e.g.: \cmd{\Name*\{Henry, VIII\}[Tudor]} \Name*{Henry, VIII}[Tudor]. The new syntax is preferred.
% \item A proper form for the old syntax is \cmd{\Name*\{Henry\}[VIII]}: \Name*{Henry}[VIII].
% \item \bgroup\IndexInactive\cmd{\Name[Henry]\{VIII\}} is a malformed Western name: ``\Name*[Henry]{VIII}'' and ``\Name[Henry]{VIII}.'' Likewise \cmd{\Name[Henry]\{VIII\}[Tudor]}: ``\Name*[Henry]{VIII}[Tudor]'' and ``\Name[Henry]{VIII}[Tudor].'' Both have the incorrect index entry ``\ShowIdxPageref*[Henry]{VIII}''.\egroup
% \end{itemize}
%
% \begin{center}\bfseries Standard Warnings\end{center}
%
% \begin{itemize}
% \item If one defines shorthand macros in the \texttt{nameauth} environment whose control sequence already exists, warnings always appear. For example:
% \begin{quote}\small\StartNum
% \begin{verbatim}
%\PretagName[E.\,B.]{White}{White, E.B.}
%\begin{nameauth}
%  \< White & E.B.   & White & > % v.1
%  \< White & E.\,B. & White & > % v.2
%\end{nameauth}\end{verbatim}
% \end{quote}
%\begin{nameauth}
%  \< White & E.B.   & White & >
%  \< White & E.\,B. & White & >
%\end{nameauth}
% \item \cmd{\White} gives ``\White''. Its pattern is: \texttt{\ShowPattern[E.\,B.]{White}}. We lost the first version. We forget \White\ for later.\ForgetName[E.\,B.]{White} There should be two package warnings in this section for the redefinition of \cmd{\White}, because we defined it in the driver, then again here twice.
% \item This could be a problem if a name shorthand replaces an actual macro that is used for something else and breaks that macro.
% \item Then again, if one uses, e.g., a new \texttt{nameauth} environment per chapter, these warnings may be harmless. User discretion is advised.
% \end{itemize}
%
% \begin{center}\bfseries Verbose Warnings\end{center}
%
% \noindent Package warnings result from the following \emph{only} when the \texttt{verbose} option is used because we do not want the default to be ``chatty'':
% \begin{itemize}
% \item Creating an index page reference after using a name as an xref or excluding it. Not allowed.
% \item Creating the same cross-reference multiple times. Not allowed.
% \item Using \cmd{\ExcludeName} on an xref. Not allowed.
% \item Using\Version{3.3} \cmd{\IncludeName} on an xref. Not allowed.
% \item Using \cmd{\ExcludeName} to exclude a name that exists is allowed, but a warning still results.
% \item \cmd{\PretagName} sorts xrefs, but also creates ``informational warnings.''
% \item Using \cmd{\TagName} and \cmd{\UntagName} on xrefs. No tag allowed.
% \end{itemize}
% 
% \begin{center}\bfseries Error Protection Strategies\end{center}
%
% \noindent The\Info{extra spaces} \textsf{nameauth} package trims extra spaces \emph{around} name arguments to prevent errors like multiple index entries that appear due to extra spaces. \LaTeX\ usually compacts internal spaces. For example, instead of being two different names, below we have the same name in a first, then subsequent use:
%
% \begin{center}\small\MyStretch\ForgetName[Martin Luther]{King, Jr.}
% \begin{tabular}{ll}
% \cmd{\Name*[Martin Luther]\{King,Jr.\}} & \Name*[Martin Luther]{King,Jr.}\\
% \cmd{\Name*}\texttt{[\textvisiblespace\textvisiblespace Martin\textvisiblespace\textvisiblespace Luther\textvisiblespace\textvisiblespace]\{\textvisiblespace\textvisiblespace King\textvisiblespace\textvisiblespace,\textvisiblespace\textvisiblespace Jr.\textvisiblespace\textvisiblespace\}} & \Name*[  Martin  Luther  ]{  King  ,  Jr.  }\\
% \end{tabular}
% \end{center}
% 
% This does not include explicit spaces from \cmd{\space}, etc. For example, the pattern \texttt{\ShowPattern[Martin Luther]{King, Jr.}} comes from \cmd{\Name[Martin Luther]\{King, Jr.\}} while \texttt{\ShowPattern[Martin~Luther]{King, Jr.}} comes from \cmd{\Name[Martin}|~|\texttt{Luther]\{King, Jr.\}}. The tilde appears as a non-breaking space in the text.
% \newpage
% 
% Full\Info{full stop\break detection} stops appear in one's initials and in affixes like ``Jr.'' (junior), ``Sr.'' (senior), ``d.\,J.'' (\textit{der Jüngere}), and ``d.\,Ä.'' (\textit{der Ältere}). The naming macros and some of the alternate name macros (Section~\ref{sec:AKA}) check if the printed name ends with a full stop and is followed by one. They gobble the extra full stop:
% \begin{quote}\small\ForgetName[Martin Luther]{King, Jr.}
% |This is Rev. Dr. \Name[Martin Luther]{King, Jr.}.|\\
% This is Rev. Dr. \Name[Martin Luther]{King, Jr.}. \hfill Full stop is gobbled.\\[2ex]
% |This is Rev. Dr. \Name[Martin Luther]{King, Jr.}.|\\
% This is Rev. Dr. \Name[Martin Luther]{King, Jr.}. \hfill Full stop is not gobbled.\\[2ex]
% |Again we speak fully of \Name*[Martin Luther]{King, Jr.}.|\\
% Again we speak fully of \Name*[Martin Luther]{King, Jr.}. \hfill Full stop is gobbled.\\[2ex]
% |We drop the affix: \DropAffix\Name*[Martin Luther]{King, Jr.}.|\\
% We drop the affix: \DropAffix\Name*[Martin Luther]{King, Jr.}. \hfill Full stop is not gobbled.\\[2ex]
% |His initials are \FName[Martin Luther]{King, Jr.}[M.L.].|\\
% His initials are \FName[Martin Luther]{King, Jr.}[M.L.]. \hfill Full stop is gobbled.
% \end{quote}
%
% Take\Info{grouping\break issues} care when using braces and spaces with a name at the end of a sentence. Braces can change name arguments, even though they look the same. We disable indexing for the three points below:\IndexInactive
% \begin{itemize}\small
% \item If one encapsulates a name in braces, the punctuation detection fails:\\
% |This is Rev. Dr. {\Name*[Martin Luther]{King, Jr.}}.|\\
% This is Rev. Dr. {\Name*[Martin Luther]{King, Jr.}}. \hfill Full stop is not gobbled.
% \item[] A solution encapsulates both the name and the full stop:\\
% |This is Rev. Dr. {\Name*[Martin Luther]{King, Jr.}.}|\\
% This is Rev. Dr. {\Name*[Martin Luther]{King, Jr.}.} \hfill Full stop is gobbled.
% \item If one encapsulates \meta{Affix} in braces, the punctuation detection fails:\\
% |This is Rev. Dr. \Name*[Martin Luther]{King, {Jr.}}.|\\
% This is Rev. Dr. \Name*[Martin Luther]{King, {Jr.}}. \hfill Full stop is not gobbled.
% \item[] The solution leaves the full stop in \meta{Affix} outside the braces:\\
% |This is Rev. Dr. \Name*[Martin Luther]{King, {Jr}.}.|\\
% This is Rev. Dr. \Name*[Martin Luther]{King, {Jr}.}. \hfill Full stop is gobbled.
% \item[] The name patterns, however, are different, creating different names:\\
% \texttt{\ShowPattern[Martin Luther]{King, {Jr.}}}\\
% \texttt{\ShowPattern[Martin Luther]{King, {Jr}.}}
% \item If one leaves an extra space after a name, the punctuation detection fails:\\
% |This is Rev. Dr. \Name*[Martin Luther]{King, Jr.}|\texttt{\textvisiblespace.}\\
% This is Rev. Dr. \Name*[Martin Luther]{King, Jr.} . \hfill Full stop is not gobbled.
% \item[] The solution removes the extra space:\\
% |This is Rev. Dr. \Name*[Martin Luther]{King, Jr.}.|\\
% This is Rev. Dr. \Name*[Martin Luther]{King, Jr.}. \hfill Full stop is gobbled.
% \end{itemize}\IndexActive
%
% Variations\Info{active chars\break and macros} in the use of active characters and control sequences also change name arguments and index sorting (Section~\ref{sec:IndexSort}; cf.~\ref{sec:Unicode} and~\ref{sec:TeXengines}):
% \begin{itemize}\small
% \item |\Name*{Æthelred, II}| \Name*{Æthelred, II};\hfill Pattern: \texttt{\ShowPattern{Æthelred, II}}\footnote{With \texttt{pdflatex} / \texttt{latex}, the glyphs \texttt{ÃĘ} correspond to \cmd{\IeC\{\textbackslash AE\}}.}\newline We have seen this name earlier.
% \item |\SkipIndex\Name{\AE thelred, II}| \SkipIndex\Name{\AE thelred, II};\hfill Pattern: \texttt{\ShowPattern{\AE thelred, II}}\newline This is a new name that looks the same.\newpage
% \item |\Name{Bo\"ethius}| \Name{Bo\"ethius};\hfill Pattern: \texttt{\ShowPattern{Bo\"ethius}}\newline We introduce this new name.
% \item |\SkipIndex\Name{Boëthius}| \SkipIndex\Name{Boëthius} \hfill Pattern: \texttt{\ShowPattern{Boëthius}}\footnote{With \texttt{pdflatex} / \texttt{latex}, the glyphs \texttt{Ãń} correspond to \cmd{\IeC\{\textbackslash"e\}}.}\newline This is a different name that looks the same.
% \item |\SkipIndex\Name{Bo{\"e}thius}| \SkipIndex\Name{Bo{\"e}thius};\hfill Pattern: \texttt{\ShowPattern{Bo{\"e}thius}}\newline This also is a different name that looks the same.
% \end{itemize}
% 
% Omit\Info{formatting\break initials} spaces between initials; see Bringhurst, \textit{Elements of Typographic Style}. If a style guide requires spaces, try thin spaces. Use \cmd{\PretagName} to sort those names (Section~\ref{sec:IndexSort}). Below we use no formatting:\bigskip
%
% \leavevmode\quad\begin{minipage}[c]{0.45\textwidth}\small
% \StartNum
% \begin{verbatim}
%\PretagName[E.\,B.]{White}%
%  {White, E.B.}
%\begin{nameauth}
%  \< White & E.\,B. & White & >
%\end{nameauth}\end{verbatim}
% \end{minipage}
% \begin{minipage}[c]{0.35\textwidth}\normalsize\renewcommand*\NamesFormat{}
% \def\sep{\vrule width0.5pt\kern-0.5pt}%
% \begin{tabular}{@{}ll@{}}\hline
% & \sep\hphantom{E.}\sep\,\sep\hphantom{B.}\sep\hphantom{\ }\sep\hphantom{White}\\
% \cmd{\White} & \hspace{0.48pt}\White\\
% & \sep\hphantom{E.}\sep\,\sep\hphantom{B.}\sep\hphantom{\ }\sep\hphantom{White}\\\hline
% & \sep\hphantom{E.}\sep\ \sep\hphantom{B.}\sep\ \sep\hphantom{White}\\
% Normal text:\hfill & E. B. White\\
% & \sep\hphantom{E.}\sep\ \sep\hphantom{B.}\sep\ \sep\hphantom{White}\\\hline
% \end{tabular}
% \end{minipage}\bigskip
%
% English\Info{hyphenation} contains names from many cultures. The rules for hyphenation go to the heart of how names with non-English origins should be pronounced. With \textsf{nameauth}, one can use either optional hyphens or the \textsf{babel}\,/\,\textsf{polyglossia} packages to handle such names:
% \NameAddInfo[John]{\de{Strietelmeier}}{late professor at Valparaiso University}
% \begin{quote}\small
% \StartNum
% \begin{verbatim}
%\newcommand\de[1]{\foreignlanguage{ngerman}{#1}}
% % or polyglossia: \newcommand\de[1]{\textgerman{#1}}
%\NameAddInfo[John]{\de{Strietelmeier}}%
%  {late professor at Valparaiso University}
%\begin{nameauth}
%  \< Striet & John & \de{Strietelmeier} & >
%\end{nameauth}
%\PretagName[John]{\de{Strietelmeier}}{Strietelmeier, John}\end{verbatim}
% \end{quote}
%
% \noindent \emph{Not fixed:}\\
% \noindent In English, some names come from other cultures. These names, like \SkipIndex\Name[John]{Strietelmeier} \cmd{\SkipIndex}\cmd{\Name[John]\{Strietelmeier\}} can break badly.\smallskip
%
% \noindent \emph{Fixed with discretionary hyphens:}\\
% \noindent In English, some names come from other cultures. These names, like \SkipIndex\Name[John]{Strie\-tel\-meier}, \cmd{\SkipIndex}\cmd{\Name[John]\{Strie\textbackslash-tel\textbackslash-meier\}} could break badly.\smallskip
%
% \noindent \emph{Fixed with language packages:}\\
% \noindent In English, some names come from other cultures. These names, like \Striet, \cmd{\Striet} could break badly.\medskip
%
% \Striet\ (\NameQueryInfo[John]{\de{Strietelmeier}}) is neither pronounced nor hyphenated as ``Stri-etel-meier''; rather, it is pronounced and hyphenated as ``Strie-tel-meier''. See Sections~\ref{sec:NameParticles} and~\ref{sec:AltFormat} when using macros in name arguments.
% Using \textsf{babel} or \textsf{polyglossia} likely is best.
%
% \ReturnLink
% \newpage
%
% \section{Detailed Usage}
% 
% \subsection{Package Options}
% \label{sec:PkgOptions}
%
% One includes the \textsf{nameauth} package thus:
% \begin{quote}
% \fbox{\mystrut\ \cmd{\usepackage[}\meta{\(option_1\)}\texttt{,}\meta{\(option_2\)}\texttt{,}\dots\texttt{,}\meta{\(option_n\)}\texttt{]\{nameauth\} }}
% \end{quote}
% The options have no required order. Still, we discuss them from the general to the specific, as the headings below indicate. In the listings below, {\bfseries implicit default options are boldface and need not be invoked by the user.} {\color{nared}Non-default options are in red and must be invoked explicitly.}
%
% \begin{center}\bfseries Choosing Features\end{center}
%
% \noindent{\bfseries Enable Package Warnings}\medskip\\
% \bgroup\begin{tabular}{p{0.22\textwidth}p{0.6\textwidth}}
% \leavevmode\color{nared}\quad\texttt{verbose} & \leavevmode\color{nared}Show warnings about index cross-references.\medskip\\
% \end{tabular}\egroup
% 
% \noindent The default\Version{3.0} suppresses package warnings from the indexing macros. Warnings from the \texttt{nameauth} environment are not suppressed.\bigskip
%
% \noindent{\bfseries Choose Formatting}\medskip\\
% \bgroup\begin{tabular}{p{0.22\textwidth}p{0.6\textwidth}}
% \bfseries\quad\texttt{mainmatter} & \bfseries Start with ``main-matter names'' and formatting hooks (see also page~\pageref{page:PostProcess}).\smallskip\\
% \leavevmode\color{nared}\quad\texttt{frontmatter} & \leavevmode\color{nared}Start with ``front-matter names'' and hooks.\smallskip\\
% \leavevmode\color{nared}\quad\texttt{alwaysformat} & \leavevmode\color{nared}Use only respective ``first use'' formatting hooks.\smallskip\\
% \leavevmode\color{nared}\quad\texttt{formatAKA} & \leavevmode\color{nared}Format the first use of a name with \cmd{\AKA} like the first use of a name with \cmd{\Name}.\smallskip\\
% \leavevmode\color{nared}\quad\texttt{oldAKA} & \leavevmode\color{nared}Force \cmd{\AKA*} to act like it did before version 3.0.\smallskip\\
% \leavevmode\color{nared}\quad\texttt{oldreset} & \leavevmode\color{nared}Reset per-use name flags locally within the naming macros, as before version 3.3.\smallskip\\
% \leavevmode\color{nared}\quad\texttt{oldpass} & \leavevmode\color{nared}When \cmd{\Justindex} is called, allow long\,/\,short flags to pass through, as before version 3.3.\medskip\\
% \end{tabular}\egroup
% 
% \noindent The \texttt{mainmatter} option and the \texttt{frontmatter} option enable two different systems of name use and formatting. They are mutually exclusive. \cmd{\NamesActive} starts the main matter system when \texttt{frontmatter} is used. See Section~\ref{sec:Formatting}.
%
% The \texttt{alwaysformat} option forces ``first use'' hooks globally in both naming systems. Its use is limited in current versions of \textsf{nameauth}.
%
% The\Version{3.1} \texttt{formatAKA} option permits \cmd{\AKA} to use the ``first use'' formatting hooks. This enables \cmd{\ForceName} to trigger those hooks at will (Section~\ref{sec:AKA}). Otherwise \cmd{\AKA} uses ``subsequent use'' hooks.
%
% Using the\Version{3.0} \texttt{oldAKA} option forces \cmd{\AKA*} always to print a ``forename'' argument in the text, as in versions before 3.0. Otherwise the current behavior of \cmd{\AKA*} prints in the same fashion as \cmd{\FName} (see Sections~\ref{sec:FName} and~\ref{sec:AKA}).
%
% Used together,\Version{3.3} the next two options restore pre-version 3.3 handling of flags that could lead to undocumented behavior. The \texttt{oldreset} option causes all Boolean flags related to the prefix macros and long\,/\,short name forms to be reset locally. The new default is to reset them  globally (Section~\ref{sec:Customize}). Likewise, the \texttt{oldpass} option allows the long\,/\,short flags to pass through \cmd{\JustIndex} instead of being reset (Section~\ref{sec:IndexControl}).
% \newpage
%
% \noindent{\bfseries Enable\,/\,Disable Indexing}\medskip\\
% \bgroup\begin{tabular}{p{0.22\textwidth}p{0.6\textwidth}}
% \quad\bfseries\texttt{index} & \bfseries Create index entries in place with names.\smallskip\\
% \leavevmode\color{nared}\quad\texttt{noindex} & \leavevmode\color{nared}Suppress indexing of names.\medskip\\
% \end{tabular}\egroup
% 
% \noindent These options and related macros do not affect the normal use of \cmd{\index}. They apply only to the \textsf{nameauth} package macros. The default \texttt{index} option enables name indexing right away. The \texttt{noindex} option disables the indexing of names until \cmd{\IndexActive} enables it. \emph{Caution:}\Warn{} using \texttt{noindex} and \cmd{\IndexInactive} prevents index tags until you call \cmd{\IndexActive}, as explained also in Section~\ref{sec:IndexControl}.\bigskip
%
% \noindent{\bfseries Enable\,/\,Disable Index Sorting}\medskip\\
% \bgroup\begin{tabular}{p{0.22\textwidth}p{0.6\textwidth}}
% \quad\bfseries\texttt{pretag} & \bfseries Create sort keys used with \texttt{makeindex}.\smallskip\\
% \leavevmode\color{nared}\quad\texttt{nopretag} & \leavevmode\color{nared}Do not create sort keys.\medskip\\
% \end{tabular}\egroup
% 
% \noindent The default allows \cmd{\PretagName} to create sort keys used with \texttt{makeindex}. The \texttt{nopretag} option disables the sorting mechanism, e.g., if a different sorting method is used with \texttt{xindy}. See Section~\ref{sec:IndexSort}.
%
% \begin{center}\bfseries Affect the Syntax of Names\end{center}
%
% \noindent{\bfseries Show\,/\,Hide Affix Commas}\medskip\\
% \bgroup\begin{tabular}{p{0.22\textwidth}p{0.6\textwidth}}
% \quad\bfseries\texttt{nocomma} & \bfseries Suppress commas between surnames and affixes, following the \textit{Chicago Manual of Style} and other conventions.\smallskip\\
% \leavevmode\color{nared}\quad\texttt{comma} & \leavevmode\color{nared}Retain commas between surnames and affixes.\medskip\\
% \end{tabular}\egroup
% 
% \noindent If you use \emph{modern standards}, choose the default \texttt{nocomma} option to get, e.g., \Name[J.E.]{Carter, Jr.}[James Earl]. If you need to adopt \emph{older standards} that use commas between surnames and affixes, you have two choices:
% \begin{enumerate}
% \item The \texttt{comma} option globally produces, e.g., \ShowComma\Name*[J.E.]{Carter, Jr.}[James Earl].
% \item Section~\ref{sec:Affix} shows how one can use \cmd{\ShowComma} with the \texttt{nocomma} option and \cmd{\NoComma} with the \texttt{comma} option to get per-name results.
% \end{enumerate}
% In both cases, the display of commas (or the lack thereof) does not affect the appearance or the sorting of index entries.\bigskip
%
% \noindent{\bfseries Capitalize Entire Surnames}\medskip\\
% \bgroup\begin{tabular}{p{0.22\textwidth}p{0.6\textwidth}}
% \quad\bfseries\texttt{normalcaps} & \bfseries Do not perform any special capitalization.\smallskip\\
% \leavevmode\color{nared}\quad\texttt{allcaps} & \leavevmode\color{nared}Capitalize entire surnames, e.g., romanized Eastern names.\medskip\\
% \end{tabular}\egroup
% 
% \noindent This only capitalizes names printed in the body text. English standards usually do not propagate typographic changes into the index.
%
% Still, you can use this package with non-English conventions (just not via these options). You can add, e.g., uppercase or small caps in surnames, formatting them also in the index. See also Sections~\ref{sec:AltFormat} and~\ref{sec:Hooksiii}. The simplified interface aids the embedding of control sequences in names. Section~\ref{sec:Eastern} deals with capitalization on a section-level and per-name basis.
% \newpage
%
% \noindent{\bfseries Reverse Name Order}\medskip\\
% \bgroup\begin{tabular}{p{0.22\textwidth}p{0.6\textwidth}}
% \quad\bfseries\texttt{notreversed} & \bfseries Print names in the order specified by \cmd{\Name} and the other macros.\smallskip\\
% \leavevmode\color{nared}\quad\texttt{allreversed} & \leavevmode\color{nared}Print all name forms in ``smart'' reverse order.\smallskip\\
% \leavevmode\color{nared}\quad\texttt{allrevcomma} & \leavevmode\color{nared}Print all names in ``Surname, Forenames'' order, meant for Western names.\medskip\\
% \end{tabular}\egroup
% 
% \noindent These three options are mutually exclusive. Section~\ref{sec:Eastern} speaks more about reversing. The \texttt{allreversed} option, \cmd{\ReverseActive}, and \cmd{\RevName} work with all names and override \texttt{allrevcomma} and its macros.
%
% So-called ``last-comma-first'' lists of names via \texttt{allrevcomma} and the reversing macros \cmd{\ReverseCommaActive} and \cmd{\RevComma} (Section~\ref{sec:LastFirst}) are \emph{not} the same as the \texttt{comma} option. They only affect Western names.
%
% \phantomsection
% \label{page:PostProcess}
%
% \begin{center}\bfseries Typographic Post-Processing\end{center}
%
% \noindent{\bfseries Formatting Attributes}\medskip\\
% \bgroup\begin{tabular}{p{0.22\textwidth}p{0.6\textwidth}}
% \quad\bfseries\texttt{noformat} & \bfseries Do not define a default format.\smallskip\\
% \leavevmode\color{nared}\quad\texttt{smallcaps} & \leavevmode\color{nared}First use of a main-matter name in small caps.\smallskip\\
% \leavevmode\color{nared}\quad\texttt{italic} & \leavevmode\color{nared}First use of a main-matter name in italic.\smallskip\\
% \leavevmode\color{nared}\quad\texttt{boldface} & \leavevmode\color{nared}First use of a main-matter name in boldface.\medskip\\
% \end{tabular}\egroup
% 
% \noindent Current\Version{2.5} versions assign no default formatting to names. Most users have preferred the \texttt{noformat} option as the default and then design their own hooks as needed.\footnote{For those that want the old default option from the early days of this package, one can recover that behavior with the \texttt{smallcaps} option.}
% The options above are ``quick'' solutions based on English typography.
%
% What\Version{2.4} was ``typographic formatting'' has become a generalized concept of ``post-processing'' via hook macros.\footnote{This package was designed with type hierarchies and flexibility in mind. See Robert Bringhurst, \textit{The Elements of Typographic Style}, version 3.2 (Point Roberts, Washington: Hartley \& Marks, 2008), 53--60. Typographic inspiration comes from Bernhard Lohse, \textit{Luthers Theologie} (G\"ottingen: Vandenhoeck \& Ruprecht, 1995) and Jaroslav J. Pelikan Jr., \textit{The Christian Tradition: A History of the Development of Doctrine}, 5 vols. (Chicago: Chicago UP, 1971--89).}
% Post-processing does not affect the index. Sections \ref{sec:Formatting}, \ref{sec:Hooksi}, \ref{sec:Hooksii}, and~\ref{sec:Hooksiii} explain these hooks in greater detail:
% \begin{itemize}
% \item \cmd{\NamesFormat} formats first uses of main-matter names.
% \item \cmd{\MainNameHook} formats subsequent uses of main-matter names.
% \item \cmd{\FrontNamesFormat} formats first uses of front-matter names.
% \item \cmd{\FrontNameHook} formats subsequent uses of front-matter names.
% \end{itemize}
%
% Changes\Info{\cmd{\global}} to the formatting hooks apply within the scope where they are made. Use \cmd{\global} explicitly to alter that. \cmd{\NamesFormat} originally was the only hook, so any oddity in the naming of these hooks results from the need for backward compatibility with old versions.
%
% Section~\ref{sec:AKA} discusses how \cmd{\AKA} does not respect these formatting systems and uses the hooks differently. To avoid using the \texttt{formatAKA} option and \cmd{\ForceName} with \cmd{\AKA}, Section~\ref{sec:IndexXref} shows how to use \cmd{\IndexRef} and \cmd{\Name} instead.
% \newpage
%
% \begin{center}\bfseries Alternate or Continental Formatting\end{center}
%
% \noindent{\bfseries Alternate Formatting}\medskip\\
% \bgroup\begin{tabular}{p{0.22\textwidth}p{0.6\textwidth}}
% \leavevmode\color{nared}\quad\texttt{altformat} & \leavevmode\color{nared}Make available the alternate formatting framework from the start of the document. Activate formatting by default.\medskip\\
% \end{tabular}\egroup
% 
% \noindent A built-in\Version{3.1} framework provides an alternate formatting mechanism that can be used for ``Continental'' formatting that one sees in German, French, and so on. Continental standards often format surnames only, both in the text and in the index. Section~\ref{sec:AltFormat} introduces the topic and should be sufficient for most users, while Section~\ref{sec:Hooksiii} goes into greater detail.
%
% The previous methods that produced Continental formatting were rather complex compared to the current, simplified manner of doing so. Yet it is likely that these older solutions still ought to work. The error protection that prevents \cmd{\CapThis} from breaking alternately formatted names remains available to these older solutions by using \texttt{altformat} or the related macros (Section~\ref{sec:AltFormat}).\bigskip
%
% \phantomsection
% \label{page:Priorities}
% \begin{center}\bfseries Feature Priority\end{center}
%
% \noindent The following table shows the relative priority of package options and macros related to indexing, capitalization, and reversing. The darker the row, the lower the priority. Those macros or options in a particular category (column) that have higher priority (row) tend to override similar macros that have lower priority in that same category.
% 
% Thus, \cmd{\IndexInactive} overrides \cmd{\JustIndex}, which overrides \cmd{\SkipIndex}; using \cmd{\SeeAlso} depends entirely on the interaction of the three others.
%
% \begin{center}\small\MyStretch
% \begin{tabular}{llll}\toprule
%   \bfseries Indexing & \bfseries Capitalization & \bfseries Reversing & \bfseries Name Forms, \\
%   & & & \bfseries Commas, Breaks\\\midrule
%   \texttt{index}       & \texttt{normalcaps}    & \texttt{notreversed}    & \cmd{\ForgetThis}\\
%   \texttt{noindex}     & \texttt{allcaps}       & \texttt{allreversed}    & \cmd{\DropAffix}\\
%   \cmd{\IndexActive}   & \cmd{\AllCapsInactive} & \cmd{\ReverseActive}    \\
%   \cmd{\IndexInactive} & \cmd{\AllCapsActive}   & \cmd{\ReverseInactive}  \\
%   \rowcolor{black!7!white}\cmd{\JustIndex} & \cmd{\CapName} & \cmd{\RevName} & \cmd{\SubvertThis}\\
%   \rowcolor{black!7!white} &                    &                         & \cmd{\ForceName}\\
%   \rowcolor{black!7!white} &                    &                         & \cmd{\NoComma}\\
%   \rowcolor{black!12!white}\cmd{\SkipIndex} & \cmd{\AccentCapThis} & \texttt{allrevcomma} & \cmd{\KeepName}\\
%   \rowcolor{black!12!white} &                   & \cmd{\RevCommaActive}   & \cmd{\ForceFN}\\
%   \rowcolor{black!12!white} &                   & \cmd{\RevCommaInactive} & \cmd{\ShowComma}\\
%   \rowcolor{black!17!white}\cmd{\SeeAlso} & \cmd{\CapThis} & \cmd{\RevComma} & \cmd{\KeepAffix}\\\bottomrule
% \end{tabular}
% \end{center}
%
% \ReturnLink
% \newpage
%
% \subsection{Naming Macros}
%
% In this manual we modify the formatting hooks to show first and later name uses, forcing such uses as needed (Sections~\ref{sec:Formatting} and~\ref{sec:NameControl}). All naming macros create index entries before and after a name for when a name straddles a page break.
%
% \subsubsection{\texttt{\textbackslash Name} and \texttt{\textbackslash Name*}}
% \label{sec:Naming}
%
% \DescribeMacro{\Name}
% \cmd{\Name} displays and indexes names. It always prints the \meta{SNN} argument. \cmd{\Name} prints the full name at the first occurrence,
% \DescribeMacro{\Name*}
% then usually just the \meta{SNN} argument thereafter. \cmd{\Name*} always prints the full name:
% \begin{quote}
% \fbox{\vbox{\hbox{\mystrut\ \cmd{\Name}| |\oarg{FNN}\marg{SNN|,| Affix}\oarg{Alternate} }\par
% \hbox{\mystrut\ \cmd{\Name}|*|\oarg{FNN}\marg{SNN|,| Affix}\oarg{Alternate} }}}
% \end{quote}
%
% In\Version{3.0} the body text, not the index, the \meta{Alternate} argument replaces either \meta{FNN} or, if \meta{FNN} is absent, \meta{Affix}.\footnote{If \meta{Alternate} is \cmd{\ignorespaces}, the Western long form of \cmd{\Name} looks like the short form. ``Native'' Eastern and ancient forms would have an extra trailing space.}
% If both \meta{FNN} and \meta{Affix} are absent, then the obsolete syntax is used (Section~\ref{sec:Obsolete}).
%
% \begin{quote}\small
% \StartNum
% \begin{verbatim}
%\begin{nameauth}
%  \< Einstein  & Albert & Einstein        & >
%  \< Cicero    & M.T.   & Cicero          & >
%  \< Confucius &        & Confucius       & >
%  \< Miyaz     &        & Miyazaki, Hayao & >
%  \< Eliz      &        & Elizabeth, I    & >
%\end{nameauth}\end{verbatim}
% \end{quote}
%
% \begin{center}\small\MyStretch
% \begin{tabular}{ll}\toprule
% |\Name [Albert]{Einstein}| or |\Einstein| & \Name [Albert]{Einstein}\\
% |\Name*[Albert]{Einstein}| or |\LEinstein| & \LEinstein\\
% |\Name [Albert]{Einstein}| or |\Einstein| & \Einstein\\
% \rowcolor{black!7!white}|\Name [M.T.]{Cicero}| or |\Cicero| & \Name [M.T.]{Cicero}\\
% \rowcolor{black!7!white}|\Name*[M.T.]{Cicero}[Marcus Tullius]| & \Name*[M.T.]{Cicero}[Marcus Tullius]\\
% \rowcolor{black!7!white}|\Name [M.T.]{Cicero}| or |\Cicero| & \Cicero\\
% |\Name {Confucius}|, |\Confucius| & \Confucius\\
% \footnotesize Same for all variants; no \meta{Affix} or \meta{Alternate}. & \Confucius\\
% \rowcolor{black!7!white}|\Name {Miyazaki, Hayao}| or |\Miyaz| & \ForgetThis\Name {Miyazaki, Hayao}\\
% \rowcolor{black!7!white}|\Name*{Miyazaki, Hayao}[Sensei]| & \Name*{Miyazaki, Hayao}[Sensei]\\
% \rowcolor{black!7!white}|\Name {Miyazaki, Hayao}| or |\Miyaz| & \Name {Miyazaki, Hayao}\\
% |\Name {Elizabeth, I}| or |\Eliz| & \ForgetThis\Eliz\\
% |\Name*{Elizabeth, I}| or |\LEliz| & \Name*{Elizabeth, I}\\
% |\Name {Elizabeth, I}| or |\Eliz| & \Name {Elizabeth, I}\\\bottomrule
% \end{tabular}
% \end{center}
%
% When using the simplified interface, the preferred way to get alternate names is |\LCicero[Marcus Tullius]| and |\LMiyaz[Sensei]|: \LCicero[Marcus Tullius] and \LMiyaz[Sensei]. The alternate forename is not shown in subsequent short name references e.g., |\Cicero[Marcus Tullius]| \Cicero[Marcus Tullius]. 
%
% \newpage
%
% \subsubsection{Forenames: \texttt{\textbackslash FName}}
% \label{sec:FName}
%
% \DescribeMacro{\FName}
% \cmd{\FName} and its synonym \cmd{\FName*} print personal names only in subsequent name uses. They print full names for first uses.
% \DescribeMacro{\FName*}
% These synonyms let one add an \texttt{F} either to \cmd{\Name} or \cmd{\Name*} to get the same effect:
% \begin{quote}
% \fbox{\vbox{\hbox{\mystrut\ \cmd{\FName\ }\oarg{FNN}\marg{SNN|,| Affix}\oarg{Alternate} }\par
% \hbox{\mystrut\ \cmd{\FName*}\oarg{FNN}\marg{SNN|,| Affix}\oarg{Alternate} }}}
% \end{quote}
%
% \DescribeMacro{\ForceFN}
% These macros work with both Eastern and Western names, but to get an Eastern personal name, one must precede these macros with \cmd{\ForceFN}.
% See\Version{3.0} also Sections~\ref{sec:NameParticles} and~\ref{sec:NameControl} on how to vary some of the forms below:
%
% \begin{center}\small\MyStretch
% \begin{tabular}{ll}\toprule
% |\FName[Albert]{Einstein}| or |\SEinstein| & \SEinstein\\
% \rowcolor{black!7!white}|\FName[M.T.]{Cicero}[Marcus Tullius]| & \\
% \rowcolor{black!7!white}or |\SCicero[Marcus Tullius]| & \SCicero[Marcus Tullius]\\
% |\FName{Confucius}|  or |\SConfucius |& \FName{Confucius}\\
% \rowcolor{black!7!white}|\FName{Miyazaki, Hayao}| or |\SMiyaz| & \FName{Miyazaki, Hayao}\\
% |\ForceFN\FName{Miyazaki, Hayao}| & \\
% or |\ForceFN\SMiyaz| & \ForceFN\FName{Miyazaki, Hayao} \\
% \rowcolor{black!7!white}|\ForceFN\FName{Miyazaki, Hayao}[Sensei]| & \\
% \rowcolor{black!7!white}or |\ForceFN\SMiyaz[Sensei]| & \ForceFN\FName{Miyazaki, Hayao}[Sensei]\\
% |\FName{Elizabeth, I}| or |\SEliz| & \SEliz\\
% \rowcolor{black!7!white}|\ForceFN\SEliz[Good Queen Bess]| & \ForceFN\SEliz[Good Queen Bess]\\\bottomrule
% \end{tabular}
% \end{center}
%
% The \meta{Alternate} argument replaces forenames in the text, which strongly shapes the use of \cmd{\FName}.\footnote{If \meta{Alternate} is \cmd{\ignorespaces}, the Western long form of \cmd{\FName} looks like the short form of \cmd{\Name}, while the Western short form of \cmd{\FName} acts like \cmd{\leavevmode} and prints nothing. ``Native'' Eastern and ancient forms would have an extra trailing space.}
% We recap what we saw on page~\pageref{page:ArgIV}, emphasizing forenames:
% \begin{quote}\small
% \StartNum
% \begin{verbatim}
%\begin{nameauth}
%  \< Lewis & Clive Staples & Lewis             &        >
%  \< CSL   & Clive Staples & Lewis             & C.S.   >
%  \< Ches  & Chesley B.    & Sullenberger, III &        >
%  \< Sully & Chesley B.    & Sullenberger, III & Sully  >
%  \< Miyaz &               & Miyazaki, Hayao   &        >
%  \< MSens &               & Miyazaki, Hayao   & Sensei >
%\end{nameauth}\end{verbatim}
% \end{quote}
%
% These share name patterns: \cmd{\SCSL} \SCSL, \cmd{\SLewis} \SLewis; \cmd{\SChes} \SChes, \cmd{\SSully} \SSully; \cmd{\SMiyaz} \SMiyaz, \cmd{\SMSens} \SMSens.
%
% Equivalents: \cmd{\SCSL} \SCSL, \cmd{\SLewis[C.S.]} \SLewis[C.S.]; \cmd{\SSully} \SSully, \cmd{\SChes[Sully]} \SChes[Sully]; |\ForceFN\SMSens| \ForceFN\SMSens, |\ForceFN\SMiyaz[Sensei]|: \ForceFN\SMiyaz[Sensei].
% 
% These\Warn{} fail: \cmd{\SCSL[Jack]}: \SCSL[Jack]; \cmd{\SSully[Chesley]}: \SSully[Chesley]; and |\ForceFN\SMSens[Hayao]|: \ForceFN\SMSens[Hayao]. Whenever \meta{arg4} of the \texttt{nameauth} environment is used, the respective shorthands cannot take optional arguments.
%
% \ReturnLink
% \newpage
%
% \subsubsection{Variant Names}
% \label{sec:VarNames}
%
% This\Version{3.1} section explains how to manage more complicated variants, which gives one the skills needed to implement a name authority. We draw from Sections~\ref{sec:Formatting}, \ref{sec:IndexXref}, \ref{sec:IndexSort}, \ref{sec:NameControl}, and \ref{sec:AKA}. One might want to consult those sections also.
% 
% We\Info{variant forenames} begin with the easier kind of variant names, namely, variant forenames indexed under a canonical name entry:
% \begin{quote}\small
% \StartNum
% \begin{verbatim}
%\begin{nameauth}
%  \< Tyson & Mike & Tyson & >
%  \< Iron & Mike & Tyson & Iron Mike >
%\end{nameauth}\end{verbatim}
%
% \begin{tabular}{@{}lrlrl}
% Same pattern: & \cmd{\Iron} & \Iron & \cmd{\LTyson} & \LTyson\\
% & \cmd{\SIron} & \SIron & \cmd{\STyson} & \STyson\\
% \end{tabular}
% \end{quote}
% Since \LIron\ is indexed as ``\ShowIdxPageref*[Mike]{Tyson}'' throughout the document, we can use |\IndexRef{Iron Mike}{Tyson, Mike}| with no output in the text or |\AKA[Mike]{Tyson}{Iron Mike}| \AKA[Mike]{Tyson}{Iron Mike} to print a name. Both create the cross-reference ``Iron Mike \textit{see} Tyson, Mike'' in the index.
%
% Variant\Info{variant surnames} family names are more complicated than variant personal names. For surname variants, one can use the following method to get fairly good results, depending on the trade-offs that one wishes to accept:
% \begin{quote}\small
% \StartNum
% \begin{verbatim}
%\begin{nameauth}
%  \< DuBois    & W.E.B. & Du~Bois & >
%  \< AltDuBois & W.E.B. & DuBois  & >
%\end{nameauth}
%\PretagName[W.E.B.]{Du~Bois}{Dubois, W.E.B.}\end{verbatim}
% \end{quote}
% \begin{enumerate}
% \item We decide the canonical name form: \cmd{\DuBois} \ForgetThis\DuBois.
% \item Both\Warn{} \cmd{\Name[W.E.B.]\{Du Bois\}} and \cmd{\Name[W.E.B.]\{DuBois\}} have the pattern ``\texttt{\ShowPattern[W.E.B.]{Du Bois}}'' (Section~\ref{sec:NamePatterns}). Here we use |Du~Bois| as the argument because we want no breaks, giving us ``\texttt{\ShowPattern[W.E.B.]{Du~Bois}}''.
% \item We set the sort key for both names to be \texttt{\{Dubois, W.E.B.\}}. If it were of the form \texttt{\{Du Bois, W.E.B.\}}, they would sort differently (Section~\ref{sec:IndexSort}). One must check a style manual for proper sorting.
% \item Instead of using \cmd{\SkipIndex}\cmd{\AltDuBois} many times, we create a cross-reference in the preamble so that no page entry for the alternate form will occur in the index:\IndexRef[W.E.B.]{DuBois}{Du Bois, W.E.B.}\smallskip\\
% |  \IndexRef[W.E.B.]{DuBois}{Du Bois, W.E.B.}|
% \item We can use |\JustIndex\DuBois\AltDuBois| \JustIndex\DuBois\AltDuBois, keep full stop detection, and check if the name straddles a page break in order to append |\JustIndex\DuBois| if needed.
% \item If we create a macro like the one below, we lose full stop detection but then we do not have to check if the name straddles a page break. Normally, the name macros create two index entries each in order to handle this issue automatically:\smallskip\\
% |  \newcommand\NewDuBois%|\\
% |    {\JustIndex\DuBois\AltDubois\JustIndex\DuBois}|
% \end{enumerate}
% 
% \newpage
% \begin{center}\bfseries Example Name Authority\end{center}
% 
% Below are a couple of names from a name authority created for a translation of \textit{De Diaconis et Diaconissis Veteris Ecclesiae Liber Commentarius} by \Name[Caspar]{Ziegler}, of which the present author was the editor.\footnote{The book, \textit{The Diaconate of the Ancient and Medieval Church}, originally was typeset using \LaTeX, but had to be converted to a different format. Using \LaTeX, the present author has published Charles P. Schaum and Albert B. Collver III, \textit{Breath of God, Yet Work of Man: Scripture, Philosophy, Dialogue, and Conflict} (St. Louis: Concordia Publishing House, 2019).}
% 
% Constructing that name authority was a challenge. In order to get the names right\,---\,the deceased translator unfortunately had left them in abbreviated Latin, as well as leaving many place names in Latin or translating them incorrectly\,---\,the present author used the following sources, among several others:
% 
% \begin{itemize}
%   \item CERL Thesaurus: \url{https://data.cerl.org/thesaurus/_search}
%   \item Virtual International Authority File: \url{http://viaf.org/}
%   \item EDIT16: \url{http://edit16.iccu.sbn.it/web_iccu/ehome.htm}
%   \item WorldCat: \url{https://www.worldcat.org/}
%   \item An older version of Graesse, \textit{Orbis Latinus}:\\ \url{http://www.columbia.edu/acis/ets/Graesse/contents.html}
% \end{itemize}
% 
% This author followed the scholarly standards for determining the canonical name forms and used the alternate names (which were the ones actually in the original text) to refer to the canonical forms. I just translated all the place-names.
% 
% Below we have candidates for sorting with \cmd{\PretagName} (Section~\ref{sec:IndexSort}) and potential use of \cmd{\CapThis} (Section~\ref{sec:NameParticles}). After using \cmd{\IndexRef} with a particular name, using \cmd{\Name} with that same name will not create a page reference from that point onward (Section~\ref{sec:IndexXref}).
% 
% \begin{quote}
%   \StartNum
%   \begin{verbatim}
%\PretagName[Jacques]{De~Pamele}{Depamele, Jacques}
%\Name[Jacques]{De~Pamele}[Jacques de~Joigny]
%\IndexRef[Jacobus]{Pamelius}{De~Pamele, Jacques}
%\Name[Jacobus]{Pamelius}
%
%\PretagName[Giovanni]{d'Andrea}{Dandrea, Giovanni}
%\Name[Giovanni]{d'Andrea}
%\IndexRef[Ioannes]{Andreae}{d'Andrea, Giovanni}
%\Name[Ioannes]{Andreae}\end{verbatim}
%
% \medskip \begin{tabular}{ll}\toprule
%   Canonical Name & Alternate Name\\\midrule
%   \Name[Jacques]{De~Pamele}[Jacques de~Joigny] &
%   \leavevmode\IndexRef[Jacobus]{Pamelius}{De~Pamele, Jacques}\Name[Jacobus]{Pamelius}\\
%   \rowcolor{black!7!white}\Name[Giovanni]{d'Andrea} &
%   \leavevmode\IndexRef[Ioannes]{Andreae}{d'Andrea, Giovanni}\Name[Ioannes]{Andreae}\\\bottomrule
%   \end{tabular}
% \end{quote}
% \CapThis\Name[Giovanni]{d'Andrea} |\CapThis\Name[Giovanni]{d'Andrea}| can be used at the beginning of a sentence. |\Name[Jacques]{De~Pamele}| gives \Name[Jacques]{De~Pamele}.
% 
% \ReturnLink
% \newpage
%
% \subsection{Language Topics}
% \label{sec:Lang}
% This section looks at how \textsf{nameauth} addresses grammar, usage, and cultural standards. The concept of comma-delimited affixes dominates much of this section.
%
% \subsubsection[Affixes]{Affixes Require Commas}
% \label{sec:Affix}
%
% A comma is required to separate a Western surname and affix, an Eastern family name and personal name, and an ancient name and affix. Yet we must take care because an example like \cmd{\Name\{}\cmd{\textsc\{a Name, Problem\}\}} will halt \LaTeX\ with errors (Section~\ref{sec:AltFormat}). Spaces around the comma are ignored (Section~\ref{sec:ErrorProt}).
%
% \begin{center}\small\ForgetName{Sun, Yat-sen}\MyStretch
% \begin{tabular}{ll}\toprule
% |\Name[Oskar]{Hammerstein, II}| & \KeepAffix\Name[Oskar]{Hammerstein, II}\\
% |\Name[Oskar]{Hammerstein, II}| & \Name[Oskar]{Hammerstein, II}\\
% \rowcolor{black!7!white}|\Name{Louis, XIV}| & \KeepAffix\Name{Louis, XIV}\\
% \rowcolor{black!7!white}|\Name{Louis, XIV}| & \Name{Louis, XIV}\\
% |\Name{Sun, Yat-sen}| & \KeepAffix\Name{Sun, Yat-sen}\\
% |\Name{Sun, Yat-sen}| & \Name{Sun, Yat-sen}\\\bottomrule
% \end{tabular}
% \end{center}
%
% Western\Warn{} names with affixes must use the comma-delimited syntax. Using the obsolete syntax, |\SkipIndex\Name[Oskar]{Hammerstein}[II]| produces \SkipIndex\Name[Oskar]{Hammerstein}[II] which is an error. See also (Section~\ref{sec:AKA}).\medskip
%
% \DescribeMacro{\KeepAffix}
% In the text only, \cmd{\KeepAffix} turns the space \emph{between} \meta{SNN} and \meta{Affix} into a non-breaking space. This holds for a Western surname and affix, an ancient name and affix, and a ``native'' Eastern family name and personal name.\medskip
%
% \DescribeMacro{\KeepName}
% In the text only, \cmd{\KeepName} turns all spaces \emph{between} name elements \meta{FNN}, \meta{SNN}, and \meta{Affix} into non-breaking spaces;\Version{3.1} |\KeepName\LJWG[von]| \KeepName\LJWG[von] will not break. This macro does not alter spaces \emph{within} name elements like \meta{FNN} (French or German forenames) and \meta{SNN} (Spanish surnames). Both \cmd{\KeepAffix} and \cmd{\KeepName} can affect \textsf{nameauth} macros that print in the text.\medskip
%
% \DescribeMacro{\DropAffix}
% Preceding the naming macros with \cmd{\DropAffix} will suppress an affix in a Western name. |\DropAffix\Name*[Oskar]{Hammerstein, II}| produces\Version{3.0} ``\DropAffix\Name*[Oskar]{Hammerstein, II}.'' This does not affect non-Western names.\medskip
%
% \DescribeMacro{\ShowComma}
% \cmd{\ShowComma} forces a comma between a Western name and its affix. It works like the \texttt{comma} option on a per-name basis, and only in the body text.
% \DescribeMacro{\NoComma}
% \cmd{\NoComma} works like the \texttt{nocomma} option in the body text on a per-name basis.
% Neither\Version{2.6} of these macros affect the use of \cmd{\RevComma}, which always prints a comma.
% \begin{center}\small\MyStretch
% \begin{tabular}{ll}\toprule
% |\ShowComma\Name*[Louis]{Gossett, Jr.}| & \ShowComma\Name*[Louis]{Gossett, Jr.}\\
% \rowcolor{black!7!white}|\NoComma\Name*[Louis]{Gossett, Jr.}| & \NoComma\Name*[Louis]{Gossett, Jr.}\\\bottomrule
% \end{tabular}
% \end{center}
%
% \ReturnLink
% \newpage
%
% \subsubsection[Listing by Surname]{Listing Western names by Surname}
% \label{sec:LastFirst}
%
% \DescribeMacro{\ReverseCommaActive}
% In addition to the options for reversed comma listing (Section~\ref{sec:PkgOptions}), the macros \cmd{\ReverseCommaActive} and \cmd{\ReverseCommaInactive}
% \DescribeMacro{\ReverseCommaInactive}
% function the same way with blocks of text. They all override \cmd{\RevComma}.
% \DescribeMacro{\RevComma}
% These all reorder long Western name forms (via \cmd{\Name*} and the like). The first two are broad toggles, while the third works on a per-name basis. These\Version{3.0} macros only affect Western and ``non-native'' Eastern name forms.
% \begin{center}\small\MyStretch
% \begin{tabular}{lll}\toprule
% \ForgetThis\VBuren & \RevComma\LVBuren & OK\\
% \ForgetThis\Name[Oskar]{Hammerstein, II} & \RevComma\Name*[Oskar]{Hammerstein, II} & OK\\
% \ForgetThis\LNoguchi & \RevComma\LNoguchi\dag & OK\\
% \rowcolor{black!7!white}\ForgetThis\Aeth & \RevComma\LAeth & no change\\
% \rowcolor{black!7!white}\ForgetThis\Name{Chiang}[Kai-shek] & \RevComma\Name*{Chiang}[Kai-shek] & no change\\
% \rowcolor{black!7!white}\ForceName\Name{Confucius} & \RevComma\Name{Confucius} & no change\\\bottomrule
% \end{tabular}
% \end{center}
%
% Both\Info{\cmd{\global}} \cmd{\ReverseCommaActive} and \cmd{\ReverseCommaInactive} can be used either as a pair or singly within a local scope. Use \cmd{\global} to force a global effect.
%
% \ReturnLink
%
% \subsubsection{Eastern Names}
% \label{sec:Eastern}
%
% One\Info{``non-native''} produces a ``non-native'' Eastern name in the text by reversing a Western without \meta{Affix} using \cmd{\RevName}:
% \begin{quote}
%   \fbox{\mystrut\ \cmd{\RevName}\cmd{\Name*}\oarg{FNN}\marg{SNN}\oarg{Alternate} }
% \end{quote}
%
% The index entry of this name form looks like \meta{SNN}, \meta{FNN} (including the comma). This is a Western index entry. This form is used also for Hungarian names, e.g.: |\RevName\Name[Frenec]{Molnár}| \ForgetThis\RevName\Name[Frenec]{Molnár}\dag, \RevName\Name[Frenec]{Molnár}\dag.\medskip
%
% In\Info{``native''} contrast, ``native'' Eastern names use either comma-delimited syntax or the obsolete syntax. They have Eastern-form index entries \meta{SNN} \meta{Affix/Alternate} (no comma). The new syntax permits alternate names; the obsolete does not. These forms work also with ancient and medieval names:
%
% \begin{quote}
% \fbox{\ \parbox{0.45\textwidth}{%
%   \mystrut\cmd{\Name}\marg{SNN, Affix}\oarg{Alternate}\\
%   \mystrut\cmd{\Name}\marg{SNN}\oarg{Alternate}}
% \parbox{0.225\textwidth}{\color{nared}\mystrut\texttt{\%} \textit{new syntax}\\ \mystrut\texttt{\%} \textit{obsolete syntax}}}
% \end{quote}
%
% People\Info{avoid error} can make mistakes that these forms help one to avoid. For example, in an otherwise excellent German-language history textbook series, one finds an index entry for ``Yat-sen, Sun''. It should be ``Sun Yat-sen''.\footnote{See Immanuel Geiss, \textit{Personen: Die biographische Dimension der Weltgeschichte}, Geschichte Griffbereit vol. 2 (Munich: Wissen Media Verlag, 2002), 720. Errors arising from cultural differences and basic mistakes give justification for the design of \textsf{nameauth}.}
% The form |\Name{Sun, Yat-sen}| \Sun\ ensures the correct entry.\medskip
%
% \DescribeMacro{\ReverseActive}
% In addition to the options for reversing (Section~\ref{sec:PkgOptions}), \cmd{\ReverseActive} and \cmd{\ReverseInactive} reverse name order for blocks of text.
% \DescribeMacro{\ReverseInactive}
% These all override the use of \cmd{\RevName}, which reverses once per name.
% \DescribeMacro{\RevName}
% These macros do not affect the index. They work also with \cmd{\AKA} and its derivatives. The reverse mechanism shows only in full names, but it does not force full names. ``Non-native'' forms are shown by a dagger (\dag) in the next table:
% \begin{center}\small\MyStretch
% \begin{tabular}{rll}\toprule
%  & unchanged & |\RevName|\\\midrule
% |\LNoguchi| & \LNoguchi & \RevName\LNoguchi\dag\\
% |\LNoguchi[Doctor]| & \LNoguchi[Doctor] & \raise0.5ex\hbox to 5em{\hfil\rule{3em}{0.6pt}} \\
% |\LNoguchi[Sensei]| & \raise0.5ex\hbox to 5em{\hfil\rule{3em}{0.6pt}} & \RevName\LNoguchi[Sensei]\dag\\
% |\Noguchi| & \Noguchi & \RevName\Noguchi\dag\\
% |\SNoguchi| & \SNoguchi & \RevName\SNoguchi\dag\\
% \rowcolor{black!7!white}|\LYamt| & \LYamt & \RevName\LYamt\\
% \rowcolor{black!7!white}|\LYamt[Admiral]| & \raise0.5ex\hbox to 5em{\hfil\rule{3em}{0.6pt}} & \RevName\LYamt[Admiral]\\
% \rowcolor{black!7!white}|\Yamt| & \Yamt & \RevName\Yamt\\
% \rowcolor{black!7!white}|\SYamt| & \SYamt & \RevName\SYamt\\
% \rowcolor{black!7!white}|\ForceFN\SYamt| & \ForceFN\SYamt & \ForceFN\RevName\SYamt\\\bottomrule
% \end{tabular}\medskip\\
% \end{center}
%
% Both\Info{\cmd{\global}} \cmd{\ReverseActive} and \cmd{\ReverseInactive} can be used either as a pair or singly within an explicitly local scope. Use \cmd{\global} to force a global effect.\medskip
%
% \DescribeMacro{\AllCapsActive}
% In addition to the options for capitalizing (Section~\ref{sec:PkgOptions}), \cmd{\AllCapsActive} and \cmd{\AllCapsInactive} work for blocks of text.
% \DescribeMacro{\AllCapsInactive}
% All override \cmd{\CapName}, which works once per name.
% \DescribeMacro{\CapName}
% These capitalize \meta{SNN} in the body text only. They also work with \cmd{\AKA} and friends. For caps in the text and index see Sections~\ref{sec:AltFormat} and~\ref{sec:Hooksiii}. We show ``non-native'' Eastern forms with a dagger ({\dag}) and the old syntax with a double dagger(\ddag).
%
% \begin{center}\small\MyStretch\AllCapsActive
% \begin{tabular}{lll}\toprule
%  & |\CapName| only & |\CapName\RevName|\\\midrule
% |\Name*[Yoko]{Kanno}| & \CapName\Name*[Yoko]{Kanno} & \CapName\RevName\Name*[Yoko]{Kanno}\dag\\
% \rowcolor{black!7!white}|\Name*{Arai, Akino}| & \Name*{Arai, Akino} & \RevName\Name*{Arai, Akino}\\
% |\Name*{Ishida}[Yoko]| & \CapName\Name*{Ishida}[Yoko]\ddag & \CapName\RevName\Name*{Ishida}[Yoko]\ddag\\
% \rowcolor{black!7!white}|\Name*{Yohko}| & \Name*{Yohko} & \RevName\Name*{Yohko}\\\bottomrule
% \end{tabular}\AllCapsInactive
% \end{center}
% 
% Both\Info{\cmd{\global}} \cmd{\AllCapsActive} and \cmd{\AllCapsInactive} can be used either as a pair or singly within an explicitly local scope. Use \cmd{\global} to force a global effect.
%
% \ReturnLink
%
% \subsubsection[Particles / Ancient]{Particles, Medieval Names, and Ancient Names}
% \label{sec:NameParticles}
%
% English\Info{cap rules} names with particles \textit{de}, \textit{de\ la}, \textit{d'}, \textit{von}, \textit{van}, and \textit{ten} often keep them with the last name, using varied capitalization.\footnote{According to [\hyperlink{Mulvany}{Mulvany}, 165f.] and the \textit{Chicago Manual of Style}.} \textit{Le}, \textit{La}, and \textit{L'} always are capitalized unless preceded by \textit{de}. See also Sections~\ref{sec:SimpleStart}, \ref{sec:NamePatterns}, \ref{sec:VarNames}, and \ref{sec:AltFormat}.\medskip
%
% We\Info{non-breaking\break spaces} recommend inserting a tilde (active character for a non-breaking space) or \cmd{\nobreakspace} between some particles and names to prevent bad breaks, sorting them with \cmd{\PretagName} (Section~\ref{sec:IndexSort}). 
% Some particles look similar: \textit{L'} (L+apostrophe) and \textit{d'} (d+apostrophe) are two separate glyphs each. In contrast, \textit{Ľ} (L+caron) and \textit{ď} (d+caron) are one Unicode glyph each (Section~\ref{sec:Unicode}).
%\newpage
%
% \phantomsection
% \label{page:CapThis}
% \DescribeMacro{\CapThis}
% In English and modern Romance languages, e.g., \ForgetThis\Soto\ shows that these particles go in the \meta{SNN} argument of \cmd{\Name}: \Soto. When the particle appears at the beginning of a sentence, one must capitalize it:
% \begin{quote}\small
% |\CapThis\Soto\| \CapThis\Soto\ was a famous Spanish explorer in North America.
% \end{quote}
%
% \cmd{\CapName} overrides the \meta{SNN} created by \cmd{\CapThis}. \cmd{\CapThis}\Version{3.2} should work with all of the Unicode characters available in the T1 encoding (its mechanism is explained in Section~\ref{sec:Unicode} and on page~\pageref{page:CapSystem}). For a broader set of Unicode characters, consider using \texttt{xelatex} and \texttt{lualatex}.\medskip
%
% For\Info{surname variants} another example, we mention poet \Name[e.e.]{cummings}. One can have formatted name caps and inflections, e.g.: \ExcludeName[e.e.]{cummings's}``\SubvertThis\CapThis\Name[e.e.]{cummings's} motif of the goat-footed balloon man has underlying sexual themes that nevertheless have a childish facade.'' The easiest way to do that is from Section~\ref{sec:IndexXref}:
% \begin{quote}\small
% |\ExcludeName[e.e.]{cummings's}|\\
% . . .\\
% |\SubvertThis\CapThis\Name[e.e.]{cummings's}%|\\
% |\IndexName[e.e.]{cummings}|\dotfill \SubvertThis\CapThis\Name[e.e.]{cummings's}\IndexName[e.e.]{cummings}
% \end{quote}
%
% One\Warn{} must use \cmd{\SubvertThis} only for the first use to avoid ``\ForgetThis\CapThis\Name[e.e.]{cummings's}''; all name elements are capped with \cmd{\CapThis}. Using \cmd{\ExcludeName} keeps one from having to use \cmd{\SkipIndex} every time. With \textsf{nameauth} we can use both simple and complex solutions to name variation. See also Section~\ref{sec:VarNames}.
%
% Section~\ref{sec:AltFormat} explains how to use \cmd{\CapThis} with alternate formatting when using macros in name arguments. Page~\pageref{page:Inflections} describes how automation lends itself to Continental (French, German, etc.) formats and grammatical inflections.\medskip
%
% \DescribeMacro{\AccentCapThis}
% If one uses this package on a system that does not handle Unicode, one can use \cmd{\AccentCapThis} instead of \cmd{\CapThis}\Version{3.0} to handle active initial characters. Otherwise, one should not need to use \cmd{\AccentCapThis}.
%
% \begin{center}\bfseries Examples\end{center}
% 
% Medieval\Info{medieval names} names present some interesting difficulties, often based on the expected standards of the context in which they are used:
% \begin{quote}\small
% \StartNum
% \begin{verbatim}
%\PretagName{Thomas, à~Kempis}{Thomas Akempis}  % medieval
%\PretagName[Thomas]{à~Kempis}{Akempis, Thomas} % Western
%\IndexRef[Thomas]{à~Kempis}{Thomas à~Kempis}    % xref
%\ExcludeName{Thomas,\`a~Kempis} % alternate form excluded
%\begin{nameauth}
%  \< KempMed & & Thomas,   à~Kempis & >         % medieval
%  \< KempW     & Thomas  & à~Kempis & >         % Western
%\end{nameauth}\end{verbatim}
% \IndexRef[Thomas]{à~Kempis}{Thomas à~Kempis}
% \ExcludeName{Thomas,\`a~Kempis}
% \end{quote}
%
% The medieval forms\Version{3.1} \KempMed\ and \KempMed\ are indexed as ``\ShowIdxPageref*{Thomas, à~Kempis}.'' The place name |\ForceFN\SKempMed| ``\ForceFN\SKempMed'' (Latin for \textit{von Kempen}) technically is not a Western surname.
% \CapThis\ForceFN\SKempMed\ |\CapThis\ForceFN\SKempMed| starts a sentence. \Name{Thomas,\`a~Kempis} |\Name{Thomas,\`a~Kempis}| is different. \CapThis\SubvertThis\ForceFN\FName{Thomas, \`a~Kempis} is |\CapThis\SubvertThis\ForceFN\FName{Thomas,\`a~Kempis}|.
% One should use \cmd{\PretagName} to sort the index entry (Section~\ref{sec:IndexSort}). We excluded this alternate form (Section~\ref{sec:IndexXref}).
% \newpage
%
% Western forms\Warn{} like |\KempW|: \KempW\ are very different from  medieval forms and create different index entries. |\CapThis\KempW| gives ``\CapThis\KempW'' in the text and ``\ShowIdxPageref*[Thomas]{à~Kempis}'' in the index.
% 
% Above, we created a cross-reference from the Western form to the medieval form, preventing page entries (Section~\ref{sec:IndexXref}). If we sorted the cross-reference using |\PretagName[Thomas]{à~Kempis}{a Kempis, Thomas}|, it would precede \texttt{aardvark}. We use |\PretagName[Thomas]{à~Kempis}{Akempis, Thomas}|, which sorts the cross-reference between \texttt{ajar} and \texttt{alkaline}. One should check a style manual for correct sorting (Section~\ref{sec:IndexSort}).\medskip
% 
% \phantomsection
% \label{page:Sobriquets}
% \begingroup%^^A
% \renewcommand*\NamesFormat{}%^^A
% \renewcommand*\MainNameHook{}%^^A
% Ancient\Info{ancient names} contexts may or may not bind particles to surnames. The \meta{alternate} argument, \cmd{\PretagName}, and \cmd{\TagName} address this (Sections~\ref{sec:IndexSort}, \ref{sec:IndexTag}).
% 
% The next examples do not use the formatting conventions of this manual and sometimes hide details that are specific to this manual in order to keep things simple and reflect normal document usage. See the \texttt{dtx} source code for more information. First we use variants with \meta{alternate}:\footnote{Copies of these examples are in \texttt{examples.tex}, collocated with this manual.}
% \NameAddInfo{Demetrius, I}{ Soter}
% \begin{quote}\small
% \StartNum
% \begin{verbatim}
%\NameAddInfo{Demetrius, I}{ Soter}
%\PretagName{Demetrius, I}{Demetrius 1}
%\TagName{Demetrius, I}{ Soter, king}
%\begin{nameauth}
%  \< Dem & & Demetrius, I & >
%\end{nameauth}\end{verbatim}
%
% \smallskip
%   \begin{tabular}{ll}\toprule
%     |\Dem[I Soter]| & \Dem[I Soter]\\
%     \rowcolor{black!7!white}|\LDem| & \LDem\\
%     |\Dem|          & \Dem\\\bottomrule
%   \end{tabular}
% \end{quote}
%
% \makeatletter\renewcommand*\NamesFormat[1]{\begingroup%^^A
% \protected@edef\temp{\endgroup{#1%^^A
% \noexpand\NameQueryInfo[\unexpanded\expandafter{\the\@nameauth@toksa}]
% {\unexpanded\expandafter{\the\@nameauth@toksb}}
% [\unexpanded\expandafter{\the\@nameauth@toksc}]}}\temp}\makeatother
% For a more automated approach, we can use ``text tags'' in the formatting macros (see Sections~\ref{sec:TextTags}, \ref{sec:Hooksii}).
%
% \begin{quote}\small
% \ContinueNum
% \begin{verbatim}
%\makeatletter
%\renewcommand*\NamesFormat[1]{%
%  \begingroup%
%  \protected@edef\temp{\endgroup%
%    {#1\noexpand\NameQueryInfo
%      [\unexpanded\expandafter{\the\@nameauth@toksa}]
%      {\unexpanded\expandafter{\the\@nameauth@toksb}}
%      [\unexpanded\expandafter{\the\@nameauth@toksc}]%
%    }%
%  }%
%  \temp%
%}
%\makeatother\end{verbatim}
%
% \smallskip
%   \begin{tabular}{ll}\toprule
%     |\ForgetThis\Dem| & \ForgetThis\Dem\\
%     \rowcolor{black!7!white}|\LDem| & \LDem\\
%     |\Dem|            & \Dem\\\bottomrule
%   \end{tabular}
% \end{quote}
% 
% The Roman naming\Info{Roman names} system does present some challenges. As long as we do not use \cmd{\CapThis}, we do not need alternate formatting (Section~\ref{sec:AltFormat}). Earlier we treated \LCicero[Marcus Tullius] as a Western name. Now we show how to handle Roman names more properly.
% 
% Roman names have a \textit{praenomen}, a personal name, then a \textit{nomen}, a clan name, followed by a \textit{cognomen}, a ``nickname,'' except it could be inherited from one's father to denote clan branches. Added to that are \textit{agnomina}, affixed names.
% 
% Popular\Info{popular works} sources tend to treat the \textit{cognomen} as we might a surname, with the indexed form: \ShowIdxPageref*[\meta{praenomen} \meta{nomen}]{\meta{cognomen} \meta{agnomen}}.\footnote{See Geiss, \textit{Geschichte Griffbereit}; Kinder and Hilgemann, \textit{dtv-Atlas zur Weltgeschichte}, 2 vols., 29th printing (1964; Munich: Deutscher Taschenbuch Verlag, 1993). For further resources see: \url{http://books.infotoday.com/books/Indexing-names.shtml}. See also \url{https://en.wikipedia.org/wiki/Roman_naming_conventions}.}
% We want all names in the index, so we define macros in \meta{FNN} and \meta{SNN} that expand to become one or two components: \textit{praenomen} and \textit{nomen}; \textit{cognomen} and \textit{agnomen}. We begin by defining a name with macros using \cmd{\noexpand} to prevent error:
%\newif\ifSkipGens
%\newif\ifNoGens
%\newif\ifSkipAgnomen
%\newif\ifNoAgnomen
%\newcommand*\SCIPi{\ifNoGens Publius\else
%                   Publius Cornelius\fi}
%\newcommand*\SCIPii{\ifNoAgnomen Scipio\else
%                    Scipio Africanus\fi}
%\newcommand*\ScipioOnly{\SkipAgnomentrue\Scipio}
%\renewcommand*\NamesFormat[1]%^^A
%  {\ifSkipGens\NoGenstrue\fi\ifSkipAgnomen\NoAgnomentrue\fi#1%^^A
%  \global\SkipGensfalse\global\SkipAgnomenfalse}
%\renewcommand*\MainNameHook[1]%^^A
%  {\ifSkipGens\NoGenstrue\fi\ifSkipAgnomen\NoAgnomentrue\fi#1%^^A
%  \global\SkipGensfalse\global\SkipAgnomenfalse}
% \begin{quote}\small
% \StartNum
% \begin{verbatim}
%\begin{nameauth}
%  \< Scipio & \noexpand\SCIPi & \noexpand\SCIPii & >
%\end{nameauth}
%\PretagName[\noexpand\SCIPi]{\noexpand\SCIPii}{Scipio Africanus}\end{verbatim}
% \end{quote}
%
% We define the flags and macros by which the name will change. The global state of \cmd{\NoGens} and\cmd{\NoAgnomen} determine the index form. The local scope in the formatting hooks allows changes that are reset when exiting that scope. The logic is inverted; false prints long, true prints short:
% \begin{quote}\small
% \ContinueNum
% \begin{verbatim}
%\newif\ifSkipGens
%\newif\ifNoGens
%\newif\ifSkipAgnomen
%\newif\ifNoAgnomen
%\newcommand*\SCIPi{\ifNoGens Publius\else
%                   Publius Cornelius\fi}
%\newcommand*\SCIPii{\ifNoAgnomen Scipio\else
%                    Scipio Africanus\fi}
%\newcommand*\ScipioOnly{\SkipAgnomentrue\Scipio}
%\renewcommand*\NamesFormat[1]%^^A
%  {\ifSkipGens\NoGenstrue\fi\ifSkipAgnomen\NoAgnomentrue\fi#1%
%  \global\SkipGensfalse\global\SkipAgnomenfalse}
%\renewcommand*\MainNameHook[1]%^^A
%  {\ifSkipGens\NoGenstrue\fi\ifSkipAgnomen\NoAgnomentrue\fi#1%
%  \global\SkipGensfalse\global\SkipAgnomenfalse}\end{verbatim}
%\smallskip
% \ForgetThis\ScipioOnly\ \cmd{\ScipioOnly} was born around 236 \textsc{bc} into the Scipio branch of the Cornelius clan, one of six large patrician clans. \ScipioOnly\ \cmd{\ScipioOnly} rose to military fame during the Second Punic War. Thereafter he was known as \Scipio\ \cmd{\Scipio}.
% \end{quote}
% 
% An advantage of the popular format is that one can drop both \textit{praenomen} and \textit{nomen} automatically in subsequent uses. Yet in any case, one can define helper macros to change Boolean flags. The raw index entry is fairly lengthy by necessity, governed by the global state of the Boolean flags, and expanding to:
% \begin{quote}\small
% \ShowIdxPageref[\noexpand\SCIPi]{\noexpand\SCIPii}
% \end{quote}
% 
% The\Info{scholarly works} \textit{Oxford Classical Dictionary} and other scholarly sources index under the \textit{nomen}. That requires a similar approach, but it moves the \textit{nomen} from \meta{FNN} to \meta{SNN}. Although we will not index the name, we will show how to set up \SkipGenstrue\Scipio\ to work in that alternate configuration.
% \newpage
% 
% We keep the Boolean flags and formatting hooks from above. We redefine the name in the following manner:
% \begin{quote}\small
% \StartNum
% \begin{verbatim}
%\begin{nameauth}
%  \< OScipio & Publius & \noexpand\CSA & >
%\end{nameauth}
%\PretagName[Publius]{\noexpand\CSA}{Cornelius Scipio Africanus}\end{verbatim}
% \end{quote}
% 
%\newcommand*\CSA{\ifNoGens
%                   \ifNoAgnomen
%                   Scipio\else
%                   Scipio Africanus\fi
%                 \else\ifNoAgnomen
%                   Cornelius Scipio\else
%                   Cornelius Scipio Africanus\fi\fi}
% \ExcludeName[Publius]{\noexpand\CSA}
% We use a nested conditional in \meta{SNN}. The default still is to show all names so that they can be indexed that way. This time we decided to index under the popular form instead of the scholarly one, so we exclude the scholarly form:
% \begin{quote}\small
% \StartNum
% \begin{verbatim}
%\newcommand*\CSA{\ifNoGens
%                   \ifNoAgnomen
%                   Scipio\else
%                   Scipio Africanus\fi
%                 \else\ifNoAgnomen
%                   Cornelius Scipio\else
%                   Cornelius Scipio Africanus\fi\fi}
%\ExcludeName[Publius]{\noexpand\CSA}\end{verbatim}
% \end{quote}
% 
% The scholarly form has a different name pattern, so it is not compatible with the popular version. Nevertheless, we show what the raw index entry of the scholarly form would be. We include some of the more meaningful forms of both versions:
% \begin{quote}\small
%   \cmd{\ShowPattern[Publius]\{}\cmd{\noexpand}\cmd{\CSA\}}:\\
%   \hbox{}\quad \ShowPattern[Publius]{\noexpand\CSA}
%   
%   \cmd{\ShowIdxPageref[Publius]\{}\cmd{\noexpand}\cmd{\CSA\}}:\\
%   \hbox{}\quad \ShowIdxPageref[Publius]{\noexpand\CSA}
%   
%   \textbf{First use:}\\[0.5ex]
%   \cmd{\OScipio}:\quad \OScipio\\
%   \cmd{\Scipio\ }:\quad \ForgetThis\Scipio
%   
%   \textbf{Subsequent use:}\\[0.5ex]
%   \hphantom{\cmd{\SkipGenstrue}}\cmd{\OScipio}:\quad \OScipio\\
%   \cmd{\SkipGenstrue}\cmd{\OScipio}:\quad \SkipGenstrue\OScipio\\
%   \hphantom{\cmd{\SkipGenstrue}}\cmd{\Scipio\ }:\quad \SkipGenstrue\Scipio
%   
%   \textbf{Subsequent use, full, no \textit{agnomen}:}\\[0.5ex]
%   \cmd{\SkipAgnomentrue}\cmd{\LOScipio}:\quad \SkipAgnomentrue\LOScipio\\
%   \cmd{\SkipAgnomentrue}\cmd{\LScipio\ }:\quad \SkipAgnomentrue\LScipio
%   
%   \textbf{Subsequent use, shortest forms:}\\[0.5ex]
%   \hphantom{\cmd{\SkipGenstrue}}\cmd{\SkipAgnomentrue}\cmd{\OScipio}:\quad \SkipAgnomentrue\OScipio\\
%   \cmd{\SkipGenstrue}\cmd{\SkipAgnomentrue}\cmd{\OScipio}:\quad \SkipGenstrue\SkipAgnomentrue\OScipio\\
%   \hphantom{\cmd{\SkipGenstrue}}\cmd{\SkipAgnomentrue}\cmd{\Scipio\ }:\quad \SkipAgnomentrue\Scipio
%   
%   \textbf{Subsequent use, personal name:}\\[0.5ex]
%   \hphantom{\cmd{\SkipGenstrue}}\cmd{\SOScipio}:\quad \SOScipio\\
%   \cmd{\SkipGenstrue}\cmd{\SScipio\ }:\quad \SkipGenstrue\SScipio
% \end{quote}
% 
% See Sections~\ref{sec:ErrorProt}, \ref{sec:AltFormat}, and \ref{sec:Hooks} for more guidance on avoiding errors when using name arguments that contain macros.
% \endgroup
% 
% \ReturnLink
% \newpage
%
% \subsection{Basic Formatting}
% \label{sec:Formatting}
%
% Below are many of the forms and formats that names can have:
% 
% \ifDoTikZ
% \begin{tcolorbox}[colframe=naslate,sidebyside,lower separated=true,adjusted title={\hfil Full Forms, Front Matter\hspace{5em} Short Forms, Front matter}]\centering
%   {\cmd{\NamesInactive}}\NamesInactive
%   \begin{tcolorbox}[left=2mm,right=2mm,bottom=1mm,colback=white,adjusted title={\sffamily\bfseries\hfil First Use (Default)}]\small
%     \begin{tabular}{@{}ll@{}}
%       \cmd{\Name}  & \ForgetThis\Pat\\
%                    & \ForgetThis\Eliz\\
%                    & \ForgetThis\Yamt\\
%       \cmd{\Name*} & \ForgetThis\LPat\\
%       \cmd{\FName} & \ForgetThis\SPat\\
%     \end{tabular}
%   \end{tcolorbox}
%   \begin{tcolorbox}[left=2mm,right=2mm,bottom=1mm,colback=white,adjusted title={\sffamily\bfseries\hfil Later Use (\texttt{*} or \cmd{\L}\meta{macro})}]\small
%     \begin{tabular}{@{}ll@{}}
%       \cmd{\Name*} & \LPat\\
%                    & \LEliz\\
%                    & \LYamt\\
%     \end{tabular}
%   \end{tcolorbox}
%   \begin{tcolorbox}[left=2mm,right=2mm,bottom=1mm,colback=white,adjusted title={\sffamily\bfseries\hfil Long, with \cmd{\DropAffix}}]\small
%     \begin{tabular}{@{}ll@{}}
%       \cmd{\DropAffix}\cmd{\LPat}\\ \SubvertThis\DropAffix\LPat\\
%     \end{tabular}
%   \end{tcolorbox}
%   \tcblower\centering
%   {\cmd{\NamesInactive}}\NamesInactive
%   \begin{tcolorbox}[left=2mm,right=2mm,bottom=1mm,colback=white,adjusted title={\sffamily\bfseries\hfil Later Use (Default)}]\small
%     \begin{tabular}{@{}ll@{}}
%       \cmd{\Name}          & \SubvertThis\Pat; \SubvertThis\Eliz\\
%                            & \SubvertThis\Yamt\\
%       \cmd{\FName},        & \SubvertThis\SPat; \SubvertThis\SEliz\\
%       \cmd{\S}\meta{macro} & \SubvertThis\SYamt\\
%     \end{tabular}
%   \end{tcolorbox}\vspace{1.4ex}
%   \begin{tcolorbox}[left=2mm,right=2mm,bottom=1mm,colback=white,adjusted title={\sffamily\bfseries\hfil Later Use (\cmd{\ForceName})}]\small
%     \begin{tabular}{@{}ll@{}}
%       \cmd{\Name}          & \SubvertThis\ForceName\Pat; \SubvertThis\ForceName\Eliz\\
%                            & \SubvertThis\ForceName\Yamt\\
%       \cmd{\FName},        & \SubvertThis\ForceName\SPat; \SubvertThis\ForceName\SEliz\\
%       \cmd{\S}\meta{macro} & \SubvertThis\ForceName\SYamt\\
%     \end{tabular}
%   \end{tcolorbox}\vspace{1.4ex}
%   \begin{tcolorbox}[left=2mm,right=2mm,bottom=1mm,colback=white,adjusted title={\sffamily\bfseries\hfil Later Use (\cmd{\ForceFN})}]\small
%     \begin{tabular}{@{}ll@{}}
%       \cmd{\FName}, \cmd{\S}\meta{macro} & \SubvertThis\ForceFN\SYamt\\
%     \end{tabular}
%   \end{tcolorbox}  
% \end{tcolorbox}
% \vfil
% \begin{tcolorbox}[colframe=naslate,sidebyside,lower separated=true,adjusted title={\hfil Full Forms, Main Matter\hspace{5em} Short Forms, Main Matter}]\centering
%   {\cmd{\NamesActive}}\NamesActive
%   \begin{tcolorbox}[left=2mm,right=2mm,bottom=1mm,colback=white,adjusted title={\sffamily\bfseries\hfil First Use (Default)}]\small
%     \begin{tabular}{@{}ll@{}}
%       \cmd{\Name}  & \ForgetThis\Pat\\
%                    & \ForgetThis\Eliz\\
%                    & \ForgetThis\Yamt\\
%       \cmd{\Name*} & \ForgetThis\LPat\\
%       \cmd{\FName} & \ForgetThis\SPat\\
%     \end{tabular}
%   \end{tcolorbox}
%   \begin{tcolorbox}[left=2mm,right=2mm,bottom=1mm,colback=white,adjusted title={\sffamily\bfseries\hfil Later Use (\texttt{*} or \cmd{\L}\meta{macro})}]\small
%     \begin{tabular}{@{}ll@{}}
%       \cmd{\Name*} & \LPat\\
%                    & \LEliz\\
%                    & \LYamt\\
%     \end{tabular}
%   \end{tcolorbox}
%   \begin{tcolorbox}[left=2mm,right=2mm,bottom=1mm,colback=white,adjusted title={\sffamily\bfseries\hfil Long, with \cmd{\DropAffix}}]\small
%     \begin{tabular}{@{}ll@{}}
%       \cmd{\DropAffix}\cmd{\LPat}\\ \SubvertThis\DropAffix\LPat\\
%     \end{tabular}
%   \end{tcolorbox}
%   \tcblower\centering
%   {\cmd{\NamesActive}}\NamesActive
%   \begin{tcolorbox}[left=2mm,right=2mm,bottom=1mm,colback=white,adjusted title={\sffamily\bfseries\hfil Later Use (Default)}]\small
%     \begin{tabular}{@{}ll@{}}
%       \cmd{\Name}          & \SubvertThis\Pat; \SubvertThis\Eliz\\
%                            & \SubvertThis\Yamt\\
%       \cmd{\FName},        & \SubvertThis\SPat; \SubvertThis\SEliz\\
%       \cmd{\S}\meta{macro} & \SubvertThis\SYamt\\
%     \end{tabular}
%   \end{tcolorbox}\vspace{1.4ex}
%   \begin{tcolorbox}[left=2mm,right=2mm,bottom=1mm,colback=white,adjusted title={\sffamily\bfseries\hfil Later Use (\cmd{\ForceName})}]\small
%     \begin{tabular}{@{}ll@{}}
%       \cmd{\Name}          & \SubvertThis\ForceName\Pat; \SubvertThis\ForceName\Eliz\\
%                            & \SubvertThis\ForceName\Yamt\\
%       \cmd{\FName},        & \SubvertThis\ForceName\SPat; \SubvertThis\ForceName\SEliz\\
%       \cmd{\S}\meta{macro} & \SubvertThis\ForceName\SYamt\\
%     \end{tabular}
%   \end{tcolorbox}\vspace{1.4ex}
%   \begin{tcolorbox}[left=2mm,right=2mm,bottom=1mm,colback=white,adjusted title={\sffamily\bfseries\hfil Later Use (\cmd{\ForceFN})}]\small
%     \begin{tabular}{@{}ll@{}}
%       \cmd{\FName}, \cmd{\S}\meta{macro} & \SubvertThis\ForceFN\SYamt\\
%     \end{tabular}
%   \end{tcolorbox}  
% \end{tcolorbox}
% \else\bigskip
% \begin{center}\small\NamesInactive
%   \begin{tabular}{@{\hspace{4em}}c@{\hspace{6em}}c}
%     Full Forms, Front Matter & Short Forms, Front Matter\\
%     \cmd{\NamesInactive} & \cmd{\NamesInactive}\medskip\\
%   \end{tabular}
%   
%   \begin{tabular}{llll}\toprule
%       \cmd{\Name}   & \ForgetThis\Pat  & \cmd{\Name} & \SubvertThis\Pat, \SubvertThis\Eliz\\
%                     & \ForgetThis\Eliz & & \SubvertThis\Yamt\\
%                     & \ForgetThis\Yamt & \cmd{\FName}, & \SubvertThis\SPat; \SubvertThis\SEliz\\
%       \cmd{\Name*}  & \ForgetThis\LPat & \cmd{\S}\meta{macro} & \SubvertThis\SYamt\\
%       \cmd{\FName}  & \ForgetThis\SPat & \\\midrule
%                     &                  & Using \cmd{\ForceName} \\
%       \cmd{\Name*}, & \LPat            & \cmd{\Name} & \SubvertThis\ForceName\Pat; \SubvertThis\ForceName\Eliz\\
%       \cmd{\L}\meta{macro} & \LEliz    & & \SubvertThis\ForceName\Yamt\\
%                     & \LYamt           & \cmd{\FName}, & \SubvertThis\ForceName\SPat; \SubvertThis\ForceName\SEliz\\
%                     &                  & \cmd{\S}\meta{macro} & \SubvertThis\ForceName\SYamt\\\midrule
%                     &                  & Using \cmd{\ForceFN} \\
%       \cmd{\DropAffix}\cmd{\LPat} & \SubvertThis\DropAffix\LPat & \cmd{\FName}, \cmd{\S}\meta{macro} & \SubvertThis\ForceFN\SYamt\\\bottomrule%   \end{tabular}
% \end{center}\bigskip
% \begin{center}\small\NamesActive
%   \begin{tabular}{@{\hspace{4em}}c@{\hspace{6em}}c}
%     Full Forms, Main Matter & Short Forms, Main Matter\\
%     \cmd{\NamesActive} & \cmd{\NamesActive}\medskip\\
%   \end{tabular}
%   
%   \begin{tabular}{llll}\toprule
%       \cmd{\Name}   & \ForgetThis\Pat  & \cmd{\Name} & \SubvertThis\Pat; \SubvertThis\Eliz\\
%                     & \ForgetThis\Eliz & & \SubvertThis\Yamt\\
%                     & \ForgetThis\Yamt & \cmd{\FName}, & \SubvertThis\SPat; \SubvertThis\SEliz\\
%       \cmd{\Name*}  & \ForgetThis\LPat & \cmd{\S}\meta{macro} & \SubvertThis\SYamt\\
%       \cmd{\FName}  & \ForgetThis\SPat & \\\midrule
%                     &                  & Using \cmd{\ForceName} \\
%       \cmd{\Name*}, & \LPat            & \cmd{\Name} & \SubvertThis\ForceName\Pat; \SubvertThis\ForceName\Eliz\\
%       \cmd{\L}\meta{macro} & \LEliz    & & \SubvertThis\ForceName\Yamt\\
%                     & \LYamt           & \cmd{\FName}, & \SubvertThis\ForceName\SPat; \SubvertThis\ForceName\SEliz\\
%                     &                  & \cmd{\S}\meta{macro} & \SubvertThis\ForceName\SYamt\\\midrule
%                     &                  & Using \cmd{\ForceFN} \\
%       \cmd{\DropAffix}\cmd{\LPat} & \SubvertThis\DropAffix\LPat & \cmd{\FName}, \cmd{\S}\meta{macro} & \SubvertThis\ForceFN\SYamt\\\bottomrule
%   \end{tabular}
% \end{center}
% \fi
% \newpage
%
% These formatting features of \textsf{nameauth} can work with name control macros (Section~\ref{sec:NameControl}) in, for example, \textsf{beamer} overlays to define consistently the context and outcome of how names appear. There are two kinds of formatting at work that interact with each other:
% \begin{enumerate}\small
% \item \emph{Syntactic Formatting:} Reversing and caps normally occur only in the body text, not the index. Yet macros in name arguments affect both text and index.
% \item \emph{Name Post-Processing:} Hook macros apply formatting only to the printed form of a name after parsing. See also Section~\ref{sec:Hooksiii}.
% \end{enumerate}
%
% \DescribeMacro{\NamesFormat}
% Independent ``main-matter'' and ``front-matter'' systems are used to format first and subsequent name uses.
% \DescribeMacro{\MainNameHook}
% The main-matter system uses \cmd{\NamesFormat} to post-process first occurrences of names and \cmd{\MainNameHook} for subsequent uses.
% \DescribeMacro{\FrontNamesFormat}
% The front-matter system uses \cmd{\FrontNamesFormat} for first uses and
% \DescribeMacro{\FrontNameHook}
% \cmd{\FrontNameHook} for subsequent uses. The \texttt{alwaysformat} option\Version{2.5} causes only \cmd{\NamesFormat} and \cmd{\FrontNamesFormat} to be used (cf. Section~\ref{sec:NamePatterns}).\footnote{The names of these macros may seem poorly conceived. When starting work on this package, this author was ignorant of the breadth of how names might be handled. Designed to meet the needs of a master's thesis, this package has evolved to meet the needs of several published works. At one time, \cmd{\NamesFormat} was the only macro that did any formatting. The rest came later. A certain degree of cargo cult programming arose, to be corrected in the 3.0 series of \textsf{nameauth}.}\medskip
% 
% \DescribeMacro{\NamesActive}
% \cmd{\NamesInactive} and the \texttt{frontmatter} option make names use the front matter system. \cmd{\NamesActive} switches names to the main matter system.
% \DescribeMacro{\NamesInactive}\medskip
%
% These\Info{\cmd{\global}} two macros can be used explicitly as a pair or singly within an explicit local scope. Use \cmd{\global} to force a global effect.\medskip
%
% The two formatting systems are distinct, useful for front matter and main matter, text and footnotes, etc. We show this with different colors:
% \begin{quote}\small
% \StartNum
% \begin{verbatim}
%\colorlet{nared}{red!50!black}
%\colorlet{nagreen}{green!35!black}
%\colorlet{nablue}{blue!50!black}
%\colorlet{nabrown}{brown!55!black}
%\renewcommand*\FrontNamesFormat[1]{\color{nared}\sffamily #1}
%\renewcommand*\FrontNameHook[1]{\color{nagreen}\sffamily #1}
%\renewcommand*\NamesFormat[1]{\color{nablue}\sffamily #1}
%\renewcommand*\MainNameHook[1]{\color{nabrown}\sffamily #1}\end{verbatim}
% \end{quote}
%
% \begin{center}\small\MyStretch
% \NamesInactive
% \begin{tabular}{ll}\toprule
% Front-matter system: & \cmd{\NamesInactive}\\\midrule
% |\Name[Rudolph]{Carnap}| & \Name[Rudolph]{Carnap}\\
% \rowcolor{black!7!white}|\Name[Rudolph]{Carnap}| & \Name[Rudolph]{Carnap}\\
% |\Name[Nicolas]{Malebranche}| & \Name[Nicolas]{Malebranche}\\
% \rowcolor{black!7!white}|\Name[Nicolas]{Malebranche}| & \Name[Nicolas]{Malebranche}\\\bottomrule
% \end{tabular}\vfil
%
% \NamesActive
% \begin{tabular}{ll}\toprule
% Main-matter system: & \cmd{\NamesActive}\\\midrule
% |\Name[Rudolph]{Carnap}| & \Name[Rudolph]{Carnap}\\
% \rowcolor{black!7!white}|\Name[Rudolph]{Carnap}| & \Name[Rudolph]{Carnap}\\
% |\Name[Nicolas]{Malebranche}| & \Name[Nicolas]{Malebranche}\\
% \rowcolor{black!7!white}|\Name[Nicolas]{Malebranche}| & \Name[Nicolas]{Malebranche}\\\bottomrule
% \end{tabular}
% \end{center}
% \newpage
% 
% \DescribeMacro{\ForceName}
% Use this prefix macro to force ``first use'' formatting for the next \cmd{\Name}, etc. This will not force a full name reference like \cmd{\ForgetThis}.\Version{3.1} One must use the \texttt{formatAKA} option when using this with \cmd{\AKA}, etc. We show \cmd{\ForceName} in Sections~\ref{sec:NameControl}, \ref{sec:AKA}, and~\ref{sec:Hooksii}.\medskip
%
% Below\Info{\texttt{alwaysformat}} we simulate the \texttt{alwaysformat} option by manipulating the package internals. Using first-use hooks will not force full name references. This option made more sense when \cmd{\NamesFormat} was the only formatting hook.
% \makeatletter\@nameauth@AlwaysFormattrue\makeatother%
% \ForgetName[M.T.]{Cicero}\ForgetName{Elizabeth, I}%
% \begin{itemize}
% \item \NamesInactive Using \texttt{alwaysformat} in the front matter will produce: \Name[Albert]{Einstein}, then  \Name[Albert]{Einstein}; \Name{Confucius}, then \Name{Confucius}.
% \item \global\NamesActive Using \texttt{alwaysformat} in the main matter will produce: \Name[M.T.]{Cicero}[Marcus Tullius], then \Name[M.T.]{Cicero}[Marcus Tullius]; \Name{Elizabeth, I}, then \Name{Elizabeth, I}.
% \end{itemize}
% \makeatletter\@nameauth@AlwaysFormatfalse\makeatother
%
% The internal\Info{hook caveats} hook dispatcher calls the formatting hooks using the pattern \cmd{\bgroup}\meta{Hook}\texttt{\{\#1\}}\cmd{\egroup}. Thus one can use, e.g., \cmd{\itshape} in a local scope. One also can use macros that take one argument (cf. Section~\ref{sec:Hooksiii}), e.g., |\renewcommand*\NamesFormat{\sffamily\color{nablue}\textit}| will create\break the forms \bgroup\renewcommand*\NamesFormat{\sffamily\color{nablue}\textit}\ForgetThis\Einstein\ and \Einstein.\egroup\medskip
%
% The\Info{applied to\break footnotes} independent systems or ``species'' of names fit independent text elements, like front matter or even footnotes. Names in the body text, such as \Name[John Maynard]{Keynes}, also affect names in the footnotes.\footnote{We have \Name[John Maynard]{Keynes} from \cmd{\Name}\texttt{[John Maynard]\{Keynes\}} instead of \ForgetThis\Name[John Maynard]{Keynes}.}
% In footnote \arabic{footnote} \cmd{\MainNameHook} is called instead of \cmd{\NamesFormat} because \Name[John Maynard]{Keynes} already had occurred above.
%
% If we wanted to format names differently in the footnotes than in the body text, an easy way to do that is to use the front-matter system. For example:
%
% \makeatletter
% \let\@oldfntext\@makefntext
% \long\def\@makefntext#1{\NamesInactive\@oldfntext{#1}\NamesActive}
% \makeatother
% \begin{quote}\small
% \StartNum
% \begin{verbatim}
%\makeatletter
% \let\@oldfntext\@makefntext
% \long\def\@makefntext#1{\NamesInactive\@oldfntext{#1}\NamesActive}
%\makeatother\end{verbatim}
% \end{quote}
% 
% When we create another footnote, we see very different results.\footnote{We have \Name[John Maynard]{Keynes} from \cmd{\Name}\texttt{[John Maynard]\{Keynes\}}, then \Name[John Maynard]{Keynes}.}
% Footnote \arabic{footnote} shows a completely independent formatting. One also can synchronize the two systems with \cmd{\ForgetThis} and \cmd{\SubvertThis} (Section~\ref{sec:NameDecisions} and its subsections).
% 
% To finish this example, we change footnotes back to normal:
% \begin{quote}\small
% \ContinueNum
% \begin{verbatim}
%\makeatletter
%\let\@makefntext\@oldfntext
%\makeatother\end{verbatim}
% \end{quote}
% \makeatletter\let\@makefntext\@oldfntext\makeatother
% 
% Of cource, one can force long and short forms as needed (Section~\ref{sec:NameControl}). Yet the main point of \textsf{nameauth} is to do the complex work once, then use that in automated fashion for the rest of the document.
% 
% \ReturnLink
% \newpage
%
% \subsection{Alternate Formatting}
% \label{sec:AltFormat}
% \begingroup\AltFormatActive
%
% \noindent The formatting hooks only affect names in the body text. Continental formatting occurs in both the text and in the index. One needs to format those names with macros in the name arguments. The basic way formats names in both text and index. The advanced way allows changes in the text, but keeps the index consistent.
%
% \subsubsection{Basic Features}
% \label{sec:AltBasic}
%
%Section~\ref{sec:ErrorProt} showed us that changing a control sequence will change the index entry of a name, even if one cannot see differences on the page. Alternate formatting helps one avert spurious index entries.
%
% Using,\Warn{} e.g., \cmd{\Name\{}\cmd{\textsc\{a Name, Problem\}\}} will halt \LaTeX\ because the comma tries to break \cmd{\textsc} and its argument into two elements. We fix that with: \cmd{\Name\{}\cmd{\textsc\{a Name\},} \cmd{\textsc\{Problem\}\}}. Yet \cmd{\CapThis} still needs alternate formatting, given that \cmd{\textsc} is robust (Section~\ref{sec:AltAdvanced}).\footnote{Pre-version 3.1 methods of Continental formatting should work if one uses the \texttt{altformat} option or \cmd{\AltFormatActive} to protect against the default behavior of \cmd{\CapThis}.}\medskip
%
% \DescribeMacro{\AltFormatActive}
% Both the \texttt{altformat} option and \cmd{\AltFormatActive} enable and activate alternate formatting. Both cause \cmd{\CapThis} to work via \cmd{\AltCaps} instead of the normal way. \cmd{\AltFormatActive} countermands \cmd{\AltFormatActive*}.
% \begin{itemize}
% \item \emph{Enabled} means that the alternate formatting mechanism inhibits the normal behavior of \cmd{\CapThis}.
% \item \emph{Activated} means that \cmd{\textSC} and other alternate formatting macros (see below) format their arguments. When deactivated, they do not format their arguments.
% \end{itemize}
%
% At\Version{3.1} the start of this section we used \cmd{\AltFormatActive} to enable alternate formatting and ``switch on'' the alternate formatting macros. That is the basic set of conditions for the simple use of alternate formatting. \medskip
%
% \DescribeMacro{\AltFormatActive*}
% The starred form \cmd{\AltFormatActive*} enables alternate formatting but deactivates the special formatting macros, preventing them from changing their arguments. It countermands both the \texttt{altformat} option and \cmd{\AltFormatActive}. It causes \cmd{\CapThis} only to work via \cmd{\AltCaps}.\medskip
%
% \DescribeMacro{\AltFormatInactive}
% To both disable alternate formatting and deactivate the alternate formatting macros, use \cmd{\AltFormatInactive}. This reverts globally to standard formatting and the normal function of \cmd{\CapThis}.
%
% \begin{center}
% \begin{tabular}{lcc}
%                          & Enabled & Activated\\\midrule
% \cmd{\AltFormatActive}   & \YES    & \YES\\
% \cmd{\AltFormatActive*}  & \YES    & \NO\\
% \cmd{\AltFormatInactive} & \NO     & \NO\\\bottomrule
% \end{tabular}
% \end{center}
%
% On the next page we describe the formatting macros that are built in to \textsf{nameauth} in order to use the basic features of alternate formatting and provide a foundation for the advanced features. One should use \cmd{\PretagName} (Section~\ref{sec:IndexSort}) to sort the related index entries for these names.
% \newpage
%
% \DescribeMacro{\textSC}
% Continental formatting can be as simple as using the short macro \cmd{\textSC}. Three other macros also implement alternate formatting.
% \DescribeMacro{\textIT}
% These macros make changes only when alternate formatting is active.
% \DescribeMacro{\textBF}
% We sort the index entry with \cmd{\PretagName} and demonstrate the formatting.
% \DescribeMacro{\textUC}
% \begin{quote}\small
% \StartNum
% \begin{verbatim}
%\PretagName[Greta]{\textSC{Garbo}}{Garbo, Greta}
%\PretagName[Ada]{\textIT{Lovelace}}{Lovelace, Ada}
%\PretagName[Charles]{\textBF{Babbage}}{Babbage, Charles}
%\PretagName{\textUC{Tokugawa}, Ieyasu}{Tokugawa Ieyasu}\end{verbatim}
%
% \smallskip\MyStretch
% |\Name[Greta]{\textSC{Garbo}}|\dotfill \Name[Greta]{\textSC{Garbo}}; \Name[Greta]{\textSC{Garbo}}\\
% |\Name[Ada]{\textIT{Lovelace}}|\dotfill \Name[Ada]{\textIT{Lovelace}}; \Name[Ada]{\textIT{Lovelace}}\\
% |\Name[Charles]{\textBF{Babbage}}|\dotfill \Name[Charles]{\textBF{Babbage}}; \Name[Charles]{\textBF{Babbage}}\\
% |\Name{\textUC{Tokugawa}, Ieyasu}|\dotfill \Name{\textUC{Tokugawa}, Ieyasu}; \Name{\textUC{Tokugawa}, Ieyasu}
% \end{quote}
%
% Since we switch to Latin Modern Sans in the formatting hooks, the switch to small caps in \Name[Greta]{\textSC{Garbo}} forces a substitution to Latin Modern Roman. This action varies with the font being used.
% 
%  Using basic alternate formatting, these macros \emph{always format their arguments} with the \texttt{altformat} option or \cmd{\AltFormatActive}. Likewise, they \emph{never format their arguments} when \cmd{\AltFormatActive*} is used. To change the formatting of the name arguments, one must use the advanced features. Whenever\Warn{} a naming macro writes to the index, the formatting macros must be in the same Boolean state to avoid spurious index entries. The next section explains more.
%
% As with normal formatting, \cmd{\CapName} interacts with alternate formatting only in the text. Thus \CapName\Name*[Greta]{\textSC{Garbo}} instead of \Name[Greta]{\textSC{Garbo}}. \cmd{\RevComma} likewise gives \RevComma\Name*[Ada]{\textIT{Lovelace}}. \cmd{\RevName} produces \RevName\Name*{\textUC{Tokugawa}, Ieyasu}.\medskip
%
% A comma\Info{comma karma} delimiter splits the mandatory macro argument into a root and an affix. To avoid errors, format the name and suffix separately.
% \begin{quote}\small
% \StartNum
% \begin{verbatim}
%\PretagName[John David]{\textSC{Rockefeller},\textSC{III}}
%  {Rockefeller, John David 3}
%\PretagName{\textUC{Fukuyama}, Takeshi}{Fukuyama Takeshi}
%\begin{nameauth}
%  \< JRIII & John David & \textSC{Rockefeller},\textSC{III} & >
%  \< Fukuyama  & & \textUC{Fukuyama}, Takeshi &         >
%  \< OFukuyama & & \textUC{Fukuyama}          & Takeshi >
%\end{nameauth}\end{verbatim}
% \end{quote}
%
% From above we get \JRIII, then \JRIII. For non-Western names, the new syntax and the older syntax produce the same control sequence that identifies names. Again we are careful to avoid putting the comma delimiter within a container macro. 
% \begin{center}\small\MyStretch
% \begin{tabular}{rl}\toprule
% |\Fukuyama| & \Fukuyama\\
% \rowcolor{black!7!white}|\OFukuyama| & \OFukuyama\\
% |\LOFukuyama| & \LOFukuyama\\
% \rowcolor{black!7!white}|\Fukuyama| & \Fukuyama\\\bottomrule
% \end{tabular}
% \end{center}
%
% Only the new syntax allows one to use alternate names in the text (Section~\ref{sec:FName}). For example, ``|\LFukuyama[Sensei]| \LFukuyama[Sensei] wrote \textit{Nihon Fukuin R\=uteru Ky\=okai Shi} in 1954, after studying in the US in the 1930s.''
% \newpage
%
% \subsubsection{Advanced Features}
% \label{sec:AltAdvanced}
%
% \noindent A more complex version of alternate formatting allows us to make formatting and other changes in the text while keeping the index consistent. In order to do this, we will be using \cmd{\textSC}, \cmd{\textIT}, \cmd{\textBF}, and \cmd{\textUC} with \cmd{\noexpand} and special triggering macros. Below we briefly see the difference:
% \begin{center}\small\MyStretch
% \begin{tabular}{ll}
% |\Name[Martin]{\textSC{Luther}}| & {\color{nared}|%| \textit{basic alternate formatting}}\\
% |\Name[Martin]{\noexpand\textSC{Luther}}| & {\color{nared}|%| \textit{advanced version}}\\
% \end{tabular}
% \end{center}
% The reason for this approach is that indexing operations occur outside the formatting hooks, never within the hooks, and \cmd{\noexpand} keeps the two separate.
% 
% \ifDoTikZ
% \begin{tcolorbox}[colback=white,colframe=nared]
% \centering Using \cmd{\noexpand} is key to consistent index entries.
% \end{tcolorbox}\bigskip
% \else
% \begin{center}\bfseries Using \cmd{\noexpand} is key to consistent index entries.\end{center}
% \fi
%
% \cmd{\CapThis}\DescribeMacro{\AltCaps}\ causes \cmd{\AltCaps} to cap its argument only in a formatting hook. It is enabled whenever alternate formatting is enabled. \cmd{\AltCaps} works independently of \cmd{\AltOn} and \cmd{\AltOff}:
% \begin{quote}
%   \fbox{\mystrut\ \cmd{\noexpand}\cmd{\AltCaps}\marg{Arg} }
% \end{quote}
% In the example below we redefine \cmd{\MainNameHook} to suppress formatting:
% \renewcommand*\MainNameHook{\color{nabrown}\sffamily\AltOff}
% \begin{quote}\small
% \StartNum
% \begin{verbatim}
%\renewcommand*\MainNameHook%
%  {\color{nabrown}\sffamily\AltOff}% we match the manual
%
%\IndexInactive
%What's in a \Name{\noexpand\AltCaps{a} Name}?
%\CapThis\Name{\noexpand\AltCaps{a} Name} smells not,
%but a rose does. We avoid \Name{
%  \noexpand\textSC{\noexpand\AltCaps{a} Name},
%  \noexpand\textSC{Problem}}.
%\CapThis\Name*{
%  \noexpand\textSC{\noexpand\AltCaps{a} Name},
%  \noexpand\textSC{Problem}} will not occur,
% even if it smells like a rose.\end{verbatim}
%
% \smallskip\IndexInactive
% What's in a \Name{\noexpand\AltCaps{a} Name}?
% \CapThis\Name{\noexpand\AltCaps{a} Name} smells not,
% but a rose does. We avoid \Name{
%   \noexpand\textSC{\noexpand\AltCaps{a} Name},
%   \noexpand\textSC{Problem}}.
% \CapThis\Name*{
%   \noexpand\textSC{\noexpand\AltCaps{a} Name},
%   \noexpand\textSC{Problem}} will not occur,
% even if it smells like a rose.
% \end{quote}
%
% Like\DescribeMacro{\AltOff}\ a manual automobile clutch and gearbox, \cmd{\AltOff} deactivates \cmd{\textSC}, \cmd{\textBF}, \cmd{\textIT}, and \cmd{\textUC} only in a formatting hook.\medskip
%
% \cmd{\AltOn}\DescribeMacro{\AltOn}\ activates \cmd{\textSC}, \cmd{\textBF}, \cmd{\textIT}, and \cmd{\textUC} only in a formatting hook. To summarize:
% 
% \begin{itemize}
%  \item \cmd{\AltFormatActive} and \cmd{\AltFormatActive*} set global states.
%  \item \cmd{\AltFormatActive} causes formatting in the text and index, as well as forcing the use of \cmd{\AltCaps}.
%  \item With \cmd{\AltFormatActive*} inhibits formatting, but still requires one to use \cmd{\AltCaps}.
%  \item \cmd{\AltOn} and \cmd{\AltOff} change local state only in the formatting hooks.
%  \item The user adds \cmd{\AltOn} and \cmd{\AltOff} to the hooks as needed.
%  \item The actual formatting happens via macros in the name arguments.
% \end{itemize}
% \newpage
% 
% Keeping the \cmd{\MainNameHook} example above, we have:
% \begin{quote}\small
% \StartNum
% \begin{verbatim}
%\begin{nameauth}
%  \< Luth & Martin & \noexpand\textSC{Luther} & >
%\end{nameauth}
%\PretagName[Martin]{\noexpand\textSC{Luther}}{Luther, Martin}\end{verbatim}
% \end{quote}
%
% We first mention \cmd{\Luth} \Luth. Then again, \cmd{\Luth} \Luth. Medieval Italian differs from modern Italian with respect to particles. Below the index entry should be ``\ShowIdxPageref*[Catherine \noexpand\AltCaps{d}e']{\noexpand\textSC{Medici}}\,'' instead of ``\ShowIdxPageref*[Catherine]{de~\textSC{Medici}}'':
% \begin{quote}\small
% \StartNum
% \begin{verbatim}
%\begin{nameauth}
%  \< Cath & Catherine \noexpand\AltCaps{d}e'
%          & \noexpand\textSC{Medici} & >
%\end{nameauth}
%\PretagName[Catherine \noexpand\AltCaps{d}e']
%            {\noexpand\textSC{Medici}}{Medici, Catherine de}\end{verbatim}
% \end{quote}
%This gives us \Cath\ and \Cath. To get \ForceName\CapThis\LCath[\noexpand\AltCaps{d}e'] and \CapThis\LCath[\noexpand\AltCaps{d}e'] in the text, use |\CapThis\LCath[\noexpand\AltCaps{d}e']|.\medskip
%
% \phantomsection
% \label{page:Inflections}
% We\Info{name inflections\break\dbend} can use alternate formatting for grammatical inflections (cf. Section~\ref{sec:Hooksiii}). We tell the same set of lies that we did on page~\pageref{page:Sobriquets}. \cmd{\DoGentrue} occurs only in the formatting hook, thereby keeping the index entries consistent:\footnote{A copy of this example is in \texttt{examples.tex}, collocated with this manual.}
%
% \newif\ifGenitive
% \newif\ifDoGen
% \renewcommand*\NamesFormat[1]{\ifGenitive\DoGentrue\fi#1\global\Genitivefalse}
% \renewcommand*\MainNameHook[1]{\ifGenitive\DoGentrue\fi\AltOff#1\global\Genitivefalse}
% \newcommand\JEFF{\ifDoGen\textSC{Jefferson's}\else\textSC{Jefferson}\fi}
% \begin{quote}\small
% \StartNum
% \begin{verbatim}
%\newif\ifGenitive
%\newif\ifDoGen
%\renewcommand*\NamesFormat[1]
%  {\ifGenitive\DoGentrue\fi#1\global\Genitivefalse}
%\renewcommand*\MainNameHook[1]
%  {\ifGenitive\DoGentrue\fi\AltOff#1\global\Genitivefalse}
%\begin{nameauth}
%  \< Jeff & Thomas & \noexpand\JEFF & >
%\end{nameauth}
%\PretagName[Thomas]{\noexpand\JEFF}{Jefferson, Thomas}
%\TagName[Thomas]{\noexpand\JEFF}{, pres.|hyperpage}
%\newcommand\JEFF{\ifDoGen\textSC{Jefferson's}\else
%  \textSC{Jefferson}\fi}
%
%Consider \Genitivetrue\Jeff\ legacy. More on \Jeff\ later.
%\Genitivetrue\Jeff\ reputation has declined in recent decades.\end{verbatim}
%
% \smallskip
% Consider \Genitivetrue\Jeff\ legacy. More on \Jeff\ later.
% \Genitivetrue\Jeff\ reputation has declined in recent decades.
% \end{quote}
% 
% For highly inflected languages, this would require two Boolean flags per case and nested conditional statements. Now we resume normal formatting with \cmd{\AltFormatInactive} and we do not use the names in this section outside of it.\footnote{In a \texttt{dtx} file it is best to put the \texttt{nameauth} environment, \cmd{\PretagName}, and \cmd{\TagName} macros in the driver section, especially when names contain macros.}
% \AltFormatInactive\endgroup
% 
% \ReturnLink
% \newpage
%
% \subsection{Indexing Macros}
%
% \subsubsection[Entries \& Control]{Index Entries and Control}
% \label{sec:IndexControl}
%
% \DescribeMacro{\IndexName}
% Both package users and the naming macros themselves use this macro to create index entries. It prints nothing in the body text:
% \begin{quote}
%   \fbox{\mystrut\ \cmd{\IndexName}\oarg{FNN}\marg{SNN|,| Affix}\oarg{Alternate} }
% \end{quote}
%
% If \meta{FNN} is present, it ignores \meta{Alternate} for Western and ``native'' Eastern name forms. If \meta{FNN} is absent, \cmd{\IndexName} may use the current or obsolete non-Western syntax (Section~\ref{sec:Obsolete}). Indexing follows [\hyperlink{Mulvany}{Mulvany}, 152--82].
%
% If \cmd{\IndexInactive} or the \texttt{noindex} option are used, this macro does nothing until \cmd{\IndexActive} appears. Additionally, it will not create index entries for cross-references made by \cmd{\IndexRef} and \cmd{\AKA}. It will not index names excluded by \cmd{\ExcludeName}. This provides some error protection for professional indexing.
%
% \cmd{\IndexName} and \cmd{\IndexRef} call \cmd{\@nameauth@Index}, a macro that assembles an index entry from the sort tag (Section~\ref{sec:IndexXref}), name arguments, and index tag (Section~\ref{sec:IndexTag}).
% Different standards exist for index entries and cross-references. Check with your publisher, style guide, and docs for \textsf{xindy} and \textsf{makeindex}.\medskip
%
% \DescribeMacro{\IndexActive}
% The \texttt{noindex} option deactivates the indexing function of this package until \cmd{\IndexActive} enables indexing.
% \DescribeMacro{\IndexInactive}
% Another macro, \cmd{\IndexInactive}, will deactivate indexing again. These can be used throughout the document. {\bfseries \cmd{\IndexInactive} suppresses index sorting and tagging macros.} Compare the use of macros \cmd{\ExcludeName} and \cmd{\IncludeName} (Section~\ref{sec:IndexXref}).\medskip
%
% \DescribeMacro{\IndexProtect}
% Both the core name engine \cmd{\@nameauth@Name} and \cmd{\AKA} have locks that prevent them from being re-entrant.\VersionWarn{3.3} This protects the text. Usually, one does not put naming macros in the index. Just in case, now one can use \cmd{\IndexProtect} right before \cmd{\printindex} to prevent \textsf{nameauth} macros from producing any output.
%
% This example shows the difference between the effects of the older and newer approaches. We use the tag \S\ in this manual's index, but not below:
% \begin{center}\small
%   \begin{tabular}{llll}\toprule
%     \bfseries Macro & \bfseries Text & \bfseries \texttt{.ind} file & \bfseries Index\\\midrule
%     no protection\\
%     |\Name{foo\Name{bar}}| & \Name{foo\Name{bar}} & |\item foo\Name {bar}| & foo{\NamesFormat bar}\\
%     (next iteration adds) \(\rightarrow\)  &      & |\item bar|            & bar\\
%     \rowcolor{black!7!white}\cmd{\IndexProtect} & & & \\
%     \rowcolor{black!7!white}|\Name{foo\Name{bar}}| & \Name{foo\Name{bar}} &
%       |\item foo\Name {bar}| & foo\\
%     \rowcolor{black!7!white}(no further output results) & & & \\\bottomrule
%   \end{tabular}
% \end{center}
%
% \cmd{\IndexActive}\Info{\cmd{\global}} and \cmd{\IndexInactive} can be used as a pair or singly within a group. These macros override any prefix macros. \cmd{\IndexProtect} also can be used in a local scope. Use \cmd{\global} with these macros to force a global effect.\medskip
%
% \DescribeMacro{\SkipIndex}
% The prefix macro \cmd{\SkipIndex} will suppress indexing for just one instance of a naming or cross-referencing macro.\VersionWarn{3.1} It will not alter name forms or formatting. For example, |\SkipIndex\Name[Monty]{Python}| produces \SkipIndex\Name[Monty]{Python} in the text with no index entry. The same thing again yields \SkipIndex\Name[Monty]{Python}. Since prefix macros are meant for macros that print a name, both \cmd{\IndexName} and \cmd{\IndexRef} ignore \cmd{\SkipIndex} and allow the Boolean flags set by the prefix macros, to ``pass through'' to the next naming macro. That may seem counter-intuitive.
% \newpage
%
% \DescribeMacro{\JustIndex}
% This prefix macro makes \cmd{\Name}, \cmd{\Name*}, \cmd{\Fname}, and the shorthands act like a one-time call to \cmd{\IndexName}.\VersionWarn{3.3} Flags set by the prefix macros ``pass through'' to the next naming macro except these three: \cmd{\@nameauth@JustIndexfalse} (obviously), but also \cmd{\@nameauth@FullNamefalse} and \cmd{\@nameauth@FirstNamefalse}.
% \begin{itemize}
% \item Both \cmd{\AKA} and \cmd{\PName} ignore and reset the flag set by \cmd{\JustIndex}.
% \item \cmd{\SkipIndex} \cmd{\JustIndex} \cmd{\Name\{A\}} \cmd{\Name\{B\}} is just like \cmd{\JustIndex} \cmd{\Name\{A\}} \cmd{\SkipIndex} \cmd{\Name\{B\}}. See the table on page~\pageref{page:Priorities}.
% \item  Version 3.3 eliminates the undocumented behavior that used to occur when not using, e.g., \cmd{\JustIndex}\cmd{\Wash}. Now any version will do:\smallskip\\
% \bgroup\small\begin{tabular}{ll@{ }ll@{ }l}
%   |\JustIndex\LWash \Wash| & old: & \makeatletter\@nameauth@OldPasstrue\makeatother\JustIndex\LWash \Wash & new: & \Wash\\
%   |\JustIndex\SWash \Wash| & old: & \makeatletter\@nameauth@OldPasstrue\makeatother\JustIndex\SWash \Wash & new: & \Wash\\
% \end{tabular}\egroup 
% \item The \texttt{oldpass} option restores the old behavior. Cf. Section~\ref{sec:Customize}.
% \end{itemize}
%
% \ReturnLink
%
% \subsubsection{Cross-References}
% \label{sec:IndexXref}
%
% \DescribeMacro{\IndexRef}
% This macro emerged from the macros in Section~\ref{sec:AKA}. By default, \cmd{\IndexRef} creates a \textit{see} reference\Version{3.0} from the name defined by its first three arguments to the target in its final argument:
% \begin{quote}\small
%   \fbox{\mystrut\ \cmd{\IndexRef}\oarg{FNN}\marg{SNN|,| Affix}\oarg{Alternate}\marg{reference target} }
% \end{quote}
%
% The name parsing is like \cmd{\IndexName}, except that the final argument is neither parsed nor checked if a target entry exists. For example, to cross-reference ``Sun King'' with \Name*{Louis, XIV} use: \cmd{\IndexRef}\texttt{\{Sun King\}\{Louis XIV\}}\IndexRef{Sun King}{Louis XIV}.
%
% When\Warn{} \cmd{\IndexRef} calls \cmd{\@nameauth@Index}, a preexisting tag of the form \meta{some text}\texttt{\textbar}\meta{some macro} is reduced to \meta{some text}. One cannot tag an extant cross-reference, but one can tag a name, then later create a \textit{see also} reference. For related warnings activated by the \texttt{verbose} option, see Section~\ref{sec:ErrorProt}.
%
% Next we look at variant names and cross-references. Some can be handled with the \meta{Alternate} argument. Others require more work to implement (Section~\ref{sec:VarNames}).
% \begin{itemize}
% \item Variant names potentially can have page numbers in index entries. Cross-references cannot have page numbers.
% \item |\DropAffix\ForgetThis\Name[J.E.]{Carter, Jr.}[Jimmy]| gives a variant name: \DropAffix\ForgetThis\Name[J.E.]{Carter, Jr.}[Jimmy] indexed under ``\ShowIdxPageref*[J.E.]{Carter, Jr.}''
% \item |\IndexRef[Jimmy]{Carter}{Carter, J.E., Jr.}| makes an xref but prints nothing. We need only create this cross-reference once.\IndexRef[Jimmy]{Carter}{Carter, J.E., Jr.}
% \item By contrast, \cmd{\AKA} automatically formats the cross-reference name in the text and in the index.
% \item Yet \cmd{\AKA} has limited formatting. Instead, after creating the xref with \cmd{\IndexRef}, one can use |\SubvertThis\Name*[Jimmy]{Carter}| \SubvertThis\Name*[Jimmy]{Carter} with full formatting, but without creating any page entries.
% \item \cmd{\SubvertThis} syncs the variant with the canonical form \cmd{\DropAffix} \cmd{\Name*[J.E.]\{Carter, Jr.\}[Jimmy]} \DropAffix\Name*[J.E.]{Carter, Jr.}[Jimmy]. Otherwise, they would act as different names. See also Section~\ref{sec:NameTests}.
% \item If we use |\Name[Jimmy]{Carter}| \Name[Jimmy]{Carter} we have to index this alternate name with the canonical one: |\IndexName[J.E.]{Carter, Jr.}|
% \end{itemize}
%
% \DescribeMacro{\SeeAlso}
% Put \cmd{\SeeAlso} before \cmd{\IndexRef}, \cmd{\AKA}, and \cmd{\PName} to make a \textit{see also} reference for a name that has appeared already in the index.\Version{3.0}
% Yet one should mind the caveats:
% \begin{itemize}
% \item If |\SeeAlso\IndexRef{Bar}{Foo}| occurs on page 10, \cmd{\Name\{Bar\}} will not create index entries thereafter. A \textit{see also} ref follows all page refs. 
% \item If |\SeeAlso\IndexRef{Bar}{Foo}| occurs on page 10, \cmd{\Name\{Foo\}} will create index entries thereafter because it is the target of ``Bar.'' 
% \item If |\Name{Baz}| occurs on page 12 and |\IndexRef{Baz}{Meschugge}| on page 16, no xref will be created. A \textit{see} reference has no page refs.
% \end{itemize}
%
% \DescribeMacro{\ExcludeName}
% This macro prevents a name from being used as either an index entry or as an index cross-reference.\Version{3.0} It will not exclude extant cross-references:
% \begin{quote}
%   \fbox{\mystrut\ \cmd{\ExcludeName}\oarg{FNN}\marg{SNN|,| Affix}\oarg{Alternate} }
% \end{quote}
% \cmd{\IndexRef} works best if one needs a cross-reference from a variant to the canonical name. If no cross-reference is needed, then \cmd{\ExcludeName} is used. Unlike \cmd{\IndexInactive} and \cmd{\IndexActive}, this macro works only on a per-name basis. Below we keep specific names and cross-references out of the index:\ExcludeName[Kris]{Kringle}\ExcludeName[Santa]{Claus}\ExcludeName{Grinch}\vspace{2.5ex}
%
%\leavevmode\quad\begin{minipage}[b]{0.9\textwidth}\small
% \StartNum
% \begin{verbatim}
%\ExcludeName[Kris]{Kringle}
%\ExcludeName[Santa]{Claus}
%\ExcludeName{Grinch}\end{verbatim}
% \end{minipage}\medskip
%
%\leavevmode\quad\begin{minipage}[b]{0.5\textwidth}\small
% \ContinueNum
% \begin{verbatim}
%\Name[Kris]{Kringle}
%\Name[Kris]{Kringle}
%\AKA[Kris]{Kringle}[Santa]{Claus}\end{verbatim}
% \end{minipage}
% \begin{minipage}[b]{0.3\textwidth}\small
% \Name[Kris]{Kringle}\\
% \Name[Kris]{Kringle}\\
% \AKA[Kris]{Kringle}[Santa]{Claus}
% \end{minipage}\vspace{2ex}
%
% For more examples of using \cmd{\ExcludeName} to handle variants, see Sections~\ref{sec:VarNames} and \ref{sec:NameParticles}, among others. We will check on the \Name{Grinch} later.\medskip
%
% \phantomsection
% \label{page:ExPage}
% \DescribeMacro{\IncludeName}
% For those who might need to break the indexing rules set by \textsf{nameauth}, these two macros get the job done.
% \DescribeMacro{\IncludeName*}
% They remove the protections used for exclusion and cross-referencing. These macros have the same syntax as \cmd{\ExcludeName}:\Version{3.0}
% \begin{quote}
% \fbox{\vbox{\hbox{\mystrut\ \cmd{\IncludeName\ }\oarg{FNN}\marg{SNN|,| Affix}\oarg{Alternate} }\par
% \hbox{\mystrut\ \cmd{\IncludeName*}\oarg{FNN}\marg{SNN|,| Affix}\oarg{Alternate} }}}
% \end{quote}
%
% \cmd{\IncludeName} only removes an excluded reference created by \cmd{\ExcludeName} while \cmd{\IncludeName*} completely un-protects a cross-reference. Thereafter, one may create page entries for it like a name.
%
% For example, we used |\ExcludeName{Attila, the Hun}| at the end of Section~\ref{sec:SimpleStart}. Using \cmd{\IfAKA\{Attila, the Hun\}}|{|\meta{an xref}|}{|\meta{no xref}|}{|\meta{excluded}|}| tells us that he is \IfAKA{Attila, the Hun}{\meta{an xref}}{\meta{no xref}}{\meta{excluded}} (cf. Section~\ref{sec:NameTests}).
% 
% Once we use |\IncludeName{Attila, the Hun}|\IncludeName{Attila, the Hun}, using |\LAttil| \LAttil\ will create a name and an index entry on this page. \cmd{\IfAKA} now tells us that he is \IfAKA{Attila, the Hun}{\meta{an xref}}{\meta{no xref}}{\meta{excluded}}. We again have a name that can be indexed.
%
% Cross-references get more protection. \cmd{\IfAKA[Jay]\{Rockefeller\}} (a reference from Section~\ref{sec:SimpleStart}) tells us that he is \IfAKA[Jay]{Rockefeller}{\meta{an xref}}{\meta{no xref}}{}. If we follow the previous example and use |\IncludeName[Jay]{Rockefeller}|\IncludeName[Jay]{Rockefeller} he still is \IfAKA[Jay]{Rockefeller}{\meta{an xref}}{\meta{no xref}}{}. After using |\IncludeName*[Jay]{Rockefeller}|\IncludeName*[Jay]{Rockefeller} he finally becomes \IfAKA[Jay]{Rockefeller}{\meta{an xref}}{\meta{no xref}}{}, removing all protection from that cross-reference.
%
% \phantomsection
% \label{page:ManualXref}
% \begin{center}\bfseries Advanced Cross-Referencing\end{center}
%
% \noindent \cmd{\IndexRef}\Info{combining xrefs} will not merge multiple cross-references. One must manually merge cross-references: |\IndexRef{Bar}{Baz; Foo}| makes the index entry ``Bar, \textit{see} Baz; Foo.'' The preferred standard (in the humanities) suggests that one avoid something like |\IndexRef{Bar}{Baz} \IndexRef{Bar}{Foo}|.\medskip
%
% There\Info{one xref\break many targets} is a special case where one cross-reference can point to multiple targets, such as demonstrated in the example below:
% \begin{quote}\small
% \StartNum
% \begin{verbatim}
%\PretagName{\textit{Snellius}}{Snellius}
%\IndexRef{\textit{Snellius}}{Snel van Royen, R.; Snel van Royen, W.}
%
%Both \Name[W.]{Snel van Royen}[Willebrord] and
%his son \Name[R.]{Snel van Royen}[Rudolph] were known
%by the Latin moniker \Name{\textit{Snellius}}.\end{verbatim}
%
% \smallskip
% \IndexRef{\textit{Snellius}}{Snel van Royen, R.; Snel van Royen, W.}
% Both \Name[W.]{Snel van Royen}[Willebrord] and
% his son \Name[R.]{Snel van Royen}[Rudolph] were known
% by the Latin moniker \Name{\textit{Snellius}}.
% \end{quote}
%
% \cmd{\IndexRef}\Info{location matters} prevents page numbers in cross-references, so one must plan how to set up complex cross-references. Above, |\Name{\textit{Snellius}}| produces no index entry because \cmd{\IndexRef} comes first.\medskip
%
% Below,\Info{Multiple\break connections} two names are indexed with page numbers. They have \textit{see also} cross-references to each other. One of those names also has a \textit{see} reference to it:
% \begin{itemize}
% \item We use the canonical name to set up page references:\smallskip\\
% \hbox{}\qquad|\Name{Maimonides}|\dotfill\Name{Maimonides}
% \item \Name{Maimonides} has two other names, one more used than the other. We set up his least-used name as the \textit{see} reference:\smallskip\\
% \hbox{}\qquad|\IndexRef{Moses, ben-Maimon}{Maimonides}|\IndexRef{Moses, ben-Maimon}{Maimonides}\\
% \hbox{}\qquad|\Name{Moses, ben-Maimon}|\dotfill\Name{Moses, ben-Maimon}
% \item We now have a main name with a page entry and a ``\textit{see} reference'' to that name. \Name*{Moses, ben-Maimon} has no page entries because we made the xref before we started to use the name.
% \item Before creating \textit{see also} cross-references, we use the other alternate name so that all the page entries precede the cross-references:\smallskip\\
% \hbox{}\qquad|\Name{Rambam}|\dotfill\Name{Rambam}
% \item All \textit{see also} references must come after all page references. For example, one could put both of these macros at the end of the document:\smallskip\\
% \hbox{}\qquad|\SeeAlso\IndexRef{Maimonides}{Rambam}|\SeeAlso\IndexRef{Maimonides}{Rambam}\\
% \hbox{}\qquad|\SeeAlso\IndexRef{Rambam}{Maimonides}|\SeeAlso\IndexRef{Rambam}{Maimonides}
% \end{itemize}
% 
% \BigBlank
% \newpage
% 
% \begin{center}\bfseries Continental Format Reference Work\end{center}
%
% Let us create a macro for entries in a reference work using the basic form of Continental formatting from Section~\ref{sec:AltBasic}. We enable alternate formatting, set up tags, and define an article with head-words:
% \AltFormatActive
%\newcommand{\RefArticle}[4]{%^^A
%  \def\check{#2}%^^A
%  \ifx\check\empty
%    \noindent\ForgetThis#1\ {#4}
%  \else
%    \noindent\ForceName#1\ ``\ForceName#2''
%    \ForceName#3\ {#4}
%  \fi
%}
% \begin{quote}\small
% \StartNum
% \begin{verbatim}
%\AltFormatActive
%\PretagName[Greta]{\textSC{Garbo}}{Garbo, Greta}
%\PretagName[Heinz]{\textSC{Rühmann}}{Ruehmann, Heinz}
%\PretagName[Heinrich Wilhelm]{\textSC{Rühmann}}%
%  {Ruehmann, Heinrich Wilhelm}
%
%\newcommand{\RefArticle}[4]{%
%  \def\check{#2}%
%  \ifx\check\empty
%    \noindent\ForgetThis#1\ {#4}
%  \else
%    \noindent\ForceName#1\ ``\ForceName#2''
%    \ForceName#3\ {#4}
%  \fi
%}\end{verbatim}
% \end{quote}
%
% \cmd{\RefArticle} either formats the name from the first argument and appends the fourth argument, ignoring the others if the second is empty, or it formats the first three arguments and appends the fourth. We determine what those arguments mean by including specific naming macros.
% \begin{quote}\small
% \ContinueNum
% \begin{verbatim}
%\RefArticle%
%  {\Name[Greta]{\textSC{Garbo}}}%
%  {}{}%
%  {(18 September 1905\,--\,15 April 1990) was a Swedish
%   film actress during the 1920s and 1930s.}
%
%\RefArticle%
%  {%
%    \IndexRef[Heinrich Wilhelm]{\textSC{Rühmann}}%
%      {\textSC{Rühmann}, Heinz}%
%    \SubvertThis\FName[Heinrich Wilhelm]{\textSC{Rühmann}}%
%  }%
%  {\SubvertThis\FName[Heinz]{\textSC{Rühmann}}}%
%  {\Name[Heinz]{\textSC{Rühmann}}}%
%  {(7 March 1902\,--\,3 October 1994) was a German actor
%   in over 100 films.}
%   
%\AltFormatInactive\end{verbatim}
% \end{quote}
%
% \begin{quote}
%\RefArticle%^^A
%  {\Name[Greta]{\textSC{Garbo}}}%^^A
%  {}{}%^^A
%  {(18 September 1905\,--\,15 April 1990) was a Swedish
%   film actress during the 1920s and 1930s.}
%
%\RefArticle%^^A
%  {%^^A
%    \IndexRef[Heinrich Wilhelm]{\textSC{Rühmann}}%^^A
%      {\textSC{Rühmann}, Heinz}%^^A
%    \SubvertThis\FName[Heinrich Wilhelm]{\textSC{Rühmann}}%^^A
%  }%^^A
%  {\SubvertThis\FName[Heinz]{\textSC{Rühmann}}}%^^A
%  {\Name[Heinz]{\textSC{Rühmann}}}%^^A
%  {(7 March 1902\,--\,3 October 1994) was a German actor
%   in over 100 films.}
% \end{quote}
%
% \AltFormatInactive
% \ReturnLink
% \newpage
%
% \subsubsection{Index Sorting}
% \label{sec:IndexSort}
%
% \DescribeMacro{\IndexActual}
% The general practice for sorting with \texttt{makeindex -s} involves creating your own |.ist| file (pages 659--65 in \textit{The Latex Companion}). The following form works with both \texttt{makeindex} and \texttt{texindy}: |\index{|\meta{sort key}|@|\meta{actual}|}|. By default, the ``actual'' character is |@|. If one needs to change the ``actual'' character, such as when using \texttt{gind.ist} with \texttt{.dtx} files, one would put |\IndexActual{=}| in the preamble (or driver section) before using \cmd{\PretagName}.\medskip
%
% \noindent \DescribeMacro{\PretagName}
% The \textsf{nameauth} package enables automatic index sorting using a ``pretag'' (see Section~\ref{sec:NamePatterns}).\Version{2.0} \cmd{\PretagName} creates a sort key terminated with the ``actual'' character. Do not put the ``actual'' character in the ``pretag'':
% \begin{quote}
%   \fbox{\mystrut\ \cmd{\PretagName}\oarg{FNN}\marg{SNN|,| Affix}\oarg{Alternate}\marg{tag} }
% \end{quote}
%
% One need only ``pretag'' names once in the preamble. Thereafter, they will be sorted automatically. For example:
% \begin{quote}\small
% \StartNum
% \begin{verbatim}
%\PretagName[Jan]{Łukasiewicz}{Lukasiewicz, Jan}
%\PretagName{Æthelred, II}{Aethelred 2}
%\PretagName[W.E.B.]{Du~Bois}{Dubois, W.E.B.}\end{verbatim}
% \end{quote}
%
% Every reference to \Name*[Jan]{Łukasiewicz},  \LAeth, and \LDuBois\ is automatically tagged and sorted. One also must ``pretag'' names that contain spaces, macros, active characters, control spaces, non-breaking spaces, and anything that is not basic ASCII. That can differ when using \texttt{xindy} and Unicode-based \LaTeX.\medskip
%
% For example, the\Info{particles and\break languages} sort tag \texttt{de Soto} precedes \texttt{deal} due to the space: \texttt{de\textvisiblespace}. The sort tag \texttt{Desoto} falls between \texttt{derp} and \texttt{determinism}. German \textsf{ä ö ü ß} map to English \textsf{ae oe ue ss}. Yet Norwegian \textsf{æ ø å} follow \textsf{z} in that order. Check a style guide regarding collating sequences, spaces, and sorting. This is where using \texttt{xindy} can be very helpful. See also Section~\ref{sec:NameParticles}.\medskip
%
% One\Info{sub-entries} can sort names by creating sub-entries, which depends on the index style and formatting files: \cmd{\PretagName[Some]\{Name\}\{}\meta{category}\texttt{!Name, Some\}}. See the documentation for \texttt{xindy} and \texttt{makeindex}.
%
% Below we show how \cmd{\PretagName} helps one to avoid manually sorting cross-references (cf. Section~\ref{sec:NameParticles}):
%
% \begin{quote}\small
% \StartNum
% \begin{verbatim}
%\PretagName{\textit{Doctor angelicus}}{Doctor angelicus}
%\IndexRef{\textit{Doctor angelicus}}{Thomas, Aquinas}
%
%Perhaps the greatest medieval theologian was
%\Name{Thomas, Aquinas}, later known as
%\Name{\textit{Doctor angelicus}}.\end{verbatim}
%
% \smallskip
% \PretagName{\textit{Doctor angelicus}}{Doctor Angelicus}
% \IndexRef{\textit{Doctor angelicus}}{Thomas, Aquinas}
% Perhaps the greatest medieval theologian was
% \Name{Thomas, Aquinas}, later known as
% \Name{\textit{Doctor angelicus}}.
% \end{quote}
%
% \cmd{\PretagName} differs from the other tagging macros because its function is sorting entries, not appending information to entries:
% \begin{itemize}
% \item You can ``pretag'' any name and any cross-reference.
% \item You can ``tag'' and ``untag'' only page-reference names, not xrefs, but you can turn a page-reference name into a \textit{see also} xref. 
% \item You can undo a ``tag'' but you cannot undo a ``pretag.''
% \end{itemize}
% \newpage
% 
% \begin{center}\bfseries Debugging Problems with Sorting\end{center}
%
% \bgroup If an entry is incorrect in the index, check the following:
% \begin{itemize}
%   \item Are there any active characters, internal spaces, or control sequences in the name arguments? Use \cmd{\PretagName}.
%   \item Is alternate formatting used consistently? Are any names used within sections of alternate formatting ever used outside of them?
%   \item Are macros in the name arguments that can expand differently under different conditions preceded by \cmd{\noexpand}?
% \end{itemize}
%
% Since 2018 changes in the way that Unicode characters are handled in \texttt{pdflatex} and \texttt{latex} have made indexing simpler and more intuitive, e.g.\medskip
% 
% \def\arrow{\ \(\rightarrow\)\ }
% \def\midrowa{\arrow\quad}
% \def\midrowb{\hphantom{\arrow}\quad}
% \def\midrow{\midrowb}
% \begin{center}\MyStretch
% \begin{tabular}{ll>{\hspace{-1em}\midrow}lll>{\hspace{-1em}\midrow}l}\toprule
%   pre-2018 & text & index & post-2018 & text & index\gdef\midrow{\midrowa}\\\midrule
%   & ä & \texttt{\cmd{\IeC\textvisiblespace}\{\cmd{\"a}\}} & & ä & \texttt{ä}\\
%   \rowcolor{black!7!white} & æ & \texttt{\cmd{\IeC\textvisiblespace}\{\cmd{\ae\textvisiblespace}\}}
%     & & æ & \texttt{æ}\\\bottomrule
% \end{tabular}
% \end{center}
% One can test for this change and take different approaches with:
% \begin{quote}
%   \cmd{\IfFileExists\{utf8-2018.def\}}\marg{yes}\marg{no}
% \end{quote}
% One also should look at the entries in the \texttt{.idx} or \texttt{.ind} files to see how the name arguments and other index entry components are turned into index entries. If there are entries that do not work, one can find the corresponding page numbers in order to identify the problem.
%
% Extra spaces\Warn{} are significant when sorting index entries, yet usually are not significant in the body text. Hidden spaces, tokens, macros, and control sequences create unique index entries that look similar, yet expand and sort differently. Some macros can add spaces to index entries. For example, index tags in this manual that include \cmd{\dag} show up as \cmd{\dag\textvisiblespace\textvisiblespace} in the index (two trailing spaces). Below we show a general form of macro that adds extra spaces to index entries:
% \begin{quote}\small
% \StartNum
% \begin{verbatim}
%\newcommand\Idx[1]{%
%  \protected@edef\arg{#1}%
%  \index{\arg}}\end{verbatim}
% \end{quote}
%
% \begin{center}\MyStretch\footnotesize
% \begin{tabular}{l>{\hspace{-1em}\arrow\ }l}\toprule
% |\Idx{\textsc{football}}| & \cmd{\indexentry}\texttt{\{\cmd{\textsc\textvisiblespace\textvisiblespace}\{football\}\}\{}\meta{page}\texttt{\}}\\
% \rowcolor{black!7!white}|\index{\textsc{football}}| & \cmd{\indexentry}\texttt{\{\cmd{\textsc}\{football\}\}\{}\meta{page}\texttt{\}}\\\bottomrule
% \end{tabular}
% \end{center}\egroup
%
% \ReturnLink
%
% \subsubsection{Index Tags}
% \label{sec:IndexTag}
%
% \DescribeMacro{\TagName}
% This macro creates a tag appended to all index entries for a corresponding \cmd{\Name}.
% \DescribeMacro{\UntagName}
% The tag persists until one changes it with \cmd{\TagName} or destroys it with \cmd{\UntagName}.
% Tags can include life dates, regnal dates, and other information. Both \cmd{\TagName} and \cmd{\UntagName} handle their arguments like \cmd{\IndexName}:
% \begin{quote}
% \fbox{\vbox{\hbox{\mystrut\ \cmd{\TagName}\oarg{FNN}\marg{SNN|,| Affix}\oarg{Alternate}\marg{tag} }\par
% \hbox{\mystrut\ \cmd{\UntagName}\oarg{FNN}\marg{SNN|,| Affix}\oarg{Alternate} }}}
% \end{quote}
% \newpage
% 
% All the indexing macros are keyed to the name patterns. \cmd{\PretagName} generates the leading sort key. \cmd{\TagName} and \cmd{\UntagName} affect the trailing content:
%
% \begin{center}\small\MyStretch
% \begin{tabular}{r@{\,\vrule width0.5pt\,}c@{\,\vrule width0.5pt\,}c@{\,\vrule width0.5pt\,}l}
% & \cmd{\PretagName} & \cmd{\IndexName}\\
% \large\bfseries|\index{| & \large\bfseries|Aethelred 2@| & \large\bfseries|Æthelred II| & \large\bfseries|, king}|\\
% & & & \quad\cmd{\TagName}\\
% & & & \quad\cmd{\UntagName}\\
% \end{tabular}
% \end{center}
% \noindent 
%
% Tags\Info{scholarly\break helps} created by \cmd{\TagName} can be helpful in the indexes of academic texts by adding dates, titles, etc. \cmd{\TagName} causes the \textsf{nameauth} indexing macros to append ``\texttt{,\textvisiblespace pope}'' to the index entries for the popes below:
% \begin{quote}\small
% \StartNum
% \begin{verbatim}
%\TagName{Leo, I}{, pope}
%\TagName{Gregory, I}{, pope}
%Pope \Name{Leo, I} was known as \AKA{Leo, I}{Leo, the Great}.\\
%Pope \Name{Gregory, I} was known as \Name{Gregory, I}
%``\ForceFN\AKA*{Gregory, I}{Gregory, the Great}.''\end{verbatim}
%
% \smallskip
% Pope \Name{Leo, I} was known as \AKA{Leo, I}{Leo, the Great}.\\
% Pope \Name{Gregory, I} was known as \Name{Gregory, I}
% ``\ForceFN\AKA*{Gregory, I}{Gregory, the Great}.''
% \end{quote}
%
% \cmd{\TagName} works with all names, but not with cross-references from \cmd{\IndexRef}, \cmd{\AKA}, etc. (cf. Sections~\ref{sec:IndexXref}, \ref{sec:AKA}). Tags also can be daggers, asterisks, and so on. For example, all fictional names in the index of this manual are tagged with \S. One must add any desired spaces to the start of the tag.\medskip
%
% We\Info{same name\break game} can format and index one name as two different people with \cmd{\TagName} and \cmd{\ForgetThis} (Section~\ref{sec:NameControl}). The index tags group together their respective entries. In a normal \LaTeX\ document one would write, e.g.:
%
% \begin{quote}\small
% \StartNum
% \begin{verbatim}
%\TagName[E.]{Humperdinck}{ (composer)}
%This refers to the classical composer:
%\Name[E.]{Humperdinck}[Engelbert].
%
%\TagName[E.]{Humperdinck}{ (singer)}
%This refers to the pop singer from the 60s and 70s:
%\ForgetThis\Name[E.]{Humperdinck}[Engelbert].\end{verbatim}
%
% \smallskip
% \TagName[E.]{Humperdinck}{ (composer)\string|hyperpage}
% This refers to the classical composer:
% \Name[E.]{Humperdinck}[Engelbert].
%
% \TagName[E.]{Humperdinck}{ (singer)\string|hyperpage}
% This refers to the pop singer from the 60s and 70s:
% \ForgetThis\Name[E.]{Humperdinck}[Engelbert].
% \end{quote}
%
% One\Info{special tags} can use \cmd{\TagName} to create ``special'' index entries for names with the general form \cmd{\TagName\{}\meta{Name}\texttt{\}\{\textbar}\meta{Macro}\texttt{\}}, when \cmd{\def}\texttt{\textbackslash}\meta{Macro}\texttt{\#1\{\#1\}} exists. These\Version{3.3} tags are compatible with \textsf{hyperref}.\footnote{Before version 3.3 these special tags did not work with \textsf{hyperref}. The fix was inspired by the answer of \Name[Heiko]{Oberdiek} in: \url{https://tex.stackexchange.com/questions/201720/index-produces-invalid-idx-entry-with-manual-style-commaparse-hyperref}}
%
% For example, using the \textsf{ltxdoc} class with \textsf{hypdoc} does not create hyperlinked page entries with \textsf{nameauth}. This behavior does not affect normal \LaTeX\ documents that use \textsf{nameauth} and \textsf{hyperref}. When creating this manual, we had to tag every name with: \cmd{\TagName\{}\meta{Name}\}\texttt{\{\textbar hyperpage\}} in the driver section of the \texttt{dtx} file. 
% 
% In the ``commented'' package documentation part of a \texttt{dtx} file, the vertical bar is active. This adds an extra layer of complexity. Index tags in the documentation part must use the form: \cmd{\TagName\{}\meta{Name}\texttt{\}\{}\cmd{\string}\texttt{\textbar hyperpage\}}.
% \newpage
% 
% Below we use the conventions of this manual to create a special tag:
% \begin{quote}\small
% \StartNum
% \begin{verbatim}
%\newcommand\Orphan[2]{#1(\hyperpage{#2})}
%\TagName[Lost]{Name}{\,\S|Orphan{perdit}}
%\Name[Lost]{Name}\end{verbatim}
%
% \smallskip
% \MyStretch\Name[Lost]{Name}\\
% \texttt{idx} file: |\indexentry{Name, Lost\,\S  |\texttt{\textbar}|Orphan{perdit}}{|\meta{page}|}|\\
% \texttt{ind} file: |\item Name, Lost\,\S  \pfill \Orphan{perdit}{|\meta{page}|}|
% \end{quote}
%
% The \textsf{microtype} package and its \texttt{Spacing} environment may be the best solution to fix index entries and sub-entries that break badly across columns or pages. Suppose, however, that we wanted to insert manual breaks into an index at will, preferably after the final page reference in an entry.
% 
% We cannot just insert something like \cmd{\newpage}. In order to accomplish our goal, we need a helper macro that can take an argument. Below we use \cmd{\newpage}, but if we instead make use of the \textsf{multicol} or \textsf{idxlayout} packages we can replace that with \cmd{\columnbreak}. Two macros illustrate a similar concept:
% \begin{quote}\small
% \StartNum
% \begin{verbatim}
%\newcommand*{\EndBreak}[1]{#1\newpage}
%\makeatletter
%\newcommand*{\MidBreak}[1]{#1\newpage\@gobble}
%\makeatother\end{verbatim}
% \end{quote}
%
% Putting a break into the middle of an index entry is quite sketchy and probably should be avoided, but it can be done by using \cmd{\@gobble} to eat the comma after the break. Instead, breaking after the entry is preferable. That entry is a list of page numbers corresponding to several pages:
% \begin{quote}
%   (page 10) \cmd{\Name\{Some, Name\}}\\
%   \dots\\
%   (page 15) \cmd{\Name\{Some, Name\}}\\
%   \dots\\
%   (page 18) \cmd{\TagName\{Some, Name\textbar EndBreak\}\%}\\
%   \hphantom{(page 18)} \cmd{\Name\{Some, Name\}}
% \end{quote}
% 
% If all instances of \cmd{\Name\{Some, Name\}} on page 18 have the same tag, there will be no duplicate page entries, \textsf{hyperref} will work, and the index will break after:
% \begin{quote}
%  \texttt{Some Name \dots\ 10, 15, 18}
% \end{quote}
% 
% We\Info{manual entries} can use the same macros in manual index entries. We may need to look at the \texttt{idx} or \texttt{ind} files to craft matching entries on the page that corresponds to the placement of the break:\footnote{Results vary, depending on what distribution of \LaTeX\ is being used and how old it is. As we saw in the previous section, any name with active characters needs to be handled differently before 2018 than after 2018.}
% \begin{quote}
%   (page 18) \cmd{\SkipIndex}\cmd{\Name\{Some, Name\}\%}\\
%   \hphantom{(page 18)} \cmd{\index\{Some Name\textbar EndBreak\}}
% \end{quote}
% 
% \ReturnLink
% \newpage
%
% \subsection{``Text Tags''}
% \label{sec:TextTags}
%
% \DescribeMacro{\NameAddInfo}
% Unlike index tags, ``text tags'' are not printed automatically with every name managed by \textsf{nameauth}. Sections~\ref{sec:NameTests} and~\ref{sec:Hooksii} have more examples. The macro is \cmd{\long}, allowing for some complexity in the \meta{tag} argument:
% \begin{quote}
%   \fbox{\mystrut\ \cmd{\NameAddInfo}\oarg{FNN}\marg{SNN|,| Affix}\oarg{Alternate}\marg{tag} }
% \end{quote}
%
% For example, |\NameAddInfo[George]{Washington}{(1732--99)}|\NameAddInfo[George]{Washington}{(1732--99)} makes a text tag but does not print whenever \cmd{\Wash} ``\Wash'' is used.\medskip
%
% \DescribeMacro{\NameQueryInfo}
% To print the text tag macro associated with a name, we use \cmd{\NameQueryInfo}, which calls the appropriate macro in the name info data set:
% \begin{quote}
%   \fbox{\mystrut\ \cmd{\NameQueryInfo}\oarg{FNN}\marg{SNN|,| Affix}\oarg{Alternate} }
% \end{quote}
%
% \NameAddInfo[Ulysses S.]{Grant}{(president from 1869 to 1877)}
% \NameAddInfo[Schuyler]{Colfax}{\footnote{He was the seventeenth
% US vice-president, holding office during the first term (1869--73)
% of \Name[Ulysses S.]{Grant} \NameQueryInfo[Ulysses S.]{Grant}.}}
% |\NameQueryInfo[George]{Washington}| expands to \NameQueryInfo[George]{Washington}. One can insert a space at its start or use signs like asterisks, daggers, and even footnotes, such as one for \Name[Schuyler]{Colfax}.\NameQueryInfo[Schuyler]{Colfax} Below is the source for footnote \arabic{footnote}:
% \begin{quote}\small
% \StartNum
% \begin{verbatim}
%\NameAddInfo[Ulysses S.]{Grant}{(president from 1869 to 1877)}%
%\NameAddInfo[Schuyler]{Colfax}{\footnote{He was the seventeenth
%US vice-president, holding office during the first term (1869--73)
%of \Name[Ulysses S.]{Grant} \NameQueryInfo[Ulysses S.]{Grant}.}}\end{verbatim}
%
% \texttt{\dots}|\Name[Schuyler]{Colfax}.\NameQueryInfo[Schuyler]{Colfax}|
% \end{quote}
%
% Since\Warn{} one can nest ``text tags'' or have them call each other, one can build complex relations. Yet one must protect against a stack overflow by using Boolean flags to stop the recursion:
% \begin{quote}\small
% \newif\ifA
% \newif\ifB
% \NameAddInfo{A}{%^^A
%   \Atrue A \ifB Stop \else \NameQueryInfo{B} \fi \Afalse}
% \NameAddInfo{B}{%^^A
%   \Btrue B \ifA Stop \else \NameQueryInfo{A} \fi \Bfalse}
% \StartNum
% \begin{verbatim}
%\newif\ifA
%\newif\ifB
%\NameAddInfo{A}{%
%  \Atrue A \ifB Stop \else \NameQueryInfo{B} \fi \Afalse}
%\NameAddInfo{B}{%
%  \Btrue B \ifA Stop \else \NameQueryInfo{A} \fi \Bfalse}\end{verbatim}
% \medskip
%
% \begin{tabular}{@{}l@{ \(\rightarrow\) }l}
% \cmd{\NameQueryInfo\{A\}} & \NameQueryInfo{A}\\
% \cmd{\NameQueryInfo\{B\}} & \NameQueryInfo{B}\\
% \end{tabular}
% \end{quote}
% 
% \DescribeMacro{\NameClearInfo}
% \cmd{\NameAddInfo} will replace one text tag with another text tag, but it does not delete a text tag. That is the role of \cmd{\NameClearInfo}. The syntax is:
% \begin{quote}\small
%   \fbox{\mystrut\ \cmd{\NameClearInfo}\oarg{FNN}\marg{SNN|,| Affix}\oarg{Alternate} }
% \end{quote}
%
% After using |\NameClearInfo[George]{Washington}|\NameClearInfo[George]{Washington}, the next attempt to query the tag |\NameQueryInfo[George]{Washington}| will produce nothing\NameQueryInfo[George]{Washington}.\footnote{Had any information from a text tag been present, it would have appeared between ``nothing'' and the full stop.}
%
% \ReturnLink
% \newpage
%
% \subsection{Name Decisions}
% \label{sec:NameDecisions}
%
% The macros in this section force and detect name states. Below we keep names consistent with \textsf{beamer} overlays using some of the macros explained in this section. Otherwise, name forms will change as one advances the slides:\footnote{A copy of this example is in \texttt{examples.tex}, collocated with this manual.}
% \begin{quote}\small
% \StartNum
% \begin{verbatim}
%\documentclass{beamer}
%\usepackage{nameauth}
%\mode<presentation>
%\beamerdefaultoverlayspecification{<+->}
%
%\begin{document}
%
%\begin{frame}{Move Text Without Retyping Names}
%  \begin{itemize}\footnotesize
%  \item<1-> Original\ForgetName[George]{Washington}%
%                    \ForgetName[George]{Washington's}\\
%            \Name[Martin]{Van Buren} changes
%            after the first overlay.
%  \begin{enumerate}
%  \item<2-> \IfMainName[George]{Washington's}{He}%
%            {\Name[George]{Washington}}
%            became the first president
%            of the United States.
%  \item<3-> \IfMainName[George]{Washington}{His}%
%            {\SkipIndex\Name*[George]{Washington's}}
%            military successes during the Seven Years War
%            readied him to command the army
%            of the Continental Congress.
%  \end{enumerate}
%  \item<1-> Reordered\ForgetName[George]{Washington}%
%                     \ForgetName[George]{Washington's}\\
%            \ForgetThis\Name[Ulysses S.]{Grant}
%            does not change.
%  \begin{enumerate}
%  \item<3-> \IfMainName[George]{Washington}{His}%
%            {\SkipIndex\Name*[George]{Washington's}}
%            military successes during the Seven Years War
%            readied him to command the army
%            of the Continental Congress.
%  \item<2-> \IfMainName[George]{Washington's}{He}%
%            {\Name[George]{Washington}}
%            became the first president
%            of the United States.
%  \end{enumerate}
%  \end{itemize}
%\end{frame}
%
%\end{document}\end{verbatim}
%\IndexName[George]{Washington}
%\IndexName[Martin]{Van Buren}
%\IndexName[Ulysses S.]{Grant}
% \end{quote}
%
% The overlays, numbered progressively from one to three, begin by deleting name control sequence patterns. Uncontrolled names will change. Name conditionals ensure specific, context-dependent forms based on what name has appeared. These conditionals allow the text to be order-independent.
%
% \subsubsection{Making Decisions}
% \label{sec:NameControl}
%
% By\Info{Naming system\break behavior} default, the macros below produce global effects. They change both the \texttt{!MN} and \texttt{!NF} data sets (Section~\ref{sec:NamePatterns}). With \cmd{\ForceName} (Section~\ref{sec:Formatting}), they change formatting. Apart from \cmd{\ForceName}, they also change long or short name forms and the outcome of the testing macros in the next section:
% 
% \begin{center}\small\MyStretch
% \begin{tabular}{lccc}\toprule
% \bfseries Defaults      & \bfseries Name Length & \bfseries Format Hooks & \bfseries Test Path\\\midrule
% \bfseries Pre-First Use & Long                  & First                  & False\\
% \rowcolor{black!7!white}\bfseries Subsequent Use\quad\hbox{} & Long or short & Subsequent        & True\\\bottomrule
% \end{tabular}\medskip\\
%
% \begin{tabular}{lll}\toprule
% \bfseries Modifications & \bfseries Form & \bfseries Function\\\midrule
% \hfill|\SubvertThis\LAnth| & \SubvertThis\LAnth & force subsequent use \hbox to 3.02em{\hfill}\\
%                            &                    & force long form\\
% \rowcolor{black!7!white} |\ForceName\SAnth| & \ForceName\SAnth & default subsequent use\\
% \rowcolor{black!7!white}   &                    & force first-use format\\
% |\ForgetThis\SAnth|  & \ForgetThis\SAnth  & force first use; default long\\
%                            &                    & default first-use format\\
% \rowcolor{black!7!white}|\SAnth| & \SAnth & default subsequent use\\
% \rowcolor{black!7!white}   &                    & default short form\\\bottomrule
% \end{tabular}
% \end{center}
% 
% \DescribeMacro{\ForgetName}
% This macro takes the same arguments as \cmd{\Name}. It ``forgets'' a name, forcing a ``pre-first use'' state that will trigger a first-time name use:
% \begin{quote}
% \fbox{\mystrut\ \cmd{\ForgetName}\oarg{FNN}\marg{SNN|,| Affix}\oarg{Alternate}\ }
% \end{quote}
%
% \DescribeMacro{\ForgetThis}
% This prefix macro causes the next instance of a naming macro or shorthand to ``forget'' a name before printing it.\Version{3.1} After knowing |\Einstein| ``\Einstein'' we forget him and again have a first reference: |\ForgetThis\Einstein| ``\ForgetThis\Einstein.''\medskip
%
% \DescribeMacro{\SubvertName}
% This macro takes the same arguments as \cmd{\Name}. It ``subverts'' a name, creating a name pattern control sequence and forcing a ``subsequent use'' case:
% \begin{quote}\small
% \fbox{\mystrut\ \cmd{\SubvertName}\oarg{FNN}\marg{SNN|,| Affix}\oarg{Alternate}\ }
% \end{quote}
%
% \DescribeMacro{\SubvertThis}
% This prefix macro causes the next instance of a naming macro or shorthand to ``subvert'' a name before printing it.\Version{3.1} As in the table on page~\pageref{page:Priorities}, \cmd{\ForgetThis} has a higher priority than \cmd{\SubvertThis} and nullifies it when used together.\medskip
%
% \DescribeMacro{\LocalNames}
% \cmd{\LocalNames} restricts the effects of the macros above to the current naming system.
% \DescribeMacro{\GlobalNames}
% \cmd{\GlobalNames} restores the default behavior. We define a macro that reports whether a name exists in the main matter, front matter, both, or none:
% \def\CheckChuck{{\bfseries\IfFrontName[Charlie]{Chaplin}%
%   {\IfMainName[Charlie]{Chaplin}{both}{front}}%
%   {\IfMainName[Charlie]{Chaplin}{main}{none}}}}%
% \begin{quote}\small
% \StartNum
% \begin{verbatim}
%\def\CheckChuck{{\bfseries\IfFrontName[Charlie]{Chaplin}%
%  {\IfMainName[Charlie]{Chaplin}{both}{front}}%
%  {\IfMainName[Charlie]{Chaplin}{main}{none}}}}\end{verbatim}
% \end{quote}
%
% We start with no extant name:
% \begin{quote}\small\MyStretch
% |\CheckChuck|\dotfill\CheckChuck\qquad\qquad\qquad\hbox{}
% \end{quote}
%
% We create a name in the ``main matter'':
% \begin{quote}\small\MyStretch
% |\Name*[Charlie]{Chaplin}|\dotfill\Name*[Charlie]{Chaplin}\qquad\qquad\qquad\hbox{}\\
% |\CheckChuck|\dotfill\CheckChuck\qquad\qquad\qquad\hbox{}
% \end{quote}
%
% We switch to the ``front-matter'' and create a name, but since we are using the \texttt{quote} environment, we add \cmd{\global}:
% \begin{quote}\small\MyStretch
% |\global\NamesInactive|\global\NamesInactive\\
% |\Name*[Charlie]{Chaplin}|\dotfill\Name*[Charlie]{Chaplin}\qquad\qquad\qquad\hbox{}\\
% |\CheckChuck|\dotfill\CheckChuck\qquad\qquad\qquad\hbox{}
% \end{quote}
%
% We now have two names. Their patterns are:
% \begin{quote}\small\MyStretch
%   \texttt{\ShowPattern[Charlie]{Chaplin}!MN}\\
%   \texttt{\ShowPattern[Charlie]{Chaplin}!NF}
% \end{quote}
% 
% We use \cmd{\Localnames} to make \cmd{\ForgetName} and \cmd{\SubvertName} work with only the front-matter system. Then we ``forget'' the front-matter name:
% \begin{quote}\small\MyStretch
% |\LocalNames|\LocalNames\\
% |\ForgetName[Charlie]{Chaplin}|\ForgetName[Charlie]{Chaplin}\\
% |\CheckChuck|\dotfill\CheckChuck\qquad\qquad\qquad\hbox{}
% \end{quote}
%
% Next we ``subvert'' the front-matter name to ``remember'' it again and switch to main matter, again using \cmd{\global} to ignore scoping.
% \begin{quote}\small\MyStretch
% |\SubvertName[Charlie]{Chaplin}|\SubvertName[Charlie]{Chaplin}\\
% |\global\NamesActive|\global\NamesActive\\
% |\CheckChuck|\dotfill\CheckChuck\qquad\qquad\qquad\hbox{}
% \end{quote}
%
%  We forget the main-matter name and additionally reset the default behavior so that \cmd{\ForgetName} and \cmd{\SubvertName} work with both systems:
% \begin{quote}\small\MyStretch
% |\ForgetName[Charlie]{Chaplin}|\ForgetName[Charlie]{Chaplin}\\
% |\GlobalNames|\GlobalNames\\
% |\CheckChuck|\dotfill\CheckChuck\qquad\qquad\qquad\hbox{}
% \end{quote}
%
% Finally, we forget everything. Even though we are in a main-matter section, the front-matter name also goes away:
% \begin{quote}\small\MyStretch
% |\ForgetName[Charlie]{Chaplin}|\ForgetName[Charlie]{Chaplin}\\
% |\CheckChuck|\dotfill\CheckChuck\qquad\qquad\qquad\hbox{}
% \end{quote}
%
% \ReturnLink
% 
% \subsubsection{Testing Decisions}
% \label{sec:NameTests}
%
% The macros in this section test for the presence or absence of a name, then expand based on the result. For example, they can synchronize information between a float and body text by each testing whether a name exists and making decisions about the information accordingly.\medskip
%
% \DescribeMacro{\IfMainName}
% In order to test whether or not a ``main matter'' name control sequence exists, use this long macro that can accommodate paragraph breaks:
% \begin{quote}\small
% \fbox{\mystrut\ \cmd{\IfMainName}\oarg{FNN}\marg{SNN|,| Affix}\oarg{Alternate}\marg{yes}\marg{no} }
% \end{quote}
% 
% For example, because we have not seen the equivalent of |\Name[Bob]{Hope}| or |\SubvertName[Bob]{Hope}|, we try the following test and get:
% \begin{quote}\small\MyStretch
% |\IfMainName[Bob]{Hope}{Bob here!}{No Bob.}|\dotfill \IfMainName[Bob]{Hope}{Bob here!}{No Bob.}
% \end{quote}
%
% Still, we can create an index entry here with \IndexName[Bob]{Hope}|\IndexName[Bob]{Hope}| and it will not affect the test above. Since, however, we have encountered the equivalent of |\Name{Elizabeth,I}| many times in the document, we get the following result:
% \begin{quote}\small
% |\IfMainName{Elizabeth,I}{Bess here!}{No Bess.}|\dotfill\IfMainName{Elizabeth,I}{Bess here!}{No Bess.}
% \end{quote}
%
% \DescribeMacro{\IfFrontName}
% In order to test whether or not a ``front matter'' name control sequence exists, use this long macro that can accommodate paragraph breaks. Its syntax is:
% \begin{quote}\small
% \fbox{\mystrut\ \cmd{\IfFrontName}\oarg{FNN}\marg{SNN|,| Affix}\oarg{Alternate}\marg{yes}\marg{no} }
% \end{quote}
% This macro works like \cmd{\IfMainName}, except using the ``front matter'' name control sequences as the test subject.
%
% \phantomsection\label{page:Carnap}
% For example, based on Section~\ref{sec:Formatting}, we see that ``\ignorespaces
% \IfFrontName[Rudolph]{Carnap}%
% {\IfMainName[Rudolph]{Carnap}%
%   {\Name[Rudolph]{Carnap} is both}%
%   {\Name[Rudolph]{Carnap} is only front-matter}}%
% {\IfMainName[Rudolph]{Carnap}%
%   {\Name[Rudolph]{Carnap} is only main-matter}%
%   {\Name[Rudolph]{Carnap} is not mentioned}}'' a main-matter and front-matter name with the following test:
% \begin{quote}\small
% \StartNum
% \begin{verbatim}
%\IfFrontName[Rudolph]{Carnap}%
%{%
%  \IfMainName[Rudolph]{Carnap}%
%    {\Name[Rudolph]{Carnap} is both}%
%    {\Name[Rudolph]{Carnap} is only front-matter}%
%}%
%{%
%  \IfMainName[Rudolph]{Carnap}%
%    {\Name[Rudolph]{Carnap} is only main-matter}%
%    {\Name[Rudolph]{Carnap} is not mentioned}%
%}\end{verbatim}
% \end{quote}
%
% \DescribeMacro{\IfAKA}
% This macro tests whether or not a regular or excluded form of cross-reference control sequence exists. The syntax is:
% \begin{quote}\small
% \fbox{\mystrut\ \cmd{\IfAKA}\oarg{FNN}\marg{SNN|,| Affix}\oarg{Alternate}\marg{y}\marg{n}\marg{excluded} }
% \end{quote}
%
% This macro works like \cmd{\IfMainName}, although it has an additional \meta{excluded} branch in order to detect the activity of \cmd{\ExcludeName} (Section~\ref{sec:IndexXref}).
%
% Cross-references are governed by name control sequences ending in \texttt{!PN} (Section~\ref{sec:NamePatterns}). Regular cross-reference control sequences (the \marg{y} path) expand to empty. Excluded control sequences (the \marg{excluded} path) expand to \texttt{!}.
%
%\cmd{\ExcludeName} creates excluded xrefs; \cmd{\IncludeName} destroys them. Regular xrefs are created by \cmd{\IndexRef}, \cmd{\AKA}, \cmd{\AKA*}, and \cmd{\PName}; they are destroyed by \cmd{\IncludeName*}. Here is how we use this logic:
%
% \begin{enumerate}
% \item In the text we refer to former pro-wrestler and Minnesota governor \Name[Jesse]{Ventura}, |\Name[Jesse]{Ventura}|.
% \item We establish his lesser-known legal name as an alias: ``\IndexRef[James]{Janos}{Ventura, Jesse}\Name[James]{Janos},''\\ |\IndexRef[James]{Janos}{Ventura,|\,|Jesse}\Name[James]{Janos}|.
% 
% \item We get the result: ``\IfAKA[James]{Janos}{\Name*[Jesse]{Ventura} is a stage name}{\Name*[Jesse]{Ventura} is a regular name}{}.'' If we do not use \cmd{\ExcludeName}, we can leave the \marg{excluded} branch empty:
% \begin{quote}\small
% \StartNum
% \begin{verbatim}
%\IfAKA[James]{Janos}%
%  {\Name*[Jesse]{Ventura} is a stage name}%
%  {\Name*[Jesse]{Ventura} is a regular name}%
%  {}\end{verbatim}
% \end{quote}
% \end{enumerate}
%
% Otherwise, based on Section~\ref{sec:IndexXref}, we get: ``\IfAKA{Grinch}{\Name{Grinch} is an alias}{\Name{Grinch} is not an alias}{\Name{Grinch} is excluded}'':
% \begin{quote}\small
% \StartNum
% \begin{verbatim}
%\IfAKA{Grinch}%
%  {\Name{Grinch} is an alias}%
%  {\Name{Grinch} is not an alias}%
%  {\Name{Grinch} is excluded}\end{verbatim}
% \end{quote}
%
% We can combine all these macros create a complete test:
% \begin{quote}\small
% \StartNum
% \begin{verbatim}
%\IfAKA[FNN]{SNN, Affix}[Alternate]%
%  {true; it is a pseudonym}%
%  {% if not a pseudonym:
%    \IfFrontName[FNN]{SNN, Affix}[Alternate]% yes path
%    {\IfMainName[FNN]{SNN, Affix}[Alternate]%
%      {both}%
%      {front}%
%    }%
%    {\IfMainName[FNN]{SNN, Affix}[Alternate]% no path
%      {main}%
%      {does not exist}%
%    }%
%  }%
%  {excluded path}\end{verbatim}
% \end{quote}
%
% We can use the text tag macros with the conditional macros to present information that depends on what names have already occurred. One must avoid unbounded recursion that results in a stack overflow (Section~\ref{sec:TextTags}):
% \begin{quote}\small
% \StartNum
% \begin{verbatim}
%\IndexRef{Paul}{Saul of Tarsus}
%\NameAddInfo{Saul, of Tarsus}{\IfMainName{Jesus, Christ}
%    {\IfMainName[Lucius]{Sergius Paulus}
%      {renames himself \Name{Paul}}
%      {a preacher to the Gentiles}}
%    {wrote that he persecuted Christians}}
%
%\Name{Saul, of Tarsus} \NameQueryInfo{Saul, of Tarsus}. He
%saw a vision of \Name{Jesus, Christ} on the road to Damascus
%and became \NameQueryInfo{Saul, of Tarsus}. After converting
%\Name[Lucius]{Sergius Paulus}, \Name{Saul, of Tarsus}
%\NameQueryInfo{Saul, of Tarsus} in honor of that.\end{verbatim}
%
% \smallskip
% \IndexRef{Paul}{Saul of Tarsus}
% \NameAddInfo{Saul, of Tarsus}{\IfMainName{Jesus, Christ}
%   {\IfMainName[Lucius]{Sergius Paulus}
%   {renamed himself \Name{Paul}}
%   {a preacher to the Gentiles}}
%   {wrote that he persecuted Christians}}
%
% \Name{Saul, of Tarsus} \NameQueryInfo{Saul, of Tarsus}. He
% saw a vision of \Name{Jesus, Christ} on the road to Damascus
% and became \NameQueryInfo{Saul, of Tarsus}. After converting
% \Name[Lucius]{Sergius Paulus}, \Name{Saul, of Tarsus}
% \NameQueryInfo{Saul, of Tarsus} in honor of that.
% \end{quote}
%
% Using these tests inside other macros or passing control sequences to them may create false results (see \textit{The \TeX book}, 212--15). That is why \textsf{nameauth} uses token registers to save name arguments (Section~\ref{sec:Hooksii}. Consider using \cmd{\noexpand} in macros passed as name arguments and see also Section~\ref{sec:Unicode}. Using the \textsf{trace} package, \cmd{\show}, or \cmd{\meaning} can help one mitigate problems.
%
% \ReturnLink
% \newpage
%
% \subsection{Alternate Name Macros}
% \label{sec:AKA}
%
% The\Version{3.0} macros in this section predate \cmd{\IndexRef} and have a syntax and behavior recalling early package versions.\footnote{Before version 3.0 the lack of modularity resulted in separate name parsing, name display, and indexing for the naming macros and the alternate name macros. The version 3 series has corrected many early missteps while remaining compatible.}
% Using \cmd{\IndexRef} with \cmd{\Name} can be more flexible (cf. page~\pageref{page:ManualXref}). To save space, we show the syntax of these macros using \meta{SAFX} as the equivalent of \meta{SNN|,| Affix}. Common properties include:
%
% \begin{itemize}
% \item These macros do not create page references.
% \item The target \oarg{FNN}\marg{SAFX} comes before the xref printed in the text: \oarg{xref FNN}\marg{xref SAFX}\oarg{xref Alternate}.
% \item The obsolete syntax cannot be used with \oarg{FNN}\marg{SAFX}.
% \item Only the \meta{SAFX} and \meta{xref SAFX} arguments are able to use comma-delimited suffixes.
% \item One cannot use \cmd{\TagName} with a cross-reference, but one can sort it with \cmd{\PretagName}\oarg{xref FNN}\marg{xref SAFX}\marg{sort tag}.
% \end{itemize}
%
% \DescribeMacro{\AKA}
% \cmd{\AKA} (\textit{also known as}) and its starred form display an alias in the text and create a cross-reference in the index.
% \DescribeMacro{\AKA*}
%  They display and format names differently than the name macros:
% \begin{quote}\small
% \fbox{\vbox{\hbox{\mystrut\ \cmd{\AKA\ }\oarg{FNN}\marg{SAFX}\oarg{xref FNN}\marg{xref SAFX}\oarg{xref Alternate} }\par
% \hbox{\mystrut\ \cmd{\AKA*}\oarg{FNN}\marg{SAFX}\oarg{xref FNN}\marg{xref SAFX}\oarg{xref Alternate} }}}
% \end{quote}
%
% Both macros create a cross-reference in the index from the \meta{xref FNN}, \meta{xref SAFX}, and \meta{xref Alternate} arguments to a target defined by \meta{FNN} and \meta{SAFX}, regardless of whether that name exists. The order of the name and cross-reference in \cmd{\AKA} is opposite that of \cmd{\IndexRef}. Otherwise the \meta{xref Alternate} argument would be ambiguous with \meta{FNN}. \cmd{\AKA} prints a long form of the cross-reference name in the text. \cmd{\SeeAlso} works with \cmd{\AKA}, \cmd{\AKA*}, and \cmd{\PName}.
%
% \cmd{\AKA} prints the \meta{xref FNN} and \meta{xref SAFX} arguments in the body text. If \meta{xref Alternate} is present with \meta{xref FNN}, \cmd{\AKA} swaps them in the text. If \meta{xref Alternate} is present without \meta{xref FNN}, the old syntax is triggered, which we do not recommend. The caps and reversing macros work with \cmd{\AKA}.
%
% \cmd{\AKA*}\Version{3.0} prints short name references like \cmd{\FName}, meaning that \cmd{\ForceFN} works with it in the same manner. For the older behavior of \cmd{\AKA*} use the \texttt{oldAKA} option or always precede \cmd{\AKA*} with \cmd{\ForceFN}.
%
% We make cross-references to \Name[Bob]{Hope}, where all of the forms below create the cross-reference ``\ShowIdxPageref*[Leslie Townes]{Hope} \textit{see} \ShowIdxPageref*[Bob]{Hope}'':
% \begin{center}\small\MyStretch
% \begin{tabular}{p{0.6\textwidth}l}\toprule
% \footnotesize|\AKA[Bob]{Hope}[Leslie Townes]{Hope}| & \AKA[Bob]{Hope}[Leslie Townes]{Hope}\\
% \rowcolor{black!7!white}\footnotesize|\RevComma\AKA[Bob]{Hope}[Leslie Townes]{Hope}| & \RevComma\AKA[Bob]{Hope}[Leslie Townes]{Hope}\\
% \footnotesize|\AKA[Bob]{Hope}[Leslie Townes]{Hope}[Lester T.]| & \AKA[Bob]{Hope}[Leslie Townes]{Hope}[Lester T.]\\
% \rowcolor{black!7!white}\footnotesize|\AKA*[Bob]{Hope}[Leslie Townes]{Hope}| & \AKA*[Bob]{Hope}[Leslie Townes]{Hope}\\
% \footnotesize|\AKA*[Bob]{Hope}[Leslie Townes]{Hope}[Lester]| & \AKA*[Bob]{Hope}[Leslie Townes]{Hope}[Lester]\\\bottomrule
% \end{tabular}
% \end{center}
% \newpage
%
% Next we have references to \KeepAffix\Name*{Louis, XIV}, \Name{Lao-tzu}, and  \KeepAffix\Name*{Gregory, I}, as well as \Name[Lafcadio]{Hearn} and \Name[Charles]{du Fresne}:
% \begin{center}\small\MyStretch
% \begin{tabular}{p{0.6\textwidth}l}\toprule
% |\AKA{Louis, XIV}{Sun King}| & \AKA{Louis, XIV}{Sun King}\\
% \rowcolor{black!7!white}|\AKA*{Louis, XIV}{Sun King}| & \AKA{Louis, XIV}{Sun King}\\
% |\AKA{Lao-tzu}{Li, Er}| & \AKA{Lao-tzu}{Li, Er}\\
% \rowcolor{black!7!white}|\AKA*{Lao-tzu}{Li, Er}| & \AKA*{Lao-tzu}{Li, Er}\\
% |\AKA[Charles]{du Fresne}{du Cange}| & \AKA[Charles]{du Fresne}{du Cange}\\
% \rowcolor{black!7!white}|\CapThis\AKA[Charles]{du Fresne}{du Cange}| & \CapThis\AKA[Charles]{du Fresne}{du Cange}\\
% \footnotesize|\CapName\AKA[Lafcadio]{Hearn}{Koizumi, Yakumo}| & \CapName\AKA[Lafcadio]{Hearn}{Koizumi, Yakumo}\\
% \rowcolor{black!7!white}\footnotesize|\RevName\AKA[Lafcadio]{Hearn}{Koizumi, Yakumo}| & \RevName\AKA[Lafcadio]{Hearn}{Koizumi, Yakumo}\\
% \footnotesize|\AKA{Gregory, I}{Gregory}[the Great]| & \AKA{Gregory, I}{Gregory}[the Great]\\
% \rowcolor{black!7!white}\footnotesize|\AKA*{Gregory, I}{Gregory}[the Great]| & \AKA*{Gregory, I}{Gregory}[the Great]\\
% \footnotesize|\ForceFN\AKA*{Gregory, I}{Gregory}[the Great]| & \ForceFN\AKA*{Gregory, I}{Gregory}[the Great]\\\bottomrule
% \end{tabular}
% \end{center}\medskip
%
% \noindent\cmd{\AKA}\Info{\texttt{formatAKA}} and its derivatives use \cmd{\MainNamesHook} and \cmd{\FrontNamesHook} to print the cross-reference because that helped keep cross-references distinct from names in early package versions.
% 
% The \texttt{formatAKA} package option allows first-use formatting of alternate names, but cross-references use their own system for being ``first'' (Section~\ref{sec:NamePatterns}). We simulate \texttt{formatAKA} and use |\AKA{Elizabeth,I}[Good Queen]{Bess}|. The colors indicate which hooks are used.
% \begin{quote}\small\MyStretch
% \makeatletter\@nameauth@AKAFormattrue\makeatother\NamesInactive
% \emph{Front Matter:} \ForgetThis\LEliz\ was known as ``\AKA{Elizabeth,I}[Good Queen]{Bess}.''\\
% Again we mention Queen \Eliz, ``\AKA{Elizabeth, I}[Good Queen]{Bess}.''\\
% \cmd{\ForceName}: \ForceName\AKA{Elizabeth, I}[Good Queen]{Bess}.
%
% \NamesActive
% \emph{Main Matter:} \ForgetThis\LEliz\ was known as ``\AKA{Elizabeth,I}[Good Queen]{Bess}.''\\
% Again we mention Queen \Eliz, ``\AKA{Elizabeth, I}[Good Queen]{Bess}.''\\
% \cmd{\ForceName}: \ForceName\AKA{Elizabeth, I}[Good Queen]{Bess}.
% \end{quote}
%
% The first appearance of the cross-reference uses the first-use hooks of whatever naming system is active. Thereafter we only use the subsequent-use hooks of both systems unless we use \cmd{\ForceName}.\medskip
% 
% Below\Info{\texttt{alwaysformat}} we compare the behavior of the \texttt{alwaysformat} option, where all regular names use only \cmd{\NamesFormat} and \cmd{\FrontNamesFormat}:
% 
% \begin{quote}\small\MyStretch
% \makeatletter\@nameauth@AlwaysFormattrue\makeatother\NamesInactive
% \emph{Front Matter:} \ForgetThis\LEliz\ was known as ``\AKA{Elizabeth, I}[Good Queen]{Bess}.''\\
% Again we mention Queen \Eliz, ``\AKA{Elizabeth, I}[Good Queen]{Bess}.''\\
% \cmd{\ForceName}: \ForceName\AKA{Elizabeth, I}[Good Queen]{Bess}.
%
% \NamesActive
% \emph{Main Matter:} \ForgetThis\LEliz\ was known as ``\AKA{Elizabeth, I}[Good Queen]{Bess}.''\\
% Again we mention Queen \Eliz, ``\AKA{Elizabeth, I}[Good Queen]{Bess}.''\\
% \cmd{\ForceName}: \ForceName\AKA{Elizabeth, I}[Good Queen]{Bess}.
% \end{quote}
%
% \DescribeMacro{\PName}
% These convenience macros (an early feature) print a main name and a cross-reference in parentheses:
% \DescribeMacro{\PName*}
% \begin{quote}\small
% \fbox{\vbox{\hbox{\mystrut\ \cmd{\PName\ }\oarg{FNN}\marg{SAFX}\oarg{xref\,FNN}\marg{xref\,SAFX}\oarg{xref\,Alternate}\ }\par
% \hbox{\mystrut\ \cmd{\PName*}\oarg{FNN}\marg{SAFX}\oarg{xref\,FNN}\marg{xref\,SAFX}\oarg{xref\,Alternate}\ }}}
% \end{quote}
% The starred form \cmd{\PName*} is like the starred form \cmd{\Name*} to the extent that it prints a long form of \meta{FNN}\meta{SAFX}. It does not affect the cross-reference arguments \meta{xref FNN}\meta{xref SAFX}\meta{xref Alternate}.
% 
% Except \cmd{\SkipIndex}, prefix macros only affect \meta{FNN}\meta{SAFX}, not the cross-reference, which always takes a long form. \cmd{\SkipIndex} keeps both names out of the index. \cmd{\PName} cannot use the obsolete syntax for the main name, but it can do so for the alternate name.
% 
% \begin{center}\footnotesize\MyStretch
% \begin{tabular}{p{0.45\textwidth}l}\toprule
% \bfseries\small Recommended Macro\,/\,Output & \bfseries\small Index\\\midrule
% |\PName[Mark]{Twain}%|\newline|  [Samuel L.]{Clemens}| & \\
% \PName[Mark]{Twain}[Samuel L.]{Clemens} & \ShowIdxPageref*[Samuel L.]{Clemens} \textit{see} \ShowIdxPageref*[Mark]{Twain}\\
% \PName[Mark]{Twain}[Samuel L.]{Clemens} & \ShowIdxPageref*[Samuel L.]{Clemens} \textit{see} \ShowIdxPageref*[Mark]{Twain}\\
% \rowcolor{black!7!white}|\PName*[Mark]{Twain}%|\newline|  [Samuel L.]{Clemens}[Sam]| & \\
% \rowcolor{black!7!white}\PName*[Mark]{Twain}[Samuel L.]{Clemens}[Sam] & \ShowIdxPageref*[Samuel L.]{Clemens} \textit{see} \ShowIdxPageref*[Mark]{Twain}\\
% |\PName{Voltaire}%|\newline|  [François-Marie]{Arouet}| & \\
% \PName{Voltaire}[François-Marie]{Arouet} & \ShowIdxPageref*[François-Marie]{Arouet} \textit{see} \ShowIdxPageref*{Voltaire}\\
% \PName{Voltaire}[François-Marie]{Arouet} & \ShowIdxPageref*[François-Marie]{Arouet} \textit{see} \ShowIdxPageref*{Voltaire}\\
% \rowcolor{black!7!white}|\PName{William, I}%|\newline|  {William, the Conqueror}| & \\
% \rowcolor{black!7!white}\PName{William, I}{William, the Conqueror} & \ShowIdxPageref*{William, the Conqueror} \textit{see} \ShowIdxPageref*{William, I}\\
% \rowcolor{black!7!white}\PName{William, I}{William, the Conqueror} & \ShowIdxPageref*{William, the Conqueror} \textit{see} \ShowIdxPageref*{William, I}\\
% |\PName*{William, I}%|\newline|  {William, the Conqueror}| & \\
% \PName*{William, I}{William, the Conqueror} & \ShowIdxPageref*{William, the Conqueror} \textit{see} \ShowIdxPageref*{William, I}\\
% \rowcolor{black!7!white}|\PretagName%|\newline|  {\textit{Doctor mellifluus}}%|\newline|  {Doctor mellifluus}| & \\
% \rowcolor{black!7!white}|\PName{Bernard, of Clairvaux}%|\newline|  {\textit{Doctor mellifluus}}| & \\
% \rowcolor{black!7!white}\PName{Bernard, of Clairvaux}{\textit{Doctor mellifluus}} & \ShowIdxPageref*{\textit{Doctor mellifluus}} \textit{see} \ShowIdxPageref*{Bernard, of Clairvaux}\\
% \rowcolor{black!7!white}\PName{Bernard, of Clairvaux}{\textit{Doctor mellifluus}} & \ShowIdxPageref*{\textit{Doctor mellifluus}} \textit{see} \ShowIdxPageref*{Bernard, of Clairvaux}\\
% |\ForgetThis\PName{Lao-tzu}{Li, Er}| & \\
% \ForgetThis\PName{Lao-tzu}{Li, Er} & \ShowIdxPageref*{Li, Er} \textit{see} \ShowIdxPageref*{Lao-tzu}\\
% \PName{Lao-tzu}{Li, Er} & \ShowIdxPageref*{Li, Er} \textit{see} \ShowIdxPageref*{Lao-tzu}\\\midrule
% \bfseries\small Discouraged Macro\,/\,Output & \bfseries\small Index\\\midrule
% |\PName{William, I}{William}%|\newline|  [the Conqueror]| & \\
% \PName{William, I}{William}[the Conqueror] & \ShowIdxPageref*{William}[the Conqueror] \textit{see} \ShowIdxPageref*{William, I}\\
% \rowcolor{black!7!white}|\PName{Lao-tzu}{Li}[Er]| & \\
% \rowcolor{black!7!white}\PName{Lao-tzu}{Li}[Er] & \ShowIdxPageref*{Li}[Er] \textit{see} \ShowIdxPageref*{Lao-tzu}\\\bottomrule
% \end{tabular}
% \end{center}
%
% The newer non-Western syntax does not work with \cmd{\PName}. If we attempted to use |\SkipIndex\PName*{William, I}[William]{the Conqueror}|, this macro would put ``\SkipIndex\PName*{William, I}[William]{the Conqueror}'' in the body text, but its index entry would be incorrect: ``\ShowIdxPageref*[William]{the Conqueror} \textit{see} \ShowIdxPageref*{William, I}''.
% 
% \ReturnLink
% \newpage
%
% \subsection{Longer Examples}
% \label{sec:Hooks}
%
% Examples from the remainder of this manual are in \texttt{examples.tex}, included with the \textsf{nameauth} documentation.
%
% When\Info{\texttt{dtx} files} creating package documentation, any name that has a macro in its argument should be set up in the driver section (the \texttt{nameauth} environment and tags from \cmd{\PretagName} and \cmd{\TagName}). Otherwise, errors can result.
%
% \begingroup^^A Start of hook macro redefinition.
% \subsubsection{Hooks: Intro}
% \label{sec:Hooksi}
%
% In these sections on advanced topics we reset all formatting hooks to do nothing. This helps us focus on the modifications made hereafter.
% \renewcommand*\NamesFormat{}
% \renewcommand*\FrontNamesFormat{}
% \renewcommand*\MainNameHook{}
% \renewcommand*\FrontNameHook{}
%
% Before we get to the use of text tags and name conditionals in name formatting, we seek to illustrate that something more complex than a font switch can occur in \cmd{\NamesFormat}. Below we put the first mention of a name in boldface, along with a marginal notation if possible.
% \begin{quote}\small
% \StartNum
% \begin{verbatim}
%\let\OldFormat\NamesFormat
%\renewcommand*\NamesFormat[1]{\textbf{#1}\unless\ifinner
%   \marginpar{\raggedleft\scriptsize #1}\fi}
%\let\NamesFormat\OldFormat
%\PretagName{Vlad, Ţepeş}{Vlad Tepes} % for accented names
%\TagName{Vlad, II}{ Dracul}          % for index information 
%\TagName{Vlad, III}{ Dracula}\end{verbatim}
%
% \medskip Within the document after the preamble:
% \let\OldFormat\NamesFormat
% \renewcommand*\NamesFormat[1]{\textbf{#1}\unless\ifinner
%   \marginpar{\raggedleft\scriptsize #1}\fi}
% \ContinueNum
% \begin{verbatim}\Name{Vlad, III}[III Dracula], known as
%\AKA{Vlad III}{Vlad, Ţepeş} (the Impaler)
%after his death, was the son of \Name{Vlad, II}[II Dracul],
%a member of the Order of the Dragon. Later references to
%``\Name*{Vlad, III}'' and ``\Name{Vlad, III}'' appear thus.\end{verbatim}
%
% \Name{Vlad, III}[III Dracula], known as
% \AKA{Vlad III}{Vlad, Ţepeş} (the Impaler)
% after his death, was the son of \Name{Vlad, II}[II Dracul],
% a member of the Order of the Dragon. Later references to
% ``\Name*{Vlad, III}'' and ``\Name{Vlad, III}'' appear thus.
%
% \let\NamesFormat\OldFormat
% \begin{verbatim}\let\NamesFormat\OldFormat\end{verbatim}
% \end{quote}
% Now we have reverted to the default \cmd{\NamesFormat} and we get:
% \begin{itemize}
%   \item |\ForgetThis\Name{Vlad, III}[III Dracula]|\dotfill \ForgetThis\Name{Vlad, III}[III Dracula]
%   \item |\Name*{Vlad, III}|\dotfill \Name*{Vlad, III}
%   \item |\Name{Vlad, III}|\dotfill \Name{Vlad, III}
% \end{itemize}
% We also set up the cross-reference |\IndexRef{Dracula}{Vlad III}|\IndexRef{Dracula}{Vlad III}. Compare the examples for \LDem\ in Section~\ref{sec:NameParticles}.
% 
% \ReturnLink
% \newpage
% 
% \subsubsection{Hooks: Life Dates}
% \label{sec:Hooksii}
% We can use name conditionals (Section~\ref{sec:NameTests}) and text tags (Section~\ref{sec:TextTags}) to add life information to names when desired.
%
% \DescribeMacro{\if@nameauth@InName}
% The example \cmd{\NamesFormat} below adds a text tag to the first occurrences of main-matter names.
% \DescribeMacro{\if@nameauth@InAKA}
% It uses internal macros of \cmd{\@nameauth@Name}. To prevent errors, the Boolean values \texttt{\textbackslash if@nameauth@InName} and \texttt{\textbackslash if@nameauth@InAKA} are true only within the scope of \cmd{\@nameauth@Name} and \cmd{\AKA} respectively.
%
% \DescribeMacro{\@nameauth@toksa}
% This package makes three token registers available to facilitate using the name conditional macros as we do below.
% \DescribeMacro{\@nameauth@toksb}
% These registers are necessary for names that contain accents and diacritics.\footnote{In \cmd{\AKA} these registers correspond to the \emph{last} three arguments, the xref.}
% \DescribeMacro{\@nameauth@toksc}
% 
%
% Below the first use of a name is in small caps. Text tags are in boldface with naming macros, and roman with \cmd{\AKA}. Just because we set up a cross-reference does not mean that we use \cmd{\AKA} by default, as was the case in early versions of \textsf{nameauth}. We use \cmd{\ForceName} to use it more than once with \cmd{\AKA}:
% \begin{quote}\small
% \StartNum
% \begin{verbatim}
%\newif\ifNoTag% allows us to work around \ForgetName
%\let\OldFormat\NamesFormat
%\let\OldFrontFormat\FrontNamesFormat
%\makeatletter
%\renewcommand*\NamesFormat[1]{\begingroup%
%  \protected@edef\temp{\endgroup\textsc{#1}%
%  \unless\ifNoTag
%    \if@nameauth@InName
%      {\bfseries\noexpand\NameQueryInfo
%      [\unexpanded\expandafter{\the\@nameauth@toksa}]
%      {\unexpanded\expandafter{\the\@nameauth@toksb}}
%      [\unexpanded\expandafter{\the\@nameauth@toksc}]}\fi
%    \if@nameauth@InAKA
%      {\normalfont\noexpand\NameQueryInfo
%      [\unexpanded\expandafter{\the\@nameauth@toksa}]
%      {\unexpanded\expandafter{\the\@nameauth@toksb}}
%      [\unexpanded\expandafter{\the\@nameauth@toksc}]}\fi
%  \fi}\temp\global\NoTagfalse%
%}
%\makeatother
%\let\FrontNamesFormat\NamesFormat\end{verbatim}
% \end{quote}
% \let\OldFormat\NamesFormat
% \let\OldFrontFormat\FrontNamesFormat
% \makeatletter
% \renewcommand*\NamesFormat[1]{\begingroup%^^A
%   \protected@edef\temp{\endgroup\textsc{#1}%^^A
%   \unless\ifNoTag
%     \if@nameauth@InName
%       {\bfseries\noexpand\NameQueryInfo
%       [\unexpanded\expandafter{\the\@nameauth@toksa}]
%       {\unexpanded\expandafter{\the\@nameauth@toksb}}
%       [\unexpanded\expandafter{\the\@nameauth@toksc}]}\fi
%     \if@nameauth@InAKA
%       {\normalfont\noexpand\NameQueryInfo
%       [\unexpanded\expandafter{\the\@nameauth@toksa}]
%       {\unexpanded\expandafter{\the\@nameauth@toksb}}
%       [\unexpanded\expandafter{\the\@nameauth@toksc}]}\fi
%   \fi}\temp\global\NoTagfalse}%^^A
% \makeatother
%
% We print tags in the first use hooks unless \cmd{\NoTag} is set true. This method uses the two \(\epsilon\)-\TeX{} primitives \cmd{\noexpand} and \cmd{\unexpanded} to avoid repetition of \cmd{\expandafter}. Since \textsf{nameauth} depends on \textsf{etoolbox}, we assume \(\epsilon\)-\TeX.
% 
% Before we can refer to any text tags, we must create them. Using the approach above, we include a leading space in the text tags. The leading space is needed only when a text tag appears.\footnote{Another way to add that space is to put it in the conditional path of the formatting hook and leave it out of the text tags entirely: \dots\texttt{\{ \}}\cmd{\noexpand}\cmd{\NameQueryInfo}\dots}
% We also set up a cross-reference, which we will use regardless of whether we also use \cmd{\AKA}. The cross-reference will be created only once and skipped thereafter:
% \begin{quote}\small
% \ContinueNum
% \begin{verbatim}
%\NameAddInfo[George]{Washington}{ (1732--99)}
%\NameAddInfo[Mustafa]{Kemal}{ (1881--1938)}
%\NameAddInfo{Atatürk}{ (in 1934, a special surname)}
%\IndexRef{Atatürk}{Kemal, Mustafa}\end{verbatim}
% \end{quote}
% \NameAddInfo[George]{Washington}{ (1732--99)}
% \NameAddInfo[Mustafa]{Kemal}{ (1881--1938)}
% \NameAddInfo{Atatürk}{ (in 1934, a special surname)}
% \IndexRef{Atatürk}{Kemal, Mustafa}
% \newpage
% 
% Now we begin with the first example, which, after all the setup, looks deceptively simple, but highly reusable without extra work:
% \begin{quote}\small
% \ContinueNum
% \begin{verbatim}
%\ForgetThis\Wash held office 1789--97.
%No tags: \Wash.\\
%First use, dates suppressed: \NoTagtrue\ForgetThis\Wash.\end{verbatim}
%
% \smallskip
% \ForgetThis\Wash\ held office 1789--97.\\
% No tags: \Wash.\\
% First use, dates suppressed: \NoTagtrue\ForgetThis\Wash.
% \end{quote}
%
% Since we already set up a cross-reference with \cmd{\IndexRef}, we can use just the the naming macros with ``Atatürk'' and get the desired formatting without any page references in the index:
% \begin{quote}\small
% \ContinueNum
% \begin{verbatim}
%\Name[Mustafa]{Kemal} was granted the name
%\Name{Atatürk}. We mention \Name[Mustafa]{Kemal}
%and \Name{Atatürk} again.
%
%First use, no tag:
%\NoTagtrue\ForgetThis\Name{Atatürk}.\end{verbatim}
%
% \smallskip
% \Name[Mustafa]{Kemal} was granted the name
% \Name{Atatürk}. We mention \Name[Mustafa]{Kemal}
% and \Name{Atatürk} again.
%
% First use, no tag:
% \NoTagtrue\ForgetThis\Name{Atatürk}.
% \end{quote}
%
% Since we set up distinct formatting for \cmd{\AKA} (\cmd{\normalfont} instead of boldface for text tags associated with cross-references), we now simulate the \texttt{formatAKA} package option and use \cmd{\ForceName} with \cmd{\AKA}:
% \begin{quote}\small
% \ContinueNum
% \begin{verbatim}
%\makeatletter\@nameauth@AKAFormattrue\makeatother
%\ForgetThis\Name[Mustafa]{Kemal} was granted the name
%\ForceName\AKA[Mustafa]{Kemal}{Atatürk}. We mention
%\Name[Mustafa]{Kemal} and \AKA[Mustafa]{Kemal}{Atatürk} again.
%
%First use, no tag:
%\NoTagtrue\ForceName\AKA[Mustafa]{Kemal}{Atatürk}.\end{verbatim}
%
% \smallskip
% \makeatletter\@nameauth@AKAFormattrue\makeatother
% \ForgetThis\Name[Mustafa]{Kemal} was granted the name
% \ForceName\AKA[Mustafa]{Kemal}{Atatürk}. We mention
% \Name[Mustafa]{Kemal} and \AKA[Mustafa]{Kemal}{Atatürk} again.
%
% First use, no tag:
% \NoTagtrue\ForceName\AKA[Mustafa]{Kemal}{Atatürk}.
% \end{quote}%
% Please remember to reset the formatting:
% \begin{quote}\small
% \ContinueNum
% \begin{verbatim}
%\let\NamesFormat\OldFormat
%\let\FrontNamesFormat\OldFrontFormat\end{verbatim}
% \end{quote}
% \let\NamesFormat\OldFormat
% \let\FrontNamesFormat\OldFrontFormat
%
% \ReturnLink
% \newpage
% 
% \subsubsection{Hooks: Advanced}
% \label{sec:Hooksiii}
% \AltFormatActive\begingroup
%
% \begin{center}\bfseries Alternate Formatting\end{center}
%
% \noindent The\Version{3.1} alternate formatting framework provides features that aid both error protection and ease of use. This section uses \cmd{\AltFormatActive}. We do not use the names in this section elsewhere. A name designed for the alternate formatting regime may cause spurious index entries when used in the default formatting regime.
%
% Both \cmd{\AltFormatActive} and \cmd{\AltFormatActive*} set the internal Boolean flag \cmd{\@nameauth@AltFormattrue}, enabling alternate formatting. \cmd{\AltFormatActive} sets \cmd{\@nameauth@DoAlttrue}, which activates formatting. \cmd{\AltFormatInactive} sets both flags false.
%
% \cmd{\AltFormatActive*} normally suppresses formatting changes but it still forces  \cmd{\CapThis} to work through \cmd{\AltCaps}. This produces the default look of \textsf{nameauth} and prevents Continental formatting, but it also reduces spurious index entries and errors if many names use macros in their arguments.
% 
% Alternate formatting protects against errors created when \cmd{\@nameauth@Cap} (used by \cmd{\CapThis}) gets a failure result from \cmd{\@nameauthUTFtest}, but that result is neither a letter nor a macro that expands to a sequence of letters. Protected macros and other cases may create errors if \cmd{\MakeUppercase} is applied to them. \cmd{\AltCaps} and \cmd{\CapThis} work together to avoid this problem (Section~\ref{sec:AltFormat}).
%
% \begin{center}\bfseries Continental Format\end{center}
%
% \noindent Here we look in greater detail at how \textsf{nameauth} implements the advanced version of Continental formatting. Font changes occur in the short macros \cmd{\textSC}, \cmd{\textIT}, \cmd{\textBF}, and \cmd{\textUC}. They all look similar to \cmd{\textSC}:
% \begin{quote}\small
% \StartNum
% \begin{verbatim}
%\newcommand*\textSC[1]{%
%  \if@nameauth@DoAlt\textsc{#1}\else#1\fi
%}\end{verbatim}
% \end{quote}
% 
% If the \texttt{altformat} option or \cmd{\AltFormatActive} is used, formatting occurs in both the text and in the index. We want small caps on by default in the text and index, then off in subsequent uses. Thus, we use \cmd{\AltFormatActive}, then redefine \cmd{\MainNameHook} because it is the subsequent use hook. \cmd{\AltOff} deactivates formatting only in the formatting hooks:
% \begin{quote}\small
% \ContinueNum
% \begin{verbatim}
%\newcommand*\AltOff{%
%  \if@nameauth@InHook\@nameauth@DoAltfalse\fi
%}\end{verbatim}
% \end{quote}
%
% \cmd{\CapThis} now triggers \cmd{\AltCaps} to capitalize its argument:
% \begin{quote}\small
% \ContinueNum
% \begin{verbatim}
%\newcommand*\AltCaps[1]{%
%  \if@nameauth@InHook
%    \if@nameauth@DoCaps\MakeUppercase{#1}\else#1\fi
%  \else#1\fi
%}\end{verbatim}
% \end{quote}
%
% We must put \cmd{\noexpand} before \cmd{\textSC}, \cmd{\AltCaps}, and so on to prevent them from expanding outside of the formatting hooks.
%
% Before we alter the formatting hooks, we either can \cmd{\let} the hook macros to recall them later or we can use \cmd{\begingroup} and \cmd{\endgroup} to create a new scope that localizes any changes. We use scoping in this section.
%
% This final step \emph{does not come} from the \textsf{nameauth} source. We must redefine the formatting hooks ourselves. One of the simplest ways to do this when using the \texttt{altformat} option or \cmd{\AltFormatActive} is:
% \begin{quote}\small
% \ContinueNum
% \begin{verbatim}
%\renewcommand*\MainNameHook{\AltOff}
%\let\FrontNameHook\MainNameHook\end{verbatim}
% \end{quote}
%
% \renewcommand*\MainNameHook{\AltOff}\let\FrontNameHook\MainNameHook
% To suppress all formatting in the front-matter text, one need simply to use |\let\FrontNamesFormat\MainNameHook|. Continental formatting usually alters at least one element in the required name argument, as we see below:
% \begin{quote}\small
% \ContinueNum
% \begin{verbatim}
%\begin{nameauth}
%  \< Adams   & John  & \noexpand\textSC{Adams}        & >
%  \< SDJR    & Sammy & \noexpand\textSC{Davis},
%                       \noexpand\textSC{Jr}.          & >
%  \< HAR     &       & Harun, \noexpand\textSC%
%                       {\noexpand\AltCaps{a}l-Rashid} & >
%  \< Mencius &       & \noexpand\textSC{Mencius}      & >
%\end{nameauth}\end{verbatim}
% \end{quote}
%
% Now we must ensure that these names are sorted properly in the index. When sorting names, be sure to use \cmd{\noexpand} before the macros:
% \begin{quote}\small
% \ContinueNum
% \begin{verbatim}
%\PretagName[John]{\noexpand\textSC{Adams}}{Adams, John}
%\PretagName[Sammy]%
%  {\noexpand\textSC{Davis}, \noexpand\textSC{Jr}.}%
%  {Davis, Sammy, Jr.}
%\PretagName{Harun, \noexpand\textSC%
%  {\noexpand\AltCaps{a}l-Rashid}}{Harun al-Rashid}
%\PretagName{\noexpand\textSC{Mencius}}{Mencius}\end{verbatim}
% \end{quote}
%
% \begin{center}
% \small\noindent\begin{tabular}{llll}\toprule
% First & Next & Long & Short \\\midrule
% |\Adams| & |\Adams| & |\LAdams| & |\SAdams|\\
% \Adams & \Adams & \LAdams & \SAdams\\
% \rowcolor{black!7!white}|\SDJR| & |\SDJR| & |\LSDJR| & |\SSDJR|\\
% \rowcolor{black!7!white}\SDJR & \SDJR & \LSDJR & \SSDJR\\
% |\HAR| & |\HAR| & |\LHAR| & |\SHAR|\\
% \HAR & \HAR & \LHAR & \SHAR\\
% \rowcolor{black!7!white}|\Mencius| & |\Mencius| & |\LMencius| & |\SMencius|\\
% \rowcolor{black!7!white}\Mencius & \Mencius & \LMencius & \SMencius\\\bottomrule
% \end{tabular}
% \end{center}
%
% \begin{itemize}\small
% \item Punctuation detection works: \ForceName\LSDJR. Also \LSDJR. Then \ForceName\SDJR. Now \SDJR. (We used \cmd{\ForceName} for formatting.)
% \item \cmd{\ForceName}\cmd{\DropAffix}\cmd{\LSDJR} gives \ForceName\DropAffix\LSDJR. Otherwise, only using the macro \cmd{\DropAffix}\cmd{\LSDJR} gives \DropAffix\LSDJR.
% \item \cmd{\RevComma}\cmd{\LAdams} yields \RevComma\LAdams. All the reversing macros work.
% \item \cmd{\ForceName}\cmd{\ForceFN}\cmd{\SHAR} produces \ForceName\ForceFN\SHAR. \cmd{\ForceFN}\cmd{\SHAR} produces \ForceFN\SHAR. If we add \cmd{\CapThis} we get \CapThis\ForceName\ForceFN\SHAR\ and \CapThis\ForceFN\SHAR.\footnote{The way that Continental resources treat certain affixes relates to similar issues in [\hyperlink{Mulvany}{Mulvany}, 168--73]. Handling non-Western names in Western sources can be a gray area. One ought take care to be culturally sensitive in these matters.}
% \item One must include all the macros in the name arguments.
% \end{itemize}
%
% \makeatletter\@nameauth@AKAFormattrue\makeatother
% If we use the \texttt{formatAKA} option we can refer to \Mencius\ as \AKA{\noexpand\textSC{Mencius}}{\noexpand\textSC{Meng}, Ke}, and again \AKA{\noexpand\textSC{Mencius}}{\noexpand\textSC{Meng}, Ke}. We get that with:
% \begin{quote}\small
% \ContinueNum
% \begin{verbatim}
%\PretagName{\noexpand\textSC{Meng}, Ke}{Meng Ke}
%\AKA{\noexpand\textSC{Mencius}}{\noexpand\textSC{Meng}, Ke}\end{verbatim}
% \end{quote}
% \makeatletter\@nameauth@AKAFormatfalse\makeatother
%
% \begin{center}\bfseries Rolling Your Own: Basic\end{center}
%
% \noindent Here we set out on the path to custom formatting by using package features that have been implemented already and look similar to the solutions in Section~\ref{sec:AltFormat}.
%
% When\Warn{} redesigning formatting hooks, one should use \cmd{\AltFormatActive} or the \texttt{altformat} option to enable alternate formatting and prevent \cmd{\CapThis} from breaking custom formatting macros.
%
% We recommend examining the internal package flag \cmd{\@nameauth@DoAlt}, which activates alternate formatting, \cmd{\@nameauth@DoCaps}, which handles capitalization, and \cmd{\@nameauth@InHook}, which is true when the formatting hooks are called. See page~\pageref{page:Hooks} and following. If you create your own macros, they will look similar:
% \begin{quote}\small
% \StartNum
% \begin{verbatim}
%\makeatletter
%\newcommand*\Fbox[1]{%
%  \if@nameauth@DoAlt\protect\fbox{#1}\else#1\fi
%}
%\makeatother\end{verbatim}
% \end{quote}
% \makeatletter\newcommand*\Fbox[1]{\if@nameauth@DoAlt\protect\fbox{#1}\else#1\fi}\makeatother
%
% Since \cmd{\AltCaps} is part of \textsf{nameauth}, you need not reinvent that wheel. Just use it. The final step is redefining the hooks, which can be as simple as:
% \begin{quote}\small
% \ContinueNum
% \begin{verbatim}
%\renewcommand*\MainNameHook{\AltOff}
%\let\FrontNameHook\MainNameHook\end{verbatim}
% \end{quote}
%
% When sorting names, be sure to use \cmd{\noexpand} as shown previously:
% \begin{quote}\small
% \ContinueNum
% \begin{verbatim}
%\PretagName[Pierre-Jean]%
%  {\noexpand\Fbox{\noexpand\AltCaps{d}e Smet}}%
%  {de Smet, Pierre-Jean}
%
%\begin{nameauth}
%  \< deSmet & Pierre-Jean &
%     \noexpand\Fbox{\noexpand\AltCaps{d}e Smet} & >
%\end{nameauth}\end{verbatim}
% \end{quote}
%
% Now we show how the formatting hooks work in the body text. One can check the index to see that it is formatted properly and consistently.
%
% \begin{center}\footnotesize
% \begin{tabular}{llll}\toprule
% First & Next & Long & Short \\\midrule
% |\deSmet| & |\deSmet| & |\LdeSmet| & |\SdeSmet|\\
% \deSmet & \deSmet & \LdeSmet & \SdeSmet\\\bottomrule
% \end{tabular}
% \end{center}
%
% The capitalized version |\CapThis\deSmet| is \CapThis\deSmet. This also works for a formatted use via \cmd{\ForceName}: \ForceName\CapThis\deSmet.
%
% Some formatting, such as the use of \cmd{\textSC}, is fairly standard. Other formatting, such as \cmd{\Fbox} above, is ornamental and may be handled better with custom features (Section~\ref{sec:Customize}), but those features appear only in the text.
% \newpage
%
% \begin{center}\bfseries Rolling Your Own: Intermediate\end{center}
%
% \noindent \hspace{-0.47em}``Intermediate'' and ``advanced'' refer to the way hooks were designed before version 3.1. We begin the journey to more customized formatting by looking at \cmd{\NameParser}, whose logic Sections~\ref{sec:InternalMacros} and~\ref{sec:UserInterface} show in detail.\medskip
%
% \DescribeMacro{\NameParser}
% This user-accessible parser (Section~\ref{sec:UserInterface}) builds a printed name from the internal macros \cmd{\FNN}, \cmd{\SNN}, \cmd{\rootb} and \cmd{\suffb}.\Version{3.1} It uses the following Boolean flags:\footnote{These exclude all capitalization macros.}
% \begin{quote}\small\MyStretch
% Only one or the other of these can be true to avoid undocumented behavior.\smallskip\\
% |\if@nameauth@FullName|\hfill Print a full name if true.\\
% |\if@nameauth@FirstName|\hfill Print a first name if true.\smallskip\\
% Reversing without commas overrides reversing with commas.\smallskip\\
% |\if@nameauth@RevThis|\hfill Reverse name order if true.\\
% |\if@nameauth@EastFN|\hfill toggled by \cmd{\ForceFN}.\\
% |\if@nameauth@RevThisComma|\hfill Reverse Western name, add comma.
% \end{quote}
%
% We create a hook that can ignore ignore the output of \cmd{\@nameauth@Name}, which is the \texttt{\#1} in the hook dispatcher's code \cmd{\bgroup}\meta{Hook}\texttt{\{\#1\}}\cmd{\egroup}:
% \begin{quote}\small\MyStretch
% |\renewcommand*|\meta{FirstHook}|[1]{|\texttt{\dots}\cmd{\NameParser}\texttt{\dots}|}|
% \end{quote}
%
% With the \texttt{altformat} option or \cmd{\AltFormatActive} we can design a subsequent-use hook that deactivates formatting inside of it:
% \begin{quote}\small\MyStretch
% |\renewcommand*|\meta{SubsequentHook}|[1]{|\texttt{\dots}\cmd{\AltOff}\cmd{\NameParser}\texttt{\dots}|}|
% \end{quote}
%
% If we used \cmd{\AltFormatActive*}, where the formatting macros are enabled, but deactivated, then we might want a hook that activates the macros:
% \begin{quote}\small\MyStretch
% |\renewcommand*|\meta{Hook}|[1]{|\texttt{\dots}\cmd{\AltOn}\cmd{\NameParser}\texttt{\dots}|}|
% \end{quote}
% 
% Within the hooks we can use the user-level parser as often as we want. We also can change internal Boolean flags, for example:
% \begingroup
% \begin{quote}\small
% \StartNum
% \begin{verbatim}
%\makeatletter
%\renewcommand*\NamesFormat[1]{\small%
%  \hbox to 3.5em{[now]\hfill}\space\NameParser\\%
%  \@nameauth@FullNametrue%
%  \hbox to 3.5em{[long]\hfill}\space\NameParser\\%
%  \@nameauth@FullNamefalse%
%  \@nameauth@FirstNametrue%
%  \hbox to 3.5em{[short]\hfill}\space\NameParser}
%\makeatother
%\let\MainNameHook\NamesFormat\end{verbatim}
%
% \smallskip
% \makeatletter
% \renewcommand*\NamesFormat[1]{\small%^^A
%   \hbox to 3.5em{[now]\hfill}\space\NameParser\\%^^A
%   \@nameauth@FullNametrue%^^A
%   \hbox to 3.5em{[long]\hfill}\space\NameParser\\%^^A
%   \@nameauth@FullNamefalse%^^A
%   \@nameauth@FirstNametrue%^^A
%   \hbox to 3.5em{[short]\hfill}\space\NameParser}
% \makeatother
% \let\MainNameHook\NamesFormat
%
% |\JRIV| displays:
% 
% \JRIV
% \end{quote}
% \endgroup
%
% The proof of concept above is interesting, but not very useful. Now we move on toward more useful designs, based on Sections~\ref{sec:Hooksi} and~\ref{sec:Hooksii}.
% \newpage
% 
% We begin by defining a name that is composed only of macros:
% \begingroup
% \newif\ifSpecialFN
% \newif\ifSpecialSN
% \newif\ifRevertSN
% \newcommand*\WM{\ifSpecialFN Wm.\else William\fi}
% \newcommand*\SHK{\ifRevertSN \textSC{Shakespeare}\else
%                  \ifSpecialSN \noexpand\AltCaps{t}he Bard\else
%                  \textSC{Shakespeare}\fi\fi}
% \newcommand*\Revert{\RevertSNtrue}
% \makeatletter
% \renewcommand*\NamesFormat[1]{%^^A
%   \RevertSNfalse\SpecialFNfalse\SpecialSNfalse#1%^^A
%   \unless\ifinner\marginpar{%^^A
%     \footnotesize\raggedleft%^^A
%     \@nameauth@FullNametrue%^^A
%     \@nameauth@FirstNamefalse%^^A
%     \@nameauth@EastFNfalse%^^A
%     \SpecialFNtrue\SpecialSNfalse%^^A
%     \NameParser}%^^A
%   \fi\global\RevertSNfalse}
% \renewcommand*\MainNameHook[1]{%^^A
%   \AltOff\SpecialFNfalse\SpecialSNtrue#1%^^A
%   \unless\ifinner
%     \unless\ifRevertSN
%       \marginpar{%^^A
%       \footnotesize\raggedleft%^^A
%       \@nameauth@FullNamefalse%^^A
%       \@nameauth@FirstNamefalse%^^A
%       \@nameauth@EastFNfalse%^^A
%       \SpecialFNfalse\SpecialSNfalse%^^A
%       \NameParser}%^^A
%     \fi
%   \fi\global\RevertSNfalse}
% \makeatother
% \begin{quote}\small
% \StartNum
% \begin{verbatim}
%\begin{nameauth}
%  \< Shak & \noexpand\WM & \noexpand\SHK & >
%\end{nameauth}
%\PretagName[\noexpand\WM]{\noexpand\SHK}{Shakespeare, William}
%\PretagName[Robert]{\textSC{Burns}}{Burns, Robert}\end{verbatim}
% \end{quote}
%
% Now we define the flags by which the macros \cmd{\WM} and \cmd{\SHK} expand differently in the formatting hooks than in the index:
% \begin{quote}\small
% \ContinueNum
% \begin{verbatim}
%\newif\ifSpecialFN
%\newif\ifSpecialSN
%\newif\ifRevertSN
%\newcommand*\WM{\ifSpecialFN Wm.\else William\fi}
%\newcommand*\SHK{\ifRevertSN \textSC{Shakespeare}\else
%                 \ifSpecialSN \noexpand\AltCaps{t}he Bard\else
%                 \textSC{Shakespeare}\fi\fi}
%\newcommand*\Revert{\RevertSNtrue}
%\makeatletter\end{verbatim}
% \end{quote}
%
% Finally, we define the two formatting hooks that trigger these changes:
% \begin{quote}\small
% \ContinueNum
% \begin{verbatim}
%\renewcommand*\NamesFormat[1]{%
%  \RevertSNfalse\SpecialFNfalse\SpecialSNfalse#1%
%  \unless\ifinner\marginpar{%
%    \footnotesize\raggedleft%
%    \@nameauth@FullNametrue%
%    \@nameauth@FirstNamefalse%
%    \@nameauth@EastFNfalse%
%    \SpecialFNtrue\SpecialSNfalse%
%    \NameParser}%
%  \fi\global\RevertSNfalse}
%\renewcommand*\MainNameHook[1]{%
%  \AltOff\SpecialFNfalse\SpecialSNtrue#1%
%  \unless\ifinner
%    \unless\ifRevertSN
%      \marginpar{%
%      \footnotesize\raggedleft%
%      \@nameauth@FullNamefalse%
%      \@nameauth@FirstNamefalse%
%      \@nameauth@EastFNfalse%
%      \SpecialFNfalse\SpecialSNfalse%
%      \NameParser}%
%    \fi
%  \fi\global\RevertSNfalse}
%\makeatother\end{verbatim}
%
% \smallskip
% \Shak\ \cmd{\Shak} is the national poet of England in much the same way as \Name[Robert]{\textSC{Burns}} |\Name[Robert]{\textSC{Burns}}| is that of Scotland. With the latter's rise of influence in the 1800s, \Revert\Shak\ \cmd{\Revert}\cmd{\Shak} became known as ``\Shak'' \cmd{\Shak}.
% \end{quote}
% \endgroup
%
% First, we put macros \cmd{\WM} and \cmd{\SHK} in name arguments using \cmd{\noexpand}. That will make the index work properly. We use \cmd{\PretagName} to sort the names. We set up three flags. One is for \cmd{\WM} and two are for \cmd{\SHK}. \cmd{\Revert} is used to print a last name without a margin note.
%
%In the first-use hook we allow only the canonical name via \cmd{\RevertSNfalse}, \cmd{\SpecialFNfalse}, and \cmd{\SpecialSNfalse}. The default global formatting state is set by \cmd{\AltFormatActive}. We print the canonical name in the body text. If not in inner horizontal mode, we print a margin paragraph with an alternate full name using \cmd{\NameParser} and two flags. Both hooks set \cmd{\RevertSNfalse} so that \cmd{\Revert} works on a per-name basis. The subsequent-use hook disables formatting with \cmd{\AltOff}, but it allows variant forms.
%
% \begin{center}\bfseries Rolling Your Own: Advanced\end{center}
%
% Here\Warn{} is how formatting hooks were designed before version 3.0. Updating these older hooks is helpful, but may not be necessary. Here we do not use the internal package macros. We only use \cmd{\NameParser} in the hooks to produce output. We still recommend using \cmd{\AltFormatActive} to prevent problems with \cmd{\CapThis}.
%
% \newif\ifFbox
% \newif\ifFirstCap
% \newif\ifInHook
% \Fboxtrue
% Three flags replace package internals. \texttt{\textbackslash @nameauth@DoAlt} activates formatting; \texttt{\textbackslash @nameauth@DoCaps} is set by \cmd{\CapThis}; and \texttt{\textbackslash @nameauth@InHook} is set by the hook dispatcher. Setting \texttt{\textbackslash Fboxtrue} is equivalent to using \cmd{\AltFormatActive}:
% \begin{quote}\small
% \StartNum
% \begin{verbatim}
%\newif\ifFbox%		Replaces \@nameauth@DoAlt
%\newif\ifFirstCap%	Replaces \@nameauth@DoCaps
%\newif\ifInHook%		Replaces \@nameauth@InHook
%\Fboxtrue\end{verbatim}
% \end{quote}
%
% \renewcommand*\Fbox[1]{\ifFbox\protect\fbox{#1}\else#1\fi}
% The formatting macro is like what we have seen, except it refers to \texttt{\textbackslash ifFbox}:
% \begin{quote}\small
% \ContinueNum
% \begin{verbatim}
%\renewcommand*\Fbox[1]{%
%  \ifFbox\protect\fbox{#1}\else#1\fi
%}\end{verbatim}
% \end{quote}
%
% \renewcommand*\AltCaps[1]{\ifInHook
%     \ifFirstCap\MakeUppercase{#1}\else#1\fi
%   \else
%     #1\fi}
% Our new \cmd{\AltCaps} works like the built-in version, except it does not use the internal macros and flags:
% \begin{quote}\small
% \ContinueNum
% \begin{verbatim}
%\renewcommand*\AltCaps[1]{%
%  \ifInHook
%    \ifFirstCap\MakeUppercase{#1}\else#1\fi
%  \else
%    #1%
%  \fi
%}\end{verbatim}
% \end{quote}
%
% \renewcommand*\CapThis{\FirstCaptrue}
% \renewcommand*\NamesFormat[1]
%   {\InHooktrue\NameParser\global\FirstCapfalse}
% \renewcommand*\MainNameHook[1]
%   {\Fboxfalse\InHooktrue\NameParser\global\FirstCapfalse}
% \let\FrontNamesFormat\Namesformat
% \let\FrontNameHook\MainNameHook
% Here we redefine \cmd{\CapThis} to use our flag instead of the internal flag:
% \begin{quote}\small
% \ContinueNum
% \begin{verbatim}
%\renewcommand*\CapThis{\FirstCaptrue}\end{verbatim}
% \end{quote}
%
% We have to reproduce the logic and macros that the package would have provided. That means defining everything, including \cmd{\NamesFormat}, from scratch: 
% \begin{quote}\small
% \ContinueNum
% \begin{verbatim}
%\renewcommand*\NamesFormat[1]
%  {\InHooktrue\NameParser\global\FirstCapfalse}\end{verbatim}
% \end{quote}
%
% Changes to \texttt{\textbackslash ifInHook} (default false) and \texttt{\textbackslash ifFbox} (default true) are local to the scope in which the hook macros are called. \texttt{\textbackslash ifFirstCap} must be set globally. Below we reproduce the logic of \cmd{\AltOff} before \cmd{\NameParser}:
% \begin{quote}\small
% \ContinueNum
% \begin{verbatim}
%\renewcommand*\MainNameHook[1]
%{\Fboxfalse\InHooktrue\NameParser\global\FirstCapfalse}\end{verbatim}
% \end{quote}
%
% We avoid spurious index entries in the front matter by using the same hooks.
% \begin{quote}\small
% \ContinueNum
% \begin{verbatim}
%\let\FrontNamesFormat\Namesformat
%\let\FrontNameHook\MainNameHook\end{verbatim}
% \end{quote}
%
% Because we use \cmd{\noexpand}, our ``old-style'' macros will index the following names under the same entry as the ``new-style'' macros.
% \begin{center}\footnotesize
% \begin{tabular}{llll}\toprule
% First & Next & Long & Short \\\midrule
% |\deSmet| & |\deSmet| & |\LdeSmet| & |\SdeSmet|\\
% \ForgetThis\deSmet & \deSmet & \LdeSmet & \SdeSmet\\\bottomrule
% \end{tabular}
% \end{center}
%
% The capitalized version |\CapThis\deSmet| is \CapThis\deSmet. This also works for a formatted use via \cmd{\ForceName}: \ForceName\CapThis\deSmet.
%
% We can reuse\Warn{} new-style names with old-style macros, shown below in abbreviated fashion. We keep the flags \texttt{\textbackslash ifFirstCap} and \texttt{\textbackslash ifInHook}. We also keep the redefined \cmd{\AltCaps}, \cmd{\CapThis}, and \cmd{\NamesFormat}. We then add:
% \newif\ifCaps
% \Capstrue
% \renewcommand*\textSC[1]{\ifCaps\textsc{#1}\else#1\fi}
% \renewcommand*\MainNameHook[1]
%   {\Capsfalse\InHooktrue\NameParser\global\FirstCapfalse}
% \let\FrontNameHook\MainNameHook
% \begin{quote}\small
% \StartNum
% \begin{verbatim}
%\newif\ifCaps
%\Capstrue
%\renewcommand*\textSC[1]{%
%  \ifCaps\textsc{#1}\else#1\fi
%}
%\renewcommand*\MainNameHook[1]
%{%
%  \Capsfalse\InHooktrue\NameParser%
%  \global\FirstCapfalse%
%}
%\let\FrontNameHook\MainNameHook\end{verbatim}
% \end{quote}
%
% The names below have the same declarations and index entries as they did above. They look and work the same but use different back-end macros:
% \begin{center}
% \small\noindent\begin{tabular}{llll}\toprule
% First & Next & Long & Short \\\midrule
% \ForgetThis\Adams & \Adams & \LAdams & \SAdams\\
% \rowcolor{black!7!white}\ForgetThis\SDJR & \SDJR & \LSDJR & \SSDJR\\
% \ForgetThis\HAR & \HAR & \LHAR & \SHAR\\
% \rowcolor{black!7!white}\ForgetThis\Mencius & \Mencius & \LMencius & \SMencius\\\bottomrule
% \end{tabular}
% \end{center}
%
% \begin{itemize}\small
% \item Punctuation detection works: \ForceName\LSDJR. Also \LSDJR. Then \ForceName\SDJR. Now \SDJR. (We used \cmd{\ForceName} for formatting.)
% \item \cmd{\ForceName}\cmd{\DropAffix}\cmd{\LSDJR} gives \ForceName\DropAffix\LSDJR. Otherwise, only using the macro \cmd{\DropAffix}\cmd{\LSDJR} gives \DropAffix\LSDJR.
% \item \cmd{\RevComma}\cmd{\LAdams} yields \RevComma\LAdams. All the reversing macros work.
% \item \cmd{\ForceName}\cmd{\ForceFN}\cmd{\SHAR} produces \ForceName\ForceFN\SHAR. \cmd{\ForceFN}\cmd{\SHAR} produces \ForceFN\SHAR. If we add \cmd{\CapThis} we get \CapThis\ForceName\ForceFN\SHAR\ and \CapThis\ForceFN\SHAR.
% \end{itemize}
%
% We now resume normal formatting with \cmd{\AltFormatInactive} and close the scope that we began at the start of Section~\ref{sec:Hooksi}.
% \AltFormatInactive\endgroup
%
% \ReturnLink
% \newpage
%
% \subsubsection{Customization}
% \label{sec:Customize}
%
% Assuming\Warn{} that redefining hooks and adding control sequences is insufficient, one could redesign the core name macros partially or wholly, then hook those modifications into the \textsf{nameauth} package without needing to patch the style file itself.
%
% \DescribeMacro{\NameauthName}
% All these macros are set by default to \cmd{\@nameauth@Name}, the internal name parser.
% \DescribeMacro{\NameauthLName}
% \cmd{\Name}, or an unmodified shorthand, calls \cmd{\NameauthName}. \cmd{\Name*}, or an L-shorthand, sets \cmd{\@nameauth@FullNametrue}, then calls \cmd{\NameauthLName}.
% \DescribeMacro{\NameauthFName}
% \cmd{\FName}, or an S-shorthand, sets \cmd{\@nameauth@FirstNametrue}, then calls \cmd{\NameauthFName}. One should not modify \cmd{\Name} and \cmd{\FName} directly.
%
% Next we see a minimal working example that implements the obsolete syntax. We use few internal Boolean values, save those governing name forms. We do not implement short forms or any other features in \textsf{nameauth}. We must index the names with \cmd{\IndexName}. This example shows how to hook these redefined macros into the user interface:
%
% \begin{quote}\small
% \StartNum
% \begin{verbatim}
%\makeatletter
%\newcommandx*\MyName[3][1=\@empty, 3=\@empty]{%
%  \protected@edef\a{\trim@spaces{#1}}%
%  \protected@edef\b{\trim@spaces{#2}}%
%  \protected@edef\c{\trim@spaces{#3}}%
%  \ifx\b\empty fail \else
%    \ifx\a\empty
%      \ifx\c\empty \hbox to 5em{Mononym:\hfill} {\b}\else
%      \hbox to 5em{Eastern:\hfill} {\b\ \c}\fi
%    \else
%      \ifx\c\empty \hbox to 5em{Western:\hfill} {\a\ \b}\else
%      \hbox to 5em{Alternate:\hfill} {\c\ \b}\fi
%    \fi
%  \fi
%  \global\@nameauth@FullNamefalse%
%  \global\@nameauth@FirstNamefalse%
%}
%\makeatother
%\let\MyLName\MyName
%\let\MyFName\MyName
%\renewcommand*\NameauthName{\MyName}
%\renewcommand*\NameauthLName{\MyLName}
%\renewcommand*\NameauthFName{\MyFName}
%\IndexName[George]{Washington}
%\IndexName[M.T.]{Cicero}
%\IndexName{Dagobert}[I]
%\IndexName{Aristotle}\end{verbatim}
% \makeatletter
% \newcommandx*\MyName[3][1=\@empty, 3=\@empty]{%^^A
%   \protected@edef\a{\trim@spaces{#1}}%^^A
%   \protected@edef\b{\trim@spaces{#2}}%^^A
%   \protected@edef\c{\trim@spaces{#3}}%^^A
%   \ifx\b\empty fail \else
%     \ifx\a\empty
%       \ifx\c\empty \hbox to 5em{Mononym:\hfill} {\b}\else
%       \hbox to 5em{Eastern:\hfill} {\b\ \c}\fi
%     \else
%       \ifx\c\empty \hbox to 5em{Western:\hfill} {\a\ \b}\else
%       \hbox to 5em{Alternate:\hfill} {\c\ \b}\fi
%     \fi
%   \fi
%   \global\@nameauth@FullNamefalse%^^A
%   \global\@nameauth@FirstNamefalse%^^A
% }
% \makeatother
% \let\MyLName\MyName
% \let\MyFName\MyName
% \renewcommand*\NameauthName{\MyName}
% \renewcommand*\NameauthLName{\MyLName}
% \renewcommand*\NameauthFName{\MyFName}
% \IndexName[George]{Washington}
% \IndexName[M.T.]{Cicero}
% \IndexName{Dagobert}[I]
% \IndexName{Aristotle}\bigskip
%
% \begin{tabular}{rl}
%   \cmd{\Wash} & \Wash \\
%   \cmd{\Cicero[Marcus Tullius]} & \Cicero[Marcus Tullius] \\
%   \cmd{\Dagb} & \Dagb \\
%   \cmd{\Aris} & \Aris \\
% \end{tabular}
% \end{quote}
%
% The previous example is not particularly useful. There is, however, a more practical use for these macros. One could choose to implement additional features, then pass the information in the name argument token registers to the extant parsing macros of \textsf{nameauth} (cf. Section~\ref{sec:Hooksii}).
% \newpage
%
% Below we introduce formatting that is additional to, inter-operative with, yet distinct from the formatting hooks:
% \begin{quote}\small
% \StartNum
% \begin{verbatim}
%\makeatletter
%\newcommandx*\MyName[3][1=\@empty, 3=\@empty]{%
%  \@nameauth@toksa\expandafter{#1}%
%  \@nameauth@toksb\expandafter{#2}%
%  \@nameauth@toksc\expandafter{#3}%
%  \hbox to 4em{Normal: \hfill}%
%  \fcolorbox{black}{gray!25!white}{\@nameauth@Name[#1]{#2}[#3]}%
%}
%\newcommandx*\MyLName[3][1=\@empty, 3=\@empty]{%
%  \@nameauth@toksa\expandafter{#1}%
%  \@nameauth@toksb\expandafter{#2}%
%  \@nameauth@toksc\expandafter{#3}%
%  \hbox to 4em{Long: \hfill}%
%  \fcolorbox{black}{green!25!white}{\@nameauth@Name[#1]{#2}[#3]}%
%}
%\newcommandx*\MyFName[3][1=\@empty, 3=\@empty]{%
%  \@nameauth@toksa\expandafter{#1}%
%  \@nameauth@toksb\expandafter{#2}%
%  \@nameauth@toksc\expandafter{#3}%
%  \hbox to 4em{Short: \hfill}%
%  \fcolorbox{black}{yellow!25!white}{\@nameauth@Name[#1]{#2}[#3]}%
%}
%\makeatother
%\renewcommand*\NamesFormat[1]
%  {\hbox to 9em{\hfil\scshape#1\hfil}}
%\renewcommand*\MainNameHook[1]{\hbox to 9em{\hfil#1\hfil}}
%\renewcommand*\NameauthName{\MyName}
%\renewcommand*\NameauthLName{\MyLName}
%\renewcommand*\NameauthFName{\MyFName}\end{verbatim}
%
% \makeatletter
% \newcommandx*\MyName[3][1=\@empty, 3=\@empty]{%^^A
%   \@nameauth@toksa\expandafter{#1}%^^A
%   \@nameauth@toksb\expandafter{#2}%^^A
%   \@nameauth@toksc\expandafter{#3}%^^A
%   \hbox to 4em{Normal: \hfill}%^^A
%   \fcolorbox{black}{gray!25!white}{\@nameauth@Name[#1]{#2}[#3]}%^^A
% }
% \newcommandx*\MyLName[3][1=\@empty, 3=\@empty]{%^^A
%   \@nameauth@toksa\expandafter{#1}%^^A
%   \@nameauth@toksb\expandafter{#2}%^^A
%   \@nameauth@toksc\expandafter{#3}%^^A
%   \hbox to 4em{Long: \hfill}%^^A
%   \fcolorbox{black}{green!25!white}{\@nameauth@Name[#1]{#2}[#3]}%^^A
% }
% \newcommandx*\MyFName[3][1=\@empty, 3=\@empty]{%^^A
%   \@nameauth@toksa\expandafter{#1}%^^A
%   \@nameauth@toksb\expandafter{#2}%^^A
%   \@nameauth@toksc\expandafter{#3}%^^A
%   \hbox to 4em{Short: \hfill}%^^A
%   \fcolorbox{black}{yellow!25!white}{\@nameauth@Name[#1]{#2}[#3]}%^^A
% }
% \makeatother
% \renewcommand*\NamesFormat[1]{\hbox to 9em{\hfil\scshape#1\hfil}}
% \renewcommand*\MainNameHook[1]{\hbox to 9em{\hfil#1\hfil}}
% \renewcommand*\NameauthName{\MyName}%
% \renewcommand*\NameauthLName{\MyLName}%
% \renewcommand*\NameauthFName{\MyFName}
%
% \smallskip
% |\ForgetName[Adolf]{Harnack}|\ForgetName[Adolf]{Harnack}\\[1ex]
% \begin{tabular}{@{}rl}
% |\Harnack| & \Harnack\\
% |\LHarnack[Adolf von]| & \LHarnack[Adolf von]\\
% |\Harnack| & \Harnack\\
% |\SHarnack| & \SHarnack\\
% \end{tabular}
% \end{quote}
%
% After\Version{3.3} the name is printed in the body text, the internal macros \emph{globally} set \cmd{\@nameauth@FullNamefalse} and \cmd{\@nameauth@FirstNamefalse}, as well as other flags related to the prefix macros. This prevents certain cases of undocumented behavior in versions of \textsf{nameauth} before 3.3, where resetting flags locally could cause unexpected name forms. If an existing document leverages the local resetting of flags, one can use the \texttt{oldreset} option. Compare Section~\ref{sec:IndexControl}.\medskip
%
% Like\Info{\cmd{\global}} many of the macros in this package, these naming macros can be redefined or used locally within a scope without making global changes to the document unless you specifically use \cmd{\global}.
% 
% \ReturnLink
% \newpage
% \endgroup^^A End of hook macro redefinition.
%
% \subsection{Technical Notes}
% \label{sec:TechNotes}
% 
% \hfil This manual was created with
% \fbox{\mystrut\ \bfseries\ifxetex xelatex (pdf)%^^A
% \else
%   \ifluatex
%     \ifpdf lualatex (pdf)%^^A
%     \else lualatex (dvi)%^^A
%     \fi
%   \else
%     \ifpdf pdflatex%^^A
%     \else latex (dvi)%^^A
%     \fi
%   \fi
% \fi\ } 
%
% \ifDoTikZ\begin{tcolorbox}[colback=white,colframe=nared,adjusted title={\hfil Thanks}]\else
% \begin{center}\large\bfseries Thanks\end{center}\fi
% \noindent Thanks to \Name[Marc van]{Dongen}, \Name[Enrico]{Gregorio}, \Name[Philipp]{Stephani}, \Name*[Heiko]{Oberdiek}, \Name[Uwe]{Lueck}, \Name[Dan]{Luecking} and \Name[Robert]{Schlicht} for assistance in early versions of this package. Thanks also to users for valuable feedback.\vfil
% \ifDoTikZ\end{tcolorbox}\fi
% 
% \subsubsection{General}
% \label{sec:GenNotes}
% 
% About the package itself:
% \begin{itemize}
% \item For version 3.2 behavior, use both the \texttt{oldpass} and \texttt{oldreset} options.
% \item For version 2.6 behavior, use \texttt{oldpass}, \texttt{oldreset}, and \texttt{oldAKA}.
% \item The package works with both \texttt{xindy} and \texttt{makeindex}.
% \item Name\Version{3.0} output, index entry creation, and index cross-reference creation occur in independent modules.
% \item Use\Version{3.0} the \texttt{verbose} option for warnings about indexing.
% \item The \texttt{nameauth} environment always will emit warnings as needed.
% \item The\Version{2.6} \texttt{comma} option and the older syntax are no longer restrictive, save with \cmd{\AKA} and its derivatives. See Sections~\ref{sec:Obsolete}, \ref{sec:Affix}, and~\ref{sec:AKA}.
% \item No\Version{2.5} formatting is selected by default.
% \end{itemize}
% \noindent About the manual:
% \begin{itemize}
% \item This manual is the test suite.
% \item This manual is designed for both current and older \LaTeX\ distributions.
% \item This\Version{3.3} manual has been redesigned.
% \item It is compatible with both A4 and US letter formats.
% \item We mention when this manual changes package internals.
% \end{itemize}
% About package building:
% \begin{itemize}
% \item The \textsf{nameauth} package requires \textsf{etoolbox}, \textsf{suffix}, \textsf{trimspaces}, and \textsf{xargs}. The \texttt{dtx} file encoding is UTF-8; we assume Unicode support.
% \item We tested this release in \texttt{dvi} mode (\texttt{latex} and \texttt{dvilualatex}), and in \texttt{pdf} mode (\texttt{pdflatex}, \texttt{lualatex}, and \texttt{xelatex}). We used \texttt{makeindex}.
% \item This release has been tested on GNU/Linux (distro TL 2017 and vanilla TL 2019), and Windows (Mik\TeX, using GNU make via Cygwin.)
% \item The release uploaded to CTAN is generated using \texttt{pdflatex} in GNU/Linux.
% \end{itemize}
% \newpage
% 
% \subsubsection{Active Unicode}
% \label{sec:Unicode}
%
% With |\usepackage[T1]{fontenc}| we can use many active Unicode characters automatically.\footnote{As of release, most documents typeset with \texttt{latex} and \texttt{pdflatex} do not require explicit loading of either \textsf{inputenc} or \textsf{inputenx}.}
% We already covered using \cmd{\PretagName} to sort names with these characters (Section~\ref{sec:IndexSort}).
% Below we group by accents and diacritical marks:
%
% \begin{center}\small\setstretch{1.4}
% \def\Bullet#1{\setbox0\hbox{#1}\raise 0.4ex\hbox to\wd0{\hfil\tiny\textbullet\hfil}}
% \def\bullet#1{\setbox0\hbox{#1}\raise 0.2ex\hbox to\wd0{\hfil\tiny\textbullet\hfil}}
% \begin{tabular}{lll}\toprule
% acute & Á Ć É Ǵ \Bullet{H} Í Ĺ Ń Ó Ŕ Ś Ú Ý Ź & á ć é ǵ \bullet{h} í ĺ ń ó ŕ ś ú ý ź \strut\\
% \rowcolor{black!7!white}grave & À \Bullet{C} È \Bullet{G} \Bullet{H} Ì Ò Ù & à \bullet{c} è \bullet{g} \bullet{h} ì ò ù \strut\\
% circumflex & Â Ĉ Ê Ĝ Ĥ Î Ĵ Ô Ŝ Û Ŵ Ŷ  & â ĉ ê ĝ ĥ î ĵ ô ŝ û ŵ ŷ \strut\\
% \rowcolor{black!7!white}tilde & Ã \Bullet{C} \Bullet{E} \Bullet{G} \Bullet{H} Ĩ Ñ Õ Ũ & ã \bullet{c} \bullet{e} \bullet{g} \Bullet{h} ĩ ñ õ ũ \strut\\
% diaresis\footnotemark & Ä \Bullet{C} Ë \Bullet{G} \Bullet{H} Ï Ö Ü Ÿ & ä \bullet{c} ë \bullet{g} \bullet{h} ï ö ü ÿ \strut\\
% \rowcolor{black!7!white}cedilla & \Bullet{A} Ç \Bullet{E} Ģ Ķ Ļ Ņ Ŗ Ş Ţ  & \bullet{a} ç \bullet{e} ģ ķ ļ ņ ŗ ş ţ \strut\\
% macron & Ā \Bullet{C} Ē Ḡ \Bullet{H} Ī Ō Ū Ǣ Ȳ & ā \bullet{c} ē ḡ \bullet{h} ī ō ū ǣ ȳ \strut\\
% \rowcolor{black!7!white}breve & Ă \Bullet{C} \Bullet{E} Ğ \Bullet{H} Ĭ Ŏ Ŭ & ă \bullet{c} \bullet{e} ğ \bullet {h} ĭ ŏ ŭ \strut\\
% dot\,/\,dotless & Ḃ Ċ Ė Ġ \Bullet{H} İ Ż & ḃ ċ ė ġ \bullet{h} ı ż \strut\\
% \rowcolor{black!7!white}ogonek & Ą \Bullet{C} Ę \Bullet{G} \Bullet{H} Į Ǫ Ų & ą \bullet{c} ę \bullet{g} \bullet{h} į ǫ ų \strut\\
% caron & Ǎ Č Ď Ě Ǧ Ǐ Ǩ Ľ Ň Ǒ Ř Š Ť Ǔ Ž & ǎ č ď ě ǧ ǐ ǰ ǩ ľ ň ǒ ř š ť ǔ ž \strut\\
% \rowcolor{black!7!white}various & Å Æ Ð (eth) Đ (stroke) IJ\ Ł Ŋ & å æ ð đ ij\ ł ŋ \strut\\
% \rowcolor{black!7!white}        & Ø Œ Ő Ů Ű Ș Ț Þ & ø œ ő ů ű ș ß ț þ \strut\\\bottomrule
% \end{tabular}
% \footnotetext{A diaresis mark is one way to indicate an umlaut, a sound change. German originally used a superscript \textsf{e} over \textsf{a}, \textsf{o}, and \textsf{u}. The cursive form of \textsf{e} simplified to a diaresis mark in the 1800s. A diaresis mark also signals a diaresis: reading a diphthong as two monophthongs.}
% \end{center}
% 
% Additional Unicode characters can be made available when using fonts with TS1 glyphs (pages 455--463 in \textit{The Latex Companion}). Compare the list: \url{http://tug.ctan.org/info/symbols/comprehensive/} or \texttt{texdoc comprehensive}.
%
% When using a font with TS1 glyphs and slots, the following preamble snippet lets one add more Unicode characters. That enables one to write, ``In Congreſs, July 4, 1776'' as |``In Congreſs, July 4, 1776''|:
% \begin{quote}\small
% \StartNum
% \begin{verbatim}
%\usepackage[utf8]{inputenc} % For older TL releases
%\usepackage[TS1,T1]{fontenc}
%\usepackage{lmodern}% Contains TS1 glyph 115
%\usepackage{newunicodechar}
%\DeclareTextSymbolDefault{\textlongs}{TS1}
%\DeclareTextSymbol{\textlongs}{TS1}{115}
%\newunicodechar{ſ}{\textlongs}\end{verbatim}
% \end{quote}
% 
% Many\Warn{} Unicode characters have native support in \texttt{xelatex} and \texttt{lualatex}, but not in \texttt{pdflatex}. Yet the latter has certain features (e.g., with respect to \textsf{microtype}) that others lack. The features of \texttt{makeindex} do not always equate to those in \texttt{xindy}. Those differences impact design choices.
%
% Before\Warn{} 2018, some index styles excluded characters with macrons, e.g., \textsf{ā}. Even now, control sequences like \cmd{\=a} in the index create undocumented behavior when using \texttt{makeindex} and \texttt{gind.ist}, which changes the ``actual'' character from~\texttt{@} to~\texttt{=}. Since 2018, names like \ifPDFTeX\IfFileExists{utf8-2018.def}{\Name{Ghazāli}}{\Name{Ghazali}}\else\Name{Ghazāli}\fi\ work properly due to new Unicode conventions. We allow for backward compatibility using the \textsf{iftex} package thus: 
% \begin{quote}\small
% \StartNum
% \begin{verbatim}
%\ifPDFTeX
%  \IfFileExists{utf8-2018.def}%
%    {\Name{Ghazāli}}{\Name{Ghazali}}%
%  \else\Name{Ghazāli}%
%\fi\end{verbatim}
% \end{quote}
%
% \TeX\Warn{} macros that partition their arguments can break active Unicode characters. Consider the simple macro |\def\foo#1#2#3!{<#1#2><#3>}|. It takes three undelimited arguments and groups the first two, then the third:
%
% \begin{center}\MyStretch
% \def\foo#1#2#3!{<#1#2><#3>}
% \begin{tabular}{llll}\toprule
% Argument & Macro & Engine & Result\\\midrule
% |abc| & |\foo abc!| & (any) & \foo abc!\\
% |{æ}bc| & |\foo {æ}bc!| & (any) & \foo {æ}bc!\\
% |\ae bc| & |\foo \ae bc!| & (any) & \foo \ae bc!\\
% \rowcolor{black!7!white}|æbc| & |\foo æbc!| & \texttt{xelatex} & \ifxetex\foo æbc!\else<æb><c>\fi\\
% \rowcolor{black!7!white}|æbc| & |\foo æbc!| & \texttt{lualatex} & \ifluatex\foo æbc!\else<æb><c>\fi\\
% |æbc| & |\foo æbc!| & \texttt{pdflatex} & \unless\ifxetex\unless\ifluatex\foo æbc!\else<æ><bc>\fi\else<æ><bc>\fi\\
% |æbc| & |\foo æbc!| & \texttt{latex} & \unless\ifxetex\unless\ifluatex\foo æbc!\else<æ><bc>\fi\else<æ><bc>\fi\\\bottomrule
% \end{tabular}
% \end{center}
%
% The letter \texttt{a} is one argument. Since \texttt{\{æ\}} is in a group, it is one argument. The macro \cmd{\ae} also is one argument. Thus, the first two glyphs are grouped together in |#1#2| and \texttt{c} is left by itself in |#3|. Both \texttt{xelatex} and \texttt{lualatex} likewise treat the Unicode letter \texttt{æ} as one argument.
% 
% In \texttt{latex} and \texttt{pdflatex}, however, \texttt{æ} is an active Unicode control sequence that uses two arguments: |#1#2|. The tail of the input, \texttt{bc}, is crowded into |#3|.  Any macro where this |#1#2| pair is divided into |#1| and |#2| will produce one of two errors: \texttt{Unicode char \dots not set up for LaTeX} or \texttt{Argument of \textbackslash UTFviii@two@octets has an extra \}}.
%
% We\Version{3.0} test if \cmd{\Umathchar} is not defined. If so, we check if the leading token of the argument matches the start of an active Unicode control sequence: If \cmd{\@car}\meta{test}\cmd{\@nil} is equal to \cmd{\@car ß}\cmd{\@nil} (page~\pageref{page:CapSystem}) we capitalize |#1#2|, otherwise just |#1|. Should |#1| be a protected macro or something that does not expand to a sequence of letters, we use alternate formatting and \cmd{\AltCaps} (Section~\ref{sec:AltAdvanced}).
%
% A\Warn{} macro defined like |\edef\foo{\CapThis\Name{bar}}| will fail. However, |\CapThis\Name{bar}| can be an argument to a macro defined with \cmd{\edef} or \cmd{\xdef}.
%
% \LaTeX\Warn{} removes spaces between undelimited macro arguments, except the trailing argument. We use \cmd{\trim@spaces} to address this in \textsf{nameauth}. Explicit spacing macros change the results, but also require sorting with \cmd{\PretagName}. See also Sections~\ref{sec:ErrorProt} and~\ref{sec:NameParticles}, as well as Section~\ref{sec:IndexSort}.
%
% \ReturnLink
% \newpage
% 
% \subsubsection{\LaTeX\ Engines}
% \label{sec:TeXengines}
% 
% The following preamble snippet lets one build \textsf{nameauth} also with older TL versions. We do not load \texttt{iftex.sty} if it does not exist. We load the transitional packages when \textsf{iftex} is absent or older than 2019:\footnote{A copy of this example is in \texttt{examples.tex}, collocated with this manual.}
% \begin{quote}\small
% \StartNum
% \begin{verbatim}
%\IfFileExists{iftex.sty}{\usepackage{iftex}}{}
%\unless\ifdefined\RequireTUTeX
%  \usepackage{ifxetex}
%  \usepackage{ifluatex}
%  \usepackage{ifpdf}
%\fi\end{verbatim}
% \end{quote}
%
% Next we test for the \LaTeX\ engine and include packages accordingly. We could just include \textsf{inputenc} either way, but we are illustrating a point about testing. Some statements below should be modified, depending on one's workflow.
% \begin{quote}\small
% \ContinueNum
% \begin{verbatim}
%\newif\ifDoTikZ                        % Perhaps not needed
%\ifxetex
%  \usepackage{fontspec}
%  \usepackage{polyglossia}
%  \setdefaultlanguage{american}        % Use own language
%  \usepackage{tikz}
%  \DoTikZtrue	                         % Perhaps not needed
%\else
%  \ifluatex
%    \ifpdf
%      \usepackage{fontspec}
%      \usepackage{polyglossia}
%      \setdefaultlanguage{american}    % Use own language
%      \usepackage{tikz}
%      \DoTikZtrue                      % Perhaps not needed
%    \else
%      \IfFileExists{utf8-2018.def}{}
%      {\usepackage[utf8]{inputenc}}
%      \usepackage[TS1,T1]{fontenc}
%      \usepackage[american]{babel}     % Use own language
%      \usepackage{lmodern}
%      % Perhaps add \usepackage{tikz}
%    \fi
%  \else
%    \IfFileExists{utf8-2018.def}{}
%    {\usepackage[utf8]{inputenc}}
%    \usepackage[TS1,T1]{fontenc}
%    \usepackage[american]{babel}       % Use own language
%    \usepackage{lmodern}
%    \ifpdf                             % Perhaps not needed
%      \usepackage{tikz}
%      \DoTikZtrue                      % Perhaps not needed
%    \fi
%  \fi
%\fi\end{verbatim}
% \end{quote}
% \newpage
% 
% For the sake of comparing \texttt{dvi} viewers \texttt{xdvi}, \texttt{yap}, and others, we load \textsf{tikz} only when making a \texttt{pdf} because some \texttt{dvi} viewers crash otherwise. This may be wholly unnecessary in a \texttt{dvips} workflow or the like. With \textsf{fontspec}, Latin Modern is the default. If we only make \texttt{pdf} documents, the test simplifies to testing for \cmd{\Umathchar}, then loading either \textsf{fontspec} (success) or \textsf{fontenc} (failure).
%
%In the body text we can use something like the test below for \fbox{\ifDoTikZ doing \texttt{pdf} things\else doing \texttt{dvi} things\fi}
% \begin{quote}\small
% \begin{verbatim}
%\ifDoTikZ
%  doing \texttt{pdf} things\else
%  doing \texttt{dvi} things\fi\end{verbatim}
% \end{quote}
%
% The following equivalent conditional statements can help a macro or just the body text to work under multiple engines:
% \begin{quote}\small
% \StartNum
% \begin{verbatim}
%\ifxetex xelatex%
%\else
%  \ifluatex
%    \ifpdf lualatex (pdf)%
%    \else lualatex (dvi)%
%    \fi
%  \else
%    \ifpdf pdflatex%
%    \else latex (dvi)%
%    \fi
%  \fi
%\fi\end{verbatim}
% \end{quote}
% 
% \begin{quote}\small
% \StartNum
% \begin{verbatim}
%\unless\ifxetex
%  \unless\ifluatex
%    \ifpdf pdflatex%
%    \else latex (dvi)%
%    \fi
%  \else
%    \ifpdf lualatex (pdf)%
%    \else lualatex (dvi)%
%    \fi
%  \fi
%\else xelatex%
%\fi\end{verbatim}
% \end{quote}
% 
% \ReturnLink
% \BigBlank
% \newpage
% 
% \StopEventually{^^A
%  \let\emph\oldemph
%  \newgeometry{textwidth=160mm,textheight=237mm,right=25mm}
%  \IndexProtect\PrintChanges\newpage\PrintIndex
% }
%
% \section{Implementation}
%
% \iffalse
%<*package>
% \fi
% \small
% \subsection{Flags and Registers}
%
% The flags below are grouped according to function. We begin with flow control\medskip
%
% \noindent{\large\bfseries Who Called Me?}\medskip\\
% Various macros use these flags to protect against stack overflows or choose the right output.
%    \begin{macrocode}
\newif\if@nameauth@InAKA
\newif\if@nameauth@InName
\newif\if@nameauth@Xref
%    \end{macrocode}\smallskip
%
% \noindent{\large\bfseries Core Macro Locks}\medskip\\
% The macros \cmd{\@nameauth@Name} and \cmd{\AKA}, with some auxiliary macros, process names in a ``locked'' state to avoid a stack overflow. The \texttt{BigLock} always locks the macros, preventing execution. See also Sections~\ref{sec:Hooksii} and~\ref{sec:Hooksiii}.
%    \begin{macrocode}
\newif\if@nameauth@Lock
\newif\if@nameauth@BigLock
\newif\if@nameauth@InHook
%    \end{macrocode}\smallskip
%
% \noindent{\large\bfseries Indexing}\medskip\\
% The indexing flags permit or prevent indexing and tags. \cmd{\IndexActive} and \cmd{\IndexInctive} or the \texttt{index} and \texttt{noindex} options toggle the first flag; \cmd{\SkipIndex} toggles the second. \cmd{\JustIndex} toggles the third, which makes the core naming engine act like a call to \cmd{\IndexName}:
%    \begin{macrocode}
\newif\if@nameauth@DoIndex
\newif\if@nameauth@SkipIndex
\newif\if@nameauth@JustIndex
%    \end{macrocode}
% The \texttt{pretag} and \texttt{nopretag} options toggle the flag below, which allows or prevents the insertion of index sort keys.
%    \begin{macrocode}
\newif\if@nameauth@Pretag
%    \end{macrocode}
% This flag determines whether \cmd{\IndexRef} creates a \textit{see} reference or a \textit{see also} reference.
%    \begin{macrocode}
\newif\if@nameauth@SeeAlso
%    \end{macrocode}\smallskip
%
% \noindent{\large\bfseries Formatting}\medskip\\
% \cmd{\NamesActive} and \cmd{\NamesInactive}, with the \texttt{mainmatter} and \texttt{frontmatter} options, toggle formatting hooks via \texttt{\textbackslash if@nameauth@MainFormat}. \texttt{\textbackslash if@nameauth@AKAFormat} permits \cmd{\AKA} to call the first-use hooks once.
%    \begin{macrocode}
\newif\if@nameauth@MainFormat
\newif\if@nameauth@AKAFormat
%    \end{macrocode}\medskip
% The next flag works with \cmd{\LocalNames} and \cmd{\GlobalNames}.
%    \begin{macrocode}
\newif\if@nameauth@LocalNames
%    \end{macrocode}\medskip
% These two flags are used only for backward compatibility. The first broadly determines how per-name flags are reset, while the second affects the behavior of \cmd{\JustIndex}.
%    \begin{macrocode}
\newif\if@nameauth@OldReset
\newif\if@nameauth@OldPass
%    \end{macrocode}\medskip
% These two flags trigger \cmd{\ForgetName} and \cmd{\SubvertName} within \cmd{\@nameauth@Name}.
%    \begin{macrocode}
\newif\if@nameauth@Forget
\newif\if@nameauth@Subvert
%    \end{macrocode}\medskip
% \texttt{\textbackslash if@nameauth@FirstFormat} triggers the first-use hooks to be called; otherwise the second-use hooks are called. Additionally, \texttt{\textbackslash if@nameauth@AlwaysFormat} forces this true, except when \texttt{\textbackslash if@nameauth@AKAFormat} is false.
%    \begin{macrocode}
\newif\if@nameauth@FirstFormat
\newif\if@nameauth@AlwaysFormat
%    \end{macrocode}\medskip
%
% \noindent Next we move from general flow control to specific modification of name forms.\medskip
%
% \noindent{\large\bfseries Affix Commas}\medskip\\
% The \texttt{comma} and \texttt{nocomma} options toggle the first flag value below. \cmd{\ShowComma} and \cmd{\NoComma} respectively toggle the second and third.
%    \begin{macrocode}
\newif\if@nameauth@AlwaysComma
\newif\if@nameauth@ShowComma
\newif\if@nameauth@NoComma
%    \end{macrocode}\medskip
%
% \noindent{\large\bfseries Name Breaking}\medskip\\
% \cmd{\KeepAffix} toggles the first flag below, while \cmd{\KeepName} toggles the second. Both affect the use of non-breaking spaces in the text.
%    \begin{macrocode}
\newif\if@nameauth@NBSP
\newif\if@nameauth@NBSPX
%    \end{macrocode}\medskip
%
% \noindent{\large\bfseries Detect Punctuation}\medskip\\
% This Boolean value is used to prevent double full stops at the end of a name in the text.
%    \begin{macrocode}
\newif\if@nameauth@Punct
%    \end{macrocode}\medskip
%
% \noindent{\large\bfseries Long and Short Names}\medskip\\
% \texttt{\textbackslash if@nameauth@FullName} is true for a long name reference. \texttt{\textbackslash if@nameauth@FirstName} disables full-name references and causes only Western forenames to be displayed. The default is to reset both globally on a per-name basis.
%
% \texttt{\textbackslash if@nameauth@AltAKA} is toggled respectively by \cmd{\AKA} and \cmd{\AKA*} to print a longer or shorter name. \texttt{\textbackslash if@nameauth@OldAKA} forces the pre-3.0 behavior of \cmd{\AKA*}.
%
% \texttt{\textbackslash if@nameauth@ShortSNN} is used with \cmd{\DropAffix} to suppress the affix of a Western name. \texttt{\textbackslash if@nameauth@EastFN} toggles the forced printing of Eastern forenames.
%    \begin{macrocode}
\newif\if@nameauth@FullName
\newif\if@nameauth@FirstName
\newif\if@nameauth@AltAKA
\newif\if@nameauth@OldAKA
\newif\if@nameauth@ShortSNN
\newif\if@nameauth@EastFN
%    \end{macrocode}\medskip
%
% \noindent{\large\bfseries Eastern Names}\medskip\\
% The next flags values govern name reversing and full surname capitalization. The first of each pair is a global state. The second of each pair is an individual state.
%    \begin{macrocode}
\newif\if@nameauth@RevAll
\newif\if@nameauth@RevThis
\newif\if@nameauth@AllCaps
\newif\if@nameauth@AllThis
%    \end{macrocode}\medskip
%
% \noindent{\large\bfseries Last-Comma-First}\medskip\\
% This pair of flags deals with Western names reordered in a list according to surname.
%    \begin{macrocode}
\newif\if@nameauth@RevAllComma
\newif\if@nameauth@RevThisComma
%    \end{macrocode}\medskip
%
% \noindent{\large\bfseries Cap First Letter and Format}\medskip\\
% The next flags deal with first-letter capitalization. \cmd{\CapThis} sets the first Boolean value. The second is triggered by \cmd{\@nameauth@UTFtest} when it encounters an active Unicode character. The third is a fallback triggered by \cmd{\AccentCapThis}. The fourth disables \cmd{\CapThis} for alternate formatting. The fifth toggles alternate formatting.
%    \begin{macrocode}
\newif\if@nameauth@DoCaps
\newif\if@nameauth@UTF
\newif\if@nameauth@Accent
\newif\if@nameauth@AltFormat
\newif\if@nameauth@DoAlt
%    \end{macrocode}\medskip
%
% \noindent{\large\bfseries Warning Levels}\medskip\\
% This flag controls how many warnings you get. Defaults to few warnings. Verbose gives you plenty of warnings about cross-references in the index.
%    \begin{macrocode}
\newif\if@nameauth@Verbose
%    \end{macrocode}\medskip
%
% \noindent{\large\bfseries Name Argument Token Registers}\vspace{-1.5ex}
% \begin{macro}{\@nameauth@toksa}
% \begin{macro}{\@nameauth@toksb}
% \begin{macro}{\@nameauth@toksc}
% These three token registers contain the current values of the name arguments passed to \cmd{\Name}, its variants, and the cross-reference arguments of \cmd{\AKA}. Users can access them especially in formatting hooks.
%    \begin{macrocode}
\newtoks\@nameauth@toksa%
\newtoks\@nameauth@toksb%
\newtoks\@nameauth@toksc%
%    \end{macrocode}
% \end{macro}
% \end{macro}
% \end{macro}
% These three token registers contain the current values of the name arguments in each line of the \texttt{nameauth} environment.
%    \begin{macrocode}
\newtoks\@nameauth@etoksb%
\newtoks\@nameauth@etoksc%
\newtoks\@nameauth@etoksd%
%    \end{macrocode}
%
% \subsection{Hooks}
% \begin{macro}{\NamesFormat}
% Post-process ``first'' instance of final complete name form in text. See Sections~\ref{sec:Formatting} and~\ref{sec:Hooksi}f. Called when both |\@nameauth@MainFormat| and |\@nameauth@FirstFormat| are true.
%    \begin{macrocode}
\newcommand*\NamesFormat{}
%    \end{macrocode}
% \end{macro}
% \begin{macro}{\MainNameHook}
% \changes{2.4}{2016/03/15}{Added}
% Post-process subsequent instance of final complete name form in main-matter text. See Sections~\ref{sec:Formatting} and~\ref{sec:Hooksi}f. Called when |\@nameauth@MainFormat| is true and the Boolean flag |\@nameauth@FirstFormat| is false.
%    \begin{macrocode}
\newcommand*\MainNameHook{}
%    \end{macrocode}
% \end{macro}
% \begin{macro}{\FrontNamesFormat}
% \changes{2.5}{2016/04/06}{Added}
% Post-process ``first'' instance of final complete name form in front-matter text. Called when |\@nameauth@MainFormat| is false and |\@nameauth@FirstFormat| is true.
%    \begin{macrocode}
\newcommand*\FrontNamesFormat{}
%    \end{macrocode}
% \end{macro}
% \begin{macro}{\FrontNameHook}
% \changes{2.4}{2016/03/15}{Added}
% Post-process subsequent instance of final complete name form in front-matter text. Called when |\@nameauth@MainFormat| is false and |\@nameauth@FirstFormat| is false.
%    \begin{macrocode}
\newcommand*\FrontNameHook{}
%    \end{macrocode}
% \end{macro}
% \begin{macro}{\NameauthName}
% \changes{2.2}{2015/12/01}{Added}
% The last three hooks usually point to \cmd{\@nameauth@Name}. See Section~\ref{sec:Customize}.
%    \begin{macrocode}
\newcommand*\NameauthName{\@nameauth@Name}
%    \end{macrocode}
% \end{macro}
% \begin{macro}{\NameauthLName}
% \changes{2.3}{2016/01/05}{Added}
% Customization hook called after \cmd{\@nameauth@FullName} is set true. See Section~\ref{sec:Customize}.
%    \begin{macrocode}
\newcommand*\NameauthLName{\@nameauth@Name}
%    \end{macrocode}
% \end{macro}
% \begin{macro}{\NameauthFName}
% \changes{2.2}{2015/12/01}{Added}
% Customization hook called after \cmd{\@nameauth@FirstName} is set true. See Section~\ref{sec:Customize}.
%    \begin{macrocode}
\newcommand*\NameauthFName{\@nameauth@Name}
%    \end{macrocode}
% \end{macro}
%
% \subsection{Package Options}
% The following package options interact with many of the prior Boolean values.
%    \begin{macrocode}
\DeclareOption{comma}{\@nameauth@AlwaysCommatrue}
\DeclareOption{nocomma}{\@nameauth@AlwaysCommafalse}
\DeclareOption{mainmatter}{\@nameauth@MainFormattrue}
\DeclareOption{frontmatter}{\@nameauth@MainFormatfalse}
\DeclareOption{formatAKA}{\@nameauth@AKAFormattrue}
\DeclareOption{oldAKA}{\@nameauth@OldAKAtrue}
\DeclareOption{oldreset}{\@nameauth@OldResettrue}
\DeclareOption{oldpass}{\@nameauth@OldPasstrue}
\DeclareOption{index}{\@nameauth@DoIndextrue}
\DeclareOption{noindex}{\@nameauth@DoIndexfalse}
\DeclareOption{pretag}{\@nameauth@Pretagtrue}
\DeclareOption{nopretag}{\@nameauth@Pretagfalse}
\DeclareOption{allcaps}{\@nameauth@AllCapstrue}
\DeclareOption{normalcaps}{\@nameauth@AllCapsfalse}
\DeclareOption{allreversed}%
  {\@nameauth@RevAlltrue\@nameauth@RevAllCommafalse}
\DeclareOption{allrevcomma}%
  {\@nameauth@RevAllfalse\@nameauth@RevAllCommatrue}
\DeclareOption{notreversed}%
  {\@nameauth@RevAllfalse\@nameauth@RevAllCommafalse}
\DeclareOption{alwaysformat}{\@nameauth@AlwaysFormattrue}
\DeclareOption{smallcaps}{\renewcommand*\NamesFormat{\scshape}}
\DeclareOption{italic}{\renewcommand*\NamesFormat{\itshape}}
\DeclareOption{boldface}{\renewcommand*\NamesFormat{\bfseries}}
\DeclareOption{noformat}{\renewcommand*\NamesFormat{}}
\DeclareOption{verbose}{\@nameauth@Verbosetrue}
\DeclareOption{altformat}{%
  \@nameauth@AltFormattrue\@nameauth@DoAlttrue}
\ExecuteOptions%
  {nocomma,mainmatter,index,pretag,%
   normalcaps,notreversed,noformat}
\ProcessOptions\relax
%    \end{macrocode}
%
% Now we load the required packages. They facilitate the first\,/\,subsequent name uses, the parsing of arguments, and the implementation of starred forms.
%    \begin{macrocode}
\RequirePackage{etoolbox}
\RequirePackage{suffix}
\RequirePackage{trimspaces}
\RequirePackage{xargs}
%    \end{macrocode}
%
% The \textsf{etoolbox} package is essential for processing name control sequences. Using \textsf{xargs} allows the optional arguments to work. Using \textsf{suffix} facilitated the starred form of macros. Finally, \textsf{trimspaces} helps the fault tolerance of name arguments.
% \BigBlank
% \newpage
%
% \subsection{Internal Macros}
% \label{sec:InternalMacros}
%
% \noindent{\large\bfseries Name Control Sequence: Who Am I?}
% \begin{macro}{\@nameauth@Clean}
% Thanks to \Name*[Heiko]{Oberdiek}, this macro produces a ``sanitized'' string used to make a (hopefully) unique control sequence for a name. We can test the existence of that control string to determine first occurrences of a name or cross-reference.
%    \begin{macrocode}
\newcommand*\@nameauth@Clean[1]
  {\expandafter\zap@space\detokenize{#1} \@empty}
%    \end{macrocode}
% \end{macro}\medskip
%
% \noindent{\large\bfseries Parsing: Root and Suffix}
% \begin{macro}{\@nameauth@Root}
% The following two macros return everything before a comma in \meta{SNN}.
%    \begin{macrocode}
\newcommand*\@nameauth@Root[1]{\@nameauth@@Root#1,\\}
%    \end{macrocode}
% \end{macro}
%
% \begin{macro}{\@nameauth@@Root}
% \changes{2.0}{2015/11/11}{Trim spaces}
% \changes{3.0}{2016/10/26}{Redesigned}
% \changes{3.2}{2017/03/22}{Renamed}
% Throw out the comma and suffix, return the radix.
%    \begin{macrocode}
\def\@nameauth@@Root#1,#2\\{\trim@spaces{#1}}
%    \end{macrocode}
% \end{macro}
%
% \begin{macro}{\@nameauth@TrimTag}
% \changes{3.0}{2016/10/26}{Added}
% \changes{3.2}{2017/03/22}{Renamed}
% The following two macros return everything before a vertical bar (\verb+|+) in an index tag.
%    \begin{macrocode}
\newcommand*\@nameauth@TrimTag[1]{\@nameauth@@TrimTag#1|\\}
%    \end{macrocode}
% \end{macro}
%
% \begin{macro}{\@nameauth@@TrimTag}
% \changes{3.0}{2016/10/26}{Added}
% Throw out the bar and suffix, return the radix.
%    \begin{macrocode}
\def\@nameauth@@TrimTag#1|#2\\{#1}
%    \end{macrocode}
% \end{macro}
%
% \begin{macro}{\@nameauth@Suffix}
% \changes{0.9}{2012/02/10}{Added}
% The following two macros parse \meta{SNN} into a radix and a comma-delimited suffix, returning only the suffix after a comma in the argument, or nothing.
%    \begin{macrocode}
\newcommand*\@nameauth@Suffix[1]{\@nameauth@@Suffix#1,,\\}
%    \end{macrocode}
% \end{macro}
%
% \begin{macro}{\@nameauth@@Suffix}
% \changes{0.9}{2012/02/10}{Added}
% \changes{1.5}{2013/02/22}{Trim spaces}
% \changes{3.0}{2016/10/26}{New test}
% \changes{3.2}{2017/03/22}{Renamed}
% Throw out the radix; return the suffix with no leading spaces. We use this when printing the suffix.
%    \begin{macrocode}
\def\@nameauth@@Suffix#1,#2,#3\\%
  {\ifx\\#2\\\@empty\else\trim@spaces{#2}\fi}
%    \end{macrocode}
% \end{macro}
%
% \begin{macro}{\@nameauth@GetSuff}
% \changes{3.2}{2017/03/22}{Added}
% The following two macros just grab the suffix for testing if the first non-space character is an active character from \textsf{inputenc}.
%    \begin{macrocode}
\newcommand*\@nameauth@GetSuff[1]{\@nameauth@@GetSuff#1,,\\}
%    \end{macrocode}
% \end{macro}
%
% \begin{macro}{\@nameauth@@GetSuff}
% \changes{3.2}{2017/03/22}{Added}
% Throw out the radix; return the suffix.
%    \begin{macrocode}
\def\@nameauth@@GetSuff#1,#2,#3\\{#2}
%    \end{macrocode}
% \end{macro}
%
% \noindent{\large\bfseries Parsing: Capitalization}
% \begin{macro}{\@nameauth@TestToks}
% \phantomsection
% \label{page:CapSystem}
% \changes{3.2}{2017/03/22}{Added}
% Test if the leading token is the same as the leading token of an active Unicode character, using an \textit{Esszett} (\texttt{ß}) as the control. We only run this macro if we are in the \textsf{inputenc} regime.
%    \begin{macrocode}
\newcommand*\@nameauth@TestToks[1]
{%
  \toks@\expandafter{\@car#1\@nil}%
  \edef\one{\the\toks@}%
  \toks@\expandafter{\@carß\@nil}%
  \edef\two{\the\toks@}%
  \ifx\one\two\@nameauth@UTFtrue\else\@nameauth@UTFfalse\fi
}
%    \end{macrocode}
% \end{macro}
%
% \begin{macro}{\@nameauth@UTFtest}
% \changes{3.0}{2016/10/26}{Added}
% \changes{3.1}{2017/01/13}{Override bypasses test}
% \changes{3.2}{2017/03/22}{Non-suffix only}
% Before we attempt at capitalizing anything, we need to determine if we are running under \texttt{xelatex} or \texttt{lualatex} by testing for \cmd{\Umathchar}. Then we see if \textsf{inputenc} is loaded. We set up the comparison and pass off to \cmd{\@nameauth@TestToks}.
%    \begin{macrocode}
\newcommand*\@nameauth@UTFtest[1]
{%
  \def\testarg{#1}%
  \ifdefined\Umathchar
    \@nameauth@UTFfalse%
  \else
    \ifdefined\UTFviii@two@octets
      \if@nameauth@Accent
        \@nameauth@UTFtrue\@nameauth@Accentfalse%
      \else
        \expandafter\@nameauth@TestToks\expandafter{\testarg}%
      \fi
    \else
      \@nameauth@UTFfalse%
    \fi
  \fi
}
%    \end{macrocode}
% \end{macro}
%
% \begin{macro}{\@nameauth@UTFtestS}
% \changes{3.2}{2017/03/22}{Added}
% This test is like the one above, but a special case when we have a suffix. We have to do a bit more in the way of expansion to get the comparison to work properly. Moreover, we only use this when the regular suffix macro is not \cmd{\@empty}.
%    \begin{macrocode}
\newcommand*\@nameauth@UTFtestS[1]
{%
  \let\ex\expandafter%
  \ex\def\ex\testarg\ex{\@nameauth@GetSuff{#1}}%
  \ex\toks@\ex\ex\ex{\testarg}%
  \ex\def\ex\test@rg\ex{\the\toks@}%
  \ifdefined\Umathchar
    \@nameauth@UTFfalse%
  \else
    \ifdefined\UTFviii@two@octets
      \if@nameauth@Accent
        \@nameauth@UTFtrue\@nameauth@Accentfalse%
      \else
        \expandafter\@nameauth@TestToks\expandafter{\test@rg}%
      \fi
    \else
      \@nameauth@UTFfalse%
    \fi
  \fi
}
%    \end{macrocode}
% \end{macro}
%
% \begin{macro}{\@nameauth@Cap}
% \changes{3.1}{2017/01/13}{Redesigned}
% \changes{3.2}{2017/03/22}{Non-UTF}
% The following two macros cap the first letter of the argument.
%    \begin{macrocode}
\newcommand*\@nameauth@Cap[1]{\@nameauth@C@p#1\\}
%    \end{macrocode}
% \end{macro}
%
% \begin{macro}{\@nameauth@C@p}
% \changes{3.1}{2017/01/13}{Added}
% \changes{3.2}{2017/03/22}{Renamed, use \cmd{\MakeUppercase}}
% Helper macro for the one above.
%    \begin{macrocode}
\def\@nameauth@C@p#1#2\\%
  {\expandafter\trim@spaces\expandafter{\MakeUppercase{#1}#2}}
%    \end{macrocode}
% \end{macro}
%
% \begin{macro}{\@nameauth@CapUTF}
% \changes{3.2}{2017/03/22}{Added}
% The following two macros cap the first active Unicode letter under \textsf{inputenc}.
%    \begin{macrocode}
\newcommand*\@nameauth@CapUTF[1]{\@nameauth@C@pUTF#1\\}
%    \end{macrocode}
% \end{macro}
%
% \begin{macro}{\@nameauth@C@pUTF}
% \changes{3.1}{2017/01/13}{Added}
% \changes{3.2}{2017/03/22}{Use \cmd{\MakeUppercase}}
% Helper macro for the one above.
%    \begin{macrocode}
\def\@nameauth@C@pUTF#1#2#3\\%
  {\expandafter\trim@spaces\expandafter{\MakeUppercase{#1#2}#3}}
%    \end{macrocode}
% \end{macro}
%
% \noindent{\large\bfseries Parsing: Punctuation Detection}
% \begin{macro}{\@nameauth@TestDot}
% This macro, based on a snippet by \Name*[Uwe]{Lueck}, checks for a period at the end of its argument. It determines whether we need to call \cmd{\@nameauth@CheckDot} below.
%    \begin{macrocode}
\newcommand*\@nameauth@TestDot[1]
{%
  \def\TestDot##1.\TestEnd##2\\{\TestPunct{##2}}%
  \def\TestPunct##1{%
    \ifx\TestPunct##1\TestPunct%
    \else
      \@nameauth@Puncttrue%
    \fi
  }%
  \@nameauth@Punctfalse%
  \TestDot#1\TestEnd.\TestEnd\\%
}
%    \end{macrocode}
% \end{macro}
%
% \begin{macro}{\@nameauth@CheckDot}
% We assume that \cmd{\expandafter} precedes the invocation of \cmd{\@nameauth@CheckDot}, which only is called if the terminal character of the input is a period. We evaluate the lookahead \cmd{\@token} while keeping it on the list of input tokens.
%    \begin{macrocode}
\newcommand*\@nameauth@CheckDot%
  {\futurelet\@token\@nameauth@EvalDot}
%    \end{macrocode}
% \end{macro}
%
% \begin{macro}{\@nameauth@EvalDot}
%  If \cmd{\@token} is a full stop, we gobble the token.
%    \begin{macrocode}
\newcommand*\@nameauth@EvalDot%
{%
  \let\@period=.%
  \ifx\@token\@period\expandafter\@gobble \fi
}
%    \end{macrocode}
% \end{macro}
%
% \noindent{\large\bfseries Error Detection}
% \begin{macro}{\@nameauth@Error}
% \changes{3.0}{2016/10/26}{Added}
% \changes{3.01}{2016/10/27}{Fixed}
% One can cause \textsf{nameauth} to halt with an error by leaving a required name argument empty, providing an argument that expands to empty, or creating an empty root within a root\,/\,suffix pair.
%    \begin{macrocode}
\newcommand*\@nameauth@Error[2]
{%
  \edef\msga{#2 SNN arg empty}%
  \edef\msgb{#2 SNN arg malformed}%
  \protected@edef\testname{\trim@spaces{#1}}%
  \protected@edef\testroot{\@nameauth@Root{#1}}%
  \ifx\testname\@empty
    \PackageError{nameauth}{\msga}%
  \fi
  \ifx\testroot\@empty
      \PackageError{nameauth}{\msgb}%
  \fi
}
%    \end{macrocode}
% \end{macro}
% \newpage
%
% \noindent{\large\bfseries Core Name Engine}
%
% \begin{macro}{\@nameauth@Name}
% \changes{0.85}{2012/02/05}{Hide commas}
% \changes{1.5}{2013/02/22}{Reversing\,/\,caps}
% \changes{2.0}{2015/11/11}{Trim spaces; fix tags}
% \changes{2.1}{2015/11/24}{Fix Unicode}
% \changes{2.3}{2016/01/05}{Internal}
% \changes{2.4}{2016/03/15}{Set token regs}
% \changes{2.41}{2016/03/17}{Fix token regs}
% \changes{2.5}{2016/04/06}{Fix old syntax}
% \changes{2.6}{2016/09/19}{Better indexing}
% \changes{3.0}{2016/10/26}{Redesigned}
% \changes{3.1}{2017/01/13}{New workflow}
% \changes{3.3}{2020/02/20}{global flag reset}
% Here is the heart of the package. \Name*[Marc van]{Dongen} provided the original basic structure. Parsing, indexing, and formatting are more discrete than in earlier versions.
%    \begin{macrocode}
\newcommandx*\@nameauth@Name[3][1=\@empty, 3=\@empty]
{%
%    \end{macrocode}
% Both \cmd{\@nameauth@Name} and \cmd{\AKA} engage the lock below, preventing a stack overflow. Tell the formatting mechanism that it is being called from \cmd{\@nameauth@Name}.
%    \begin{macrocode}
  \if@nameauth@BigLock\@nameauth@Locktrue\fi
  \unless\if@nameauth@Lock
    \@nameauth@Locktrue%
    \@nameauth@InNametrue%
%    \end{macrocode}
% Test for malformed input.
%    \begin{macrocode}
    \@nameauth@Error{#2}{macro \string\@nameauth@name}%
%    \end{macrocode}
% If we use \cmd{\JustIndex} then skip everything else. The \texttt{oldpass} option restores what we did before version 3.3, where we locally reset \cmd{\@nameauth@JustIndexfalse} and were done. Now, however, the default is a global reset to avoid undocumented behavior.
%    \begin{macrocode}
    \if@nameauth@JustIndex
      \IndexName[#1]{#2}[#3]%
      \if@nameauth@OldPass
        \@nameauth@JustIndexfalse%
      \else
        \if@nameauth@OldReset
          \@nameauth@FullNamefalse%
          \@nameauth@FirstNamefalse%
          \@nameauth@JustIndexfalse%
        \else
          \global\@nameauth@FullNamefalse%
          \global\@nameauth@FirstNamefalse%
          \global\@nameauth@JustIndexfalse%
        \fi
      \fi
    \else
%    \end{macrocode}
% Delete\,/\,create name cseq if directed. If the delete flag is set, the create flag is ignored. Ensure that names are printed in horizontal mode. Wrap the name with two index entries in case a page break occurs between name elements.
%    \begin{macrocode}
      \if@nameauth@Forget
        \ForgetName[#1]{#2}[#3]%
      \else
        \if@nameauth@Subvert
          \SubvertName[#1]{#2}[#3]%
        \fi
      \fi
      \leavevmode\hbox{}%
      \unless\if@nameauth@SkipIndex\IndexName[#1]{#2}[#3]\fi
      \if@nameauth@MainFormat
        \@nameauth@Parse[#1]{#2}[#3]{!MN}%
      \else
        \@nameauth@Parse[#1]{#2}[#3]{!NF}%
      \fi
      \unless\if@nameauth@SkipIndex\IndexName[#1]{#2}[#3]\fi
%    \end{macrocode}
% Reset all the ``per name'' Boolean values. The default is global.
%    \begin{macrocode}
      \if@nameauth@OldReset
        \@nameauth@SkipIndexfalse%
        \@nameauth@Forgetfalse%
        \@nameauth@Subvertfalse%
        \@nameauth@NBSPfalse%
        \@nameauth@NBSPXfalse%
        \@nameauth@DoCapsfalse%
        \@nameauth@Accentfalse%
        \@nameauth@AllThisfalse%
        \@nameauth@ShowCommafalse%
        \@nameauth@NoCommafalse%
        \@nameauth@RevThisfalse%
        \@nameauth@RevThisCommafalse%
        \@nameauth@ShortSNNfalse%
        \@nameauth@EastFNfalse%
      \else
        \global\@nameauth@SkipIndexfalse%
        \global\@nameauth@Forgetfalse%
        \global\@nameauth@Subvertfalse%
        \global\@nameauth@NBSPfalse%
        \global\@nameauth@NBSPXfalse%
        \global\@nameauth@DoCapsfalse%
        \global\@nameauth@Accentfalse%
        \global\@nameauth@AllThisfalse%
        \global\@nameauth@ShowCommafalse%
        \global\@nameauth@NoCommafalse%
        \global\@nameauth@RevThisfalse%
        \global\@nameauth@RevThisCommafalse%
        \global\@nameauth@ShortSNNfalse%
        \global\@nameauth@EastFNfalse%
      \fi
    \fi
    \@nameauth@Lockfalse%
    \@nameauth@InNamefalse%
%    \end{macrocode}
% Close the ``locked'' branch.
%    \begin{macrocode}
  \fi
%    \end{macrocode}
% Call the full stop detection.
%    \begin{macrocode}
  \if@nameauth@Punct\expandafter\@nameauth@CheckDot\fi
}
%    \end{macrocode}
% \end{macro}
%
% \begin{macro}{\@nameauth@Parse}
% \changes{3.0}{2016/10/26}{Added}
% \changes{3.1}{2017/01/13}{New workflow, caps}
% \changes{3.2}{2017/03/22}{Fix alt. format, affixes, use \cmd{\MakeUppercase}}
% Parse and print a name in the text. The final required argument tells us which naming system we are in (Section~\ref{sec:NamePatterns}). Both \cmd{\@nameauth@Name} and \cmd{\AKA} call this parser.
%    \begin{macrocode}
\newcommandx*\@nameauth@Parse[4][1=\@empty, 3=\@empty]
{%
  \if@nameauth@BigLock\@nameauth@Lockfalse\fi
  \if@nameauth@Lock
    \let\ex\expandafter%
%    \end{macrocode}
% We want these arguments to expand to \cmd{\@empty} (or not) when we test them.
%    \begin{macrocode}
    \protected@edef\arga{\trim@spaces{#1}}%
    \protected@edef\rootb{\@nameauth@Root{#2}}%
    \protected@edef\suffb{\@nameauth@Suffix{#2}}%
    \protected@edef\argc{\trim@spaces{#3}}%
%    \end{macrocode}
% \newpage
% \noindent If global caps. reversing, and commas are true, set the local flags true.
%    \begin{macrocode}
    \if@nameauth@AllCaps\@nameauth@AllThistrue\fi
    \if@nameauth@RevAll\@nameauth@RevThistrue\fi
    \if@nameauth@RevAllComma\@nameauth@RevThisCommatrue\fi
%    \end{macrocode}
% Make (usually) unique control sequence values from the name arguments.
%    \begin{macrocode}
    \def\csb{\@nameauth@Clean{#2}}%
    \def\csbc{\@nameauth@Clean{#2,#3}}%
    \def\csab{\@nameauth@Clean{#1!#2}}%
%    \end{macrocode}
% Make token register copies of the current name args to be available for the hook macros.
%    \begin{macrocode}
    \@nameauth@toksa\expandafter{#1}%
    \@nameauth@toksb\expandafter{#2}%
    \@nameauth@toksc\expandafter{#3}%
%    \end{macrocode}
% Implement capitalization on demand in the body text if not in Continental mode.
%    \begin{macrocode}
    \if@nameauth@DoCaps
      \let\carga\arga%
      \let\crootb\rootb%
      \let\csuffb\suffb%
      \let\cargc\argc%
      \unless\if@nameauth@AltFormat
%    \end{macrocode}
% We test the first optarg for active Unicode characters. Then we capitalize the first letter. 
%    \begin{macrocode}
        \unless\ifx\arga\@empty
          \def\test{#1}%
          \ex\@nameauth@UTFtest\ex{\test}%
          \if@nameauth@UTF
            \ex\def\ex\carga\ex{\ex\@nameauth@CapUTF\ex{\test}}%
          \else
            \ex\def\ex\carga\ex{\ex\@nameauth@Cap\ex{\test}}%
          \fi
        \fi
%    \end{macrocode}
% We test the root surname for active Unicode characters. Then we capitalize the first letter.
%    \begin{macrocode}
        \def\test{#2}%
        \ex\@nameauth@UTFtest\ex{\test}%
        \if@nameauth@UTF
          \ex\def\ex\crootb\ex{\ex\@nameauth@CapUTF\ex{\rootb}}%
        \else
          \ex\def\ex\crootb\ex{\ex\@nameauth@Cap\ex{\rootb}}%
        \fi
%    \end{macrocode}
% We test the suffix for active Unicode characters. Then we capitalize the first letter.
%    \begin{macrocode}
        \unless\ifx\suffb\@empty
          \def\test{#2}%
          \ex\@nameauth@UTFtestS\ex{\test}%
          \protected@edef\test{\@nameauth@GetSuff{#2}}%
          \if@nameauth@UTF
            \protected@edef\test{\@nameauth@Suffix{#2}}%
            \ex\def\ex\csuffb\ex{\ex\@nameauth@CapUTF\ex{\test}}%
          \else
            \edef\test{\@nameauth@Suffix{#2}}%
            \ex\def\ex\csuffb\ex{\ex\@nameauth@Cap\ex{\test}}%
          \fi
        \fi
%    \end{macrocode}
% We test the final optarg for active Unicode characters. Then we capitalize the first letter. 
%    \begin{macrocode}
        \unless\ifx\argc\@empty
          \def\test{#3}%
          \ex\@nameauth@UTFtest\ex{\test}%
          \if@nameauth@UTF
            \ex\def\ex\cargc\ex{\ex\@nameauth@CapUTF\ex{\test}}%
          \else
            \ex\def\ex\cargc\ex{\ex\@nameauth@Cap\ex{\test}}%
          \fi
        \fi
      \fi
      \let\arga\carga%
      \let\rootb\crootb%
      \let\suffb\csuffb%
      \let\argc\cargc%
    \fi
%    \end{macrocode}
% We capitalize the entire surname when desired; different from above.
%    \begin{macrocode}
    \if@nameauth@AllThis
      \protected@edef\rootb{\MakeUppercase{\@nameauth@Root{#2}}}%
    \fi
%    \end{macrocode}
% Use non-breaking spaces and commas as desired.
%    \begin{macrocode}
    \edef\Space{\space}%
    \edef\SpaceX{\space}%
    \if@nameauth@NBSP\edef\Space{\nobreakspace}\fi
    \if@nameauth@NBSPX\edef\SpaceX{\nobreakspace}\fi
    \unless\ifx\arga\@empty
      \if@nameauth@AlwaysComma
        \edef\Space{,\space}%
        \if@nameauth@NBSP\edef\Space{,\nobreakspace}\fi
      \fi
      \if@nameauth@ShowComma
        \edef\Space{,\space}%
        \if@nameauth@NBSP\edef\Space{,\nobreakspace}\fi
      \fi
      \if@nameauth@NoComma
        \edef\Space{\space}%
        \if@nameauth@NBSP\edef\Space{\nobreakspace}\fi
      \fi
    \fi
%    \end{macrocode}
% We parse names by attaching ``meaning'' to patterns of macro arguments primarily via \cmd{\FNN} and \cmd{\SNN}. Then we call the name printing macros, based on the optional arguments.
%    \begin{macrocode}
    \let\SNN\rootb%
    \ifx\arga\@empty
      \ifx\argc\@empty
%    \end{macrocode}
% When \cmd{\arga}, \cmd{\argc}, and \cmd{\suffb} are empty, we have a mononym. When \cmd{\suffb} is not empty, we have a ``native'' Eastern name or non-Western name.
%    \begin{macrocode}
        \let\FNN\suffb%
        \let\SNN\rootb%
        \@nameauth@NonWest{\csb#4}%
      \else
%    \end{macrocode}
% When \cmd{\arga} and \cmd{\suffb} are empty, but \cmd{\argc} is not, we have the older syntax. When \cmd{\arga} is empty, but \cmd{\argc} and \cmd{\suffb} are not, we have alternate names for non-Western names.
%    \begin{macrocode}
        \ifx\suffb\@empty
          \let\FNN\argc%
          \let\SNN\rootb%
          \@nameauth@NonWest{\csbc#4}%
        \else
          \let\FNN\argc%
          \let\SNN\rootb%
          \@nameauth@NonWest{\csb#4}%
        \fi
      \fi
    \else
%    \end{macrocode}
% When \cmd{\arga} is not empty, we have either a Western name or a ``non-native'' Eastern name. When \cmd{\argc} is not empty, we use alternate names. When \cmd{\suffb} is not empty we use suffixed forms.
%    \begin{macrocode}
      \ifx\argc\@empty
        \let\FNN\arga%
      \else
        \let\FNN\argc%
      \fi
      \unless\ifx\suffb\@empty
        \def\SNN{\rootb\Space\suffb}%
        \if@nameauth@ShortSNN\let\SNN\rootb\fi
      \fi
      \@nameauth@West{\csab#4}%
    \fi
  \fi
}
%    \end{macrocode}
% \end{macro}
%
% \begin{macro}{\@nameauth@NonWest}
% \changes{3.0}{2016/10/26}{Added}
% \changes{3.02}{2016/11/01}{Restrict \cmd{\ForceFN}}
% \changes{3.3}{2020/02/20}{global flag reset}
% Print non-Western names from \cmd{\@nameauth@name} and \cmd{\AKA}. We inherit internal macros from the parser and do nothing apart from the locked state.
%    \begin{macrocode}
\newcommand*\@nameauth@NonWest[1]
{%
  \if@nameauth@BigLock\@nameauth@Lockfalse\fi
  \if@nameauth@Lock
    \unless\ifcsname#1\endcsname
      \@nameauth@FirstFormattrue%
    \fi
    \if@nameauth@InAKA
      \if@nameauth@AltAKA
        \if@nameauth@OldAKA\@nameauth@EastFNtrue\fi
        \@nameauth@FullNamefalse%
        \@nameauth@FirstNametrue%
      \else
        \@nameauth@FullNametrue%
        \@nameauth@FirstNamefalse%
      \fi
    \else
      \unless\ifcsname#1\endcsname
        \@nameauth@FullNametrue%
        \@nameauth@FirstNamefalse%
      \fi
    \fi
    \if@nameauth@FirstName
      \@nameauth@FullNamefalse%
    \fi
    \ifx\FNN\@empty
      \@nameauth@Hook{\SNN}%
    \else
      \if@nameauth@FullName
        \if@nameauth@RevThis
          \@nameauth@Hook{\FNN\Space\SNN}%
        \else
          \@nameauth@Hook{\SNN\Space\FNN}%
        \fi
      \else
        \if@nameauth@FirstName
          \if@nameauth@EastFN
            \@nameauth@Hook{\FNN}%
          \else
            \@nameauth@Hook{\SNN}%
          \fi
        \else
          \@nameauth@Hook{\SNN}%
        \fi
      \fi
    \fi
    \unless\ifcsname#1\endcsname
      \unless\if@nameauth@InAKA\csgdef{#1}{}\fi
    \fi
%    \end{macrocode}
% We have to reset these flags here because both the naming and cross-referencing macros use the parser.
%    \begin{macrocode}
    \if@nameauth@OldReset
      \@nameauth@FullNamefalse%
      \@nameauth@FirstNamefalse%
    \else
      \global\@nameauth@FullNamefalse%
      \global\@nameauth@FirstNamefalse%
    \fi
  \fi
}
%    \end{macrocode}
% \end{macro}
%
% \begin{macro}{\@nameauth@West}
% \changes{3.0}{2016/10/26}{Added}
% \changes{3.3}{2020/02/20}{global flag reset}
% Print Western names and ``non-native'' Eastern names from \cmd{\@nameauth@name} and \cmd{\AKA}. We inherit internal macros from the parser and do nothing apart from the locked state.
%    \begin{macrocode}
\newcommand*\@nameauth@West[1]
{%
  \if@nameauth@BigLock\@nameauth@Lockfalse\fi
  \if@nameauth@Lock
    \unless\ifcsname#1\endcsname
      \@nameauth@FirstFormattrue%
    \fi
    \if@nameauth@InAKA
      \if@nameauth@AltAKA
        \@nameauth@FullNamefalse%
        \@nameauth@FirstNametrue%
      \else
        \@nameauth@FullNametrue%
        \@nameauth@FirstNamefalse%
      \fi
    \else
      \unless\ifcsname#1\endcsname
        \@nameauth@FullNametrue%
        \@nameauth@FirstNamefalse%
      \fi
    \fi
    \if@nameauth@FirstName
      \@nameauth@FullNamefalse%
    \fi
    \if@nameauth@FullName
      \if@nameauth@RevThis
        \@nameauth@Hook{\SNN\SpaceX\FNN}%
      \else
        \if@nameauth@RevThisComma
          \edef\RevSpace{,\SpaceX}%
          \@nameauth@Hook{\SNN\RevSpace\FNN}%
        \else
          \@nameauth@Hook{\FNN\SpaceX\SNN}%
        \fi
      \fi
    \else
      \if@nameauth@FirstName
        \@nameauth@Hook{\FNN}%
      \else
        \@nameauth@Hook{\rootb}%
      \fi
    \fi
    \unless\ifcsname#1\endcsname
      \unless\if@nameauth@InAKA\csgdef{#1}{}\fi
    \fi
%    \end{macrocode}
% We have to reset these flags here because both the naming and cross-referencing macros use the parser.
%    \begin{macrocode}
    \if@nameauth@OldReset
      \@nameauth@FullNamefalse%
      \@nameauth@FirstNamefalse%
    \else
      \global\@nameauth@FullNamefalse%
      \global\@nameauth@FirstNamefalse%
    \fi
  \fi
}
%    \end{macrocode}
% \end{macro}
%
% \noindent{\large\bfseries Format Hook Dispatcher}
% \begin{macro}{\@nameauth@Hook}
% \changes{2.4}{2016/03/15}{Current form}
% \changes{2.5}{2016/04/06}{Improve hooks}
% \changes{3.0}{2016/10/26}{Fix punct. detection}
% Flags help the dispatcher invoke the correct formatting hooks. The flags control which hook is called (first\,/\,subsequent use, name type). The first set of tests handles formatting within \cmd{\AKA}. The second set of tests handles regular name formatting.
%    \begin{macrocode}
\newcommand*\@nameauth@Hook[1]
{%
  \if@nameauth@BigLock\@nameauth@Lockfalse\fi
  \if@nameauth@Lock
    \@nameauth@InHooktrue%
    \protected@edef\test{#1}%
    \expandafter\@nameauth@TestDot\expandafter{\test}%
    \if@nameauth@InAKA
      \if@nameauth@AlwaysFormat
        \@nameauth@FirstFormattrue%
      \else
        \unless\if@nameauth@AKAFormat
        \@nameauth@FirstFormatfalse\fi
      \fi
      \if@nameauth@MainFormat
        \if@nameauth@FirstFormat
          \bgroup\NamesFormat{#1}\egroup%
        \else
          \bgroup\MainNameHook{#1}\egroup%
        \fi
      \else
        \if@nameauth@FirstFormat
          \bgroup\FrontNamesFormat{#1}\egroup%
        \else
          \bgroup\FrontNameHook{#1}\egroup%
        \fi
      \fi
    \else
      \if@nameauth@AlwaysFormat
        \@nameauth@FirstFormattrue%
      \fi
      \if@nameauth@MainFormat
        \if@nameauth@FirstFormat
          \bgroup\NamesFormat{#1}\egroup%
        \else
          \bgroup\MainNameHook{#1}\egroup%
        \fi
      \else
        \if@nameauth@FirstFormat
          \bgroup\FrontNamesFormat{#1}\egroup%
        \else
          \bgroup\FrontNameHook{#1}\egroup%
        \fi
      \fi
    \fi
%    \end{macrocode}
% We have to reset this flag here because both the naming and cross-referencing macros use the parser.
%    \begin{macrocode}
    \if@nameauth@OldReset
      \@nameauth@FirstFormatfalse%
    \else
      \global\@nameauth@FirstFormatfalse%
    \fi
    \@nameauth@InHookfalse%
  \fi
}
%    \end{macrocode}
% \end{macro}
%
% \noindent{\large\bfseries Indexing Internals}
% \begin{macro}{\@nameauth@Index}
% \changes{0.94}{2012/02/15}{Added}
% \changes{2.0}{2015/11/11}{New tagging}
% \changes{3.3}{2020/02/20}{Tags support hyperref}
% If the indexing flag is true, create an index entry, otherwise do nothing. Add tags automatically if they exist.
%    \begin{macrocode}
\newcommand*\@nameauth@Index[2]
{%
  \let\ex\expandafter%
  \if@nameauth@DoIndex
    \ifcsname#1!TAG\endcsname
      \protected@edef\Tag{\csname#1!TAG\endcsname}%
      \ex\def\ex\ShortTag\ex{\ex\@nameauth@TrimTag\ex{\Tag}}%
      \ifcsname#1!PRE\endcsname
        \protected@edef\Pre{\csname#1!PRE\endcsname}%
        \if@nameauth@Xref
          \protected@edef\Entry{\Pre#2\ShortTag}%
        \else
          \protected@edef\Entry{\Pre#2\Tag}%
        \fi
      \else
        \if@nameauth@Xref
          \protected@edef\Entry{#2\ShortTag}%
        \else
          \protected@edef\Entry{#2\Tag}%
        \fi
      \fi
    \else
      \ifcsname#1!PRE\endcsname
        \protected@edef\Pre{\csname#1!PRE\endcsname}%
        \protected@edef\Entry{\Pre#2}%
      \else
        \protected@edef\Entry{#2}%
      \fi
    \fi
    \ex\index\ex{\Entry}%
  \fi
}
%    \end{macrocode}
% \end{macro}
%
% \begin{macro}{\@nameauth@Actual}
% \changes{2.0}{2015/11/11}{Added}
% This sets the ``actual'' character used by \textsf{nameauth} for index sorting.
%    \begin{macrocode}
\newcommand*\@nameauth@Actual{@}
%    \end{macrocode}
% \end{macro}
%
% \noindent{\large\bfseries Debugging Help}
% \begin{macro}{\@nameauth@Debug}
% \changes{3.3}{2020/02/20}{added}
% This Swiss-army knife for debugging shows name control sequence patterns, full index entries with tags, and short index entries with just the name. Other macros call it to get the desired info. We set up a local scope, redefine \cmd{\index} to print an argument in the text, force indexing to occur, and ignore whether we are working with xrefs.
%    \begin{macrocode}
\newcommandx*\@nameauth@Debug[3][1=\@empty, 3=\@empty]
{%
  \bgroup%
    \def\index##1{##1}%
    \@nameauth@DoIndextrue%
    \protected@edef\arga{\trim@spaces{#1}}%
    \protected@edef\argc{\trim@spaces{#3}}%
    \protected@edef\suffb{\@nameauth@Suffix{#2}}%
    \def\csb{\@nameauth@Clean{#2}}%
    \def\csbc{\@nameauth@Clean{#2,#3}}%
    \def\csab{\@nameauth@Clean{#1!#2}}%
    \@nameauth@Error{#2}{macro \string\@nameauth@Debug}%
%    \end{macrocode}
% We interleave printing name patterns (\cmd{\ShowPattern}), printing full index entries as if they were page refs (\cmd{\ShowIdxPageref*}), and printing short index entries (\cmd{\ShowIdxPageref}). Since we are in a local scope we delete the tag and xref control sequences as needed. They will be restored when the scope ends. We do not care about xrefs because we just want to see what happens with the name. We can always go to the \texttt{idx} and \texttt{ind} files if needed.
%    \begin{macrocode}
    \ifx\arga\@empty
      \ifx\argc\@empty
        \ifdefined\ShortIdxEntry
          \csundef{\csb!PRE}%
          \csundef{\csb!TAG}%
          \csundef{\csb!PN}%
          \IndexName[#1]{#2}[#3]%
        \else
          \ifdefined\LongIdxEntry
            \csundef{\csb!PN}%
            \IndexName[#1]{#2}[#3]%
          \else
            \csb%
          \fi
        \fi
      \else
        \ifx\suffb\@empty
          \ifdefined\ShortIdxEntry
            \csundef{\csbc!PRE}%
            \csundef{\csbc!TAG}%
            \csundef{\csbc!PN}%
            \IndexName[#1]{#2}[#3]%
          \else
            \ifdefined\LongIdxEntry
              \csundef{\csbc!PN}%
              \IndexName[#1]{#2}[#3]%
            \else
              \csbc%
            \fi
          \fi
        \else
          \ifdefined\ShortIdxEntry
            \csundef{\csb!PRE}%
            \csundef{\csb!TAG}%
            \csundef{\csb!PN}%
            \IndexName[#1]{#2}[#3]%
          \else
            \ifdefined\LongIdxEntry
              \csundef{\csb!PN}%
              \IndexName[#1]{#2}[#3]%
            \else
              \csb%
            \fi
          \fi
        \fi
      \fi
    \else
      \ifdefined\ShortIdxEntry
        \csundef{\csab!PRE}%
        \csundef{\csab!TAG}%
        \csundef{\csab!PN}%
        \IndexName[#1]{#2}[#3]%
      \else
        \ifdefined\LongIdxEntry
          \csundef{\csab!PN}%
          \IndexName[#1]{#2}[#3]%
        \else
          \csab%
        \fi
      \fi
    \fi
    \global\undef{\LongIdxEntry}%
    \global\undef{\ShortIdxEntry}%
  \egroup%
}
%    \end{macrocode}
% \end{macro}
%
% \subsection[Prefix Macros]{User Interface Macros: Prefix Macros}
% \noindent{\large\bfseries Syntactic Formatting\,---\,Capitalization}
% \begin{macro}{\CapThis}
% \changes{0.94}{2012/02/15}{Added}
% Tells the root capping macro to cap the first character of all name elements.
%    \begin{macrocode}
\newcommand*\CapThis{\@nameauth@DoCapstrue}
%    \end{macrocode}
% \end{macro}
%
% \begin{macro}{\AccentCapThis}
% \changes{2.1}{2015/11/24}{Added}
% Overrides the automatic test for active Unicode characters. This is a fall-back in case the automatic test for active Unicode characters fails.
%    \begin{macrocode}
\newcommand*\AccentCapThis%
  {\@nameauth@Accenttrue\@nameauth@DoCapstrue}
%    \end{macrocode}
% \end{macro}
%
% \begin{macro}{\CapName}
% \changes{1.5}{2013/02/22}{Added}
% Capitalize entire required name. Overrides \cmd{\CapThis} for surnames.
%    \begin{macrocode}
\newcommand*\CapName{\@nameauth@AllThistrue}
%    \end{macrocode}
% \end{macro}
%
% \begin{macro}{\AllCapsInactive}
% Turn off global surname capitalization.
% \changes{1.5}{2013/02/22}{Added}
%    \begin{macrocode}
\newcommand*\AllCapsInactive{\@nameauth@AllCapsfalse}
%    \end{macrocode}
% \end{macro}
%
% \begin{macro}{\AllCapsActive}
% \changes{1.5}{2013/02/22}{Added}
% Turn on global surname capitalization. Activates \cmd{\CapName} for every name.
%    \begin{macrocode}
\newcommand*\AllCapsActive{\@nameauth@AllCapstrue}
%    \end{macrocode}
% \end{macro}
%
% \noindent{\large\bfseries Syntactic Formatting\,---\,Reversing}
% \begin{macro}{\RevName}
% \changes{1.5}{2013/02/22}{Added}
% Reverse name order.
%    \begin{macrocode}
\newcommand*\RevName{\@nameauth@RevThistrue}
%    \end{macrocode}
% \end{macro}
%
% \begin{macro}{\ReverseInactive}
% Turn off global name reversing.
% \changes{1.5}{2013/02/22}{Added}
%    \begin{macrocode}
\newcommand*\ReverseInactive{\@nameauth@RevAllfalse}
%    \end{macrocode}
% \end{macro}
%
% \begin{macro}{\ReverseActive}
% Turn on global name reversing. Activates \cmd{\RevName} for every name.
% \changes{1.5}{2013/02/22}{Added}
%    \begin{macrocode}
\newcommand*\ReverseActive{\@nameauth@RevAlltrue}
%    \end{macrocode}
% \end{macro}
%
% \begin{macro}{\ForceFN}
% Force the printing of an Eastern forename or ancient affix in the text, but only when using the ``short name'' macro \cmd{\FName} and the \cmd{\S}\meta{macro}.
% \changes{3.0}{2016/10/26}{Added}
%    \begin{macrocode}
\newcommand*\ForceFN{\@nameauth@EastFNtrue}
%    \end{macrocode}
% \end{macro}
%
% \noindent{\large\bfseries Syntactic Formatting\,---\,Reversing with Commas}
% \begin{macro}{\RevComma}
% \changes{1.5}{2013/02/22}{Added}
% Last name, comma, first name.
%    \begin{macrocode}
\newcommand*\RevComma%
  {\@nameauth@RevThisCommatrue}
%    \end{macrocode}
% \end{macro}
%
% \begin{macro}{\ReverseCommaInactive}
% Turn off global ``last-name-comma-first.''
% \changes{1.5}{2013/02/22}{Added}
%    \begin{macrocode}
\newcommand*\ReverseCommaInactive%
  {\@nameauth@RevAllCommafalse}
%    \end{macrocode}
% \end{macro}
%
% \begin{macro}{\ReverseCommaActive}
% Turn on global ``last-name-comma-first.'' Activates \cmd{\RevComma} for every name.
% \changes{1.5}{2013/02/22}{Added}
%    \begin{macrocode}
\newcommand*\ReverseCommaActive%
  {\@nameauth@RevAllCommatrue}
%    \end{macrocode}
% \end{macro}
%
% \noindent{\large\bfseries Alternate Formatting}
% \phantomsection\label{page:Hooks}
% \begin{macro}{\AltFormatActive}
% \changes{3.1}{2017/01/13}{Added}
% Turn on alternate formatting, engage the formatting macros.
%    \begin{macrocode}
\newcommand*\AltFormatActive{%
  \global\@nameauth@AltFormattrue%
  \global\@nameauth@DoAlttrue%
}
%    \end{macrocode}
% \end{macro}
%
% \begin{macro}{\AltFormatActive*}
% \changes{3.1}{2017/01/13}{Added}
% Turn on alternate formatting, disengage the formatting macros.
%    \begin{macrocode}
\WithSuffix{\newcommand*}\AltFormatActive*{%
  \global\@nameauth@AltFormattrue%
  \global\@nameauth@DoAltfalse%
}
%    \end{macrocode}
% \end{macro}
%
% \begin{macro}{\AltFormatInactive}
% Turn off alternate formatting altogether.
% \changes{3.1}{2017/01/13}{Added}
%    \begin{macrocode}
\newcommand*\AltFormatInactive{%
  \global\@nameauth@AltFormatfalse%
  \global\@nameauth@DoAltfalse%
}
%    \end{macrocode}
% \end{macro}
%
% \begin{macro}{\AltOn}
% Locally turn on alternate formatting.
% \changes{3.1}{2017/01/13}{Added}
%    \begin{macrocode}
\newcommand*\AltOn{%
  \if@nameauth@InHook
    \if@nameauth@AltFormat\@nameauth@DoAlttrue\fi
  \fi
}
%    \end{macrocode}
% \end{macro}
%
% \begin{macro}{\AltOff}
% Locally turn off alternate formatting.
% \changes{3.1}{2017/01/13}{Added}
%    \begin{macrocode}
\newcommand*\AltOff{%
  \if@nameauth@InHook
    \if@nameauth@AltFormat\@nameauth@DoAltfalse\fi
  \fi
}
%    \end{macrocode}
% \end{macro}
%
% \begin{macro}{\AltCaps}
% \changes{3.1}{2017/01/13}{Added}
% \changes{3.2}{2017/03/22}{Use \cmd{\MakeUppercase}}
% Alternate discretionary capping macro triggered by \cmd{\CapThis}.
%    \begin{macrocode}
\newcommand*\AltCaps[1]{%
  \if@nameauth@InHook
    \if@nameauth@DoCaps\MakeUppercase{#1}\else#1\fi
  \else#1%
  \fi
}
%    \end{macrocode}
% \end{macro}
%
% \begin{macro}{\textSC}
% \changes{3.1}{2017/01/13}{Added}
% Alternate formatting macro: small caps when active.
%    \begin{macrocode}
\newcommand*\textSC[1]{%
  \if@nameauth@DoAlt\textsc{#1}\else#1\fi}
%    \end{macrocode}
% \end{macro}
% 
% \begin{macro}{\textUC}
% \changes{3.1}{2017/01/13}{Added}
% \changes{3.2}{2017/03/22}{Use \cmd{\MakeUppercase}}
% Alternate formatting macro: uppercase when active.
%    \begin{macrocode}
\newcommand*\textUC[1]{%
  \if@nameauth@DoAlt\MakeUppercase{#1}\else#1\fi}
%    \end{macrocode}
% \end{macro}
%  
% \begin{macro}{\textIT}
% \changes{3.1}{2017/01/13}{Added}
% Alternate formatting macro: italic when active.
%    \begin{macrocode}
\newcommand*\textIT[1]{%
  \if@nameauth@DoAlt\textit{#1}\else#1\fi}
%    \end{macrocode}
% \end{macro}
%  
% \begin{macro}{\textBF}
% \changes{3.1}{2017/01/13}{Added}
% Alternate formatting macro: boldface when active.
%    \begin{macrocode}
\newcommand*\textBF[1]{%
  \if@nameauth@DoAlt\textbf{#1}\else#1\fi}
%    \end{macrocode}
% \end{macro}
%
% \noindent{\large\bfseries Syntactic Formatting\,---\,Affixes}
% \begin{macro}{\ShowComma}
% \changes{1.4}{2012/07/24}{Added}
% Put comma between name and suffix one time.
%    \begin{macrocode}
\newcommand*\ShowComma{\@nameauth@ShowCommatrue}
%    \end{macrocode}
% \end{macro}
%
% \begin{macro}{\NoComma}
% \changes{2.6}{2016/09/19}{Added}
% Remove comma between name and suffix one time (with \texttt{comma} option).
%    \begin{macrocode}
\newcommand*\NoComma{\@nameauth@NoCommatrue}
%    \end{macrocode}
% \end{macro}
%
% \begin{macro}{\DropAffix}
% \changes{3.0}{2016/10/26}{Added}
% Suppress the affix in a long Western name.
%    \begin{macrocode}
\newcommand*\DropAffix{\@nameauth@ShortSNNtrue}
%    \end{macrocode}
% \end{macro}
%
% \begin{macro}{\KeepAffix}
% \changes{1.9}{2015/07/09}{Added}
% Trigger a name-suffix pair to be separated by a non-breaking space.
%    \begin{macrocode}
\newcommand*\KeepAffix{\@nameauth@NBSPtrue}
%    \end{macrocode}
% \end{macro}
%
% \begin{macro}{\KeepName}
% \changes{3.1}{2017/01/13}{Added}
% Use non-breaking spaces between name syntactic forms.
%    \begin{macrocode}
\newcommand*\KeepName{\@nameauth@NBSPtrue\@nameauth@NBSPXtrue}
%    \end{macrocode}
% \end{macro}
%
% \noindent{\large\bfseries Post-Processing\,---\,Main Versus Front Matter}
% \begin{macro}{\NamesInactive}
% Switch to the ``non-formatted'' species of names.
%    \begin{macrocode}
\newcommand*\NamesInactive{\@nameauth@MainFormatfalse}
%    \end{macrocode}
% \end{macro}
%
% \begin{macro}{\NamesActive}
% Switch to the ``formatted'' species of names.
%    \begin{macrocode}
\newcommand*\NamesActive{\@nameauth@MainFormattrue}
%    \end{macrocode}
% \end{macro}
%
% \noindent{\large\bfseries Name Decisions\,---\,First\,/\,Subsequent Reference}
% \begin{macro}{\ForgetThis}
% \changes{3.1}{2017/01/13}{Added}
% Have the naming engine \cmd{\@nameauth@Name} call \cmd{\ForgetName} internally.
%    \begin{macrocode}
\newcommand*\ForgetThis{\@nameauth@Forgettrue}
%    \end{macrocode}
% \end{macro}
%
% \begin{macro}{\SubvertThis}
% \changes{3.1}{2017/01/13}{Added}
% Have the naming engine \cmd{\@nameauth@Name} call \cmd{\SubvertName} internally.
%    \begin{macrocode}
\newcommand*\SubvertThis{\@nameauth@Subverttrue}
%    \end{macrocode}
% \end{macro}
%
% \begin{macro}{\ForceName}
% \changes{3.1}{2017/01/13}{Added}
% Set \cmd{\@nameauth@FirstFormat} to be true even for subsequent name uses. Works for one name only.
%    \begin{macrocode}
\newcommand*\ForceName{\@nameauth@FirstFormattrue}
%    \end{macrocode}
% \end{macro}
%
% \noindent{\large\bfseries Name Occurrence Tweaks}
% \begin{macro}{\LocalNames}
% \changes{2.3}{2016/01/05}{Added}
% \changes{2.4}{2016/03/15}{Ensure global}
% \cmd{\LocalNames} sets |@nameauth@LocalNames| true so \cmd{\ForgetName} and \cmd{\SubvertName} do not affect both main and front matter naming systems.
%    \begin{macrocode}
\newcommand*\LocalNames{\global\@nameauth@LocalNamestrue}
%    \end{macrocode}
% \end{macro}
%
% \begin{macro}{\GlobalNames}
% \changes{2.3}{2016/01/05}{Added}
% \changes{2.4}{2016/03/15}{Ensure global}
% \cmd{\GlobalNames} sets |@nameauth@LocalNames| false. This restores the default behavior of \cmd{\ForgetName} and \cmd{\SubvertName}.
%    \begin{macrocode}
\newcommand*\GlobalNames{\global\@nameauth@LocalNamesfalse}
%    \end{macrocode}
% \end{macro}
%
% \noindent{\large\bfseries Index Operations}
% \begin{macro}{\IndexInactive}
% \changes{0.94}{2012/02/15}{Added}
% Turn off global indexing of names.
%    \begin{macrocode}
\newcommand*\IndexInactive{\@nameauth@DoIndexfalse}
%    \end{macrocode}
% \end{macro}
%
% \begin{macro}{\SkipIndex}
% \changes{3.1}{2017/01/13}{Added}
% Turn off the next instance of indexing in \cmd{\Name}, \cmd{\FName}, and starred forms.
%    \begin{macrocode}
\newcommand*\SkipIndex{\@nameauth@SkipIndextrue}
%    \end{macrocode}
% \end{macro}
%
% \begin{macro}{\JustIndex}
% \changes{3.1}{2017/01/13}{Added}
% Makes the next call to \cmd{\Name}, \cmd{\FName}, and starred forms act like \cmd{\IndexName}. Overrides \cmd{\SkipIndex}.
%    \begin{macrocode}
\newcommand*\JustIndex{\@nameauth@JustIndextrue}
%    \end{macrocode}
% \end{macro}
%
% \begin{macro}{\IndexActive}
% \changes{0.94}{2012/02/15}{Added}
% Turn on global indexing of names.
%    \begin{macrocode}
\newcommand*\IndexActive{\@nameauth@DoIndextrue}
%    \end{macrocode}
% \end{macro}
%
% \begin{macro}{\IndexActual}
% \changes{2.0}{2015/11/11}{Added}
% Change the ``actual'' character from the default.
%    \begin{macrocode}
\newcommand*\IndexActual[1]
  {\global\renewcommand*\@nameauth@Actual{#1}}
%    \end{macrocode}
% \end{macro}
%
% \begin{macro}{\SeeAlso}
% \changes{3.0}{2016/10/26}{Added}
% Change the type of cross-reference from a \textit{see} reference to a \textit{see also} reference. Works once per xref, unless one uses \cmd{\Include*}, in which case, take care!
%    \begin{macrocode}
\newcommand*\SeeAlso{\@nameauth@SeeAlsotrue}
%    \end{macrocode}
% \end{macro}
%
% \subsection[General User Interface]{User Interface Macros: General}
% \label{sec:UserInterface}
%
% \begin{macro}{\ShowPattern}
% \changes{3.3}{2020/02/20}{added}
% This displays the pattern that the name arguments generate; maybe useful for debugging.
%    \begin{macrocode}
\newcommand*\ShowPattern{\@nameauth@Debug}
%    \end{macrocode}
% \end{macro}
%
% \begin{macro}{\ShowIdxPageref}
% \changes{3.3}{2020/02/20}{added}
% This displays the index entry that will be generated. This may be useful for debugging.
%    \begin{macrocode}
\newcommand*\ShowIdxPageref%
  {\def\LongIdxEntry{}\ShowPattern}
%    \end{macrocode}
% \end{macro}
%
% \begin{macro}{\ShowIdxPageref*}
% \changes{3.3}{2020/02/20}{added}
% This displays an index entry with no tag. This may be useful for debugging.
%    \begin{macrocode}
\WithSuffix{\newcommand*}\ShowIdxPageref*%
  {\def\ShortIdxEntry{}\ShowPattern}
%    \end{macrocode}
% \end{macro}
%
% \begin{macro}{\NameParser}
% \changes{3.0}{2016/10/26}{Added}
% \changes{3.03}{2016/11/01}{First name only for short forms}
% \changes{3.1}{2017/01/13}{Fix old syntax; add NBSP}
% \changes{3.2}{2017/03/22}{Fix alt. format, affixes}
% Generate a name form based on the current state of the \textsf{nameauth} macros in the locked path. Available for use only in the hook macros.
%    \begin{macrocode}
\newcommand*\NameParser
{%
  \if@nameauth@InHook
    \let\SNN\rootb%
    \ifx\arga\@empty
%    \end{macrocode}
%  If the first optarg is empty, it is a non-Western name. The forename will be either the suffix or the final optarg.
%    \begin{macrocode}
      \ifx\argc\@empty
        \let\FNN\suffb%
      \else
        \let\FNN\argc%
      \fi
      \ifx\suffb\@empty
%    \end{macrocode}
%  Mononym case
%    \begin{macrocode}
        \ifx\FNN\@empty
          \SNN%
        \else
%    \end{macrocode}
%  Eastern or ancient name, using the older syntax, with name reversing and forcing
%    \begin{macrocode}
          \if@nameauth@FullName%
            \if@nameauth@RevThis
              \FNN\Space\SNN%
            \else
              \SNN\Space\FNN%
            \fi
          \else
            \if@nameauth@FirstName
              \if@nameauth@EastFN
                \FNN%
              \else
                \SNN%
              \fi
            \else
              \SNN%
            \fi
          \fi
        \fi
      \else
%    \end{macrocode}
%  Eastern or ancient name, using the new syntax, with name reversing and forcing
%    \begin{macrocode}
        \if@nameauth@FullName
          \if@nameauth@RevThis
            \FNN\Space\SNN%
          \else
            \SNN\Space\FNN%
          \fi
        \else
          \if@nameauth@FirstName
            \if@nameauth@EastFN
              \FNN%
            \else
              \SNN%
            \fi
          \else
            \SNN%
          \fi
        \fi
      \fi
    \else
%    \end{macrocode}
%  Western name with name reversing and suffixes
%    \begin{macrocode}
      \ifx\argc\@empty
        \let\FNN\arga%
      \else
        \let\FNN\argc%
      \fi
      \unless\ifx\suffb\@empty
        \def\SNN{\rootb\Space\suffb}%
        \if@nameauth@ShortSNN\let\SNN\rootb\fi%
      \fi
      \if@nameauth@FullName
        \if@nameauth@RevThis
          \SNN\SpaceX\FNN%
        \else
          \if@nameauth@RevThisComma
            \SNN\RevSpace\FNN%
          \else
            \FNN\SpaceX\SNN%
          \fi
        \fi
      \else
        \if@nameauth@FirstName
          \FNN%
        \else
          \let\SNN\rootb%
          \SNN%
        \fi
      \fi
    \fi
  \fi
}
%    \end{macrocode}
% \end{macro}
%
% \noindent{\large\bfseries Traditional Naming Interface}
% \begin{macro}{\Name}
% \changes{2.3}{2016/01/05}{Interface macro}
% \cmd{\Name} calls \cmd{\NameauthName}, the interface hook.
%    \begin{macrocode}
\newcommand\Name{\NameauthName}
%    \end{macrocode}
% \end{macro}
%
% \begin{macro}{\Name*}
% \changes{2.3}{2016/01/05}{Interface macro}
% \cmd{\Name*} sets up a long name reference and calls \cmd{\NameauthLName}, the interface hook.
%    \begin{macrocode}
\WithSuffix{\newcommand*}\Name*%
  {\@nameauth@FullNametrue\NameauthLName}
%    \end{macrocode}
% \end{macro}
%
% \begin{macro}{\FName}
% \changes{0.9}{2012/02/10}{Added}
% \changes{2.3}{2016/01/05}{Interface macro}
% \cmd{\FName} sets up a short name reference and calls \cmd{\NameauthFName}, the interface hook.
%    \begin{macrocode}
\newcommand\FName{\@nameauth@FirstNametrue\NameauthFName}
%    \end{macrocode}
% \end{macro}
%
% \begin{macro}{\FName*}
% \changes{2.3}{2016/01/05}{Interface macro}
% \cmd{\FName} and \cmd{\FName*} are identical in function.
%    \begin{macrocode}
\WithSuffix{\newcommand*}\FName*%
  {\@nameauth@FirstNametrue\NameauthFName}
%    \end{macrocode}
% \end{macro}
% 
% \noindent{\large\bfseries Index Operations}
% \begin{macro}{\IndexProtect}
% \changes{3.3}{2020/02/20}{added}
% We shut down all output from the naming and indexing macros to protect against problems in the index in case a macro in an index entry should expand into one of the naming macros.
%    \begin{macrocode}
\newcommand*\IndexProtect
{%
  \@nameauth@DoIndexfalse%
  \@nameauth@BigLocktrue%
}
%    \end{macrocode}
% \end{macro}
%
% \begin{macro}{\IndexName}
% \changes{0.75}{2012/01/19}{Current args}
% \changes{0.85}{2012/02/05}{Hide commas}
% \changes{1.26}{2012/04/24}{Fix affixes}
% \changes{2.0}{2015/11/11}{Fix spaces, tagging}
% \changes{2.6}{2016/09/19}{Fix commas}
% \changes{3.0}{2016/10/26}{Redesigned}
% \changes{3.1}{2017/01/13}{Better tests}
% This creates an index entry with page references. It issues warnings if the \texttt{verbose} option is selected. It prints nothing. First we make copies of the arguments.
%    \begin{macrocode}
\newcommandx*\IndexName[3][1=\@empty, 3=\@empty]
{%
  \protected@edef\arga{\trim@spaces{#1}}%
  \protected@edef\rootb{\@nameauth@Root{#2}}%
  \protected@edef\suffb{\@nameauth@Suffix{#2}}%
  \protected@edef\argc{\trim@spaces{#3}}%
  \def\csb{\@nameauth@Clean{#2}}%
  \def\csbc{\@nameauth@Clean{#2,#3}}%
  \def\csab{\@nameauth@Clean{#1!#2}}%
%    \end{macrocode}
%  Test for malformed input.
%    \begin{macrocode}
  \@nameauth@Error{#2}{macro \string\IndexName}%
%    \end{macrocode}
% We create the appropriate index entries, calling \cmd{\@nameauth@Index} to handle sorting and tagging. We do not create an index entry for a cross-reference (code !PN for pseudonym), used by \cmd{\IndexRef}, \cmd{\Excludename}, \cmd{\Includename}, \cmd{\AKA}, and \cmd{\PName}. If the first optarg is empty, it is a non-Western name.
%    \begin{macrocode}
  \ifx\arga\@empty
    \ifx\argc\@empty
      \ifcsname\csb!PN\endcsname
        \if@nameauth@Verbose
          \PackageWarning{nameauth}%
          {macro \IndexName: XRef: #2 exists}%
        \fi
      \else
        \ifx\suffb\@empty
%    \end{macrocode}
%  mononym or Eastern\,/\,ancient name, new syntax
%    \begin{macrocode}
          \@nameauth@Index{\csb}{\rootb}%
        \else
          \@nameauth@Index{\csb}{\rootb\space\suffb}%
        \fi
      \fi
    \else
      \ifx\suffb\@empty
        \ifcsname\csbc!PN\endcsname
          \if@nameauth@Verbose
            \PackageWarning{nameauth}%
            {macro \IndexName: XRef: #2 #3 exists}%
          \fi
        \else
%    \end{macrocode}
%  Eastern or ancient name, older syntax
%    \begin{macrocode}
          \@nameauth@Index{\csbc}{\rootb\space\argc}%
        \fi
      \else
        \ifcsname\csb!PN\endcsname
          \if@nameauth@Verbose
            \PackageWarning{nameauth}%
            {macro \IndexName: XRef: #2 exists}%
          \fi
        \else
%    \end{macrocode}
%  Eastern or ancient name, new syntax, alternate name ignored
%    \begin{macrocode}
          \@nameauth@Index{\csb}{\rootb\space\suffb}%
        \fi
      \fi
    \fi
  \else
    \ifcsname\csab!PN\endcsname
      \if@nameauth@Verbose
        \PackageWarning{nameauth}%
        {macro \IndexName: XRef: #1 #2 exists}%
      \fi
    \else
%    \end{macrocode}
%  Western name, without and with affix
%    \begin{macrocode}
      \ifx\suffb\@empty
        \@nameauth@Index{\csab}%
        {\rootb,\space\arga}%
      \else
        \@nameauth@Index{\csab}%
        {\rootb,\space\arga,\space\suffb}%
      \fi
    \fi
  \fi
}
%    \end{macrocode}
% \end{macro}
%
% \begin{macro}{\IndexRef}
% \changes{3.0}{2016/10/26}{Added}
% \changes{3.1}{2017/01/13}{Better tests}
% \changes{3.3}{2020/02/20}{global flag reset}
% This creates an index cross-reference that is not already a pseudonym. It prints nothing. First we make copies of the arguments to test them and make parsing decisions.
%    \begin{macrocode}
\newcommandx*\IndexRef[4][1=\@empty, 3=\@empty]
{%
  \protected@edef\arga{\trim@spaces{#1}}%
  \protected@edef\rootb{\@nameauth@Root{#2}}%
  \protected@edef\suffb{\@nameauth@Suffix{#2}}%
  \protected@edef\argc{\trim@spaces{#3}}%
  \protected@edef\target{#4}%
  \def\csb{\@nameauth@Clean{#2}}%
  \def\csbc{\@nameauth@Clean{#2,#3}}%
  \def\csab{\@nameauth@Clean{#1!#2}}%
  \let\ex\expandafter%
%    \end{macrocode}
%  Test for malformed input.
%    \begin{macrocode}
  \@nameauth@Error{#2}{macro \string\IndexRef}%
  \@nameauth@Xreftrue%
%    \end{macrocode}
% We create either \textit{see also} entries or \textit{see} entries. The former are unrestricted. The latter are only created if they do not already exist as main entries.
%    \begin{macrocode}
  \ifx\arga\@empty
    \ifx\argc\@empty
      \ifcsname\csb!PN\endcsname
        \if@nameauth@Verbose
          \PackageWarning{nameauth}%
          {macro \IndexRef: XRef: #2 exists}%
        \fi
      \else
        \ifx\suffb\@empty
%    \end{macrocode}
%  mononym or Eastern\,/\,ancient name, new syntax
%    \begin{macrocode}
          \if@nameauth@SeeAlso
            \@nameauth@Index{\csb}{\rootb|seealso{\target}}%
          \else
            \@nameauth@Index{\csb}{\rootb|see{\target}}%
          \fi
        \else
          \if@nameauth@SeeAlso
            \@nameauth@Index{\csb}%
              {\rootb\space\suffb|seealso{\target}}%
          \else
            \@nameauth@Index{\csb}%
              {\rootb\space\suffb|see{\target}}%
          \fi
        \fi
        \csgdef{\csb!PN}{}%
      \fi
    \else
      \ifx\suffb\@empty
        \ifcsname\csbc!PN\endcsname
          \if@nameauth@Verbose
            \PackageWarning{nameauth}%
            {macro \IndexRef: XRef: #2 #3 exists}%
          \fi
        \else
%    \end{macrocode}
%  Eastern or ancient name, older syntax
%    \begin{macrocode}
          \if@nameauth@SeeAlso
            \@nameauth@Index{\csbc}%
              {\rootb\space\argc|seealso{\target}}%
          \else
            \@nameauth@Index{\csbc}%
              {\rootb\space\argc|see{\target}}%
          \fi
          \csgdef{\csbc!PN}{}%
        \fi
      \else
        \ifcsname\csb!PN\endcsname
          \if@nameauth@Verbose
            \PackageWarning{nameauth}%
            {macro \IndexRef: XRef: #2 exists}%
          \fi
        \else
%    \end{macrocode}
%  Eastern or ancient name, new syntax, alternate name ignored
%    \begin{macrocode}
          \if@nameauth@SeeAlso
            \@nameauth@Index{\csb}%
              {\rootb\space\suffb|seealso{\target}}%
          \else
            \@nameauth@Index{\csb}%
              {\rootb\space\suffb|see{\target}}%
          \fi
          \csgdef{\csb!PN}{}%
        \fi
      \fi
    \fi
  \else
    \ifcsname\csab!PN\endcsname
      \if@nameauth@Verbose
        \PackageWarning{nameauth}%
        {macro \IndexRef: XRef: #1 #2 exists}%
      \fi
    \else
%    \end{macrocode}
%  Western name, without and with affix
%    \begin{macrocode}
      \ifx\suffb\@empty
        \if@nameauth@SeeAlso
          \@nameauth@Index{\csab}%
          {\rootb,\space\arga|seealso{\target}}%
        \else
          \@nameauth@Index{\csab}%
          {\rootb,\space\arga|see{\target}}%
        \fi
      \else
        \if@nameauth@SeeAlso
          \@nameauth@Index{\csab}%
          {\rootb,\space\arga,\space\suffb|seealso{\target}}%
        \else
          \@nameauth@Index{\csab}%
          {\rootb,\space\arga,\space\suffb|see{\target}}%
        \fi
      \fi
      \csgdef{\csab!PN}{}%
    \fi
  \fi
  \@nameauth@Xreffalse%
%    \end{macrocode}
% This may not be necessary, but we do it for consistency.
%    \begin{macrocode}
  \if@nameauth@OldReset
    \@nameauth@SeeAlsofalse%
  \else
    \global\@nameauth@SeeAlsofalse%
  \fi
}
%    \end{macrocode}
% \end{macro}
%
% \begin{macro}{\ExcludeName}
% \changes{0.94}{2012/02/15}{Added}
% \changes{2.3}{2016/01/05}{New xref test}
% \changes{3.0}{2016/10/26}{Redesigned}
% \changes{3.3}{2020/02/20}{More accurate warnings}
% This macro prevents a name from being indexed.
%    \begin{macrocode}
\newcommandx*\ExcludeName[3][1=\@empty, 3=\@empty]
{%
  \protected@edef\arga{\trim@spaces{#1}}%
  \protected@edef\argc{\trim@spaces{#3}}%
  \protected@edef\suffb{\@nameauth@Suffix{#2}}%
  \def\csb{\@nameauth@Clean{#2}}%
  \def\csbc{\@nameauth@Clean{#2,#3}}%
  \def\csab{\@nameauth@Clean{#1!#2}}%
%    \end{macrocode}
% Below we parse the name arguments and create a non-empty pseudonym macro.
%    \begin{macrocode}
  \@nameauth@Error{#2}{macro \string\ExcludeName}%
  \ifx\arga\@empty
    \ifx\argc\@empty
      \if@nameauth@Verbose
        \ifcsname\csb!MN\endcsname
          \PackageWarning{nameauth}%
          {macro \ExcludeName: Reference: #2 exists}%
        \fi
        \ifcsname\csb!NF\endcsname
          \PackageWarning{nameauth}%
          {macro \ExcludeName: Reference: #2 exists}%
        \fi
      \fi
      \ifcsname\csb!PN\endcsname
        \if@nameauth@Verbose
          \PackageWarning{nameauth}%
          {macro \ExcludeName: Xref: #2 exists}%
        \fi
      \else
        \csgdef{\csb!PN}{!}%
      \fi
    \else
      \ifx\suffb\@empty
        \if@nameauth@Verbose
          \ifcsname\csbc!MN\endcsname
            \PackageWarning{nameauth}%
            {macro \ExcludeName: Reference: #2 #3 exists}%
          \fi
          \ifcsname\csbc!NF\endcsname
            \PackageWarning{nameauth}%
            {macro \ExcludeName: Reference: #2 #3 exists}%
          \fi
        \fi
        \csgdef{\csbc!PN}{!}%
        \ifcsname\csbc!PN\endcsname
          \if@nameauth@Verbose
            \PackageWarning{nameauth}%
            {macro \ExcludeName: Xref: #2 #3 exists}%
          \fi
        \else
          \csgdef{\csbc!PN}{!}%
        \fi
      \else
        \if@nameauth@Verbose
          \ifcsname\csb!MN\endcsname
            \PackageWarning{nameauth}%
            {macro \ExcludeName: Reference: #2 exists}%
          \fi
          \ifcsname\csb!NF\endcsname
            \PackageWarning{nameauth}%
            {macro \ExcludeName: Reference: #2 exists}%
          \fi
        \fi
        \ifcsname\csb!PN\endcsname
          \if@nameauth@Verbose
            \PackageWarning{nameauth}%
            {macro \ExcludeName: Xref: #2 exists}%
          \fi
        \else
          \csgdef{\csb!PN}{!}%
        \fi
      \fi
    \fi
  \else
    \if@nameauth@Verbose
      \ifcsname\csab!MN\endcsname
        \PackageWarning{nameauth}%
        {macro \ExcludeName: Reference: #1 #2 exists}%
      \fi
      \ifcsname\csab!NF\endcsname
        \PackageWarning{nameauth}%
        {macro \ExcludeName: Reference: #1 #2 exists}%
      \fi
    \fi
    \ifcsname\csab!PN\endcsname
      \if@nameauth@Verbose
        \PackageWarning{nameauth}%
        {macro \ExcludeName: Xref: #1 #2 exists}%
      \fi
    \else
      \csgdef{\csab!PN}{!}%
    \fi
  \fi
}
%    \end{macrocode}
% \end{macro}
%
% \begin{macro}{\IncludeName}
% \changes{3.0}{2016/10/26}{Added}
% \changes{3.3}{2020/02/20}{Added warnings}
% This macro allows a name to be indexed if it is not a cross-reference.
%    \begin{macrocode}
\newcommandx*\IncludeName[3][1=\@empty, 3=\@empty]
{%
  \protected@edef\arga{\trim@spaces{#1}}%
  \protected@edef\argc{\trim@spaces{#3}}%
  \protected@edef\suffb{\@nameauth@Suffix{#2}}%
  \def\csb{\@nameauth@Clean{#2}}%
  \def\csbc{\@nameauth@Clean{#2,#3}}%
  \def\csab{\@nameauth@Clean{#1!#2}}%
%    \end{macrocode}
% Below we parse the name arguments and undefine only an ``excluded'' name.
%    \begin{macrocode}
  \@nameauth@Error{#2}{macro \string\IncludeName}%
  \ifx\arga\@empty
    \ifx\argc\@empty
      \ifcsname\csb!PN\endcsname
        \edef\testex{\csname\csb!PN\endcsname}%
        \unless\ifx\testex\@empty\global\csundef{\csb!PN}%
        \else
          \if@nameauth@Verbose
            \PackageWarning{nameauth}%
            {macro \IncludeName: Xref: #2 exists}%
          \fi
        \fi
      \fi
    \else
      \ifx\suffb\@empty
        \ifcsname\csbc!PN\endcsname
          \edef\testex{\csname\csbc!PN\endcsname}%
          \unless\ifx\testex\@empty\global\csundef{\csbc!PN}%
          \else
            \if@nameauth@Verbose
              \PackageWarning{nameauth}%
              {macro \IncludeName: Xref: #2 #3 exists}%
            \fi
          \fi
        \fi
      \else
        \ifcsname\csb!PN\endcsname
          \edef\testex{\csname\csb!PN\endcsname}%
          \unless\ifx\testex\@empty\global\csundef{\csb!PN}%
          \else
            \if@nameauth@Verbose
              \PackageWarning{nameauth}%
              {macro \IncludeName: Xref: #2 exists}%
            \fi
          \fi
        \fi
      \fi
    \fi
  \else
    \ifcsname\csab!PN\endcsname
      \edef\testex{\csname\csab!PN\endcsname}%
      \unless\ifx\testex\@empty\global\csundef{\csab!PN}%
      \else
        \if@nameauth@Verbose
          \PackageWarning{nameauth}%
          {macro \IncludeName: Xref: #1 #2 exists}%
        \fi
      \fi
    \fi
  \fi
}
%    \end{macrocode}
% \end{macro}
%
% \begin{macro}{\IncludeName*}
% \changes{3.0}{2016/10/26}{Added}
% \changes{3.1}{2017/01/13}{Fixed}
% This macro allows any name to be indexed, undefining cross-references.
%    \begin{macrocode}
\WithSuffix{\newcommandx*}\IncludeName*[3][1=\@empty, 3=\@empty]
{%
  \protected@edef\arga{\trim@spaces{#1}}%
  \protected@edef\argc{\trim@spaces{#3}}%
  \protected@edef\suffb{\@nameauth@Suffix{#2}}%
  \def\csb{\@nameauth@Clean{#2}}%
  \def\csbc{\@nameauth@Clean{#2,#3}}%
  \def\csab{\@nameauth@Clean{#1!#2}}%
%    \end{macrocode}
% Below we parse the name arguments and undefine an xref control sequence.
%    \begin{macrocode}
  \@nameauth@Error{#2}{macro \string\IncludeName*}%
  \ifx\arga\@empty
    \ifx\argc\@empty
      \global\csundef{\csb!PN}%
    \else
      \ifx\suffb\@empty
        \global\csundef{\csbc!PN}%
      \else
        \global\csundef{\csb!PN}%
      \fi
    \fi
  \else
    \global\csundef{\csab!PN}%
  \fi
}
%    \end{macrocode}
% \end{macro}
%
% \begin{macro}{\PretagName}
% \changes{2.0}{2015/11/11}{Added}
% This creates an index entry tag that is applied before a name.
%    \begin{macrocode}
\newcommandx*\PretagName[4][1=\@empty, 3=\@empty]
{%
  \protected@edef\arga{\trim@spaces{#1}}%
  \protected@edef\argc{\trim@spaces{#3}}%
  \protected@edef\suffb{\@nameauth@Suffix{#2}}%
  \def\csb{\@nameauth@Clean{#2}}%
  \def\csbc{\@nameauth@Clean{#2,#3}}%
  \def\csab{\@nameauth@Clean{#1!#2}}%
%    \end{macrocode}
% We parse the arguments, defining the sort tag control sequences used by\newline \cmd{\@nameauth@Index}.
%    \begin{macrocode}
  \@nameauth@Error{#2}{macro \string\PretagName}%
  \ifx\arga\@empty
    \ifx\argc\@empty
      \ifcsname\csb!PN\endcsname
        \if@nameauth@Verbose
          \PackageWarning{nameauth}%
          {macro \PretagName: tagging xref: #2}%
        \fi
      \fi
      \if@nameauth@Pretag\csgdef{\csb!PRE}{#4\@nameauth@Actual}\fi
    \else
      \ifx\suffb\@empty
        \ifcsname\csbc!PN\endcsname
          \if@nameauth@Verbose
            \PackageWarning{nameauth}%
            {macro \PretagName: tagging xref: #2 #3}%
          \fi
        \fi
        \if@nameauth@Pretag\csgdef{\csbc!PRE}{#4\@nameauth@Actual}\fi
      \else
        \ifcsname\csb!PN\endcsname
          \if@nameauth@Verbose
            \PackageWarning{nameauth}%
            {macro \PretagName: tagging xref: #2}%
          \fi
        \fi
        \if@nameauth@Pretag\csgdef{\csb!PRE}{#4\@nameauth@Actual}\fi
      \fi
    \fi
  \else
    \ifcsname\csab!PN\endcsname
      \if@nameauth@Verbose
        \PackageWarning{nameauth}%
        {macro \PretagName: tagging xref: #1 #2}%
      \fi
    \fi
    \if@nameauth@Pretag\csgdef{\csab!PRE}{#4\@nameauth@Actual}\fi
  \fi
}
%    \end{macrocode}
% \end{macro}
% 
% \begin{macro}{\TagName}
% \changes{1.2}{2012/02/25}{Added}
% \changes{1.9}{2015/07/09}{Fix cs collisions}
% \changes{2.0}{2015/11/11}{Redesign tagging}
% This creates an index entry tag for a name that is not used as a cross-reference.
%    \begin{macrocode}
\newcommandx*\TagName[4][1=\@empty, 3=\@empty]
{%
  \protected@edef\arga{\trim@spaces{#1}}%
  \protected@edef\argc{\trim@spaces{#3}}%
  \protected@edef\suffb{\@nameauth@Suffix{#2}}%
  \def\csb{\@nameauth@Clean{#2}}%
  \def\csbc{\@nameauth@Clean{#2,#3}}%
  \def\csab{\@nameauth@Clean{#1!#2}}%
%    \end{macrocode}
% We parse the arguments, defining the macros used by \cmd{\@nameauth@Index}.
%    \begin{macrocode}
  \@nameauth@Error{#2}{macro \string\TagName}%
  \ifx\arga\@empty
    \ifx\argc\@empty
      \ifcsname\csb!PN\endcsname
        \if@nameauth@Verbose
          \PackageWarning{nameauth}%
          {macro \TagName: not tagging xref: #2}%
        \fi
      \else
        \csgdef{\csb!TAG}{#4}%
      \fi
    \else
      \ifx\suffb\@empty
        \ifcsname\csbc!PN\endcsname
          \if@nameauth@Verbose
            \PackageWarning{nameauth}%
            {macro \TagName: not tagging xref: #2 #3}%
          \fi
        \else
          \csgdef{\csbc!TAG}{#4}%
        \fi
      \else
        \ifcsname\csb!PN\endcsname
          \if@nameauth@Verbose
            \PackageWarning{nameauth}%
            {macro \TagName: not tagging xref: #2}%
          \fi
        \else
          \csgdef{\csb!TAG}{#4}%
        \fi
      \fi
    \fi
  \else
    \ifcsname\csab!PN\endcsname
      \if@nameauth@Verbose
        \PackageWarning{nameauth}%
        {macro \TagName: not tagging xref: #1 #2}%
      \fi
    \else
      \csgdef{\csab!TAG}{#4}%
    \fi
  \fi
}
%    \end{macrocode}
% \end{macro}
% 
% \begin{macro}{\UntagName}
% \changes{1.2}{2012/02/25}{Added}
% \changes{1.9}{2015/07/09}{Global undef, no cs collisions}
% \changes{2.0}{2015/11/11}{Redesign tagging}
% This deletes an index tag.
%    \begin{macrocode}
\newcommandx*\UntagName[3][1=\@empty, 3=\@empty]
{%
  \protected@edef\arga{\trim@spaces{#1}}%
  \protected@edef\argc{\trim@spaces{#3}}%
  \protected@edef\suffb{\@nameauth@Suffix{#2}}%
  \def\csb{\@nameauth@Clean{#2}}%
  \def\csbc{\@nameauth@Clean{#2,#3}}%
  \def\csab{\@nameauth@Clean{#1!#2}}%
%    \end{macrocode}
% We parse the arguments, undefining the index tag macros.
%    \begin{macrocode}
  \@nameauth@Error{#2}{macro \string\UntagName}%
  \ifx\arga\@empty
    \ifx\argc\@empty
      \global\csundef{\csb!TAG}%
    \else
      \ifx\suffb\@empty
        \global\csundef{\csbc!TAG}%
      \else
        \global\csundef{\csb!TAG}%
      \fi
    \fi
  \else
    \global\csundef{\csab!TAG}%
  \fi
}
%    \end{macrocode}
% \end{macro}
% 
% \noindent{\large\bfseries Name Info Data Set: ``Text Tags''}
% \begin{macro}{\NameAddInfo}
% \changes{2.4}{2016/03/15}{Added}
% This creates a macro that expands to information associated with a given name, similar to an index tag, but usable in the body text.
%    \begin{macrocode}
\newcommandx\NameAddInfo[4][1=\@empty, 3=\@empty]
{%
  \protected@edef\arga{\trim@spaces{#1}}%
  \protected@edef\argc{\trim@spaces{#3}}%
  \protected@edef\Suff{\@nameauth@Suffix{#2}}%
  \def\csb{\@nameauth@Clean{#2}}%
  \def\csbc{\@nameauth@Clean{#2,#3}}%
  \def\csab{\@nameauth@Clean{#1!#2}}%
%    \end{macrocode}
% We parse the arguments, defining the text tag control sequences.
%    \begin{macrocode}
  \@nameauth@Error{#2}{macro \string\NameAddInfo}%
  \ifx\arga\@empty
    \ifx\argc\@empty
      \csgdef{\csb!DB}{#4}%
    \else
      \ifx\Suff\@empty
        \csgdef{\csbc!DB}{#4}%
      \else
        \csgdef{\csb!DB}{#4}%
      \fi
    \fi
  \else
    \csgdef{\csab!DB}{#4}%
  \fi
}
%    \end{macrocode}
% \end{macro}
%
% \begin{macro}{\NameQueryInfo}
% \changes{2.4}{2016/03/15}{Added}
% \changes{3.1}{2017/01/13}{Short macro}
% \changes{3.3}{2020/02/20}{lock added}
% This prints the information created by \cmd{\NameAddInfo} if it exists.
%    \begin{macrocode}
\newcommandx*\NameQueryInfo[3][1=\@empty, 3=\@empty]
{%
  \protected@edef\arga{\trim@spaces{#1}}%
  \protected@edef\argc{\trim@spaces{#3}}%
  \protected@edef\Suff{\@nameauth@Suffix{#2}}%
  \def\csb{\@nameauth@Clean{#2}}%
  \def\csbc{\@nameauth@Clean{#2,#3}}%
  \def\csab{\@nameauth@Clean{#1!#2}}%
  \unless\if@nameauth@BigLock
%    \end{macrocode}
% We parse the arguments, invoking the tag macros to expand to their contents.
%    \begin{macrocode}
    \@nameauth@Error{#2}{macro \string\NameQueryInfo}%
    \ifx\arga\@empty
      \ifx\argc\@empty
        \ifcsname\csb!DB\endcsname\csname\csb!DB\endcsname\fi
      \else
        \ifx\Suff\@empty
          \ifcsname\csbc!DB\endcsname\csname\csbc!DB\endcsname\fi
        \else
          \ifcsname\csb!DB\endcsname\csname\csb!DB\endcsname\fi
        \fi
      \fi
    \else
      \ifcsname\csab!DB\endcsname\csname\csab!DB\endcsname\fi
    \fi
  \fi
}
%    \end{macrocode}
% \end{macro}
%
% \begin{macro}{\NameClearInfo}
% \changes{2.4}{2016/03/15}{Added}
% This deletes a text tag. It has the same structure as \cmd{\UntagName}.
%    \begin{macrocode}
\newcommandx*\NameClearInfo[3][1=\@empty, 3=\@empty]
{%
  \protected@edef\arga{\trim@spaces{#1}}%
  \protected@edef\argc{\trim@spaces{#3}}%
  \protected@edef\Suff{\@nameauth@Suffix{#2}}%
  \def\csb{\@nameauth@Clean{#2}}%
  \def\csbc{\@nameauth@Clean{#2,#3}}%
  \def\csab{\@nameauth@Clean{#1!#2}}%
%    \end{macrocode}
% We parse the arguments, undefining the text tag control sequences.
%    \begin{macrocode}
  \@nameauth@Error{#2}{macro \string\NameClearInfo}%
  \ifx\arga\@empty
    \ifx\argc\@empty
      \global\csundef{\csb!DB}%
    \else
      \ifx\Suff\@empty
        \global\csundef{\csbc!DB}%
      \else
        \global\csundef{\csb!DB}%
      \fi
    \fi
  \else
    \global\csundef{\csab!DB}%
  \fi
}
%    \end{macrocode}
% \end{macro}
% 
% \noindent{\large\bfseries Name Decisions}
% \begin{macro}{\IfMainName}
% \changes{2.3}{2016/01/05}{Added}
% This macro expands one path if a main matter name exists, or else the other.
%    \begin{macrocode}
\newcommandx\IfMainName[5][1=\@empty, 3=\@empty]
{%
  \protected@edef\arga{\trim@spaces{#1}}%
  \protected@edef\argc{\trim@spaces{#3}}%
  \protected@edef\suffb{\@nameauth@Suffix{#2}}%
  \def\csb{\@nameauth@Clean{#2}}%
  \def\csbc{\@nameauth@Clean{#2,#3}}%
  \def\csab{\@nameauth@Clean{#1!#2}}%
%    \end{macrocode}
% Below we parse the name arguments and choose the path.
%    \begin{macrocode}
  \@nameauth@Error{#2}{macro \string\IfMainName}%
  \ifx\arga\@empty
    \ifx\argc\@empty
      \ifcsname\csb!MN\endcsname{#4}\else{#5}\fi
    \else
      \ifx\suffb\@empty
        \ifcsname\csbc!MN\endcsname{#4}\else{#5}\fi
      \else
        \ifcsname\csb!MN\endcsname{#4}\else{#5}\fi
      \fi
    \fi
  \else
    \ifcsname\csab!MN\endcsname{#4}\else{#5}\fi
  \fi
}
%    \end{macrocode}
% \end{macro}
% 
% \begin{macro}{\IfFrontName}
% \changes{2.3}{2016/01/05}{Added}
% This macro expands one path if a front matter name exists, or else the other.
%    \begin{macrocode}
\newcommandx\IfFrontName[5][1=\@empty, 3=\@empty]
{%
  \protected@edef\arga{\trim@spaces{#1}}%
  \protected@edef\argc{\trim@spaces{#3}}%
  \protected@edef\suffb{\@nameauth@Suffix{#2}}%
  \def\csb{\@nameauth@Clean{#2}}%
  \def\csbc{\@nameauth@Clean{#2,#3}}%
  \def\csab{\@nameauth@Clean{#1!#2}}%
%    \end{macrocode}
% Below we parse the name arguments and choose the path.
%    \begin{macrocode}
  \@nameauth@Error{#2}{macro \string\IfFrontName}%
  \ifx\arga\@empty
    \ifx\argc\@empty
      \ifcsname\csb!NF\endcsname{#4}\else{#5}\fi
    \else
      \ifx\suffb\@empty
        \ifcsname\csbc!NF\endcsname{#4}\else{#5}\fi
      \else
        \ifcsname\csb!NF\endcsname{#4}\else{#5}\fi
      \fi
    \fi
  \else
    \ifcsname\csab!NF\endcsname{#4}\else{#5}\fi
  \fi
}
%    \end{macrocode}
% \end{macro}
% 
% \begin{macro}{\IfAKA}
% \changes{2.3}{2016/01/05}{Added}
% \changes{2.4}{2016/03/15}{Test for excluded}
% \changes{3.0}{2016/10/26}{Redesigned}
% This macro expands one path if a cross-reference exists, another if it does not exist, and a third if it is excluded.
%    \begin{macrocode}
\newcommandx\IfAKA[6][1=\@empty, 3=\@empty]
{%
  \protected@edef\arga{\trim@spaces{#1}}%
  \protected@edef\argc{\trim@spaces{#3}}%
  \protected@edef\suffb{\@nameauth@Suffix{#2}}%
  \def\csb{\@nameauth@Clean{#2}}%
  \def\csbc{\@nameauth@Clean{#2,#3}}%
  \def\csab{\@nameauth@Clean{#1!#2}}%
%    \end{macrocode}
% For each class of name we test first if a cross-reference exists, then if it is excluded.
%    \begin{macrocode}
  \@nameauth@Error{#2}{macro \string\IfAKA}%
  \ifx\arga\@empty
    \ifx\argc\@empty
      \ifcsname\csb!PN\endcsname
        \edef\testex{\csname\csb!PN\endcsname}%
        \ifx\testex\@empty{#4}\else{#6}\fi
      \else{#5}\fi
    \else
      \ifx\suffb\@empty
        \ifcsname\csbc!PN\endcsname
          \edef\testex{\csname\csbc!PN\endcsname}%
          \ifx\testex\@empty{#4}\else{#6}\fi
        \else{#5}\fi
      \else
        \ifcsname\csb!PN\endcsname
          \edef\testex{\csname\csb!PN\endcsname}%
          \ifx\testex\@empty{#4}\else{#6}\fi
        \else{#5}\fi
      \fi
    \fi
  \else
    \ifcsname\csab!PN\endcsname
      \edef\testex{\csname\csab!PN\endcsname}%
      \ifx\testex\@empty{#4}\else{#6}\fi
    \else{#5}\fi
  \fi
}
%    \end{macrocode}
% \end{macro}
%
% \noindent{\large\bfseries Changing Name Decisions}
% \begin{macro}{\ForgetName}
% \changes{0.75}{2012/01/19}{Current args}
% \changes{1.9}{2015/07/09}{Global undef}
% \changes{2.3}{2016/01/05}{Global or local}
% This undefines a control sequence to force a ``first use.''
%    \begin{macrocode}
\newcommandx*\ForgetName[3][1=\@empty, 3=\@empty]
{%
  \protected@edef\arga{\trim@spaces{#1}}%
  \protected@edef\argc{\trim@spaces{#3}}%
  \protected@edef\suffb{\@nameauth@Suffix{#2}}%
  \def\csb{\@nameauth@Clean{#2}}%
  \def\csbc{\@nameauth@Clean{#2,#3}}%
  \def\csab{\@nameauth@Clean{#1!#2}}%
  \@nameauth@Error{#2}{macro \string\ForgetName}%
%    \end{macrocode}
% Now we parse the arguments, undefining the control sequences either by current name type (via |@nameauth@MainFormat|) or completely (toggled by |@nameauth@LocalNames|).
%    \begin{macrocode}
  \ifx\arga\@empty
    \ifx\argc\@empty
      \if@nameauth@LocalNames
        \if@nameauth@MainFormat
          \global\csundef{\csb!MN}%
        \else
          \global\csundef{\csb!NF}%
        \fi
      \else
        \global\csundef{\csb!MN}%
        \global\csundef{\csb!NF}%
      \fi
    \else
      \ifx\suffb\@empty
        \if@nameauth@LocalNames
          \if@nameauth@MainFormat
            \global\csundef{\csbc!MN}%
          \else
            \global\csundef{\csbc!NF}%
          \fi
        \else
          \global\csundef{\csbc!MN}%
          \global\csundef{\csbc!NF}%
        \fi
      \else
        \if@nameauth@LocalNames
          \if@nameauth@MainFormat
            \global\csundef{\csb!MN}%
          \else
            \global\csundef{\csb!NF}%
          \fi
        \else
          \global\csundef{\csb!MN}%
          \global\csundef{\csb!NF}%
        \fi
      \fi
    \fi
  \else
    \if@nameauth@LocalNames
      \if@nameauth@MainFormat
        \global\csundef{\csab!MN}%
      \else
        \global\csundef{\csab!NF}%
      \fi
    \else
      \global\csundef{\csab!MN}%
      \global\csundef{\csab!NF}%
    \fi
  \fi
}
%    \end{macrocode}
% \end{macro}
%
% \begin{macro}{\SubvertName}
% \changes{0.9}{2012/02/10}{Added}
% \changes{2.3}{2016/01/05}{Global or local}
% \changes{3.1}{2017/01/13}{Fix old syntax}
% This defines a control sequence to force a ``subsequent use.''
%    \begin{macrocode}
\newcommandx*\SubvertName[3][1=\@empty, 3=\@empty]
{%
  \protected@edef\arga{\trim@spaces{#1}}%
  \protected@edef\argc{\trim@spaces{#3}}%
  \protected@edef\suffb{\@nameauth@Suffix{#2}}%
  \def\csb{\@nameauth@Clean{#2}}%
  \def\csbc{\@nameauth@Clean{#2,#3}}%
  \def\csab{\@nameauth@Clean{#1!#2}}%
%    \end{macrocode}
% We make copies of the arguments to test them.
%    \begin{macrocode}
  \@nameauth@Error{#2}{macro \string\SubvertName}%
%    \end{macrocode}
% Now we parse the arguments, defining the control sequences either locally by section type or globally. |@nameauth@LocalNames| toggles the local or global behavior, while we select the type of name with |@nameauth@MainFormat|.
%    \begin{macrocode}
  \ifx\arga\@empty
    \ifx\argc\@empty
      \if@nameauth@LocalNames
        \if@nameauth@MainFormat
          \csgdef{\csb!MN}{}%
        \else
          \csgdef{\csb!NF}{}%
        \fi
      \else
        \csgdef{\csb!MN}{}%
        \csgdef{\csb!NF}{}%
      \fi
    \else
      \ifx\suffb\@empty
        \if@nameauth@LocalNames
          \if@nameauth@MainFormat
            \csgdef{\csbc!MN}{}%
          \else
            \csgdef{\csbc!NF}{}%
          \fi
        \else
          \csgdef{\csbc!MN}{}%
          \csgdef{\csbc!NF}{}%
        \fi
      \else  
        \if@nameauth@LocalNames
          \if@nameauth@MainFormat
            \csgdef{\csb!MN}{}%
          \else
            \csgdef{\csb!NF}{}%
          \fi
        \else
          \csgdef{\csb!MN}{}%
          \csgdef{\csb!NF}{}%
        \fi
      \fi
    \fi
  \else
    \if@nameauth@LocalNames
      \if@nameauth@MainFormat
        \csgdef{\csab!MN}{}%
      \else
        \csgdef{\csab!NF}{}%
      \fi
    \else
      \csgdef{\csab!MN}{}%
      \csgdef{\csab!NF}{}%
    \fi
  \fi
}
%    \end{macrocode}
% \end{macro}
% 
% \noindent{\large\bfseries Alternate Names}
% \begin{macro}{\AKA}
% \changes{0.85}{2012/02/05}{Hide commas}
% \changes{1.26}{2012/04/24}{Fix affixes}
% \changes{1.5}{2013/02/22}{Reversing, caps}
% \changes{2.0}{2015/11/11}{Trim spaces; fix tagging}
% \changes{2.1}{2015/11/24}{Fix Unicode}
% \changes{2.3}{2016/01/05}{Fix starred mode}
% \changes{2.41}{2016/03/17}{Fix token regs}
% \changes{2.6}{2016/09/19}{Fix index commas}
% \changes{3.0}{2016/10/26}{Redesigned}
% \changes{3.1}{2017/01/13}{Can skip index}
% \changes{3.3}{2020/02/20}{global flag reset}
% \cmd{\AKA} prints an alternate name and creates index cross-references. It prevents multiple generation of cross-references and suppresses double periods.
%    \begin{macrocode}
\newcommandx*\AKA[5][1=\@empty, 3=\@empty, 5=\@empty]
{%
%    \end{macrocode}
% Prevent entering \cmd{\AKA} via itself or \cmd{\@nameauth@Name}. Prevent the index-only flag. Tell the formatting system that \cmd{\AKA} is running.
%    \begin{macrocode}
  \if@nameauth@BigLock\@nameauth@Locktrue\fi
  \unless\if@nameauth@Lock
    \@nameauth@Locktrue%
    \@nameauth@InAKAtrue%
    \if@nameauth@OldReset
      \@nameauth@JustIndexfalse%
    \else
      \global\@nameauth@JustIndexfalse%
    \fi
%    \end{macrocode}
% Test for malformed input.
%    \begin{macrocode}
    \@nameauth@Error{#2}{macro \string\AKA}%
    \@nameauth@Error{#4}{macro \string\AKA}%
%    \end{macrocode}
% Names occur in horizontal mode; we ensure that. Next we make copies of the target name arguments and we parse and print the cross-reference name.
%    \begin{macrocode}
    \leavevmode\hbox{}%
    \protected@edef\argi{\trim@spaces{#1}}%
    \protected@edef\rooti{\@nameauth@Root{#2}}%
    \protected@edef\suffi{\@nameauth@Suffix{#2}}%
    \@nameauth@Parse[#3]{#4}[#5]{!PN}%
%    \end{macrocode}
% Create an index cross-reference based on the arguments.
%    \begin{macrocode}
    \unless\if@nameauth@SkipIndex
      \ifx\argi\@empty
        \ifx\suffi\@empty
          \IndexRef[#3]{#4}[#5]{\rooti}%
        \else
          \IndexRef[#3]{#4}[#5]{\rooti\space\suffi}%
        \fi
      \else
        \ifx\suffi\@empty
          \IndexRef[#3]{#4}[#5]{\rooti,\space\argi}%
        \else
          \IndexRef[#3]{#4}[#5]{\rooti,\space\argi,\space\suffi}%
        \fi
      \fi
    \fi
%    \end{macrocode}
% Reset all the ``per name'' Boolean values. The default is global.
%    \begin{macrocode}
    \if@nameauth@OldReset
      \@nameauth@SkipIndexfalse%
      \@nameauth@AltAKAfalse%
      \@nameauth@NBSPfalse%
      \@nameauth@NBSPXfalse%
      \@nameauth@DoCapsfalse%
      \@nameauth@Accentfalse%
      \@nameauth@AllThisfalse%
      \@nameauth@ShowCommafalse%
      \@nameauth@NoCommafalse%
      \@nameauth@RevThisfalse%
      \@nameauth@RevThisCommafalse%
      \@nameauth@ShortSNNfalse%
      \@nameauth@EastFNfalse%
    \else
      \global\@nameauth@SkipIndexfalse%
      \global\@nameauth@AltAKAfalse%
      \global\@nameauth@NBSPfalse%
      \global\@nameauth@NBSPXfalse%
      \global\@nameauth@DoCapsfalse%
      \global\@nameauth@Accentfalse%
      \global\@nameauth@AllThisfalse%
      \global\@nameauth@ShowCommafalse%
      \global\@nameauth@NoCommafalse%
      \global\@nameauth@RevThisfalse%
      \global\@nameauth@RevThisCommafalse%
      \global\@nameauth@ShortSNNfalse%
      \global\@nameauth@EastFNfalse%
    \fi
    \@nameauth@Lockfalse%
    \@nameauth@InAKAfalse%
%    \end{macrocode}
% Close the ``locked'' branch.
%    \begin{macrocode}
  \fi
%    \end{macrocode}
% Call the full stop detection.
%    \begin{macrocode}
  \if@nameauth@Punct\expandafter\@nameauth@CheckDot\fi
}
%    \end{macrocode}
% \end{macro}
%
% \begin{macro}{\AKA*}
% \changes{0.9}{2012/02/10}{Added}
% This starred form sets a Boolean to print only the alternate name argument, if that exists, and calls \cmd{\AKA}.
%    \begin{macrocode}
\WithSuffix{\newcommand*}\AKA*{\@nameauth@AltAKAtrue\AKA}
%    \end{macrocode}
% \end{macro}
%
% \begin{macro}{\PName}
% \changes{2.3}{2016/01/05}{Work with hooks}
% \changes{3.1}{2017/01/13}{Can skip index}
% \cmd{\PName} is a convenience macro that calls \cmd{\NameauthName}, then \cmd{\AKA}. It prevents the index-only feature from triggering.
%    \begin{macrocode}
\newcommandx*\PName[5][1=\@empty,3=\@empty,5=\@empty]
{%
  \@nameauth@JustIndexfalse%
  \if@nameauth@SkipIndex
    \NameauthName[#1]{#2}\space(\SkipIndex\AKA[#1]{#2}[#3]{#4}[#5])%
  \else
    \NameauthName[#1]{#2}\space(\AKA[#1]{#2}[#3]{#4}[#5])%
  \fi
}
%    \end{macrocode}
% \end{macro}
%
% \begin{macro}{\PName*}
% This sets up a long name reference and calls \cmd{\PName}.
%    \begin{macrocode}
\WithSuffix{\newcommand*}\PName*{\@nameauth@FullNametrue\PName}
%    \end{macrocode}
% \end{macro}
%
% \noindent{\large\bfseries Simplified Interface}
% \begin{environment}{nameauth}
% \changes{1.6}{2013/03/10}{Environment added}
% \changes{1.9}{2015/07/09}{Bugfix}
% \changes{2.0}{2015/11/11}{Better arg handling}
% \changes{2.11}{2015/11/29}{Bugfix}
% \changes{2.41}{2016/03/17}{No local \cmd{\newtoks}}
% The \texttt{nameauth} environment creates macro shorthands. First we define a control sequence \cmd{\<} that takes four parameters, delimited by three ampersands and \texttt{>}.
%    \begin{macrocode}
\newenvironment{nameauth}{%
  \begingroup%
  \let\ex\expandafter%
  \csdef{<}##1&##2&##3&##4>{%
    \protected@edef\@arga{\trim@spaces{##1}}%
    \protected@edef\@testb{\trim@spaces{##2}}%
    \protected@edef\@testd{\trim@spaces{##4}}%
    \@nameauth@etoksb\expandafter{##2}%
    \@nameauth@etoksc\expandafter{##3}%
    \@nameauth@etoksd\expandafter{##4}%
%    \end{macrocode}
% The first argument must have some text to create a set of control sequences with it. The third argument is the required name argument. Redefining a shorthand creates a warning.
%    \begin{macrocode}
    \ifx\@arga\@empty
      \PackageError{nameauth}%
      {environment nameauth: Control sequence missing}%
    \fi
    \@nameauth@Error{##3}{environment nameauth}%
    \ifcsname\@arga\endcsname
      \PackageWarning{nameauth}%
      {environment nameauth: Shorthand macro already exists}%
    \fi
%    \end{macrocode}
% Set up shorthands according to name form. We have to use \cmd{\expandafter}, not the \(\epsilon\)-\TeX{} way, due to \cmd{\protected@edef} in the naming macros.
%
% We begin with mononyms and non-Western names that use the new syntax. We use one \cmd{\expandafter} per token because we only have one argument to expand first.
%    \begin{macrocode}
    \ifx\@testd\@empty
      \ifx\@testb\@empty
        \ex\csgdef\ex{\ex\@arga\ex}%
          \ex{\ex\NameauthName\ex{\the\@nameauth@etoksc}}%
        \ex\csgdef\ex{\ex L\ex\@arga\ex}%
          \ex{\ex\@nameauth@FullNametrue%
          \ex\NameauthLName\ex{\the\@nameauth@etoksc}}%
        \ex\csgdef\ex{\ex S\ex\@arga\ex}%
          \ex{\ex\@nameauth@FirstNametrue%
          \ex\NameauthFName\ex{\the\@nameauth@etoksc}}%
      \else
%    \end{macrocode}
% Next we have Western names with no alternate names. Here we have two arguments to expand in reverse order, so we need three, then one uses of \cmd{\expandafter} per token.
%    \begin{macrocode}
        \ex\ex\ex\csgdef\ex\ex\ex{\ex\ex\ex\@arga\ex\ex\ex}%
          \ex\ex\ex{\ex\ex\ex\NameauthName%
          \ex\ex\ex[\ex\the\ex\@nameauth@etoksb\ex]%
          \ex{\the\@nameauth@etoksc}}%
        \ex\ex\ex\csgdef\ex\ex\ex{\ex\ex\ex L\ex\ex\ex\@arga%
          \ex\ex\ex}\ex\ex\ex{\ex\ex\ex\@nameauth@FullNametrue%
          \ex\ex\ex\NameauthLName%
          \ex\ex\ex[\ex\the\ex\@nameauth@etoksb\ex]%
          \ex{\the\@nameauth@etoksc}}%
        \ex\ex\ex\csgdef\ex\ex\ex{\ex\ex\ex S\ex\ex\ex\@arga%
          \ex\ex\ex}\ex\ex\ex{\ex\ex\ex\@nameauth@FirstNametrue%
          \ex\ex\ex\NameauthFName%
          \ex\ex\ex[\ex\the\ex\@nameauth@etoksb\ex]%
          \ex{\the\@nameauth@etoksc}}%
      \fi
    \else
%    \end{macrocode}
% Below are ``native'' Eastern names with alternates and the older syntax. Again, we have three or one use of  \cmd{\expandafter} per step before the respective arguments.
%    \begin{macrocode}
      \ifx\@testb\@empty
        \ex\ex\ex\csgdef\ex\ex\ex{\ex\ex\ex\@arga\ex\ex\ex}%
          \ex\ex\ex{\ex\ex\ex\NameauthName%
          \ex\ex\ex{\ex\the\ex\@nameauth@etoksc\ex}%
          \ex[\the\@nameauth@etoksd]}%
        \ex\ex\ex\csgdef\ex\ex\ex{\ex\ex\ex L\ex\ex\ex\@arga%
          \ex\ex\ex}\ex\ex\ex{\ex\ex\ex\@nameauth@FullNametrue%
          \ex\ex\ex\NameauthLName%
          \ex\ex\ex{\ex\the\ex\@nameauth@etoksc\ex}%
          \ex[\the\@nameauth@etoksd]}%
        \ex\ex\ex\csgdef\ex\ex\ex{\ex\ex\ex S\ex\ex\ex\@arga%
          \ex\ex\ex}\ex\ex\ex{\ex\ex\ex\@nameauth@FirstNametrue%
          \ex\ex\ex\NameauthFName%
          \ex\ex\ex{\ex\the\ex\@nameauth@etoksc\ex}%
          \ex[\the\@nameauth@etoksd]}%
      \else
%    \end{macrocode}
% Here are Western names with alternates. We have three arguments to expand, so we have seven, three, and one respective use of \cmd{\expandafter}.
%    \begin{macrocode}
        \ex\ex\ex\ex\ex\ex\ex\csgdef\ex\ex\ex\ex\ex\ex\ex{%
          \ex\ex\ex\ex\ex\ex\ex\@arga\ex\ex\ex\ex\ex\ex\ex}%
          \ex\ex\ex\ex\ex\ex\ex{\ex\ex\ex\ex\ex\ex\ex\NameauthName%
          \ex\ex\ex\ex\ex\ex\ex[\ex\ex\ex\the%
          \ex\ex\ex\@nameauth@etoksb\ex\ex\ex]%
          \ex\ex\ex{\ex\the\ex\@nameauth@etoksc\ex}%
          \ex[\the\@nameauth@etoksd]}%
        \ex\ex\ex\ex\ex\ex\ex\csgdef\ex\ex\ex\ex\ex\ex\ex{%
          \ex\ex\ex\ex\ex\ex\ex L\ex\ex\ex\ex\ex\ex\ex\@arga%
          \ex\ex\ex\ex\ex\ex\ex}\ex\ex\ex\ex\ex\ex\ex{%
          \ex\ex\ex\ex\ex\ex\ex\@nameauth@FullNametrue%
          \ex\ex\ex\ex\ex\ex\ex\NameauthLName%
          \ex\ex\ex\ex\ex\ex\ex[\ex\ex\ex\the\ex\ex\ex%
          \@nameauth@etoksb\ex\ex\ex]%
          \ex\ex\ex{\ex\the\ex\@nameauth@etoksc\ex}%
          \ex[\the\@nameauth@etoksd]}%
        \ex\ex\ex\ex\ex\ex\ex\csgdef\ex\ex\ex\ex\ex\ex\ex{%
          \ex\ex\ex\ex\ex\ex\ex S\ex\ex\ex\ex\ex\ex\ex\@arga%
          \ex\ex\ex\ex\ex\ex\ex}\ex\ex\ex\ex\ex\ex\ex{%
          \ex\ex\ex\ex\ex\ex\ex\@nameauth@FirstNametrue%
          \ex\ex\ex\ex\ex\ex\ex\NameauthFName%
          \ex\ex\ex\ex\ex\ex\ex[\ex\ex\ex\the\ex\ex\ex%
          \@nameauth@etoksb\ex\ex\ex]%
          \ex\ex\ex{\ex\the\ex\@nameauth@etoksc\ex}%
          \ex[\the\@nameauth@etoksd]}%
      \fi
    \fi\ignorespaces%
  }\ignorespaces%
}{\endgroup\ignorespaces}
%    \end{macrocode}
% \end{environment}
% \Finale
\endinput
%\iffalse
%</package>
%\fi

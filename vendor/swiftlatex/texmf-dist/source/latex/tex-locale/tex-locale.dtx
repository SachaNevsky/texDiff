%\iffalse
% tex-locale.dtx generated using makedtx version 1.2 (c) Nicola Talbot
% Command line args:
%   -src "tex-locale.sty\Z=>tex-locale.sty"
%   -src "tex-locale.tex\Z=>tex-locale.tex"
%   -src "tex-locale-scripts-enc.def\Z=>tex-locale-scripts-enc.def"
%   -src "tex-locale-encodings.def\Z=>tex-locale-encodings.def"
%   -src "tex-locale-support.def\Z=>tex-locale-support.def"
%   -author "Nicola Talbot"
%   -section "chapter"
%   -doc "tex-locale-manual.tex"
%   tex-locale
% Created on 2018/8/26 19:14
%\fi
%\iffalse
%<*package>
%% \CharacterTable
%%  {Upper-case    \A\B\C\D\E\F\G\H\I\J\K\L\M\N\O\P\Q\R\S\T\U\V\W\X\Y\Z
%%   Lower-case    \a\b\c\d\e\f\g\h\i\j\k\l\m\n\o\p\q\r\s\t\u\v\w\x\y\z
%%   Digits        \0\1\2\3\4\5\6\7\8\9
%%   Exclamation   \!     Double quote  \"     Hash (number) \#
%%   Dollar        \$     Percent       \%     Ampersand     \&
%%   Acute accent  \'     Left paren    \(     Right paren   \)
%%   Asterisk      \*     Plus          \+     Comma         \,
%%   Minus         \-     Point         \.     Solidus       \/
%%   Colon         \:     Semicolon     \;     Less than     \<
%%   Equals        \=     Greater than  \>     Question mark \?
%%   Commercial at \@     Left bracket  \[     Backslash     \\
%%   Right bracket \]     Circumflex    \^     Underscore    \_
%%   Grave accent  \`     Left brace    \{     Vertical bar  \|
%%   Right brace   \}     Tilde         \~}
%</package>
%\fi
% \iffalse
% Doc-Source file to use with LaTeX2e
% Copyright (C) 2018 Nicola Talbot, all rights reserved.
% \fi
% \iffalse
%<*driver>
\documentclass[report,widecs,inlinetitle]{nlctdoc}

\DeleteShortVerb{|}

\usepackage{metalogo}

\let\orgtheindex\theindex
\let\orgendtheindex\endtheindex
\usepackage{imakeidx}
\usepackage[a4paper,left=1.5in,right=0.5in]{geometry}
\usepackage[colorlinks,
            bookmarks,
            hyperindex=false,
            pdfauthor={Nicola L.C. Talbot},
            pdftitle={tex-locale: setup document locale}]{hyperref}

\CheckSum{3313}

\newcommand{\file}[1]{\texttt{#1}}

\doxitem{Option}{option}{package options}

\renewcommand{\nlctdocmarginfmt}{\scriptsize\raggedright}
\renewcommand\MacroFont{\ttfamily\mdseries}

\renewcommand*{\usage}[1]{\textit{\hyperpage{#1}}}
\renewcommand*{\main}[1]{\underline{\hyperpage{#1}}}
\PageIndex
\setcounter{IndexColumns}{2}
\newcommand*{\PrintCodeIndex}{%
 \bgroup
  \let\theindex\orgtheindex
  \let\endtheindex\orgendtheindex
  \PrintIndex
 \egroup
}
\IndexPrologue{%
\clearpage\phantomsection
\addcontentsline{toc}{chapter}{Code Index}%
\chapter*{Code Index}\markboth{Code Index}{Code Index}%
}

\newenvironment{code}%
{\begin{flushleft}\ttfamily\obeylines\obeyspaces}
{\end{flushleft}\ignorespacesafterend}

\makeindex[name=user,title=Main Index,intoc]
\renewcommand{\iapp}[1]{\index[user]{#1@\appfmt{#1}|hyperpage}}
\renewcommand{\iterm}[1]{\index[user]{#1|hyperpage}}
\renewcommand{\ics}[1]{\cs{#1}\index[user]{#1@\protect\cs{#1}|hyperpage}}
\renewcommand*{\ipkgopt}[2][]{%
 \ifstrempty{#1}%
 {\index[user]{package options:!#2@\pkgoptfmt{#2}|hyperpage}}%
 {\index[user]{package options:!#2@\pkgoptfmt{#2}!#1@\pkgoptfmt{#1}|hyperpage}}%
}
\renewcommand*{\pkgopt}[2][]{%
 \pkgoptfmt{#2}\ifstrempty{#1}%
 {\index[user]{package options:!#2@\pkgoptfmt{#2}|hyperpage}}%
 {\index[user]{package options:!#2@\pkgoptfmt{#2}!#1@\pkgoptfmt{#1}|hyperpage}}%
}
\renewcommand*{\isty}[1]{%
 \index[user]{#1 package@\styfmt{#1} package|hyperpage}}

\newcommand{\attrfmt}[1]{\texttt{#1}}

\newcommand{\attr}[2]{%
 \attrfmt{#2}%
 \index[user]{attributes!#1!#2@\protect\attrfmt{#2}}}

\newcommand{\mattr}[2]{%
  \attr{#1@\protect\meta{#1}}{#2}%
}

\newcommand{\dattr}[1]{\mattr{dialect}{#1}}
\newcommand{\rattr}[1]{\mattr{region}{#1}}
\newcommand{\cattr}[1]{\mattr{currency}{#1}}

\newcommand{\longarg}[1]{\texttt{-{}-#1}}
\newcommand{\shortarg}[1]{\texttt{-#1}}

\newcommand{\PDFLaTeX}{PDF\LaTeX}

\setcounter{secnumdepth}{3}
\setcounter{tocdepth}{3}

\begin{document}
\DocInput{tex-locale.dtx}
\end{document}
%</driver>
%\fi
%
%\MakeShortVerb{"}
%
%\title{tex-locale v1.0:
%setup document locale}
%\author{Nicola L. C. Talbot\\\url{http://www.dickimaw-books.com/}}
%
%\date{2018-08-26}
%\maketitle
%
%\begin{abstract}
%The generic \TeX\ \file{tex-locale.tex} code uses both \sty{tracklang}
%(at least v1.3.4) and \sty{texosquery} (at least v1.2, but newest
%version recommended) to look up the locale information from the
%operating system and provide commands that can access
%locale-dependent information, such as the currency symbol and
%numeric separators. This works best with the shell escape when
%building the document. (Some \hyperref[sec:noshell]{adjustments} are
%needed to work without the shell escape.) \TeX\ Live 2017 now
%includes \app{texosquery-jre8} on the restricted list, but the
%\sty{texosquery} installation needs to be set up to enable this.
%Set up instructions are in the file \file{texosquery.cfg}, which is
%distributed with the \sty{texosquery} package.  If the application
%isn't on the restricted list, then you'll need to enable the
%unrestricted mode, but take care as this mode is insecure.
%
%Example plain \XeTeX\ document:
%\begin{verbatim}
%\font\nimbus="NimbusRoman-Regular" at 10pt
%\nimbus
%
%\input locale
%
%Currency: \CurrentLocaleCurrency.
%Date: \CurrentLocaleDate.
%Time: \CurrentLocaleTime.
%\bye
%\end{verbatim}
%
%The \LaTeX\ package \file{tex-locale.sty} can additionally load 
%\sty{babel} or \sty{polyglossia} with the locale's language 
%setting, as well as various other 
%packages such as \sty{fontspec} (\XeLaTeX\ or \LuaLaTeX) or 
%\sty{fontenc} and \sty{inputenc}. Packages that provide currency
%symbols can also be loaded automatically (\sty{textcomp} by default).
%Example \LaTeX\ document:
%\begin{verbatim}
%\documentclass{article}
%
%\usepackage{tex-locale}
%
%\begin{document}
%Currency: \CurrentLocaleCurrency.
%Date: \CurrentLocaleDate.
%Time: \CurrentLocaleTime.
%\end{document}
%\end{verbatim}
%\end{abstract}
%
%\begin{important}
%\TeX's restricted mode prohibits the use of quotes in the shell
%escape for security reasons. This means that if you want the file
%modification date and you have spaces in your filename you must use
%the unrestricted mode to allow the filename to be delimited by
%quotes. In general it's best to avoid spaces in file names.
%\end{important}
%
%
%\tableofcontents
%\chapter{Introduction}
%
%The \sty{tex-locale} package is designed to set up the document's
%locale-sensitive information by querying the relevant information
%from the operating system using \app{texosquery}.
%\sectionref{sec:generic} describes generic commands that can be used
%in \LaTeX\ or plain \TeX\ documents. \sectionref{sec:latex}
%describes the \LaTeX-specific package \file{tex-locale.sty}, which does
%more than simply input \file{tex-locale.tex}.
%
%The generic code \file{tex-locale.tex} requires: \file{tracklang.tex} and
%\file{texosquery.tex}. The \LaTeX\ package \file{tex-locale.sty}
%additionally requires: \file{tracklang.sty}, \file{texosquery.sty},
%\sty{etoolbox}, \sty{xfor}, \sty{ifxetex}, \sty{ifluatex}, \sty{xkeyval}
%and optionally: 
%\begin{itemize}
%\item \sty{datetime2} (and associated language modules);
%\item \sty{textcomp} or \sty{fontawesome};
%\item \sty{fontspec} or \sty{fontenc}\&\sty{inputenc};
%\item \sty{polyglossia} or \sty{babel};
%\item \sty{CJK}, \sty{CJKutf8} or \sty{xeCJK};
%\item \sty{tracklang-scripts} (provided with \sty{tracklang}).
%\end{itemize}
%
%The \sty{texosquery} package (distributed separately) comes with generic 
%\TeX\ code \file{texosquery.tex}, a simple \LaTeX\ package wrapper
%\file{texosquery.sty} and a Java application that comes in three
%variants:
%\begin{itemize}
%\item \file{texosquery-jre8.jar}: the full application, requires at
%least Java 8.
%\item \file{texosquery.jar}: a slightly cut-down version of the 
%application with less locale support, requires at least Java 7.
%\item \file{texosquery-jre5.jar}: a significantly reduced version of
%the application with poor locale support, requires at least Java 5.
%\emph{Less secure than the other variants.}
%\end{itemize}
%The Java~8 variant (\app{texosquery-jre8}) is on \TeX~Live
%2017's restricted list, so it's possible to use it with  
%the restricted shell escape.  The other variants should not be added
%to the restricted list as old versions of Java are deprecated and
%considered security risks. In particular, the Java~5 variant's file
%listing actions are less secure as they allow file listings outside
%of the current directory path, which the Java~7 and 8 variants
%prohibit. If you have Java~8 installed, I recommend that you make
%\app{texosquery-jre8} the default. This can be done by editing the
%configuration file \file{texosquery.cfg}. See the \sty{texosquery}
%documentation for further details. (This document assumes
%\app{texosquery-jre8} in the examples. Substitute the appropriate command if you
%have a different set up.)
%
%\begin{important}
%MiKTeX users will need to enable piped shell escape with
%\longarg{enable-pipes} (unless the non-shell escape method described
%in \sectionref{sec:noshell} is used).
%\end{important}
%
%The information returned by \app{texosquery} has special markup
%that's converted by \cs{TeXOSQuery} or \cs{TeXOSQueryFromFile}. For
%example, \ics{fhyn} is used to represent a literal hyphen (category
%code~12) whereas \ics{thyn} is used for a textual hyphen.
%For example, if a file name containing a hyphen is returned, the
%hyphen will be marked as \cs{fhyn}, whereas a date containing a
%hyphen will use \cs{thyn}.
%
%\section{Using the package without the shell escape}
%\label{sec:noshell}
%
%The \sty{tex-locale} package is designed for use with the piped shell escape
%(preferably the restricted mode for greater security), but it's
%still possible to use \file{tex-locale.tex} when the shell escape is disabled,
%although it's less convenient. (You will need at least
%\sty{texosquery} v1.4 for this method.) 
%
%First compile your document with the shell escape disabled. The dry
%run mode will automatically be on, and the \sty{texosquery} command
%\cs{TeXOSQuery} will simply write the command it would've tried to
%the transcript. This will be prefixed by \texttt{TeXOSQuery:}
%
%For example, suppose the file \texttt{test.tex} contains:
%\begin{verbatim}
%\def\LocaleMain{en-GB}
%\def\LocaleOther{de-CH-1996,fr-BE}
%
%\input locale
%
%Currency: \CurrentLocaleCurrency.
%
%\bye
%\end{verbatim} 
%Here the main locale has been explicitly set to
%\texttt{en-GB} and the other locales have been set to
%\texttt{de-CH-1996} and \texttt{fr-BE}, so the transcript file
%\texttt{test.log} will include:
%\begin{verbatim}
%TeXOSQuery: texosquery-jre8 -o -r -a -n -N -C -d 'test.tex' -D en-GB
%-D de-CH-1996 -D fr-BE
%\end{verbatim}
%Copy and paste this command into a command prompt or terminal and redirect the
%output to a file. (Don't include the initial \texttt{TeXOSQuery:}
%and don't include any line breaks.) For example (omit line break):
%\begin{verbatim}
%texosquery-jre8 -o -r -a -n -N -C -d 'test.tex' -D en-GB
%-D de-CH-1996 -D fr-BE > localesettings.tex
%\end{verbatim}
%Then define
%\begin{definition}[\DescribeMacro\LocaleQueryFile]
%\cs{LocaleQueryFile}
%\end{definition}
%to the file name before loading \sty{tex-locale}:
%\begin{verbatim}
%\def\LocaleQueryFile{localesettings}
%\def\LocaleMain{en-GB}
%\def\LocaleOther{de-CH-1996,fr-BE}
%
%\input locale
%
%Currency: \CurrentLocaleCurrency.
%
%\bye
%\end{verbatim} 
%If you change any of the document settings you'll need to re-run
%\app{texosquery} to update the query result file.
%
%\begin{important}
%If you want the current date or time, the result files will need to
%be updated before every document build.
%\end{important}
%
%For example, if the file \texttt{test.tex} is simply:
%\begin{verbatim}
%\input locale
%
%Today: \CurrentLocaleShortDate.
%Currency: \CurrentLocaleCurrency.
%
%\bye
%\end{verbatim}
%Then the document build process would be:
%\begin{verbatim}
%texosquery-jre8 -o -r -a -n -N -C -d 'test.tex' -D > testsettings.tex
%etex '\def\LocaleQueryFile{testsettings}\input test'
%\end{verbatim}
%(Replace \app{etex} with \app{pdftex} or \app{xetex}, as required.)
%
%\begin{important}
%If \cs{LocaleQueryFile} is defined and non-empty, the non-shell escape method 
%will automatically be implemented even if the shell escape is actually
%enabled.
%\end{important}
%
%If you're using \LaTeX\ (\file{tex-locale.sty}), there may be an initial query with
%\texttt{-b} or \texttt{-C} (or both) before \file{tex-locale.tex} is
%input. The result of this will also need to be captured in a file.
%For example:
%\begin{verbatim}
%texosquery-jre8 -b -C > localestysettings.tex
%\end{verbatim}
%This file name should be provided in the command
%\begin{definition}[\DescribeMacro\LocaleStyQueryFile]
%\cs{LocaleStyQueryFile}
%\end{definition}
%In this case you only need to update the file if you change the
%document locales, encoding or engine. (For example, if you change the
%value of the package options \pkgopt{main}, \pkgopt{other},
%\pkgopt{support}, \pkgopt{inputenc} or \pkgopt{fontenc}.)
%
%\LaTeX\ example:
%\begin{verbatim}
%\documentclass{article}
%\newcommand{\LocaleStyQueryFile}{localestysettings}
%\newcommand{\LocaleQueryFile}{localesettings}
%\usepackage[main=en-GB,other={de-CH-1996,fr-BE}]{tex-locale}
%\begin{document}
%Currency: \CurrentLocaleCurrency.
%\end{document}
%\end{verbatim} 
%
%\begin{important}
%Even if the shell escape is on, if \cs{LocaleQueryFile} or
%\cs{LocaleStyQueryFile} have been set (to a non-empty value), 
%they'll be used instead.
%\end{important}
%
%It's possible to define one but not the other command. For example:
%\begin{verbatim}
%\documentclass{article}
%\newcommand{\LocaleStyQueryFile}{localestysettings}
%\usepackage[main=en-GB,other={de-CH-1996,fr-BE}]{tex-locale}
%\begin{document}
%Currency: \CurrentLocaleCurrency.
%\end{document}
%\end{verbatim} 
%This requires the shell escape for the main locale information
%(used in \file{tex-locale.tex}) but omits the \LaTeX-specific shell
%escape used in \file{tex-locale.sty}.
%
%\section{Encoding}
%\label{sec:encoding}
%
%The default encoding used by \app{texosquery} when writing the
%results to STDOUT (which are piped in through the shell escape) 
%can be obtained with \longarg{codeset} (from v1.6). For example:
%\begin{verbatim}
%texosquery-jre8 --codeset
%\end{verbatim}
%The short form of this switch is \shortarg{cs}. So the above is
%equivalent to
%\begin{verbatim}
%texosquery-jre8 -cs
%\end{verbatim}
%For example, if the encoding is UTF-8, then this returns
%\begin{verbatim}
%UTF\fhyn 8
%\end{verbatim}
%
%There's a similar switch \longarg{codeset-lcs} (from v1.2) that
%converts the encoding name to lower case and strips hyphens.
%For example
%\begin{verbatim}
%texosquery-jre8 --codeset-lcs
%\end{verbatim}
%The short form is \shortarg{C}
%\begin{verbatim}
%texosquery-jre8 -C
%\end{verbatim}
%For example, if the encoding is UTF-8, then this returns
%\begin{verbatim}
%utf8
%\end{verbatim}
%(The switches are case-sensitive. The lower case \shortarg{c} has a
%different meaning.)
%
%The document encoding must match the encoding used by
%\app{texosquery} when the query is made (through \cs{TeXOSQuery} or
%\cs{TeXOSQueryFromFile}). This is why the \LaTeX\ package
%\file{tex-locale.sty} makes an initial query with \shortarg{C} so that
%it can set up the appropriate input encoding when used with the
%\pkgoptfmt{inputenc=auto} package option. If the document doesn't use
%this interface, then the encoding needs to be correctly set before
%the query is made.
%
%You have a choice of either changing the document encoding to match
%the encoding used by \app{texosquery} or by changing
%\app{texosquery}'s encoding to match all your documents.
%For example, if your Java runtime environment is set up so that the
%default file encoding is ISO~8859-1 (Latin-1) but you always use
%UTF-8 in your source code, then you can instruct \app{texosquery} to
%use UTF-8 with the \longarg{encoding} option (from v1.6). For example:
%\begin{verbatim}
%texosquery-jre8 --encoding UTF-8 -N
%\end{verbatim}
%will ensure that the currency symbol is written as a UTF-8
%character. The short form is \shortarg{enc}. For example:
%\begin{verbatim}
%texosquery-jre8 -enc UTF-8 -N
%\end{verbatim}
%You can append this option to the definition of
%\cs{TeXOSInvokerName}. For example:
%\begin{verbatim}
%\def\TeXOSInvokerName{texosquery-jre8 -enc UTF-8}
%\end{verbatim}
%
%There are two hooks which, if defined, are used immediately before
%or immediately after the main query is made in \file{tex-locale.tex}.
%The hook used before the query is
%\begin{definition}[\DescribeMacro\localeprequery]
%\cs{localeprequery}
%\end{definition}
%and the hook used after the query is
%\begin{definition}[\DescribeMacro\localepostquery]
%\cs{localepostquery}
%\end{definition}
%These may start and end a grouping, if required, as the result from
%the query is given a global assignment, so these may be used to
%temporarily switch the input encoding just for the query. This is
%done by \file{tex-locale.sty} if it detects that the document requires
%\sty{CJKutf8}. In this case it defines the hooks to locally switch
%the encoding while the CJK characters are read from the query result.
%
%In addition to using markup commands like \cs{fhyn} and \cs{thyn},
%\app{texosquery} also wraps non-ASCII characters in the argument of
%\cs{fwrp} (which expands to \cs{texosquerynonasciidetokwrap}) and 
%\cs{twrp} (which expands to \cs{texosquerynonasciiwrap}). The
%expanded versions of these commands may be changed to deal with
%non-ASCII characters, but in general it's simpler to either use the
%same encoding as the document or use the hooks to temporarily switch
%encoding while the information is read.
%
%\chapter{Generic Use}
%\label{sec:generic}
%
%The commands available for generic use are defined in
%\file{tex-locale.tex}. Plain \TeX\ users can load this using \cs{input}:
%\begin{verbatim}
%\input locale
%\end{verbatim}
%\LaTeX\ users are also able to do this, but are better off loading
%\file{tex-locale.sty} instead:
%\begin{verbatim}
%\usepackage{tex-locale}
%\end{verbatim}
%This does more than simply inputting \file{tex-locale.tex} as it also
%loads other packages as well, such as \sty{babel} or
%\sty{polyglossia}.  See \sectionref{sec:latex} for \LaTeX\ package options.
%
%By default \file{tex-locale.tex} assumes that the document locale should
%match your own locale as identified by your operating system. This
%means that the same document code will produce different results
%when compiled in different locations. If you want information for a
%specific locale, then you need to identify the document's main
%locale and optionally other locales. For plain \TeX\ users, this
%means defining \cs{LocaleMain} and \cs{LocaleOther} \emph{before} loading
%\file{tex-locale.tex}. For \LaTeX\ users, this is done through the
%package options provided by \file{tex-locale.sty}.
%
%\begin{definition}[\DescribeMacro\LocaleMain]
%\cs{LocaleMain}
%\end{definition}
%If this is defined before \file{tex-locale.tex} is input, then this may
%be defined as the keyword \texttt{locale}:
%\begin{verbatim}
%\def\LocaleMain{locale}
%\end{verbatim}
%or the relevant language tag. For example:
%\begin{verbatim}
%\def\LocaleMain{de-CH-1996}
%\end{verbatim}
%The \texttt{locale} keyword indicates that the main locale should be
%found by querying the operating system (using \app{texosquery}).
%After \file{tex-locale.tex} has been input, this command will be set to
%the language tag for the main locale. If \cs{LocaleMain} is undefined before
%\file{tex-locale.tex} is input, then \texttt{locale} is assumed.
%
%\begin{definition}[\DescribeMacro\LocaleOther]
%\cs{LocaleOther}
%\end{definition}
%If this is defined before \file{tex-locale.tex} is input, then this
%should be a comma-separated list of language tags for the other
%locales. If \cs{LocaleMain} has been defined to a language tag, the
%list may also include the keyword \texttt{locale} to indicate the
%operating system's locale.
%
%Note that the generic \file{tex-locale.tex} doesn't load \sty{babel} or
%\sty{polyglossia}, but it does track each language using
%\sty{tracklang}'s interface. You can switch to any of these locales
%using\label{pg:selectlocale}
%\begin{definition}[\DescribeMacro\selectlocale]
%\cs{selectlocale}\marg{locale}
%\end{definition}
%where \meta{locale} is either a language tag or \sty{tracklang}
%dialect label. If the locale is unrecognised, an error will occur
%unless \sty{texosquery}'s dry run mode is on, in which case it will
%just be a warning.
%
%This uses \sty{tracklang}'s
%\cs{SetCurrentTrackedDialect}\marg{dialect} but additionally, if
%\cs{selectlanguage} has been defined, it will try to determine the
%correct label to pass to \ics{selectlanguage}.
%
%This means that if you use \LaTeX\ with \file{tex-locale.sty}, which
%loads \sty{babel} or \sty{polyglossia}, then you don't need to
%remember, for example, that \texttt{en-GB} has the \sty{babel}
%dialect label \texttt{british}, and can simply do
%\begin{verbatim}
%\selectlocale{en-GB}
%\end{verbatim}
%If \cs{selectlanguage} isn't defined, then this command doesn't
%switch the document language (that is, it doesn't change the
%hyphenation patterns or \cs{languagename}) but it does allow 
%the \cs{CurrentLocale\ldots} commands (described in \sectionref{sec:currentlocale}) to reflect
%the change in locale.
% 
%Here's a plain \XeTeX\ document that sets up three locales for the
%document. The first (main one) is obtained from the operating system
%(because \cs{LocaleMain} isn't defined) and the other two are
%explicitly set by defining \cs{LocaleOther} before \file{tex-locale.tex}
%is input:
%\begin{verbatim}
%\def\LocaleOther{fr-BE,de-CH-1996}
%
%\input locale
%
%Currency: \CurrentLocaleCurrency.
%
%\selectlocale{fr-BE}
%
%Currency: \CurrentLocaleCurrency.
%
%\selectlocale{de-CH-1996}
%
%Currency: \CurrentLocaleCurrency.
%\bye
%\end{verbatim}
%If this document is in a file called \file{test.tex}, then it can be
%compiled using:
%\begin{verbatim}
%pdftex --shell-escape test
%\end{verbatim}
%or
%\begin{verbatim}
%etex --shell-escape test
%\end{verbatim}
%or
%\begin{verbatim}
%xetex --shell-escape test
%\end{verbatim}
%or
%\begin{verbatim}
%luatex --shell-escape test
%\end{verbatim}
%(Omit \longarg{shell-escape} if the application given by
%\cs{TeXOSInvokerName} is on the restricted list.)
%
%The equivalent \LaTeX\ document is:
%\begin{verbatim}
%\documentclass{article}
%
%\usepackage[other={fr-BE,de-CH-1996}]{tex-locale}
%
%\begin{document}
%Currency: \CurrentLocaleCurrency.
%
%\selectlocale{fr-BE}
%
%Currency: \CurrentLocaleCurrency.
%
%\selectlocale{de-CH-1996}
%
%Currency: \CurrentLocaleCurrency.
%
%\end{document}
%\end{verbatim}
%If this document is in a file called \file{test.tex}, then it can be
%compiled using:
%\begin{verbatim}
%pdflatex --shell-escape test
%\end{verbatim}
%or
%\begin{verbatim}
%latex --shell-escape test
%\end{verbatim}
%or
%\begin{verbatim}
%xelatex --shell-escape test
%\end{verbatim}
%or
%\begin{verbatim}
%lualatex --shell-escape test
%\end{verbatim}
%(Again, omit \longarg{shell-escape} if the application given by
%\cs{TeXOSInvokerName} is on the restricted list.)
%
%In the first two cases (\texttt{pdflatex} or \texttt{latex}),
%\sty{babel} will be loaded (and also \sty{fontenc} and \sty{inputenc}
%and various other packages). Whereas in the last two cases 
%(\texttt{xelatex} or \texttt{lualatex}),
%\sty{polyglossia} will be loaded (and also \sty{fontspec} and
%various other packages). This means that \cs{selectlocale} will also
%use \cs{selectlanguage} to switch the document language. In the
%plain \TeX\ version, \cs{languagename} remains undefined.
%
%The other difference between the plain \TeX\ and the \LaTeX\
%examples, is the default output produced by
%\cs{CurrentLocaleCurrency}. The plain \TeX\ version uses the
%currency codes (for example, EUR) whereas the \LaTeX\ version uses
%currency symbols. These differences are described in more detail
%below.
%
%\section{General Information}
%\label{sec:general}
%
%Each invocation of \app{texosquery} requires starting up the
%Java Virtual Machine, so \file{tex-locale.tex} minimises this overhead by
%only using a single system call. The \file{tex-locale.sty} \LaTeX\
%package may need additional information that also requires a
%\app{texosquery} call before \file{tex-locale.tex} is input. So \LaTeX\
%users will have at most two system calls to \app{texosquery} when
%loading \file{tex-locale.sty} whereas plain \TeX\ users will only have a
%single system call when loading just \file{tex-locale.tex}.
%
%Since it's necessary to run \app{texosquery} to obtain all the
%locale information, \file{tex-locale.tex} uses the opportunity to also
%fetch more general information that may be of use.
%
%\begin{definition}[\DescribeMacro\LocaleOSname]
%\cs{LocaleOSname}
%\end{definition}
%This expands to the operating system name. For example,
%\texttt{Linux}.
%
%\begin{definition}[\DescribeMacro\LocaleOSversion]
%\cs{LocaleOSversion}
%\end{definition}
%This expands to the operating system version. For example,
%\verb|4.1.13-100.fc21.x86_64|. (Note that the underscore in this
%case has been detokenized.)
%
%\begin{definition}[\DescribeMacro\LocaleOSarch]
%\cs{LocaleOSarch}
%\end{definition}
%This expands to the architecture. For example,
%\texttt{amd64}.
%
%\begin{definition}[\DescribeMacro\LocaleOScodeset]
%\cs{LocaleOScodeset}
%\end{definition}
%This expands to the operating system's default file encoding (or the
%value of the Java setting \texttt{file.encoding}). This uses
%\app{texosquery}'s \shortarg{C} switch, which converts the encoding
%name to lower case and strips the hyphens. (For example, \texttt{UTF-8}
%is converted to \texttt{utf8}.) If you prefer to use the
%\shortarg{cs} (or \longarg{codeset}) switch, you can define
%\begin{definition}[\DescribeMacro\LocaleQueryCodesetParam]
%\cs{LocaleQueryCodesetParam}
%\end{definition}
%to the required switch. For example:
%\begin{verbatim}
%\def\LocaleQueryCodesetParam{-cs}
%\input locale
%Encoding: \LocaleOScodeset.
%\bye
%\end{verbatim}
%(Note that \cs{codeset} was added to \app{texosquery} v1.6.)
%
%\begin{definition}[\DescribeMacro\LocaleOStag]
%\cs{LocaleOStag}
%\end{definition}
%This expands to the operating system's language tag, but note that
%this doesn't mean that this locale has been tracked (either as the
%main locale or one of the other document locales). It's for
%informational purposes only.
%
%\begin{definition}[\DescribeMacro\LocaleNowStamp]
%\cs{LocaleNowStamp}
%\end{definition}
%This expands to the current date-time in PDF format. This is like
%the PDF primitive \ics{pdfcreationdate} (also available with \LuaTeX\
%through \cs{pdffeedback}). Since \cs{pdfcreationdate} isn't provided
%by \XeTeX, \cs{LocaleNowStamp} can be used instead if required.
%
%You can also obtain the file modification date of the document
%source file. The file is assumed to be \cs{jobname}\texttt{.tex},
%but you can choose another instead by first defining
%\begin{definition}[\DescribeMacro\LocaleMainFile]
%\cs{LocaleMainFile}
%\end{definition}
%before loading \file{tex-locale.tex}. The value should be the file name.
%For example:
%\begin{verbatim}
%\def\LocaleMainFile{mydoc.tex}
%\input locale
%\end{verbatim}
%You can define this command to empty if you don't want the file 
%modification PDF date-stamp. For example:
%\begin{verbatim}
%\def\LocaleMainFile{}
%\input locale
%\end{verbatim}
%If the file name contains spaces, you can't use the restricted shell
%escape and will have to use the less secure unrestricted mode instead.
%
%If \cs{LocaleMainFile} is not empty, the modification date can be
%accessed using
%\begin{definition}[\DescribeMacro\LocaleFileMod]
%\cs{LocaleFileMod}
%\end{definition}
%For example:
%\begin{verbatim}
%\input locale
%
%\pdfinfo{
%  /CreationDate (\LocaleNowStamp)
%  /ModDate (\LocaleFileMod)
%}
%
%Now: \LocaleNowStamp.
%Mod: \LocaleFileMod.
%
%\bye
%\end{verbatim}
%This command will be empty if the modification date wasn't found.
%
%\section{Attributes}
%\label{sec:attributes}
%
%The \file{tex-locale.tex} package stores and references information
%through the use of attributes. These are set using:
%\begin{definition}[\DescribeMacro\LocaleSetAttribute]
%\cs{LocaleSetAttribute}\marg{label}\marg{attribute}\marg{value}
%\end{definition}
%and accessed using
%\begin{definition}[\DescribeMacro\LocaleGetAttribute]
%\cs{LocaleGetAttribute}\marg{label}\marg{attribute}
%\end{definition}
%The \meta{label} identifies the type of information,
%\meta{attribute} is the attribute's label and \meta{value} is the
%attribute's value.
%
%There are three particular types of identifiers: locales, regions
%and currencies. For convenience, there are some shortcut commands so
%you don't need to use the special prefix part of the label.
%
%For locales, where \meta{dialect} is the \sty{tracklang} dialect
%label associated with that locale:
%\begin{definition}[\DescribeMacro\LocaleSetDialectAttribute]
%\cs{LocaleSetDialectAttribute}\marg{dialect}\marg{attribute}\marg{value}
%\end{definition}
%\begin{definition}[\DescribeMacro\LocaleGetDialectAttribute]
%\cs{LocaleGetDialectAttribute}\marg{dialect}\marg{attribute}
%\end{definition}
%
%Region attributes are set using:
%\begin{definition}[\DescribeMacro\LocaleSetRegionAttribute]
%\cs{LocaleSetRegionAttribute}\marg{region code}\marg{attribute}\marg{value}
%\end{definition}
%where \meta{region code} is the ISO territory code (for example,
%\texttt{GB} or \texttt{CH}).
%You can fetch a region's attribute value using:
%\begin{definition}[\DescribeMacro\LocaleGetRegionAttribute]
%\cs{LocaleGetRegionAttribute}\marg{region code}\marg{attribute}
%\end{definition}
%
%Currency attributes are set using:
%\begin{definition}[\DescribeMacro\LocaleSetCurrencyAttribute]
%\cs{LocaleSetCurrencyAttribute}\marg{currency code}\marg{attribute}\marg{value}
%\end{definition}
%where \meta{currency code} is the ISO currency code (for example,
%\texttt{GBP} or \texttt{EUR}).
%You can fetch a currency's attribute value using:
%\begin{definition}[\DescribeMacro\LocaleGetCurrencyAttribute]
%\cs{LocaleGetCurrencyAttribute}\marg{currency code}\marg{attribute}
%\end{definition}
%
%There's a debugging command available that uses \ics{show} to show
%an attribute's value (or an error if undefined):
%\begin{definition}[\DescribeMacro\localeshowattribute]
%\cs{localeshowattribute}\marg{label}\marg{attribute}
%\end{definition}
%
%Again there are shortcuts for the dialect, region and currency attribute groups:
%\begin{definition}[\DescribeMacro\localeshowdialectattribute]
%\cs{localeshowdialectattribute}\marg{dialect}\marg{attribute}
%\end{definition}
%\begin{definition}[\DescribeMacro\localeshowregionattribute]
%\cs{localeshowregionattribute}\marg{region code}\marg{attribute}
%\end{definition}
%\begin{definition}[\DescribeMacro\localeshowcurrencyattribute]
%\cs{localeshowcurrencyattribute}\marg{currency code}\marg{attribute}
%\end{definition}
%Further details of these and other attribute commands can be found in 
%\sectionref{sec:code.attributes}.
%
%For the most part, the attributes are stored and accessed internally
%by the various commands described in this chapter, so you
%won't usually need to worry about them. However, you can modify the
%attributes. For example, the \texttt{de-CH-1996} locale uses an
%apostrophe \verb|'| as the numeric group separator. This is actually
%returned by \app{texosquery} as the 0x27 straight apostrophe from
%the Basic Latin block. This isn't a problem for plain \TeX\ using
%the default Computer Modern font, as illustrated by the following
%document:
%\begin{verbatim}
%1'234.5
%\bye
%\end{verbatim}
%This uses a curly quote in the resulting 1'234.5. However, changing the font can result in a
%straight apostrophe instead, as illustrated in this plain \XeTeX\
%document:
%\begin{verbatim}
%\font\nimbus="NimbusRoman-Regular" at 10pt
%\nimbus
%
%1'234.5
%\bye
%\end{verbatim}
%This isn't a problem with \XeLaTeX\ if the \sty{fontspec} package is
%used:
%\begin{verbatim}
%\documentclass{article}
%\usepackage{fontspec}
%\setmainfont{NimbusRoman-Regular}
%
%\begin{document}
%1'234.5
%\end{document}
%\end{verbatim}
%
%So a plain \XeTeX\ user might prefer to change the numeric group
%separator to the right single quote character ' (0x2019).
%The numeric group separator is provided by the \dattr{groupsep}
%dialect attribute. The \sty{tracklang} dialect label corresponding
%to the \texttt{de-CH-1996} locale is \texttt{nswissgerman}. So the 
%group separator for this dialect can be changed by setting the
%attribute to the new value using:
%\begin{code}
%\cs{LocaleSetDialectAttribute}\{nswissgerman\}\{groupsep\}\{\textquoteright\}
%\end{code}
%(See \sectionref{sec:generalattr} to convert a language tag to
%dialect label.)
%
%\subsection{Attribute Lists}
%\label{sec:attrlists}
%
%Some attribute values may be comma-separated lists. You can
%add a unique item to a list using:
%\begin{definition}[\DescribeMacro\LocaleAddToAttributeList]
%\cs{LocaleAddToAttributeList}\marg{label}\marg{attribute}\marg{item}
%\end{definition}
%This will do nothing if the item is already in the list, otherwise
%it will append \meta{item} to the list. No
%expansion is performed on \meta{item}. The item may be empty.
%
%\begin{definition}[\DescribeMacro\LocaleXpAddToAttributeList]
%\cs{LocaleXpAddToAttributeList}\marg{label}\marg{attribute}\marg{item}
%\end{definition}
%As above but the first token of \meta{item} is expanded before
%being added to the list.
%
%You can test if an item is in one of those lists using:
%\begin{definition}[\DescribeMacro\LocaleIfInAttributeList]
%\cs{LocaleIfInAttributeList}\marg{label}\marg{attribute}\marg{item}\marg{true}\marg{false}
%\end{definition}
%where \meta{item} is the item. This does \meta{true} if
%\meta{item} is found in the \meta{attribute} list for
%\meta{label}, otherwise it does \meta{false}. If the attribute
%hasn't been defined, \meta{false} is done.
%
%There's a similar command that expands the first token of
%\meta{item} before performing the test:
%\begin{definition}[\DescribeMacro\LocaleIfInAttributeList]
%\cs{LocaleIfXpInAttributeList}\marg{label}\marg{attribute}\marg{item}\marg{true}\marg{false}
%\end{definition}
%
%You can iterate over the list using:
%\begin{definition}[\DescribeMacro\LocaleForEachInAttributeList]
%\cs{LocaleForEachInAttributeList}\marg{label}\marg{attribute}\marg{cs}\marg{body}
%\end{definition}
%This iterates over each element of the list, setting
%the control sequence \meta{cs} to the current item and performing
%\meta{body}. Note that this doesn't skip empty items.
%
%There are shortcut commands for the dialect, region and currency
%attributes:
%
%\begin{definition}[\DescribeMacro\LocaleAddToDialectAttributeList]
%\cs{LocaleAddToDialectAttributeList}\marg{label}\marg{attribute}\marg{item}
%\end{definition}
%
%\begin{definition}[\DescribeMacro\LocaleXpAddToDialectAttributeList]
%\cs{LocaleXpAddToDialectAttributeList}\marg{label}\marg{attribute}\marg{item}
%\end{definition}
%
%\begin{definition}[\DescribeMacro\LocaleIfInDialectAttributeList]
%\cs{LocaleIfInDialectAttributeList}\marg{label}\marg{attribute}\marg{item}\marg{true}\marg{false}
%\end{definition}
%
%\begin{definition}[\DescribeMacro\LocaleIfInDialectAttributeList]
%\cs{LocaleIfXpInDialectAttributeList}\marg{label}\marg{attribute}\marg{item}\marg{true}\marg{false}
%\end{definition}
%
%\begin{definition}[\DescribeMacro\LocaleForEachInDialectAttributeList]
%\cs{LocaleForEachInDialectAttributeList}\marg{label}\marg{attribute}\marg{cs}\marg{body}
%\end{definition}
%
%\begin{definition}[\DescribeMacro\LocaleAddToRegionAttributeList]
%\cs{LocaleAddToRegionAttributeList}\marg{label}\marg{attribute}\marg{item}
%\end{definition}
%
%\begin{definition}[\DescribeMacro\LocaleXpAddToRegionAttributeList]
%\cs{LocaleXpAddToRegionAttributeList}\marg{label}\marg{attribute}\marg{item}
%\end{definition}
%
%\begin{definition}[\DescribeMacro\LocaleIfInRegionAttributeList]
%\cs{LocaleIfInRegionAttributeList}\marg{label}\marg{attribute}\marg{item}\marg{true}\marg{false}
%\end{definition}
%
%\begin{definition}[\DescribeMacro\LocaleIfInRegionAttributeList]
%\cs{LocaleIfXpInRegionAttributeList}\marg{label}\marg{attribute}\marg{item}\marg{true}\marg{false}
%\end{definition}
%
%\begin{definition}[\DescribeMacro\LocaleForEachInRegionAttributeList]
%\cs{LocaleForEachInRegionAttributeList}\marg{label}\marg{attribute}\marg{cs}\marg{body}
%\end{definition}
%
%\begin{definition}[\DescribeMacro\LocaleAddToCurrencyAttributeList]
%\cs{LocaleAddToCurrencyAttributeList}\marg{label}\marg{attribute}\marg{item}
%\end{definition}
%
%\begin{definition}[\DescribeMacro\LocaleXpAddToCurrencyAttributeList]
%\cs{LocaleXpAddToCurrencyAttributeList}\marg{label}\marg{attribute}\marg{item}
%\end{definition}
%
%\begin{definition}[\DescribeMacro\LocaleIfInCurrencyAttributeList]
%\cs{LocaleIfInCurrencyAttributeList}\marg{label}\marg{attribute}\marg{item}\marg{true}\marg{false}
%\end{definition}
%
%\begin{definition}[\DescribeMacro\LocaleIfInCurrencyAttributeList]
%\cs{LocaleIfXpInCurrencyAttributeList}\marg{label}\marg{attribute}\marg{item}\marg{true}\marg{false}
%\end{definition}
%
%\begin{definition}[\DescribeMacro\LocaleForEachInCurrencyAttributeList]
%\cs{LocaleForEachInCurrencyAttributeList}\marg{label}\marg{attribute}\marg{cs}\marg{body}
%\end{definition}
%
%\subsection{General Attributes}
%\label{sec:generalattr}
%
%The language tag attribute type \mattr{lang tag}{tagtodialect} is set using 
%\begin{code}
%\cs{LocaleSetAttribute}\marg{label}\{tagtodialect\}\marg{value}
%\end{code}
%and can be accessed using
%\begin{code}
%\cs{LocaleGetAttribute}\marg{label}\{tagtodialect\}
%\end{code}
%The \meta{label} is a language tag (such as \texttt{de-CH-1996})
%and the \meta{value} is the associated \sty{tracklang} dialect label
%(for example, \texttt{nswissgerman}).
%This attribute is used by \ics{selectlocale} to convert the language
%tag to the corresponding dialect label.
%
%If the time zone information was fetched with \app{texosquery}'s
%\texttt{-Z} switch, then the time zone IDs will be stored
%in the time zone \attr{timezone}{id} attribute list that can be fetched with
%\begin{verbatim}
%\LocaleGetAttribute{timezone}{id}
%\end{verbatim}
%or iterated over using \cs{LocaleForEachInAttributeList}.
%For example:
%\begin{verbatim}
%\def\LocaleIfDateTimePatternsSupported#1#2{#1}
%
%\input locale
%
%Time zone IDs:
%
%\LocaleForEachInAttributeList{timezone}{id}{\ThisId}{\ThisId\endgraf}
%
%\bye
%\end{verbatim}
%This simply lists all the known time zone identifiers. (Since
%\cs{LocaleForEachInAttributeList} is a short command,
%\cs{endgraf} is used to create a paragraph break in the above
%example.)
%
%There are two general currency attribute lists, where the
%\meta{label} is \texttt{currencies} and the attribute values
%are \attr{currencies}{official} for the list of official currency codes
%and \attr{currencies}{regional} for the regional currency codes.
%For example:
%\begin{verbatim}
%\def\LocaleMain{en-GB}
%\def\LocaleOther{en-IM,en-GG,fr-BE,nl-BE,de-DE-1996}
%
%\input locale
%
%Official: \LocaleGetAttribute{currencies}{official}.
%Regional: \LocaleGetAttribute{currencies}{regional}.
%
%\bye
%\end{verbatim}
%produces:
%\begin{quote}
%Official: GBP,EUR. Regional: GBP,IMP,GGP,EUR.
%\end{quote}
%
%You can iterate over these lists using
%\cs{LocaleForEachInAttributeList}. For example:
%\begin{verbatim}
%% arara: xetex: {shell: on}
%\font\nimbus="NimbusRoman-Regular" at 10pt
%\nimbus
%
%\def\LocaleMain{en-GB}
%\def\LocaleOther{en-IM,en-GG,fr-BE,nl-BE,de-DE-1996}
%
%\input locale
%
%\LocaleForEachInAttributeList{currencies}{regional}{\thiscode}%
%{\thiscode\ (\LocaleGetCurrencyAttribute{\thiscode}{sym}) }
%
%\bye
%\end{verbatim}
%which produces:
%\begin{quote}
%GBP (\pounds) IMP (M\pounds) GGP (\pounds) EUR (\texteuro)
%\end{quote}
%
%\subsection{Dialect Attributes}
%\label{sec:dialectattr}
%
%The following attributes are associated with dialects and can be set
%using
%\begin{code}
%\cs{LocaleSetDialectAttribute}\marg{dialect}\marg{attribute}\marg{value}
%\end{code}
%or fetched with
%\begin{code}
%\cs{LocaleGetDialectAttribute}\marg{dialect}\marg{attribute}
%\end{code}
%where \meta{dialect} is the \sty{tracklang} dialect label.
%Known values of \meta{attribute} are listed below.
%
%\subsubsection{General Dialect Attributes}
%\label{sec:dialectgenattr}
%
%\begin{itemize}
%\item\dattr{langtag}: the \meta{value} is the language tag associated with the 
%dialect label \meta{dialect}.
%\item\dattr{langname}: the \meta{value} is the language name associated with the 
%dialect label \meta{dialect}.
%\item\dattr{nativelangname}: the \meta{value} is the native language name 
%associated with the dialect label \meta{dialect}.
%\item\dattr{regionname}: the \meta{value} is the region name associated with the 
%dialect label \meta{dialect}.
%\item\dattr{nativeregionname}: the \meta{value} is the native region name 
%associated with the dialect label \meta{dialect}.
%\item\dattr{variantname}: the \meta{value} is the variant name associated with the 
%dialect label \meta{dialect}.
%\item\dattr{nativevariantname}: the \meta{value} is the native variant name associated with the 
%dialect label \meta{dialect}.
%\end{itemize}
%
%\subsubsection{Dates and Times Dialect Attributes}
%\label{sec:attrdatetime}
%
%The commands described in \sectionref{sec:datetimes} use these attributes.
%
%\begin{itemize}
%\item\dattr{fulldate}: the \meta{value} is the full date associated with the 
%dialect label \meta{dialect}. (Used by \ics{LocaleFullDate}.)
%\item\dattr{longdate}: the \meta{value} is the long date associated with the 
%dialect label \meta{dialect}. (Used by \ics{LocaleLongDate}.)
%\item\dattr{meddate}: the \meta{value} is the medium date associated with the 
%dialect label \meta{dialect}. (Used by \ics{LocaleMediumDate}.)
%\item\dattr{shortdate}: the \meta{value} is the short date associated with the 
%dialect label \meta{dialect}. (Used by \ics{LocaleShortDate}.)
%\item\dattr{firstday}: the \meta{value} is the index of the first day of the week associated with the 
%dialect label \meta{dialect}. (Used by \ics{LocaleFirstDayIndex}.)
%\item\dattr{fulltime}: the \meta{value} is the full time associated with the 
%dialect label \meta{dialect}. (Used by \ics{LocaleFullTime}.)
%\item\dattr{longtime}: the \meta{value} is the long time associated with the 
%dialect label \meta{dialect}. (Used by \ics{LocaleLongTime}.)
%\item\dattr{medtime}: the \meta{value} is the medium time associated with the 
%dialect label \meta{dialect}. (Used by \ics{LocaleMediumTime}.)
%\item\dattr{shorttime}: the \meta{value} is the short time associated with the 
%dialect label \meta{dialect}. (Used by \ics{LocaleShortTime}.)
%\item\dattr{fulldatetime}: the \meta{value} is the full date and time 
%associated with the dialect label \meta{dialect}. (Used by
%\ics{LocaleFullDateTime}.)
%\item\dattr{longdatetime}: the \meta{value} is the long date and time 
%associated with the dialect label \meta{dialect}. (Used by
%\ics{LocaleLongDateTime}.)
%\item\dattr{meddatetime}: the \meta{value} is the medium date and time 
%associated with the dialect label \meta{dialect}. (Used by
%\ics{LocaleMediumDateTime}.)
%\item\dattr{shortdatetime}: the \meta{value} is the short date and time 
%associated with the dialect label \meta{dialect}. (Used by
%\ics{LocaleShortDateTime}.)
%\end{itemize}
%Note that the combined date and time attributes (such as
%\attrfmt{fulldatetime}) aren't used by
%\ics{CurrentLocaleDateTime}.
%
%\paragraph{Patterns}
%The following attributes store date or time patterns (see
%\sectionref{sec:datetimepat}).
%\begin{itemize}
%\item\dattr{fulldatefmt}: the \meta{value} is the full date format associated 
%with the dialect label \meta{dialect}.
%\item\dattr{longdatefmt}: the \meta{value} is the long date format associated 
%with the dialect label \meta{dialect}.
%\item\dattr{meddatefmt}: the \meta{value} is the medium date format associated 
%with the dialect label \meta{dialect}.
%\item\dattr{shortdatefmt}: the \meta{value} is the short date format associated 
%with the dialect label \meta{dialect}.
%\item\dattr{fulltimefmt}: the \meta{value} is the full time format associated 
%with the dialect label \meta{dialect}.
%\item\dattr{longtimefmt}: the \meta{value} is the long time format associated 
%with the dialect label \meta{dialect}.
%\item\dattr{medtimefmt}: the \meta{value} is the medium time format associated 
%with the dialect label \meta{dialect}.
%\item\dattr{shorttimefmt}: the \meta{value} is the short time format associated 
%with the dialect label \meta{dialect}.
%\item\dattr{fulldatetimefmt}: the \meta{value} is the full date time format 
%associated with the dialect label \meta{dialect}.
%\item\dattr{longdatetimefmt}: the \meta{value} is the long date time format 
%associated with the dialect label \meta{dialect}.
%\item\dattr{meddatetimefmt}: the \meta{value} is the medium date time format 
%associated with the dialect label \meta{dialect}.
%\item\dattr{shortdatetimefmt}: the \meta{value} is the short date time format 
%associated with the dialect label \meta{dialect}.
%\end{itemize}
%
%\paragraph{Day Names}
%
%The commands \ics{LocaleDayName}, \ics{LocaleShortDayName},
%\ics{LocaleStandaloneDayName} and \ics{LocaleStandaloneShortDayName}
%use these attributes.
%
%\begin{itemize}
%\item\dattr{day.0}: the \meta{value} is the name for day~0 (Monday)
%associated with the dialect label \meta{dialect}.
%\item\dattr{day.1}: the \meta{value} is the name for day~1 (Tuesday) 
%associated with the dialect label \meta{dialect}.
%\item\dattr{day.2}: the \meta{value} is the name for day~2 (Wednesday) 
%associated with the dialect label \meta{dialect}.
%\item\dattr{day.3}: the \meta{value} is the name for day~3 (Thursday) 
%associated with the dialect label \meta{dialect}.
%\item\dattr{day.4}: the \meta{value} is the name for day~4 (Friday) 
%associated with the dialect label \meta{dialect}.
%\item\dattr{day.5}: the \meta{value} is the name for day~5 (Saturday) 
%associated with the dialect label \meta{dialect}.
%\item\dattr{day.6}: the \meta{value} is the name for day~6 (Sunday) 
%associated with the dialect label \meta{dialect}.
%\item\dattr{shortday.0}: the \meta{value} is the short name for day~0 (Monday)
%associated with the dialect label \meta{dialect}.
%\item\dattr{shortday.1}: the \meta{value} is the short name for day~1 (Tuesday) 
%associated with the dialect label \meta{dialect}.
%\item\dattr{shortday.2}: the \meta{value} is the short name for day~2 (Wednesday) 
%associated with the dialect label \meta{dialect}.
%\item\dattr{shortday.3}: the \meta{value} is the short name for day~3 (Thursday) 
%associated with the dialect label \meta{dialect}.
%\item\dattr{shortday.4}: the \meta{value} is the short name for day~4 (Friday) 
%associated with the dialect label \meta{dialect}.
%\item\dattr{shortday.5}: the \meta{value} is the short name for day~5 (Saturday) 
%associated with the dialect label \meta{dialect}.
%\item\dattr{shortday.6}: the \meta{value} is the short name for day~6 (Sunday) 
%associated with the dialect label \meta{dialect}.
%\item\dattr{standalone.day.0}: the \meta{value} is the name for day~0 (Monday)
%associated with the dialect label \meta{dialect}.
%\item\dattr{standalone.day.1}: the \meta{value} is the standalone name for day~1 (Tuesday) 
%associated with the dialect label \meta{dialect}.
%\item\dattr{standalone.day.2}: the \meta{value} is the standalone name for day~2 (Wednesday) 
%associated with the dialect label \meta{dialect}.
%\item\dattr{standalone.day.3}: the \meta{value} is the standalone name for day~3 (Thursday) 
%associated with the dialect label \meta{dialect}.
%\item\dattr{standalone.day.4}: the \meta{value} is the standalone name for day~4 (Friday) 
%associated with the dialect label \meta{dialect}.
%\item\dattr{standalone.day.5}: the \meta{value} is the standalone name for day~5 (Saturday) 
%associated with the dialect label \meta{dialect}.
%\item\dattr{standalone.day.6}: the \meta{value} is the standalone name for day~6 (Sunday) 
%associated with the dialect label \meta{dialect}.
%\item\dattr{standalone.shortday.0}: the \meta{value} is the standalone short name for day~0 (Monday)
%associated with the dialect label \meta{dialect}.
%\item\dattr{standalone.shortday.1}: the \meta{value} is the standalone short name for day~1 (Tuesday) 
%associated with the dialect label \meta{dialect}.
%\item\dattr{standalone.shortday.2}: the \meta{value} is the standalone short name for day~2 (Wednesday) 
%associated with the dialect label \meta{dialect}.
%\item\dattr{standalone.shortday.3}: the \meta{value} is the standalone short name for day~3 (Thursday) 
%associated with the dialect label \meta{dialect}.
%\item\dattr{standalone.shortday.4}: the \meta{value} is the standalone short name for day~4 (Friday) 
%associated with the dialect label \meta{dialect}.
%\item\dattr{standalone.shortday.5}: the \meta{value} is the standalone short name for day~5 (Saturday) 
%associated with the dialect label \meta{dialect}.
%\item\dattr{standalone.shortday.6}: the \meta{value} is the standalone short name for day~6 (Sunday) 
%associated with the dialect label \meta{dialect}.
%\end{itemize}
%
%\paragraph{Month Names}
%
%The commands \ics{LocaleMonthName}, \ics{LocaleShortMonthName},
%\ics{LocaleStandaloneMonthName} and \ics{LocaleStandaloneShortMonthName}
%use these attributes.
%
%\begin{itemize}
%\item\dattr{month.1}: the \meta{value} is the name for month~1
%(January) associated with the dialect label \meta{dialect}.
%\item\dattr{month.2}: the \meta{value} is the name for month~2
%(February) associated with the dialect label \meta{dialect}.
%\item\dattr{month.3}: the \meta{value} is the name for month~3
%(March) associated with the dialect label \meta{dialect}.
%\item\dattr{month.4}: the \meta{value} is the name for month~4
%(April) associated with the dialect label \meta{dialect}.
%\item\dattr{month.5}: the \meta{value} is the name for month~5
%(May) associated with the dialect label \meta{dialect}.
%\item\dattr{month.6}: the \meta{value} is the name for month~6
%(June) associated with the dialect label \meta{dialect}.
%\item\dattr{month.7}: the \meta{value} is the name for month~7
%(July) associated with the dialect label \meta{dialect}.
%\item\dattr{month.8}: the \meta{value} is the name for month~8
%(August) associated with the dialect label \meta{dialect}.
%\item\dattr{month.9}: the \meta{value} is the name for month~9
%(September) associated with the dialect label \meta{dialect}.
%\item\dattr{month.10}: the \meta{value} is the name for month~10
%(October) associated with the dialect label \meta{dialect}.
%\item\dattr{month.11}: the \meta{value} is the name for month~11
%(November) associated with the dialect label \meta{dialect}.
%\item\dattr{month.12}: the \meta{value} is the name for month~12
%(December) associated with the dialect label \meta{dialect}.
%\item\dattr{shortmonth.1}: the \meta{value} is the short name for month~1
%(January) associated with the dialect label \meta{dialect}.
%\item\dattr{shortmonth.2}: the \meta{value} is the short name for month~2
%(February) associated with the dialect label \meta{dialect}.
%\item\dattr{shortmonth.3}: the \meta{value} is the short name for month~3
%(March) associated with the dialect label \meta{dialect}.
%\item\dattr{shortmonth.4}: the \meta{value} is the short name for month~4
%(April) associated with the dialect label \meta{dialect}.
%\item\dattr{shortmonth.5}: the \meta{value} is the short name for month~5
%(May) associated with the dialect label \meta{dialect}.
%\item\dattr{shortmonth.6}: the \meta{value} is the short name for month~6
%(June) associated with the dialect label \meta{dialect}.
%\item\dattr{shortmonth.7}: the \meta{value} is the short name for month~7
%(July) associated with the dialect label \meta{dialect}.
%\item\dattr{shortmonth.8}: the \meta{value} is the short name for month~8
%(August) associated with the dialect label \meta{dialect}.
%\item\dattr{shortmonth.9}: the \meta{value} is the short name for month~9
%(September) associated with the dialect label \meta{dialect}.
%\item\dattr{shortmonth.10}: the \meta{value} is the short name for month~10
%(October) associated with the dialect label \meta{dialect}.
%\item\dattr{shortmonth.11}: the \meta{value} is the short name for month~11
%(November) associated with the dialect label \meta{dialect}.
%\item\dattr{shortmonth.12}: the \meta{value} is the short name for month~12
%(December) associated with the dialect label \meta{dialect}.
%\item\dattr{standalone.month.1}: the \meta{value} is the standalone name for month~1
%(January) associated with the dialect label \meta{dialect}.
%\item\dattr{standalone.month.2}: the \meta{value} is the standalone name for month~2
%(February) associated with the dialect label \meta{dialect}.
%\item\dattr{standalone.month.3}: the \meta{value} is the standalone name for month~3
%(March) associated with the dialect label \meta{dialect}.
%\item\dattr{standalone.month.4}: the \meta{value} is the standalone name for month~4
%(April) associated with the dialect label \meta{dialect}.
%\item\dattr{standalone.month.5}: the \meta{value} is the standalone name for month~5
%(May) associated with the dialect label \meta{dialect}.
%\item\dattr{standalone.month.6}: the \meta{value} is the standalone name for month~6
%(June) associated with the dialect label \meta{dialect}.
%\item\dattr{standalone.month.7}: the \meta{value} is the standalone name for month~7
%(July) associated with the dialect label \meta{dialect}.
%\item\dattr{standalone.month.8}: the \meta{value} is the standalone name for month~8
%(August) associated with the dialect label \meta{dialect}.
%\item\dattr{standalone.month.9}: the \meta{value} is the standalone name for month~9
%(September) associated with the dialect label \meta{dialect}.
%\item\dattr{standalone.month.10}: the \meta{value} is the standalone name for month~10
%(October) associated with the dialect label \meta{dialect}.
%\item\dattr{standalone.month.11}: the \meta{value} is the standalone name for month~11
%(November) associated with the dialect label \meta{dialect}.
%\item\dattr{standalone.month.12}: the \meta{value} is the standalone name for month~12
%(December) associated with the dialect label \meta{dialect}.
%\item\dattr{standalone.shortmonth.1}: the \meta{value} is the standalone short name for month~1
%(January) associated with the dialect label \meta{dialect}.
%\item\dattr{standalone.shortmonth.2}: the \meta{value} is the standalone short name for month~2
%(February) associated with the dialect label \meta{dialect}.
%\item\dattr{standalone.shortmonth.3}: the \meta{value} is the standalone short name for month~3
%(March) associated with the dialect label \meta{dialect}.
%\item\dattr{standalone.shortmonth.4}: the \meta{value} is the standalone short name for month~4
%(April) associated with the dialect label \meta{dialect}.
%\item\dattr{standalone.shortmonth.5}: the \meta{value} is the standalone short name for month~5
%(May) associated with the dialect label \meta{dialect}.
%\item\dattr{standalone.shortmonth.6}: the \meta{value} is the standalone short name for month~6
%(June) associated with the dialect label \meta{dialect}.
%\item\dattr{standalone.shortmonth.7}: the \meta{value} is the standalone short name for month~7
%(July) associated with the dialect label \meta{dialect}.
%\item\dattr{standalone.shortmonth.8}: the \meta{value} is the standalone short name for month~8
%(August) associated with the dialect label \meta{dialect}.
%\item\dattr{standalone.shortmonth.9}: the \meta{value} is the standalone short name for month~9
%(September) associated with the dialect label \meta{dialect}.
%\item\dattr{standalone.shortmonth.10}: the \meta{value} is the standalone short name for month~10
%(October) associated with the dialect label \meta{dialect}.
%\item\dattr{standalone.shortmonth.11}: the \meta{value} is the standalone short name for month~11
%(November) associated with the dialect label \meta{dialect}.
%\item\dattr{standalone.shortmonth.12}: the \meta{value} is the standalone short name for month~12
%(December) associated with the dialect label \meta{dialect}.
%\end{itemize}
%
%\subsubsection{Time Zones}
%
%These attributes won't be set if the time zone information wasn't
%fetched (with \app{texosquery}'s \texttt{-Z} switch). These attributes 
%include the time zone identifier in the
%attribute label. (The list of known identifiers is stored
%in the \texttt{timezone} \attr{timezone}{id} attribute.) For example, \texttt{Europe/London} is the
%identifier for the UK time zone. So the attribute
%\texttt{timezone.Europe/London.short} is for the short name for that
%zone (\texttt{GMT}) and \texttt{timezone.Europe/London.shortdst}
%is for the short daylight saving name for that zone (\texttt{BST}).
%
%\begin{itemize}
%\item\dattr{timezone.\meta{zone}.short}: the \meta{value} is the
%short name for the zone identified by \meta{zone} 
%associated with the dialect label \meta{dialect}.
%\item\dattr{timezone.\meta{zone}.long}: the \meta{value} is the
%long name for the zone identified by \meta{zone} 
%associated with the dialect label \meta{dialect}.
%\item\dattr{timezone.\meta{zone}.shortdst}: the \meta{value} is the
%short daylight saving name for the zone identified by \meta{zone} 
%associated with the dialect label \meta{dialect}.
%\item\dattr{timezone.\meta{zone}.longdst}: the \meta{value} is the
%long daylight saving name for the zone identified by \meta{zone} 
%associated with the dialect label \meta{dialect}.
%\end{itemize}
%
%\subsubsection{Numeric}
%\label{sec:attrnum}
%
%\begin{itemize}
%\item\dattr{groupsep}: the \meta{value} is the numeric group
%separator associated with the dialect label \meta{dialect}. (Used by
%\ics{LocaleNumericGroupSep}.)
%\item\dattr{decsep}: the \meta{value} is the numeric decimal
%separator associated with the dialect label \meta{dialect}. (Used by
%\ics{LocaleNumericDecimalSep}.)
%\item\dattr{currencysep}: the \meta{value} is the numeric currency
%separator associated with the dialect label \meta{dialect}. (Used by
%\ics{LocaleNumericMonetarySep}.)
%\item\dattr{exp}: the \meta{value} is the exponent symbol associated 
%with the dialect label \meta{dialect}. (Used by
%\ics{LocaleNumericExponent}.)
%\item\dattr{usesgroup}: the \meta{value} (either 1 or 0) indicates if
%the number group separator should be used 
%with the dialect label \meta{dialect}. (Used by
%\ics{LocaleIfNumericUsesGroup}.)
%\item\dattr{percent}: the \meta{value} is the percentage symbol associated 
%with the dialect label \meta{dialect}. (Used by
%\ics{LocaleNumericPercent}.)
%\item\dattr{permill}: the \meta{value} is the per mill symbol associated 
%with the dialect label \meta{dialect}. (Used by
%\ics{LocaleNumericPermill}.)
%\item\dattr{decfmt}: the \meta{value} is the decimal format associated 
%with the dialect label \meta{dialect}. (Used by
%\ics{CurrentLocaleDecimalPattern}.)
%\item\dattr{intfmt}: the \meta{value} is the integer format associated 
%with the dialect label \meta{dialect}. (Used by
%\ics{CurrentLocaleIntegerPattern}.)
%\item\dattr{curfmt}: the \meta{value} is the currency format associated 
%with the dialect label \meta{dialect}. (Used by
%\ics{CurrentLocaleCurrencyPattern}.)
%\item\dattr{perfmt}: the \meta{value} is the percent format associated 
%with the dialect label \meta{dialect}. (Used by
%\ics{CurrentLocalePercentPattern}.)
%\item\dattr{currency}: the \meta{value} is the currency code associated 
%with the dialect label \meta{dialect}. (Used by
%\ics{LocaleCurrencyLabel}.)
%\item\dattr{regionalcurrency}: the \meta{value} is the regional currency code 
%associated with the dialect label \meta{dialect}. (Used by
%\ics{LocaleCurrencyRegionalLabel}.)
%\item\dattr{currencysym}: the \meta{value} is the currency symbol associated 
%with the dialect label \meta{dialect}. (Used by
%\ics{LocaleCurrencySymbol}, if the \attrfmt{currency} 
%attribute isn't \texttt{XXX}.)
%\item\dattr{currencytex}: the \meta{value} is the \TeX\ code
%representing the currency symbol associated 
%with the dialect label \meta{dialect}. (Used by
%\ics{LocaleCurrencyTeXSymbol}, if the \attrfmt{currency} 
%attribute isn't \texttt{XXX}.)
%\end{itemize}
%
%Note that the currency information is also available through
%the currency attributes \cattr{official}, \cattr{sym} and \cattr{tex}.
%The dialect attribute allows for a different symbol to be
%used within the context of a specific dialect. For example,
%if a country has multiple scripts (such as Latin and Cyrillic)
%then the dialect currency symbol can reflect a specific script,
%whereas the currency \cattr{sym} attribute might be in the default
%script.
%
%
%\subsection{Region Attributes}
%\label{sec:regionattr}
%
%The following attributes are associated with region and can be set
%using
%\begin{code}
%\cs{LocaleSetRegionAttribute}\marg{region code}\marg{attribute}\marg{value}
%\end{code}
%or fetched with
%\begin{code}
%\cs{LocaleGetRegionAttribute}\marg{region code}\marg{attribute}
%\end{code}
%
%\begin{itemize}
%\item\rattr{currency}: the regional currency code (for
%example, \texttt{IMP}).
%\item\rattr{dialect}: a comma-separated list of the dialects
%associated with this region. For example:
%\begin{verbatim}
%\def\LocaleMain{en-GB}
%\def\LocaleOther{fr-BE,nl-BE}
%\input locale
%
%GB: \LocaleGetRegionAttribute{GB}{dialect}.
%BE: \LocaleGetRegionAttribute{BE}{dialect}.
%
%\bye
%\end{verbatim}
%produces: GB: british. BE: belgique,flemish.
%\end{itemize}
%
%
%\subsection{Currency Attributes}
%\label{sec:currencyattr}
%
%The following attributes are associated with currency and can be set
%using
%\begin{code}
%\cs{LocaleSetCurrencyAttribute}\marg{code}\marg{attribute}\marg{value}
%\end{code}
%or fetched with
%\begin{code}
%\cs{LocaleGetCurrencyAttribute}\marg{code}\marg{attribute}
%\end{code}
%where \meta{code} is the ISO currency code or regional currency
%code.
%
%\begin{itemize}
%\item\cattr{region}: a comma-separated list of regions that
%use this currency. If the regional currency code is different from
%the official ISO currency code, then both codes will have this
%attribute set. For example:
%\begin{verbatim}
%\def\LocaleMain{en-GB}
%\def\LocaleOther{en-IM}
%
%\input locale
%
%GBP: \LocaleGetCurrencyAttribute{GBP}{region}.
%IMP: \LocaleGetCurrencyAttribute{IMP}{region}.
%
%\bye
%\end{verbatim}
%produces: GBP: GB,IM. IMP: IM.
%
%\item \cattr{official}: the official ISO currency code for
%\meta{code}. In most cases this will be the same as \meta{code}.
%For example:
%\begin{verbatim}
%\def\LocaleMain{en-GB}
%\def\LocaleOther{en-IM}
%
%\input locale
%
%GBP: \LocaleGetCurrencyAttribute{GBP}{official}.
%IMP: \LocaleGetCurrencyAttribute{IMP}{official}.
%
%\bye
%\end{verbatim}
%produces: GBP: GBP. IMP: GBP.
%
%\item\cattr{sym}: the currency symbol (using characters, not
%\TeX\ commands, except for \verb|\$|) for regional currency
%\meta{code}.
%
%\item\cattr{tex}: the currency symbol using \TeX\ code (such
%as \cs{texosquerycurrencypound}) for regional currency
%\meta{code}.
%\end{itemize}
%
%\section{Patterns}
%\label{sec:patterns}
%
%The \sty{texosquery} package provides for two types of patterns:
%date-time and numeric. The date-time patterns may just display a
%date, or just a time or both. They may or may not include a time
%zone. The numeric patterns can be used to format integers, decimals,
%scientific notation, percentages or currency. New patterns can be
%defined by commands provided in \file{texosquery.tex}. The 
%\sty{tex-locale} package can also fetch the patterns for the document
%locales so that they can be applied to data in the document
%to match the current locale.
%
%The numeric patterns are fairly easy to use but are tricky to
%define. The date-time patterns are fairly easy to define but are
%tricky to use.
%
%\subsection{Numeric Patterns}
%\label{sec:numpatterns}
%
%The \sty{texosquery} numeric pattern symbol commands are redefined
%by \sty{tex-locale} to use the appropriate symbol from the current
%locale. For example, \ics{texosquerypatfmtgroupsep} is redefined to
%use \cs{CurrentLocaleNumericGroupSep}. (See the \sty{texosquery}
%documentation for further details of these pattern commands.)
%
%You can access a numeric pattern for a particular locale by querying
%one of the dialect attributes listed in \sectionref{sec:attrnum}: \dattr{intfmt} (integer pattern), 
%\dattr{decfmt} (decimal pattern), \dattr{curfmt} (currency pattern) or 
%\dattr{perfmt} (percent pattern).
%
%These can be accessed explicitly using
%\begin{definition}
%\ics{LocaleGetDialectAttribute}\marg{dialect}\marg{attribute}
%\end{definition}
%For example:
%\begin{verbatim}
%\LocaleGetDialectAttribute{british}{intfmt}
%\end{verbatim}
%However it's simpler to use the shortcut commands described in 
%\sectionref{sec:currpat}. For example:
%\begin{verbatim}
%\CurrentLocaleIntegerPattern
%\end{verbatim}
%
%A pattern can be used in two ways. If it's simply used in the
%document, then it just reproduces the pattern specification. For
%example:
%\begin{verbatim}
%\def\LocaleMain{en-GB}
%\input locale
%
%{\tt \CurrentLocaleDecimalPattern}
%\bye
%\end{verbatim}
%This simply produces:
%\begin{quote}
%\texttt{\#,\#\#\#,\#\#\#,\#\#0.\#\#\#\#\#\#\#\#\#\#}
%\end{quote}
%This can be used for debugging purposes to check the pattern.
%
%The second use is to apply the pattern to a number. This is done
%with \sty{texosquery}'s \cs{texosqueryfmtnumber} command:
%\begin{definition}
%\ics{texosqueryfmtnumber}\marg{pattern}\marg{int part}\marg{frac
%part}\marg{mantissa}
%\end{definition}
%where \meta{pattern} is the pattern (defined using
%\cs{texosquerydefpattern} or fetched with the \app{texosquery}
%application), \meta{int part} is the integer part, \meta{frac part}
%is the fractional part and \meta{mantissa} is the exponent part.
%
%For example:
%\begin{verbatim}
%% arara: xetex: {shell: on}
%
%\def\LocaleMain{en-GB}
%\input locale
%
%\texosqueryfmtnumber{\CurrentLocaleIntegerPattern}{12}{34567}{4}
%
%\texosqueryfmtnumber{\CurrentLocaleCurrencyPattern}{12345}{67}{0}
%
%\texosqueryfmtnumber{\CurrentLocaleDecimalPattern}{123}{4567}{2}
%
%\texosqueryfmtnumber{\CurrentLocalePercentPattern}{123}{4567}{2}
%\bye
%\end{verbatim}
%This produces:
%\begin{quote}
%123,456
%
%GBP12,345.67
%
%12,345.67
%
%1,234,567\%
%\end{quote}
%There are some shortcut commands provided, to allow for more
%convenient code.
%
%A numeric pattern can be applied to a numeric value using
%\begin{definition}[\DescribeMacro\localenumfmt]
%\cs{localenumfmt}\marg{pattern}\marg{decimal}
%\end{definition}
%where \meta{pattern} is the pattern (as for \cs{texosqueryfmtnumber})
%and \meta{decimal} is the numeric value in the form
%\meta{int}\texttt{.}\meta{frac}\texttt{E}\meta{mantissa}
%(where \texttt{.} represents the decimal separator and \texttt{E} represents 
%the exponent separator, regardless of locale). The \texttt{.}\meta{frac} and 
%\texttt{E}\meta{mantissa} parts are optional. 
%
%The \cs{localenumfmt} command splits up \meta{decimal} into the
%\meta{int part}, \meta{frac part} and \meta{mantissa} required by
%\cs{texosqueryfmtnumber}. It then does
%\begin{code}
%\meta{fmt cs}\{\cs{texosqueryfmtnumber}\marg{pattern}\marg{int part}\marg{frac part}\marg{mantissa}
%\end{code}
%where \meta{fmt cs} is one of:
%\begin{definition}[\DescribeMacro\localenumfmtneg]
%\cs{localenumfmtneg}\marg{text}
%\end{definition}
%if the \meta{int part} is negative, 
%\begin{definition}[\DescribeMacro\localenumfmtzero]
%\cs{localenumfmtzero}\marg{text}
%\end{definition}
%if the \meta{int part}, \meta{frac part} and \meta{mantissa} are all
%zero, otherwise  
%\begin{definition}[\DescribeMacro\localenumfmtpos]
%\cs{localenumfmtpos}\marg{text}
%\end{definition}
%These three commands all simply default to just doing \meta{text}.
%
%\begin{definition}[\DescribeMacro\localeint]
%\cs{localeint}\marg{value}
%\end{definition}
%This is equivalent to
%\begin{definition}
%\cs{localenumfmt}\{\cs{CurrentLocaleIntegerPattern}\}\marg{value}
%\end{definition}
%
%\begin{definition}[\DescribeMacro\localedec]
%\cs{localedec}\marg{decimal}
%\end{definition}
%As above but uses \cs{CurrentLocaleDecimalPattern}.
%
%\begin{definition}[\DescribeMacro\localecur]
%\cs{localecur}\marg{decimal}
%\end{definition}
%As above but uses \cs{CurrentLocaleCurrencyPattern}.
%
%\begin{definition}[\DescribeMacro\localeper]
%\cs{localeper}\marg{decimal}
%\end{definition}
%As above but uses \cs{CurrentLocalePercentPattern}.
%
%The above example can be simplified to:
%\begin{verbatim}
%\def\LocaleMain{en-GB}
%\input locale
%
%\localeint{12.34567E4}
%
%\localecur{12345.67}
%
%\localedec{123.4567E2}
%
%\localeper{123.4567E2}
%\bye
%\end{verbatim}
%This produces the same result as before.
%
%As described in \sectionref{sec:currnumsym}, the currency unit can
%be switched to a symbol by redefining \cs{localecurrchoice}. For
%example:
%\begin{verbatim}
%\font\nimbus="NimbusRoman-Regular" at 10pt
%\nimbus
%
%\def\LocaleMain{en-GB}
%\def\LocaleOther{pt-BR,de-CH-1996}
%\input locale
%
%\def\localecurrchoice#1#2#3#4{#3}
%
%en-GB: \localecur{12345.67}
%
%\selectlocale{pt-BR}
%pt-BR: \localecur{12345.67}
%
%\selectlocale{de-CH-1996}
%de-CH-1996: \localecur{12345.67}
%
%\bye
%\end{verbatim}
%This produces:
%\begin{quote}
%en-GB: \pounds 12,345.67
%
%pt-BR: R\$ 12.345,67
%
%de-CH-1996: SFr. 12'345.67
%\end{quote}
%
%You can define your own pattern using the commands provided by
%\sty{texosquery}. For example:
%\begin{verbatim}
%\texosquerydefpattern{\sinumpattern}{%
%  \sinumfmt
%  {\decfmt{\#\,\#\#\#\,\#\#\#\,\#\#\0}{\0\#\#\#\#\#\#\#\#\#}}%
%  {\-\#\#\#\#\#\#\#\#\0\0}%
%}
%\end{verbatim}
%This can then be used with \cs{localenumfmt}. The \sty{texosquery}
%documentation includes other examples, including a currency pattern
%that shifts the sign before the currency symbol for negative
%amounts:
%\begin{verbatim}
%\texosquerydefpattern{\curpattern}{%
%\pmnumfmt
%{\pcur{\decfmt{\#\,\#\#\#\,\#\#\#\,\#\#\0}{\0\0\#\#\#\#\#\#\#\#}}{}}%
%{\pcur{\decfmt{\#\,\#\#\#\,\#\#\#\,\#\#\0}{\0\0\#\#\#\#\#\#\#\#}}{\-}}}
%\end{verbatim}
%(No sign is used for positive amounts.)
%
%Here's a complete document:
%\begin{verbatim}
%\font\nimbus="NimbusRoman-Regular" at 10pt
%\nimbus
%
%\def\LocaleMain{en-GB}
%\def\LocaleOther{pt-BR,de-CH-1996}
%\input locale
%
%\def\localecurrchoice#1#2#3#4{#3}
%
%\texosquerydefpattern{\sinumpattern}{%
%  \sinumfmt
%  {\decfmt{\#\,\#\#\#\,\#\#\#\,\#\#\0}{\0\#\#\#\#\#\#\#\#\#}}%
%  {\-\#\#\#\#\#\#\#\#\0\0}%
%}
%
%\texosquerydefpattern{\curpattern}{%
%\pmnumfmt
%{\pcur{\decfmt{\#\,\#\#\#\,\#\#\#\,\#\#\0}{\0\0\#\#\#\#\#\#\#\#}}{}}%
%{\pcur{\decfmt{\#\,\#\#\#\,\#\#\#\,\#\#\0}{\0\0\#\#\#\#\#\#\#\#}}{\-}}}
%
%en-GB: 
%\localenumfmt{\sinumpattern}{1.2345E-3}.
%\localenumfmt{\curpattern}{12345.6}.
%\localenumfmt{\curpattern}{-12345.6}.
%\localenumfmt{\curpattern}{0}.
%
%\selectlocale{pt-BR}
%pt-BR: 
%\localenumfmt{\sinumpattern}{1.2345E-3}.
%\localenumfmt{\curpattern}{12345.6}.
%\localenumfmt{\curpattern}{-12345.6}.
%\localenumfmt{\curpattern}{0}.
%
%\selectlocale{de-CH-1996}
%de-CH-1996:
%\localenumfmt{\sinumpattern}{1.2345E-3}.
%\localenumfmt{\curpattern}{12345.6}.
%\localenumfmt{\curpattern}{-12345.6}.
%\localenumfmt{\curpattern}{0}.
%
%\bye
%\end{verbatim}
%This produces:
%\begin{quote}
%en-GB: 1.2345E$-$03. \pounds 12,345.60. $-$\pounds12,345.60.
%\pounds0.00.
%
%pt-BR: 1,2345E$-$03. R\$12.345,60. $-$R\$12.345,60.
%R\$0,00.
%
%de-CH-1996: 1.2345E$-$03. SFr.12'345.60. $-$SFr.12'345.60.
%SFr.0.00.
%\end{quote}
%
%Here's a \LaTeX\ alternative that redefines the formatting commands
%used by \cs{localenumfmt}:
%\begin{verbatim}
%\documentclass{article}
%
%\usepackage{color}
%\usepackage[main={en-GB},other={pt-BR,de-CH-1996}]{tex-locale}
%\setmainfont{NimbusRoman-Regular}
%
%\texosquerydefpattern{\sinumpattern}{%
%  \sinumfmt
%  {\decfmt{\#\,\#\#\#\,\#\#\#\,\#\#\0}{\0\#\#\#\#\#\#\#\#\#}}%
%  {\-\#\#\#\#\#\#\#\#\0\0}%
%}
%
%\texosquerydefpattern{\curpattern}{%
%\pmnumfmt
%{\pcur{\decfmt{\#\,\#\#\#\,\#\#\#\,\#\#\0}{\0\0\#\#\#\#\#\#\#\#}}{}}%
%{\pcur{\decfmt{\#\,\#\#\#\,\#\#\#\,\#\#\0}{\0\0\#\#\#\#\#\#\#\#}}{\-}}}
%
%\renewcommand*{\localenumfmtpos}[1]{\textcolor{green}{#1}}
%\renewcommand*{\localenumfmtneg}[1]{\textcolor{red}{#1}}
%\renewcommand*{\localenumfmtzero}[1]{\textbf{#1}}
%
%\begin{document}
%
%en-GB: 
%\localenumfmt{\sinumpattern}{1.2345E-3}.
%\localenumfmt{\curpattern}{12345.6}.
%\localenumfmt{\curpattern}{-12345.6}.
%\localenumfmt{\curpattern}{0}.
%
%\selectlocale{pt-BR}
%pt-BR: 
%\localenumfmt{\sinumpattern}{1.2345E-3}.
%\localenumfmt{\curpattern}{12345.6}.
%\localenumfmt{\curpattern}{-12345.6}.
%\localenumfmt{\curpattern}{0}.
%
%\selectlocale{de-CH-1996}
%de-CH-1996:
%\localenumfmt{\sinumpattern}{1.2345E-3}.
%\localenumfmt{\curpattern}{12345.6}.
%\localenumfmt{\curpattern}{-12345.6}.
%\localenumfmt{\curpattern}{0}.
%\end{document}
%\end{verbatim}
%This produces:
%\begin{quote}
%en-GB: \textcolor{green}{1.2345E$-$03}. 
%\textcolor{green}{\pounds 12,345.60}. 
%\textcolor{red}{$-$\pounds12,345.60}.
%\textbf{\pounds0.00}.
%
%pt-BR: \textcolor{green}{1,2345E$-$03}. 
%\textcolor{green}{R\$12.345,60}. 
%\textcolor{red}{$-$R\$12.345,60}.
%\textbf{R\$0,00}.
%
%de-CH-1996: \textcolor{green}{1.2345E$-$03}. 
%\textcolor{green}{SFr.12'345.60}. 
%\textcolor{red}{$-$SFr.12'345.60}.
%\textbf{SFr.0.00}.
%\end{quote}
%
%\subsection{Date-Time Patterns}
%\label{sec:datetimepat}
%
%In addition to providing locale-sensitive strings for the current
%date-time, the \app{texosquery} application can also provide the
%locale's pattern that's used to describe how the dates and times
%should be formatted. Additionally, it can provide time-zone mappings
%from identifying labels to locale-sensitive names. Both these things
%require extra overhead and so these functions aren't on by default.
%
%You can enable them by defining
%\begin{definition}[\DescribeMacro\LocaleIfDateTimePatternsSupported]
%\cs{LocaleIfDateTimePatternsSupported}\marg{true}\marg{false}
%\end{definition}
%\emph{before} you input \file{tex-locale.tex}. \LaTeX\ users can use the
%more convenient \pkgopt{timedata} package option:
%\begin{verbatim}
%\usepackage[timedata]{tex-locale}
%\end{verbatim}
%Plain \TeX\ users need to define
%\cs{LocaleIfDateTimePatternsSupported} so that it does the \meta{true}
%argument and ignores the \meta{false} argument:
%\begin{verbatim}
%\def\LocaleIfDateTimePatternsSupported#1#2{#1}
%\input locale
%\end{verbatim}
%
%If enabled, \sty{tex-locale} will add the \texttt{-M} and \texttt{-Z}
%switches to the \app{texosquery} command invocation. This will
%additionally redefine \sty{texosquery}'s pattern format commands 
%(such as \cs{texosqueryfmtpatMMM}) to use the appropriate commands
%for the current locale. For example, \cs{texosqueryfmtpatMMM}\marg{n} is
%redefined to use \cs{CurrentLocaleShortMonthName}\marg{n}. (See the
%\sty{texosquery} documentation for further details on these
%commands.)
%
%If date-time patterns are supported then the current date-time data will be 
%stored in
%\begin{definition}[\DescribeMacro\LocaleDateTimeInfo]
%\cs{LocaleDateTimeInfo}
%\end{definition}
%This will be empty if date-time patterns aren't supported.
%
%The value of this command contains all the information for the current date-time provided
%in a format that can easily be parsed by date or time patterns.
%(This is done through \sty{texosquery}'s \ics{texosqueryfmtdatetime}
%command, which is internally used by
%\ics{LocaleApplyDateTimePattern}, described below.)
%For example, \cs{LocaleDateTimeInfo} might be set to
%\begin{verbatim}
%{1}{2017}{2017}{3}{12}{4}{85}{26}{4}{7}{1}{19}{19}{7}{7}{16}{39}{136}
%{{1}{0}{Europe/London}{1}}
%\end{verbatim}
%(Line break inserted above for clarity.) See the \sty{texosquery}
%documentation for details of this syntax.
%
%Alternatively you can use \sty{texosquery}'s pattern generator to
%create your own date-time pattern and explicitly use
%\cs{texosqueryfmtdatetime}. In which case \cs{LocaleDateTimeInfo}
%provides a convenient way of applying the pattern to the current
%date-time.  For example:
%\begin{verbatim}
%\def\LocaleIfDateTimePatternsSupported#1#2{#1}
%\def\LocaleMain{en-GB}
%\input locale
%
%\texosquerydefpattern{\pattern}{\%3E \%1d \%3M \%4y \%2H:\%2m:\%2s \%2z}
%
%\ifx\LocaleDateTimeInfo\empty
%\else
%  \expandafter\texosqueryfmtdatetime\expandafter\pattern\LocaleDateTimeInfo
%\fi
%\bye
%\end{verbatim}
%This produces:
%\begin{quote}
%Sun 26 Mar 2017 21:38:06 BST
%\end{quote}
%(assuming that was the date and time of the document build). The
%textual elements (\qt{Sun}, \qt{Mar} and \qt{BST}) are obtained from
%the current locale.
%
%If the date-time patterns are supported, you can apply a pattern to
%a specific date or time using 
%\begin{definition}[\DescribeMacro\LocaleApplyDateTimePattern]
%\cs{LocaleApplyDateTimePattern}\marg{dialect}\marg{attribute}\marg{data}
%\end{definition}
%This uses \sty{texosquery}'s \cs{texosqueryfmtdatetime} command to
%format a date or time according to a date-time pattern (identified
%by \meta{attribute}) for the locale identified by \meta{dialect}.
%The \sty{texosquery} manual provides further details on this
%command, but essentially the first argument of \cs{texosqueryfmtdatetime}
%should be a control sequence containing the special pattern markup
%and the remaining arguments are the date-time data, which are
%supplied in the \meta{data} argument of \cs{LocaleApplyDateTimePattern}.
%
%The \ics{LocaleDateTimeInfo} command will either be empty or set to
%the correct data for the current date and time. This can be used
%in the \meta{data} argument. Since \cs{LocaleDateTimeInfo} might be
%empty, \cs{LocaleApplyDateTimePattern} will first test for this and
%not try applying the pattern in that case. (Remember that you can
%also define your own custom date-time patterns, as mentioned in
%\sectionref{sec:general}.)
%
%Note that the \meta{dialect} is used to fetch the pattern data, but
%the month and day names will be obtained using the
%\cs{CurrentLocale\ldots} commands described in
%\sectionref{sec:currdatetime}, which may not match \meta{dialect}.
%In general it's easiest to use the shortcut 
%\ics{CurrentLocaleApplyDateTimePattern} (\sectionref{sec:currpat}) instead to ensure the
%pattern matches the current locale.
%
%The attributes associated with date-time patterns are listed below
%(see \sectionref{sec:attrdatetime}).
%\begin{itemize}
%\item \dattr{fulldatefmt}: the full date format pattern (as used in 
%\cs{LocaleFullDate}).
%\item \dattr{longdatefmt}: the long date format pattern (as used in 
%\cs{LocaleLongDate}).
%\item \dattr{meddatefmt}: the medium date format pattern (as used in 
%\cs{LocaleMediumDate}).
%\item \dattr{shortdatefmt}: the short date format pattern (as used in 
%\cs{LocaleShortDate}).
%\item \dattr{fulltimefmt}: the full time format pattern (as used in 
%\cs{LocaleFullTime}).
%\item \dattr{longtimefmt}: the long time format pattern (as used in 
%\cs{LocaleLongTime}).
%\item \dattr{medtimefmt}: the medium time format pattern (as used in 
%\cs{LocaleMediumTime}).
%\item \dattr{shorttimefmt}: the short time format pattern (as used in 
%\cs{LocaleShortTime}).
%\item \dattr{fulldatetimefmt}: the full date and time format pattern.
%\item \dattr{longdatetimefmt}: the long date and time format pattern.
%\item \dattr{meddatetimefmt}: the medium date and time format pattern.
%\item \dattr{shortdatetimefmt}: the short date and time format pattern.
%\end{itemize}
%
%For example:
%\begin{verbatim}
%\def\LocaleIfDateTimePatternsSupported#1#2{#1}
%\def\LocaleMain{en-GB}
%
%\input locale
%
%Now:
%\CurrentLocaleApplyDateTimePattern{fulldatetimefmt}{\LocaleDateTimeInfo}.
%\bye
%\end{verbatim}
%This displays:
%\begin{quote}
%Now: Sunday, 26 March 2017 21:12:10 British Summer Time.
%\end{quote}
%However, it's simpler to just use:
%\begin{verbatim}
%\def\LocaleIfDateTimePatternsSupported#1#2{#1}
%\def\LocaleMain{en-GB}
%
%\input locale
%
%\def\localedatechoice#1#2#3#4{#1}
%\def\localetimechoice#1#2#3#4{#1}
%
%Now: \CurrentLocaleDateTime.
%\bye
%\end{verbatim}
%or with \LaTeX:
%\begin{verbatim}
%\documentclass{article}
%
%\usepackage[timedata,date=full,time=full,main={en-GB}]{tex-locale}
%
%\begin{document}
%Now: \CurrentLocaleDateTime.
%\end{document}
%\end{verbatim}
%
%The date-time patterns are therefore more useful when applied to
%something other that the current date-time. However, it's more
%complicated to work out the data.
%
%For example:
%\begin{verbatim}
%\def\LocaleIfDateTimePatternsSupported#1#2{#1}
%\def\LocaleMain{en-GB}
%
%\input locale
%
%\CurrentLocaleApplyDateTimePattern{fulldatetimefmt}%
%{{1}{2017}{2017}{3}{12}{4}{85}{26}{4}{7}{1}{21}{21}{9}{9}{23}{36}{140}%
%{{1}{0}{Europe/London}{1}}}
%\bye
%\end{verbatim}
%This produces:
%\begin{quote}
%Sunday, 26 March 2017 21:23:36 British Summer Time.
%\end{quote}
%
%
%\section{Locale Information}
%\label{sec:localedata}
%
%Information about each locale is fetched for each tracked dialect (the
%main locale, identified by \cs{LocaleMain}, and the other locales,
%identified by \cs{LocaleOther}). This information can be accessed
%using the commands described below, where \meta{dialect} is a
%\sty{tracklang} dialect label that identifies the locale. There are convenient commands
%(\cs{CurrentLocale\ldots}) that select the appropriate command with
%the label provided by the currently selected dialect. (See
%\sectionref{sec:currentlocale}.)
%
%Most of the commands in this package are intended to be used in an
%expandable context, and so will expand to nothing if the dialect isn't
%recognised.
%
%\begin{definition}[\DescribeMacro\LocaleMainDialect]
%\cs{LocaleMainDialect}
%\end{definition}
%This expands to the main locale's dialect label. For example with:
%\begin{verbatim}
%\def\LocaleMain{en-GB}
%\input locale
%\end{verbatim}
%the main dialect is \texttt{british}.
%
%\begin{definition}[\DescribeMacro\LocaleMainRegion]
%\cs{LocaleMainRegion}
%\end{definition}
%This expands to the main locale's region. For example, if the main
%locale is \texttt{en-GB}, then the main region is \texttt{GB}.
%
%\begin{definition}[\DescribeMacro\LocaleLanguageTag]
%\cs{LocaleLanguageTag}\marg{dialect}
%\end{definition}
%Expands to the language tag for the given dialect. For example, if
%the document locales are set using:
%\begin{verbatim}
%\def\LocaleMain{en-GB}
%\def\LocaleOther{pt-BR,fr-BE,de-CH-1996}
%\input locale
%\end{verbatim}
%then
%\begin{verbatim}
%\LocaleLanguageTag{british}
%\end{verbatim}
%will expand to \texttt{en-GB} and
%\begin{verbatim}
%\LocaleLanguageTag{nswissgerman}
%\end{verbatim}
%will expand to \texttt{de-CH-1996}.
%
%\begin{definition}[\DescribeMacro\LocaleLanguageName]
%\cs{LocaleLanguageName}\marg{dialect}
%\end{definition}
%The name of the language associated with \meta{dialect} provided in
%the Java virtual machine's default language.
%
%\begin{definition}[\DescribeMacro\LocaleLanguageNativeName]
%\cs{LocaleLanguageNativeName}\marg{dialect}
%\end{definition}
%The name of the language associated with \meta{dialect} provided in
%that language.
%
%For example:
%\begin{verbatim}
%\font\nimbus="NimbusRoman-Regular" at 10pt
%\nimbus
%
%\def\LocaleMain{en-GB}
%\def\LocaleOther{pt-BR,fr-BE}
%
%\input locale
%
%en-GB: \LocaleLanguageName{british}.
%pt-BR: \LocaleLanguageName{brazilian}.
%fr-BE: \LocaleLanguageName{belgique}.
%\bye
%\end{verbatim}
%For me this produces:
%\begin{quote}
%en-GB: English. pt-BR: Portuguese. fr-BE: French.
%\end{quote}
%because my operating system's default language is English. If I
%edit the \app{texosquery-jre8} bash script so that it includes
%\begin{verbatim}
%-Duser.language=fr
%\end{verbatim}
%in the \app{java} arguments, then the same document will produce:
%\begin{quote}
%en-GB: anglais. pt-BR: portugais. fr-BE: fran\c{c}ais.
%\end{quote}
%So this isn't necessarily in the same language as the main locale (or
%any of the other locales tracked in the document). It's in the Java
%virtual machine's default language. Compare this document with:
%\begin{verbatim}
%\font\nimbus="NimbusRoman-Regular" at 10pt
%\nimbus
%
%\def\LocaleMain{en-GB}
%\def\LocaleOther{pt-BR,fr-BE}
%
%\input locale
%
%en-GB: \LocaleLanguageNativeName{british}.
%pt-BR: \LocaleLanguageNativeName{brazilian}.
%fr-BE: \LocaleLanguageNativeName{belgique}.
%\bye
%\end{verbatim}
%This produces:
%\begin{quote}
%en-GB: English. pt-BR: portugu\^es. fr-BE: fran\c{c}ais.
%\end{quote}
%So in this case each name is displayed according to its own language.
%
%There are similar commands for the region and variant.
%\begin{definition}[\DescribeMacro\LocaleRegionName]
%\cs{LocaleRegionName}\marg{dialect}
%\end{definition}
%The name of the region associated with \meta{dialect} provided in
%the Java virtual machine's default language. This will be an empty
%string if the region wasn't supplied.
%
%\begin{definition}[\DescribeMacro\LocaleRegionNativeName]
%\cs{LocaleRegionNativeName}\marg{dialect}
%\end{definition}
%The name of the region associated with \meta{dialect} provided in
%that language. This will be an empty string if the region wasn't supplied.
%
%\begin{definition}[\DescribeMacro\LocaleVariantName]
%\cs{LocaleVariantName}\marg{dialect}
%\end{definition}
%The name of the variant associated with \meta{dialect} provided in
%the Java virtual machine's default language. This will be an empty
%string if a variant wasn't supplied.
%
%\begin{definition}[\DescribeMacro\LocaleVariantNativeName]
%\cs{LocaleVariantNativeName}\marg{dialect}
%\end{definition}
%The name of the variant associated with \meta{dialect} provided in
%that language. This will be an empty string if a variant wasn't supplied.
%This may well be the same as \cs{LocaleVariantName} as the variants
%are often identifiers (such as \texttt{1996} in the case of
%\texttt{de-CH-1996}) that remain the same in different languages.
%
%You can find out if the language name, region name or variant have
%been set using the commands listed below. In each case,
%\meta{dialect} is again the dialect label, \meta{true} is the code
%to do if the condition is true and \meta{false} is the code to do if
%the condition is false.
%
%\begin{definition}[\DescribeMacro\LocaleIfHasLanguageName]
%\cs{LocaleIfHasLanguageName}\marg{dialect}\marg{true}\marg{false}
%\end{definition}
%This tests if the dialect has an associated language name. This will
%typically be true.
%
%\begin{definition}[\DescribeMacro\LocaleIfHasRegionName]
%\cs{LocaleIfHasRegionName}\marg{dialect}\marg{true}\marg{false}
%\end{definition}
%This tests if the dialect has an associated region name. Some
%language tags don't have regions supplied. For example:
%\begin{verbatim}
%\def\LocaleMain{en}
%\input locale
%\end{verbatim}
%In this case, there's no region associated with the main locale.
%
%\begin{definition}[\DescribeMacro\LocaleIfHasVariantName]
%\cs{LocaleIfHasVariantName}\marg{dialect}\marg{true}\marg{false}
%\end{definition}
%This tests if the dialect has an associated variant name. For
%example, there's no variant in \texttt{de-DE} but there is a variant
%in \texttt{de-DE-1996}.
%
%\section{Dates and Times}
%\label{sec:datetimes}
%
%Date and time information is fetched for each tracked dialect. 
%The \file{tex-locale.tex} code doesn't
%modify \ics{today} (although the \file{tex-locale.sty} \LaTeX\ package
%can load \sty{datetime2}, which does). Instead, the dates or times
%for a specific locale can be obtained using the commands listed
%below. The \meta{dialect} argument indicates the \sty{tracklang}
%dialect label. See \sectionref{sec:currentlocale} for shortcut
%commands that select the appropriate command with the label of the
%currently selected locale.
%
%Dates and times are wrapped in a formatting command:
%\begin{definition}[\DescribeMacro\localedatetimefmt]
%\cs{localedatetimefmt}\marg{date-time text}
%\end{definition}
%By default this simply does its argument. For example:
%\begin{verbatim}
%\def\LocaleMain{en-GB}
%
%\input locale
%
%\def\localedatetimefmt#1{{\it #1}}
%
%Date: \LocaleFullDate{british}.
%
%\bye
%\end{verbatim}
%This displays the date in italic. The \LaTeX\ equivalent is:
%\begin{verbatim}
%\documentclass{article}
%\usepackage[main=en-GB]{tex-locale}
%
%\renewcommand*{\localedatetimefmt}[1]{\textit{#1}}
%
%\begin{document}
%Date: \LocaleFullDate{british}.
%\end{document}
%\end{verbatim}
%
%\subsection{Dates}
%\label{sec:localedates}
%
%\begin{definition}[\DescribeMacro\LocaleFullDate]
%\cs{LocaleFullDate}\marg{dialect}
%\end{definition}
%This displays the full date for the locale identified by \meta{dialect}
%as provided by \app{texosquery}.
%
%\begin{definition}[\DescribeMacro\LocaleLongDate]
%\cs{LocaleLongDate}\marg{dialect}
%\end{definition}
%This displays the long date for the locale identified by \meta{dialect}
%as provided by \app{texosquery}.
%
%\begin{definition}[\DescribeMacro\LocaleMediumDate]
%\cs{LocaleMediumDate}\marg{dialect}
%\end{definition}
%This displays the medium date for the locale identified by \meta{dialect}
%as provided by \app{texosquery}.
%
%\begin{definition}[\DescribeMacro\LocaleShortDate]
%\cs{LocaleShortDate}\marg{dialect}
%\end{definition}
%This displays the short date for the locale identified by \meta{dialect}
%as provided by \app{texosquery}.
%
%For example:
%\begin{verbatim}
%\def\LocaleMain{en-GB}
%
%\input locale
%
%Full: \LocaleFullDate{british}.
%Long: \LocaleLongDate{british}.
%Medium: \LocaleMediumDate{british}.
%Short: \LocaleShortDate{british}.
%\bye
%\end{verbatim}
%In the above, the main locale is explicitly set by defining \cs{LocaleMain}
%before loading \file{tex-locale.tex}. The \sty{tracklang} package
%identifies the \texttt{en-GB} locale with the label
%\texttt{british}.
%
%Different locales have different concepts of full, long, medium and
%short dates. In most cases the short form is numeric and the full
%form is textual. The medium form may be numeric or it may use an
%abbreviated month name. The locale provided used by Java may also
%influence the result. For example, if you are using
%\app{texosquery-jre8} with \texttt{java.locale.providers} set to
%\texttt{CLDR,JRE} then the above document will display (assuming
%today is 2017-03-26):
%\begin{quote}
%Full: Sunday, 26 March 2017. Long: 26 March 2017. Medium: 26 Mar
%2017. Short: 26/03/2017.
%\end{quote}
%However, if I instead use the Java 7 version (\file{texosquery.jar})
%which doesn't support the CLDR, then the above document will
%display:
%\begin{quote}
%Full: Sunday, 26 March 2017. Long: 26 March 2017. Medium:
%26-Mar-2017. Short: 26/03/17.
%\end{quote}
%The CLDR provides more extensive locale support than the JRE. In the
%case of \texttt{en-GB}, the difference is minor (as shown above),
%but in some cases a language may be supported in the CLDR but not in
%the JRE. For example, if in the above document I change the main
%locale to \texttt{cy-GB} (which maps to the \sty{tracklang dialect
%label \texttt{GBwelsh}}):
%\begin{verbatim}
%\def\LocaleMain{cy-GB}
%
%\input locale
%
%Full: \LocaleFullDate{GBwelsh}.
%Long: \LocaleLongDate{GBwelsh}.
%Medium: \LocaleMediumDate{GBwelsh}.
%Short: \LocaleShortDate{GBwelsh}.
%\bye
%\end{verbatim}
%then the Java 7 version produces US English dates:
%\begin{quote}
%Full: Sunday, March 26, 2017. Long: March 26, 2017. Medium: Mar 26,
%2017. Short: 3/26/17.
%\end{quote}
%This is because Welsh isn't supported by the JRE, but it is
%supported by the CLDR, so the Java 8 version (\app{texosquery-jre8})
%does work (provided \texttt{java.locale.providers} is set to
%\texttt{CLDR,JRE}):
%\begin{quote}
%Full: Dydd Sul, 26 Mawrth 2017. Long: 26 Mawrth 2017. Medium: 26
%Mawrth 2017. Short: 26/03/2017.
%\end{quote}
%
%\subsubsection{Week Days}
%\label{sec:weekdays}
%
%In order to be compatible with \sty{pgfcalendar} (and hence
%\sty{datetime2}), the \sty{tex-locale} package uses
%a zero-based indexing where 0 represents Monday, 1 represents
%Tuesday, etc. If you can't remember the index for a particular day, 
%you can use the following commands:
%\begin{definition}[\DescribeMacro\dtmMondayIndex]
%\cs{dtmMondayIndex}
%\end{definition}
%This expands to 0, the index for Monday.
%
%\begin{definition}[\DescribeMacro\dtmTuesdayIndex]
%\cs{dtmTuesdayIndex}
%\end{definition}
%This expands to 1, the index for Tuesday.
%
%\begin{definition}[\DescribeMacro\dtmWednesdayIndex]
%\cs{dtmWednesdayIndex}
%\end{definition}
%This expands to 2, the index for Wednesday.
%
%\begin{definition}[\DescribeMacro\dtmThursdayIndex]
%\cs{dtmThursdayIndex}
%\end{definition}
%This expands to 3, the index for Thursday.
%
%\begin{definition}[\DescribeMacro\dtmFridayIndex]
%\cs{dtmFridayIndex}
%\end{definition}
%This expands to 4, the index for Friday.
%
%\begin{definition}[\DescribeMacro\dtmSaturdayIndex]
%\cs{dtmSaturdayIndex}
%\end{definition}
%This expands to 5, the index for Saturday.
%
%\begin{definition}[\DescribeMacro\dtmSundayIndex]
%\cs{dtmSundayIndex}
%\end{definition}
%This expands to 6, the index for Sunday.
%
%You can get the day of week name identified by the Monday=0 based
%index for a particular dialect using
%\begin{definition}[\DescribeMacro\LocaleDayName]
%\cs{LocaleDayName}\marg{dialect}\marg{index}
%\end{definition}
%For example:
%\begin{verbatim}
%\def\LocaleMain{en-GB}
%\def\LocaleOther{pt-BR}
%
%\input locale
%
%en-GB: \LocaleDayName{british}{0}.
%pt-BR: \LocaleDayName{brazilian}{0}.
%\bye
%\end{verbatim}
%This produces:
%\begin{quote}
%en-GB: Monday. pt-BR: Segunda-feira.
%\end{quote}
%
%The abbreviated day of week name can be obtained with:
%\begin{definition}[\DescribeMacro\LocaleShortDayName]
%\cs{LocaleShortDayName}\marg{dialect}\marg{index}
%\end{definition}
%This has the same syntax as the previous command. As with the date
%commands described in the \hyperref[sec:localedates]{previous
%section}, the support for the given locale depends on the locale
%provider used by \app{texosquery}.
%
%Some languages have a different form for day of week names when used in a
%standalone context, such as in a column header. These can be
%obtained using the commands below, \emph{but only with
%\app{texosquery-jre8}}. If you are using Java 7 or earlier, these
%commands will produce identical results to the analogous command
%above.
%
%\begin{definition}[\DescribeMacro\LocaleStandaloneDayName]
%\cs{LocaleStandaloneDayName}\marg{dialect}\marg{index}
%\end{definition}
%The same syntax as above, this produces the standalone day of week
%name. The abbreviated name is produced with:
%\begin{definition}[\DescribeMacro\LocaleStandaloneShortDayName]
%\cs{LocaleStandaloneShortDayName}\marg{dialect}\marg{index}
%\end{definition}
%
%The first day of the week varies according to region. In some
%locales, Monday is considered the first day of the week (for
%example, en-GB), but in other locales, Sunday is the first day
%(for example, pt-BR). You can find out which day of the week is
%considered the first day using:
%\begin{definition}[\DescribeMacro\LocaleFirstDayIndex]
%\cs{LocaleFirstDayIndex}\marg{dialect}
%\end{definition}
%This expands to an integer index identifying the day.
%
%For example, Monday (0) is the first day of the week in \texttt{en-GB},
%but Sunday (6) is the first day of the week in \texttt{pt-BR}.
%\begin{verbatim}
%\def\LocaleMain{en-GB}
%\def\LocaleOther{pt-BR}
%
%\input locale
%
%First day (en-GB): \LocaleFirstDayIndex{british}.
%First day (pt-BR): \LocaleFirstDayIndex{brazilian}.
%
%\bye
%\end{verbatim}
%So the above produces:
%\begin{quote}
%First day (en-GB): 0. First day (pt-BR): 6.
%\end{quote}
%
%Since it's possible that you may want to use a different indexing
%system, there are commands provided to convert between them:
%\begin{definition}[\DescribeMacro\LocaleDayIndexFromZeroMonToOneSun]
%\cs{LocaleDayIndexFromZeroMonToOneSun}\marg{index}
%\end{definition}
%This converts a day index from the Monday=0 based system to the
%Sunday=1 based system. For example, in Monday=0 indexing then the
%index 3 represents Thursday. In the Sunday=1 based system, the index
%5 represents Thursday, so
%\begin{verbatim}
%\LocaleDayIndexFromZeroMonToOneSun{3}
%\end{verbatim}
%expands to 5.
%
%\begin{definition}[\DescribeMacro\LocaleDayIndexFromZeroMonToOneMon]
%\cs{LocaleDayIndexFromZeroMonToOneMon}\marg{index}
%\end{definition}
%This converts a day index from the Monday=0 (Sunday=6) based system to 
%the ISO-8601 Monday=1 (Sunday=7) based system. For example
%\begin{verbatim}
%\LocaleDayIndexFromZeroMonToOneMon{3}
%\end{verbatim}
%expands to 4.
%
%\begin{definition}[\DescribeMacro\LocaleDayIndexFromOneSunToZeroMon]
%\cs{LocaleDayIndexFromOneSunToZeroMon}\marg{index}
%\end{definition}
%This converts a day index from the Sunday=1 based system to the
%Monday=0 based system. For example
%\begin{verbatim}
%\LocaleDayIndexFromOneSunToZeroMon{5}
%\end{verbatim}
%expands to 3. (That is, it performs the reverse of 
%\cs{LocaleDayIndexFromZeroMonToOneSun}.)
%
%\begin{definition}[\DescribeMacro\LocaleDayIndexFromOneMonToZeroMon]
%\cs{LocaleDayIndexFromOneMonToZeroMon}\marg{index}
%\end{definition}
%This converts a day index from the ISO-8601 Monday=1 (Sunday=7) based system
%to the Monday=0 (Sunday=6) based system. For example
%\begin{verbatim}
%\LocaleDayIndexFromOneMonToZeroMon{4}
%\end{verbatim}
%expands to 3. (That is, it performs the reverse of 
%\cs{LocaleDayIndexFromZeroMonToOneMon}.)
%
%\begin{definition}[\DescribeMacro\LocaleDayIndexFromRegion]
%\cs{LocaleDayIndexFromRegion}\marg{dialect}\marg{index}
%\end{definition}
%This converts the region's day of the week (starting from 1 for the
%region's first day) to Monday=0 based indexing. The region is identified by
%\meta{dialect}. This first uses
%\cs{LocaleFirstDayIndex}\marg{dialect} to find the
%1-based first day of week index for the region and then converts it
%to the Monday=0 based system. An invalid \meta{index} results in
%\texttt{-1}.
%
%For example, the \texttt{en-GB} locale has Monday as the first day
%of the week. So if the \meta{index} argument is 1
%\begin{verbatim}
%\LocaleDayIndexFromRegion{british}{1}
%\end{verbatim}
%that indicates Monday. This is then converted to 0 (using
%\cs{LocaleDayIndexFromOneMonToZeroMon}). If the \meta{index}
%argument is 7
%\begin{verbatim}
%\LocaleDayIndexFromRegion{british}{7}
%\end{verbatim}
%that indicates Sunday, so this would be converted to 6.
%
%The \texttt{pt-BR} locale has Sunday as the first day of the week.
%So if the \meta{index} argument is 2
%\begin{verbatim}
%\LocaleDayIndexFromRegion{brazilian}{2}
%\end{verbatim}
%that indicates Monday. This is
%converted to 0 (using \cs{LocaleDayIndexFromOneSunToZeroMon}). If
%the \meta{index} argument is 1
%\begin{verbatim}
%\LocaleDayIndexFromRegion{brazilian}{1}
%\end{verbatim}
%that indicates Sunday, so this would be converted to 6.
%
%\begin{definition}[\DescribeMacro\LocaleDayIndexToRegion]
%\cs{LocaleDayIndexToRegion}\marg{dialect}\marg{index}
%\end{definition}
%This performs the reverse operation. In this case the \meta{index}
%argument is Monday=0 based and the result is the locale's day index
%(starting with 1 for the first day of the week).
%
%\subsubsection{Month Names}
%\label{sec:monthnames}
%
%The month name for a given locale can be obtained using
%\begin{definition}[\DescribeMacro\LocaleMonthName]
%\cs{LocaleMonthName}\marg{dialect}\marg{index}
%\end{definition}
%where \meta{dialect} is the \sty{tracklang} dialect label that
%identifies the locale and \meta{index} is an integer from 1
%(January) to 12 (December) indicating the month.
%
%For example:
%\begin{verbatim}
%\def\LocaleMain{en-GB}
%\def\LocaleOther{pt-BR}
%
%\input locale
%
%en-GB: \LocaleMonthName{british}{1}.
%pt-BR: \LocaleMonthName{brazilian}{1}.
%
%\bye
%\end{verbatim}
%produces:
%\begin{quote}
%en-GB: January. pt-BR: Janeiro.
%\end{quote}
%
%The abbreviated month name can be obtained using
%\begin{definition}[\DescribeMacro\LocaleShortMonthName]
%\cs{LocaleShortMonthName}\marg{dialect}\marg{index}
%\end{definition}
%which has the same syntax as the previous command.
%
%Some languages have a different form for month names when used in a
%standalone context, such as in a column header. These can be
%obtained using the commands below, \emph{but only with
%\app{texosquery-jre8}}. If you are using Java 7 or earlier, these
%commands will produce identical results to the analogous command
%above.
%
%\begin{definition}[\DescribeMacro\LocaleStandaloneMonthName]
%\cs{LocaleStandaloneMonthName}\marg{dialect}\marg{index}
%\end{definition}
%The same syntax as above, this produces the standalone month
%name. The abbreviated name is produced with:
%\begin{definition}[\DescribeMacro\LocaleStandaloneShortMonthName]
%\cs{LocaleStandaloneShortMonthName}\marg{dialect}\marg{index}
%\end{definition}
%
%\subsection{Times}
%\label{sec:localetimes}
%
%\begin{definition}[\DescribeMacro\LocaleFullTime]
%\cs{LocaleFullTime}\marg{dialect}
%\end{definition}
%This displays the full time for the locale identified by \meta{dialect}
%as provided by \app{texosquery}.
%
%\begin{definition}[\DescribeMacro\LocaleLongTime]
%\cs{LocaleLongTime}\marg{dialect}
%\end{definition}
%This displays the long time for the locale identified by \meta{dialect}
%as provided by \app{texosquery}.
%
%\begin{definition}[\DescribeMacro\LocaleMediumTime]
%\cs{LocaleMediumTime}\marg{dialect}
%\end{definition}
%This displays the medium time for the locale identified by \meta{dialect}
%as provided by \app{texosquery}.
%
%\begin{definition}[\DescribeMacro\LocaleShortTime]
%\cs{LocaleShortTime}\marg{dialect}
%\end{definition}
%This displays the short time for the locale identified by \meta{dialect}
%as provided by \app{texosquery}.
%
%As with \hyperref[sec:localedates]{dates}, the results are
%determined by the locale provider used by \app{texosquery}.
%For example, the Java 8 variant (\app{texosquery-jre8}) with 
%\texttt{java.locale.providers} set to \texttt{CLDR,JRE} may produce
%a different result to a variant of \app{texosquery} that only uses
%the JRE provider.
%
%For example:
%\begin{verbatim}
%\def\LocaleMain{en-GB}
%
%\input locale
%
%Full: \LocaleLongTime{british}.
%Long: \LocaleLongTime{british}.
%Medium: \LocaleMediumTime{british}.
%Short: \LocaleShortTime{british}.
%
%\bye
%\end{verbatim}
%This produces:
%\begin{quote}
%Full: 13:43:53 BST. Long: 13:43:53 BST. Medium: 13:43:53. Short:
%13:43.
%\end{quote}
%In this particular case there's no difference between using the CLDR
%and the JRE locale providers, but note that there's no difference in
%the full and long forms.
%
%There are also commands for the combined date and time:
%\begin{definition}[\DescribeMacro\LocaleFullDateTime]
%\cs{LocaleFullDateTime}\marg{dialect}
%\end{definition}
%This displays the full date and time for the locale identified by \meta{dialect}
%as provided by \app{texosquery}.
%
%\begin{definition}[\DescribeMacro\LocaleLongDateTime]
%\cs{LocaleLongDateTime}\marg{dialect}
%\end{definition}
%This displays the long date and time for the locale identified by \meta{dialect}
%as provided by \app{texosquery}.
%
%\begin{definition}[\DescribeMacro\LocaleMediumDateTime]
%\cs{LocaleMediumDateTime}\marg{dialect}
%\end{definition}
%This displays the medium date and time for the locale identified by \meta{dialect}
%as provided by \app{texosquery}.
%
%\begin{definition}[\DescribeMacro\LocaleShortDateTime]
%\cs{LocaleShortDateTime}\marg{dialect}
%\end{definition}
%This displays the short date and time for the locale identified by \meta{dialect}
%as provided by \app{texosquery}.
%
%
%\section{Numbers}
%\label{sec:numbers}
%
%Different locales use different symbols to denote the decimal mark
%or the number group separator when displaying numbers. There are
%other numeric-related symbols that may also vary according to
%region, such as the currency symbol, exponent, percent or per mill
%signs. Since most of these are outside of the Basic Latin set, the
%examples here use \XeTeX\ with a font that supports those symbols.
%If you don't have that font installed, you will need to adapt the
%examples accordingly.
%
%As with date and times, the commands describe here need the
%\sty{tracklang} dialect label that identifies the required locale.
%See \sectionref{sec:currentlocale} for shortcut commands that
%automatically select the current locale's dialect label.
%
%In general, most of these commands won't need to be used explicitly
%as it's easier to use a numeric pattern instead (see
%\sectionref{sec:numpatterns}).
%
%\subsection{Numeric Symbols}
%\label{sec:numsyms}
%
%The number group separator for a particular locale can be obtained
%using:
%\begin{definition}[\DescribeMacro\LocaleNumericGroupSep]
%\cs{LocaleNumericGroupSep}\marg{dialect}
%\end{definition}
%where \meta{dialect} is the \sty{tracklang} dialect label
%representing the locale.
%
%You can determine whether or not the locale uses a number group
%separator using:
%\begin{definition}[\DescribeMacro\LocaleIfNumericUsesGroup]
%\cs{LocaleIfNumericUsesGroup}\marg{dialect}\marg{true
%code}\marg{false code}
%\end{definition}
%This does \meta{true code} if the locale identified by \meta{dialect}
%uses a number group separator otherwise it does \meta{false code}.
%
%The decimal mark is obtained using:
%\begin{definition}[\DescribeMacro\LocaleNumericDecimalSep]
%\cs{LocaleNumericDecimalSep}\marg{dialect}
%\end{definition}
%
%The monetary separator is often the same as the decimal separator,
%but isn't always, so there's a separate command for it:
%\begin{definition}[\DescribeMacro\LocaleNumericMonetarySep]
%\cs{LocaleNumericMonetarySep}\marg{dialect}
%\end{definition}
%
%The exponent symbol is obtained using:
%\begin{definition}[\DescribeMacro\LocaleNumericExponent]
%\cs{LocaleNumericExponent}\marg{dialect}
%\end{definition}
%
%The percent symbol is obtained using:
%\begin{definition}[\DescribeMacro\LocaleNumericPercent]
%\cs{LocaleNumericPercent}\marg{dialect}
%\end{definition}
%This will typically use \verb|\%| (the percent symbol \%).
%
%The per mill symbol is obtained using:
%\begin{definition}[\DescribeMacro\LocaleNumericPermill]
%\cs{LocaleNumericPermill}\marg{dialect}
%\end{definition}
%This will typically be a character outside the Basic Latin set, so
%the document will need to support this symbol if required.
%
%\subsection{Currencies}
%\label{sec:currencies}
%
%The official currency code (such as \texttt{GBP} or \texttt{USD}) is
%obtained using:
%\begin{definition}[\DescribeMacro\LocaleCurrencyLabel]
%\cs{LocaleCurrencyLabel}\marg{dialect}
%\end{definition}
%This first checks if the \dattr{currency} attribute for the given
%dialect is \texttt{XXX} (which denotes an unknown currency,
%typically because there's no region associated with \meta{dialect}).
%If the currency for \meta{dialect} is known (that is, the attribute
%value isn't \texttt{XXX}), then that currency's official code is
%produced.  If the currency is unknown, then it will try to fallback
%on the main dialect's currency code if the main dialect has an
%associated region, otherwise it will fallback on the OS currency
%code.
%
%Remember that with the language tag, you're not restricted to using
%official languages for a given region. For example, if you're
%writing in English in Belgium, it's valid to use \texttt{en-BE} as a
%locale. For example:
%\begin{verbatim}
%\def\LocaleMain{en-GB}
%\def\LocaleOther{en-BE,pt-BR}
%
%\input locale
%
%en-GB: \LocaleCurrencyLabel{british}.
%en-BE: \LocaleCurrencyLabel{enBE}.
%pt-BR: \LocaleCurrencyLabel{brazilian}.
%
%\bye
%\end{verbatim}
%This produces:
%\begin{quote}
%en-GB: GBP. en-BE: EUR. pt-BR: BRL.
%\end{quote}
%
%There are a few currencies that have an unofficial currency code.
%These are typically currencies pegged to another currency with a
%fixed exchange rate of 1.0. The
%official ISO currency code for one
%region may be the code for the other currency. For example, the Isle
%of Man currency is the Manx pound, which is kept in parity with 
%pound sterling. The currency code for \texttt{en-IM} is returned as
%\texttt{GBP} by Java (since IMP has no official recognition in
%ISO~4217), but \app{texosquery} recognises that this
%region has an unofficial currency code \texttt{IMP}. This can be
%obtained using
%\begin{definition}[\DescribeMacro\LocaleCurrencyRegionalLabel]
%\cs{LocaleCurrencyRegionalLabel}\marg{dialect}
%\end{definition}
%This command is much the same as the previous command in that it
%will fallback on the main or OS currency code if not known.
%The dialect attribute \dattr{regionalcurrency} is queried for the
%required information.
%For most regions, this command will return the same as
%\cs{LocaleCurrencyLabel}.
%
%For example:
%\begin{verbatim}
%\def\LocaleMain{en-GB}
%\def\LocaleOther{en-IM,pt-BR}
%
%\input locale
%
%en-GB: \LocaleCurrencyRegionalLabel{british}.
%en-IM: \LocaleCurrencyRegionalLabel{isleofmanenglish}.
%pt-BR: \LocaleCurrencyRegionalLabel{brazilian}.
%
%\bye
%\end{verbatim}
%This produces:
%\begin{quote}
%en-GB: GBP. en-IM: IMP. pt-BR: BRL.
%\end{quote}
%If I had used \cs{LocaleCurrencyLabel} instead, the \texttt{en-IM}
%currency label would've been displayed as GBP.
%
%The currency symbol (for example, \$ or \pounds) is obtained using
%\begin{definition}[\DescribeMacro\LocaleCurrencySymbol]
%\cs{LocaleCurrencySymbol}\marg{dialect}
%\end{definition}
%Except in the case of \$ this will include symbols outside the Basic
%Latin set. This means that if you use this command in your document,
%you need to ensure that the document encoding has been set up before
%\file{tex-locale.tex} is loaded. (The \LaTeX\ \file{tex-locale.sty} package
%can automatically load \sty{fontspec} or
%\sty{inputenc} \& \sty{fontenc}, if required.) As with the previous
%commands, if the currency is unknown it will try to fallback on the
%main or OS currency. If the \dattr{currencysym} dialect attribute
%is the same as the currency code, then the \cattr{sym} currency
%attribute will be used instead.
%
%Modifying the above example:
%\begin{verbatim}
%\font\nimbus="NimbusRoman-Regular" at 10pt
%\nimbus
%
%\def\LocaleMain{en-GB}
%\def\LocaleOther{en-IM,pt-BR}
%
%\input locale
%
%en-GB: \LocaleCurrencySymbol{british}.
%en-IM: \LocaleCurrencySymbol{isleofmanenglish}.
%pt-BR: \LocaleCurrencySymbol{brazilian}.
%
%\bye
%\end{verbatim}
%This produces:
%\begin{quote}
%en-GB: \pounds. en-IM: M\pounds. pt-BR: R\$.
%\end{quote}
%The \XeLaTeX\ equivalent is:
%\begin{verbatim}
%\documentclass{article}
%
%\usepackage[main={en-GB},other={en-IM,pt-BR}]{tex-locale}
%\setmainfont{NimbusRoman-Regular}
%
%\begin{document}
%en-GB: \LocaleCurrencySymbol{british}.
%en-IM: \LocaleCurrencySymbol{isleofmanenglish}.
%pt-BR: \LocaleCurrencySymbol{brazilian}.
%\end{document}
%\end{verbatim}
%
%Alternatively, you can use:
%\begin{definition}[\DescribeMacro\LocaleCurrencyTeXSymbol]
%\cs{LocaleCurrencyTeXSymbol}\marg{dialect}
%\end{definition}
%This works in the same way as the previous command, except
%that it checks the \dattr{currencytex} dialect attribute.
%If this attribute is the same as the currency code, then
%the \cattr{tex} currency attribute is used instead.
%
%This command uses control sequences instead of the actual currency
%character. Since both \sty{texosquery} and \sty{tex-locale} are designed
%for generic \TeX\ use, \sty{texosquery} just performs some limited
%tests for the existence of common command names (such as \cs{pounds}
%or \cs{euro}). If it can't find an appropriate command to use, the
%currency commands simply expand to a textual tag.
%
%For example, when \file{texosquery.tex} is input, it checks for the
%existence of \cs{faGbp} and \cs{pounds}. If one of those commands
%exist, then \ics{texosquerycurrencypound} is defined to that
%commands, otherwise it's defined to just \qt{pound}. For the Euro
%symbol (\ics{texosquerycurrencyeuro}) the code checks for
%\cs{faEuro}, \cs{texteuro} and \cs{euro}. (See the \sty{texosquery}
%package for more information about these currency commands.)
%
%The \LaTeX\ \file{tex-locale.sty} package will automatically load the
%\sty{textcomp} package by default. You can switch this to another
%package (for example, \sty{fontawesome}) using the \pkgopt{symbols}
%package option (for example, \pkgoptfmt{symbols=fontawesome}).
%
%\section{Current Locale}
%\label{sec:currentlocale}
%
%The commands described above mostly require a recognised \sty{tracklang}
%dialect label in the argument to identify the locale. These labels
%aren't intuitively obvious. There are some predefined dialect labels
%that try to be compatible with known \sty{babel} dialects, but
%neither \sty{babel} nor \sty{polyglossia} provide as much detail
%about the non-language aspects of the dialect (such as the region or
%variant). For example, there's no \sty{babel} or \sty{polyglossia}
%setting for English in the Isle of Man (\texttt{en-IM}). Users need
%to select the closest matching dialect (\texttt{british} in the case
%of \texttt{en-IM}). The \sty{tracklang} package also allows an
%unofficial language and region combination.
%For example, a document written in English but with the regional information,
%such as currencies, matching those for Belgium, should be identified
%with \texttt{en-BE}. This isn't recognised as a predefined
%\sty{tracklang} dialect, so \sty{tracklang} creates its own dialect
%label (\texttt{enBE} in this case) when it parses this language tag.
%
%This makes it a bit awkward to directly use the above commands and
%it's most likely that the required dialect will be the currently selected
%language setting, so \file{tex-locale.tex} provides some convenient
%wrapper commands, listed below. These \cs{CurrentLocale\ldots} commands are redefined every
%time the locale is changed (except for \cs{CurrentLocaleDateTime}). For a document that doesn't use
%\sty{babel} or \sty{polyglossia}, these commands can be reset using
%\ics{selectlocale} (described on page~\pageref{pg:selectlocale}). If \file{tex-locale.tex} detects
%any language hooks \cs{captions}\meta{language} (such as
%\cs{captionsenglish} or \cs{captionsbritish}) then code is added to
%those hooks to ensure that the locale is switched when the language
%changes.
%
%\TeX's \ics{show} command provides a useful way of checking the
%current settings. For example:
%\begin{verbatim}
%\def\LocaleMain{en-GB}
%\def\LocaleOther{en-IM,pt-BR}
%
%\input locale
%
%\show\CurrentLocaleMonthName
%
%\selectlocale{en-IM}
%\show\CurrentLocaleMonthName
%
%\selectlocale{pt-BR}
%\show\CurrentLocaleMonthName
%
%\bye
%\end{verbatim}
%These cause three interruptions to the \TeX\ build, which look like
%errors but are the results from each \cs{show} command. In the first
%case the transcript shows:
%\begin{verbatim}
%> \CurrentLocaleMonthName=macro:
%->\LocaleMonthName {british}.
%\end{verbatim}
%I haven't specifically set the locale at this point. At the end of
%the \file{tex-locale.tex} code, the main locale was automatically
%selected. So the current locale dialect label is \texttt{british}
%and \cs{CurrentLocaleMonthName} has been defined as
%\begin{verbatim}
%\LocaleMonthName{british}
%\end{verbatim}
%Note that there's no check for \cs{languagename} or similar command
%here. The dialect label is hard-coded into the definition of
%\cs{CurrentLocaleMonthName}, so it's fully-expandable (unless the
%month name contains any awkward non-expandable characters, which can
%occur with \sty{inputenc} and non-ASCII characters).
%
%The next instance of \cs{show} produces the following lines in the
%transcript:
%\begin{verbatim}
%> \CurrentLocaleMonthName=macro:
%->\LocaleMonthName {isleofmanenglish}.
%\end{verbatim}
%The command \cs{CurrentLocaleMonthName} was redefined when 
%\verb|\selectlocale{en-IM}| was used.
%
%Similarly for the final instance of \cs{show}:
%\begin{verbatim}
%> \CurrentLocaleMonthName=macro:
%->\LocaleMonthName {brazilian}.
%\end{verbatim}
%
%All the commands that are redefined with each instance of \cs{selectlocale} 
%are listed below. Note that \cs{selectlocale} also uses
%\sty{tracklang}'s \cs{SetCurrentTrackedDialect}, which also defines
%a set of commands that can be used to identify information about the
%current dialect. (See the \sty{tracklang} documentation for further
%details.)
%
%\begin{definition}[\DescribeMacro\CurrentLocaleLanguageName]
%\cs{CurrentLocaleLanguageName}
%\end{definition}
%This is a shortcut for \ics{LocaleLanguageName}\marg{dialect}.
%
%\begin{definition}[\DescribeMacro\CurrentLocaleLanguageNativeName]
%\cs{CurrentLocaleLanguageNativeName}
%\end{definition}
%This is a shortcut for \ics{LocaleLanguageNativeName}\marg{dialect}.
%
%\begin{definition}[\DescribeMacro\CurrentLocaleRegionName]
%\cs{CurrentLocaleRegionName}
%\end{definition}
%This is a shortcut for \ics{LocaleRegionName}\marg{dialect}.
%
%\begin{definition}[\DescribeMacro\CurrentLocaleRegionNativeName]
%\cs{CurrentLocaleRegionNativeName}
%\end{definition}
%This is a shortcut for \ics{LocaleRegionNativeName}\marg{dialect}.
%
%\begin{definition}[\DescribeMacro\CurrentLocaleVariantName]
%\cs{CurrentLocaleVariantName}
%\end{definition}
%This is a shortcut for \ics{LocaleVariantName}\marg{dialect}.
%
%\begin{definition}[\DescribeMacro\CurrentLocaleVariantNativeName]
%\cs{CurrentLocaleVariantNativeName}
%\end{definition}
%This is a shortcut for \ics{LocaleRegionNativeName}\marg{dialect}.
%
%\subsection{Dates and Times}
%\label{sec:currdatetime}
%
%\begin{definition}[\DescribeMacro\CurrentLocaleFirstDayIndex]
%\cs{CurrentLocaleFirstDayIndex}
%\end{definition}
%This is a shortcut for \ics{LocaleFirstDayIndex}\marg{dialect}.
%
%\begin{definition}[\DescribeMacro\CurrentLocaleDayIndexFromRegion]
%\cs{CurrentLocaleDayIndexFromRegion}\marg{index}
%\end{definition}
%This is a shortcut for
%\ics{LocaleDayIndexFromRegion}\marg{dialect}\marg{index}.
%
%\begin{definition}[\DescribeMacro\CurrentLocaleDayName]
%\cs{CurrentLocaleDayName}\marg{index}
%\end{definition}
%This is a shortcut for \ics{LocaleDayName}\marg{dialect}\marg{index}.
%
%\begin{definition}[\DescribeMacro\CurrentLocaleShortDayName]
%\cs{CurrentLocaleShortDayName}\marg{index}
%\end{definition}
%This is a shortcut for \ics{LocaleShortDayName}\marg{dialect}\marg{index}.
%
%\begin{definition}[\DescribeMacro\CurrentLocaleStandaloneDayName]
%\cs{CurrentLocaleStandaloneDayName}\marg{index}
%\end{definition}
%This is a shortcut for \ics{LocaleStandaloneDayName}\marg{dialect}\marg{index}.
%
%\begin{definition}[\DescribeMacro\CurrentLocaleStandaloneShortDayName]
%\cs{CurrentLocaleStandaloneShortDayName}\marg{index}
%\end{definition}
%This is a shortcut for \ics{LocaleStandaloneShortDayName}\marg{dialect}\marg{index}.
%
%\begin{definition}[\DescribeMacro\CurrentLocaleMonthName]
%\cs{CurrentLocaleMonthName}\marg{index}
%\end{definition}
%This is a shortcut for \ics{LocaleMonthName}\marg{dialect}\marg{index}.
%
%\begin{definition}[\DescribeMacro\CurrentLocaleShortMonthName]
%\cs{CurrentLocaleShortMonthName}\marg{index}
%\end{definition}
%This is a shortcut for \ics{LocaleShortMonthName}\marg{dialect}\marg{index}.
%
%\begin{definition}[\DescribeMacro\CurrentLocaleStandaloneMonthName]
%\cs{CurrentLocaleStandaloneMonthName}\marg{index}
%\end{definition}
%This is a shortcut for \ics{LocaleStandaloneMonthName}\marg{dialect}\marg{index}.
%
%\begin{definition}[\DescribeMacro\CurrentLocaleStandaloneShortMonthName]
%\cs{CurrentLocaleStandaloneShortMonthName}\marg{index}
%\end{definition}
%This is a shortcut for \ics{LocaleStandaloneShortMonthName}\marg{dialect}\marg{index}.
%
%\begin{definition}[\DescribeMacro\CurrentLocaleDate]
%\cs{CurrentLocaleDate}
%\end{definition}
%This is slightly more complicated that the above. It uses
%\begin{definition}[\DescribeMacro\localedatechoice]
%\cs{localedatechoice}\marg{full}\marg{long}\marg{medium}\marg{short}
%\end{definition}
%to determine whether to use the full date \ics{LocaleFullDate}\marg{dialect},
%the long date \ics{LocaleLongDate}\marg{dialect},
%the medium date \ics{LocaleMediumDate}\marg{dialect} or
%the short date \ics{LocaleShortDate}\marg{dialect}.
%
%This command may be defined before \file{tex-locale.tex} is loaded or
%redefined afterwards. For example, to ensure that
%\cs{CurrentLocaleDate} uses the medium form:
%\begin{verbatim}
%\def\localedatechoice#1#2#3#4{#3}
%\input locale
%\end{verbatim}
%\LaTeX\ users should instead use the \pkgopt{date} package option in
%the \file{tex-locale.sty} package interface:
%\begin{verbatim}
%\usepackage[date=medium]{tex-locale}
%\end{verbatim}
%\LaTeX\ users can also redefine the command afterwards. For example:
%\begin{verbatim}
%\renewcommand*{\localedatechoice}[4]{#3}
%\end{verbatim}
%
%There is a similar command for the time:
%\begin{definition}[\DescribeMacro\CurrentLocaleTime]
%\cs{CurrentLocaleTime}
%\end{definition}
%This uses
%\begin{definition}[\DescribeMacro\localetimechoice]
%\cs{localetimechoice}\marg{full}\marg{long}\marg{medium}\marg{short}
%\end{definition}
%to determine whether to use the full time \ics{LocaleFullTime}\marg{dialect},
%the long time \ics{LocaleLongTime}\marg{dialect},
%the medium time \ics{LocaleMediumTime}\marg{dialect} or
%the short time \ics{LocaleShortTime}\marg{dialect}. As with the date choice, this command
%may be defined by plain \TeX\ users before \file{tex-locale.tex} is
%input. \LaTeX\ users should use the \pkgopt{time} package option
%instead. In both formats, the command may be redefined after
%\sty{tex-locale} has been loaded.
%
%There is also a command for the combined date and time:
%\begin{definition}[\DescribeMacro\CurrentLocaleDateTime]
%\cs{CurrentLocaleDateTime}
%\end{definition}
%This doesn't use the analogous \cs{LocaleFullDateTime} etc
%commands, but instead is simply defined as:
%\begin{code}
%\cs{CurrentLocaleDate}\cs{space}\cs{CurrentLocaleTime}
%\end{code}
%This allows for a mix of date and time styles, reflecting
%the definitions of \cs{localedatechoice} and \cs{localetimechoice}.
%
%If you want a specific style (ignoring \cs{localedatechoice}
%and \cs{localetimechoice}), you can use the following commands:
%\begin{definition}[\DescribeMacro\CurrentLocaleFullDate]
%\cs{CurrentLocaleFullDate}
%\end{definition}
%This just uses \cs{LocaleFullDate}\marg{dialect}.
%
%\begin{definition}[\DescribeMacro\CurrentLocaleLongDate]
%\cs{CurrentLocaleLongDate}
%\end{definition}
%This just uses \cs{LocaleLongDate}\marg{dialect}.
%
%\begin{definition}[\DescribeMacro\CurrentLocaleMediumDate]
%\cs{CurrentLocaleMediumDate}
%\end{definition}
%This just uses \cs{LocaleMediumDate}\marg{dialect}.
%
%\begin{definition}[\DescribeMacro\CurrentLocaleShortDate]
%\cs{CurrentLocaleShortDate}
%\end{definition}
%This just uses \cs{LocaleShortDate}\marg{dialect}.
%
%\begin{definition}[\DescribeMacro\CurrentLocaleFullTime]
%\cs{CurrentLocaleFullTime}
%\end{definition}
%This just uses \cs{LocaleFullTime}\marg{dialect}.
%
%\begin{definition}[\DescribeMacro\CurrentLocaleLongTime]
%\cs{CurrentLocaleLongTime}
%\end{definition}
%This just uses \cs{LocaleLongTime}\marg{dialect}.
%
%\begin{definition}[\DescribeMacro\CurrentLocaleMediumTime]
%\cs{CurrentLocaleMediumTime}
%\end{definition}
%This just uses \cs{LocaleMediumTime}\marg{dialect}.
%
%\begin{definition}[\DescribeMacro\CurrentLocaleShortTime]
%\cs{CurrentLocaleShortTime}
%\end{definition}
%This just uses \cs{LocaleShortTime}\marg{dialect}.
%
%\begin{definition}[\DescribeMacro\CurrentLocaleFullDateTime]
%\cs{CurrentLocaleFullDateTime}
%\end{definition}
%This just uses \cs{LocaleFullDateTime}\marg{dialect}.
%
%\begin{definition}[\DescribeMacro\CurrentLocaleLongDateTime]
%\cs{CurrentLocaleLongDateTime}
%\end{definition}
%This just uses \cs{LocaleLongDateTime}\marg{dialect}.
%
%\begin{definition}[\DescribeMacro\CurrentLocaleMediumDateTime]
%\cs{CurrentLocaleMediumDateTime}
%\end{definition}
%This just uses \cs{LocaleMediumDateTime}\marg{dialect}.
%
%\begin{definition}[\DescribeMacro\CurrentLocaleShortDateTime]
%\cs{CurrentLocaleShortDateTime}
%\end{definition}
%This just uses \cs{LocaleShortDateTime}\marg{dialect}.
%
%
%\subsection{Numeric Symbols}
%\label{sec:currnumsym}
%
%The currency for the current locale is similar in construct to the
%current locale's date and time commands listed above.
%\begin{definition}[\DescribeMacro\CurrentLocaleCurrency]
%\cs{CurrentLocaleCurrency}
%\end{definition}
%This uses
%\begin{definition}[\DescribeMacro\localecurrchoice]
%\cs{localecurrchoice}\marg{label}\marg{regional}\marg{symbol}\marg{\TeX}
%\end{definition}
%to determine whether to use \ics{LocaleCurrencyLabel}\marg{dialect},
%\ics{LocaleCurrencyRegionalLabel}\marg{dialect},
%\ics{LocaleCurrencySymbol}\marg{dialect} or
%\ics{LocaleCurrencyTeXSymbol}\marg{dialect}.
%This can similarly be defined before \file{tex-locale.tex} is input or
%redefined afterwards. \LaTeX\ users can use the \pkgopt{currency}
%package option.
%
%\begin{definition}[\DescribeMacro\CurrentLocaleNumericGroupSep]
%\cs{CurrentLocaleNumericGroupSep}
%\end{definition}
%This is a shortcut for \ics{LocaleNumericGroupSep}\marg{dialect}.
%
%\begin{definition}[\DescribeMacro\CurrentLocaleIfNumericUsesGroup]
%\cs{CurrentLocaleIfNumericUsesGroup}\marg{true code}\marg{false code}
%\end{definition}
%This is a shortcut for
%\ics{LocaleIfNumericUsesGroup}\marg{dialect}\marg{true
%code}\marg{false code}.
%
%The next few commands have a slightly different pattern to the
%shortcut control sequence name. For brevity, the shortcut command
%omits the \qt{Numeric} part of the corresponding name.
%
%\begin{definition}[\DescribeMacro\CurrentLocaleDecimalSep]
%\cs{CurrentLocaleDecimalSep}
%\end{definition}
%This is a shortcut for \ics{LocaleNumericDecimalSep}\marg{dialect}.
%
%\begin{definition}[\DescribeMacro\CurrentLocaleMonetarySep]
%\cs{CurrentLocaleMonetarySep}
%\end{definition}
%This is a shortcut for \ics{LocaleNumericMonetarySep}\marg{dialect}.
%
%\begin{definition}[\DescribeMacro\CurrentLocaleExponent]
%\cs{CurrentLocaleExponent}
%\end{definition}
%This is a shortcut for \ics{LocaleNumericExponent}\marg{dialect}.
%
%\subsection{Current Locale Patterns}
%\label{sec:currpat}
%
%\begin{definition}[\DescribeMacro\CurrentLocaleIntegerPattern]
%\cs{CurrentLocaleIntegerPattern}
%\end{definition}
%This is a convenient shortcut to access the integer pattern for the
%current dialect.
%
%\begin{definition}[\DescribeMacro\CurrentLocaleDecimalPattern]
%\cs{CurrentLocaleDecimalPattern}
%\end{definition}
%This is a convenient shortcut to access the decimal pattern for the
%current dialect.
%
%\begin{definition}[\DescribeMacro\CurrentLocaleCurrencyPattern]
%\cs{CurrentLocaleCurrencyPattern}
%\end{definition}
%This is a convenient shortcut to access the currency pattern for the
%current dialect.
%
%\begin{definition}[\DescribeMacro\CurrentLocalePercentPattern]
%\cs{CurrentLocalePercentPattern}
%\end{definition}
%This is a convenient shortcut to access the percent pattern for the
%current dialect.
%
%\begin{definition}[\DescribeMacro\CurrentLocaleApplyDateTimePattern]
%\cs{CurrentLocaleApplyDateTimePattern}
%\end{definition}
%This is a convenient shortcut for
%\ics{LocaleApplyDateTimePattern}\marg{dialect}.
%
%\chapter{\texorpdfstring{\LaTeX}{LaTeX}\ Use}
%\label{sec:latex}
%
%The \sty{tex-locale} package is loaded in \LaTeX\ with the usual
%\cs{usepackage} syntax:
%\begin{verbatim}
%\usepackage{tex-locale}
%\end{verbatim}
%This does more than simply input \file{tex-locale.tex}. The options
%are listed below.
%
%\begin{description}
%\item[\pkgopt{main}=\marg{tag}] This option identifies the main locale for the
%document. The value should be a valid language tag. If this option
%is omitted, the main language will be obtained from the Java Runtime
%Environment, which should match your operating system's default
%locale. If any languages have already been tracked in your document with 
%\sty{tracklang} prior to loading \sty{tex-locale}, the main
%language is set to the first tracked dialect. Normally you don't
%need to explicitly load \sty{tracklang}.
%
%\item[\pkgopt{other}=\marg{list}] This option identifies other
%locales required by the document. The value should be a
%comma-separated list of language tags or the keyword \pkgoptfmt{locale}
%to indicate the system's default locale. The value must be grouped
%to protect any commas from the key=value parser.
%\begin{important}
%This option has a cumulative effect.
%\end{important}
%
%\item[\pkgopt{symbols}=\marg{name}] This option identifies the
%symbol package to automatically load. For example,
%\pkgoptfmt{symbols=textcomp} or \pkgoptfmt{symbols=fontawesome}. If
%another package is required, \sty{texosquery}'s currency symbol commands
%will need to be redefined as appropriate.
%
%The keyword \pkgoptfmt{none} indicates that no package is required. The
%default is \pkgoptfmt{symbols=none} for \XeLaTeX\ or \LuaLaTeX,
%otherwise it's \pkgoptfmt{symbols=textcomp}.
%
%\item[\pkgopt{support}=\marg{value}] This option identifies the
%language support value. Available values are:
%\begin{itemize}
%\item \pkgoptfmt{none}: don't load any language support package;
%\item \pkgoptfmt{auto}: load \sty{polyglossia} for \XeLaTeX\ or
%\LuaLaTeX, otherwise load \sty{babel};
%\item \pkgoptfmt{babel}: load \sty{babel} regardless of the \LaTeX\
%format;
%\item \pkgoptfmt{polyglossia}: load \sty{polyglossia} without checking
%the \LaTeX\ format (but remember that \sty{polyglossia} doesn't work
%with \PDFLaTeX);
%\item \pkgoptfmt{cjk}: load CJK support with either \sty{xeCJK}
%(\XeLaTeX/\LuaLaTeX) or \sty{CJKutf8}. This isn't fully tested, and
%there's no support for non-UTF encoding.
%\end{itemize}
%
%\item[\pkgopt{fontenc}=\marg{value}] This option indicates whether
%or not to load the \sty{fontenc} package. The value should be one
%of: \pkgoptfmt{none} (don't load), \pkgoptfmt{auto} (automatically load
%with the encoding determined from the language settings), or
%\meta{option} (indicating the package option to pass to
%\sty{fontenc}). The default is \pkgoptfmt{fontenc=auto} unless
%\sty{fontenc} has already been loaded.
%
%This option is ignored with \XeLaTeX\ or \LuaLaTeX.
%
%\item[\pkgopt{inputenc}=\marg{value}] This option indicates whether
%or not to load the \sty{inputenc} package. The value should be one
%of: \pkgoptfmt{none} (don't load), \pkgoptfmt{auto} (automatically load
%with the encoding determined from the default obtained from
%\sty{texosquery}), or \meta{option} (indicating the package option
%to pass to \sty{inputenc}). The default is \pkgoptfmt{inputenc=auto}
%but \sty{inputenc} won't be loaded if the appropriate encoding can't
%be determined. Note that loading \sty{inputenc} before the
%\sty{texosquery} lookup is performed will cause a problem with
%non-ASCII characters appearing in the result.
%
%This option is ignored with \XeLaTeX\ or \LuaLaTeX.
%
%\item[\pkgopt{datetime}=\meta{value}] This option indicates whether
%or not to load the \sty{datetime2} package. The value may be one of:
%\begin{itemize}
%\item \pkgoptfmt{false}: don't load \sty{datetime2};
%\item \pkgoptfmt{iso}: load \sty{datetime2} with the package option
%\pkgoptfmt{useregional=false} and sets the date-time style to \pkgoptfmt{iso};
%\item \pkgoptfmt{text}: load \sty{datetime2} with the package option
%\pkgoptfmt{useregional=text};
%\item \pkgoptfmt{num} or \pkgoptfmt{numeric}: load \sty{datetime2} with the 
%package option \pkgoptfmt{useregional=numeric};
%\item \pkgoptfmt{locale}: don't load \sty{datetime2} and set \ics{today}
%to \ics{CurrentLocaleDate}, if available.
%\end{itemize}
%
%\item[\pkgopt{date}=\meta{value}] This option indicates the
%preferred date style for \ics{CurrentLocaleDate}. The value may be
%one of: \pkgoptfmt{full}, \pkgoptfmt{long}, \pkgoptfmt{medium}
%\pkgoptfmt{short}. This option doesn't affect \ics{today} unless you
%have used \pkgoptfmt{datetime=locale}.
%
%\item[\pkgopt{time}=\meta{value}] This option indicates the
%preferred time style for \ics{CurrentLocaleTime}. The value may be
%one of: \pkgoptfmt{full}, \pkgoptfmt{long}, \pkgoptfmt{medium}
%\pkgoptfmt{short}.
%
%\item[\pkgopt{timedate}=\meta{boolean}] This option indicates
%whether or not to support date and time patterns. (See
%\sectionref{sec:datetimepat}.)
%
%\item[\pkgopt{currency}=\meta{value}] This option indicates the
%preferred currency style for \ics{CurrentLocaleCurrency}. Available
%values: \pkgoptfmt{official} (official identifier), \pkgoptfmt{unofficial}
%unofficial identified, \pkgoptfmt{sym} (the symbol, which may include
%non-ASCII characters) or \pkgoptfmt{tex} (use the commands provided by
%\sty{texosquery}). The default is \pkgoptfmt{currency=sym} for
%\XeLaTeX/\LuaLaTeX\ or \pkgoptfmt{currency=tex} otherwise.
%
%\end{description}
%
%The \sty{tex-locale} package will use \sty{texosquery} to lookup the
%default language tag and encoding, and then attempt to load the
%appropriate encoding and language support packages.
%
%For example, consider the following document:
%\begin{verbatim}
%\documentclass{article}
%\usepackage{tex-locale}
%\begin{document}
%Currency: \CurrentLocaleCurrency.
%\end{document}
%\end{verbatim}
%My default locale is \texttt{en-GB} with UTF-8 as the default file
%encoding, so if I compile this document with \PDFLaTeX, then this is
%essentially equivalent to:
%\begin{verbatim}
%\documentclass{article}
%\usepackage[en-GB]{tracklang}
%\usepackage[T1]{fontenc}
%\usepackage[utf8]{inputenc}
%\usepackage[british]{babel}
%\usepackage[useregional]{datetime2}
%\begin{document}
%Currency: \pounds.
%\end{document}
%\end{verbatim}
%(Other dependent packages omitted for clarity.) Whereas if I compile
%the document with \XeLaTeX, then this is essentially equivalent to:
%\begin{verbatim}
%\documentclass{article}
%\usepackage[en-GB]{tracklang}
%\usepackage{fontspec}
%\usepackage{polyglossia}
%\setmainlanguage[variant=uk]{english}
%\usepackage[useregional]{datetime2}
%\begin{document}
%Currency: £.
%\end{document}
%\end{verbatim}
%If I still want to use \sty{babel}, then I can enforce this with:
%\begin{verbatim}
%\usepackage[support=babel]{tex-locale}
%\end{verbatim}
%Or if I don't want \sty{babel} or \sty{polyglossia}:
%\begin{verbatim}
%\usepackage[support=none]{tex-locale}
%\end{verbatim}
%
%You can determine which language package was used with:
%\begin{definition}[\DescribeMacro\LocaleSupportPackageCase]
%\cs{LocaleSupportPackageCase}\marg{babel}\marg{polyglossia}\marg{neither}
%\end{definition}
%If either package has been loaded \ics{selectlocale}\marg{locale}
%should automatically be implemented when the document language
%changes, so you can the use the current locale commands described in
%\sectionref{sec:currentlocale}.
%
%Example:
%\begin{verbatim}
%\usepackage[main={sr-Cyrl-RS},other={en-GB}]{tex-locale}
%
%\LocaleSupportPackageCase
%{\newcommand{\textenglish}[1]{\foreignlanguage{british}{#1}}}% babel
%{% polyglossia
%  \setmainfont{Liberation Serif}
%  \newfontfamily\cyrillicfont{Liberation Serif}
%}
%{\newcommand{\textenglish}[1]{#1}}% none
%\end{verbatim}
%
%\StopEventually{%
% \printindex[user]
% \PrintCodeIndex
% \PrintChanges
%}
%
%
%\chapter{The Code}
%\iffalse
%    \begin{macrocode}
%<*tex-locale.sty>
%    \end{macrocode}
%\fi
%\changes{1.0}{2018-08-26}{Initial release}
%\section{\LaTeX\ Code (\texttt{tex-locale.sty})}
%    \begin{macrocode}
\NeedsTeXFormat{LaTeX2e}
\ProvidesPackage{tex-locale}[2018/08/26 v1.0 (NLCT) localisation support]
%    \end{macrocode}
%Make life easier by using \sty{etoolbox}:
%    \begin{macrocode}
\RequirePackage{etoolbox}
%    \end{macrocode}
% Also need \sty{xfor} to break out of \cs{@for} loop.
%    \begin{macrocode}
\RequirePackage{xfor}
%    \end{macrocode}
%Load \sty{tracklang} using package interface just in case any
%languages have been passed through the document class options.
%Require at least v1.3.3 since the dialect label mappings are needed
%in this file.
%    \begin{macrocode}
\RequirePackage{tracklang}[2016/11/03]
%    \end{macrocode}
%If any languages have been tracked, set the main language to the
%first tracked dialect.
%    \begin{macrocode}
\AnyTrackedLanguages
{%
   \ForEachTrackedDialect{\locale@this@dialect}
   {
      \ifx\LocaleMain\undefined
        \edef\LocaleMain{\GetTrackedLanguageTag{\locale@this@dialect}}
        \let\@locale@trackedmain\LocaleMain
      \else
        \ifx\LocaleOther\undefined
          \edef\LocaleOther{\GetTrackedLanguageTag{\locale@this@dialect}}
        \else
          \edef\LocaleOther{\LocaleOther,\GetTrackedLanguageTag{\locale@this@dialect}}
        \fi
      \fi
   }
}
{}
%    \end{macrocode}
%Need to determine if we're using \XeLaTeX\ or \LuaLaTeX.
%    \begin{macrocode}
\RequirePackage{ifxetex}
\RequirePackage{ifluatex}
%    \end{macrocode}
%\begin{macro}{\@locale@ifxeorlua}
%Short cut to check if we're using either:
%    \begin{macrocode}
\ifxetex
  \newcommand*{\@locale@ifxeorlua}[2]{#1}
\else
  \ifluatex 
    \newcommand*{\@locale@ifxeorlua}[2]{#1}
  \else
    \newcommand*{\@locale@ifxeorlua}[2]{#2}
  \fi
\fi
%    \end{macrocode}
%\end{macro}
%
%Need \sty{xkeyval} for key-value interface:
%    \begin{macrocode}
\RequirePackage{xkeyval}
%    \end{macrocode}
%
%Define package options.
%
% Each \cs{DeclareOptionX} needs a corresponding \cs{DeclareOption}
% so that it can be passed as a document class option, so define a
% command that will implement both.
%\begin{macro}{\@locale@declareoption}
%    \begin{macrocode}
\newcommand*{\@locale@declareoption}[2]{%
  \DeclareOptionX{#1}{#2}%
  \DeclareOption{#1}{#2}%
}
%    \end{macrocode}
%\end{macro}
%
%\begin{option}{main}
%This option identifies the main locale. This will
%default to the operating system's locale if not set. The value
%should be the language tag or the keyword \texttt{locale}.
%    \begin{macrocode}
\define@key{tex-locale.sty}{main}{%
%    \end{macrocode}
%If this has already been set in the earlier tracked dialect check,
%move the original main to the other list.
%    \begin{macrocode}
  \ifx\@locale@trackedmain\undefined
  \else
    \ifx\LocaleOther\undefined
      \let\LocaleOther\@locale@trackedmain
    \else
      \edef\LocaleOther{\@locale@trackedmain,\LocaleOther}%
    \fi
    \let\@locale@trackedmain\undefined
  \fi
  \def\LocaleMain{#1}%
}
%    \end{macrocode}
%\end{option}
%\begin{option}{other}
%This option identifies the other locales. The value
%should be a comma-separated list of language tags or the keyword
%\texttt{locale}. The value must be grouped to protect any commas
%from the key-value parser. Note this option has a cumulative
%effect.
%    \begin{macrocode}
\define@key{tex-locale.sty}{other}{%
  \ifx\LocaleOther\undefined
    \def\LocaleOther{#1}%
  \else
    \edef\LocaleOther{\LocaleOther,#1}%
  \fi
}
%    \end{macrocode}
%\end{option}
%\begin{option}{symbols}
%This option identifies the symbol package to automatically load.
%For example, \sty{textcomp} or \texttt{fontawesome}. If another
%package is required, \sty{texosquery}'s currency symbol commands
%will need to be redefined as appropriate. The key word \texttt{none}
%indicates that no package is required. The default is \texttt{none}
%for \XeLaTeX\ or \LuaLaTeX\ and \sty{textcomp} otherwise.
%\texttt{textcomp}
%    \begin{macrocode}
\@locale@ifxeorlua{\def\@locale@symbols{none}}{\def\@locale@symbols{textcomp}}
\define@key{tex-locale.sty}{symbols}{\def\@locale@symbols{#1}}
%    \end{macrocode}
%\end{option}
%
%To make it easier to add extra support options, assign a numeric
%value to each option so that \cs{ifcase} can be used. Recognised
%values: 0 (none), 1 (auto), 2 (babel), 3 (polyglossia), 4 (cjk).
%\begin{macro}{\@locale@supportopt}
%    \begin{macrocode}
\newcount\@locale@supportopt
%    \end{macrocode}
% The default value is auto unless a known language package has 
% already been loaded.
%    \begin{macrocode}
\@ifpackageloaded{polyglossia}
{\@locale@supportopt=0\relax}
{
  \@ifpackageloaded{babel}
  {\@locale@supportopt=0\relax}
  {
    \@ifpackageloaded{CJK}
    {\@locale@supportopt=0\relax}
    {\@locale@supportopt=1\relax}
  }
}
%    \end{macrocode}
%\end{macro}
%
%\begin{option}{support}
%This option identifies the language support package.
%    \begin{macrocode}
\define@choicekey{tex-locale.sty}{support}%
 [\@locale@support@val\@locale@support@nr]%
 {none,auto,babel,polyglossia,cjk}
{\@locale@supportopt=\@locale@support@nr\relax}
%    \end{macrocode}
%\end{option}
%\begin{option}{fontspec}
%This option identifies whether or not to load \sty{fontspec} if 
%\XeLaTeX\ or \LuaLaTeX\ is in use. Defaults to true and is ignored
%if neither \XeLaTeX\ nor \LuaLaTeX\ are in use.
%    \begin{macrocode}
\define@boolkey{tex-locale.sty}[@locale@]{fontspec}[true]{}
\@locale@ifxeorlua{\@locale@fontspectrue}{\@locale@fontspecfalse}
%    \end{macrocode}
%\end{option}
%\begin{option}{fontenc}
%This option identifies whether or not to load \sty{fontenc} if 
%neither \XeLaTeX\ nor \LuaLaTeX\ is in use. The value should be
%either \pkgoptfmt{none} (don't load) or \pkgoptfmt{auto}
%(determined from the main language or script) or the option to pass to
%\sty{fontenc}. If the \pkgoptfmt{auto} option is used but the
%appropriate encoding can't be determined, \sty{fontenc} won't be
%loaded. The default value is \pkgoptfmt{auto} unless \sty{fontenc}
%has already been loaded. This option is ignored if 
% either \XeLaTeX\ or \LuaLaTeX\ are in use.
%    \begin{macrocode}
\define@key{tex-locale.sty}{fontenc}{%
 \edef\@locale@fontenc{#1}%
 \ifdefstring{\@locale@fontenc}{false}{\def\@locale@fontenc{none}}{}%
}
\@ifpackageloaded{fontenc}
{\def\@locale@fontenc{none}}
{\def\@locale@fontenc{auto}}
%    \end{macrocode}
%\end{option}
%\begin{option}{inputenc}
%This option identifies whether or not to load \sty{inputenc} if 
%neither \XeLaTeX\ nor \LuaLaTeX\ is in use. The value should be
%either \pkgoptfmt{none} (don't load) or \pkgoptfmt{auto}
%(determined from the locale's default encoding) or the option to pass to
%\sty{inputenc}. If the \pkgoptfmt{auto} option is used but the
%appropriate encoding can't be determined, \sty{inputenc} won't be
%loaded. The default value is \pkgoptfmt{auto}. This option is ignored if 
% either \XeLaTeX\ or \LuaLaTeX\ are in use. Note that loading
% \sty{inputenc} before the query is performed will cause a problem
% with non-ASCII characters appearing in the result.
%    \begin{macrocode}
\define@key{tex-locale.sty}{inputenc}{%
  \edef\@locale@inputenc{#1}%
  \ifdefstring{\@locale@inputenc}{false}{\def\@locale@inputenc{none}}{}%
}
\@ifpackageloaded{inputenc}
{\def\@locale@inputenc{none}}
{\def\@locale@inputenc{auto}}
%    \end{macrocode}
%\end{option}
%
%\begin{macro}{\@locale@load@dtm}
%This command is used to load \sty{datetime2}, if required, and setup the style.
%    \begin{macrocode}
\@ifpackageloaded{datetime2}
{
%    \end{macrocode}
% User has already loaded \sty{datetime2}. Assume they have already
% set their preferred style. (They would also need to have loaded
% \sty{babel}\slash\sty{polyglossia} before \sty{datetime2}, which
% rather reduces the point of the \sty{tex-locale} package.)
%    \begin{macrocode}
  \newcommand\@locale@load@dtm{}
}
{
  \newcommand\@locale@load@dtm{%
%    \end{macrocode}
% First check if \sty{datetime2} is installed. If it isn't, then
% default to \pkgopt[locale]{datetime2}.
%    \begin{macrocode}
    \IfFileExists{datetime2.sty}%
      {\@locale@load@regional@dtm}%
      {\@locale@set@today}%
  }%
}
%    \end{macrocode}
%\end{macro}
%
%\begin{macro}{\@locale@load@regional@dtm}
%Load \sty{datetime2} with regional settings.
%    \begin{macrocode}
\newcommand\@locale@load@regional@dtm{%
  \localedatechoice
    {%full
     \PassOptionsToPackage{showdow}{datetime2}%
     \PassOptionsToPackage{useregional=text}{datetime2}%
    }%
    {%long
      \PassOptionsToPackage{useregional=text}{datetime2}%
    }%
    {%medium
      \PassOptionsToPackage{useregional=text}{datetime2}%
    }%
    {%short
      \PassOptionsToPackage{useregional=numeric}{datetime2}%
    }%
  \RequirePackage{datetime2}%
  \localedatechoice{}{}{\DTMlangsetup*{abbr}}{}%
}%
%    \end{macrocode}
%\end{macro}
%
%\begin{macro}{\@locale@set@today}
%Redefine \cs{today}.
%    \begin{macrocode}
\newcommand*{\@locale@set@today}{%
%    \end{macrocode}
% If \cs{CurrentLocaleDate} is empty then the query failed, in which
% case don't change \cs{today}.
%    \begin{macrocode}
  \ifdefempty\CurrentLocaleDate
  {}%
  {%
    \renewcommand{\today}{\CurrentLocaleDate}%
    \ForEachTrackedDialect{\locale@this@dialect}
    {%
%    \end{macrocode}
%Add to \cs{date}\meta{lang} hook.
%    \begin{macrocode}
      \SetCurrentTrackedDialect{\locale@this@dialect}%
      \@TrackLangAddToHook
       {\renewcommand{\today}{\CurrentLocaleDate}}
       {date}%
    }%
  }%
}
%    \end{macrocode}
%\end{macro}
%
%
%\begin{option}{datetime}
%Provide an option to adjust the date and time settings.
%    \begin{macrocode}
\define@choicekey{tex-locale.sty}{datetime}%
 [\@locale@datetime@val\@locale@datetime@nr]%
 {false,iso,text,num,numeric,locale}
{%
  \ifcase\@locale@datetime@nr
    \def\@locale@load@dtm{}%
  \or
    \def\@locale@load@dtm{%
      \PassOptionsToPackage{useregional=false}{datetime2}%
      \RequirePackage{datetime2}%
      \DTMsetstyle{iso}%
    }%
  \or
    \def\@locale@load@dtm{%
      \@locale@load@regional@dtm
    }%
  \or
    \def\@locale@load@dtm{%
      \PassOptionsToPackage{useregional=numeric}{datetime2}%
      \RequirePackage{datetime2}%
    }%
  \or
    \def\@locale@load@dtm{%
      \PassOptionsToPackage{useregional=numeric}{datetime2}%
      \RequirePackage{datetime2}%
    }%
  \or
    \def\@locale@load@dtm{\@locale@set@today}%
  \fi
}
%    \end{macrocode}
%\end{option}
%
%\begin{option}{iso}
%Shortcut for \pkgopt[iso]{datetime2}.
%    \begin{macrocode}
\@locale@declareoption{iso}{%
  \def\@locale@load@dtm{%
    \PassOptionsToPackage{useregional=false}{datetime2}%
    \RequirePackage{datetime2}%
    \DTMsetstyle{iso}%
  }%
}
%    \end{macrocode}
%\end{option}
%
%\begin{option}{date}
%Preferred date style for \cs{CurrentLocaleDate}.
%    \begin{macrocode}
\define@choicekey{tex-locale.sty}{date}%
[\@locale@date@val\@locale@date@nr]{full,long,medium,short}
{%
  \ifcase\@locale@date@nr
    \def\localedatechoice##1##2##3##4{##1}%
  \or
    \def\localedatechoice##1##2##3##4{##2}%
  \or
    \def\localedatechoice##1##2##3##4{##3}%
  \or
    \def\localedatechoice##1##2##3##4{##4}%
  \fi
}
\newcommand*{\localedatechoice}[4]{#2}%
%    \end{macrocode}
%\end{option}
%
%\begin{option}{time}
%Preferred date style for \cs{CurrentLocaleTime}.
%    \begin{macrocode}
\define@choicekey{tex-locale.sty}{time}%
[\@locale@time@val\@locale@time@nr]{full,long,medium,short}
{%
  \ifcase\@locale@time@nr
    \def\localetimechoice##1##2##3##4{##1}%
  \or
    \def\localetimechoice##1##2##3##4{##2}%
  \or
    \def\localetimechoice##1##2##3##4{##3}%
  \or
    \def\localetimechoice##1##2##3##4{##4}%
  \fi
}
\newcommand*{\localetimechoice}[4]{#3}%
%    \end{macrocode}
%\end{option}
%
%\begin{option}{timedata}
%Determine whether or not to use \texttt{-M} and \texttt{-Z}.
%    \begin{macrocode}
\define@choicekey{tex-locale.sty}{timedata}%
[\@locale@timedata@val\@locale@timedata@nr]{true,false}[true]
{%
  \ifcase\@locale@timedata@nr
    \def\LocaleIfDateTimePatternsSupported##1##2{##1}%
  \or
    \def\LocaleIfDateTimePatternsSupported##1##2{##2}%
  \fi
}
%    \end{macrocode}
%\end{option}
%
%\begin{option}{currency}
%Preferred currency style for \cs{CurrentLocaleCurrency}.
%    \begin{macrocode}
\define@choicekey{tex-locale.sty}{currency}%
[\@locale@currency@val\@locale@currency@nr]%
{official,unofficial,sym,tex}
{%
  \ifcase\@locale@currency@nr
    \def\localecurrchoice##1##2##3##4{##1}%
  \or
    \def\localecurrchoice##1##2##3##4{##2}%
  \or
    \def\localecurrchoice##1##2##3##4{##3}%
  \or
    \def\localecurrchoice##1##2##3##4{##4}%
  \fi
}
\@locale@ifxeorlua
 {\newcommand*{\localecurrchoice}[4]{#3}}
 {\newcommand*{\localecurrchoice}[4]{#4}}
%    \end{macrocode}
%\end{option}
%
% Process package options. First process any options that have been
% passed via the document class.
%    \begin{macrocode}
\@for\CurrentOption:=\@declaredoptions\do{%
  \ifx\CurrentOption\@empty
  \else
    \@expandtwoargs
      \in@{,\CurrentOption,}{,\@classoptionslist,\@curroptions,}%
    \ifin@
      \@use@ption
      \expandafter\let\csname ds@\CurrentOption\endcsname\@empty
    \fi
  \fi
}
%    \end{macrocode}
% Now process options passed to the package:
%    \begin{macrocode}
\ProcessOptionsX
%    \end{macrocode}
%
% Check if \sty{babel} or \sty{polyglossia} have already been loaded
% (user may have used \pkgopt[babel]{support} or \pkgopt[polyglossia]{support}
% without realising it's already been loaded). Since
% \sty{polyglossia} pretends \sty{babel} has been loaded, only
% \sty{babel} test is required.
%    \begin{macrocode}
\@ifpackageloaded{babel}
{%
  \@locale@supportopt=0\relax
}
{}
%    \end{macrocode}
% If the main language hasn't been set, define it (needed before
% the generic code is input.)
%    \begin{macrocode}
\ifx\LocaleMain\undefined
  \def\LocaleMain{locale}
\fi
\ifx\LocaleOther\undefined
  \def\LocaleOther{}
\fi
%    \end{macrocode}
%
%\begin{macro}{\@locale@postparse@hook}
%Post-parser hook.
%    \begin{macrocode}
\newcommand*{\@locale@postparse@hook}{%
  \input{tex-locale-support.def}%
}
%    \end{macrocode}
%\end{macro}
%
%If symbol support is required, load the package now:
%    \begin{macrocode}
\ifdefstring{\@locale@symbols}{none}
{}
{\RequirePackage{\@locale@symbols}}
%    \end{macrocode}
%Load \sty{texosquery} using package interface (needs to be done
%after the symbol support is loaded since \texttt{texosquery.tex}
%detects some common symbol commands).
%    \begin{macrocode}
\RequirePackage{texosquery}
%    \end{macrocode}
%The encoding needs to be set up before the main query in case there
%are any non-ASCII characters returned by the query (for example,
%the currency symbol or in month or day names).
%
% May need to track all the listed dialects before
% \texttt{tex-locale.tex} is loaded.
%\begin{macro}{\@locale@trackall}
% Track the main language if set.
%    \begin{macrocode}
\newcommand\@locale@trackall{%
  \ifdefstring\LocaleMain{locale}%
  {%
    \ifx\LocaleOStag\empty
      \PackageWarning{tex-locale}{Unable to determine locale
        (check shell escape)}%
    \else
      \TrackLanguageTag{\LocaleOStag}%
    \fi
  }%
  {\TrackLanguageTag{\LocaleMain}}%
%    \end{macrocode}
% Track the other languages.
%    \begin{macrocode}
  \@for\locale@this@dialect:=\LocaleOther\do{%
    \ifdefstring\locale@this@dialect{locale}
    {%
      \ifx\LocaleOStag\empty
        \PackageWarning{tex-locale}{Unable to determine locale
          (check shell escape)}%
      \else
        \TrackLanguageTag{\LocaleOStag}%
      \fi
    }%
    {\TrackLanguageTag{\locale@this@dialect}}%
  }%
  \let\@locale@trackall\relax
}
%    \end{macrocode}
%\end{macro}
%
%\begin{macro}{\@locale@loadinputenc}
%    \begin{macrocode}
\newcommand{\@locale@loadinputenc}{}
%    \end{macrocode}
%\end{macro}
%
%It may be necessary to use texosquery here to determine the
%language tag ("-b") and\slash or the codeset ("-C"). It makes more 
%sense to minimise the number of shell escapes, so try to determine 
%what extra information we need here. The required parameters are stored in
%\cs{@locale@pre@query@params} 
%\begin{macro}{\@locale@pre@query@params}
%    \begin{macrocode}
\newcommand*\@locale@pre@query@params{}%
%    \end{macrocode}
%\end{macro}
%Now we need some commands that will parse the result, depending on
%the parameters.
%\begin{macro}{\@locale@pre@query@parsetag}
%Only the language tag ("-b") is required. The shell escape will
%only return a single result. \cs{LocaleOStag} will either be set to
%the language tag, if successful, or will be empty, if in dry run
%mode.
%    \begin{macrocode}
\newcommand*\@locale@pre@query@parsetag{%
  \ifx\LocaleStyQueryFile\undefined
   \TeXOSQuery{\@locale@result}{\@locale@pre@query@params}%
  \else
   \ifx\LocaleStyQueryFile\empty
     \TeXOSQuery{\@locale@result}{\@locale@pre@query@params}%
   \else
     \ifx\TeXOSQueryFromFile\undefined
       \PackageError{locale}{texosquery too old to support
       \string\LocaleStyQueryFile. At least v1.4 required}
       {You need to update your version of texosquery}
       \def\@locale@result{}%
     \else
       \PackageInfo{locale}{Fetching query results from `\LocaleStyQueryFile'}%
       \TeXOSQueryFromFile{\@locale@result}{\LocaleStyQueryFile}%
     \fi
   \fi
  \fi
  \edef\LocaleOStag{\@locale@result}%
}%
%    \end{macrocode}
%\end{macro}
%\begin{macro}{\@locale@pre@query@parsecodeset}
%Only the codeset ("-C") is required. The shell escape will
%only return a single result. \cs{LocaleOScodeset} will either be set to
%the codeset, if successful, or will be empty, if in dry run
%mode.
%    \begin{macrocode}
\newcommand*\@locale@pre@query@parsecodeset{%
  \TeXOSQuery{\@locale@result}{\@locale@pre@query@params}%
  \edef\LocaleOScodeset{\@locale@result}%
}%
%    \end{macrocode}
%\end{macro}
%\begin{macro}{\@locale@pre@query@parsetagcodeset}
%Both the language tag ("-b") and the codeset ("-C") are required.
%The shell escape will return two arguments.
%    \begin{macrocode}
\newcommand*\@locale@pre@query@parsetagcodeset{%
  \def\LocaleOStag{}%
  \def\LocaleOScodeset{}%
  \TeXOSQuery{\@locale@result}{\@locale@pre@query@params}%
  \ifx\@locale@result\@empty
  \else
    \edef\LocaleOStag{\expandafter\@firstoftwo\@locale@result}%
    \edef\LocaleOScodeset{\expandafter\@secondoftwo\@locale@result}%
  \fi
}%
%    \end{macrocode}
%\end{macro}
%\begin{macro}{\@locale@pre@query@parse}
%This will be set to the appropriate command after determining what
%information is actually required.
%    \begin{macrocode}
\let\@locale@pre@query@parse\relax
%    \end{macrocode}
%\end{macro}
%
%If \XeLaTeX\ or \LuaLaTeX\ is in use and the \pkgopt{fontspec} option
%is on, load \sty{fontspec}.
%    \begin{macrocode}
\@locale@ifxeorlua
{%
  \if@locale@fontspec\RequirePackage{fontspec}\fi
%    \end{macrocode}
%The language tag is needed if \pkgopt[auto]{support}.
%    \begin{macrocode}
  \ifnum\@locale@supportopt=1\relax
    \ifdefstring\LocaleMain{locale}
    {%
       \def\@locale@pre@query@params{\string-b }%
       \let\@locale@pre@query@parse\@locale@pre@query@parsetag
    }%
    {%
      \@for\locale@this@dialect:=\LocaleOther\do{%
       \ifdefstring\locale@this@dialect{locale}
       {%
         \def\@locale@pre@query@params{\string-b }%
         \let\@locale@pre@query@parse\@locale@pre@query@parsetag
         \@endfortrue
       }%
       {}%
      }%
    }%
%    \end{macrocode}
%Perform the shell escape if required.
%    \begin{macrocode}
  \@locale@pre@query@parse
  \fi
}
{%
%    \end{macrocode}
%The language tag is needed if \pkgopt[auto]{fontenc}.
%    \begin{macrocode}
  \ifdefstring{\@locale@fontenc}{auto}
  {%
    \ifdefstring\LocaleMain{locale}
    {%
       \def\@locale@pre@query@params{\string-b }%
       \let\@locale@pre@query@parse\@locale@pre@query@parsetag
    }%
    {%
      \@for\locale@this@dialect:=\LocaleOther\do{%
       \ifdefstring\locale@this@dialect{locale}
       {%
         \def\@locale@pre@query@params{\string-b }%
         \let\@locale@pre@query@parse\@locale@pre@query@parsetag
         \@endfortrue
       }%
       {}%
      }%
    }%
  }%
  {}%
%    \end{macrocode}
%The codeset is needed if \pkgopt[auto]{inputenc}. 
%The "-C" (\texttt{-{}-codeset-lcs}) switch is used rather
%than the "-cs" (\texttt{-{}-codeset}) switch to make it more
%compatible with \sty{inputenc}.
%    \begin{macrocode}
  \ifdefstring{\@locale@inputenc}{auto}
  {%
     \ifx\@locale@pre@query@params\@empty
       \let\@locale@pre@query@parse\@locale@pre@query@parsecodeset
     \else
       \let\@locale@pre@query@parse\@locale@pre@query@parsetagcodeset
     \fi
     \edef\@locale@pre@query@params{\@locale@pre@query@params\string-C}%
  }%
  {}%
%    \end{macrocode}
%Perform the shell escape if required.
%    \begin{macrocode}
  \@locale@pre@query@parse
%    \end{macrocode}
% Do we need to load \sty{fontenc}?
%    \begin{macrocode}
  \ifdefstring{\@locale@fontenc}{none}
  {}
  {%
    \ifdefstring{\@locale@fontenc}{auto}
    {%
%    \end{macrocode}
% If \pkgopt[auto]{fontenc}, set up some script to \sty{fontenc} mappings.
%    \begin{macrocode}
     \input{tex-locale-scripts-enc.def}%
     \@locale@trackall
%    \end{macrocode}
% Iterate through all the dialects.
%    \begin{macrocode}
     \ForEachTrackedDialect{\locale@this@dialect}%
     {
       \edef\@locale@lang{\TrackedLanguageFromDialect{\locale@this@dialect}}%
       \@locale@if@langenc@map{\locale@this@dialect}%
       {%
           \edef\@locale@fontenc@opt{%
             \@locale@get@langenc@map{\locale@this@dialect}}%
           \expandafter\PassOptionsToPackage\expandafter
             {\@locale@fontenc@opt}{fontenc}%
       }%
       {%
         \@locale@if@langenc@map{\@locale@lang}%
         {%
           \edef\@locale@fontenc@opt{%
             \@locale@get@langenc@map{\@locale@lang}}%
           \expandafter\PassOptionsToPackage\expandafter
             {\@locale@fontenc@opt}{fontenc}%
         }%
         {%
           \edef\@locale@script{\GetTrackedDialectScript{\locale@this@dialect}}%
           \ifx\@locale@script\empty
            \edef\@locale@script{\TrackLangGetDefaultScript{\@locale@lang}}%
           \fi
           \@locale@if@scriptenc@map{\@locale@script}%
           {%
             \edef\@locale@fontenc@opt{%
               \@locale@get@scriptenc@map{\@locale@script}}%
             \expandafter\PassOptionsToPackage\expandafter
               {\@locale@fontenc@opt}{fontenc}%
           }%
           {}%
         }%
       }%
     }%
     \ifx\@locale@fontenc@opt\empty
       \PackageWarning{tex-locale}{Option `fontenc=auto' failed. 
         Can't determine an appropriate font encoding for dialect(s). 
         (Dialect list: \@tracklang@dialects.)
         Either set the encoding explicitly or switch to XeLaTeX%
       }%
     \else
       \RequirePackage{fontenc}%
     \fi
    }
    {%
      \expandafter\PassOptionsToPackage\expandafter{\@locale@fontenc}{fontenc}%
      \RequirePackage{fontenc}%
    }
  }
%    \end{macrocode}
% Do we need to load \sty{inputenc}? This needs to be done before
% the main \texttt{tex-locale.tex} shell escape.
%    \begin{macrocode}
  \ifdefstring{\@locale@inputenc}{none}
  {}
  {%
    \ifdefstring{\@locale@inputenc}{auto}
    {%
%    \end{macrocode}
%If the earlier shell escape was successful, \cs{LocaleOScodeset} will be set.
%    \begin{macrocode}
       \ifx\LocaleOScodeset\empty
%    \end{macrocode}
%Query failed.
%    \begin{macrocode}
        \PackageWarning{tex-locale}{Option `inputenc=auto' failed.
        (Check shell escape.) Default file encoding unavailable}
       \else
%    \end{macrocode}
%If the file \cs{LocaleOScodeset}\texttt{.def} exists, then the
%codeset should hopefully be valid.
%    \begin{macrocode}
        \IfFileExists{\LocaleOScodeset.def}%
        {%
          \renewcommand{\@locale@loadinputenc}{%
            \RequirePackage{inputenc}%
            \inputencoding{\LocaleOScodeset}%
          }
          \let\@locale@inputenc\LocaleOScodeset
        }%
        {%
%    \end{macrocode}
%May have a different name. Try known encoding mappings.
%    \begin{macrocode}
           \input{tex-locale-encodings.def}%
           \@locale@ifhasencmap{\LocaleOScodeset}
           {
              \edef\@locale@inputenc{\@locale@getencmap\LocaleOScodeset}
              \renewcommand{\@locale@loadinputenc}{%
                \RequirePackage{inputenc}%
                \inputencoding{\@locale@inputenc}%
              }
           }
           {%
             \PackageWarning{tex-locale}{Option `inputenc=auto' failed.
              Don't know how to interpret codeset `\LocaleOScodeset'}%
           }%
        }
       \fi
    }
    {%
      \renewcommand{\@locale@loadinputenc}{%
        \RequirePackage{inputenc}%
        \inputencoding{\@locale@inputenc}%
      }
    }
  }
}
%    \end{macrocode}
%
%\begin{macro}{\@locale@loadscripts}
%Only load \sty{tracklang-scripts} if required.
%    \begin{macrocode}
\newcommand{\@locale@loadscripts}{%
  \RequirePackage{tracklang-scripts}%
  \let\@locale@loadscripts\relax
}
%    \end{macrocode}
%\end{macro}
%
%The \LaTeX\ kernel provides \cs{@thirdofthree} but not
%\cs{@secondofthree}.
%\begin{macro}{\@secondofthree}
%    \begin{macrocode}
\providecommand*{\@secondofthree}[3]{#2}
%    \end{macrocode}
%\end{macro}
%
%\begin{macro}{\@locale@ifsupportbabelorpoly}
%\begin{definition}
%\cs{@locale@ifsupportbabelorpoly}\marg{neither
%case}\marg{babel case}\marg{polyglossia case}
%\end{definition}
%Should \sty{babel} or \sty{polyglossia} be loaded? Initialise to false.
%    \begin{macrocode}
\newcommand*{\@locale@ifsupportbabelorpoly}[3]{#1}
%    \end{macrocode}
%\end{macro}
%
%\begin{macro}{\@locale@ifsupportcjk}
%Should \sty{CJK} be loaded? Initialise to false.
%    \begin{macrocode}
\newcommand*{\@locale@ifsupportcjk}[2]{#2}
%    \end{macrocode}
%\end{macro}
%
%\begin{macro}{\@locale@ifsupportpinyin}
%Should \sty{pinyin} be loaded? Initialise to false.
%    \begin{macrocode}
\newcommand*{\@locale@ifsupportpinyin}[2]{#2}
%    \end{macrocode}
%\end{macro}
%
%\begin{macro}{\@locale@cjklist}
%List of languages supported by \sty{CJK}. This just makes it easier to test the
%language without multiple conditions.
%    \begin{macrocode}
\newcommand*{\@locale@cjklist}{}
\listadd{\@locale@cjklist}{chinese}
\listadd{\@locale@cjklist}{japanese}
\listadd{\@locale@cjklist}{korean}
\listadd{\@locale@cjklist}{thai}
%    \end{macrocode}
%\end{macro}
%
%\begin{macro}{\@locale@ifcjk}
%\begin{definition}
%\cs{@locale@ifcjk}\marg{language}\marg{true case}\marg{false case}
%\end{definition}
%Check if given root language is in the CJK list.
%    \begin{macrocode}
\newcommand*{\@locale@ifcjk}[1]{%
  \xifinlist{#1}{\@locale@cjklist}%
}
%    \end{macrocode}
%\end{macro}
%
%\begin{macro}{\@locale@iflatinscript}
%\begin{definition}
%\cs{@locale@iflatinscript}\marg{dialect}\marg{true case}\marg{false case}
%\end{definition}
%Check if given dialect has the script explicitly set to Latn.
%    \begin{macrocode}
\newcommand*{\@locale@iflatinscript}[1]{%
  \ifcsstring{@tracklang@script@#1}{Latn}%
}
%    \end{macrocode}
%\end{macro}
%
%Should \sty{babel} or \sty{polyglossia} be loaded?
%    \begin{macrocode}
\ifcase\@locale@supportopt
%    \end{macrocode}
% No support required.
%    \begin{macrocode}
\or
%    \end{macrocode}
% \pkgopt{auto} option.
%    \begin{macrocode}
  \@locale@trackall
  \ForEachTrackedDialect{\locale@this@dialect}%
  {%
%    \end{macrocode}
% Get root language name.
%    \begin{macrocode}
    \edef\this@root@lang{\TrackedLanguageFromDialect{\locale@this@dialect}}%
%    \end{macrocode}
% Is this language in the CJK list?
%    \begin{macrocode}
    \@locale@ifcjk{\this@root@lang}%
    {
      \let\@locale@ifsupportcjk\@firstoftwo
%    \end{macrocode}
% Is pinyin needed?
%    \begin{macrocode}
      \@locale@iflatinscript{\locale@this@dialect}%
      {\let\@locale@ifsupportpinyin\@firstoftwo}%
      {}
    }
    {}
%    \end{macrocode}
% No point checking for \sty{polyglossia} support if not using
% \XeLaTeX\ or \LuaLaTeX.
%    \begin{macrocode}
    \@locale@ifxeorlua
    {
      \@locale@ifsupportbabelorpoly
      {
%    \end{macrocode}
% Haven't yet determined support for \sty{babel} or \sty{polyglossia}.
% Does the file \texttt{gloss-}\meta{language}\texttt{.ldf} exist?
%    \begin{macrocode}
        \IfFileExists{gloss-\this@root@lang.ldf}
        {
          \let\@locale@ifsupportbabelorpoly\@thirdofthree
        }
        {
%    \end{macrocode}
% Does the file \meta{language}\texttt{.ldf} exist?
%    \begin{macrocode}
          \IfFileExists{\this@root@lang.ldf}
          {\let\@locale@ifsupportbabelorpoly\@secondofthree}
          {}
        }
      }
%    \end{macrocode}
% Don't bother checking if already determined that \sty{babel} needs
% to be used.
%    \begin{macrocode}
      {}
      {
%    \end{macrocode}
% Already found one language supported by \sty{polyglossia}. Now
% check this one.
% Does the file \texttt{gloss-}\meta{language}\texttt{.ldf} exist?
%    \begin{macrocode}
        \IfFileExists{gloss-\this@root@lang.ldf}
        {}
        {
%    \end{macrocode}
% No support for this language with \sty{polyglossia}.
% Does the file \meta{language}\texttt{.ldf} exist?
%    \begin{macrocode}
          \IfFileExists{\this@root@lang.ldf}
          {\let\@locale@ifsupportbabelorpoly\@secondofthree}
          {}
        }
      }
    }
    {
%    \end{macrocode}
% Not using \XeLaTeX\ or \LuaLaTeX\ so no \sty{polyglossia} support.
% Does the file \meta{language}\texttt{.ldf} exist?
%    \begin{macrocode}
      \IfFileExists{\this@root@lang.ldf}
      {\let\@locale@ifsupportbabelorpoly\@secondofthree}
      {}
    }
  }
\or
%    \end{macrocode}
% \pkgopt{babel} option.
%    \begin{macrocode}
  \let\@locale@ifsupportbabelorpoly\@secondofthree
\or
%    \end{macrocode}
% \pkgopt{polyglossia} option.
%    \begin{macrocode}
  \let\@locale@ifsupportbabelorpoly\@thirdofthree
\or
%    \end{macrocode}
% \pkgopt{cjk} option.
%    \begin{macrocode}
  \let\@locale@currentiscjk\@secondoftwo
%    \end{macrocode}
% Is pinyin needed?
%    \begin{macrocode}
  \ForEachTrackedDialect{\locale@this@dialect}%
  {%
    \edef\this@root@lang{\TrackedLanguageFromDialect{\locale@this@dialect}}%
    \@locale@ifcjk{\this@root@lang}%
    {
      \@locale@iflatinscript{\locale@this@dialect}%
      {\let\@locale@ifsupportpinyin\@firstoftwo}%
      {}
    }
    {}
  }
%    \end{macrocode}
%End of case statement
%    \begin{macrocode}
\fi
%    \end{macrocode}
%
% Does \sty{CJK} need to be loaded?
%    \begin{macrocode}
\@locale@ifsupportcjk
{
%    \end{macrocode}
% Is UTF-8 support needed?
%    \begin{macrocode}
  \@locale@ifxeorlua
  {
    \RequirePackage{xeCJK}%
  }
  {
    \ifdefstring\@locale@inputenc{utf8}%
    {%
      \RequirePackage{CJKutf8}%
%    \end{macrocode}
% \sty{CJKutf8} automatically loads
% \sty{inputenc} with the \pkgopt{utf8} option.
%    \begin{macrocode}
      \renewcommand{\@locale@loadinputenc}{}%
%    \end{macrocode}
% Need to ensure UTF-8 characters are correctly set up when the
% query is made.
%    \begin{macrocode}
      \newcommand*{\localeprequery}{\begin{CJK}{UTF8}{}\makeatletter}
      \newcommand*{\localepostquery}{\end{CJK}}
    }%
    {%
      \RequirePackage{CJK}%
%    \end{macrocode}
% Non-UTF encoding. Not implemented as I'm
% not familiar with these encodings.
%    \begin{macrocode}
      \PackageWarning{tex-locale}{Unsupported encoding `\@locale@loadinputenc'}%
    }%
  }
%    \end{macrocode}
% Load \sty{pinyin} if needed.
%    \begin{macrocode}
  \@locale@ifsupportpinyin
  {\RequirePackage{pinyin}}%
  {}%
}
{}
%    \end{macrocode}

% Load \sty{inputenc} if required.
%    \begin{macrocode}
\@locale@loadinputenc
%    \end{macrocode}
%
%\begin{macro}{\localenopolypunct}
%Just does its argument (scoped) if \sty{polyglossia} hasn't been
%loaded. (Made robust if \sty{polyglossia} is loaded.)
%    \begin{macrocode}
\newcommand{\localenopolypunct}[1]{{#1}}
%    \end{macrocode}
%\end{macro}
%
%\begin{macro}{\@locale@nopolypunct}
%Robust form used with \sty{polyglossia}.
%    \begin{macrocode}
\newrobustcmd{\@locale@nopolypunct}[1]{%
  {%
    \@tracklang@ifundef{no\languagename @punctuation}{}%
    {\csname no\languagename @punctuation\endcsname}%
    #1%
  }%
}
%    \end{macrocode}
%\end{macro}
%
%
%Load the generic code.
%    \begin{macrocode}
% tex-locale.ins generated using makedtx version 1.2 2018/8/26 19:14
\input docstrip

\preamble

 tex-locale.dtx
 Copyright 2018 Nicola Talbot

 This work may be distributed and/or modified under the
 conditions of the LaTeX Project Public License, either version 1.3
 of this license or (at your option) any later version.
 The latest version of this license is in
   http://www.latex-project.org/lppl.txt
 and version 1.3 or later is part of all distributions of LaTeX
 version 2005/12/01 or later.

 This work has the LPPL maintenance status `maintained'.

 The Current Maintainer of this work is Nicola Talbot.

 This work consists of the files tex-locale.dtx and tex-locale.ins and the derived files tex-locale.sty, tex-locale.tex, tex-locale-scripts-enc.def, tex-locale-encodings.def, tex-locale-support.def.

\endpreamble

\askforoverwritefalse

\generate{\file{tex-locale.sty}{\usepreamble\defaultpreamble
\usepostamble\defaultpostamble\from{tex-locale.dtx}{tex-locale.sty,package}}
\file{tex-locale.tex}{\usepreamble\defaultpreamble
\usepostamble\defaultpostamble\from{tex-locale.dtx}{tex-locale.tex,package}}
\file{tex-locale-scripts-enc.def}{\usepreamble\defaultpreamble
\usepostamble\defaultpostamble\from{tex-locale.dtx}{tex-locale-scripts-enc.def,package}}
\file{tex-locale-encodings.def}{\usepreamble\defaultpreamble
\usepostamble\defaultpostamble\from{tex-locale.dtx}{tex-locale-encodings.def,package}}
\file{tex-locale-support.def}{\usepreamble\defaultpreamble
\usepostamble\defaultpostamble\from{tex-locale.dtx}{tex-locale-support.def,package}}
}

\endbatchfile

%    \end{macrocode}
%Load \sty{datetime2} if required:
%    \begin{macrocode}
\@locale@load@dtm
%    \end{macrocode}
%
%\begin{macro}{\LocaleSupportPackageCase}
%\begin{definition}
%\cs{LocaleSupportPackageCase}\marg{babel}\marg{polyglossia}\marg{neither}
%\end{definition}
%Provide a user-level command to determine whether \sty{babel} or
%\sty{polyglossia} was used. This doesn't test if \sty{CJK} was
%loaded, which may have additionally been loaded.
%    \begin{macrocode}
\@ifpackageloaded{polyglossia}
{\newcommand{\LocaleSupportPackageCase}[3]{#2}}
{%
  \@ifpackageloaded{babel}
  {\newcommand{\LocaleSupportPackageCase}[3]{#1}}%
  {\newcommand{\LocaleSupportPackageCase}[3]{#3}}%
}
%    \end{macrocode}
%\end{macro}
%\iffalse
%    \begin{macrocode}
%</tex-locale.sty>
%    \end{macrocode}
%\fi
%\iffalse
%    \begin{macrocode}
%<*tex-locale.tex>
%    \end{macrocode}
%\fi
%\section{Generic Code (\texttt{tex-locale.tex})}
% Does the category code of \verb|@| need changing?
%\begin{macro}{\@locale@restore@at}
%    \begin{macrocode}
\ifnum\catcode`\@=11\relax
  \def\@locale@restore@at{}%
\else
  \expandafter\edef\csname @locale@restore@at\endcsname{%
    \noexpand\catcode`\noexpand\@=\number\catcode`\@\relax
  }%
 \catcode`\@=11\relax
\fi
%    \end{macrocode}
%\end{macro}
% First check if this file has already been loaded:
%    \begin{macrocode}
\ifx\@locale@parse@query\undefined
\else
  \@locale@restore@at
  \expandafter\endinput
\fi
%    \end{macrocode}
% Version info.
%    \begin{macrocode}
\expandafter\def\csname ver@tex-locale.tex\endcsname{2018/08/26 v1.0
(NLCT) localisation support}
%    \end{macrocode}
% Load \styfmt{tracklang} and \styfmt{texosquery}:
%    \begin{macrocode}
\input tracklang
\input texosquery
%    \end{macrocode}
%
%\begin{macro}{\@locale@err}
%    \begin{macrocode}
\ifx\PackageError\undefined
  \def\@locale@err#1#2{%
    \errhelp{#2}%
    \errmessage{tex-locale: #1}}
\else
  \def\@locale@err#1#2{\PackageError{tex-locale}{#1}{#2}}
\fi
%    \end{macrocode}
%\end{macro}
%
%\begin{macro}{\@locale@warn}
%Use \sty{tracklang}'s warning to allow all warnings to be switched
%off at the same time.
%    \begin{macrocode}
\def\@locale@warn{\@tracklang@pkgwarn{tex-locale}}
%    \end{macrocode}
%\end{macro}
%
%\begin{macro}{\@locale@info}
%Information message.
%    \begin{macrocode}
\ifx\PackageInfo\undefined
  \def\@locale@info#1{%
    {%
      \newlinechar=`\^^J
      \def\MessageBreak{^^J}%
      \message{^^Jtex-locale Info: #1^^J}%
    }%
  }
\else
  \def\@locale@info#1{\PackageInfo{tex-locale}{#1}}
\fi
%    \end{macrocode}
%\end{macro}
%
%Check \sty{tracklang} is at least v1.3.4.
%    \begin{macrocode}
\ifx\@tracklang@pkgwarn\undefined
  \@locale@err{tracklang version is too old. At least v1.3.4 required}
   {You need to update tracklang to at least v1.3.4}%
\fi
%    \end{macrocode}
%
%Check \sty{texosquery} is at least v1.4.
%    \begin{macrocode}
\ifx\@texosquery@argquote\undefined
  \@locale@err{texosquery version is too old. At least v1.4 required}
   {You need to update texosquery to at least v1.4}%
\fi
%    \end{macrocode}
%
%\begin{macro}{\LocaleMain}
%If \cs{LocaleMain} hasn't been defined, define it. This macro
%stores the language tag of the document's main region or
%\texttt{locale} to use the OS locale. The default is
%\texttt{locale}. This is a user-level command so it can be set before loading
%\file{tex-locale.tex}.
%    \begin{macrocode}
\ifx\LocaleMain\undefined
  \def\LocaleMain{locale}
\fi
%    \end{macrocode}
%Sanitize just in case.
%    \begin{macrocode}
\@tracklang@sanitize\LocaleMain
%    \end{macrocode}
%\end{macro}
%
%\begin{macro}{\LocaleOther}
%If \cs{LocaleOther} hasn't been defined, define it. This macro
%stores a comma-separated list of language tags or
%\texttt{locale} for additional regions. The default is empty. This
%is a user-level command so it can be set before loading
%\file{tex-locale.tex}.
%    \begin{macrocode}
\ifx\LocaleOther\undefined
  \def\LocaleOther{}
\else
%    \end{macrocode}
%Sanitize just in case.
%    \begin{macrocode}
  \@tracklang@sanitize\LocaleOther
\fi
%    \end{macrocode}
%\end{macro}
%
%\begin{macro}{\@locale@os@tag}
%Command to keep track of the OS locale. This is
%initialised as empty but will be set if the \sty{texosquery} call
%is successful.
%    \begin{macrocode}
\def\@locale@os@tag{}
%    \end{macrocode}
%\end{macro}
%
%\begin{macro}{\@locale@os@default}
%The keyword used to indicate the OS locale.
%    \begin{macrocode}
\def\@locale@os@default{locale}
%    \end{macrocode}
%Sanitize since it needs to be compared with \cs{LocaleMain}.
%    \begin{macrocode}
\@tracklang@sanitize\@locale@os@default
%    \end{macrocode}
%\end{macro}
%
%\begin{macro}{\@locale@unknown@currency}
%Unknown currency designator.
%    \begin{macrocode}
\def\@locale@unknown@currency{XXX}
%    \end{macrocode}
%\end{macro}
%
%\begin{macro}{\@locale@os@region}
%The OS region. Initialised as empty but will be set if the
%\sty{texosquery} call is successful.
%    \begin{macrocode}
\def\@locale@os@region{}
%    \end{macrocode}
%\end{macro}
%
%\begin{macro}{\@locale@os@groupsep}
%The OS numeric group sep. Initialised as empty but will be set if the
%\sty{texosquery} call is successful.
%    \begin{macrocode}
\def\@locale@os@groupsep{}
%    \end{macrocode}
%\end{macro}
%
%\begin{macro}{\@locale@os@decsep}
%The OS numeric group sep. Initialised as empty but will be set if the
%\sty{texosquery} call is successful.
%    \begin{macrocode}
\def\@locale@os@decsep{}
%    \end{macrocode}
%\end{macro}
%
%\begin{macro}{\@locale@os@cursep}
%The OS currency separator. Initialised as empty but will be set if the
%\sty{texosquery} call is successful.
%    \begin{macrocode}
\def\@locale@os@cursep{}
%    \end{macrocode}
%\end{macro}
%
%\begin{macro}{\@locale@os@exp}
%The OS exponent symbol. Initialised as empty but will be set if the
%\sty{texosquery} call is successful.
%    \begin{macrocode}
\def\@locale@os@exp{}
%    \end{macrocode}
%\end{macro}
%
%\begin{macro}{\@locale@os@usesgroup}
%The OS numeric uses group value. Initialised as empty but will be set if the
%\sty{texosquery} call is successful.
%    \begin{macrocode}
\def\@locale@os@usesgroup{}
%    \end{macrocode}
%\end{macro}
%
%\begin{macro}{\@locale@os@currencycode}
%The OS currency code. Initialised as unknown code but will be set if the
%\sty{texosquery} call is successful.
%    \begin{macrocode}
\def\@locale@os@currencycode{XXX}
%    \end{macrocode}
%\end{macro}
%
%\begin{macro}{\@locale@os@regionalcurrencycode}
%The OS regional currency code. Initialised as empty but will be set if the
%\sty{texosquery} call is successful.
%    \begin{macrocode}
\def\@locale@os@regionalcurrencycode{XXX}
%    \end{macrocode}
%\end{macro}
%
%\begin{macro}{\@locale@os@currencysym}
%The OS currency symbol. Initialised as empty but will be set if the
%\sty{texosquery} call is successful.
%    \begin{macrocode}
\def\@locale@os@currencysym{}
%    \end{macrocode}
%\end{macro}
%
%\begin{macro}{\@locale@os@currencytex}
%The OS currency \TeX\ code. Initialised as empty but will be set if the
%\sty{texosquery} call is successful.
%    \begin{macrocode}
\def\@locale@os@currencytex{}
%    \end{macrocode}
%\end{macro}
%
%\begin{macro}{\LocaleMainFile}
%The document's main file. (The modification date is queried if not
%empty.) If \cs{jobname} includes double-quotes, these need to be
%stripped to avoid interfering with the shell-escape (especially in
%restricted mode).
%    \begin{macrocode}
\ifx\LocaleMainFile\undefined
  \edef\LocaleMainFile{\expandafter\texosquerystripquotes{\jobname}.tex}
\fi
%    \end{macrocode}
%\end{macro}
%
%\begin{macro}{\LocaleIfDateTimePatternsSupported}
%The \texttt{-M} and \texttt{-Z} lookup is optional. This may be
%defined before \file{texosquery.tex} is input.
%    \begin{macrocode}
\ifx\LocaleIfDateTimePatternsSupported\undefined
 \def\LocaleIfDateTimePatternsSupported#1#2{#2}
\fi
%    \end{macrocode}
%\end{macro}
%
%\begin{macro}{\localedatetimefmt}
%Allow the date/time to be wrapped in a formatting command.
%    \begin{macrocode}
\def\localedatetimefmt#1{#1}
%    \end{macrocode}
%\end{macro}
%
%\subsection{Setting Command Line Switches}
%
%\begin{macro}{\LocaleQueryCodesetParam}
%\app{texosquery} has two different actions for obtaining the
%codeset: \texttt{-{}-codeset-lcs} ("-C") and \texttt{-{}-codeset}
%(-cs). The second returns the codeset name as recognised by Java.
%The first returns a modified version that's closer
%to the \sty{inputenc} setting. This macro defaults to using "-C"
%but may be defined before this file is loaded to use "-cs" instead.
%(This command isn't used by \file{locale.sty} when determining the
%input encoding for \sty{inputenc}.)
%    \begin{macrocode}
\ifx\LocaleQueryCodesetParam\undefined
  \edef\LocaleQueryCodesetParam{\string-C}
\fi
%    \end{macrocode}
%\end{macro}
%
%\begin{macro}{\@locale@query@params}
%It's more efficient to have a single \sty{texosquery} call, but the
%parameters need to be determined first. (Use short arg with
%\cs{string} just in case the hyphen character has a special
%meaning.)
%    \begin{macrocode}
\edef\@locale@query@params{%
%    \end{macrocode}
%Since we need to use \sty{texosquery} anyway, may as well look up
%the OS information at the same time.
%    \begin{macrocode}
  \string-o \string-r \string-a
%    \end{macrocode}
% May as well get the PDF date in case we have a TeX format that
% doesn't provide \ics{pdfcreationdate}.
%    \begin{macrocode}
  \string-n
%    \end{macrocode}
% Default locale data (requires \sty{texosquery} v1.2):
%    \begin{macrocode}
  \string-N
%    \end{macrocode}
% May as well get the default codeset (requires \sty{texosquery}
% v1.2). Note that \texttt{locale.sty} may have already found this
% if the \sty{inputenc} package was automatically loaded.
%    \begin{macrocode}
  \LocaleQueryCodesetParam\space
}
%    \end{macrocode}
%The remaining arguments need to be appended programatically.
%
%If the date-time patterns are needed, get the full date-time
%information.
%    \begin{macrocode}
\LocaleIfDateTimePatternsSupported
{%
  \edef\@locale@query@params{\@locale@query@params \string-M }
}
{}
%    \end{macrocode}
%
% May as well get the modification date of the document file since
% we need to make a system call anyway, but only if
% \cs{LocaleMainFile} isn't empty.
%    \begin{macrocode}
\ifx\LocaleMainFile\empty
\else
 \edef\@locale@query@params{\@locale@query@params
   \string-d \@texosquery@argquote{\LocaleMainFile}
 }
\fi
%    \end{macrocode}
%The \texttt{-D} switch also requires \sty{texosquery} v1.2.
%First deal with the main language.
%    \begin{macrocode}
\ifx\LocaleMain\@locale@os@default
 \LocaleIfDateTimePatternsSupported
 {%
   \edef\@locale@query@params{\@locale@query@params
     \string-D \string-Z
   }
 }
 {
   \edef\@locale@query@params{\@locale@query@params
     \string-D 
   }
 }
\else
 \LocaleIfDateTimePatternsSupported
 {%
   \edef\@locale@query@params{\@locale@query@params
     \string-D \LocaleMain\space\string-Z \LocaleMain\space
   }
 }
 {
   \edef\@locale@query@params{\@locale@query@params
     \string-D \LocaleMain\space
   }
 }
\fi
%    \end{macrocode}
%Iterate through the list of other languages.
%    \begin{macrocode}
\@tracklang@for\@locale@tag:=\LocaleOther\do{%
  \ifx\@locale@tag\@locale@os@default
   \LocaleIfDateTimePatternsSupported
   {%
     \edef\@locale@query@params{\@locale@query@params
       \string-D \string-Z
     }
   }
   {%
     \edef\@locale@query@params{\@locale@query@params
       \string-D 
     }
   }
  \else
   \LocaleIfDateTimePatternsSupported
   {%
     \edef\@locale@query@params{\@locale@query@params
       \string-D \@locale@tag\space\string-Z \@locale@tag\space
     }
   }
   {%
     \edef\@locale@query@params{\@locale@query@params
       \string-D \@locale@tag\space
     }
   }
  \fi
}
%    \end{macrocode}
%\end{macro}
%
%\subsection{System Call}
%
%\begin{macro}{\localeprequery}
%Allow for a hook immediately before the query.
%    \begin{macrocode}
\csname localeprequery\endcsname
%    \end{macrocode}
%\end{macro}
%
%Now run \sty{texosquery} unless \cs{LocaleQueryFile} has been
%defined, in which case use \cs{TeXOSQueryFromFile}
%    \begin{macrocode}
\ifx\LocaleQueryFile\undefined
%    \end{macrocode}
% No file provided.
%    \begin{macrocode}
  \TeXOSQuery{\@locale@result}{\@locale@query@params}
\else
  \ifx\LocaleQueryFile\empty
    \TeXOSQuery{\@locale@result}{\@locale@query@params}
  \else
%    \end{macrocode}
%\cs{TeXOSQueryFromFile} was added to \sty{texosquery} v1.4, so
%check it's available.
%    \begin{macrocode}
    \ifx\TeXOSQueryFromFile\undefined
      \@locale@err{texosquery too old to support
      \string\LocaleQueryFile. At least v1.4 required}
      {You need to update your version of texosquery}
      \def\@locale@result{}
    \else
      \@locale@info{Fetching query results from `\LocaleQueryFile'}%
      \TeXOSQueryFromFile{\@locale@result}{\LocaleQueryFile}
    \fi
  \fi
\fi
%    \end{macrocode}
%Make the result global in case the pre and post query hooks have
%introduced a local scope.
%    \begin{macrocode}
\global\let\@locale@result\@locale@result
%    \end{macrocode}
%
%\begin{macro}{\localepostquery}
%Allow for a hook immediately after the query.
%    \begin{macrocode}
\csname localepostquery\endcsname
%    \end{macrocode}
%\end{macro}
%
%If the result is empty then the query failed (\sty{texosquery} not
%installed correctly or JRE not installed or shell escape not
%permitted or dry run mode on). The result now needs to be parsed,
%but first define some convenient commands that can be used by the
%parser.
%
%\subsection{Attributes}\label{sec:code.attributes}
%\begin{macro}{\LocaleSetAttribute}
%\begin{definition}
%\cs{LocaleSetAttribute}\marg{label}\marg{attribute
%label}\marg{attribute value}
%\end{definition}
%Provide convenient way of defining attributes. The label depends on
%the attribute type. For example, it could be the dialect label or
%the region code.
%    \begin{macrocode}
\def\LocaleSetAttribute#1#2#3{%
  \expandafter\def\csname locale@#2@#1\endcsname{#3}%
}
%    \end{macrocode}
%\end{macro}
%
%\begin{macro}{\LocaleAppToAttribute}
%\begin{definition}
%\cs{LocaleAppToAttribute}\marg{label}\marg{attribute
%label}\marg{value}
%\end{definition}
%Append \meta{value} to this attribute value.
%    \begin{macrocode}
\def\LocaleAppToAttribute#1#2#3{%
  \LocaleIfHasAttribute{#1}{#2}%
  {%
    \expandafter\expandafter\expandafter\def
      \expandafter\expandafter\csname locale@#2@#1\expandafter\endcsname
      \expandafter\expandafter\expandafter{\csname locale@#2@#1\endcsname#3}%
  }%
  {\expandafter\def\csname locale@#2@#1\endcsname{#3}}%
}
%    \end{macrocode}
%\end{macro}
%
%\begin{macro}{\LocaleXpAppToAttribute}
%\begin{definition}
%\cs{LocaleXpAppToAttribute}\marg{label}\marg{attribute
%label}\marg{value}
%\end{definition}
%As above but expand first token of \meta{value}.
%    \begin{macrocode}
\def\LocaleXpAppToAttribute#1#2#3{%
  \LocaleIfHasAttribute{#1}{#2}%
  {%
    \expandafter\expandafter\expandafter\def
      \expandafter\expandafter\csname locale@#2@#1\expandafter\endcsname
      \expandafter\expandafter\expandafter{%
        \csname locale@#2@#1\expandafter\endcsname#3}%
  }%
  {%
   \expandafter\def\csname locale@#2@#1\expandafter\endcsname\expandafter{#3}%
  }%
}
%    \end{macrocode}
%\end{macro}
%
%\begin{macro}{\LocaleAddToAttributeList}
%\begin{definition}
%\cs{LocaleAddToAttributeList}\marg{label}\marg{attribute
%label}\marg{value}
%\end{definition}
%Adds \meta{value} to this attribute's list (without repetition).
%    \begin{macrocode}
\def\LocaleAddToAttributeList#1#2#3{%
  \LocaleIfHasAttribute{#1}{#2}%
  {%
    \LocaleIfInAttributeList{#1}{#2}{#3}%
    {}%
    {\LocaleAppToAttribute{#1}{#2}{,#3}}%
  }%
  {\LocaleSetAttribute{#1}{#2}{#3}}%
}
%    \end{macrocode}
%\end{macro}
%
%\begin{macro}{\LocaleXpAddToAttributeList}
%\begin{definition}
%\cs{LocaleXpAddToAttributeList}\marg{label}\marg{attribute
%label}\marg{value}
%\end{definition}
%Adds \meta{value} to this attribute's list (without repetition).
%    \begin{macrocode}
\def\LocaleXpAddToAttributeList#1#2#3{%
  \LocaleIfHasAttribute{#1}{#2}%
  {%
    \LocaleIfXpInAttributeList{#1}{#2}{#3}%
    {}%
    {\LocaleXpAppToAttribute{#1}{#2}{\expandafter,#3}}%
  }%
  {\expandafter\def\csname locale@#2@#1\expandafter\endcsname\expandafter{#3}}%
}
%    \end{macrocode}
%\end{macro}
%
%\begin{macro}{\localeshowattribute}
%Debugging command.
%    \begin{macrocode}
\def\localeshowattribute#1#2{%
  \LocaleIfHasAttribute{#1}{#2}%
  {%
    \expandafter\show\csname locale@#2@#1\endcsname
  }%
  {\@locale@err{Attribute `#2' not defined for `#1'}%
  {\string\localeshowattribute\space was asked to show this
   attribute for the given attribute type, but the
   associated command hasn't been defined}}%
}
%    \end{macrocode}
%\end{macro}
%
%\begin{macro}{\LocaleProvideAttribute}
%\begin{definition}
%\cs{LocaleProvideAttribute}\marg{label}\marg{attribute
%label}\marg{attribute value}
%\end{definition}
%Only set the attribute if it hasn't already been set for this label.
%    \begin{macrocode}
\def\LocaleProvideAttribute#1#2#3{%
  \LocaleIfHasAttribute{#1}{#2}%
  {}%
  {\LocaleSetAttribute{#1}{#2}{#3}}%
}
%    \end{macrocode}
%\end{macro}
%
%\begin{macro}{\LocaleLetAttribute}
%\begin{definition}
%\cs{LocaleLetAttribute}\marg{label}\marg{attribute
%label}\marg{cs}
%\end{definition}
%Set the attribute value to the definition of the control sequence \meta{cs}.
%    \begin{macrocode}
\def\LocaleLetAttribute#1#2#3{%
  \expandafter\let\csname locale@#2@#1\endcsname#3%
}
%    \end{macrocode}
%\end{macro}
%
%\begin{macro}{\LocaleGetAttributeOrDefValue}
%\begin{definition}
%\cs{LocaleGetAttributeOrDefValue}\marg{label}\marg{attribute label}\marg{def
%value}
%\end{definition}
%Gets the attribute or the default value if unset.
%    \begin{macrocode}
\def\LocaleGetAttributeOrDefValue#1#2#3{%
  \@tracklang@ifundef{locale@#2@#1}%
  {#3\@locale@undef@action{#1}{#2}}%
  {\csname locale@#2@#1\endcsname}%
}
%    \end{macrocode}
%\end{macro}
%
%\begin{macro}{\LocaleGetAttribute}
%\begin{definition}
%\cs{LocaleGetAttribute}\marg{label}\marg{attribute label}
%\end{definition}
%    \begin{macrocode}
\def\LocaleGetAttribute#1#2{%
 \LocaleGetAttributeOrDefValue{#1}{#2}{}%
}
%    \end{macrocode}
%\end{macro}
%
%\begin{macro}{\@locale@undef@action}
%Action if an undefined attribute is referenced.
%Does nothing by default but may be redefined for debugging
%purposes.
%    \begin{macrocode}
\def\@locale@undef@action#1#2{}
%    \end{macrocode}
%\end{macro}
%
%\begin{macro}{\LocaleIfHasAttribute}
%\begin{definition}
%\cs{LocaleIfHasAttribute}\marg{label}\marg{attribute
%label}\marg{true}\marg{false}
%\end{definition}
%Tests if the given attribute has been assigned.
%    \begin{macrocode}
\def\LocaleIfHasAttribute#1#2#3#4{%
  \@tracklang@ifundef{locale@#2@#1}%
  {#4}%
  {#3}%
}
%    \end{macrocode}
%\end{macro}
%
%\begin{macro}{\LocaleForEachInAttributeList}
%\begin{definition}
%\cs{LocaleForEachInAttributeList}\marg{label}\marg{attribute
%label}\marg{cs}\marg{body}
%\end{definition}
%Where an attribute value is a comma-separated list, this iterates
%over each item in that list, setting \meta{cs} to that item and
%performing \meta{body}.
%    \begin{macrocode}
\def\LocaleForEachInAttributeList#1#2#3#4{%
  \LocaleIfHasAttribute{#1}{#2}%
  {%
    \expandafter\@tracklang@for\expandafter#3\expandafter:\expandafter
      =\csname locale@#2@#1\endcsname\do{#4}%
  }%
  {}%
}
%    \end{macrocode}
%\end{macro}
%
%\begin{macro}{\LocaleIfInAttributeList}
%\begin{definition}
%\cs{LocaleIfInAttributeList}\marg{label}\marg{attribute
%label}\marg{item}\marg{true}\marg{false}
%\end{definition}
%Where an attribute value is a comma-separated list, this tests if
%\meta{item} is in that list.
%    \begin{macrocode}
\def\LocaleIfInAttributeList#1#2#3#4#5{%
  \LocaleIfHasAttribute{#1}{#2}%
  {%
    \expandafter\let\expandafter\@locale@attrlist\csname locale@#2@#1\endcsname
    \@tracklang@ifinlist{#3}{\@locale@attrlist}{#4}{#5}%
  }%
  {#5}%
}
%    \end{macrocode}
%\end{macro}
%
%\begin{macro}{\LocaleIfXpInAttributeList}
%\begin{definition}
%\cs{LocaleIfXpInAttributeList}\marg{label}\marg{attribute
%label}\marg{item}\marg{true}\marg{false}
%\end{definition}
%As above but expands the first token of \meta{item}.
%    \begin{macrocode}
\def\LocaleIfXpInAttributeList#1#2#3#4#5{%
  \LocaleIfHasAttribute{#1}{#2}%
  {%
    \expandafter\let\expandafter\@locale@attrlist\csname locale@#2@#1\endcsname
    \expandafter\@tracklang@ifinlist\expandafter{#3}{\@locale@attrlist}{#4}{#5}%
  }%
  {#5}%
}
%    \end{macrocode}
%\end{macro}
%
%\begin{macro}{\LocaleIfHasNonEmptyAttribute}
%\begin{definition}
%\cs{LocaleIfHasNonEmptyAttribute}\marg{label}\marg{attribute
%label}\marg{true}\marg{false}
%\end{definition}
%Tests if the given attribute has been assigned a non-empty value.
%    \begin{macrocode}
\def\LocaleIfHasNonEmptyAttribute#1#2#3#4{%
  \@tracklang@ifundef{locale@#2@#1}%
  {#4}%
  {%
    \expandafter\ifx\csname locale@#2@#1\endcsname\empty
      #4%
    \else
      #3%
    \fi
  }%
}
%    \end{macrocode}
%\end{macro}
%
%\begin{macro}{\LocaleIfAttributeEqCs}
%\begin{definition}
%\cs{LocaleIfAttributeEqCs}\marg{label}\marg{attribute
%label}\marg{cs}\marg{true}\marg{false}
%\end{definition}
%If the attribute value is the same as the definition of the control
%sequence \meta{cs} to \meta{true} otherwise do false.
%    \begin{macrocode}
\def\LocaleIfAttributeEqCs#1#2#3#4#5{%
  \expandafter\ifx\csname locale@#2@#1\endcsname#3%
    #4%
  \else
    #5%
  \fi
}
%    \end{macrocode}
%\end{macro}
%
%\begin{macro}{\LocaleIfAttributeEqCsName}
%\begin{definition}
%\cs{LocaleIfAttributeEqCsName}\marg{label}\marg{attribute
%label}\marg{cs name}\marg{true}\marg{false}
%\end{definition}
%If the attribute value is the same as the definition of the control
%sequence name \meta{cs name} to \meta{true} otherwise do false.
%    \begin{macrocode}
\def\LocaleIfAttributeEqCsName#1#2#3#4#5{%
  \expandafter\ifx
    \csname locale@#2@#1\expandafter\endcsname
    \csname #3\endcsname
    #4%
  \else
    #5%
  \fi
}
%    \end{macrocode}
%\end{macro}
%
%\begin{macro}{\LocaleIfAttributeEqNum}
%\begin{definition}
%\cs{LocaleIfAttributeEqNum}\marg{label}\marg{attribute
%label}\marg{n}\marg{true}\marg{false}
%\end{definition}
%If the numeric attribute value is equal to \meta{n} do \meta{true} 
%otherwise do \meta{false}.
%    \begin{macrocode}
\def\LocaleIfAttributeEqNum#1#2#3#4#5{%
  \LocaleIfHasNonEmptyAttribute{#1}{#2}%
  {%
    \expandafter\ifnum\csname locale@#2@#1\endcsname=#3
      #4%
    \else
      #5%
    \fi
  }%
  {#5}%
}
%    \end{macrocode}
%\end{macro}
%
%\begin{macro}{\LocaleIfSameAttributeValues}
%\begin{definition}
%\cs{LocaleIfSameAttributeValues}\marg{label}\marg{attribute
%1}\marg{attribute 2}\marg{true}\marg{false}
%\end{definition}
%Tests if two different attributes for the same \meta{label} have
%matching values.
%    \begin{macrocode}
\def\LocaleIfSameAttributeValues#1#2#3#4#5{%
  \expandafter\ifx
    \csname locale@#2@#1\expandafter\endcsname
    \csname locale@#3@#1\endcsname
    #4%
  \else
    #5%
  \fi
}
%    \end{macrocode}
%\end{macro}
%
%\begin{macro}{\localeshowdialectattribute}
%Debugging command.
%    \begin{macrocode}
\def\localeshowdialectattribute#1#2{%
  \localeshowattribute{#1}{dialect@#2}%
}
%    \end{macrocode}
%\end{macro}
%
%\begin{macro}{\LocaleSetDialectAttribute}
%\begin{definition}
%\cs{LocaleSetDialectAttribute}\marg{dialect label}\marg{attribute
%label}\marg{attribute value}
%\end{definition}
%Sets dialect attribute.
%    \begin{macrocode}
\def\LocaleSetDialectAttribute#1#2#3{%
  \LocaleSetAttribute{#1}{dialect@#2}{#3}%
}
%    \end{macrocode}
%\end{macro}
%
%\begin{macro}{\LocaleProvideDialectAttribute}
%\begin{definition}
%\cs{LocaleProvideDialectAttribute}\marg{dialect label}\marg{attribute
%label}\marg{attribute value}
%\end{definition}
%Provides dialect attribute.
%    \begin{macrocode}
\def\LocaleProvideDialectAttribute#1#2#3{%
  \LocaleProvideAttribute{#1}{dialect@#2}{#3}%
}
%    \end{macrocode}
%\end{macro}
%
%\begin{macro}{\LocaleLetDialectAttribute}
%\begin{definition}
%\cs{LocaleLetDialectAttribute}\marg{dialect label}\marg{attribute
%label}\marg{cs}
%\end{definition}
%Set the dialect attribute value to the definition of the control sequence \meta{cs}.
%    \begin{macrocode}
\def\LocaleLetDialectAttribute#1#2#3{%
  \LocaleLetAttribute{#1}{dialect@#2}{#3}%
}
%    \end{macrocode}
%\end{macro}
%
%\begin{macro}{\LocaleAppToDialectAttribute}
%\begin{definition}
%\cs{LocaleAppToDialectAttribute}\marg{dialect label}\marg{attribute
%label}\marg{attribute value}
%\end{definition}
%Append to dialect attribute value.
%    \begin{macrocode}
\def\LocaleAppToDialectAttribute#1#2#3{%
  \LocaleAppToAttribute{#1}{dialect@#2}{#3}%
}
%    \end{macrocode}
%\end{macro}
%
%\begin{macro}{\LocaleXpAppToDialectAttribute}
%\begin{definition}
%\cs{LocaleXpAppToDialectAttribute}\marg{dialect label}\marg{attribute
%label}\marg{value}
%\end{definition}
%Append to dialect attribute value (expand first token of
%\meta{value}).
%    \begin{macrocode}
\def\LocaleXpAppToDialectAttribute#1#2#3{%
  \LocaleXpAppToAttribute{#1}{dialect@#2}{#3}%
}
%    \end{macrocode}
%\end{macro}
%
%\begin{macro}{\LocaleAddToDialectAttributeList}
%\begin{definition}
%\cs{LocaleAddToDialectAttributeList}\marg{label}\marg{attribute
%label}\marg{value}
%\end{definition}
%Adds \meta{value} to this attribute's list (without repetition).
%    \begin{macrocode}
\def\LocaleAddToDialectAttributeList#1#2{%
  \LocaleAddToAttributeList{#1}{dialect@#2}%
}
%    \end{macrocode}
%\end{macro}
%
%\begin{macro}{\LocaleXpAddToDialectAttributeList}
%\begin{definition}
%\cs{LocaleXpAddToDialectAttributeList}\marg{label}\marg{attribute
%label}\marg{value}
%\end{definition}
%Adds \meta{value} to this attribute's list (without repetition).
%    \begin{macrocode}
\def\LocaleXpAddToDialectAttributeList#1#2{%
  \LocaleXpAddToAttributeList{#1}{dialect@#2}%
}
%    \end{macrocode}
%\end{macro}
%
%\begin{macro}{\LocaleGetDialectAttribute}
%\begin{definition}
%\cs{LocaleGetDialectAttribute}\marg{dialect label}\marg{attribute
%label}
%\end{definition}
%Gets dialect attribute.
%    \begin{macrocode}
\def\LocaleGetDialectAttribute#1#2{%
   \LocaleGetAttribute{#1}{dialect@#2}%
}
%    \end{macrocode}
%\end{macro}
%
%\begin{macro}{\LocaleGetDialectAttributeOrDefValue}
%\begin{definition}
%\cs{LocaleGetDialectAttributeOrDefValue}\marg{dialect label}\marg{attribute
%label}\marg{def value}
%\end{definition}
%Gets attribute for given dialect or \meta{def value} if unset.
%    \begin{macrocode}
\def\LocaleGetDialectAttributeOrDefValue#1#2{%
   \LocaleGetAttributeOrDefValue{#1}{dialect@#2}%
}
%    \end{macrocode}
%\end{macro}
%
%\begin{macro}{\LocaleIfHasDialectAttribute}
%\begin{definition}
%\cs{LocaleIfHasDialectAttribute}\marg{dialect label}\marg{attribute
%label}\marg{true}\marg{false}
%\end{definition}
%    \begin{macrocode}
\def\LocaleIfHasDialectAttribute#1#2{%
  \LocaleIfHasAttribute{#1}{dialect@#2}%
}
%    \end{macrocode}
%\end{macro}
%
%\begin{macro}{\LocaleForEachInDialectAttributeList}
%\begin{definition}
%\cs{LocaleForEachInDialectAttributeList}\marg{label}\marg{attribute
%label}\marg{cs}\marg{body}
%\end{definition}
%Where an attribute value is a comma-separated list, this iterates
%over each item in that list, setting \meta{cs} to that item and
%performing \meta{body}.
%    \begin{macrocode}
\def\LocaleForEachInDialectAttributeList#1#2{%
  \LocaleForEachInAttributeList{#1}{dialect@#2}%
}
%    \end{macrocode}
%\end{macro}
%
%\begin{macro}{\LocaleIfInDialectAttributeList}
%\begin{definition}
%\cs{LocaleIfInDialectAttributeList}\marg{label}\marg{attribute
%label}\marg{item}\marg{true}\marg{false}
%\end{definition}
%Where an attribute value is a comma-separated list, this tests if
%\meta{item} is in that list.
%    \begin{macrocode}
\def\LocaleIfInDialectAttributeList#1#2{%
  \LocaleIfInAttributeList{#1}{dialect@#2}%
}
%    \end{macrocode}
%\end{macro}
%
%\begin{macro}{\LocaleIfXpInDialectAttributeList}
%\begin{definition}
%\cs{LocaleIfXpInDialectAttributeList}\marg{label}\marg{attribute
%label}\marg{item}\marg{true}\marg{false}
%\end{definition}
%As above but expands the first token of \meta{item}.
%    \begin{macrocode}
\def\LocaleIfXpInDialectAttributeList#1#2{%
  \LocaleIfXpInAttributeList{#1}{dialect@#2}%
}
%    \end{macrocode}
%\end{macro}
%
%\begin{macro}{\LocaleIfHasDialectNonEmptyAttribute}
%\begin{definition}
%\cs{LocaleIfHasDialectNonEmptyAttribute}\marg{dialect label}\marg{attribute
%label}\marg{true}\marg{false}
%\end{definition}
%    \begin{macrocode}
\def\LocaleIfHasDialectNonEmptyAttribute#1#2{%
  \LocaleIfHasNonEmptyAttribute{#1}{dialect@#2}%
}
%    \end{macrocode}
%\end{macro}
%
%\begin{macro}{\LocaleIfDialectAttributeEqCs}
%\begin{definition}
%\cs{LocaleIfDialectAttributeEqCs}\marg{dialect label}\marg{attribute
%label}\marg{cs}\marg{true}\marg{false}
%\end{definition}
%If the attribute value for the given dialect is the same as the definition of the control
%sequence \meta{cs} do \meta{true} otherwise do \meta{false}.
%    \begin{macrocode}
\def\LocaleIfDialectAttributeEqCs#1#2#3{%
  \LocaleIfAttributeEqCs{#1}{dialect@#2}{#3}%
}
%    \end{macrocode}
%\end{macro}
%
%\begin{macro}{\LocaleIfDialectAttributeEqCsName}
%\begin{definition}
%\cs{LocaleIfDialectAttributeEqCsName}\marg{dialect label}\marg{attribute
%label}\marg{cs name}\marg{true}\marg{false}
%\end{definition}
%If the attribute value for the given dialect is the same as the definition of the control
%sequence name \meta{cs name} do \meta{true} otherwise do \meta{false}.
%    \begin{macrocode}
\def\LocaleIfDialectAttributeEqCsName#1#2#3{%
  \LocaleIfAttributeEqCsName{#1}{dialect@#2}{#3}%
}
%    \end{macrocode}
%\end{macro}
%
%\begin{macro}{\LocaleIfDialectAttributeEqNum}
%\begin{definition}
%\cs{LocaleIfDialectAttributeEqNum}\marg{dialect label}\marg{attribute
%label}\marg{n}\marg{true}\marg{false}
%\end{definition}
%If the numeric attribute value for the given dialect is equal to
%the number \meta{n} do \meta{true} otherwise do \meta{false}.
%    \begin{macrocode}
\def\LocaleIfDialectAttributeEqNum#1#2#3{%
  \LocaleIfAttributeEqNum{#1}{dialect@#2}{#3}%
}
%    \end{macrocode}
%\end{macro}
%
%\begin{macro}{\LocaleIfSameDialectAttributeValues}
%\begin{definition}
%\cs{LocaleIfSameDialectAttributeValues}\marg{dialect label}\marg{attribute
%1}\marg{attribute 2}\marg{true}\marg{false}
%\end{definition}
%Tests if two different dialect attributes for the same
%\meta{dialect label} have
%matching values.
%    \begin{macrocode}
\def\LocaleIfSameDialectAttributeValues#1#2#3{%
  \LocaleIfSameAttributeValues{#1}{dialect@#2}{dialect@#3}%
}
%    \end{macrocode}
%\end{macro}
%
%\begin{macro}{\localeshowregionattribute}
%Debugging command.
%    \begin{macrocode}
\def\localeshowregionattribute#1#2{%
  \localeshowattribute{#1}{region@#2}%
}
%    \end{macrocode}
%\end{macro}
%
%\begin{macro}{\LocaleSetRegionAttribute}
%\begin{definition}
%\cs{LocaleSetRegionAttribute}\marg{region code}\marg{attribute
%label}\marg{attribute value}
%\end{definition}
%Sets region attribute.
%    \begin{macrocode}
\def\LocaleSetRegionAttribute#1#2#3{%
  \LocaleSetAttribute{#1}{region@#2}{#3}%
}
%    \end{macrocode}
%\end{macro}
%
%\begin{macro}{\LocaleProvideRegionAttribute}
%\begin{definition}
%\cs{LocaleProvideRegionAttribute}\marg{region code}\marg{attribute
%label}\marg{attribute value}
%\end{definition}
%Provides region attribute.
%    \begin{macrocode}
\def\LocaleProvideRegionAttribute#1#2#3{%
  \LocaleProvideAttribute{#1}{region@#2}{#3}%
}
%    \end{macrocode}
%\end{macro}
%
%\begin{macro}{\LocaleLetRegionAttribute}
%\begin{definition}
%\cs{LocaleLetRegionAttribute}\marg{region code}\marg{attribute
%label}\marg{cs}
%\end{definition}
%Set the region attribute value to the definition of the control sequence \meta{cs}.
%    \begin{macrocode}
\def\LocaleLetRegionAttribute#1#2#3{%
  \LocaleLetAttribute{#1}{region@#2}{#3}%
}
%    \end{macrocode}
%\end{macro}
%
%\begin{macro}{\LocaleAppToRegionAttribute}
%\begin{definition}
%\cs{LocaleAppToRegionAttribute}\marg{region code}\marg{attribute
%label}\marg{value}
%\end{definition}
%Append to region attribute.
%    \begin{macrocode}
\def\LocaleAppToRegionAttribute#1#2#3{%
  \LocaleAppToAttribute{#1}{region@#2}{#3}%
}
%    \end{macrocode}
%\end{macro}
%
%\begin{macro}{\LocaleXpAppToRegionAttribute}
%\begin{definition}
%\cs{LocaleXpAppToRegionAttribute}\marg{region code}\marg{attribute
%label}\marg{value}
%\end{definition}
%Append to region attribute (expand first token of
%\meta{value}).
%    \begin{macrocode}
\def\LocaleXpAppToRegionAttribute#1#2#3{%
  \LocaleXpAppToAttribute{#1}{region@#2}{#3}%
}
%    \end{macrocode}
%\end{macro}
%
%\begin{macro}{\LocaleAddToRegionAttributeList}
%\begin{definition}
%\cs{LocaleAddToRegionAttributeList}\marg{label}\marg{attribute
%label}\marg{value}
%\end{definition}
%Adds \meta{value} to this attribute's list (without repetition).
%    \begin{macrocode}
\def\LocaleAddToRegionAttributeList#1#2{%
  \LocaleAddToAttributeList{#1}{region@#2}%
}
%    \end{macrocode}
%\end{macro}
%
%\begin{macro}{\LocaleXpAddToRegionAttributeList}
%\begin{definition}
%\cs{LocaleXpAddToRegionAttributeList}\marg{label}\marg{attribute
%label}\marg{value}
%\end{definition}
%Adds \meta{value} to this attribute's list (without repetition).
%    \begin{macrocode}
\def\LocaleXpAddToRegionAttributeList#1#2{%
  \LocaleXpAddToAttributeList{#1}{region@#2}%
}
%    \end{macrocode}
%\end{macro}
%
%\begin{macro}{\LocaleGetRegionAttribute}
%\begin{definition}
%\cs{LocaleGetRegionAttribute}\marg{region code}\marg{attribute
%label}
%\end{definition}
%Gets region attribute.
%    \begin{macrocode}
\def\LocaleGetRegionAttribute#1#2{%
   \LocaleGetAttribute{#1}{region@#2}%
}
%    \end{macrocode}
%\end{macro}
%
%\begin{macro}{\LocaleGetRegionAttributeOrDefValue}
%\begin{definition}
%\cs{LocaleGetRegionAttributeOrDefValue}\marg{region code}\marg{attribute
%label}\marg{def value}
%\end{definition}
%Gets attribute for given region or \meta{def value} if unset.
%    \begin{macrocode}
\def\LocaleGetRegionAttributeOrDefValue#1#2{%
   \LocaleGetAttributeOrDefValue{#1}{region@#2}%
}
%    \end{macrocode}
%\end{macro}
%
%\begin{macro}{\LocaleIfHasRegionAttribute}
%\begin{definition}
%\cs{LocaleIfHasRegionAttribute}\marg{region code}\marg{attribute
%label}\marg{true}\marg{false}
%\end{definition}
%    \begin{macrocode}
\def\LocaleIfHasRegionAttribute#1#2{%
  \LocaleIfHasAttribute{#1}{region@#2}%
}
%    \end{macrocode}
%\end{macro}
%
%\begin{macro}{\LocaleForEachInRegionAttributeList}
%\begin{definition}
%\cs{LocaleForEachInRegionAttributeList}\marg{label}\marg{attribute
%label}\marg{cs}\marg{body}
%\end{definition}
%Where an attribute value is a comma-separated list, this iterates
%over each item in that list, setting \meta{cs} to that item and
%performing \meta{body}.
%    \begin{macrocode}
\def\LocaleForEachInRegionAttributeList#1#2{%
  \LocaleForEachInAttributeList{#1}{region@#2}%
}
%    \end{macrocode}
%\end{macro}
%
%\begin{macro}{\LocaleIfInRegionAttributeList}
%\begin{definition}
%\cs{LocaleIfInRegionAttributeList}\marg{label}\marg{attribute
%label}\marg{item}\marg{true}\marg{false}
%\end{definition}
%Where an attribute value is a comma-separated list, this tests if
%\meta{item} is in that list.
%    \begin{macrocode}
\def\LocaleIfInRegionAttributeList#1#2{%
  \LocaleIfInAttributeList{#1}{region@#2}%
}
%    \end{macrocode}
%\end{macro}
%
%\begin{macro}{\LocaleIfXpInRegionAttributeList}
%\begin{definition}
%\cs{LocaleIfXpInRegionAttributeList}\marg{label}\marg{attribute
%label}\marg{item}\marg{true}\marg{false}
%\end{definition}
%As above but expands the first token of \meta{item}.
%    \begin{macrocode}
\def\LocaleIfXpInRegionAttributeList#1#2{%
  \LocaleIfXpInAttributeList{#1}{region@#2}%
}
%    \end{macrocode}
%\end{macro}
%
%\begin{macro}{\LocaleIfHasRegionNonEmptyAttribute}
%\begin{definition}
%\cs{LocaleIfHasRegionNonEmptyAttribute}\marg{region code}\marg{attribute
%label}\marg{true}\marg{false}
%\end{definition}
%    \begin{macrocode}
\def\LocaleIfHasRegionNonEmptyAttribute#1#2{%
  \LocaleIfHasNonEmptyAttribute{#1}{region@#2}%
}
%    \end{macrocode}
%\end{macro}
%
%\begin{macro}{\LocaleIfRegionAttributeEqCs}
%\begin{definition}
%\cs{LocaleIfRegionAttributeEqCs}\marg{region code}\marg{attribute
%label}\marg{cs}\marg{true}\marg{false}
%\end{definition}
%If the attribute value for the given region is the same as the definition of the control
%sequence \meta{cs} do \meta{true} otherwise do \meta{false}.
%    \begin{macrocode}
\def\LocaleIfRegionAttributeEqCs#1#2#3{%
  \LocaleIfAttributeEqCs{#1}{region@#2}{#3}%
}
%    \end{macrocode}
%\end{macro}
%
%\begin{macro}{\LocaleIfRegionAttributeEqCsName}
%\begin{definition}
%\cs{LocaleIfRegionAttributeEqCsName}\marg{region code}\marg{attribute
%label}\marg{cs name}\marg{true}\marg{false}
%\end{definition}
%If the attribute value for the given region is the same as the definition of the control
%sequence name \meta{cs name} do \meta{true} otherwise do \meta{false}.
%    \begin{macrocode}
\def\LocaleIfRegionAttributeEqCsName#1#2#3{%
  \LocaleIfAttributeEqCsName{#1}{region@#2}{#3}%
}
%    \end{macrocode}
%\end{macro}
%
%\begin{macro}{\LocaleIfRegionAttributeEqNum}
%\begin{definition}
%\cs{LocaleIfRegionAttributeEqNum}\marg{region label}\marg{attribute
%label}\marg{n}\marg{true}\marg{false}
%\end{definition}
%If the numeric attribute value for the given region is equal to
%the number \meta{n} do \meta{true} otherwise do \meta{false}.
%    \begin{macrocode}
\def\LocaleIfRegionAttributeEqNum#1#2#3{%
  \LocaleIfAttributeEqNum{#1}{region@#2}{#3}%
}
%    \end{macrocode}
%\end{macro}
%
%\begin{macro}{\LocaleIfSameRegionAttributeValues}
%\begin{definition}
%\cs{LocaleIfSameRegionAttributeValues}\marg{region code}\marg{attribute
%1}\marg{attribute 2}\marg{true}\marg{false}
%\end{definition}
%Tests if two different region attributes for the same
%\meta{region code} have matching values.
%    \begin{macrocode}
\def\LocaleIfSameRegionAttributeValues#1#2#3{%
  \LocaleIfSameAttributeValues{#1}{region@#2}{region@#3}%
}
%    \end{macrocode}
%\end{macro}
%
%\begin{macro}{\localeshowcurrencyattribute}
%Debugging command.
%    \begin{macrocode}
\def\localeshowcurrencyattribute#1#2{%
  \localeshowattribute{#1}{currency@#2}%
}
%    \end{macrocode}
%\end{macro}
%
%\begin{macro}{\LocaleSetCurrencyAttribute}
%\begin{definition}
%\cs{LocaleSetCurrencyAttribute}\marg{currency code}\marg{attribute
%label}\marg{attribute value}
%\end{definition}
%Sets currency attribute.
%    \begin{macrocode}
\def\LocaleSetCurrencyAttribute#1#2#3{%
  \LocaleSetAttribute{#1}{currency@#2}{#3}%
}
%    \end{macrocode}
%\end{macro}
%
%\begin{macro}{\LocaleProvideCurrencyAttribute}
%\begin{definition}
%\cs{LocaleProvideCurrencyAttribute}\marg{currency code}\marg{attribute
%label}\marg{attribute value}
%\end{definition}
%Provides currency attribute.
%    \begin{macrocode}
\def\LocaleProvideCurrencyAttribute#1#2#3{%
  \LocaleProvideAttribute{#1}{currency@#2}{#3}%
}
%    \end{macrocode}
%\end{macro}
%
%\begin{macro}{\LocaleLetCurrencyAttribute}
%\begin{definition}
%\cs{LocaleLetCurrencyAttribute}\marg{currency code}\marg{attribute
%label}\marg{cs}
%\end{definition}
%Set the currency attribute value to the definition of the control sequence \meta{cs}.
%    \begin{macrocode}
\def\LocaleLetCurrencyAttribute#1#2#3{%
  \LocaleLetAttribute{#1}{currency@#2}{#3}%
}
%    \end{macrocode}
%\end{macro}
%
%\begin{macro}{\LocaleAppToCurrencyAttribute}
%\begin{definition}
%\cs{LocaleAppToCurrencyAttribute}\marg{currency code}\marg{attribute
%label}\marg{value}
%\end{definition}
%Append to currency attribute value.
%    \begin{macrocode}
\def\LocaleAppToCurrencyAttribute#1#2#3{%
  \LocaleAppToAttribute{#1}{currency@#2}{#3}%
}
%    \end{macrocode}
%\end{macro}
%
%\begin{macro}{\LocaleXpAppToCurrencyAttribute}
%\begin{definition}
%\cs{LocaleXpAppToCurrencyAttribute}\marg{currency code}\marg{attribute
%label}\marg{value}
%\end{definition}
%Append to currency attribute value (expand first token of
%\meta{value}).
%    \begin{macrocode}
\def\LocaleXpAppToCurrencyAttribute#1#2#3{%
  \LocaleXpAppToAttribute{#1}{currency@#2}{#3}%
}
%    \end{macrocode}
%\end{macro}
%
%\begin{macro}{\LocaleAddToCurrencyAttributeList}
%\begin{definition}
%\cs{LocaleAddToCurrencyAttributeList}\marg{label}\marg{attribute
%label}\marg{value}
%\end{definition}
%Adds \meta{value} to this attribute's list (without repetition).
%    \begin{macrocode}
\def\LocaleAddToCurrencyAttributeList#1#2{%
  \LocaleAddToAttributeList{#1}{currency@#2}%
}
%    \end{macrocode}
%\end{macro}
%
%\begin{macro}{\LocaleXpAddToCurrencyAttributeList}
%\begin{definition}
%\cs{LocaleXpAddToCurrencyAttributeList}\marg{label}\marg{attribute
%label}\marg{value}
%\end{definition}
%Adds \meta{value} to this attribute's list (without repetition).
%    \begin{macrocode}
\def\LocaleXpAddToCurrencyAttributeList#1#2{%
  \LocaleXpAddToAttributeList{#1}{currency@#2}%
}
%    \end{macrocode}
%\end{macro}
%
%\begin{macro}{\LocaleGetCurrencyAttribute}
%\begin{definition}
%\cs{LocaleGetCurrencyAttribute}\marg{currency code}\marg{attribute
%label}
%\end{definition}
%Gets currency attribute.
%    \begin{macrocode}
\def\LocaleGetCurrencyAttribute#1#2{%
   \LocaleGetAttribute{#1}{currency@#2}%
}
%    \end{macrocode}
%\end{macro}
%
%\begin{macro}{\LocaleGetCurrencyAttributeOrDefValue}
%\begin{definition}
%\cs{LocaleGetCurrencyAttributeOrDefValue}\marg{currency code}\marg{attribute
%label}\marg{def value}
%\end{definition}
%Gets attribute for given currency or \meta{def value} if unset.
%    \begin{macrocode}
\def\LocaleGetCurrencyAttributeOrDefValue#1#2{%
   \LocaleGetAttributeOrDefValue{#1}{currency@#2}%
}
%    \end{macrocode}
%\end{macro}
%
%\begin{macro}{\LocaleIfHasCurrencyAttribute}
%\begin{definition}
%\cs{LocaleIfHasCurrencyAttribute}\marg{currency code}\marg{attribute
%label}\marg{true}\marg{false}
%\end{definition}
%    \begin{macrocode}
\def\LocaleIfHasCurrencyAttribute#1#2{%
  \LocaleIfHasAttribute{#1}{currency@#2}%
}
%    \end{macrocode}
%\end{macro}
%
%\begin{macro}{\LocaleForEachInCurrencyAttributeList}
%\begin{definition}
%\cs{LocaleForEachInCurrencyAttributeList}\marg{label}\marg{attribute
%label}\marg{cs}\marg{body}
%\end{definition}
%Where an attribute value is a comma-separated list, this iterates
%over each item in that list, setting \meta{cs} to that item and
%performing \meta{body}.
%    \begin{macrocode}
\def\LocaleForEachInCurrencyAttributeList#1#2{%
  \LocaleForEachInAttributeList{#1}{currency@#2}%
}
%    \end{macrocode}
%\end{macro}
%
%\begin{macro}{\LocaleIfInCurrencyAttributeList}
%\begin{definition}
%\cs{LocaleIfInCurrencyAttributeList}\marg{label}\marg{attribute
%label}\marg{item}\marg{true}\marg{false}
%\end{definition}
%Where an attribute value is a comma-separated list, this tests if
%\meta{item} is in that list.
%    \begin{macrocode}
\def\LocaleIfInCurrencyAttributeList#1#2{%
  \LocaleIfInAttributeList{#1}{currency@#2}%
}
%    \end{macrocode}
%\end{macro}
%
%\begin{macro}{\LocaleIfXpInCurrencyAttributeList}
%\begin{definition}
%\cs{LocaleIfXpInCurrencyAttributeList}\marg{label}\marg{attribute
%label}\marg{item}\marg{true}\marg{false}
%\end{definition}
%As above but expands the first token of \meta{item}.
%    \begin{macrocode}
\def\LocaleIfXpInCurrencyAttributeList#1#2{%
  \LocaleIfXpInAttributeList{#1}{currency@#2}%
}
%    \end{macrocode}
%\end{macro}
%
%\begin{macro}{\LocaleIfHasCurrencyNonEmptyAttribute}
%\begin{definition}
%\cs{LocaleIfHasCurrencyNonEmptyAttribute}\marg{currency code}\marg{attribute
%label}\marg{true}\marg{false}
%\end{definition}
%    \begin{macrocode}
\def\LocaleIfHasCurrencyNonEmptyAttribute#1#2{%
  \LocaleIfHasNonEmptyAttribute{#1}{currency@#2}%
}
%    \end{macrocode}
%\end{macro}
%
%\begin{macro}{\LocaleIfCurrencyAttributeEqCs}
%\begin{definition}
%\cs{LocaleIfCurrencyAttributeEqCs}\marg{currency code}\marg{attribute
%label}\marg{cs}\marg{true}\marg{false}
%\end{definition}
%If the attribute value for the given currency is the same as the definition of the control
%sequence \meta{cs} do \meta{true} otherwise do \meta{false}.
%    \begin{macrocode}
\def\LocaleIfCurrencyAttributeEqCs#1#2#3{%
  \LocaleIfAttributeEqCs{#1}{currency@#2}{#3}%
}
%    \end{macrocode}
%\end{macro}
%
%\begin{macro}{\LocaleIfCurrencyAttributeEqCsName}
%\begin{definition}
%\cs{LocaleIfCurrencyAttributeEqCsName}\marg{currency code}\marg{attribute
%label}\marg{cs name}\marg{true}\marg{false}
%\end{definition}
%If the attribute value for the given currency is the same as the definition of the control
%sequence name \meta{cs name} do \meta{true} otherwise do
%\meta{false}.
%    \begin{macrocode}
\def\LocaleIfCurrencyAttributeEqCsName#1#2#3{%
  \LocaleIfAttributeEqCsName{#1}{currency@#2}{#3}%
}
%    \end{macrocode}
%\end{macro}
%
%\begin{macro}{\LocaleIfCurrencyAttributeEqNum}
%\begin{definition}
%\cs{LocaleIfCurrencyAttributeEqNum}\marg{currency label}\marg{attribute
%label}\marg{n}\marg{true}\marg{false}
%\end{definition}
%If the numeric attribute value for the given currency is equal to
%the number \meta{n} do \meta{true} otherwise do \meta{false}.
%    \begin{macrocode}
\def\LocaleIfCurrencyAttributeEqNum#1#2#3{%
  \LocaleIfAttributeEqNum{#1}{currency@#2}{#3}%
}
%    \end{macrocode}
%\end{macro}
%
%\begin{macro}{\LocaleIfSameCurrencyAttributeValues}
%\begin{definition}
%\cs{LocaleIfSameCurrencyAttributeValues}\marg{currency code}\marg{attribute
%1}\marg{attribute 2}\marg{true}\marg{false}
%\end{definition}
%Tests if two different currency attributes for the same
%\meta{currency code} have matching values.
%    \begin{macrocode}
\def\LocaleIfSameCurrencyAttributeValues#1#2#3{%
  \LocaleIfSameAttributeValues{#1}{currency@#2}{currency@#3}%
}
%    \end{macrocode}
%\end{macro}
%
%\subsection{Parser Commands}
%
%Define commands that are needed to parse the result.
%\begin{macro}{\@locale@parse@result}
%Start parsing the result. There are more than nine arguments, so do
%this in bits. The first six arguments are: OS name, OS version, OS
%architecture, PDF date-time, BCP 47 tag for default locale,
%the default file encoding.
%    \begin{macrocode}
\def\@locale@parse@result#1#2#3#4#5#6{%
  \def\LocaleOSname{#1}%
  \def\LocaleOSversion{#2}%
  \def\LocaleOSarch{#3}%
  \def\LocaleNowStamp{#4}%
  \@locale@parse@default#5% remove outer group
  \def\LocaleOScodeset{#6}%
%    \end{macrocode}
% Is date-time pattern support required?
%    \begin{macrocode}
 \LocaleIfDateTimePatternsSupported
 {%
    \let\@locale@next\@locale@parse@datetimeinfo
 }%
 {%
    \def\LocaleDateTimeInfo{}%
%    \end{macrocode}
% Was a file modification date included?
%    \begin{macrocode}
    \ifx\LocaleMainFile\empty
      \def\LocaleFileMod{}%
      \let\@locale@next\@locale@parse@maindata
    \else
      \let\@locale@next\@locale@parse@filemod
    \fi
  }%
  \@locale@next
}
%    \end{macrocode}
%\end{macro}
%
%\begin{macro}{\@locale@parse@datetimeinfo}
%    \begin{macrocode}
\def\@locale@parse@datetimeinfo#1{%
  \def\LocaleDateTimeInfo{#1}%
%    \end{macrocode}
% Was a file modification date included?
%    \begin{macrocode}
  \ifx\LocaleMainFile\empty
    \def\LocaleFileMod{}%
    \let\@locale@next\@locale@parse@maindata
  \else
    \let\@locale@next\@locale@parse@filemod
  \fi
  \@locale@next
}
%    \end{macrocode}
%\end{macro}
%
%\begin{macro}{\@locale@parse@default}
%Parse the result of \texttt{-N}
%    \begin{macrocode}
\def\@locale@parse@default#1#2#3#4#5#6#7#8#9{%
  \def\LocaleOStag{#1}%
%    \end{macrocode}
%Parse (but don't track) tag.
%    \begin{macrocode}
  \@tracklang@parselangtag{#1}%
%    \end{macrocode}
%Can now get the region code.
%    \begin{macrocode}
  \let\@locale@os@region\@TrackLangEnvTerritory
  \def\@locale@os@groupsep{#2}%
  \def\@locale@os@decsep{#3}%
  \def\@locale@os@exp{#4}%
  \def\@locale@os@usesgroup{#5}%
  \def\@locale@os@currencycode{#6}%
  \def\@locale@os@regionalcurrencycode{#7}%
  \def\@locale@os@currencysym{#8}%
  \def\@locale@os@currencytex{#9}%
%    \end{macrocode}
%Provide currency attributes.
%    \begin{macrocode}
  \LocaleSetRegionAttribute{\@locale@os@region}{currency}{#7}%
  \LocaleProvideCurrencyAttribute{#7}{official}{#6}%
  \LocaleProvideCurrencyAttribute{#7}{sym}{#8}%
  \LocaleProvideCurrencyAttribute{#7}{tex}{#9}%
  \@locale@parse@default@cursep
}
%    \end{macrocode}
%\end{macro}
%\begin{macro}{\@locale@parse@default@cursep}
%    \begin{macrocode}
\def\@locale@parse@default@cursep#1{%
  \def\@locale@os@cursep{#1}%
}
%    \end{macrocode}
%\end{macro}
%
%\begin{macro}{\@locale@parse@filemod}
%    \begin{macrocode}
\def\@locale@parse@filemod#1{%
  \def\LocaleFileMod{#1}%
  \@locale@parse@maindata
}
%    \end{macrocode}
%\end{macro}
%
%\begin{macro}{\@locale@parse@maindata}
%    \begin{macrocode}
\def\@locale@parse@maindata#1{%
  \@locale@parse@maindatablock#1% remove outer group
}
%    \end{macrocode}
%\end{macro}
%
%\begin{macro}{\@locale@parse@maindatablock}
%    \begin{macrocode}
\def\@locale@parse@maindatablock#1{%
  \@locale@parse@maindatalocaleblock#1% remove outer group
  \@locale@parse@dateblock
}
%    \end{macrocode}
%\end{macro}
%
%\begin{macro}{\@locale@parse@maindatalocaleblock}
%    \begin{macrocode}
\def\@locale@parse@maindatalocaleblock#1#2#3#4#5#6#7{%
  \def\LocaleMain{#1}%
  \TrackLanguageTag{#1}%
  \let\LocaleMainDialect\TrackLangLastTrackedDialect
  \let\@locale@dialect\TrackLangLastTrackedDialect
%    \end{macrocode}
%Get the region code if provided.
%    \begin{macrocode}
  \IfTrackedLanguageHasIsoCode{3166-1}{\@locale@dialect}%
  {%
    \edef\@locale@region{%
     \TrackedIsoCodeFromLanguage{3166-1}{\@locale@dialect}}%
  }%
  {\def\@locale@region{}}%
  \let\LocaleMainRegion\@locale@region
  \ifx\LocaleMainRegion\empty
  \else
    \LocaleLetRegionAttribute{\LocaleMainRegion}{dialect}{\@locale@dialect}%
  \fi
%    \end{macrocode}
%Save language tag. (Provides a convenient mapping from dialect
%label to tag.)
%    \begin{macrocode}
  \LocaleSetDialectAttribute{\@locale@dialect}{langtag}{#1}%
%    \end{macrocode}
%Provide reverse mapping from tag to dialect label.
%    \begin{macrocode}
  \LocaleLetAttribute{#1}{tagtodialect}{\@locale@dialect}%
%    \end{macrocode}
%Save display names.
%    \begin{macrocode}
  \LocaleSetDialectAttribute{\@locale@dialect}{langname}{#2}%
  \LocaleSetDialectAttribute{\@locale@dialect}{nativelangname}{#3}%
  \LocaleSetDialectAttribute{\@locale@dialect}{regionname}{#4}%
  \LocaleSetDialectAttribute{\@locale@dialect}{nativeregionname}{#5}%
  \LocaleSetDialectAttribute{\@locale@dialect}{variantname}{#6}%
  \LocaleSetDialectAttribute{\@locale@dialect}{nativevariantname}{#7}%
}
%    \end{macrocode}
%\end{macro}
%
%\begin{macro}{\@locale@parse@otherdatalocaleblock}
%    \begin{macrocode}
\def\@locale@parse@otherdatalocaleblock#1#2#3#4#5#6#7{%
  \TrackLanguageTag{#1}%
  \let\@locale@dialect\TrackLangLastTrackedDialect
%    \end{macrocode}
%Get the region code if provided.
%    \begin{macrocode}
  \IfTrackedLanguageHasIsoCode{3166-1}{\@locale@dialect}%
  {%
    \edef\@locale@region{%
     \TrackedIsoCodeFromLanguage{3166-1}{\@locale@dialect}}%
     \ifx\@locale@region\empty
     \else
       \LocaleXpAddToRegionAttributeList
         {\@locale@region}{dialect}{\@locale@dialect}%
     \fi
  }%
  {\def\@locale@region{}}%
%    \end{macrocode}
%Save language tag.
%    \begin{macrocode}
  \LocaleSetDialectAttribute{\@locale@dialect}{langtag}{#1}%
%    \end{macrocode}
%Provide reverse mapping from tag to dialect label.
%    \begin{macrocode}
  \LocaleLetAttribute{#1}{tagtodialect}{\@locale@dialect}%
%    \end{macrocode}
%Save attributes.
%    \begin{macrocode}
  \LocaleSetDialectAttribute{\@locale@dialect}{langname}{#2}%
  \LocaleSetDialectAttribute{\@locale@dialect}{nativelangname}{#3}%
  \LocaleSetDialectAttribute{\@locale@dialect}{regionname}{#4}%
  \LocaleSetDialectAttribute{\@locale@dialect}{nativeregionname}{#5}%
  \LocaleSetDialectAttribute{\@locale@dialect}{variantname}{#6}%
  \LocaleSetDialectAttribute{\@locale@dialect}{nativevariantname}{#7}%
}
%    \end{macrocode}
%\end{macro}
%
%\begin{macro}{\@locale@parse@dateblock}
%    \begin{macrocode}
\def\@locale@parse@dateblock#1{%
  \@locale@parse@dates#1% remove outer group
  \@locale@parse@datefmtblock
}
%    \end{macrocode}
%\end{macro}
%
%\begin{macro}{\@locale@parse@dates}
%    \begin{macrocode}
\def\@locale@parse@dates#1#2#3#4#5{%
  \LocaleSetDialectAttribute{\@locale@dialect}{fulldate}{#1}%
  \LocaleSetDialectAttribute{\@locale@dialect}{longdate}{#2}%
  \LocaleSetDialectAttribute{\@locale@dialect}{meddate}{#3}%
  \LocaleSetDialectAttribute{\@locale@dialect}{shortdate}{#4}%
  \LocaleSetDialectAttribute{\@locale@dialect}{firstday}{#5}%
}
%    \end{macrocode}
%\end{macro}
%
%\begin{macro}{\@locale@parse@datefmtblock}
%    \begin{macrocode}
\def\@locale@parse@datefmtblock#1{%
  \@locale@parse@datefmts#1% remove outer group
  \@locale@parse@timeblock
}
%    \end{macrocode}
%\end{macro}
%
%\begin{macro}{\@locale@parse@datefmts}
%    \begin{macrocode}
\def\@locale@parse@datefmts#1#2#3#4{%
  \LocaleSetDialectAttribute{\@locale@dialect}{fulldatefmt}{#1}%
  \LocaleSetDialectAttribute{\@locale@dialect}{longdatefmt}{#2}%
  \LocaleSetDialectAttribute{\@locale@dialect}{meddatefmt}{#3}%
  \LocaleSetDialectAttribute{\@locale@dialect}{shortdatefmt}{#4}%
}
%    \end{macrocode}
%\end{macro}
%
%\begin{macro}{\@locale@parse@timeblock}
%    \begin{macrocode}
\def\@locale@parse@timeblock#1{%
  \@locale@parse@times#1% remove outer group
  \@locale@parse@timefmtblock
}
%    \end{macrocode}
%\end{macro}
%
%\begin{macro}{\@locale@parse@times}
%    \begin{macrocode}
\def\@locale@parse@times#1#2#3#4{%
  \LocaleSetDialectAttribute{\@locale@dialect}{fulltime}{#1}%
  \LocaleSetDialectAttribute{\@locale@dialect}{longtime}{#2}%
  \LocaleSetDialectAttribute{\@locale@dialect}{medtime}{#3}%
  \LocaleSetDialectAttribute{\@locale@dialect}{shorttime}{#4}%
}
%    \end{macrocode}
%\end{macro}
%
%\begin{macro}{\@locale@parse@timefmtblock}
%    \begin{macrocode}
\def\@locale@parse@timefmtblock#1{%
  \@locale@parse@timefmts#1% remove outer group
  \@locale@parse@datetimeblock
}
%    \end{macrocode}
%\end{macro}
%
%\begin{macro}{\@locale@parse@timefmts}
%    \begin{macrocode}
\def\@locale@parse@timefmts#1#2#3#4{%
  \LocaleSetDialectAttribute{\@locale@dialect}{fulltimefmt}{#1}%
  \LocaleSetDialectAttribute{\@locale@dialect}{longtimefmt}{#2}%
  \LocaleSetDialectAttribute{\@locale@dialect}{medtimefmt}{#3}%
  \LocaleSetDialectAttribute{\@locale@dialect}{shorttimefmt}{#3}%
}
%    \end{macrocode}
%\end{macro}
%
%\begin{macro}{\@locale@parse@datetimeblock}
%    \begin{macrocode}
\def\@locale@parse@datetimeblock#1{%
  \@locale@parse@datetimes#1% remove outer group
  \@locale@parse@datetimefmtblock
}
%    \end{macrocode}
%\end{macro}
%
%\begin{macro}{\@locale@parse@datetimes}
%    \begin{macrocode}
\def\@locale@parse@datetimes#1#2#3#4{%
  \LocaleSetDialectAttribute{\@locale@dialect}{fulldatetime}{#1}%
  \LocaleSetDialectAttribute{\@locale@dialect}{longdatetime}{#2}%
  \LocaleSetDialectAttribute{\@locale@dialect}{meddatetime}{#3}%
  \LocaleSetDialectAttribute{\@locale@dialect}{shortdatetime}{#4}%
}
%    \end{macrocode}
%\end{macro}
%
%\begin{macro}{\@locale@parse@datetimefmtblock}
%    \begin{macrocode}
\def\@locale@parse@datetimefmtblock#1{%
  \@locale@parse@datetimefmts#1% remove outer group
  \@locale@parse@weekdayblock
}
%    \end{macrocode}
%\end{macro}
%
%\begin{macro}{\@locale@parse@datetimefmts}
%    \begin{macrocode}
\def\@locale@parse@datetimefmts#1#2#3#4{%
  \LocaleSetDialectAttribute{\@locale@dialect}{fulldatetimefmt}{#1}%
  \LocaleSetDialectAttribute{\@locale@dialect}{longdatetimefmt}{#2}%
  \LocaleSetDialectAttribute{\@locale@dialect}{meddatetimefmt}{#3}%
  \LocaleSetDialectAttribute{\@locale@dialect}{shortdatetimefmt}{#4}%
}
%    \end{macrocode}
%\end{macro}
%
%\begin{macro}{\@locale@parse@weekdayblock}
%    \begin{macrocode}
\def\@locale@parse@weekdayblock#1{%
  \@locale@parse@weekdays#1% remove outer group
  \@locale@parse@shortweekdayblock
}
%    \end{macrocode}
%\end{macro}
%
%\begin{macro}{\@locale@parse@weekdays}
%    \begin{macrocode}
\def\@locale@parse@weekdays#1#2#3#4#5#6#7{%
  \LocaleSetDialectAttribute{\@locale@dialect}{day.0}{#1}%
  \LocaleSetDialectAttribute{\@locale@dialect}{day.1}{#2}%
  \LocaleSetDialectAttribute{\@locale@dialect}{day.2}{#3}%
  \LocaleSetDialectAttribute{\@locale@dialect}{day.3}{#4}%
  \LocaleSetDialectAttribute{\@locale@dialect}{day.4}{#5}%
  \LocaleSetDialectAttribute{\@locale@dialect}{day.5}{#6}%
  \LocaleSetDialectAttribute{\@locale@dialect}{day.6}{#7}%
}
%    \end{macrocode}
%\end{macro}
%
%\begin{macro}{\@locale@parse@shortweekdayblock}
%    \begin{macrocode}
\def\@locale@parse@shortweekdayblock#1{%
  \@locale@parse@shortweekdays#1% remove outer group
  \@locale@parse@monthblock
}
%    \end{macrocode}
%\end{macro}
%
%\begin{macro}{\@locale@parse@shortweekdays}
%    \begin{macrocode}
\def\@locale@parse@shortweekdays#1#2#3#4#5#6#7{%
  \LocaleSetDialectAttribute{\@locale@dialect}{shortday.0}{#1}%
  \LocaleSetDialectAttribute{\@locale@dialect}{shortday.1}{#2}%
  \LocaleSetDialectAttribute{\@locale@dialect}{shortday.2}{#3}%
  \LocaleSetDialectAttribute{\@locale@dialect}{shortday.3}{#4}%
  \LocaleSetDialectAttribute{\@locale@dialect}{shortday.4}{#5}%
  \LocaleSetDialectAttribute{\@locale@dialect}{shortday.5}{#6}%
  \LocaleSetDialectAttribute{\@locale@dialect}{shortday.6}{#7}%
}
%    \end{macrocode}
%\end{macro}
%
%\begin{macro}{\@locale@parse@monthblock}
%    \begin{macrocode}
\def\@locale@parse@monthblock#1{%
  \@locale@parse@months#1% remove outer group
  \@locale@parse@shortmonthblock
}
%    \end{macrocode}
%\end{macro}
%
%\begin{macro}{\@locale@parse@months}
%    \begin{macrocode}
\def\@locale@parse@months#1#2#3#4#5#6#7#8#9{%
  \LocaleSetDialectAttribute{\@locale@dialect}{month.1}{#1}%
  \LocaleSetDialectAttribute{\@locale@dialect}{month.2}{#2}%
  \LocaleSetDialectAttribute{\@locale@dialect}{month.3}{#3}%
  \LocaleSetDialectAttribute{\@locale@dialect}{month.4}{#4}%
  \LocaleSetDialectAttribute{\@locale@dialect}{month.5}{#5}%
  \LocaleSetDialectAttribute{\@locale@dialect}{month.6}{#6}%
  \LocaleSetDialectAttribute{\@locale@dialect}{month.7}{#7}%
  \LocaleSetDialectAttribute{\@locale@dialect}{month.8}{#8}%
  \LocaleSetDialectAttribute{\@locale@dialect}{month.9}{#9}%
%    \end{macrocode}
%Grab the remaining three arguments:
%    \begin{macrocode}
  \@locale@parse@endmonths
}
%    \end{macrocode}
%\end{macro}
%
%\begin{macro}{\@locale@parse@endmonths}
%    \begin{macrocode}
\def\@locale@parse@endmonths#1#2#3{%
  \LocaleSetDialectAttribute{\@locale@dialect}{month.10}{#1}%
  \LocaleSetDialectAttribute{\@locale@dialect}{month.11}{#2}%
  \LocaleSetDialectAttribute{\@locale@dialect}{month.12}{#3}%
}
%    \end{macrocode}
%\end{macro}
%
%\begin{macro}{\@locale@parse@shortmonthblock}
%    \begin{macrocode}
\def\@locale@parse@shortmonthblock#1{%
  \@locale@parse@shortmonths#1% remove outer group
  \@locale@parse@standalone@weekdayblock
}
%    \end{macrocode}
%\end{macro}
%
%\begin{macro}{\@locale@parse@shortmonths}
%    \begin{macrocode}
\def\@locale@parse@shortmonths#1#2#3#4#5#6#7#8#9{%
  \LocaleSetDialectAttribute{\@locale@dialect}{shortmonth.1}{#1}%
  \LocaleSetDialectAttribute{\@locale@dialect}{shortmonth.2}{#2}%
  \LocaleSetDialectAttribute{\@locale@dialect}{shortmonth.3}{#3}%
  \LocaleSetDialectAttribute{\@locale@dialect}{shortmonth.4}{#4}%
  \LocaleSetDialectAttribute{\@locale@dialect}{shortmonth.5}{#5}%
  \LocaleSetDialectAttribute{\@locale@dialect}{shortmonth.6}{#6}%
  \LocaleSetDialectAttribute{\@locale@dialect}{shortmonth.7}{#7}%
  \LocaleSetDialectAttribute{\@locale@dialect}{shortmonth.8}{#8}%
  \LocaleSetDialectAttribute{\@locale@dialect}{shortmonth.9}{#9}%
%    \end{macrocode}
%Grab the remaining three arguments:
%    \begin{macrocode}
  \@locale@parse@endshortmonths
}
%    \end{macrocode}
%\end{macro}
%
%\begin{macro}{\@locale@parse@endshortmonths}
%    \begin{macrocode}
\def\@locale@parse@endshortmonths#1#2#3{%
  \LocaleSetDialectAttribute{\@locale@dialect}{shortmonth.10}{#1}%
  \LocaleSetDialectAttribute{\@locale@dialect}{shortmonth.11}{#2}%
  \LocaleSetDialectAttribute{\@locale@dialect}{shortmonth.12}{#3}%
}
%    \end{macrocode}
%\end{macro}
%
%\begin{macro}{\@locale@parse@standalone@weekdayblock}
%    \begin{macrocode}
\def\@locale@parse@standalone@weekdayblock#1{%
  \@locale@parse@standalone@weekdays#1% remove outer group
  \@locale@parse@standalone@shortweekdayblock
}
%    \end{macrocode}
%\end{macro}
%
%\begin{macro}{\@locale@parse@standalone@weekdays}
%    \begin{macrocode}
\def\@locale@parse@standalone@weekdays#1#2#3#4#5#6#7{%
  \LocaleSetDialectAttribute{\@locale@dialect}{standalone.day.0}{#1}%
  \LocaleSetDialectAttribute{\@locale@dialect}{standalone.day.1}{#2}%
  \LocaleSetDialectAttribute{\@locale@dialect}{standalone.day.2}{#3}%
  \LocaleSetDialectAttribute{\@locale@dialect}{standalone.day.3}{#4}%
  \LocaleSetDialectAttribute{\@locale@dialect}{standalone.day.4}{#5}%
  \LocaleSetDialectAttribute{\@locale@dialect}{standalone.day.5}{#6}%
  \LocaleSetDialectAttribute{\@locale@dialect}{standalone.day.6}{#7}%
}
%    \end{macrocode}
%\end{macro}
%
%\begin{macro}{\@locale@parse@standalone@shortweekdayblock}
%    \begin{macrocode}
\def\@locale@parse@standalone@shortweekdayblock#1{%
  \@locale@parse@standalone@shortweekdays#1% remove outer group
  \@locale@parse@standalone@monthblock
}
%    \end{macrocode}
%\end{macro}
%
%\begin{macro}{\@locale@parse@standalone@shortweekdays}
%    \begin{macrocode}
\def\@locale@parse@standalone@shortweekdays#1#2#3#4#5#6#7{%
  \LocaleSetDialectAttribute{\@locale@dialect}{standalone.shortday.0}{#1}%
  \LocaleSetDialectAttribute{\@locale@dialect}{standalone.shortday.1}{#2}%
  \LocaleSetDialectAttribute{\@locale@dialect}{standalone.shortday.2}{#3}%
  \LocaleSetDialectAttribute{\@locale@dialect}{standalone.shortday.3}{#4}%
  \LocaleSetDialectAttribute{\@locale@dialect}{standalone.shortday.4}{#5}%
  \LocaleSetDialectAttribute{\@locale@dialect}{standalone.shortday.5}{#6}%
  \LocaleSetDialectAttribute{\@locale@dialect}{standalone.shortday.6}{#7}%
}
%    \end{macrocode}
%\end{macro}
%
%\begin{macro}{\@locale@parse@standalone@monthblock}
%    \begin{macrocode}
\def\@locale@parse@standalone@monthblock#1{%
  \@locale@parse@standalone@months#1% remove outer group
  \@locale@parse@standalone@shortmonthblock
}
%    \end{macrocode}
%\end{macro}
%
%\begin{macro}{\@locale@parse@standalone@months}
%    \begin{macrocode}
\def\@locale@parse@standalone@months#1#2#3#4#5#6#7#8#9{%
  \LocaleSetDialectAttribute{\@locale@dialect}{standalone.month.1}{#1}%
  \LocaleSetDialectAttribute{\@locale@dialect}{standalone.month.2}{#2}%
  \LocaleSetDialectAttribute{\@locale@dialect}{standalone.month.3}{#3}%
  \LocaleSetDialectAttribute{\@locale@dialect}{standalone.month.4}{#4}%
  \LocaleSetDialectAttribute{\@locale@dialect}{standalone.month.5}{#5}%
  \LocaleSetDialectAttribute{\@locale@dialect}{standalone.month.6}{#6}%
  \LocaleSetDialectAttribute{\@locale@dialect}{standalone.month.7}{#7}%
  \LocaleSetDialectAttribute{\@locale@dialect}{standalone.month.8}{#8}%
  \LocaleSetDialectAttribute{\@locale@dialect}{standalone.month.9}{#9}%
%    \end{macrocode}
%Grab the remaining three arguments:
%    \begin{macrocode}
  \@locale@parse@endstandalone@months
}
%    \end{macrocode}
%\end{macro}
%
%\begin{macro}{\@locale@parse@endstandalone@months}
%    \begin{macrocode}
\def\@locale@parse@endstandalone@months#1#2#3{%
  \LocaleSetDialectAttribute{\@locale@dialect}{standalone.month.10}{#1}%
  \LocaleSetDialectAttribute{\@locale@dialect}{standalone.month.11}{#2}%
  \LocaleSetDialectAttribute{\@locale@dialect}{standalone.month.12}{#3}%
}
%    \end{macrocode}
%\end{macro}
%
%\begin{macro}{\@locale@parse@standalone@shortmonthblock}
%    \begin{macrocode}
\def\@locale@parse@standalone@shortmonthblock#1{%
  \@locale@parse@standalone@shortmonths#1% remove outer group
  \@locale@parse@numericblock
}
%    \end{macrocode}
%\end{macro}
%
%\begin{macro}{\@locale@parse@standalone@shortmonths}
%    \begin{macrocode}
\def\@locale@parse@standalone@shortmonths#1#2#3#4#5#6#7#8#9{%
  \LocaleSetDialectAttribute{\@locale@dialect}{standalone.shortmonth.1}{#1}%
  \LocaleSetDialectAttribute{\@locale@dialect}{standalone.shortmonth.2}{#2}%
  \LocaleSetDialectAttribute{\@locale@dialect}{standalone.shortmonth.3}{#3}%
  \LocaleSetDialectAttribute{\@locale@dialect}{standalone.shortmonth.4}{#4}%
  \LocaleSetDialectAttribute{\@locale@dialect}{standalone.shortmonth.5}{#5}%
  \LocaleSetDialectAttribute{\@locale@dialect}{standalone.shortmonth.6}{#6}%
  \LocaleSetDialectAttribute{\@locale@dialect}{standalone.shortmonth.7}{#7}%
  \LocaleSetDialectAttribute{\@locale@dialect}{standalone.shortmonth.8}{#8}%
  \LocaleSetDialectAttribute{\@locale@dialect}{standalone.shortmonth.9}{#9}%
%    \end{macrocode}
%Grab the remaining three arguments:
%    \begin{macrocode}
  \@locale@parse@endstandalone@shortmonths
}
%    \end{macrocode}
%\end{macro}
%
%\begin{macro}{\@locale@parse@endstandalone@shortmonths}
%    \begin{macrocode}
\def\@locale@parse@endstandalone@shortmonths#1#2#3{%
  \LocaleSetDialectAttribute{\@locale@dialect}{standalone.shortmonth.10}{#1}%
  \LocaleSetDialectAttribute{\@locale@dialect}{standalone.shortmonth.11}{#2}%
  \LocaleSetDialectAttribute{\@locale@dialect}{standalone.shortmonth.12}{#3}%
}
%    \end{macrocode}
%\end{macro}
%
%\begin{macro}{\@locale@parse@numericblock}
%    \begin{macrocode}
\def\@locale@parse@numericblock#1{%
  \@locale@parse@numeric#1% remove outer group
  \@locale@parse@numericfmtblock
}
%    \end{macrocode}
%\end{macro}
%
%\begin{macro}{\@locale@parse@numeric}
%    \begin{macrocode}
\def\@locale@parse@numeric#1#2#3#4#5#6#7#8#9{%
  \LocaleSetDialectAttribute{\@locale@dialect}{groupsep}{#1}%
  \LocaleSetDialectAttribute{\@locale@dialect}{decsep}{#2}%
  \LocaleSetDialectAttribute{\@locale@dialect}{exp}{#3}%
  \LocaleSetDialectAttribute{\@locale@dialect}{usesgroup}{#4}%
  \LocaleSetDialectAttribute{\@locale@dialect}{currency}{#5}%
  \LocaleSetDialectAttribute{\@locale@dialect}{regionalcurrency}{#6}%
  \LocaleSetDialectAttribute{\@locale@dialect}{currencysym}{#7}%
  \LocaleSetDialectAttribute{\@locale@dialect}{currencytex}{#8}%
  \LocaleSetDialectAttribute{\@locale@dialect}{currencysep}{#9}%
%    \end{macrocode}
% Set currency attributes.
%    \begin{macrocode}
  \LocaleProvideCurrencyAttribute{#6}{official}{#5}%
  \LocaleProvideCurrencyAttribute{#6}{sym}{#7}%
  \LocaleProvideCurrencyAttribute{#6}{tex}{#8}%
  \LocaleAddToAttributeList{currencies}{official}{#5}%
  \LocaleAddToAttributeList{currencies}{regional}{#6}%
%    \end{macrocode}
%If this dialect has an associated region, map the region to the
%currency code.
%    \begin{macrocode}
  \ifx\@locale@region\empty
  \else
    \LocaleProvideRegionAttribute{\@locale@region}{currency}{#6}%
    \LocaleXpAddToCurrencyAttributeList{#5}{region}{\@locale@region}%
    \LocaleIfSameDialectAttributeValues{\@locale@dialect}%
    {regionalcurrency}{currency}%
    {}%
    {%
      \LocaleXpAddToCurrencyAttributeList{#6}{region}{\@locale@region}%
    }%
  \fi
%    \end{macrocode}
%Grab remaining arguments.
%    \begin{macrocode}
  \@locale@parse@persym
}
%    \end{macrocode}
%\end{macro}
%\begin{macro}{\@locale@parse@persym}
%    \begin{macrocode}
\def\@locale@parse@persym#1#2{%
  \LocaleSetDialectAttribute{\@locale@dialect}{percent}{#1}%
  \LocaleSetDialectAttribute{\@locale@dialect}{permill}{#2}%
}
%    \end{macrocode}
%\end{macro}
%
%\begin{macro}{\@locale@parse@numericfmtblock}
%    \begin{macrocode}
\def\@locale@parse@numericfmtblock#1{%
  \@locale@parse@numericfmt#1% remove outer group
  \LocaleIfDateTimePatternsSupported
  {\@locale@parse@timezones}%
  {\@locale@parse@otherdata}%
}
%    \end{macrocode}
%\end{macro}
%
%\begin{macro}{\@locale@parse@numericfmt}
%    \begin{macrocode}
\def\@locale@parse@numericfmt#1#2#3#4{%
  \LocaleSetDialectAttribute{\@locale@dialect}{decfmt}{#1}%
  \LocaleSetDialectAttribute{\@locale@dialect}{intfmt}{#2}%
  \LocaleSetDialectAttribute{\@locale@dialect}{curfmt}{#3}%
  \LocaleSetDialectAttribute{\@locale@dialect}{perfmt}{#4}%
}
%    \end{macrocode}
%\end{macro}
%
%\begin{macro}{\@locale@parse@timezones}
%    \begin{macrocode}
\def\@locale@parse@timezones#1{%
  \@locale@parse@timezonemap#1\relax
  \@locale@parse@otherdata
}
%    \end{macrocode}
%\end{macro}
%
%\begin{macro}{\@locale@parse@timezonemap}
%    \begin{macrocode}
\def\@locale@parse@timezonemap#1{%
  \ifx\relax#1\relax
    \let\@locale@next\relax
  \else
    \@locale@save@timezonemap#1%
    \let\@locale@next\@locale@parse@timezonemap
  \fi
  \@locale@next
}%
%    \end{macrocode}
%\end{macro}
%
%\begin{macro}{\@locale@save@timezonemap}
%    \begin{macrocode}
\def\@locale@save@timezonemap#1#2#3#4#5{%
  \LocaleSetDialectAttribute{\@locale@dialect}{timezone.#1.short}{#2}%
  \LocaleSetDialectAttribute{\@locale@dialect}{timezone.#1.long}{#3}%
  \LocaleSetDialectAttribute{\@locale@dialect}{timezone.#1.shortdst}{#4}%
  \LocaleSetDialectAttribute{\@locale@dialect}{timezone.#1.longdst}{#5}%
  \LocaleAddToAttributeList{timezone}{id}{#1}%
}
%    \end{macrocode}
%\end{macro}
%
%\begin{macro}{\@locale@parse@otherdata}
%    \begin{macrocode}
\def\@locale@endparse@result{}
%    \end{macrocode}
%\end{macro}
%\begin{macro}{\@locale@parse@otherdata}
%    \begin{macrocode}
\def\@locale@parse@otherdata#1{%
 \ifx\@locale@endparse@result#1\relax
%    \end{macrocode}
%Finished.
%    \begin{macrocode}
  \let\@locale@next\relax
 \else
  \def\@locale@next{\@locale@parse@otherdata@localeblock#1}%
 \fi
 \@locale@next
}
%    \end{macrocode}
%\end{macro}
%
%\begin{macro}{\@locale@parse@otherdata@localeblock}
%    \begin{macrocode}
\def\@locale@parse@otherdata@localeblock#1{%
  \@locale@parse@otherdatalocaleblock#1% remove outer group
  \@locale@parse@dateblock
}
%    \end{macrocode}
%\end{macro}
%
%    \begin{macrocode}
\ifx\@locale@result\empty
%    \end{macrocode}
%Set defaults.
%    \begin{macrocode}
  \ifTeXOSQueryDryRun
    \@locale@warn{Dry run mode on. No data provided by texosquery. Check 
    TeX's shell escape status}%
  \else
  \@locale@warn{No data provided by texosquery. Check 
    TeX's shell escape status and texosquery's setup}%
  \fi
  \def\LocaleOSname{}%
  \def\LocaleOSversion{}%
  \def\LocaleOSarch{}%
  \def\LocaleNowStamp{}%
  \def\LocaleOStag{}%
  \def\LocaleOScodeset{}%
  \def\LocaleFileMod{}%
  \def\LocaleMain{}%
  \def\LocaleMainDialect{}%
  \def\LocaleMainRegion{}%
  \def\LocaleDateTimeInfo{}%
\else
  \expandafter\@locale@parse@result\@locale@result\@locale@endparse@result
\fi
%    \end{macrocode}
%
%\subsection{Locale User Commands}
%
%\begin{macro}{\LocaleLanguageTag}
%\begin{definition}
%\cs{LocaleLanguageTag}\marg{dialect}
%\end{definition}
%Get the language tag of the given dialect.
%    \begin{macrocode}
\def\LocaleLanguageTag#1{%
 \LocaleGetDialectAttribute{#1}{langtag}%
}
%    \end{macrocode}
%\end{macro}
%\begin{macro}{\LocaleLanguageName}
%\begin{definition}
%\cs{LocaleLanguageName}\marg{dialect}
%\end{definition}
%Get the language name of the given dialect according to the OS
%default language.
%    \begin{macrocode}
\def\LocaleLanguageName#1{%
 \LocaleGetDialectAttribute{#1}{langname}%
}
%    \end{macrocode}
%\end{macro}
%\begin{macro}{\LocaleLanguageNativeName}
%\begin{definition}
%\cs{LocaleLanguageNativeName}\marg{dialect}
%\end{definition}
%Get the language name of the given dialect according to that
%dialect.
%    \begin{macrocode}
\def\LocaleLanguageNativeName#1{%
 \LocaleGetDialectAttribute{#1}{nativelangname}%
}
%    \end{macrocode}
%\end{macro}
%
%\begin{macro}{\LocaleRegionName}
%\begin{definition}
%\cs{LocaleRegionName}\marg{dialect}
%\end{definition}
%Get the region name of the given dialect according to the OS
%default language.
%    \begin{macrocode}
\def\LocaleRegionName#1{%
 \LocaleGetDialectAttribute{#1}{regionname}%
}
%    \end{macrocode}
%\end{macro}
%\begin{macro}{\LocaleRegionNativeName}
%\begin{definition}
%\cs{LocaleRegionNativeName}\marg{dialect}
%\end{definition}
%Get the region name of the given dialect according to that
%dialect.
%    \begin{macrocode}
\def\LocaleRegionNativeName#1{%
 \LocaleGetDialectAttribute{#1}{nativeregionname}%
}
%    \end{macrocode}
%\end{macro}
%
%\begin{macro}{\LocaleVariantName}
%\begin{definition}
%\cs{LocaleVariantName}\marg{dialect}
%\end{definition}
%Get the variant name of the given dialect according to the OS
%default language.
%    \begin{macrocode}
\def\LocaleVariantName#1{%
 \LocaleGetDialectAttribute{#1}{variantname}%
}
%    \end{macrocode}
%\end{macro}
%\begin{macro}{\LocaleVariantNativeName}
%\begin{definition}
%\cs{LocaleVariantNativeName}\marg{dialect}
%\end{definition}
%Get the variant name of the given dialect according to that
%dialect.
%    \begin{macrocode}
\def\LocaleVariantNativeName#1{%
 \LocaleGetDialectAttribute{#1}{nativevariantname}%
}
%    \end{macrocode}
%\end{macro}
%
%\begin{macro}{\LocaleFullDate}
%\begin{definition}
%\cs{LocaleFullDate}\marg{dialect}
%\end{definition}
%Get the full date for the given dialect.
%    \begin{macrocode}
\def\LocaleFullDate#1{%
 \localedatetimefmt{\LocaleGetDialectAttribute{#1}{fulldate}}%
}
%    \end{macrocode}
%\end{macro}
%
%\begin{macro}{\LocaleLongDate}
%\begin{definition}
%\cs{LocaleLongDate}\marg{dialect}
%\end{definition}
%Get the long date for the given dialect.
%    \begin{macrocode}
\def\LocaleLongDate#1{%
 \localedatetimefmt{\LocaleGetDialectAttribute{#1}{longdate}}%
}
%    \end{macrocode}
%\end{macro}
%
%\begin{macro}{\LocaleMediumDate}
%\begin{definition}
%\cs{LocaleMediumDate}\marg{dialect}
%\end{definition}
%Get the medium date for the given dialect.
%    \begin{macrocode}
\def\LocaleMediumDate#1{%
 \localedatetimefmt{\LocaleGetDialectAttribute{#1}{meddate}}%
}
%    \end{macrocode}
%\end{macro}
%
%\begin{macro}{\LocaleShortDate}
%\begin{definition}
%\cs{LocaleShortDate}\marg{dialect}
%\end{definition}
%Get the short date for the given dialect.
%    \begin{macrocode}
\def\LocaleShortDate#1{%
 \localedatetimefmt{\LocaleGetDialectAttribute{#1}{shortdate}}%
}
%    \end{macrocode}
%\end{macro}
%
%\begin{macro}{\LocaleFirstDayIndex}
%\begin{definition}
%\cs{LocaleFirstDayIndex}\marg{dialect}
%\end{definition}
%Get the index for the first day of the week for the given dialect.
%This starts with 0 for Monday to be compatible with
%\sty{pgfcalendar} (and \sty{datetime2} which similar does this to
%be compatible with \sty{pgfcalendar}).
%    \begin{macrocode}
\def\LocaleFirstDayIndex#1{%
 \LocaleGetDialectAttributeOrDefValue{#1}{firstday}{-1}%
}
%    \end{macrocode}
%\end{macro}
%
%\begin{macro}{\dtmMondayIndex}
%    \begin{macrocode}
\def\dtmMondayIndex{0}
%    \end{macrocode}
%\end{macro}
%
%\begin{macro}{\dtmTuesdayIndex}
%    \begin{macrocode}
\def\dtmTuesdayIndex{1}
%    \end{macrocode}
%\end{macro}
%
%\begin{macro}{\dtmWednesdayIndex}
%    \begin{macrocode}
\def\dtmWednesdayIndex{2}
%    \end{macrocode}
%\end{macro}
%
%\begin{macro}{\dtmThursdayIndex}
%    \begin{macrocode}
\def\dtmThursdayIndex{3}
%    \end{macrocode}
%\end{macro}
%
%\begin{macro}{\dtmFridayIndex}
%    \begin{macrocode}
\def\dtmFridayIndex{4}
%    \end{macrocode}
%\end{macro}
%
%\begin{macro}{\dtmSaturdayIndex}
%    \begin{macrocode}
\def\dtmSaturdayIndex{5}
%    \end{macrocode}
%\end{macro}
%
%\begin{macro}{\dtmSundayIndex}
%    \begin{macrocode}
\def\dtmSundayIndex{6}
%    \end{macrocode}
%\end{macro}
%
%The ISO-8601 standard has day of week indexing starting with 
%1 for Monday. The \sty{datetime} package has day of week indexing
%starting with 1 for Sunday, so provide convenient ways to convert
%if necessary.
%\begin{macro}{\LocaleDayIndexFromZeroMonToOneSun}
%Convert from \sty{pgfcalendar}'s Monday=0 indexing to
%\sty{datetime}'s Sunday=1 indexing.
%    \begin{macrocode}
\def\LocaleDayIndexFromZeroMonToOneSun#1{%
  \ifcase#1
  2% Monday 0 -> 2
  \or
  3% Tuesday 1 -> 3
  \or
  4% Wednesday 2 -> 4
  \or
  5% Thursday 3 -> 5
  \or
  6% Friday 4 -> 6
  \or
  7% Saturday 5 -> 7
  \or
  1% Sunday 6 -> 1
  \else
  -1% invalid
  \fi
}
%    \end{macrocode}
%\end{macro}
%
%\begin{macro}{\LocaleDayIndexFromZeroMonToOneMon}
%Convert from \sty{pgfcalendar}'s Monday=0 indexing to
%ISO-8601's Monday=1 indexing.
%    \begin{macrocode}
\def\LocaleDayIndexFromZeroMonToOneMon#1{%
  \ifcase#1
  1% Monday 0 -> 1
  \or
  2% Tuesday 1 -> 2
  \or
  3% Wednesday 2 -> 3
  \or
  4% Thursday 3 -> 4
  \or
  5% Friday 4 -> 5
  \or
  6% Saturday 5 -> 6
  \or
  7% Sunday 6 -> 7
  \else
  -1% invalid
  \fi
}
%    \end{macrocode}
%\end{macro}
%
%\begin{macro}{\LocaleDayIndexFromOneSunToZeroMon}
%Convert from \sty{datetime}'s Sunday=1 indexing to \sty{pgfcalendar}'s
%Monday=0 indexing.
%    \begin{macrocode}
\def\LocaleDayIndexFromOneSunToZeroMon#1{%
  \ifcase#1
  % no 0 
  \or
  6% Sunday 1 -> 6
  \or
  0% Monday 2 -> 0
  \or
  1% Tuesday 3 -> 1
  \or
  2% Wednesday 4 -> 2
  \or
  3% Thursday 5 -> 3
  \or
  4% Friday 6 -> 4
  \or
  5% Saturday 7 -> 5
  \else
  -1% invalid
  \fi
}
%    \end{macrocode}
%\end{macro}
%
%\begin{macro}{\LocaleDayIndexFromOneMonToZeroMon}
%Convert from ISO-8601's Monday=1 indexing to \sty{pgfcalendar}'s
%Monday=0 indexing.
%    \begin{macrocode}
\def\LocaleDayIndexFromOneMonToZeroMon#1{%
  \ifcase#1
  % no 0 
  \or
  0% Monday 1 -> 0
  \or
  1% Tuesday 2 -> 1
  \or
  2% Wednesday 3 -> 2
  \or
  3% Thursday 4 -> 3
  \or
  4% Friday 5 -> 4
  \or
  5% Saturday 6 -> 5
  \or
  6% Sunday 7 -> 6
  \else
  -1% invalid
  \fi
}
%    \end{macrocode}
%\end{macro}
%
%\begin{macro}{\LocaleDayIndexFromRegion}
%Convert from the region's day of the week (starting from 1)
%to \sty{pgfcalendar}'s 0=Monday indexing. The first argument is the
%dialect label.
%    \begin{macrocode}
\def\LocaleDayIndexFromRegion#1#2{%
%    \end{macrocode}
%Get the region's first day of the week.
%    \begin{macrocode}
  \ifcase\LocaleFirstDayIndex{#1}
%    \end{macrocode}
%Monday is the first day of the week. That is, the region's using 1=Monday
%indexing.
%    \begin{macrocode}
    \LocaleDayIndexFromOneMonToZeroMon{#2}%
  \or
%    \end{macrocode}
%Tuesday is the first day of the week. That is, the region's using
%1=Tuesday indexing. (These mid-week cases are unlikely, but add
%them for completeness.)
%    \begin{macrocode}
    \ifnum#2=7 % Monday
      0%
    \else
      #2% 
    \fi
  \or
%    \end{macrocode}
%Wednesday is the first day of the week. That is, the region's using
%1=Wednesday indexing.
%    \begin{macrocode}
    \ifcase#1
    % no 0 
    \or
    2% Wednesday 1 -> 2
    \or
    3% Thursday 2 -> 3
    \or
    4% Friday 3 -> 4
    \or
    5% Saturday 4 -> 5
    \or
    6% Sunday 5 -> 6
    \or
    0% Monday 6 -> 0
    \or
    1% Tuesday 7 -> 1
    \else
    -1% invalid
    \fi
  \or
%    \end{macrocode}
%Thursday is the first day of the week. That is, the region's using
%1=Thursday indexing.
%    \begin{macrocode}
    \ifcase#1
    % no 0 
    \or
    3% Thursday 1 -> 3
    \or
    4% Friday 2 -> 4
    \or
    5% Saturday 3 -> 5
    \or
    6% Sunday 4 -> 6
    \or
    0% Monday 5 -> 0
    \or
    1% Tuesday 6 -> 1
    \or
    2% Wednesday 7 -> 2
    \else
    -1% invalid
    \fi
  \or
%    \end{macrocode}
%Friday is the first day of the week. That is, the region's using
%1=Friday indexing.
%    \begin{macrocode}
    \ifcase#1
    % no 0 
    \or
    4% Friday 1 -> 4
    \or
    5% Saturday 2 -> 5
    \or
    6% Sunday 3 -> 6
    \or
    0% Monday 4 -> 0
    \or
    1% Tuesday 5 -> 1
    \or
    2% Wednesday 6 -> 2
    \or
    3% Thursday 7 -> 3
    \else
    -1% invalid
    \fi
  \or
%    \end{macrocode}
%Saturday is the first day of the week. That is, the region's using
%1=Saturday indexing.
%    \begin{macrocode}
    \ifcase#1
    % no 0 
    \or
    5% Saturday 1 -> 5
    \or
    6% Sunday 2 -> 6
    \or
    0% Monday 3 -> 0
    \or
    1% Tuesday 4 -> 1
    \or
    2% Wednesday 5 -> 2
    \or
    3% Thursday 6 -> 3
    \or
    4% Friday 7 -> 4
    \else
    -1% invalid
    \fi
  \or
%    \end{macrocode}
%Sunday is the first day of the week. That is, the region's using
%1=Sunday indexing.
%    \begin{macrocode}
    \LocaleDayIndexFromOneSunToZeroMon{#2}%
  \else
    #2%
  \fi
}
%    \end{macrocode}
%\end{macro}
%
%\begin{macro}{\LocaleDayIndexToRegion}
%Convert from \sty{pgfcalendar}'s 0=Monday indexing to the region's 
%first day of the week (starting from 1). The first argument is the
%dialect label.
%    \begin{macrocode}
\def\LocaleDayIndexToRegion#1#2{%
%    \end{macrocode}
%Get the region's first day of the week.
%    \begin{macrocode}
  \ifcase\LocaleFirstDayIndex{#1}
%    \end{macrocode}
%Monday is the first day of the week. That is, the region's using 1=Monday
%indexing.
%    \begin{macrocode}
    \LocaleDayIndexFromZeroMonToOneMon{#2}%
  \or
%    \end{macrocode}
%Tuesday is the first day of the week. That is, the region's using
%1=Tuesday indexing. (These mid-week cases are unlikely, but add
%them for completeness.)
%    \begin{macrocode}
    \ifnum#2=0 % Monday
      7%
    \else
      #2% 
    \fi
  \or
%    \end{macrocode}
%Wednesday is the first day of the week. That is, the region's using
%1=Wednesday indexing.
%    \begin{macrocode}
    \ifcase#1
     6% Monday 0 -> 6
    \or
     7% Tuesday 1 -> 7
    \or
     1% Wednesday 2 -> 1
    \or
     2% Thursday 3 -> 2
    \or
     3% Friday 4 -> 3
    \or
     4% Saturday 5 -> 4
    \or
     5% Sunday 6 -> 5
    \else
    -1% invalid
    \fi
  \or
%    \end{macrocode}
%Thursday is the first day of the week. That is, the region's using
%1=Thursday indexing.
%    \begin{macrocode}
    \ifcase#1
     5% Monday 0 -> 5
    \or
     6% Tuesday 1 -> 6
    \or
     7% Wednesday 2 -> 7
    \or
     1% Thursday 3 -> 1
    \or
     2% Friday 4 -> 2
    \or
     3% Saturday 5 -> 3
    \or
     4% Sunday 6 -> 4
    \else
    -1% invalid
    \fi
  \or
%    \end{macrocode}
%Friday is the first day of the week. That is, the region's using
%1=Friday indexing.
%    \begin{macrocode}
    \ifcase#1
     4% Monday 0 -> 4
    \or
     5% Tuesday 1 -> 5
    \or
     6% Wednesday 2 -> 6
    \or
     7% Thursday 3 -> 7
    \or
     1% Friday 4 -> 1
    \or
     2% Saturday 5 -> 2
    \or
     3% Sunday 6 -> 3
    \else
    -1% invalid
    \fi
  \or
%    \end{macrocode}
%Saturday is the first day of the week. That is, the region's using
%1=Saturday indexing.
%    \begin{macrocode}
    \ifcase#1
     3% Monday 0 -> 3
    \or
     4% Tuesday 1 -> 4
    \or
     5% Wednesday 2 -> 5
    \or
     6% Thursday 3 -> 6
    \or
     7% Friday 4 -> 7
    \or
     1% Saturday 5 -> 1
    \or
     2% Sunday 6 -> 2
    \else
    -1% invalid
    \fi
  \or
%    \end{macrocode}
%Sunday is the first day of the week. That is, the region's using
%1=Sunday indexing.
%    \begin{macrocode}
    \LocaleDayIndexFromZeroMonToOneSun{#2}%
  \else
    #2%
  \fi
}
%    \end{macrocode}
%\end{macro}
%
%\begin{macro}{\LocaleFullTime}
%\begin{definition}
%\cs{LocaleFullTime}\marg{dialect}
%\end{definition}
%Get the full time for the given dialect.
%    \begin{macrocode}
\def\LocaleFullTime#1{%
 \localedatetimefmt{\LocaleGetDialectAttribute{#1}{fulltime}}%
}
%    \end{macrocode}
%\end{macro}
%
%\begin{macro}{\LocaleLongTime}
%\begin{definition}
%\cs{LocaleLongTime}\marg{dialect}
%\end{definition}
%Get the long time for the given dialect.
%    \begin{macrocode}
\def\LocaleLongTime#1{%
 \localedatetimefmt{\LocaleGetDialectAttribute{#1}{longtime}}%
}
%    \end{macrocode}
%\end{macro}
%
%\begin{macro}{\LocaleMediumTime}
%\begin{definition}
%\cs{LocaleMediumTime}\marg{dialect}
%\end{definition}
%Get the medium time for the given dialect.
%    \begin{macrocode}
\def\LocaleMediumTime#1{%
 \localedatetimefmt{\LocaleGetDialectAttribute{#1}{medtime}}%
}
%    \end{macrocode}
%\end{macro}
%
%\begin{macro}{\LocaleShortTime}
%\begin{definition}
%\cs{LocaleShortTime}\marg{dialect}
%\end{definition}
%Get the short time for the given dialect.
%    \begin{macrocode}
\def\LocaleShortTime#1{%
 \localedatetimefmt{\LocaleGetDialectAttribute{#1}{shorttime}}%
}
%    \end{macrocode}
%\end{macro}
%
%\begin{macro}{\LocaleFullDateTime}
%\begin{definition}
%\cs{LocaleFullDateTime}\marg{dialect}
%\end{definition}
%Get the full date and time for the given dialect.
%    \begin{macrocode}
\def\LocaleFullDateTime#1{%
 \localedatetimefmt{\LocaleGetDialectAttribute{#1}{fulldatetime}}%
}
%    \end{macrocode}
%\end{macro}
%
%\begin{macro}{\LocaleLongDateTime}
%\begin{definition}
%\cs{LocaleLongDateTime}\marg{dialect}
%\end{definition}
%Get the long date and time for the given dialect.
%    \begin{macrocode}
\def\LocaleLongDateTime#1{%
 \localedatetimefmt{\LocaleGetDialectAttribute{#1}{longdatetime}}%
}
%    \end{macrocode}
%\end{macro}
%
%\begin{macro}{\LocaleMediumDateTime}
%\begin{definition}
%\cs{LocaleMediumDateTime}\marg{dialect}
%\end{definition}
%Get the medium date and time for the given dialect.
%    \begin{macrocode}
\def\LocaleMediumDateTime#1{%
 \localedatetimefmt{\LocaleGetDialectAttribute{#1}{meddatetime}}%
}
%    \end{macrocode}
%\end{macro}
%
%\begin{macro}{\LocaleShortDateTime}
%\begin{definition}
%\cs{LocaleShortDateTime}\marg{dialect}
%\end{definition}
%Get the short date and time for the given dialect.
%    \begin{macrocode}
\def\LocaleShortDateTime#1{%
 \localedatetimefmt{\LocaleGetDialectAttribute{#1}{shortdatetime}}%
}
%    \end{macrocode}
%\end{macro}
%
%\begin{macro}{\LocaleDayName}
%\begin{definition}
%\cs{LocaleDayName}\marg{dialect}\marg{n}
%\end{definition}
%Gets the day of week name identified by the index \meta{n} (0 for
%Monday, 1 for Tuesday, etc) in the given dialect.
%    \begin{macrocode}
\def\LocaleDayName#1#2{%
 \LocaleGetDialectAttribute{#1}{day.\number#2}%
}
%    \end{macrocode}
%\end{macro}
%
%\begin{macro}{\LocaleShortDayName}
%\begin{definition}
%\cs{LocaleShortDayName}\marg{dialect}\marg{n}
%\end{definition}
%Gets the abbreviated day of week name identified by the index \meta{n} (0 for
%Monday, 1 for Tuesday, etc) in the given dialect.
%    \begin{macrocode}
\def\LocaleShortDayName#1#2{%
 \LocaleGetDialectAttribute{#1}{shortday.\number#2}%
}
%    \end{macrocode}
%\end{macro}
%
%\begin{macro}{\LocaleMonthName}
%\begin{definition}
%\cs{LocaleMonthName}\marg{dialect}\marg{n}
%\end{definition}
%Gets the month name identified by the index \meta{n} (1 for
%January, 2 for February, etc) in the given dialect.
%    \begin{macrocode}
\def\LocaleMonthName#1#2{%
 \LocaleGetDialectAttribute{#1}{month.\number#2}%
}
%    \end{macrocode}
%\end{macro}
%
%\begin{macro}{\LocaleShortMonthName}
%\begin{definition}
%\cs{LocaleShortMonthName}\marg{dialect}\marg{n}
%\end{definition}
%Gets the month week name identified by the index \meta{n} (1 for
%January, 2 for February, etc) in the given dialect.
%    \begin{macrocode}
\def\LocaleShortMonthName#1#2{%
 \LocaleGetDialectAttribute{#1}{shortmonth.\number#2}%
}
%    \end{macrocode}
%\end{macro}
%
%Some languages have a different form for month or week day names
%used in a standalone context, such as in a column header. These are
%provided with the \cs{LocaleStandalone\ldots} commands defined
%below. For most languages they will be the same as the above.
%
%\begin{macro}{\LocaleStandaloneDayName}
%\begin{definition}
%\cs{LocaleStandaloneDayName}\marg{dialect}\marg{n}
%\end{definition}
%Gets the day of week name identified by the index \meta{n} (0 for
%Monday, 1 for Tuesday, etc) in the given dialect.
%    \begin{macrocode}
\def\LocaleStandaloneDayName#1#2{%
 \LocaleGetDialectAttribute{#1}{standalone.day.\number#2}%
}
%    \end{macrocode}
%\end{macro}
%
%\begin{macro}{\LocaleStandaloneShortDayName}
%\begin{definition}
%\cs{LocaleStandaloneShortDayName}\marg{dialect}\marg{n}
%\end{definition}
%Gets the abbreviated day of week name identified by the index \meta{n} (0 for
%Monday, 1 for Tuesday, etc) in the given dialect.
%    \begin{macrocode}
\def\LocaleStandaloneShortDayName#1#2{%
 \LocaleGetDialectAttribute{#1}{standalone.shortday.\number#2}%
}
%    \end{macrocode}
%\end{macro}
%
%\begin{macro}{\LocaleStandaloneMonthName}
%\begin{definition}
%\cs{LocaleStandaloneMonthName}\marg{dialect}\marg{n}
%\end{definition}
%Gets the month week name identified by the index \meta{n} (1 for
%January, 2 for February, etc) in the given dialect.
%    \begin{macrocode}
\def\LocaleStandaloneMonthName#1#2{%
 \LocaleGetDialectAttribute{#1}{standalone.month.\number#2}%
}
%    \end{macrocode}
%\end{macro}
%
%\begin{macro}{\LocaleStandaloneShortMonthName}
%\begin{definition}
%\cs{LocaleStandaloneShortMonthName}\marg{dialect}\marg{n}
%\end{definition}
%Gets the month week name identified by the index \meta{n} (1 for
%January, 2 for February, etc) in the given dialect.
%    \begin{macrocode}
\def\LocaleStandaloneShortMonthName#1#2{%
 \LocaleGetDialectAttribute{#1}{standalone.shortmonth.\number#2}%
}
%    \end{macrocode}
%\end{macro}
%
%\begin{macro}{\LocaleNumericGroupSep}
%\begin{definition}
%\cs{LocaleNumericGroupSep}\marg{dialect}
%\end{definition}
%Gets the numeric group separator for the given dialect.
%    \begin{macrocode}
\def\LocaleNumericGroupSep#1{%
 \LocaleGetDialectAttribute{#1}{groupsep}%
}
%    \end{macrocode}
%\end{macro}
%
%\begin{macro}{\LocaleNumericDecimalSep}
%\begin{definition}
%\cs{LocaleNumericDecimalSep}\marg{dialect}
%\end{definition}
%Gets the numeric decimal separator for the given dialect.
%    \begin{macrocode}
\def\LocaleNumericDecimalSep#1{%
 \LocaleGetDialectAttribute{#1}{decsep}%
}
%    \end{macrocode}
%\end{macro}
%
%\begin{macro}{\LocaleNumericMonetarySep}
%\begin{definition}
%\cs{LocaleNumericMonetarySep}\marg{dialect}
%\end{definition}
%Gets the numeric monetary separator for the given dialect.
%    \begin{macrocode}
\def\LocaleNumericMonetarySep#1{%
 \LocaleGetDialectAttribute{#1}{currencysep}%
}
%    \end{macrocode}
%\end{macro}
%
%\begin{macro}{\LocaleNumericExponent}
%\begin{definition}
%\cs{LocaleNumericExponent}\marg{dialect}
%\end{definition}
%Gets the exponent symbol for the given dialect.
%    \begin{macrocode}
\def\LocaleNumericExponent#1{%
 \LocaleGetDialectAttribute{#1}{exp}%
}
%    \end{macrocode}
%\end{macro}
%
%\begin{macro}{\LocaleCurrencyLabel}
%\begin{definition}
%\cs{LocaleCurrencyLabel}\marg{dialect}
%\end{definition}
%Gets the currency label for the given dialect.
%    \begin{macrocode}
\def\LocaleCurrencyLabel#1{%
 \LocaleIfDialectAttributeEqCs{#1}{currency}{\@locale@unknown@currency}%
 {%
%    \end{macrocode}
%Unknown currency symbol.
%    \begin{macrocode}
   \ifx\LocaleMainRegion\empty
     \@locale@os@currencycode
   \else
     \LocaleGetDialectAttributeOrDefValue{\LocaleMainDialect}{currency}%
     {\@locale@os@currencycode}%
   \fi
 }%
 {\LocaleGetDialectAttribute{#1}{currency}}%
}
%    \end{macrocode}
%\end{macro}
%
%\begin{macro}{\LocaleCurrencyRegionalLabel}
%\begin{definition}
%\cs{LocaleCurrencyRegionalLabel}\marg{dialect}
%\end{definition}
%Gets the regional currency label for the given dialect.
%    \begin{macrocode}
\def\LocaleCurrencyRegionalLabel#1{%
 \LocaleIfDialectAttributeEqCs{#1}{regionalcurrency}%
 {\@locale@unknown@currency}%
 {%
%    \end{macrocode}
%Unknown currency symbol.
%    \begin{macrocode}
   \ifx\LocaleMainRegion\empty
     \@locale@os@regionalcurrencycode
   \else
     \LocaleGetDialectAttributeOrDefValue{\LocaleMainDialect}{regionalcurrency}%
     {\@locale@os@regionalcurrencycode}%
   \fi
 }%
 {\LocaleGetDialectAttribute{#1}{regionalcurrency}}%
}
%    \end{macrocode}
%\end{macro}
%
%\begin{macro}{\LocaleCurrencySymbol}
%\begin{definition}
%\cs{LocaleCurrencySymbol}\marg{dialect}
%\end{definition}
%Gets the currency symbol for the given dialect. (May have non-ASCII
%characters.)
%    \begin{macrocode}
\def\LocaleCurrencySymbol#1{%
 \LocaleIfDialectAttributeEqCs{#1}{currency}{\@locale@unknown@currency}%
 {%
%    \end{macrocode}
%Unknown currency code. Does the main locale have a region?
%    \begin{macrocode}
   \ifx\LocaleMainRegion\empty
%    \end{macrocode}
%No main region, try the default OS currency.
%    \begin{macrocode}
     \ifx\@locale@os@currencysym\empty
       \LocaleGetDialectAttribute{#1}{currencysym}%
     \else
       \@locale@os@currencysym
     \fi
   \else
%    \end{macrocode}
%Try the main dialect attribute.
%    \begin{macrocode}
     \LocaleGetDialectAttributeOrDefValue{\LocaleMainDialect}{currencysym}%
     {\LocaleGetDialectAttribute{#1}{currencysym}}%
   \fi
 }%
 {%
   \LocaleIfSameDialectAttributeValues{#1}%
   {currency}{currencysym}%
   {%
     \LocaleGetCurrencyAttributeOrDefValue
     {\LocaleGetDialectAttribute{#1}{currency}}{sym}%
     {\LocaleGetDialectAttribute{#1}{currencysym}}%
   }%
   {%
     \LocaleGetDialectAttribute{#1}{currencysym}%
   }%
 }%
}
%    \end{macrocode}
%\end{macro}
%
%\begin{macro}{\LocaleCurrencyTeXSymbol}
%\begin{definition}
%\cs{LocaleCurrencyTeXSymbol}\marg{dialect}
%\end{definition}
%Gets the currency symbol for the given dialect. (May include
%\sty{texosquery} currency control symbols.)
%    \begin{macrocode}
\def\LocaleCurrencyTeXSymbol#1{%
 \LocaleIfDialectAttributeEqCs{#1}{currency}{\@locale@unknown@currency}%
 {%
%    \end{macrocode}
%Unknown currency code. Does the main locale have a region?
%    \begin{macrocode}
   \ifx\LocaleMainRegion\empty
     \ifx\@locale@os@currencytex\empty
       \LocaleGetDialectAttribute{#1}{currencytex}%
     \else
       \@locale@os@currencytex
     \fi
   \fi
 }%
 {%
   \LocaleIfSameDialectAttributeValues{#1}%
   {currency}{currencytex}%
   {%
     \LocaleGetCurrencyAttributeOrDefValue
     {\LocaleGetDialectAttribute{#1}{currency}}{tex}%
     {\LocaleGetDialectAttribute{#1}{currencytex}}%
   }%
   {%
     \LocaleGetDialectAttribute{#1}{currencytex}%
   }%
 }%
}
%    \end{macrocode}
%\end{macro}
%
%\begin{macro}{\LocaleNumericPercent}
%\begin{definition}
%\cs{LocaleNumericPercent}\marg{dialect}
%\end{definition}
%Gets the percent symbol for this dialect.
%    \begin{macrocode}
\def\LocaleNumericPercent#1{%
 \LocaleGetDialectAttribute{#1}{percent}%
}
%    \end{macrocode}
%\end{macro}
%
%\begin{macro}{\LocaleNumericPermill}
%\begin{definition}
%\cs{LocaleNumericPermill}\marg{dialect}
%\end{definition}
%Gets the permill symbol for this dialect.
%    \begin{macrocode}
\def\LocaleNumericPermill#1{%
 \LocaleGetDialectAttribute{#1}{permill}%
}
%    \end{macrocode}
%\end{macro}
%
%\subsection{Patterns}
%
%\begin{macro}{\LocaleApplyDateTimePattern}
%\begin{definition}
%\cs{LocaleApplyDateTimePattern}\marg{dialect}\marg{attribute}\marg{date-time data}
%\end{definition}
%    \begin{macrocode}
\def\LocaleApplyDateTimePattern#1#2#3{%
 \LocaleIfHasDialectNonEmptyAttribute{#1}{#2}%
 {%
%    \end{macrocode}
%The data might be provided with the control sequence
%\cs{LocaleDateTimeInfo}, which may be empty and will need expanding.
%    \begin{macrocode}
   \ifx\empty#3\empty
     \@locale@warn{No date-time data for pattern 
       (attribute `#2', dialect `#1')}%
   \else
    \localedatetimefmt
     {\expandafter\texosqueryfmtdatetime
      \csname locale@dialect@#2@#1\expandafter\endcsname#3}%
   \fi
 }%
 {%
   \@locale@warn{No date-time pattern attribute `#2' for dialect `#1'}%
 }%
}
%    \end{macrocode}
%\end{macro}
%
%If the date-time patterns are required, the name commands such as
%\cs{texosqueryfmtpatMMM} need defining
%    \begin{macrocode}
\LocaleIfDateTimePatternsSupported
{%
%    \end{macrocode}
%\begin{macro}{\texosqueryfmtpatMMM}
%Short month name.
%    \begin{macrocode}
  \def\texosqueryfmtpatMMM#1{%
    \CurrentLocaleShortMonthName{#1}%
  }
%    \end{macrocode}
%\end{macro}
%
%\begin{macro}{\texosqueryfmtpatMMMM}
%Full month name.
%    \begin{macrocode}
  \def\texosqueryfmtpatMMMM#1{%
    \CurrentLocaleMonthName{#1}%
  }
%    \end{macrocode}
%\end{macro}
%
%\begin{macro}{\texosqueryfmtpatLLL}
%Short standalone month name.
%    \begin{macrocode}
  \def\texosqueryfmtpatLLL#1{%
    \CurrentLocaleStandaloneShortMonthName{#1}%
  }
%    \end{macrocode}
%\end{macro}
%
%\begin{macro}{\texosqueryfmtpatLLLL}
%Full standalone month name.
%    \begin{macrocode}
  \def\texosqueryfmtpatLLLL#1{%
    \CurrentLocaleStandaloneMonthName{#1}%
  }
%    \end{macrocode}
%\end{macro}
%
%\begin{macro}{\texosqueryfmtpatEEE}
%Short day of week name.
%    \begin{macrocode}
  \def\texosqueryfmtpatEEE#1{%
    \CurrentLocaleShortDayName{\LocaleDayIndexFromOneMonToZeroMon{#1}}%
  }
%    \end{macrocode}
%\end{macro}
%
%\begin{macro}{\texosqueryfmtpatEEEE}
%Full day of week name.
%    \begin{macrocode}
  \def\texosqueryfmtpatEEEE#1{%
    \CurrentLocaleDayName{\LocaleDayIndexFromOneMonToZeroMon{#1}}%
  }
%    \end{macrocode}
%\end{macro}
%
%\begin{macro}{\texosquerytimezonefmt}
%Allow the time zone name to be wrapped in a formatting command.
%    \begin{macrocode}
  \def\texosquerytimezonefmt#1{#1}
%    \end{macrocode}
%\end{macro}
%
%\begin{macro}{\texosqueryshorttimezone}
%    \begin{macrocode}
  \def\texosqueryshorttimezone#1{%
    \texosquerytimezonefmt
     {\LocaleGetDialectAttribute{\CurrentTrackedDialect}{timezone.#1.short}}%
  }
%    \end{macrocode}
%\end{macro}
%
%\begin{macro}{\texosqueryshortdstzone}
%    \begin{macrocode}
  \def\texosqueryshortdstzone#1{%
    \texosquerytimezonefmt
     {\LocaleGetDialectAttribute{\CurrentTrackedDialect}{timezone.#1.shortdst}}%
  }
%    \end{macrocode}
%\end{macro}
%
%\begin{macro}{\texosquerylongtimezone}
%    \begin{macrocode}
  \def\texosquerylongtimezone#1{%
    \texosquerytimezonefmt
     {\LocaleGetDialectAttribute{\CurrentTrackedDialect}{timezone.#1.long}}%
  }
%    \end{macrocode}
%\end{macro}
%
%\begin{macro}{\texosquerylongdstzone}
%    \begin{macrocode}
  \def\texosquerylongdstzone#1{%
    \texosquerytimezonefmt
     {\LocaleGetDialectAttribute{\CurrentTrackedDialect}{timezone.#1.longdst}}%
  }
%    \end{macrocode}
%\end{macro}
%End of first argument of \cs{LocaleIfDateTimePatternsSupported}.
%(Do nothing if date-time patterns not required.)
%    \begin{macrocode}
}
{}
%    \end{macrocode}
%
%Set up the \sty{texosquery} numbering commands to use the current
%locale symbols.
%\begin{macro}{\texosquerypatfmtgroupsep}
%    \begin{macrocode}
\def\texosquerypatfmtgroupsep{\CurrentLocaleNumericGroupSep}
%    \end{macrocode}
%\end{macro}
%
%\begin{macro}{\texosquerypatfmtdecsep}
%    \begin{macrocode}
\def\texosquerypatfmtdecsep{\CurrentLocaleDecimalSep}
%    \end{macrocode}
%\end{macro}
%
%\begin{macro}{\texosquerypatfmtcurdecsep}
%    \begin{macrocode}
\def\texosquerypatfmtcurdecsep{\CurrentLocaleMonetarySep}
%    \end{macrocode}
%\end{macro}
%
%\begin{macro}{\texosquerypatfmtexp}
%    \begin{macrocode}
\def\texosquerypatfmtexp{\CurrentLocaleExponent}
%    \end{macrocode}
%\end{macro}
%
%\begin{macro}{\texosquerypatfmtcurrencysign}
%    \begin{macrocode}
\def\texosquerypatfmtcurrencysign{\CurrentLocaleCurrency}
%    \end{macrocode}
%\end{macro}
%
%\begin{macro}{\texosquerypatfmtpercentsign}
%    \begin{macrocode}
\def\texosquerypatfmtpercentsign{\CurrentLocalePercent}
%    \end{macrocode}
%\end{macro}
%
%\begin{macro}{\texosquerypatfmtpermillsign}
%    \begin{macrocode}
\def\texosquerypatfmtpermillsign{\CurrentLocalePermill}
%    \end{macrocode}
%\end{macro}
%
%\subsection{Conditionals}
%\begin{macro}{\LocaleIfNumericUsesGroup}
%\begin{definition}
%\cs{LocaleIfNumericUsesGroup}\marg{dialect}\marg{true}\marg{false}
%\end{definition}
%Determines if the locale uses numerical group separator. The
%\texttt{usesgroup} attribute will be 1 for true and 0 for false, so
%we need a numerical comparison.
%    \begin{macrocode}
\def\LocaleIfNumericUsesGroup#1{%
 \LocaleIfDialectAttributeEqNum{#1}{usesgroup}{1}%
}
%    \end{macrocode}
%\end{macro}
%
%\begin{macro}{\LocaleIfHasLanguageName}
%\begin{definition}
%\cs{LocaleIfHasLanguageName}\marg{dialect}\marg{true}\marg{false}
%\end{definition}
%Determines if the given dialect has a language name.
%    \begin{macrocode}
\def\LocaleIfHasLanguageName#1{%
 \LocaleIfHasDialectNonEmptyAttribute{#1}{langname}%
}
%    \end{macrocode}
%\end{macro}
%No need for a separate check for the native version. If it has
%one, then it will have the other.
%
%\begin{macro}{\LocaleIfHasRegionName}
%\begin{definition}
%\cs{LocaleIfHasRegionName}\marg{dialect}\marg{true}\marg{false}
%\end{definition}
%Determines if the given dialect has a region name.
%    \begin{macrocode}
\def\LocaleIfHasRegionName#1{%
 \LocaleIfHasDialectNonEmptyAttribute{#1}{regionname}%
}
%    \end{macrocode}
%\end{macro}
%
%\begin{macro}{\LocaleIfHasVariantName}
%\begin{definition}
%\cs{LocaleIfHasVariantName}\marg{dialect}\marg{true}\marg{false}
%\end{definition}
%Determines if the given dialect has a variant name.
%    \begin{macrocode}
\def\LocaleIfHasVariantName#1{%
 \LocaleIfHasDialectNonEmptyAttribute{#1}{variantname}%
}
%    \end{macrocode}
%\end{macro}
%
%\subsection{Post-Parsing}
%If this hook has been defined, use it.
%    \begin{macrocode}
\ifx\@locale@postparse@hook\undefined
\else
  \@locale@postparse@hook
\fi
%    \end{macrocode}
%
%\subsection{Initialising Current Locale Commands}
%These commands are all initialised to do nothing (except for
%indexes which are initialised to \texttt{-1}), in the event that
%the dry run mode is on.
%    \begin{macrocode}
\def\CurrentLocaleLanguageName{}%
\def\CurrentLocaleLanguageNativeName{}%
\def\CurrentLocaleRegionName{}%
\def\CurrentLocaleRegionNativeName{}%
\def\CurrentLocaleVariantName{}%
\def\CurrentLocaleVariantNativeName{}%
\def\CurrentLocaleFirstDayIndex{-1}%
\def\CurrentLocaleDayIndexFromRegion{-1}%
\def\CurrentLocaleDayName{}%
\def\CurrentLocaleShortDayName{}%
\def\CurrentLocaleStandaloneDayName{}%
\def\CurrentLocaleStandaloneShortDayName{}%
\def\CurrentLocaleMonthName{}%
\def\CurrentLocaleShortMonthName{}%
\def\CurrentLocaleStandaloneMonthName{}%
\def\CurrentLocaleStandaloneShortMonthName{}%
\def\CurrentLocaleFullDate{}%
\def\CurrentLocaleLongDate{}%
\def\CurrentLocaleMediumDate{}%
\def\CurrentLocaleShortDate{}%
\def\CurrentLocaleFullTime{}%
\def\CurrentLocaleLongTime{}%
\def\CurrentLocaleMediumTime{}%
\def\CurrentLocaleShortTime{}%
\def\CurrentLocaleFullDateTime{}%
\def\CurrentLocaleLongDateTime{}%
\def\CurrentLocaleMediumDateTime{}%
\def\CurrentLocaleShortDateTime{}%
\def\CurrentLocaleDate{}%
\def\CurrentLocaleTime{}%
\def\CurrentLocaleCurrency{}%
\def\CurrentLocaleNumericGroupSep{}%
\def\CurrentLocaleIfNumericUsesGroup{}%
\def\CurrentLocaleDecimalSep{}%
\def\CurrentLocaleMonetarySep{}%
\def\CurrentLocaleExponent{}%
\def\CurrentLocalePercent{}%
\def\CurrentLocalePermill{}%
\def\CurrentLocaleIntegerPattern{}%
\def\CurrentLocaleDecimalPattern{}%
\def\CurrentLocaleCurrencyPattern{}%
\def\CurrentLocalePercentPattern{}%
\def\CurrentLocaleApplyDateTimePattern{}%
\def\CurrentTrackedDialect{}%
%    \end{macrocode}
%
%\subsection{Switching Locale}
%\begin{macro}{\@locale@select}
%Used when a language is changed. The argument is the dialect label.
%    \begin{macrocode}
\def\@locale@select#1{%
  \SetCurrentTrackedDialect{#1}%
  \def\CurrentLocaleLanguageName{\LocaleLanguageName{#1}}%
  \def\CurrentLocaleLanguageNativeName{\LocaleLanguageNativeName{#1}}%
  \def\CurrentLocaleRegionName{\LocaleRegionName{#1}}%
  \def\CurrentLocaleRegionNativeName{\LocaleRegionNativeName{#1}}%
  \def\CurrentLocaleVariantName{\LocaleVariantName{#1}}%
  \def\CurrentLocaleVariantNativeName{\LocaleVariantNativeName{#1}}%
  \def\CurrentLocaleFirstDayIndex{\LocaleFirstDayIndex{#1}}%
  \def\CurrentLocaleDayIndexFromRegion{\LocaleDayIndexFromRegion{#1}}%
  \def\CurrentLocaleDayName{\LocaleDayName{#1}}%
  \def\CurrentLocaleShortDayName{\LocaleShortDayName{#1}}%
  \def\CurrentLocaleStandaloneDayName{%
    \LocaleStandaloneDayName{#1}}%
  \def\CurrentLocaleStandaloneShortDayName{%
    \LocaleStandaloneShortDayName{#1}}%
  \def\CurrentLocaleMonthName{\LocaleMonthName{#1}}%
  \def\CurrentLocaleShortMonthName{\LocaleShortMonthName{#1}}%
  \def\CurrentLocaleStandaloneMonthName{%
    \LocaleStandaloneMonthName{#1}}%
  \def\CurrentLocaleStandaloneShortMonthName{%
    \LocaleStandaloneShortMonthName{#1}}%
  \def\CurrentLocaleFullDate{\LocaleFullDate{#1}}%
  \def\CurrentLocaleLongDate{\LocaleLongDate{#1}}%
  \def\CurrentLocaleMediumDate{\LocaleMediumDate{#1}}%
  \def\CurrentLocaleShortDate{\LocaleShortDate{#1}}%
  \def\CurrentLocaleFullTime{\LocaleFullTime{#1}}%
  \def\CurrentLocaleLongTime{\LocaleLongTime{#1}}%
  \def\CurrentLocaleMediumTime{\LocaleMediumTime{#1}}%
  \def\CurrentLocaleShortTime{\LocaleShortTime{#1}}%
  \def\CurrentLocaleFullDateTime{\LocaleFullDateTime{#1}}%
  \def\CurrentLocaleLongDateTime{\LocaleLongDateTime{#1}}%
  \def\CurrentLocaleMediumDateTime{\LocaleMediumDateTime{#1}}%
  \def\CurrentLocaleShortDateTime{\LocaleShortDateTime{#1}}%
  \def\CurrentLocaleDate{%
    \localedatechoice
     {\LocaleFullDate{#1}}%
     {\LocaleLongDate{#1}}%
     {\LocaleMediumDate{#1}}%
     {\LocaleShortDate{#1}}%
  }%
  \def\CurrentLocaleTime{%
    \localetimechoice
     {\LocaleFullTime{#1}}%
     {\LocaleLongTime{#1}}%
     {\LocaleMediumTime{#1}}%
     {\LocaleShortTime{#1}}%
  }%
  \def\CurrentLocaleCurrency{%
    \localecurrchoice
     {\LocaleCurrencyLabel{#1}}%
     {\LocaleCurrencyRegionalLabel{#1}}%
     {\LocaleCurrencySymbol{#1}}%
     {\LocaleCurrencyTeXSymbol{#1}}%
  }%
  \def\CurrentLocaleNumericGroupSep{%
    \LocaleNumericGroupSep{#1}%
  }%
  \def\CurrentLocaleIfNumericUsesGroup{%
    \LocaleIfNumericUsesGroup{#1}%
  }%
  \def\CurrentLocaleDecimalSep{%
    \LocaleNumericDecimalSep{#1}%
  }%
  \def\CurrentLocaleMonetarySep{%
    \LocaleNumericMonetarySep{#1}%
  }%
  \def\CurrentLocaleExponent{%
    \LocaleNumericExponent{#1}%
  }%
  \def\CurrentLocalePercent{%
    \LocaleNumericPercent{#1}%
  }%
  \def\CurrentLocalePermill{%
    \LocaleNumericPermill{#1}%
  }%
  \def\CurrentLocaleIntegerPattern{%
    \LocaleGetDialectAttribute{#1}{intfmt}%
  }%
  \def\CurrentLocaleDecimalPattern{%
    \LocaleGetDialectAttribute{#1}{decfmt}%
  }%
  \def\CurrentLocaleCurrencyPattern{%
    \LocaleGetDialectAttribute{#1}{curfmt}%
  }%
  \def\CurrentLocalePercentPattern{%
    \LocaleGetDialectAttribute{#1}{perfmt}%
  }%
  \def\CurrentLocaleApplyDateTimePattern{%
    \LocaleApplyDateTimePattern{#1}%
  }%
}
%    \end{macrocode}
%\end{macro}
%
%Provide the choice commands:
%\begin{macro}{\localedatechoice}
%    \begin{macrocode}
\ifx\localedatechoice\undefined
  \def\localedatechoice#1#2#3#4{#2}
\fi
%    \end{macrocode}
%\end{macro}
%
%\begin{macro}{\localetimechoice}
%    \begin{macrocode}
\ifx\localetimechoice\undefined
  \def\localetimechoice#1#2#3#4{#3}
\fi
%    \end{macrocode}
%\end{macro}
%
%\begin{macro}{\localecurrchoice}
%    \begin{macrocode}
\ifx\localecurrchoice\undefined
  \def\localecurrchoice#1#2#3#4{#2}
\fi
%    \end{macrocode}
%\end{macro}
%
%\begin{macro}{\CurrentLocaleDateTime}
%Shortcut for date and time.
%    \begin{macrocode}
\def\CurrentLocaleDateTime{\CurrentLocaleDate\space\CurrentLocaleTime}
%    \end{macrocode}
%\end{macro}
%
%Provide convenient shortcut commands for formatting numbers.
%These need a bit of help to split the argument on "."
%and on "E". First provide a general purpose command.
%
%\begin{macro}{\localenumfmt}
%The first argument is the pattern. The second argument is the
%value.
%    \begin{macrocode}
\def\localenumfmt#1#2{%
  \@locale@decfmt#2\empty.0E0\relax
 \ifnum\@locale@decfmt@int<0
   \let\@localenum@fmt\localenumfmtneg
 \else
   \ifnum\@locale@decfmt@int=0
     \ifnum\@locale@decfmt@frac=0
       \ifnum\@locale@decfmt@exp=0
         \let\@localenum@fmt\localenumfmtzero
       \else
         \let\@localenum@fmt\localenumfmtpos
       \fi
     \else
       \let\@localenum@fmt\localenumfmtpos
     \fi
   \else
     \let\@localenum@fmt\localenumfmtpos
   \fi
 \fi
 \@localenum@fmt
 {%
   \texosqueryfmtnumber{#1}%
   {\@locale@decfmt@int}%
   {\@locale@decfmt@frac}%
   {\@locale@decfmt@exp}%
 }%
}
%    \end{macrocode}
%\end{macro}
%
%\begin{macro}{\localenumfmtpos}
%Wrapper for positive values.
%    \begin{macrocode}
\def\localenumfmtpos#1{#1}
%    \end{macrocode}
%\end{macro}
%
%\begin{macro}{\localenumfmtneg}
%Wrapper for negative values.
%    \begin{macrocode}
\def\localenumfmtneg#1{#1}
%    \end{macrocode}
%\end{macro}
%
%\begin{macro}{\localenumfmtzero}
%Wrapper for zero values.
%    \begin{macrocode}
\def\localenumfmtzero#1{#1}
%    \end{macrocode}
%\end{macro}
%
%\begin{macro}{\localeint}
%    \begin{macrocode}
\def\localeint#1{%
 \localenumfmt{\CurrentLocaleIntegerPattern}{#1}%
}
%    \end{macrocode}
%\end{macro}
%
%
%\begin{macro}{\localedec}
%    \begin{macrocode}
\def\localedec#1{%
 \localenumfmt{\CurrentLocaleDecimalPattern}{#1}%
}
%    \end{macrocode}
%\end{macro}
%
%\begin{macro}{\localecur}
%    \begin{macrocode}
\def\localecur#1{%
 \localenumfmt{\CurrentLocaleCurrencyPattern}{#1}%
}
%    \end{macrocode}
%\end{macro}
%
%\begin{macro}{\localeper}
%    \begin{macrocode}
\def\localeper#1{%
 \localenumfmt{\CurrentLocalePercentPattern}{#1}%
}
%    \end{macrocode}
%\end{macro}
%
%\begin{macro}{\@locale@decfmt}
%    \begin{macrocode}
\def\@locale@decfmt#1.#2E#3\relax{%
%    \end{macrocode}
%Allow for mantissa present but missing fractional part.
%    \begin{macrocode}
 \@locale@decfmt@split#1E\empty\relax
 \ifx\@locale@decfmt@exp\empty
   \edef\@locale@decfmt@frac{\@locale@gobbleemptytorelax#2\empty\relax}%
   \edef\@locale@decfmt@exp{\@locale@gobbleemptytorelax#3\empty\relax}%
 \else
   \def\@locale@decfmt@frac{0}%
 \fi
}
%    \end{macrocode}
%\end{macro}
%
%\begin{macro}{\@locale@gobbleemptytorelax}
%Strip everything between \cs{empty} and \cs{relax}
%    \begin{macrocode}
\def\@locale@gobbleemptytorelax#1\empty#2\relax{#1}
%    \end{macrocode}
%\end{macro}
%
%\begin{macro}{\@locale@decfmt@split}
%Split on "E". (In case "E" but no "." supplied.)
%    \begin{macrocode}
\def\@locale@decfmt@split#1E#2\empty#3\relax{%
 \edef\@locale@decfmt@int{\@locale@gobbleemptytorelax#1\empty\relax}%
 \edef\@locale@decfmt@exp{\@locale@gobbleemptytorelax#2\empty\relax}%
}
%    \end{macrocode}
%\end{macro}
%
%Iterate through all languages and add to caption hook.
%    \begin{macrocode}
\ForEachTrackedDialect{\locale@this@dialect}{%
  \SetCurrentTrackedDialect{\locale@this@dialect}%
  \expandafter\@TrackLangAddToHook\expandafter
   {\expandafter\@locale@select\expandafter{\locale@this@dialect}}
   {captions}%
}
%    \end{macrocode}
%
%Provide a command to select a language using the \sty{tracklang}
%dialect label or the language tag.
%\begin{macro}{\selectlocale}
%    \begin{macrocode}
\def\selectlocale#1{%
  \ifx\selectlanguage\undefined
    \IfTrackedDialect{#1}%
    {\@locale@select{#1}}%
    {%
      \LocaleIfHasAttribute{#1}{tagtodialect}
      {%
        \edef\@locale@dialect{\LocaleGetAttribute{#1}{tagtodialect}}%
        \expandafter\@locale@select\expandafter{\@locale@dialect}%
      }%
      {%
        \ifTeXOSQueryDryRun
          \@locale@warn{Unknown locale `#1'}%
        \else
          \@locale@err{Unknown locale `#1'}%
          {The argument to \string\selectlocale\space must be either a
           tracklang dialect label or a tracked dialect language tag}%
        \fi
      }%
    }%
  \else
%    \end{macrocode}
% Need to find the correct label to use in \cs{selectlanguage}
%    \begin{macrocode}
    \IfTrackedDialect{#1}%
    {%
      \@locale@select@dialect{#1}%
    }%
    {%
      \LocaleIfHasAttribute{#1}{tagtodialect}
      {\@locale@select@dialect{\LocaleGetAttribute{#1}{tagtodialect}}}%
      {%
        \ifTeXOSQueryDryRun
          \@locale@warn{Unknown locale `#1'}%
        \else
          \@locale@err{Unknown locale `#1'}%
          {The argument to \string\selectlocale\space must be either a
           tracklang dialect label or a tracked dialect language tag}%
        \fi
      }%
    }%
  \fi
}
%    \end{macrocode}
%\end{macro}
%
%\begin{macro}{\@locale@select@dialect}
%Select language from \sty{tracklang} dialect label.
%    \begin{macrocode}
\def\@locale@select@dialect#1{%
  \IfTrackedDialectHasMapping{#1}%
  {\edef\@locale@dialect{\GetTrackedDialectToMapping{#1}}}%
  {\edef\@locale@dialect{#1}}%
%    \end{macrocode}
% Is there a date hook available? (Match the check used by
% \cs{selectlanguage}.)
%    \begin{macrocode}
  \@tracklang@ifundef{date\@locale@dialect}%
  {%
%    \end{macrocode}
% Try the root language.
%    \begin{macrocode}
    \edef\@locale@dialect{\TrackedLanguageFromDialect{#1}}%
%    \end{macrocode}
% Is there a date hook available?
%    \begin{macrocode}
    \@tracklang@ifundef{date\@locale@dialect}%
    {%
      \@locale@warn{Can't determine correct label for
       \string\selectlanguage\space from tracklang dialect `#1'}%
      \SetCurrentTrackedDialect{#1}%
    }%
    {\expandafter\selectlanguage\expandafter{\@locale@dialect}}%
  }%
  {\expandafter\selectlanguage\expandafter{\@locale@dialect}}%
}
%    \end{macrocode}
%\end{macro}
%
%Select the main locale (but don't try any language change
%as this might upset \sty{polyglossia} if the fonts haven't been set
%yet), so use the internal
%\cs{@locale@select}.
%    \begin{macrocode}
\ifx\LocaleMainDialect\empty
\else
  \expandafter\@locale@select\expandafter{\LocaleMainDialect}
\fi
%    \end{macrocode}
%
%Restore category code for \texttt{@} if necessary.
%    \begin{macrocode}
\@locale@restore@at
%    \end{macrocode}

%\iffalse
%    \begin{macrocode}
%</tex-locale.tex>
%    \end{macrocode}
%\fi
%\iffalse
%    \begin{macrocode}
%<*tex-locale-scripts-enc.def>
%    \end{macrocode}
%\fi
%\section{Scripts to \styfmt{fontenc} mappings
%(\texttt{tex-locale-scripts-enc.def})}
%\label{sec:locale-scripts-enc.def}
%Provides mappings from scripts to font encodings. Not all
%supported. This file is only loaded if the \pkgopt[auto]{fontenc}
%option is used. The \texttt{@} character is assumed to have
%category code when this file is input.
%\begin{macro}{\@locale@scriptenc@map}
%Maps ISO 15924 script label to \LaTeX\ font encoding name.
%    \begin{macrocode}
\def\@locale@scriptenc@map#1#2{%
  \@tracklang@namedef{@locale@scriptenc@map@#1}{#2}%
}
%    \end{macrocode}
%\end{macro}
%\begin{macro}{\@locale@get@scriptenc@map}
%Get the mapping.
%    \begin{macrocode}
\def\@locale@get@scriptenc@map#1{%
  \@tracklang@ifundef{@locale@scriptenc@map@#1}%
  {}%
  {\csname @locale@scriptenc@map@#1\endcsname}%
}
%    \end{macrocode}
%\end{macro}
%\begin{macro}{\@locale@if@scriptenc@map}
%Test if there's a script mapping.
%    \begin{macrocode}
\def\@locale@if@scriptenc@map#1#2#3{%
  \@tracklang@ifundef{@locale@scriptenc@map@#1}%
  {#3}% false
  {#2}% true
}
%    \end{macrocode}
%\end{macro}
%
%\begin{macro}{\@locale@langenc@map}
%Maps tracklang language or dialect labels to \LaTeX\ font encoding name.
%    \begin{macrocode}
\def\@locale@langenc@map#1#2{%
  \@tracklang@namedef{@locale@langenc@map@#1}{#2}%
}
%    \end{macrocode}
%\end{macro}
%
%\begin{macro}{\@locale@if@langenc@map}
%Test if there's a lang mapping.
%    \begin{macrocode}
\def\@locale@if@langenc@map#1#2#3{%
  \@tracklang@ifundef{@locale@langenc@map@#1}%
  {#3}% false
  {#2}% true
}
%    \end{macrocode}
%\end{macro}
%
%\begin{macro}{\@locale@get@langenc@map}
%Get the mapping.
%    \begin{macrocode}
\def\@locale@get@langenc@map#1{%
  \@tracklang@ifundef{@locale@langenc@map@#1}%
  {}%
  {\csname @locale@langenc@map@#1\endcsname}%
}
%    \end{macrocode}
%\end{macro}
%
%Define mappings. Only a few currently supported. It's better to use
%\sty{fontspec} instead of \sty{fontenc} for non-Latin scripts.
%
% Scripts first.
%    \begin{macrocode}
\@locale@scriptenc@map{Latn}{T1}
\@locale@scriptenc@map{Latf}{T1}
\@locale@scriptenc@map{Latg}{T1}
\@locale@scriptenc@map{Cyrl}{T2A,T2B,T2C}
%    \end{macrocode}
%Now languages or dialect labels:
%    \begin{macrocode}
\@locale@langenc@map{vietnamese}{T5}
\@locale@langenc@map{polish}{OT4}
\@locale@langenc@map{armenian}{OT6}
\@locale@langenc@map{greek}{LGR}
%    \end{macrocode}
%\iffalse
%    \begin{macrocode}
%</tex-locale-scripts-enc.def>
%    \end{macrocode}
%\fi
%\iffalse
%    \begin{macrocode}
%<*tex-locale-encodings.def>
%    \end{macrocode}
%\fi
%\section{texosquery to \styfmt{inputenc} mappings
%(\texttt{tex-locale-encodings.def})}
%Encoding mappings.
%\begin{macro}{\@locale@newencmap}
%    \begin{macrocode}
\def\@locale@newencmap#1#2{%
  \@tracklang@namedef{@locale@encmap@#1}{#2}}
%    \end{macrocode}
%\end{macro}
%\begin{macro}{\@locale@ifhasencmap}
%    \begin{macrocode}
\def\@locale@ifhasencmap#1#2#3{%
  \@tracklang@ifundef{@locale@encmap@#1}{#3}{#2}}
%    \end{macrocode}
%\end{macro}
%\begin{macro}{\@locale@getencmap}
%    \begin{macrocode}
\def\@locale@getencmap#1{%
  \@tracklang@nameuse{@locale@encmap@#1}}
%    \end{macrocode}
%\end{macro}
%    \begin{macrocode}
\@locale@newencmap{iso88591}{latin1}
\@locale@newencmap{iso88592}{latin2}
\@locale@newencmap{iso88593}{latin3}
\@locale@newencmap{iso88594}{latin4}
\@locale@newencmap{iso88595}{latin5}
\@locale@newencmap{iso88599}{latin9}
\@locale@newencmap{windows1250}{cp1250}
\@locale@newencmap{windows1252}{cp1252}
\@locale@newencmap{windows1257}{cp1257}
\@locale@newencmap{ibm850}{cp850}
\@locale@newencmap{ibm852}{cp852}
\@locale@newencmap{ibm437}{cp437}
\@locale@newencmap{ibm865}{cp865}
\@locale@newencmap{usascii}{ascii}
\@locale@newencmap{xmaccentraleurope}{macce}
%    \end{macrocode}
%\iffalse
%    \begin{macrocode}
%</tex-locale-encodings.def>
%    \end{macrocode}
%\fi
%\iffalse
%    \begin{macrocode}
%<*tex-locale-support.def>
%    \end{macrocode}
%\fi
%\section{Language Support Setup}
%The language support setup performed in \cs{@locale@postparse@hook}.
%Load \sty{babel} or \sty{polyglossia}.
%    \begin{macrocode}
\@locale@ifsupportbabelorpoly
{}
{%
%    \end{macrocode}
%(\sty{babel} support.)
%Set up some extra mappings that aren't in \sty{tracklang} version
%1.3. There's no intuitive way of testing if a dialect label is
%provided by \sty{babel}. There's only a test for the root language
%by checking for the existence of the ldf file, so these are known
%\sty{babel} options. This means that if there isn't a mapping, the
%root language will be used instead.
%    \begin{macrocode}
  \def\@locale@providemap#1#2{%
    \IfTrackedDialectHasMapping{#1}%
    {}%
    {\SetTrackedDialectLabelMap{#1}{#2}}%
  }%
  \@locale@providemap{bahasa}{bahasa}%
  \@locale@providemap{indonesian}{indonesian}%
  \@locale@providemap{indon}{indon}%
  \@locale@providemap{bahasam}{bahasam}%
  \@locale@providemap{malay}{malay}%
  \@locale@providemap{melayu}{melayu}%
  \@locale@providemap{USenglish}{USenglish}%
  \@locale@providemap{american}{american}%
  \@locale@providemap{UKenglish}{UKenglish}%
  \@locale@providemap{british}{british}%
  \@locale@providemap{canadian}{canadian}%
  \@locale@providemap{australian}{australian}%
  \@locale@providemap{newzealand}{newzealand}%
  \@locale@providemap{francais}{francais}%
  \@locale@providemap{canadien}{canadien}%
  \@locale@providemap{acadian}{acadian}%
  \@locale@providemap{austrian}{austrian}%
  \@locale@providemap{germanb}{germanb}%
  \@locale@providemap{ngerman}{ngerman}%
  \@locale@providemap{naustrian}{naustrian}%
  \@locale@providemap{nswissgerman}{nswissgerman}%
  \@locale@providemap{swissgerman}{swissgerman}%
  \@locale@providemap{polutonikogreek}{greek}%
  \@locale@providemap{nynorsk}{nynorsk}%
  \@locale@providemap{portuguese}{portuguese}%
  \@locale@providemap{brazilian}{brazilian}%
  \@locale@providemap{brazil}{brazil}%
%    \end{macrocode}
%Work out how to pass the options since we may need to convert a
%\sty{tracklang} dialect label to a recognised \sty{babel} label.
%    \begin{macrocode}
     \def\@locale@bbl@options{}%
     \ForEachTrackedDialect{\this@dialect}%
     {%
       \edef\this@root@lang{%
         \TrackedLanguageFromDialect{\this@dialect}}%
%    \end{macrocode}
%Is there a language file?
%    \begin{macrocode}
       \IfFileExists{\this@root@lang.ldf}
       {%
%    \end{macrocode}
%Check for Serbian, since we need to know the script.
%    \begin{macrocode}
         \ifdefstring{\this@root@lang}{serbian}%
         {%
           \@locale@loadscripts
           \IfTrackedDialectIsScriptCs{\this@dialect}%
           {\TrackLangScriptLatn}%
           {\def\locale@bbl@dialect{serbian}}%
           {\def\locale@bbl@dialect{serbianc}}%
         }%
         {%
           \IfFileExists{\this@dialect.ldf}%
           {%
             \let\locale@bbl@dialect\this@dialect
           }%
           {%
%    \end{macrocode}
%Do we have a mapping for this dialect label?
%    \begin{macrocode}
             \IfTrackedDialectHasMapping{\this@dialect}%
             {%
               \edef\locale@bbl@dialect{%
                 \GetTrackedDialectToMapping{\this@dialect}}%
             }%
             {%
               \let\locale@bbl@dialect\this@root@lang
             }%
           }%
         }%
         \SetTrackedDialectLabelMap{\this@dialect}{\locale@bbl@dialect}%
         \ifx\this@dialect\LocaleMainDialect
           \ifx\@locale@bbl@options\empty
             \edef\@locale@bbl@options{main=\locale@bbl@dialect}%
           \else
             \edef\@locale@bbl@options{\@locale@bbl@options,%
               main=\locale@bbl@dialect}%
           \fi
         \else
           \ifx\@locale@bbl@options\empty
             \edef\@locale@bbl@options{\locale@bbl@dialect}%
           \else
             \edef\@locale@bbl@options{\@locale@bbl@options,%
               \locale@bbl@dialect}%
           \fi
         \fi
       }%
       {}%
     }%
     \ifx\@locale@bbl@options\@empty
       \ifTeXOSQueryDryRun
         \@locale@err{Can't determine `babel' package
          options\MessageBreak (texosquery's dry run mode is on)}{}
       \else
         \@locale@err{Can't determine `babel' package
          options (perhaps the shell escape failed, check
          `\jobname.log')}{}
       \fi
     \else
        \expandafter\PassOptionsToPackage\expandafter
           {\@locale@bbl@options}{babel}
        \RequirePackage{babel}%
     \fi
}%
{%
%    \end{macrocode}
% (\sty{polyglossia} support).
%    \begin{macrocode}
%    \end{macrocode}
% Need a way of mapping scripts, regions and variants to known
% \sty{polyglossia} keys. A lot of these variants aren't official
% BCP 47 tags.
%    \begin{macrocode}
   \def\@set@locale@poly@map@script#1#2#3{%
     \@tracklang@namedef{@local@poly@map@script@#1@#2}{#3}%
   }
   \def\@if@locale@poly@map@script#1#2#3#4{%
     \@tracklang@ifundef{@local@poly@map@script@#1@#2}{#4}{#3}%
   }
   \def\@get@locale@poly@map@script#1#2{%
     \@tracklang@nameuse{@local@poly@map@script@#1@#2}%
   }
   \def\@set@locale@poly@map@region#1#2#3{%
     \@tracklang@namedef{@local@poly@map@region@#1@#2}{#3}%
   }
   \def\@if@locale@poly@map@region#1#2#3#4{%
     \@tracklang@ifundef{@local@poly@map@region@#1@#2}{#4}{#3}%
   }
   \def\@get@locale@poly@map@region#1#2{%
     \@tracklang@nameuse{@local@poly@map@region@#1@#2}%
   }
   \def\@set@locale@poly@map@variant#1#2#3{%
     \@tracklang@namedef{@local@poly@map@variant@#1@#2}{#3}%
   }
   \def\@if@locale@poly@map@variant#1#2#3#4{%
     \@tracklang@ifundef{@local@poly@map@variant@#1@#2}{#4}{#3}%
   }
   \def\@get@locale@poly@map@variant#1#2{%
     \@tracklang@nameuse{@local@poly@map@variant@#1@#2}%
   }
   \def\@set@locale@poly@map@sublang#1#2#3{%
     \@tracklang@namedef{@local@poly@map@sublang@#1@#2}{#3}%
   }
   \def\@if@locale@poly@map@sublang#1#2#3#4{%
     \@tracklang@ifundef{@local@poly@map@sublang@#1@#2}{#4}{#3}%
   }
   \def\@get@locale@poly@map@sublang#1#2{%
     \@tracklang@nameuse{@local@poly@map@sublang@#1@#2}%
   }
   \@set@locale@poly@map@region{arabic}{IQ}{locale=mashriq}
   \@set@locale@poly@map@region{arabic}{SY}{locale=mashriq}
   \@set@locale@poly@map@region{arabic}{JO}{locale=mashriq}
   \@set@locale@poly@map@region{arabic}{LB}{locale=mashriq}
   \@set@locale@poly@map@region{arabic}{PS}{locale=mashriq}
   \@set@locale@poly@map@region{arabic}{LY}{locale=libya}
   \@set@locale@poly@map@region{arabic}{DZ}{locale=algeria}
   \@set@locale@poly@map@region{arabic}{TN}{locale=tunisia}
   \@set@locale@poly@map@region{arabic}{MA}{locale=morocco}
   \@set@locale@poly@map@region{arabic}{MR}{locale=mauritania}
   \@set@locale@poly@map@variant{arabic}{islamic}{calendar=islamic}
   \@set@locale@poly@map@variant{arabic}{maghrib}{numerals=maghrib}
   \@set@locale@poly@map@variant{arabic}{abjad}{abjadjimnotail}
   \@set@locale@poly@map@variant{bengali}{western}{numerals=Western}
   \@set@locale@poly@map@variant{bengali}{devanagari}{numerals=Devanagari}
   \@set@locale@poly@map@variant{bengali}{bengali}{numerals=Bengali}
   \@set@locale@poly@map@region{english}{US}{variant=us}
   \@set@locale@poly@map@region{english}{GB}{variant=uk}
   \@set@locale@poly@map@region{english}{AU}{variant=australian}
   \@set@locale@poly@map@region{english}{NZ}{variant=newzealand}
   \@set@locale@poly@map@variant{farsi}{western}{numerals=western}
   \@set@locale@poly@map@variant{farsi}{eastern}{numerals=eastern}
   \@set@locale@poly@map@region{german}{DE}{variant=german}
   \@set@locale@poly@map@region{german}{AU}{variant=austrian}
   \@set@locale@poly@map@region{german}{CH}{variant=swiss}
   \@set@locale@poly@map@variant{german}{1996}{spelling=new}
   \@set@locale@poly@map@variant{german}{1901}{spelling=old}
   \@set@locale@poly@map@script{german}{Latf}{script=fraktur}
   \@set@locale@poly@map@variant{greek}{monoton}{variant=monotonic}
   \@set@locale@poly@map@variant{greek}{polyton}{variant=polytonic}
   \@set@locale@poly@map@variant{greek}{ancient}{variant=ancient}
   \@set@locale@poly@map@variant{greek}{arabic}{numerals=arabic}
   \@set@locale@poly@map@variant{hebrew}{arabic}{numerals=arabic}
   \@set@locale@poly@map@variant{hebrew}{gregorian}{calendar=gregorian}
   \@set@locale@poly@map@variant{hindi}{western}{numerals=Western}
   \@set@locale@poly@map@variant{hindi}{devanagari}{numerals=Devanagari}
   \@set@locale@poly@map@variant{latin}{classic}{variant=classic}
   \@set@locale@poly@map@variant{latin}{modern}{variant=modern}
   \@set@locale@poly@map@variant{latin}{medieval}{variant=medieval}
   \@set@locale@poly@map@sublang{russian}{orv}{spelling=old}
   \@set@locale@poly@map@variant{russian}{luna1918}{spelling=new}
   \@set@locale@poly@map@script{serbian}{Latn}{script=Latin}
   \@set@locale@poly@map@script{serbian}{Cyrl}{script=Cyrillic}
   \@set@locale@poly@map@variant{syriac}{western}{numerals=western}
   \@set@locale@poly@map@variant{syriac}{eastern}{numerals=eastern}
%    \end{macrocode}
% Load polyglossia.
%    \begin{macrocode}
     \RequirePackage{polyglossia}
     \ForEachTrackedDialect{\this@dialect}%
     {%
       \edef\this@root@lang{%
         \TrackedLanguageFromDialect{\this@dialect}}%
       \edef\this@sublang{%
         \GetTrackedDialectSubLang{\this@dialect}}%
       \edef\this@region{%
         \TrackedIsoCodeFromLanguage{3166-1}{\this@dialect}}%
       \edef\this@script{%
         \GetTrackedDialectScript{\this@dialect}}%
       \edef\this@variant{%
         \GetTrackedDialectVariant{\this@dialect}}%
%    \end{macrocode}
% Try to determine the options. Check the script mappings.
%    \begin{macrocode}
     \def\@locale@poly@options{}%
     \ifx\this@script\empty
     \else
       \@if@locale@poly@map@script{\this@root@lang}{\this@script}%
       {%
         \edef\@locale@poly@options{%
           \@get@locale@poly@map@script
             {\this@root@lang}{\this@script}}%
       }%
       {}%
     \fi
%    \end{macrocode}
% Check the region mappings.
%    \begin{macrocode}
     \ifx\this@region\empty
     \else
       \@if@locale@poly@map@region{\this@root@lang}{\this@region}%
       {%
         \ifx\@locale@poly@options\empty
           \edef\@locale@poly@options{%
             \@get@locale@poly@map@region{\this@root@lang}{\this@region}}%
         \else
           \edef\@locale@poly@options{\@locale@poly@options,%
             \@get@locale@poly@map@region{\this@root@lang}{\this@region}}%
         \fi
       }%
       {}%
     \fi
%    \end{macrocode}
% Check the sub-language mappings.
%    \begin{macrocode}
     \ifx\this@sublang\empty
     \else
       \@if@locale@poly@map@sublang{\this@root@lang}{\this@sublang}%
       {%
         \ifx\@locale@poly@options\empty
           \edef\@locale@poly@options{%
             \@get@locale@poly@map@sublang
               {\this@root@lang}{\this@sublang}}%
         \else
           \edef\@locale@poly@options{\@locale@poly@options,%
             \@get@locale@poly@map@sublang
               {\this@root@lang}{\this@sublang}}%
         \fi
       }%
       {}%
     \fi
%    \end{macrocode}
% Check the variant mappings.
%    \begin{macrocode}
     \ifx\this@subvariant\empty
     \else
       \@if@locale@poly@map@variant{\this@root@lang}{\this@variant}%
       {%
         \ifx\@locale@poly@options\empty
           \edef\@locale@poly@options{%
             \@get@locale@poly@map@variant
               {\this@root@lang}{\this@variant}}%
         \else
           \edef\@locale@poly@options{\@locale@poly@options,%
             \@get@locale@poly@map@variant
               {\this@root@lang}{\this@variant}}%
         \fi
       }%
       {}%
     \fi
%    \end{macrocode}
% Set the language using either \cs{setmainlanguage} or
% \cs{setotherlanguage}.
%    \begin{macrocode}
     \ifx\this@dialect\LocaleMainDialect
       \edef\@locale@tmp{\noexpand\setmainlanguage
           [\@locale@poly@options]{\this@root@lang}}%
     \else
        \edef\@locale@tmp{\noexpand\setotherlanguage
           [\@locale@poly@options]{\this@root@lang}}%
     \fi
     \@locale@tmp
   }%
%    \end{macrocode}
%Redefine command to temporarily switch off punctuation
%adjustments. For example, when formatting the date or time.
%    \begin{macrocode}
   \def\localenopolypunct{\@locale@nopolypunct}%
}
%    \end{macrocode}
%\iffalse
%    \begin{macrocode}
%</tex-locale-support.def>
%    \end{macrocode}
%\fi
%\Finale
\endinput

% xeplain.tex: macros for nonformatting.  Written 1989--94 by (mostly)
% Karl Berry. Some additions/changes 1997--98 by Adam Lewenberg,
% with subsequent contributions from many people.
% These macros are in the public domain.
% 
% This is the ``extended plain'' TeX format that's described in
% `eplain.texi', which you should have received with this file. 
% It assumes that plain has been loaded.
%
% N.B.: A version number is defined at the beginning and end of this file;
% please change those numbers whenever the file is modified!
% And it's best to rename the file if you're going to distribute a
% modified version.
%
% Some macros were written and/or suggested by Paul Abrahams.
% Other sources (e.g., The TeXbook) are cited at the appropriate places.
% 
%% @texfile{
%%   author = "Karl Berry, Steven Smith, Oleg Katsitadze, and others",
%%   version = "REPLACE-WITH-VERSION",
%%   date = "REPLACE-WITH-DATE",
%%   filename = "xeplain.tex",
%%   email = "bug-eplain@tug.org",
%%   checksum = "REPLACE-WITH-CHECKSUM",
%%   codetable = "ASCII",
%%   supported = "yes",
%%   docstring = "This file defines macros that extend and expand on
%%                plain TeX. eplain.tex is xeplain.tex and the other
%%                source files with comments stripped; see the original
%%                files for author credits, etc.  The original sources
%%                can be found in Eplain sources in your TeX
%%                distribution, on CTAN or on Eplain's home page at
%%                http://tug.org/eplain.  Please base diffs or
%%                other contributions on xeplain.tex, not the
%%                stripped-down eplain.tex.",
%% }
%
%
% Load eplain.tex only once (to avoid using up \new's). 
% \eplain is defined at the end of this file, so we can test \eplain to
% detect whether eplain.tex has been loaded already or not. 
% 
% We use \expandafter because the merge script strips any lines
% with an \endinput.
%
\ifx\eplain\undefined
  \let\next\relax
\else
  \expandafter\let\expandafter\next\csname endinput\endcsname
\fi
\next
% The ifpdf.sty file included below was written by Heiko Oberdiek.
% See the complete source file (e.g., in this distribution) for
% comments.
%% [[[include ifpdf.sty]]]
% 
% 
% Category codes, etc.
%
\def\makeactive#1{\catcode`#1 = \active \ignorespaces}%
\chardef\letter = 11
\chardef\other = 12
% The following two macros were adopted from miniltx.tex of graphics.
\def\makeatletter{%
  \edef\resetatcatcode{\catcode`\noexpand\@\the\catcode`\@\relax}%
  \catcode`\@11\relax
}%
\def\makeatother{%
  \edef\resetatcatcode{\catcode`\noexpand\@\the\catcode`\@\relax}%
  \catcode`\@12\relax
}%
% 
% 
% So we can have user-inaccessible control sequences.
% 
\edef\leftdisplays{\the\catcode`@}%
\catcode`@ = \letter
\let\@eplainoldatcode = \leftdisplays
%
% Save miniscule amounts of memory and time by writing \toks@ii instead
% of \toks2.
\toksdef\toks@ii = 2
% 
% 
% This macro is defined in The TeXbook, but it never made it
% into plain TeX.  \dospecials is defined there, though.
% 
\def\uncatcodespecials{%
   \def\do##1{\catcode`##1 = \other}%
   \dospecials
}%
%
% 
% Here is a way to do \let^^M = \cs, where the \let need not be global.
{%
   \makeactive\^^M %
   \long\gdef\letreturn#1{\let^^M = #1}%
}%
% 
% 
% Swallow parameters, etc.
% 
\let\@eattoken = \relax  % Define this, so \eattoken can be used in \edef.
\def\eattoken{\let\@eattoken = }%
\def\gobble#1{}%
\def\gobbletwo#1#2{}%
\def\gobblethree#1#2#3{}%
%
% We can't just use \empty as the identity function, since then outer
% braces which would supposedly delimit the argument would define a group.
\def\identity#1{#1}%
% 
% True if #1 is the empty string, i.e., called like `\ifempty{}'.
% Use notes:
% So far, \ifempty works in the following cases:
% 1. \ifempty{}\message{empty}\else\message{not empty}\fi --> empty
% 2. \ifempty\undefined\message{empty}\else\message{not empty}\fi --> not empty
%   But does NOT work in the case
% 3. \def\empty{}
%    \ifempty\empty\message{empty}\else\message{not empty}\fi --> not empty
% Question: When should \ifempty be true?
\def\@emptymarkA{\@emptymarkB} 
% The above line suggested by Stanislav Brabec.
\def\ifempty#1{\@@ifempty #1\@emptymarkA\@emptymarkB}%
\def\@@ifempty#1#2\@emptymarkB{\ifx #1\@emptymarkA}%
%
% True if #1 is an integer.  From the UK List of TeX Frequently Asked
% Questions, http://www.tex.ac.uk/cgi-bin/texfaq2html.
\def\@gobbleminus#1{\ifx-#1\else#1\fi}%
\def\ifinteger#1{\ifcat_\ifnum9<1\@gobbleminus#1 _\else A\fi}%
% This is a convenience to be used in places where TeX might be
% skipping tokens, e.g., in conditionals.  Usage:
%   \if\isinteger{<subject of test>}%
%     <deal with integer>%
%   \else
%     <deal with non-integer>%
%   \fi
\def\isinteger{TT\fi\ifinteger}%
%
% Turn a definition into the characters that compose it.  See
% ``Sanitizing control sequences under \write'', by Ron Whitney, TUGboat
% 11(4), p.620.
\def\@gobblemeaning#1:->{}%
\def\sanitize{\expandafter\@gobblemeaning\meaning}%
%
% 
% From p.308 of the TeXbook.  This cannot be used in places where TeX
% might be skipping tokens, e.g., in conditionals.
% 
\def\ifundefined#1{\expandafter\ifx\csname#1\endcsname\relax}%
% 
% 
% \csname constructions come up an awful lot, so we save typing with the
% following.  (But the extra macro expansion does take time, so we don't
% use these in frequently-executed code.)
% 
\def\csn#1{\csname#1\endcsname}%
\def\ece#1#2{\expandafter#1\csname#2\endcsname}%
%
% 
% \expandonce{TOKEN} abbreviates \expandafter\noexpand TOKEN.
% 
\def\expandonce{\expandafter\noexpand}%
%
%
% Don't show our register allocations in the log.
% 
\let\@plainwlog = \wlog
\let\wlog = \gobble
%
% 
% Make it convenient to put newlines in error messages.
% 
\newlinechar = `^^J
% 
%
% Sometimes it is convenient to have everything in the transcript file
% and nothing on the terminal.  We don't just call \tracingall here,
% since that produces some useless output on the terminal.
%
\def\gloggingall{\begingroup \globaldefs = 1 \loggingall \endgroup}%
\def\loggingall{\tracingcommands\tw@\tracingstats\tw@
   \tracingpages\@ne\tracingoutput\@ne\tracinglostchars\@ne
   \tracingmacros\tw@\tracingparagraphs\@ne\tracingrestores\@ne
   \showboxbreadth\maxdimen\showboxdepth\maxdimen
}%
% Show the complete contents of boxes.
%
\def\tracingboxes{\showboxbreadth = \maxdimen \showboxdepth = \maxdimen}%
%
% Don't trace anything, except restore \showbox... to plain's values.
% 
\def\gtracingoff{\begingroup \globaldefs = 1 \tracingoff \endgroup}%
\def\tracingoff{\tracingonline\z@\tracingcommands\z@\tracingstats\z@
  \tracingpages\z@\tracingoutput\z@\tracinglostchars\z@
  \tracingmacros\z@\tracingparagraphs\z@\tracingrestores\z@
  \showboxbreadth5 \showboxdepth3
}%
%
% 
% Definitions to produce actual `{' (et al.) characters in an output
% file via \write.  We omit the line break after the first }, since we
% have no comment character at that point.
% 
\begingroup
  \catcode`\{ = 12 \catcode`\} = 12
  \catcode`\[ = 1 \catcode`\] = 2
  \gdef\lbracechar[{]%
  \gdef\rbracechar[}]%
  \catcode`\% = \other
  \gdef\percentchar[%]\endgroup
%
% 
% Leave horizontal mode (if we're in it), then insert a penalty.
% And conversely.
% 
\def\vpenalty{\ifhmode\par\fi \penalty}%
\def\hpenalty{\ifvmode\leavevmode\fi \penalty}%
%
% 
% Make \else usable in \loop.  From Victor Eijkhout's TeX by Topic (page
% 104).  See also Alois Kabelschacht, TUGboat 8(2), page 184.
%
\def\iterate{%
  \let\loop@next\relax
  \body
  \let\loop@next\iterate
  \fi
  \loop@next
}%
% 
% 
% Add #2 (which is expanded in an \edef) to the end of the definition of
% #1 (which must be a previously-defined control sequence).  This is a
% way to construct simple lists.
% 
\def\edefappend#1#2{%
  \toks@ = \expandafter{#1}%
  \edef#1{\the\toks@ #2}%
}%
%
% 
% \allowhyphens, from TeXbook, p. 395.  Allows following & preceding word
% to be hyphenated.
%
\def\allowhyphens{\nobreak\hskip\z@skip}%
%
% 
% 
%  Hooks.
%
% \hookaction{HOOK}{TOKENS} adds TOKENS to the list of actions for
% HOOK.  We avoid defining a \toks register for each hook, although
% maybe that isn't so important.
% 
% \hookappend and \hookprepend add TOKENS specificially to the end or
% the beginning.  When the argument is used, \toks@ will be the previous
% value of the hook, and \toks@ii the new tokens.
%
\long\def\hookprepend{\@hookassign{\the\toks@ii \the\toks@}}%
\long\def\hookappend{\@hookassign{\the\toks@ \the\toks@ii}}%
\let\hookaction = \hookappend % either one should be ok
%
% 
% \@hookassign{LAST-DEF}{HOOK}{TOKENS} makes \toks@ the previous value
% of HOOK, and \toks@ii TOKENS, and then assigns the new value using
% LASTDEF.  We store the hook in a control sequence \@HOOKhook.
% 
\long\def\@hookassign#1#2#3{%
  % Make \toks@ be the expansion (to one level) of \@HOOKhook, or empty.
  \expandafter\ifx\csname @#2hook\endcsname \relax
    % If \@HOOKhook was undefined, let it be empty.
    \toks@ = {}%
  \else
    % Otherwise, expand it to one level.  We can't just assign from
    % \expandafter{\csname ...} since then the \toks register would
    % contain the control sequence, not its definition.
    \expandafter\let\expandafter\temp \csname @#2hook\endcsname
    \toks@ = \expandafter{\temp}%
  \fi
  \toks2 = {#3}% Don't expand the argument all the way.
  \ece\edef{@#2hook}{#1}%
}%
%
% 
% \hookactiononce{HOOK}\CS adds `\global\let\CS=\relax' to the
% definition of \CS, then adds to HOOK.  Thus, \CS is expanded the next
% time HOOK is called, but then it goes away.  This only works if \CS is
% expandable, though.
% 
\long\def\hookactiononce#1#2{%
  \edefappend#2{\global\let\noexpand#2\relax}
  \hookaction{#1}#2%
}%
%
% 
% \hookrun{HOOKNAME} runs whatever actions have been defined for HOOK.
% 
\def\hookrun#1{%
  \expandafter\ifx\csname @#1hook\endcsname \relax \else
    % Isn't this fun?  We want to get rid of the \fi before expanding
    % the actions, so that they can read what's coming up next.
    \def\temp{\csname @#1hook\endcsname}%
    \expandafter\temp
  \fi
}%
%
% 
% 
%  Properties a la Lisp.
%
% \setproperty{ATOM}{PROPNAME}{VALUE} defines the property PROPNAME on the
% ``atom'' ATOM to have VALUE.
% 
\def\setproperty#1#2#3{\ece\edef{#1@p#2}{#3}}%
\def\setpropertyglobal#1#2#3{\ece\xdef{#1@p#2}{#3}}%
%
% 
% \getproperty{ATOM}{PROPNAME} expands to the value of the property
% PROPNAME on ATOM, or to nothing (i.e., \empty), if the property isn't
% present.
% 
\def\getproperty#1#2{%
  \expandafter\ifx\csname#1@p#2\endcsname\relax
  % then \empty
  \else \csname#1@p#2\endcsname
  \fi
}%
%
% 
% 
%  Macros to support BibTeX are in a separate file, btxmac.tex.
% 
% (They are maintained separately, too, by Oren Patashnik,
% opbibtex@cs.stanford.edu.)  btxmac.tex also defines other macros we
% want to use and make available.
% 
% But not all people want to read the BibTeX macros, because of either
% space or time considerations.  Therefore, we look for \nobibtex,
% which, if defined, causes btxmac.tex not to be read.  But we still
% have to get \tokstostring et al. defined---so eplain.tex contains
% those definitions, automatically edited in from btxmac.tex.  All the
% documentation has been removed, so you must read btxmac.tex if you
% want the comments.
% 
% 
% We want to give a slightly different message than btxmac if no .aux
% file exists (unless the person using us has already define some
% message, possibly empty.)
% 
\ifx\@undefinedmessage\@undefined
  \def\@undefinedmessage
    {No .aux file; I won't warn you about undefined labels.}%
\fi
%
% 
% We use a token register to define all the BibTeX definitions, to avoid
% problems with the \if... constructions when they are conditionally
% read.
% 
%% [[[here is the first set of common definitions from btxmac]]]
\toks0 = {%
  %% [[[here are the BibTeX-specific definitions from btxmac]]]
}%
\ifx\nobibtex\@undefined \the\toks0 \fi
%% [[[here is the second set of common definitions from btxmac]]]
%
% Here are the control sequences that btxmac.tex defines using an @,
% because btxmac.tex wants to absolutely minimize the chance of
% conflicts.  But these control sequence implement documented features
% of eplain, so we want to allow people to use them without the @.
% 
\let\auxfile = \@auxfile
\let\for = \@for
\let\futurenonspacelet = \@futurenonspacelet
\def\iffileexists{\if@fileexists}%
\let\innerdef = \@innerdef
\let\innernewcount = \@innernewcount
\let\innernewdimen = \@innernewdimen
\let\innernewif = \@innernewif
\let\innernewwrite = \@innernewwrite
\let\linenumber = \@linenumber
\let\readauxfile = \@readauxfile
\let\spacesub = \@spacesub
\let\testfileexistence = \@testfileexistence
\let\writeaux = \@writeaux
% 
%
% btxmac.tex defines \innerdef.  Let's use it to make an abbreviation
% for \innerdef\inner<name>{<name>}.
% 
\def\innerinnerdef#1{\expandafter\innerdef\csname inner#1\endcsname{#1}}%
%
% Use that in turn to make non-outer versions of the rest of plain TeX's
% allocation macros.  (btxmac.tex already did a few of them.)
% 
\innerinnerdef{newbox}%
\innerinnerdef{newfam}%
\innerinnerdef{newhelp}%
\innerinnerdef{newinsert}%
\innerinnerdef{newlanguage}%
\innerinnerdef{newmuskip}%
\innerinnerdef{newread}%
\innerinnerdef{newskip}%
\innerinnerdef{newtoks}%
% 
% 
% Besides doing a \write to the aux file, we also need to do an
% \immediate\write.
% 
\def\immediatewriteaux#1{%
  \ifx\noauxfile\@undefined
    \immediate\write\@auxfile{#1}%
  \fi
}%
%
%
% We want \biblabelprint to define hyperlink destination.  For that,
% we save the label for the current bibliography entry from within
% \bblitemhook.
\def\bblitemhook#1{\gdef\@hlbblitemlabel{#1}}%
%
\def\biblabelprint#1{%
   \noindent
   \hbox to \biblabelwidth{%
      \hldest@impl{bib}{\@hlbblitemlabel}%
      \biblabelprecontents
      \biblabelcontents{#1}%
      \biblabelpostcontents
   }%
   \kern\biblabelextraspace
}%
% We want each citation item to be hyperlink to bibliography entry.
% To avoid unnecessary warnings about undefined destinations (e.g.,
% from pdfTeX), produce the link only when .bbl file (which does the
% \hldest's) is present.
\def\eplainprintcitepreitem#1{\hlstart@impl{cite}{#1}}%
\def\eplainprintcitepostitem{\hlend@impl{cite}}%
%
%
\def\printcitepreitem#1{%
  \testfileexistence[\bblfilebasename]{bbl}%
  \iffileexists
    \global\let\printcitepreitem\eplainprintcitepreitem
    \global\let\printcitepostitem\eplainprintcitepostitem
  \else
    \global\let\printcitepreitem\gobble
    \global\let\printcitepostitem\relax
  \fi
  \printcitepreitem{#1}%
}%
%
% btxmac.tex defines a \@for loop.  We provide an alternative \For
% loop, extended to strip an optional leading space off list items.
\def\@Nnil{\@Nil}%
\def\@Fornoop#1\@@#2#3{}%
%
\def\For#1:=#2\do#3{%
   \edef\@Fortmp{#2}%
   \ifx\@Fortmp\empty \else
      \expandafter\@Forloop#2,\@Nil,\@Nil\@@#1{#3}%
   \fi
}%
%
\def\@Forloop#1,#2,#3\@@#4#5{\@Fordef#1\@@#4\ifx #4\@Nnil \else
       #5\@Fordef#2\@@#4\ifx #4\@Nnil \else#5\@iForloop #3\@@#4{#5}\fi\fi
}%
%
\def\@iForloop#1,#2\@@#3#4{\@Fordef#1\@@#3\ifx #3\@Nnil
       \let\@Nextwhile=\@Fornoop \else
      #4\relax\let\@Nextwhile=\@iForloop\fi\@Nextwhile#2\@@#3{#4}%
}%
% \@Fordef<TEXT>\@@<COMMAND> defines COMMAND to the TEXT trimmed off
% optional space tokens at the beginning.  Assume spaces are of
% category 10 (won't work when \obeyspaces is in effect).
\def\@Forspc{ }%
%
\def\@Fordef{\futurelet\@Fortmp\@@Fordef}% Peep at the next token.
%
\def\@@Fordef{%
  \expandafter\ifx\@Forspc\@Fortmp % Next token is a space.
    \expandafter\@Fortrim
  \else
    \expandafter\@@@Fordef
  \fi
}%
%
\expandafter\def\expandafter\@Fortrim\@Forspc#1\@@{\@Fordef#1\@@}%
%
\def\@@@Fordef#1\@@#2{\def#2{#1}}%
%
% From texinfo.tex.  Emulate eTeX's \scantokens with a temporary file
% in ordinary TeX.
\def\tmpfileextension{.tmp}%
\let\tmpfilebasename = \jobname
\ifx\eTeXversion\undefined
  \innernewwrite\eplain@tmpfile
  \def\scantokens#1{%
    \toks@={#1}%
    \immediate\openout\eplain@tmpfile=\tmpfilebasename\tmpfileextension
    \immediate\write\eplain@tmpfile{\the\toks@}%
    \immediate\closeout\eplain@tmpfile
    \input \tmpfilebasename\tmpfileextension\relax
  }%
\fi
% 
% 
% 
%  Macros that produce output.
% 
% \obeywhitespace makes both end-of-lines and spaces in the input be
% respected in the output.  Even spaces at the beginning of lines turn
% into blank space the size of the natural space of the current font.
% The reason why plain TeX's \obeyspaces does not do this last is that
% it produces actual space characters, i.e., glue, and glue is discarded
% at a(n output) line break, and so if line breaks in the input are
% line breaks in the output...
% 
% Tabs are not affected; they will still produce glue (a single space).
%
\begingroup
   \makeactive\^^M \makeactive\ % No spaces or ^^M's from here on.
\gdef\obeywhitespace{%
\makeactive\^^M\def^^M{\par\futurelet\next\@finishobeyedreturn}%
\makeactive\ \let =\ %
%
% The group we use here is the one \obeywhitespace must be enclosed in.
% If we don't do this, then if the obeyed stuff ends in a newline, the
% indent produced by the definition of ^^M will make that line indented,
% even if it isn't the end of the paragraph.
\aftergroup\@removebox%
\futurelet\next\@finishobeywhitespace%
}%
%
% \@finishobeywhitespace eats any spaces and/or the end-of-line after
% the \obeywhitespace command itself.  The group here is the one that it
% itself creates.
%
\gdef\@finishobeywhitespace{{%
\ifx\next %
\aftergroup\@obeywhitespaceloop%
\else\ifx\next^^M%
\aftergroup\gobble%
\fi\fi}}%
%
% \@finishobeyedreturn is invoked at the end of every input line.  We
% check if the next thing is also a return, and, if so, insert extra
% space.  Then we start the next line.
% 
\gdef\@finishobeyedreturn{%
\ifx\next^^M\vskip\blanklineskipamount\fi%
\indent%
}%
\endgroup
%
% The argument here is the space that we are supposed to eat after the
% \obeywhitespace command.
% 
\def\@obeywhitespaceloop#1{\futurelet\next\@finishobeywhitespace}%
% 
% 
% This removes the last box, if it was a empty box of width \parindent.
% We might have been called inside a \vbox, so we have to test if we are
% in horizontal mode before using \lastbox.
% 
\def\@removebox{%
  \ifhmode
    \setbox0 = \lastbox
    \ifdim\wd0=\parindent
      \setbox2 = \hbox{\unhcopy0}% Preserve \box0 so we can put it back.
      \ifdim\wd2=0pt
        % Don't put it back: it was an indentation box.
        % This \ignorespaces ignores spaces after the group.
        \ignorespaces
      \else
        \box0 % Put it back: it wasn't empty.
      \fi
    \else
       \box0 % Put it back: it wasn't the right width.
    \fi
  \fi
}%
%
% We allow for extra (possibly negative) space when we hit blank lines.
% 
\newskip\blanklineskipamount
\blanklineskipamount = 0pt
%
% 
% A good way to print fractions in text when you don't want
% to use \over (which should be most of the time), and yet
% just `1/2' doesn't look right.  (From the TeXbook,
% the answer to exercise 11.6, p.311.)
% 
\def\frac#1/#2{\leavevmode
   \kern.1em \raise .5ex \hbox{\the\scriptfont0 #1}%
   \kern-.1em $/$%
   \kern-.15em \lower .25ex \hbox{\the\scriptfont0 #2}%
}%
%
% 
% The `e' just means `Eplain', as in `Eplain's hrule'.  The advantage
% to using these is that you can change the default thickness.
% 
\newdimen\hruledefaultheight  \hruledefaultheight = 0.4pt
\newdimen\hruledefaultdepth   \hruledefaultdepth = 0.0pt
\newdimen\vruledefaultwidth   \vruledefaultwidth = 0.4pt
%
\def\ehrule{\hrule height\hruledefaultheight depth\hruledefaultdepth}%
\def\evrule{\vrule width\vruledefaultwidth}%
%
%
% The texnames.sty and path.sty files included below were originally
% written by Nelson Beebe and Philip Taylor, respectively. See the
% complete source files (e.g., in this distribution) for comments.
%% [[[include texnames.sty]]]
%% [[[include path.sty]]]
%
% 
% More TeX logos.
%
% Adapted from tugboat.dtx.
\ifx\eTeX\undefined
  \def\eTeX{\hbox{\mathsurround=0pt $\varepsilon$-\kern-.125em\TeX}}%
\fi
%
\ifx\ExTeX\undefined
  \def\ExTeX{\hbox{\mathsurround=0pt
    $\textstyle\varepsilon_{\kern-0.15em\cal{X}}$\kern-.2em\TeX}}%
\fi
%
% XeTeX require the graphics package for reflecting/rotating.  From
% tugboat.dtx:  "Also, at Barbara's suggestion, if the current font is
% slanted, we rotate by 180 instead of reflecting so there is at least
% a chance to look ok.  (The magic values here seem more or less ok
% for \texttt{cmsl} and \texttt{cmti}.)".
\def\eplain@Xe@reflect#1{%
  \ifx\reflectbox\undefined
    \errmessage{A graphics package must be loaded for \string\XeTeX}%
  \else
    \ifdim \fontdimen1\font>0pt
      \raise 1.75ex \hbox{\kern.1em\rotatebox{180}{#1}}\kern-.1em
    \else
      \reflectbox{#1}%
    \fi
  \fi
}%
\def\eplain@Xe#1{\leavevmode
  \smash{\hbox{X%
    \setbox0=\hbox{\TeX}\setbox2=\hbox{E}%
    \lower\dp0\hbox{\raise\dp2\hbox{\kern-.125em\eplain@Xe@reflect{E}}}%
    \kern-.1667em #1}}}%
\ifx\XeTeX\undefined
  \def\XeTeX{\eplain@Xe\TeX}%
\fi
\ifx\XeLaTeX\undefined
  \def\XeLaTeX{\eplain@Xe{\thinspace\LaTeX}}%
\fi
%
%
% A square box, suitable for being a marker in lists.
% 
\def\blackbox{\vrule height .8ex width .6ex depth -.2ex \relax}% square bullet
%
% 
% From p.311 of the TeXbook.
% 
% Make an unfilled rectangle with the dimensions of \box0.  #1 is the
% height of the rules, #2 the depth (i.e., the thicknesses).
% 
\def\makeblankbox#1#2{%
  \ifvoid0
    \errhelp = \@makeblankboxhelp
    \errmessage{Box 0 is void}%
  \fi
  \hbox{\lower\dp0
    \vbox{\hidehrule{#1}{#2}%
      \kern -#1% overlap rules
      \hbox to \wd0{\hidevrule{#1}{#2}%
        \raise\ht0\vbox to #1{}% vrule height
        \lower\dp0\vtop to #1{}% vrule depth
        \hfil\hidevrule{#2}{#1}%
      }%
      \kern-#1\hidehrule{#2}{#1}%
    }%
  }%
}%
%
\newhelp\@makeblankboxhelp{Assigning to the dimensions of a void^^J%
  box has no effect.  Do `\string\setbox0=\string\null' before you^^J%
  define its dimensions.}%
%
% 
% Produce an hrule with height #1 and depth #2, and insert kerning so it
% doesn't change the current position.
% 
\def\hidehrule#1#2{\kern-#1\hrule height#1 depth#2 \kern-#2}%
%
% Produce a vrule with width #1+#2, kerning so as not to change the
% current position.
% 
\def\hidevrule#1#2{%
  \kern-#1%
  \dimen@=#1\advance\dimen@ by #2%
  \vrule width\dimen@
  \kern-#2%
}%
% 
% 
% The \boxit macro from the TeXbook, trivially generalized to allow
% something other than 3pt around the TeX box being boxed.
% 
\newdimen\boxitspace \boxitspace = 3pt
%
\long\def\boxit#1{%
  \vbox{%
    \ehrule
    \hbox{%
      \evrule
      \kern\boxitspace
      \vbox{\kern\boxitspace \parindent = 0pt #1\kern\boxitspace}%
      \kern\boxitspace
      \evrule
    }%
    \ehrule
  }%
}%
%
% 
% Produce the written-out form of a number.
% 
\def\numbername#1{\ifcase#1%
   zero%
   \or one%
   \or two%
   \or three%
   \or four%
   \or five%
   \or six%
   \or seven%
   \or eight%
   \or nine%
   \or ten%
   \or #1%
   \fi
}%
%
% The following arrow macros were written by Steven Smith. See arrow.tex.
\let\@plainnewif = \newif
\let\@plainnewdimen = \newdimen
\let\newif = \innernewif
\let\newdimen = \innernewdimen
\edef\@eplainoldandcode{\the\catcode`& }%
\catcode`& = 11
\toks0 = {%
  %% [[[include arrow1]]]
}%
\catcode`& = 4
\toks2 = {%
  %% [[[include arrow2]]]
}%
\let\newif = \@plainnewif
\let\newdimen = \@plainnewdimen
\ifx\noarrow\@undefined \the\toks0 \the\toks2 \fi
\catcode`& = \@eplainoldandcode
% 
% 
% 
%  Environments.
% 
% Define an ``environment'': arbitrary text, enclosed by \begingroup and
% \endgroup.  But you get to label the group, so that if you forget an
% \environment or an \endenvironment, you will probably get an error
% message about it.
% 
% Since the environment names appear in \errmessage arguments, it's best
% to keep them to `letter' and `other' characters.  I suppose we could
% call \tokstostring to allow more general labels.
% 
% These macros improve slightly on the answer to exercise 5.7 in
% The TeXbook, by making some checks on \begingroup and \endgroup, as
% well as just making sure \environment and \endenvironment's match.
% 
%
\def\environment#1{%
   \ifx\@groupname\@undefined\else
      % This gets invoked if we have two \environments (and no matching
      % \endenvironment to the first) with an \endgroup in between.
      \errhelp = \@unnamedendgrouphelp
      \errmessage{`\@groupname' was not closed by \string\endenvironment}%
   \fi
   % Use \edef in case we are passed a macro that contains the name,
   % instead of the name.
   \edef\@groupname{#1}%
   \begingroup
      \let\@groupname = \@undefined
}%
%
\def\endenvironment#1{%
   \endgroup
   \edef\@thearg{#1}%
   \ifx\@groupname\@thearg
   \else
      \ifx\@groupname\@undefined
         % Unfortunately, one gets an `extra \endgroup' message before
         % seeing this.  But we have to restore \@groupname, so I see no
         % alternative. 
         \errhelp = \@isolatedendenvironmenthelp
         \errmessage{Isolated \string\endenvironment\space for `#1'}%
      \else
         \errhelp = \@mismatchedenvironmenthelp
         \errmessage{Environment `#1' ended, but `\@groupname' started}%
         \endgroup % Probably a typo in the names.
      \fi
   \fi
   \let\@groupname = \@undefined
}%
%
\newhelp\@unnamedendgrouphelp{Most likely, you just forgot an^^J%
   \string\endenvironment.  Maybe you should try inserting another^^J%
   \string\endgroup to recover.}%
%
\newhelp\@isolatedendenvironmenthelp{You ended an environment X, but^^J%
   no \string\environment{X} to start it is anywhere in sight.^^J%
   You might also be at an \string\endenvironment\space that would match^^J%
   a \string\begingroup, i.e., you forgot an \string\endgroup.}%
%
\newhelp\@mismatchedenvironmenthelp{You started an environment named X, but^^J%
   you ended one named Y.  Maybe you made a typo in one^^J%
   or the other of the names?}%
% 
% 
% The above sort of environment allows nesting.  But environments
% shouldn't always be allowed to nest (like the \flushright,
% \flushleft, and \center ones defined below).  Here are some macros to
% help deal with that.
% 
% \checkenv goes at the beginning of a macro that is
% going to define the environment.
% 
\newif\ifenvironment
\def\checkenv{\ifenvironment \errhelp = \@interwovenenvhelp
   \errmessage{Interwoven environments}%
   \egroup \fi
}%
%
\newhelp\@interwovenenvhelp{Perhaps you forgot to end the previous^^J%
   environment? I'm finishing off the current group,^^J%
   hoping that will fix it.}%
%
% 
% 
%  Mathematics displays.
%
% By default, TeX centers displayed material.  To get left-justified
% displays, say \leftdisplays.  To go back to centered displays, say
% \centereddisplays.
% 
% This is based on an approach developed by Donald Arseneau,
% asnd@triumfrg.bitnet.
% 
\newtoks\previouseverydisplay
% Here we want to make ordinary math displays flush left, 
% indented by the dimen \leftdisplayindent, which defaults
% to \parindent.
%
% How do you want the first column aligned?
\let\@leftleftfill\relax % as it was
%\let\@leftleftfill\hfill % makes more sense, but could be too ugly
%
% Surely it makes more sense to not sum \leftdisplayindent+\parindent
\newdimen\leftdisplayindent \leftdisplayindent=\parindent
\newif\if@leftdisplays
%
\def\leftdisplays{%
  \if@leftdisplays\else
    \previouseverydisplay = \everydisplay
    \everydisplay = {\the\previouseverydisplay \leftdisplaysetup}%
    \let\@save@maybedisableeqno = \@maybedisableeqno
    \let\@saveeqno = \eqno
    \let\@saveleqno = \leqno
    \let\@saveeqalignno = \eqalignno
    \let\@saveleqalignno = \leqalignno
    \let\@maybedisableeqno = \relax
    \def\eqno{\hfill\textstyle\enspace}%
    \def\leqno{%
      \hfill
      \hbox to0pt\bgroup
        \kern-\displaywidth
        % was: \kern-\displayindent % really \displayindent?  
        \kern-\leftdisplayindent    % I'll use just \leftdisplayindent
        $\aftergroup\@leftleqnoend  % inserted after ending $
    }%
    \@redefinealignmentdisplays
    \@leftdisplaystrue
  \fi
}%
\newbox\@lignbox
\newdimen\disprevdepth

% In order to use $$ for left-aligned equation we have to
% put something like \leftline{$\displaystyle ...$} in the
% display.  Then \eqno works basically like \hfill.
% In order to make $$<assignments>\eqalignno{...}$$ work as
% expected, including page breaks, we have to get rid of the
% horizontal box, and un-vbox the alignment.  Sadly, \unvbox
% does not perform baselineskip handing, so we need to get
% the \prevdepth ourselves...in order to get the prevdepth,
% the outermost display must be an alignment display.
% Therefore, an ordinary $$ a=b $$ becomes:
% $$\halign{#\cr\noalign{\disprevdepth = \prevdepth
%       \leftline{$\displaystyle a=b$}
%   }}$$
%
\def\centereddisplays{%
  % If \leftdisplays hasn't been called, don't try to restore all the
  % stuff it changes.
  % 
  \if@leftdisplays
    \everydisplay = \previouseverydisplay
    \let\@maybedisableeqno = \@save@maybedisableeqno
    \let\eqno = \@saveeqno
    \let\leqno = \@saveleqno
    \let\eqalignno = \@saveeqalignno
    \let\leqalignno = \@saveleqalignno
    \@leftdisplaysfalse
  \fi
}%
%
\def\leftdisplaysetup{%
%  surely not this *and* \leftdisplayindent? :  \dimen@ = \parindent
   \dimen@ = \leftdisplayindent
   \advance\dimen@ by \leftskip
   \advance\displayindent by \dimen@
   \advance\displaywidth by -\dimen@
%  this outermost alignment doesn't align anything.
   \halign\bgroup##\cr \noalign\bgroup
      \disprevdepth = \prevdepth
      \setbox\z@ = \hbox to\displaywidth\bgroup
      %  Why strut??  \strut
      %  Why this??   \advance\hsize by -\displayindent
      $\displaystyle
      \aftergroup\@lefteqend % inserted after ending $
}
%
\def\@lefteqend{% gets inserted between the ending $$
   \hfil\egroup% end box 0
   \@putdisplay}
% gets inserted between trailing $$.
\def\@leftleqnoend{\hss \egroup $}% end the \hbox to 0pt for \leqno, restore $
%
\def\@putdisplay{%
   \ifvoid\@lignbox %  Ordinary display; use it.
     \moveright\displayindent\box\z@ 
   \else % alignment display; unwrap alignment
     \prevdepth = \dp\@lignbox % affects the skip *below*
     \unvbox\@lignbox
   \fi
   \egroup\egroup % end \noalign, end outer \halign
   $% restore first $ of trailing $$
}
%
\def\@redefinealignmentdisplays{%
  \def\displaylines##1{
    \global\setbox\@lignbox\vbox{%
      \prevdepth = \disprevdepth
      \displ@y
      \tabskip\displayindent
      \halign{\hbox to\displaywidth
        {$\@lign\displaystyle####\hfil$\hfil}\crcr
              ##1\crcr}}}%
  \def\eqalignno##1{%
    \def\eqno{&}%
    \def\leqno{&}%
    \global\setbox\@lignbox\vbox{%
      \prevdepth = \disprevdepth
      \displ@y
      \advance\displaywidth by \displayindent
      \tabskip\displayindent
      \halign to\displaywidth{%
         \hfil $\@lign\displaystyle{####}$\@leftleftfill\tabskip\z@skip
        &$\@lign\displaystyle{{}####}$\hfil\tabskip\centering
        &\llap{$\@lign####$}\tabskip\z@skip\crcr
        ##1\crcr}}}%
  \def\leqalignno##1{%
    \def\eqno{&}%
    \def\leqno{&}%
    \global\setbox\@lignbox\vbox{%
      \prevdepth = \disprevdepth
      \displ@y
      \advance\displaywidth by \displayindent
      \tabskip\displayindent
      \halign to\displaywidth{%
         \hfil $\@lign\displaystyle{####}$\@leftleftfill\tabskip\z@skip
        &$\@lign\displaystyle{{}####}$\hfil\tabskip\centering
        &\kern-\displaywidth 
%         \showthe\displayindent \showthe\leftdisplayindent
         \rlap{\kern\displayindent \kern-\leftdisplayindent$\@lign####$}%
         \tabskip\displaywidth\crcr
        ##1\crcr}}}%
}%
% 
% \noalign is typically used to insert a few words (`and', for example)
% between two aligned equations.  So I don't think the \noaligned
% material should be indented.  Since \noalign takes <vertical mode
% material>, we would end up with double indentation, anyway: one
% because we're indenting the whole display, and one at the start of the
% <v.m.m.>.  (If you want to change any of this, you can put something in
% \@everynoalign.)  So, we use this definition for \noalign in a
% left-justified \eqalignno:
% 
\let\@primitivenoalign = \noalign
\newtoks\@everynoalign
\def\@lefteqalignonoalign#1{%
  \@primitivenoalign{%
    % Is it right to set \leftskip=0pt first, thus perhaps making this
    % work in lists and so forth?  We just compensate for the other ways
    % the display is indented here.
    \advance\leftskip by -\parindent
    \advance\leftskip by -\leftdisplayindent
    \parskip = 0pt
    %
    % We use \parindent=0pt instead of \noindent because the latter
    % starts unrestricted horizontal mode, which means the alignment
    % we're inside will wind up being as wide as the page.  When the arg
    % is just vertical material, this is wrong.  For example, using
    % \matrix inside \eqalignno fails if \noindent is used.
    \parindent = 0pt
    \the\@everynoalign
    #1%
  }%
}%
% 
% 
% 
%  Time macros.
% 
% TeX sets \time, \day, \month, and \year when it begins.  (And does not
% update them as it runs!)
% 
% 
% \monthname produces the name of the month, abbreviated to three
% letters.  The primitive \month should never be zero.
% 
\def\monthname{%
   \ifcase\month
      \or Jan\or Feb\or Mar\or Apr\or May\or Jun%
      \or Jul\or Aug\or Sep\or Oct\or Nov\or Dec%
   \fi
}%
%
% \fullmonthname is like \monthname, except it doesn't abbreviate.
% 
\def\fullmonthname{%
   \ifcase\month
      \or January\or February\or March\or April\or May\or June%
      \or July\or August\or September\or October\or November\or December%
   \fi
}%
%
% \timestring produces the current time, in a format like `1:14 p.m.'.
% 
\def\timestring{\begingroup
   \count0 = \time
   \divide\count0 by 60
   \count2 = \count0   % The hour, from zero to 23.
   %
   \count4 = \time
   \multiply\count0 by 60
   \advance\count4 by -\count0   % The minute, from zero to 59.
   % But we need the minutes with a leading zero, if necessary.
   \ifnum\count4<10
      \toks1 = {0}%
   \else
      \toks1 = {}%
   \fi
   % 
   % Convert the hour into `a.m.' or `p.m.', and make it mod 12.
   \ifnum\count2<12
      \toks0 = {a.m.}%
   \else
      \toks0 = {p.m.}%
      \advance\count2 by -12
   \fi
   %
   % If it's midnight, call it `12', not `0'.
   \ifnum\count2=0
      \count2 = 12
   \fi
   %
   % Produce the output.
   \number\count2:\the\toks1 \number\count4 \thinspace \the\toks0
\endgroup}%
%
%
% \timestamp produces a text string for the whole thing like
% `23 Apr 1964  1:14 p.m.'.
%
\def\timestamp{\number\day\space\monthname\space\number\year\quad\timestring}%
%
% 
% \today produces the current date, as in `23 April 1964'.
% 
\def\today{\the\day\ \fullmonthname\ \the\year}%
%
% 
% 
%  (Typographical) lists.
% 
% These macros can produce numbered or unnumbered lists.
% 
% You can change the spacing by assigning new values to these registers.
% They are used by both kinds of lists.  \listleftindent is relative to
% the current paragraph indentation, while \listrightindent is an
% absolute value.  I do this for two reasons: (1) it is more useful, if not
% more ``logical'', to make list indentation depend on the paragraph
% indentation; (2) footnotes do not work if \parindent is zero, and
% having a footnote in a list item is perfectly reasonable.
% 
% If you change \baselineskip and want \abovelistskip and \belowlistskip
% to retain their meanings here, you will have to reassign to them.  The
% \baselineskip here is the value at the time eplain.tex is read, i.e.,
% 12pt (most likely).
% 
% If the items in your lists are very long, you might want to
% make \interitemskipamount nonzero.  
% 
\newskip\abovelistskipamount      \abovelistskipamount = .5\baselineskip
  \newcount\abovelistpenalty      \abovelistpenalty    = 10000
  \def\abovelistskip{\vpenalty\abovelistpenalty \vskip\abovelistskipamount}%
\newskip\interitemskipamount      \interitemskipamount = 0pt
  \newcount\belowlistpenalty      \belowlistpenalty    = -50
  \def\belowlistskip{\vpenalty\belowlistpenalty \vskip\belowlistskipamount}%
\newskip\belowlistskipamount      \belowlistskipamount = .5\baselineskip
  \newcount\interitempenalty      \interitempenalty    = 0
  \def\interitemskip{\vpenalty\interitempenalty \vskip\interitemskipamount}%
\newdimen\listleftindent    \listleftindent = 0pt
\newdimen\listrightindent   \listrightindent = 0pt        
\let\listmarkerspace = \enspace
%
% To do arbitrary things at the start of each list:
\newtoks\everylist
%
% If you want no space between items for a particular list
% (perhaps because the items in it are short), you can say,
% e.g., \numberedlist\listcompact.
% 
\def\listcompact{\interitemskipamount = 0pt \relax}%
%
% This is called to set up the parameters by both sorts of lists.
% Because we set \rightskip, we finish off the current paragraph.
% 
\newdimen\@listindent
%
\def\beginlist{%
  % Insert the space above this list, before we change \leftskip
  % (because the \vskip in here might be what ends the paragraph).
  \abovelistskip
  %
  \@listindent = \parindent
  \advance\@listindent by \listleftindent
  %
  % \leftskip shifts nested lists to the right on the page.
  \advance\leftskip by \@listindent
  \advance\rightskip by \listrightindent
  %
  % We always need \itemnumber, so we can know whether an item is the
  % first one or not.
  \itemnumber = 1
  %
  \the\everylist
}%
%
% A list item, for both kinds of lists.
% 
\def\li{\@getoptionalarg\@finli}%
\def\@finli{%
  % Save \@optionalarg in case \interitemskip garble it.
  \let\@lioptarg\@optionalarg
  % Write xref definition but do not define hyperlink destination
  % (\printitem will do it).
  \ifx\@lioptarg\empty \else
    \begingroup
      \@@hldestoff
      \expandafter\writeitemxref\expandafter{\@lioptarg}%
    \endgroup
  \fi
  \ifnum\itemnumber=1 \else \interitemskip \fi
  % Add hyperlink destination in front of \marker before doing \printitem.
  \begingroup
    \ifx\@lioptarg\empty \else
      \toks@ = \expandafter{\@lioptarg}%
      \let\li@nohldest@marker\marker
      \edef\marker{\noexpand\hldest@impl{li}{\the\toks@}\noexpand\li@nohldest@marker}%
    \fi
    \printitem
  \endgroup
  \advance\itemnumber by 1
  \advance\itemletter by 1
  \advance\itemromannumeral by 1
  %
  % Just in case somebody creeps in with an argument or something.
  \ignorespaces
}%
%
% \writeitemxref{LABEL} writes out a definition for LABEL to be \marker
% for the aux file.
% 
\def\writeitemxref#1{\definexref{#1}\marker{item}}%
%
% \printitem is used to print items by both sorts of lists.  A \par gets
% produced before every item -- even the first one.  We also want to
% make paragraphs after the first appear to be indented -- i.e., they
% will have double indentation.  It is usually bad exposition to have
% lists with multiparagraph items, but sometimes it is unavoidable.
% 
\def\printitem{%
  \par
  \nobreak
  \vskip-\parskip
  \noindent
  \printmarker\marker
}%
%
% Output the list marker.
% 
\def\printmarker#1{\llap{\marker \enspace}}%
%
% Common ending.
% 
\def\endlist{\belowlistskip}%
%
% 
% \numberedlist produces items which are numbered sequentially, starting
% from one.  You start items with \li (`list item').  End the list with
% \endnumberedlist.
% 
% A nested \numberedlist produces items labelled `(a)', `(b)', etc.  A
% doubly (and deeper) nested \numberedlist labels items with `*'.
% 
% These registers keep track of where we are.
% 
\newcount\numberedlistdepth
\newcount\itemnumber
\newcount\itemletter
\newcount\itemromannumeral
%
\def\numberedmarker{%
  \ifcase\numberedlistdepth
      (impossible)%
  \or \printitemnumber
  \or \printitemletter
  \or \printitemromannumeral
  \else *%
  \fi
}%
%
% These produce the text of the labels.  We use \the\itemletter so that
% the value will expand.
% 
\def\printitemnumber{\number\itemnumber}%
\def\printitemletter{\char\the\itemletter}%
\def\printitemromannumeral{\romannumeral\itemromannumeral}%
\def\numberedprintmarker#1{\llap{#1) \listmarkerspace}}%
%
\def\numberedlist{\environment{@numbered-list}%
  % This is set back to zero by getting to the end of the group.
  \advance\numberedlistdepth by 1
  \itemletter = `a
  \itemromannumeral = 1
  \beginlist
  \let\marker = \numberedmarker
  \let\printmarker = \numberedprintmarker
}%
%
\def\endnumberedlist{%
  \par
  \endenvironment{@numbered-list}%
  \endlist
}%
%
% Allow synonyms for \numberedlist.
\let\orderedlist = \numberedlist
\let\endorderedlist = \endnumberedlist
% 
% 
% 
% \unorderedlist produces items which are labelled with bullets.  You
% start an item with \li, just as with numbered lists.  You end the list
% with \endunorderedlist.
% 
% A nested \unorderedlist produces items labelled with em-dashes.  A
% doubly (and deeper) nested \unorderedlist uses `*'.
% 
\newcount\unorderedlistdepth
%
\def\unorderedmarker{%
  \ifcase\unorderedlistdepth
      (impossible)%
  \or \blackbox
  \or ---%
  \else *%
  \fi
}%
\def\unorderedprintmarker#1{\llap{#1\listmarkerspace}}%
%
\def\unorderedlist{\environment{@unordered-list}%
  \advance\unorderedlistdepth by 1
  \beginlist
  \let\marker = \unorderedmarker
  \let\printmarker = \unorderedprintmarker
}%
%
\def\endunorderedlist{%
  \par
  \endenvironment{@unordered-list}%
  \endlist
}%
%
% 
% 
%  Verbatim listing.
% 
% ... well, almost verbatim.  We assume the font \tt has all the
% characters that will appear.  Control characters, except for tabs and
% form feeds (and returns) won't produce anything useful.  Tabs produce
% a fixed amount of space, and form feeds produce a page break.
% 
% This is based on Knuth's ideas in Appendix D of the TeXbook, p. 380.
% The argument should be a filename.
% 
% if you need to do something more for your particular fonts and/or
% environment before the file is input, give a definition to
% \setuplistinghook.  If you want line numbers on the output, you can
% say \let\setuplistinghook = \linenumberedlisting.  To omit last
% (empty) line of input, say \let\setuplistinghook = \nolastlinelisting.
% (This works with line numbers, too, but only if \printlistinglineno
% consists of boxes only.)
\def\listing#1{%
   \par \begingroup
   \@setuplisting
   \setuplistinghook
   \input #1
   \endgroup
}%
%
\let\setuplistinghook = \relax
%
\def\linenumberedlisting{%
  \ifx\lineno\undefined \innernewcount\lineno \fi
  \lineno = 0
  \everypar = {\advance\lineno by 1 \printlistinglineno}%
}%
\def\printlistinglineno{\llap{[\the\lineno]\quad}}%
% Remove the last line by removing all boxes (indentation \hbox and/or
% line number from \everypar).
\def\nolastlinelisting{\aftergroup\@removeboxes}%
% Remove boxes until we remove all of them, or until we
% encounter something other than a box.
\def\@removeboxes{%
  \setbox0 = \lastbox
  \ifvoid0
    \ignorespaces % Ignore spaces after the \obeywhitespace's group.
  \else
    \expandafter\@removeboxes
  \fi
}%
%
%
% 
% \uncatcodespecials must come before \obeywhitespace, lest a space
% character in the input produce character 32 from the \tt font.
% Ensure ^^L is active and non-outer.
% 
{%
  \makeactive\^^L
  \let^^L = \relax
  \gdef\@setuplisting{%
     \uncatcodespecials
     \obeywhitespace
     \makeactive\`
     \makeactive\^^I
     \makeactive\^^L
     \def^^L{\vfill\break}%
     \parskip = 0pt
     \listingfont
  }%
}%
\def\listingfont{\tt}%
%
% Give definitions to the characters we want to be special.
% 
% Do ` separately, so can use ` in the \catcode commands elsewhere.
% 
{%
   \makeactive\`
   \gdef`{\relax\lq}% Defeat ligatures.
}%
{%
   \makeactive\^^I
   \gdef^^I{\hskip8\fontdimen2\font \relax}%
}%
%
% 
% \verbatim ... |endverbatim typesets the ... in typewriter.  To produce a |
% in the ..., use ||.  This macro was contributed by beebe@math.utah.edu.
% Generalized to characters other than | by dorai@cs.rice.edu.
% 
\def\verbatimescapechar#1{%
  \gdef\@makeverbatimescapechar{%
    \@makeverbatimdoubleescape #1%
    \catcode`#1 = 0
  }%
}%
\def\@makeverbatimdoubleescape#1{%
  \catcode`#1 = \other
  \begingroup
    \lccode`\* = `#1%
    \lowercase{\endgroup \ece\def*{*}}%
}%
\verbatimescapechar\|  % initially escapechar is |
%
\def\verbatim{\begingroup
  \uncatcodespecials
  \makeactive\` % make space character a single space, not stretchable
  \@makeverbatimescapechar
  \tt\obeywhitespace}
\let\endverbatim = \endgroup
% 
% 
% 
%  Table of contents, list of figures, etc.
% 
% Entries for the table of contents are recorded in \jobname.toc, which
% we open for writing at the first \writetocentry.  Actually, we use
% \tocfilebasename for the root of the filename to read; \jobname is
% the default.
%
\def\definecontentsfile#1{%
  \ece\innernewwrite{#1file}%
  \ece\innernewif{if@#1fileopened}%
  \ece\let{#1filebasename} = \jobname
  \ece\def{open#1file}{\opencontentsfile{#1}}%
  \ece\def{write#1entry}{\writecontentsentry{#1}}%
  \ece\def{writenumbered#1entry}{\writenumberedcontentsentry{#1}}%
  \ece\def{writenumbered#1line}{\writenumberedcontentsline{#1}}%
  \ece\innernewif{ifrewrite#1file} \csname rewrite#1filetrue\endcsname
  \ece\def{read#1file}{\readcontentsfile{#1}}%
}%
% 
% We provide \opentocfile, \readtocfile, etc., by default.
\definecontentsfile{toc}%
%
% And `toc' is just the argument to this macro.
\def\opencontentsfile#1{%
  \csname if@#1fileopened\endcsname \else
     \ece{\immediate\openout}{#1file} = \csname #1filebasename\endcsname.#1
     \ece\global{@#1fileopenedtrue}%
  \fi
}%
%
% \writetocentry#1#2 produces a line in the .toc file that
% looks like:
% \toc#1entry{#2}{page number}
% e.g., 
% \tocchapterentry{Introduction}{2}
% would be written by
% \writetocentry{chapter}{Introduction}
% if the chapter started on page two.
% 
% Thus, #1 is intended to be something like `chapter' or `section', #2
% to be the text of the title.
% 
% One special case is for #1 to be an integer (once fully expanded).
% If it is, we produce a line in the .toc file that looks like:
%   \tocentry{#1}{#2}{page number}
% This might be useful if you want to write one generic \tocentry
% macro to format all TOC entries on the basis of their depth (#1).
%
% Of course, if you want, you can \write\tocfile yourself with whatever
% you like.  In that case, you must also call \opentocfile.
% 
% By the way, it would be wrong to put a \percentchar at the end of the
% output line.  Then, when the .toc file is read, if each line is turned
% into a \leftline, say, there would be no legal breakpoint between the
% boxes, and one extremely long line would result.
% 
% `toc' is the first argument to this; \writetocentry is defined by
% \definecontentsfile.
\def\writecontentsentry#1#2#3{\writenumberedcontentsentry{#1}{#2}{#3}{}}%
%
% Sometimes you want the control sequence to take another number (e.g.,
% a chapter number) as a parameter.  (Although you can pass anything you
% want as the third parameter, naturally.)  The third parameter is
% expanded at the point of the \writenumberedtocentry, not when the
% \write actually happens.  This makes the usual case---the third
% parameter being \the\someregister---work.
% 
% For example:
% \writenumberedtocentry{chapter}{The second chapter}{2}
% would produce:
% \tocchapterentry{The second chapter}{2}{14}
% 
% if the second chapter started on page 14.
% 
% `toc' is the first argument, as above.
\def\writenumberedcontentsentry#1#2#3#4{%
  \csname ifrewrite#1file\endcsname
    % Set \toks0 to the initial part (command name and possible PART argument).
    \writenumberedcontents@cmdname{#1}{#2}%
    %
    \def\temp{#3}% the text
    %
    % Usually #4 is just `\the\register', which we want to expand.  But
    % if it's not a number at all -- e.g., if it's an author's name, we
    % don't want to expand control sequences for accents and the like.
    % So we play some games here.
    \toks2 = \expandafter{#4}%
    \edef\cs{\the\toks2}%
    \edef\@wr{%
      \write\csname #1file\endcsname{%
        \the\toks0 % the \toc...entry control sequence
        {\sanitize\temp}% the text
        \ifx\empty\cs\else {\sanitize\cs}\fi % a secondary number, or nothing
        {\noexpand\folio}% the page number
      }%
    }%
    \@wr
  \fi
  \ignorespaces
}%
% This is very similar to \writenumberedcontentsentry, differing only
% in the order of parameters it takes and writes for the
% \toc...entry control sequence.  #3 is the chapter (section, etc.)
% number, and #4 is the text.
\def\writenumberedcontentsline#1#2#3#4{%
  \csname ifrewrite#1file\endcsname
    % Set \toks0 to the initial part (command name and possible PART argument).
    \writenumberedcontents@cmdname{#1}{#2}%
    %
    \def\temp{#4}% the text
    %
    % Usually #3 is just `\the\register', which we want to expand.  But
    % if it's not a number at all -- e.g., if it's an author's name, we
    % don't want to expand control sequences for accents and the like.
    % So we play some games here.
    \toks2 = \expandafter{#3}%
    \edef\cs{\the\toks2}%
    \edef\@wr{%
      \write\csname #1file\endcsname{%
        \the\toks0 % the \toc...entry control sequence
        \ifx\empty\cs\else {\sanitize\cs}\fi % a secondary number, or nothing
        {\sanitize\temp}% the text
        {\noexpand\folio}% the page number
      }%
    }%
    \@wr
  \fi
  \ignorespaces
}%
% This is the common part of \writenumberedcontentsentry and
% \writenumberedcontentsline.  It takes `toc' and PART arguments,
% opens toc file and sets \toks0 to `\tocentry{PART}' (if PART is an
% integer) or `\tocPARTentry' (otherwise).
\def\writenumberedcontents@cmdname#1#2{%
  \csname open#1file\endcsname
  \edef\temp{#2}% Expand PART fully and see if this produced an integer.
  \expandafter\if\expandafter\isinteger\expandafter{\temp}%
    \toks0 = {\expandafter\noexpand \csname #1entry\endcsname}%
    \edef\temp{\the\toks0{\temp}}%
    \toks0 = \expandafter{\temp}%
  \else
    \toks0 = {\expandafter\noexpand \csname #1#2entry\endcsname}%
  \fi
}%
% 
% The entries are read in when the user invokes \readtocfile (which
% should be before the first \writetocentry).  We do not open the .toc
% file to allow reading it in twice to make a short contents or some
% such.  This should not cause any problems because each of
% \writecontentsentry and \writenumberedcontentsentry open the .toc
% file for writing (i.e., empty it) before trying to write to it.
\def\readcontentsfile#1{%
   \edef\temp{%
     \noexpand\testfileexistence[\csname #1filebasename\endcsname]{#1}%
   }\temp
   \if@fileexists
      \input \csname #1filebasename\endcsname.#1\relax
   \fi
}%
%
% Here are some sample definitions of the \toc...entry macros.  Perhaps
% you or your book designer can come up with a better way of handling
% contents than leaders.  These definitions are just examples, not
% something you might want to actually use to print a document.
\def\tocchapterentry#1#2{\line{\bf #1 \dotfill\ #2}}%
\def\tocsectionentry#1#2{\line{\quad\sl #1 \dotfill\ \rm #2}}%
\def\tocsubsectionentry#1#2{\line{\qquad\rm #1 \dotfill\ #2}}%
\def\tocentry#1#2#3{\line{\rm\hskip#1em #2 \dotfill\ #3}}%
%
% 
% 
%  Cross-references.
% 
% Definitions of references are recorded in \jobname.aux, called
% \auxfile in the macros, which btxmac.tex has opened.
% 
% When a label isn't defined, we only want to complain if
% \xrefwarningtrue; btxmac uses \if@citewarning for this, so we have to
% reuse that name.  We can't just say \let\ifxrefwarning =
% \if@citewarning, since then changes to the latter won't be reflected
% in the former.  On the other hand, we have to have a true \if...
% command, so \if's and \fi's match properly.  What a mess.
%
\let\ifxrefwarning = \iftrue
\def\xrefwarningtrue{\@citewarningtrue \let\ifxrefwarning = \iftrue}%
\def\xrefwarningfalse{\@citewarningfalse \let\ifxrefwarning = \iffalse}%
%
% 
% \xref{foo} produces ``p.\thinspace <page#>''.  \xrefn{foo} produces
% ``<page#>''.  \xrdef{foo} produces nothing, but defines the label
% `foo' to be on the current page.
%
% As usual, it takes two passes to get the cross-references right.
%
% We check for multiply defined labels during the reading of the aux
% file, not at the time a cross-reference macro is called by the user.
% (From Tomas Penicka <tom.penicka@centrum.cz>.)
%
% 
% \xrlabel{LABEL} expands to a cross-reference internal name.  We append
% an _ character to NAME, to help avoid conflicts.  And we append an `x'
% so that we don't redefine \_ on an empty label.
% 
\begingroup
  % Mike Spivak's MathTime macros for Times Roman fonts changes the
  % catcode of _ to be active. Undo that. (From adam@symcom.math.uiuc.edu.)
  \catcode`\_ = 8
  \gdef\xrlabel#1{#1_x}%
\endgroup
%
%
% \xrdef{LABEL} defines LABEL to be the current page number.  But we
% don't define the label here, because the page number might be off: if
% this is not the first time through, the label would already be
% defined, and we would redefine it with the wrong information.
% 
\def\xrdef#1{%
  \begingroup
    % Define hyperlink destination.
    \hldest@impl{xrdef}{#1}%
    % Define cross-reference but not hyperlink destination.
    \begingroup
      \@@hldestoff
      \definexref{#1}{\noexpand\folio}{page}%
    \endgroup
  \endgroup
  \ignorespaces
}%
%
% \definexref{LABEL}{DEFINITION}{CLASS} defines a cross-reference named
% LABEL of label class CLASS to be DEFINITION.  (Or LABEL can be a
% control sequence; it's expanded to get the label text.)  To get a
% possible page number right, we have to write the definition out to the
% auxiliary file, instead of only defining it directly.
% 
\def\definexref#1#2#3{%
  % Define a hyperlink destination LABEL.
  \hldest@impl{definexref}{#1}%
  % Remember what we're given; it might be `\@optionalarg', which
  % \readauxfile trashes.  (No loss of generality here, since \csname
  % will fully expand the label anyway.)
  \edef\temp{#1}%
  %
  % Be sure we've read the aux file before we zap it:
  \readauxfile
  %
  % When we read in the aux file next time, define the label:
  \edef\@wr{\noexpand\writeaux{\string\@definelabel{\temp}{#2}{#3}}}%
  \@wr
  \ignorespaces
}%
%
% \@definelabel{LABEL}{DEFINITION}{CLASS} actually defines LABEL of
% label class CLASS to be DEFINITION.
% 
\def\@definelabel#1{% #2 and #3 will be read later.
  % The following group will keep the save stack from overflowing:
  % We are reading the .aux file inside a group.  For undefined
  % commands, \csname...\endcsname makes them equivalent to \relax
  % locally in that group; when we globally redefine them, TeX enters
  % the definitions on the save stack, once for each label.  When done
  % in the topmost group, this can overflow stack of any size, given
  % sufficient number of label definitions.  So, we put these
  % \csname...\endcsname's inside another group, to pop the stack for
  % every label.
  \begingroup % Will be ended in \@definelabel@nocheck.
    % Warn if we see that another label with the same name has been
    % defined.  Remember, we are called when the aux file is being read,
    % which means that no labels have been defined yet except the ones
    % which come earlier in this aux file.
    \expandafter\ifx\csname\xrlabel{#1}\endcsname \relax
      \expandafter\@definelabel@nocheck
    \else
      \expandafter\@definelabel@warn
    \fi
    % Define the control sequences or warn.
    {#1}%
}%
% Define a label without checking for redefinitions.
\def\@definelabel@nocheck#1#2#3{%
    % Define the control sequence.
    \expandafter\gdef\csname\xrlabel{#1}\endcsname{#2}%
    %
    % Remember what kind of label this is, so \ref will know what to do.
%    \global\setproperty{\xrlabel{#1}}{class}{#3}%
    \setpropertyglobal{\xrlabel{#1}}{class}{#3}%
  \endgroup % From \@definelabel.
}%
% Warn and then redefine a label.
\def\@definelabel@warn#1#2#3{%
  \message{^^J\linenumber Label `#1' multiply defined,
           value `#2' of class `#3' overwriting value
           `\csname\xrlabel{#1}\endcsname' of class
           `\getproperty{\xrlabel{#1}}{class}'.}%
  \@definelabel@nocheck{#1}{#2}{#3}%
}%
%
% \reftie is used by several cross-referencing macros to separate
% optional text from label reference.  \refspace is used by \ref to
% separate optional text from \CLASSword and by \xref to separate
% optional text from `p. '.
%
\def\reftie{\penalty\@M \ }% Do not rely on `~' being defined as a tie.
\let\refspace\ 
% 
% Typeset a reference to the label #1.  If optional argument is
% present it will be tied with \reftie to the reference and become
% part of the hyperlink.
% 
\def\xrefn{\@getoptionalarg\@finxrefn}%
\def\@finxrefn#1{%
  % Hyperlink and optional TEXT.
  \hlstart@impl{ref}{#1}%
  \ifx\@optionalarg\empty \else
    % Save \@optionalarg before it is trashed by \readauxfile.
    \let\@xrefnoptarg\@optionalarg
    % Read .aux file -- \@optionalarg may contain cross-references.
    \readauxfile
    % Typeset the \@optionalarg
    {\@@hloff\@xrefnoptarg}\reftie
  \fi
  % Typeset the reference.
  \plain@xrefn{#1}%
  \hlend@impl{ref}%
}%
% This is equivalent to \xrefn but without the hyperlink stuff and the
% optional parameter so that its expansion does not contain the many
% \let's, \def's and such which the expansion of \xrefn contains.  It
% is needed by \eqdefn and \eqsubdefn to be able to `cleanly' write
% texts of equation labels to .aux file.  It is also used when
% functionality of \xrefn is needed without the hyperlink and the
% optional parameter.
\def\plain@xrefn#1{%
  \readauxfile
  %
  \expandafter \ifx\csname\xrlabel{#1}\endcsname\relax
    \if@citewarning
       \message{\linenumber Undefined label `#1'.}%
    \fi
    %
    % Give it a dummy definition, though, to stop multiple error messages.
    \expandafter\def\csname\xrlabel{#1}\endcsname{%
      `{\tt
        \escapechar = -1
        \expandafter\string\csname#1\endcsname
      }'%
    }%
  \fi
  \csname\xrlabel{#1}\endcsname % Always produce something.
}%
%
% \refn is just a synonym.
% 
\let\refn = \xrefn
%
% One common case: print `p. ' before the page number.
% 
\def\xrefpageword{p.\thinspace}%
%
\def\xref{\@getoptionalarg\@finxref}%
\def\@finxref#1{%
  % Hyperlink and optional TEXT.
  \hlstart@impl{xref}{#1}%
  \ifx\@optionalarg\empty \else
    {\@@hloff\@optionalarg}\refspace
  \fi
  % Typeset page reference but omit hyperlink.
  \xrefpageword\plain@xrefn{#1}%
  \hlend@impl{xref}%
}%
%
% \ref[TEXT]{LABEL} typesets TEXT, \CLASSword for LABEL's class (if it's defined)
% and then does \refn on LABEL. But amstex also has a \ref, so tell the
% user if they try to use \ref and have loaded amsppt.sty.
% 
% \refs{LABEL} is similar, but puts the letter `s' after the \...word, thus
% producing (for example) `Figures 1.2' (presumably to be followed by
% `and~\refn{fig-1.3}').
% 
% Note that \ref takes an optional TEXT, while \refs does not.  This
% text (together with \CLASSword) will become part of the hyperlink
% generated by \ref.  \refs does not include \CLASSword in the
% hyperlink.
%
\def\@maybewarnref{%
  \ifundefined{amsppt.sty}%
    % No amsppt.sty, so just use ours.
  \else
    \message{Warning: amsppt.sty and Eplain both define \string\ref. See
             the Eplain manual.}%
    % Remember their definition.
    \let\amsref = \ref
  \fi
  \let\ref = \eplainref
  \ref
}
\let\ref = \@maybewarnref
%
\def\eplainref{\@getoptionalarg\@fineplainref}%
\def\@fineplainref{\@generalref{1}{}}%
\def\refs{\let\@optionalarg\empty \@generalref{0}s}%
%
% #1 is an integer flag which tells whether hyperlink should include
% \@optionalarg and class word (non-0) or not (0).  #2 is the text to
% follow the \...word.  Both #1 and #2 are supplied by the macros
% above.  #3 comes from the document, and is the LABEL.
%
% \reftie separates \@optionalarg or \..word from \refn; \refspace
% separates \@optionalarg from \..word.
%
\def\@generalref#1#2#3{%
  % Save \@optionalarg before it is trashed by \readauxfile.
  \let\@generalrefoptarg\@optionalarg
  \readauxfile
  % Maybe start hyperlink here.
  \ifcase#1 \else \hlstart@impl{ref}{#3}\fi
  % Get the class of the label.
  \edef\@generalref@class{\getproperty{\xrlabel{#3}}{class}}%
  % If the word for this class is not defined, don't complain.
  \expandafter\ifx\csname \@generalref@class word\endcsname\relax
    % Produce \@optionalarg followed by a \reftie, if \@optionalarg is
    % not empty.
    \ifx\@generalrefoptarg\empty \else {\@@hloff\@generalrefoptarg\reftie}\fi
  \else
    \begingroup
      \@@hloff
      % Produce \@optionalarg, if not empty, followed by a \refspace.
      \ifx\@generalrefoptarg\empty \else \@generalrefoptarg \refspace \fi
      % Produce the word.
      \csname \@generalref@class word\endcsname
      % Add the suffix and then put in a \reftie before the \refn.
      #2\reftie
    \endgroup
  \fi
  % Maybe start hyperlink here.
  \ifcase#1 \hlstart@impl{ref}{#3}\fi
  % Typeset the reference omitting hyperlinks.
  \plain@xrefn{#3}%
  \hlend@impl{ref}%
}%
%
%
% References to equations are similar.
%
% \eqref{foo} produces ``(<text for equation label foo>)''.
% \eqdefn{foo} advances \eqnumber, resets \eqsubnumber, and defines
%   `foo' to be the new number. 
% \eqsubdefn{foo} advances \eqsubnumber and defines `foo'.  \eqref works
%   for both equations and subequations,
% \eqdef{foo} does \eqdefn, then inserts an \eqno and \eqref.
% \eqsubdef{foo} does \eqsubdefn, then what \eqdef does.
% 
% The non-``sub'' macros also take an optional argument; if it's
% present, we use it as the text for the equation label, instead of the
% various counters.
%
% Because there are no page break issues with equations, we can
% immediately define the control sequence.  But we also need to write
% the definition out, in case the user wants to forward reference an
% equation (bad style as that may be).
%
% The current equation number is in \eqnumber; we just advance it by one
% for each \eqdef.  You can handle fancier equation numbers (e.g., ones
% that include a chapter number) by redefining \eqprint, below, and
% using your own counters.  We do provide for one level of substructure,
% since that's more painful to implement than superstructures.
%
\newcount\eqnumber
\newcount\subeqnumber
%
%
% \eqdefn[TEXT]{LABEL} defines LABEL to be TEXT (if it's present),
% otherwise it advances \eqnumber and defines LABEL to be that.  It
% doesn't produce anything.
% 
\def\eqdefn{\@getoptionalarg\@fineqdefn}%
\def\@fineqdefn#1{%
  \ifx\@optionalarg\empty
    \global\advance\eqnumber by 1
    % We call \eqconstruct here instead of in \@eqdefn because we don't
    % want to expand it for \eqsubdefn -- \eqsubdefn already includes an
    % \eqrefn which includes the text of the label which was \eqconstructed.
    \def\temp{\eqconstruct{\number\eqnumber}}%
  \else
    % In the next \def there is (I believe) a spurious \noexpand. 
    % I leave in the old definition, albeit commented out, in case the
    % \noexpand really _is_ necessary. But I don't thikn so. Adam Lewenberg
%    \def\temp{\noexpand\@optionalarg}%
    \def\temp{\@optionalarg}% 
  \fi
  %
  % Always reset the current subequation number:
  \global\subeqnumber = 0
  %
  % Remember this label, so that we can define subequations:
  \gdef\@currenteqlabel{#1}%
  \toks0 = \expandafter{\@currenteqlabel}%
  %
  % Actually do the definition, replacing \xrefn and \eqrefn with
  % \plain@xrefn and taking precautions not to expand \plain@xrefn
  % in what we output to the aux file.  \plain@xrefn expands to many things,
  % including \count@'s and \edef's and the expansion of \xrlabel, and
  % it's just a real mess.
  \begingroup
    \def\eqrefn{\noexpand\plain@xrefn}%
    \def\xrefn{\noexpand\plain@xrefn}%
    \edef\temp{\noexpand\@eqdefn{\the\toks0}{\temp}}%
    \temp
  \endgroup
}%
% 
% 
% \eqsubdefn defines its argument as a ``subequation'' of the last \eqdef.
%
\def\eqsubdefn#1{%
  \global\advance\subeqnumber by 1
  \toks0 = {#1}%
  %
  % Get the text of the label;
  \toks2 = \expandafter{\@currenteqlabel}%
  %
  % We must expand \@currenteqlabel.  We have to not expand
  % \eqsubreftext here, as well as \plain@xrefn (which we substitute
  % for \eqrefn and \xrefn), since the first arg to \eqsubreftext
  % could also include lots of complicated things.
  \begingroup
    \def\eqrefn{\noexpand\plain@xrefn}%
    \def\xrefn{\noexpand\plain@xrefn}%
    \def\eqsubreftext{\noexpand\eqsubreftext}%
    \edef\temp{%
      \noexpand\@eqdefn
        {\the\toks0}%
        {\eqsubreftext{\eqrefn{\the\toks2}}{\the\subeqnumber}}%
    }%
    \temp           
  \endgroup
}%
% 
% \@eqdefn{LABEL}{REF-TEXT} actually handles the equation number
% definitions and writing to the aux file.
% 
% In contrast to \xrdef, we define LABEL right away (as REF-TEXT).  We
% can do this since we know right now what the right equation number is.
% This eliminates some unnecessary warning.  It also lets the user put
% \eqdef{} on all equations and have it work, since then \eqref
% refers to the just-defined new value.  This however needs special
% treatment with respect to hyperlink labels which should not be
% duplicated.  For each \eqdef{}, we define a destination with a
% unique hyperlink label.  These labels can be customized with:
\newcount\phantomeqnumber
\def\phantomeqlabel{PHEQ\the\phantomeqnumber}%
%
\def\@eqdefn#1#2{%
  % Write out xref into the aux file to allow forward references, but
  % only if the label is not empty.  Otherwise numerous redefinitions
  % of the empty label might be reported.
  \ifempty{#1}%
    % LABEL is empty, so generate a `phantom' label.
    \global\advance\phantomeqnumber by 1
    \edef\hl@eqlabel{\phantomeqlabel}%
    % Do not write to the aux file but read it anyway, so that
    % \@definelabel's get a chance to do the checks for multiple
    % definitions before we define any cross-references.
    \readauxfile
  \else
    % LABEL is not empty, so use it for hyperlinks.
    \def\hl@eqlabel{#1}%
    % Write the definition to the aux file, but do not create
    % hyperlink destination (we'll create it ourselves below).
    % This also reads the aux file.
    {\@@hldestoff \definexref{#1}{#2}{eq}}%
  \fi
  % Define hyperlink destination.
  \hldest@impl{eq}{\hl@eqlabel}%
  % Define the LABEL omitting redefinition check.
  \begingroup % \@definelabel@nocheck will end this group.
    \@definelabel@nocheck{#1}{#2}{eq}%
}%
% 
% \eqdef{LABEL} defines LABEL, with \eqdefn, then prints it.  We allow
% an optional argument to explicitly specify the text which we define
% the label as.
%
\def\eqdef{\@getoptionalarg\@fineqdef}%
\def\@fineqdef{%
  \toks0 = \expandafter{\@optionalarg}%
  \edef\temp{\noexpand\@eqdef{\noexpand\eqdefn[\the\toks0]}}%
  \temp
}%
%
% \eqsubdef is to \eqdef as \eqsubdefn is to \eqdefn.  No optional
% argument allowed here.
% 
\def\eqsubdef{\@eqdef\eqsubdefn}%
%
% \@eqdef{DEFN-CMD}{LABEL} defines LABEL, using DEFN-CMD.  Then it
% inserts an \eqno (unless it's called when an \eqno would be invalid).
% Then it prints the newly-defined value using \eqprint.
% 
\def\@eqdef#1#2{%
  \@maybedisableeqno
  \eqnum #1{#2}% Define the label and hyperlink destination.
        \let\@optionalarg\empty % \@fineqref will examine \@optionalarg.
        {\@@hloff\@fineqref{#2}}% Print the text without a hyperlink.
  \@mayberestoreeqno
  \ignorespaces
}%
% 
% 
% If we are in an alignment or some other inner place, \eqno won't work.
% 
\let\@mayberestoreeqno = \relax
%
\def\@maybedisableeqno{%
  \ifinner
    \global\let\eqno = \relax
    \global\let\leqno = \relax
    \global\let\@mayberestoreeqno = \@restoreeqno
  \fi
}%
%
% This makes `\eqno' mean \eqno again.
% 
\let\@primitiveeqno = \eqno
\let\@primitiveleqno = \leqno
\def\@restoreeqno{%
  \global\let\eqno = \@primitiveeqno
  \global\let\leqno = \@primitiveleqno
  \global\let\@mayberestoreeqno = \empty
}%
%
%
% \righteqnumbers/\lefteqnumbers configure \eqnum, \eqdef
% and \eqalignnum to place the equation number against right/left
% margin respectively.
%
\def\righteqnumbers{%
  \def\eqnum{\eqno}%
  \def\eqalignnum{\eqalignno}%
}%
%
\def\lefteqnumbers{%
  \def\eqnum{\leqno}%
  \def\eqalignnum{\leqalignno}%
}%
%
\righteqnumbers
%
% 
% \eqrefn[TEXT]{LABEL} produces the text for the equation label LABEL, or
% something suitable if LABEL is undefined.  (It possibly issues a
% warning in the latter case as well.)  TEXT followed by a \reftie is
% prepended to the equation text as part of the hyperlink.
%
\def\eqrefn{\@getoptionalarg\@fineqrefn}%
\def\@fineqrefn#1{%
  \eqref@start{#1}%
  % Typeset the equation reference but do not produce hyperlink.
  \plain@xrefn{#1}%
  \hlend@impl{eq}%
}%
%
% \eqref[TEXT]{LABEL} is the usual way to refer to equation labels; it
% calls \eqprint on the text of LABEL, prepending it with TEXT and a
% \reftie as part of the hyperlink.
% 
\def\eqref{\@getoptionalarg\@fineqref}%
\def\@fineqref#1{%
  \eqref@start{#1}%
  % Typeset the equation reference but do not produce hyperlink.
  \eqprint{\plain@xrefn{#1}}%
  \hlend@impl{eq}%
}%
% Common code executed at the start of \@fineqrefn and \@fineqref.
\def\eqref@start#1{%
  % Save \@optionalarg in case \phantomeqlabel is redefined by the
  % user to something that trashes it.
  \let\@eqrefoptarg\@optionalarg
  % Hyperlink (fetch last `phantom' equation label if LABEL is empty).
  \ifempty{#1}%
    \hlstart@impl{eq}{\phantomeqlabel}%
  \else
    \hlstart@impl{eq}{#1}%
  \fi
  % Optional TEXT followed by a \reftie.
  \ifx\@eqrefoptarg\empty \else
    {\@@hloff\@eqrefoptarg\reftie}%
  \fi
}%
%
% 
% \eqconstruct{EQ-TEXT} constructs an equation number, i.e., the text to
% be defined as the value of a label.
% 
\let\eqconstruct = \identity
%
% \eqprint{EQ-TEXT} produces the typeset equation number EQ-TEXT.
%
\def\eqprint#1{(#1)}%
%
% \eqsubreftext{EQ-TEXT}{SUBEQ-TEXT} produces the text of a subequation
% reference.  (\eqprint is later called on the result of this to produce
% output for subequations; I didn't define any \subeqprint.)
% 
\def\eqsubreftext#1#2{#1.#2}%
%
%
% 
%  Indexing.
% 
% \defineindex{PREFIX} defines an index with ``prefix'' PREFIX.  The
% prefix is used to construct the output filename and the various
% commands.  We just define all the index commands for this index to
% call the general commands with PREFIX.
%
\let\extraidxcmdsuffixes = \empty
%
\def\defineindex#1{%
  \def\@idxprefix{#1}%
  %
  % Define a switch to control opening and writing of the index file
  % for this prefix.
  \expandafter\innernewif\csname if\@idxprefix dx\endcsname
  \csname \@idxprefix dxtrue\endcsname
  %
  % Define the indexing commands for this prefix. 
  \for\@idxcmd:=,marked,submarked,name%
                \extraidxcmdsuffixes\do
  {%
    \@defineindexcmd\@idxcmd
  }%
  %
  % Allocate a stream for the output.
  \ece\innernewwrite{@#1indexfile}%
  %
  % And a conditional to test whether we've opened the file.
  \ece\innernewif{if@#1indexfileopened}%
}%
%
%
% \@defineindexcmd{SUFFIX} defines both silent and non-silent index
% command for prefix \@idxprefix with suffix SUFFIX.  That is, we define
% both `\@idxprefix dxSUFFIX' and `\s\@idxprefix dxSUFFIX' to call the
% corresponding generic command with \@idxprefix.  \silentindexentry is
% used to decide whether we should ignore following spaces.
% 
\newif\ifsilentindexentry
%
\def\@defineindexcmd#1{%
  \@defineoneindexcmd{s}{#1}\silentindexentrytrue
  \@defineoneindexcmd{}{#1}\silentindexentryfalse
}%
%
%
% \@defineoneindexcmd{PREFIX}{SUFFIX}{PRECALL} does just one silent or
% non-silent commands.  We define the command `\@@PREFIXidxSUFFIX' to do
% PRECALL, then define \@idxprefix, then call \@idxgetrange with an
% argument of `\@@{,s}idxSUFFIX'.  (So far every indexing command 
% should allow a range.  If not, you could redefine `\@@{,s}idxSUFFIX'
% after this macro is called.)
% 
\def\@defineoneindexcmd#1#2#3{%
  \toks@ = {#3}%
  \edef\temp{%
    \def
      % We have to restrict expansion because the generic (\@@...)
      % commands will be defined after the first call to \defineindex.
      % Not expanding the user (\idx...) commands is unnecessary unless
      % the user has defined some new commands, but may as well be cautious.
      \expandonce\csname#1\@idxprefix dx#2\endcsname % e.g., \idx or \sidxname.
      {\def\noexpand\@idxprefix{\@idxprefix}% define \@idxprefix
       % call, e.g., \@@idx or \@@sidxname:
       \expandonce\csname @@#1idx#2\endcsname
      }%
    \def
      \expandonce\csname @@#1idx#2\endcsname{% e.g., \@@idx
        % First do PRECALL.
        \the\toks@
        % Then call \@idxgetrange with, e.g., \@idx or \@sidxname as its arg.
        \noexpand\@idxgetrange\expandonce\csname @#1idx#2\endcsname
      }%
  }%
  \temp
}%
%
%
% \@idxwrite{TERM}{PAGENO} writes a general index entry for TERM on page
% PAGENO to the index file `\@idxprefix indexfile'.  We open the stream
% as `\indexfilebasename.\@idxprefix dx' if it isn't already open.  We
% only write out the index term if this has not been disabled with
% \csname\@idxprefix dxfalse\endcsname.
%
\let\indexfilebasename = \jobname
%
\def\@idxwrite#1#2{%
  \csname if\@idxprefix dx\endcsname
    % Be sure the file is opened.
    \@openidxfile
    %
    % Save the index term.
    \def\temp{#1}%
    %
    % Write the index term and page number.
    \edef\@wr{%
      \expandafter\write\csname @\@idxprefix indexfile\endcsname{%
        \string\indexentry
        {\sanitize\temp}%
        {\noexpand#2}%
      }%
    }%
    \@wr
  \else
    % Produce a whatsit anyway, to ensure consistent page-breaking.
    \write-1{}%
  \fi
  % 
  % Marginalize the index term, if desired.  We \sanitize the term to
  % avoid expansion of any (user-defined) active characters that made
  % it through (e.g., because they had not been added to \dospecials
  % and hence were not read verbatim by the indexing commands).
  \ifindexproofing
    \def\temp{#1}%
    \edef\temp{%
      \insert\@indexproof{\noexpand\indexproofterm{\sanitize\temp}}%
    }%
    \temp
    % Allow `infinitesimal' in \sidx{Infinitesimal}infinitesimal to be
    % hyphenated.
    \ifhmode\allowhyphens\fi
  \fi
  %
  % We just appended at least one non-discardable item (namely, the
  % whatsit from the \write) to the current list.  So in case glue comes
  % next (not unlikely), be sure we don't inadvertently make that glue a
  % valid breakpoint, if it wouldn't have been without us.
  \hookrun{afterindexterm}%
  %
  % This is the end of the index entry processing.  If this was a silent
  % entry, ignore following spaces.
  \ifsilentindexentry \expandafter\ignorespaces\fi
}%
%
\def\@openidxfile{%
  \csname if@\@idxprefix indexfileopened\endcsname \else
    \expandafter\immediate\openout\csname @\@idxprefix indexfile\endcsname =
      \indexfilebasename.\@idxprefix dx
    \expandafter\global\csname @\@idxprefix indexfileopenedtrue\endcsname
  \fi
}%
%
%
% If this conditional is true, we output the index terms on the page
% where they occur.
\newif\ifindexproofing
%
% We need a new insertion class to collect the proofed terms.
\newinsert\@indexproof
\dimen\@indexproof = \maxdimen                  % No limit on number of terms.
\count\@indexproof = 0  \skip\@indexproof = 0pt % They take up no space.
%
% This actually typesets the proofed term.  We don't go to any lengths
% to provide nice-looking output; since the term might have all kinds of
% weird characters in it, we just dump it in the smallest standard
% Computer Modern typewriter font.
% 
% We put the term in an \hbox, even though that might make the output
% run off the page, since we don't really need to see all of it, and
% I think it's better to opt for simplicity -- one term per line.
%
\font\indexprooffont = cmtt8
\def\indexproofterm#1{\hbox{\strut \indexprooffont #1}}%
%
%
% If \output doesn't use \makeheadline, or redefines it, it's up to the
% new \output to define index hyperlink destination and to call
% \indexproofunbox.
%
\let\@plainmakeheadline = \makeheadline
\def\makeheadline{%
  % Define index page destination only when this `page anchor' is defined.
  \expandafter\ifx\csname\idxpageanchor{\folio}\endcsname\relax \else
    % Say \@@hldeston in case a page break happened at an unfortunate
    % time when user said \hldestoff.
    {\@@hldeston \hldest@impl{idx}{\hlidxpagelabel{\folio}}}%
  \fi
  \indexproofunbox
  \@plainmakeheadline
}%
%
% We want to put the proof index terms in the margin, outside the
% printed area. So if \outsidemargin (for odd pages) and \insidemargin
% (for even pages) are undefined, we define them (both) to be the default
% TeX margin -- one inch + \hoffset.
\def\indexsetmargins{%
  \ifx\undefined\outsidemargin
    \dimen@ = 1truein
    \advance\dimen@ by \hoffset
    \edef\outsidemargin{\the\dimen@}%
    \let\insidemargin = \outsidemargin
  \fi
}%
%
% We always put the terms in the right-hand margin, so long terms run
% off the page, instead of into the text.
\def\indexproofunbox{%
  \ifvoid\@indexproof\else
    \indexsetmargins
    \rlap{%
      \kern\hsize
      \ifodd\pageno \kern\outsidemargin \else \kern\insidemargin \fi
      \vbox to 0pt{\unvbox\@indexproof\vss}%
    }\nointerlineskip
  \fi
}%
%
%
% \@idxgetrange\CS parses an optional argument which, if present, should
% be either `begin' or `end', marking the beginning or ending of a range
% for the index entry.  If we find this, we set the appropriate one of
% \@idxrangestr.  Then we call \CS.
% 
% If the optional argument is `see' or `seealso' we read another
% argument, namely, the entry to see.
% 
\def\idxrangebeginword{begin}%
\def\idxbeginrangemark{(}% the corresponding magic char to go in the idx file
%
\def\idxrangeendword{end}%
\def\idxendrangemark{)}%
%
\def\idxseecmdword{see}%
\def\idxseealsocmdword{seealso}%
\newif\if@idxsee
\newif\if@hlidxencap
\let\@idxseenterm = \relax
%
\def\idxpagemarkupcmdword{pagemarkup}%
\let\@idxpagemarkup = \relax
%
\def\@idxgetrange#1{%
  \let\@idxrangestr = \empty
  \let\@afteridxgetrange = #1%
  % Since \@getoptionalarg scans ahead, it might scan the \idxargopen
  % character of the following non-optional argument if the optional
  % argument is missing.  To make sure that \idxargopen gets the right
  % catcode, we need to set it up before calling \@getoptionalarg.
  \begingroup
    \catcode\idxargopen=1
    \@getoptionalarg\@finidxgetopt
}%
\def\@finidxgetopt{%
    \global\let\@idxgetrange@arg\@optionalarg
  \endgroup
  %
  \@hlidxencaptrue
  %
  \for\@idxarg:=\@idxgetrange@arg\do{%
    % These are ordered by my guess at frequency of use.
    \expandafter\@idxcheckpagemarkup\@idxarg=,%
    %
    \ifx\@idxarg\idxrangebeginword
      \def\@idxrangestr{\idxencapoperator\idxbeginrangemark}%
    \else
      \ifx\@idxarg\idxrangeendword
        \def\@idxrangestr{\idxencapoperator\idxendrangemark}%
        \@hlidxencapfalse
      \else
        \ifx\@idxarg\idxseecmdword
          \def\@idxpagemarkup{indexsee}%
          \@idxseetrue
          \@hlidxencapfalse
        \else
          \ifx\@idxarg\idxseealsocmdword
            \def\@idxpagemarkup{indexseealso}%
            \@idxseetrue
            \@hlidxencapfalse
          \else
             \ifx\@idxpagemarkup\relax
               \errmessage{Unrecognized index option `\@idxarg'}%
             \fi
          \fi
        \fi
      \fi
    \fi
  }%
  % Stick hyperlink encapsulator into \@idxpagemarkup.
  \ifnum\hldest@place@idx < 0 \else
    \if@hlidxencap
      \ifx\@idxpagemarkup\relax
        % Even when user gives no pagemarkup command, we still do
        % insert our hyperlink encapsulator.
        \let\@idxpagemarkup\empty
      \fi
      \ifcase\hldest@place@idx \relax
        % \hldest@place@idx = 0, dests point to a page with a term.
        \edef\@idxpagemarkup{hlidxpage{\@idxpagemarkup}}%
        % We want to define index \hldest's only on those
        % pages which contain at least one index term, so this
        % `page anchor' will tell \makeheadline on which pages to
        % generate an \hldest.
        \definepageanchor{\noexpand\folio}%
      \else
        % \hldest@place@idx = 1, dests point to exact location of a term.
        \global\advance\hlidxlabelnumber by 1
        \edef\@idxpagemarkup{hlidx{\hlidxlabel}{\@idxpagemarkup}}%
        \hldest@impl{idx}{\hlidxlabel}%
      \fi
    \fi
  \fi
  %
  \@afteridxgetrange
}%
%
%
% Check for a command of the form `pagemarkup=\cmd', and if found, set
% \@idxpagemarkup to `cmd'.
% 
\def\@idxcheckpagemarkup#1=#2,{%
  \def\temp{#1}%
  \ifx\temp\idxpagemarkupcmdword
    \if ,#2, % If #2 is empty, complain.
      \errmessage{Missing markup command to `pagemarkup'}%
    \else
      % Remove a trailing =.
      \def\temp##1={##1}%
      \edef\@idxpagemarkup{\temp\string#2}%
    \fi
  \fi
}%
%
%
% \hlidxpage and \hlidx are hyperlink encapsulators for the two types
% of hyperlink destinations for index terms.
%
% \hldest@place@idx defines which type is selected.  When defined to 0,
% we generate destinations pointing to the page on which the indexed
% term is located.  When defined to 1, we generate destinations
% pointing to exact location of the indexed term.  When negative, we
% generate no hyperlinks / destinations.
\def\hldest@place@idx{-1}%
%
% \idxpageanchor{PAGE} expands to index page anchor internal name.
% This page anchor is used when index entries point to pages.  Like in
% \xrlabel, we append an _ character to NAME, to help avoid conflicts.
% And we append a `p' so that we don't redefine \_ on an empty label.
\begingroup
  % Mike Spivak's MathTime macros for Times Roman fonts changes the
  % catcode of _ to be active. Undo that. (From adam@symcom.math.uiuc.edu.)
  \catcode`\_ = 8
  \gdef\idxpageanchor#1{#1_p}%
\endgroup
% \definepageanchor{PAGE} defines a page anchor for the page PAGE.
% \makeheadline will then define a hyperlink destination on top of
% each page for which an anchor is defined.  To get a possible page
% number right, we have to write the definition out to the auxiliary
% file, instead of only defining it directly.
\def\definepageanchor#1{%
  % Be sure we've read the aux file before we zap it:
  \readauxfile
  %
  % When we read in the aux file next time, define the label:
  \edef\@wr{\noexpand\writeaux{\string\@definepageanchor{#1}}}%
  \@wr
  \ignorespaces
}%
% \@definepageanchor{PAGE} actually defines page anchor for the PAGE.
\def\@definepageanchor#1{%
  \expandafter\gdef\csname\idxpageanchor{#1}\endcsname{}%
}%
%
% Hyperlink labels for both types of destinations.
\newcount\hlidxlabelnumber
\def\hlidxlabel{IDX\the\hlidxlabelnumber}%
\def\hlidxpagelabel#1{IDXPG#1}%
%
% \hlidx{HYPERLINK-LABEL}{PAGEENCAP}{PAGESPEC}
\def\hlidx#1#2#3{%
  \ifempty{#2}%
    \let\@idxpageencap\relax
  \else
    \expandafter\let\expandafter\@idxpageencap\csname #2\endcsname
  \fi
  \hlstart@impl{idx}{#1}%
  \@idxpageencap{#3}%
  \hlend@impl{idx}%
}%
%
% \hlidxpage{PAGEENCAP}{PAGESPEC}
%
% We expect PAGESPEC to be one of the following:
%   14          (single page number)
%   14, 15      (list of two consecutive page numbers)
%   14--15      (page range)
%
% If you configure MakeIndex to use different page list separator
% (delim_n parameter in .mst file) / different page range separator
% (delim_r parameter), you should call \setidxpagelistdelimiter /
% \setidxpagerangedelimiter to setup the new separators.
%
% In case of a single page number, we call \PAGEENCAP on the page
% number and turn the result into a hyperlink.
%
% In case of a two-page list, we break the list, call \PAGEENCAP on
% each of the page numbers separately, and turn each result into a
% hyperlink, producing page list separator between the page numbers.
%
% In case of a page range, we do not break the range, call \PAGEENCAP
% on the entire range and turn what results into a hyperlink for the
% first of the page numbers.
\def\hlidxpage#1#2{%
  % Set \@idxpagei[i][ref] and \@idxpagesep
  \@hlidxgetpages{#2}%
  % Alias for \PAGEENCAP.
  \ifempty{#1}%
    \let\@idxpageencap\relax
  \else
    \expandafter\let\expandafter\@idxpageencap\csname #1\endcsname
  \fi
  % Now comes the first part, to be done in any of the three cases.
  \hlstart@impl{idx}{\hlidxpagelabel{\@idxpageiref}}%
  \expandafter\@idxpageencap\expandafter{\@idxpagei}%
  \hlend@impl{idx}%
  % The second part, to be done only if there is a second page number.
  \ifx\@idxpageii\empty \else
    \@idxpagesep
    \hlstart@impl{idx}{\hlidxpagelabel{\@idxpageiiref}}%
    \expandafter\@idxpageencap\expandafter{\@idxpageii}%
    \hlend@impl{idx}%
  \fi
}%
%
% This macro parses PAGESPEC parameter to \hlidxpage.  It sets
% \@idxpagei to the first page number, \@idxpageii to the second (if
% present; if not, \@idxpageii will be set to \empty).  For the first
% page number, use \@idxpageiref to construct hyperlink label; for
% the second page number, use \@idxpageiiref.  Use \@idxpagesep to
% separate the two page numbers.
%
% NOTE:  we use \@idxpagei[i], not \idxpagei[i], because the user may
% use \idxparselist / \idxparserange in his \PAGEENCAP, in which case
% our \idxpagei[i]'s will get garbled.
\def\@hlidxgetpages#1{%
  % Try to parse a two-page list.
  \idxparselist{#1}%
  \ifx\idxpagei\empty
    % It is not a two-page list, try page range.
    \idxparserange{#1}%
    \ifx\idxpagei\empty
      % It is neither a two-page list nor a page range, so we assume
      % it is a single page number.
      \def\@idxpageiref{#1}% Label for \hlstart.
    \else
      % It is a page range.
      \let\@idxpageiref\idxpagei % Label for \hlstart.
    \fi
    \def\@idxpagei{#1}%
    \let\@idxpageii\empty
  \else
    % It is a two-page list.
    \let\@idxpagei\idxpagei
    \let\@idxpageii\idxpageii
    \let\@idxpageiref\idxpagei % Label for \hlstart.
    \let\@idxpageiiref\idxpageii % Label for \hlstart.
    \let\@idxpagesep\idxpagelistdelimiter
  \fi
}%
%
% Set up a macro \@idxparselist and user-accessible \idxparselist for
% parsing two-page list.  Takes list separator (which is saved in
% \idxpagelistdelimiter).  If the list is found, \@idxparselist will
% set \idxpagei to the first page number, \idxpageii to the second;
% if not found, it will set \idxpagei to \empty (this implies that
% the list's first page number should never be empty).
\def\setidxpagelistdelimiter#1{%
  \gdef\idxpagelistdelimiter{#1}%
  \gdef\@removepagelistdelimiter##1#1{##1}%
  \gdef\@idxparselist##1#1##2\@@{%
    \ifempty{##2}%
      \let\idxpagei\empty
    \else
      \def\idxpagei{##1}%
      \edef\idxpageii{\@removepagelistdelimiter##2}%
    \fi
  }%
  \gdef\idxparselist##1{\@idxparselist##1#1\@@}%
}%
%
% Same as previous, but \@idxparserange and \idxparserange will parse
% page range, range separator is saved in \idxpagerangedelimiter.
\def\setidxpagerangedelimiter#1{%
  \gdef\idxpagerangedelimiter{#1}%
  \gdef\@removepagerangedelimiter##1#1{##1}%
  \gdef\@idxparserange##1#1##2\@@{%
    \ifempty{##2}%
      \let\idxpagei\empty
    \else
      \def\idxpagei{##1}%
      \edef\idxpageii{\@removepagerangedelimiter##2}%
    \fi
  }%
  \gdef\idxparserange##1{\@idxparserange##1#1\@@}%
}%
%
% Set up the default delimiters.
\setidxpagelistdelimiter{, }%
\setidxpagerangedelimiter{--}%
%
%
% \@idxtokscollect uses \@idxmaintoks as the token list for the main
% part of an index entry and \@idxsubtoks for the subpart.  Then it
% calls \@idxwrite.
% 
\def\idxsubentryseparator{!}%
\def\idxencapoperator{|}%
\def\idxmaxpagenum{99999}%
%
\newtoks\@idxmaintoks
\newtoks\@idxsubtoks
%
\def\@idxtokscollect{%
  % Remember the subentry.
  \edef\temp{\the\@idxsubtoks}%
  %
  % We want to expand the conditions, but not the terms.  The index
  % entry starts simply with \@idxmaintoks and \@idxsubtoks.
  \edef\@indexentry{%
    \the\@idxmaintoks
    \ifx\temp\empty\else \idxsubentryseparator\the\@idxsubtoks \fi
    \@idxrangestr
  }%
  %
  % If this is a `see' or `see also' entry, we need to read one more
  % arg.  We use a giant page number so the entry will be last (for the
  % benefit of `see also's).  MakeIndex rejects page numbers >=1000.
  % 
  \if@idxsee
    \@idxseefalse % Reset so the next term won't be a `see'.
    \edef\temp{\noexpand\idx@getverbatimarg
      {\noexpand\@finidxtokscollect{\idxmaxpagenum}}}%
  \else
    \def\temp{\@finfinidxtokscollect\folio}%
  \fi
  \temp
}%
%
%
% \@finidxtokscollect{PAGENO}{REAL-TERM} reads the final term for
% see/see also entries.  We do not check if the person has put both a
% range and a see in the same index term (which will confuse makeindex).
% 
\def\@finidxtokscollect#1#2{%
  \def\@idxseenterm{#2}%
  \@finfinidxtokscollect{#1}%
}%
%
% \@finfinidxtokscollect{PAGENO} writes \@indexentry for page PAGENO.
% Besides \@indexentry, if \@idxpagemarkup is not \relax we output an
% index entry \@indexentry|\@idxpagemarkup{PAGENO} (but omitting | if
% this is a range term, because in that case | will have been added
% together with \@idxrangestr in \@idxtokscollect).  And if
% \@idxseenterm is not \relax we output {\@idxseenterm} after the
% \@idxpagemarkup.  (This will become an argument to the ``markup''
% command, which will be \indexsee or \indexseealso.)
% 
\def\@finfinidxtokscollect#1{%
  % If we've got a page markup command, append it.
  \ifx\@idxpagemarkup\relax \else
    \toks@ = \expandafter{\@indexentry}%
    \edef\@indexentry{%
      \the\toks@
      % Add | only if this is not a range term, otherwise | has
      % already been added with \@idxrangestr by \@idxtokscollect.
      \ifx\@idxrangestr\empty \idxencapoperator \fi
      \@idxpagemarkup
    }%
    \let\@idxpagemarkup = \relax
  \fi
  %
  % If we've got an argument to the ``page markup'' command, append it.
  \ifx\@idxseenterm\relax \else
    \toks@ = \expandafter{\@indexentry}%
    \edef\@indexentry{\the\toks@{\sanitize\@idxseenterm}}%
    \let\@idxseenterm = \relax
  \fi
  %
  % Finally, write what we've constructed.
  \expandafter\@idxwrite\expandafter{\@indexentry}{#1}%
}%
%
%
% \@idxcollect{MAIN}{SUB} sets up the token registers
% \@idx{main,sub}toks, then calls \@idxtokscollect.  This is convenient
% for some of the macros below.
% 
\def\@idxcollect#1#2{%
  \@idxmaintoks = {#1}%
  \@idxsubtoks = {#2}%
  \@idxtokscollect
}%
%
%
% Macros for reading verbatim the TERM, SUBTERM, SEE and SEEALSO
% arguments of indexing commands.
%
% These can be customized by the user to different beginning and
% end of group characters, so that `{' and `}' can be used inside
% index terms.
\def\idxargopen{`\{}%
\def\idxargclose{`\}}%
% \idx@getverbatimarg#1{ARG} reads ARG verbatim and then calls #1 with
% that argument.  We use it to read all TERM, (non-optional) SUBTERM,
% SEE and SEEALSO arguments.
\def\idx@getverbatimarg#1{%
  \def\idx@getverbatimarg@cmd{#1}% Save the command, allowing it to be
                                 % more than just a command sequence.
  \begingroup
    \uncatcodespecials
    \catcode\idxargopen=1
    \catcode\idxargclose=2
    \catcode`\ =10   % Swallow multiple consequent spaces.
    \catcode`\^^I=10 % Just in case, to avoid dependence on
    \catcode`\^^M=5  % current meanings of tabs and newlines.
    \idx@fingetverbatimarg
}%
\def\idx@fingetverbatimarg#1{\endgroup\idx@getverbatimarg@cmd{#1}}%
% \idx@getverboptarg#1[ARG] reads ARG verbatim and then calls #1 with
% that argument.  We use it to read all optional SUBTERM arguments.
% This is adopted from btxmac.tex's \@getoptionalarg.
\def\idx@getverboptarg#1{%
  \def\idx@optionaltemp{#1}% Save the command, allowing it to be more
            % than just a command sequence (unlike \@getoptionalarg).
  \let\idx@optionalnext = \relax
  % If this is a SEE or SEEALSO entry, we know that another
  % non-optional arg follows, delimited by the \idxargopen and
  % \idxargclose characters.  Since \@futurenonspacelet scans ahead,
  % it might scan \idxargopen if the optional argument is missing.  To
  % make sure that \idxargopen gets the right catcode, we need to set
  % it up before calling \@futurenonspacelet.
  \begingroup
    \if@idxsee \catcode\idxargopen=1 \fi
    \@futurenonspacelet\idx@optionalnext\idx@bracketcheck
}%
% The \expandafter's in this macro let us avoid the use of \aftergroup,
% which is somewhat more expensive.
\def\idx@bracketcheck{%
  \ifx [\idx@optionalnext
    \endgroup % Cancel \catcode\idxargopen=1.
    \expandafter\idx@finbracketcheck
  \else
    \endgroup % Cancel \catcode\idxargopen=1.
    \let\@optionalarg = \empty
    % We can't do the \temp after the \fi, because then the \temp gets
    % in the way of reading the optional argument from the input, if
    % we do have one.
    \expandafter\idx@optionaltemp
  \fi
}%
%
\def\idx@finbracketcheck{%
  \begingroup
    \uncatcodespecials
    % `[' should already be \other since \idx@bracketcheck succeeded;
    % we assume that the user also didn't give `]' a funky catcode
    % (otherwise it should have been added to \dospecials, anyway).
    %\catcode`\[=\other
    %\catcode`\]=\other
    \catcode`\ =10   % Swallow multiple consequent spaces.
    \catcode`\^^I=10 % Just in case, to avoid dependence on
    \catcode`\^^M=5  % current meanings of tabs and newlines.
    \idx@@getoptionalarg
}%
%
\def\idx@@getoptionalarg[#1]{%
  \endgroup
  \def\@optionalarg{#1}%
  \idx@optionaltemp
}%
%
% Produce term (which has been scanned verbatim) as output, rescanning
% with `real' catcodes.  \endinput ensures TeX does not see the ending
% newline.  Stepan Kasal verified that it is necessary both for e-TeX
% and for ordinary TeX (see texinfo.tex).  We reset catcode of newline
% because \scantokens (both the e-TeX primitive and the Eplain's
% emulation) generate a trailing newline when newline is catcode 13
% (putting \scantokens in an \hbox also fixes that newline--strange).
% We don't write `\endinput' directly because the merge script strips
% any lines with an \endinput.
{\catcode`\&=0
\gdef\idx@scanterm#1{%
  \edef\idx@scanterm@nl@catcode{\the\catcode`\^^M\relax}%
  \catcode`\^^M=5
  \scantokens{#1&endinput}%
  \catcode`\^^M=\idx@scanterm@nl@catcode
}}%
%
%
% Following are the TeX macros that correspond to the commands
% that actually appear in the document.
%
% \@idx{TERM} produces TERM in the output and then makes the index entry
% for TERM as usual.  We don't allow a [SUBTERM] here since then we
% would lose spaces after the command, which would be very inconvenient.
%
% As with all our index commands, we've already defined \@idxprefix (in
% \idx or whatever), to save passing it around, and we've looked for a
% range argument before TERM.
% 
\def\@idx{\idx@getverbatimarg\@finidx}%
\def\@finidx#1{%
  \idx@scanterm{#1}% Produce TERM as output.
  \@idxcollect{#1}{}%
}%
%
% \@sidx{TERM}[SUBTERM] produces an index entry TERM and no output.  If
% SUBTERM is present, this is a subentry.  (At the moment, I don't
% provide for subsubentries, since I've never needed that.)
% 
\def\@sidx{\idx@getverbatimarg\@finsidx}%
\def\@finsidx#1{\@idxmaintoks = {#1}\idx@getverboptarg\@finfinsidx}%
\def\@finfinsidx{%
  \@idxsubtoks = \expandafter{\@optionalarg}%
  \@idxtokscollect
}%
%
%
% \@idxconstructmarked{TOKS-REG}\CS{TERM}
% 
\def\idxsortkeysep{@}% This `@' is catcode 11, but it doesn't matter.
%
\def\@idxconstructmarked#1#2#3{%
  \toks@ = {#2}% The control sequence.
  \toks2 = {#3}% The term.
  %
  % Construct TERM@\CS{TERM} as the string to write.
  \edef\temp{\the\toks2 \idxsortkeysep \the\toks@{\the\toks2}}%
  %
  % Save it in TOKS-REG.
  #1 = \expandafter{\temp}%
}%
%
%
% \@idxmarked\CS{TERM} outputs \CS{TERM} and then makes an index entry
% sorted by TERM but producing \CS{TERM}.
%
\def\@idxmarked#1{\idx@getverbatimarg{\@finidxmarked{#1}}}%
\def\@finidxmarked#1#2{%
  \idx@scanterm{#1{#2}}% Produce \CS{TERM} as output.
  \@idxconstructmarked\@idxmaintoks{#1}{#2}%
  \@idxsubtoks = {}%
  \@idxtokscollect
}%
%
% \@sidxmarked\CS{TERM}[SUBTERM] outputs an index entry sorted by TERM
% but producing \CS{TERM}.
% 
\def\@sidxmarked#1{\idx@getverbatimarg{\@finsidxmarked{#1}}}%
\def\@finsidxmarked#1#2{%
  \@idxconstructmarked\toks@{#1}{#2}%
  \edef\temp{{\the\toks@}}%
  \expandafter\@finsidx\temp
}%
%
%
% \@idxsubmarked{TERM}\CS{SUBTERM} is like \@idxmarked, except that it's
% SUBTERM that's marked instead of TERM.
% 
\def\@idxsubmarked{%
  \let\sidxsubmarked@print\idxsubmarked@print
  \idx@getverbatimarg\@finsidxsubmarked
}%
\def\idxsubmarked@print{\expandafter\idx@scanterm\expandafter}%
%
% \@sidxsubmarked{TERM}\CS{SUBTERM} is to \@sidxmarked as \@idxsubmarked
% is to \@idxmarked.
% 
\def\@sidxsubmarked{%
  \let\sidxsubmarked@print\gobble
  \idx@getverbatimarg\@finsidxsubmarked
}%
\def\@finsidxsubmarked#1{\@idxmaintoks = {#1}\@@finsidxsubmarked}% Reads TERM.
\def\@@finsidxsubmarked#1{\idx@getverbatimarg{\@@@finsidxsubmarked{#1}}}% Reads \CS.
\def\@@@finsidxsubmarked#1#2{% Reads {\CS}{SUBTERM}.
  \sidxsubmarked@print % Prints for \@idxsubmarked, gobbles for \@sidxsubmarked.
    {\the\@idxmaintoks\space #1{#2}}% Maybe produce `TERM \CS{SUBTERM}' as output.
  \@idxconstructmarked\@idxsubtoks{#1}{#2}%
  \@idxtokscollect
}%
%
%
% \@idxcollectname{FIRST}{LAST} puts `LAST, FIRST' into \temp. (Well,
% we use \idxnameseparator instead of hardwiring `, '.) If FIRST is
% empty, don't include the separator.
% 
\def\idxnameseparator{, }% as in `Tachikawa, Elizabeth'
%
\def\@idxcollectname#1#2{%
  \def\temp{#1}%
  \ifx\temp\empty
    \toks@ = {}%
  \else
    \toks@ = \expandafter{\idxnameseparator #1}%
  \fi
  \toks2 = {#2}%
  %
  \edef\temp{\the\toks2 \the\toks@}%
}%
%
%
% \@idxname{FIRST}{LAST} also produces `FIRST LAST' in the output and an
% index entry for `LAST, FIRST'.
%
\def\@idxname{\idx@getverbatimarg\@finidxname}%
\def\@finidxname#1{\idx@getverbatimarg{\@finfinidxname{#1}}}%
\def\@finfinidxname#1#2{%
  \idx@scanterm{#1 #2}% Separate the names by a space in the output.
  \@idxcollectname{#1}{#2}%
  \expandafter\@idxcollect\expandafter{\temp}{}%
}%
%
% \@sidxname{FIRST}{LAST}[SUBTERM] is to \@sidx as \@idxname is to
% \@idx.
% 
\def\@sidxname{\idx@getverbatimarg\@finsidxname}%
\def\@finsidxname#1{\idx@getverbatimarg{\@finfinsidxname{#1}}}%
\def\@finfinsidxname#1#2{%
  \@idxcollectname{#1}{#2}%
  \expandafter\@finsidx\expandafter{\temp}%
}%
%
%
% Now we come to actually producing the index, i.e., implementing the
% formatting commands that MakeIndex outputs.
%
% \readindexfile is responsible for formatting and printing the index.
% It reads \indexfilebasename.ind.  We implement the same commands that
% LaTeX does.  I suppose we could allow for different indices having
% different basenames, but I can't imagine when that would be useful.
% 
\let\indexfonts = \relax
%
\def\readindexfile#1{%
  \edef\@idxprefix{#1}%
  %
  % Does the output file exist?
  \testfileexistence[\indexfilebasename]{\@idxprefix nd}%
  \iffileexists \begingroup
    % If no \begin or \end, define them. The argument will be `{theindex}'.
    \ifx\begin\undefined
      \def\begin##1{\@beginindex}%
      \let\end = \@gobble
    \fi
    %
    % Read the file:
    \input \indexfilebasename.\@idxprefix nd
    % 
    % \doublecolumns isn't affected by groups.
    \singlecolumn
  \endgroup
  \else
    \message{No index file \indexfilebasename.\@idxprefix nd.}%
  \fi
}%
%
% Here's the default for `\begin{theindex}', if \begin isn't defined.
\def\@beginindex{%
  % Define the commands MakeIndex outputs.
  \let\item = \@indexitem
  \let\subitem = \@indexsubitem
  \let\subsubitem = \@indexsubsubitem
  %
  % Set up the default formatting:
  \indexfonts
  \doublecolumns
  \parindent = 0pt
  %
  % Let the user override the defaults.
  \hookrun{beginindex}%
}%
%
% MakeIndex puts \indexspace between groups in the ind file.
\let\indexspace = \bigbreak
%
% You can make \afterindexterm appear after the term and before the
% first page with the following in the ist file:
% delim_0 "\\afterindexterm "
% delim_1 "\\afterindexterm "
% delim_2 "\\afterindexterm "
\let\afterindexterm = \quad
%
%
% Top-level index entries start with \item.
\newskip\aboveindexitemskipamount  \aboveindexitemskipamount = 0pt plus2pt
\def\aboveindexitemskip{\vskip\aboveindexitemskipamount}%
%
\def\@indexitem{\begingroup
  \@indexitemsetup
  \leftskip = 0pt
  \aboveindexitemskip
  \penalty-100 % Encourage page breaks before items.
  % 
  % But forbid page breaks after items, in case a subitem follows.
  \def\par{\endgraf\endgroup\nobreak}%
}%
%
% Secondary index entries.
\def\@indexsubitem{%
  \@indexitemsetup
  \leftskip = 1em
}%
%
% And tertiary entries.
\def\@indexsubsubitem{%
  \@indexitemsetup
  \leftskip = 2em
}%
%
% Common setup for the formatting.
\def\@indexitemsetup{%
  \par
  \hangindent = 1em
  \raggedright
  \hyphenpenalty = 10000
  \hookrun{indexitem}%
}%
%
%
% \indexsee{TERM}{PAGENO} ignores PAGENO, and says `see TERM'.
\def\seevariant{\it}%
\def\indexseeword{see}%
\def\indexsee{\idx@getverbatimarg\@finindexsee}%
\def\@finindexsee#1#2{{\seevariant \indexseeword\/ }\idx@scanterm{#1}}%
%
% \indexseealso{TERM}{PAGENO} is similar.
\def\indexseealsowords{see also}%
\def\indexseealso{\idx@getverbatimarg\@finindexseealso}%
\def\@finindexseealso#1#2{{\seevariant \indexseealsowords\/ }\idx@scanterm{#1}}%
%
%
% We provide one index by default; commands are \idx, \sidx, etc.
\defineindex{i}%
%
%
% 
%  Justification of multiple input lines.
%
% You use these by saying 
% {\flushright 
% <flush right text>
% }
% 
% and similarly for \flushleft and \center.  The command must be
% embedded in a group.  The lines are set in paragraphs as usual, i.e.,
% blank lines start a new paragraph (by virtue of the
% \blanklineskipamount vertical glue being inserted).
% 
% \environment ... \endenvironment isn't appropriate in this case, since
% these ``environments'' can't be nested.
% 
\begingroup
  \catcode `\^^M = \active %
  \gdef\flushleft{%
    \def\@endjustifycmd{\@endflushleft}%
    \def\@eoljustifyaction{\null\hfil\break}%
    \let\@firstlinejustifyaction = \relax
    \@startjustify %
  }%
  \gdef\flushright{%
    \def\@endjustifycmd{\@endflushright}%
    \def\@eoljustifyaction{\break\null\hfil}%
    \def\@firstlinejustifyaction{\hfil\null}%
    \@startjustify %
  }%
  \gdef\center{%
    \def\@endjustifycmd{\@endcenter}%
    \def\@eoljustifyaction{\hfil\break\null\hfil}%
    \def\@firstlinejustifyaction{\hfil\null}%
    \@startjustify %
  }%
  %
  % We do this before starting any of the justification commands.
  \gdef\@startjustify{%
    \parskip = 0pt
    \catcode`\^^M = \active %
    \def^^M{\futurelet\next\@finjustifyreturn}%
    %
    % \@eateol is called at the beginning of each justified paragraph.
    \def\@eateol##1^^M{%
      \def\temp{##1}%
      \@firstlinejustifyaction %
      \ifx\temp\empty\else \temp^^M\fi %
    }%
    \expandafter\aftergroup\@endjustifycmd %
    \checkenv \environmenttrue %
    \par\noindent %
    \@eateol %
  }%
  %
  % If the next thing is a ^^M, insert \blanklineskipamount glue.  Then
  % do \@eoljustifyaction (which each justification command defines).
  \gdef\@finjustifyreturn{%
    \@eoljustifyaction %
    \ifx\next^^M%
      % Insert extra glue when the \@end... command does the \par.
      \def\par{\endgraf\vskip\blanklineskipamount \global\let\par = \endgraf}%
      \@endjustifycmd %
      % Get back into horizontal mode for the next line.
      \noindent %
      \@firstlinejustifyaction %
    \fi %
  }%
\endgroup
% 
\def\@endflushleft{\unpenalty{\parfillskip = 0pt plus1fil\par}\ignorespaces}%
\def\@endflushright{% Remove the \hfil\null\break we just put on.
   \unskip \setbox0=\lastbox \unpenalty
   % We have fil glue at the left of the line; \parfillskip shouldn't
   % affect that.
   {\parfillskip = 0pt \par}\ignorespaces
}%
\def\@endcenter{% Remove the \hfil\null\break we just put on.
   \unskip \setbox0=\lastbox \unpenalty
   % We have fil glue at the left of the line; \parfillskip must balance it.
   {\parfillskip = 0pt plus1fil \par}\ignorespaces
}%
% 
\ifx\@undefined\raggedleft
% like plain's \raggedright except for \parfillskip.
\innernewskip\raggedleftskip \raggedleftskip=0pt plus2em
\def\raggedleft{\leftskip\raggedleftskip \parindent=0pt
  \spaceskip.3333em \xspaceskip.5em \parfillskip0pt \relax} 
\fi % \raggedleft undefined
%
% 
%  Automatically-columnated tables.
% 
% \makecolumns N/K: organizes the entries on the following N lines into
% K columns.  If N is too small, some text beyond the end of the table
% will be incorporated into the table, probably producing an error
% message.  If N is too large, some of the entries will appear after the
% table, probably looking very out of place.
% 
% You can adjust the position of the table on the page by changing
% \parindent (space to the left of the block) and \hsize (distance from
% the left margin to the right of the block).  (No doubt inside a
% group.)  And you can allow a page break above the valign by changing
% \abovecolumnspenalty.
% 
\newcount\abovecolumnspenalty   \abovecolumnspenalty = 10000
\newcount\@linestogo         % Lines remaining to process.
\newcount\@linestogoincolumn % Lines remaining in column.
\newcount\@columndepth       % Number of lines in a column.
\newdimen\@columnwidth       % Width of each column.
\newtoks\crtok  \crtok = {\cr}%
\newcount\currentcolumn
%
% The space matches an end-of-line that will probably be there.
% 
\def\makecolumns#1/#2: {\par \begingroup
   % Set \@columndepth to the number of items we will put in a column:
   % ceiling(N/K), i.e.  (N - 1) / K + 1.
   \@columndepth = #1
   \advance\@columndepth by -1
   \divide \@columndepth by #2
   \advance\@columndepth by 1
   \@linestogoincolumn = \@columndepth
   \@linestogo = #1
   %
   % We start in the first column.
   \currentcolumn = 1
   %
   \def\@endcolumnactions{%
      \ifnum \@linestogo<2 
         \the\crtok \egroup \endgroup \par % End \valign and \makecolumns.
      \else
         % We've done one more line out of the total.
         \global\advance\@linestogo by -1
         % 
         % How many left in the column?
         % 
         \ifnum\@linestogoincolumn<2
            % End this column, that was the last line.
            \global\advance\currentcolumn by 1
            \global\@linestogoincolumn = \@columndepth
            \the\crtok
         \else
            % Still got more lines to go.
            &\global\advance\@linestogoincolumn by -1
         \fi
      \fi
   }%
   %
   % Set up to read the table.
   % 
   \makeactive\^^M
   \letreturn \@endcolumnactions
   % 
   % Figure out how wide our columns are going to be; each column has
   % exactly the same template, so we can use the feature described on
   % p.241 of the TeXbook for repeating preambles.
   % 
   \@columnwidth = \hsize
     \advance\@columnwidth by -\parindent
     \divide\@columnwidth by #2
   \penalty\abovecolumnspenalty
   \noindent % It's not a paragraph (usually).
   \valign\bgroup
     &\hbox to \@columnwidth{\strut \hsize = \@columnwidth ##\hfil}\cr
     %
     % The next end-of-line starts everything going.
}%
%
% 
% 
% \numberedfootnote is like plain TeX's \footnote, but automatically
% numbered.  When you want to reset the footnote number, say
% \footnotenumber = 0.
% 
% We also provide for more general formatting than \footnote:
%   \footnotemarkseparation is the space between the reference mark and
%     the footnote text;
%  \interfootnoteskip is the space between footnotes;
%  \everyfootnote is expanded just before we typeset the footnote.
% 
% The dimensions of the footnote rule are controlled by
% \footnoterulewidth and \footnoteruleheight (the depth is always zero);
% the space after the rule is \belowfootnoterulespace.
% 
\newcount\footnotenumber
\newcount\hlfootlabelnumber
\newdimen\footnotemarkseparation \footnotemarkseparation = .5em
\newskip\interfootnoteskip \interfootnoteskip = 0pt
\newtoks\everyfootnote
\newdimen\footnoterulewidth \footnoterulewidth = 2in
\newdimen\footnoteruleheight \footnoteruleheight = 0.4pt
\newdimen\belowfootnoterulespace \belowfootnoterulespace = 2.6pt
%
\let\@plainfootnote = \footnote
\let\@plainvfootnote = \vfootnote
% Hyperlink labels for forward and back references.
\def\hlfootlabel{FOOT\the\hlfootlabelnumber}%
\def\hlfootbacklabel{FOOTB\the\hlfootlabelnumber}%
%
\def\@eplainfootnote#1{\let\@sf\empty % parameter #2 (the text) is read later
  \ifhmode\edef\@sf{\spacefactor\the\spacefactor}\/\fi
  \global\advance\hlfootlabelnumber by 1
  \hlstart@impl{foot}{\hlfootlabel}%
  \hldest@impl{footback}{\hlfootbacklabel}%
  #1%
  \hlend@impl{foot}%
  \@sf\vfootnote{#1}%
}%
%
\let\footnote\@eplainfootnote
%
\def\vfootnote#1{\insert\footins\bgroup
  \interlinepenalty\interfootnotelinepenalty
  \splittopskip\ht\strutbox % top baseline for broken footnotes
  \advance\splittopskip by \interfootnoteskip
  \splitmaxdepth\dp\strutbox
  \floatingpenalty\@MM
  \leftskip\z@skip \rightskip\z@skip \spaceskip\z@skip \xspaceskip\z@skip
  \everypar = {}%
  \parskip = 0pt % because of the vskip
  % Even if typesetting in multicolumns, do footnotes in normal page width.
  % (We don't have any provision in the output routine for having
  % footnotes per column, anyway.)
  \ifnum\@numcolumns > 1 \hsize = \@normalhsize \fi
  \the\everyfootnote
  \vskip\interfootnoteskip
  \indent\llap{%
    \hlstart@impl{footback}{\hlfootbacklabel}%
    \hldest@impl{foot}{\hlfootlabel}%
    #1%
    \hlend@impl{footback}%
    \kern\footnotemarkseparation
  }%
  \footstrut\futurelet\next\fo@t
}%
%
\def\footnoterule{\dimen@ = \footnoteruleheight
  \advance\dimen@ by \belowfootnoterulespace
  \kern-\dimen@
  \hrule width\footnoterulewidth height\footnoteruleheight depth0pt
  \kern\belowfootnoterulespace
  \vskip-\interfootnoteskip
}%
%
\def\numberedfootnote{%
  \global\advance\footnotenumber by 1
  \@eplainfootnote{$^{\number\footnotenumber}$}%
}%
%
%
% 
%  Margins.
% 
% TeX's primitives determine the type area.  But some users prefer to
% think in terms of margins.  These definitions allow one to say, for
% example, `\topmargin = 2in', instead of `\voffset=1in\advance\vsize by
% -1in'.  Constructions like `\advance\topmargin by 1in' give an error
% message, though, since \topmargin is not a parameter.  Instead, the
% macro \advancetopmargin has to be used.
% 
\newdimen\paperheight 
\ifnum\mag=1000
  \paperheight = 11in % No magnification (yet).
\else
  \paperheight = 11truein % We already have a magnification. 
\fi
%
\def\topmargin{\afterassignment\@finishtopmargin \dimen@}%
\def\@finishtopmargin{%
  \dimen2 = \voffset		% Remember the old \voffset.
  \voffset = \dimen@ \advance\voffset by -1truein
  \advance\dimen2 by -\voffset	% Compute the change in \voffset.
  \advance\vsize by \dimen2	% Change type area accordingly.
}%
\def\advancetopmargin{%
  \dimen@ = 0pt \afterassignment\@finishadvancetopmargin \advance\dimen@
}%
\def\@finishadvancetopmargin{%
  \advance\voffset by \dimen@
  \advance\vsize by -\dimen@
}%
%
%
\def\bottommargin{\afterassignment\@finishbottommargin \dimen@}%
\def\@finishbottommargin{%
  \@computebottommargin		% Result in \dimen2.
  \advance\dimen2 by -\dimen@	% Compute the change in the bottom margin.
  \advance\vsize by \dimen2	% Change the type area.
}%
\def\advancebottommargin{%
  \dimen@ = 0pt \afterassignment\@finishadvancebottommargin \advance\dimen@
}%
\def\@finishadvancebottommargin{%
  \advance\vsize by -\dimen@
}%
%
% Find the current bottom margin, putting the result in \dimen2.
% 
\def\@computebottommargin{%
  \dimen2 = \paperheight	% The total paper size.
  \advance\dimen2 by -\vsize	% Less the text size.
  \advance\dimen2 by -\voffset	% Less the offset at the top.
  \advance\dimen2 by -1truein	% Less the default offset.
}%
% 
% 
\newdimen\paperwidth
\ifnum\mag=1000
  \paperwidth = 8.5in % No magnification (yet).
\else
  \paperwidth = 8.5truein % We already have a magnification. 
\fi
%
\def\leftmargin{\afterassignment\@finishleftmargin \dimen@}%
\def\@finishleftmargin{%
  \dimen2 = \hoffset		% Remember the old \hoffset.
  \hoffset = \dimen@ \advance\hoffset by -1truein
  \advance\dimen2 by -\hoffset	% Compute the change in \hoffset.
  \advance\hsize by \dimen2	% Change type area accordingly.
}%
\def\advanceleftmargin{%
  \dimen@ = 0pt \afterassignment\@finishadvanceleftmargin \advance\dimen@
}%
\def\@finishadvanceleftmargin{%
  \advance\hoffset by \dimen@
  \advance\hsize by -\dimen@
}%
%
%
\def\rightmargin{\afterassignment\@finishrightmargin \dimen@}%
\def\@finishrightmargin{%
  \@computerightmargin		% Result in \dimen2.
  \advance\dimen2 by -\dimen@	% Compute the change in the right margin.
  \advance\hsize by \dimen2	% Change the type area.
}%
\def\advancerightmargin{%
  \dimen@ = 0pt \afterassignment\@finishadvancerightmargin \advance\dimen@
}%
\def\@finishadvancerightmargin{%
  \advance\hsize by -\dimen@
}%
%
% Find the current right margin, putting the result in \dimen2.
% 
\def\@computerightmargin{%
  \dimen2 = \paperwidth		% The total paper size.
  \advance\dimen2 by -\hsize	% Less the text size.
  \advance\dimen2 by -\hoffset	% Less the offset at the left.
  \advance\dimen2 by -1truein	% Less the default offset.
}%
% 
% There is a potential problem when using the margin macros at a true
% dimension and then calling \magnification. So we redefine
% \magnification to address this. 
% 
\let\@plainm@g = \m@g
\def\m@g{\@plainm@g \paperwidth = 8.5 true in \paperheight = 11 true in}%
%
% 
% 
%  Double column output.
%
% \doublecolumns begins double column output.  It can be called
% in the midst of a page.  \singlecolumn restores single column 
% output.  (It would be wrong to require \enddoublecolumns, because 
% often one wants double column mode to continue to the end of 
% the document.)
%
% The basic approach is that of Appendix E of the TeXbook, p.417.
% David Guichard made significant improvements to my original implementation.
%
% The glue here (the default is intended to be one linespace) is inserted
% before double columns start, and after they end.
%
\newskip\abovecolumnskip \abovecolumnskip = \bigskipamount
\newskip\belowcolumnskip \belowcolumnskip = \bigskipamount
%
% Space between the columns. It can be changed as desired.
\newdimen\gutter \gutter = 2pc
%
% These registers are needed for dealing with switching back and forth.
\newbox\@partialpage
\newdimen\@normalhsize
\newdimen\@normalvsize  % The original (before multi-columns) \vsize.
\newtoks\previousoutput
%
% Some synonymous ways to refer to multiple column modes.
\def\quadcolumns{\@columns4}%
\def\triplecolumns{\@columns3}%
\def\doublecolumns{\@columns2}%
\def\begincolumns#1{\ifcase#1\relax \or \singlecolumn \or \@columns2 \or
                            \@columns3 \or \@columns4 \else \relax \fi}%
\def\endcolumns{\singlecolumn}%
\let\@ndcolumns = \relax
%
% Set this by default so \vfootnote can unconditionally inspect it.
\chardef\@numcolumns = 1
%
\mathchardef\@ejectpartialpenalty = 10141
%
%
% \@columns:  Start typesetting with #1 columns.
%
% Before we actually start, we have to make sure that there are at least
\chardef\@col@minlines = 3
% free lines.
% (It could be 2, or even 1, but it might give ugly results; at least one
% line is absolutely necessary, or the output routine might get confused.)
% We have to be careful, so that eg.
%	\hbox{TITLE}
%	\nobreak
%	\doublecolumns
% won't break between the title and the start of the columned output.
%
% To achieve this, we add vskip of fixed size equal to
%	@col@minlines * baselineskip
% and then eject the page.
% The output routine then catches the pages ejected:
%  1) if it's a normal page, it is processed by previous output routine;
%  2) if it's the last one, it is saved and the added skip is removed.
%
% When the box is processed according to 1), an underfull vbox can appear,
% but it's not our problem, the manuscript (or its macros) has to be fixed.
%
% Gather this many baselines
\chardef\@col@extralines = 3
% of additional material per each column than the combined height of the
% columns, before attempting to break the columns.  This sometimes
% considerably improves the last column, when it would become too spaced
% out because TeX had to squeeze some lines into the previous columns
% due to unfortunate column breaks.
%
% The larger \@col@extralines, the better, but we don't want to make
% it too large.  If there are less than \@col@extralines * num_columns
% lines left before we start gathering these extra lines, we'll hit
% \@endcolumns while gathering material for this page.  If the extra
% lines will not actually fit in the columns, we'll have to move them
% to the next page.  But we won't move them if there are any
% insertions on this page (see \@balancecolumns), so in such case
% we'll stuff everything on the current page, producing an overfull
% box, even though these extra lines could successfully be moved to
% the next page.
%
\newdimen\@col@extraheight
%
% Another note: all assignments are global; it is not possible to call
% \doublecolumns in a group.
%
\def\@columns#1{%
  \@ndcolumns
  %
  \global\let\@ndcolumns = \@endcolumns
  \global\chardef\@numcolumns = #1
  \global\previousoutput = \expandafter{\the\output}%
  %
  % Grab the page so far (i.e., the material BEFORE \@columns was called)
  % and save it in \@partialpage.
  \global\output = {%
    \ifnum\outputpenalty = -\@ejectpartialpenalty
      \dimen@ = \vsize
      \advance\dimen@ by \@col@minlines\baselineskip
      \global\setbox\@partialpage =
        \vbox  \ifdim \pagetotal > \vsize  to \dimen@  \fi  {%
	  \unvbox255 \unskip
	}%
    \else
      \the\previousoutput
    \fi
  }%
  %
  \vskip \abovecolumnskip
  \vskip \@col@minlines\baselineskip
  % now execute the output routine:
  \penalty -\@ejectpartialpenalty
  %
  % Reset \output to prepare for the first real page break.
  \global\output = {\@columnoutput}%
  %
  \global\@normalhsize = \hsize
  \global\@normalvsize = \vsize
  %
  % Figure out how wide the columns should be -- for n columns,
  % decrement by n - 1 gutters.
  \count@ = \@numcolumns
  \advance\count@ by -1
  \global\advance\hsize by -\count@\gutter
  \global\divide\hsize by \@numcolumns
  %
  % Compute \vsize based on what's already on the page
  % and the number of columns. Also change the mag factor for insertions.
  %
  \advance\vsize by -\ht\@partialpage
  %
  \advance\vsize by -\ht\footins
  \ifvoid\footins\else \advance\vsize by -\skip\footins \fi
  \multiply\count\footins by \@numcolumns
  %
  \advance\vsize by -\ht\topins
  \ifvoid\topins\else \advance\vsize by -\skip\topins \fi
  \multiply\count\topins by \@numcolumns
  %
  \global\vsize = \@numcolumns\vsize
  %
  % Gather some extra material, to improve the last column.
  \@col@extraheight=\@col@extralines\baselineskip
  \multiply\@col@extraheight by \@numcolumns
  \global\advance\vsize by \@col@extraheight
}%
%
% When this is invoked box 255 contains just the right amount of
% material, whether triggered by an output routine or a change in the
% number of columns. Because columns have to contain an integral number
% of lines of type, we take a bit of care with balancing the heights of
% the columns to prevent either losing material or having a very short
% last column.
%
% When a page ends due to \bye or \eject, box 255 will contain lots of
% white space, so the columns will not look balanced. To fix this use
% \singlecolumn before ending the page.
%
% [gutterbox material added by AHL on 5 November 1997.]
% \gutterbox is a hook to enable the placement of arbitrary vertical
% material in the gutter between columns. For example, to put a
% vertical line between the columns you could say 
%
%   \def\gutterbox{\vbox to \dimen0{\vfil\hbox{\vrule height\dimen0}\vfil}}%
%
% The dimension counter \dimen0 contains the height of the column. 
%
% By default, we define \gutterbox to be "empty": 
\def\gutterbox{\vbox to \dimen0{\vfil\hbox{\hfil}\vfil}}%
%
\def\@columnsplit{%
  \splittopskip = \topskip
  \splitmaxdepth = \baselineskip
  %
  % \dimen@ should be the height that columns on this page should
  % have, i.e., the height of the page (\singlecolumvsize) minus
  % single-column material, which includes insertions.  (If you want
  % your insertions to respect the columns, you will have to change
  % the output routine.)  If you add more insertions, they should be
  % taken into account in \@columns and \@endcolumns.
  %
  % Unfortunately, we lose on flexible glue because we must
  % \vsplit to a <dimen>.
  %
  % Split the long scroll into columns.
  \begingroup
    % We do not want to see underfull \vbox messages unless the final
    % page is underfull.
    \vbadness = 10000
    %
    % The first (leftmost) column.
    %d\message{^^J starting columnsplit, splitting to \the\dimen@}%
    %d\showbox255
    \global\setbox1 = \vsplit255 to \dimen@  \global\wd1 = \hsize
    %d\message{^^J split off to this box1:}%
    %d\showbox1
    %
    % The second column.
    \global\setbox3 = \vsplit255 to \dimen@  \global\wd3 = \hsize
    %
    \ifnum\@numcolumns > 2
      % The third column, if requested.
      \global\setbox5 = \vsplit255 to \dimen@ \global\wd5 = \hsize
    \fi
    \ifnum\@numcolumns > 3
      % The fourth column, likewise if requested.
      \global\setbox7 = \vsplit255 to \dimen@ \global\wd7 = \hsize
    \fi
  \endgroup
  %
  % Set up \box255 with the real output page, as the previous output
  % routine expects.
  \setbox0 = \box255
  \global\setbox255 = \vbox{%
    \unvbox\@partialpage
    \ifcase\@numcolumns \relax\or\relax
      \or \hbox to \@normalhsize{\box1\hfil\gutterbox\hfil\box3}%
      \or \hbox to \@normalhsize{\box1\hfil\gutterbox\hfil\box3%
                                      \hfil\gutterbox\hfil\box5}%
      \or \hbox to \@normalhsize{\box1\hfil\gutterbox\hfil\box3%
                                      \hfil\gutterbox\hfil\box5%
                                      \hfil\gutterbox\hfil\box7}%
    \fi
  }%
  %
  % Save what's left over in a private register before calling their
  % output routine.
  \setbox\@partialpage = \box0
}%
%
% Our output routine splits the columns and then calls the previous one.
% 
\def\@columnoutput{%
  %d\message{^^J starting columnoutput, ht255: \the\ht255}%
  \dimen@ = \ht255
    \advance\dimen@ by -\@col@extraheight
    %d\message{^^J minus col@extraheight: \the\@col@extraheight}%
    \divide\dimen@ by \@numcolumns
  \@columnsplit
  \@recoverclubpenalty 
  \hsize = \@normalhsize % Local to \output's group.
  \vsize = \@normalvsize
  \the\previousoutput
  %
  % Put back what didn't fit.
  \unvbox\@partialpage
  \penalty\outputpenalty
  %
  % The correct vsize is the original vsize times the
  % number of columns, plus the ``extra height''.
  \global\vsize = \@numcolumns\@normalvsize
  \global\advance\vsize by \@col@extraheight
}%
%
% Go back to single-column typesetting.  Assume \doublecolumns has
% been called.
% 
\def\singlecolumn{%
  \@ndcolumns
  \chardef\@numcolumns = 1
  \vskip\belowcolumnskip
  \nointerlineskip
}%
%
\newbox\@singlecolumnbox 
\newdimen\column@pagegoal
\newdimen\column@vsize
%
\def\@endcolumns{%
  \global\let\@ndcolumns = \relax
  \par % Shouldn't start in horizontal mode.
  % Save the combined height of the columns and the page goal.  We
  % have to be careful -- the last line of the multi-column material
  % might have taken \pagetotal just over \pagegoal, in which case we
  % have to use \pagegoal for the height, otherwise the box produced
  % when splitting the columns will not fit on the page.
  \column@pagegoal = \pagegoal
  \advance\column@pagegoal by-\@col@extraheight
  \ifdim \pagetotal > \column@pagegoal
    \column@vsize = \column@pagegoal
  \else
    \column@vsize = \pagetotal
  \fi
  % Grab whatever is left of the multi-column material.
  \global\output = {\global\setbox1 = \box255}%
  \pagegoal = \pagetotal
  \break                     % Exercise the page builder, i.e., \output.
  \setbox2 = \box1           % Save material in box2 in case of overflow.
  % We won't need \columnoutput anymore.
  \global\output = \expandafter{\the\previousoutput}%
  % Save single column material (in case the multi-column material
  % ends on the same page where it starts; otherwise \@partialpage
  % will be void).  This also voids \@partialpage.
  \setbox\@singlecolumnbox = \box\@partialpage
  % Try to fit what's left into the columns.
  \@balancecolumns
}%
%
% There are many caveats when balancing columns at the end of
% multi-column material.  The core of all problems is the following.
% Eplain collects multi-column material in one long scroll until the
% scroll's length is at least \vsize * \@numcolumns.  But when we try
% to split that scroll into columns, there is no guarantee that all
% the collected material will fit (for example, because a column break
% occured inside a large unbreakable block, so we had to carry it over
% to the next column and stretch out the previous column).  So every
% time we call \@columnsplit, we should expect it to leave something
% for us in \@partialpage.
%
% Now, what should we do with these left-overs?  We increase column
% height a bit and try to split the scroll again, and so on until
% everything fits.
%
% However, while doing this, we might increase column height so much
% that the columns no longer fit on a page.  What we do then is output
% the highest fitting columns, break the page and then restart the
% whole process on what's left.
%
% However, there's another catch -- if a page contains insertions,
% there's a slight chance that the footnotes or top insertions were
% inserted by the multi-column material we are going to carry over to
% the next page, so those footnotes or top insertions will appear one
% page ahead of their references.  The worst thing is that the user
% will get no warning of this.  Therefore, when we have left-overs on
% a page with insertions, we just stuff them into columns to produce
% an overfull box warning.
%
\def\@balancecolumns{%
  % Split the scroll to the new column height.
  \global\setbox255 = \copy2  % Retrieve what the fake \output set.
  \dimen@ = \column@vsize
    \divide\dimen@ by \@numcolumns
  \@columnsplit
  %
  \ifvoid\@partialpage
    % Everything fits -- we're done.
    \global\vsize = \@normalvsize
    \global\hsize = \@normalhsize
    \dump@balanced@columns
    \let\next\relax
  \else
    % There is some left-over.  Increase column height.
    \advance \column@vsize by \@numcolumns pt
    % Check what we should do with the left-over.
    \test@spill@columns
    \ifspill@columns
      % We are to break the page here.  Make up a page from the
      % single-column material followed by the columns that we've
      % already split off into \box255.
      \begingroup
        \vsize = \@normalvsize
        \hsize = \@normalhsize
        \dump@balanced@columns
        \break
        \@recoverclubpenalty
      \endgroup
      % Now put back what didn't fit and process it recursively.
      \unvbox\@partialpage
      \let\next\@endcolumns
    \else
      % We should continue incrementing column height.
      \setbox0=\box\@partialpage % Merely to void \@partialpage.
      \let\next\@balancecolumns
    \fi
  \fi
  \next
}%
%
\def\dump@balanced@columns{%
  \ifvoid\topins\else\topinsert\unvbox\topins\endinsert\fi
  \unvbox\@singlecolumnbox
  % Avoid interline glue here -- we didn't (couldn't) account for it
  % when assessing \vsize for the columns in \@columns, which means
  % the columns may not fit if we also add the interline glue.
  \nointerlineskip
  \box255
}%
%
% If we hit page bottom while balancing columns and there are no
% insertions on the page, we can let the left-over spill to the next
% page.  If there are insertions on the page, we shouldn't let
% \@partialpage to the next page, to avoid separating a possible
% reference from an insertion.  The following flag and test are to
% check these conditions.
\newif\ifspill@columns
\def\test@spill@columns{%
  \spill@columnsfalse
  \ifdim \column@vsize > \column@pagegoal
    \ifvoid\footins
      \ifvoid\topins
        \spill@columnstrue
      \fi
    \fi
  \fi
}%
%
% [\columnfill]
% We don't have any way to force a column eject, since the \output
% routine is only prepared to split up a full page of material. Instead,
% we provide the following as a guess at enough space to fill up the
% current column.
%
% [April 30, 1998] This definition is from
% Helmut Jarausch <jarausch@IGPM.Rwth-Aachen.DE> with some
% modifications by A. Lewenberg <ahl@uakron.edu>. 
%
% Here is the main difficulty: when \vsplitting the long page into the
% component columns, \vsplit prepends to the top of each column glue
% in the amount \topskip - (height of top box). But this happens
% in the output routine _after_ the \columnfill does its
% calculations. The result of this is that if we are not careful
% \columnfill will insert too much space attempting to ``eject'' the
% current column. There is no simple way around this, so what I have
% done is have is make \columnfill insert less space than needed, and
% then not allow any \club lines by setting \clubpenalty to its
% maximum value. Not a pretty solution, but until something better
% comes along, it will have to do. 
%
\def\@saveclubpenalty{% save the current value of \clubpenalty
  \edef\@recoverclubpenalty{%
     \global\clubpenalty=\the\clubpenalty\relax%
     \global\let\noexpand\@recoverclubpenalty\relax
  }% the \noexpand handles infinite self-reference
}%
\let\@recoverclubpenalty\relax
\newdimen\temp@dimen
\def\columnfill{%
  \par
  \dimen@ = \pagetotal  % The height of the text so far. 
  %d\message{^^J pagetotal start (dimen@): \the\dimen@}%
  %
  \temp@dimen = \vsize  % = \@numcolumns * columnheight
  %d\message{^^J temp@dimen (total height): \the\temp@dimen}%
  %
  \advance\temp@dimen by -\@col@extraheight
  %d\message{^^J reducing temp@dimen \the\temp@dimen (-@col@extraheight)}%
  %
  \divide\temp@dimen by \@numcolumns  % Compute column height
  %d\message{^^J col height (tempdimen@): \the\temp@dimen}%
  %
  \loop
    %d\message{^^J looping}%
    \ifdim \dimen@ > \temp@dimen  % more material than a column?
      \advance \dimen@ by -\temp@dimen
      \advance \dimen@ by \topskip % fudge factor
      %d\message{^^J reducing dimen@ to \the\dimen@ (tempdimen@ + \the\topskip)}%
  \repeat
  \advance \temp@dimen by -\dimen@
  %d\message{^^J reducing temp@dimen to \the\temp@dimen (-dimen@)}%
  %
  \advance \temp@dimen by -\prevdepth
  %d\message{ reducing temp@dimen to \the\temp@dimen (-prevdepth)}%
  %
  \@saveclubpenalty 
  \clubpenalty=10000\relax
  \hrule height\temp@dimen width0pt depth0pt\relax  % fill space with rule
  \nointerlineskip
  \par
  \nointerlineskip
  \allowbreak \vfil % To allow a page break here.
  \relax
}%
%
%
%  Hypertext links support.
%
% Hyperlink destinations (driver-independent code).
%
% \hldest{TYPE}{OPTIONS}{LABEL} defines a hyperlink destination
% LABEL.  OPTIONS is a comma-separated list of option assignments of
% the form `opt=value'; permitted values for TYPE and OPTIONS depend
% on selected hyperlink driver.
%
% \hldest will be \let to \@hldest by \enablehyperlinks.  TYPE,
% OPTIONS and LABEL will be read by \hl@getparam.
\def\@hldest{%
  \def\hl@prefix{hldest}%
  \let\after@hl@getparam\hldest@aftergetparam
  % Start the group which will isolate option settings.  It will be
  % ended in \hldest@aftergetparam
  \begingroup
    \hl@getparam
}%
% This actually produces hyperlink destination.  It will be called at
% the end of \hl@getparam, after the parameters are parsed.
\def\hldest@aftergetparam{%
  \ifvmode
    % In vertical mode we don't raise the destination, so it can go
    % directly into the vertical list.
    \hldest@driver
  \else
    % In horizontal mode, the destination is raised \hldest@opt@raise
    % above the baseline and placed inside a zero-width/height/depth
    % box; the box is surrounded by \allowhyphens in case it is
    % placed next to a word, to allow hyphenation of that word.
    \allowhyphens
    \smash{\ifx\hldest@opt@raise\empty \else \raise\hldest@opt@raise\fi
             \hbox{\hldest@driver}}%
    \allowhyphens
  \fi
  % End the group which was isolating option settings (it was started
  % in \@hldest).
  \endgroup
}%
%
% Hyperlinks (driver-independent code).
%
% \hlstart{TYPE}{OPTIONS}{LABEL} starts a hyperlink to destination
% LABEL.  OPTIONS is a comma-separated list of option assignments of
% the form `opt=value'; permitted values of TYPE and OPTIONS depend on
% selected hyperlink driver.  End the link with \hlend.
%
% \hlstart will be \let to \@hlstart by \enablehyperlinks.  TYPE,
% OPTIONS and LABEL will be read by \hl@getparam.
\def\@hlstart{%
  \leavevmode
  \def\hl@prefix{hl}%
  \let\after@hl@getparam\hlstart@aftergetparam
  % Start the group which will isolate option settings and color
  % changes.  It will be ended in \@hlend
  \begingroup
    \hl@getparam
}%
%
\def\hlstart@aftergetparam{%
  % Set the color for the link.
  \ifx\color\undefined \else
    \ifx\hl@opt@color\empty \else
      \ifx\hl@opt@colormodel\empty
        \edef\temp{\noexpand\color{\hl@opt@color}}%
      \else
        \edef\temp{\noexpand\color[\hl@opt@colormodel]{\hl@opt@color}}%
      \fi
      \temp
    \fi
  \fi
  % Call the driver.
  \hl@driver
}%
% \hlend will be \let to \@hlend by \enablehyperlinks.  \@hlend will
% be defined by a driver.
%
% Macros which are used commonly by \hldest and \hlstart to parse and
% save parameters.  \hl@prefix must be set to `hldest' by \hldest and
% to `hl' by \hlstart.
%
% \hl@getparam{TYPE}{OPTIONS}{LABEL} reads, parses and saves the
% parameters for \hldest or \hlstart into \@hltype, \hl[dest]@opt@...
% and \@hllabel.  After doing that it calls \after@hl@getparam which
% should be defined by \@hldest and \@hlstart to produce destination /
% link using the saved parameters.
\def\hl@getparam#1#2{% We'll read #3 (LABEL) later.
  % Save TYPE in \@hltype (use default if empty).
  \edef\@hltype{#1}%
  \ifx\@hltype\empty
    \expandafter\let\expandafter\@hltype
      \csname \hl@prefix @type\endcsname
  \fi
  % For each supported destination / link type TYPE, a driver should
  % define \hl[dest]@typeh@TYPE handler.
  \expandafter\ifx\csname \hl@prefix @typeh@\@hltype\endcsname \relax
    \errmessage{Invalid hyperlink type `\@hltype'}%
  \fi
  % \for will expand #2 once so user can pass a shortcut macro.  We
  % also ignore empty \hl@arg, so that the following
  %     \hldest{TYPE}{\myopt,height=200pt}{LABEL}
  % would be legal even when \myopt happens to be empty.
  \For\hl@arg:=#2\do{%
    \ifx\hl@arg\empty \else
      \expandafter\hl@set@opt\hl@arg=,%
    \fi
  }%
  % Now read the third argument, LABEL.  Do so inside a group with
  % \uncatcodespecials, to allow `#' and `~' in it (LABEL can be a URL
  % for some link types).
  \bgroup
    \uncatcodespecials
    \catcode`\{=1 \catcode`\}=2
    \@hl@getparam
}%
%
\def\@hl@getparam#1{%
  \egroup
  % Save LABEL in \@hllabel.
  \edef\@hllabel{#1}%
  % Execute the commands specific to destination / link
  \after@hl@getparam
  % Ignore spaces after \hlstart and \hldest.
  \ignorespaces
}%
% Parse and set a (default, not a group) option.
\def\hl@set@opt#1=#2,{%
  % For each supported option OPTION, a driver should define
  % \hl[dest]@opt@OPTION.
  \expandafter\ifx\csname \hl@prefix @opt@#1\endcsname \relax
    \errmessage{Invalid hyperlink option `#1'}%
  \fi
  % Save the value of the option.
  \if ,#2, % if #2 is empty, complain.
    \errmessage{Missing value for option `#1'}%
  \else
    % Remove a trailing =.
    \def\temp##1={##1}%
    \expandafter\edef\csname \hl@prefix @opt@#1\endcsname{\temp#2}%
  \fi
}%
% \hl{dest,start,end}@impl{GROUP}{LABEL} will generate `implicit' destination /
% hyperlink, if the user has not turned off this kind of implicit
% destinations / hyperlinks.  This is used by Eplain's cross-reference
% macros.
\def\hldest@impl#1{%
  \expandafter\ifcase\csname hldest@on@#1\endcsname
    \relax\expandafter\gobble
  \else
    \toks@=\expandafter{\csname hldest@type@#1\endcsname}%
    \toks@ii=\expandafter{\csname hldest@opts@#1\endcsname}%
    \edef\temp{\noexpand\hldest{\the\toks@}{\the\toks@ii}}%
    \expandafter\temp
  \fi
}%
\def\hlstart@impl#1{%
  \expandafter\ifcase\csname hl@on@#1\endcsname
    % Still produce \leavevmode, to be consistent with \hloff.
    \leavevmode\expandafter\gobble
  \else
    \toks@=\expandafter{\csname hl@type@#1\endcsname}%
    \toks@ii=\expandafter{\csname hl@opts@#1\endcsname}%
    \edef\temp{\noexpand\hlstart{\the\toks@}{\the\toks@ii}}%
    \expandafter\temp
  \fi
}%
\def\hlend@impl#1{%
  \expandafter\ifcase\csname hl@on@#1\endcsname
  \else
    \hlend
  \fi
}%
%
% Setting options and types.
%
\def\hl@asterisk@word{*}%
\def\hl@opts@word{opts}%
\newif\if@params@override
% We define hyperlink / destination groups.  A group is a macro or a
% group of macros which implicitly generate hyperlink / destination.
% The user can set parameters for each group individually, as well as
% the default parameters, with the macros defined below.  Group
% settings will override the default hyperlink / destination
% parameters.
\def\hldest@groups{definexref,xrdef,li,eq,bib,foot,footback,idx}%
\def\hl@groups{ref,xref,eq,cite,foot,footback,idx,url,hrefint,hrefext}%
% \hldesttype [GROUPS]{VALUE}
% \hldestopts [GROUPS]{VALUE}
% \hltype     [GROUPS]{VALUE}
% \hlopts     [GROUPS]{VALUE}
%
% Set hyperlink or destination parameter (type / opts) to VALUE for
% each group in GROUPS.  An empty `group' will set default value for
% the parameter.  A star (*) `group' stands for all groups (except the
% empty `group').  If the macro is followed by an exclamation mark
% (like \hlopts!...), the parameters will be overridden; otherwise,
% they will be updated (this has effect only on group option list).
\def\hldesttype{%
  \def\hl@prefix{hldest}%
  \def\hl@param{type}%
  \let\hl@all@groups\hldest@groups
  \futurelet\hl@excl\hl@param@read@excl
}%
\def\hldestopts{%
  \def\hl@prefix{hldest}%
  \def\hl@param{opts}%
  \let\hl@all@groups\hldest@groups
  \futurelet\hl@excl\hl@param@read@excl
}%
\def\hltype{%
  \def\hl@prefix{hl}%
  \def\hl@param{type}%
  \let\hl@all@groups\hl@groups
  \futurelet\hl@excl\hl@param@read@excl
}%
\def\hlopts{%
  \def\hl@prefix{hl}%
  \def\hl@param{opts}%
  \let\hl@all@groups\hl@groups
  \futurelet\hl@excl\hl@param@read@excl
}%
\def\hl@param@read@excl{%
  \ifx!\hl@excl
    \let\next\hl@param@read@opt@arg
    \@params@overridetrue
  \else
    \def\next{\hl@param@read@opt@arg{!}}%
    \@params@overridefalse
  \fi
  \next
}%
\def\hl@param@read@opt@arg#1{% #1 is the `!', ignore it.
  \@getoptionalarg\hl@setparam
}%
% Set the parameter \hl@param to #1 for each group in \@optionalarg.
% This will become \hl@setparam in \enablehyperlinks.
\def\@hl@setparam#1{%
  \ifx\@optionalarg\empty
    \hl@setparam@default{#1}% Set default.
  \else
    % If we find an asterisk in the list, we have no choice but to
    % finish the list and then call \hl@setparam again, now with
    % \hl@all@groups for the list of groups.
    \let\hl@do@all@groups\gobble
%
    \For\hl@group:=\@optionalarg\do{%
      \ifx\hl@group\hl@asterisk@word
        \def\hl@do@all@groups{\let\@optionalarg\hl@all@groups \hl@setparam}%
      \else
        \hl@setparam@group{#1}%
      \fi
    }%
%
    \hl@do@all@groups{#1}%
  \fi
}%
% Set a parameter (\hl@param) for one group (\hl@group) to the value
% (#1).  The group may be empty, in which case we call
% \hl@setparam@default
\def\hl@setparam@group#1{%
  \ifx\hl@group\empty
    \hl@setparam@default{#1}%
  \else
    \expandafter\ifx\csname\hl@prefix @\hl@param @\hl@group\endcsname\relax
      \errmessage{Hyperlink group `\hl@prefix:\hl@param:\hl@group' is not defined}%
    \fi
    \ifx\hl@param\hl@opts@word
      % For the `opts' parameter, we want to expand the first token of
      % #1 once, in case the user passed a macro containing the option
      % list.  Even if we simply need to override the old option list,
      % we still call \hl@update@opts@with@list to go through the
      % options and trim possible leading space token in option keys.
      \if@params@override
        \expandafter\let\csname\hl@prefix @\hl@param @\hl@group\endcsname\empty
      \fi
      \hl@update@opts@with@list{#1}% #1 will be used in the \for
                                % loop, so it will be expanded once.
    \else
      % Do not use \edef here to define the parameter, so the user can
      % define it to, e.g., \normalbaselineskip, and make the parameter
      % adjustable to a situation.
      \ece\def{\hl@prefix @\hl@param @\hl@group}{#1}%
    \fi
  \fi
}%
% Set default parameter values.  We have to treat `opts' (list of
% options) specially, because for option defaults we don't store a
% list of options (like we do for the group options) but set each
% option individually.
\def\hl@setparam@default#1{%
  \ifx\hl@param\hl@opts@word
    % `opts'.
    \For\hl@opt:=#1\do{%
      \ifx\hl@opt\empty \else
        \expandafter\hl@set@opt\hl@opt=,%
      \fi
    }%
  \else
    % Everything except `opts'.
    \expandafter\ifx\csname\hl@prefix @\hl@param\endcsname\relax
      \message{Default hyperlink parameter `\hl@prefix:\hl@param' is not defined}%
    \fi
    % Should not use \edef, so the user could define this to, e.g.,
    % \normalbaselineskip, to make the parameter adjustable to a
    % situation.
    \ece\def{\hl@prefix @\hl@param}{#1}%
  \fi
}%
% For each option in the list (#1), call \hl@update@opts@with@opt to update
% the group's option list (\csname\hl@prefix @opts@ \hl@group\endcsname)
% with this new option.
\def\hl@update@opts@with@list#1{%
  % Start with the current list of the group.
  \global\expandafter\let\expandafter\hl@update@new@list
    \csname \hl@prefix @opts@\hl@group\endcsname
  % We have to isolate the \for loop inside a (TeX) group, to avoid
  % clashes with the loop in \hl@setparam
  \begingroup
    \For\hl@opt:=#1\do{%
      \hl@update@opts@with@opt
    }%
  \endgroup
  % Save the final list back in the option list for the group.
  \ece\let{\hl@prefix @opts@\hl@group}\hl@update@new@list
}%
% Go through the option list (\hl@update@new@list) and construct the
% new list (in \hl@update@new@list), replacing the old definition of
% the option with the new one (\hl@opt).
\def\hl@update@opts@with@opt{%
  % Save the old list and the new option.
  \global\let\hl@update@old@list\hl@update@new@list
  \global\let\hl@update@new@list\empty
  \global\let\hl@update@new@opt\hl@opt
  % Get the key of the new option and save it.
  \expandafter\hl@parse@opt@key\hl@opt=,%
  \let\hl@update@new@key\hl@update@key
  % We will set this to real comma after the first entry.
  \global\let\hl@update@comma\empty
  % We have to isolate the \for loop inside a (TeX) group, to avoid
  % clashes with the loop in \hl@update@opts@with@list
  \begingroup
    \for\hl@opt:=\hl@update@old@list\do{%
      \ifx\hl@opt\empty \else % Skip empty `options'.
        % Get the key of this option.
        \expandafter\hl@parse@opt@key\hl@opt=,%
        % If the key matches, replace the option definition with the
        % new definition, otherwise, repeat the old definition.
        \toks@=\expandafter{\hl@update@new@list}%
        \ifx\hl@update@key\hl@update@new@key
          \ifx\hl@update@new@opt\empty \else % Skip multiple options.
            \toks@ii=\expandafter{\hl@update@new@opt}%
            \xdef\hl@update@new@list{\the\toks@\hl@update@comma\the\toks@ii}%
            \global\let\hl@update@new@opt\empty
            \global\def\hl@update@comma{,}%
          \fi
        \else
          \toks@ii=\expandafter{\hl@opt}%
          \xdef\hl@update@new@list{\the\toks@\hl@update@comma\the\toks@ii}%
          \global\def\hl@update@comma{,}%
        \fi
      \fi
    }%
  \endgroup
  % If nothing was replaced, add the new option to the end of the new list.
  \ifx\hl@update@new@opt\empty \else
    \toks@=\expandafter{\hl@update@new@list}%
    \toks@ii=\expandafter{\hl@update@new@opt}%
    \xdef\hl@update@new@list{\the\toks@\hl@update@comma\the\toks@ii}%
  \fi
}%
% Parse the key of the option and save it in \hl@update@key
\def\hl@parse@opt@key#1=#2,{\def\hl@update@key{#1}}%
%
% Default and group parameters (options and types).
%
% Option `raise' will determine how much to raise hyperlink
% destinations above the baseline.  It will be supported by all
% drivers, since it is handled outside the drivers, in
% \hldest@aftergetparam.
\def\hldest@opt@raise{\normalbaselineskip}%
% Options `colormodel' and `color' will also be handled outside the
% drivers, in \hlstart@aftergetparam.
\def\hl@opt@colormodel{cmyk}%
\def\hl@opt@color{0.28,1,1,0.35}%
%
% Parameters for destinations and links produced implicitly by
% cross-reference macros.  Note that each driver will additionally
% define \hldest@type and \hl@type parameters which will be used when
% one of the below is empty, and default values for destination and
% link options (which are driver-specific).
%
% Destination on/off flags (0=off, 1=on).  Changing them here has no
% effect, modify \enablehyperlinks to set defaults.
\def\hldest@on@definexref{0}%
\def\hldest@on@xrdef{0}%
\def\hldest@on@li{0}%
\def\hldest@on@eq{0}% \eqdef and friends
\def\hldest@on@bib{0}% \biblabelprint (BibTeX)
\def\hldest@on@foot{0}% \footnote / \numberedfootnote
\def\hldest@on@footback{0}% back-ref for \footnote / \numberedfootnote
\def\hldest@on@idx{0}% both `page' dests and `exact' dests
% Types of destinations.
\let\hldest@type@definexref\empty
\let\hldest@type@xrdef\empty
\let\hldest@type@li\empty
\let\hldest@type@eq\empty % \eqdef and friends
\let\hldest@type@bib\empty % \biblabelprint (BibTeX)
\let\hldest@type@foot\empty % \footnote / \numberedfootnote
\let\hldest@type@footback\empty % back-ref for \footnote / \numberedfootnote
\let\hldest@type@idx\empty % both `page' dests and `exact' dests
% Options for destinations.
\let\hldest@opts@definexref\empty
\let\hldest@opts@xrdef\empty
\let\hldest@opts@li\empty
\def\hldest@opts@eq{raise=1.7\normalbaselineskip}% \eqdef and friends
\let\hldest@opts@bib\empty % \biblabelprint (BibTeX)
\let\hldest@opts@foot\empty % \footnote / \numberedfootnote
\let\hldest@opts@footback\empty % back-ref for \footnote / \numberedfootnote
\let\hldest@opts@idx\empty % both `page' dests and `exact' dests
%
% Hyperlink on/off flags (0=off, 1=on).  Changing them here has no
% effect, modify \enablehyperlinks to set defaults.
\def\hl@on@ref{0}% \refn and \xrefn, \ref, \refs
\def\hl@on@xref{0}%
\def\hl@on@eq{0}% \eqref and \eqrefn
\def\hl@on@cite{0}% \cite (BibTeX)
\def\hl@on@foot{0}% \footnote / \numberedfootnote
\def\hl@on@footback{0}% back-reference for \footnote / \numberedfootnote
\def\hl@on@idx{0}%
\def\hl@on@url{0}% \url from url.sty
\def\hl@on@hrefint{0}% \href with internal #labels
\def\hl@on@hrefext{0}% \href with external labels (URLs)
% Types of links.
\let\hl@type@ref\empty % \refn and \xrefn, \ref, \refs
\let\hl@type@xref\empty
\let\hl@type@eq\empty % \eqref and \eqrefn
\let\hl@type@cite\empty % \cite (BibTeX)
\let\hl@type@foot\empty % \footnote / \numberedfootnote
\let\hl@type@footback\empty % back-reference for \footnote / \numberedfootnote
\let\hl@type@idx\empty
\let\hl@type@url\empty % \url from url.sty (this will be set to `url' by
                       % drivers which support the `url' type)
\let\hl@type@hrefint\empty % \href with internal #labels
\let\hl@type@hrefext\empty % \href with external labels (URLs) (this
      % will be set to `url' by drivers which support the `url' type)
% Options for links.
\let\hl@opts@ref\empty % \refn and \xrefn, \ref, \refs
\let\hl@opts@xref\empty
\let\hl@opts@eq\empty % \eqref and \eqrefn
\let\hl@opts@cite\empty % \cite (BibTeX)
\let\hl@opts@foot\empty % \footnote / \numberedfootnote
\let\hl@opts@footback\empty % back-reference for \footnote / \numberedfootnote
\let\hl@opts@idx\empty
\let\hl@opts@url\empty % \url from url.sty
\let\hl@opts@hrefint\empty % \href with internal #labels
\let\hl@opts@hrefext\empty % \href with external labels (URLs)
%
% \@hlon[GROUPS]
% \@hloff[GROUPS]
% \@hldeston[GROUPS]
% \@hldestoff[GROUPS]
% \@@hlon
% \@@hloff
% \@@hldeston
% \@@hldestoff
%
% Macros to switch hyperlinks / destinations on/off.
%
% The optional arg is the list of groups.  It can contain a star (*)
% which will make the macros affect all groups (but not the low-level
% macros \hlstart, \hlend and \hldest).
%
% \@hlon, \@hldeston, \@hloff and \@hldestoff will turn low-level
% macros on/off only when they are used either without the optional
% arg or with an empty `group' in the optional arg, otherwise only the
% specified groups are affected.
%
% The single-`@' variants (\@hl...) are for the user.  In your macros,
% if you want to (temporarily) turn low-level macros on/off, it's
% better to use the double-`@' variants (\@@hl...), because they are
% much faster and won't clobber \@optionalarg or anything else.
%
\def\@hlon{\@hlonoff@value@stub{hl}\@@hlon1 }%
\def\@hloff{\@hlonoff@value@stub{hl}\@@hloff0 }%
\def\@hldeston{\@hlonoff@value@stub{hldest}\@@hldeston1 }%
\def\@hldestoff{\@hlonoff@value@stub{hldest}\@@hldestoff0 }%
%
\def\@hlonoff@value@stub#1#2#3{%
  \def\hl@prefix{#1}%
  \let\hl@on@empty#2%
  \def\hl@value{#3}%
  \expandafter\let\expandafter\hl@all@groups
    \csname \hl@prefix @groups\endcsname
  \@getoptionalarg\@finhlswitch
}%
%
\def\@finhlswitch{%
  \ifx\@optionalarg\empty
    \hl@on@empty
  \fi
  % If we find an asterisk in the list, we have no choice but to
  % finish the list and then call \@finhlswitch again, now with
  % \hl@all@groups for the list of groups.
  \let\hl@do@all@groups\relax
%
  \For\hl@group:=\@optionalarg\do{%
    \ifx\hl@group\hl@asterisk@word
      \let\@optionalarg\hl@all@groups
      \let\hl@do@all@groups\@finhlswitch
    \else
      \ifx\hl@group\empty
        \hl@on@empty
      \else
        \expandafter\ifx\csname\hl@prefix @on@\hl@group\endcsname \relax
          \errmessage{Hyperlink group `\hl@prefix:on:\hl@group'
                      is not defined}%
        \fi
        \ece\edef{\hl@prefix @on@\hl@group}{\hl@value}%
      \fi
    \fi
  }%
%
  \hl@do@all@groups
}%
% Turn low-level macros on/off.
\def\@@hlon{%
  \let\hlstart\@hlstart
  \let\hlend\@hlend
}%
\def\@@hloff{%
  \def\hlstart##1##2##3{\leavevmode\ignorespaces}%
  \let\hlend\relax
}%
\def\@@hldeston{%
  \let\hldest\@hldest
}%
\def\@@hldestoff{%
  \def\hldest##1##2##3{\ignorespaces}%
}%
%
% Hyperlink drivers.
%
% \enablehyperlinks[OPTIONS] will enable hyperlinks.  OPTIONS is a
% list of comma-separated options.  An option is one of the following:
%
%     idxexact      Point index links to exact locations of the term
%     idxpage       Point index links to pages with the term (default)
%     idxnone       No links for index entries
%     <driver-name> Force the hyperlink driver
%
% If <driver-name> is omitted, appropriate driver will be detected, if
% possible; if not, we fall back on `hypertex'.
\def\hl@idxexact@word{idxexact}%
\def\hl@idxpage@word{idxpage}%
\def\hl@idxnone@word{idxnone}%
\def\hl@raw@word{raw}%
%
\def\enablehyperlinks{\@getoptionalarg\@finenablehyperlinks}%
\def\@finenablehyperlinks{%
  \let\hl@selecteddriver\empty
  % By default we generate `idxpage' index hyperlinks.
  \def\hldest@place@idx{0}%
  % Go through the option list.
  \for\hl@arg:=\@optionalarg\do{%
    \ifx\hl@arg\hl@idxexact@word
      \def\hldest@place@idx{1}%
    \else
      \ifx\hl@arg\hl@idxnone@word
        \def\hldest@place@idx{-1}%
      \else
        \ifx\hl@arg\hl@idxpage@word
          \def\hldest@place@idx{0}%
        \else
          \let\hl@selecteddriver\hl@arg
        \fi
      \fi
    \fi
  }%
  % Check the driver name.
  \ifx\hl@selecteddriver\empty
    % The user did not specify a driver, detect.
    \ifpdf
      \def\hl@selecteddriver{pdftex}%
      \message{^^JEplain: using `pdftex' hyperlink driver.}%
    \else
      \def\hl@selecteddriver{hypertex}%
      \message{^^JEplain: using `hypertex' hyperlink driver.}%
    \fi
  \else
    % Check that the requested driver's initialization routine is
    % available.
    \expandafter\ifx\csname hldriver@\hl@selecteddriver\endcsname \relax
      \errmessage{No hyperlink driver `\hl@selecteddriver' available}%
    \fi
  \fi
  % Enable \hltype, \hlopts, \hldest and \hldestopts now (the driver's
  % initialization routine may change this).
  \let\hl@setparam\@hl@setparam
  % Call the driver's initialization routine.
  \csname hldriver@\hl@selecteddriver\endcsname
  % Driver should not be changed later.
  \def\@finenablehyperlinks{\errmessage{Hyperlink driver `\hl@selecteddriver'
                                        already selected}}%
  % Free memory taken up by the drivers.
  \let\hldriver@nolinks\undefined
  \let\hldriver@hypertex\undefined
  \let\hldriver@pdftex \undefined
  \let\hldriver@dvipdfm\undefined
  % The user can use these to turn the links / destinations on/off
  % (see comments to the driver `nolinks').
  \let\hloff\@hloff
  \let\hlon\@hlon
  \let\hldestoff\@hldestoff
  \let\hldeston\@hldeston
  % By default turn everything on except the footnotes.
  \hlon[*,]\hloff[foot,footback]%
  \hldeston[*,]\hldestoff[foot,footback]%
}%
%
% Driver `nolinks'.
%
% Select this driver to suppress any hyperlinks / destinations in your
% document.
%
% NOTE:  selecting this driver is quite different from not selecting
% any driver at all, or from selecting some driver and then turning
% off links and destinations for the entire document with \hloff and
% \hldestoff.
%
% The purpose of \hldestoff and \hloff is to mark (parts) of document
% where links should never appear.  (Imagine you want to prevent a
% cross-referencing macro from generating a link at a certain spot in
% your document.)
%
% If instead you have prepared a document with links and just want to
% compile a version without the links, it is better to select the
% driver `nolinks'.  This will ensure that spacing and pagebreaking
% will be the same as what you were getting with hyperlinks enabled.
%
% The reason for this is that hyperlinks are produced by \special
% commands.  Each \special is placed inside a whatsit which may
% introduce a legitimate breakpoint at places where none would exist
% without the whatsit.  The macros \hldestoff and \hloff disable
% hyperlink macros so drastically that no whatsits are produced.
%
% On the other hand, `nolinks' driver does not completely disable
% hyperlink macros.  Instead, it defines them to write to the log
% file (what gets written is not really important).  This will produce
% the whatsits imitating the whatsits from the \special's.  (This
% trick was borrowed from graphics bundle.)
%
% Another reason for using `nolinks' is that in horizontal mode
% \hldest places destinations inside zero-width/height/depth boxes.
% When you say \hldestoff, \hldest will omit both destination specs
% and these boxes.  The missing boxes can cause typesetting to be
% inconsistent with what you were getting with destinations enabled.
% Again, `nolinks' driver helps here by defining \hldest to still
% produce the empty boxes.
%
% Additionally, `nolinks' driver defines the \hldesttype, \hldestopts,
% \hltype, \hlopts macros to gobble their parameters, to avoid error
% messages about "unknown" options and types under the `nolinks'
% driver.
\def\hldriver@nolinks{%
  \def\@hldest##1##2##3{%
    \edef\temp{\write-1{hldest: ##3}}%
    \ifvmode
      \temp
    \else
      \allowhyphens
      \expandafter\smash\expandafter{\temp}%
      \allowhyphens
    \fi
    \ignorespaces
  }%
  \def\@hlstart##1##2##3{%
    \leavevmode
    \begingroup % Start the color group.
    \edef\temp{\write-1{hlstart: ##3}}%
    \temp
    \ignorespaces
  }%
  \def\@hlend{%
    \edef\temp{\write-1{hlend}}%
    \temp
    \endgroup % End the color group from \@hlstart.
  }%
  % Make \hltype, \hlopts, \hldesttype and \hldestopts ignore their
  % parameters.
  \let\hl@setparam\gobble
}%
%
% Driver `hypertex'.
%
{\catcode`\#=\other
\gdef\hlhash{#}}%
%
\def\hldriver@hypertex{%
  %
  % Hyperlink destinations.
  %
  % Default type.
  \def\hldest@type{xyz}%
  % Set defaults for the options (this also tells \hl@set@opt what
  % options we support).  (We do not define \hldest@opt@raise,
  % \hl@opt@colormodel and \hl@opt@color, they are defined and used
  % outside the drivers.)
  \let\hldest@opt@cmd \empty
  % Multiplexer for all supported destination types.
  \def\hldest@driver{%
    % Special case for `raw' destinations.
    \ifx\@hltype\hl@raw@word
      \csname \hldest@opt@cmd \endcsname
    \else
      \special{html:<a name="\@hllabel">}\special{html:</a>}%
    \fi
  }%
  % Define handlers for each supported destination type (this also
  % tells \hl@getparam what types we support).
  \let\hldest@typeh@raw \empty
  \let\hldest@typeh@xyz \empty
  %
  % Hyperlinks.
  %
  % Default type.
  \def\hl@type{name}%
  % We support `url' hyperlinks, so set some group types.
  \ifx\hl@type@url\empty
    \def\hl@type@url{url}%
  \fi
  \ifx\hl@type@hrefext\empty
    \def\hl@type@hrefext{url}%
  \fi
  % Set defaults for the options (this also tells \hl@set@opt what
  % options we support).
  \let\hl@opt@cmd  \empty
  \let\hl@opt@ext  \empty
  \let\hl@opt@file \empty
  % Multiplexer for all supported link types.
  \def\hl@driver{%
    % Special case for `raw' links.
    \ifx\@hltype\hl@raw@word
      \csname \hl@opt@cmd \endcsname
    \else
      % Construct common preamble of a link.
      \def\hlstart@preamble{html:<a href="}%
      % Call the handler.
      \csname hl@typeh@\@hltype\endcsname
    \fi
  }%
  % Define handlers for each supported link type (this also tells
  % \hl@getparam what types we support).
  \let\hl@typeh@raw \empty
  \def\hl@typeh@name{\special{\hlstart@preamble \hlhash\@hllabel">}}%
  \def\hl@typeh@filename{%
    \special{%
      \hlstart@preamble
        file:\hl@opt@file\hl@opt@ext
        \ifempty\@hllabel \else \hlhash\@hllabel\fi
      ">%
    }%
  }%
  \def\hl@typeh@url{%
    \special{%
      \hlstart@preamble
        \@hllabel
      ">%
    }%
  }%
  %
  \def\@hlend{\special{html:</a>}\endgroup}% End the group from \@hlstart.
}%
%
% Driver `pdftex'.
%
\def\hldriver@pdftex{%
\ifpdf % PDF output is enabled.
  %
  % Hyperlink destinations.
  %
  % Default type.
  \def\hldest@type{xyz}%
  % Set defaults for the options (this also tells \hl@set@opt what
  % options we support).  (We do not define \hldest@opt@raise,
  % \hl@opt@colormodel and \hl@opt@color, they are defined and used
  % outside the drivers.)
  \let\hldest@opt@width  \empty
  \let\hldest@opt@height \empty
  \let\hldest@opt@depth  \empty
  \let\hldest@opt@zoom   \empty
  \let\hldest@opt@cmd    \empty
  % Multiplexer for all supported destination types.
  \def\hldest@driver{%
    % Special case for `raw' destinations.
    \ifx\@hltype\hl@raw@word
      \csname \hldest@opt@cmd \endcsname
    \else
      \pdfdest name{\@hllabel}\@hltype
        \csname hldest@typeh@\@hltype\endcsname
    \fi
  }%
  % Define handlers for each supported destination type (this also
  % tells \hl@getparam what types we support).
  \let\hldest@typeh@raw   \empty
  \let\hldest@typeh@fit   \empty
  \let\hldest@typeh@fith  \empty
  \let\hldest@typeh@fitv  \empty
  \let\hldest@typeh@fitb  \empty
  \let\hldest@typeh@fitbh \empty
  \let\hldest@typeh@fitbv \empty
  \def\hldest@typeh@fitr{%
    \ifx\hldest@opt@width  \empty \else width  \hldest@opt@width  \fi
    \ifx\hldest@opt@height \empty \else height \hldest@opt@height \fi
    \ifx\hldest@opt@depth  \empty \else depth  \hldest@opt@depth  \fi
  }%
  \def\hldest@typeh@xyz{%
    \ifx\hldest@opt@zoom\empty \else zoom \hldest@opt@zoom \fi
  }%
  %
  % Hyperlinks.
  %
  % Default type.
  \def\hl@type{name}%
  % We support `url' hyperlinks, so set some group types.
  \ifx\hl@type@url\empty
    \def\hl@type@url{url}%
  \fi
  \ifx\hl@type@hrefext\empty
    \def\hl@type@hrefext{url}%
  \fi
  % Set defaults for the options (this also tells \hl@set@opt what
  % options we support).
  \let\hl@opt@width   \empty
  \let\hl@opt@height  \empty
  \let\hl@opt@depth   \empty
  \def\hl@opt@bstyle  {S}%
  \def\hl@opt@bwidth  {1}%
  \let\hl@opt@bcolor  \empty
  \let\hl@opt@hlight  \empty
  \let\hl@opt@bdash   \empty
  \let\hl@opt@pagefit \empty
  \let\hl@opt@cmd     \empty
  \let\hl@opt@file    \empty
  \let\hl@opt@newwin  \empty
  % Multiplexer for all supported link types.
  \def\hl@driver{%
    % Special case for `raw' links.
    \ifx\@hltype\hl@raw@word
      \csname \hl@opt@cmd \endcsname
    \else
      % See if we will construct a /BS spec.  We want to bother only
      % if any of \hl@opt@bstyle, \hl@opt@bwidth and \hl@opt@bdash is
      % not empty.
      \let\hl@BSspec\relax % construct
      \ifx\hl@opt@bstyle \empty
        \ifx\hl@opt@bwidth \empty
          \ifx\hl@opt@bdash \empty
            \let\hl@BSspec\empty % don't construct
          \fi
        \fi
      \fi
      % Construct common preamble of a link.
      \def\hlstart@preamble{%
        \pdfstartlink
          \ifx\hl@opt@width  \empty \else width  \hl@opt@width  \fi
          \ifx\hl@opt@height \empty \else height \hl@opt@height \fi
          \ifx\hl@opt@depth  \empty \else depth  \hl@opt@depth \fi
          attr{%
            \ifx\hl@opt@bcolor\empty\else /C[\hl@opt@bcolor]\fi
            \ifx\hl@opt@hlight\empty\else /H/\hl@opt@hlight\fi
            \ifx\hl@BSspec\relax
              /BS<<%
                /Type/Border%
                \ifx\hl@opt@bstyle\empty\else /S/\hl@opt@bstyle\fi
                \ifx\hl@opt@bwidth\empty\else /W \hl@opt@bwidth\fi
                \ifx\hl@opt@bdash\empty \else /D[\hl@opt@bdash]\fi
              >>%
            \fi
          }%
      }%
      % Call the handler.
      \csname hl@typeh@\@hltype\endcsname
    \fi
  }%
  % Define handlers for each supported link type (this also tells
  % \hl@getparam what types we support).
  \let\hl@typeh@raw\empty
  \def\hl@typeh@name{\hlstart@preamble goto name{\@hllabel}}%
  \def\hl@typeh@num{\hlstart@preamble  goto num \@hllabel}%
  \def\hl@typeh@page{%
    % PDF requires pages to start from 0, so adjust page number.
    \count@=\@hllabel
    \advance\count@ by-1
    %
    \hlstart@preamble
    user{%
      /Subtype/Link%
      /Dest%
        [\the\count@
          \ifx\hl@opt@pagefit\empty/Fit\else\hl@opt@pagefit\fi]%
    }%
  }%
  \def\hl@typeh@filename{\hl@file{(\@hllabel)}}%
  \def\hl@typeh@filepage{%
    % PDF requires pages to start from 0, so adjust page number.
    \count@=\@hllabel
    \advance\count@ by-1
    %
    \hl@file{%
      [\the\count@ \ifx\hl@opt@pagefit\empty/Fit\else\hl@opt@pagefit\fi]%
    }%
  }%
  \def\hl@file##1{%
    \hlstart@preamble
    user{%
      /Subtype/Link%
      /A<<%
        /Type/Action%
        /S/GoToR%
        /D##1%
        /F(\hl@opt@file)%
        \ifx\hl@opt@newwin\empty \else
          /NewWindow \ifcase\hl@opt@newwin false\else true\fi
        \fi
      >>%
    }%
  }%
  \def\hl@typeh@url{%
    \hlstart@preamble
    user{%
      /Subtype/Link%
      /A<<%
        /Type/Action%
        /S/URI%
        /URI(\@hllabel)%
      >>%
    }%
  }%
  %
  \def\@hlend{\pdfendlink\endgroup}% End the group from the \@hlstart.
%
\else % PDF output is not enabled.
  \message{Eplain warning: `pdftex' hyperlink driver: PDF output is^^J
           \space not enabled, falling back on `nolinks' driver.}%
  \hldriver@nolinks
\fi
}%
%
% Driver `dvipdfm'.
%
\def\hldriver@dvipdfm{%
  %
  % Hyperlink destinations.
  %
  % Default type.
  \def\hldest@type{xyz}%
  % Set defaults for the options (this also tells \hl@set@opt what
  % options we support).  (We do not define \hldest@opt@raise,
  % \hl@opt@colormodel and \hl@opt@color, they are defined and used
  % outside the drivers.)
  \let\hldest@opt@left   \empty
  \let\hldest@opt@top    \empty
  \let\hldest@opt@right  \empty
  \let\hldest@opt@bottom \empty
  \let\hldest@opt@zoom   \empty
  \let\hldest@opt@cmd    \empty
  % Multiplexer for all supported destination types.
  \def\hldest@driver{%
    % Special case for `raw' destinations.
    \ifx\@hltype\hl@raw@word
      \csname \hldest@opt@cmd \endcsname
    \else
      % Construct common preamble of a destination.
      \def\hldest@preamble{%
        pdf: dest (\@hllabel) [@thispage
      }%
      % Call the handler.
      \csname hldest@typeh@\@hltype\endcsname
    \fi
  }%
  % Define handlers for each supported destination type (this also
  % tells \hl@getparam what types we support).
  \let\hldest@typeh@raw\empty
  \def\hldest@typeh@fit{%
    \special{\hldest@preamble /Fit]}%
  }%
  \def\hldest@typeh@fith{%
    \special{\hldest@preamble /FitH
      \ifx\hldest@opt@top\empty @ypos \else \hldest@opt@top \fi]}%
  }%
  \def\hldest@typeh@fitv{%
    \special{\hldest@preamble /FitV
      \ifx\hldest@opt@left\empty @xpos \else \hldest@opt@left \fi]}%
  }%
  \def\hldest@typeh@fitb{%
    \special{\hldest@preamble /FitB]}%
  }%
  \def\hldest@typeh@fitbh{%
    \special{\hldest@preamble /FitBH
      \ifx\hldest@opt@top\empty @ypos \else \hldest@opt@top \fi]}%
  }%
  \def\hldest@typeh@fitbv{%
    \special{\hldest@preamble /FitBV
      \ifx\hldest@opt@left\empty @xpos \else \hldest@opt@left \fi]}%
  }%
  \def\hldest@typeh@fitr{%
    \special{\hldest@preamble /FitR
      \ifx\hldest@opt@left\empty @xpos\else\hldest@opt@left\fi\space
      \ifx\hldest@opt@bottom\empty @ypos\else\hldest@opt@bottom\fi\space
      \ifx\hldest@opt@right\empty @xpos\else\hldest@opt@right\fi\space
      \ifx\hldest@opt@top\empty @ypos\else\hldest@opt@top \fi]}%
  }%
  \def\hldest@typeh@xyz{%
    \begingroup
      % Convert zoom factor:  12345 -> 12.345
      \ifx\hldest@opt@zoom\empty
        \count1=\z@ \count2=\z@
      \else
        \count2=\hldest@opt@zoom
        \count1=\count2 \divide\count1 by 1000
        \count3=\count1 \multiply\count3 by 1000
        \advance\count2 by -\count3
      \fi
      \special{\hldest@preamble /XYZ
        \ifx\hldest@opt@left\empty @xpos\else\hldest@opt@left\fi\space
        \ifx\hldest@opt@top\empty @ypos\else\hldest@opt@top\fi\space
        \the\count1.\the\count2]}%
    \endgroup
  }%
  %
  % Hyperlinks.
  %
  % Default type.
  \def\hl@type{name}%
  % We support `url' hyperlinks, so set some group types.
  \ifx\hl@type@url\empty
    \def\hl@type@url{url}%
  \fi
  \ifx\hl@type@hrefext\empty
    \def\hl@type@hrefext{url}%
  \fi
  % Set defaults for the options (this also tells \hl@set@opt what
  % options we support).
  \def\hl@opt@bstyle  {S}%
  \def\hl@opt@bwidth  {1}%
  \let\hl@opt@bcolor  \empty
  \let\hl@opt@hlight  \empty
  \let\hl@opt@bdash   \empty
  \let\hl@opt@pagefit \empty
  \let\hl@opt@cmd     \empty
  \let\hl@opt@file    \empty
  \let\hl@opt@newwin  \empty
  % Multiplexer for all supported link types.
  \def\hl@driver{%
    % Special case for `raw' links.
    \ifx\@hltype\hl@raw@word
      \csname \hl@opt@cmd \endcsname
    \else
      % See if we will construct a /BS spec.  We want to bother only
      % if any of \hl@opt@bstyle, \hl@opt@bwidth and \hl@opt@bdash is
      % not empty.
      \let\hl@BSspec\relax % construct
      \ifx\hl@opt@bstyle \empty
        \ifx\hl@opt@bwidth \empty
          \ifx\hl@opt@bdash \empty
            \let\hl@BSspec\empty % don't construct
          \fi
        \fi
      \fi
      % Construct common preamble of a link.
      \def\hlstart@preamble{%
        pdf: beginann
          <<%
            /Type/Annot%
            /Subtype/Link%
            \ifx\hl@opt@bcolor\empty\else /C[\hl@opt@bcolor]\fi
            \ifx\hl@opt@hlight\empty\else /H/\hl@opt@hlight\fi
            \ifx\hl@BSspec\relax
              /BS<<%
                /Type/Border%
                \ifx\hl@opt@bstyle\empty\else /S/\hl@opt@bstyle\fi
                \ifx\hl@opt@bwidth\empty\else /W \hl@opt@bwidth\fi
                \ifx\hl@opt@bdash\empty \else /D[\hl@opt@bdash]\fi
              >>%
            \fi
      }%
      % Call the handler.
      \csname hl@typeh@\@hltype\endcsname
    \fi
  }%
  % Define handlers for each supported link type (this also tells
  % \hl@getparam what types we support).
  \let\hl@typeh@raw\empty
  \def\hl@typeh@name{\special{\hlstart@preamble /Dest(\@hllabel)>>}}%
  \def\hl@typeh@page{%
    % PDF requires pages to start from 0, so adjust page number.
    \count@=\@hllabel
    \advance\count@ by-1
    %
    \special{%
      \hlstart@preamble
      /Dest[\the\count@
            \ifx\hl@opt@pagefit\empty/Fit\else\hl@opt@pagefit\fi]%
     >>%
    }%
  }%
  \def\hl@typeh@filename{\hl@file{(\@hllabel)}}%
  \def\hl@typeh@filepage{%
    % PDF requires pages to start from 0, so adjust page number.
    \count@=\@hllabel
    \advance\count@ by-1
    %
    \hl@file{%
      [\the\count@ \ifx\hl@opt@pagefit\empty/Fit\else\hl@opt@pagefit\fi]%
    }%
  }%
  \def\hl@file##1{%
    \special{%
      \hlstart@preamble
      /A<<%
        /Type/Action%
        /S/GoToR%
        /D##1%
        /F(\hl@opt@file)%
        \ifx\hl@opt@newwin\empty \else
          /NewWindow \ifcase\hl@opt@newwin false\else true\fi
        \fi
      >>%
     >>%
    }%
  }%
  \def\hl@typeh@url{%
    \special{%
      \hlstart@preamble
      /A<<%
        /Type/Action%
        /S/URI%
        /URI(\@hllabel)%
      >>%
     >>%
    }%
  }%
  %
  \def\@hlend{\special{pdf: endann}\endgroup}% End the group from \@hlstart.
}%
%
% Miscellaneous hyperlink macros.
%
%
% \href{URL}{TEXT} typesets TEXT as a link to the URL.  If URL starts
% with a #, the rest of the URL is assumed to be this document's local
% anchor.  Special chars (like # and ~) in URL don't need to be
% escaped in any way.
\def\href{%
  % Read #1 (URL) inside a group with \uncatcodespecials, to get the #
  % and ~ right.
  \bgroup
    \uncatcodespecials
    \catcode`\{=1 \catcode`\}=2
    \@href
}%
%
\def\@href#1{% We'll read #2 (TEXT) later.
  \egroup
  \edef\@hreftmp{\ifempty{#1}{}\fi}% Parameter stuffing for \@@href.
  \expandafter\@@href\@hreftmp#1\@@
}%
%
\def\href@end@int{\hlend@impl{hrefint}}%
\def\href@end@ext{\hlend@impl{hrefext}}%
% Split out the first token and check if it is a #.
\def\@@href#1#2\@@{%
  \def\@hreftmp{#1}%
  \ifx\@hreftmp\hlhash
    \let\href@end\href@end@int
    \hlstart@impl{hrefint}{#2}%
  \else
    \let\href@end\href@end@ext
    \hlstart@impl{hrefext}{#1#2}%
  \fi
  \@@@href
}%
% Now some tricks to avoid reading the TEXT as an argument (from the
% \footnote definition in plain TeX).
\def\@@@href{%
  \futurelet\@hreftmp\href@
}%
%
\def\href@{%
  \ifcat\bgroup\noexpand\@hreftmp
    \let\@hreftmp\href@@
  \else
    \let\@hreftmp\href@@@
  \fi
  \@hreftmp
}%
%
\def\href@@{\bgroup\aftergroup\href@end \let\@hreftmp}%
%
\def\href@@@#1{#1\href@end}%
%
% Make all user-visible \hl* macros to give errors until hyperlinks
% are explicitly enabled with \enablehyperlinks.
\def\hldeston{\errmessage{Please enable hyperlinks with
  \string\enablehyperlinks\space before using hyperlink commands
  (consider selecting the `nolinks' driver to ignore all hyperlink
  commands in your document)}}%
\let\hldestoff\hldeston \let\hlon\hldeston \let\hloff\hldeston
\let\hlstart\hldeston \let\hlend\hldeston \let\hldest\hldeston
% This catches \hltype, \hlopts, \hldesttype, \hldestopts.
\let\hl@setparam\hldeston
% Turn off all groups to make sure \hlstart@impl, \hlend@impl and
% \hldest@impl do not call \hlstart, \hlend and \hldest until
% hyperlinks are enabled.
\@hloff[*]\@hldestoff[*]%
%
%
%  Support for LaTeX packages under plain TeX.
%
% We use miniltx.tex from the LaTeX graphics collection and build on
% it to provide package options support, package version check,
% recursive package loading with \RequirePackage, proper handling of
% \AtBeginDocument and \AtEndOfPackage.
%
% Much of the following was borrowed from LaTeX.
%
% The internal variables are quite a mess, so here is a hint:
%
%  - \usepkg@pkg, \usepkg@options, \usepkg@date are used by
%    \usepackage to save its parameters.
%
%  - When \RequirePackage is called within a package, the above
%    variables are saved in \usepkg@save@VAR@RECURSIONLEVEL, where
%    VAR={pkg,options,date}, and RECURSIONLEVEL is incremented for
%    each nested package inclusion.  This way the variables can be
%    restored after the (nested) package will have been loaded.
%
%  - Options for package PACKAGE (no .sty extension) are accumulated
%    in \usepkg@options@PACKAGE.
%
%  - For each declared option OPTION in package PACKAGE, we save the
%    code which enables OPTION in \usepkg@option@PACKAGE@OPTION.
%    There may be a star (`*') option declaration, the code from which
%    will be used to process options not declared by the package
%    (without it, an undeclared option will cause an error).
%
%  - For each loaded PACKAGE / FILE.EXT we declare \ver@PACKAGE.sty /
%    \ver@FILE.EXT.  We use \ver@PACKAGE.sty to detect reloading
%    of packages.  Some packages also use these macros.
%
%  - Calls to \AtBeginDocument accumulate the code in
%    \usepkg@at@begin@document.  We will expand it at the end of the
%    \beginpackages...\endpackages `environment'.
%
%  - Calls to \AtEndOfPackage accumulate the code in
%    \usepkg@at@end@of@package.  We will expand it after the package
%    is loaded.  To allow recursive package loading, \RequirePackage
%    saves \usepkg@at@end@of@package analogous to \usepkg@pkg.
%
\newif\ifusepkg@miniltx@loaded
\newcount\usepkg@recursion@level
\def\usepkg@rcrs{\the\usepkg@recursion@level}%
\let\usepkg@at@begin@document\empty
\let\usepkg@at@end@of@package\empty
\def\usepkg@word@autopict{autopict}%
\def\usepkg@word@psfrag{psfrag}%
%
% \beginpackages...\endpackages
%
% All packages must be loaded within this `environment' (a kind of
% LaTeX's preamble).  This is so to give us a chance to expand the
% accumulated \AtBeginDocument commands.  Usually, you will want to
% restrict all \usepackage's to a single \beginpackages...\endpackages,
% just like in LaTeX, where you load all packages within a (single)
% preamble.
%
% At the end of the \beginpackages...\endpackages block, we restore
% the \input command to whatever it was defined before the
% package(s), because miniltx.tex and some packages make (sometimes
% incompatible) redefinitions of \input, and under plain TeX
% non-primitive \input is probably not what the user expects.  But in
% case the redefined \input is actually needed, we save it as
% \packageinput.
% 
% Define \DoNotLoadEpstopdf so that graphics.cfg does not try to load
% epstopdf.sty, which is a complication surely not wanted by default.
%
\long\def\beginpackages#1\endpackages{%
  \let\usepackage\real@usepackage
  \let\DoNotLoadEpstopdf=t
  \let\eplaininput=\input
  #1%
  \usepkg@at@begin@document
  \global\let\usepkg@at@begin@document\empty
  \global\let\usepackage\fake@usepackage
  \let\packageinput=\input
  \let\input=\eplaininput
  % should always be defined (from miniltx), but just in case:
  \ifx\resetatcatcode\@undefined \else\resetatcatcode \fi
}%
%
% \fake@usepackage
%
% This is what \usepackage is defined to outside of
% \beginpackages...\endpackages `environment'.
\def\fake@usepackage{\errmessage{You should not use \string\usepackage\space outside of^^J
  \@spaces\string\beginpackages...\string\endpackages\space environment}%
}%
% \RequirePackage[OPTIONS]{PACKAGES}[YYYY/MM/DD]
%
% Save parameters for the package being loaded and call \usepackage to
% load PACKAGES.  We depend on \usepackage to restore the saved
% parameters.  We allow for nested calls to \RequirePackage.
\def\eplain@RequirePackage{%
  \global\ece\let{usepkg@save@pkg\usepkg@rcrs}\usepkg@pkg
  \global\ece\let{usepkg@save@options\usepkg@rcrs}\usepkg@options
  \global\ece\let{usepkg@save@date\usepkg@rcrs}\usepkg@date
  \global\ece\let{usepkg@at@end@of@package\usepkg@rcrs}\usepkg@at@end@of@package
  \global\advance\usepkg@recursion@level by\@ne
  \real@usepackage
}%
% \usepackage[OPTIONS]{PACKAGES}[YYYY/MM/DD]
%
% Load each of the PACKAGES with OPTIONS, checking that the package
% date is at least YYYY/MM/DD; if it is not, issue a warning.
\let\usepackage\fake@usepackage
\def\real@usepackage{\@getoptionalarg\@finusepackage}%
\def\@finusepackage#1{%
  \let\usepkg@options\@optionalarg
  \ifempty{#1}%
    \errmessage{No packages specified}%
  \fi
  % Load miniltx.tex, if it is not loaded.
  \ifusepkg@miniltx@loaded \else
    \testfileexistence[miniltx]{tex}%
    \if@fileexists
      \input miniltx.tex
      \global\usepkg@miniltx@loadedtrue
      % Redefine some macros from miniltx.tex
      \global\let\RequirePackage\eplain@RequirePackage
      \global\let\DeclareOption\eplain@DeclareOption
      \global\let\PassOptionsToPackage\eplain@PassOptionsToPackage
      \global\let\ExecuteOptions\eplain@ExecuteOptions
      \gdef\ProcessOptions{\@ifstar\eplain@ProcessOptions
                                   \eplain@ProcessOptions}%
      \global\let\AtBeginDocument\eplain@AtBeginDocument
      \global\let\AtEndOfPackage\eplain@AtEndOfPackage
      \global\let\ProvidesFile\eplain@ProvidesFile
      \global\let\ProvidesPackage\eplain@ProvidesPackage
    \else
      \errmessage{miniltx.tex not found, cannot load LaTeX packages}%
    \fi
  \fi
  % Read the trailing optional arg (YYYY/MM/DD).
  \@ifnextchar[%]
    {\@finfinusepackage{#1}}%
    {\@finfinusepackage{#1}[]}%
}%
%
\def\@finfinusepackage#1[#2]{%
  \edef\usepkg@date{#2}%
  % Load each package from the list.  Do it outside the \for loop to
  % avoid clashes with any \for loops executed by the package.
  \let\usepkg@load@list\empty
  \for\usepkg@pkg:=#1\do{%
    \toks@=\expandafter{\usepkg@load@list}%
    \edef\usepkg@load@list{%
      \the\toks@
      \noexpand\usepkg@load@pkg{\usepkg@pkg}%
    }%
  }%
  \usepkg@load@list
  % Restore parameters in case we were called from \RequirePackage.
  \ifnum\usepkg@recursion@level>0
    \global\advance\usepkg@recursion@level by\m@ne
    \expandafter\let\expandafter\usepkg@pkg\csname usepkg@save@pkg\usepkg@rcrs\endcsname
    \expandafter\let\expandafter\usepkg@options\csname usepkg@save@options\usepkg@rcrs\endcsname
    \expandafter\let\expandafter\usepkg@date\csname usepkg@save@date\usepkg@rcrs\endcsname
    \expandafter\let\expandafter\usepkg@at@end@of@package\csname usepkg@at@end@of@package\usepkg@rcrs\endcsname
    % Free the memory.
    \global\ece\let{usepkg@save@pkg\usepkg@rcrs}\undefined
    \global\ece\let{usepkg@save@options\usepkg@rcrs}\undefined
    \global\ece\let{usepkg@save@date\usepkg@rcrs}\undefined
    \global\ece\let{usepkg@at@end@of@package\usepkg@rcrs}\undefined
  \fi
}%
% Load one package.  We assume packages have `sty' extension.
\def\usepkg@load@pkg#1{%
  % Special cases for `autopict' and `psfrag' packages:
  %   - `psfrag' is loaded by psfrag.tex.
  %   - `autopict' is loaded by picture.tex, but the package file is
  %      autopict.sty.
  \def\usepkg@pkg{#1}%
  \ifx\usepkg@pkg\usepkg@word@autopict
    \testfileexistence[picture]{tex}%
    \if@fileexists \else
      \errmessage{Loader `picture.tex' for package `\usepkg@pkg' not found}%
    \fi
  \else
    \ifx\usepkg@pkg\usepkg@word@psfrag
      \testfileexistence[psfrag]{tex}%
      \if@fileexists \else
        \errmessage{Loader `psfrag.tex' for package `\usepkg@pkg' not found}%
      \fi
    \fi
  \fi
  % See if the package was loaded.  If it was, we just bail out.
  % (Maybe we should skip reloading it/warn?  We don't want to go into
  % checking that the package was first loaded with a superset of
  % options which are requested now.)  \ProvidePackage and
  % \ProvideFile define the macro \ver@PACKAGE.sty (see
  % \eplain@pr@videpackage below).
  \ifundefined{ver@\usepkg@pkg.sty}%
    \expandafter\@finusepkg@load@pkg
  \else
    \immediate\write-1{^^J\linenumber Eplain: package `\usepkg@pkg' already
             loaded, skipping reloading}%
  \fi
}%
\def\@finusepkg@load@pkg{%
  \testfileexistence[\usepkg@pkg]{sty}%
  \if@fileexists \else
    \errmessage{Package `\usepkg@pkg' not found}%
  \fi
  % Construct option list for the package.  Include any options
  % passed to us by \PassOptionsToPackage and \RequirePackage.
  \expandafter\let\expandafter\temp\csname usepkg@options@\usepkg@pkg\endcsname
  \ifx\temp\relax
    \let\temp\empty
  \fi
  \ifx\temp\empty
    \let\temp\usepkg@options
  \else
    \ifx\usepkg@options\empty \else
      \edef\temp{\temp,\usepkg@options}%
    \fi
  \fi
  \global\ece\let{usepkg@options@\usepkg@pkg}\temp
  % For the duration of the package, we want any calls to \usepackage
  % to be mapped to \RequirePackage, to allow nested package loads
  % without clobbering up anything.  (Maybe packages never use
  % \usepackage instead of \RequirePackage, but this won't hurt.)
  \let\usepackage\eplain@RequirePackage
  % Clear \AtEndOfPackage commands (can be non-empty during recursive
  % package loading).
  \global\let\usepkg@at@end@of@package\empty
  % Load the package.  If the package asks to load other package,
  % \RequirePackage will save our \usepkg@{pkg,options,date} and
  % \AtEndOfPackage commands, and \usepackage will restore them after
  % recursively loading that package, so we don't worry about our
  % setup being clobbered.
  \ifx\usepkg@pkg\usepkg@word@autopict
    \input picture.tex
  \else
    \ifx\usepkg@pkg\usepkg@word@psfrag
      \input \usepkg@pkg.tex
    \else
      \input \usepkg@pkg.sty
    \fi
  \fi
  % Execute the accumulated \AtEndOfPackage commands, and reset them
  % to free the memory.
  \usepkg@at@end@of@package
  \global\let\usepkg@at@end@of@package\empty
  % Restore the `real' \usepackage definition.
  \let\usepackage\real@usepackage
  % Clear the option list for the loaded package (we won't load a
  % second copy anyway).
  \global\ece\let{usepkg@options@\usepkg@pkg}\undefined
  % Set \Url@HyperHook for url.sty
  \def\Url@HyperHook##1{\hlstart@impl{url}{\Url@String}##1\hlend@impl{url}}%
  % We check package version in \ProvidePackage, before reading the
  % rest of the package, so that in case of errors (which can be
  % numerous) the warning hopefully comes before the errors.
%  \@ifl@ter{sty}\usepkg@pkg\usepkg@date{}%
%    {\message{Warning: you have requested package `\usepkg@pkg', version \usepkg@date,^^J
%       \@spaces but only version `\csname ver@\usepkg@pkg.sty\endcsname' is available.}}%
}%
% \DeclareOption{OPTION}{CODE}
%
% Save CODE in `usepkg@option@PACKAGE@OPTION'.  Starred version is not
% treated any different here, but when defined it will be used by
% \ProcessOptions and \ExecuteOptions to process undeclared options.
\def\eplain@DeclareOption#1#2{%
  \toks@{#2}%
  \expandafter\xdef\csname usepkg@option@\usepkg@pkg @#1\endcsname{\the\toks@}%
}%
% \PassOptionsToPackage{OPTIONS}{PACKAGES}
%
% Add OPTIONS to the option list for each of the PACKAGES.
\def\eplain@PassOptionsToPackage#1#2{%
  \ifempty{#1}\else
    \for\usepkg@temp:=#2\do{%
      \expandafter\let\expandafter\temp\csname usepkg@options@\usepkg@temp\endcsname
      \ifx\temp\relax
        \let\temp\empty
      \fi
      \ifx\temp\empty
        \edef\temp{#1}%
      \else
        \edef\temp{\temp,#1}%
      \fi
      \global\ece\let{usepkg@options@\usepkg@temp}\temp
    }%
  \fi
}%
% \ExecuteOptions{OPTIONS}
% \ProcessOptions
%
% \ExecuteOptions executes each option from OPTIONS, \ProcessOptions
% executes each option from the option list for the current package.
% If the star (`*') option is declared, it will be used to process
% undeclared options; otherwise, undeclared option will cause an
% error.
\def\usepkg@exec@options#1{%
  % The iterator \CurrentOption below was used purposely.  Some
  % packages use it in the argument to \DeclareOption.
  \for\CurrentOption:=#1\do{%
    \expandafter\let\expandafter\usepkg@temp
      \csname usepkg@option@\usepkg@pkg @\CurrentOption\endcsname
%
    \ifx\usepkg@temp\relax
      % Option is not declared.  If a `default' option handler is
      % declared, use it.
      \expandafter\let\expandafter\temp\csname usepkg@option@\usepkg@pkg @*\endcsname
      \ifx\temp\relax
        \errmessage{Unknown option `\CurrentOption' to package `\usepkg@pkg'}%
      \else
        \temp
      \fi
    \else
      % Option is declared.
      \usepkg@temp
    \fi
  }%
}%
%
\let\eplain@ExecuteOptions\usepkg@exec@options
\def\eplain@ProcessOptions{%
  \expandafter\usepkg@exec@options\csname usepkg@options@\usepkg@pkg\endcsname
}%
% \AtBeginDocument{CODE}
% \AtEndOfPackage{CODE}
%
% miniltx.tex defines \AtBeginDocument to execute its parameter right
% away, but some packages depend on \AtBeginDocument to be executed
% after packages are processed.  So we redefine \AtBeginDocument to
% accumulate its argument, to be executed by us at the end of
% \beginpackages...\endpackages.  \AtEndOfPackage is not defined by
% miniltx.tex at all; we define it similar to \AtBeginDocument.
%
\def\usepkg@accumulate@cmds#1#2{%
  \toks@=\expandafter{#1}%
  \toks@ii={#2}%
  \xdef#1{\the\toks@\the\toks@ii}%
}%
\def\eplain@AtBeginDocument{\usepkg@accumulate@cmds\usepkg@at@begin@document}%
\def\eplain@AtEndOfPackage{\usepkg@accumulate@cmds\usepkg@at@end@of@package}%
%
% \ProvidesFile{FILENAME.EXT}[VERSION]
% \ProvidesPackage{PACKAGENAME}[VERSION]
%
% miniltx.tex defines \ProvidesFile and \ProvidesPackage to only log a
% message.  We want to define \ver@PACKAGE.sty / \ver@FILENAME.EXT, as
% we depend on these to know when a package/file has been loaded, and
% some packages depend on them, too.
\def\eplain@ProvidesPackage#1{%
  \@ifnextchar[%]
    {\eplain@pr@videpackage{#1.sty}}{\eplain@pr@videpackage#1[]}%
}%
\def\eplain@pr@videpackage#1[#2]{%
  \wlog{#1: #2}%
  % This will flag the package as loaded.
  \expandafter\xdef\csname ver@#1\endcsname{#2}%
%  \expandafter\edef\expandafter\temp{\csname usepkg@options@\usepkg@pkg\endcsname}%
%  \message{^^JPackage:\usepkg@pkg^^JOptions:\usepkg@options^^J+ passed options:\temp^^J}%
  % Check package version.
  \@ifl@t@r{#2}\usepkg@date{}%
    {\message{Warning: you have requested package `\usepkg@pkg', version \usepkg@date,^^J
       \@spaces but only version `\csname ver@#1\endcsname' is available.}}%
}%
%
\def\eplain@ProvidesFile#1{%
  \@ifnextchar[%]
    {\eplain@pr@videfile{#1}}{\eplain@pr@videfile#1[]}%
}%
\def\eplain@pr@videfile#1[#2]{%
  \wlog{#1: #2}%
  % This will flag the file as loaded.
  \expandafter\xdef\csname ver@#1\endcsname{#2}%
  % We don't check file version.  graphics calls \ProvideFile to
  % \includegraphics, and it does not give any date at the beginning
  % of #2, so checking the date will cause an error.
}%
% Check package version.  From LaTeX.
\def\@ifl@ter#1#2{%
  \expandafter\@ifl@t@r
    \csname ver@#2.#1\endcsname
}%
\def\@ifl@t@r#1#2{%
  \ifnum\expandafter\@parse@version#1//00\@nil<%
        \expandafter\@parse@version#2//00\@nil
    \expandafter\@secondoftwo
  \else
    \expandafter\@firstoftwo
  \fi
}%
\def\@parse@version#1/#2/#3#4#5\@nil{#1#2#3#4 }%
% For the `draft' option to graphic{s,x}.sty.
\let\ttfamily\tt
\def\strip@prefix#1>{}%
% Definitions for epstopdf.sty.
\def\@ifpackageloaded#1{%
  \expandafter\ifx\csname ver@#1.sty\endcsname\relax
    \expandafter\@secondoftwo
  \else
    \expandafter\@firstoftwo
  \fi
}%
\long\def\g@addto@macro#1#2{%
  \begingroup
    \toks@\expandafter{#1#2}%
    \xdef#1{\the\toks@}%
  \endgroup
}%
% These are for the warnings from epstopdf.sty when graphics.sty is
% not loaded.  From pdftex.def.
\def\PackageWarning#1#2{%
  \begingroup
    \newlinechar=10 %
    \def\MessageBreak{%
      ^^J(#1)\@spaces\@spaces\@spaces\@spaces
    }%
    \immediate\write16{^^JPackage #1 Warning: #2\on@line.^^J}%
  \endgroup
}%
\def\PackageWarningNoLine#1#2{%
  \PackageWarning{#1}{#2\@gobble}%
}%
\def\on@line{ on input line \the\inputlineno}%
% Needed by some packages.
\def\@spaces{\space\space\space\space}%
\def\@inmatherr#1{%
   \relax
   \ifmmode
     \errmessage{The command is invalid in math mode}%
   \fi
}%
\let\protected@edef\edef
%
%
%
\let\wlog = \@plainwlog
\catcode`@ = \@eplainoldatcode
%
\def\fmtname{eplain}%
\def\eplain{t}%
{\edef\plainversion{\fmtversion}%
 \xdef\fmtversion{REPLACE-WITH-VERSION: REPLACE-WITH-DAY-MONTH-YEAR (and plain \plainversion)}%
}%
%
% 
% 
% 
% Local variables:
% page-delimiter: "^% \f"
% End:
